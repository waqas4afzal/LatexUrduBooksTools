% \iffalse meta-comment
%
% File: zref.dtx
% Version: 2020-07-03 v2.32
% Info: A new reference scheme for LaTeX
%
% Copyright (C)
%    2006-2012  Heiko Oberdiek
%    2016-2020  Oberdiek Package Support Group
%    https://github.com/ho-tex/zref/issues
%
% This work may be distributed and/or modified under the
% conditions of the LaTeX Project Public License, either
% version 1.3c of this license or (at your option) any later
% version. This version of this license is in
%    https://www.latex-project.org/lppl/lppl-1-3c.txt
% and the latest version of this license is in
%    https://www.latex-project.org/lppl.txt
% and version 1.3 or later is part of all distributions of
% LaTeX version 2005/12/01 or later.
%
% This work has the LPPL maintenance status "maintained".
%
% The Current Maintainers of this work are
% Heiko Oberdiek and the Oberdiek Package Support Group
% https://github.com/ho-tex/zref/issues
%
% This work consists of the main source file zref.dtx
% and the derived files
%    zref.sty, zref.pdf, zref.ins, zref.drv, zref-base.sty,
%    zref-abspage.sty, zref-abspos.sty, zref-counter.sty,
%    zref-dotfill.sty, zref-env.sty, zref-hyperref.sty,
%    zref-lastpage.sty, zref-marks.sty, zref-nextpage.sty,
%    zref-pageattr.sty, zref-pagelayout.sty, zref-perpage.sty,
%    zref-runs.sty, zref-savepos.sty, zref-thepage.sty,
%    zref-titleref.sty, zref-totpages.sty, zref-user.sty,
%    zref-xr.sty, zref-example.tex, zref-example-lastpage.tex,
%    zref-example-nextpage.tex, zref-test1.tex,
%    zref-test-base.tex, zref-test-runs.tex,
%    zref-test-titleref-memoir.tex.
%
% Distribution:
%    CTAN:macros/latex/contrib/zref/zref.dtx
%    CTAN:macros/latex/contrib/zref/zref.pdf
%
% Unpacking:
%    (a) If zref.ins is present:
%           tex zref.ins
%    (b) Without zref.ins:
%           tex zref.dtx
%    (c) If you insist on using LaTeX
%           latex \let\install=y% \iffalse meta-comment
%
% File: zref.dtx
% Version: 2020-07-03 v2.32
% Info: A new reference scheme for LaTeX
%
% Copyright (C)
%    2006-2012  Heiko Oberdiek
%    2016-2020  Oberdiek Package Support Group
%    https://github.com/ho-tex/zref/issues
%
% This work may be distributed and/or modified under the
% conditions of the LaTeX Project Public License, either
% version 1.3c of this license or (at your option) any later
% version. This version of this license is in
%    https://www.latex-project.org/lppl/lppl-1-3c.txt
% and the latest version of this license is in
%    https://www.latex-project.org/lppl.txt
% and version 1.3 or later is part of all distributions of
% LaTeX version 2005/12/01 or later.
%
% This work has the LPPL maintenance status "maintained".
%
% The Current Maintainers of this work are
% Heiko Oberdiek and the Oberdiek Package Support Group
% https://github.com/ho-tex/zref/issues
%
% This work consists of the main source file zref.dtx
% and the derived files
%    zref.sty, zref.pdf, zref.ins, zref.drv, zref-base.sty,
%    zref-abspage.sty, zref-abspos.sty, zref-counter.sty,
%    zref-dotfill.sty, zref-env.sty, zref-hyperref.sty,
%    zref-lastpage.sty, zref-marks.sty, zref-nextpage.sty,
%    zref-pageattr.sty, zref-pagelayout.sty, zref-perpage.sty,
%    zref-runs.sty, zref-savepos.sty, zref-thepage.sty,
%    zref-titleref.sty, zref-totpages.sty, zref-user.sty,
%    zref-xr.sty, zref-example.tex, zref-example-lastpage.tex,
%    zref-example-nextpage.tex, zref-test1.tex,
%    zref-test-base.tex, zref-test-runs.tex,
%    zref-test-titleref-memoir.tex.
%
% Distribution:
%    CTAN:macros/latex/contrib/zref/zref.dtx
%    CTAN:macros/latex/contrib/zref/zref.pdf
%
% Unpacking:
%    (a) If zref.ins is present:
%           tex zref.ins
%    (b) Without zref.ins:
%           tex zref.dtx
%    (c) If you insist on using LaTeX
%           latex \let\install=y% \iffalse meta-comment
%
% File: zref.dtx
% Version: 2020-07-03 v2.32
% Info: A new reference scheme for LaTeX
%
% Copyright (C)
%    2006-2012  Heiko Oberdiek
%    2016-2020  Oberdiek Package Support Group
%    https://github.com/ho-tex/zref/issues
%
% This work may be distributed and/or modified under the
% conditions of the LaTeX Project Public License, either
% version 1.3c of this license or (at your option) any later
% version. This version of this license is in
%    https://www.latex-project.org/lppl/lppl-1-3c.txt
% and the latest version of this license is in
%    https://www.latex-project.org/lppl.txt
% and version 1.3 or later is part of all distributions of
% LaTeX version 2005/12/01 or later.
%
% This work has the LPPL maintenance status "maintained".
%
% The Current Maintainers of this work are
% Heiko Oberdiek and the Oberdiek Package Support Group
% https://github.com/ho-tex/zref/issues
%
% This work consists of the main source file zref.dtx
% and the derived files
%    zref.sty, zref.pdf, zref.ins, zref.drv, zref-base.sty,
%    zref-abspage.sty, zref-abspos.sty, zref-counter.sty,
%    zref-dotfill.sty, zref-env.sty, zref-hyperref.sty,
%    zref-lastpage.sty, zref-marks.sty, zref-nextpage.sty,
%    zref-pageattr.sty, zref-pagelayout.sty, zref-perpage.sty,
%    zref-runs.sty, zref-savepos.sty, zref-thepage.sty,
%    zref-titleref.sty, zref-totpages.sty, zref-user.sty,
%    zref-xr.sty, zref-example.tex, zref-example-lastpage.tex,
%    zref-example-nextpage.tex, zref-test1.tex,
%    zref-test-base.tex, zref-test-runs.tex,
%    zref-test-titleref-memoir.tex.
%
% Distribution:
%    CTAN:macros/latex/contrib/zref/zref.dtx
%    CTAN:macros/latex/contrib/zref/zref.pdf
%
% Unpacking:
%    (a) If zref.ins is present:
%           tex zref.ins
%    (b) Without zref.ins:
%           tex zref.dtx
%    (c) If you insist on using LaTeX
%           latex \let\install=y% \iffalse meta-comment
%
% File: zref.dtx
% Version: 2020-07-03 v2.32
% Info: A new reference scheme for LaTeX
%
% Copyright (C)
%    2006-2012  Heiko Oberdiek
%    2016-2020  Oberdiek Package Support Group
%    https://github.com/ho-tex/zref/issues
%
% This work may be distributed and/or modified under the
% conditions of the LaTeX Project Public License, either
% version 1.3c of this license or (at your option) any later
% version. This version of this license is in
%    https://www.latex-project.org/lppl/lppl-1-3c.txt
% and the latest version of this license is in
%    https://www.latex-project.org/lppl.txt
% and version 1.3 or later is part of all distributions of
% LaTeX version 2005/12/01 or later.
%
% This work has the LPPL maintenance status "maintained".
%
% The Current Maintainers of this work are
% Heiko Oberdiek and the Oberdiek Package Support Group
% https://github.com/ho-tex/zref/issues
%
% This work consists of the main source file zref.dtx
% and the derived files
%    zref.sty, zref.pdf, zref.ins, zref.drv, zref-base.sty,
%    zref-abspage.sty, zref-abspos.sty, zref-counter.sty,
%    zref-dotfill.sty, zref-env.sty, zref-hyperref.sty,
%    zref-lastpage.sty, zref-marks.sty, zref-nextpage.sty,
%    zref-pageattr.sty, zref-pagelayout.sty, zref-perpage.sty,
%    zref-runs.sty, zref-savepos.sty, zref-thepage.sty,
%    zref-titleref.sty, zref-totpages.sty, zref-user.sty,
%    zref-xr.sty, zref-example.tex, zref-example-lastpage.tex,
%    zref-example-nextpage.tex, zref-test1.tex,
%    zref-test-base.tex, zref-test-runs.tex,
%    zref-test-titleref-memoir.tex.
%
% Distribution:
%    CTAN:macros/latex/contrib/zref/zref.dtx
%    CTAN:macros/latex/contrib/zref/zref.pdf
%
% Unpacking:
%    (a) If zref.ins is present:
%           tex zref.ins
%    (b) Without zref.ins:
%           tex zref.dtx
%    (c) If you insist on using LaTeX
%           latex \let\install=y\input{zref.dtx}
%        (quote the arguments according to the demands of your shell)
%
% Documentation:
%    (a) If zref.drv is present:
%           latex zref.drv
%    (b) Without zref.drv:
%           latex zref.dtx; ...
%    The class ltxdoc loads the configuration file ltxdoc.cfg
%    if available. Here you can specify further options, e.g.
%    use A4 as paper format:
%       \PassOptionsToClass{a4paper}{article}
%
%    Programm calls to get the documentation (example):
%       pdflatex zref.dtx
%       makeindex -s gind.ist zref.idx
%       pdflatex zref.dtx
%       makeindex -s gind.ist zref.idx
%       pdflatex zref.dtx
%
% Installation:
%    TDS:tex/latex/zref/zref.sty
%    TDS:tex/latex/zref/zref-base.sty
%    TDS:tex/latex/zref/zref-abspage.sty
%    TDS:tex/latex/zref/zref-abspos.sty
%    TDS:tex/latex/zref/zref-counter.sty
%    TDS:tex/latex/zref/zref-dotfill.sty
%    TDS:tex/latex/zref/zref-env.sty
%    TDS:tex/latex/zref/zref-hyperref.sty
%    TDS:tex/latex/zref/zref-lastpage.sty
%    TDS:tex/latex/zref/zref-marks.sty
%    TDS:tex/latex/zref/zref-nextpage.sty
%    TDS:tex/latex/zref/zref-pageattr.sty
%    TDS:tex/latex/zref/zref-pagelayout.sty
%    TDS:tex/latex/zref/zref-perpage.sty
%    TDS:tex/latex/zref/zref-runs.sty
%    TDS:tex/latex/zref/zref-savepos.sty
%    TDS:tex/latex/zref/zref-thepage.sty
%    TDS:tex/latex/zref/zref-titleref.sty
%    TDS:tex/latex/zref/zref-totpages.sty
%    TDS:tex/latex/zref/zref-user.sty
%    TDS:tex/latex/zref/zref-xr.sty
%    TDS:doc/latex/zref/zref.pdf
%    TDS:doc/latex/zref/zref-example.tex
%    TDS:doc/latex/zref/zref-example-lastpage.tex
%    TDS:doc/latex/zref/zref-example-nextpage.tex
%    TDS:source/latex/zref/zref.dtx
%
%<*ignore>
\begingroup
  \catcode123=1 %
  \catcode125=2 %
  \def\x{LaTeX2e}%
\expandafter\endgroup
\ifcase 0\ifx\install y1\fi\expandafter
         \ifx\csname processbatchFile\endcsname\relax\else1\fi
         \ifx\fmtname\x\else 1\fi\relax
\else\csname fi\endcsname
%</ignore>
%<*install>
\input docstrip.tex
\Msg{************************************************************************}
\Msg{* Installation}
\Msg{* Package: zref 2020-07-03 v2.32 A new reference scheme for LaTeX (HO)}
\Msg{************************************************************************}

\keepsilent
\askforoverwritefalse

\let\MetaPrefix\relax
\preamble

This is a generated file.

Project: zref
Version: 2020-07-03 v2.32

Copyright (C)
  2006-2012 Heiko Oberdiek
  2016-2020 Oberdiek Package Support Group

This work may be distributed and/or modified under the
conditions of the LaTeX Project Public License, either
version 1.3c of this license or (at your option) any later
version. This version of this license is in
   https://www.latex-project.org/lppl/lppl-1-3c.txt
and the latest version of this license is in
   https://www.latex-project.org/lppl.txt
and version 1.3 or later is part of all distributions of
LaTeX version 2005/12/01 or later.

This work has the LPPL maintenance status "maintained".

The Current Maintainers of this work are
Heiko Oberdiek and the Oberdiek Package Support Group
https://github.com/ho-tex/zref/issues


This work consists of the main source file zref.dtx
and the derived files
   zref.sty, zref.pdf, zref.ins, zref.drv, zref-base.sty,
   zref-abspage.sty, zref-abspos.sty, zref-counter.sty,
   zref-dotfill.sty, zref-env.sty, zref-hyperref.sty,
   zref-lastpage.sty, zref-marks.sty, zref-nextpage.sty,
   zref-pageattr.sty, zref-pagelayout.sty, zref-perpage.sty,
   zref-runs.sty, zref-savepos.sty, zref-thepage.sty,
   zref-titleref.sty, zref-totpages.sty, zref-user.sty,
   zref-xr.sty, zref-example.tex, zref-example-lastpage.tex,
   zref-example-nextpage.tex, zref-test1.tex,
   zref-test-base.tex, zref-test-runs.tex,
   zref-test-titleref-memoir.tex.

\endpreamble
\let\MetaPrefix\DoubleperCent

\generate{%
  \file{zref.ins}{\from{zref.dtx}{install}}%
  \file{zref.drv}{\from{zref.dtx}{driver}}%
  \usedir{tex/latex/zref}%
  \file{zref.sty}{\from{zref.dtx}{package}}%
  \file{zref-base.sty}{\from{zref.dtx}{base}}%
  \file{zref-abspage.sty}{\from{zref.dtx}{abspage}}%
  \file{zref-abspos.sty}{\from{zref.dtx}{abspos}}%
  \file{zref-counter.sty}{\from{zref.dtx}{counter}}%
  \file{zref-dotfill.sty}{\from{zref.dtx}{dotfill}}%
  \file{zref-env.sty}{\from{zref.dtx}{env}}%
  \file{zref-hyperref.sty}{\from{zref.dtx}{hyperref}}%
  \file{zref-lastpage.sty}{\from{zref.dtx}{lastpage}}%
  \file{zref-marks.sty}{\from{zref.dtx}{marks}}%
  \file{zref-nextpage.sty}{\from{zref.dtx}{nextpage}}%
  \file{zref-pageattr.sty}{\from{zref.dtx}{pageattr}}%
  \file{zref-pagelayout.sty}{\from{zref.dtx}{pagelayout}}%
  \file{zref-perpage.sty}{\from{zref.dtx}{perpage}}%
  \file{zref-runs.sty}{\from{zref.dtx}{runs}}%
  \file{zref-savepos.sty}{\from{zref.dtx}{savepos}}%
  \file{zref-thepage.sty}{\from{zref.dtx}{thepage}}%
  \file{zref-titleref.sty}{\from{zref.dtx}{titleref}}%
  \file{zref-totpages.sty}{\from{zref.dtx}{totpages}}%
  \file{zref-user.sty}{\from{zref.dtx}{user}}%
  \file{zref-xr.sty}{\from{zref.dtx}{xr}}%
  \usedir{doc/latex/zref}%
  \file{zref-example.tex}{\from{zref.dtx}{example}}%
  \file{zref-example-lastpage.tex}{\from{zref.dtx}{example-lastpage}}%
  \file{zref-example-nextpage.tex}{\from{zref.dtx}{example-nextpage}}%
%  \usedir{doc/latex/zref/test}%
%  \file{zref-test1.tex}{\from{zref.dtx}{test1}}%
%  \file{zref-test-base.tex}{\from{zref.dtx}{test-base}}%
%  \file{zref-test-runs.tex}{\from{zref.dtx}{test-runs}}%
%  \file{zref-test-titleref-memoir.tex}{\from{zref.dtx}{test-titleref-memoir}}%
}

\catcode32=13\relax% active space
\let =\space%
\Msg{************************************************************************}
\Msg{*}
\Msg{* To finish the installation you have to move the following}
\Msg{* files into a directory searched by TeX:}
\Msg{*}
\Msg{*     zref.sty, zref-base.sty, zref-abspage.sty, zref-abspos.sty,}
\Msg{*     zref-counter.sty, zref-dotfill.sty, zref-env.sty,}
\Msg{*     zref-hyperref.sty, zref-lastpage.sty, zref-marks.sty,}
\Msg{*     zref-nextpage.sty, zref-pageattr.sty, zref-pagelayout.sty,}
\Msg{*     zref-perpage.sty, zref-runs.sty, zref-savepos.sty,}
\Msg{*     zref-thepage.sty, zref-titleref.sty, zref-totpages.sty,}
\Msg{*     zref-user.sty, zref-xr.sty}
\Msg{*}
\Msg{* To produce the documentation run the file `zref.drv'}
\Msg{* through LaTeX.}
\Msg{*}
\Msg{* Happy TeXing!}
\Msg{*}
\Msg{************************************************************************}

\endbatchfile
%</install>
%<*ignore>
\fi
%</ignore>
%<*driver>
\NeedsTeXFormat{LaTeX2e}
\ProvidesFile{zref.drv}%
  [2020-07-03 v2.32 A new reference scheme for LaTeX (HO)]%
\documentclass{ltxdoc}
\usepackage{holtxdoc}[2011/11/22]
\setlength{\emergencystretch}{1em}
\begin{document}
  \DocInput{zref.dtx}%
\end{document}
%</driver>
% \fi
%
%
%
% \GetFileInfo{zref.drv}
%
% \title{The \xpackage{zref} package}
% \date{2020-07-03 v2.32}
% \author{Heiko Oberdiek\thanks
% {Please report any issues at \url{https://github.com/ho-tex/zref/issues}}}
%
% \maketitle
%
% \begin{abstract}
% Package \xpackage{zref} tries to get rid of the restriction
% in \hologo{LaTeX}'s reference system that only two properties are
% supported. The package implements an extensible referencing
% system, where properties are handled in a more flexible way.
% It offers an interface for macro programmers for the access
% to the system and some applications that uses the new
% reference scheme.
% \end{abstract}
%
% \tableofcontents
%
% \newcommand*{\metacs}[1]{\texttt{\itshape #1}}
%
% \newcommand{\markexp}{^^A
%   \textsuperscript{exp}^^A
%   \space
% }
% \newcommand{\markexptwo}{^^A
%   \textsuperscript{exp2}^^A
%   \space
% }
% \newcommand{\markrobust}{^^A
%   \textsuperscript{robust}^^A
%   \space
% }
% \newcommand{\marknowarn}{^^A
%   \textsuperscript{nowarn}^^A
%   \space
% }
% \newcommand{\markbabel}{^^A
%   \textsuperscript{babel}^^A
%   \space
% }
%
% \section{Introduction}
%
% Standard \hologo{LaTeX}'s reference system with \cs{label}, \cs{ref},
% and \cs{pageref} supports two properties, the apperance of
% the counter that is last incremented by \cs{refstepcounter} and
% the page with the \cs{label} command.
%
% Unhappily \hologo{LaTeX} does not provide an interface for adding
% another properties. Packages such as \xpackage{hyperref},
% \xpackage{nameref}, or \xpackage{titleref} are forced to
% use ugly hacks to extend the reference system.
% These ugly hacks are one of the causes for \xpackage{hyperref}'s
% difficulty regarding compatibility with other packages.
%
% \subsection{Standard \hologo{LaTeX} behaviour}
%
% References are created by the \cs{label} command:
% \begin{quote}
%   |\chapter{Second chapter}|\\
%   |\section{First section on page 7} % section 2.1|\\
%   |\label{myref}|
% \end{quote}
% Now \hologo{LaTeX} records the section number |2.1| and the page |7|
% in the reference. Internally the reference is a list with
% two entries:
% \begin{quote}
%   |\r@myref| $\rightarrow$ |{2.1}{7}|
% \end{quote}
% The length of the list if fixed in the \hologo{LaTeX} kernel,
% An interface for adding new properties is missing.
%
% There are several tries to add new properties:
% \begin{description}
% \item[\xpackage{hyperref}] uses a list of five
%   properties instead of the standard list with two entries.
%   This causes many compatibility problems with \hologo{LaTeX} and
%   other packages.
% \item[\xpackage{titleref}] stores its title data into the
%   first entry in the list. \hologo{LaTeX} is happy because it
%   does only see its list with two entries. The situation
%   becomes more difficult, if more properties are added
%   this way. Then the macros form a nested structure
%   inside the first reference argument for the label.
%   Expandable extractions will then become painful.
% \end{description}
%
% \subsection{Basic idea}
%
% Some time ago Morten H\o gholm sent me an experimental
% cross referencing mechanism as ``expl3'' code. His
% idea is:
% \begin{quote}
%   |\g_xref_mylabel_plist| $\rightarrow$\\
%   \mbox{}\qquad|\xref_dance_key{salsa}\xref_name_key{Morten}|\dots
% \end{quote}
% The entries have the following format:
% \begin{quote}
%   |\xref_|\meta{your key}|_key{|\meta{some text}|}|
% \end{quote}
% This approach is much more flexible:
% \begin{itemize}
%   \item New properties can easily be added, just use a new key.
%   \item The length of the list is not fixed. A reference
%         can use a subset of the keys.
%   \item The order of the entries does not matter.
% \end{itemize}
% Unhappily I am not familiar with the experimental code
% for \hologo{LaTeX3} that will need some time before its first release.
% Thus I have implemented it as \hologo{LaTeXe} package without disturbing
% the existing \hologo{LaTeX} reference system.
%
% \subsection{Interfaces}
%
% The package provides a generic \emph{interface for programmers}.
% Commands of this interface are prefixed by \cs{zref@}.
%
% Option \xoption{user} enabels the \emph{user interface}.
% Here the commands are prefixed by \cs{z} to avoid name
% clashes with existing macros.
%
% Then the packages provides some \emph{modules}. They
% are applications for the reference system and can also be
% considered as examples how to use the reference system.
%
% The modules can be loaded as packages. The package
% name is prefixed with |zref-|, for example:
%\begin{quote}
%\begin{verbatim}
%\RequirePackage{zref-abspage}
%\end{verbatim}
%\end{quote}
% This is the preferred way if the package is loaded from
% within other packages to avoid option clashes.
%
% As alternative package |zref| can be used and the modules
% are given as options:
%\begin{quote}
%\begin{verbatim}
%\usepackage[perpage,user]{zref}
%\end{verbatim}
%\end{quote}
%
% \section{Interface for programmers}
%
% The user interface is described in the next section \ref{sec:user}.
%
% \subsection{Entities}
%
% \paragraph{Reference.}
% Internally a reference is a list of key value pairs:
% \begin{quote}
%   |\Z@R@myref| $\rightarrow$
%   |\default{2.1}\page{7}|
% \end{quote}
% The generic format of a entry is:
% \begin{quote}
%   |\Z@R@|\meta{refname} $\rightarrow$
%   |\|\meta{propname}|{|\meta{value}|}|
% \end{quote}
% \meta{refname} is the name that denoted references
% (the name used in \cs{label} and \cs{ref}). \meta{propname}
% is the name of the property or key. The property key macro
% is never executed, it is used in parameter text matching only.
%
% \paragraph{Property.}
% Because the name of a property is used in a macro name
% that must survive the \xfile{.aux} file, the name is
% restricted to letters and `@'.
%
% \paragraph{Property list.}
% Often references are used for special purposes. Thus it
% saves memory if just the properties are used in this
% reference that are necessary for its purpose.
%
% Therefore this package uses the concept of \emph{property lists}.
% A property list is a set of properties. The set of properties
% that is used by the default \cs{label} command is the
% \emph{main property list}.
%
% \subsection{Property list}
%
% \markexp means that the implementation of the
% marked macro is expandable. \markexptwo goes a step further
% and marks the macro expandable in exact two expansion steps.
%
% \begin{declcs}{zref@newlist}
%   \M{listname}
% \end{declcs}
% Declares a new empty property list.
%
% \begin{declcs}{zref@addprop}
%   \M{listname} \M{propname}\\
%   \cs{zref@localaddprop} \M{listname} \M{propname}
% \end{declcs}
% Adds the property \meta{propname}
% to the property list \meta{listname}.
% The property and list must exist.
% The addition is global by \cs{zref@addprop} and limited to
% local scope by \cs{zref@localaddprop}.
% Between 2010/04/19 v2.13 and 2010/10/22 v2.19 a comma separated list
% of properties could be used as argument \meta{propname}.
% Since 2010/10/22 v2.19 the addition of several properties
% at once is supported by \cs{zref@addprops}.
%
% \begin{declcs}{zref@addprops}
%   \M{listname} \M{propname list}\\
%   \cs{zref@localaddprops} \M{listname} \M{propname list}
% \end{declcs}
% These macros add a comma separated list of properties
% \meta{propname list} to list \meta{listname}.
% \cs{zref@addprops} works globally and \cs{zref@localaddprops}
% locally. Since 2010/10/22 v2.19.
%
% \begin{declcs}{zref@listexists}
%   \M{listname} \M{then}
% \end{declcs}
% Executes \meta{then} if the property list \meta{listname} exists
% or raise an error otherwise.
%
% \begin{declcs}{zref@iflistundefined}\markexp \M{listname}
%   \M{then} \M{else}
% \end{declcs}
% Executes \meta{then} if the list exists or \meta{else}
% otherwise.
%
% \begin{declcs}{zref@iflistcontainsprop}
%   \M{listname} \M{propname} \M{then} \M{else}
% \end{declcs}
% Executes \meta{then} if the property \meta{propname} is part
% of property list \meta{listname} or otherwise it
% runs the \meta{else} part.
%
% \subsection{Property}
%
% \begin{declcs}{zref@newprop}
%   \* \M{propname} |[|\meta{default}|]| \M{value}
% \end{declcs}
% This command declares and configures a new property with
% name \meta{propname}.
%
% In case of unknown references
% or the property does not exist in the reference, the
% \meta{default} is used as value. If it is not specified
% here, a global default is used, see \cs{zref@setdefault}.
%
% The correct values of some properties are not known
% immediately but at page shipout time. Prominent example
% is the page number. These properties are declared with
% the star form of the command.
%
% \begin{declcs}{zref@setcurrent}
%   \M{propname} \M{value}
% \end{declcs}
% This sets the current value of the property \meta{propname}.
% It is a generalization of setting \hologo{LaTeX}'s \cs{currentlabel}.
%
% \begin{declcs}{zref@getcurrent}\markexptwo
%   \M{propname}
% \end{declcs}
% This returns the current value of the property \meta{propname}.
% The value may not be correct, especially if the property
% is bound to a page (start form of \cs{zref@newprop}) and the
% right value is only known at shipout time (e.g. property `page').
% In case of errors (e.g. unknown property) the empty string is
% returned.
%
% Since version 2010/04/22 v2.14 \cs{zref@getcurrent} supports
% \cs{zref@wrapper@unexpanded}.
%
% \begin{declcs}{zref@propexists}
%   \M{propname} \M{then}
% \end{declcs}
% Calls \meta{then} if the property \meta{propname} is available
% or generates an error message otherwise.
%
% \begin{declcs}{zref@ifpropundefined}\markexp
%   \M{propname} \M{then} \M{else}
% \end{declcs}
% Calls \meta{then} or \meta{else} depending on the existence
% of property \meta{propname}.
%
% \subsection{Reference generation}
%
% \begin{declcs}{zref@label}
%   \M{refname}
% \end{declcs}
% This works similar to \cs{label}. The reference \meta{refname} is
% created and put into the \xfile{.aux} file with the
% properties of the main property list.
%
% \begin{declcs}{zref@labelbylist}
%   \M{refname} \M{listname}
% \end{declcs}
% Same as \cs{zref@label} except that the properties are
% taken from the specified property list \meta{listname}.
%
% \begin{declcs}{zref@labelbyprops}
%   \M{refname} |{|\meta{propnameA}|,|\meta{propnameB}|,|\dots|}|
% \end{declcs}
% Same as \cs{zref@label} except that these properties
% are used that are given as comma separated list in the
% second argument.
%
% \begin{declcs}{zref@newlabel}
%   \M{refname} |{|\dots|}|
% \end{declcs}
% This is the macro that is used in the \xfile{.aux} file.
% It is basically the same as \cs{newlabel} apart from
% the format of the data in the second argument.
%
% \subsection{Data extraction}
%
% \begin{declcs}{zref@extractdefault}\markexptwo
%   \M{refname} \M{propname} \M{default}
% \end{declcs}
% This is the basic command that refernces the value of a
% property \meta{propname} for the reference \meta{refname}.
% In case of errors such as undefined reference the \meta{default}
% is used instead.
%
% \begin{declcs}{zref@extract}\markexptwo
%   \M{refname} \M{propname}
% \end{declcs}
% The command is an abbreviation for \cs{zref@extractdefault}.
% As default the default of the property is taken, otherwise
% the global default.
%
% Example for page references:
% \begin{quote}
% \begin{tabular}{@{}ll}
% \hologo{LaTeX}:& |\pageref{foobar}|\\
% \xpackage{zref}:& |\zref@extract{foobar}{page}|
% \end{tabular}
% \end{quote}
% Both \cs{zref@extract} and \cs{zref@extractdefault} are
% expandable. That means, these macros can directly
% be used in expandable calculations, see the example file.
% On the other side, babel's shorthands are not supported,
% there are no warnings in case of undefined references.
%
% If an user interface doesn't need expandable macros
% then it can use \cs{zref@refused} and \cs{zref@wrapper@babel}
% for its user macros.
%
% \begin{declcs}{zref@refused}
%   \M{refname}
% \end{declcs}
% This command is not expandable. It causes the warnings if
% the reference \meta{refname} is not defined. Use the
% \cs{zref@extract} commands inside expandable contexts and
% mark their use outside by \cs{zref@refused}, see the example
% file.
%
% \begin{declcs}{zref@def@extract}%
%     \M{cmd} \M{refname} \M{propname}\\
%   \cs{zref@def@extractdefault}
%     \M{cmd} \M{refname} \M{propname} \M{default}
% \end{declcs}
% Both macros extract the property \meta{propname} from the reference
% \meta{refname} the same way as macros \cs{zref@extract} and
% \cs{zref@extractdefault}. The result is stored in macro \meta{cmd}.
% Also \cs{zref@refused} is called to notify \hologo{LaTeX} that
% the reference \meta{refname} is used. Added in 2011/10/04 v2.22.
%
% \begin{declcs}{zref@ifrefundefined}\markexp
%   \M{refname} \M{then} \M{else}
% \end{declcs}
% Macro \cs{zref@ifrefundefined} calls arguments \meta{then} or \meta{else}
% dependent on the existence of the reference \meta{refname}.
%
% \begin{declcs}{zifrefundefined}
%   \M{refname} \M{then} \M{else}
% \end{declcs}
% Macro \cs{zifrefundefined} calls \cs{ref@refused} before
% executing \cs{zref@ifrefundefined}. Babel shorthands are
% supported in \meta{refname}.
%
% \begin{declcs}{zref@ifrefcontainsprop}\markexp
%   \M{refname} \M{propname} \M{then} \M{else}
% \end{declcs}
% Test whether a reference provides a property.
%
% \subsection{Setup}
%
% \begin{declcs}{zref@default}
% \end{declcs}
% Holds the global default for unknown values.
%
% \begin{declcs}{zref@setdefault}
%   |{|\meta{value}|}|
% \end{declcs}
% Sets the global default for unknown values.
% The global default is used, if a property does not specify
% an own default and the value for a property
% cannot be extracted. This can happen if the reference is
% unknown or the reference does not have the property.
%
% \begin{declcs}{zref@setmainlist}
%   |{|\meta{value}|}|
% \end{declcs}
% Sets the name of the main property list. The package
% sets and uses |main|.
%
% \subsection{Declared properties}
%
% \begin{quote}
% \renewcommand*{\arraystretch}{1.05}
% \setlength{\tabcolsep}{2\tabcolsep}
% \def\VV{\vphantom{\large \"Ag}}
% \def\EMPTY{{\itshape\textless empty\textgreater}}
% \def\pl#1{&#1&thepage&\cs{number}\cs{#1}}
% \begin{tabular}{@{}>{\sffamily}l|>{\ttfamily}l|>{\ttfamily}l|>{\ttfamily}l@{}}
%   \rmfamily Module&
%   \rmfamily Property&
%   \rmfamily Property list&
%   \rmfamily Default\\
%   \hline
%   (base)
%     & default & main & \EMPTY\\
%     & page & main & \EMPTY\\
%   \hline
%   abspage
%     & abspage & main & 0\\
%   \hline
%   counter
%     & counter & main & \EMPTY\\
%   \hline
%   hyperref
%     & anchor & main & \EMPTY\\
%     & url    & & \EMPTY\\
%   \hline
%   pageattr
%     & pdfpageattr & thepage & \dots\\
%     & pdfpagesattr & LastPage & \dots\\
%   \hline
%   pagelayout\footnotemark
%     \pl{mag}\\
%     \pl{paperwidth}\\
%     \pl{paperheight}\\
%     \pl{stockwidth}\\
%     \pl{stockheight}\\
%     \pl{pdfpageheight}\\
%     \pl{pdfpagewidth}\\
%     \pl{pdfhorigin}\\
%     \pl{pdfvorigin}\\
%     \pl{hoffset}\\
%     \pl{voffset}\\
%     \pl{topmargin}\\
%     \pl{oddsidemargin}\\
%     \pl{evensidemargin}\\
%     \pl{textwidth}\\
%     \pl{textheight}\\
%     \pl{headheight}\\
%     \pl{headsep}\\
%     \pl{footskip}\\
%     \pl{marginparwidth}\\
%     \pl{marginparsep}\\
%     \pl{columnwidth}\\
%     \pl{columnsep}\\
%   \hline
%   perpage
%     & pagevalue & perpage & 0\\
%     & page & perpage & \EMPTY\\
%     & abspage & perpage & 0\\
%   \hline
%   savepos
%     & posx & savepos & 0\\
%     & posy & savepos & 0\\
%   \hline
%   titleref
%     & title & main & \EMPTY\\
%   \hline
%   xr
%     & anchor   & & \EMPTY\\
%     & externaldocument & & \EMPTY\\
%     & theotype & & \EMPTY\\
%     & title    & & \EMPTY\\
%     & url      & & \EMPTY\\
% \end{tabular}
% \footnotetext{Module \xmodule{pagelayout} only defines properties
% if the parameter exists.}%
% \end{quote}
%
% \subsection{Wrapper for advanced situations}
%
% \begin{declcs}{zref@wrapper@babel}
%   |{|\dots|}| \M{name}
% \end{declcs}
% This macro helps to add shorthand support. The second
% argument is protected, then the code of the first argument
% is called with the protected name appended. Examples
% are in the sources.
%
% \begin{declcs}{zref@wrapper@immediate}
%   |{|\dots|}|
% \end{declcs}
% There are situations where a label must be written instantly
% to the \xfile{.aux} file, for example after the last page.
% If the \cs{zlabel} or \cs{label} command is put inside this wrapper,
% immediate writing is enabled. See the implementation for
% module \xmodule{lastpage} for an example of its use.
%
% \begin{declcs}{zref@wrapper@unexpanded}
%   |{|\dots|}|
% \end{declcs}
% Assuming someone wants to extract a value for property |bar|
% and store the result in a macro |\foo| without traces of
% the expanding macros and without expanding the value.
% This (theoretical?) problem can be solved by this wrapper:
% \begin{quote}
%\begin{verbatim}
%\zref@wrapper@unexpanded{%
%  \edef\foo{%
%    \zref@extract{someref}{bar}%
%  }%
%}
%\end{verbatim}
% \end{quote}
% The \cs{edef} forces the expansion of \cs{zref@extract},
% but the extraction of the value is prevented by
% the wrapper that uses \hologo{eTeX}'\ \cs{unexpanded} for this
% purpose. Supported macros are \cs{zref@extract}, \cs{zref@extractdefault}
% and since version 2010/04/22 v2.14 macro \cs{zref@getcurrent}.
%
% \subsection{Counter for unique names}
%
% Some modules (\xmodule{titleref} and \xmodule{dotfillmin})
% need unique names for automatically generated label names.
%
% \begin{declcs}{zref@require@unique}
% \end{declcs}
% This command creates the unique counter |zref@unique| if the
% counter does not already exist.
%
% \begin{declcs}{thezref@unique}
% \end{declcs}
% This command is used to generate unique label names.
%
% \section{User interface}\label{sec:user}
%
% \subsection{Module \xmodule{user}}
%
% The user interface for this package and its modules
% is enabled by \xpackage{zref}'s package option \xmodule{user}
% or package \xpackage{zref-user}.
% The names of user commands are prefixed by |z| in order
% to avoid name clashes with existing macros of the same
% functionality. Thus the package does not disturb the
% traditional reference scheme, both can be used together.
%
% The syntax descriptions contain the following markers
% that are intended as hints for programmers:
% \begin{quote}
% \begin{tabular}{@{}ll@{}}
%   \markbabel\unskip  & Babel shorthands are allowed.\\
%   \markrobust\unskip & Robust macro.\\
%   \markexp\unskip    & Expandable version:\\
%                      & \textbullet\ robust,
%                        unless the extracted values are fragile,\\
%                      & \textbullet\ no babel shorthand suport.\\
%   \markexptwo\unskip & Expandable like \markexp and:\\
%                      & \textbullet\ expandable in exact two steps.\\
% \end{tabular}
% \end{quote}
%
% The basic user interface of the package without modules
% are commands that mimic the standard \hologo{LaTeX} behaviour
% of \cs{label}, \cs{ref}, and \cs{pageref}:
%
% \begin{declcs}{zlabel}
%   \M{refname}\markbabel
% \end{declcs}
% Similar to \cs{label}. It generates a label with name
% \meta{refname} in the new reference scheme.
%
% \begin{declcs}{zref}
%   |[|\meta{propname}|]| \M{refname}\markbabel
% \end{declcs}
% Without optional argument similar to \cs{ref}, it returns
% the default reference property. This property is named
% |default|:
% \[
%   |\zref{|x|}| \equiv |\zref[default]{|x|}|
% \]
%
% \begin{declcs}{zpageref}
%   \M{refname}\markbabel
% \end{declcs}
% Convenience macro, similar to \cs{pageref}.
% \[
%   |\zpageref{|x|}| \equiv |\zref[page]{|x|}|
% \]
%
% \begin{declcs}{zrefused}
%   \M{refname}\markbabel
% \end{declcs}
% Some of the user commands in the modules are expandable.
% The use of such commands do not cause any undefined
% reference warnings, because inside of expandable
% contexts this is not possible. However, if there is
% a place outside of expandable contexts, \cs{refused}
% is strongly recommended. The reference \meta{refname}
% is marked as used, undefined ones will generate
% warnings.
%
% \subsection{Module \xmodule{abspage}}
%
% With the help of package \xpackage{atbegshi}
% a new counter |abspage| with absolute page numbers is provided.
% For technical and historical reasons the counter itself is zero based:
% if you use it directly on the first page, e.g
% with |\arabic{abspage}| you will get 0 as value. When using |\zref|
% the first page will be page 1 as expected.
% Also a new property |abspage| is defined and added to the
% main property list. Thus you can reference the absolute
% page number:
% \begin{quote}
%   |Section \zref{foo} is on page \zpageref{foo}.|\\
%   |This is page \zref[abspage]{foo}|\\
%   |of \zref[abspage]{LastPage}.|
% \end{quote}
% The example also makes use of module \xmodule{lastpage}.
%
% \subsection{Module \xmodule{lastpage}}
%
% Provides the functionality of package \xpackage{lastpage}
% \cite{lastpage} in the new reference scheme.
% The label |LastPage| is put at the end of the document.
% You can refer the last page number with:
% \begin{quote}
%   |\zref@extract{LastPage}{page}| (+ |\zref@refused{LastPage}|)\\
% \end{quote}
% or
% \begin{quote}
%   |\zpageref{LastPage}| (module \xmodule{user})
% \end{quote}
%
% Since version 2008/10/01 v2.3 the module defines the list |LastPage|.
% In addition to the properties of the main list label |LastPage| also
% stores the properties of this list |LastPage|. The default of this
% list is empty. The list can be used by the user to add additional
% properties for label |LastPage|.
%
% \subsubsection{Tests for last page}
%
% Since version 2010/03/26 v2.8 the macros \cs{zref@iflastpage}
% and \cs{ziflastpage} were added. They test the reference,
% whether it is a reference of the last page.
%
% \begin{declcs}{zref@iflastpage}\markexp \M{refname} \M{then} \M{else}
% \end{declcs}
% Macro \cs{zref@iflastpage} compares the references \meta{refname}
% with \meta{LastPage}. Basis of the comparison is the value of
% property |abspage|, because the values are different for different
% pages. This is not ensured by property |page|. Therefore module
% \xmodule{abspage} is loaded by module \xmodule{lastpage}.
% If both values of property |abspage| are present and match,
% then \meta{then} is executed, otherwise code \meta{else} is called.
% If one or both references are undefined or lack the property |abspage|,
% then \meta{else} is executed.
%
% Macro \cs{zref@iflastpage} is expandable, therefore \cs{zref@refused}
% should be called on \meta{refname} and \meta{LastPage}.
%
% \begin{declcs}{ziflastpage} \M{refname} \M{then} \M{else}
% \end{declcs}
% Macro \cs{ziflastpage} has the same function as \cs{zref@iflastpage},
% but adds support for babel shorthands in \meta{refname} and
% calls \cs{zref@refused}. However macro \cs{ziflastpage} is not
% expandable.
%
% \subsubsection{Example}
%
%    \begin{macrocode}
%<*example-lastpage>
%<<END_EXAMPLE
\NeedsTeXFormat{LaTeX2e}
\documentclass{report}

\newcounter{foo}
\renewcommand*{\thefoo}{\Alph{foo}}

\usepackage{zref-lastpage,zref-user}[2019/11/29]

\makeatletter
\zref@newprop{thefoo}{\thefoo}
\zref@newprop{valuefoo}{\the\value{foo}}
\zref@newprop{chapter}{\thechapter}
\zref@addprops{LastPage}{thefoo,valuefoo,chapter}
\makeatother

\newcommand*{\foo}{%
  \stepcounter{foo}%
  [Current foo: \thefoo]%
}

\begin{document}
  \chapter{First chapter}
  Last page is \zref{LastPage}.\\
  Last chapter is \zref[chapter]{LastPage}.\\
  Last foo is \zref[thefoo]{LastPage}.\\
  Last value of foo is \zref[valuefoo]{LastPage}.\\
  \foo
  \chapter{Second chapter}
  \foo\foo\foo
  \chapter{Last chapter}
  \foo
\end{document}
%END_EXAMPLE
%</example-lastpage>
%    \end{macrocode}
%
% \subsection{Module \xmodule{thepage}}
%
% This module \xmodule{thepage} loads module \xmodule{abspage},
% constructs a reference name using the absolute page number
% and remembers property |page|. Other properties can be added
% by adding them to the property list |thepage|.
%
% \begin{declcs}{zthepage} \M{absolute page number}\\
% \end{declcs}
% Macro \cs{zthepage} is basically a \cs{zpageref}. The reference
% name is yield by the \meta{absolute page number}. If the reference
% is not defined, then the default for property |page| is used.
%
% \begin{declcs}{zref@thepage@name}\markexp \M{absolute page number}
% \end{declcs}
% Macro \cs{zref@thepage@name} returns the internal reference
% name that is constructed using the \meta{absolute page number}.
% The internal reference name should not be used directly, because
% it might change in future versions.
%
% \begin{declcs}{zref@thepage}\markexp \M{absolute page number}\\
%   \cs{zref@thepage@refused} \M{absolute page number}
% \end{declcs}
% Macro \cs{zref@thepage} returns the page number (\cs{thepage})
% of \meta{absolute page number}. Because this macro is expandable,
% \cs{zref@thepage@refused} is used outside an expandable context
% to mark the reference as used.
%
%
% \subsection{Module \xmodule{nextpage}}
%
% \begin{declcs}{znextpage}
% \end{declcs}
% Macro \cs{znextpage} prints \cs{thepage} of the following page.
% It gets the current absolute page number by using a label.
% There are three cases for the next page:
% \begin{enumerate}
% \item The next page is not known yet because of undefined references.
%   Then \cs{zunknownnextpagename} is used instead. The default for
%   this macro is the default of property |page|.
% \item This page is the last page. Then \cs{znonextpagename}
%   is used. Its default is empty.
% \item The next page is known, then \cs{thepage} of the next
%   page is used (the value of property |page| of the next page).
% \end{enumerate}
%
% \subsubsection{Configuration}
%
% The behaviour can be configured by the following macros.
%
% \begin{declcs}{zunknownnextpagename}\\
%   \cs{znonextpagename}
% \end{declcs}
% If the next page is not known or available, then
% \cs{znextpage} uses these name macros as default.
% \cs{zunknownnextpagename} is used in case of undefined
% references. Default is the value of property |page| of the
% next page (\cs{thepage}). Module \xmodule{thepage} is used.
%
% Macro \cs{znonextpagename} is used, if the next page does
% not exists. That means that the current page is last page.
% The default is empty.
%
% \begin{declcs}{znextpagesetup} \M{unknown} \M{no next}
%   \M{next}
% \end{declcs}
% Acording to the case (see \cs{znextpage}) macro \cs{znextpage}
% calls an internal macro with an argument. The argument is
% either \cs{thepage} of the next page or one of \cs{zunknownnextpagename}
% or \cs{znonextpagename}. These internal macro can be changed by
% \cs{znextpagesetup}. It expects the definition texts for these
% three cases of a macro with one argument.
% The default is
% \begin{quote}
%   |\znextpagesetup{#1}{#1}{#1}|
% \end{quote}
%
% \subsubsection{Example}
%
%    \begin{macrocode}
%<*example-nextpage>
%<<END_EXAMPLE
\documentclass{book}

\usepackage{zref-nextpage}[2019/11/29]
\znextpagesetup
  {\thepage}% next page is unknown
  {\thepage\ (#1)}% this page is last page
  {\thepage\ $\rightarrow$ #1}% next page is known
\renewcommand*{\znonextpagename}{last page}

\usepackage{fancyhdr}
\pagestyle{fancy}
\fancyhf{}
\fancyhead[LE,RO]{\znextpage}
\fancypagestyle{plain}{%
  \fancyhf{}%
  \fancyhead[LE,RO]{\znextpage}%
}

\begin{document}
\frontmatter
  \tableofcontents
\mainmatter
  \chapter{Hello World}
  \clearpage
  \section{Last section}
\end{document}
%END_EXAMPLE
%</example-nextpage>
%    \end{macrocode}
%
%
% \subsection{Module \xmodule{totpages}}
%
% For the total number of pages of a document you need to
% know the absolute page number of the last page.
% Both modules \xmodule{abspage} and \xmodule{lastpage}
% are necessary and automatically enabled.
%
% \begin{declcs}{ztotpages}\markexp
% \end{declcs}
% Prints the total number of pages or |0| if this
% number is not yet known. It expands to an explicit number
% and can also used even in expandable calculations (\cs{numexpr})
% or counter assignments.
%
% \subsection{Module \xmodule{pagelayout}}
%
% The module defines additional properties for each parameter
% of the page layout that is effective during page shipout.
% The value of length parameters is given in sp without the
% unit as plain number.
%
% Some parameters are specific for a class (e.g.\@ \xoption{stockwidth}
% and \xoption{stockheight} for class \xclass{memoir}) or the
% \hologo{TeX} engine like \hologo{pdfTeX}. If the parameter is not
% available, then the property will not be defined. The default value
% of the property is the current setting of the parameter.
%
% The module \xmodule{thepage} is loaded that generates a label
% for each page. The properties of module \xmodule{pagelayout}
% are added to the property list |thepage| of module \xmodule{thepage}.
%
% List of properties:
% \begin{quote}
%   \def\x#1{\cs{#1}&#1}%
%   \begin{tabular}{l>{\sffamily}ll}
%     \textrm{parameter} & \textrm{property} & remarks\\
%     \hline
%     \x{mag}\\
%     \x{paperwidth}\\
%     \x{paperheight}\\
%     \x{stockwidth}& class \xclass{memoir}\\
%     \x{stockheight}& class \xclass{memoir}\\
%     \x{pdfpagewidth}& \hologo{pdfTeX}, \hologo{LuaTeX}\\
%     \x{pdfpageheight}& \hologo{pdfTeX}, \hologo{LuaTeX}\\
%     \x{pdfhorigin}& \hologo{pdfTeX}, \hologo{LuaTeX}\\
%     \x{pdfvorigin}& \hologo{pdfTeX}, \hologo{LuaTeX}\\
%     \x{hoffset}\\
%     \x{voffset}\\
%     \x{topmargin}\\
%     \x{oddsidemargin}\\
%     \x{evensidemargin}\\
%     \x{textwidth}\\
%     \x{textheight}\\
%     \x{headheight}\\
%     \x{headsep}\\
%     \x{footskip}\\
%     \x{marginparwidth}\\
%     \x{marginparsep}\\
%     \x{columnwidth}\\
%     \x{columnsep}\\
%   \end{tabular}
% \end{quote}
%
% \begin{declcs}{zlistpagelayout}
% \end{declcs}
% At the end of document the page layout parameter for each page are
% printed into the \xext{log} file if macro \cs{zlistpagelayout}
% is called before |\end{document}| (preamble is a good place).
%
% \subsection{Module \xmodule{marks}}
%
% ToDo.
%
% \subsection{Module \xmodule{runs}}
%
% Module \xmodule{runs} counts the \hologo{LaTeX} runs since last
% \xext{aux} file creation and prints the number in the
% \xext{log} file.
%
% \begin{declcs}{zruns}\markexp
% \end{declcs}
% Prints the the total number of \hologo{LaTeX} runs including
% the current one. It expands to an explicit number.
% Before |begin{document}| the value is zero meaning
% the \xext{aux} file is not read yet. If a previous
% \xext{aux} file exists, the value found there increased
% by one is the new number. Otherwise \cs{zruns} is set to one.
% \hologo{LaTeX}~runs where the \xext{aux} files are not rewritten
% are not counted (see \cs{nofiles}).
%
% \subsection{Module \xmodule{perpage}}
%
% With \cs{@addtoreset} or \cs{numberwithin} a counter can be
% reset if another counter is incremented. This do not work
% well if the other counter is the page counter. The page counter is
% incremented in the output routine that is often called asynchronous
% somewhere on the next page. A reference mechanism costs at least
% two \hologo{LaTeX} runs, but ensures correct page counter values.
%
% \begin{declcs}{zmakeperpage}
%   |[|\meta{reset}|]| \M{counter}
% \end{declcs}
% At the of a new page counter \meta{counter} starts
% counting with value \meta{reset} (default is \texttt{1}).
% The macro has the same
% syntax and semantics as \cs{MakePerPage} of package
% \xpackage{perpage} \cite{perpage}.
% Also \xoption{perpage} of package \xpackage{footmisc} \cite{footmisc}
% can easily be simulated by
% \begin{quote}
% |\zmakeperpage{footnote}| \itshape |% \usepackage[perpage]{footmisc}|
%\end{quote}
% If footnote symbols are used, some people dislike the
% first symbol \ensuremath{\dagger}. It can easily be skipped:
% \begin{quote}
% |\zmakeperpage[2]{footnote}|
% \end{quote}
%
% \begin{declcs}{thezpage}\\
%   counter |zpage|
% \end{declcs}
% If the formatted counter value of the counter that is reset at
% a new page contains the page value, then you can use \cs{thezpage},
% the page number of the current page. Or counter |zpage| can be
% used, if the page number should be formatted differently from
% the current page number. Example:
%\begin{quote}
%\begin{verbatim}
%\newcounter{foobar}
%\zmakeperpage{foobar}
%\renewcommand*{\thefoobar}{\thezpage-\arabic{foobar}}
% % or
%\renewcommand*{\thefoobar}{\roman{zpage}-\arabic{foobar}}
%\end{verbatim}
%\end{quote}
%
% \begin{declcs}{zunmakeperpage}
%   \M{counter}
% \end{declcs}
% The reset mechanism for this counter is deactivated.
%
%
% \subsection{Module \xmodule{counter}}
%
% This option just add the property |counter| to the main
% property list. The property stores the counter name,
% that was responsible for the reference. This is the
% property \xpackage{hyperref}'s \cs{autoref} feature
% uses. Thus this property |counter| may be useful
% for a reimplementation of the autoref feature,
% see the section \ref{sec:todo} with the todo list.
%
% \subsection{Module \xmodule{titleref}}
%
% This option makes section and caption titles available
% to the reference system similar to packages \xpackage{titleref}
% or \xpackage{nameref}.
%
% \begin{declcs}{ztitleref}
%   \M{refname}\markbabel
% \end{declcs}
% Print the section or caption title of reference \meta{refname},
% similar to \cs{nameref} or \cs{titleref}.
%
% \begin{declcs}{ztitlerefsetup}
%   |{|\textit{key$_1$=value$_1$, key$_2$=value$_2$, \dots}|}|
% \end{declcs}
% This command allows to configure the behaviour of module
% \xmodule{titleref}. The following keys are available:
% \begin{description}
% \let\bfseries\relax
% \item[\texttt{title=}\meta{value}]\mbox{}\\*
%   Sets the current title.
% \item[\texttt{stripperiod=true\string|false}]\mbox{}\\*
%   Follow package \xpackage{nameref} that
%   removes a last period. Default: |true|.
% \item[\texttt{expand=true\string|false}]\mbox{}\\*
%   Package \cs{titleref} expands the title first. This way
%   garbage and dangerous commands can be removed, e.g.
%   \cs{label}, \cs{index}\dots. See implementation section
%   for more details. Default is |false|.
% \item[\texttt{cleanup=\{}\textrm{\dots}\texttt{\}}]\mbox{}\\*
%   Hook to add own cleanup code, if method |expand| is used.
%   See implementation section for more details.
% \end{description}
%
% \subsection{Module \xmodule{savepos}}
%
% This option supports a feature that \pdfTeX\ provides
% (and \hologo{XeTeX}).
% \pdfTeX\ is able to tell the current position on the
% page. The page position is not instantly known. First the
% page must be constructed by \hologo{TeX}'s asynchronous output
% routine. Thus the time where the position is known is
% the page shipout time. Thus a reference system where the
% information is recorded in the first run and made available
% for use in the second run comes in handy.
%
% \begin{declcs}{zsavepos}
%   \M{refname}
% \end{declcs}
% It generates a reference with name \meta{refname}.
% The reference stores the location where \cs{zsavepos}
% is executed in properties |posx| and |posy|.
%
% \begin{declcs}{zsaveposx}
%   \M{refname}\\
%   \cs{zsaveposy}
%   \M{refname}
% \end{declcs}
% Same as \cs{zsavepos} except that only the |x| or |y| component
% of the position is stored. Since 2011/12/05 v2.23.
%
% \begin{declcs}{zposx}\markexp
%   \M{refname}\\
%   \SpecialUsageIndex{\zposy}\cs{zposy}\markexp \M{refname}
% \end{declcs}
% Get the position as number. Unit is sp. Horizontal
% positions by \cs{zposx} increase from left to right.
% Vertical positions by \cs{zposy} from bottom to top.
%
% Do not rely on absolute page numbers. Because of problems
% with the origin the numbers may differ in DVI or PDF mode
% of \pdfTeX. Therefore work with relative values by comparisons.
%
% Both \cs{zposx} and \cs{zposy} are expandable and can be
% used inside calculations (\cs{setcounter}, \cs{addtocounter},
% package \xpackage{calc}, \cs{numexpr}). However this property
% prevents from notifying \hologo{LaTeX} that the reference is actually
% used (the notifying is not expandable). Therefore you should
% mark the reference as used by \cs{zrefused}.
%
% This module uses \pdfTeX's \cs{pdfsavepos}, \cs{pdflastxpos},
% and \cs{pdflastypos}. They are available in PDF mode and
% since version 1.40.0 also in DVI mode.
%
% \begin{declcs}{zref@savepos}
% \end{declcs}
% Macro \cs{zref@savepos} performs the first part of \cs{zsavepos}
% by calling \cs{pdfsavepos} (if \xext{aux} files are writable).
%
% Thus \cs{zsavepos} is basically \cs{zref@savepos} followed
% by |\zref@labelbylist{|\texttt{\meta{refname}}|}{savepos}|.
% If \cs{TeXXeTstate} is detected and enabled, \cs{savepos}
% also adds \cs{zref@savepos} at the end to support \cs{beginR}
% where the whatits are processed in reverse order.
% The property list |savepos| contains the properties |posx|
% and |posy|.
%
% \subsection{Module \xmodule{abspos}}
% Module \xmodule{abspos} allows to get various values of the page layout. There is no user
% command, only a number of internal commands. For example:
% \begin{quote}
%   |\zref@absposx{|\meta{label}|}{|\meta{value}|}{|\meta{position}|}|\\
%   |\zref@absposy{|\meta{label}|}{|\meta{value}|}{|\meta{position}|}|
% \end{quote}
%
% The return value is like in the module \xmodule{savepos} a number representing a length in sp.
% The length are measured from the bottom left of the page.
%
% \meta{label} is a label set with \cs{zlabel} or \cs{zsavepos} that allows to retrieve the
% absolute page number.
%
% \meta{position} is for the x-command one of |left|, |right| or |center|. For the y-command
%  it is one of |top|, |bottom|, |center|.
%
% The possible content of \meta{value} can be seen in the following table. Be aware that the code
% makes some assumptions which are perhaps not always true -- for example that the left of
% the head is identical to the left of the body.
%
% \medskip
% \begin{tabular}{lll}
%  \textbf{value} & \textbf{axis} & \textbf{comments}\\\hline
%   media         & x             &   left=0, right=\cs{pdfpagewidth} \\
%   paper         & x             &   left=0, right=\cs{paperwidth}\\
%   stock         & x             &   derived from paper\\
%   media         & y             & bottom=0, top=\cs{pdfpageheigh}\\
%   paper         & y             & top=\cs{pdfpageheight}, bottom=top-\cs{paperheight}\\
%   stock         & y             & top  derived from paper\\
%   head          & x             & calculated with hoffset, horigin, etc\\
%   head          & y             & calculated\\
%   body          & x             & = head value\\%
%   body          & y             & = head bottom - \cs{headsep}\\
%   foot          & x             & = head\\
%   foot          & y             & calculated from body bottom and \cs{footskip}\\
%   marginpar     & x             &different on odd/even pages! \\
%   marginpar     & y             & = body %
% \end{tabular}
%
% \subsection{Module \xmodule{dotfill}}
%
% \begin{declcs}{zdotfill}
% \end{declcs}
% This package provides the command \cs{zdotfill} that works
% similar to \cs{dotfill}, but can be configured. Especially
% it suppresses the dots if a minimum number of dots cannot be set.
%
% \begin{declcs}{zdotfillsetup}
%   |{|\textit{key$_1$=value$_1$, key$_2$=value$_2$, \dots}|}|
% \end{declcs}
% This command allows to configure the behaviour of \cs{zdotfill}.
% The following keys are available:
% \begin{description}
% \let\bfseries\relax
% \item[\texttt{min=}\meta{count value}]\mbox{}\\*
%   If the actual number of dots are smaller than \meta{count value},
%   then the dots are suppressed. Default: |2|.
% \item[\texttt{unit=}\meta{dimen value}]\mbox{}\\*
%   The width of a dot unit is given by \meta{dimen value}.
%   Default: |0.44em| (same as the unit in \cs{dotfill}).
% \item[\texttt{dot=}\meta{value}]\mbox{}\\*
%   The dot itself is given by \meta{value}. Default: |.| (dot, same as
%   the dot in \cs{dotfill}).
% \end{description}
%
% \subsection{Module \xmodule{env}}
%
% This module defines two properties |envname| and |envline|. They
% remember the name of the environment and the line number at
% the start of the environment.
%
% \subsection{Module \xmodule{xr}}
%
% This package provides the functionality of package \xpackage{xr},
% see \cite{xr}. It also supports the syntax of \xpackage{xr-hyper}.
%
% \begin{declcs}{zexternaldocument}
%   \* |[|\meta{prefix}|]|\markbabel \M{external document}
%   |[|\meta{url}|]|
% \end{declcs}
% See \cs{externaldocument} for a description of this option.
% The found labels also get a property |externaldocument| that
% remembers \meta{external document}.
% The standard reference scheme and the scheme of this
% package use different name spaces for reference names.
% If the external document uses both systems. Then one
% import statement would put the names in one namespace
% and probably causing problems with multiple references
% of the same name. Thus the star form only looks for
% \cs{newlabel} in the \xfile{.aux} files, whereas without
% star only \cs{zref@newlabel}s are used.
%
% In the star form it tries to detect labels from
% \xpackage{hyperref}, \xpackage{titleref}, and \xpackage{ntheorem}.
% If such an extended property from the packages before cannot
% be found or are empty, they are not included in the
% imported reference.
%
% Warnings are given if a reference name is already in use
% and the item is ignored. Unknown properties will automatically
% be declared.
%
% If the external references contain |anchor| properties,
% then we need also a url to be able to address the external
% file. As default the filename is taken with a default
% extension.
%
% \begin{declcs}{zxrsetup}
%   |{|\textit{key$_1$=value$_1$, key$_2$=value$_2$, \dots}|}|
% \end{declcs}
% The following setup options are available:
% \begin{description}
% \item[\xoption{ext}:] It sets the default extension.
% \item[\xoption{tozreflabel}:] Boolean option. The found references
%   are imported as zref labels. This is enabled by default.
% \item[\xoption{toltxlabel}:] Boolean option. The found references
%   are imported as \hologo{LaTeX} labels. Packages \xpackage{nameref},
%   \xpackage{titleref} and class \xclass{memoir} are supported.
% \item[\xoption{urluse}:] Boolean option. If enabled, then a URL
%   is stored in a macro and the macro is put in property `urluse'.
%   The URL is not put in property `url'.
%   The purpose is to save \hologo{TeX} memory.
% \item[\xoption{verbose}:] Boolean option. List the imported labels in the
%   \xext{log} file. Default is |false|.
% \end{description}
%
% \begin{declcs}{zref@xr@ext}
% \end{declcs}
% If the \meta{url} is not specified in \cs{zref@externaldocument},
% then the url will be constructed with the file name and
% this macro as extension. \cs{XR@ext} is used if
% \xpackage{hyperref} is loaded, otherwise |pdf|.
%
% \subsection{Module \xmodule{pageattr}}
%
% This module allows to recover the content of the register
% \cs{pdfpageattr} and \cs{pdfpagesattr} in pdftex and the equivalent register
% in luatex. There is no user command. Programmer commands are
%
% \begin{quote}
% |\zref@pdfpageattr{|\meta{absolute page number}|}|
%
% |\zref@pdfpagesattr{|\meta{absolute page number}|}|
% \end{quote}
% \section{ToDo}\label{sec:todo}
%
% Among other things the following issues are left for future work:
% \begin{itemize}
% \item Other applications: autoref, hyperref, \dots
% \end{itemize}
%
% \section{Example}
%
%    \begin{macrocode}
%<*example>
%    \end{macrocode}
%
%    \begin{macrocode}
\documentclass{book}

\usepackage[ngerman]{babel}%

\usepackage[savepos,totpages,titleref,dotfill,counter,user]{zref}

%    \end{macrocode}
%
%    Chapters are wrapped inside \cs{ChapterStart} and
%    \cs{ChapterStop}. The first argument \verb|#1| of \cs{ChapterStart}
%    is used to form a label id \verb|chap:#1|. At the end of
%    the chapter another label is set by \cs{zref@wrapper@immediate}, because
%    otherwise at the end of document a deferred write would not
%    be written, because there is no page for shipout.
%
%    Also this example shows how chapter titles can be recorded.
%    A new property \verb|chaptitle| is declared and added to the
%    main property list. In \cs{ChapterStart} the current value
%    of the property is updated.
%    \begin{macrocode}
\makeatletter
\zref@newprop{chaptitle}{}
\zref@addprop{main}{chaptitle}

\newcommand*{\ChapterStart}[2]{%
  \cleardoublepage
  \def\current@chapid{#1}%
  \zref@setcurrent{chaptitle}{#2}%
  \chapter{#2}%
  \zlabel{chap:#1}%
}
\newcommand*{\ChapterStop}{%
  \cleardoublepage
  \zref@wrapper@immediate{%
    \zref@labelbyprops{chapend:\current@chapid}{abspage}%
  }%
}
%    \end{macrocode}
%    \cs{ChapterPages} calculates and returns the number of pages of the
%    referenced chapter.
%    \begin{macrocode}
\newcommand*{\ChapterPages}[1]{%
  \zrefused{chap:#1}%
  \zrefused{chapend:#1}%
  \number\numexpr
    \zref@extract{chapend:#1}{abspage}%
    -\zref@extract{chap:#1}{abspage}%
    +1\relax
}
\makeatother
%    \end{macrocode}
%    \begin{macrocode}
\begin{document}
%    \end{macrocode}
%    As exception we use \cs{makeatletter} here, because
%    this is just an example file that also should show some of
%    programmer's interface.
%    \begin{macrocode}
\makeatletter

\frontmatter
\zlabel{documentstart}

\begin{itemize}
\item
  The frontmatter part has
  \number\numexpr\zref@extract{chap:first}{abspage}-1\relax
  ~pages.
\item
  Chapter \zref{chap:first} has \ChapterPages{first} page(s).
\item
  Section \zref{hello} is on the
  \ifcase\numexpr
    \zref@extractdefault{hello}{page}{0}%
    -\zref@extractdefault{chap:first}{page}{0}%
    +1\relax
    ??\or first\or second\or third\or forth\fi
  ~page inside its chapter.
\item
  The document has
  \zref[abspage]{LastPage} pages.
  This number is \ifodd\ztotpages odd\else even\fi.
\item
  The last page is labeled with \zpageref{LastPage}.
\item
  The title of chapter \zref{chap:next} %
  is ``\zref[chaptitle]{chap:next}''.
\end{itemize}

\tableofcontents

\mainmatter
\ChapterStart{first}{First chapter}

%    \end{macrocode}
%    The user level commands should protect babel shorthands
%    where possible. On the other side, expandable extracting
%    macros are useful in calculations, see above the
%    examples with \cs{numexpr}.
%    \begin{macrocode}
\section{Test}
\zlabel{a"o}
Section \zref{a"o} on page
\zref@wrapper@babel\zref@extract{a"o}{page}.

Text.
\newpage

\section{Hello World}
\zlabel{hello}

\ChapterStop

\ChapterStart{next}{Next chapter with \emph{umlauts}: "a"o"u"s}

%    \end{macrocode}
%
%    Here an example follows that makes use of \pdfTeX's ``savepos''
%    feature. The position on the page is not known before the
%    page is constructed and shipped out. Therefore the position
%    ist stored in references and are available for calculations
%    in the next \hologo{LaTeX} compile run.
%    \begin{macrocode}
The width of the first column is
  \the\dimexpr \zposx{secondcol}sp - \zposx{firstcol}sp\relax,\\
the height difference of the two baselines is
  \the\dimexpr \zposy{firstcol}sp - \zposy{secondline}sp\relax:\\
\begin{tabular}{ll}
  \zsavepos{firstcol}Hello&\zsavepos{secondcol}World\\
  \zsavepos{secondline}Second line&foobar\\
\end{tabular}

%    \end{macrocode}
%    With \cs{zrefused} \hologo{LaTeX} is notified, if the references
%    are not yet available and \hologo{LaTeX} can generate the rerun
%    hint.
%    \begin{macrocode}
\zrefused{firstcol}
\zrefused{secondcol}
\zrefused{secondline}

\ChapterStop
%    \end{macrocode}
%    Test for module \cs{dotfill}.
%    \begin{macrocode}
\ChapterStart{dotfill}{Test for dotfill feature}
\newcommand*{\dftest}[1]{%
  #1&
  [\makebox[{#1}]{\dotfill}]&
  [\makebox[{#1}]{\zdotfill}]\\
}
\begin{tabular}{rll}
& [\verb|\dotfill|] & [\verb|\zdotfill|]\\
\dftest{0.43em}
\dftest{0.44em}
\dftest{0.45em}
\dftest{0.87em}
\dftest{0.88em}
\dftest{0.89em}
\dftest{1.31em}
\dftest{1.32em}
\dftest{1.33em}
\end{tabular}
\ChapterStop
\end{document}
%    \end{macrocode}
%
%    \begin{macrocode}
%</example>
%    \end{macrocode}
%
% \StopEventually{
% }
%
% \section{Implementation}
%
% \subsection{Package \xpackage{zref}}
%
% \subsubsection{Identification}
%
%    \begin{macrocode}
%<*package>
\NeedsTeXFormat{LaTeX2e}
\ProvidesPackage{zref}
  [2020-07-03 v2.32 A new reference scheme for LaTeX (HO)]%
%    \end{macrocode}
%
% \subsubsection{Load basic module}
%
%    \begin{macrocode}
\RequirePackage{zref-base}[2019/11/29]
%    \end{macrocode}
%    Abort package loading if \xpackage{zref-base} could not be
%    loaded successfully.
%    \begin{macrocode}
\@ifundefined{ZREF@base@ok}{\endinput}{}
%    \end{macrocode}
%
% \subsubsection{Process options}
%
%    Known modules are loaded and the release date is checked.
%    \begin{macrocode}
\def\ZREF@temp#1{%
  \DeclareOption{#1}{%
    \AtEndOfPackage{%
      \RequirePackage{zref-#1}[2019/11/29]%
    }%
  }%
}
\ZREF@temp{abspage}
\ZREF@temp{counter}
\ZREF@temp{dotfill}
\ZREF@temp{hyperref}
\ZREF@temp{lastpage}
\ZREF@temp{marks}
\ZREF@temp{nextpage}
\ZREF@temp{pageattr}
\ZREF@temp{pagelayout}
\ZREF@temp{perpage}
\ZREF@temp{runs}
\ZREF@temp{savepos}
\ZREF@temp{thepage}
\ZREF@temp{titleref}
\ZREF@temp{totpages}
\ZREF@temp{user}
\ZREF@temp{xr}
%    \end{macrocode}
%    \begin{macrocode}
\ProcessOptions\relax
%</package>
%    \end{macrocode}
%
%
% \subsection{Module \xmodule{base}}
%
% \subsubsection{Prefixes}
%
%    This package uses the following prefixes for macro names:
%    \begin{description}
%    \item[\cs{zref@}:] Macros of the programmer's interface.
%    \item[\cs{ZREF@}:] Internal macros.
%    \item[\cs{Z@L@}\metacs{listname}:]
%      The properties of the list \meta{listname}.
%    \item[\cs{Z@D@}\metacs{propname}:]
%      The default value for property \meta{propname}.
%    \item[\cs{Z@E@}\metacs{propname}:]
%      Extract function for property \meta{propname}.
%    \item[\cs{Z@X@}\metacs{propname}:]
%      Information whether a property value for property
%      \meta{propname} is expanded
%      immediately or at shipout time.
%    \item[\cs{Z@C@}\metacs{propname}:]
%      Current value of the property \meta{propname}.
%    \item[\cs{Z@R@}\metacs{labelname}:]
%      Data for reference \meta{labelname}.
%    \item[\cs{ZREF@org@}:] Original versions of patched commands.
%    \item[\cs{z}:] For macros in user land, defined if
%      module \xmodule{user} is set.
%    \end{description}
%    The following family names are used for keys defined
%    according to the \xpackage{keyval} package:
%    \begin{description}
%    \item[\texttt{ZREF@TR}:] Setup for module \xmodule{titleref}.
%    \end{description}
%
% \subsubsection{Identification}
%
%    \begin{macrocode}
%<*base>
\NeedsTeXFormat{LaTeX2e}
\ProvidesPackage{zref-base}%
  [2020-07-03 v2.32 Module base for zref (HO)]%
%    \end{macrocode}
%
% \subsubsection{Utilities}
%
%    \begin{macrocode}
\providecommand\IfFormatAtLeastTF{\@ifl@t@r\fmtversion}
\RequirePackage{ltxcmds}[2010/12/02]
\RequirePackage{infwarerr}[2010/04/08]
\RequirePackage{kvsetkeys}[2010/03/01]
\RequirePackage{kvdefinekeys}[2010/03/01]
\RequirePackage{pdftexcmds}[2010/04/01]
%    \end{macrocode}
%
%    \begin{macro}{\ZREF@name}
%    Several times the package name is used, thus we store it
%    in \cmd{\ZREF@name}.
%    \begin{macrocode}
\def\ZREF@name{zref}
%    \end{macrocode}
%    \end{macro}
%
%    \begin{macrocode}
\ltx@IfUndefined{protected}{%
  \RequirePackage{makerobust}[2006/03/18]%
%    \end{macrocode}
%    \begin{macro}{\ZREF@Robust}
%    \begin{macrocode}
  \def\ZREF@Robust#1#2{%
    \def\ZREF@temp{\MakeRobustcommand#2}%
    \afterassignment\ZREF@temp
    #1#2%
  }%
%    \end{macrocode}
%    \end{macro}
%    \begin{macrocode}
}{%
%    \end{macrocode}
%    \begin{macro}{\ZREF@Robust}
%    \begin{macrocode}
  \def\ZREF@Robust#1{%
    \protected#1%
  }%
%    \end{macrocode}
%    \end{macro}
%    \begin{macrocode}
}
%    \end{macrocode}
%    \begin{macro}{\ZREF@IfDefinable}
%    \begin{macrocode}
\def\ZREF@IfDefinable#1#2#3{%
  \@ifdefinable{#1}{%
    \ZREF@Robust{#2}#1#3%
  }%
}
%    \end{macrocode}
%    \end{macro}
%
%    \begin{macro}{\ZREF@UpdatePdfTeX}
%    \cs{ZREF@UpdatePdfTeX} is used as help message text in
%    error messages.
%    \begin{macrocode}
\def\ZREF@UpdatePdfTeX{Update pdfTeX.}
%    \end{macrocode}
%    \end{macro}
%
%    \begin{macro}{\ifZREF@found}
%    The following switch is usded in list processing.
%    \begin{macrocode}
\newif\ifZREF@found
%    \end{macrocode}
%    \end{macro}
%
%    \begin{macro}{\ZREF@patch}
%    Macro \cs{ZREF@patch} first checks the existence of
%    the command and safes it.
%    \begin{macrocode}
\def\ZREF@patch#1{%
  \ltx@IfUndefined{#1}{%
    \ltx@gobble
  }{%
    \expandafter\let\csname ZREF@org@#1\expandafter\endcsname
    \csname #1\endcsname
    \ltx@firstofone
  }%
}
%    \end{macrocode}
%    \end{macro}
%
% \subsubsection{Check for \hologo{eTeX}}
%
%    The use of \hologo{eTeX} should be standard nowadays for \hologo{LaTeX}.
%    We test for \hologo{eTeX} in order to use its features later.
%    \begin{macrocode}
\ltx@IfUndefined{eTeXversion}{%
  \PackageError\ZREF@name{%
    Missing support for eTeX; package is abandoned%
  }{%
    Use a TeX compiler that support eTeX and enable eTeX %
    in the format.%
  }%
  \endinput
}{}%
%    \end{macrocode}
%
%    \begin{macrocode}
\RequirePackage{etexcmds}[2007/09/09]
\ifetex@unexpanded
\else
  \PackageError\ZREF@name{%
    Missing e-TeX's \string\unexpanded.\MessageBreak
    Add \string\RequirePackage\string{etexcmds\string} before %
    \string\documentclass%
  }{%
    Probably you are using some package (e.g. ConTeXt) that %
    redefines \string\unexpanded%
  }%
  \expandafter\endinput
\fi
%    \end{macrocode}
%
% \subsubsection{Auxiliary file stuff}
%
%    We are using some commands in the \xfile{.aux} files.
%    However sometimes these auxiliary files are interpreted
%    by \hologo{LaTeX} processes that haven't loaded this package
%    (e.g. package \xpackage{xr}).
%    Therefore we provide dummy definitions.
%    \begin{macrocode}
\RequirePackage{auxhook}
\AddLineBeginAux{%
  \string\providecommand\string\zref@newlabel[2]{}%
}
%    \end{macrocode}
%    \begin{macro}{\ZREF@RefPrefix}
%    \begin{macrocode}
\def\ZREF@RefPrefix{Z@R}
%    \end{macrocode}
%    \end{macro}
%    \begin{macro}{\zref@newlabel}
%    For the implementation of \cs{zref@newlabel} we call
%    the same internal macro \cs{@newl@bel} that is used
%    in \cs{newlabel}. Thus we have for free:
%    \begin{itemize}
%    \item \cs{Z@R@}\metacs{labelname} is defined.
%    \item \hologo{LaTeX}'s check for multiple references.
%    \item \hologo{LaTeX}'s check for changed references.
%    \end{itemize}
%    \begin{macrocode}
\ZREF@Robust\edef\zref@newlabel{%
  \noexpand\@newl@bel{\ZREF@RefPrefix}%
}
%    \end{macrocode}
%    \end{macro}
%
% \subsubsection{Property lists}
%
%    \begin{macro}{\zref@newlist}
%    Property lists are stored as list of property names
%    enclosed in curly braces. \cs{zref@newlist}
%    creates a new list as empty list. Assignments to
%    property lists are global.
%    \begin{macrocode}
\ZREF@Robust\def\zref@newlist#1{%
  \zref@iflistundefined{#1}{%
    \@ifdefinable{Z@L@#1}{%
     \global\expandafter\let\csname Z@L@#1\endcsname\ltx@empty
     \PackageInfo\ZREF@name{New property list: #1}%
    }%
  }{%
    \PackageError\ZREF@name{%
      Property list `#1' already exists%
    }\@ehc
  }%
}
%    \end{macrocode}
%    \end{macro}
%
%    \begin{macro}{\zref@iflistundefined}
%    \cs{zref@iflistundefined} checks the existence of the
%    property list \verb|#1|. If the property list is present,
%    then \verb|#2| is executed and \verb|#3| otherwise.
%    \begin{macrocode}
\def\zref@iflistundefined#1{%
  \ltx@ifundefined{Z@L@#1}%
}
%    \end{macrocode}
%    \end{macro}
%    \begin{macro}{\zref@listexists}
%    \cs{zref@listexists} only executes \verb|#2| if the
%    property list \verb|#1| exists and raises an error
%    message otherwise.
%    \begin{macrocode}
\ZREF@Robust\def\zref@listexists#1{%
  \zref@iflistundefined{#1}{%
    \PackageError\ZREF@name{%
      Property list `#1' does not exist%
    }\@ehc
  }%
}
%    \end{macrocode}
%    \end{macro}
%
%    \begin{macro}{\zref@iflistcontainsprop}
%    \cs{zref@iflistcontainsprop} checks, whether a property \verb|#2| is
%    already present in a property list \verb|#1|.
%    \begin{macrocode}
\ZREF@Robust\def\zref@iflistcontainsprop#1#2{%
  \zref@iflistundefined{#1}{%
    \ltx@secondoftwo
  }{%
    \begingroup\expandafter\endgroup
    \expandafter\in@
    \csname#2\expandafter\expandafter\expandafter\endcsname
    \expandafter\expandafter\expandafter{\csname Z@L@#1\endcsname}%
    \csname ltx@\ifin@ first\else second\fi oftwo\endcsname
  }%
}
%    \end{macrocode}
%    \end{macro}
%
%    \begin{macro}{\zref@listforloop}
%    \begin{macrocode}
\def\zref@listforloop#1#2{%
  \zref@listexists{#1}{%
    \expandafter\expandafter\expandafter\@tfor
    \expandafter\expandafter\expandafter\zref@prop
    \expandafter\expandafter\expandafter:%
    \expandafter\expandafter\expandafter=%
    \csname Z@L@#1\endcsname
    \do{%
      \begingroup
        \escapechar=-1 %
        \edef\x{\endgroup
          \def\noexpand\zref@prop{%
            \expandafter\string\zref@prop
          }%
        }%
      \x
      #2\zref@prop
    }%
  }%
}
%    \end{macrocode}
%    \end{macro}
%
%    \begin{macro}{\zref@addprops}
%    \cs{zref@addprop} adds the properties \verb|#2| to the property
%    list \verb|#1|, if the property is not already in the list.
%    Otherwise a warning is given.
%    \begin{macrocode}
\ZREF@Robust\def\zref@addprops#1#2{%
  \zref@listexists{#1}{%
    \comma@parse{#2}{%
      \zref@propexists\comma@entry{%
        \zref@iflistcontainsprop{#1}\comma@entry{%
          \PackageWarning\ZREF@name{%
            Property `\comma@entry' is already in list `#1'%
          }%
        }{%
          \begingroup\expandafter\endgroup
          \expandafter\g@addto@macro
          \csname Z@L@#1\expandafter\endcsname
          \expandafter{\csname\comma@entry\endcsname}%
        }%
      }%
      \ltx@gobble
    }%
  }%
}
%    \end{macrocode}
%    \end{macro}
%    \begin{macro}{\zref@addprop}
%    \cs{zref@addprop} adds the property \verb|#2| to the property
%    list \verb|#1|, if the property is not already in the list.
%    Otherwise a warning is given.
%    \begin{macrocode}
\ZREF@Robust\def\zref@addprop#1#2{%
  \zref@listexists{#1}{%
    \zref@propexists{#2}{%
      \zref@iflistcontainsprop{#1}{#2}{%
        \PackageWarning\ZREF@name{%
          Property `#2' is already in list `#1'%
        }%
      }{%
        \begingroup\expandafter\endgroup
        \expandafter\g@addto@macro
        \csname Z@L@#1\expandafter\endcsname
        \expandafter{\csname#2\endcsname}%
      }%
    }%
  }%
}
%    \end{macrocode}
%    \end{macro}
%
%    \begin{macro}{\zref@localaddprops}
%    \begin{macrocode}
\ZREF@Robust\def\zref@localaddprops#1#2{%
  \zref@listexists{#1}{%
    \comma@parse{#2}{%
      \zref@propexists\comma@entry{%
        \zref@iflistcontainsprop{#1}\comma@entry{%
          \PackageWarning\ZREF@name{%
            Property `\comma@entry' is already in list `#1'%
          }%
        }{%
          \begingroup\expandafter\endgroup
          \expandafter\ltx@LocalAppendToMacro
          \csname Z@L@#1\expandafter\endcsname
          \expandafter{\csname\comma@entry\endcsname}%
        }%
      }%
      \ltx@gobble
    }%
  }%
}
%    \end{macrocode}
%    \end{macro}
%    \begin{macro}{\zref@localaddprop}
%    \begin{macrocode}
\ZREF@Robust\def\zref@localaddprop#1#2{%
  \zref@listexists{#1}{%
    \zref@propexists{#2}{%
      \zref@iflistcontainsprop{#1}{#2}{%
        \PackageWarning\ZREF@name{%
          Property `#2' is already in list `#1'%
        }%
      }{%
        \begingroup\expandafter\endgroup
        \expandafter\ltx@LocalAppendToMacro
        \csname Z@L@#1\expandafter\endcsname
        \expandafter{\csname#2\endcsname}%
      }%
    }%
  }%
}
%    \end{macrocode}
%    \end{macro}
%
%    \begin{macrocode}
\ltx@IfUndefined{pdf@strcmp}{%
%    \end{macrocode}
%    \begin{macro}{\zref@delprop}
%    \begin{macrocode}
  \ZREF@Robust\def\zref@delprop{%
    \ZREF@delprop\gdef
  }%
%    \end{macrocode}
%    \end{macro}
%    \begin{macro}{\zref@localdelprop}
%    \begin{macrocode}
  \ZREF@Robust\def\zref@localdelprop{%
    \ZREF@delprop\def
  }%
%    \end{macrocode}
%    \end{macro}
%    \begin{macro}{\ZREF@delprop}
%    \begin{macrocode}
  \def\ZREF@delprop#1#2#3{%
    \zref@listexists{#2}{%
      \begingroup
        \escapechar=-1 %
        \def\ZREF@param{#3}%
        \@onelevel@sanitize\ZREF@param
        \toks@{}%
        \expandafter\expandafter\expandafter\ZREF@@delprop
        \csname Z@L@#2\endcsname!%
      \expandafter\endgroup
      \expandafter#1\csname Z@L@#2\expandafter\endcsname
      \expandafter{%
        \the\toks@
      }%
    }%
  }%
%    \end{macrocode}
%    \end{macro}
%    \begin{macro}{\ZREF@@delprop}
%    \begin{macrocode}
  \def\ZREF@@delprop#1{%
    \expandafter\ZREF@@@delprop\expandafter{\string#1}#1%
  }%
%    \end{macrocode}
%    \end{macro}
%    \begin{macro}{\ZREF@@@delprop}
%    \begin{macrocode}
  \def\ZREF@@@delprop#1#2{%
    \ifx#2!%
    \else
      \def\ZREF@temp{#1}%
      \@onelevel@sanitize\ZREF@temp
      \ifx\ZREF@param\ZREF@temp
      \else
        \toks@\expandafter{%
          \the\expandafter\toks@\csname#1\endcsname
        }%
      \fi
      \expandafter\ZREF@@delprop
    \fi
  }%
%    \end{macrocode}
%    \end{macro}
%    \begin{macrocode}
}{%
%    \end{macrocode}
%    \begin{macro}{\zref@delprop}
%    \begin{macrocode}
  \ZREF@Robust\def\zref@delprop{%
    \ZREF@delprop\xdef
  }%
%    \end{macrocode}
%    \end{macro}
%    \begin{macro}{\zref@localdelprop}
%    \begin{macrocode}
  \ZREF@Robust\def\zref@localdelprop{%
    \ZREF@delprop\edef
  }%
%    \end{macrocode}
%    \end{macro}
%    \begin{macro}{\ZREF@delprop}
%    \begin{macrocode}
  \def\ZREF@delprop#1#2#3{%
    \zref@listexists{#2}{%
      \def\ZREF@param{#3}%
      \edef\ZREF@SavedEscapechar{\the\escapechar}%
      \escapechar=-1 %
      \expandafter#1\csname Z@L@#2%
      \expandafter\expandafter\expandafter\endcsname{%
        \expandafter\expandafter\expandafter\ZREF@@delprop
        \csname Z@L@#2\endcsname!%
      }%
      \escapechar=\ZREF@SavedEscapechar\relax
    }%
  }%
%    \end{macrocode}
%    \end{macro}
%    \begin{macro}{\ZREF@@delprop}
%    Caution: |#1| might be an \cs{if} or similar token.
%    \begin{macrocode}
  \def\ZREF@@delprop#1{%
    \expandafter\ZREF@@@delprop\expandafter{\string#1}#1%
  }%
%    \end{macrocode}
%    \end{macro}
%    \begin{macro}{\ZREF@@@delprop}
%    \begin{macrocode}
  \def\ZREF@@@delprop#1#2{%
    \ifx#2!%
    \else
      \ifnum\pdf@strcmp{#1}{\ZREF@param}=\ltx@zero
      \else
        \expandafter\noexpand\csname#1\endcsname
      \fi
      \expandafter\ZREF@@delprop
    \fi
  }%
%    \end{macrocode}
%    \end{macro}
%    \begin{macrocode}
}
%    \end{macrocode}
%
% \subsubsection{Properties}
%
%    \begin{macro}{\zref@ifpropundefined}
%    \cs{zref@ifpropundefined} checks the existence of the
%    property \verb|#1|. If the property is present,
%    then \verb|#2| is executed and \verb|#3| otherwise.
%    \begin{macrocode}
\def\zref@ifpropundefined#1{%
  \ltx@ifundefined{Z@E@#1}%
}
%    \end{macrocode}
%    \end{macro}
%    \begin{macro}{\zref@propexists}
%    Some macros rely on the existence of a property.
%    \cs{zref@propexists} only executes \verb|#2| if the
%    property \verb|#1| exists and raises an error
%    message otherwise.
%    \begin{macrocode}
\ZREF@Robust\def\zref@propexists#1{%
  \zref@ifpropundefined{#1}{%
    \PackageError\ZREF@name{%
      Property `#1' does not exist%
    }\@ehc
  }%
}
%    \end{macrocode}
%    \end{macro}
%
%    \begin{macro}{\zref@newprop}
%    A new property is declared by \cs{zref@newprop}, the
%    property name \meta{propname} is given in \verb|#1|. The
%    property is created and configured. If the star form
%    is given, then the expansion of the property value
%    is delayed to page shipout time, when the reference
%    is written to the \xfile{.aux} file.
%    \begin{description}
%    \item[\cs{Z@D@}\metacs{propname}:]
%      Stores the default value for this property.
%    \item[\cs{Z@E@}\metacs{propname}:]
%      Extract function.
%    \item[\cs{Z@X@}\metacs{propname}:]
%      Information whether the expansion of the property value
%      is delayed to shipout time.
%    \item[\cs{Z@C@}\metacs{propname}:]
%      Current value of the property.
%    \end{description}
%    \begin{macrocode}
\ZREF@Robust\def\zref@newprop{%
  \@ifstar{%
    \let\ZREF@X\noexpand
    \ZREF@newprop
  }{%
    \let\ZREF@X\ltx@empty
    \ZREF@newprop
  }%
}
%    \end{macrocode}
%    \end{macro}
%    \begin{macro}{\ZREF@newprop}
%    \begin{macrocode}
\def\ZREF@newprop#1{%
  \edef\ZREF@P{#1}%
  \@onelevel@sanitize\ZREF@P
  \begingroup
  \ifx\ZREF@P\ZREF@par
    \@PackageError\ZREF@name{%
      Invalid property name `\ZREF@P'%
    }{%
      The property name `par' is not allowed %
      because of internal reasons.%
      \MessageBreak
      \@ehc
    }%
    \def\ZREF@@newprop[##1]##2{\endgroup}%
  \else
    \zref@ifpropundefined\ZREF@P{%
      \endgroup
      \PackageInfo\ZREF@name{%
        New property: \ZREF@P
      }%
    }{%
      \@PackageError\ZREF@name{%
        Property `\ZREF@P' already exists%
      }\@ehc
      \def\ZREF@@newprop[##1]##2{\endgroup}%
    }%
  \fi
  \@ifnextchar[\ZREF@@newprop{\ZREF@@newprop[\zref@default]}%
}
%    \end{macrocode}
%    \end{macro}
%    \begin{macro}{\ZREF@par}
%    \begin{macrocode}
\def\ZREF@par{par}
\@onelevel@sanitize\ZREF@par
%    \end{macrocode}
%    \end{macro}
%    \begin{macro}{\ZREF@@newprop}
%    \begin{macrocode}
\def\ZREF@@newprop[#1]{%
  \global\@namedef{Z@D@\ZREF@P}{#1}%
  \global\expandafter\let\csname Z@X@\ZREF@P\endcsname\ZREF@X
  \begingroup\expandafter\endgroup
  \expandafter\ZREF@@@newprop\csname\ZREF@P\endcsname
  \expandafter\gdef\csname Z@C@\ZREF@P\endcsname{}%
  \zref@setcurrent\ZREF@P
}
\def\ZREF@@@newprop#1{%
  \expandafter
  \gdef\csname Z@E@\ZREF@P\endcsname##1#1##2##3\ZREF@nil{##2}%
}
%    \end{macrocode}
%    \end{macro}
%
%    \begin{macro}{\zref@showprop}
%    \begin{macrocode}
\ZREF@Robust\def\zref@showprop#1{%
  \zref@ifpropundefined{#1}{%
    \@PackageInfoNoLine{\ZREF@name}{%
      Show property `#1': <undefined>%
    }%
  }{%
    \begingroup
      \toks@\expandafter\expandafter\expandafter{%
        \csname Z@C@#1\endcsname
      }%
      \edef\ZREF@value{\the\toks@}%
      \ltx@onelevel@sanitize\ZREF@value
      \toks@\expandafter\expandafter\expandafter{%
        \csname Z@D@#1\endcsname
      }%
      \edef\ZREF@default{\the\toks@}%
      \ltx@onelevel@sanitize\ZREF@default
      \@PackageInfoNoLine{\ZREF@name}{%
        Show property `#1':\MessageBreak
        \expandafter\ifx\csname Z@X@#1\endcsname\ltx@empty
          Immediate %
        \else
          Delayed %
        \fi
        value: [\ZREF@value]\MessageBreak
        Default: [\ZREF@default]%
      }%
    \endgroup
  }%
}
%    \end{macrocode}
%    \end{macro}
%
%    \begin{macro}{\zref@setcurrent}
%    \cs{zref@setcurrent} sets the current value for a property.
%    \begin{macrocode}
\ZREF@Robust\def\zref@setcurrent#1#2{%
  \zref@propexists{#1}{%
    \expandafter\def\csname Z@C@#1\endcsname{#2}%
  }%
}
%    \end{macrocode}
%    \end{macro}
%
%    \begin{macro}{\ZREF@getcurrent}
%    \cs{zref@getcurrent} gets the current value for a property.
%    \begin{macrocode}
\def\ZREF@getcurrent#1{%
  \romannumeral0%
  \ltx@ifundefined{Z@C@#1}{%
    \ltx@space
  }{%
    \expandafter\expandafter\expandafter\ltx@space
    \csname Z@C@#1\endcsname
  }%
}
%    \end{macrocode}
%    \end{macro}
%    \begin{macro}{\ZREF@u@getcurrent}
%    \begin{macrocode}
\def\ZREF@wu@getcurrent#1{%
  \etex@unexpanded\expandafter\expandafter\expandafter{%
    \ZREF@getcurrent{#1}%
  }%
}
%    \end{macrocode}
%    \end{macro}
%    \begin{macro}{\zref@getcurrent}
%    \begin{macrocode}
\let\zref@getcurrent\ZREF@getcurrent
%    \end{macrocode}
%    \end{macro}
%
% \subsubsection{Reference generation}
%
%    \begin{macro}{\zref@label}
%    Label macro that uses the main property list.
%    \begin{macrocode}
\ZREF@Robust\def\zref@label#1{%
  \zref@labelbylist{#1}\ZREF@mainlist
}
%    \end{macrocode}
%    \end{macro}
%    \begin{macro}{\zref@labelbylist}
%    Label macro that stores the properties, specified in the
%    property list \verb|#2|.
%    \begin{macrocode}
\ZREF@Robust\def\zref@labelbylist#1#2{%
  \@bsphack
    \zref@listexists{#2}{%
      \expandafter\expandafter\expandafter\ZREF@label
      \expandafter\expandafter\expandafter{%
        \csname Z@L@#2\endcsname
      }{#1}%
    }%
  \@esphack
}
%    \end{macrocode}
%    \end{macro}
%    \begin{macro}{\zref@labelbyprops}
%    The properties are directly specified in a comma separated list.
%    \begin{macrocode}
\ZREF@Robust\def\zref@labelbyprops#1#2{%
  \@bsphack
    \begingroup
      \toks@{}%
      \comma@parse{#2}{%
        \zref@ifpropundefined\comma@entry{%
          \PackageWarning\ZREF@name{%
            Property `\comma@entry' is not known%
          }%
        }{%
          \toks@\expandafter{%
            \the\expandafter\toks@\csname\comma@entry\endcsname
          }%
        }%
        \ltx@gobble
      }%
    \expandafter\endgroup
    \expandafter\ZREF@label\expandafter{\the\toks@}{#1}%
  \@esphack
}
%    \end{macrocode}
%    \end{macro}
%    \begin{macro}{\zref@labelbykv}
%    \begin{macrocode}
\ZREF@Robust\def\zref@labelbykv#1#2{%
  \@bsphack
    \begingroup
      \let\Z@L@ZREF@temp\ltx@empty
      \kvsetkeys{ZREF@LABEL}{#1}%
      \ifZREF@immediate
        \expandafter\zref@wrapper@immediate\expandafter{%
          \expandafter\ZREF@label\expandafter{\Z@L@ZREF@temp}{#2}%
        }%
      \else
        \expandafter\ZREF@label\expandafter{\Z@L@ZREF@temp}{#2}%
      \fi
    \endgroup
  \@esphack
}
%    \end{macrocode}
%    \end{macro}
%    \begin{macrocode}
\kv@define@key{ZREF@LABEL}{prop}{%
  \edef\ZREF@param{#1}%
  \zref@propexists\ZREF@param{%
    \zref@iflistcontainsprop{ZREF@temp}\ZREF@param{}{%
      \begingroup\expandafter\endgroup
      \expandafter\ltx@LocalAppendToMacro
      \expandafter\Z@L@ZREF@temp
      \expandafter{\csname\ZREF@param\endcsname}%
    }%
  }%
}
\kv@define@key{ZREF@LABEL}{list}{%
  \zref@listforloop{#1}{%
    \zref@iflistcontainsprop{ZREF@temp}\zref@prop{}{%
      \begingroup\expandafter\endgroup
      \expandafter\ltx@LocalAppendToMacro
      \expandafter\Z@L@ZREF@temp
      \expandafter{\csname\zref@prop\endcsname}%
    }%
    \ltx@gobble
  }%
}
\kv@define@key{ZREF@LABEL}{delprop}{%
  \zref@propexists{#1}{%
    \zref@localdelprop{ZREF@temp}{#1}%
  }%
}
\kv@define@key{ZREF@LABEL}{immediate}[true]{%
  \edef\ZREF@param{#1}%
  \ifx\ZREF@param\ZREF@true
    \ZREF@immediatetrue
  \else
    \ifx\ZREF@param\ZREF@false
      \ZREF@immediatefalse
    \else
      \PackageWarning\ZREF@name{%
        Option `immediate' expects `true' or `false'.\MessageBreak
        Ignoring invalid value `\ZREF@param'%
      }%
    \fi
  \fi
}
%    \end{macrocode}
%    \begin{macro}{\ZREF@false}
%    \begin{macrocode}
\def\ZREF@false{false}
%    \end{macrocode}
%    \end{macro}
%    \begin{macro}{\ZREF@true}
%    \begin{macrocode}
\def\ZREF@true{true}
%    \end{macrocode}
%    \end{macro}
%
%    \begin{macrocode}
\kv@define@key{ZREF@LABEL}{values}[]{%
  \kv@parse{#1}{%
    \ifx\kv@value\relax
      \@PackageWarning\ZREF@name{%
        Missing value for property `\kv@key'%
      }%
      \expandafter\ltx@gobbletwo
    \else
      \expandafter\zref@setcurrent
    \fi
  }%
}
%    \end{macrocode}
%
%    \begin{macro}{\ifZREF@immediate}
%    The switch \cs{ifZREF@immediate} tells us, whether the
%    label should be written immediately or at page shipout time.
%    \cs{ZREF@label} need to be notified about this, because
%    it must disable the deferred execution of property values,
%    if the label is written immediately.
%    \begin{macrocode}
\newif\ifZREF@immediate
%    \end{macrocode}
%    \end{macro}
%    \begin{macro}{\zref@wrapper@immediate}
%    The argument of \cs{zref@wrapper@immediate} is executed inside
%    a group where \cs{write} is redefined by adding
%    \cs{immediate} before its execution. Also \cs{ZREF@label}
%    is notified via the switch \cs{ifZREF@immediate}.
%    \begin{macrocode}
\ZREF@Robust{\long\def}\zref@wrapper@immediate#1{%
  \begingroup
    \ZREF@immediatetrue
    \let\ZREF@org@write\write
    \def\write{\immediate\ZREF@org@write}%
    #1%
  \endgroup
}
%    \end{macrocode}
%    \end{macro}
%
%    \begin{macro}{\ZREF@label}
%    \cs{ZREF@label} writes the data in the \xfile{.aux} file.
%    \verb|#1| contains the list of valid properties,
%    \verb|#2| the name of the reference. In case of immediate
%    writing, the deferred execution of property values is
%    disabled. Also \thepage is made expandable in this case.
%    \begin{macrocode}
\def\ZREF@label#1#2{%
  \if@filesw
    \begingroup
      \ifZREF@immediate
        \let\ZREF@org@thepage\thepage
      \fi
      \protected@write\@auxout{%
        \ifZREF@immediate
          \let\thepage\ZREF@org@thepage
        \fi
        \let\ZREF@temp\ltx@empty
        \@tfor\ZREF@P:=#1\do{%
          \begingroup
            \escapechar=-1 %
            \edef\x{\endgroup
              \def\noexpand\ZREF@P{%
                \expandafter\string\ZREF@P
              }%
            }%
          \x
          \expandafter\ifx
              \csname
                \ifZREF@immediate
                  relax%
                \else
                  Z@X@\ZREF@P%
                \fi
              \endcsname
              \noexpand
            \expandafter\let\csname Z@C@\ZREF@P\endcsname\relax
          \fi
          \toks@\expandafter{\ZREF@temp}%
          \edef\ZREF@temp{%
            \the\toks@
            \ltx@backslashchar\ZREF@P{%
              \expandafter\noexpand\csname Z@C@\ZREF@P\endcsname
            }%
          }%
        }%
      }{%
        \string\zref@newlabel{#2}{\ZREF@temp}%
      }%
    \endgroup
  \fi
}
\def\ZREF@addtoks#1{%
  \toks@\expandafter\expandafter\expandafter{%
    \expandafter\the\expandafter\toks@#1%
  }%
}
%    \end{macrocode}
%    \end{macro}
%
% \subsubsection{Reference querying and extracting}
%
%    Design goal for the extracting macros is that the extraction
%    process is full expandable. Thus these macros can be used
%    in expandable contexts. But there are problems that cannot
%    be solved by full expandable macros:
%    \begin{itemize}
%    \item In standard \hologo{LaTeX} undefined references sets a flag
%      and generate a warning. Both actions are not expandable.
%    \item Babel's support for its shorthand uses commands that
%      use non-expandable assignments. However currently there
%      is hope, that primitives are added to \pdfTeX\ that
%      allows the detection of contexts. Then the shorthand
%      can detect, if they are executed inside \cs{csname}
%      and protect themselves automatically.
%    \end{itemize}
%    \begin{macro}{\zref@ifrefundefined}
%    If a reference \verb|#1| is undefined, then macro
%    \cs{zref@ifrefundefined} calls \verb|#2| and \verb|#3| otherwise.
%    \begin{macrocode}
\def\zref@ifrefundefined#1{%
  \ltx@ifundefined{Z@R@#1}%
}
%    \end{macrocode}
%    \end{macro}
%
%    \begin{macro}{\zifrefundefined}
%    If a reference \verb|#1| is undefined, then macro
%    \cs{zref@ifrefundefined} calls \verb|#2| and \verb|#3| otherwise.
%    Also the reference is marked used.
%    \begin{macrocode}
\ZREF@IfDefinable\zifrefundefined\def{%
  #1{%
    \zref@wrapper@babel\ZREF@ifrefundefined{#1}%
  }%
}
%    \end{macrocode}
%    \end{macro}
%    \begin{macro}{\ZREF@ifrefundefined}
%    \begin{macrocode}
\def\ZREF@ifrefundefined#1{%
  \zref@refused{#1}%
  \zref@ifrefundefined{#1}%
}
%    \end{macrocode}
%    \end{macro}
%
%    \begin{macro}{\zref@refused}
%    The problem with undefined references is addressed by
%    the macro \cs{zref@refused}. This can be used outside
%    the expandable context. In case of an undefined reference
%    the flag is set to notify \hologo{LaTeX} and a warning is given.
%    \begin{macrocode}
\ZREF@Robust\def\zref@refused#1{%
  \zref@wrapper@babel\ZREF@refused{#1}%
}
%    \end{macrocode}
%    \end{macro}
%    \begin{macro}{\ZREF@refused}
%    \begin{macrocode}
\def\ZREF@refused#1{%
  \zref@ifrefundefined{#1}{%
    \protect\G@refundefinedtrue
    \@latex@warning{%
      Reference `#1' on page \thepage \space undefined%
    }%
  }{}%
}
%    \end{macrocode}
%    \end{macro}
%
%    \begin{macro}{\zref@ifrefcontainsprop}
%    \cs{zref@ifrefcontainsprop} looks, if the reference \verb|#1|
%    has the property \verb|#2| and calls then \verb|#3| and
%    \verb|#4| otherwise.
%    \begin{macrocode}
\def\zref@ifrefcontainsprop#1#2{%
  \zref@ifrefundefined{#1}{%
    \ltx@secondoftwo
  }{%
    \expandafter\ZREF@ifrefcontainsprop
    \csname Z@E@#2\expandafter\endcsname
    \csname#2\expandafter\expandafter\expandafter\endcsname
    \expandafter\expandafter\expandafter{%
      \csname Z@R@#1\endcsname
    }%
  }%
}
\def\ZREF@ifrefcontainsprop#1#2#3{%
  \expandafter\ifx\expandafter\ZREF@novalue
  #1#3#2\ZREF@novalue\ZREF@nil\ltx@empty
    \expandafter\ltx@secondoftwo
  \else
    \expandafter\ltx@firstoftwo
  \fi
}
\def\ZREF@novalue{\ZREF@NOVALUE}
%    \end{macrocode}
%    \end{macro}
%
%    \begin{macro}{\zref@extract}
%    \cs{zref@extract} is an abbreviation for the case
%    that the default of the property is used as default
%    value.
%    \begin{macrocode}
\def\ZREF@extract#1#2{%
  \romannumeral0%
  \ltx@ifundefined{Z@D@#2}{%
    \expandafter\ltx@space\zref@default
  }{%
    \expandafter\expandafter\expandafter\ZREF@@extract
    \expandafter\expandafter\expandafter{%
      \csname Z@D@#2\endcsname
    }{#1}{#2}%
  }%
}
%    \end{macrocode}
%    \end{macro}
%    \begin{macro}{\ZREF@@extract}
%    \begin{macrocode}
\def\ZREF@@extract#1#2#3{%
  \expandafter\expandafter\expandafter\ltx@space
  \zref@extractdefault{#2}{#3}{#1}%
}
%    \end{macrocode}
%    \end{macro}
%    \begin{macro}{\ZREF@wu@extract}
%    \begin{macrocode}
\def\ZREF@wu@extract#1#2{%
  \etex@unexpanded\expandafter\expandafter\expandafter{%
    \ZREF@extract{#1}{#2}%
  }%
}
%    \end{macrocode}
%    \end{macro}
%    \begin{macro}{\zref@extract}
%    \begin{macrocode}
\let\zref@extract\ZREF@extract
%    \end{macrocode}
%    \end{macro}
%
%    \begin{macro}{\ZREF@extractdefault}
%    The basic extracting macro is \cs{zref@extractdefault} with
%    the reference name in \verb|#1|, the property in \verb|#2| and
%    the default value in \verb|#3| in case for problems.
%    \begin{macrocode}
\def\ZREF@extractdefault#1#2#3{%
  \romannumeral0%
  \zref@ifrefundefined{#1}\ltx@firstoftwo{%
    \zref@ifpropundefined{#2}\ltx@firstoftwo\ltx@secondoftwo
  }{%
    \ltx@space
    #3%
  }{%
    \expandafter\expandafter\expandafter\ltx@space
    \csname Z@E@#2\expandafter\expandafter\expandafter\endcsname
    \csname Z@R@#1\expandafter\endcsname
    \csname#2\endcsname{#3}\ZREF@nil
  }%
}
%    \end{macrocode}
%    \end{macro}
%    \begin{macro}{\ZREF@wu@extractdefault}
%    \begin{macrocode}
\def\ZREF@wu@extractdefault#1#2#3{%
  \etex@unexpanded\expandafter\expandafter\expandafter{%
    \ZREF@extractdefault{#1}{#2}{#3}%
  }%
}
%    \end{macrocode}
%    \end{macro}
%    \begin{macro}{\zref@extractdefault}
%    \begin{macrocode}
\let\zref@extractdefault\ZREF@extractdefault
%    \end{macrocode}
%    \end{macro}
%
%    \begin{macro}{\zref@def@extract}
%    \begin{macrocode}
\ZREF@Robust\def\zref@def@extract#1{%
  \zref@wrapper@babel{\ZREF@def@extract{#1}}%
}
%    \end{macrocode}
%    \end{macro}
%    \begin{macro}{\ZREF@def@extract}
%    \begin{macrocode}
\def\ZREF@def@extract#1#2#3{%
  \zref@refused{#2}%
  \expandafter\expandafter\expandafter\def
  \expandafter\expandafter\expandafter#1%
  \expandafter\expandafter\expandafter{%
    \zref@extract{#2}{#3}%
  }%
}
%    \end{macrocode}
%    \end{macro}
%    \begin{macro}{\zref@def@extractdefault}
%    \begin{macrocode}
\ZREF@Robust\def\zref@def@extractdefault#1{%
  \zref@wrapper@babel{\ZREF@def@extractdefault{#1}}%
}
%    \end{macrocode}
%    \end{macro}
%    \begin{macro}{\ZREF@def@extractdefault}
%    \begin{macrocode}
\def\ZREF@def@extractdefault#1#2#3#4{%
  \zref@refused{#2}%
  \expandafter\expandafter\expandafter\def
  \expandafter\expandafter\expandafter#1%
  \expandafter\expandafter\expandafter{%
    \zref@extractdefault{#2}{#3}{#4}%
  }%
}
%    \end{macrocode}
%    \end{macro}
%
%    \begin{macro}{\ZREF@wrapper@unexpanded}
%    \begin{macrocode}
\ZREF@Robust{\long\def}\ZREF@wrapper@unexpanded#1{%
  \let\zref@wrapper@unexpanded\ltx@firstofone
  \let\zref@getcurrent\ZREF@wu@getcurrent
  \let\zref@extractdefault\ZREF@wu@extractdefault
  \let\zref@extract\ZREF@wu@extract
  #1%
  \let\zref@wrapper@unexpanded\ZREF@wrapper@unexpanded
  \let\zref@getcurrent\ZREF@getcurrent
  \let\zref@extractdefault\ZREF@extractdefault
  \let\zref@extract\ZREF@extract
}
%    \end{macrocode}
%    \end{macro}
%    \begin{macro}{\zref@wrapper@unexpanded}
%    \begin{macrocode}
\ltx@IfUndefined{etex@unexpanded}{%
  \let\zref@wrapper@unexpanded\ltx@firstofone
}{%
  \let\zref@wrapper@unexpanded\ZREF@wrapper@unexpanded
}
%    \end{macrocode}
%    \end{macro}
%
% \subsubsection{Compatibility with \xpackage{babel}}
%
%    \begin{macro}{\zref@wrapper@babel}
%    \begin{macrocode}
\ZREF@Robust{\long\def}\zref@wrapper@babel#1#2{%
  \ifcsname if@safe@actives\endcsname
    \expandafter\ltx@firstofone
  \else
    \expandafter\ltx@secondoftwo
  \fi
  {%
    \if@safe@actives
      \expandafter\ltx@secondoftwo
    \else
      \expandafter\ltx@firstoftwo
    \fi
    {%
      \begingroup
        \csname @safe@activestrue\endcsname
        \edef\x{#2}%
      \expandafter\endgroup
      \expandafter\ZREF@wrapper@babel\expandafter{\x}{#1}%
    }%
  }{%
    #1{#2}%
  }%
}
\long\def\ZREF@wrapper@babel#1#2{%
  #2{#1}%
}
%    \end{macrocode}
%    \end{macro}
%
% \subsubsection{Unique counter support}
%
%    \begin{macro}{\zref@require@unique}
%    Generate the counter |zref@unique| if the counter does not
%    already exist.
%    \begin{macrocode}
\ZREF@Robust\def\zref@require@unique{%
  \@ifundefined{c@zref@unique}{%
    \begingroup
      \let\@addtoreset\ltx@gobbletwo
      \newcounter{zref@unique}%
    \endgroup
%    \end{macrocode}
%    \end{macro}
%
%    \begin{macro}{\thezref@unique}
%    \cs{thezref@unique} is used for automatically generated unique
%    labelnames.
%    \begin{macrocode}
    \renewcommand*{\thezref@unique}{%
      zref@\number\c@zref@unique
    }%
  }{}%
}
%    \end{macrocode}
%    \end{macro}
%
% \subsubsection{Utilities}
%
%    \begin{macro}{\ZREF@number}
%    \begin{macrocode}
\ltx@IfUndefined{numexpr}{%
  \def\ZREF@number#1{\number#1}%
}{%
  \def\ZREF@number#1{\the\numexpr(#1)\relax}%
}
%    \end{macrocode}
%    \end{macro}
%
% \subsubsection{Setup}
%
%    \begin{macro}{\zref@setdefault}
%    Standard \hologo{LaTeX} prints ``??'' in bold face if a reference
%    is not known. \cs{zref@default} holds the text that is
%    printed in case of unknown references and is used, if
%    the default was not specified during the definition
%    of the new property by \cs{ref@newprop}.
%    The global default value can be set by \cs{zref@setdefault}.
%    \begin{macrocode}
\ZREF@Robust\def\zref@setdefault#1{%
  \def\zref@default{#1}%
}
%    \end{macrocode}
%    \end{macro}
%    \begin{macro}{\zref@default}
%    Now we initialize \cs{zref@default} with the same value
%    that \hologo{LaTeX} uses for its undefined references.
%    \begin{macrocode}
\zref@setdefault{%
  \nfss@text{\reset@font\bfseries ??}%
}
%    \end{macrocode}
%    \end{macro}
%
% \paragraph{Main property list.}
%
%    \begin{macro}{\zref@setmainlist}
%    The name of the default property list is stored in
%    \cs{ZREF@mainlist} and can be set by \cs{zref@setmainlist}.
%    \begin{macrocode}
\ZREF@Robust\def\zref@setmainlist#1{%
  \def\ZREF@mainlist{#1}%
}
\zref@setmainlist{main}
%    \end{macrocode}
%    \end{macro}
%    Now we create the list.
%    \begin{macrocode}
\zref@newlist\ZREF@mainlist
%    \end{macrocode}
%
% \paragraph{Main properties.}
%
%    The two properties \texttt{default} and \texttt{page}
%    are created and added to the main property list.
%    They store the data that
%    standard \hologo{LaTeX} uses in its references created by
%    \cs{label}.
%    \begin{description}
%    \item[\texttt{default}] the apperance of the latest
%      counter that is incremented by \cs{refstepcounter}
%    \item[\texttt{page}] the apperance of the page counter
%    \end{description}
%    \begin{macrocode}
\zref@newprop{default}{\@currentlabel}
\zref@newprop*{page}{\thepage}
\zref@addprops\ZREF@mainlist{default,page}
%    \end{macrocode}
%
% \paragraph{Properties}
%
%    \begin{macro}{\ZREF@NewPropAnchor}
%    \begin{macrocode}
\def\ZREF@NewPropAnchor{%
  \zref@newprop{anchor}{%
    \ltx@ifundefined{@currentHref}{}{\@currentHref}%
  }%
  \global\let\ZREF@NewPropAnchor\relax
}
%    \end{macrocode}
%    \end{macro}
%    \begin{macro}{\zref@titleref@current}
%    Later we will redefine the section and caption macros
%    to catch the current title and remember the value in
%    \cs{zref@titleref@current}.
%    \begin{macro}{\ZREF@NewPropTitle}
%    \begin{macrocode}
\def\ZREF@NewPropTitle{%
  \gdef\zref@titleref@current{}%
  \zref@newprop{title}{\zref@titleref@current}%
  \global\let\ZREF@NewPropTitle\relax
}
%    \end{macrocode}
%    \end{macro}
%    \end{macro}
%    \begin{macro}{\ZREF@NewPropTheotype}
%    \begin{macrocode}
\def\ZREF@NewPropTheotype{%
  \zref@newprop{theotype}{}%
  \global\let\ZREF@NewPropTheotype\relax
}
%    \end{macrocode}
%    \end{macro}
%    \begin{macro}{\ZREF@NewPropPageValue}
%    \begin{macrocode}
\def\ZREF@NewPropPageValue{%
  \zref@newprop*{pagevalue}[0]{\number\c@page}%
  \global\let\ZREF@NewPropPageValue\relax
}
%    \end{macrocode}
%    \end{macro}
%
% \paragraph{Mark successful loading}
%
%    \begin{macrocode}
\let\ZREF@base@ok=Y
%</base>
%    \end{macrocode}
%
%
% \subsection{Module \xmodule{user}}
%
%    \begin{macrocode}
%<*user>
\NeedsTeXFormat{LaTeX2e}
\ProvidesPackage{zref-user}%
  [2020-07-03 v2.32 Module user for zref (HO)]%
\RequirePackage{zref-base}[2019/11/29]
\ifx\ZREF@base@ok Y%
\else
  \expandafter\endinput
\fi
%    \end{macrocode}
%
%    Module \xmodule{user} enables a small user interface.
%    All macros are prefixed by \cs{z}.
%
%    First we define the pendants to the standard \hologo{LaTeX} referencing
%    commands \cs{label}, \cs{ref}, and \cs{pageref}.
%
%    \begin{macro}{\zlabel}
%    Similar to \cs{label} the macro \cs{zlabel} writes a reference
%    entry in the \xfile{.aux} file. The main property list is used.
%    Also we add the babel patch. The \cs{label} command can also be
%    used inside section titles, but it must not go into the
%    table of contents. Therefore we have to check this situation.
%    \begin{macrocode}
\newcommand*\zlabel{%
  \ifx\label\ltx@gobble
    \expandafter\ltx@gobble
  \else
    \expandafter\zref@wrapper@babel\expandafter\zref@label
  \fi
}%
%    \end{macrocode}
%    \end{macro}
%
%    \begin{macro}{\zkvlabel}
%    \begin{macrocode}
\newcommand*{\zkvlabel}[1]{%
  \ifx\label\ltx@gobble
    \expandafter\ltx@gobblethree
  \fi
  \zref@wrapper@babel{\zref@labelbykv{#1}}%
}%
%    \end{macrocode}
%    \end{macro}
%
%    \begin{macro}{\zref}
%    Macro \cs{zref} is the corresponding macro for \cs{ref}.
%    Also it provides an optional argument in order to
%    select another property.
%    \begin{macrocode}
\newcommand*{\zref}[2][default]{% robust because of optional argument
  \zref@propexists{#1}{%
    \zref@wrapper@babel\ZREF@zref{#2}{#1}%
  }%
}%
\def\ZREF@zref#1{%
  \zref@refused{#1}%
  \zref@extract{#1}%
}%
%    \end{macrocode}
%    \end{macro}
%
%    \begin{macro}{\zpageref}
%    For macro \cs{zpageref} we just call \cs{zref}
%    with property \verb|page|.
%    \begin{macrocode}
\ZREF@IfDefinable\zpageref\def{%
  {\zref[page]}%
}
%    \end{macrocode}
%    \end{macro}
%
%    \begin{macro}{\zrefused}
%    For the following expandible user macros \cs{zrefused}
%    should be used to notify \hologo{LaTeX} in case of undefined
%    references.
%    \begin{macrocode}
\ZREF@IfDefinable\zrefused\def{%
  {\zref@refused}%
}
%    \end{macrocode}
%    \end{macro}
%
%    \begin{macrocode}
%</user>
%    \end{macrocode}
%
%
% \subsection{Module \xmodule{abspage}}
%
%    \begin{macrocode}
%<*abspage>
\NeedsTeXFormat{LaTeX2e}
\ProvidesPackage{zref-abspage}%
  [2020-07-03 v2.32 Module abspage for zref (HO)]%
\RequirePackage{zref-base}[2019/11/29]
\ifx\ZREF@base@ok Y%
\else
  \expandafter\endinput
\fi
%    \end{macrocode}
%
%    Module \xmodule{abspage} adds a new property \verb|abspage|
%    to the \verb|main| property list for absolute page numbers.
%    These are recorded by the help of package \xpackage{atbegshi}.
%    \begin{macrocode}
\RequirePackage{atbegshi}[2011/10/05]%
%    \end{macrocode}
%    The counter \verb|abspage| must not go in the clear list
%    of \verb|@ckpt| that is used to set counters in
%    \xfile{.aux} files of included \hologo{TeX} files.
%    \begin{macrocode}
\begingroup
  \let\@addtoreset\ltx@gobbletwo
  \newcounter{abspage}%
\endgroup
\setcounter{abspage}{0}%
\AtBeginShipout{%
  \stepcounter{abspage}%
}%
\zref@newprop*{abspage}[0]{\the\c@abspage}%
\zref@addprop\ZREF@mainlist{abspage}%
%    \end{macrocode}
%    Note that counter \verb|abspage| shows the
%    previous page during page processing. Before
%    shipout the counter is incremented.
%    Thus the property is correctly written with
%    deferred writing. If the counter is written
%    using \cs{zref@wrapper@immediate}, then the number
%    is too small by one.
%
%    \begin{macrocode}
%</abspage>
%    \end{macrocode}
%
%
% \subsection{Module \xmodule{counter}}
%
%    \begin{macrocode}
%<*counter>
\NeedsTeXFormat{LaTeX2e}
\ProvidesPackage{zref-counter}%
  [2020-07-03 v2.32 Module counter for zref (HO)]%
\RequirePackage{zref-base}[2019/11/29]
\ifx\ZREF@base@ok Y%
\else
  \expandafter\endinput
\fi
%    \end{macrocode}
%
%    For features such as \xpackage{hyperref}'s \cs{autoref}
%    we need the name of the counter. The property
%    \verb|counter| is defined and added to the main property
%    list. Starting with \LaTeX\ 2020-10-01 the proper can use
%    \@currentcounter. In older formats
%    \cs{refstepcounter} has to be patched but this can fail in some
%    cases, see issue \#5.
%    \begin{macrocode}
\@ifl@t@r\fmtversion{2020-10-01}
 {
  \zref@newprop{counter}{\@currentcounter}
  \zref@addprop\ZREF@mainlist{counter}
 }
 {
  \zref@newprop{counter}{}
  \zref@addprop\ZREF@mainlist{counter}
  \AtBeginDocument{%
    \ZREF@patch{refstepcounter}{%
      \def\refstepcounter#1{%
        \zref@setcurrent{counter}{#1}%
        \ZREF@org@refstepcounter{#1}%
      }%
    }%
  }
 }
%    \end{macrocode}
%    \begin{macrocode}
%</counter>
%    \end{macrocode}
%
%
% \subsection{Module \xmodule{lastpage}}
%
%    \begin{macrocode}
%<*lastpage>
\NeedsTeXFormat{LaTeX2e}
\ProvidesPackage{zref-lastpage}%
  [2020-07-03 v2.32 Module lastpage for zref (HO)]%
\RequirePackage{zref-base}[2019/11/29]
\RequirePackage{zref-abspage}[2019/11/29]
\RequirePackage{atveryend}[2009/12/07]
\ifx\ZREF@base@ok Y%
\else
  \expandafter\endinput
\fi
%    \end{macrocode}
%
%    The module \xmodule{lastpage} implements the service
%    of package \xpackage{lastpage} by setting a reference
%    \verb|LastPage| at the end of the document.
%    If module \xmodule{abspage} is given, also the absolute
%    page number is available, because the properties of
%    the main property list are used.
%    \begin{macrocode}
\zref@newlist{LastPage}
\AfterLastShipout{%
  \if@filesw
    \begingroup
      \advance\c@page\m@ne
      \toks@\expandafter\expandafter\expandafter{%
        \expandafter\Z@L@main
        \Z@L@LastPage
      }%
      \expandafter\zref@wrapper@immediate\expandafter{%
        \expandafter\ZREF@label\expandafter{\the\toks@}{LastPage}%
      }%
    \endgroup
  \fi
}
%    \end{macrocode}
%    \begin{macro}{\zref@iflastpage}
%    \begin{macrocode}
\def\zref@iflastpage#1{%
  \ifnum\zref@extractdefault{#1}{abspage}{-1}=%
        \zref@extractdefault{LastPage}{abspage}{-2} %
    \expandafter\ltx@firstoftwo
  \else
    \expandafter\ltx@secondoftwo
  \fi
}
%    \end{macrocode}
%    \end{macro}
%    \begin{macro}{\ziflastpage}
%    \begin{macrocode}
\ZREF@IfDefinable\ziflastpage\def{%
  {\zref@wrapper@babel\ZREF@iflastpage}%
}
%    \end{macrocode}
%    \end{macro}
%    \begin{macro}{ZREF@iflastpage}
%    \begin{macrocode}
\def\ZREF@iflastpage#1{%
  \zref@refused{LastPage}%
  \zref@refused{#1}%
  \zref@iflastpage{#1}%
}
%    \end{macrocode}
%    \end{macro}
%    \begin{macrocode}
%</lastpage>
%    \end{macrocode}
%
%
% \subsection{Module \xmodule{thepage}}
%
%    \begin{macrocode}
%<*thepage>
\NeedsTeXFormat{LaTeX2e}
\ProvidesPackage{zref-thepage}%
  [2020-07-03 v2.32 Module thepage for zref (HO)]%
\RequirePackage{zref-base}[2019/11/29]
\ifx\ZREF@base@ok Y%
\else
  \expandafter\endinput
\fi
%    \end{macrocode}
%
%    \begin{macrocode}
\RequirePackage{atbegshi}[2011/10/05]
\RequirePackage{zref-abspage}[2019/11/29]
%    \end{macrocode}
%    \begin{macrocode}
\zref@newlist{thepage}
\zref@addprop{thepage}{page}
\ZREF@NewPropPageValue
%    \end{macrocode}
%    \begin{macro}{\zref@thepage@atbegshi@hook}
%    \begin{macrocode}
\let\zref@thepage@atbegshi@hook\ltx@empty
%    \end{macrocode}
%    \end{macro}
%    \begin{macrocode}
\zref@addprop{thepage}{pagevalue}
\AtBeginShipout{%
  \AtBeginShipoutAddToBox{%
    \zref@thepage@atbegshi@hook
    \zref@labelbylist{thepage\the\value{abspage}}{thepage}%
  }%
}
%    \end{macrocode}
%    \begin{macro}{\zref@thepage@name}
%    \begin{macrocode}
\ltx@IfUndefined{numexpr}{%
  \def\zref@thepage@name#1{thepage\number#1}%
}{%
  \def\zref@thepage@name#1{thepage\the\numexpr#1}%
}
%    \end{macrocode}
%    \end{macro}
%    \begin{macro}{\zref@thepage}
%    \begin{macrocode}
\def\zref@thepage#1{%
  \zref@extract{\zref@thepage@name{#1}}{page}%
}%
%    \end{macrocode}
%    \end{macro}
%    \begin{macro}{\zref@thepage@refused}
%    \begin{macrocode}
\ZREF@Robust\def\zref@thepage@refused#1{%
  \zref@refused{\zref@thepage@name{#1}}%
}%
%    \end{macrocode}
%    \end{macro}
%    \begin{macro}{\zthepage}
%    \begin{macrocode}
\ZREF@IfDefinable\zthepage\def{%
  #1{%
    \zref@thepage@refused{#1}%
    \zref@thepage{#1}%
  }%
}
%    \end{macrocode}
%    \end{macro}
%    \begin{macrocode}
%</thepage>
%    \end{macrocode}
%
%
% \subsection{Module \xmodule{nextpage}}
%
%    \begin{macrocode}
%<*nextpage>
\NeedsTeXFormat{LaTeX2e}
\ProvidesPackage{zref-nextpage}%
  [2020-07-03 v2.32 Module nextpage for zref (HO)]%
\RequirePackage{zref-base}[2019/11/29]
\ifx\ZREF@base@ok Y%
\else
  \expandafter\endinput
\fi
%    \end{macrocode}
%
%    \begin{macrocode}
\RequirePackage{zref-abspage}[2019/11/29]
\RequirePackage{zref-thepage}[2019/11/29]
\RequirePackage{zref-lastpage}[2019/11/29]
\RequirePackage{uniquecounter}[2009/12/18]
%    \end{macrocode}
%    \begin{macrocode}
\UniqueCounterNew{znextpage}

\newcommand*{\znextpagesetup}{%
  \afterassignment\ZREF@np@setup@i
  \def\ZREF@np@call@unknown##1%
}
\def\ZREF@np@setup@i{%
  \afterassignment\ZREF@np@setup@ii
  \def\ZREF@np@call@nonext##1%
}
\def\ZREF@np@setup@ii{%
  \def\ZREF@np@call@next##1%
}
\def\ZREF@np@call@unknown#1{#1}
\def\ZREF@np@call@nonext#1{#1}
\def\ZREF@np@call@next#1{#1}
\ZREF@IfDefinable\znextpage\def{%
  {\UniqueCounterCall{znextpage}{\ZREF@nextpage}}%
}%
\newcommand*{\znonextpagename}{}
\newcommand*{\zunknownnextpagename}{\Z@D@page}
\def\ZREF@nextpage#1{%
  \begingroup
    \def\ZREF@refname@this{zref@np#1}%
    \zref@labelbyprops\ZREF@refname@this{abspage}%
    \chardef\ZREF@call=0 % unknown
    \ZREF@ifrefundefined\ZREF@refname@this{%
    }{%
      \edef\ZREF@pagenum@this{%
        \zref@extractdefault\ZREF@refname@this{abspage}{0}%
      }%
      \edef\ZREF@refname@next{%
        \zref@thepage@name{%
          \the\numexpr\ZREF@pagenum@this+1%
        }%
      }%
      \ifnum\ZREF@pagenum@this>0 %
        \ZREF@ifrefundefined{LastPage}{%
          \zref@ifrefundefined\ZREF@refname@next{%
          }{%
            \chardef\ZREF@call=2 % next page
          }%
        }{%
          \edef\ZREF@pagenum@last{%
            \zref@extractdefault{LastPage}{abspage}{0}%
          }%
          \ifnum\ZREF@pagenum@this<\ZREF@pagenum@last\ltx@space
            \ZREF@ifrefundefined\ZREF@refname@next{%
            }{%
              \chardef\ZREF@call=2 % next page
            }%
          \else
            \ifnum\ZREF@pagenum@this=\ZREF@pagenum@this\ltx@space
              \chardef\ZREF@call=1 % no next page
            \fi
          \fi
        }%
      \fi
    }%
    \edef\x{%
      \endgroup
      \ifcase\ZREF@call
        \noexpand\ZREF@np@call@unknown{%
          \noexpand\zunknownnextpagename
        }%
      \or
        \noexpand\ZREF@np@call@nonext{%
          \noexpand\znonextpagename
        }%
      \else
        \noexpand\ZREF@np@call@next{%
          \noexpand\zref@extract{\ZREF@refname@next}{page}%
        }%
      \fi
    }%
  \x
}
%    \end{macrocode}
%    \begin{macrocode}
%</nextpage>
%    \end{macrocode}
%
%
% \subsection{Module \xmodule{totpages}}
%
%    \begin{macrocode}
%<*totpages>
\NeedsTeXFormat{LaTeX2e}
\ProvidesPackage{zref-totpages}%
  [2020-07-03 v2.32 Module totpages for zref (HO)]%
\RequirePackage{zref-base}[2019/11/29]
\ifx\ZREF@base@ok Y%
\else
  \expandafter\endinput
\fi
%    \end{macrocode}
%
%    The absolute page number of the last page is the total
%    page number.
%    \begin{macrocode}
\RequirePackage{zref-abspage}[2019/11/29]
\RequirePackage{zref-lastpage}[2019/11/29]
%    \end{macrocode}
%
%    \begin{macro}{\ztotpages}
%    Macro \cs{ztotpages} contains the number of pages. It can be
%    used inside expandable calculations. It expands to zero if
%    the reference is not yet available.
%    \begin{macrocode}
\newcommand*{\ztotpages}{%
  \zref@extractdefault{LastPage}{abspage}{0}%
}
%    \end{macrocode}
%    \end{macro}
%    Also we mark the reference |LastPage| as used:
%    \begin{macrocode}
\AtBeginDocument{%
  \zref@refused{LastPage}%
}
%    \end{macrocode}
%    \begin{macrocode}
%</totpages>
%    \end{macrocode}
%
%
% \subsection{Module \xmodule{pagelayout}}
%
%    \begin{macrocode}
%<*pagelayout>
\NeedsTeXFormat{LaTeX2e}
\ProvidesPackage{zref-pagelayout}%
  [2020-07-03 v2.32 Module pagelayout for zref (HO)]%
\RequirePackage{zref-base}[2019/11/29]
\ifx\ZREF@base@ok Y%
\else
  \expandafter\endinput
\fi
%    \end{macrocode}
%
%    \begin{macrocode}
\RequirePackage{zref-thepage}[2019/11/29]
\RequirePackage{iftex}[2019/11/07]%
\RequirePackage{atveryend}[2010/03/24]
%    \end{macrocode}
%
%
% \subsubsection{Define layout properties}
%
%    \begin{macrocode}
\def\ZREF@temp#1{%
  \begingroup
    \escapechar=-1 %
  \ltx@ifundefined{\string#1}{\endgroup}{%
    \edef\x{%
      \endgroup
      \noexpand\zref@newprop*{\string#1}%
                             [\noexpand\number\noexpand#1]% hash-ok
                             {\noexpand\number\noexpand#1}%
      \noexpand\zref@addprop{thepage}{\string#1}%
    }%
    \x
  }%
}
\ZREF@temp\mag
\ZREF@temp\paperwidth
\ZREF@temp\paperheight
\ZREF@temp\stockwidth % memoir.cls, crop.sty
\ZREF@temp\stockheight % memoir.cls, crop.sty
\ZREF@temp\mediawidth % VTeX
\ZREF@temp\mediaheight % VTeX
\ifluatex
\zref@newprop*{pdfvorigin}%
                             [\number\pdfvariable vorigin]% hash-ok
                             {\number\pdfvariable vorigin}%
\zref@addprop{thepage}{pdfvorigin}
\zref@newprop*{pdfhorigin}%
                             [\number\pdfvariable horigin]% hash-ok
                             {\number\pdfvariable horigin}%
\zref@addprop{thepage}{pdfhorigin}
\zref@newprop*{pdfpageheight}%
                             [\number\pageheight]% hash-ok
                             {\number\pageheight}%
\zref@addprop{thepage}{pdfpageheight}
\zref@newprop*{pdfpagewidth}%
                             [\number\pagewidth]% hash-ok
                             {\number\pagewidth}%
\zref@addprop{thepage}{pdfpagewidth}
\else
\ZREF@temp\pdfpagewidth
\ZREF@temp\pdfpageheight
\ZREF@temp\pdfhorigin
\ZREF@temp\pdfvorigin
\fi
\ZREF@temp\hoffset
\ZREF@temp\voffset
\ZREF@temp\topmargin
\ZREF@temp\oddsidemargin
\ZREF@temp\evensidemargin
\ZREF@temp\textwidth
\ZREF@temp\textheight
\ZREF@temp\headheight
\ZREF@temp\headsep
\ZREF@temp\footskip
\ZREF@temp\marginparwidth
\ZREF@temp\marginparsep
\ZREF@temp\columnwidth
\ZREF@temp\columnsep
\ZREF@temp\trimedge % memoir.cls
\ZREF@temp\spinemargin % memoir.cls
\ZREF@temp\foremargin % memoir.cls
\ZREF@temp\trimtop % memoir.cls
\ZREF@temp\uppermargin % memoir.cls
\ZREF@temp\headmargin % memoir.cls
%    \end{macrocode}
%
%    \begin{macrocode}
\IfFormatAtLeastTF{2020/10/01}
 {
   \zref@newprop*{outputboxwd}[0pt]{\ShipoutBoxWidth}
   \zref@newprop*{outputboxht}[0pt]{\ShipoutBoxHeight}
   \zref@newprop*{outputboxdp}[0pt]{\ShipoutBoxDepth}
 }
 {
   \zref@newprop*{outputboxwd}[0pt]{\AtBeginShipoutBoxWidth}
   \zref@newprop*{outputboxht}[0pt]{\AtBeginShipoutBoxHeight}
   \zref@newprop*{outputboxdp}[0pt]{\AtBeginShipoutBoxDepth}
 }
%    \end{macrocode}
%    \begin{macrocode}
\zref@addprops{thepage}{outputboxwd,outputboxht,outputboxdp}
%    \end{macrocode}
%
%    \begin{macro}{\ifZREF@pl@list}
%    \begin{macrocode}
\ltx@newif\ifZREF@pl@list
%    \end{macrocode}
%    \end{macro}
%    \begin{macro}{\zref@listpagelayout}
%    \begin{macrocode}
\ZREF@IfDefinable\zlistpagelayout\def{%
  {\global\ZREF@pl@listtrue}%
}
%    \end{macrocode}
%    \end{macro}
%    \begin{macro}{\ZREF@pl@AfterLastShipout}
%    \begin{macrocode}
\def\ZREF@pl@AfterLastShipout{%
  \ifZREF@pl@list
    \edef\ZREF@page@max{\the\value{abspage}}%
    \ltx@ifundefined{ZREF@org@testdef}{%
      \let\ZREF@org@testdef\@testdef
      \def\@testdef##1##2##3{%
        \ZREF@org@testdef{##1}{##2}{##3}%
        \def\ZREF@temp{##1}%
        \ifx\ZREF@temp\ZREF@RefPrefix
          \expandafter\gdef\csname##1@##2\endcsname{##3}%
        \fi
      }%
    }{}%
    \AtVeryEndDocument{\ZREF@pl@AtVeryEnd}%
  \fi
}
%    \end{macrocode}
%    \end{macro}
%    \begin{macro}{\ZREF@pl@AtVeryEnd}
%    \begin{macrocode}
\def\ZREF@pl@AtVeryEnd{%
  \begingroup
    \toks@{Page layout parameters:\MessageBreak}%
    \count@=1 %
    \ZREF@pl@ListPage
    \edef\x{\endgroup
      \noexpand\@PackageInfoNoLine{zref-pagelayout}{\the\toks@}%
    }%
  \x
}
%    \end{macrocode}
%    \end{macro}
%    \begin{macro}{\ZREF@pl@ListPage}
%    \begin{macrocode}
\def\ZREF@pl@ListPage{%
  \edef\x{%
    \toks@={%
      \the\toks@
      Page \the\count@:\noexpand\MessageBreak
      \zref@ifrefundefined{thepage\the\count@}{}{%
        \ltx@space\ltx@space mag = %
        \zref@extract{thepage\the\count@}{mag}%
        \noexpand\MessageBreak
        \ZREF@pl@ListEntry{paperwidth}%
        \ZREF@pl@ListEntry{paperheight}%
        \ZREF@pl@ListEntry{stockwidth}%
        \ZREF@pl@ListEntry{stockheight}%
        \ZREF@pl@ListEntry{mediawidth}%
        \ZREF@pl@ListEntry{mediaheight}%
        \ZREF@pl@ListEntry{pdfpagewidth}%
        \ZREF@pl@ListEntry{pdfpageheight}%
        \ZREF@pl@ListEntry{pdfhorigin}%
        \ZREF@pl@ListEntry{pdfvorigin}%
        \ZREF@pl@ListEntry{hoffset}%
        \ZREF@pl@ListEntry{voffset}%
        \ZREF@pl@ListEntry{topmargin}%
        \ZREF@pl@ListEntry{oddsidemargin}%
        \ZREF@pl@ListEntry{evensidemargin}%
        \ZREF@pl@ListEntry{textwidth}%
        \ZREF@pl@ListEntry{textheight}%
        \ZREF@pl@ListEntry{headheight}%
        \ZREF@pl@ListEntry{headsep}%
        \ZREF@pl@ListEntry{footskip}%
        \ZREF@pl@ListEntry{marginparwidth}%
        \ZREF@pl@ListEntry{marginparsep}%
        \ZREF@pl@ListEntry{columnwidth}%
        \ZREF@pl@ListEntry{columnsep}%
        \ZREF@pl@ListEntry{trimedge}%
        \ZREF@pl@ListEntry{spinemargin}%
        \ZREF@pl@ListEntry{foremargin}%
        \ZREF@pl@ListEntry{trimtop}%
        \ZREF@pl@ListEntry{uppermargin}%
        \ZREF@pl@ListEntry{headmargin}%
      }%
    }%
  }\x
  \ifnum\ZREF@page@max>\count@
    \advance\count@ by\ltx@one
  \else
    \expandafter\ltx@gobble
  \fi
  \ZREF@pl@ListPage
}
%    \end{macrocode}
%    \end{macro}
%    \begin{macro}{\ZREF@pl@ListEntry}
%    \begin{macrocode}
\def\ZREF@pl@ListEntry#1{%
  \zref@ifpropundefined{#1}{%
  }{%
    \zref@ifrefcontainsprop{thepage\the\count@}{#1}{%
      \ltx@space\ltx@space#1 = %
      \zref@extract{thepage\the\count@}{#1}sp = %
      \the\dimexpr\zref@extract{thepage\the\count@}{#1}sp\relax
      \noexpand\MessageBreak
    }{}%
  }%
}
%    \end{macrocode}
%    \end{macro}
%    \begin{macrocode}
\AfterLastShipout{%
  \ZREF@pl@AfterLastShipout
}
%    \end{macrocode}
%
%    \begin{macrocode}
%</pagelayout>
%    \end{macrocode}
%
%
% \subsection{Module \xmodule{pageattr}}
%
%    \begin{macrocode}
%<*pageattr>
\NeedsTeXFormat{LaTeX2e}
\ProvidesPackage{zref-pageattr}%
  [2020-07-03 v2.32 Module pageattr for zref (HO)]%
\RequirePackage{zref-base}[2019/11/29]
\ifx\ZREF@base@ok Y%
\else
  \expandafter\endinput
\fi
%    \end{macrocode}
%
%    \begin{macrocode}
\RequirePackage{iftex}[2019/11/07]%
%    \end{macrocode}
%
%
%    \begin{macrocode}
\let\ZREF@temp=N%
\ifluatex
\expandafter\@firstoftwo
\else
\expandafter\@secondoftwo
\fi
{%luatex
 \RequirePackage{zref-thepage}[2019/11/29]
 \RequirePackage{zref-lastpage}[2019/11/29]%
 \zref@newprop*{pdfpageattr}[]{\zref@hex{\the\pdfvariable pageattr}}%
 \zref@addprop{thepage}{pdfpageattr}%
 \zref@newprop*{pdfpagesattr}[]{\zref@hex{\the\pdfvariable pagesattr}}%
 \zref@addprop{LastPage}{pdfpagesattr}%
 \let\ZREF@temp=Y%
}
{%other
 \ltx@IfUndefined{pdfpageattr}{%
  \@PackageInfoNoLine{zref-pageattr}{%
    \string\pdfpageattr\space is not available%
  }%
  \def\zref@pdfpageattr#1{}%
  \def\zref@pdfpageattr@used#1{}%
 }{%
  \RequirePackage{zref-thepage}[2019/11/29]%
  \zref@newprop*{pdfpageattr}[]{\zref@hex{\the\pdfpageattr}}%
  \zref@addprop{thepage}{pdfpageattr}%
  \let\ZREF@temp=Y%
 }
 \ltx@IfUndefined{pdfpagesattr}{%
  \@PackageInfoNoLine{zref-pageattr}{%
    \string\pdfpagesattr\space is not available%
  }%
  \def\zref@pdfpagesattr{}%
  \def\zref@pdfpagesattr@used{}%
 }{%
  \RequirePackage{zref-lastpage}[2019/11/29]%
  \zref@newprop*{pdfpagesattr}[]{\zref@hex{\the\pdfpagesattr}}%
  \zref@addprop{LastPage}{pdfpagesattr}%
  \let\ZREF@temp=Y%
 }%
}%
%    \end{macrocode}
%    \begin{macrocode}
\ifx\ZREF@temp N%
  \expandafter\endinput
\fi
%    \end{macrocode}
%    \begin{macrocode}
\RequirePackage{zref-abspage}[2019/11/29]
\RequirePackage{atveryend}[2010/03/24]
\RequirePackage{pdftexcmds}[2010/04/01]
\let\ZREF@temp=Y%
\ltx@IfUndefined{pdf@escapehex}{\let\ZREF@temp=N}{}
\ltx@IfUndefined{pdf@unescapehex}{\let\ZREF@temp=N}{}
\ifx\ZREF@temp N%
  \let\zref@hex\ltx@firstofone
  \let\zref@unhex\ltx@firstofone
\else
  \let\zref@hex\pdf@escapehex
  \let\zref@unhex\pdf@unescapehex
\fi
%    \end{macrocode}
%
%    \begin{macro}{\ifZREF@pa@list}
%    \begin{macrocode}
\ltx@newif\ifZREF@pa@list
%    \end{macrocode}
%    \end{macro}
%    \begin{macro}{\zref@listpageattr}
%    \begin{macrocode}
\ZREF@IfDefinable\zlistpageattr\def{%
  {\ZREF@pa@listtrue}%
}
%    \end{macrocode}
%    \end{macro}
%    \begin{macro}{\ZREF@pa@AfterLastShipout}
%    \begin{macrocode}
\def\ZREF@pa@AfterLastShipout{%
  \ifZREF@pa@list
    \edef\ZREF@page@max{\the\value{abspage}}%
    \ltx@ifundefined{ZREF@org@testdef}{%
      \let\ZREF@org@testdef\@testdef
      \def\@testdef##1##2##3{%
        \ZREF@org@testdef{##1}{##2}{##3}%
        \def\ZREF@temp{##1}%
        \ifx\ZREF@temp\ZREF@RefPrefix
          \expandafter\xdef\csname##1@##2\endcsname{##3}%
        \fi
      }%
    }{}%
    \AtVeryEndDocument{\ZREF@pa@AtVeryEnd}%
  \fi
}
%    \end{macrocode}
%    \end{macro}
%    \begin{macro}{\ZREF@pa@AtVeryEnd}
%    \begin{macrocode}
\let\ZREF@temp=Y%
\ltx@IfUndefined{pdfpageattr}{}{\let\ZREF@temp=N}
\ifluatex \let\ZREF@temp=N \fi
\ifx\ZREF@temp Y
 \expandafter\@firstoftwo
\else
 \expandafter\@secondoftwo
\fi
 {%
  \def\ZREF@pa@AtVeryEnd{}%
 }
 {%
  \def\ZREF@pa@AtVeryEnd{%
    \begingroup
      \toks@{List of \ltx@backslashchar
             \ifluatex pdfvariable\else pdf\fi
             pdfpageattr:\MessageBreak}%
      \count@=1 %
      \ZREF@pa@ListPage
      \edef\x{\endgroup
        \noexpand\@PackageInfoNoLine{zref-pageattr}{%
          \the\toks@
        }%
      }%
    \x
  }%
%    \end{macrocode}
%    \end{macro}
%    \begin{macro}{\zref@pageattr}
%    \begin{macrocode}
\def\zref@pdfpageattr#1{%
  \zref@unhex{%
    \zref@extract{thepage\ZREF@number{#1}}{pdfpageattr}%
  }%
}
% compability, \zref@pageattr was defined in older versions
\let\zref@pageattr\zref@pdfpageattr
%    \end{macrocode}
%    \end{macro}
%    \begin{macro}{\zref@pageattr@used}
%    \begin{macrocode}
\ZREF@Robust\def\zref@pageattr@used#1{%
  \zref@refused{thepage\ZREF@number{#1}}%
}
%    \end{macrocode}
%    \end{macro}
%    \begin{macro}{\ZREF@pa@ListPage}
%    \begin{macrocode}
  \def\ZREF@pa@ListPage{%
    \edef\x{%
      \toks@={%
        \the\toks@
        Page \the\count@:%
        \noexpand\MessageBreak
        \zref@ifrefundefined{thepage\the\count@}{}{%
          <<\zref@pdfpageattr\count@>>%
          \noexpand\MessageBreak
        }%
      }%
    }\x
    \ifnum\ZREF@page@max>\count@
      \advance\count@ by\ltx@one
    \else
      \expandafter\ltx@gobble
    \fi
    \ZREF@pa@ListPage
  }%
}
%    \end{macrocode}
%    \end{macro}
%    \begin{macrocode}
\let\ZREF@temp=Y%
\ltx@IfUndefined{pdfpagesattr}{}{\let\ZREF@temp=N}
\ifluatex \let\ZREF@temp=N \fi
\ifx\ZREF@temp N
 \expandafter\@firstofone
\fi
 {%
%    \end{macrocode}
%    \begin{macro}{\zref@pdfpagesattr}
%    \begin{macrocode}
  \def\zref@pdfpagesattr{%
    \zref@unhex{%
      \zref@extract{LastPage}{pdfpagesattr}%
    }%
  }%
%    \end{macrocode}
%    \end{macro}
%    \begin{macro}{\zref@pdfpagesattr@used}
%    \begin{macrocode}
  \ZREF@Robust\def\zref@pdfpagesattr@used{%
    \zref@refused{LastPage}%
  }%
%    \end{macrocode}
%    \end{macro}
%    \begin{macrocode}
  \ltx@LocalAppendToMacro\ZREF@pa@AtVeryEnd{%
    \@PackageInfoNoLine{zref-pageattr}{%
      \ltx@backslashchar
      \ifluatex pdfvariable\else pdf\fi
      pagesattr:\MessageBreak
      <<\zref@pdfpagesattr>>%
      \MessageBreak
    }%
  }%
}
\AfterLastShipout{%
  \ZREF@pa@AfterLastShipout
}
%    \end{macrocode}
%    \begin{macrocode}
%</pageattr>
%    \end{macrocode}
%
%
% \subsection{Module \xmodule{marks}}
%
%    \begin{macrocode}
%<*marks>
\NeedsTeXFormat{LaTeX2e}
\ProvidesPackage{zref-marks}%
  [2020-07-03 v2.32 Module marks for zref (HO)]%
\RequirePackage{zref-base}[2019/11/29]
\ifx\ZREF@base@ok Y%
\else
  \expandafter\endinput
\fi
%    \end{macrocode}
%
%    \begin{macrocode}
\newcommand*{\zref@marks@register}[3][]{%
  \edef\ZREF@TempName{#1}%
  \edef\ZREF@TempNum{\ZREF@number{#2}}%
  \ifnum\ZREF@TempNum<\ltx@zero %
    \PackageError\ZREF@name{%
      \string\zref@marks@register\ltx@space is called with invalid%
      \MessageBreak
      marks register number (\ZREF@TempNum)%
    }{%
      Use `0' or the command, defined by \string\newmarks.\MessageBreak
      \@ehc
    }%
  \else
    \ifx\ZREF@TempName\ltx@empty
      \edef\ZREF@TempName{mark\romannumeral\ZREF@TempNum}%
    \else
      \edef\ZREF@TempName{marks\ZREF@TempName}%
    \fi
    \ZREF@MARKS@DefineProp{top}%
    \ZREF@MARKS@DefineProp{first}%
    \ZREF@MARKS@DefineProp{bot}%
    \kv@parse{#3}{%
      \ifx\kv@value\relax
        \def\kv@value{top,first,bot}%
      \fi
      \edef\ZREF@temp{\expandafter\ltx@car\kv@key X\@nil}%
      \ifx\ZREF@temp\ZREF@STAR
        \edef\kv@key{\expandafter\ltx@cdr\kv@key\@nil}%
        \zref@newlist\kv@key
      \fi
      \expandafter\comma@parse\expandafter{\kv@value}{%
        \ifcase0\ifx\comma@entry\ZREF@NAME@top 1\else
                \ifx\comma@entry\ZREF@NAME@first 1\else
                \ifx\comma@entry\ZREF@NAME@bot 1\fi\fi\fi\ltx@space
          \PackageWarning{zref-marks}{%
            Use `top', `first' or `bot' for the list values%
            \MessageBreak
            in the third argument of \string\zref@marks@register.%
            \MessageBreak
            Ignoring unkown value `\comma@entry'%
          }%
        \else
          \zref@addprop{\kv@key}{\comma@entry\ZREF@TempName}%
        \fi
        \ltx@gobble
      }%
      \ltx@gobbletwo
    }%
  \fi
}
\def\ZREF@STAR{*}
\def\ZREF@NAME@top{top}
\def\ZREF@NAME@first{first}
\def\ZREF@NAME@bot{bot}
\def\ZREF@MARKS@DefineProp#1{%
  \zref@ifpropundefined{#1\ZREF@TempName}{%
    \ifnum\ZREF@TempNum=\ltx@zero
      \begingroup
        \edef\x{\endgroup
          \noexpand\zref@newprop*{#1\ZREF@TempName}[]{%
            \expandafter\noexpand\csname#1mark\endcsname
          }%
        }%
      \x
    \else
      \begingroup
        \edef\x{\endgroup
          \noexpand\zref@newprop*{#1\ZREF@TempName}[]{%
            \expandafter\noexpand\csname#1marks\endcsname
            \ZREF@TempNum
          }%
        }%
      \x
    \fi
  }{%
    \PackageWarning{zref-marks}{%
      \string\zref@marks@register\ltx@space does not generate the%
      \MessageBreak
      new property `#1\ZREF@TempName', because\MessageBreak
      it is already defined%
    }%
  }%
}
%    \end{macrocode}
%    \begin{macrocode}
%</marks>
%    \end{macrocode}
%
% \subsection{Module \xmodule{runs}}
%
%    This module does not use the label-reference-system.
%    The reference changes with each \hologo{LaTeX}~run and would
%    force a rerun warning always.
%    \begin{macrocode}
%<*runs>
\NeedsTeXFormat{LaTeX2e}
\ProvidesPackage{zref-runs}%
  [2020-07-03 v2.32 Module runs for zref (HO)]%
%    \end{macrocode}
%
%    \begin{macro}{\zruns}
%    \begin{macrocode}
\providecommand*{\zruns}{0}%
\AtBeginDocument{%
  \edef\zruns{\number\numexpr\zruns+1}%
  \begingroup
    \def\on@line{}%
    \PackageInfo{zref-runs}{LaTeX runs: \zruns}%
    \if@filesw
      \immediate\write\@mainaux{%
        \string\gdef\string\zruns{\zruns}%
      }%
    \fi
  \endgroup
}
%    \end{macrocode}
%    \end{macro}
%
%    \begin{macrocode}
%</runs>
%    \end{macrocode}
%
%
% \subsection{Module \xmodule{perpage}}
%
%    \begin{macrocode}
%<*perpage>
\NeedsTeXFormat{LaTeX2e}
\ProvidesPackage{zref-perpage}%
  [2020-07-03 v2.32 Module perpage for zref (HO)]%
\RequirePackage{zref-base}[2019/11/29]
\ifx\ZREF@base@ok Y%
\else
  \expandafter\endinput
\fi
%    \end{macrocode}
%
%    This module resets a counter at page boundaries.
%    Because of the asynchronous output routine
%    page counter properties cannot be asked directly,
%    references are necessary.
%
%    For detecting changed pages module \xmodule{abspage}
%    is loaded.
%    \begin{macrocode}
\RequirePackage{zref-abspage}[2019/11/29]
%    \end{macrocode}
%
%    We group the properties for the needed references
%    in the property list |perpage|. The property |pagevalue|
%    records the correct value of the page counter.
%    \begin{macrocode}
\ZREF@NewPropPageValue
\zref@newlist{perpage}
\zref@addprops{perpage}{abspage,page,pagevalue}
%    \end{macrocode}
%    The page value, known by the reference mechanism, will be
%    stored in counter |zpage|.
%    \begin{macrocode}
\newcounter{zpage}
%    \end{macrocode}
%    Counter |zref@unique| helps in generating unique reference
%    names.
%    \begin{macrocode}
\zref@require@unique
%    \end{macrocode}
%    In order to be able to reset the counter, we hook here
%    into \cs{stepcounter}. In fact two nested hooks are used
%    to allow other packages to use the first hook at the beginning
%    of \cs{stepcounter}.
%    \begin{macrocode}
\let\ZREF@org@stepcounter\stepcounter
\def\stepcounter#1{%
  \ifcsname @stepcounterhook@#1\endcsname
    \csname @stepcounterhook@#1\endcsname
  \fi
  \ZREF@org@stepcounter{#1}%
}
%    \end{macrocode}
%    \cs{@stpelt} must be adapted due to the change in latex
%    2015-01, see https://github.com/ho-tex/zref/issues/26
%    \begin{macrocode}
\let\ZREF@org@@stpelt\@stpelt
\def\@stpelt#1{%
  \ifcsname ZREF@perpage@#1\endcsname
    \begingroup
      \let\stepcounter\ZREF@org@stepcounter
      \ZREF@org@@stpelt{#1}%
    \endgroup
    \expandafter\ltx@gobbletwo
  \fi
  \ZREF@org@@stpelt{#1}%
}
%    \end{macrocode}
%
%    \begin{macro}{\zmakeperpage}
%    Makro \cs{zmakeperpage} resets a counter at each page
%    break. It uses the same syntax and semantics as
%    \cs{MakePerPage} from package \xpackage{perpage} \cite{perpage}.
%    The initial start value can be given by the optional
%    argument. Default is one that means after the first
%    \cs{stepcounter} on a new page the counter starts with
%    one.
%    \begin{macrocode}
\ZREF@IfDefinable\zmakeperpage\def{%
  {%
    \@ifnextchar[\ZREF@makeperpage@opt{\ZREF@@makeperpage[\ltx@zero]}%
  }%
}
%    \end{macrocode}
%    \end{macro}
%    We hook before the counter is incremented in \cs{stepcounter},
%    package \xpackage{perpage} afterwards. Thus a little
%    calculation is necessary.
%    \begin{macrocode}
\def\ZREF@makeperpage@opt[#1]{%
  \begingroup
    \edef\x{\endgroup
      \noexpand\ZREF@@makeperpage[\number\numexpr#1-1\relax]%
    }%
  \x
}
%    \end{macrocode}
%    \begin{macrocode}
\def\ZREF@@makeperpage[#1]#2{%
  \@ifundefined{@stepcounterhook@#2}{%
    \expandafter\gdef\csname @stepcounterhook@#2\endcsname{}%
  }{}%
  \expandafter\gdef\csname ZREF@perpage@#2\endcsname{%
    \ZREF@@perpage@step{#2}{#1}%
  }%
  \expandafter\g@addto@macro\csname @stepcounterhook@#2\endcsname{%
    \ifcsname ZREF@perpage@#2\endcsname
      \csname ZREF@perpage@#2\endcsname
    \fi
  }%
}
%    \end{macrocode}
%    \begin{macro}{\ZREF@@perpage@step}
%    The heart of this module follows.
%    \begin{macrocode}
\def\ZREF@@perpage@step#1#2{%
%    \end{macrocode}
%    First the reference is generated.
%    \begin{macrocode}
  \global\advance\c@zref@unique\ltx@one
  \begingroup
    \expandafter
    \zref@labelbylist\expandafter{\thezref@unique}{perpage}%
%    \end{macrocode}
%    The \cs{expandafter} commands are necessary, because \cs{ZREF@temp}
%    is also used inside of \cs{zref@labelbylist}.
%
%    The evaluation of the reference follows. If the reference
%    is not yet kwown, we use the page counter as approximation.
%    \begin{macrocode}
    \zref@ifrefundefined\thezref@unique{%
      \global\c@zpage=\c@page
      \global\let\thezpage\thepage
      \expandafter\xdef\csname ZREF@abspage@#1\endcsname{%
        \number\c@abspage
      }%
    }{%
%    \end{macrocode}
%    The reference is used to set \cs{thezpage} and counter |zpage|.
%    \begin{macrocode}
      \global\c@zpage=\zref@extract\thezref@unique{pagevalue}\relax
      \xdef\thezpage{\noexpand\zref@extract{\thezref@unique}{page}}%
      \expandafter\xdef\csname ZREF@abspage@#1\endcsname{%
        \zref@extractdefault\thezref@unique
            {abspage}{\number\c@abspage}%
      }%
    }%
%    \end{macrocode}
%    Page changes are detected by a changed absolute page number.
%    \begin{macrocode}
    \expandafter\ifx\csname ZREF@abspage@#1\expandafter\endcsname
                    \csname ZREF@currentabspage@#1\endcsname
    \else
      \global\csname c@#1\endcsname=#2\relax
      \global\expandafter\let
          \csname ZREF@currentabspage@#1\expandafter\endcsname
          \csname ZREF@abspage@#1\endcsname
    \fi
  \endgroup
}
%    \end{macrocode}
%    \end{macro}
%    \begin{macro}{\zunmakeperpage}
%    Macro \cs{zunmakeperpage} cancels the effect of \cs{zmakeperpage}.
%    \begin{macrocode}
\ZREF@IfDefinable\zunmakeperpage\def{%
  #1{%
    \global\expandafter
    \let\csname ZREF@perpage@#1\endcsname\@undefined
  }%
}
%    \end{macrocode}
%    \end{macro}
%
%    \begin{macrocode}
%</perpage>
%    \end{macrocode}
%
%
% \subsection{Module \xmodule{titleref}}
%
%    \begin{macrocode}
%<*titleref>
\NeedsTeXFormat{LaTeX2e}
\ProvidesPackage{zref-titleref}%
  [2020-07-03 v2.32 Module titleref for zref (HO)]%
\RequirePackage{zref-base}[2019/11/29]
\ifx\ZREF@base@ok Y%
\else
  \expandafter\endinput
\fi
\RequirePackage{gettitlestring}[2009/12/08]
%    \end{macrocode}
%
% \subsubsection{Implementation}
%    \begin{macrocode}
\RequirePackage{keyval}
%    \end{macrocode}
%
%    This module makes section and caption titles available
%    for the reference system. It uses some of the ideas
%    of package \xpackage{nameref} and \xpackage{titleref}.
%
%    Now we can add the property \verb|title| is added to
%    the main property list.
%    \begin{macrocode}
\ZREF@NewPropTitle
\zref@addprop\ZREF@mainlist{title}%
%    \end{macrocode}
%
%    The title strings go into the \xfile{.aux} file, thus
%    they need some kind of protection.
%    Package \xpackage{titleref} uses a protected expansion
%    method. The advantage is that this can be used to
%    cleanup the string and to remove \cs{label}, \cs{index}
%    and other macros unwanted for referencing.
%    But there is the risk that fragile stuff can break.
%
%    Therefore package \xpackage{nameref} does not expand
%    the string. Thus the entries can safely be written to
%    the \xfile{.aux} file. But potentially dangerous
%    macros such as \cs{label} remain in the string and
%    can cause problems when using the string in references.
%
%    \begin{macro}{\ifzref@titleref@expand}
%    The switch \cs{ifzref@titleref@expand} distinguishes
%    between the both methods. Package \xpackage{nameref}'s
%    behaviour is achieved by setting the switch to false,
%    otherwise \xpackage{titleref}'s expansion is used.
%    Default is false.
%    \begin{macrocode}
\newif\ifzref@titleref@expand
%    \end{macrocode}
%    \end{macro}
%
%    \begin{macro}{\ZREF@titleref@hook}
%    The hook \cs{ZREF@titleref@hook} allows to extend
%    the cleanup for the expansion method. Thus
%    unnecessary macros can be removed or dangerous commands
%    removed. The hook is executed before the expansion of
%    \cs{zref@titleref@current}.
%    \begin{macrocode}
\let\ZREF@titleref@hook\ltx@empty
%    \end{macrocode}
%    \end{macro}
%    \begin{macro}{\zref@titleref@cleanup}
%    The hook should not be used directly, instead we provide the
%    macro \cs{zref@titleref@cleanup} to add stuff to the hook
%    and prevents that a previous non-empty content is not discarded
%    accidently.
%    \begin{macrocode}
\ZREF@Robust\def\zref@titleref@cleanup#1{%
   \begingroup
    \toks@\expandafter{%
      \ZREF@titleref@hook
      #1%
    }%
  \expandafter\endgroup
  \expandafter\def\expandafter\ZREF@titleref@hook\expandafter{%
    \the\toks@
  }%
}%
%    \end{macrocode}
%    \end{macro}
%
%    \begin{macro}{\ifzref@titleref@stripperiod}
%    Sometimes a title contains a period at the end.
%    Package \xpackage{nameref} removes this. This behaviour
%    is controlled by the switch \cs{ifzref@titleref@stripperiod}
%    and works regardless of the setting of option \xoption{expand}.
%    Period stripping is the default.
%    \begin{macrocode}
\newif\ifzref@titleref@stripperiod
\zref@titleref@stripperiodtrue
%    \end{macrocode}
%    \end{macro}
%
%    \begin{macro}{\zref@titleref@setcurrent}
%    Macro \cs{zref@titleref@setcurrent} sets a new current title
%    stored in \cs{zref@titleref@current}. Some cleanup and
%    expansion is performed that can be controlled by the
%    previous switches.
%    \begin{macrocode}
\ZREF@Robust\def\zref@titleref@setcurrent#1{%
  \ifzref@titleref@expand
    \GetTitleStringExpand{#1}%
  \else
    \GetTitleStringNonExpand{#1}%
  \fi
  \edef\zref@titleref@current{%
    \detokenize\expandafter{\GetTitleStringResult}%
  }%
  \ifzref@titleref@stripperiod
    \edef\zref@titleref@current{%
      \expandafter\ZREF@stripperiod\zref@titleref@current
      \ltx@empty.\ltx@empty\@nil
    }%
  \fi
}%
\GetTitleStringDisableCommands{%
  \ZREF@titleref@hook
}
%    \end{macrocode}
%    \end{macro}
%    \begin{macro}{\ZREF@stripperiod}
%    If \cs{ZREF@stripperiod} is called, the argument consists of
%    space tokens and tokens with catcode 12 (other), because
%    of \hologo{eTeX}'s \cs{detokenize}.
%    \begin{macrocode}
\def\ZREF@stripperiod#1.\ltx@empty#2\@nil{#1}%
%    \end{macrocode}
%    \end{macro}
%
% \subsubsection{User interface}
%
%    \begin{macro}{\ztitlerefsetup}
%    The behaviour of module \xmodule{titleref} is controlled
%    by switches and a hook. They can be set by \cs{ztitlerefsetup}
%    with a key value interface, provided by package \xpackage{keyval}.
%    Also the current title can be given explicitly by the key
%    \verb|title|.
%    \begin{macrocode}
\define@key{ZREF@TR}{expand}[true]{%
  \csname zref@titleref@expand#1\endcsname
}%
\define@key{ZREF@TR}{stripperiod}[true]{%
  \csname zref@titleref@stripperiod#1\endcsname
}%
\define@key{ZREF@TR}{cleanup}{%
  \zref@titleref@cleanup{#1}%
}%
\define@key{ZREF@TR}{title}{%
  \def\zref@titleref@current{#1}%
}%
\ZREF@IfDefinable\ztitlerefsetup\def{%
  {\kvsetkeys{ZREF@TR}}%
}%
%    \end{macrocode}
%    \end{macro}
%
%    \begin{macro}{\ztitleref}
%    The user command \cs{ztitleref} references the title.
%    For safety \cs{label} is disabled to prevent multiply
%    defined references.
%    \begin{macrocode}
\ZREF@IfDefinable\ztitleref\def{%
  {\zref@wrapper@babel\ZREF@titleref}%
}%
\def\ZREF@titleref#1{%
  \begingroup
    \zref@refused{#1}%
    \let\label\ltx@gobble
    \zref@extract{#1}{title}%
  \endgroup
}%
%    \end{macrocode}
%    \end{macro}
%
% \subsubsection{Patches for section and caption commands}
%
%    The section and caption macros are patched to extract the
%    title data.
%
%    Captions of figures and tables.
%    \begin{macrocode}
\AtBeginDocument{%
  \ZREF@patch{@caption}{%
    \long\def\@caption#1[#2]{%
      \zref@titleref@setcurrent{#2}%
      \ZREF@org@@caption{#1}[{#2}]%
    }%
  }%
%    \end{macrocode}
%
%    Section commands without star. The title version for the
%    table of contents is used because it is usually shorter and
%    more robust.
%    \begin{macrocode}
  \ZREF@patch{@part}{%
    \def\@part[#1]{%
      \zref@titleref@setcurrent{#1}%
      \ZREF@org@@part[{#1}]%
    }%
  }%
  \ZREF@patch{@chapter}{%
    \def\@chapter[#1]{%
      \zref@titleref@setcurrent{#1}%
      \ZREF@org@@chapter[{#1}]%
    }%
  }%
  \ZREF@patch{@sect}{%
    \def\@sect#1#2#3#4#5#6[#7]{%
      \zref@titleref@setcurrent{#7}%
      \ZREF@org@@sect{#1}{#2}{#3}{#4}{#5}{#6}[{#7}]%
    }%
  }%
%    \end{macrocode}
%    The star versions of the section commands.
%    \begin{macrocode}
  \ZREF@patch{@spart}{%
    \def\@spart#1{%
      \zref@titleref@setcurrent{#1}%
      \ZREF@org@@spart{#1}%
    }%
  }%
  \ZREF@patch{@schapter}{%
    \def\@schapter#1{%
      \zref@titleref@setcurrent{#1}%
      \ZREF@org@@schapter{#1}%
    }%
  }%
  \ZREF@patch{@ssect}{%
    \def\@ssect#1#2#3#4#5{%
      \zref@titleref@setcurrent{#5}%
      \ZREF@org@@ssect{#1}{#2}{#3}{#4}{#5}%
    }%
  }%
%    \end{macrocode}
%
% \subsubsection{Environment \textsf{description}}
%
%    \begin{macrocode}
  \ZREF@patch{descriptionlabel}{%
    \def\descriptionlabel#1{%
      \zref@titleref@setcurrent{#1}%
      \ZREF@org@descriptionlabel{#1}%
    }%
  }%
%    \end{macrocode}
%
% \subsubsection{Class \xclass{memoir}}
%
%    \begin{macrocode}
  \@ifclassloaded{memoir}{%
    \ltx@IfUndefined{ifheadnameref}{}{%
      \def\@chapter[#1]#2{%
        \ltx@IfUndefined{ch@pt@c}{%
          \zref@titleref@setcurrent{#1}%
        }{%
          \ifx\ch@pt@c\ltx@empty
            \zref@titleref@setcurrent{#2}%
          \else
            \def\NR@temp{#1}%
            \ifx\NR@temp\ltx@empty
              \expandafter\zref@titleref@setcurrent
              \expandafter{\ch@pt@c}%
            \else
              \ifheadnameref
                \zref@titleref@setcurrent{#1}%
              \else
                \expandafter\zref@titleref@setcurrent
                \expandafter{\ch@pt@c}%
              \fi
            \fi
          \fi
        }%
        \ZREF@org@@chapter[{#1}]{#2}%
      }%
      \ZREF@patch{M@sect}{%
        \def\M@sect#1#2#3#4#5#6[#7][#8]{%
          \ifheadnameref
            \zref@titleref@setcurrent{#8}%
          \else
            \zref@titleref@setcurrent{#7}%
          \fi
          \ZREF@org@M@sect{#1}{#2}{#3}{#4}{#5}{#6}[{#7}][{#8}]%
        }%
      }%
    }%
  }{}%
%    \end{macrocode}
%
% \subsubsection{Class \xclass{beamer}}
%
%    \begin{macrocode}
  \@ifclassloaded{beamer}{%
    \ZREF@patch{beamer@section}{%
      \long\def\beamer@section[#1]{%
        \zref@titleref@setcurrent{#1}%
        \ZREF@org@beamer@section[{#1}]%
      }%
    }%
    \ZREF@patch{beamer@subsection}{%
      \long\def\beamer@subsection[#1]{%
        \zref@titleref@setcurrent{#1}%
        \ZREF@org@beamer@subsection[{#1}]%
      }%
    }%
    \ZREF@patch{beamer@subsubsection}{%
      \long\def\beamer@subsubsection[#1]{%
        \zref@titleref@setcurrent{#1}%
        \ZREF@org@beamer@subsubsection[{#1}]%
      }%
    }%
  }{}%
%    \end{macrocode}
%
% \subsubsection{Package \xpackage{titlesec}}
%
%    \begin{macrocode}
  \@ifpackageloaded{titlesec}{%
    \ZREF@patch{ttl@sect@i}{%
      \def\ttl@sect@i#1#2[#3]#4{%
        \zref@titleref@setcurrent{#4}%
        \ZREF@org@ttl@sect@i{#1}{#2}[{#3}]{#4}%
      }%
    }%
    \ZREF@patch{ttl@straight@i}{%
      \def\ttl@straight@i#1[#2]#3{%
        \def\ZREF@temp{#2}%
        \ifx\ZREF@temp\ltx@empty
          \zref@titleref@setcurrent{#3}%
        \else
          \zref@titleref@setcurrent{#2}%
        \fi
        \ZREF@org@ttl@straight@i{#1}[{#2}]{#3}%
      }%
    }%
  }{}%
%    \end{macrocode}
%
% \subsubsection{Package \xpackage{longtable}}
%
%    Package \xpackage{longtable}: some support for its \cs{caption}.
%    However \cs{label} inside the caption is not supported.
%    \begin{macrocode}
  \@ifpackageloaded{longtable}{%
    \ZREF@patch{LT@c@ption}{%
      \def\LT@c@ption#1[#2]#3{%
        \ZREF@org@LT@c@ption{#1}[{#2}]{#3}%
        \zref@titleref@setcurrent{#2}%
      }%
    }%
  }{}%
%    \end{macrocode}
%
% \subsubsection{Package \xpackage{listings}}
%
%    Package \xpackage{listings}: support for its caption.
%    \begin{macrocode}
  \@ifpackageloaded{listings}{%
    \ZREF@patch{lst@MakeCaption}{%
      \def\lst@MakeCaption{%
        \ifx\lst@label\ltx@empty
        \else
          \expandafter\zref@titleref@setcurrent\expandafter{%
            \lst@@caption
          }%
        \fi
        \ZREF@org@lst@MakeCaption
      }%
    }%
  }{}%
%    \end{macrocode}
%
% \subsubsection{Theorems}
%
%    \begin{macrocode}
  \ZREF@patch{@opargbegintheorem}{%
    \def\@opargbegintheorem#1#2#3{%
      \zref@titleref@setcurrent{#3}%
      \ZREF@org@@opargbegintheorem{#1}{#2}{#3}%
    }%
  }%
  \@ifpackageloaded{amsthm}{%
    \begingroup
      \edef\x{macro:\string#1\string#2[\string#3]}%
      \@onelevel@sanitize\x
      \def\y#1->#2\@nil{#1}%
      \edef\z{\expandafter\y\meaning\@begintheorem->\@nil}%
      \@onelevel@sanitize\z
    \expandafter\endgroup
    \ifx\x\z
      \ZREF@patch{@begintheorem}{%
        \def\@begintheorem#1#2[#3]{%
          \zref@titleref@setcurrent{#3}%
          \ZREF@org@@begintheorem{#1}{#2}[{#3}]%
        }%
      }%
    \fi
  }{}%
%    \end{macrocode}
%
%    \begin{macrocode}
}
%    \end{macrocode}
%
%    \begin{macrocode}
%</titleref>
%    \end{macrocode}
%
%
% \subsection{Module \xmodule{xr}}
%
%    \begin{macrocode}
%<*xr>
\NeedsTeXFormat{LaTeX2e}
\ProvidesPackage{zref-xr}%
  [2020-07-03 v2.32 Module xr for zref (HO)]%
\RequirePackage{zref-base}[2019/11/29]
\ifx\ZREF@base@ok Y%
\else
  \expandafter\endinput
\fi
%    \end{macrocode}
%    \begin{macrocode}
\RequirePackage{keyval}
\RequirePackage{kvoptions}[2010/02/22]
%    \end{macrocode}
%
%    We declare property |url|, because this is added,
%    if a reference is imported and has not already set this field.
%    Or if |hyperref| is used, then this property can be asked.
%    \begin{macrocode}
\zref@newprop{url}{}%
\zref@newprop{urluse}{}%
%    \end{macrocode}
%    \begin{macrocode}
\zref@newprop{externaldocument}{}%
%    \end{macrocode}
%
%    Most code, especially the handling of the \xfile{.aux}
%    files are taken from David Carlisle's \xpackage{xr} package.
%    Therefore I drop the documentation for these macros here.
%
%    \begin{macro}{\zref@xr@ext}
%    If the URL is not specied, then assume
%    processed file with a guessed extension.
%    Use the setting of \xpackage{hyperref} if available.
%    \begin{macrocode}
\providecommand*{\zref@xr@ext}{%
  \ltx@ifundefined{XR@ext}{pdf}{\XR@ext}%
}%
%    \end{macrocode}
%    \end{macro}
%    \begin{macro}{\ifZREF@xr@zreflabel}
%    The use of the star form of \cs{zexternaldocument}
%    is remembered in the switch \cs{ifZREF@xr@zreflabel}.
%    \begin{macrocode}
\newif\ifZREF@xr@zreflabel
%    \end{macrocode}
%    \end{macro}
%    \begin{macrocode}
\SetupKeyvalOptions{%
  family=ZREF@XR,%
  prefix=ZREF@xr@%
}
\DeclareBoolOption[true]{tozreflabel}
\DeclareBoolOption[false]{toltxlabel}
\DeclareBoolOption{verbose}
\define@key{ZREF@XR}{ext}{%
  \def\zref@xr@{#1}%
}
\DeclareBoolOption[false]{urluse}
%    \end{macrocode}
%    \begin{macro}{\zxrsetup}
%    \begin{macrocode}
\newcommand*{\zxrsetup}{%
  \kvsetkeys{ZREF@XR}%
}%
%    \end{macrocode}
%    \end{macro}
%
%    \begin{macro}{\ZREF@xr@URL}
%    \begin{macrocode}
\newcount\ZREF@xr@URL
\ZREF@xr@URL=\ltx@zero
%    \end{macrocode}
%    \end{macro}
%    \begin{macro}{\ZREF@xr@AddURL}
%    \begin{macrocode}
\def\ZREF@xr@AddURL#1{%
  \begingroup
    \def\ZREF@temp{#1}%
    \count@=\ltx@one
    \ZREF@xr@@AddUrl
  \endgroup
}
%    \end{macrocode}
%    \end{macro}
%    \begin{macro}{\ZREF@xr@@AddUrl}
%    \begin{macrocode}
\def\ZREF@xr@@AddUrl{%
  \ifnum\count@>\ZREF@xr@URL
    \global\advance\ZREF@xr@URL by\ltx@one
    \xdef\ZREF@xr@theURL{\romannumeral\ZREF@xr@URL}%
    \global\expandafter\let
        \csname Z@U@\ZREF@xr@theURL\endcsname\ZREF@temp
    \@PackageInfo{zref-xr}{%
      \ltx@backslashchar Z@U@\ZREF@xr@theURL:\MessageBreak
      \ZREF@temp\MessageBreak
    }%
  \else
    \expandafter
    \ifx\csname Z@U@\romannumeral\count@\endcsname\ZREF@temp
      \xdef\ZREF@xr@theURL{\romannumeral\count@}%
    \else
      \expandafter\expandafter\expandafter\ZREF@xr@@AddUrl
    \fi
  \fi
}
%    \end{macrocode}
%    \end{macro}
%
%    \begin{macro}{\zexternaldocument}
%    In its star form it looks for \cs{newlabel},
%    otherwise for \cs{zref@newlabel}.
%     Later we
%    will read \xfile{.aux} files that expects
%    |@| to have catcode 11 (letter).
%    \begin{macrocode}
\ZREF@IfDefinable\zexternaldocument\def{%
  {%
    \ZREF@NewPropAnchor
    \ZREF@NewPropTitle
    \begingroup
      \csname @safe@actives@true\endcsname
      \makeatletter
      \@ifstar{%
        \ZREF@xr@zreflabelfalse
        \@testopt\ZREF@xr@externaldocument{}%
      }{%
        \ZREF@xr@zreflabeltrue
        \@testopt\ZREF@xr@externaldocument{}%
      }%
  }%
}%
%    \end{macrocode}
%    \end{macro}
%
%    If the \cs{include} featuer was used, there can be
%    several \xfile{.aux} files. These files are read
%    one after another, especially they are not recursively
%    read in order to save read registers. Thus it can
%    happen that the read order of the newlabel commands
%    differs from \hologo{LaTeX}'s order using \cs{input}.
%
%    \begin{macro}{\ZREF@xr@externaldocument}
%    It reads the remaining arguments. \cs{newcommand}
%    comes in handy for the optional argument.
%    \begin{macrocode}
\def\ZREF@xr@externaldocument[#1]#2{%
    \def\ZREF@xr@prefix{#1}%
    \let\ZREF@xr@filelist\ltx@empty
    \edef\ZREF@xr@externalfile{#2}%
    \edef\ZREF@xr@file{\ZREF@xr@externalfile.aux}%
    \filename@parse{#2}%
    \@testopt\ZREF@xr@graburl{#2.\zref@xr@ext}%
}%
\def\ZREF@xr@graburl[#1]{%
    \edef\ZREF@xr@url{#1}%
    \ifZREF@xr@urluse
      \expandafter\ZREF@xr@AddURL\expandafter{\ZREF@xr@url}%
      \expandafter\def\expandafter\ZREF@xr@url
      \expandafter{\csname Z@U@\ZREF@xr@theURL\endcsname}%
    \fi
    \ZREF@xr@checkfile
  \endgroup
}%
%    \end{macrocode}
%    \end{macro}
%    \begin{macro}{\ZREF@xr@processfile}
%    We follow \xpackage{xr} here, \cs{IfFileExists} offers
%    a nicer test, but we have to open the file anyway.
%    \begin{macrocode}
\def\ZREF@xr@checkfile{%
  \openin\@inputcheck\ZREF@xr@file\relax
  \ifeof\@inputcheck
    \PackageWarning{zref-xr}{%
      File `\ZREF@xr@file' not found or empty,\MessageBreak
      labels not imported%
    }%
  \else
    \PackageInfo{zref-xr}{%
      Label \ifZREF@xr@zreflabel (zref) \fi
      import from `\ZREF@xr@file'%
    }%
    \def\ZREF@xr@found{0}%
    \def\ZREF@xr@ignored@empty{0}%
    \def\ZREF@xr@ignored@zref{0}%
    \def\ZREF@xr@ignored@ltx{0}%
    \ZREF@xr@processfile
    \closein\@inputcheck
    \begingroup
      \let\on@line\ltx@empty
      \PackageInfo{zref-xr}{%
        Statistics for `\ZREF@xr@file':\MessageBreak
        \ZREF@xr@found\space
        \ifZREF@xr@zreflabel zref\else LaTeX\fi\space
        label(s) found%
        \ifnum\ZREF@xr@ignored@empty>0 %
          ,\MessageBreak
          \ZREF@xr@ignored@empty\space empty label(s) ignored%
        \fi
        \ifnum\ZREF@xr@ignored@zref>0 %
          ,\MessageBreak
          \ZREF@xr@ignored@zref\space
          duplicated zref label(s) ignored%
        \fi
        \ifnum\ZREF@xr@ignored@ltx>0 %
          ,\MessageBreak
          \ZREF@xr@ignored@ltx\space
          duplicated latex label(s) ignored%
        \fi
      }%
    \endgroup
  \fi
  \ifx\ZREF@xr@filelist\ltx@empty
  \else
    \edef\ZREF@xr@file{%
      \expandafter\ltx@car\ZREF@xr@filelist\@nil
    }%
    \edef\ZREF@xr@filelist{%
      \expandafter\ltx@cdr\ZREF@xr@filelist\ltx@empty\@nil
    }%
    \expandafter\ZREF@xr@checkfile
  \fi
}%
%    \end{macrocode}
%    \end{macro}
%
%    \begin{macro}{\ZREF@xr@processfile}
%    \begin{macrocode}
\def\ZREF@xr@processfile{%
  \read\@inputcheck to\ZREF@xr@line
  \expandafter\ZREF@xr@processline\ZREF@xr@line..\ZREF@nil
  \ifeof\@inputcheck
  \else
    \expandafter\ZREF@xr@processfile
  \fi
}%
%    \end{macrocode}
%    \end{macro}
%    \begin{macro}{\ZREF@xr@processline}
%    The most work must be done for analyzing the
%    arguments of \cs{newlabel}.
%    \begin{macrocode}
\long\def\ZREF@xr@processline#1#2#3\ZREF@nil{%
  \def\x{#1}%
  \toks@{#2}%
  \ifZREF@xr@zreflabel
    \ifx\x\ZREF@xr@zref@newlabel
      \expandafter
      \ZREF@xr@process@zreflabel\ZREF@xr@line...\ZREF@nil
    \fi
  \else
    \ifx\x\ZREF@xr@newlabel
      \expandafter
      \ZREF@xr@process@label\ZREF@xr@line...[]\ZREF@nil
    \fi
  \fi
  \ifx\x\ZREF@xr@@input
    \edef\ZREF@xr@filelist{%
      \etex@unexpanded\expandafter{\ZREF@xr@filelist}%
      {\filename@area\the\toks@}%
    }%
  \fi
}%
\def\ZREF@xr@process@zreflabel\zref@newlabel#1#2#3\ZREF@nil{%
  \edef\ZREF@xr@refname{Z@R@\ZREF@xr@prefix#1}%
  \edef\ZREF@xr@found{\the\numexpr\ZREF@xr@found+1\relax}%
  \def\x{#2}%
  \edef\ZREF@xr@tempname{$temp$}%
  \edef\ZREF@xr@temprefname{Z@R@\ZREF@xr@tempname}%
  \let\ZREF@xr@list\x
  \ifx\ZREF@xr@list\ltx@empty
    \PackageWarningNoLine{zref-xr}{%
      Label `#1' without properties ignored\MessageBreak
      in file `\ZREF@xr@file'%
    }%
    \edef\ZREF@xr@ignored@empty{%
      \the\numexpr\ZREF@xr@ignored@empty+1\relax
    }%
  \else
    \expandafter\ZREF@xr@checklist\x\ZREF@nil
    \expandafter\let\csname\ZREF@xr@temprefname\endcsname\x
    \expandafter\ltx@LocalAppendToMacro
    \csname\ZREF@xr@temprefname\expandafter\endcsname
    \expandafter{%
      \expandafter\externaldocument\expandafter{%
        \ZREF@xr@externalfile
      }%
    }%
    \ZREF@xr@urlcheck\ZREF@xr@tempname
    \ifZREF@xr@tozreflabel
      \@ifundefined{\ZREF@xr@refname}{%
        \ifZREF@xr@verbose
          \PackageInfo{zref-xr}{%
            Import to zref label `\ZREF@xr@tempname#1'%
          }%
        \fi
        \global\expandafter
        \let\csname\ZREF@xr@refname\expandafter\endcsname
        \csname\ZREF@xr@temprefname\endcsname
      }{%
        \ZREF@xr@zref@ignorewarning{\ZREF@xr@prefix#1}%
      }%
    \fi
    \ifZREF@xr@toltxlabel
      \ZREF@xr@tolabel{\ZREF@xr@tempname}{\ZREF@xr@prefix#1}%
    \fi
  \fi
}%
\def\ZREF@xr@process@label\newlabel#1#2#3[#4]#5\ZREF@nil{%
  \def\ZREF@xr@refname{Z@R@\ZREF@xr@prefix#1}%
  \edef\ZREF@xr@found{\the\numexpr\ZREF@xr@found+1\relax}%
  \def\x{#2}%
  \edef\ZREF@xr@tempname{$temp$}%
  \edef\ZREF@xr@temprefname{Z@R@\ZREF@xr@tempname}%
  \expandafter\ZREF@xr@scanparams
      \csname\ZREF@xr@temprefname\expandafter\endcsname
      \x{}{}{}{}{}\ZREF@nil
  \ifx\\#4\\%
  \else
    % ntheorem knows an optional argument at the end of \newlabel
    \ZREF@NewPropTheotype
    \expandafter\ltx@LocalAppendToMacro
        \csname\ZREF@xr@temprefname\endcsname{\theotype{#4}}%
  \fi
  \expandafter\ltx@LocalAppendToMacro
  \csname\ZREF@xr@temprefname\expandafter\endcsname\expandafter{%
    \expandafter\externaldocument\expandafter{%
      \ZREF@xr@externalfile
    }%
  }%
  \ZREF@xr@urlcheck\ZREF@xr@tempname
  \ifZREF@xr@tozreflabel
    \@ifundefined{\ZREF@xr@refname}{%
      \ifZREF@xr@verbose
        \PackageInfo{zref-xr}{%
          Import to zref label `\ZREF@xr@prefix#1'%
        }%
      \fi
      \global\expandafter
      \let\csname\ZREF@xr@refname\expandafter\endcsname
      \csname\ZREF@xr@temprefname\endcsname
    }{%
      \ZREF@xr@zref@ignorewarning{\ZREF@xr@prefix#1}%
    }%
  \fi
  \ifZREF@xr@toltxlabel
    \ZREF@xr@tolabel{\ZREF@xr@tempname}{\ZREF@xr@prefix#1}%
  \fi
}
\def\ZREF@xr@zref@newlabel{\zref@newlabel}%
\def\ZREF@xr@newlabel{\newlabel}%
\def\ZREF@xr@@input{\@input}%
\def\ZREF@xr@relax{\relax}%
%    \end{macrocode}
%    \end{macro}
%    \begin{macro}{\ZREF@xr@tolabel}
%    \begin{macrocode}
\def\ZREF@xr@tolabel#1#2{%
  \ifZREF@xr@verbose
    \PackageInfo{zref-xr}{%
      Import to LaTeX label `#2'%
    }%
  \fi
  \zref@wrapper@unexpanded{%
    \expandafter\xdef\csname r@#2\endcsname{%
      {%
        \ltx@ifundefined{M@TitleReference}{%
          \ltx@ifundefined{TR@TitleReference}{%
            \zref@extractdefault{#1}{default}{}%
          }{%
            \noexpand\TR@TitleReference
            {\zref@extractdefault{#1}{default}{}}%
            {\zref@extractdefault{#1}{title}{}}%
          }%
        }{%
          \noexpand\M@TitleReference
          {\zref@extractdefault{#1}{default}{}}%
          {\zref@extractdefault{#1}{title}{}}%
        }%
      }%
      {\zref@extractdefault{#1}{page}{}}%
      \ltx@ifpackageloaded{nameref}{%
        {\zref@extractdefault{#1}{title}{}}%
        {\zref@extractdefault{#1}{anchor}{}}%
        \zref@ifrefcontainsprop{#1}{urluse}{%
          {\zref@extractdefault{#1}{urluse}{}}%
        }{%
          {\zref@extractdefault{#1}{url}{}}%
        }%
      }{}%
    }%
  }%
}
%    \end{macrocode}
%    \end{macro}
%
%    \begin{macro}{\ZREF@xr@zref@ignorewarning}
%    \begin{macrocode}
\def\ZREF@xr@zref@ignorewarning#1{%
  \PackageWarningNoLine{zref-xr}{%
    Zref label `#1' is already in use\MessageBreak
    in file `\ZREF@xr@file'%
  }%
  \edef\ZREF@xr@ignored@zref{%
    \the\numexpr\ZREF@xr@ignored@zref+1%
  }%
}%
%    \end{macrocode}
%    \end{macro}
%
%    \begin{macro}{\ZREF@xr@ltx@ignorewarning}
%    \begin{macrocode}
\def\ZREF@xr@ltx@ignorewarning#1{%
  \PackageWarningNoLine{zref-xr}{%
    LaTeX label `#1' is already in use\MessageBreak
    in file `\ZREF@xr@file'%
  }%
  \edef\ZREF@xr@ignored@ltx{%
    \the\numexpr\ZREF@xr@ignored@ltx+1%
  }%
}%
%    \end{macrocode}
%    \end{macro}
%
%    \begin{macro}{\ZREF@xr@checklist}
%    \begin{macrocode}
\def\ZREF@xr@checklist#1#2#3\ZREF@nil{%
  \ifx\@undefined#1\relax
    \expandafter\ZREF@xr@checkkey\string#1\@nil
  \fi
  \ifx\\#3\\%
  \else
    \ltx@ReturnAfterFi{%
      \ZREF@xr@checklist#3\ZREF@nil
    }%
  \fi
}%
\def\ZREF@xr@checkkey#1#2\@nil{%
  \zref@ifpropundefined{#2}{%
    \zref@newprop{#2}{}%
  }{}%
}%
%    \end{macrocode}
%    \end{macro}
%
%    \begin{macro}{\ZREF@xr@scanparams}
%    \begin{macrocode}
\def\ZREF@xr@scanparams#1#2#3#4#5#6#7\ZREF@nil{%
  \let#1\ltx@empty
  \ZREF@foundfalse
  \ZREF@xr@scantitleref#1#2\TR@TitleReference{}{}\ZREF@nil
  \ifZREF@found
  \else
    \ltx@LocalAppendToMacro#1{\default{#2}}%
  \fi
  % page
  \ltx@LocalAppendToMacro#1{\page{#3}}%
  % nameref title
  \ifZREF@found
  \else
    \ifx\\#4\\%
    \else
      \def\ZREF@xr@temp{#4}%
      \ifx\ZREF@xr@temp\ZREF@xr@relax
      \else
        \ltx@LocalAppendToMacro#1{\title{#4}}%
      \fi
    \fi
  \fi
  % anchor
  \ifx\\#5\\%
  \else
    \ltx@LocalAppendToMacro#1{\anchor{#5}}%
  \fi
  \ifx\\#6\\%
  \else
    \ifZREF@xr@urluse
      \ZREF@xr@AddURL{#6}%
      \expandafter\ltx@LocalAppendToMacro\expandafter#1%
      \expandafter{%
        \expandafter\urluse\expandafter{%
          \csname Z@U@\ZREF@xr@theURL\endcsname
        }%
      }%
    \else
      \ltx@LocalAppendToMacro#1{\url{#6}}%
    \fi
  \fi
}%
%    \end{macrocode}
%    \end{macro}
%
%    \begin{macro}{\ZREF@xr@scantitleref}
%    \begin{macrocode}
\def\ZREF@xr@scantitleref#1#2\TR@TitleReference#3#4#5\ZREF@nil{%
  \ifx\\#5\\%
  \else
    \ltx@LocalAppendToMacro#1{%
      \default{#3}%
      \title{#4}%
    }%
    \ZREF@foundtrue
  \fi
}%
%    \end{macrocode}
%    \end{macro}
%
%    \begin{macro}{\ZREF@xr@urlcheck}%
%    \begin{macrocode}
\def\ZREF@xr@urlcheck#1{%
  \zref@ifrefcontainsprop{#1}{anchor}{%
    \zref@ifrefcontainsprop{#1}{url}{%
    }{%
      \expandafter
      \ltx@LocalAppendToMacro\csname Z@R@#1\expandafter\endcsname
      \expandafter{%
        \csname url\ifZREF@xr@urluse use\fi
        \expandafter\endcsname\expandafter{\ZREF@xr@url}%
      }%
    }%
  }{%
  }%
}%
%    \end{macrocode}
%    \end{macro}
%
%    \begin{macrocode}
%</xr>
%    \end{macrocode}
%
%
% \subsection{Module \xmodule{hyperref}}
%
%    UNFINISHED :-(
%
%    \begin{macrocode}
%<*hyperref>
\NeedsTeXFormat{LaTeX2e}
\ProvidesPackage{zref-hyperref}%
  [2020-07-03 v2.32 Module hyperref for zref (HO)]%
\RequirePackage{zref-base}[2019/11/29]
\ifx\ZREF@base@ok Y%
\else
  \expandafter\endinput
\fi
%    \end{macrocode}
%
%    \begin{macrocode}
\ZREF@NewPropAnchor
\zref@addprop\ZREF@mainlist{anchor}%
%    \end{macrocode}
%
%    \begin{macrocode}
%</hyperref>
%    \end{macrocode}
%
%
% \subsection{Module \xmodule{savepos}}
%
%    Module \xmodule{savepos} provides an interface for
%    \pdfTeX's \cmd{\pdfsavepos}, see the manual for
%    \pdfTeX.
%
% \subsubsection{Identification}
%
%    \begin{macrocode}
%<*savepos>
\NeedsTeXFormat{LaTeX2e}
\ProvidesPackage{zref-savepos}%
  [2020-07-03 v2.32 Module savepos for zref (HO)]%
\RequirePackage{zref-base}[2019/11/29]
\ifx\ZREF@base@ok Y%
\else
  \expandafter\endinput
\fi
%    \end{macrocode}
%
% \subsubsection{Availability}
%
%    First we check, whether the feature is available.
%    \begin{macrocode}
\ifx\directlua\@undefined
\ltx@IfUndefined{pdfsavepos}{%
  \PackageError\ZREF@name{%
    \string\pdfsavepos\space is not supported.\MessageBreak
    It is provided by pdfTeX (1.40) or XeTeX%
  }\ZREF@UpdatePdfTeX
  \endinput
}{}%
\fi
%    \end{macrocode}
%    In PDF mode we are done. However support for DVI mode
%    was added later in version 1.40.0. In earlier versions
%    \cs{pdfsavepos} is defined, but its execution raises
%    an error. Note that \hologo{XeTeX} also provides \cs{pdfsavepos}.
%    \begin{macrocode}
\ifpdf
\else
  \ltx@IfUndefined{pdftexversion}{%
  }{%
    \ifnum\pdftexversion<140 %
      \PackageError\ZREF@name{%
        \string\pdfsavepos\space is not supported in DVI mode%
        \MessageBreak
        of this pdfTeX version%
      }\ZREF@UpdatePdfTeX
      \expandafter\expandafter\expandafter\endinput
    \fi
  }%
\fi
%    \end{macrocode}
%
% \subsubsection{Setup}
%
%    \begin{macrocode}
\zref@newlist{savepos}
\ifx\directlua\@undefined
  \zref@newprop*{posx}[0]{\the\pdflastxpos}
  \zref@newprop*{posy}[0]{\the\pdflastypos}
\else
  \zref@newprop*{posx}[0]{\the\lastxpos}
  \zref@newprop*{posy}[0]{\the\lastypos}
\fi
\zref@addprops{savepos}{posx,posy}
%    \end{macrocode}
%
% \subsubsection{User macros}
%
%    \begin{macro}{\zref@savepos}
%    \begin{macrocode}
\ifx\directlua\@undefined
  \def\zref@savepos{%
    \if@filesw
      \pdfsavepos
    \fi
  }
\else
  \def\zref@savepos{%
    \if@filesw
      \savepos
    \fi
  }
\fi
%    \end{macrocode}
%    \end{macro}
%
%    \begin{macro}{\ZREF@zsavepos}
%    \begin{macrocode}
\def\ZREF@zsavepos#1#2#3{%
  \@bsphack
  \if@filesw
    \zref@savepos
    #1{#3}{#2}%
    \ltx@IfUndefined{TeXXeTstate}{%
    }{%
      \ifnum\TeXXeTstate=\ltx@zero
      \else
        \zref@savepos
      \fi
    }%
  \fi
  \@esphack
}
%    \end{macrocode}
%    \end{macro}
%
%    \begin{macro}{\zsavepos}
%    The current location is stored in a reference with the
%    given name.
%    \begin{macrocode}
\ZREF@IfDefinable\zsavepos\def{%
  {%
    \ZREF@zsavepos\zref@labelbylist{savepos}%
  }%
}
%    \end{macrocode}
%    \end{macro}
%
%    \begin{macro}{\zsaveposx}
%    \begin{macrocode}
\ZREF@IfDefinable\zsaveposx\def{%
  {%
    \ZREF@zsavepos\zref@labelbyprops{posx}%
  }%
}
%    \end{macrocode}
%    \end{macro}
%    \begin{macro}{\zsaveposy}
%    \begin{macrocode}
\ZREF@IfDefinable\zsaveposy\def{%
  {%
    \ZREF@zsavepos\zref@labelbyprops{posy}%
  }%
}
%    \end{macrocode}
%    \end{macro}
%
%    \begin{macro}{\zposx}
%    \begin{macro}{\zposy}
%    The horizontal and vertical position are available by
%    \cs{zposx} and \cs{zposy}. Do not rely on absolute
%    positions. They differ in DVI and PDF mode of \pdfTeX.
%    Use differences instead. The unit of the position numbers
%    is sp.
%    \begin{macrocode}
\newcommand*{\zposx}[1]{%
  \zref@extract{#1}{posx}%
}%
\newcommand*{\zposy}[1]{%
  \zref@extract{#1}{posy}%
}%
%    \end{macrocode}
%    \end{macro}
%    \end{macro}
%
%    Typically horizontal and vertical positions are used
%    inside calculations. Therefore the extracting macros
%    should be expandable and babel's patch is not applyable.
%
%    Also it is in the responsibility of the user to marked
%    used positions by \cs{zrefused}
%    in order to notify \hologo{LaTeX} about undefined references.
%
%    \begin{macro}{\ZREF@savepos@ok}
%    \begin{macrocode}
\let\ZREF@savepos@ok=Y
%    \end{macrocode}
%    \end{macro}
%
%    \begin{macrocode}
%</savepos>
%    \end{macrocode}
%
%
% \subsection{Module \xmodule{abspos}}
%
% \subsubsection{Identification}
%
%    \begin{macrocode}
%<*abspos>
\NeedsTeXFormat{LaTeX2e}
\ProvidesPackage{zref-abspos}%
  [2020-07-03 v2.32 Module abspos for zref (HO)]%
\RequirePackage{zref-base}[2019/11/29]
\ifx\ZREF@base@ok Y%
\else
  \expandafter\endinput
\fi
%    \end{macrocode}
%
%    \begin{macrocode}
\RequirePackage{zref-savepos}[2019/11/29]
\ifx\ZREF@savepos@ok Y%
\else
  \expandafter\endinput
\fi
%    \end{macrocode}
%    \begin{macrocode}
\RequirePackage{zref-pagelayout}[2019/11/29]
\zref@addprop{savepos}{abspage}
\zref@addprop{savepos}{pagevalue}
%    \end{macrocode}
%
%    \begin{macro}{\zref@absposx}
%    \begin{macrocode}
\newcommand*{\zref@absposx}[3]{%
  \number
    \expandafter\zref@absposnumx\expandafter{%
      \number\zref@extractdefault{#1}{abspage}{0}%
    }{#2}{#3}%
  \ltx@space
}
%    \end{macrocode}
%    \end{macro}
%
%    \begin{macro}{\zref@absposy}
%    \begin{macrocode}
\newcommand*{\zref@absposy}[3]{%
  \number
  \expandafter\zref@absposnumy\expandafter{%
    \number\zref@extractdefault{#1}{abspage}{0}%
  }{#2}{#3}%
  \ltx@space
}
%    \end{macrocode}
%    \end{macro}
%
%    \begin{macro}{\zref@absposnumx}
%    \begin{macrocode}
\newcommand*{\zref@absposnumx}[3]{%
  \number
%  \ifnum#1>\ltx@zero
%    \zref@ifrefundefined{thepage#1}{%
%      0%
%    }{%
      \numexpr\ZREF@absposnum{thepage#1}{#2}{x}{#3}\relax
%    }%
%  \else
%    0%
%  \fi
}
%    \end{macrocode}
%    \end{macro}
%
%    \begin{macro}{\zref@absposnumy}
%    \begin{macrocode}
\newcommand*{\zref@absposnumy}[3]{%
  \number
%  \ifnum#1>\ltx@zero
%    \zref@ifrefundefined{thepage#1}{%
%      0%
%    }{%
      \numexpr\ZREF@absposnum{thepage#1}{#2}{y}{#3}\relax
%    }%
%  \else
%    0%
%  \fi
}
%    \end{macrocode}
%    \end{macro}
%
%    \begin{macro}{\ZREF@absposnum}
%    \begin{macrocode}
\def\ZREF@absposnum#1#2#3#4{%
  \ltx@ifundefined{ZREF@abspos@#2@#3@#4}{%
    0%
  }{%
    \csname ZREF@abspos@#2@#3@#4\endcsname{#1}%
  }%
}
%    \end{macrocode}
%    \end{macro}
%
%    \begin{macro}{\zref@def@absposx}
%    \begin{macrocode}
\ZREF@Robust\def\zref@def@absposx#1{%
  \zref@wrapper@babel{\ZREF@def@abspos{#1}\zref@absposx}%
}
%    \end{macrocode}
%    \end{macro}
%
%    \begin{macro}{\zref@def@absposy}
%    \begin{macrocode}
\ZREF@Robust\def\zref@def@absposy#1{%
  \zref@wrapper@babel{\ZREF@def@abspos{#1}\zref@absposy}%
}
%    \end{macrocode}
%    \end{macro}
%
%    \begin{macro}{\zref@def@absposnumx}
%    \begin{macrocode}
\ZREF@Robust\def\zref@def@absposnumx#1{%
  \zref@wrapper@babel{\ZREF@def@abspos{#1}\zref@absposnumx}%
}
%    \end{macrocode}
%    \end{macro}
%
%    \begin{macro}{\zref@def@absposnumy}
%    \begin{macrocode}
\ZREF@Robust\def\zref@def@absposnumy#1{%
  \zref@wrapper@babel{\ZREF@def@abspos{#1}\zref@absposnumy}%
}
%    \end{macrocode}
%    \end{macro}
%    \begin{macro}{\ZREF@def@abspos}
%    \begin{macrocode}
\def\ZREF@def@absposnumy#1#2#3#4#5{%
  \edef#1{#2{#3}{#4}{#5}}%
}
%    \end{macrocode}
%    \end{macro}
%
%    \begin{macro}{\zref@absposused}
%    \begin{macrocode}
\ZREF@Robust\def\zref@absposused{%
  \zref@wrapper@babel\ZREF@abspos@used
}
%    \end{macrocode}
%    \end{macro}
%    \begin{macro}{\ZREF@abspos@used}
%    \begin{macrocode}
\def\ZREF@abspos@used#1{%
  \zref@refused{#1}%
  \zref@ifrefundefined{#1}{%
  }{%
    \begingroup
      \edef\ZREF@temp{%
        \zref@extractdefault{#1}{abspage}{0}%
      }%
      \ifnum\ZREF@temp>\ltx@zero
        \zref@refused{thepage\ZREF@temp}%
      \else
        \@PackageError{zref-abspos}{%
          \string\zref@pos@label@used\ltx@space
          needs property `abspage'\MessageBreak
          in label `#1'%
        }\@ehc
      \fi
    \endgroup
  }%
}
%    \end{macrocode}
%    \end{macro}
%
%    \begin{macro}{\zref@absposnumused}
%    \begin{macrocode}
\newcommand*{\zref@absposnumused}[1]{%
  \ifnum#1>\ltx@zero
    \zref@refused{thepage\number#1}%
  \else
    \@PackageError{zref-abspos}{%
      Invalid absolute page number (#1)\MessageBreak
      for \string\zref@pos@num@used.\MessageBreak
      A positive integer is expected%
    }\@ehc
  \fi
}
%    \end{macrocode}
%    \end{macro}
%
%    \begin{macro}{\zref@ifabsposundefined}
%    \begin{macrocode}
\def\zref@ifabsposundefined#1{%
  \zref@ifrefundefined{#1}\ltx@firsttwo{%
    \expandafter\zref@ifabsposnumundefined\expandafter{%
      \number\zref@extractdefault{#1}{abspage}{0}%
    }%
  }%
}
%    \end{macrocode}
%    \end{macro}
%
%    \begin{macro}{\zref@ifabsposnumundefined}
%    \begin{macrocode}
\def\zref@ifabsposnumundefined#1{%
  \ifnum\ZREF@number{#1}>\ltx@zero
    \zref@ifrefundefined{thepage#1}%
        \ltx@firstoftwo\ltx@secondoftwo
  \else
    \expandafter\ltx@firstoftwo
  \fi
}
%    \end{macrocode}
%    \end{macro}
%
% \subsubsection{Media}
%
%    \begin{macro}{\ZREF@abspos@media@width}
%    \begin{macrocode}
\edef\ZREF@abspos@media@width{%
  \ltx@ifundefined{pdfpagewidth}{%
    \ltx@ifundefined{mediawidth}{%
      \ltx@ifundefined{stockwidth}{%
        paperwidth%
      }{%
        stockwidth%
      }%
    }{%
      mediawidth%
    }%
  }{%
    pdfpagewidth%
  }%
}
\ifluatex
\def\ZREF@abspos@media@width{pdfpagewidth}%
\fi
%    \end{macrocode}
%    \end{macro}
%    \begin{macro}{\ZREF@abspos@media@height}
%    \begin{macrocode}
\edef\ZREF@abspos@media@height{%
  \ltx@ifundefined{pdfpageheight}{%
    \ltx@ifundefined{mediaheight}{%
      \ltx@ifundefined{stockheight}{%
        paperheight%
      }{%
        stockheight%
      }%
    }{%
      mediaheight%
    }%
  }{%
    \noexpand\ifcase\pdfpageheight
      \ltx@ifundefined{stockheight}{%
        paperheight%
      }{%
        stockheight%
      }%
    \noexpand\else
      pdfpageheight%
    \noexpand\fi
  }%
}
\ifluatex
\edef\ZREF@abspos@media@height{%
   \noexpand\ifcase\pageheight
      \ltx@ifundefined{stockheight}{%
        paperheight%
      }{%
        stockheight%
      }%
    \noexpand\else
      pdfpageheight%
    \noexpand\fi}%
\fi
%    \end{macrocode}
%    \end{macro}
%
%    \begin{macro}{\ZREF@abspos@media@x@left}
%    \begin{macrocode}
\def\ZREF@abspos@media@x@left#1{%
  0%
}
%    \end{macrocode}
%    \end{macro}
%    \begin{macro}{\ZREF@abspos@media@x@right}
%    \begin{macrocode}
\def\ZREF@abspos@media@x@right#1{%
  \zref@extract{#1}\ZREF@abspos@media@width
}
%    \end{macrocode}
%    \end{macro}
%    \begin{macro}{\ZREF@abspos@media@x@center}
%    \begin{macrocode}
\def\ZREF@abspos@media@x@center#1{%
  \ZREF@abspos@media@x@left{#1}%
  +\zref@extract{#1}\ZREF@abspos@media@width/2%
}
%    \end{macrocode}
%    \end{macro}
%
%    \begin{macro}{\ZREF@abspos@media@y@top}
%    \begin{macrocode}
\def\ZREF@abspos@media@y@top#1{%
  \zref@extract{#1}\ZREF@abspos@media@height
}
%    \end{macrocode}
%    \end{macro}
%    \begin{macro}{\ZREF@abspos@media@y@bottom}
%    \begin{macrocode}
\def\ZREF@abspos@media@y@bottom#1{%
  0%
}
%    \end{macrocode}
%    \end{macro}
%    \begin{macro}{\ZREF@abspos@media@y@center}
%    \begin{macrocode}
\def\ZREF@abspos@media@y@center#1{%
  \zref@extract{#1}\ZREF@abspos@media@height/2%
}
%    \end{macrocode}
%    \end{macro}
%
% \subsubsection{Paper}
%
%    \begin{macro}{\ZREF@abspos@paper@x@left}
%    \begin{macrocode}
\def\ZREF@abspos@paper@x@left#1{%
  0%
}
%    \end{macrocode}
%    \end{macro}
%    \begin{macro}{\ZREF@abspos@paper@x@right}
%    \begin{macrocode}
\def\ZREF@abspos@paper@x@right#1{%
  \zref@extract{#1}{paperwidth}%
}
%    \end{macrocode}
%    \end{macro}
%    \begin{macro}{\ZREF@abspos@paper@x@center}
%    \begin{macrocode}
\def\ZREF@abspos@paper@x@center#1{%
  \zref@extract{#1}{paperwidth}/2%
}
%    \end{macrocode}
%    \end{macro}
%
%    \begin{macro}{\ZREF@abspos@paper@y@top}
%    \begin{macrocode}
\let\ZREF@abspos@paper@y@top\ZREF@abspos@media@y@top
%    \end{macrocode}
%    \end{macro}
%    \begin{macro}{\ZREF@abspos@paper@y@bottom}
%    \begin{macrocode}
\def\ZREF@abspos@paper@y@bottom#1{%
  \ZREF@abspos@paper@y@top{#1}%
  -\zref@extract{#1}{paperheight}%
}
%    \end{macrocode}
%    \end{macro}
%    \begin{macro}{\ZREF@abspos@paper@y@center}
%    \begin{macrocode}
\def\ZREF@abspos@paper@y@center#1{%
  \ZREF@abspos@paper@y@top{#1}%
  -\zref@extract{#1}{paperheight}/2%
}
%    \end{macrocode}
%    \end{macro}
%
% \subsubsection{Origin}
%
% There doesn't seem a good reason to make these tests depend on pdf mode
%  in current engines, so comment out the \verb|\ifpdf| tests.
%    \begin{macro}{\ZREF@abspos@origin@x}
%    \begin{macrocode}
\let\ZREF@temp\ltx@two
\ltx@IfUndefined{pdfhorigin}{}{%
%  \ifpdf
    \let\ZREF@temp\ltx@zero
%  \fi
}
\ifluatex
% \ifpdf
  \let\ZREF@temp\ltx@zero
% \fi
\fi

\ifx\ZREF@temp\ltx@two
  \ifnum\mag=1000 %
    \let\ZREF@temp\ltx@one
  \fi
\fi
\ifcase\ZREF@temp
  \def\ZREF@abspos@origin@x#1{%
    \zref@extract{#1}{pdfhorigin}%
  }%
\or
  \def\ZREF@abspos@origin@x#1{%
    4736286%
  }%
\or
  \def\ZREF@abspos@origin@x#1{%
    \numexpr\mag/1000*\dimexpr 1truein\relax\relax
  }%
\fi
%    \end{macrocode}
%    \end{macro}
%    \begin{macro}{\ZREF@abspos@origin@y}
%    \begin{macrocode}
\let\ZREF@temp\ltx@two
\ltx@IfUndefined{pdfvorigin}{}{%
%  \ifpdf
    \let\ZREF@temp\ltx@zero
%  \fi
}
\ifluatex
%  \ifpdf
    \let\ZREF@temp\ltx@zero
%  \fi
\fi
\ifx\ZREF@temp\ltx@two
  \ifnum\mag=1000 %
    \let\ZREF@temp\ltx@one
  \fi
\fi
\ifcase\ZREF@temp
  \def\ZREF@abspos@origin@y#1{%
    \zref@extract{#1}{pdfvorigin}%
  }%
\or
  \def\ZREF@abspos@origin@y#1{%
    4736286%
  }%
\or
  \def\ZREF@abspos@origin@y#1{%
    \numexpr\mag/1000*\dimexpr 1truein\relax\relax
   }%
\fi
%    \end{macrocode}
%    \end{macro}
%
% \subsubsection{Header}
%
%    \begin{macro}{\ZREF@abspos@head@x@left}
%    \begin{macrocode}
\def\ZREF@abspos@head@x@left#1{%
  \ZREF@abspos@paper@x@left{#1}%
  +\ZREF@abspos@origin@x{#1}%
  +\zref@extract{#1}{hoffset}%
  +\ifodd\zref@extractdefault{#1}{pagevalue}{\number\c@page} %
     \zref@extract{#1}{oddsidemargin}%
   \else
     \zref@extract{#1}{evensidemargin}%
   \fi
}
%    \end{macrocode}
%    \end{macro}
%    \begin{macro}{\ZREF@abspos@head@x@right}
%    \begin{macrocode}
\def\ZREF@abspos@head@x@right#1{%
  \ZREF@abspos@head@x@left{#1}%
  +\zref@extract{#1}{textwidth}%
}
%    \end{macrocode}
%    \end{macro}
%    \begin{macro}{\ZREF@abspos@head@x@center}
%    \begin{macrocode}
\def\ZREF@abspos@head@x@center#1{%
  \ZREF@abspos@head@x@left{#1}%
  +\zref@extract{#1}{textwidth}/2%
}
%    \end{macrocode}
%    \end{macro}
%
%    \begin{macro}{\ZREF@abspos@head@y@top}
%    \begin{macrocode}
\def\ZREF@abspos@head@y@top#1{%
  \ZREF@abspos@paper@y@top{#1}%
  -\ZREF@abspos@origin@y{#1}%
  -\zref@extract{#1}{voffset}%
  -\zref@extract{#1}{topmargin}%
}
%    \end{macrocode}
%    \end{macro}
%    \begin{macro}{\ZREF@abspos@head@y@bottom}
%    \begin{macrocode}
\def\ZREF@abspos@head@y@bottom#1{%
  \ZREF@abspos@head@y@top{#1}%
  -\zref@extract{#1}{headheight}%
}
%    \end{macrocode}
%    \end{macro}
%    \begin{macro}{\ZREF@abspos@head@y@center}
%    \begin{macrocode}
\def\ZREF@abspos@head@y@center#1{%
  \ZREF@abspos@head@y@top{#1}%
  -\zref@extract{#1}{headheight}/2%
}
%    \end{macrocode}
%    \end{macro}
%
% \subsubsection{Body}
%
%    \begin{macro}{\ZREF@abspos@body@x@left}
%    \begin{macrocode}
\let\ZREF@abspos@body@x@left\ZREF@abspos@head@x@left
%    \end{macrocode}
%    \end{macro}
%    \begin{macro}{\ZREF@abspos@body@x@right}
%    \begin{macrocode}
\let\ZREF@abspos@body@x@right\ZREF@abspos@head@x@right
%    \end{macrocode}
%    \end{macro}
%    \begin{macro}{\ZREF@abspos@body@x@center}
%    \begin{macrocode}
\let\ZREF@abspos@body@x@center\ZREF@abspos@head@x@center
%    \end{macrocode}
%    \end{macro}
%
%    \begin{macro}{\ZREF@abspos@body@y@top}
%    \begin{macrocode}
\def\ZREF@abspos@body@y@top#1{%
  \ZREF@abspos@head@y@bottom{#1}%
  -\zref@extract{#1}{headsep}%
}
%    \end{macrocode}
%    \end{macro}
%    \begin{macro}{\ZREF@abspos@body@y@bottom}
%    \begin{macrocode}
\def\ZREF@abspos@body@y@bottom#1{%
  \ZREF@abspos@body@y@top{#1}%
  -\zref@extract{#1}{textheight}%
}
%    \end{macrocode}
%    \end{macro}
%    \begin{macro}{\ZREF@abspos@body@y@center}
%    \begin{macrocode}
\def\ZREF@abspos@body@y@center#1{%
  \ZREF@abspos@body@y@top{#1}%
  -\zref@extract{#1}{textheight}/2%
}
%    \end{macrocode}
%    \end{macro}
%
% \subsubsection{Footer}
%
%    \begin{macro}{\ZREF@abspos@foot@x@left}
%    \begin{macrocode}
\let\ZREF@abspos@foot@x@left\ZREF@abspos@head@x@left
%    \end{macrocode}
%    \end{macro}
%    \begin{macro}{\ZREF@abspos@foot@x@right}
%    \begin{macrocode}
\let\ZREF@abspos@foot@x@right\ZREF@abspos@head@x@right
%    \end{macrocode}
%    \end{macro}
%    \begin{macro}{\ZREF@abspos@foot@x@center}
%    \begin{macrocode}
\let\ZREF@abspos@foot@x@center\ZREF@abspos@head@x@center
%    \end{macrocode}
%    \end{macro}
%
%    \begin{macro}{\ZREF@abspos@foot@y@bottom}
%    \begin{macrocode}
\def\ZREF@abspos@foot@y@bottom#1{%
  \ZREF@abspos@body@y@bottom{#1}%
  -\zref@extract{#1}{footskip}%
}
%    \end{macrocode}
%    \end{macro}
%
% \subsubsection{Marginal notes}
%
%    \begin{macro}{\ZREF@abspos@marginpar@x@left}
%    \begin{macrocode}
\def\ZREF@abspos@marginpar@x@left#1{%
  \ifodd\zref@extractdefault{#1}{pagevalue}{\number\c@page} %
    \ZREF@abspos@body@x@right{#1}%
    +\zref@extract{#1}{marginparsep}%
  \else
    \ZREF@abspos@body@x@left{#1}%
    -\zref@extract{#1}{marginparsep}%
    -\zref@extract{#1}{marginparwidth}%
  \fi
}
%    \end{macrocode}
%    \end{macro}
%    \begin{macro}{\ZREF@abspos@marginpar@x@right}
%    \begin{macrocode}
\def\ZREF@abspos@marginpar@x@right#1{%
  \ZREF@abspos@marginpar@x@left{#1}%
  +\zref@extract{#1}{marginparwidth}%
}
%    \end{macrocode}
%    \end{macro}
%    \begin{macro}{\ZREF@abspos@marginpar@x@center}
%    \begin{macrocode}
\def\ZREF@abspos@marginpar@x@center#1{%
  \ZREF@abspos@marginpar@x@left{#1}%
  +\zref@extract{#1}{marginparwidth}/2%
}
%    \end{macrocode}
%    \end{macro}
%
%    \begin{macro}{\ZREF@abspos@marginpar@y@top}
%    \begin{macrocode}
\let\ZREF@abspos@marginpar@y@top\ZREF@abspos@body@y@top
%    \end{macrocode}
%    \end{macro}
%    \begin{macro}{\ZREF@abspos@marginpar@y@bottom}
%    \begin{macrocode}
\let\ZREF@abspos@marginpar@y@bottom\ZREF@abspos@body@y@bottom
%    \end{macrocode}
%    \end{macro}
%    \begin{macro}{\ZREF@abspos@marginpar@y@center}
%    \begin{macrocode}
\let\ZREF@abspos@marginpar@y@center\ZREF@abspos@body@y@center
%    \end{macrocode}
%    \end{macro}
%
% \subsubsection{Stock paper}
%
%    \begin{macro}{\ZREF@abspos@stock@x@left}
%    \begin{macrocode}
\let\ZREF@abspos@stock@x@left\ZREF@abspos@paper@x@left
%    \end{macrocode}
%    \end{macro}
%    \begin{macro}{\ZREF@abspos@stock@x@right}
%    \begin{macrocode}
\let\ZREF@abspos@stock@x@right\ZREF@abspos@paper@x@right
%    \end{macrocode}
%    \end{macro}
%    \begin{macro}{\ZREF@abspos@stock@x@center}
%    \begin{macrocode}
\let\ZREF@abspos@stock@x@center\ZREF@abspos@paper@x@center
%    \end{macrocode}
%    \end{macro}
%
%    \begin{macro}{\ZREF@abspos@stock@y@top}
%    \begin{macrocode}
\let\ZREF@abspos@stock@y@top\ZREF@abspos@paper@y@top
%    \end{macrocode}
%    \end{macro}
%    \begin{macro}{\ZREF@abspos@stock@y@bottom}
%    \begin{macrocode}
\let\ZREF@abspos@stock@y@bottom\ZREF@abspos@paper@y@bottom
%    \end{macrocode}
%    \end{macro}
%    \begin{macro}{\ZREF@abspos@stock@y@center}
%    \begin{macrocode}
\let\ZREF@abspos@stock@y@center\ZREF@abspos@paper@y@center
%    \end{macrocode}
%    \end{macro}
%
%    \begin{macrocode}
%</abspos>
%    \end{macrocode}
%
%
% \subsection{Module \xmodule{dotfill}}
%
%    \begin{macrocode}
%<*dotfill>
\NeedsTeXFormat{LaTeX2e}
\ProvidesPackage{zref-dotfill}%
  [2020-07-03 v2.32 Module dotfill for zref (HO)]%
\RequirePackage{zref-base}[2019/11/29]
\ifx\ZREF@base@ok Y%
\else
  \expandafter\endinput
\fi
%    \end{macrocode}
%
%    For measuring the width of \cs{zdotfill} we use the features
%    provided by module \xmodule{savepos}.
%    \begin{macrocode}
\RequirePackage{zref-savepos}[2019/11/29]
%    \end{macrocode}
%    For automatically generated label names we use the unique
%    counter of module \xmodule{base}.
%    \begin{macrocode}
\zref@require@unique
%    \end{macrocode}
%    Configuration is done by the key value interface of package
%    \xpackage{keyval}.
%    \begin{macrocode}
\RequirePackage{keyval}
%    \end{macrocode}
%    The definitions of the keys follow.
%    \begin{macrocode}
\define@key{ZREF@DF}{unit}{%
  \def\ZREF@df@unit{#1}%
}
\define@key{ZREF@DF}{min}{%
  \def\ZREF@df@min{#1}%
}
\define@key{ZREF@DF}{dot}{%
  \def\ZREF@df@dot{#1}%
}
%    \end{macrocode}
%    Defaults are set, see user interface.
%    \begin{macrocode}
\providecommand\ZREF@df@min{2}
\providecommand\ZREF@df@unit{.44em}
\providecommand\ZREF@df@dot{.}
%    \end{macrocode}
%    \begin{macro}{\zdotfillsetup}
%    Configuration of \cs{zdotfill} is done by \cs{zdotfillsetup}.
%    \begin{macrocode}
\newcommand*{\zdotfillsetup}{\kvsetkeys{ZREF@DF}}
%    \end{macrocode}
%    \end{macro}
%
%    \begin{macro}{\zdotfill}
%    \cs{zdotfill} sets labels at the left and the right to get
%    the horizontal position. \cs{zsavepos} is not used, because
%    we do not need the vertical position.
%    \begin{macrocode}
\ZREF@IfDefinable\zdotfill\def{%
  {%
    \leavevmode
    \global\advance\c@zref@unique\ltx@one
    \begingroup
      \def\ZREF@temp{zref@\number\c@zref@unique}%
      \pdfsavepos
      \zref@labelbyprops{\thezref@unique L}{posx}%
      \setlength{\dimen@}{\ZREF@df@unit}%
      \zref@ifrefundefined{\thezref@unique R}{%
        \ZREF@dotfill
      }{%
        \ifnum\numexpr\zposx{\thezref@unique R}%
                      -\zposx{\thezref@unique L}\relax
            <\dimexpr\ZREF@df@min\dimen@\relax
          \hfill
        \else
          \ZREF@dotfill
        \fi
      }%
      \pdfsavepos
      \zref@labelbyprops{\thezref@unique R}{posx}%
    \endgroup
    \kern\z@
  }%
}
%    \end{macrocode}
%    \end{macro}
%    \begin{macro}{\ZREF@dotfill}
%    Help macro that actually sets the dots.
%    \begin{macrocode}
\def\ZREF@dotfill{%
  \cleaders\hb@xt@\dimen@{\hss\ZREF@df@dot\hss}\hfill
}
%    \end{macrocode}
%    \end{macro}
%
%    \begin{macrocode}
%</dotfill>
%    \end{macrocode}
%
% \subsection{Module \xmodule{env}}
%
%    \begin{macrocode}
%<*env>
\NeedsTeXFormat{LaTeX2e}
\ProvidesPackage{zref-env}%
  [2020-07-03 v2.32 Module env for zref (HO)]%
\RequirePackage{zref-base}[2019/11/29]
\ifx\ZREF@base@ok Y%
\else
  \expandafter\endinput
\fi
%    \end{macrocode}
%
%    \begin{macrocode}
\zref@newprop{envname}[]{\@currenvir}
\zref@newprop{envline}[]{\zref@env@line}
%    \end{macrocode}
%
%    \begin{macro}{\zref@env@line}
%    Macro \cs{zref@env@line} extracts the line number
%    from \cs{@currenvline}.
%    \begin{macrocode}
\def\zref@env@line{%
  \ifx\@currenvline\ltx@empty
  \else
    \expandafter
    \ZREF@ENV@line\@currenvline\ltx@empty line \ltx@empty\@nil
  \fi
}
%    \end{macrocode}
%    \end{macro}
%    \begin{macro}{\ZREF@ENV@line}
%    \begin{macrocode}
\def\ZREF@ENV@line#1line #2\ltx@empty#3\@nil{#2}%
%    \end{macrocode}
%    \end{macro}
%
%    \begin{macrocode}
%</env>
%    \end{macrocode}
%
%% \section{Installation}
%
% \subsection{Download}
%
% \paragraph{Package.} This package is available on
% CTAN\footnote{\CTANpkg{zref}}:
% \begin{description}
% \item[\CTAN{macros/latex/contrib/zref/zref.dtx}] The source file.
% \item[\CTAN{macros/latex/contrib/zref/zref.pdf}] Documentation.
% \end{description}
%
%
% \paragraph{Bundle.} All the packages of the bundle `zref'
% are also available in a TDS compliant ZIP archive. There
% the packages are already unpacked and the documentation files
% are generated. The files and directories obey the TDS standard.
% \begin{description}
% \item[\CTANinstall{install/macros/latex/contrib/zref.tds.zip}]
% \end{description}
% \emph{TDS} refers to the standard ``A Directory Structure
% for \TeX\ Files'' (\CTANpkg{tds}). Directories
% with \xfile{texmf} in their name are usually organized this way.
%
% \subsection{Bundle installation}
%
% \paragraph{Unpacking.} Unpack the \xfile{zref.tds.zip} in the
% TDS tree (also known as \xfile{texmf} tree) of your choice.
% Example (linux):
% \begin{quote}
%   |unzip zref.tds.zip -d ~/texmf|
% \end{quote}
%
%
% \subsection{Package installation}
%
% \paragraph{Unpacking.} The \xfile{.dtx} file is a self-extracting
% \docstrip\ archive. The files are extracted by running the
% \xfile{.dtx} through \plainTeX:
% \begin{quote}
%   \verb|tex zref.dtx|
% \end{quote}
%
% \paragraph{TDS.} Now the different files must be moved into
% the different directories in your installation TDS tree
% (also known as \xfile{texmf} tree):
% \begin{quote}
% \def\t{^^A
% \begin{tabular}{@{}>{\ttfamily}l@{ $\rightarrow$ }>{\ttfamily}l@{}}
%   zref.sty & tex/latex/zref/zref.sty\\
%   zref-base.sty & tex/latex/zref/zref-base.sty\\
%   zref-abspage.sty & tex/latex/zref/zref-abspage.sty\\
%   zref-abspos.sty & tex/latex/zref/zref-abspos.sty\\
%   zref-counter.sty & tex/latex/zref/zref-counter.sty\\
%   zref-dotfill.sty & tex/latex/zref/zref-dotfill.sty\\
%   zref-env.sty & tex/latex/zref/zref-env.sty\\
%   zref-hyperref.sty & tex/latex/zref/zref-hyperref.sty\\
%   zref-lastpage.sty & tex/latex/zref/zref-lastpage.sty\\
%   zref-marks.sty & tex/latex/zref/zref-marks.sty\\
%   zref-nextpage.sty & tex/latex/zref/zref-nextpage.sty\\
%   zref-pageattr.sty & tex/latex/zref/zref-pageattr.sty\\
%   zref-pagelayout.sty & tex/latex/zref/zref-pagelayout.sty\\
%   zref-perpage.sty & tex/latex/zref/zref-perpage.sty\\
%   zref-runs.sty & tex/latex/zref/zref-runs.sty\\
%   zref-savepos.sty & tex/latex/zref/zref-savepos.sty\\
%   zref-thepage.sty & tex/latex/zref/zref-thepage.sty\\
%   zref-titleref.sty & tex/latex/zref/zref-titleref.sty\\
%   zref-totpages.sty & tex/latex/zref/zref-totpages.sty\\
%   zref-user.sty & tex/latex/zref/zref-user.sty\\
%   zref-xr.sty & tex/latex/zref/zref-xr.sty\\
%   zref.pdf & doc/latex/zref/zref.pdf\\
%   zref-example.tex & doc/latex/zref/zref-example.tex\\
%   zref-example-lastpage.tex & doc/latex/zref/zref-example-lastpage.tex\\
%   zref-example-nextpage.tex & doc/latex/zref/zref-example-nextpage.tex\\
%   zref.dtx & source/latex/zref/zref.dtx\\
% \end{tabular}^^A
% }^^A
% \sbox0{\t}^^A
% \ifdim\wd0>\linewidth
%   \begingroup
%     \advance\linewidth by\leftmargin
%     \advance\linewidth by\rightmargin
%   \edef\x{\endgroup
%     \def\noexpand\lw{\the\linewidth}^^A
%   }\x
%   \def\lwbox{^^A
%     \leavevmode
%     \hbox to \linewidth{^^A
%       \kern-\leftmargin\relax
%       \hss
%       \usebox0
%       \hss
%       \kern-\rightmargin\relax
%     }^^A
%   }^^A
%   \ifdim\wd0>\lw
%     \sbox0{\small\t}^^A
%     \ifdim\wd0>\linewidth
%       \ifdim\wd0>\lw
%         \sbox0{\footnotesize\t}^^A
%         \ifdim\wd0>\linewidth
%           \ifdim\wd0>\lw
%             \sbox0{\scriptsize\t}^^A
%             \ifdim\wd0>\linewidth
%               \ifdim\wd0>\lw
%                 \sbox0{\tiny\t}^^A
%                 \ifdim\wd0>\linewidth
%                   \lwbox
%                 \else
%                   \usebox0
%                 \fi
%               \else
%                 \lwbox
%               \fi
%             \else
%               \usebox0
%             \fi
%           \else
%             \lwbox
%           \fi
%         \else
%           \usebox0
%         \fi
%       \else
%         \lwbox
%       \fi
%     \else
%       \usebox0
%     \fi
%   \else
%     \lwbox
%   \fi
% \else
%   \usebox0
% \fi
% \end{quote}
% If you have a \xfile{docstrip.cfg} that configures and enables \docstrip's
% TDS installing feature, then some files can already be in the right
% place, see the documentation of \docstrip.
%
% \subsection{Refresh file name databases}
%
% If your \TeX~distribution
% (\TeX\,Live, \mikTeX, \dots) relies on file name databases, you must refresh
% these. For example, \TeX\,Live\ users run \verb|texhash| or
% \verb|mktexlsr|.
%
% \subsection{Some details for the interested}
%
% \paragraph{Unpacking with \LaTeX.}
% The \xfile{.dtx} chooses its action depending on the format:
% \begin{description}
% \item[\plainTeX:] Run \docstrip\ and extract the files.
% \item[\LaTeX:] Generate the documentation.
% \end{description}
% If you insist on using \LaTeX\ for \docstrip\ (really,
% \docstrip\ does not need \LaTeX), then inform the autodetect routine
% about your intention:
% \begin{quote}
%   \verb|latex \let\install=y\input{zref.dtx}|
% \end{quote}
% Do not forget to quote the argument according to the demands
% of your shell.
%
% \paragraph{Generating the documentation.}
% You can use both the \xfile{.dtx} or the \xfile{.drv} to generate
% the documentation. The process can be configured by the
% configuration file \xfile{ltxdoc.cfg}. For instance, put this
% line into this file, if you want to have A4 as paper format:
% \begin{quote}
%   \verb|\PassOptionsToClass{a4paper}{article}|
% \end{quote}
% An example follows how to generate the
% documentation with pdf\LaTeX:
% \begin{quote}
%\begin{verbatim}
%pdflatex zref.dtx
%makeindex -s gind.ist zref.idx
%pdflatex zref.dtx
%makeindex -s gind.ist zref.idx
%pdflatex zref.dtx
%\end{verbatim}
% \end{quote}
%
% \begin{thebibliography}{99}
% \bibpackage{footmisc}{Robin Fairbairns}{2004/01/23 v5.3a}
%   {\CTANpkg{footmisc}}
% \bibpackage{hyperref}{Sebastian Rahtz, Heiko Oberdiek}{2006/08/16 v6.75c}
%   {\CTANpkg{hyperref}}
% \bibpackage{lastpage}{Jeff Goldberg}{1994/06/25 v0.1b}
%   {\CTANpkg{lastpage}}
% \bibpackage{nameref}{Sebastian Rahtz, Heiko Oberdiek}{2006/02/12 v2.24}
%   {\CTANpkg{nameref}}
% \bibpackage{perpage}{David Kastrup}{2002/12/20 v1.0}
%   {\CTANpkg{perpage}}
% \bibpackage{titleref}{Donald Arsenau}{2001/04/05 v3.1}
%   {\CTANpkg{titleref}}
% \bibpackage{totpages}{Wilhelm M\"uller}{1999/07/14 v1.00}
%   {\CTANpkg{totpages}}
% \bibpackage{xr}{David Carlisle}{1994/05/28 v5.02}
%   {\CTANpkg{xr}}
% \bibpackage{xr-hyper}{David Carlisle}{2000/03/22 v6.00beta4}
%   {\CTANpkg{xr-hyper}}
% \end{thebibliography}
%
% \begin{History}
%   \begin{Version}{2006/02/20 v1.0}
%   \item
%     First version.
%   \end{Version}
%   \begin{Version}{2006/05/03 v1.1}
%   \item
%     Module \xmodule{perpage} added.
%   \item
%     Module redesign as packages.
%   \end{Version}
%   \begin{Version}{2006/05/25 v1.2}
%   \item
%     Module \xmodule{dotfillmin} added.
%   \item
%     Module \xmodule{base}: macros \cs{zref@require@uniqe}
%     and \cs{thezref@unique} added
%     (used by modules \xmodule{titleref} and \xmodule{dotfillmin}).
%   \end{Version}
%   \begin{Version}{2006/09/08 v1.3}
%   \item
%     Typo fixes and English cleanup by Per Starback.
%   \end{Version}
%   \begin{Version}{2007/01/23 v1.4}
%   \item
%     Typo in macro name fixed in documentation.
%   \end{Version}
%   \begin{Version}{2007/02/18 v1.5}
%   \item
%     \cs{zref@getcurrent} added (suggestion of Igor Akkerman).
%   \item
%     Module \xmodule{savepos} also supports \hologo{XeTeX}.
%   \end{Version}
%   \begin{Version}{2007/04/06 v1.6}
%   \item
%     Fix in modules \xmodule{abspage} and \xmodule{base}:
%     Now counter |abspage| and |zref@unique|
%     are not remembered by \cs{include}.
%   \item
%     Beamer support for module \xmodule{titleref}.
%   \end{Version}
%   \begin{Version}{2007/04/17 v1.7}
%   \item
%     Package \xpackage{atbegshi} replaces \xpackage{everyshi}.
%   \end{Version}
%   \begin{Version}{2007/04/22 v1.8}
%   \item
%     \cs{zref@wrapper@babel} and \cs{zref@refused} are now expandable
%     if babel is not used or \cs{if@safe@actives} is already set to true.
%     (Feature request of Josselin Noirel)
%   \end{Version}
%   \begin{Version}{2007/05/02 v1.9}
%   \item
%     Module \xmodule{titleref}: Some support for \cs{caption} of package
%     \xpackage{longtable}, but only if \cs{label} is given after
%     \cs{caption}.
%   \end{Version}
%   \begin{Version}{2007/05/06 v2.0}
%   \item
%     Uses package \xpackage{etexcmds} for accessing \hologo{eTeX}'s \cs{unexpanded}.
%   \end{Version}
%   \begin{Version}{2007/05/28 v2.1}
%   \item
%     Module \xmodule{titleref} supports caption of package \xpackage{listings}.
%   \item
%     Fixes in module \xmodule{titleref} for support
%     of packages \xpackage{titlesec} and \xpackage{longtable}.
%   \end{Version}
%   \begin{Version}{2008/09/21 v2.2}
%   \item
%     Module \xmodule{base}: \cs{zref@iflistcontainsprop} is documented, but
%     a broken \cs{zref@listcontainsprop} implemented. Name and
%     implementation fixed (thanks Ohad Kammar).
%   \end{Version}
%   \begin{Version}{2008/10/01 v2.3}
%   \item
%     \cs{zref@localaddprop} added (feature request of Ohad Kammar).
%   \item
%     Module \xmodule{lastpage}: list `LastPage' added. Label `LastPage' will use
%     the properties of this list (default is empty) along with the
%     properties of the main list.
%   \end{Version}
%   \begin{Version}{2009/08/07 v2.4}
%   \item
%     Module \xmodule{runs} added.
%   \end{Version}
%   \begin{Version}{2009/12/06 v2.5}
%   \item
%     Module \xmodule{lastpage}: Uses package \xpackage{atveryend}.
%   \item
%     Module \xmodule{titleref}: Further commands are disabled
%     during string expansion, imported from package \xpackage{nameref}.
%   \end{Version}
%   \begin{Version}{2009/12/07 v2.6}
%   \item
%     Version date added for package \xpackage{atveryend}.
%   \end{Version}
%   \begin{Version}{2009/12/08 v2.7}
%   \item
%     Module \xmodule{titleref}: Use of package \xpackage{gettitlestring}.
%   \end{Version}
%   \begin{Version}{2010/03/26 v2.8}
%   \item
%     \cs{zifrefundefined} added.
%   \item
%     Module \xmodule{lastpage}: Macros \cs{zref@iflastpage}
%     and \cs{ziflastpage} added.
%   \item
%     Module \xmodule{thepage} added.
%   \item
%     Module \xmodule{nextpage} added.
%   \end{Version}
%   \begin{Version}{2010/03/29 v2.9}
%   \item
%     Module \xmodule{marks} added (without documentation).
%   \item
%     \cs{zref@addprop} now adds expanded property to list.
%   \item
%     Useless \cs{ZREF@ErrorNoLine} removed.
%   \end{Version}
%   \begin{Version}{2010/04/08 v2.10}
%   \item
%     Module \xmodule{xr} remembers the external document name
%     in property `externaldocument'.
%   \end{Version}
%   \begin{Version}{2010/04/15 v2.11}
%   \item
%     Module \xmodule{titleref}: Better support of class \xclass{memoir}.
%   \item
%     Module \xmodule{titleref}: Support of theorems.
%   \end{Version}
%   \begin{Version}{2010/04/17 v2.12}
%   \item
%     Module \xmodule{base}: \cs{zref@newprop} ensures global
%     empty default.
%   \item
%     Module \xmodule{xr}: Setup options \xoption{tozreflabel}
%     and \xoption{toltxlabel} added.
%   \end{Version}
%   \begin{Version}{2010/04/19 v2.13}
%   \item
%     \cs{zref@setcurrent} throws an error if the property
%     does not exist (Florent Chervet).
%   \item
%     \cs{zref@getcurrent} the documentation is fixed (Florent Chervet).
%     Also it returns the empty string in case of errors.
%   \item
%     \cs{zref@addprop} and \cs{zref@localaddprop} now take a list of
%     property names (feature request of Florent Chervet).
%   \item
%     Example for \cs{zref@wrapper@unexpanded} corrected (Florent Chervet).
%   \end{Version}
%   \begin{Version}{2010/04/22 v2.14}
%   \item
%     Bug fix for \cs{zref@getcurrent} second argument wasn't eaten in
%     case of unknown property.
%   \item
%     \cs{zref@getcurrent} supports \cs{zref@wrapper@unexpanded}.
%   \item
%     \cs{zref@wrapper@unexpanded} added for \cs{ZREF@xr@tolabel}.
%   \item
%     \cs{zref@extract}, \cs{zref@extractdefault}, \cs{zref@getcurrent}
%     are expandable in exact two steps
%     except inside \cs{zref@wrapper@unexpanded}.
%   \end{Version}
%   \begin{Version}{2010/04/23 v2.15}
%   \item
%     \cs{zexternaldocument} fixed for property `url' when
%     importing \cs{new@label} (bug found by Victor Ivrii).
%   \item
%     Two expansion steps also in \cs{zref@wrapper@unexpanded}.
%   \item
%     Nested calls of \cs{zref@wrapper@unexpanded} possible.
%   \end{Version}
%   \begin{Version}{2010/04/28 v2.16}
%   \item
%     More consequent use of package `ltxcmds' and `hologo'.
%   \item
%     Module \xmodule{pagelayout} added.
%   \item
%     Module \xmodule{pageattr} added.
%   \item
%     Robustness introduced for non-expandable interface macros.
%   \item
%     Internal change of the data format of property lists
%     (suggestion of Florent Chervet).
%   \item
%     Module \xmodule{titleref}: Support of environment \textsf{description}.
%   \end{Version}
%   \begin{Version}{2010/05/01 v2.17}
%   \item
%     \cs{zref@newprop} throws an error if the property already exists.
%   \item
%     Module \xmodule{xr}: Bug fix for the case of several \xext{aux} files
%     (bug found by Victor Ivrii).
%   \item
%     Module \xmodule{xr}: Property `urluse' and option \xoption{urluse} added.
%   \end{Version}
%   \begin{Version}{2010/05/13 v2.18}
%   \item
%     Module \xmodule{env} added.
%   \item
%     Module \xmodule{savepos}: \cs{zref@savepos} added.
%   \end{Version}
%   \begin{Version}{2010/10/22 v2.19}
%   \item
%     \cs{zref@addprop} and \cs{zref@localaddprop} are limited
%     to one property only
%     (incompatibility to versions v2.13 to v2.18).
%   \item
%     \cs{zref@addprops} and \cs{zref@localaddprops} added.
%   \item
%     \cs{zref@delprop} and \cs{zref@localdelprop} added.
%   \item
%     \cs{zref@labelbykv} and \cs{zkvlabel} (module \xmodule{user})
%     with keys \xoption{prop}, \xoption{list}, \xoption{delprop},
%     \xoption{immediate}, \xoption{values} added.
%   \end{Version}
%   \begin{Version}{2011/02/12 v2.20}
%   \item
%     Fix for warning in \xpackage{zref-xr}.
%   \end{Version}
%   \begin{Version}{2011/03/18 v2.21}
%   \item
%     Fix in module \xmodule{pagelayout} for \cs{zlistpagelayout}.
%   \item
%     Fix for \cs{zref@localaddprop} (probably since v2.19).
%   \end{Version}
%   \begin{Version}{2011/10/05 v2.22}
%   \item
%     Documentation fixed for \cs{zref@(local)addprop(s)}.
%   \item
%     Module \xmodule{base}: \cs{zref@def@extract}, \cs{zref@def@extractdefault}
%     added.
%   \item
%     Fix in module \xmodule{pagelayout}: Because of missing \cs{noexpand}
%     commands the values of the pagelayout properties on all pages were
%     the values at package loading.
%   \item
%     Module \xmodule{base}: \cs{zref@showprop} added.
%   \end{Version}
%   \begin{Version}{2011/12/05 v2.23}
%   \item
%     Module \xmodule{savepos}: \cs{zsaveposx} and \cs{zsaveposy} added.
%   \end{Version}
%   \begin{Version}{2012/04/04 v2.24}
%   \item
%     Module \xmodule{titleref}, package \xpackage{titlesec}:
%     some support for class `straight' (\cs{ttl@straight@i}) added.
%   \end{Version}
%   \begin{Version}{2016/05/16 v2.25}
%   \item
%     Documentation updates.
%   \end{Version}
%   \begin{Version}{2016/05/21 v2.26}
%   \item
%     update zref-savepos for new luatex
%   \end{Version}
%   \begin{Version}{2018/11/21 v2.27}
%   \item
%     adapted zref-perpage, see issue https://github.com/ho-tex/zref/issues/2
%   \end{Version}
%   \begin{Version}{2019/11/29 v2.28}
%   \item
%     Documentation updates.
%   \item
%     Use \xpackage{iftex} directly.
%   \end{Version}
%   \begin{Version}{2020-03-03 v2.29}
%   \item
%     adapted in module zref-pagelayout the properties pdfhorigin, pdfvorigin,
%     pdfpagewidth, pdfpageheight for luatex to the right primitives.
%   \item
%     Removed no longer needed code for older lualatex versions.
%   \item added some documentation of the abspos module.
%   \item adapted the abspos module to the new luatex primitives.
%   \item adapted pageattr module to the new luatex primitives.
%   \item added short documentation for pageattr module
%   \item use luatex command names directly in zref-savepos rather than defining pdftex compatibility names.
%   \item allow zref-abspos to use \verb|\pdf[vh]origin| in dvi mode.
%   \end{Version}
%   \begin{Version}{2020-03-04 v2.30}
%   \item
%     add pagevalue property to savepos in the abspos module (issue 1)
%   \end{Version}
%   \begin{Version}{2020-05-28 v2.31}
%   \item Adapted module zref-counter to use \cs{@currentcounter} in the next
%   \LaTeX version (issue 5)
%   \end{Version}
%   \begin{Version}{2020-07-03 v2.32}
%   \item Changed in zref-pagelayout the names of the shipout box dimensions
%    to adapt to the new hook management.
%   \end{Version}
% \end{History}
%
% \PrintIndex
%
% \Finale
\endinput

%        (quote the arguments according to the demands of your shell)
%
% Documentation:
%    (a) If zref.drv is present:
%           latex zref.drv
%    (b) Without zref.drv:
%           latex zref.dtx; ...
%    The class ltxdoc loads the configuration file ltxdoc.cfg
%    if available. Here you can specify further options, e.g.
%    use A4 as paper format:
%       \PassOptionsToClass{a4paper}{article}
%
%    Programm calls to get the documentation (example):
%       pdflatex zref.dtx
%       makeindex -s gind.ist zref.idx
%       pdflatex zref.dtx
%       makeindex -s gind.ist zref.idx
%       pdflatex zref.dtx
%
% Installation:
%    TDS:tex/latex/zref/zref.sty
%    TDS:tex/latex/zref/zref-base.sty
%    TDS:tex/latex/zref/zref-abspage.sty
%    TDS:tex/latex/zref/zref-abspos.sty
%    TDS:tex/latex/zref/zref-counter.sty
%    TDS:tex/latex/zref/zref-dotfill.sty
%    TDS:tex/latex/zref/zref-env.sty
%    TDS:tex/latex/zref/zref-hyperref.sty
%    TDS:tex/latex/zref/zref-lastpage.sty
%    TDS:tex/latex/zref/zref-marks.sty
%    TDS:tex/latex/zref/zref-nextpage.sty
%    TDS:tex/latex/zref/zref-pageattr.sty
%    TDS:tex/latex/zref/zref-pagelayout.sty
%    TDS:tex/latex/zref/zref-perpage.sty
%    TDS:tex/latex/zref/zref-runs.sty
%    TDS:tex/latex/zref/zref-savepos.sty
%    TDS:tex/latex/zref/zref-thepage.sty
%    TDS:tex/latex/zref/zref-titleref.sty
%    TDS:tex/latex/zref/zref-totpages.sty
%    TDS:tex/latex/zref/zref-user.sty
%    TDS:tex/latex/zref/zref-xr.sty
%    TDS:doc/latex/zref/zref.pdf
%    TDS:doc/latex/zref/zref-example.tex
%    TDS:doc/latex/zref/zref-example-lastpage.tex
%    TDS:doc/latex/zref/zref-example-nextpage.tex
%    TDS:source/latex/zref/zref.dtx
%
%<*ignore>
\begingroup
  \catcode123=1 %
  \catcode125=2 %
  \def\x{LaTeX2e}%
\expandafter\endgroup
\ifcase 0\ifx\install y1\fi\expandafter
         \ifx\csname processbatchFile\endcsname\relax\else1\fi
         \ifx\fmtname\x\else 1\fi\relax
\else\csname fi\endcsname
%</ignore>
%<*install>
\input docstrip.tex
\Msg{************************************************************************}
\Msg{* Installation}
\Msg{* Package: zref 2020-07-03 v2.32 A new reference scheme for LaTeX (HO)}
\Msg{************************************************************************}

\keepsilent
\askforoverwritefalse

\let\MetaPrefix\relax
\preamble

This is a generated file.

Project: zref
Version: 2020-07-03 v2.32

Copyright (C)
  2006-2012 Heiko Oberdiek
  2016-2020 Oberdiek Package Support Group

This work may be distributed and/or modified under the
conditions of the LaTeX Project Public License, either
version 1.3c of this license or (at your option) any later
version. This version of this license is in
   https://www.latex-project.org/lppl/lppl-1-3c.txt
and the latest version of this license is in
   https://www.latex-project.org/lppl.txt
and version 1.3 or later is part of all distributions of
LaTeX version 2005/12/01 or later.

This work has the LPPL maintenance status "maintained".

The Current Maintainers of this work are
Heiko Oberdiek and the Oberdiek Package Support Group
https://github.com/ho-tex/zref/issues


This work consists of the main source file zref.dtx
and the derived files
   zref.sty, zref.pdf, zref.ins, zref.drv, zref-base.sty,
   zref-abspage.sty, zref-abspos.sty, zref-counter.sty,
   zref-dotfill.sty, zref-env.sty, zref-hyperref.sty,
   zref-lastpage.sty, zref-marks.sty, zref-nextpage.sty,
   zref-pageattr.sty, zref-pagelayout.sty, zref-perpage.sty,
   zref-runs.sty, zref-savepos.sty, zref-thepage.sty,
   zref-titleref.sty, zref-totpages.sty, zref-user.sty,
   zref-xr.sty, zref-example.tex, zref-example-lastpage.tex,
   zref-example-nextpage.tex, zref-test1.tex,
   zref-test-base.tex, zref-test-runs.tex,
   zref-test-titleref-memoir.tex.

\endpreamble
\let\MetaPrefix\DoubleperCent

\generate{%
  \file{zref.ins}{\from{zref.dtx}{install}}%
  \file{zref.drv}{\from{zref.dtx}{driver}}%
  \usedir{tex/latex/zref}%
  \file{zref.sty}{\from{zref.dtx}{package}}%
  \file{zref-base.sty}{\from{zref.dtx}{base}}%
  \file{zref-abspage.sty}{\from{zref.dtx}{abspage}}%
  \file{zref-abspos.sty}{\from{zref.dtx}{abspos}}%
  \file{zref-counter.sty}{\from{zref.dtx}{counter}}%
  \file{zref-dotfill.sty}{\from{zref.dtx}{dotfill}}%
  \file{zref-env.sty}{\from{zref.dtx}{env}}%
  \file{zref-hyperref.sty}{\from{zref.dtx}{hyperref}}%
  \file{zref-lastpage.sty}{\from{zref.dtx}{lastpage}}%
  \file{zref-marks.sty}{\from{zref.dtx}{marks}}%
  \file{zref-nextpage.sty}{\from{zref.dtx}{nextpage}}%
  \file{zref-pageattr.sty}{\from{zref.dtx}{pageattr}}%
  \file{zref-pagelayout.sty}{\from{zref.dtx}{pagelayout}}%
  \file{zref-perpage.sty}{\from{zref.dtx}{perpage}}%
  \file{zref-runs.sty}{\from{zref.dtx}{runs}}%
  \file{zref-savepos.sty}{\from{zref.dtx}{savepos}}%
  \file{zref-thepage.sty}{\from{zref.dtx}{thepage}}%
  \file{zref-titleref.sty}{\from{zref.dtx}{titleref}}%
  \file{zref-totpages.sty}{\from{zref.dtx}{totpages}}%
  \file{zref-user.sty}{\from{zref.dtx}{user}}%
  \file{zref-xr.sty}{\from{zref.dtx}{xr}}%
  \usedir{doc/latex/zref}%
  \file{zref-example.tex}{\from{zref.dtx}{example}}%
  \file{zref-example-lastpage.tex}{\from{zref.dtx}{example-lastpage}}%
  \file{zref-example-nextpage.tex}{\from{zref.dtx}{example-nextpage}}%
%  \usedir{doc/latex/zref/test}%
%  \file{zref-test1.tex}{\from{zref.dtx}{test1}}%
%  \file{zref-test-base.tex}{\from{zref.dtx}{test-base}}%
%  \file{zref-test-runs.tex}{\from{zref.dtx}{test-runs}}%
%  \file{zref-test-titleref-memoir.tex}{\from{zref.dtx}{test-titleref-memoir}}%
}

\catcode32=13\relax% active space
\let =\space%
\Msg{************************************************************************}
\Msg{*}
\Msg{* To finish the installation you have to move the following}
\Msg{* files into a directory searched by TeX:}
\Msg{*}
\Msg{*     zref.sty, zref-base.sty, zref-abspage.sty, zref-abspos.sty,}
\Msg{*     zref-counter.sty, zref-dotfill.sty, zref-env.sty,}
\Msg{*     zref-hyperref.sty, zref-lastpage.sty, zref-marks.sty,}
\Msg{*     zref-nextpage.sty, zref-pageattr.sty, zref-pagelayout.sty,}
\Msg{*     zref-perpage.sty, zref-runs.sty, zref-savepos.sty,}
\Msg{*     zref-thepage.sty, zref-titleref.sty, zref-totpages.sty,}
\Msg{*     zref-user.sty, zref-xr.sty}
\Msg{*}
\Msg{* To produce the documentation run the file `zref.drv'}
\Msg{* through LaTeX.}
\Msg{*}
\Msg{* Happy TeXing!}
\Msg{*}
\Msg{************************************************************************}

\endbatchfile
%</install>
%<*ignore>
\fi
%</ignore>
%<*driver>
\NeedsTeXFormat{LaTeX2e}
\ProvidesFile{zref.drv}%
  [2020-07-03 v2.32 A new reference scheme for LaTeX (HO)]%
\documentclass{ltxdoc}
\usepackage{holtxdoc}[2011/11/22]
\setlength{\emergencystretch}{1em}
\begin{document}
  \DocInput{zref.dtx}%
\end{document}
%</driver>
% \fi
%
%
%
% \GetFileInfo{zref.drv}
%
% \title{The \xpackage{zref} package}
% \date{2020-07-03 v2.32}
% \author{Heiko Oberdiek\thanks
% {Please report any issues at \url{https://github.com/ho-tex/zref/issues}}}
%
% \maketitle
%
% \begin{abstract}
% Package \xpackage{zref} tries to get rid of the restriction
% in \hologo{LaTeX}'s reference system that only two properties are
% supported. The package implements an extensible referencing
% system, where properties are handled in a more flexible way.
% It offers an interface for macro programmers for the access
% to the system and some applications that uses the new
% reference scheme.
% \end{abstract}
%
% \tableofcontents
%
% \newcommand*{\metacs}[1]{\texttt{\itshape #1}}
%
% \newcommand{\markexp}{^^A
%   \textsuperscript{exp}^^A
%   \space
% }
% \newcommand{\markexptwo}{^^A
%   \textsuperscript{exp2}^^A
%   \space
% }
% \newcommand{\markrobust}{^^A
%   \textsuperscript{robust}^^A
%   \space
% }
% \newcommand{\marknowarn}{^^A
%   \textsuperscript{nowarn}^^A
%   \space
% }
% \newcommand{\markbabel}{^^A
%   \textsuperscript{babel}^^A
%   \space
% }
%
% \section{Introduction}
%
% Standard \hologo{LaTeX}'s reference system with \cs{label}, \cs{ref},
% and \cs{pageref} supports two properties, the apperance of
% the counter that is last incremented by \cs{refstepcounter} and
% the page with the \cs{label} command.
%
% Unhappily \hologo{LaTeX} does not provide an interface for adding
% another properties. Packages such as \xpackage{hyperref},
% \xpackage{nameref}, or \xpackage{titleref} are forced to
% use ugly hacks to extend the reference system.
% These ugly hacks are one of the causes for \xpackage{hyperref}'s
% difficulty regarding compatibility with other packages.
%
% \subsection{Standard \hologo{LaTeX} behaviour}
%
% References are created by the \cs{label} command:
% \begin{quote}
%   |\chapter{Second chapter}|\\
%   |\section{First section on page 7} % section 2.1|\\
%   |\label{myref}|
% \end{quote}
% Now \hologo{LaTeX} records the section number |2.1| and the page |7|
% in the reference. Internally the reference is a list with
% two entries:
% \begin{quote}
%   |\r@myref| $\rightarrow$ |{2.1}{7}|
% \end{quote}
% The length of the list if fixed in the \hologo{LaTeX} kernel,
% An interface for adding new properties is missing.
%
% There are several tries to add new properties:
% \begin{description}
% \item[\xpackage{hyperref}] uses a list of five
%   properties instead of the standard list with two entries.
%   This causes many compatibility problems with \hologo{LaTeX} and
%   other packages.
% \item[\xpackage{titleref}] stores its title data into the
%   first entry in the list. \hologo{LaTeX} is happy because it
%   does only see its list with two entries. The situation
%   becomes more difficult, if more properties are added
%   this way. Then the macros form a nested structure
%   inside the first reference argument for the label.
%   Expandable extractions will then become painful.
% \end{description}
%
% \subsection{Basic idea}
%
% Some time ago Morten H\o gholm sent me an experimental
% cross referencing mechanism as ``expl3'' code. His
% idea is:
% \begin{quote}
%   |\g_xref_mylabel_plist| $\rightarrow$\\
%   \mbox{}\qquad|\xref_dance_key{salsa}\xref_name_key{Morten}|\dots
% \end{quote}
% The entries have the following format:
% \begin{quote}
%   |\xref_|\meta{your key}|_key{|\meta{some text}|}|
% \end{quote}
% This approach is much more flexible:
% \begin{itemize}
%   \item New properties can easily be added, just use a new key.
%   \item The length of the list is not fixed. A reference
%         can use a subset of the keys.
%   \item The order of the entries does not matter.
% \end{itemize}
% Unhappily I am not familiar with the experimental code
% for \hologo{LaTeX3} that will need some time before its first release.
% Thus I have implemented it as \hologo{LaTeXe} package without disturbing
% the existing \hologo{LaTeX} reference system.
%
% \subsection{Interfaces}
%
% The package provides a generic \emph{interface for programmers}.
% Commands of this interface are prefixed by \cs{zref@}.
%
% Option \xoption{user} enabels the \emph{user interface}.
% Here the commands are prefixed by \cs{z} to avoid name
% clashes with existing macros.
%
% Then the packages provides some \emph{modules}. They
% are applications for the reference system and can also be
% considered as examples how to use the reference system.
%
% The modules can be loaded as packages. The package
% name is prefixed with |zref-|, for example:
%\begin{quote}
%\begin{verbatim}
%\RequirePackage{zref-abspage}
%\end{verbatim}
%\end{quote}
% This is the preferred way if the package is loaded from
% within other packages to avoid option clashes.
%
% As alternative package |zref| can be used and the modules
% are given as options:
%\begin{quote}
%\begin{verbatim}
%\usepackage[perpage,user]{zref}
%\end{verbatim}
%\end{quote}
%
% \section{Interface for programmers}
%
% The user interface is described in the next section \ref{sec:user}.
%
% \subsection{Entities}
%
% \paragraph{Reference.}
% Internally a reference is a list of key value pairs:
% \begin{quote}
%   |\Z@R@myref| $\rightarrow$
%   |\default{2.1}\page{7}|
% \end{quote}
% The generic format of a entry is:
% \begin{quote}
%   |\Z@R@|\meta{refname} $\rightarrow$
%   |\|\meta{propname}|{|\meta{value}|}|
% \end{quote}
% \meta{refname} is the name that denoted references
% (the name used in \cs{label} and \cs{ref}). \meta{propname}
% is the name of the property or key. The property key macro
% is never executed, it is used in parameter text matching only.
%
% \paragraph{Property.}
% Because the name of a property is used in a macro name
% that must survive the \xfile{.aux} file, the name is
% restricted to letters and `@'.
%
% \paragraph{Property list.}
% Often references are used for special purposes. Thus it
% saves memory if just the properties are used in this
% reference that are necessary for its purpose.
%
% Therefore this package uses the concept of \emph{property lists}.
% A property list is a set of properties. The set of properties
% that is used by the default \cs{label} command is the
% \emph{main property list}.
%
% \subsection{Property list}
%
% \markexp means that the implementation of the
% marked macro is expandable. \markexptwo goes a step further
% and marks the macro expandable in exact two expansion steps.
%
% \begin{declcs}{zref@newlist}
%   \M{listname}
% \end{declcs}
% Declares a new empty property list.
%
% \begin{declcs}{zref@addprop}
%   \M{listname} \M{propname}\\
%   \cs{zref@localaddprop} \M{listname} \M{propname}
% \end{declcs}
% Adds the property \meta{propname}
% to the property list \meta{listname}.
% The property and list must exist.
% The addition is global by \cs{zref@addprop} and limited to
% local scope by \cs{zref@localaddprop}.
% Between 2010/04/19 v2.13 and 2010/10/22 v2.19 a comma separated list
% of properties could be used as argument \meta{propname}.
% Since 2010/10/22 v2.19 the addition of several properties
% at once is supported by \cs{zref@addprops}.
%
% \begin{declcs}{zref@addprops}
%   \M{listname} \M{propname list}\\
%   \cs{zref@localaddprops} \M{listname} \M{propname list}
% \end{declcs}
% These macros add a comma separated list of properties
% \meta{propname list} to list \meta{listname}.
% \cs{zref@addprops} works globally and \cs{zref@localaddprops}
% locally. Since 2010/10/22 v2.19.
%
% \begin{declcs}{zref@listexists}
%   \M{listname} \M{then}
% \end{declcs}
% Executes \meta{then} if the property list \meta{listname} exists
% or raise an error otherwise.
%
% \begin{declcs}{zref@iflistundefined}\markexp \M{listname}
%   \M{then} \M{else}
% \end{declcs}
% Executes \meta{then} if the list exists or \meta{else}
% otherwise.
%
% \begin{declcs}{zref@iflistcontainsprop}
%   \M{listname} \M{propname} \M{then} \M{else}
% \end{declcs}
% Executes \meta{then} if the property \meta{propname} is part
% of property list \meta{listname} or otherwise it
% runs the \meta{else} part.
%
% \subsection{Property}
%
% \begin{declcs}{zref@newprop}
%   \* \M{propname} |[|\meta{default}|]| \M{value}
% \end{declcs}
% This command declares and configures a new property with
% name \meta{propname}.
%
% In case of unknown references
% or the property does not exist in the reference, the
% \meta{default} is used as value. If it is not specified
% here, a global default is used, see \cs{zref@setdefault}.
%
% The correct values of some properties are not known
% immediately but at page shipout time. Prominent example
% is the page number. These properties are declared with
% the star form of the command.
%
% \begin{declcs}{zref@setcurrent}
%   \M{propname} \M{value}
% \end{declcs}
% This sets the current value of the property \meta{propname}.
% It is a generalization of setting \hologo{LaTeX}'s \cs{currentlabel}.
%
% \begin{declcs}{zref@getcurrent}\markexptwo
%   \M{propname}
% \end{declcs}
% This returns the current value of the property \meta{propname}.
% The value may not be correct, especially if the property
% is bound to a page (start form of \cs{zref@newprop}) and the
% right value is only known at shipout time (e.g. property `page').
% In case of errors (e.g. unknown property) the empty string is
% returned.
%
% Since version 2010/04/22 v2.14 \cs{zref@getcurrent} supports
% \cs{zref@wrapper@unexpanded}.
%
% \begin{declcs}{zref@propexists}
%   \M{propname} \M{then}
% \end{declcs}
% Calls \meta{then} if the property \meta{propname} is available
% or generates an error message otherwise.
%
% \begin{declcs}{zref@ifpropundefined}\markexp
%   \M{propname} \M{then} \M{else}
% \end{declcs}
% Calls \meta{then} or \meta{else} depending on the existence
% of property \meta{propname}.
%
% \subsection{Reference generation}
%
% \begin{declcs}{zref@label}
%   \M{refname}
% \end{declcs}
% This works similar to \cs{label}. The reference \meta{refname} is
% created and put into the \xfile{.aux} file with the
% properties of the main property list.
%
% \begin{declcs}{zref@labelbylist}
%   \M{refname} \M{listname}
% \end{declcs}
% Same as \cs{zref@label} except that the properties are
% taken from the specified property list \meta{listname}.
%
% \begin{declcs}{zref@labelbyprops}
%   \M{refname} |{|\meta{propnameA}|,|\meta{propnameB}|,|\dots|}|
% \end{declcs}
% Same as \cs{zref@label} except that these properties
% are used that are given as comma separated list in the
% second argument.
%
% \begin{declcs}{zref@newlabel}
%   \M{refname} |{|\dots|}|
% \end{declcs}
% This is the macro that is used in the \xfile{.aux} file.
% It is basically the same as \cs{newlabel} apart from
% the format of the data in the second argument.
%
% \subsection{Data extraction}
%
% \begin{declcs}{zref@extractdefault}\markexptwo
%   \M{refname} \M{propname} \M{default}
% \end{declcs}
% This is the basic command that refernces the value of a
% property \meta{propname} for the reference \meta{refname}.
% In case of errors such as undefined reference the \meta{default}
% is used instead.
%
% \begin{declcs}{zref@extract}\markexptwo
%   \M{refname} \M{propname}
% \end{declcs}
% The command is an abbreviation for \cs{zref@extractdefault}.
% As default the default of the property is taken, otherwise
% the global default.
%
% Example for page references:
% \begin{quote}
% \begin{tabular}{@{}ll}
% \hologo{LaTeX}:& |\pageref{foobar}|\\
% \xpackage{zref}:& |\zref@extract{foobar}{page}|
% \end{tabular}
% \end{quote}
% Both \cs{zref@extract} and \cs{zref@extractdefault} are
% expandable. That means, these macros can directly
% be used in expandable calculations, see the example file.
% On the other side, babel's shorthands are not supported,
% there are no warnings in case of undefined references.
%
% If an user interface doesn't need expandable macros
% then it can use \cs{zref@refused} and \cs{zref@wrapper@babel}
% for its user macros.
%
% \begin{declcs}{zref@refused}
%   \M{refname}
% \end{declcs}
% This command is not expandable. It causes the warnings if
% the reference \meta{refname} is not defined. Use the
% \cs{zref@extract} commands inside expandable contexts and
% mark their use outside by \cs{zref@refused}, see the example
% file.
%
% \begin{declcs}{zref@def@extract}%
%     \M{cmd} \M{refname} \M{propname}\\
%   \cs{zref@def@extractdefault}
%     \M{cmd} \M{refname} \M{propname} \M{default}
% \end{declcs}
% Both macros extract the property \meta{propname} from the reference
% \meta{refname} the same way as macros \cs{zref@extract} and
% \cs{zref@extractdefault}. The result is stored in macro \meta{cmd}.
% Also \cs{zref@refused} is called to notify \hologo{LaTeX} that
% the reference \meta{refname} is used. Added in 2011/10/04 v2.22.
%
% \begin{declcs}{zref@ifrefundefined}\markexp
%   \M{refname} \M{then} \M{else}
% \end{declcs}
% Macro \cs{zref@ifrefundefined} calls arguments \meta{then} or \meta{else}
% dependent on the existence of the reference \meta{refname}.
%
% \begin{declcs}{zifrefundefined}
%   \M{refname} \M{then} \M{else}
% \end{declcs}
% Macro \cs{zifrefundefined} calls \cs{ref@refused} before
% executing \cs{zref@ifrefundefined}. Babel shorthands are
% supported in \meta{refname}.
%
% \begin{declcs}{zref@ifrefcontainsprop}\markexp
%   \M{refname} \M{propname} \M{then} \M{else}
% \end{declcs}
% Test whether a reference provides a property.
%
% \subsection{Setup}
%
% \begin{declcs}{zref@default}
% \end{declcs}
% Holds the global default for unknown values.
%
% \begin{declcs}{zref@setdefault}
%   |{|\meta{value}|}|
% \end{declcs}
% Sets the global default for unknown values.
% The global default is used, if a property does not specify
% an own default and the value for a property
% cannot be extracted. This can happen if the reference is
% unknown or the reference does not have the property.
%
% \begin{declcs}{zref@setmainlist}
%   |{|\meta{value}|}|
% \end{declcs}
% Sets the name of the main property list. The package
% sets and uses |main|.
%
% \subsection{Declared properties}
%
% \begin{quote}
% \renewcommand*{\arraystretch}{1.05}
% \setlength{\tabcolsep}{2\tabcolsep}
% \def\VV{\vphantom{\large \"Ag}}
% \def\EMPTY{{\itshape\textless empty\textgreater}}
% \def\pl#1{&#1&thepage&\cs{number}\cs{#1}}
% \begin{tabular}{@{}>{\sffamily}l|>{\ttfamily}l|>{\ttfamily}l|>{\ttfamily}l@{}}
%   \rmfamily Module&
%   \rmfamily Property&
%   \rmfamily Property list&
%   \rmfamily Default\\
%   \hline
%   (base)
%     & default & main & \EMPTY\\
%     & page & main & \EMPTY\\
%   \hline
%   abspage
%     & abspage & main & 0\\
%   \hline
%   counter
%     & counter & main & \EMPTY\\
%   \hline
%   hyperref
%     & anchor & main & \EMPTY\\
%     & url    & & \EMPTY\\
%   \hline
%   pageattr
%     & pdfpageattr & thepage & \dots\\
%     & pdfpagesattr & LastPage & \dots\\
%   \hline
%   pagelayout\footnotemark
%     \pl{mag}\\
%     \pl{paperwidth}\\
%     \pl{paperheight}\\
%     \pl{stockwidth}\\
%     \pl{stockheight}\\
%     \pl{pdfpageheight}\\
%     \pl{pdfpagewidth}\\
%     \pl{pdfhorigin}\\
%     \pl{pdfvorigin}\\
%     \pl{hoffset}\\
%     \pl{voffset}\\
%     \pl{topmargin}\\
%     \pl{oddsidemargin}\\
%     \pl{evensidemargin}\\
%     \pl{textwidth}\\
%     \pl{textheight}\\
%     \pl{headheight}\\
%     \pl{headsep}\\
%     \pl{footskip}\\
%     \pl{marginparwidth}\\
%     \pl{marginparsep}\\
%     \pl{columnwidth}\\
%     \pl{columnsep}\\
%   \hline
%   perpage
%     & pagevalue & perpage & 0\\
%     & page & perpage & \EMPTY\\
%     & abspage & perpage & 0\\
%   \hline
%   savepos
%     & posx & savepos & 0\\
%     & posy & savepos & 0\\
%   \hline
%   titleref
%     & title & main & \EMPTY\\
%   \hline
%   xr
%     & anchor   & & \EMPTY\\
%     & externaldocument & & \EMPTY\\
%     & theotype & & \EMPTY\\
%     & title    & & \EMPTY\\
%     & url      & & \EMPTY\\
% \end{tabular}
% \footnotetext{Module \xmodule{pagelayout} only defines properties
% if the parameter exists.}%
% \end{quote}
%
% \subsection{Wrapper for advanced situations}
%
% \begin{declcs}{zref@wrapper@babel}
%   |{|\dots|}| \M{name}
% \end{declcs}
% This macro helps to add shorthand support. The second
% argument is protected, then the code of the first argument
% is called with the protected name appended. Examples
% are in the sources.
%
% \begin{declcs}{zref@wrapper@immediate}
%   |{|\dots|}|
% \end{declcs}
% There are situations where a label must be written instantly
% to the \xfile{.aux} file, for example after the last page.
% If the \cs{zlabel} or \cs{label} command is put inside this wrapper,
% immediate writing is enabled. See the implementation for
% module \xmodule{lastpage} for an example of its use.
%
% \begin{declcs}{zref@wrapper@unexpanded}
%   |{|\dots|}|
% \end{declcs}
% Assuming someone wants to extract a value for property |bar|
% and store the result in a macro |\foo| without traces of
% the expanding macros and without expanding the value.
% This (theoretical?) problem can be solved by this wrapper:
% \begin{quote}
%\begin{verbatim}
%\zref@wrapper@unexpanded{%
%  \edef\foo{%
%    \zref@extract{someref}{bar}%
%  }%
%}
%\end{verbatim}
% \end{quote}
% The \cs{edef} forces the expansion of \cs{zref@extract},
% but the extraction of the value is prevented by
% the wrapper that uses \hologo{eTeX}'\ \cs{unexpanded} for this
% purpose. Supported macros are \cs{zref@extract}, \cs{zref@extractdefault}
% and since version 2010/04/22 v2.14 macro \cs{zref@getcurrent}.
%
% \subsection{Counter for unique names}
%
% Some modules (\xmodule{titleref} and \xmodule{dotfillmin})
% need unique names for automatically generated label names.
%
% \begin{declcs}{zref@require@unique}
% \end{declcs}
% This command creates the unique counter |zref@unique| if the
% counter does not already exist.
%
% \begin{declcs}{thezref@unique}
% \end{declcs}
% This command is used to generate unique label names.
%
% \section{User interface}\label{sec:user}
%
% \subsection{Module \xmodule{user}}
%
% The user interface for this package and its modules
% is enabled by \xpackage{zref}'s package option \xmodule{user}
% or package \xpackage{zref-user}.
% The names of user commands are prefixed by |z| in order
% to avoid name clashes with existing macros of the same
% functionality. Thus the package does not disturb the
% traditional reference scheme, both can be used together.
%
% The syntax descriptions contain the following markers
% that are intended as hints for programmers:
% \begin{quote}
% \begin{tabular}{@{}ll@{}}
%   \markbabel\unskip  & Babel shorthands are allowed.\\
%   \markrobust\unskip & Robust macro.\\
%   \markexp\unskip    & Expandable version:\\
%                      & \textbullet\ robust,
%                        unless the extracted values are fragile,\\
%                      & \textbullet\ no babel shorthand suport.\\
%   \markexptwo\unskip & Expandable like \markexp and:\\
%                      & \textbullet\ expandable in exact two steps.\\
% \end{tabular}
% \end{quote}
%
% The basic user interface of the package without modules
% are commands that mimic the standard \hologo{LaTeX} behaviour
% of \cs{label}, \cs{ref}, and \cs{pageref}:
%
% \begin{declcs}{zlabel}
%   \M{refname}\markbabel
% \end{declcs}
% Similar to \cs{label}. It generates a label with name
% \meta{refname} in the new reference scheme.
%
% \begin{declcs}{zref}
%   |[|\meta{propname}|]| \M{refname}\markbabel
% \end{declcs}
% Without optional argument similar to \cs{ref}, it returns
% the default reference property. This property is named
% |default|:
% \[
%   |\zref{|x|}| \equiv |\zref[default]{|x|}|
% \]
%
% \begin{declcs}{zpageref}
%   \M{refname}\markbabel
% \end{declcs}
% Convenience macro, similar to \cs{pageref}.
% \[
%   |\zpageref{|x|}| \equiv |\zref[page]{|x|}|
% \]
%
% \begin{declcs}{zrefused}
%   \M{refname}\markbabel
% \end{declcs}
% Some of the user commands in the modules are expandable.
% The use of such commands do not cause any undefined
% reference warnings, because inside of expandable
% contexts this is not possible. However, if there is
% a place outside of expandable contexts, \cs{refused}
% is strongly recommended. The reference \meta{refname}
% is marked as used, undefined ones will generate
% warnings.
%
% \subsection{Module \xmodule{abspage}}
%
% With the help of package \xpackage{atbegshi}
% a new counter |abspage| with absolute page numbers is provided.
% For technical and historical reasons the counter itself is zero based:
% if you use it directly on the first page, e.g
% with |\arabic{abspage}| you will get 0 as value. When using |\zref|
% the first page will be page 1 as expected.
% Also a new property |abspage| is defined and added to the
% main property list. Thus you can reference the absolute
% page number:
% \begin{quote}
%   |Section \zref{foo} is on page \zpageref{foo}.|\\
%   |This is page \zref[abspage]{foo}|\\
%   |of \zref[abspage]{LastPage}.|
% \end{quote}
% The example also makes use of module \xmodule{lastpage}.
%
% \subsection{Module \xmodule{lastpage}}
%
% Provides the functionality of package \xpackage{lastpage}
% \cite{lastpage} in the new reference scheme.
% The label |LastPage| is put at the end of the document.
% You can refer the last page number with:
% \begin{quote}
%   |\zref@extract{LastPage}{page}| (+ |\zref@refused{LastPage}|)\\
% \end{quote}
% or
% \begin{quote}
%   |\zpageref{LastPage}| (module \xmodule{user})
% \end{quote}
%
% Since version 2008/10/01 v2.3 the module defines the list |LastPage|.
% In addition to the properties of the main list label |LastPage| also
% stores the properties of this list |LastPage|. The default of this
% list is empty. The list can be used by the user to add additional
% properties for label |LastPage|.
%
% \subsubsection{Tests for last page}
%
% Since version 2010/03/26 v2.8 the macros \cs{zref@iflastpage}
% and \cs{ziflastpage} were added. They test the reference,
% whether it is a reference of the last page.
%
% \begin{declcs}{zref@iflastpage}\markexp \M{refname} \M{then} \M{else}
% \end{declcs}
% Macro \cs{zref@iflastpage} compares the references \meta{refname}
% with \meta{LastPage}. Basis of the comparison is the value of
% property |abspage|, because the values are different for different
% pages. This is not ensured by property |page|. Therefore module
% \xmodule{abspage} is loaded by module \xmodule{lastpage}.
% If both values of property |abspage| are present and match,
% then \meta{then} is executed, otherwise code \meta{else} is called.
% If one or both references are undefined or lack the property |abspage|,
% then \meta{else} is executed.
%
% Macro \cs{zref@iflastpage} is expandable, therefore \cs{zref@refused}
% should be called on \meta{refname} and \meta{LastPage}.
%
% \begin{declcs}{ziflastpage} \M{refname} \M{then} \M{else}
% \end{declcs}
% Macro \cs{ziflastpage} has the same function as \cs{zref@iflastpage},
% but adds support for babel shorthands in \meta{refname} and
% calls \cs{zref@refused}. However macro \cs{ziflastpage} is not
% expandable.
%
% \subsubsection{Example}
%
%    \begin{macrocode}
%<*example-lastpage>
%<<END_EXAMPLE
\NeedsTeXFormat{LaTeX2e}
\documentclass{report}

\newcounter{foo}
\renewcommand*{\thefoo}{\Alph{foo}}

\usepackage{zref-lastpage,zref-user}[2019/11/29]

\makeatletter
\zref@newprop{thefoo}{\thefoo}
\zref@newprop{valuefoo}{\the\value{foo}}
\zref@newprop{chapter}{\thechapter}
\zref@addprops{LastPage}{thefoo,valuefoo,chapter}
\makeatother

\newcommand*{\foo}{%
  \stepcounter{foo}%
  [Current foo: \thefoo]%
}

\begin{document}
  \chapter{First chapter}
  Last page is \zref{LastPage}.\\
  Last chapter is \zref[chapter]{LastPage}.\\
  Last foo is \zref[thefoo]{LastPage}.\\
  Last value of foo is \zref[valuefoo]{LastPage}.\\
  \foo
  \chapter{Second chapter}
  \foo\foo\foo
  \chapter{Last chapter}
  \foo
\end{document}
%END_EXAMPLE
%</example-lastpage>
%    \end{macrocode}
%
% \subsection{Module \xmodule{thepage}}
%
% This module \xmodule{thepage} loads module \xmodule{abspage},
% constructs a reference name using the absolute page number
% and remembers property |page|. Other properties can be added
% by adding them to the property list |thepage|.
%
% \begin{declcs}{zthepage} \M{absolute page number}\\
% \end{declcs}
% Macro \cs{zthepage} is basically a \cs{zpageref}. The reference
% name is yield by the \meta{absolute page number}. If the reference
% is not defined, then the default for property |page| is used.
%
% \begin{declcs}{zref@thepage@name}\markexp \M{absolute page number}
% \end{declcs}
% Macro \cs{zref@thepage@name} returns the internal reference
% name that is constructed using the \meta{absolute page number}.
% The internal reference name should not be used directly, because
% it might change in future versions.
%
% \begin{declcs}{zref@thepage}\markexp \M{absolute page number}\\
%   \cs{zref@thepage@refused} \M{absolute page number}
% \end{declcs}
% Macro \cs{zref@thepage} returns the page number (\cs{thepage})
% of \meta{absolute page number}. Because this macro is expandable,
% \cs{zref@thepage@refused} is used outside an expandable context
% to mark the reference as used.
%
%
% \subsection{Module \xmodule{nextpage}}
%
% \begin{declcs}{znextpage}
% \end{declcs}
% Macro \cs{znextpage} prints \cs{thepage} of the following page.
% It gets the current absolute page number by using a label.
% There are three cases for the next page:
% \begin{enumerate}
% \item The next page is not known yet because of undefined references.
%   Then \cs{zunknownnextpagename} is used instead. The default for
%   this macro is the default of property |page|.
% \item This page is the last page. Then \cs{znonextpagename}
%   is used. Its default is empty.
% \item The next page is known, then \cs{thepage} of the next
%   page is used (the value of property |page| of the next page).
% \end{enumerate}
%
% \subsubsection{Configuration}
%
% The behaviour can be configured by the following macros.
%
% \begin{declcs}{zunknownnextpagename}\\
%   \cs{znonextpagename}
% \end{declcs}
% If the next page is not known or available, then
% \cs{znextpage} uses these name macros as default.
% \cs{zunknownnextpagename} is used in case of undefined
% references. Default is the value of property |page| of the
% next page (\cs{thepage}). Module \xmodule{thepage} is used.
%
% Macro \cs{znonextpagename} is used, if the next page does
% not exists. That means that the current page is last page.
% The default is empty.
%
% \begin{declcs}{znextpagesetup} \M{unknown} \M{no next}
%   \M{next}
% \end{declcs}
% Acording to the case (see \cs{znextpage}) macro \cs{znextpage}
% calls an internal macro with an argument. The argument is
% either \cs{thepage} of the next page or one of \cs{zunknownnextpagename}
% or \cs{znonextpagename}. These internal macro can be changed by
% \cs{znextpagesetup}. It expects the definition texts for these
% three cases of a macro with one argument.
% The default is
% \begin{quote}
%   |\znextpagesetup{#1}{#1}{#1}|
% \end{quote}
%
% \subsubsection{Example}
%
%    \begin{macrocode}
%<*example-nextpage>
%<<END_EXAMPLE
\documentclass{book}

\usepackage{zref-nextpage}[2019/11/29]
\znextpagesetup
  {\thepage}% next page is unknown
  {\thepage\ (#1)}% this page is last page
  {\thepage\ $\rightarrow$ #1}% next page is known
\renewcommand*{\znonextpagename}{last page}

\usepackage{fancyhdr}
\pagestyle{fancy}
\fancyhf{}
\fancyhead[LE,RO]{\znextpage}
\fancypagestyle{plain}{%
  \fancyhf{}%
  \fancyhead[LE,RO]{\znextpage}%
}

\begin{document}
\frontmatter
  \tableofcontents
\mainmatter
  \chapter{Hello World}
  \clearpage
  \section{Last section}
\end{document}
%END_EXAMPLE
%</example-nextpage>
%    \end{macrocode}
%
%
% \subsection{Module \xmodule{totpages}}
%
% For the total number of pages of a document you need to
% know the absolute page number of the last page.
% Both modules \xmodule{abspage} and \xmodule{lastpage}
% are necessary and automatically enabled.
%
% \begin{declcs}{ztotpages}\markexp
% \end{declcs}
% Prints the total number of pages or |0| if this
% number is not yet known. It expands to an explicit number
% and can also used even in expandable calculations (\cs{numexpr})
% or counter assignments.
%
% \subsection{Module \xmodule{pagelayout}}
%
% The module defines additional properties for each parameter
% of the page layout that is effective during page shipout.
% The value of length parameters is given in sp without the
% unit as plain number.
%
% Some parameters are specific for a class (e.g.\@ \xoption{stockwidth}
% and \xoption{stockheight} for class \xclass{memoir}) or the
% \hologo{TeX} engine like \hologo{pdfTeX}. If the parameter is not
% available, then the property will not be defined. The default value
% of the property is the current setting of the parameter.
%
% The module \xmodule{thepage} is loaded that generates a label
% for each page. The properties of module \xmodule{pagelayout}
% are added to the property list |thepage| of module \xmodule{thepage}.
%
% List of properties:
% \begin{quote}
%   \def\x#1{\cs{#1}&#1}%
%   \begin{tabular}{l>{\sffamily}ll}
%     \textrm{parameter} & \textrm{property} & remarks\\
%     \hline
%     \x{mag}\\
%     \x{paperwidth}\\
%     \x{paperheight}\\
%     \x{stockwidth}& class \xclass{memoir}\\
%     \x{stockheight}& class \xclass{memoir}\\
%     \x{pdfpagewidth}& \hologo{pdfTeX}, \hologo{LuaTeX}\\
%     \x{pdfpageheight}& \hologo{pdfTeX}, \hologo{LuaTeX}\\
%     \x{pdfhorigin}& \hologo{pdfTeX}, \hologo{LuaTeX}\\
%     \x{pdfvorigin}& \hologo{pdfTeX}, \hologo{LuaTeX}\\
%     \x{hoffset}\\
%     \x{voffset}\\
%     \x{topmargin}\\
%     \x{oddsidemargin}\\
%     \x{evensidemargin}\\
%     \x{textwidth}\\
%     \x{textheight}\\
%     \x{headheight}\\
%     \x{headsep}\\
%     \x{footskip}\\
%     \x{marginparwidth}\\
%     \x{marginparsep}\\
%     \x{columnwidth}\\
%     \x{columnsep}\\
%   \end{tabular}
% \end{quote}
%
% \begin{declcs}{zlistpagelayout}
% \end{declcs}
% At the end of document the page layout parameter for each page are
% printed into the \xext{log} file if macro \cs{zlistpagelayout}
% is called before |\end{document}| (preamble is a good place).
%
% \subsection{Module \xmodule{marks}}
%
% ToDo.
%
% \subsection{Module \xmodule{runs}}
%
% Module \xmodule{runs} counts the \hologo{LaTeX} runs since last
% \xext{aux} file creation and prints the number in the
% \xext{log} file.
%
% \begin{declcs}{zruns}\markexp
% \end{declcs}
% Prints the the total number of \hologo{LaTeX} runs including
% the current one. It expands to an explicit number.
% Before |begin{document}| the value is zero meaning
% the \xext{aux} file is not read yet. If a previous
% \xext{aux} file exists, the value found there increased
% by one is the new number. Otherwise \cs{zruns} is set to one.
% \hologo{LaTeX}~runs where the \xext{aux} files are not rewritten
% are not counted (see \cs{nofiles}).
%
% \subsection{Module \xmodule{perpage}}
%
% With \cs{@addtoreset} or \cs{numberwithin} a counter can be
% reset if another counter is incremented. This do not work
% well if the other counter is the page counter. The page counter is
% incremented in the output routine that is often called asynchronous
% somewhere on the next page. A reference mechanism costs at least
% two \hologo{LaTeX} runs, but ensures correct page counter values.
%
% \begin{declcs}{zmakeperpage}
%   |[|\meta{reset}|]| \M{counter}
% \end{declcs}
% At the of a new page counter \meta{counter} starts
% counting with value \meta{reset} (default is \texttt{1}).
% The macro has the same
% syntax and semantics as \cs{MakePerPage} of package
% \xpackage{perpage} \cite{perpage}.
% Also \xoption{perpage} of package \xpackage{footmisc} \cite{footmisc}
% can easily be simulated by
% \begin{quote}
% |\zmakeperpage{footnote}| \itshape |% \usepackage[perpage]{footmisc}|
%\end{quote}
% If footnote symbols are used, some people dislike the
% first symbol \ensuremath{\dagger}. It can easily be skipped:
% \begin{quote}
% |\zmakeperpage[2]{footnote}|
% \end{quote}
%
% \begin{declcs}{thezpage}\\
%   counter |zpage|
% \end{declcs}
% If the formatted counter value of the counter that is reset at
% a new page contains the page value, then you can use \cs{thezpage},
% the page number of the current page. Or counter |zpage| can be
% used, if the page number should be formatted differently from
% the current page number. Example:
%\begin{quote}
%\begin{verbatim}
%\newcounter{foobar}
%\zmakeperpage{foobar}
%\renewcommand*{\thefoobar}{\thezpage-\arabic{foobar}}
% % or
%\renewcommand*{\thefoobar}{\roman{zpage}-\arabic{foobar}}
%\end{verbatim}
%\end{quote}
%
% \begin{declcs}{zunmakeperpage}
%   \M{counter}
% \end{declcs}
% The reset mechanism for this counter is deactivated.
%
%
% \subsection{Module \xmodule{counter}}
%
% This option just add the property |counter| to the main
% property list. The property stores the counter name,
% that was responsible for the reference. This is the
% property \xpackage{hyperref}'s \cs{autoref} feature
% uses. Thus this property |counter| may be useful
% for a reimplementation of the autoref feature,
% see the section \ref{sec:todo} with the todo list.
%
% \subsection{Module \xmodule{titleref}}
%
% This option makes section and caption titles available
% to the reference system similar to packages \xpackage{titleref}
% or \xpackage{nameref}.
%
% \begin{declcs}{ztitleref}
%   \M{refname}\markbabel
% \end{declcs}
% Print the section or caption title of reference \meta{refname},
% similar to \cs{nameref} or \cs{titleref}.
%
% \begin{declcs}{ztitlerefsetup}
%   |{|\textit{key$_1$=value$_1$, key$_2$=value$_2$, \dots}|}|
% \end{declcs}
% This command allows to configure the behaviour of module
% \xmodule{titleref}. The following keys are available:
% \begin{description}
% \let\bfseries\relax
% \item[\texttt{title=}\meta{value}]\mbox{}\\*
%   Sets the current title.
% \item[\texttt{stripperiod=true\string|false}]\mbox{}\\*
%   Follow package \xpackage{nameref} that
%   removes a last period. Default: |true|.
% \item[\texttt{expand=true\string|false}]\mbox{}\\*
%   Package \cs{titleref} expands the title first. This way
%   garbage and dangerous commands can be removed, e.g.
%   \cs{label}, \cs{index}\dots. See implementation section
%   for more details. Default is |false|.
% \item[\texttt{cleanup=\{}\textrm{\dots}\texttt{\}}]\mbox{}\\*
%   Hook to add own cleanup code, if method |expand| is used.
%   See implementation section for more details.
% \end{description}
%
% \subsection{Module \xmodule{savepos}}
%
% This option supports a feature that \pdfTeX\ provides
% (and \hologo{XeTeX}).
% \pdfTeX\ is able to tell the current position on the
% page. The page position is not instantly known. First the
% page must be constructed by \hologo{TeX}'s asynchronous output
% routine. Thus the time where the position is known is
% the page shipout time. Thus a reference system where the
% information is recorded in the first run and made available
% for use in the second run comes in handy.
%
% \begin{declcs}{zsavepos}
%   \M{refname}
% \end{declcs}
% It generates a reference with name \meta{refname}.
% The reference stores the location where \cs{zsavepos}
% is executed in properties |posx| and |posy|.
%
% \begin{declcs}{zsaveposx}
%   \M{refname}\\
%   \cs{zsaveposy}
%   \M{refname}
% \end{declcs}
% Same as \cs{zsavepos} except that only the |x| or |y| component
% of the position is stored. Since 2011/12/05 v2.23.
%
% \begin{declcs}{zposx}\markexp
%   \M{refname}\\
%   \SpecialUsageIndex{\zposy}\cs{zposy}\markexp \M{refname}
% \end{declcs}
% Get the position as number. Unit is sp. Horizontal
% positions by \cs{zposx} increase from left to right.
% Vertical positions by \cs{zposy} from bottom to top.
%
% Do not rely on absolute page numbers. Because of problems
% with the origin the numbers may differ in DVI or PDF mode
% of \pdfTeX. Therefore work with relative values by comparisons.
%
% Both \cs{zposx} and \cs{zposy} are expandable and can be
% used inside calculations (\cs{setcounter}, \cs{addtocounter},
% package \xpackage{calc}, \cs{numexpr}). However this property
% prevents from notifying \hologo{LaTeX} that the reference is actually
% used (the notifying is not expandable). Therefore you should
% mark the reference as used by \cs{zrefused}.
%
% This module uses \pdfTeX's \cs{pdfsavepos}, \cs{pdflastxpos},
% and \cs{pdflastypos}. They are available in PDF mode and
% since version 1.40.0 also in DVI mode.
%
% \begin{declcs}{zref@savepos}
% \end{declcs}
% Macro \cs{zref@savepos} performs the first part of \cs{zsavepos}
% by calling \cs{pdfsavepos} (if \xext{aux} files are writable).
%
% Thus \cs{zsavepos} is basically \cs{zref@savepos} followed
% by |\zref@labelbylist{|\texttt{\meta{refname}}|}{savepos}|.
% If \cs{TeXXeTstate} is detected and enabled, \cs{savepos}
% also adds \cs{zref@savepos} at the end to support \cs{beginR}
% where the whatits are processed in reverse order.
% The property list |savepos| contains the properties |posx|
% and |posy|.
%
% \subsection{Module \xmodule{abspos}}
% Module \xmodule{abspos} allows to get various values of the page layout. There is no user
% command, only a number of internal commands. For example:
% \begin{quote}
%   |\zref@absposx{|\meta{label}|}{|\meta{value}|}{|\meta{position}|}|\\
%   |\zref@absposy{|\meta{label}|}{|\meta{value}|}{|\meta{position}|}|
% \end{quote}
%
% The return value is like in the module \xmodule{savepos} a number representing a length in sp.
% The length are measured from the bottom left of the page.
%
% \meta{label} is a label set with \cs{zlabel} or \cs{zsavepos} that allows to retrieve the
% absolute page number.
%
% \meta{position} is for the x-command one of |left|, |right| or |center|. For the y-command
%  it is one of |top|, |bottom|, |center|.
%
% The possible content of \meta{value} can be seen in the following table. Be aware that the code
% makes some assumptions which are perhaps not always true -- for example that the left of
% the head is identical to the left of the body.
%
% \medskip
% \begin{tabular}{lll}
%  \textbf{value} & \textbf{axis} & \textbf{comments}\\\hline
%   media         & x             &   left=0, right=\cs{pdfpagewidth} \\
%   paper         & x             &   left=0, right=\cs{paperwidth}\\
%   stock         & x             &   derived from paper\\
%   media         & y             & bottom=0, top=\cs{pdfpageheigh}\\
%   paper         & y             & top=\cs{pdfpageheight}, bottom=top-\cs{paperheight}\\
%   stock         & y             & top  derived from paper\\
%   head          & x             & calculated with hoffset, horigin, etc\\
%   head          & y             & calculated\\
%   body          & x             & = head value\\%
%   body          & y             & = head bottom - \cs{headsep}\\
%   foot          & x             & = head\\
%   foot          & y             & calculated from body bottom and \cs{footskip}\\
%   marginpar     & x             &different on odd/even pages! \\
%   marginpar     & y             & = body %
% \end{tabular}
%
% \subsection{Module \xmodule{dotfill}}
%
% \begin{declcs}{zdotfill}
% \end{declcs}
% This package provides the command \cs{zdotfill} that works
% similar to \cs{dotfill}, but can be configured. Especially
% it suppresses the dots if a minimum number of dots cannot be set.
%
% \begin{declcs}{zdotfillsetup}
%   |{|\textit{key$_1$=value$_1$, key$_2$=value$_2$, \dots}|}|
% \end{declcs}
% This command allows to configure the behaviour of \cs{zdotfill}.
% The following keys are available:
% \begin{description}
% \let\bfseries\relax
% \item[\texttt{min=}\meta{count value}]\mbox{}\\*
%   If the actual number of dots are smaller than \meta{count value},
%   then the dots are suppressed. Default: |2|.
% \item[\texttt{unit=}\meta{dimen value}]\mbox{}\\*
%   The width of a dot unit is given by \meta{dimen value}.
%   Default: |0.44em| (same as the unit in \cs{dotfill}).
% \item[\texttt{dot=}\meta{value}]\mbox{}\\*
%   The dot itself is given by \meta{value}. Default: |.| (dot, same as
%   the dot in \cs{dotfill}).
% \end{description}
%
% \subsection{Module \xmodule{env}}
%
% This module defines two properties |envname| and |envline|. They
% remember the name of the environment and the line number at
% the start of the environment.
%
% \subsection{Module \xmodule{xr}}
%
% This package provides the functionality of package \xpackage{xr},
% see \cite{xr}. It also supports the syntax of \xpackage{xr-hyper}.
%
% \begin{declcs}{zexternaldocument}
%   \* |[|\meta{prefix}|]|\markbabel \M{external document}
%   |[|\meta{url}|]|
% \end{declcs}
% See \cs{externaldocument} for a description of this option.
% The found labels also get a property |externaldocument| that
% remembers \meta{external document}.
% The standard reference scheme and the scheme of this
% package use different name spaces for reference names.
% If the external document uses both systems. Then one
% import statement would put the names in one namespace
% and probably causing problems with multiple references
% of the same name. Thus the star form only looks for
% \cs{newlabel} in the \xfile{.aux} files, whereas without
% star only \cs{zref@newlabel}s are used.
%
% In the star form it tries to detect labels from
% \xpackage{hyperref}, \xpackage{titleref}, and \xpackage{ntheorem}.
% If such an extended property from the packages before cannot
% be found or are empty, they are not included in the
% imported reference.
%
% Warnings are given if a reference name is already in use
% and the item is ignored. Unknown properties will automatically
% be declared.
%
% If the external references contain |anchor| properties,
% then we need also a url to be able to address the external
% file. As default the filename is taken with a default
% extension.
%
% \begin{declcs}{zxrsetup}
%   |{|\textit{key$_1$=value$_1$, key$_2$=value$_2$, \dots}|}|
% \end{declcs}
% The following setup options are available:
% \begin{description}
% \item[\xoption{ext}:] It sets the default extension.
% \item[\xoption{tozreflabel}:] Boolean option. The found references
%   are imported as zref labels. This is enabled by default.
% \item[\xoption{toltxlabel}:] Boolean option. The found references
%   are imported as \hologo{LaTeX} labels. Packages \xpackage{nameref},
%   \xpackage{titleref} and class \xclass{memoir} are supported.
% \item[\xoption{urluse}:] Boolean option. If enabled, then a URL
%   is stored in a macro and the macro is put in property `urluse'.
%   The URL is not put in property `url'.
%   The purpose is to save \hologo{TeX} memory.
% \item[\xoption{verbose}:] Boolean option. List the imported labels in the
%   \xext{log} file. Default is |false|.
% \end{description}
%
% \begin{declcs}{zref@xr@ext}
% \end{declcs}
% If the \meta{url} is not specified in \cs{zref@externaldocument},
% then the url will be constructed with the file name and
% this macro as extension. \cs{XR@ext} is used if
% \xpackage{hyperref} is loaded, otherwise |pdf|.
%
% \subsection{Module \xmodule{pageattr}}
%
% This module allows to recover the content of the register
% \cs{pdfpageattr} and \cs{pdfpagesattr} in pdftex and the equivalent register
% in luatex. There is no user command. Programmer commands are
%
% \begin{quote}
% |\zref@pdfpageattr{|\meta{absolute page number}|}|
%
% |\zref@pdfpagesattr{|\meta{absolute page number}|}|
% \end{quote}
% \section{ToDo}\label{sec:todo}
%
% Among other things the following issues are left for future work:
% \begin{itemize}
% \item Other applications: autoref, hyperref, \dots
% \end{itemize}
%
% \section{Example}
%
%    \begin{macrocode}
%<*example>
%    \end{macrocode}
%
%    \begin{macrocode}
\documentclass{book}

\usepackage[ngerman]{babel}%

\usepackage[savepos,totpages,titleref,dotfill,counter,user]{zref}

%    \end{macrocode}
%
%    Chapters are wrapped inside \cs{ChapterStart} and
%    \cs{ChapterStop}. The first argument \verb|#1| of \cs{ChapterStart}
%    is used to form a label id \verb|chap:#1|. At the end of
%    the chapter another label is set by \cs{zref@wrapper@immediate}, because
%    otherwise at the end of document a deferred write would not
%    be written, because there is no page for shipout.
%
%    Also this example shows how chapter titles can be recorded.
%    A new property \verb|chaptitle| is declared and added to the
%    main property list. In \cs{ChapterStart} the current value
%    of the property is updated.
%    \begin{macrocode}
\makeatletter
\zref@newprop{chaptitle}{}
\zref@addprop{main}{chaptitle}

\newcommand*{\ChapterStart}[2]{%
  \cleardoublepage
  \def\current@chapid{#1}%
  \zref@setcurrent{chaptitle}{#2}%
  \chapter{#2}%
  \zlabel{chap:#1}%
}
\newcommand*{\ChapterStop}{%
  \cleardoublepage
  \zref@wrapper@immediate{%
    \zref@labelbyprops{chapend:\current@chapid}{abspage}%
  }%
}
%    \end{macrocode}
%    \cs{ChapterPages} calculates and returns the number of pages of the
%    referenced chapter.
%    \begin{macrocode}
\newcommand*{\ChapterPages}[1]{%
  \zrefused{chap:#1}%
  \zrefused{chapend:#1}%
  \number\numexpr
    \zref@extract{chapend:#1}{abspage}%
    -\zref@extract{chap:#1}{abspage}%
    +1\relax
}
\makeatother
%    \end{macrocode}
%    \begin{macrocode}
\begin{document}
%    \end{macrocode}
%    As exception we use \cs{makeatletter} here, because
%    this is just an example file that also should show some of
%    programmer's interface.
%    \begin{macrocode}
\makeatletter

\frontmatter
\zlabel{documentstart}

\begin{itemize}
\item
  The frontmatter part has
  \number\numexpr\zref@extract{chap:first}{abspage}-1\relax
  ~pages.
\item
  Chapter \zref{chap:first} has \ChapterPages{first} page(s).
\item
  Section \zref{hello} is on the
  \ifcase\numexpr
    \zref@extractdefault{hello}{page}{0}%
    -\zref@extractdefault{chap:first}{page}{0}%
    +1\relax
    ??\or first\or second\or third\or forth\fi
  ~page inside its chapter.
\item
  The document has
  \zref[abspage]{LastPage} pages.
  This number is \ifodd\ztotpages odd\else even\fi.
\item
  The last page is labeled with \zpageref{LastPage}.
\item
  The title of chapter \zref{chap:next} %
  is ``\zref[chaptitle]{chap:next}''.
\end{itemize}

\tableofcontents

\mainmatter
\ChapterStart{first}{First chapter}

%    \end{macrocode}
%    The user level commands should protect babel shorthands
%    where possible. On the other side, expandable extracting
%    macros are useful in calculations, see above the
%    examples with \cs{numexpr}.
%    \begin{macrocode}
\section{Test}
\zlabel{a"o}
Section \zref{a"o} on page
\zref@wrapper@babel\zref@extract{a"o}{page}.

Text.
\newpage

\section{Hello World}
\zlabel{hello}

\ChapterStop

\ChapterStart{next}{Next chapter with \emph{umlauts}: "a"o"u"s}

%    \end{macrocode}
%
%    Here an example follows that makes use of \pdfTeX's ``savepos''
%    feature. The position on the page is not known before the
%    page is constructed and shipped out. Therefore the position
%    ist stored in references and are available for calculations
%    in the next \hologo{LaTeX} compile run.
%    \begin{macrocode}
The width of the first column is
  \the\dimexpr \zposx{secondcol}sp - \zposx{firstcol}sp\relax,\\
the height difference of the two baselines is
  \the\dimexpr \zposy{firstcol}sp - \zposy{secondline}sp\relax:\\
\begin{tabular}{ll}
  \zsavepos{firstcol}Hello&\zsavepos{secondcol}World\\
  \zsavepos{secondline}Second line&foobar\\
\end{tabular}

%    \end{macrocode}
%    With \cs{zrefused} \hologo{LaTeX} is notified, if the references
%    are not yet available and \hologo{LaTeX} can generate the rerun
%    hint.
%    \begin{macrocode}
\zrefused{firstcol}
\zrefused{secondcol}
\zrefused{secondline}

\ChapterStop
%    \end{macrocode}
%    Test for module \cs{dotfill}.
%    \begin{macrocode}
\ChapterStart{dotfill}{Test for dotfill feature}
\newcommand*{\dftest}[1]{%
  #1&
  [\makebox[{#1}]{\dotfill}]&
  [\makebox[{#1}]{\zdotfill}]\\
}
\begin{tabular}{rll}
& [\verb|\dotfill|] & [\verb|\zdotfill|]\\
\dftest{0.43em}
\dftest{0.44em}
\dftest{0.45em}
\dftest{0.87em}
\dftest{0.88em}
\dftest{0.89em}
\dftest{1.31em}
\dftest{1.32em}
\dftest{1.33em}
\end{tabular}
\ChapterStop
\end{document}
%    \end{macrocode}
%
%    \begin{macrocode}
%</example>
%    \end{macrocode}
%
% \StopEventually{
% }
%
% \section{Implementation}
%
% \subsection{Package \xpackage{zref}}
%
% \subsubsection{Identification}
%
%    \begin{macrocode}
%<*package>
\NeedsTeXFormat{LaTeX2e}
\ProvidesPackage{zref}
  [2020-07-03 v2.32 A new reference scheme for LaTeX (HO)]%
%    \end{macrocode}
%
% \subsubsection{Load basic module}
%
%    \begin{macrocode}
\RequirePackage{zref-base}[2019/11/29]
%    \end{macrocode}
%    Abort package loading if \xpackage{zref-base} could not be
%    loaded successfully.
%    \begin{macrocode}
\@ifundefined{ZREF@base@ok}{\endinput}{}
%    \end{macrocode}
%
% \subsubsection{Process options}
%
%    Known modules are loaded and the release date is checked.
%    \begin{macrocode}
\def\ZREF@temp#1{%
  \DeclareOption{#1}{%
    \AtEndOfPackage{%
      \RequirePackage{zref-#1}[2019/11/29]%
    }%
  }%
}
\ZREF@temp{abspage}
\ZREF@temp{counter}
\ZREF@temp{dotfill}
\ZREF@temp{hyperref}
\ZREF@temp{lastpage}
\ZREF@temp{marks}
\ZREF@temp{nextpage}
\ZREF@temp{pageattr}
\ZREF@temp{pagelayout}
\ZREF@temp{perpage}
\ZREF@temp{runs}
\ZREF@temp{savepos}
\ZREF@temp{thepage}
\ZREF@temp{titleref}
\ZREF@temp{totpages}
\ZREF@temp{user}
\ZREF@temp{xr}
%    \end{macrocode}
%    \begin{macrocode}
\ProcessOptions\relax
%</package>
%    \end{macrocode}
%
%
% \subsection{Module \xmodule{base}}
%
% \subsubsection{Prefixes}
%
%    This package uses the following prefixes for macro names:
%    \begin{description}
%    \item[\cs{zref@}:] Macros of the programmer's interface.
%    \item[\cs{ZREF@}:] Internal macros.
%    \item[\cs{Z@L@}\metacs{listname}:]
%      The properties of the list \meta{listname}.
%    \item[\cs{Z@D@}\metacs{propname}:]
%      The default value for property \meta{propname}.
%    \item[\cs{Z@E@}\metacs{propname}:]
%      Extract function for property \meta{propname}.
%    \item[\cs{Z@X@}\metacs{propname}:]
%      Information whether a property value for property
%      \meta{propname} is expanded
%      immediately or at shipout time.
%    \item[\cs{Z@C@}\metacs{propname}:]
%      Current value of the property \meta{propname}.
%    \item[\cs{Z@R@}\metacs{labelname}:]
%      Data for reference \meta{labelname}.
%    \item[\cs{ZREF@org@}:] Original versions of patched commands.
%    \item[\cs{z}:] For macros in user land, defined if
%      module \xmodule{user} is set.
%    \end{description}
%    The following family names are used for keys defined
%    according to the \xpackage{keyval} package:
%    \begin{description}
%    \item[\texttt{ZREF@TR}:] Setup for module \xmodule{titleref}.
%    \end{description}
%
% \subsubsection{Identification}
%
%    \begin{macrocode}
%<*base>
\NeedsTeXFormat{LaTeX2e}
\ProvidesPackage{zref-base}%
  [2020-07-03 v2.32 Module base for zref (HO)]%
%    \end{macrocode}
%
% \subsubsection{Utilities}
%
%    \begin{macrocode}
\providecommand\IfFormatAtLeastTF{\@ifl@t@r\fmtversion}
\RequirePackage{ltxcmds}[2010/12/02]
\RequirePackage{infwarerr}[2010/04/08]
\RequirePackage{kvsetkeys}[2010/03/01]
\RequirePackage{kvdefinekeys}[2010/03/01]
\RequirePackage{pdftexcmds}[2010/04/01]
%    \end{macrocode}
%
%    \begin{macro}{\ZREF@name}
%    Several times the package name is used, thus we store it
%    in \cmd{\ZREF@name}.
%    \begin{macrocode}
\def\ZREF@name{zref}
%    \end{macrocode}
%    \end{macro}
%
%    \begin{macrocode}
\ltx@IfUndefined{protected}{%
  \RequirePackage{makerobust}[2006/03/18]%
%    \end{macrocode}
%    \begin{macro}{\ZREF@Robust}
%    \begin{macrocode}
  \def\ZREF@Robust#1#2{%
    \def\ZREF@temp{\MakeRobustcommand#2}%
    \afterassignment\ZREF@temp
    #1#2%
  }%
%    \end{macrocode}
%    \end{macro}
%    \begin{macrocode}
}{%
%    \end{macrocode}
%    \begin{macro}{\ZREF@Robust}
%    \begin{macrocode}
  \def\ZREF@Robust#1{%
    \protected#1%
  }%
%    \end{macrocode}
%    \end{macro}
%    \begin{macrocode}
}
%    \end{macrocode}
%    \begin{macro}{\ZREF@IfDefinable}
%    \begin{macrocode}
\def\ZREF@IfDefinable#1#2#3{%
  \@ifdefinable{#1}{%
    \ZREF@Robust{#2}#1#3%
  }%
}
%    \end{macrocode}
%    \end{macro}
%
%    \begin{macro}{\ZREF@UpdatePdfTeX}
%    \cs{ZREF@UpdatePdfTeX} is used as help message text in
%    error messages.
%    \begin{macrocode}
\def\ZREF@UpdatePdfTeX{Update pdfTeX.}
%    \end{macrocode}
%    \end{macro}
%
%    \begin{macro}{\ifZREF@found}
%    The following switch is usded in list processing.
%    \begin{macrocode}
\newif\ifZREF@found
%    \end{macrocode}
%    \end{macro}
%
%    \begin{macro}{\ZREF@patch}
%    Macro \cs{ZREF@patch} first checks the existence of
%    the command and safes it.
%    \begin{macrocode}
\def\ZREF@patch#1{%
  \ltx@IfUndefined{#1}{%
    \ltx@gobble
  }{%
    \expandafter\let\csname ZREF@org@#1\expandafter\endcsname
    \csname #1\endcsname
    \ltx@firstofone
  }%
}
%    \end{macrocode}
%    \end{macro}
%
% \subsubsection{Check for \hologo{eTeX}}
%
%    The use of \hologo{eTeX} should be standard nowadays for \hologo{LaTeX}.
%    We test for \hologo{eTeX} in order to use its features later.
%    \begin{macrocode}
\ltx@IfUndefined{eTeXversion}{%
  \PackageError\ZREF@name{%
    Missing support for eTeX; package is abandoned%
  }{%
    Use a TeX compiler that support eTeX and enable eTeX %
    in the format.%
  }%
  \endinput
}{}%
%    \end{macrocode}
%
%    \begin{macrocode}
\RequirePackage{etexcmds}[2007/09/09]
\ifetex@unexpanded
\else
  \PackageError\ZREF@name{%
    Missing e-TeX's \string\unexpanded.\MessageBreak
    Add \string\RequirePackage\string{etexcmds\string} before %
    \string\documentclass%
  }{%
    Probably you are using some package (e.g. ConTeXt) that %
    redefines \string\unexpanded%
  }%
  \expandafter\endinput
\fi
%    \end{macrocode}
%
% \subsubsection{Auxiliary file stuff}
%
%    We are using some commands in the \xfile{.aux} files.
%    However sometimes these auxiliary files are interpreted
%    by \hologo{LaTeX} processes that haven't loaded this package
%    (e.g. package \xpackage{xr}).
%    Therefore we provide dummy definitions.
%    \begin{macrocode}
\RequirePackage{auxhook}
\AddLineBeginAux{%
  \string\providecommand\string\zref@newlabel[2]{}%
}
%    \end{macrocode}
%    \begin{macro}{\ZREF@RefPrefix}
%    \begin{macrocode}
\def\ZREF@RefPrefix{Z@R}
%    \end{macrocode}
%    \end{macro}
%    \begin{macro}{\zref@newlabel}
%    For the implementation of \cs{zref@newlabel} we call
%    the same internal macro \cs{@newl@bel} that is used
%    in \cs{newlabel}. Thus we have for free:
%    \begin{itemize}
%    \item \cs{Z@R@}\metacs{labelname} is defined.
%    \item \hologo{LaTeX}'s check for multiple references.
%    \item \hologo{LaTeX}'s check for changed references.
%    \end{itemize}
%    \begin{macrocode}
\ZREF@Robust\edef\zref@newlabel{%
  \noexpand\@newl@bel{\ZREF@RefPrefix}%
}
%    \end{macrocode}
%    \end{macro}
%
% \subsubsection{Property lists}
%
%    \begin{macro}{\zref@newlist}
%    Property lists are stored as list of property names
%    enclosed in curly braces. \cs{zref@newlist}
%    creates a new list as empty list. Assignments to
%    property lists are global.
%    \begin{macrocode}
\ZREF@Robust\def\zref@newlist#1{%
  \zref@iflistundefined{#1}{%
    \@ifdefinable{Z@L@#1}{%
     \global\expandafter\let\csname Z@L@#1\endcsname\ltx@empty
     \PackageInfo\ZREF@name{New property list: #1}%
    }%
  }{%
    \PackageError\ZREF@name{%
      Property list `#1' already exists%
    }\@ehc
  }%
}
%    \end{macrocode}
%    \end{macro}
%
%    \begin{macro}{\zref@iflistundefined}
%    \cs{zref@iflistundefined} checks the existence of the
%    property list \verb|#1|. If the property list is present,
%    then \verb|#2| is executed and \verb|#3| otherwise.
%    \begin{macrocode}
\def\zref@iflistundefined#1{%
  \ltx@ifundefined{Z@L@#1}%
}
%    \end{macrocode}
%    \end{macro}
%    \begin{macro}{\zref@listexists}
%    \cs{zref@listexists} only executes \verb|#2| if the
%    property list \verb|#1| exists and raises an error
%    message otherwise.
%    \begin{macrocode}
\ZREF@Robust\def\zref@listexists#1{%
  \zref@iflistundefined{#1}{%
    \PackageError\ZREF@name{%
      Property list `#1' does not exist%
    }\@ehc
  }%
}
%    \end{macrocode}
%    \end{macro}
%
%    \begin{macro}{\zref@iflistcontainsprop}
%    \cs{zref@iflistcontainsprop} checks, whether a property \verb|#2| is
%    already present in a property list \verb|#1|.
%    \begin{macrocode}
\ZREF@Robust\def\zref@iflistcontainsprop#1#2{%
  \zref@iflistundefined{#1}{%
    \ltx@secondoftwo
  }{%
    \begingroup\expandafter\endgroup
    \expandafter\in@
    \csname#2\expandafter\expandafter\expandafter\endcsname
    \expandafter\expandafter\expandafter{\csname Z@L@#1\endcsname}%
    \csname ltx@\ifin@ first\else second\fi oftwo\endcsname
  }%
}
%    \end{macrocode}
%    \end{macro}
%
%    \begin{macro}{\zref@listforloop}
%    \begin{macrocode}
\def\zref@listforloop#1#2{%
  \zref@listexists{#1}{%
    \expandafter\expandafter\expandafter\@tfor
    \expandafter\expandafter\expandafter\zref@prop
    \expandafter\expandafter\expandafter:%
    \expandafter\expandafter\expandafter=%
    \csname Z@L@#1\endcsname
    \do{%
      \begingroup
        \escapechar=-1 %
        \edef\x{\endgroup
          \def\noexpand\zref@prop{%
            \expandafter\string\zref@prop
          }%
        }%
      \x
      #2\zref@prop
    }%
  }%
}
%    \end{macrocode}
%    \end{macro}
%
%    \begin{macro}{\zref@addprops}
%    \cs{zref@addprop} adds the properties \verb|#2| to the property
%    list \verb|#1|, if the property is not already in the list.
%    Otherwise a warning is given.
%    \begin{macrocode}
\ZREF@Robust\def\zref@addprops#1#2{%
  \zref@listexists{#1}{%
    \comma@parse{#2}{%
      \zref@propexists\comma@entry{%
        \zref@iflistcontainsprop{#1}\comma@entry{%
          \PackageWarning\ZREF@name{%
            Property `\comma@entry' is already in list `#1'%
          }%
        }{%
          \begingroup\expandafter\endgroup
          \expandafter\g@addto@macro
          \csname Z@L@#1\expandafter\endcsname
          \expandafter{\csname\comma@entry\endcsname}%
        }%
      }%
      \ltx@gobble
    }%
  }%
}
%    \end{macrocode}
%    \end{macro}
%    \begin{macro}{\zref@addprop}
%    \cs{zref@addprop} adds the property \verb|#2| to the property
%    list \verb|#1|, if the property is not already in the list.
%    Otherwise a warning is given.
%    \begin{macrocode}
\ZREF@Robust\def\zref@addprop#1#2{%
  \zref@listexists{#1}{%
    \zref@propexists{#2}{%
      \zref@iflistcontainsprop{#1}{#2}{%
        \PackageWarning\ZREF@name{%
          Property `#2' is already in list `#1'%
        }%
      }{%
        \begingroup\expandafter\endgroup
        \expandafter\g@addto@macro
        \csname Z@L@#1\expandafter\endcsname
        \expandafter{\csname#2\endcsname}%
      }%
    }%
  }%
}
%    \end{macrocode}
%    \end{macro}
%
%    \begin{macro}{\zref@localaddprops}
%    \begin{macrocode}
\ZREF@Robust\def\zref@localaddprops#1#2{%
  \zref@listexists{#1}{%
    \comma@parse{#2}{%
      \zref@propexists\comma@entry{%
        \zref@iflistcontainsprop{#1}\comma@entry{%
          \PackageWarning\ZREF@name{%
            Property `\comma@entry' is already in list `#1'%
          }%
        }{%
          \begingroup\expandafter\endgroup
          \expandafter\ltx@LocalAppendToMacro
          \csname Z@L@#1\expandafter\endcsname
          \expandafter{\csname\comma@entry\endcsname}%
        }%
      }%
      \ltx@gobble
    }%
  }%
}
%    \end{macrocode}
%    \end{macro}
%    \begin{macro}{\zref@localaddprop}
%    \begin{macrocode}
\ZREF@Robust\def\zref@localaddprop#1#2{%
  \zref@listexists{#1}{%
    \zref@propexists{#2}{%
      \zref@iflistcontainsprop{#1}{#2}{%
        \PackageWarning\ZREF@name{%
          Property `#2' is already in list `#1'%
        }%
      }{%
        \begingroup\expandafter\endgroup
        \expandafter\ltx@LocalAppendToMacro
        \csname Z@L@#1\expandafter\endcsname
        \expandafter{\csname#2\endcsname}%
      }%
    }%
  }%
}
%    \end{macrocode}
%    \end{macro}
%
%    \begin{macrocode}
\ltx@IfUndefined{pdf@strcmp}{%
%    \end{macrocode}
%    \begin{macro}{\zref@delprop}
%    \begin{macrocode}
  \ZREF@Robust\def\zref@delprop{%
    \ZREF@delprop\gdef
  }%
%    \end{macrocode}
%    \end{macro}
%    \begin{macro}{\zref@localdelprop}
%    \begin{macrocode}
  \ZREF@Robust\def\zref@localdelprop{%
    \ZREF@delprop\def
  }%
%    \end{macrocode}
%    \end{macro}
%    \begin{macro}{\ZREF@delprop}
%    \begin{macrocode}
  \def\ZREF@delprop#1#2#3{%
    \zref@listexists{#2}{%
      \begingroup
        \escapechar=-1 %
        \def\ZREF@param{#3}%
        \@onelevel@sanitize\ZREF@param
        \toks@{}%
        \expandafter\expandafter\expandafter\ZREF@@delprop
        \csname Z@L@#2\endcsname!%
      \expandafter\endgroup
      \expandafter#1\csname Z@L@#2\expandafter\endcsname
      \expandafter{%
        \the\toks@
      }%
    }%
  }%
%    \end{macrocode}
%    \end{macro}
%    \begin{macro}{\ZREF@@delprop}
%    \begin{macrocode}
  \def\ZREF@@delprop#1{%
    \expandafter\ZREF@@@delprop\expandafter{\string#1}#1%
  }%
%    \end{macrocode}
%    \end{macro}
%    \begin{macro}{\ZREF@@@delprop}
%    \begin{macrocode}
  \def\ZREF@@@delprop#1#2{%
    \ifx#2!%
    \else
      \def\ZREF@temp{#1}%
      \@onelevel@sanitize\ZREF@temp
      \ifx\ZREF@param\ZREF@temp
      \else
        \toks@\expandafter{%
          \the\expandafter\toks@\csname#1\endcsname
        }%
      \fi
      \expandafter\ZREF@@delprop
    \fi
  }%
%    \end{macrocode}
%    \end{macro}
%    \begin{macrocode}
}{%
%    \end{macrocode}
%    \begin{macro}{\zref@delprop}
%    \begin{macrocode}
  \ZREF@Robust\def\zref@delprop{%
    \ZREF@delprop\xdef
  }%
%    \end{macrocode}
%    \end{macro}
%    \begin{macro}{\zref@localdelprop}
%    \begin{macrocode}
  \ZREF@Robust\def\zref@localdelprop{%
    \ZREF@delprop\edef
  }%
%    \end{macrocode}
%    \end{macro}
%    \begin{macro}{\ZREF@delprop}
%    \begin{macrocode}
  \def\ZREF@delprop#1#2#3{%
    \zref@listexists{#2}{%
      \def\ZREF@param{#3}%
      \edef\ZREF@SavedEscapechar{\the\escapechar}%
      \escapechar=-1 %
      \expandafter#1\csname Z@L@#2%
      \expandafter\expandafter\expandafter\endcsname{%
        \expandafter\expandafter\expandafter\ZREF@@delprop
        \csname Z@L@#2\endcsname!%
      }%
      \escapechar=\ZREF@SavedEscapechar\relax
    }%
  }%
%    \end{macrocode}
%    \end{macro}
%    \begin{macro}{\ZREF@@delprop}
%    Caution: |#1| might be an \cs{if} or similar token.
%    \begin{macrocode}
  \def\ZREF@@delprop#1{%
    \expandafter\ZREF@@@delprop\expandafter{\string#1}#1%
  }%
%    \end{macrocode}
%    \end{macro}
%    \begin{macro}{\ZREF@@@delprop}
%    \begin{macrocode}
  \def\ZREF@@@delprop#1#2{%
    \ifx#2!%
    \else
      \ifnum\pdf@strcmp{#1}{\ZREF@param}=\ltx@zero
      \else
        \expandafter\noexpand\csname#1\endcsname
      \fi
      \expandafter\ZREF@@delprop
    \fi
  }%
%    \end{macrocode}
%    \end{macro}
%    \begin{macrocode}
}
%    \end{macrocode}
%
% \subsubsection{Properties}
%
%    \begin{macro}{\zref@ifpropundefined}
%    \cs{zref@ifpropundefined} checks the existence of the
%    property \verb|#1|. If the property is present,
%    then \verb|#2| is executed and \verb|#3| otherwise.
%    \begin{macrocode}
\def\zref@ifpropundefined#1{%
  \ltx@ifundefined{Z@E@#1}%
}
%    \end{macrocode}
%    \end{macro}
%    \begin{macro}{\zref@propexists}
%    Some macros rely on the existence of a property.
%    \cs{zref@propexists} only executes \verb|#2| if the
%    property \verb|#1| exists and raises an error
%    message otherwise.
%    \begin{macrocode}
\ZREF@Robust\def\zref@propexists#1{%
  \zref@ifpropundefined{#1}{%
    \PackageError\ZREF@name{%
      Property `#1' does not exist%
    }\@ehc
  }%
}
%    \end{macrocode}
%    \end{macro}
%
%    \begin{macro}{\zref@newprop}
%    A new property is declared by \cs{zref@newprop}, the
%    property name \meta{propname} is given in \verb|#1|. The
%    property is created and configured. If the star form
%    is given, then the expansion of the property value
%    is delayed to page shipout time, when the reference
%    is written to the \xfile{.aux} file.
%    \begin{description}
%    \item[\cs{Z@D@}\metacs{propname}:]
%      Stores the default value for this property.
%    \item[\cs{Z@E@}\metacs{propname}:]
%      Extract function.
%    \item[\cs{Z@X@}\metacs{propname}:]
%      Information whether the expansion of the property value
%      is delayed to shipout time.
%    \item[\cs{Z@C@}\metacs{propname}:]
%      Current value of the property.
%    \end{description}
%    \begin{macrocode}
\ZREF@Robust\def\zref@newprop{%
  \@ifstar{%
    \let\ZREF@X\noexpand
    \ZREF@newprop
  }{%
    \let\ZREF@X\ltx@empty
    \ZREF@newprop
  }%
}
%    \end{macrocode}
%    \end{macro}
%    \begin{macro}{\ZREF@newprop}
%    \begin{macrocode}
\def\ZREF@newprop#1{%
  \edef\ZREF@P{#1}%
  \@onelevel@sanitize\ZREF@P
  \begingroup
  \ifx\ZREF@P\ZREF@par
    \@PackageError\ZREF@name{%
      Invalid property name `\ZREF@P'%
    }{%
      The property name `par' is not allowed %
      because of internal reasons.%
      \MessageBreak
      \@ehc
    }%
    \def\ZREF@@newprop[##1]##2{\endgroup}%
  \else
    \zref@ifpropundefined\ZREF@P{%
      \endgroup
      \PackageInfo\ZREF@name{%
        New property: \ZREF@P
      }%
    }{%
      \@PackageError\ZREF@name{%
        Property `\ZREF@P' already exists%
      }\@ehc
      \def\ZREF@@newprop[##1]##2{\endgroup}%
    }%
  \fi
  \@ifnextchar[\ZREF@@newprop{\ZREF@@newprop[\zref@default]}%
}
%    \end{macrocode}
%    \end{macro}
%    \begin{macro}{\ZREF@par}
%    \begin{macrocode}
\def\ZREF@par{par}
\@onelevel@sanitize\ZREF@par
%    \end{macrocode}
%    \end{macro}
%    \begin{macro}{\ZREF@@newprop}
%    \begin{macrocode}
\def\ZREF@@newprop[#1]{%
  \global\@namedef{Z@D@\ZREF@P}{#1}%
  \global\expandafter\let\csname Z@X@\ZREF@P\endcsname\ZREF@X
  \begingroup\expandafter\endgroup
  \expandafter\ZREF@@@newprop\csname\ZREF@P\endcsname
  \expandafter\gdef\csname Z@C@\ZREF@P\endcsname{}%
  \zref@setcurrent\ZREF@P
}
\def\ZREF@@@newprop#1{%
  \expandafter
  \gdef\csname Z@E@\ZREF@P\endcsname##1#1##2##3\ZREF@nil{##2}%
}
%    \end{macrocode}
%    \end{macro}
%
%    \begin{macro}{\zref@showprop}
%    \begin{macrocode}
\ZREF@Robust\def\zref@showprop#1{%
  \zref@ifpropundefined{#1}{%
    \@PackageInfoNoLine{\ZREF@name}{%
      Show property `#1': <undefined>%
    }%
  }{%
    \begingroup
      \toks@\expandafter\expandafter\expandafter{%
        \csname Z@C@#1\endcsname
      }%
      \edef\ZREF@value{\the\toks@}%
      \ltx@onelevel@sanitize\ZREF@value
      \toks@\expandafter\expandafter\expandafter{%
        \csname Z@D@#1\endcsname
      }%
      \edef\ZREF@default{\the\toks@}%
      \ltx@onelevel@sanitize\ZREF@default
      \@PackageInfoNoLine{\ZREF@name}{%
        Show property `#1':\MessageBreak
        \expandafter\ifx\csname Z@X@#1\endcsname\ltx@empty
          Immediate %
        \else
          Delayed %
        \fi
        value: [\ZREF@value]\MessageBreak
        Default: [\ZREF@default]%
      }%
    \endgroup
  }%
}
%    \end{macrocode}
%    \end{macro}
%
%    \begin{macro}{\zref@setcurrent}
%    \cs{zref@setcurrent} sets the current value for a property.
%    \begin{macrocode}
\ZREF@Robust\def\zref@setcurrent#1#2{%
  \zref@propexists{#1}{%
    \expandafter\def\csname Z@C@#1\endcsname{#2}%
  }%
}
%    \end{macrocode}
%    \end{macro}
%
%    \begin{macro}{\ZREF@getcurrent}
%    \cs{zref@getcurrent} gets the current value for a property.
%    \begin{macrocode}
\def\ZREF@getcurrent#1{%
  \romannumeral0%
  \ltx@ifundefined{Z@C@#1}{%
    \ltx@space
  }{%
    \expandafter\expandafter\expandafter\ltx@space
    \csname Z@C@#1\endcsname
  }%
}
%    \end{macrocode}
%    \end{macro}
%    \begin{macro}{\ZREF@u@getcurrent}
%    \begin{macrocode}
\def\ZREF@wu@getcurrent#1{%
  \etex@unexpanded\expandafter\expandafter\expandafter{%
    \ZREF@getcurrent{#1}%
  }%
}
%    \end{macrocode}
%    \end{macro}
%    \begin{macro}{\zref@getcurrent}
%    \begin{macrocode}
\let\zref@getcurrent\ZREF@getcurrent
%    \end{macrocode}
%    \end{macro}
%
% \subsubsection{Reference generation}
%
%    \begin{macro}{\zref@label}
%    Label macro that uses the main property list.
%    \begin{macrocode}
\ZREF@Robust\def\zref@label#1{%
  \zref@labelbylist{#1}\ZREF@mainlist
}
%    \end{macrocode}
%    \end{macro}
%    \begin{macro}{\zref@labelbylist}
%    Label macro that stores the properties, specified in the
%    property list \verb|#2|.
%    \begin{macrocode}
\ZREF@Robust\def\zref@labelbylist#1#2{%
  \@bsphack
    \zref@listexists{#2}{%
      \expandafter\expandafter\expandafter\ZREF@label
      \expandafter\expandafter\expandafter{%
        \csname Z@L@#2\endcsname
      }{#1}%
    }%
  \@esphack
}
%    \end{macrocode}
%    \end{macro}
%    \begin{macro}{\zref@labelbyprops}
%    The properties are directly specified in a comma separated list.
%    \begin{macrocode}
\ZREF@Robust\def\zref@labelbyprops#1#2{%
  \@bsphack
    \begingroup
      \toks@{}%
      \comma@parse{#2}{%
        \zref@ifpropundefined\comma@entry{%
          \PackageWarning\ZREF@name{%
            Property `\comma@entry' is not known%
          }%
        }{%
          \toks@\expandafter{%
            \the\expandafter\toks@\csname\comma@entry\endcsname
          }%
        }%
        \ltx@gobble
      }%
    \expandafter\endgroup
    \expandafter\ZREF@label\expandafter{\the\toks@}{#1}%
  \@esphack
}
%    \end{macrocode}
%    \end{macro}
%    \begin{macro}{\zref@labelbykv}
%    \begin{macrocode}
\ZREF@Robust\def\zref@labelbykv#1#2{%
  \@bsphack
    \begingroup
      \let\Z@L@ZREF@temp\ltx@empty
      \kvsetkeys{ZREF@LABEL}{#1}%
      \ifZREF@immediate
        \expandafter\zref@wrapper@immediate\expandafter{%
          \expandafter\ZREF@label\expandafter{\Z@L@ZREF@temp}{#2}%
        }%
      \else
        \expandafter\ZREF@label\expandafter{\Z@L@ZREF@temp}{#2}%
      \fi
    \endgroup
  \@esphack
}
%    \end{macrocode}
%    \end{macro}
%    \begin{macrocode}
\kv@define@key{ZREF@LABEL}{prop}{%
  \edef\ZREF@param{#1}%
  \zref@propexists\ZREF@param{%
    \zref@iflistcontainsprop{ZREF@temp}\ZREF@param{}{%
      \begingroup\expandafter\endgroup
      \expandafter\ltx@LocalAppendToMacro
      \expandafter\Z@L@ZREF@temp
      \expandafter{\csname\ZREF@param\endcsname}%
    }%
  }%
}
\kv@define@key{ZREF@LABEL}{list}{%
  \zref@listforloop{#1}{%
    \zref@iflistcontainsprop{ZREF@temp}\zref@prop{}{%
      \begingroup\expandafter\endgroup
      \expandafter\ltx@LocalAppendToMacro
      \expandafter\Z@L@ZREF@temp
      \expandafter{\csname\zref@prop\endcsname}%
    }%
    \ltx@gobble
  }%
}
\kv@define@key{ZREF@LABEL}{delprop}{%
  \zref@propexists{#1}{%
    \zref@localdelprop{ZREF@temp}{#1}%
  }%
}
\kv@define@key{ZREF@LABEL}{immediate}[true]{%
  \edef\ZREF@param{#1}%
  \ifx\ZREF@param\ZREF@true
    \ZREF@immediatetrue
  \else
    \ifx\ZREF@param\ZREF@false
      \ZREF@immediatefalse
    \else
      \PackageWarning\ZREF@name{%
        Option `immediate' expects `true' or `false'.\MessageBreak
        Ignoring invalid value `\ZREF@param'%
      }%
    \fi
  \fi
}
%    \end{macrocode}
%    \begin{macro}{\ZREF@false}
%    \begin{macrocode}
\def\ZREF@false{false}
%    \end{macrocode}
%    \end{macro}
%    \begin{macro}{\ZREF@true}
%    \begin{macrocode}
\def\ZREF@true{true}
%    \end{macrocode}
%    \end{macro}
%
%    \begin{macrocode}
\kv@define@key{ZREF@LABEL}{values}[]{%
  \kv@parse{#1}{%
    \ifx\kv@value\relax
      \@PackageWarning\ZREF@name{%
        Missing value for property `\kv@key'%
      }%
      \expandafter\ltx@gobbletwo
    \else
      \expandafter\zref@setcurrent
    \fi
  }%
}
%    \end{macrocode}
%
%    \begin{macro}{\ifZREF@immediate}
%    The switch \cs{ifZREF@immediate} tells us, whether the
%    label should be written immediately or at page shipout time.
%    \cs{ZREF@label} need to be notified about this, because
%    it must disable the deferred execution of property values,
%    if the label is written immediately.
%    \begin{macrocode}
\newif\ifZREF@immediate
%    \end{macrocode}
%    \end{macro}
%    \begin{macro}{\zref@wrapper@immediate}
%    The argument of \cs{zref@wrapper@immediate} is executed inside
%    a group where \cs{write} is redefined by adding
%    \cs{immediate} before its execution. Also \cs{ZREF@label}
%    is notified via the switch \cs{ifZREF@immediate}.
%    \begin{macrocode}
\ZREF@Robust{\long\def}\zref@wrapper@immediate#1{%
  \begingroup
    \ZREF@immediatetrue
    \let\ZREF@org@write\write
    \def\write{\immediate\ZREF@org@write}%
    #1%
  \endgroup
}
%    \end{macrocode}
%    \end{macro}
%
%    \begin{macro}{\ZREF@label}
%    \cs{ZREF@label} writes the data in the \xfile{.aux} file.
%    \verb|#1| contains the list of valid properties,
%    \verb|#2| the name of the reference. In case of immediate
%    writing, the deferred execution of property values is
%    disabled. Also \thepage is made expandable in this case.
%    \begin{macrocode}
\def\ZREF@label#1#2{%
  \if@filesw
    \begingroup
      \ifZREF@immediate
        \let\ZREF@org@thepage\thepage
      \fi
      \protected@write\@auxout{%
        \ifZREF@immediate
          \let\thepage\ZREF@org@thepage
        \fi
        \let\ZREF@temp\ltx@empty
        \@tfor\ZREF@P:=#1\do{%
          \begingroup
            \escapechar=-1 %
            \edef\x{\endgroup
              \def\noexpand\ZREF@P{%
                \expandafter\string\ZREF@P
              }%
            }%
          \x
          \expandafter\ifx
              \csname
                \ifZREF@immediate
                  relax%
                \else
                  Z@X@\ZREF@P%
                \fi
              \endcsname
              \noexpand
            \expandafter\let\csname Z@C@\ZREF@P\endcsname\relax
          \fi
          \toks@\expandafter{\ZREF@temp}%
          \edef\ZREF@temp{%
            \the\toks@
            \ltx@backslashchar\ZREF@P{%
              \expandafter\noexpand\csname Z@C@\ZREF@P\endcsname
            }%
          }%
        }%
      }{%
        \string\zref@newlabel{#2}{\ZREF@temp}%
      }%
    \endgroup
  \fi
}
\def\ZREF@addtoks#1{%
  \toks@\expandafter\expandafter\expandafter{%
    \expandafter\the\expandafter\toks@#1%
  }%
}
%    \end{macrocode}
%    \end{macro}
%
% \subsubsection{Reference querying and extracting}
%
%    Design goal for the extracting macros is that the extraction
%    process is full expandable. Thus these macros can be used
%    in expandable contexts. But there are problems that cannot
%    be solved by full expandable macros:
%    \begin{itemize}
%    \item In standard \hologo{LaTeX} undefined references sets a flag
%      and generate a warning. Both actions are not expandable.
%    \item Babel's support for its shorthand uses commands that
%      use non-expandable assignments. However currently there
%      is hope, that primitives are added to \pdfTeX\ that
%      allows the detection of contexts. Then the shorthand
%      can detect, if they are executed inside \cs{csname}
%      and protect themselves automatically.
%    \end{itemize}
%    \begin{macro}{\zref@ifrefundefined}
%    If a reference \verb|#1| is undefined, then macro
%    \cs{zref@ifrefundefined} calls \verb|#2| and \verb|#3| otherwise.
%    \begin{macrocode}
\def\zref@ifrefundefined#1{%
  \ltx@ifundefined{Z@R@#1}%
}
%    \end{macrocode}
%    \end{macro}
%
%    \begin{macro}{\zifrefundefined}
%    If a reference \verb|#1| is undefined, then macro
%    \cs{zref@ifrefundefined} calls \verb|#2| and \verb|#3| otherwise.
%    Also the reference is marked used.
%    \begin{macrocode}
\ZREF@IfDefinable\zifrefundefined\def{%
  #1{%
    \zref@wrapper@babel\ZREF@ifrefundefined{#1}%
  }%
}
%    \end{macrocode}
%    \end{macro}
%    \begin{macro}{\ZREF@ifrefundefined}
%    \begin{macrocode}
\def\ZREF@ifrefundefined#1{%
  \zref@refused{#1}%
  \zref@ifrefundefined{#1}%
}
%    \end{macrocode}
%    \end{macro}
%
%    \begin{macro}{\zref@refused}
%    The problem with undefined references is addressed by
%    the macro \cs{zref@refused}. This can be used outside
%    the expandable context. In case of an undefined reference
%    the flag is set to notify \hologo{LaTeX} and a warning is given.
%    \begin{macrocode}
\ZREF@Robust\def\zref@refused#1{%
  \zref@wrapper@babel\ZREF@refused{#1}%
}
%    \end{macrocode}
%    \end{macro}
%    \begin{macro}{\ZREF@refused}
%    \begin{macrocode}
\def\ZREF@refused#1{%
  \zref@ifrefundefined{#1}{%
    \protect\G@refundefinedtrue
    \@latex@warning{%
      Reference `#1' on page \thepage \space undefined%
    }%
  }{}%
}
%    \end{macrocode}
%    \end{macro}
%
%    \begin{macro}{\zref@ifrefcontainsprop}
%    \cs{zref@ifrefcontainsprop} looks, if the reference \verb|#1|
%    has the property \verb|#2| and calls then \verb|#3| and
%    \verb|#4| otherwise.
%    \begin{macrocode}
\def\zref@ifrefcontainsprop#1#2{%
  \zref@ifrefundefined{#1}{%
    \ltx@secondoftwo
  }{%
    \expandafter\ZREF@ifrefcontainsprop
    \csname Z@E@#2\expandafter\endcsname
    \csname#2\expandafter\expandafter\expandafter\endcsname
    \expandafter\expandafter\expandafter{%
      \csname Z@R@#1\endcsname
    }%
  }%
}
\def\ZREF@ifrefcontainsprop#1#2#3{%
  \expandafter\ifx\expandafter\ZREF@novalue
  #1#3#2\ZREF@novalue\ZREF@nil\ltx@empty
    \expandafter\ltx@secondoftwo
  \else
    \expandafter\ltx@firstoftwo
  \fi
}
\def\ZREF@novalue{\ZREF@NOVALUE}
%    \end{macrocode}
%    \end{macro}
%
%    \begin{macro}{\zref@extract}
%    \cs{zref@extract} is an abbreviation for the case
%    that the default of the property is used as default
%    value.
%    \begin{macrocode}
\def\ZREF@extract#1#2{%
  \romannumeral0%
  \ltx@ifundefined{Z@D@#2}{%
    \expandafter\ltx@space\zref@default
  }{%
    \expandafter\expandafter\expandafter\ZREF@@extract
    \expandafter\expandafter\expandafter{%
      \csname Z@D@#2\endcsname
    }{#1}{#2}%
  }%
}
%    \end{macrocode}
%    \end{macro}
%    \begin{macro}{\ZREF@@extract}
%    \begin{macrocode}
\def\ZREF@@extract#1#2#3{%
  \expandafter\expandafter\expandafter\ltx@space
  \zref@extractdefault{#2}{#3}{#1}%
}
%    \end{macrocode}
%    \end{macro}
%    \begin{macro}{\ZREF@wu@extract}
%    \begin{macrocode}
\def\ZREF@wu@extract#1#2{%
  \etex@unexpanded\expandafter\expandafter\expandafter{%
    \ZREF@extract{#1}{#2}%
  }%
}
%    \end{macrocode}
%    \end{macro}
%    \begin{macro}{\zref@extract}
%    \begin{macrocode}
\let\zref@extract\ZREF@extract
%    \end{macrocode}
%    \end{macro}
%
%    \begin{macro}{\ZREF@extractdefault}
%    The basic extracting macro is \cs{zref@extractdefault} with
%    the reference name in \verb|#1|, the property in \verb|#2| and
%    the default value in \verb|#3| in case for problems.
%    \begin{macrocode}
\def\ZREF@extractdefault#1#2#3{%
  \romannumeral0%
  \zref@ifrefundefined{#1}\ltx@firstoftwo{%
    \zref@ifpropundefined{#2}\ltx@firstoftwo\ltx@secondoftwo
  }{%
    \ltx@space
    #3%
  }{%
    \expandafter\expandafter\expandafter\ltx@space
    \csname Z@E@#2\expandafter\expandafter\expandafter\endcsname
    \csname Z@R@#1\expandafter\endcsname
    \csname#2\endcsname{#3}\ZREF@nil
  }%
}
%    \end{macrocode}
%    \end{macro}
%    \begin{macro}{\ZREF@wu@extractdefault}
%    \begin{macrocode}
\def\ZREF@wu@extractdefault#1#2#3{%
  \etex@unexpanded\expandafter\expandafter\expandafter{%
    \ZREF@extractdefault{#1}{#2}{#3}%
  }%
}
%    \end{macrocode}
%    \end{macro}
%    \begin{macro}{\zref@extractdefault}
%    \begin{macrocode}
\let\zref@extractdefault\ZREF@extractdefault
%    \end{macrocode}
%    \end{macro}
%
%    \begin{macro}{\zref@def@extract}
%    \begin{macrocode}
\ZREF@Robust\def\zref@def@extract#1{%
  \zref@wrapper@babel{\ZREF@def@extract{#1}}%
}
%    \end{macrocode}
%    \end{macro}
%    \begin{macro}{\ZREF@def@extract}
%    \begin{macrocode}
\def\ZREF@def@extract#1#2#3{%
  \zref@refused{#2}%
  \expandafter\expandafter\expandafter\def
  \expandafter\expandafter\expandafter#1%
  \expandafter\expandafter\expandafter{%
    \zref@extract{#2}{#3}%
  }%
}
%    \end{macrocode}
%    \end{macro}
%    \begin{macro}{\zref@def@extractdefault}
%    \begin{macrocode}
\ZREF@Robust\def\zref@def@extractdefault#1{%
  \zref@wrapper@babel{\ZREF@def@extractdefault{#1}}%
}
%    \end{macrocode}
%    \end{macro}
%    \begin{macro}{\ZREF@def@extractdefault}
%    \begin{macrocode}
\def\ZREF@def@extractdefault#1#2#3#4{%
  \zref@refused{#2}%
  \expandafter\expandafter\expandafter\def
  \expandafter\expandafter\expandafter#1%
  \expandafter\expandafter\expandafter{%
    \zref@extractdefault{#2}{#3}{#4}%
  }%
}
%    \end{macrocode}
%    \end{macro}
%
%    \begin{macro}{\ZREF@wrapper@unexpanded}
%    \begin{macrocode}
\ZREF@Robust{\long\def}\ZREF@wrapper@unexpanded#1{%
  \let\zref@wrapper@unexpanded\ltx@firstofone
  \let\zref@getcurrent\ZREF@wu@getcurrent
  \let\zref@extractdefault\ZREF@wu@extractdefault
  \let\zref@extract\ZREF@wu@extract
  #1%
  \let\zref@wrapper@unexpanded\ZREF@wrapper@unexpanded
  \let\zref@getcurrent\ZREF@getcurrent
  \let\zref@extractdefault\ZREF@extractdefault
  \let\zref@extract\ZREF@extract
}
%    \end{macrocode}
%    \end{macro}
%    \begin{macro}{\zref@wrapper@unexpanded}
%    \begin{macrocode}
\ltx@IfUndefined{etex@unexpanded}{%
  \let\zref@wrapper@unexpanded\ltx@firstofone
}{%
  \let\zref@wrapper@unexpanded\ZREF@wrapper@unexpanded
}
%    \end{macrocode}
%    \end{macro}
%
% \subsubsection{Compatibility with \xpackage{babel}}
%
%    \begin{macro}{\zref@wrapper@babel}
%    \begin{macrocode}
\ZREF@Robust{\long\def}\zref@wrapper@babel#1#2{%
  \ifcsname if@safe@actives\endcsname
    \expandafter\ltx@firstofone
  \else
    \expandafter\ltx@secondoftwo
  \fi
  {%
    \if@safe@actives
      \expandafter\ltx@secondoftwo
    \else
      \expandafter\ltx@firstoftwo
    \fi
    {%
      \begingroup
        \csname @safe@activestrue\endcsname
        \edef\x{#2}%
      \expandafter\endgroup
      \expandafter\ZREF@wrapper@babel\expandafter{\x}{#1}%
    }%
  }{%
    #1{#2}%
  }%
}
\long\def\ZREF@wrapper@babel#1#2{%
  #2{#1}%
}
%    \end{macrocode}
%    \end{macro}
%
% \subsubsection{Unique counter support}
%
%    \begin{macro}{\zref@require@unique}
%    Generate the counter |zref@unique| if the counter does not
%    already exist.
%    \begin{macrocode}
\ZREF@Robust\def\zref@require@unique{%
  \@ifundefined{c@zref@unique}{%
    \begingroup
      \let\@addtoreset\ltx@gobbletwo
      \newcounter{zref@unique}%
    \endgroup
%    \end{macrocode}
%    \end{macro}
%
%    \begin{macro}{\thezref@unique}
%    \cs{thezref@unique} is used for automatically generated unique
%    labelnames.
%    \begin{macrocode}
    \renewcommand*{\thezref@unique}{%
      zref@\number\c@zref@unique
    }%
  }{}%
}
%    \end{macrocode}
%    \end{macro}
%
% \subsubsection{Utilities}
%
%    \begin{macro}{\ZREF@number}
%    \begin{macrocode}
\ltx@IfUndefined{numexpr}{%
  \def\ZREF@number#1{\number#1}%
}{%
  \def\ZREF@number#1{\the\numexpr(#1)\relax}%
}
%    \end{macrocode}
%    \end{macro}
%
% \subsubsection{Setup}
%
%    \begin{macro}{\zref@setdefault}
%    Standard \hologo{LaTeX} prints ``??'' in bold face if a reference
%    is not known. \cs{zref@default} holds the text that is
%    printed in case of unknown references and is used, if
%    the default was not specified during the definition
%    of the new property by \cs{ref@newprop}.
%    The global default value can be set by \cs{zref@setdefault}.
%    \begin{macrocode}
\ZREF@Robust\def\zref@setdefault#1{%
  \def\zref@default{#1}%
}
%    \end{macrocode}
%    \end{macro}
%    \begin{macro}{\zref@default}
%    Now we initialize \cs{zref@default} with the same value
%    that \hologo{LaTeX} uses for its undefined references.
%    \begin{macrocode}
\zref@setdefault{%
  \nfss@text{\reset@font\bfseries ??}%
}
%    \end{macrocode}
%    \end{macro}
%
% \paragraph{Main property list.}
%
%    \begin{macro}{\zref@setmainlist}
%    The name of the default property list is stored in
%    \cs{ZREF@mainlist} and can be set by \cs{zref@setmainlist}.
%    \begin{macrocode}
\ZREF@Robust\def\zref@setmainlist#1{%
  \def\ZREF@mainlist{#1}%
}
\zref@setmainlist{main}
%    \end{macrocode}
%    \end{macro}
%    Now we create the list.
%    \begin{macrocode}
\zref@newlist\ZREF@mainlist
%    \end{macrocode}
%
% \paragraph{Main properties.}
%
%    The two properties \texttt{default} and \texttt{page}
%    are created and added to the main property list.
%    They store the data that
%    standard \hologo{LaTeX} uses in its references created by
%    \cs{label}.
%    \begin{description}
%    \item[\texttt{default}] the apperance of the latest
%      counter that is incremented by \cs{refstepcounter}
%    \item[\texttt{page}] the apperance of the page counter
%    \end{description}
%    \begin{macrocode}
\zref@newprop{default}{\@currentlabel}
\zref@newprop*{page}{\thepage}
\zref@addprops\ZREF@mainlist{default,page}
%    \end{macrocode}
%
% \paragraph{Properties}
%
%    \begin{macro}{\ZREF@NewPropAnchor}
%    \begin{macrocode}
\def\ZREF@NewPropAnchor{%
  \zref@newprop{anchor}{%
    \ltx@ifundefined{@currentHref}{}{\@currentHref}%
  }%
  \global\let\ZREF@NewPropAnchor\relax
}
%    \end{macrocode}
%    \end{macro}
%    \begin{macro}{\zref@titleref@current}
%    Later we will redefine the section and caption macros
%    to catch the current title and remember the value in
%    \cs{zref@titleref@current}.
%    \begin{macro}{\ZREF@NewPropTitle}
%    \begin{macrocode}
\def\ZREF@NewPropTitle{%
  \gdef\zref@titleref@current{}%
  \zref@newprop{title}{\zref@titleref@current}%
  \global\let\ZREF@NewPropTitle\relax
}
%    \end{macrocode}
%    \end{macro}
%    \end{macro}
%    \begin{macro}{\ZREF@NewPropTheotype}
%    \begin{macrocode}
\def\ZREF@NewPropTheotype{%
  \zref@newprop{theotype}{}%
  \global\let\ZREF@NewPropTheotype\relax
}
%    \end{macrocode}
%    \end{macro}
%    \begin{macro}{\ZREF@NewPropPageValue}
%    \begin{macrocode}
\def\ZREF@NewPropPageValue{%
  \zref@newprop*{pagevalue}[0]{\number\c@page}%
  \global\let\ZREF@NewPropPageValue\relax
}
%    \end{macrocode}
%    \end{macro}
%
% \paragraph{Mark successful loading}
%
%    \begin{macrocode}
\let\ZREF@base@ok=Y
%</base>
%    \end{macrocode}
%
%
% \subsection{Module \xmodule{user}}
%
%    \begin{macrocode}
%<*user>
\NeedsTeXFormat{LaTeX2e}
\ProvidesPackage{zref-user}%
  [2020-07-03 v2.32 Module user for zref (HO)]%
\RequirePackage{zref-base}[2019/11/29]
\ifx\ZREF@base@ok Y%
\else
  \expandafter\endinput
\fi
%    \end{macrocode}
%
%    Module \xmodule{user} enables a small user interface.
%    All macros are prefixed by \cs{z}.
%
%    First we define the pendants to the standard \hologo{LaTeX} referencing
%    commands \cs{label}, \cs{ref}, and \cs{pageref}.
%
%    \begin{macro}{\zlabel}
%    Similar to \cs{label} the macro \cs{zlabel} writes a reference
%    entry in the \xfile{.aux} file. The main property list is used.
%    Also we add the babel patch. The \cs{label} command can also be
%    used inside section titles, but it must not go into the
%    table of contents. Therefore we have to check this situation.
%    \begin{macrocode}
\newcommand*\zlabel{%
  \ifx\label\ltx@gobble
    \expandafter\ltx@gobble
  \else
    \expandafter\zref@wrapper@babel\expandafter\zref@label
  \fi
}%
%    \end{macrocode}
%    \end{macro}
%
%    \begin{macro}{\zkvlabel}
%    \begin{macrocode}
\newcommand*{\zkvlabel}[1]{%
  \ifx\label\ltx@gobble
    \expandafter\ltx@gobblethree
  \fi
  \zref@wrapper@babel{\zref@labelbykv{#1}}%
}%
%    \end{macrocode}
%    \end{macro}
%
%    \begin{macro}{\zref}
%    Macro \cs{zref} is the corresponding macro for \cs{ref}.
%    Also it provides an optional argument in order to
%    select another property.
%    \begin{macrocode}
\newcommand*{\zref}[2][default]{% robust because of optional argument
  \zref@propexists{#1}{%
    \zref@wrapper@babel\ZREF@zref{#2}{#1}%
  }%
}%
\def\ZREF@zref#1{%
  \zref@refused{#1}%
  \zref@extract{#1}%
}%
%    \end{macrocode}
%    \end{macro}
%
%    \begin{macro}{\zpageref}
%    For macro \cs{zpageref} we just call \cs{zref}
%    with property \verb|page|.
%    \begin{macrocode}
\ZREF@IfDefinable\zpageref\def{%
  {\zref[page]}%
}
%    \end{macrocode}
%    \end{macro}
%
%    \begin{macro}{\zrefused}
%    For the following expandible user macros \cs{zrefused}
%    should be used to notify \hologo{LaTeX} in case of undefined
%    references.
%    \begin{macrocode}
\ZREF@IfDefinable\zrefused\def{%
  {\zref@refused}%
}
%    \end{macrocode}
%    \end{macro}
%
%    \begin{macrocode}
%</user>
%    \end{macrocode}
%
%
% \subsection{Module \xmodule{abspage}}
%
%    \begin{macrocode}
%<*abspage>
\NeedsTeXFormat{LaTeX2e}
\ProvidesPackage{zref-abspage}%
  [2020-07-03 v2.32 Module abspage for zref (HO)]%
\RequirePackage{zref-base}[2019/11/29]
\ifx\ZREF@base@ok Y%
\else
  \expandafter\endinput
\fi
%    \end{macrocode}
%
%    Module \xmodule{abspage} adds a new property \verb|abspage|
%    to the \verb|main| property list for absolute page numbers.
%    These are recorded by the help of package \xpackage{atbegshi}.
%    \begin{macrocode}
\RequirePackage{atbegshi}[2011/10/05]%
%    \end{macrocode}
%    The counter \verb|abspage| must not go in the clear list
%    of \verb|@ckpt| that is used to set counters in
%    \xfile{.aux} files of included \hologo{TeX} files.
%    \begin{macrocode}
\begingroup
  \let\@addtoreset\ltx@gobbletwo
  \newcounter{abspage}%
\endgroup
\setcounter{abspage}{0}%
\AtBeginShipout{%
  \stepcounter{abspage}%
}%
\zref@newprop*{abspage}[0]{\the\c@abspage}%
\zref@addprop\ZREF@mainlist{abspage}%
%    \end{macrocode}
%    Note that counter \verb|abspage| shows the
%    previous page during page processing. Before
%    shipout the counter is incremented.
%    Thus the property is correctly written with
%    deferred writing. If the counter is written
%    using \cs{zref@wrapper@immediate}, then the number
%    is too small by one.
%
%    \begin{macrocode}
%</abspage>
%    \end{macrocode}
%
%
% \subsection{Module \xmodule{counter}}
%
%    \begin{macrocode}
%<*counter>
\NeedsTeXFormat{LaTeX2e}
\ProvidesPackage{zref-counter}%
  [2020-07-03 v2.32 Module counter for zref (HO)]%
\RequirePackage{zref-base}[2019/11/29]
\ifx\ZREF@base@ok Y%
\else
  \expandafter\endinput
\fi
%    \end{macrocode}
%
%    For features such as \xpackage{hyperref}'s \cs{autoref}
%    we need the name of the counter. The property
%    \verb|counter| is defined and added to the main property
%    list. Starting with \LaTeX\ 2020-10-01 the proper can use
%    \@currentcounter. In older formats
%    \cs{refstepcounter} has to be patched but this can fail in some
%    cases, see issue \#5.
%    \begin{macrocode}
\@ifl@t@r\fmtversion{2020-10-01}
 {
  \zref@newprop{counter}{\@currentcounter}
  \zref@addprop\ZREF@mainlist{counter}
 }
 {
  \zref@newprop{counter}{}
  \zref@addprop\ZREF@mainlist{counter}
  \AtBeginDocument{%
    \ZREF@patch{refstepcounter}{%
      \def\refstepcounter#1{%
        \zref@setcurrent{counter}{#1}%
        \ZREF@org@refstepcounter{#1}%
      }%
    }%
  }
 }
%    \end{macrocode}
%    \begin{macrocode}
%</counter>
%    \end{macrocode}
%
%
% \subsection{Module \xmodule{lastpage}}
%
%    \begin{macrocode}
%<*lastpage>
\NeedsTeXFormat{LaTeX2e}
\ProvidesPackage{zref-lastpage}%
  [2020-07-03 v2.32 Module lastpage for zref (HO)]%
\RequirePackage{zref-base}[2019/11/29]
\RequirePackage{zref-abspage}[2019/11/29]
\RequirePackage{atveryend}[2009/12/07]
\ifx\ZREF@base@ok Y%
\else
  \expandafter\endinput
\fi
%    \end{macrocode}
%
%    The module \xmodule{lastpage} implements the service
%    of package \xpackage{lastpage} by setting a reference
%    \verb|LastPage| at the end of the document.
%    If module \xmodule{abspage} is given, also the absolute
%    page number is available, because the properties of
%    the main property list are used.
%    \begin{macrocode}
\zref@newlist{LastPage}
\AfterLastShipout{%
  \if@filesw
    \begingroup
      \advance\c@page\m@ne
      \toks@\expandafter\expandafter\expandafter{%
        \expandafter\Z@L@main
        \Z@L@LastPage
      }%
      \expandafter\zref@wrapper@immediate\expandafter{%
        \expandafter\ZREF@label\expandafter{\the\toks@}{LastPage}%
      }%
    \endgroup
  \fi
}
%    \end{macrocode}
%    \begin{macro}{\zref@iflastpage}
%    \begin{macrocode}
\def\zref@iflastpage#1{%
  \ifnum\zref@extractdefault{#1}{abspage}{-1}=%
        \zref@extractdefault{LastPage}{abspage}{-2} %
    \expandafter\ltx@firstoftwo
  \else
    \expandafter\ltx@secondoftwo
  \fi
}
%    \end{macrocode}
%    \end{macro}
%    \begin{macro}{\ziflastpage}
%    \begin{macrocode}
\ZREF@IfDefinable\ziflastpage\def{%
  {\zref@wrapper@babel\ZREF@iflastpage}%
}
%    \end{macrocode}
%    \end{macro}
%    \begin{macro}{ZREF@iflastpage}
%    \begin{macrocode}
\def\ZREF@iflastpage#1{%
  \zref@refused{LastPage}%
  \zref@refused{#1}%
  \zref@iflastpage{#1}%
}
%    \end{macrocode}
%    \end{macro}
%    \begin{macrocode}
%</lastpage>
%    \end{macrocode}
%
%
% \subsection{Module \xmodule{thepage}}
%
%    \begin{macrocode}
%<*thepage>
\NeedsTeXFormat{LaTeX2e}
\ProvidesPackage{zref-thepage}%
  [2020-07-03 v2.32 Module thepage for zref (HO)]%
\RequirePackage{zref-base}[2019/11/29]
\ifx\ZREF@base@ok Y%
\else
  \expandafter\endinput
\fi
%    \end{macrocode}
%
%    \begin{macrocode}
\RequirePackage{atbegshi}[2011/10/05]
\RequirePackage{zref-abspage}[2019/11/29]
%    \end{macrocode}
%    \begin{macrocode}
\zref@newlist{thepage}
\zref@addprop{thepage}{page}
\ZREF@NewPropPageValue
%    \end{macrocode}
%    \begin{macro}{\zref@thepage@atbegshi@hook}
%    \begin{macrocode}
\let\zref@thepage@atbegshi@hook\ltx@empty
%    \end{macrocode}
%    \end{macro}
%    \begin{macrocode}
\zref@addprop{thepage}{pagevalue}
\AtBeginShipout{%
  \AtBeginShipoutAddToBox{%
    \zref@thepage@atbegshi@hook
    \zref@labelbylist{thepage\the\value{abspage}}{thepage}%
  }%
}
%    \end{macrocode}
%    \begin{macro}{\zref@thepage@name}
%    \begin{macrocode}
\ltx@IfUndefined{numexpr}{%
  \def\zref@thepage@name#1{thepage\number#1}%
}{%
  \def\zref@thepage@name#1{thepage\the\numexpr#1}%
}
%    \end{macrocode}
%    \end{macro}
%    \begin{macro}{\zref@thepage}
%    \begin{macrocode}
\def\zref@thepage#1{%
  \zref@extract{\zref@thepage@name{#1}}{page}%
}%
%    \end{macrocode}
%    \end{macro}
%    \begin{macro}{\zref@thepage@refused}
%    \begin{macrocode}
\ZREF@Robust\def\zref@thepage@refused#1{%
  \zref@refused{\zref@thepage@name{#1}}%
}%
%    \end{macrocode}
%    \end{macro}
%    \begin{macro}{\zthepage}
%    \begin{macrocode}
\ZREF@IfDefinable\zthepage\def{%
  #1{%
    \zref@thepage@refused{#1}%
    \zref@thepage{#1}%
  }%
}
%    \end{macrocode}
%    \end{macro}
%    \begin{macrocode}
%</thepage>
%    \end{macrocode}
%
%
% \subsection{Module \xmodule{nextpage}}
%
%    \begin{macrocode}
%<*nextpage>
\NeedsTeXFormat{LaTeX2e}
\ProvidesPackage{zref-nextpage}%
  [2020-07-03 v2.32 Module nextpage for zref (HO)]%
\RequirePackage{zref-base}[2019/11/29]
\ifx\ZREF@base@ok Y%
\else
  \expandafter\endinput
\fi
%    \end{macrocode}
%
%    \begin{macrocode}
\RequirePackage{zref-abspage}[2019/11/29]
\RequirePackage{zref-thepage}[2019/11/29]
\RequirePackage{zref-lastpage}[2019/11/29]
\RequirePackage{uniquecounter}[2009/12/18]
%    \end{macrocode}
%    \begin{macrocode}
\UniqueCounterNew{znextpage}

\newcommand*{\znextpagesetup}{%
  \afterassignment\ZREF@np@setup@i
  \def\ZREF@np@call@unknown##1%
}
\def\ZREF@np@setup@i{%
  \afterassignment\ZREF@np@setup@ii
  \def\ZREF@np@call@nonext##1%
}
\def\ZREF@np@setup@ii{%
  \def\ZREF@np@call@next##1%
}
\def\ZREF@np@call@unknown#1{#1}
\def\ZREF@np@call@nonext#1{#1}
\def\ZREF@np@call@next#1{#1}
\ZREF@IfDefinable\znextpage\def{%
  {\UniqueCounterCall{znextpage}{\ZREF@nextpage}}%
}%
\newcommand*{\znonextpagename}{}
\newcommand*{\zunknownnextpagename}{\Z@D@page}
\def\ZREF@nextpage#1{%
  \begingroup
    \def\ZREF@refname@this{zref@np#1}%
    \zref@labelbyprops\ZREF@refname@this{abspage}%
    \chardef\ZREF@call=0 % unknown
    \ZREF@ifrefundefined\ZREF@refname@this{%
    }{%
      \edef\ZREF@pagenum@this{%
        \zref@extractdefault\ZREF@refname@this{abspage}{0}%
      }%
      \edef\ZREF@refname@next{%
        \zref@thepage@name{%
          \the\numexpr\ZREF@pagenum@this+1%
        }%
      }%
      \ifnum\ZREF@pagenum@this>0 %
        \ZREF@ifrefundefined{LastPage}{%
          \zref@ifrefundefined\ZREF@refname@next{%
          }{%
            \chardef\ZREF@call=2 % next page
          }%
        }{%
          \edef\ZREF@pagenum@last{%
            \zref@extractdefault{LastPage}{abspage}{0}%
          }%
          \ifnum\ZREF@pagenum@this<\ZREF@pagenum@last\ltx@space
            \ZREF@ifrefundefined\ZREF@refname@next{%
            }{%
              \chardef\ZREF@call=2 % next page
            }%
          \else
            \ifnum\ZREF@pagenum@this=\ZREF@pagenum@this\ltx@space
              \chardef\ZREF@call=1 % no next page
            \fi
          \fi
        }%
      \fi
    }%
    \edef\x{%
      \endgroup
      \ifcase\ZREF@call
        \noexpand\ZREF@np@call@unknown{%
          \noexpand\zunknownnextpagename
        }%
      \or
        \noexpand\ZREF@np@call@nonext{%
          \noexpand\znonextpagename
        }%
      \else
        \noexpand\ZREF@np@call@next{%
          \noexpand\zref@extract{\ZREF@refname@next}{page}%
        }%
      \fi
    }%
  \x
}
%    \end{macrocode}
%    \begin{macrocode}
%</nextpage>
%    \end{macrocode}
%
%
% \subsection{Module \xmodule{totpages}}
%
%    \begin{macrocode}
%<*totpages>
\NeedsTeXFormat{LaTeX2e}
\ProvidesPackage{zref-totpages}%
  [2020-07-03 v2.32 Module totpages for zref (HO)]%
\RequirePackage{zref-base}[2019/11/29]
\ifx\ZREF@base@ok Y%
\else
  \expandafter\endinput
\fi
%    \end{macrocode}
%
%    The absolute page number of the last page is the total
%    page number.
%    \begin{macrocode}
\RequirePackage{zref-abspage}[2019/11/29]
\RequirePackage{zref-lastpage}[2019/11/29]
%    \end{macrocode}
%
%    \begin{macro}{\ztotpages}
%    Macro \cs{ztotpages} contains the number of pages. It can be
%    used inside expandable calculations. It expands to zero if
%    the reference is not yet available.
%    \begin{macrocode}
\newcommand*{\ztotpages}{%
  \zref@extractdefault{LastPage}{abspage}{0}%
}
%    \end{macrocode}
%    \end{macro}
%    Also we mark the reference |LastPage| as used:
%    \begin{macrocode}
\AtBeginDocument{%
  \zref@refused{LastPage}%
}
%    \end{macrocode}
%    \begin{macrocode}
%</totpages>
%    \end{macrocode}
%
%
% \subsection{Module \xmodule{pagelayout}}
%
%    \begin{macrocode}
%<*pagelayout>
\NeedsTeXFormat{LaTeX2e}
\ProvidesPackage{zref-pagelayout}%
  [2020-07-03 v2.32 Module pagelayout for zref (HO)]%
\RequirePackage{zref-base}[2019/11/29]
\ifx\ZREF@base@ok Y%
\else
  \expandafter\endinput
\fi
%    \end{macrocode}
%
%    \begin{macrocode}
\RequirePackage{zref-thepage}[2019/11/29]
\RequirePackage{iftex}[2019/11/07]%
\RequirePackage{atveryend}[2010/03/24]
%    \end{macrocode}
%
%
% \subsubsection{Define layout properties}
%
%    \begin{macrocode}
\def\ZREF@temp#1{%
  \begingroup
    \escapechar=-1 %
  \ltx@ifundefined{\string#1}{\endgroup}{%
    \edef\x{%
      \endgroup
      \noexpand\zref@newprop*{\string#1}%
                             [\noexpand\number\noexpand#1]% hash-ok
                             {\noexpand\number\noexpand#1}%
      \noexpand\zref@addprop{thepage}{\string#1}%
    }%
    \x
  }%
}
\ZREF@temp\mag
\ZREF@temp\paperwidth
\ZREF@temp\paperheight
\ZREF@temp\stockwidth % memoir.cls, crop.sty
\ZREF@temp\stockheight % memoir.cls, crop.sty
\ZREF@temp\mediawidth % VTeX
\ZREF@temp\mediaheight % VTeX
\ifluatex
\zref@newprop*{pdfvorigin}%
                             [\number\pdfvariable vorigin]% hash-ok
                             {\number\pdfvariable vorigin}%
\zref@addprop{thepage}{pdfvorigin}
\zref@newprop*{pdfhorigin}%
                             [\number\pdfvariable horigin]% hash-ok
                             {\number\pdfvariable horigin}%
\zref@addprop{thepage}{pdfhorigin}
\zref@newprop*{pdfpageheight}%
                             [\number\pageheight]% hash-ok
                             {\number\pageheight}%
\zref@addprop{thepage}{pdfpageheight}
\zref@newprop*{pdfpagewidth}%
                             [\number\pagewidth]% hash-ok
                             {\number\pagewidth}%
\zref@addprop{thepage}{pdfpagewidth}
\else
\ZREF@temp\pdfpagewidth
\ZREF@temp\pdfpageheight
\ZREF@temp\pdfhorigin
\ZREF@temp\pdfvorigin
\fi
\ZREF@temp\hoffset
\ZREF@temp\voffset
\ZREF@temp\topmargin
\ZREF@temp\oddsidemargin
\ZREF@temp\evensidemargin
\ZREF@temp\textwidth
\ZREF@temp\textheight
\ZREF@temp\headheight
\ZREF@temp\headsep
\ZREF@temp\footskip
\ZREF@temp\marginparwidth
\ZREF@temp\marginparsep
\ZREF@temp\columnwidth
\ZREF@temp\columnsep
\ZREF@temp\trimedge % memoir.cls
\ZREF@temp\spinemargin % memoir.cls
\ZREF@temp\foremargin % memoir.cls
\ZREF@temp\trimtop % memoir.cls
\ZREF@temp\uppermargin % memoir.cls
\ZREF@temp\headmargin % memoir.cls
%    \end{macrocode}
%
%    \begin{macrocode}
\IfFormatAtLeastTF{2020/10/01}
 {
   \zref@newprop*{outputboxwd}[0pt]{\ShipoutBoxWidth}
   \zref@newprop*{outputboxht}[0pt]{\ShipoutBoxHeight}
   \zref@newprop*{outputboxdp}[0pt]{\ShipoutBoxDepth}
 }
 {
   \zref@newprop*{outputboxwd}[0pt]{\AtBeginShipoutBoxWidth}
   \zref@newprop*{outputboxht}[0pt]{\AtBeginShipoutBoxHeight}
   \zref@newprop*{outputboxdp}[0pt]{\AtBeginShipoutBoxDepth}
 }
%    \end{macrocode}
%    \begin{macrocode}
\zref@addprops{thepage}{outputboxwd,outputboxht,outputboxdp}
%    \end{macrocode}
%
%    \begin{macro}{\ifZREF@pl@list}
%    \begin{macrocode}
\ltx@newif\ifZREF@pl@list
%    \end{macrocode}
%    \end{macro}
%    \begin{macro}{\zref@listpagelayout}
%    \begin{macrocode}
\ZREF@IfDefinable\zlistpagelayout\def{%
  {\global\ZREF@pl@listtrue}%
}
%    \end{macrocode}
%    \end{macro}
%    \begin{macro}{\ZREF@pl@AfterLastShipout}
%    \begin{macrocode}
\def\ZREF@pl@AfterLastShipout{%
  \ifZREF@pl@list
    \edef\ZREF@page@max{\the\value{abspage}}%
    \ltx@ifundefined{ZREF@org@testdef}{%
      \let\ZREF@org@testdef\@testdef
      \def\@testdef##1##2##3{%
        \ZREF@org@testdef{##1}{##2}{##3}%
        \def\ZREF@temp{##1}%
        \ifx\ZREF@temp\ZREF@RefPrefix
          \expandafter\gdef\csname##1@##2\endcsname{##3}%
        \fi
      }%
    }{}%
    \AtVeryEndDocument{\ZREF@pl@AtVeryEnd}%
  \fi
}
%    \end{macrocode}
%    \end{macro}
%    \begin{macro}{\ZREF@pl@AtVeryEnd}
%    \begin{macrocode}
\def\ZREF@pl@AtVeryEnd{%
  \begingroup
    \toks@{Page layout parameters:\MessageBreak}%
    \count@=1 %
    \ZREF@pl@ListPage
    \edef\x{\endgroup
      \noexpand\@PackageInfoNoLine{zref-pagelayout}{\the\toks@}%
    }%
  \x
}
%    \end{macrocode}
%    \end{macro}
%    \begin{macro}{\ZREF@pl@ListPage}
%    \begin{macrocode}
\def\ZREF@pl@ListPage{%
  \edef\x{%
    \toks@={%
      \the\toks@
      Page \the\count@:\noexpand\MessageBreak
      \zref@ifrefundefined{thepage\the\count@}{}{%
        \ltx@space\ltx@space mag = %
        \zref@extract{thepage\the\count@}{mag}%
        \noexpand\MessageBreak
        \ZREF@pl@ListEntry{paperwidth}%
        \ZREF@pl@ListEntry{paperheight}%
        \ZREF@pl@ListEntry{stockwidth}%
        \ZREF@pl@ListEntry{stockheight}%
        \ZREF@pl@ListEntry{mediawidth}%
        \ZREF@pl@ListEntry{mediaheight}%
        \ZREF@pl@ListEntry{pdfpagewidth}%
        \ZREF@pl@ListEntry{pdfpageheight}%
        \ZREF@pl@ListEntry{pdfhorigin}%
        \ZREF@pl@ListEntry{pdfvorigin}%
        \ZREF@pl@ListEntry{hoffset}%
        \ZREF@pl@ListEntry{voffset}%
        \ZREF@pl@ListEntry{topmargin}%
        \ZREF@pl@ListEntry{oddsidemargin}%
        \ZREF@pl@ListEntry{evensidemargin}%
        \ZREF@pl@ListEntry{textwidth}%
        \ZREF@pl@ListEntry{textheight}%
        \ZREF@pl@ListEntry{headheight}%
        \ZREF@pl@ListEntry{headsep}%
        \ZREF@pl@ListEntry{footskip}%
        \ZREF@pl@ListEntry{marginparwidth}%
        \ZREF@pl@ListEntry{marginparsep}%
        \ZREF@pl@ListEntry{columnwidth}%
        \ZREF@pl@ListEntry{columnsep}%
        \ZREF@pl@ListEntry{trimedge}%
        \ZREF@pl@ListEntry{spinemargin}%
        \ZREF@pl@ListEntry{foremargin}%
        \ZREF@pl@ListEntry{trimtop}%
        \ZREF@pl@ListEntry{uppermargin}%
        \ZREF@pl@ListEntry{headmargin}%
      }%
    }%
  }\x
  \ifnum\ZREF@page@max>\count@
    \advance\count@ by\ltx@one
  \else
    \expandafter\ltx@gobble
  \fi
  \ZREF@pl@ListPage
}
%    \end{macrocode}
%    \end{macro}
%    \begin{macro}{\ZREF@pl@ListEntry}
%    \begin{macrocode}
\def\ZREF@pl@ListEntry#1{%
  \zref@ifpropundefined{#1}{%
  }{%
    \zref@ifrefcontainsprop{thepage\the\count@}{#1}{%
      \ltx@space\ltx@space#1 = %
      \zref@extract{thepage\the\count@}{#1}sp = %
      \the\dimexpr\zref@extract{thepage\the\count@}{#1}sp\relax
      \noexpand\MessageBreak
    }{}%
  }%
}
%    \end{macrocode}
%    \end{macro}
%    \begin{macrocode}
\AfterLastShipout{%
  \ZREF@pl@AfterLastShipout
}
%    \end{macrocode}
%
%    \begin{macrocode}
%</pagelayout>
%    \end{macrocode}
%
%
% \subsection{Module \xmodule{pageattr}}
%
%    \begin{macrocode}
%<*pageattr>
\NeedsTeXFormat{LaTeX2e}
\ProvidesPackage{zref-pageattr}%
  [2020-07-03 v2.32 Module pageattr for zref (HO)]%
\RequirePackage{zref-base}[2019/11/29]
\ifx\ZREF@base@ok Y%
\else
  \expandafter\endinput
\fi
%    \end{macrocode}
%
%    \begin{macrocode}
\RequirePackage{iftex}[2019/11/07]%
%    \end{macrocode}
%
%
%    \begin{macrocode}
\let\ZREF@temp=N%
\ifluatex
\expandafter\@firstoftwo
\else
\expandafter\@secondoftwo
\fi
{%luatex
 \RequirePackage{zref-thepage}[2019/11/29]
 \RequirePackage{zref-lastpage}[2019/11/29]%
 \zref@newprop*{pdfpageattr}[]{\zref@hex{\the\pdfvariable pageattr}}%
 \zref@addprop{thepage}{pdfpageattr}%
 \zref@newprop*{pdfpagesattr}[]{\zref@hex{\the\pdfvariable pagesattr}}%
 \zref@addprop{LastPage}{pdfpagesattr}%
 \let\ZREF@temp=Y%
}
{%other
 \ltx@IfUndefined{pdfpageattr}{%
  \@PackageInfoNoLine{zref-pageattr}{%
    \string\pdfpageattr\space is not available%
  }%
  \def\zref@pdfpageattr#1{}%
  \def\zref@pdfpageattr@used#1{}%
 }{%
  \RequirePackage{zref-thepage}[2019/11/29]%
  \zref@newprop*{pdfpageattr}[]{\zref@hex{\the\pdfpageattr}}%
  \zref@addprop{thepage}{pdfpageattr}%
  \let\ZREF@temp=Y%
 }
 \ltx@IfUndefined{pdfpagesattr}{%
  \@PackageInfoNoLine{zref-pageattr}{%
    \string\pdfpagesattr\space is not available%
  }%
  \def\zref@pdfpagesattr{}%
  \def\zref@pdfpagesattr@used{}%
 }{%
  \RequirePackage{zref-lastpage}[2019/11/29]%
  \zref@newprop*{pdfpagesattr}[]{\zref@hex{\the\pdfpagesattr}}%
  \zref@addprop{LastPage}{pdfpagesattr}%
  \let\ZREF@temp=Y%
 }%
}%
%    \end{macrocode}
%    \begin{macrocode}
\ifx\ZREF@temp N%
  \expandafter\endinput
\fi
%    \end{macrocode}
%    \begin{macrocode}
\RequirePackage{zref-abspage}[2019/11/29]
\RequirePackage{atveryend}[2010/03/24]
\RequirePackage{pdftexcmds}[2010/04/01]
\let\ZREF@temp=Y%
\ltx@IfUndefined{pdf@escapehex}{\let\ZREF@temp=N}{}
\ltx@IfUndefined{pdf@unescapehex}{\let\ZREF@temp=N}{}
\ifx\ZREF@temp N%
  \let\zref@hex\ltx@firstofone
  \let\zref@unhex\ltx@firstofone
\else
  \let\zref@hex\pdf@escapehex
  \let\zref@unhex\pdf@unescapehex
\fi
%    \end{macrocode}
%
%    \begin{macro}{\ifZREF@pa@list}
%    \begin{macrocode}
\ltx@newif\ifZREF@pa@list
%    \end{macrocode}
%    \end{macro}
%    \begin{macro}{\zref@listpageattr}
%    \begin{macrocode}
\ZREF@IfDefinable\zlistpageattr\def{%
  {\ZREF@pa@listtrue}%
}
%    \end{macrocode}
%    \end{macro}
%    \begin{macro}{\ZREF@pa@AfterLastShipout}
%    \begin{macrocode}
\def\ZREF@pa@AfterLastShipout{%
  \ifZREF@pa@list
    \edef\ZREF@page@max{\the\value{abspage}}%
    \ltx@ifundefined{ZREF@org@testdef}{%
      \let\ZREF@org@testdef\@testdef
      \def\@testdef##1##2##3{%
        \ZREF@org@testdef{##1}{##2}{##3}%
        \def\ZREF@temp{##1}%
        \ifx\ZREF@temp\ZREF@RefPrefix
          \expandafter\xdef\csname##1@##2\endcsname{##3}%
        \fi
      }%
    }{}%
    \AtVeryEndDocument{\ZREF@pa@AtVeryEnd}%
  \fi
}
%    \end{macrocode}
%    \end{macro}
%    \begin{macro}{\ZREF@pa@AtVeryEnd}
%    \begin{macrocode}
\let\ZREF@temp=Y%
\ltx@IfUndefined{pdfpageattr}{}{\let\ZREF@temp=N}
\ifluatex \let\ZREF@temp=N \fi
\ifx\ZREF@temp Y
 \expandafter\@firstoftwo
\else
 \expandafter\@secondoftwo
\fi
 {%
  \def\ZREF@pa@AtVeryEnd{}%
 }
 {%
  \def\ZREF@pa@AtVeryEnd{%
    \begingroup
      \toks@{List of \ltx@backslashchar
             \ifluatex pdfvariable\else pdf\fi
             pdfpageattr:\MessageBreak}%
      \count@=1 %
      \ZREF@pa@ListPage
      \edef\x{\endgroup
        \noexpand\@PackageInfoNoLine{zref-pageattr}{%
          \the\toks@
        }%
      }%
    \x
  }%
%    \end{macrocode}
%    \end{macro}
%    \begin{macro}{\zref@pageattr}
%    \begin{macrocode}
\def\zref@pdfpageattr#1{%
  \zref@unhex{%
    \zref@extract{thepage\ZREF@number{#1}}{pdfpageattr}%
  }%
}
% compability, \zref@pageattr was defined in older versions
\let\zref@pageattr\zref@pdfpageattr
%    \end{macrocode}
%    \end{macro}
%    \begin{macro}{\zref@pageattr@used}
%    \begin{macrocode}
\ZREF@Robust\def\zref@pageattr@used#1{%
  \zref@refused{thepage\ZREF@number{#1}}%
}
%    \end{macrocode}
%    \end{macro}
%    \begin{macro}{\ZREF@pa@ListPage}
%    \begin{macrocode}
  \def\ZREF@pa@ListPage{%
    \edef\x{%
      \toks@={%
        \the\toks@
        Page \the\count@:%
        \noexpand\MessageBreak
        \zref@ifrefundefined{thepage\the\count@}{}{%
          <<\zref@pdfpageattr\count@>>%
          \noexpand\MessageBreak
        }%
      }%
    }\x
    \ifnum\ZREF@page@max>\count@
      \advance\count@ by\ltx@one
    \else
      \expandafter\ltx@gobble
    \fi
    \ZREF@pa@ListPage
  }%
}
%    \end{macrocode}
%    \end{macro}
%    \begin{macrocode}
\let\ZREF@temp=Y%
\ltx@IfUndefined{pdfpagesattr}{}{\let\ZREF@temp=N}
\ifluatex \let\ZREF@temp=N \fi
\ifx\ZREF@temp N
 \expandafter\@firstofone
\fi
 {%
%    \end{macrocode}
%    \begin{macro}{\zref@pdfpagesattr}
%    \begin{macrocode}
  \def\zref@pdfpagesattr{%
    \zref@unhex{%
      \zref@extract{LastPage}{pdfpagesattr}%
    }%
  }%
%    \end{macrocode}
%    \end{macro}
%    \begin{macro}{\zref@pdfpagesattr@used}
%    \begin{macrocode}
  \ZREF@Robust\def\zref@pdfpagesattr@used{%
    \zref@refused{LastPage}%
  }%
%    \end{macrocode}
%    \end{macro}
%    \begin{macrocode}
  \ltx@LocalAppendToMacro\ZREF@pa@AtVeryEnd{%
    \@PackageInfoNoLine{zref-pageattr}{%
      \ltx@backslashchar
      \ifluatex pdfvariable\else pdf\fi
      pagesattr:\MessageBreak
      <<\zref@pdfpagesattr>>%
      \MessageBreak
    }%
  }%
}
\AfterLastShipout{%
  \ZREF@pa@AfterLastShipout
}
%    \end{macrocode}
%    \begin{macrocode}
%</pageattr>
%    \end{macrocode}
%
%
% \subsection{Module \xmodule{marks}}
%
%    \begin{macrocode}
%<*marks>
\NeedsTeXFormat{LaTeX2e}
\ProvidesPackage{zref-marks}%
  [2020-07-03 v2.32 Module marks for zref (HO)]%
\RequirePackage{zref-base}[2019/11/29]
\ifx\ZREF@base@ok Y%
\else
  \expandafter\endinput
\fi
%    \end{macrocode}
%
%    \begin{macrocode}
\newcommand*{\zref@marks@register}[3][]{%
  \edef\ZREF@TempName{#1}%
  \edef\ZREF@TempNum{\ZREF@number{#2}}%
  \ifnum\ZREF@TempNum<\ltx@zero %
    \PackageError\ZREF@name{%
      \string\zref@marks@register\ltx@space is called with invalid%
      \MessageBreak
      marks register number (\ZREF@TempNum)%
    }{%
      Use `0' or the command, defined by \string\newmarks.\MessageBreak
      \@ehc
    }%
  \else
    \ifx\ZREF@TempName\ltx@empty
      \edef\ZREF@TempName{mark\romannumeral\ZREF@TempNum}%
    \else
      \edef\ZREF@TempName{marks\ZREF@TempName}%
    \fi
    \ZREF@MARKS@DefineProp{top}%
    \ZREF@MARKS@DefineProp{first}%
    \ZREF@MARKS@DefineProp{bot}%
    \kv@parse{#3}{%
      \ifx\kv@value\relax
        \def\kv@value{top,first,bot}%
      \fi
      \edef\ZREF@temp{\expandafter\ltx@car\kv@key X\@nil}%
      \ifx\ZREF@temp\ZREF@STAR
        \edef\kv@key{\expandafter\ltx@cdr\kv@key\@nil}%
        \zref@newlist\kv@key
      \fi
      \expandafter\comma@parse\expandafter{\kv@value}{%
        \ifcase0\ifx\comma@entry\ZREF@NAME@top 1\else
                \ifx\comma@entry\ZREF@NAME@first 1\else
                \ifx\comma@entry\ZREF@NAME@bot 1\fi\fi\fi\ltx@space
          \PackageWarning{zref-marks}{%
            Use `top', `first' or `bot' for the list values%
            \MessageBreak
            in the third argument of \string\zref@marks@register.%
            \MessageBreak
            Ignoring unkown value `\comma@entry'%
          }%
        \else
          \zref@addprop{\kv@key}{\comma@entry\ZREF@TempName}%
        \fi
        \ltx@gobble
      }%
      \ltx@gobbletwo
    }%
  \fi
}
\def\ZREF@STAR{*}
\def\ZREF@NAME@top{top}
\def\ZREF@NAME@first{first}
\def\ZREF@NAME@bot{bot}
\def\ZREF@MARKS@DefineProp#1{%
  \zref@ifpropundefined{#1\ZREF@TempName}{%
    \ifnum\ZREF@TempNum=\ltx@zero
      \begingroup
        \edef\x{\endgroup
          \noexpand\zref@newprop*{#1\ZREF@TempName}[]{%
            \expandafter\noexpand\csname#1mark\endcsname
          }%
        }%
      \x
    \else
      \begingroup
        \edef\x{\endgroup
          \noexpand\zref@newprop*{#1\ZREF@TempName}[]{%
            \expandafter\noexpand\csname#1marks\endcsname
            \ZREF@TempNum
          }%
        }%
      \x
    \fi
  }{%
    \PackageWarning{zref-marks}{%
      \string\zref@marks@register\ltx@space does not generate the%
      \MessageBreak
      new property `#1\ZREF@TempName', because\MessageBreak
      it is already defined%
    }%
  }%
}
%    \end{macrocode}
%    \begin{macrocode}
%</marks>
%    \end{macrocode}
%
% \subsection{Module \xmodule{runs}}
%
%    This module does not use the label-reference-system.
%    The reference changes with each \hologo{LaTeX}~run and would
%    force a rerun warning always.
%    \begin{macrocode}
%<*runs>
\NeedsTeXFormat{LaTeX2e}
\ProvidesPackage{zref-runs}%
  [2020-07-03 v2.32 Module runs for zref (HO)]%
%    \end{macrocode}
%
%    \begin{macro}{\zruns}
%    \begin{macrocode}
\providecommand*{\zruns}{0}%
\AtBeginDocument{%
  \edef\zruns{\number\numexpr\zruns+1}%
  \begingroup
    \def\on@line{}%
    \PackageInfo{zref-runs}{LaTeX runs: \zruns}%
    \if@filesw
      \immediate\write\@mainaux{%
        \string\gdef\string\zruns{\zruns}%
      }%
    \fi
  \endgroup
}
%    \end{macrocode}
%    \end{macro}
%
%    \begin{macrocode}
%</runs>
%    \end{macrocode}
%
%
% \subsection{Module \xmodule{perpage}}
%
%    \begin{macrocode}
%<*perpage>
\NeedsTeXFormat{LaTeX2e}
\ProvidesPackage{zref-perpage}%
  [2020-07-03 v2.32 Module perpage for zref (HO)]%
\RequirePackage{zref-base}[2019/11/29]
\ifx\ZREF@base@ok Y%
\else
  \expandafter\endinput
\fi
%    \end{macrocode}
%
%    This module resets a counter at page boundaries.
%    Because of the asynchronous output routine
%    page counter properties cannot be asked directly,
%    references are necessary.
%
%    For detecting changed pages module \xmodule{abspage}
%    is loaded.
%    \begin{macrocode}
\RequirePackage{zref-abspage}[2019/11/29]
%    \end{macrocode}
%
%    We group the properties for the needed references
%    in the property list |perpage|. The property |pagevalue|
%    records the correct value of the page counter.
%    \begin{macrocode}
\ZREF@NewPropPageValue
\zref@newlist{perpage}
\zref@addprops{perpage}{abspage,page,pagevalue}
%    \end{macrocode}
%    The page value, known by the reference mechanism, will be
%    stored in counter |zpage|.
%    \begin{macrocode}
\newcounter{zpage}
%    \end{macrocode}
%    Counter |zref@unique| helps in generating unique reference
%    names.
%    \begin{macrocode}
\zref@require@unique
%    \end{macrocode}
%    In order to be able to reset the counter, we hook here
%    into \cs{stepcounter}. In fact two nested hooks are used
%    to allow other packages to use the first hook at the beginning
%    of \cs{stepcounter}.
%    \begin{macrocode}
\let\ZREF@org@stepcounter\stepcounter
\def\stepcounter#1{%
  \ifcsname @stepcounterhook@#1\endcsname
    \csname @stepcounterhook@#1\endcsname
  \fi
  \ZREF@org@stepcounter{#1}%
}
%    \end{macrocode}
%    \cs{@stpelt} must be adapted due to the change in latex
%    2015-01, see https://github.com/ho-tex/zref/issues/26
%    \begin{macrocode}
\let\ZREF@org@@stpelt\@stpelt
\def\@stpelt#1{%
  \ifcsname ZREF@perpage@#1\endcsname
    \begingroup
      \let\stepcounter\ZREF@org@stepcounter
      \ZREF@org@@stpelt{#1}%
    \endgroup
    \expandafter\ltx@gobbletwo
  \fi
  \ZREF@org@@stpelt{#1}%
}
%    \end{macrocode}
%
%    \begin{macro}{\zmakeperpage}
%    Makro \cs{zmakeperpage} resets a counter at each page
%    break. It uses the same syntax and semantics as
%    \cs{MakePerPage} from package \xpackage{perpage} \cite{perpage}.
%    The initial start value can be given by the optional
%    argument. Default is one that means after the first
%    \cs{stepcounter} on a new page the counter starts with
%    one.
%    \begin{macrocode}
\ZREF@IfDefinable\zmakeperpage\def{%
  {%
    \@ifnextchar[\ZREF@makeperpage@opt{\ZREF@@makeperpage[\ltx@zero]}%
  }%
}
%    \end{macrocode}
%    \end{macro}
%    We hook before the counter is incremented in \cs{stepcounter},
%    package \xpackage{perpage} afterwards. Thus a little
%    calculation is necessary.
%    \begin{macrocode}
\def\ZREF@makeperpage@opt[#1]{%
  \begingroup
    \edef\x{\endgroup
      \noexpand\ZREF@@makeperpage[\number\numexpr#1-1\relax]%
    }%
  \x
}
%    \end{macrocode}
%    \begin{macrocode}
\def\ZREF@@makeperpage[#1]#2{%
  \@ifundefined{@stepcounterhook@#2}{%
    \expandafter\gdef\csname @stepcounterhook@#2\endcsname{}%
  }{}%
  \expandafter\gdef\csname ZREF@perpage@#2\endcsname{%
    \ZREF@@perpage@step{#2}{#1}%
  }%
  \expandafter\g@addto@macro\csname @stepcounterhook@#2\endcsname{%
    \ifcsname ZREF@perpage@#2\endcsname
      \csname ZREF@perpage@#2\endcsname
    \fi
  }%
}
%    \end{macrocode}
%    \begin{macro}{\ZREF@@perpage@step}
%    The heart of this module follows.
%    \begin{macrocode}
\def\ZREF@@perpage@step#1#2{%
%    \end{macrocode}
%    First the reference is generated.
%    \begin{macrocode}
  \global\advance\c@zref@unique\ltx@one
  \begingroup
    \expandafter
    \zref@labelbylist\expandafter{\thezref@unique}{perpage}%
%    \end{macrocode}
%    The \cs{expandafter} commands are necessary, because \cs{ZREF@temp}
%    is also used inside of \cs{zref@labelbylist}.
%
%    The evaluation of the reference follows. If the reference
%    is not yet kwown, we use the page counter as approximation.
%    \begin{macrocode}
    \zref@ifrefundefined\thezref@unique{%
      \global\c@zpage=\c@page
      \global\let\thezpage\thepage
      \expandafter\xdef\csname ZREF@abspage@#1\endcsname{%
        \number\c@abspage
      }%
    }{%
%    \end{macrocode}
%    The reference is used to set \cs{thezpage} and counter |zpage|.
%    \begin{macrocode}
      \global\c@zpage=\zref@extract\thezref@unique{pagevalue}\relax
      \xdef\thezpage{\noexpand\zref@extract{\thezref@unique}{page}}%
      \expandafter\xdef\csname ZREF@abspage@#1\endcsname{%
        \zref@extractdefault\thezref@unique
            {abspage}{\number\c@abspage}%
      }%
    }%
%    \end{macrocode}
%    Page changes are detected by a changed absolute page number.
%    \begin{macrocode}
    \expandafter\ifx\csname ZREF@abspage@#1\expandafter\endcsname
                    \csname ZREF@currentabspage@#1\endcsname
    \else
      \global\csname c@#1\endcsname=#2\relax
      \global\expandafter\let
          \csname ZREF@currentabspage@#1\expandafter\endcsname
          \csname ZREF@abspage@#1\endcsname
    \fi
  \endgroup
}
%    \end{macrocode}
%    \end{macro}
%    \begin{macro}{\zunmakeperpage}
%    Macro \cs{zunmakeperpage} cancels the effect of \cs{zmakeperpage}.
%    \begin{macrocode}
\ZREF@IfDefinable\zunmakeperpage\def{%
  #1{%
    \global\expandafter
    \let\csname ZREF@perpage@#1\endcsname\@undefined
  }%
}
%    \end{macrocode}
%    \end{macro}
%
%    \begin{macrocode}
%</perpage>
%    \end{macrocode}
%
%
% \subsection{Module \xmodule{titleref}}
%
%    \begin{macrocode}
%<*titleref>
\NeedsTeXFormat{LaTeX2e}
\ProvidesPackage{zref-titleref}%
  [2020-07-03 v2.32 Module titleref for zref (HO)]%
\RequirePackage{zref-base}[2019/11/29]
\ifx\ZREF@base@ok Y%
\else
  \expandafter\endinput
\fi
\RequirePackage{gettitlestring}[2009/12/08]
%    \end{macrocode}
%
% \subsubsection{Implementation}
%    \begin{macrocode}
\RequirePackage{keyval}
%    \end{macrocode}
%
%    This module makes section and caption titles available
%    for the reference system. It uses some of the ideas
%    of package \xpackage{nameref} and \xpackage{titleref}.
%
%    Now we can add the property \verb|title| is added to
%    the main property list.
%    \begin{macrocode}
\ZREF@NewPropTitle
\zref@addprop\ZREF@mainlist{title}%
%    \end{macrocode}
%
%    The title strings go into the \xfile{.aux} file, thus
%    they need some kind of protection.
%    Package \xpackage{titleref} uses a protected expansion
%    method. The advantage is that this can be used to
%    cleanup the string and to remove \cs{label}, \cs{index}
%    and other macros unwanted for referencing.
%    But there is the risk that fragile stuff can break.
%
%    Therefore package \xpackage{nameref} does not expand
%    the string. Thus the entries can safely be written to
%    the \xfile{.aux} file. But potentially dangerous
%    macros such as \cs{label} remain in the string and
%    can cause problems when using the string in references.
%
%    \begin{macro}{\ifzref@titleref@expand}
%    The switch \cs{ifzref@titleref@expand} distinguishes
%    between the both methods. Package \xpackage{nameref}'s
%    behaviour is achieved by setting the switch to false,
%    otherwise \xpackage{titleref}'s expansion is used.
%    Default is false.
%    \begin{macrocode}
\newif\ifzref@titleref@expand
%    \end{macrocode}
%    \end{macro}
%
%    \begin{macro}{\ZREF@titleref@hook}
%    The hook \cs{ZREF@titleref@hook} allows to extend
%    the cleanup for the expansion method. Thus
%    unnecessary macros can be removed or dangerous commands
%    removed. The hook is executed before the expansion of
%    \cs{zref@titleref@current}.
%    \begin{macrocode}
\let\ZREF@titleref@hook\ltx@empty
%    \end{macrocode}
%    \end{macro}
%    \begin{macro}{\zref@titleref@cleanup}
%    The hook should not be used directly, instead we provide the
%    macro \cs{zref@titleref@cleanup} to add stuff to the hook
%    and prevents that a previous non-empty content is not discarded
%    accidently.
%    \begin{macrocode}
\ZREF@Robust\def\zref@titleref@cleanup#1{%
   \begingroup
    \toks@\expandafter{%
      \ZREF@titleref@hook
      #1%
    }%
  \expandafter\endgroup
  \expandafter\def\expandafter\ZREF@titleref@hook\expandafter{%
    \the\toks@
  }%
}%
%    \end{macrocode}
%    \end{macro}
%
%    \begin{macro}{\ifzref@titleref@stripperiod}
%    Sometimes a title contains a period at the end.
%    Package \xpackage{nameref} removes this. This behaviour
%    is controlled by the switch \cs{ifzref@titleref@stripperiod}
%    and works regardless of the setting of option \xoption{expand}.
%    Period stripping is the default.
%    \begin{macrocode}
\newif\ifzref@titleref@stripperiod
\zref@titleref@stripperiodtrue
%    \end{macrocode}
%    \end{macro}
%
%    \begin{macro}{\zref@titleref@setcurrent}
%    Macro \cs{zref@titleref@setcurrent} sets a new current title
%    stored in \cs{zref@titleref@current}. Some cleanup and
%    expansion is performed that can be controlled by the
%    previous switches.
%    \begin{macrocode}
\ZREF@Robust\def\zref@titleref@setcurrent#1{%
  \ifzref@titleref@expand
    \GetTitleStringExpand{#1}%
  \else
    \GetTitleStringNonExpand{#1}%
  \fi
  \edef\zref@titleref@current{%
    \detokenize\expandafter{\GetTitleStringResult}%
  }%
  \ifzref@titleref@stripperiod
    \edef\zref@titleref@current{%
      \expandafter\ZREF@stripperiod\zref@titleref@current
      \ltx@empty.\ltx@empty\@nil
    }%
  \fi
}%
\GetTitleStringDisableCommands{%
  \ZREF@titleref@hook
}
%    \end{macrocode}
%    \end{macro}
%    \begin{macro}{\ZREF@stripperiod}
%    If \cs{ZREF@stripperiod} is called, the argument consists of
%    space tokens and tokens with catcode 12 (other), because
%    of \hologo{eTeX}'s \cs{detokenize}.
%    \begin{macrocode}
\def\ZREF@stripperiod#1.\ltx@empty#2\@nil{#1}%
%    \end{macrocode}
%    \end{macro}
%
% \subsubsection{User interface}
%
%    \begin{macro}{\ztitlerefsetup}
%    The behaviour of module \xmodule{titleref} is controlled
%    by switches and a hook. They can be set by \cs{ztitlerefsetup}
%    with a key value interface, provided by package \xpackage{keyval}.
%    Also the current title can be given explicitly by the key
%    \verb|title|.
%    \begin{macrocode}
\define@key{ZREF@TR}{expand}[true]{%
  \csname zref@titleref@expand#1\endcsname
}%
\define@key{ZREF@TR}{stripperiod}[true]{%
  \csname zref@titleref@stripperiod#1\endcsname
}%
\define@key{ZREF@TR}{cleanup}{%
  \zref@titleref@cleanup{#1}%
}%
\define@key{ZREF@TR}{title}{%
  \def\zref@titleref@current{#1}%
}%
\ZREF@IfDefinable\ztitlerefsetup\def{%
  {\kvsetkeys{ZREF@TR}}%
}%
%    \end{macrocode}
%    \end{macro}
%
%    \begin{macro}{\ztitleref}
%    The user command \cs{ztitleref} references the title.
%    For safety \cs{label} is disabled to prevent multiply
%    defined references.
%    \begin{macrocode}
\ZREF@IfDefinable\ztitleref\def{%
  {\zref@wrapper@babel\ZREF@titleref}%
}%
\def\ZREF@titleref#1{%
  \begingroup
    \zref@refused{#1}%
    \let\label\ltx@gobble
    \zref@extract{#1}{title}%
  \endgroup
}%
%    \end{macrocode}
%    \end{macro}
%
% \subsubsection{Patches for section and caption commands}
%
%    The section and caption macros are patched to extract the
%    title data.
%
%    Captions of figures and tables.
%    \begin{macrocode}
\AtBeginDocument{%
  \ZREF@patch{@caption}{%
    \long\def\@caption#1[#2]{%
      \zref@titleref@setcurrent{#2}%
      \ZREF@org@@caption{#1}[{#2}]%
    }%
  }%
%    \end{macrocode}
%
%    Section commands without star. The title version for the
%    table of contents is used because it is usually shorter and
%    more robust.
%    \begin{macrocode}
  \ZREF@patch{@part}{%
    \def\@part[#1]{%
      \zref@titleref@setcurrent{#1}%
      \ZREF@org@@part[{#1}]%
    }%
  }%
  \ZREF@patch{@chapter}{%
    \def\@chapter[#1]{%
      \zref@titleref@setcurrent{#1}%
      \ZREF@org@@chapter[{#1}]%
    }%
  }%
  \ZREF@patch{@sect}{%
    \def\@sect#1#2#3#4#5#6[#7]{%
      \zref@titleref@setcurrent{#7}%
      \ZREF@org@@sect{#1}{#2}{#3}{#4}{#5}{#6}[{#7}]%
    }%
  }%
%    \end{macrocode}
%    The star versions of the section commands.
%    \begin{macrocode}
  \ZREF@patch{@spart}{%
    \def\@spart#1{%
      \zref@titleref@setcurrent{#1}%
      \ZREF@org@@spart{#1}%
    }%
  }%
  \ZREF@patch{@schapter}{%
    \def\@schapter#1{%
      \zref@titleref@setcurrent{#1}%
      \ZREF@org@@schapter{#1}%
    }%
  }%
  \ZREF@patch{@ssect}{%
    \def\@ssect#1#2#3#4#5{%
      \zref@titleref@setcurrent{#5}%
      \ZREF@org@@ssect{#1}{#2}{#3}{#4}{#5}%
    }%
  }%
%    \end{macrocode}
%
% \subsubsection{Environment \textsf{description}}
%
%    \begin{macrocode}
  \ZREF@patch{descriptionlabel}{%
    \def\descriptionlabel#1{%
      \zref@titleref@setcurrent{#1}%
      \ZREF@org@descriptionlabel{#1}%
    }%
  }%
%    \end{macrocode}
%
% \subsubsection{Class \xclass{memoir}}
%
%    \begin{macrocode}
  \@ifclassloaded{memoir}{%
    \ltx@IfUndefined{ifheadnameref}{}{%
      \def\@chapter[#1]#2{%
        \ltx@IfUndefined{ch@pt@c}{%
          \zref@titleref@setcurrent{#1}%
        }{%
          \ifx\ch@pt@c\ltx@empty
            \zref@titleref@setcurrent{#2}%
          \else
            \def\NR@temp{#1}%
            \ifx\NR@temp\ltx@empty
              \expandafter\zref@titleref@setcurrent
              \expandafter{\ch@pt@c}%
            \else
              \ifheadnameref
                \zref@titleref@setcurrent{#1}%
              \else
                \expandafter\zref@titleref@setcurrent
                \expandafter{\ch@pt@c}%
              \fi
            \fi
          \fi
        }%
        \ZREF@org@@chapter[{#1}]{#2}%
      }%
      \ZREF@patch{M@sect}{%
        \def\M@sect#1#2#3#4#5#6[#7][#8]{%
          \ifheadnameref
            \zref@titleref@setcurrent{#8}%
          \else
            \zref@titleref@setcurrent{#7}%
          \fi
          \ZREF@org@M@sect{#1}{#2}{#3}{#4}{#5}{#6}[{#7}][{#8}]%
        }%
      }%
    }%
  }{}%
%    \end{macrocode}
%
% \subsubsection{Class \xclass{beamer}}
%
%    \begin{macrocode}
  \@ifclassloaded{beamer}{%
    \ZREF@patch{beamer@section}{%
      \long\def\beamer@section[#1]{%
        \zref@titleref@setcurrent{#1}%
        \ZREF@org@beamer@section[{#1}]%
      }%
    }%
    \ZREF@patch{beamer@subsection}{%
      \long\def\beamer@subsection[#1]{%
        \zref@titleref@setcurrent{#1}%
        \ZREF@org@beamer@subsection[{#1}]%
      }%
    }%
    \ZREF@patch{beamer@subsubsection}{%
      \long\def\beamer@subsubsection[#1]{%
        \zref@titleref@setcurrent{#1}%
        \ZREF@org@beamer@subsubsection[{#1}]%
      }%
    }%
  }{}%
%    \end{macrocode}
%
% \subsubsection{Package \xpackage{titlesec}}
%
%    \begin{macrocode}
  \@ifpackageloaded{titlesec}{%
    \ZREF@patch{ttl@sect@i}{%
      \def\ttl@sect@i#1#2[#3]#4{%
        \zref@titleref@setcurrent{#4}%
        \ZREF@org@ttl@sect@i{#1}{#2}[{#3}]{#4}%
      }%
    }%
    \ZREF@patch{ttl@straight@i}{%
      \def\ttl@straight@i#1[#2]#3{%
        \def\ZREF@temp{#2}%
        \ifx\ZREF@temp\ltx@empty
          \zref@titleref@setcurrent{#3}%
        \else
          \zref@titleref@setcurrent{#2}%
        \fi
        \ZREF@org@ttl@straight@i{#1}[{#2}]{#3}%
      }%
    }%
  }{}%
%    \end{macrocode}
%
% \subsubsection{Package \xpackage{longtable}}
%
%    Package \xpackage{longtable}: some support for its \cs{caption}.
%    However \cs{label} inside the caption is not supported.
%    \begin{macrocode}
  \@ifpackageloaded{longtable}{%
    \ZREF@patch{LT@c@ption}{%
      \def\LT@c@ption#1[#2]#3{%
        \ZREF@org@LT@c@ption{#1}[{#2}]{#3}%
        \zref@titleref@setcurrent{#2}%
      }%
    }%
  }{}%
%    \end{macrocode}
%
% \subsubsection{Package \xpackage{listings}}
%
%    Package \xpackage{listings}: support for its caption.
%    \begin{macrocode}
  \@ifpackageloaded{listings}{%
    \ZREF@patch{lst@MakeCaption}{%
      \def\lst@MakeCaption{%
        \ifx\lst@label\ltx@empty
        \else
          \expandafter\zref@titleref@setcurrent\expandafter{%
            \lst@@caption
          }%
        \fi
        \ZREF@org@lst@MakeCaption
      }%
    }%
  }{}%
%    \end{macrocode}
%
% \subsubsection{Theorems}
%
%    \begin{macrocode}
  \ZREF@patch{@opargbegintheorem}{%
    \def\@opargbegintheorem#1#2#3{%
      \zref@titleref@setcurrent{#3}%
      \ZREF@org@@opargbegintheorem{#1}{#2}{#3}%
    }%
  }%
  \@ifpackageloaded{amsthm}{%
    \begingroup
      \edef\x{macro:\string#1\string#2[\string#3]}%
      \@onelevel@sanitize\x
      \def\y#1->#2\@nil{#1}%
      \edef\z{\expandafter\y\meaning\@begintheorem->\@nil}%
      \@onelevel@sanitize\z
    \expandafter\endgroup
    \ifx\x\z
      \ZREF@patch{@begintheorem}{%
        \def\@begintheorem#1#2[#3]{%
          \zref@titleref@setcurrent{#3}%
          \ZREF@org@@begintheorem{#1}{#2}[{#3}]%
        }%
      }%
    \fi
  }{}%
%    \end{macrocode}
%
%    \begin{macrocode}
}
%    \end{macrocode}
%
%    \begin{macrocode}
%</titleref>
%    \end{macrocode}
%
%
% \subsection{Module \xmodule{xr}}
%
%    \begin{macrocode}
%<*xr>
\NeedsTeXFormat{LaTeX2e}
\ProvidesPackage{zref-xr}%
  [2020-07-03 v2.32 Module xr for zref (HO)]%
\RequirePackage{zref-base}[2019/11/29]
\ifx\ZREF@base@ok Y%
\else
  \expandafter\endinput
\fi
%    \end{macrocode}
%    \begin{macrocode}
\RequirePackage{keyval}
\RequirePackage{kvoptions}[2010/02/22]
%    \end{macrocode}
%
%    We declare property |url|, because this is added,
%    if a reference is imported and has not already set this field.
%    Or if |hyperref| is used, then this property can be asked.
%    \begin{macrocode}
\zref@newprop{url}{}%
\zref@newprop{urluse}{}%
%    \end{macrocode}
%    \begin{macrocode}
\zref@newprop{externaldocument}{}%
%    \end{macrocode}
%
%    Most code, especially the handling of the \xfile{.aux}
%    files are taken from David Carlisle's \xpackage{xr} package.
%    Therefore I drop the documentation for these macros here.
%
%    \begin{macro}{\zref@xr@ext}
%    If the URL is not specied, then assume
%    processed file with a guessed extension.
%    Use the setting of \xpackage{hyperref} if available.
%    \begin{macrocode}
\providecommand*{\zref@xr@ext}{%
  \ltx@ifundefined{XR@ext}{pdf}{\XR@ext}%
}%
%    \end{macrocode}
%    \end{macro}
%    \begin{macro}{\ifZREF@xr@zreflabel}
%    The use of the star form of \cs{zexternaldocument}
%    is remembered in the switch \cs{ifZREF@xr@zreflabel}.
%    \begin{macrocode}
\newif\ifZREF@xr@zreflabel
%    \end{macrocode}
%    \end{macro}
%    \begin{macrocode}
\SetupKeyvalOptions{%
  family=ZREF@XR,%
  prefix=ZREF@xr@%
}
\DeclareBoolOption[true]{tozreflabel}
\DeclareBoolOption[false]{toltxlabel}
\DeclareBoolOption{verbose}
\define@key{ZREF@XR}{ext}{%
  \def\zref@xr@{#1}%
}
\DeclareBoolOption[false]{urluse}
%    \end{macrocode}
%    \begin{macro}{\zxrsetup}
%    \begin{macrocode}
\newcommand*{\zxrsetup}{%
  \kvsetkeys{ZREF@XR}%
}%
%    \end{macrocode}
%    \end{macro}
%
%    \begin{macro}{\ZREF@xr@URL}
%    \begin{macrocode}
\newcount\ZREF@xr@URL
\ZREF@xr@URL=\ltx@zero
%    \end{macrocode}
%    \end{macro}
%    \begin{macro}{\ZREF@xr@AddURL}
%    \begin{macrocode}
\def\ZREF@xr@AddURL#1{%
  \begingroup
    \def\ZREF@temp{#1}%
    \count@=\ltx@one
    \ZREF@xr@@AddUrl
  \endgroup
}
%    \end{macrocode}
%    \end{macro}
%    \begin{macro}{\ZREF@xr@@AddUrl}
%    \begin{macrocode}
\def\ZREF@xr@@AddUrl{%
  \ifnum\count@>\ZREF@xr@URL
    \global\advance\ZREF@xr@URL by\ltx@one
    \xdef\ZREF@xr@theURL{\romannumeral\ZREF@xr@URL}%
    \global\expandafter\let
        \csname Z@U@\ZREF@xr@theURL\endcsname\ZREF@temp
    \@PackageInfo{zref-xr}{%
      \ltx@backslashchar Z@U@\ZREF@xr@theURL:\MessageBreak
      \ZREF@temp\MessageBreak
    }%
  \else
    \expandafter
    \ifx\csname Z@U@\romannumeral\count@\endcsname\ZREF@temp
      \xdef\ZREF@xr@theURL{\romannumeral\count@}%
    \else
      \expandafter\expandafter\expandafter\ZREF@xr@@AddUrl
    \fi
  \fi
}
%    \end{macrocode}
%    \end{macro}
%
%    \begin{macro}{\zexternaldocument}
%    In its star form it looks for \cs{newlabel},
%    otherwise for \cs{zref@newlabel}.
%     Later we
%    will read \xfile{.aux} files that expects
%    |@| to have catcode 11 (letter).
%    \begin{macrocode}
\ZREF@IfDefinable\zexternaldocument\def{%
  {%
    \ZREF@NewPropAnchor
    \ZREF@NewPropTitle
    \begingroup
      \csname @safe@actives@true\endcsname
      \makeatletter
      \@ifstar{%
        \ZREF@xr@zreflabelfalse
        \@testopt\ZREF@xr@externaldocument{}%
      }{%
        \ZREF@xr@zreflabeltrue
        \@testopt\ZREF@xr@externaldocument{}%
      }%
  }%
}%
%    \end{macrocode}
%    \end{macro}
%
%    If the \cs{include} featuer was used, there can be
%    several \xfile{.aux} files. These files are read
%    one after another, especially they are not recursively
%    read in order to save read registers. Thus it can
%    happen that the read order of the newlabel commands
%    differs from \hologo{LaTeX}'s order using \cs{input}.
%
%    \begin{macro}{\ZREF@xr@externaldocument}
%    It reads the remaining arguments. \cs{newcommand}
%    comes in handy for the optional argument.
%    \begin{macrocode}
\def\ZREF@xr@externaldocument[#1]#2{%
    \def\ZREF@xr@prefix{#1}%
    \let\ZREF@xr@filelist\ltx@empty
    \edef\ZREF@xr@externalfile{#2}%
    \edef\ZREF@xr@file{\ZREF@xr@externalfile.aux}%
    \filename@parse{#2}%
    \@testopt\ZREF@xr@graburl{#2.\zref@xr@ext}%
}%
\def\ZREF@xr@graburl[#1]{%
    \edef\ZREF@xr@url{#1}%
    \ifZREF@xr@urluse
      \expandafter\ZREF@xr@AddURL\expandafter{\ZREF@xr@url}%
      \expandafter\def\expandafter\ZREF@xr@url
      \expandafter{\csname Z@U@\ZREF@xr@theURL\endcsname}%
    \fi
    \ZREF@xr@checkfile
  \endgroup
}%
%    \end{macrocode}
%    \end{macro}
%    \begin{macro}{\ZREF@xr@processfile}
%    We follow \xpackage{xr} here, \cs{IfFileExists} offers
%    a nicer test, but we have to open the file anyway.
%    \begin{macrocode}
\def\ZREF@xr@checkfile{%
  \openin\@inputcheck\ZREF@xr@file\relax
  \ifeof\@inputcheck
    \PackageWarning{zref-xr}{%
      File `\ZREF@xr@file' not found or empty,\MessageBreak
      labels not imported%
    }%
  \else
    \PackageInfo{zref-xr}{%
      Label \ifZREF@xr@zreflabel (zref) \fi
      import from `\ZREF@xr@file'%
    }%
    \def\ZREF@xr@found{0}%
    \def\ZREF@xr@ignored@empty{0}%
    \def\ZREF@xr@ignored@zref{0}%
    \def\ZREF@xr@ignored@ltx{0}%
    \ZREF@xr@processfile
    \closein\@inputcheck
    \begingroup
      \let\on@line\ltx@empty
      \PackageInfo{zref-xr}{%
        Statistics for `\ZREF@xr@file':\MessageBreak
        \ZREF@xr@found\space
        \ifZREF@xr@zreflabel zref\else LaTeX\fi\space
        label(s) found%
        \ifnum\ZREF@xr@ignored@empty>0 %
          ,\MessageBreak
          \ZREF@xr@ignored@empty\space empty label(s) ignored%
        \fi
        \ifnum\ZREF@xr@ignored@zref>0 %
          ,\MessageBreak
          \ZREF@xr@ignored@zref\space
          duplicated zref label(s) ignored%
        \fi
        \ifnum\ZREF@xr@ignored@ltx>0 %
          ,\MessageBreak
          \ZREF@xr@ignored@ltx\space
          duplicated latex label(s) ignored%
        \fi
      }%
    \endgroup
  \fi
  \ifx\ZREF@xr@filelist\ltx@empty
  \else
    \edef\ZREF@xr@file{%
      \expandafter\ltx@car\ZREF@xr@filelist\@nil
    }%
    \edef\ZREF@xr@filelist{%
      \expandafter\ltx@cdr\ZREF@xr@filelist\ltx@empty\@nil
    }%
    \expandafter\ZREF@xr@checkfile
  \fi
}%
%    \end{macrocode}
%    \end{macro}
%
%    \begin{macro}{\ZREF@xr@processfile}
%    \begin{macrocode}
\def\ZREF@xr@processfile{%
  \read\@inputcheck to\ZREF@xr@line
  \expandafter\ZREF@xr@processline\ZREF@xr@line..\ZREF@nil
  \ifeof\@inputcheck
  \else
    \expandafter\ZREF@xr@processfile
  \fi
}%
%    \end{macrocode}
%    \end{macro}
%    \begin{macro}{\ZREF@xr@processline}
%    The most work must be done for analyzing the
%    arguments of \cs{newlabel}.
%    \begin{macrocode}
\long\def\ZREF@xr@processline#1#2#3\ZREF@nil{%
  \def\x{#1}%
  \toks@{#2}%
  \ifZREF@xr@zreflabel
    \ifx\x\ZREF@xr@zref@newlabel
      \expandafter
      \ZREF@xr@process@zreflabel\ZREF@xr@line...\ZREF@nil
    \fi
  \else
    \ifx\x\ZREF@xr@newlabel
      \expandafter
      \ZREF@xr@process@label\ZREF@xr@line...[]\ZREF@nil
    \fi
  \fi
  \ifx\x\ZREF@xr@@input
    \edef\ZREF@xr@filelist{%
      \etex@unexpanded\expandafter{\ZREF@xr@filelist}%
      {\filename@area\the\toks@}%
    }%
  \fi
}%
\def\ZREF@xr@process@zreflabel\zref@newlabel#1#2#3\ZREF@nil{%
  \edef\ZREF@xr@refname{Z@R@\ZREF@xr@prefix#1}%
  \edef\ZREF@xr@found{\the\numexpr\ZREF@xr@found+1\relax}%
  \def\x{#2}%
  \edef\ZREF@xr@tempname{$temp$}%
  \edef\ZREF@xr@temprefname{Z@R@\ZREF@xr@tempname}%
  \let\ZREF@xr@list\x
  \ifx\ZREF@xr@list\ltx@empty
    \PackageWarningNoLine{zref-xr}{%
      Label `#1' without properties ignored\MessageBreak
      in file `\ZREF@xr@file'%
    }%
    \edef\ZREF@xr@ignored@empty{%
      \the\numexpr\ZREF@xr@ignored@empty+1\relax
    }%
  \else
    \expandafter\ZREF@xr@checklist\x\ZREF@nil
    \expandafter\let\csname\ZREF@xr@temprefname\endcsname\x
    \expandafter\ltx@LocalAppendToMacro
    \csname\ZREF@xr@temprefname\expandafter\endcsname
    \expandafter{%
      \expandafter\externaldocument\expandafter{%
        \ZREF@xr@externalfile
      }%
    }%
    \ZREF@xr@urlcheck\ZREF@xr@tempname
    \ifZREF@xr@tozreflabel
      \@ifundefined{\ZREF@xr@refname}{%
        \ifZREF@xr@verbose
          \PackageInfo{zref-xr}{%
            Import to zref label `\ZREF@xr@tempname#1'%
          }%
        \fi
        \global\expandafter
        \let\csname\ZREF@xr@refname\expandafter\endcsname
        \csname\ZREF@xr@temprefname\endcsname
      }{%
        \ZREF@xr@zref@ignorewarning{\ZREF@xr@prefix#1}%
      }%
    \fi
    \ifZREF@xr@toltxlabel
      \ZREF@xr@tolabel{\ZREF@xr@tempname}{\ZREF@xr@prefix#1}%
    \fi
  \fi
}%
\def\ZREF@xr@process@label\newlabel#1#2#3[#4]#5\ZREF@nil{%
  \def\ZREF@xr@refname{Z@R@\ZREF@xr@prefix#1}%
  \edef\ZREF@xr@found{\the\numexpr\ZREF@xr@found+1\relax}%
  \def\x{#2}%
  \edef\ZREF@xr@tempname{$temp$}%
  \edef\ZREF@xr@temprefname{Z@R@\ZREF@xr@tempname}%
  \expandafter\ZREF@xr@scanparams
      \csname\ZREF@xr@temprefname\expandafter\endcsname
      \x{}{}{}{}{}\ZREF@nil
  \ifx\\#4\\%
  \else
    % ntheorem knows an optional argument at the end of \newlabel
    \ZREF@NewPropTheotype
    \expandafter\ltx@LocalAppendToMacro
        \csname\ZREF@xr@temprefname\endcsname{\theotype{#4}}%
  \fi
  \expandafter\ltx@LocalAppendToMacro
  \csname\ZREF@xr@temprefname\expandafter\endcsname\expandafter{%
    \expandafter\externaldocument\expandafter{%
      \ZREF@xr@externalfile
    }%
  }%
  \ZREF@xr@urlcheck\ZREF@xr@tempname
  \ifZREF@xr@tozreflabel
    \@ifundefined{\ZREF@xr@refname}{%
      \ifZREF@xr@verbose
        \PackageInfo{zref-xr}{%
          Import to zref label `\ZREF@xr@prefix#1'%
        }%
      \fi
      \global\expandafter
      \let\csname\ZREF@xr@refname\expandafter\endcsname
      \csname\ZREF@xr@temprefname\endcsname
    }{%
      \ZREF@xr@zref@ignorewarning{\ZREF@xr@prefix#1}%
    }%
  \fi
  \ifZREF@xr@toltxlabel
    \ZREF@xr@tolabel{\ZREF@xr@tempname}{\ZREF@xr@prefix#1}%
  \fi
}
\def\ZREF@xr@zref@newlabel{\zref@newlabel}%
\def\ZREF@xr@newlabel{\newlabel}%
\def\ZREF@xr@@input{\@input}%
\def\ZREF@xr@relax{\relax}%
%    \end{macrocode}
%    \end{macro}
%    \begin{macro}{\ZREF@xr@tolabel}
%    \begin{macrocode}
\def\ZREF@xr@tolabel#1#2{%
  \ifZREF@xr@verbose
    \PackageInfo{zref-xr}{%
      Import to LaTeX label `#2'%
    }%
  \fi
  \zref@wrapper@unexpanded{%
    \expandafter\xdef\csname r@#2\endcsname{%
      {%
        \ltx@ifundefined{M@TitleReference}{%
          \ltx@ifundefined{TR@TitleReference}{%
            \zref@extractdefault{#1}{default}{}%
          }{%
            \noexpand\TR@TitleReference
            {\zref@extractdefault{#1}{default}{}}%
            {\zref@extractdefault{#1}{title}{}}%
          }%
        }{%
          \noexpand\M@TitleReference
          {\zref@extractdefault{#1}{default}{}}%
          {\zref@extractdefault{#1}{title}{}}%
        }%
      }%
      {\zref@extractdefault{#1}{page}{}}%
      \ltx@ifpackageloaded{nameref}{%
        {\zref@extractdefault{#1}{title}{}}%
        {\zref@extractdefault{#1}{anchor}{}}%
        \zref@ifrefcontainsprop{#1}{urluse}{%
          {\zref@extractdefault{#1}{urluse}{}}%
        }{%
          {\zref@extractdefault{#1}{url}{}}%
        }%
      }{}%
    }%
  }%
}
%    \end{macrocode}
%    \end{macro}
%
%    \begin{macro}{\ZREF@xr@zref@ignorewarning}
%    \begin{macrocode}
\def\ZREF@xr@zref@ignorewarning#1{%
  \PackageWarningNoLine{zref-xr}{%
    Zref label `#1' is already in use\MessageBreak
    in file `\ZREF@xr@file'%
  }%
  \edef\ZREF@xr@ignored@zref{%
    \the\numexpr\ZREF@xr@ignored@zref+1%
  }%
}%
%    \end{macrocode}
%    \end{macro}
%
%    \begin{macro}{\ZREF@xr@ltx@ignorewarning}
%    \begin{macrocode}
\def\ZREF@xr@ltx@ignorewarning#1{%
  \PackageWarningNoLine{zref-xr}{%
    LaTeX label `#1' is already in use\MessageBreak
    in file `\ZREF@xr@file'%
  }%
  \edef\ZREF@xr@ignored@ltx{%
    \the\numexpr\ZREF@xr@ignored@ltx+1%
  }%
}%
%    \end{macrocode}
%    \end{macro}
%
%    \begin{macro}{\ZREF@xr@checklist}
%    \begin{macrocode}
\def\ZREF@xr@checklist#1#2#3\ZREF@nil{%
  \ifx\@undefined#1\relax
    \expandafter\ZREF@xr@checkkey\string#1\@nil
  \fi
  \ifx\\#3\\%
  \else
    \ltx@ReturnAfterFi{%
      \ZREF@xr@checklist#3\ZREF@nil
    }%
  \fi
}%
\def\ZREF@xr@checkkey#1#2\@nil{%
  \zref@ifpropundefined{#2}{%
    \zref@newprop{#2}{}%
  }{}%
}%
%    \end{macrocode}
%    \end{macro}
%
%    \begin{macro}{\ZREF@xr@scanparams}
%    \begin{macrocode}
\def\ZREF@xr@scanparams#1#2#3#4#5#6#7\ZREF@nil{%
  \let#1\ltx@empty
  \ZREF@foundfalse
  \ZREF@xr@scantitleref#1#2\TR@TitleReference{}{}\ZREF@nil
  \ifZREF@found
  \else
    \ltx@LocalAppendToMacro#1{\default{#2}}%
  \fi
  % page
  \ltx@LocalAppendToMacro#1{\page{#3}}%
  % nameref title
  \ifZREF@found
  \else
    \ifx\\#4\\%
    \else
      \def\ZREF@xr@temp{#4}%
      \ifx\ZREF@xr@temp\ZREF@xr@relax
      \else
        \ltx@LocalAppendToMacro#1{\title{#4}}%
      \fi
    \fi
  \fi
  % anchor
  \ifx\\#5\\%
  \else
    \ltx@LocalAppendToMacro#1{\anchor{#5}}%
  \fi
  \ifx\\#6\\%
  \else
    \ifZREF@xr@urluse
      \ZREF@xr@AddURL{#6}%
      \expandafter\ltx@LocalAppendToMacro\expandafter#1%
      \expandafter{%
        \expandafter\urluse\expandafter{%
          \csname Z@U@\ZREF@xr@theURL\endcsname
        }%
      }%
    \else
      \ltx@LocalAppendToMacro#1{\url{#6}}%
    \fi
  \fi
}%
%    \end{macrocode}
%    \end{macro}
%
%    \begin{macro}{\ZREF@xr@scantitleref}
%    \begin{macrocode}
\def\ZREF@xr@scantitleref#1#2\TR@TitleReference#3#4#5\ZREF@nil{%
  \ifx\\#5\\%
  \else
    \ltx@LocalAppendToMacro#1{%
      \default{#3}%
      \title{#4}%
    }%
    \ZREF@foundtrue
  \fi
}%
%    \end{macrocode}
%    \end{macro}
%
%    \begin{macro}{\ZREF@xr@urlcheck}%
%    \begin{macrocode}
\def\ZREF@xr@urlcheck#1{%
  \zref@ifrefcontainsprop{#1}{anchor}{%
    \zref@ifrefcontainsprop{#1}{url}{%
    }{%
      \expandafter
      \ltx@LocalAppendToMacro\csname Z@R@#1\expandafter\endcsname
      \expandafter{%
        \csname url\ifZREF@xr@urluse use\fi
        \expandafter\endcsname\expandafter{\ZREF@xr@url}%
      }%
    }%
  }{%
  }%
}%
%    \end{macrocode}
%    \end{macro}
%
%    \begin{macrocode}
%</xr>
%    \end{macrocode}
%
%
% \subsection{Module \xmodule{hyperref}}
%
%    UNFINISHED :-(
%
%    \begin{macrocode}
%<*hyperref>
\NeedsTeXFormat{LaTeX2e}
\ProvidesPackage{zref-hyperref}%
  [2020-07-03 v2.32 Module hyperref for zref (HO)]%
\RequirePackage{zref-base}[2019/11/29]
\ifx\ZREF@base@ok Y%
\else
  \expandafter\endinput
\fi
%    \end{macrocode}
%
%    \begin{macrocode}
\ZREF@NewPropAnchor
\zref@addprop\ZREF@mainlist{anchor}%
%    \end{macrocode}
%
%    \begin{macrocode}
%</hyperref>
%    \end{macrocode}
%
%
% \subsection{Module \xmodule{savepos}}
%
%    Module \xmodule{savepos} provides an interface for
%    \pdfTeX's \cmd{\pdfsavepos}, see the manual for
%    \pdfTeX.
%
% \subsubsection{Identification}
%
%    \begin{macrocode}
%<*savepos>
\NeedsTeXFormat{LaTeX2e}
\ProvidesPackage{zref-savepos}%
  [2020-07-03 v2.32 Module savepos for zref (HO)]%
\RequirePackage{zref-base}[2019/11/29]
\ifx\ZREF@base@ok Y%
\else
  \expandafter\endinput
\fi
%    \end{macrocode}
%
% \subsubsection{Availability}
%
%    First we check, whether the feature is available.
%    \begin{macrocode}
\ifx\directlua\@undefined
\ltx@IfUndefined{pdfsavepos}{%
  \PackageError\ZREF@name{%
    \string\pdfsavepos\space is not supported.\MessageBreak
    It is provided by pdfTeX (1.40) or XeTeX%
  }\ZREF@UpdatePdfTeX
  \endinput
}{}%
\fi
%    \end{macrocode}
%    In PDF mode we are done. However support for DVI mode
%    was added later in version 1.40.0. In earlier versions
%    \cs{pdfsavepos} is defined, but its execution raises
%    an error. Note that \hologo{XeTeX} also provides \cs{pdfsavepos}.
%    \begin{macrocode}
\ifpdf
\else
  \ltx@IfUndefined{pdftexversion}{%
  }{%
    \ifnum\pdftexversion<140 %
      \PackageError\ZREF@name{%
        \string\pdfsavepos\space is not supported in DVI mode%
        \MessageBreak
        of this pdfTeX version%
      }\ZREF@UpdatePdfTeX
      \expandafter\expandafter\expandafter\endinput
    \fi
  }%
\fi
%    \end{macrocode}
%
% \subsubsection{Setup}
%
%    \begin{macrocode}
\zref@newlist{savepos}
\ifx\directlua\@undefined
  \zref@newprop*{posx}[0]{\the\pdflastxpos}
  \zref@newprop*{posy}[0]{\the\pdflastypos}
\else
  \zref@newprop*{posx}[0]{\the\lastxpos}
  \zref@newprop*{posy}[0]{\the\lastypos}
\fi
\zref@addprops{savepos}{posx,posy}
%    \end{macrocode}
%
% \subsubsection{User macros}
%
%    \begin{macro}{\zref@savepos}
%    \begin{macrocode}
\ifx\directlua\@undefined
  \def\zref@savepos{%
    \if@filesw
      \pdfsavepos
    \fi
  }
\else
  \def\zref@savepos{%
    \if@filesw
      \savepos
    \fi
  }
\fi
%    \end{macrocode}
%    \end{macro}
%
%    \begin{macro}{\ZREF@zsavepos}
%    \begin{macrocode}
\def\ZREF@zsavepos#1#2#3{%
  \@bsphack
  \if@filesw
    \zref@savepos
    #1{#3}{#2}%
    \ltx@IfUndefined{TeXXeTstate}{%
    }{%
      \ifnum\TeXXeTstate=\ltx@zero
      \else
        \zref@savepos
      \fi
    }%
  \fi
  \@esphack
}
%    \end{macrocode}
%    \end{macro}
%
%    \begin{macro}{\zsavepos}
%    The current location is stored in a reference with the
%    given name.
%    \begin{macrocode}
\ZREF@IfDefinable\zsavepos\def{%
  {%
    \ZREF@zsavepos\zref@labelbylist{savepos}%
  }%
}
%    \end{macrocode}
%    \end{macro}
%
%    \begin{macro}{\zsaveposx}
%    \begin{macrocode}
\ZREF@IfDefinable\zsaveposx\def{%
  {%
    \ZREF@zsavepos\zref@labelbyprops{posx}%
  }%
}
%    \end{macrocode}
%    \end{macro}
%    \begin{macro}{\zsaveposy}
%    \begin{macrocode}
\ZREF@IfDefinable\zsaveposy\def{%
  {%
    \ZREF@zsavepos\zref@labelbyprops{posy}%
  }%
}
%    \end{macrocode}
%    \end{macro}
%
%    \begin{macro}{\zposx}
%    \begin{macro}{\zposy}
%    The horizontal and vertical position are available by
%    \cs{zposx} and \cs{zposy}. Do not rely on absolute
%    positions. They differ in DVI and PDF mode of \pdfTeX.
%    Use differences instead. The unit of the position numbers
%    is sp.
%    \begin{macrocode}
\newcommand*{\zposx}[1]{%
  \zref@extract{#1}{posx}%
}%
\newcommand*{\zposy}[1]{%
  \zref@extract{#1}{posy}%
}%
%    \end{macrocode}
%    \end{macro}
%    \end{macro}
%
%    Typically horizontal and vertical positions are used
%    inside calculations. Therefore the extracting macros
%    should be expandable and babel's patch is not applyable.
%
%    Also it is in the responsibility of the user to marked
%    used positions by \cs{zrefused}
%    in order to notify \hologo{LaTeX} about undefined references.
%
%    \begin{macro}{\ZREF@savepos@ok}
%    \begin{macrocode}
\let\ZREF@savepos@ok=Y
%    \end{macrocode}
%    \end{macro}
%
%    \begin{macrocode}
%</savepos>
%    \end{macrocode}
%
%
% \subsection{Module \xmodule{abspos}}
%
% \subsubsection{Identification}
%
%    \begin{macrocode}
%<*abspos>
\NeedsTeXFormat{LaTeX2e}
\ProvidesPackage{zref-abspos}%
  [2020-07-03 v2.32 Module abspos for zref (HO)]%
\RequirePackage{zref-base}[2019/11/29]
\ifx\ZREF@base@ok Y%
\else
  \expandafter\endinput
\fi
%    \end{macrocode}
%
%    \begin{macrocode}
\RequirePackage{zref-savepos}[2019/11/29]
\ifx\ZREF@savepos@ok Y%
\else
  \expandafter\endinput
\fi
%    \end{macrocode}
%    \begin{macrocode}
\RequirePackage{zref-pagelayout}[2019/11/29]
\zref@addprop{savepos}{abspage}
\zref@addprop{savepos}{pagevalue}
%    \end{macrocode}
%
%    \begin{macro}{\zref@absposx}
%    \begin{macrocode}
\newcommand*{\zref@absposx}[3]{%
  \number
    \expandafter\zref@absposnumx\expandafter{%
      \number\zref@extractdefault{#1}{abspage}{0}%
    }{#2}{#3}%
  \ltx@space
}
%    \end{macrocode}
%    \end{macro}
%
%    \begin{macro}{\zref@absposy}
%    \begin{macrocode}
\newcommand*{\zref@absposy}[3]{%
  \number
  \expandafter\zref@absposnumy\expandafter{%
    \number\zref@extractdefault{#1}{abspage}{0}%
  }{#2}{#3}%
  \ltx@space
}
%    \end{macrocode}
%    \end{macro}
%
%    \begin{macro}{\zref@absposnumx}
%    \begin{macrocode}
\newcommand*{\zref@absposnumx}[3]{%
  \number
%  \ifnum#1>\ltx@zero
%    \zref@ifrefundefined{thepage#1}{%
%      0%
%    }{%
      \numexpr\ZREF@absposnum{thepage#1}{#2}{x}{#3}\relax
%    }%
%  \else
%    0%
%  \fi
}
%    \end{macrocode}
%    \end{macro}
%
%    \begin{macro}{\zref@absposnumy}
%    \begin{macrocode}
\newcommand*{\zref@absposnumy}[3]{%
  \number
%  \ifnum#1>\ltx@zero
%    \zref@ifrefundefined{thepage#1}{%
%      0%
%    }{%
      \numexpr\ZREF@absposnum{thepage#1}{#2}{y}{#3}\relax
%    }%
%  \else
%    0%
%  \fi
}
%    \end{macrocode}
%    \end{macro}
%
%    \begin{macro}{\ZREF@absposnum}
%    \begin{macrocode}
\def\ZREF@absposnum#1#2#3#4{%
  \ltx@ifundefined{ZREF@abspos@#2@#3@#4}{%
    0%
  }{%
    \csname ZREF@abspos@#2@#3@#4\endcsname{#1}%
  }%
}
%    \end{macrocode}
%    \end{macro}
%
%    \begin{macro}{\zref@def@absposx}
%    \begin{macrocode}
\ZREF@Robust\def\zref@def@absposx#1{%
  \zref@wrapper@babel{\ZREF@def@abspos{#1}\zref@absposx}%
}
%    \end{macrocode}
%    \end{macro}
%
%    \begin{macro}{\zref@def@absposy}
%    \begin{macrocode}
\ZREF@Robust\def\zref@def@absposy#1{%
  \zref@wrapper@babel{\ZREF@def@abspos{#1}\zref@absposy}%
}
%    \end{macrocode}
%    \end{macro}
%
%    \begin{macro}{\zref@def@absposnumx}
%    \begin{macrocode}
\ZREF@Robust\def\zref@def@absposnumx#1{%
  \zref@wrapper@babel{\ZREF@def@abspos{#1}\zref@absposnumx}%
}
%    \end{macrocode}
%    \end{macro}
%
%    \begin{macro}{\zref@def@absposnumy}
%    \begin{macrocode}
\ZREF@Robust\def\zref@def@absposnumy#1{%
  \zref@wrapper@babel{\ZREF@def@abspos{#1}\zref@absposnumy}%
}
%    \end{macrocode}
%    \end{macro}
%    \begin{macro}{\ZREF@def@abspos}
%    \begin{macrocode}
\def\ZREF@def@absposnumy#1#2#3#4#5{%
  \edef#1{#2{#3}{#4}{#5}}%
}
%    \end{macrocode}
%    \end{macro}
%
%    \begin{macro}{\zref@absposused}
%    \begin{macrocode}
\ZREF@Robust\def\zref@absposused{%
  \zref@wrapper@babel\ZREF@abspos@used
}
%    \end{macrocode}
%    \end{macro}
%    \begin{macro}{\ZREF@abspos@used}
%    \begin{macrocode}
\def\ZREF@abspos@used#1{%
  \zref@refused{#1}%
  \zref@ifrefundefined{#1}{%
  }{%
    \begingroup
      \edef\ZREF@temp{%
        \zref@extractdefault{#1}{abspage}{0}%
      }%
      \ifnum\ZREF@temp>\ltx@zero
        \zref@refused{thepage\ZREF@temp}%
      \else
        \@PackageError{zref-abspos}{%
          \string\zref@pos@label@used\ltx@space
          needs property `abspage'\MessageBreak
          in label `#1'%
        }\@ehc
      \fi
    \endgroup
  }%
}
%    \end{macrocode}
%    \end{macro}
%
%    \begin{macro}{\zref@absposnumused}
%    \begin{macrocode}
\newcommand*{\zref@absposnumused}[1]{%
  \ifnum#1>\ltx@zero
    \zref@refused{thepage\number#1}%
  \else
    \@PackageError{zref-abspos}{%
      Invalid absolute page number (#1)\MessageBreak
      for \string\zref@pos@num@used.\MessageBreak
      A positive integer is expected%
    }\@ehc
  \fi
}
%    \end{macrocode}
%    \end{macro}
%
%    \begin{macro}{\zref@ifabsposundefined}
%    \begin{macrocode}
\def\zref@ifabsposundefined#1{%
  \zref@ifrefundefined{#1}\ltx@firsttwo{%
    \expandafter\zref@ifabsposnumundefined\expandafter{%
      \number\zref@extractdefault{#1}{abspage}{0}%
    }%
  }%
}
%    \end{macrocode}
%    \end{macro}
%
%    \begin{macro}{\zref@ifabsposnumundefined}
%    \begin{macrocode}
\def\zref@ifabsposnumundefined#1{%
  \ifnum\ZREF@number{#1}>\ltx@zero
    \zref@ifrefundefined{thepage#1}%
        \ltx@firstoftwo\ltx@secondoftwo
  \else
    \expandafter\ltx@firstoftwo
  \fi
}
%    \end{macrocode}
%    \end{macro}
%
% \subsubsection{Media}
%
%    \begin{macro}{\ZREF@abspos@media@width}
%    \begin{macrocode}
\edef\ZREF@abspos@media@width{%
  \ltx@ifundefined{pdfpagewidth}{%
    \ltx@ifundefined{mediawidth}{%
      \ltx@ifundefined{stockwidth}{%
        paperwidth%
      }{%
        stockwidth%
      }%
    }{%
      mediawidth%
    }%
  }{%
    pdfpagewidth%
  }%
}
\ifluatex
\def\ZREF@abspos@media@width{pdfpagewidth}%
\fi
%    \end{macrocode}
%    \end{macro}
%    \begin{macro}{\ZREF@abspos@media@height}
%    \begin{macrocode}
\edef\ZREF@abspos@media@height{%
  \ltx@ifundefined{pdfpageheight}{%
    \ltx@ifundefined{mediaheight}{%
      \ltx@ifundefined{stockheight}{%
        paperheight%
      }{%
        stockheight%
      }%
    }{%
      mediaheight%
    }%
  }{%
    \noexpand\ifcase\pdfpageheight
      \ltx@ifundefined{stockheight}{%
        paperheight%
      }{%
        stockheight%
      }%
    \noexpand\else
      pdfpageheight%
    \noexpand\fi
  }%
}
\ifluatex
\edef\ZREF@abspos@media@height{%
   \noexpand\ifcase\pageheight
      \ltx@ifundefined{stockheight}{%
        paperheight%
      }{%
        stockheight%
      }%
    \noexpand\else
      pdfpageheight%
    \noexpand\fi}%
\fi
%    \end{macrocode}
%    \end{macro}
%
%    \begin{macro}{\ZREF@abspos@media@x@left}
%    \begin{macrocode}
\def\ZREF@abspos@media@x@left#1{%
  0%
}
%    \end{macrocode}
%    \end{macro}
%    \begin{macro}{\ZREF@abspos@media@x@right}
%    \begin{macrocode}
\def\ZREF@abspos@media@x@right#1{%
  \zref@extract{#1}\ZREF@abspos@media@width
}
%    \end{macrocode}
%    \end{macro}
%    \begin{macro}{\ZREF@abspos@media@x@center}
%    \begin{macrocode}
\def\ZREF@abspos@media@x@center#1{%
  \ZREF@abspos@media@x@left{#1}%
  +\zref@extract{#1}\ZREF@abspos@media@width/2%
}
%    \end{macrocode}
%    \end{macro}
%
%    \begin{macro}{\ZREF@abspos@media@y@top}
%    \begin{macrocode}
\def\ZREF@abspos@media@y@top#1{%
  \zref@extract{#1}\ZREF@abspos@media@height
}
%    \end{macrocode}
%    \end{macro}
%    \begin{macro}{\ZREF@abspos@media@y@bottom}
%    \begin{macrocode}
\def\ZREF@abspos@media@y@bottom#1{%
  0%
}
%    \end{macrocode}
%    \end{macro}
%    \begin{macro}{\ZREF@abspos@media@y@center}
%    \begin{macrocode}
\def\ZREF@abspos@media@y@center#1{%
  \zref@extract{#1}\ZREF@abspos@media@height/2%
}
%    \end{macrocode}
%    \end{macro}
%
% \subsubsection{Paper}
%
%    \begin{macro}{\ZREF@abspos@paper@x@left}
%    \begin{macrocode}
\def\ZREF@abspos@paper@x@left#1{%
  0%
}
%    \end{macrocode}
%    \end{macro}
%    \begin{macro}{\ZREF@abspos@paper@x@right}
%    \begin{macrocode}
\def\ZREF@abspos@paper@x@right#1{%
  \zref@extract{#1}{paperwidth}%
}
%    \end{macrocode}
%    \end{macro}
%    \begin{macro}{\ZREF@abspos@paper@x@center}
%    \begin{macrocode}
\def\ZREF@abspos@paper@x@center#1{%
  \zref@extract{#1}{paperwidth}/2%
}
%    \end{macrocode}
%    \end{macro}
%
%    \begin{macro}{\ZREF@abspos@paper@y@top}
%    \begin{macrocode}
\let\ZREF@abspos@paper@y@top\ZREF@abspos@media@y@top
%    \end{macrocode}
%    \end{macro}
%    \begin{macro}{\ZREF@abspos@paper@y@bottom}
%    \begin{macrocode}
\def\ZREF@abspos@paper@y@bottom#1{%
  \ZREF@abspos@paper@y@top{#1}%
  -\zref@extract{#1}{paperheight}%
}
%    \end{macrocode}
%    \end{macro}
%    \begin{macro}{\ZREF@abspos@paper@y@center}
%    \begin{macrocode}
\def\ZREF@abspos@paper@y@center#1{%
  \ZREF@abspos@paper@y@top{#1}%
  -\zref@extract{#1}{paperheight}/2%
}
%    \end{macrocode}
%    \end{macro}
%
% \subsubsection{Origin}
%
% There doesn't seem a good reason to make these tests depend on pdf mode
%  in current engines, so comment out the \verb|\ifpdf| tests.
%    \begin{macro}{\ZREF@abspos@origin@x}
%    \begin{macrocode}
\let\ZREF@temp\ltx@two
\ltx@IfUndefined{pdfhorigin}{}{%
%  \ifpdf
    \let\ZREF@temp\ltx@zero
%  \fi
}
\ifluatex
% \ifpdf
  \let\ZREF@temp\ltx@zero
% \fi
\fi

\ifx\ZREF@temp\ltx@two
  \ifnum\mag=1000 %
    \let\ZREF@temp\ltx@one
  \fi
\fi
\ifcase\ZREF@temp
  \def\ZREF@abspos@origin@x#1{%
    \zref@extract{#1}{pdfhorigin}%
  }%
\or
  \def\ZREF@abspos@origin@x#1{%
    4736286%
  }%
\or
  \def\ZREF@abspos@origin@x#1{%
    \numexpr\mag/1000*\dimexpr 1truein\relax\relax
  }%
\fi
%    \end{macrocode}
%    \end{macro}
%    \begin{macro}{\ZREF@abspos@origin@y}
%    \begin{macrocode}
\let\ZREF@temp\ltx@two
\ltx@IfUndefined{pdfvorigin}{}{%
%  \ifpdf
    \let\ZREF@temp\ltx@zero
%  \fi
}
\ifluatex
%  \ifpdf
    \let\ZREF@temp\ltx@zero
%  \fi
\fi
\ifx\ZREF@temp\ltx@two
  \ifnum\mag=1000 %
    \let\ZREF@temp\ltx@one
  \fi
\fi
\ifcase\ZREF@temp
  \def\ZREF@abspos@origin@y#1{%
    \zref@extract{#1}{pdfvorigin}%
  }%
\or
  \def\ZREF@abspos@origin@y#1{%
    4736286%
  }%
\or
  \def\ZREF@abspos@origin@y#1{%
    \numexpr\mag/1000*\dimexpr 1truein\relax\relax
   }%
\fi
%    \end{macrocode}
%    \end{macro}
%
% \subsubsection{Header}
%
%    \begin{macro}{\ZREF@abspos@head@x@left}
%    \begin{macrocode}
\def\ZREF@abspos@head@x@left#1{%
  \ZREF@abspos@paper@x@left{#1}%
  +\ZREF@abspos@origin@x{#1}%
  +\zref@extract{#1}{hoffset}%
  +\ifodd\zref@extractdefault{#1}{pagevalue}{\number\c@page} %
     \zref@extract{#1}{oddsidemargin}%
   \else
     \zref@extract{#1}{evensidemargin}%
   \fi
}
%    \end{macrocode}
%    \end{macro}
%    \begin{macro}{\ZREF@abspos@head@x@right}
%    \begin{macrocode}
\def\ZREF@abspos@head@x@right#1{%
  \ZREF@abspos@head@x@left{#1}%
  +\zref@extract{#1}{textwidth}%
}
%    \end{macrocode}
%    \end{macro}
%    \begin{macro}{\ZREF@abspos@head@x@center}
%    \begin{macrocode}
\def\ZREF@abspos@head@x@center#1{%
  \ZREF@abspos@head@x@left{#1}%
  +\zref@extract{#1}{textwidth}/2%
}
%    \end{macrocode}
%    \end{macro}
%
%    \begin{macro}{\ZREF@abspos@head@y@top}
%    \begin{macrocode}
\def\ZREF@abspos@head@y@top#1{%
  \ZREF@abspos@paper@y@top{#1}%
  -\ZREF@abspos@origin@y{#1}%
  -\zref@extract{#1}{voffset}%
  -\zref@extract{#1}{topmargin}%
}
%    \end{macrocode}
%    \end{macro}
%    \begin{macro}{\ZREF@abspos@head@y@bottom}
%    \begin{macrocode}
\def\ZREF@abspos@head@y@bottom#1{%
  \ZREF@abspos@head@y@top{#1}%
  -\zref@extract{#1}{headheight}%
}
%    \end{macrocode}
%    \end{macro}
%    \begin{macro}{\ZREF@abspos@head@y@center}
%    \begin{macrocode}
\def\ZREF@abspos@head@y@center#1{%
  \ZREF@abspos@head@y@top{#1}%
  -\zref@extract{#1}{headheight}/2%
}
%    \end{macrocode}
%    \end{macro}
%
% \subsubsection{Body}
%
%    \begin{macro}{\ZREF@abspos@body@x@left}
%    \begin{macrocode}
\let\ZREF@abspos@body@x@left\ZREF@abspos@head@x@left
%    \end{macrocode}
%    \end{macro}
%    \begin{macro}{\ZREF@abspos@body@x@right}
%    \begin{macrocode}
\let\ZREF@abspos@body@x@right\ZREF@abspos@head@x@right
%    \end{macrocode}
%    \end{macro}
%    \begin{macro}{\ZREF@abspos@body@x@center}
%    \begin{macrocode}
\let\ZREF@abspos@body@x@center\ZREF@abspos@head@x@center
%    \end{macrocode}
%    \end{macro}
%
%    \begin{macro}{\ZREF@abspos@body@y@top}
%    \begin{macrocode}
\def\ZREF@abspos@body@y@top#1{%
  \ZREF@abspos@head@y@bottom{#1}%
  -\zref@extract{#1}{headsep}%
}
%    \end{macrocode}
%    \end{macro}
%    \begin{macro}{\ZREF@abspos@body@y@bottom}
%    \begin{macrocode}
\def\ZREF@abspos@body@y@bottom#1{%
  \ZREF@abspos@body@y@top{#1}%
  -\zref@extract{#1}{textheight}%
}
%    \end{macrocode}
%    \end{macro}
%    \begin{macro}{\ZREF@abspos@body@y@center}
%    \begin{macrocode}
\def\ZREF@abspos@body@y@center#1{%
  \ZREF@abspos@body@y@top{#1}%
  -\zref@extract{#1}{textheight}/2%
}
%    \end{macrocode}
%    \end{macro}
%
% \subsubsection{Footer}
%
%    \begin{macro}{\ZREF@abspos@foot@x@left}
%    \begin{macrocode}
\let\ZREF@abspos@foot@x@left\ZREF@abspos@head@x@left
%    \end{macrocode}
%    \end{macro}
%    \begin{macro}{\ZREF@abspos@foot@x@right}
%    \begin{macrocode}
\let\ZREF@abspos@foot@x@right\ZREF@abspos@head@x@right
%    \end{macrocode}
%    \end{macro}
%    \begin{macro}{\ZREF@abspos@foot@x@center}
%    \begin{macrocode}
\let\ZREF@abspos@foot@x@center\ZREF@abspos@head@x@center
%    \end{macrocode}
%    \end{macro}
%
%    \begin{macro}{\ZREF@abspos@foot@y@bottom}
%    \begin{macrocode}
\def\ZREF@abspos@foot@y@bottom#1{%
  \ZREF@abspos@body@y@bottom{#1}%
  -\zref@extract{#1}{footskip}%
}
%    \end{macrocode}
%    \end{macro}
%
% \subsubsection{Marginal notes}
%
%    \begin{macro}{\ZREF@abspos@marginpar@x@left}
%    \begin{macrocode}
\def\ZREF@abspos@marginpar@x@left#1{%
  \ifodd\zref@extractdefault{#1}{pagevalue}{\number\c@page} %
    \ZREF@abspos@body@x@right{#1}%
    +\zref@extract{#1}{marginparsep}%
  \else
    \ZREF@abspos@body@x@left{#1}%
    -\zref@extract{#1}{marginparsep}%
    -\zref@extract{#1}{marginparwidth}%
  \fi
}
%    \end{macrocode}
%    \end{macro}
%    \begin{macro}{\ZREF@abspos@marginpar@x@right}
%    \begin{macrocode}
\def\ZREF@abspos@marginpar@x@right#1{%
  \ZREF@abspos@marginpar@x@left{#1}%
  +\zref@extract{#1}{marginparwidth}%
}
%    \end{macrocode}
%    \end{macro}
%    \begin{macro}{\ZREF@abspos@marginpar@x@center}
%    \begin{macrocode}
\def\ZREF@abspos@marginpar@x@center#1{%
  \ZREF@abspos@marginpar@x@left{#1}%
  +\zref@extract{#1}{marginparwidth}/2%
}
%    \end{macrocode}
%    \end{macro}
%
%    \begin{macro}{\ZREF@abspos@marginpar@y@top}
%    \begin{macrocode}
\let\ZREF@abspos@marginpar@y@top\ZREF@abspos@body@y@top
%    \end{macrocode}
%    \end{macro}
%    \begin{macro}{\ZREF@abspos@marginpar@y@bottom}
%    \begin{macrocode}
\let\ZREF@abspos@marginpar@y@bottom\ZREF@abspos@body@y@bottom
%    \end{macrocode}
%    \end{macro}
%    \begin{macro}{\ZREF@abspos@marginpar@y@center}
%    \begin{macrocode}
\let\ZREF@abspos@marginpar@y@center\ZREF@abspos@body@y@center
%    \end{macrocode}
%    \end{macro}
%
% \subsubsection{Stock paper}
%
%    \begin{macro}{\ZREF@abspos@stock@x@left}
%    \begin{macrocode}
\let\ZREF@abspos@stock@x@left\ZREF@abspos@paper@x@left
%    \end{macrocode}
%    \end{macro}
%    \begin{macro}{\ZREF@abspos@stock@x@right}
%    \begin{macrocode}
\let\ZREF@abspos@stock@x@right\ZREF@abspos@paper@x@right
%    \end{macrocode}
%    \end{macro}
%    \begin{macro}{\ZREF@abspos@stock@x@center}
%    \begin{macrocode}
\let\ZREF@abspos@stock@x@center\ZREF@abspos@paper@x@center
%    \end{macrocode}
%    \end{macro}
%
%    \begin{macro}{\ZREF@abspos@stock@y@top}
%    \begin{macrocode}
\let\ZREF@abspos@stock@y@top\ZREF@abspos@paper@y@top
%    \end{macrocode}
%    \end{macro}
%    \begin{macro}{\ZREF@abspos@stock@y@bottom}
%    \begin{macrocode}
\let\ZREF@abspos@stock@y@bottom\ZREF@abspos@paper@y@bottom
%    \end{macrocode}
%    \end{macro}
%    \begin{macro}{\ZREF@abspos@stock@y@center}
%    \begin{macrocode}
\let\ZREF@abspos@stock@y@center\ZREF@abspos@paper@y@center
%    \end{macrocode}
%    \end{macro}
%
%    \begin{macrocode}
%</abspos>
%    \end{macrocode}
%
%
% \subsection{Module \xmodule{dotfill}}
%
%    \begin{macrocode}
%<*dotfill>
\NeedsTeXFormat{LaTeX2e}
\ProvidesPackage{zref-dotfill}%
  [2020-07-03 v2.32 Module dotfill for zref (HO)]%
\RequirePackage{zref-base}[2019/11/29]
\ifx\ZREF@base@ok Y%
\else
  \expandafter\endinput
\fi
%    \end{macrocode}
%
%    For measuring the width of \cs{zdotfill} we use the features
%    provided by module \xmodule{savepos}.
%    \begin{macrocode}
\RequirePackage{zref-savepos}[2019/11/29]
%    \end{macrocode}
%    For automatically generated label names we use the unique
%    counter of module \xmodule{base}.
%    \begin{macrocode}
\zref@require@unique
%    \end{macrocode}
%    Configuration is done by the key value interface of package
%    \xpackage{keyval}.
%    \begin{macrocode}
\RequirePackage{keyval}
%    \end{macrocode}
%    The definitions of the keys follow.
%    \begin{macrocode}
\define@key{ZREF@DF}{unit}{%
  \def\ZREF@df@unit{#1}%
}
\define@key{ZREF@DF}{min}{%
  \def\ZREF@df@min{#1}%
}
\define@key{ZREF@DF}{dot}{%
  \def\ZREF@df@dot{#1}%
}
%    \end{macrocode}
%    Defaults are set, see user interface.
%    \begin{macrocode}
\providecommand\ZREF@df@min{2}
\providecommand\ZREF@df@unit{.44em}
\providecommand\ZREF@df@dot{.}
%    \end{macrocode}
%    \begin{macro}{\zdotfillsetup}
%    Configuration of \cs{zdotfill} is done by \cs{zdotfillsetup}.
%    \begin{macrocode}
\newcommand*{\zdotfillsetup}{\kvsetkeys{ZREF@DF}}
%    \end{macrocode}
%    \end{macro}
%
%    \begin{macro}{\zdotfill}
%    \cs{zdotfill} sets labels at the left and the right to get
%    the horizontal position. \cs{zsavepos} is not used, because
%    we do not need the vertical position.
%    \begin{macrocode}
\ZREF@IfDefinable\zdotfill\def{%
  {%
    \leavevmode
    \global\advance\c@zref@unique\ltx@one
    \begingroup
      \def\ZREF@temp{zref@\number\c@zref@unique}%
      \pdfsavepos
      \zref@labelbyprops{\thezref@unique L}{posx}%
      \setlength{\dimen@}{\ZREF@df@unit}%
      \zref@ifrefundefined{\thezref@unique R}{%
        \ZREF@dotfill
      }{%
        \ifnum\numexpr\zposx{\thezref@unique R}%
                      -\zposx{\thezref@unique L}\relax
            <\dimexpr\ZREF@df@min\dimen@\relax
          \hfill
        \else
          \ZREF@dotfill
        \fi
      }%
      \pdfsavepos
      \zref@labelbyprops{\thezref@unique R}{posx}%
    \endgroup
    \kern\z@
  }%
}
%    \end{macrocode}
%    \end{macro}
%    \begin{macro}{\ZREF@dotfill}
%    Help macro that actually sets the dots.
%    \begin{macrocode}
\def\ZREF@dotfill{%
  \cleaders\hb@xt@\dimen@{\hss\ZREF@df@dot\hss}\hfill
}
%    \end{macrocode}
%    \end{macro}
%
%    \begin{macrocode}
%</dotfill>
%    \end{macrocode}
%
% \subsection{Module \xmodule{env}}
%
%    \begin{macrocode}
%<*env>
\NeedsTeXFormat{LaTeX2e}
\ProvidesPackage{zref-env}%
  [2020-07-03 v2.32 Module env for zref (HO)]%
\RequirePackage{zref-base}[2019/11/29]
\ifx\ZREF@base@ok Y%
\else
  \expandafter\endinput
\fi
%    \end{macrocode}
%
%    \begin{macrocode}
\zref@newprop{envname}[]{\@currenvir}
\zref@newprop{envline}[]{\zref@env@line}
%    \end{macrocode}
%
%    \begin{macro}{\zref@env@line}
%    Macro \cs{zref@env@line} extracts the line number
%    from \cs{@currenvline}.
%    \begin{macrocode}
\def\zref@env@line{%
  \ifx\@currenvline\ltx@empty
  \else
    \expandafter
    \ZREF@ENV@line\@currenvline\ltx@empty line \ltx@empty\@nil
  \fi
}
%    \end{macrocode}
%    \end{macro}
%    \begin{macro}{\ZREF@ENV@line}
%    \begin{macrocode}
\def\ZREF@ENV@line#1line #2\ltx@empty#3\@nil{#2}%
%    \end{macrocode}
%    \end{macro}
%
%    \begin{macrocode}
%</env>
%    \end{macrocode}
%
%% \section{Installation}
%
% \subsection{Download}
%
% \paragraph{Package.} This package is available on
% CTAN\footnote{\CTANpkg{zref}}:
% \begin{description}
% \item[\CTAN{macros/latex/contrib/zref/zref.dtx}] The source file.
% \item[\CTAN{macros/latex/contrib/zref/zref.pdf}] Documentation.
% \end{description}
%
%
% \paragraph{Bundle.} All the packages of the bundle `zref'
% are also available in a TDS compliant ZIP archive. There
% the packages are already unpacked and the documentation files
% are generated. The files and directories obey the TDS standard.
% \begin{description}
% \item[\CTANinstall{install/macros/latex/contrib/zref.tds.zip}]
% \end{description}
% \emph{TDS} refers to the standard ``A Directory Structure
% for \TeX\ Files'' (\CTANpkg{tds}). Directories
% with \xfile{texmf} in their name are usually organized this way.
%
% \subsection{Bundle installation}
%
% \paragraph{Unpacking.} Unpack the \xfile{zref.tds.zip} in the
% TDS tree (also known as \xfile{texmf} tree) of your choice.
% Example (linux):
% \begin{quote}
%   |unzip zref.tds.zip -d ~/texmf|
% \end{quote}
%
%
% \subsection{Package installation}
%
% \paragraph{Unpacking.} The \xfile{.dtx} file is a self-extracting
% \docstrip\ archive. The files are extracted by running the
% \xfile{.dtx} through \plainTeX:
% \begin{quote}
%   \verb|tex zref.dtx|
% \end{quote}
%
% \paragraph{TDS.} Now the different files must be moved into
% the different directories in your installation TDS tree
% (also known as \xfile{texmf} tree):
% \begin{quote}
% \def\t{^^A
% \begin{tabular}{@{}>{\ttfamily}l@{ $\rightarrow$ }>{\ttfamily}l@{}}
%   zref.sty & tex/latex/zref/zref.sty\\
%   zref-base.sty & tex/latex/zref/zref-base.sty\\
%   zref-abspage.sty & tex/latex/zref/zref-abspage.sty\\
%   zref-abspos.sty & tex/latex/zref/zref-abspos.sty\\
%   zref-counter.sty & tex/latex/zref/zref-counter.sty\\
%   zref-dotfill.sty & tex/latex/zref/zref-dotfill.sty\\
%   zref-env.sty & tex/latex/zref/zref-env.sty\\
%   zref-hyperref.sty & tex/latex/zref/zref-hyperref.sty\\
%   zref-lastpage.sty & tex/latex/zref/zref-lastpage.sty\\
%   zref-marks.sty & tex/latex/zref/zref-marks.sty\\
%   zref-nextpage.sty & tex/latex/zref/zref-nextpage.sty\\
%   zref-pageattr.sty & tex/latex/zref/zref-pageattr.sty\\
%   zref-pagelayout.sty & tex/latex/zref/zref-pagelayout.sty\\
%   zref-perpage.sty & tex/latex/zref/zref-perpage.sty\\
%   zref-runs.sty & tex/latex/zref/zref-runs.sty\\
%   zref-savepos.sty & tex/latex/zref/zref-savepos.sty\\
%   zref-thepage.sty & tex/latex/zref/zref-thepage.sty\\
%   zref-titleref.sty & tex/latex/zref/zref-titleref.sty\\
%   zref-totpages.sty & tex/latex/zref/zref-totpages.sty\\
%   zref-user.sty & tex/latex/zref/zref-user.sty\\
%   zref-xr.sty & tex/latex/zref/zref-xr.sty\\
%   zref.pdf & doc/latex/zref/zref.pdf\\
%   zref-example.tex & doc/latex/zref/zref-example.tex\\
%   zref-example-lastpage.tex & doc/latex/zref/zref-example-lastpage.tex\\
%   zref-example-nextpage.tex & doc/latex/zref/zref-example-nextpage.tex\\
%   zref.dtx & source/latex/zref/zref.dtx\\
% \end{tabular}^^A
% }^^A
% \sbox0{\t}^^A
% \ifdim\wd0>\linewidth
%   \begingroup
%     \advance\linewidth by\leftmargin
%     \advance\linewidth by\rightmargin
%   \edef\x{\endgroup
%     \def\noexpand\lw{\the\linewidth}^^A
%   }\x
%   \def\lwbox{^^A
%     \leavevmode
%     \hbox to \linewidth{^^A
%       \kern-\leftmargin\relax
%       \hss
%       \usebox0
%       \hss
%       \kern-\rightmargin\relax
%     }^^A
%   }^^A
%   \ifdim\wd0>\lw
%     \sbox0{\small\t}^^A
%     \ifdim\wd0>\linewidth
%       \ifdim\wd0>\lw
%         \sbox0{\footnotesize\t}^^A
%         \ifdim\wd0>\linewidth
%           \ifdim\wd0>\lw
%             \sbox0{\scriptsize\t}^^A
%             \ifdim\wd0>\linewidth
%               \ifdim\wd0>\lw
%                 \sbox0{\tiny\t}^^A
%                 \ifdim\wd0>\linewidth
%                   \lwbox
%                 \else
%                   \usebox0
%                 \fi
%               \else
%                 \lwbox
%               \fi
%             \else
%               \usebox0
%             \fi
%           \else
%             \lwbox
%           \fi
%         \else
%           \usebox0
%         \fi
%       \else
%         \lwbox
%       \fi
%     \else
%       \usebox0
%     \fi
%   \else
%     \lwbox
%   \fi
% \else
%   \usebox0
% \fi
% \end{quote}
% If you have a \xfile{docstrip.cfg} that configures and enables \docstrip's
% TDS installing feature, then some files can already be in the right
% place, see the documentation of \docstrip.
%
% \subsection{Refresh file name databases}
%
% If your \TeX~distribution
% (\TeX\,Live, \mikTeX, \dots) relies on file name databases, you must refresh
% these. For example, \TeX\,Live\ users run \verb|texhash| or
% \verb|mktexlsr|.
%
% \subsection{Some details for the interested}
%
% \paragraph{Unpacking with \LaTeX.}
% The \xfile{.dtx} chooses its action depending on the format:
% \begin{description}
% \item[\plainTeX:] Run \docstrip\ and extract the files.
% \item[\LaTeX:] Generate the documentation.
% \end{description}
% If you insist on using \LaTeX\ for \docstrip\ (really,
% \docstrip\ does not need \LaTeX), then inform the autodetect routine
% about your intention:
% \begin{quote}
%   \verb|latex \let\install=y% \iffalse meta-comment
%
% File: zref.dtx
% Version: 2020-07-03 v2.32
% Info: A new reference scheme for LaTeX
%
% Copyright (C)
%    2006-2012  Heiko Oberdiek
%    2016-2020  Oberdiek Package Support Group
%    https://github.com/ho-tex/zref/issues
%
% This work may be distributed and/or modified under the
% conditions of the LaTeX Project Public License, either
% version 1.3c of this license or (at your option) any later
% version. This version of this license is in
%    https://www.latex-project.org/lppl/lppl-1-3c.txt
% and the latest version of this license is in
%    https://www.latex-project.org/lppl.txt
% and version 1.3 or later is part of all distributions of
% LaTeX version 2005/12/01 or later.
%
% This work has the LPPL maintenance status "maintained".
%
% The Current Maintainers of this work are
% Heiko Oberdiek and the Oberdiek Package Support Group
% https://github.com/ho-tex/zref/issues
%
% This work consists of the main source file zref.dtx
% and the derived files
%    zref.sty, zref.pdf, zref.ins, zref.drv, zref-base.sty,
%    zref-abspage.sty, zref-abspos.sty, zref-counter.sty,
%    zref-dotfill.sty, zref-env.sty, zref-hyperref.sty,
%    zref-lastpage.sty, zref-marks.sty, zref-nextpage.sty,
%    zref-pageattr.sty, zref-pagelayout.sty, zref-perpage.sty,
%    zref-runs.sty, zref-savepos.sty, zref-thepage.sty,
%    zref-titleref.sty, zref-totpages.sty, zref-user.sty,
%    zref-xr.sty, zref-example.tex, zref-example-lastpage.tex,
%    zref-example-nextpage.tex, zref-test1.tex,
%    zref-test-base.tex, zref-test-runs.tex,
%    zref-test-titleref-memoir.tex.
%
% Distribution:
%    CTAN:macros/latex/contrib/zref/zref.dtx
%    CTAN:macros/latex/contrib/zref/zref.pdf
%
% Unpacking:
%    (a) If zref.ins is present:
%           tex zref.ins
%    (b) Without zref.ins:
%           tex zref.dtx
%    (c) If you insist on using LaTeX
%           latex \let\install=y\input{zref.dtx}
%        (quote the arguments according to the demands of your shell)
%
% Documentation:
%    (a) If zref.drv is present:
%           latex zref.drv
%    (b) Without zref.drv:
%           latex zref.dtx; ...
%    The class ltxdoc loads the configuration file ltxdoc.cfg
%    if available. Here you can specify further options, e.g.
%    use A4 as paper format:
%       \PassOptionsToClass{a4paper}{article}
%
%    Programm calls to get the documentation (example):
%       pdflatex zref.dtx
%       makeindex -s gind.ist zref.idx
%       pdflatex zref.dtx
%       makeindex -s gind.ist zref.idx
%       pdflatex zref.dtx
%
% Installation:
%    TDS:tex/latex/zref/zref.sty
%    TDS:tex/latex/zref/zref-base.sty
%    TDS:tex/latex/zref/zref-abspage.sty
%    TDS:tex/latex/zref/zref-abspos.sty
%    TDS:tex/latex/zref/zref-counter.sty
%    TDS:tex/latex/zref/zref-dotfill.sty
%    TDS:tex/latex/zref/zref-env.sty
%    TDS:tex/latex/zref/zref-hyperref.sty
%    TDS:tex/latex/zref/zref-lastpage.sty
%    TDS:tex/latex/zref/zref-marks.sty
%    TDS:tex/latex/zref/zref-nextpage.sty
%    TDS:tex/latex/zref/zref-pageattr.sty
%    TDS:tex/latex/zref/zref-pagelayout.sty
%    TDS:tex/latex/zref/zref-perpage.sty
%    TDS:tex/latex/zref/zref-runs.sty
%    TDS:tex/latex/zref/zref-savepos.sty
%    TDS:tex/latex/zref/zref-thepage.sty
%    TDS:tex/latex/zref/zref-titleref.sty
%    TDS:tex/latex/zref/zref-totpages.sty
%    TDS:tex/latex/zref/zref-user.sty
%    TDS:tex/latex/zref/zref-xr.sty
%    TDS:doc/latex/zref/zref.pdf
%    TDS:doc/latex/zref/zref-example.tex
%    TDS:doc/latex/zref/zref-example-lastpage.tex
%    TDS:doc/latex/zref/zref-example-nextpage.tex
%    TDS:source/latex/zref/zref.dtx
%
%<*ignore>
\begingroup
  \catcode123=1 %
  \catcode125=2 %
  \def\x{LaTeX2e}%
\expandafter\endgroup
\ifcase 0\ifx\install y1\fi\expandafter
         \ifx\csname processbatchFile\endcsname\relax\else1\fi
         \ifx\fmtname\x\else 1\fi\relax
\else\csname fi\endcsname
%</ignore>
%<*install>
\input docstrip.tex
\Msg{************************************************************************}
\Msg{* Installation}
\Msg{* Package: zref 2020-07-03 v2.32 A new reference scheme for LaTeX (HO)}
\Msg{************************************************************************}

\keepsilent
\askforoverwritefalse

\let\MetaPrefix\relax
\preamble

This is a generated file.

Project: zref
Version: 2020-07-03 v2.32

Copyright (C)
  2006-2012 Heiko Oberdiek
  2016-2020 Oberdiek Package Support Group

This work may be distributed and/or modified under the
conditions of the LaTeX Project Public License, either
version 1.3c of this license or (at your option) any later
version. This version of this license is in
   https://www.latex-project.org/lppl/lppl-1-3c.txt
and the latest version of this license is in
   https://www.latex-project.org/lppl.txt
and version 1.3 or later is part of all distributions of
LaTeX version 2005/12/01 or later.

This work has the LPPL maintenance status "maintained".

The Current Maintainers of this work are
Heiko Oberdiek and the Oberdiek Package Support Group
https://github.com/ho-tex/zref/issues


This work consists of the main source file zref.dtx
and the derived files
   zref.sty, zref.pdf, zref.ins, zref.drv, zref-base.sty,
   zref-abspage.sty, zref-abspos.sty, zref-counter.sty,
   zref-dotfill.sty, zref-env.sty, zref-hyperref.sty,
   zref-lastpage.sty, zref-marks.sty, zref-nextpage.sty,
   zref-pageattr.sty, zref-pagelayout.sty, zref-perpage.sty,
   zref-runs.sty, zref-savepos.sty, zref-thepage.sty,
   zref-titleref.sty, zref-totpages.sty, zref-user.sty,
   zref-xr.sty, zref-example.tex, zref-example-lastpage.tex,
   zref-example-nextpage.tex, zref-test1.tex,
   zref-test-base.tex, zref-test-runs.tex,
   zref-test-titleref-memoir.tex.

\endpreamble
\let\MetaPrefix\DoubleperCent

\generate{%
  \file{zref.ins}{\from{zref.dtx}{install}}%
  \file{zref.drv}{\from{zref.dtx}{driver}}%
  \usedir{tex/latex/zref}%
  \file{zref.sty}{\from{zref.dtx}{package}}%
  \file{zref-base.sty}{\from{zref.dtx}{base}}%
  \file{zref-abspage.sty}{\from{zref.dtx}{abspage}}%
  \file{zref-abspos.sty}{\from{zref.dtx}{abspos}}%
  \file{zref-counter.sty}{\from{zref.dtx}{counter}}%
  \file{zref-dotfill.sty}{\from{zref.dtx}{dotfill}}%
  \file{zref-env.sty}{\from{zref.dtx}{env}}%
  \file{zref-hyperref.sty}{\from{zref.dtx}{hyperref}}%
  \file{zref-lastpage.sty}{\from{zref.dtx}{lastpage}}%
  \file{zref-marks.sty}{\from{zref.dtx}{marks}}%
  \file{zref-nextpage.sty}{\from{zref.dtx}{nextpage}}%
  \file{zref-pageattr.sty}{\from{zref.dtx}{pageattr}}%
  \file{zref-pagelayout.sty}{\from{zref.dtx}{pagelayout}}%
  \file{zref-perpage.sty}{\from{zref.dtx}{perpage}}%
  \file{zref-runs.sty}{\from{zref.dtx}{runs}}%
  \file{zref-savepos.sty}{\from{zref.dtx}{savepos}}%
  \file{zref-thepage.sty}{\from{zref.dtx}{thepage}}%
  \file{zref-titleref.sty}{\from{zref.dtx}{titleref}}%
  \file{zref-totpages.sty}{\from{zref.dtx}{totpages}}%
  \file{zref-user.sty}{\from{zref.dtx}{user}}%
  \file{zref-xr.sty}{\from{zref.dtx}{xr}}%
  \usedir{doc/latex/zref}%
  \file{zref-example.tex}{\from{zref.dtx}{example}}%
  \file{zref-example-lastpage.tex}{\from{zref.dtx}{example-lastpage}}%
  \file{zref-example-nextpage.tex}{\from{zref.dtx}{example-nextpage}}%
%  \usedir{doc/latex/zref/test}%
%  \file{zref-test1.tex}{\from{zref.dtx}{test1}}%
%  \file{zref-test-base.tex}{\from{zref.dtx}{test-base}}%
%  \file{zref-test-runs.tex}{\from{zref.dtx}{test-runs}}%
%  \file{zref-test-titleref-memoir.tex}{\from{zref.dtx}{test-titleref-memoir}}%
}

\catcode32=13\relax% active space
\let =\space%
\Msg{************************************************************************}
\Msg{*}
\Msg{* To finish the installation you have to move the following}
\Msg{* files into a directory searched by TeX:}
\Msg{*}
\Msg{*     zref.sty, zref-base.sty, zref-abspage.sty, zref-abspos.sty,}
\Msg{*     zref-counter.sty, zref-dotfill.sty, zref-env.sty,}
\Msg{*     zref-hyperref.sty, zref-lastpage.sty, zref-marks.sty,}
\Msg{*     zref-nextpage.sty, zref-pageattr.sty, zref-pagelayout.sty,}
\Msg{*     zref-perpage.sty, zref-runs.sty, zref-savepos.sty,}
\Msg{*     zref-thepage.sty, zref-titleref.sty, zref-totpages.sty,}
\Msg{*     zref-user.sty, zref-xr.sty}
\Msg{*}
\Msg{* To produce the documentation run the file `zref.drv'}
\Msg{* through LaTeX.}
\Msg{*}
\Msg{* Happy TeXing!}
\Msg{*}
\Msg{************************************************************************}

\endbatchfile
%</install>
%<*ignore>
\fi
%</ignore>
%<*driver>
\NeedsTeXFormat{LaTeX2e}
\ProvidesFile{zref.drv}%
  [2020-07-03 v2.32 A new reference scheme for LaTeX (HO)]%
\documentclass{ltxdoc}
\usepackage{holtxdoc}[2011/11/22]
\setlength{\emergencystretch}{1em}
\begin{document}
  \DocInput{zref.dtx}%
\end{document}
%</driver>
% \fi
%
%
%
% \GetFileInfo{zref.drv}
%
% \title{The \xpackage{zref} package}
% \date{2020-07-03 v2.32}
% \author{Heiko Oberdiek\thanks
% {Please report any issues at \url{https://github.com/ho-tex/zref/issues}}}
%
% \maketitle
%
% \begin{abstract}
% Package \xpackage{zref} tries to get rid of the restriction
% in \hologo{LaTeX}'s reference system that only two properties are
% supported. The package implements an extensible referencing
% system, where properties are handled in a more flexible way.
% It offers an interface for macro programmers for the access
% to the system and some applications that uses the new
% reference scheme.
% \end{abstract}
%
% \tableofcontents
%
% \newcommand*{\metacs}[1]{\texttt{\itshape #1}}
%
% \newcommand{\markexp}{^^A
%   \textsuperscript{exp}^^A
%   \space
% }
% \newcommand{\markexptwo}{^^A
%   \textsuperscript{exp2}^^A
%   \space
% }
% \newcommand{\markrobust}{^^A
%   \textsuperscript{robust}^^A
%   \space
% }
% \newcommand{\marknowarn}{^^A
%   \textsuperscript{nowarn}^^A
%   \space
% }
% \newcommand{\markbabel}{^^A
%   \textsuperscript{babel}^^A
%   \space
% }
%
% \section{Introduction}
%
% Standard \hologo{LaTeX}'s reference system with \cs{label}, \cs{ref},
% and \cs{pageref} supports two properties, the apperance of
% the counter that is last incremented by \cs{refstepcounter} and
% the page with the \cs{label} command.
%
% Unhappily \hologo{LaTeX} does not provide an interface for adding
% another properties. Packages such as \xpackage{hyperref},
% \xpackage{nameref}, or \xpackage{titleref} are forced to
% use ugly hacks to extend the reference system.
% These ugly hacks are one of the causes for \xpackage{hyperref}'s
% difficulty regarding compatibility with other packages.
%
% \subsection{Standard \hologo{LaTeX} behaviour}
%
% References are created by the \cs{label} command:
% \begin{quote}
%   |\chapter{Second chapter}|\\
%   |\section{First section on page 7} % section 2.1|\\
%   |\label{myref}|
% \end{quote}
% Now \hologo{LaTeX} records the section number |2.1| and the page |7|
% in the reference. Internally the reference is a list with
% two entries:
% \begin{quote}
%   |\r@myref| $\rightarrow$ |{2.1}{7}|
% \end{quote}
% The length of the list if fixed in the \hologo{LaTeX} kernel,
% An interface for adding new properties is missing.
%
% There are several tries to add new properties:
% \begin{description}
% \item[\xpackage{hyperref}] uses a list of five
%   properties instead of the standard list with two entries.
%   This causes many compatibility problems with \hologo{LaTeX} and
%   other packages.
% \item[\xpackage{titleref}] stores its title data into the
%   first entry in the list. \hologo{LaTeX} is happy because it
%   does only see its list with two entries. The situation
%   becomes more difficult, if more properties are added
%   this way. Then the macros form a nested structure
%   inside the first reference argument for the label.
%   Expandable extractions will then become painful.
% \end{description}
%
% \subsection{Basic idea}
%
% Some time ago Morten H\o gholm sent me an experimental
% cross referencing mechanism as ``expl3'' code. His
% idea is:
% \begin{quote}
%   |\g_xref_mylabel_plist| $\rightarrow$\\
%   \mbox{}\qquad|\xref_dance_key{salsa}\xref_name_key{Morten}|\dots
% \end{quote}
% The entries have the following format:
% \begin{quote}
%   |\xref_|\meta{your key}|_key{|\meta{some text}|}|
% \end{quote}
% This approach is much more flexible:
% \begin{itemize}
%   \item New properties can easily be added, just use a new key.
%   \item The length of the list is not fixed. A reference
%         can use a subset of the keys.
%   \item The order of the entries does not matter.
% \end{itemize}
% Unhappily I am not familiar with the experimental code
% for \hologo{LaTeX3} that will need some time before its first release.
% Thus I have implemented it as \hologo{LaTeXe} package without disturbing
% the existing \hologo{LaTeX} reference system.
%
% \subsection{Interfaces}
%
% The package provides a generic \emph{interface for programmers}.
% Commands of this interface are prefixed by \cs{zref@}.
%
% Option \xoption{user} enabels the \emph{user interface}.
% Here the commands are prefixed by \cs{z} to avoid name
% clashes with existing macros.
%
% Then the packages provides some \emph{modules}. They
% are applications for the reference system and can also be
% considered as examples how to use the reference system.
%
% The modules can be loaded as packages. The package
% name is prefixed with |zref-|, for example:
%\begin{quote}
%\begin{verbatim}
%\RequirePackage{zref-abspage}
%\end{verbatim}
%\end{quote}
% This is the preferred way if the package is loaded from
% within other packages to avoid option clashes.
%
% As alternative package |zref| can be used and the modules
% are given as options:
%\begin{quote}
%\begin{verbatim}
%\usepackage[perpage,user]{zref}
%\end{verbatim}
%\end{quote}
%
% \section{Interface for programmers}
%
% The user interface is described in the next section \ref{sec:user}.
%
% \subsection{Entities}
%
% \paragraph{Reference.}
% Internally a reference is a list of key value pairs:
% \begin{quote}
%   |\Z@R@myref| $\rightarrow$
%   |\default{2.1}\page{7}|
% \end{quote}
% The generic format of a entry is:
% \begin{quote}
%   |\Z@R@|\meta{refname} $\rightarrow$
%   |\|\meta{propname}|{|\meta{value}|}|
% \end{quote}
% \meta{refname} is the name that denoted references
% (the name used in \cs{label} and \cs{ref}). \meta{propname}
% is the name of the property or key. The property key macro
% is never executed, it is used in parameter text matching only.
%
% \paragraph{Property.}
% Because the name of a property is used in a macro name
% that must survive the \xfile{.aux} file, the name is
% restricted to letters and `@'.
%
% \paragraph{Property list.}
% Often references are used for special purposes. Thus it
% saves memory if just the properties are used in this
% reference that are necessary for its purpose.
%
% Therefore this package uses the concept of \emph{property lists}.
% A property list is a set of properties. The set of properties
% that is used by the default \cs{label} command is the
% \emph{main property list}.
%
% \subsection{Property list}
%
% \markexp means that the implementation of the
% marked macro is expandable. \markexptwo goes a step further
% and marks the macro expandable in exact two expansion steps.
%
% \begin{declcs}{zref@newlist}
%   \M{listname}
% \end{declcs}
% Declares a new empty property list.
%
% \begin{declcs}{zref@addprop}
%   \M{listname} \M{propname}\\
%   \cs{zref@localaddprop} \M{listname} \M{propname}
% \end{declcs}
% Adds the property \meta{propname}
% to the property list \meta{listname}.
% The property and list must exist.
% The addition is global by \cs{zref@addprop} and limited to
% local scope by \cs{zref@localaddprop}.
% Between 2010/04/19 v2.13 and 2010/10/22 v2.19 a comma separated list
% of properties could be used as argument \meta{propname}.
% Since 2010/10/22 v2.19 the addition of several properties
% at once is supported by \cs{zref@addprops}.
%
% \begin{declcs}{zref@addprops}
%   \M{listname} \M{propname list}\\
%   \cs{zref@localaddprops} \M{listname} \M{propname list}
% \end{declcs}
% These macros add a comma separated list of properties
% \meta{propname list} to list \meta{listname}.
% \cs{zref@addprops} works globally and \cs{zref@localaddprops}
% locally. Since 2010/10/22 v2.19.
%
% \begin{declcs}{zref@listexists}
%   \M{listname} \M{then}
% \end{declcs}
% Executes \meta{then} if the property list \meta{listname} exists
% or raise an error otherwise.
%
% \begin{declcs}{zref@iflistundefined}\markexp \M{listname}
%   \M{then} \M{else}
% \end{declcs}
% Executes \meta{then} if the list exists or \meta{else}
% otherwise.
%
% \begin{declcs}{zref@iflistcontainsprop}
%   \M{listname} \M{propname} \M{then} \M{else}
% \end{declcs}
% Executes \meta{then} if the property \meta{propname} is part
% of property list \meta{listname} or otherwise it
% runs the \meta{else} part.
%
% \subsection{Property}
%
% \begin{declcs}{zref@newprop}
%   \* \M{propname} |[|\meta{default}|]| \M{value}
% \end{declcs}
% This command declares and configures a new property with
% name \meta{propname}.
%
% In case of unknown references
% or the property does not exist in the reference, the
% \meta{default} is used as value. If it is not specified
% here, a global default is used, see \cs{zref@setdefault}.
%
% The correct values of some properties are not known
% immediately but at page shipout time. Prominent example
% is the page number. These properties are declared with
% the star form of the command.
%
% \begin{declcs}{zref@setcurrent}
%   \M{propname} \M{value}
% \end{declcs}
% This sets the current value of the property \meta{propname}.
% It is a generalization of setting \hologo{LaTeX}'s \cs{currentlabel}.
%
% \begin{declcs}{zref@getcurrent}\markexptwo
%   \M{propname}
% \end{declcs}
% This returns the current value of the property \meta{propname}.
% The value may not be correct, especially if the property
% is bound to a page (start form of \cs{zref@newprop}) and the
% right value is only known at shipout time (e.g. property `page').
% In case of errors (e.g. unknown property) the empty string is
% returned.
%
% Since version 2010/04/22 v2.14 \cs{zref@getcurrent} supports
% \cs{zref@wrapper@unexpanded}.
%
% \begin{declcs}{zref@propexists}
%   \M{propname} \M{then}
% \end{declcs}
% Calls \meta{then} if the property \meta{propname} is available
% or generates an error message otherwise.
%
% \begin{declcs}{zref@ifpropundefined}\markexp
%   \M{propname} \M{then} \M{else}
% \end{declcs}
% Calls \meta{then} or \meta{else} depending on the existence
% of property \meta{propname}.
%
% \subsection{Reference generation}
%
% \begin{declcs}{zref@label}
%   \M{refname}
% \end{declcs}
% This works similar to \cs{label}. The reference \meta{refname} is
% created and put into the \xfile{.aux} file with the
% properties of the main property list.
%
% \begin{declcs}{zref@labelbylist}
%   \M{refname} \M{listname}
% \end{declcs}
% Same as \cs{zref@label} except that the properties are
% taken from the specified property list \meta{listname}.
%
% \begin{declcs}{zref@labelbyprops}
%   \M{refname} |{|\meta{propnameA}|,|\meta{propnameB}|,|\dots|}|
% \end{declcs}
% Same as \cs{zref@label} except that these properties
% are used that are given as comma separated list in the
% second argument.
%
% \begin{declcs}{zref@newlabel}
%   \M{refname} |{|\dots|}|
% \end{declcs}
% This is the macro that is used in the \xfile{.aux} file.
% It is basically the same as \cs{newlabel} apart from
% the format of the data in the second argument.
%
% \subsection{Data extraction}
%
% \begin{declcs}{zref@extractdefault}\markexptwo
%   \M{refname} \M{propname} \M{default}
% \end{declcs}
% This is the basic command that refernces the value of a
% property \meta{propname} for the reference \meta{refname}.
% In case of errors such as undefined reference the \meta{default}
% is used instead.
%
% \begin{declcs}{zref@extract}\markexptwo
%   \M{refname} \M{propname}
% \end{declcs}
% The command is an abbreviation for \cs{zref@extractdefault}.
% As default the default of the property is taken, otherwise
% the global default.
%
% Example for page references:
% \begin{quote}
% \begin{tabular}{@{}ll}
% \hologo{LaTeX}:& |\pageref{foobar}|\\
% \xpackage{zref}:& |\zref@extract{foobar}{page}|
% \end{tabular}
% \end{quote}
% Both \cs{zref@extract} and \cs{zref@extractdefault} are
% expandable. That means, these macros can directly
% be used in expandable calculations, see the example file.
% On the other side, babel's shorthands are not supported,
% there are no warnings in case of undefined references.
%
% If an user interface doesn't need expandable macros
% then it can use \cs{zref@refused} and \cs{zref@wrapper@babel}
% for its user macros.
%
% \begin{declcs}{zref@refused}
%   \M{refname}
% \end{declcs}
% This command is not expandable. It causes the warnings if
% the reference \meta{refname} is not defined. Use the
% \cs{zref@extract} commands inside expandable contexts and
% mark their use outside by \cs{zref@refused}, see the example
% file.
%
% \begin{declcs}{zref@def@extract}%
%     \M{cmd} \M{refname} \M{propname}\\
%   \cs{zref@def@extractdefault}
%     \M{cmd} \M{refname} \M{propname} \M{default}
% \end{declcs}
% Both macros extract the property \meta{propname} from the reference
% \meta{refname} the same way as macros \cs{zref@extract} and
% \cs{zref@extractdefault}. The result is stored in macro \meta{cmd}.
% Also \cs{zref@refused} is called to notify \hologo{LaTeX} that
% the reference \meta{refname} is used. Added in 2011/10/04 v2.22.
%
% \begin{declcs}{zref@ifrefundefined}\markexp
%   \M{refname} \M{then} \M{else}
% \end{declcs}
% Macro \cs{zref@ifrefundefined} calls arguments \meta{then} or \meta{else}
% dependent on the existence of the reference \meta{refname}.
%
% \begin{declcs}{zifrefundefined}
%   \M{refname} \M{then} \M{else}
% \end{declcs}
% Macro \cs{zifrefundefined} calls \cs{ref@refused} before
% executing \cs{zref@ifrefundefined}. Babel shorthands are
% supported in \meta{refname}.
%
% \begin{declcs}{zref@ifrefcontainsprop}\markexp
%   \M{refname} \M{propname} \M{then} \M{else}
% \end{declcs}
% Test whether a reference provides a property.
%
% \subsection{Setup}
%
% \begin{declcs}{zref@default}
% \end{declcs}
% Holds the global default for unknown values.
%
% \begin{declcs}{zref@setdefault}
%   |{|\meta{value}|}|
% \end{declcs}
% Sets the global default for unknown values.
% The global default is used, if a property does not specify
% an own default and the value for a property
% cannot be extracted. This can happen if the reference is
% unknown or the reference does not have the property.
%
% \begin{declcs}{zref@setmainlist}
%   |{|\meta{value}|}|
% \end{declcs}
% Sets the name of the main property list. The package
% sets and uses |main|.
%
% \subsection{Declared properties}
%
% \begin{quote}
% \renewcommand*{\arraystretch}{1.05}
% \setlength{\tabcolsep}{2\tabcolsep}
% \def\VV{\vphantom{\large \"Ag}}
% \def\EMPTY{{\itshape\textless empty\textgreater}}
% \def\pl#1{&#1&thepage&\cs{number}\cs{#1}}
% \begin{tabular}{@{}>{\sffamily}l|>{\ttfamily}l|>{\ttfamily}l|>{\ttfamily}l@{}}
%   \rmfamily Module&
%   \rmfamily Property&
%   \rmfamily Property list&
%   \rmfamily Default\\
%   \hline
%   (base)
%     & default & main & \EMPTY\\
%     & page & main & \EMPTY\\
%   \hline
%   abspage
%     & abspage & main & 0\\
%   \hline
%   counter
%     & counter & main & \EMPTY\\
%   \hline
%   hyperref
%     & anchor & main & \EMPTY\\
%     & url    & & \EMPTY\\
%   \hline
%   pageattr
%     & pdfpageattr & thepage & \dots\\
%     & pdfpagesattr & LastPage & \dots\\
%   \hline
%   pagelayout\footnotemark
%     \pl{mag}\\
%     \pl{paperwidth}\\
%     \pl{paperheight}\\
%     \pl{stockwidth}\\
%     \pl{stockheight}\\
%     \pl{pdfpageheight}\\
%     \pl{pdfpagewidth}\\
%     \pl{pdfhorigin}\\
%     \pl{pdfvorigin}\\
%     \pl{hoffset}\\
%     \pl{voffset}\\
%     \pl{topmargin}\\
%     \pl{oddsidemargin}\\
%     \pl{evensidemargin}\\
%     \pl{textwidth}\\
%     \pl{textheight}\\
%     \pl{headheight}\\
%     \pl{headsep}\\
%     \pl{footskip}\\
%     \pl{marginparwidth}\\
%     \pl{marginparsep}\\
%     \pl{columnwidth}\\
%     \pl{columnsep}\\
%   \hline
%   perpage
%     & pagevalue & perpage & 0\\
%     & page & perpage & \EMPTY\\
%     & abspage & perpage & 0\\
%   \hline
%   savepos
%     & posx & savepos & 0\\
%     & posy & savepos & 0\\
%   \hline
%   titleref
%     & title & main & \EMPTY\\
%   \hline
%   xr
%     & anchor   & & \EMPTY\\
%     & externaldocument & & \EMPTY\\
%     & theotype & & \EMPTY\\
%     & title    & & \EMPTY\\
%     & url      & & \EMPTY\\
% \end{tabular}
% \footnotetext{Module \xmodule{pagelayout} only defines properties
% if the parameter exists.}%
% \end{quote}
%
% \subsection{Wrapper for advanced situations}
%
% \begin{declcs}{zref@wrapper@babel}
%   |{|\dots|}| \M{name}
% \end{declcs}
% This macro helps to add shorthand support. The second
% argument is protected, then the code of the first argument
% is called with the protected name appended. Examples
% are in the sources.
%
% \begin{declcs}{zref@wrapper@immediate}
%   |{|\dots|}|
% \end{declcs}
% There are situations where a label must be written instantly
% to the \xfile{.aux} file, for example after the last page.
% If the \cs{zlabel} or \cs{label} command is put inside this wrapper,
% immediate writing is enabled. See the implementation for
% module \xmodule{lastpage} for an example of its use.
%
% \begin{declcs}{zref@wrapper@unexpanded}
%   |{|\dots|}|
% \end{declcs}
% Assuming someone wants to extract a value for property |bar|
% and store the result in a macro |\foo| without traces of
% the expanding macros and without expanding the value.
% This (theoretical?) problem can be solved by this wrapper:
% \begin{quote}
%\begin{verbatim}
%\zref@wrapper@unexpanded{%
%  \edef\foo{%
%    \zref@extract{someref}{bar}%
%  }%
%}
%\end{verbatim}
% \end{quote}
% The \cs{edef} forces the expansion of \cs{zref@extract},
% but the extraction of the value is prevented by
% the wrapper that uses \hologo{eTeX}'\ \cs{unexpanded} for this
% purpose. Supported macros are \cs{zref@extract}, \cs{zref@extractdefault}
% and since version 2010/04/22 v2.14 macro \cs{zref@getcurrent}.
%
% \subsection{Counter for unique names}
%
% Some modules (\xmodule{titleref} and \xmodule{dotfillmin})
% need unique names for automatically generated label names.
%
% \begin{declcs}{zref@require@unique}
% \end{declcs}
% This command creates the unique counter |zref@unique| if the
% counter does not already exist.
%
% \begin{declcs}{thezref@unique}
% \end{declcs}
% This command is used to generate unique label names.
%
% \section{User interface}\label{sec:user}
%
% \subsection{Module \xmodule{user}}
%
% The user interface for this package and its modules
% is enabled by \xpackage{zref}'s package option \xmodule{user}
% or package \xpackage{zref-user}.
% The names of user commands are prefixed by |z| in order
% to avoid name clashes with existing macros of the same
% functionality. Thus the package does not disturb the
% traditional reference scheme, both can be used together.
%
% The syntax descriptions contain the following markers
% that are intended as hints for programmers:
% \begin{quote}
% \begin{tabular}{@{}ll@{}}
%   \markbabel\unskip  & Babel shorthands are allowed.\\
%   \markrobust\unskip & Robust macro.\\
%   \markexp\unskip    & Expandable version:\\
%                      & \textbullet\ robust,
%                        unless the extracted values are fragile,\\
%                      & \textbullet\ no babel shorthand suport.\\
%   \markexptwo\unskip & Expandable like \markexp and:\\
%                      & \textbullet\ expandable in exact two steps.\\
% \end{tabular}
% \end{quote}
%
% The basic user interface of the package without modules
% are commands that mimic the standard \hologo{LaTeX} behaviour
% of \cs{label}, \cs{ref}, and \cs{pageref}:
%
% \begin{declcs}{zlabel}
%   \M{refname}\markbabel
% \end{declcs}
% Similar to \cs{label}. It generates a label with name
% \meta{refname} in the new reference scheme.
%
% \begin{declcs}{zref}
%   |[|\meta{propname}|]| \M{refname}\markbabel
% \end{declcs}
% Without optional argument similar to \cs{ref}, it returns
% the default reference property. This property is named
% |default|:
% \[
%   |\zref{|x|}| \equiv |\zref[default]{|x|}|
% \]
%
% \begin{declcs}{zpageref}
%   \M{refname}\markbabel
% \end{declcs}
% Convenience macro, similar to \cs{pageref}.
% \[
%   |\zpageref{|x|}| \equiv |\zref[page]{|x|}|
% \]
%
% \begin{declcs}{zrefused}
%   \M{refname}\markbabel
% \end{declcs}
% Some of the user commands in the modules are expandable.
% The use of such commands do not cause any undefined
% reference warnings, because inside of expandable
% contexts this is not possible. However, if there is
% a place outside of expandable contexts, \cs{refused}
% is strongly recommended. The reference \meta{refname}
% is marked as used, undefined ones will generate
% warnings.
%
% \subsection{Module \xmodule{abspage}}
%
% With the help of package \xpackage{atbegshi}
% a new counter |abspage| with absolute page numbers is provided.
% For technical and historical reasons the counter itself is zero based:
% if you use it directly on the first page, e.g
% with |\arabic{abspage}| you will get 0 as value. When using |\zref|
% the first page will be page 1 as expected.
% Also a new property |abspage| is defined and added to the
% main property list. Thus you can reference the absolute
% page number:
% \begin{quote}
%   |Section \zref{foo} is on page \zpageref{foo}.|\\
%   |This is page \zref[abspage]{foo}|\\
%   |of \zref[abspage]{LastPage}.|
% \end{quote}
% The example also makes use of module \xmodule{lastpage}.
%
% \subsection{Module \xmodule{lastpage}}
%
% Provides the functionality of package \xpackage{lastpage}
% \cite{lastpage} in the new reference scheme.
% The label |LastPage| is put at the end of the document.
% You can refer the last page number with:
% \begin{quote}
%   |\zref@extract{LastPage}{page}| (+ |\zref@refused{LastPage}|)\\
% \end{quote}
% or
% \begin{quote}
%   |\zpageref{LastPage}| (module \xmodule{user})
% \end{quote}
%
% Since version 2008/10/01 v2.3 the module defines the list |LastPage|.
% In addition to the properties of the main list label |LastPage| also
% stores the properties of this list |LastPage|. The default of this
% list is empty. The list can be used by the user to add additional
% properties for label |LastPage|.
%
% \subsubsection{Tests for last page}
%
% Since version 2010/03/26 v2.8 the macros \cs{zref@iflastpage}
% and \cs{ziflastpage} were added. They test the reference,
% whether it is a reference of the last page.
%
% \begin{declcs}{zref@iflastpage}\markexp \M{refname} \M{then} \M{else}
% \end{declcs}
% Macro \cs{zref@iflastpage} compares the references \meta{refname}
% with \meta{LastPage}. Basis of the comparison is the value of
% property |abspage|, because the values are different for different
% pages. This is not ensured by property |page|. Therefore module
% \xmodule{abspage} is loaded by module \xmodule{lastpage}.
% If both values of property |abspage| are present and match,
% then \meta{then} is executed, otherwise code \meta{else} is called.
% If one or both references are undefined or lack the property |abspage|,
% then \meta{else} is executed.
%
% Macro \cs{zref@iflastpage} is expandable, therefore \cs{zref@refused}
% should be called on \meta{refname} and \meta{LastPage}.
%
% \begin{declcs}{ziflastpage} \M{refname} \M{then} \M{else}
% \end{declcs}
% Macro \cs{ziflastpage} has the same function as \cs{zref@iflastpage},
% but adds support for babel shorthands in \meta{refname} and
% calls \cs{zref@refused}. However macro \cs{ziflastpage} is not
% expandable.
%
% \subsubsection{Example}
%
%    \begin{macrocode}
%<*example-lastpage>
%<<END_EXAMPLE
\NeedsTeXFormat{LaTeX2e}
\documentclass{report}

\newcounter{foo}
\renewcommand*{\thefoo}{\Alph{foo}}

\usepackage{zref-lastpage,zref-user}[2019/11/29]

\makeatletter
\zref@newprop{thefoo}{\thefoo}
\zref@newprop{valuefoo}{\the\value{foo}}
\zref@newprop{chapter}{\thechapter}
\zref@addprops{LastPage}{thefoo,valuefoo,chapter}
\makeatother

\newcommand*{\foo}{%
  \stepcounter{foo}%
  [Current foo: \thefoo]%
}

\begin{document}
  \chapter{First chapter}
  Last page is \zref{LastPage}.\\
  Last chapter is \zref[chapter]{LastPage}.\\
  Last foo is \zref[thefoo]{LastPage}.\\
  Last value of foo is \zref[valuefoo]{LastPage}.\\
  \foo
  \chapter{Second chapter}
  \foo\foo\foo
  \chapter{Last chapter}
  \foo
\end{document}
%END_EXAMPLE
%</example-lastpage>
%    \end{macrocode}
%
% \subsection{Module \xmodule{thepage}}
%
% This module \xmodule{thepage} loads module \xmodule{abspage},
% constructs a reference name using the absolute page number
% and remembers property |page|. Other properties can be added
% by adding them to the property list |thepage|.
%
% \begin{declcs}{zthepage} \M{absolute page number}\\
% \end{declcs}
% Macro \cs{zthepage} is basically a \cs{zpageref}. The reference
% name is yield by the \meta{absolute page number}. If the reference
% is not defined, then the default for property |page| is used.
%
% \begin{declcs}{zref@thepage@name}\markexp \M{absolute page number}
% \end{declcs}
% Macro \cs{zref@thepage@name} returns the internal reference
% name that is constructed using the \meta{absolute page number}.
% The internal reference name should not be used directly, because
% it might change in future versions.
%
% \begin{declcs}{zref@thepage}\markexp \M{absolute page number}\\
%   \cs{zref@thepage@refused} \M{absolute page number}
% \end{declcs}
% Macro \cs{zref@thepage} returns the page number (\cs{thepage})
% of \meta{absolute page number}. Because this macro is expandable,
% \cs{zref@thepage@refused} is used outside an expandable context
% to mark the reference as used.
%
%
% \subsection{Module \xmodule{nextpage}}
%
% \begin{declcs}{znextpage}
% \end{declcs}
% Macro \cs{znextpage} prints \cs{thepage} of the following page.
% It gets the current absolute page number by using a label.
% There are three cases for the next page:
% \begin{enumerate}
% \item The next page is not known yet because of undefined references.
%   Then \cs{zunknownnextpagename} is used instead. The default for
%   this macro is the default of property |page|.
% \item This page is the last page. Then \cs{znonextpagename}
%   is used. Its default is empty.
% \item The next page is known, then \cs{thepage} of the next
%   page is used (the value of property |page| of the next page).
% \end{enumerate}
%
% \subsubsection{Configuration}
%
% The behaviour can be configured by the following macros.
%
% \begin{declcs}{zunknownnextpagename}\\
%   \cs{znonextpagename}
% \end{declcs}
% If the next page is not known or available, then
% \cs{znextpage} uses these name macros as default.
% \cs{zunknownnextpagename} is used in case of undefined
% references. Default is the value of property |page| of the
% next page (\cs{thepage}). Module \xmodule{thepage} is used.
%
% Macro \cs{znonextpagename} is used, if the next page does
% not exists. That means that the current page is last page.
% The default is empty.
%
% \begin{declcs}{znextpagesetup} \M{unknown} \M{no next}
%   \M{next}
% \end{declcs}
% Acording to the case (see \cs{znextpage}) macro \cs{znextpage}
% calls an internal macro with an argument. The argument is
% either \cs{thepage} of the next page or one of \cs{zunknownnextpagename}
% or \cs{znonextpagename}. These internal macro can be changed by
% \cs{znextpagesetup}. It expects the definition texts for these
% three cases of a macro with one argument.
% The default is
% \begin{quote}
%   |\znextpagesetup{#1}{#1}{#1}|
% \end{quote}
%
% \subsubsection{Example}
%
%    \begin{macrocode}
%<*example-nextpage>
%<<END_EXAMPLE
\documentclass{book}

\usepackage{zref-nextpage}[2019/11/29]
\znextpagesetup
  {\thepage}% next page is unknown
  {\thepage\ (#1)}% this page is last page
  {\thepage\ $\rightarrow$ #1}% next page is known
\renewcommand*{\znonextpagename}{last page}

\usepackage{fancyhdr}
\pagestyle{fancy}
\fancyhf{}
\fancyhead[LE,RO]{\znextpage}
\fancypagestyle{plain}{%
  \fancyhf{}%
  \fancyhead[LE,RO]{\znextpage}%
}

\begin{document}
\frontmatter
  \tableofcontents
\mainmatter
  \chapter{Hello World}
  \clearpage
  \section{Last section}
\end{document}
%END_EXAMPLE
%</example-nextpage>
%    \end{macrocode}
%
%
% \subsection{Module \xmodule{totpages}}
%
% For the total number of pages of a document you need to
% know the absolute page number of the last page.
% Both modules \xmodule{abspage} and \xmodule{lastpage}
% are necessary and automatically enabled.
%
% \begin{declcs}{ztotpages}\markexp
% \end{declcs}
% Prints the total number of pages or |0| if this
% number is not yet known. It expands to an explicit number
% and can also used even in expandable calculations (\cs{numexpr})
% or counter assignments.
%
% \subsection{Module \xmodule{pagelayout}}
%
% The module defines additional properties for each parameter
% of the page layout that is effective during page shipout.
% The value of length parameters is given in sp without the
% unit as plain number.
%
% Some parameters are specific for a class (e.g.\@ \xoption{stockwidth}
% and \xoption{stockheight} for class \xclass{memoir}) or the
% \hologo{TeX} engine like \hologo{pdfTeX}. If the parameter is not
% available, then the property will not be defined. The default value
% of the property is the current setting of the parameter.
%
% The module \xmodule{thepage} is loaded that generates a label
% for each page. The properties of module \xmodule{pagelayout}
% are added to the property list |thepage| of module \xmodule{thepage}.
%
% List of properties:
% \begin{quote}
%   \def\x#1{\cs{#1}&#1}%
%   \begin{tabular}{l>{\sffamily}ll}
%     \textrm{parameter} & \textrm{property} & remarks\\
%     \hline
%     \x{mag}\\
%     \x{paperwidth}\\
%     \x{paperheight}\\
%     \x{stockwidth}& class \xclass{memoir}\\
%     \x{stockheight}& class \xclass{memoir}\\
%     \x{pdfpagewidth}& \hologo{pdfTeX}, \hologo{LuaTeX}\\
%     \x{pdfpageheight}& \hologo{pdfTeX}, \hologo{LuaTeX}\\
%     \x{pdfhorigin}& \hologo{pdfTeX}, \hologo{LuaTeX}\\
%     \x{pdfvorigin}& \hologo{pdfTeX}, \hologo{LuaTeX}\\
%     \x{hoffset}\\
%     \x{voffset}\\
%     \x{topmargin}\\
%     \x{oddsidemargin}\\
%     \x{evensidemargin}\\
%     \x{textwidth}\\
%     \x{textheight}\\
%     \x{headheight}\\
%     \x{headsep}\\
%     \x{footskip}\\
%     \x{marginparwidth}\\
%     \x{marginparsep}\\
%     \x{columnwidth}\\
%     \x{columnsep}\\
%   \end{tabular}
% \end{quote}
%
% \begin{declcs}{zlistpagelayout}
% \end{declcs}
% At the end of document the page layout parameter for each page are
% printed into the \xext{log} file if macro \cs{zlistpagelayout}
% is called before |\end{document}| (preamble is a good place).
%
% \subsection{Module \xmodule{marks}}
%
% ToDo.
%
% \subsection{Module \xmodule{runs}}
%
% Module \xmodule{runs} counts the \hologo{LaTeX} runs since last
% \xext{aux} file creation and prints the number in the
% \xext{log} file.
%
% \begin{declcs}{zruns}\markexp
% \end{declcs}
% Prints the the total number of \hologo{LaTeX} runs including
% the current one. It expands to an explicit number.
% Before |begin{document}| the value is zero meaning
% the \xext{aux} file is not read yet. If a previous
% \xext{aux} file exists, the value found there increased
% by one is the new number. Otherwise \cs{zruns} is set to one.
% \hologo{LaTeX}~runs where the \xext{aux} files are not rewritten
% are not counted (see \cs{nofiles}).
%
% \subsection{Module \xmodule{perpage}}
%
% With \cs{@addtoreset} or \cs{numberwithin} a counter can be
% reset if another counter is incremented. This do not work
% well if the other counter is the page counter. The page counter is
% incremented in the output routine that is often called asynchronous
% somewhere on the next page. A reference mechanism costs at least
% two \hologo{LaTeX} runs, but ensures correct page counter values.
%
% \begin{declcs}{zmakeperpage}
%   |[|\meta{reset}|]| \M{counter}
% \end{declcs}
% At the of a new page counter \meta{counter} starts
% counting with value \meta{reset} (default is \texttt{1}).
% The macro has the same
% syntax and semantics as \cs{MakePerPage} of package
% \xpackage{perpage} \cite{perpage}.
% Also \xoption{perpage} of package \xpackage{footmisc} \cite{footmisc}
% can easily be simulated by
% \begin{quote}
% |\zmakeperpage{footnote}| \itshape |% \usepackage[perpage]{footmisc}|
%\end{quote}
% If footnote symbols are used, some people dislike the
% first symbol \ensuremath{\dagger}. It can easily be skipped:
% \begin{quote}
% |\zmakeperpage[2]{footnote}|
% \end{quote}
%
% \begin{declcs}{thezpage}\\
%   counter |zpage|
% \end{declcs}
% If the formatted counter value of the counter that is reset at
% a new page contains the page value, then you can use \cs{thezpage},
% the page number of the current page. Or counter |zpage| can be
% used, if the page number should be formatted differently from
% the current page number. Example:
%\begin{quote}
%\begin{verbatim}
%\newcounter{foobar}
%\zmakeperpage{foobar}
%\renewcommand*{\thefoobar}{\thezpage-\arabic{foobar}}
% % or
%\renewcommand*{\thefoobar}{\roman{zpage}-\arabic{foobar}}
%\end{verbatim}
%\end{quote}
%
% \begin{declcs}{zunmakeperpage}
%   \M{counter}
% \end{declcs}
% The reset mechanism for this counter is deactivated.
%
%
% \subsection{Module \xmodule{counter}}
%
% This option just add the property |counter| to the main
% property list. The property stores the counter name,
% that was responsible for the reference. This is the
% property \xpackage{hyperref}'s \cs{autoref} feature
% uses. Thus this property |counter| may be useful
% for a reimplementation of the autoref feature,
% see the section \ref{sec:todo} with the todo list.
%
% \subsection{Module \xmodule{titleref}}
%
% This option makes section and caption titles available
% to the reference system similar to packages \xpackage{titleref}
% or \xpackage{nameref}.
%
% \begin{declcs}{ztitleref}
%   \M{refname}\markbabel
% \end{declcs}
% Print the section or caption title of reference \meta{refname},
% similar to \cs{nameref} or \cs{titleref}.
%
% \begin{declcs}{ztitlerefsetup}
%   |{|\textit{key$_1$=value$_1$, key$_2$=value$_2$, \dots}|}|
% \end{declcs}
% This command allows to configure the behaviour of module
% \xmodule{titleref}. The following keys are available:
% \begin{description}
% \let\bfseries\relax
% \item[\texttt{title=}\meta{value}]\mbox{}\\*
%   Sets the current title.
% \item[\texttt{stripperiod=true\string|false}]\mbox{}\\*
%   Follow package \xpackage{nameref} that
%   removes a last period. Default: |true|.
% \item[\texttt{expand=true\string|false}]\mbox{}\\*
%   Package \cs{titleref} expands the title first. This way
%   garbage and dangerous commands can be removed, e.g.
%   \cs{label}, \cs{index}\dots. See implementation section
%   for more details. Default is |false|.
% \item[\texttt{cleanup=\{}\textrm{\dots}\texttt{\}}]\mbox{}\\*
%   Hook to add own cleanup code, if method |expand| is used.
%   See implementation section for more details.
% \end{description}
%
% \subsection{Module \xmodule{savepos}}
%
% This option supports a feature that \pdfTeX\ provides
% (and \hologo{XeTeX}).
% \pdfTeX\ is able to tell the current position on the
% page. The page position is not instantly known. First the
% page must be constructed by \hologo{TeX}'s asynchronous output
% routine. Thus the time where the position is known is
% the page shipout time. Thus a reference system where the
% information is recorded in the first run and made available
% for use in the second run comes in handy.
%
% \begin{declcs}{zsavepos}
%   \M{refname}
% \end{declcs}
% It generates a reference with name \meta{refname}.
% The reference stores the location where \cs{zsavepos}
% is executed in properties |posx| and |posy|.
%
% \begin{declcs}{zsaveposx}
%   \M{refname}\\
%   \cs{zsaveposy}
%   \M{refname}
% \end{declcs}
% Same as \cs{zsavepos} except that only the |x| or |y| component
% of the position is stored. Since 2011/12/05 v2.23.
%
% \begin{declcs}{zposx}\markexp
%   \M{refname}\\
%   \SpecialUsageIndex{\zposy}\cs{zposy}\markexp \M{refname}
% \end{declcs}
% Get the position as number. Unit is sp. Horizontal
% positions by \cs{zposx} increase from left to right.
% Vertical positions by \cs{zposy} from bottom to top.
%
% Do not rely on absolute page numbers. Because of problems
% with the origin the numbers may differ in DVI or PDF mode
% of \pdfTeX. Therefore work with relative values by comparisons.
%
% Both \cs{zposx} and \cs{zposy} are expandable and can be
% used inside calculations (\cs{setcounter}, \cs{addtocounter},
% package \xpackage{calc}, \cs{numexpr}). However this property
% prevents from notifying \hologo{LaTeX} that the reference is actually
% used (the notifying is not expandable). Therefore you should
% mark the reference as used by \cs{zrefused}.
%
% This module uses \pdfTeX's \cs{pdfsavepos}, \cs{pdflastxpos},
% and \cs{pdflastypos}. They are available in PDF mode and
% since version 1.40.0 also in DVI mode.
%
% \begin{declcs}{zref@savepos}
% \end{declcs}
% Macro \cs{zref@savepos} performs the first part of \cs{zsavepos}
% by calling \cs{pdfsavepos} (if \xext{aux} files are writable).
%
% Thus \cs{zsavepos} is basically \cs{zref@savepos} followed
% by |\zref@labelbylist{|\texttt{\meta{refname}}|}{savepos}|.
% If \cs{TeXXeTstate} is detected and enabled, \cs{savepos}
% also adds \cs{zref@savepos} at the end to support \cs{beginR}
% where the whatits are processed in reverse order.
% The property list |savepos| contains the properties |posx|
% and |posy|.
%
% \subsection{Module \xmodule{abspos}}
% Module \xmodule{abspos} allows to get various values of the page layout. There is no user
% command, only a number of internal commands. For example:
% \begin{quote}
%   |\zref@absposx{|\meta{label}|}{|\meta{value}|}{|\meta{position}|}|\\
%   |\zref@absposy{|\meta{label}|}{|\meta{value}|}{|\meta{position}|}|
% \end{quote}
%
% The return value is like in the module \xmodule{savepos} a number representing a length in sp.
% The length are measured from the bottom left of the page.
%
% \meta{label} is a label set with \cs{zlabel} or \cs{zsavepos} that allows to retrieve the
% absolute page number.
%
% \meta{position} is for the x-command one of |left|, |right| or |center|. For the y-command
%  it is one of |top|, |bottom|, |center|.
%
% The possible content of \meta{value} can be seen in the following table. Be aware that the code
% makes some assumptions which are perhaps not always true -- for example that the left of
% the head is identical to the left of the body.
%
% \medskip
% \begin{tabular}{lll}
%  \textbf{value} & \textbf{axis} & \textbf{comments}\\\hline
%   media         & x             &   left=0, right=\cs{pdfpagewidth} \\
%   paper         & x             &   left=0, right=\cs{paperwidth}\\
%   stock         & x             &   derived from paper\\
%   media         & y             & bottom=0, top=\cs{pdfpageheigh}\\
%   paper         & y             & top=\cs{pdfpageheight}, bottom=top-\cs{paperheight}\\
%   stock         & y             & top  derived from paper\\
%   head          & x             & calculated with hoffset, horigin, etc\\
%   head          & y             & calculated\\
%   body          & x             & = head value\\%
%   body          & y             & = head bottom - \cs{headsep}\\
%   foot          & x             & = head\\
%   foot          & y             & calculated from body bottom and \cs{footskip}\\
%   marginpar     & x             &different on odd/even pages! \\
%   marginpar     & y             & = body %
% \end{tabular}
%
% \subsection{Module \xmodule{dotfill}}
%
% \begin{declcs}{zdotfill}
% \end{declcs}
% This package provides the command \cs{zdotfill} that works
% similar to \cs{dotfill}, but can be configured. Especially
% it suppresses the dots if a minimum number of dots cannot be set.
%
% \begin{declcs}{zdotfillsetup}
%   |{|\textit{key$_1$=value$_1$, key$_2$=value$_2$, \dots}|}|
% \end{declcs}
% This command allows to configure the behaviour of \cs{zdotfill}.
% The following keys are available:
% \begin{description}
% \let\bfseries\relax
% \item[\texttt{min=}\meta{count value}]\mbox{}\\*
%   If the actual number of dots are smaller than \meta{count value},
%   then the dots are suppressed. Default: |2|.
% \item[\texttt{unit=}\meta{dimen value}]\mbox{}\\*
%   The width of a dot unit is given by \meta{dimen value}.
%   Default: |0.44em| (same as the unit in \cs{dotfill}).
% \item[\texttt{dot=}\meta{value}]\mbox{}\\*
%   The dot itself is given by \meta{value}. Default: |.| (dot, same as
%   the dot in \cs{dotfill}).
% \end{description}
%
% \subsection{Module \xmodule{env}}
%
% This module defines two properties |envname| and |envline|. They
% remember the name of the environment and the line number at
% the start of the environment.
%
% \subsection{Module \xmodule{xr}}
%
% This package provides the functionality of package \xpackage{xr},
% see \cite{xr}. It also supports the syntax of \xpackage{xr-hyper}.
%
% \begin{declcs}{zexternaldocument}
%   \* |[|\meta{prefix}|]|\markbabel \M{external document}
%   |[|\meta{url}|]|
% \end{declcs}
% See \cs{externaldocument} for a description of this option.
% The found labels also get a property |externaldocument| that
% remembers \meta{external document}.
% The standard reference scheme and the scheme of this
% package use different name spaces for reference names.
% If the external document uses both systems. Then one
% import statement would put the names in one namespace
% and probably causing problems with multiple references
% of the same name. Thus the star form only looks for
% \cs{newlabel} in the \xfile{.aux} files, whereas without
% star only \cs{zref@newlabel}s are used.
%
% In the star form it tries to detect labels from
% \xpackage{hyperref}, \xpackage{titleref}, and \xpackage{ntheorem}.
% If such an extended property from the packages before cannot
% be found or are empty, they are not included in the
% imported reference.
%
% Warnings are given if a reference name is already in use
% and the item is ignored. Unknown properties will automatically
% be declared.
%
% If the external references contain |anchor| properties,
% then we need also a url to be able to address the external
% file. As default the filename is taken with a default
% extension.
%
% \begin{declcs}{zxrsetup}
%   |{|\textit{key$_1$=value$_1$, key$_2$=value$_2$, \dots}|}|
% \end{declcs}
% The following setup options are available:
% \begin{description}
% \item[\xoption{ext}:] It sets the default extension.
% \item[\xoption{tozreflabel}:] Boolean option. The found references
%   are imported as zref labels. This is enabled by default.
% \item[\xoption{toltxlabel}:] Boolean option. The found references
%   are imported as \hologo{LaTeX} labels. Packages \xpackage{nameref},
%   \xpackage{titleref} and class \xclass{memoir} are supported.
% \item[\xoption{urluse}:] Boolean option. If enabled, then a URL
%   is stored in a macro and the macro is put in property `urluse'.
%   The URL is not put in property `url'.
%   The purpose is to save \hologo{TeX} memory.
% \item[\xoption{verbose}:] Boolean option. List the imported labels in the
%   \xext{log} file. Default is |false|.
% \end{description}
%
% \begin{declcs}{zref@xr@ext}
% \end{declcs}
% If the \meta{url} is not specified in \cs{zref@externaldocument},
% then the url will be constructed with the file name and
% this macro as extension. \cs{XR@ext} is used if
% \xpackage{hyperref} is loaded, otherwise |pdf|.
%
% \subsection{Module \xmodule{pageattr}}
%
% This module allows to recover the content of the register
% \cs{pdfpageattr} and \cs{pdfpagesattr} in pdftex and the equivalent register
% in luatex. There is no user command. Programmer commands are
%
% \begin{quote}
% |\zref@pdfpageattr{|\meta{absolute page number}|}|
%
% |\zref@pdfpagesattr{|\meta{absolute page number}|}|
% \end{quote}
% \section{ToDo}\label{sec:todo}
%
% Among other things the following issues are left for future work:
% \begin{itemize}
% \item Other applications: autoref, hyperref, \dots
% \end{itemize}
%
% \section{Example}
%
%    \begin{macrocode}
%<*example>
%    \end{macrocode}
%
%    \begin{macrocode}
\documentclass{book}

\usepackage[ngerman]{babel}%

\usepackage[savepos,totpages,titleref,dotfill,counter,user]{zref}

%    \end{macrocode}
%
%    Chapters are wrapped inside \cs{ChapterStart} and
%    \cs{ChapterStop}. The first argument \verb|#1| of \cs{ChapterStart}
%    is used to form a label id \verb|chap:#1|. At the end of
%    the chapter another label is set by \cs{zref@wrapper@immediate}, because
%    otherwise at the end of document a deferred write would not
%    be written, because there is no page for shipout.
%
%    Also this example shows how chapter titles can be recorded.
%    A new property \verb|chaptitle| is declared and added to the
%    main property list. In \cs{ChapterStart} the current value
%    of the property is updated.
%    \begin{macrocode}
\makeatletter
\zref@newprop{chaptitle}{}
\zref@addprop{main}{chaptitle}

\newcommand*{\ChapterStart}[2]{%
  \cleardoublepage
  \def\current@chapid{#1}%
  \zref@setcurrent{chaptitle}{#2}%
  \chapter{#2}%
  \zlabel{chap:#1}%
}
\newcommand*{\ChapterStop}{%
  \cleardoublepage
  \zref@wrapper@immediate{%
    \zref@labelbyprops{chapend:\current@chapid}{abspage}%
  }%
}
%    \end{macrocode}
%    \cs{ChapterPages} calculates and returns the number of pages of the
%    referenced chapter.
%    \begin{macrocode}
\newcommand*{\ChapterPages}[1]{%
  \zrefused{chap:#1}%
  \zrefused{chapend:#1}%
  \number\numexpr
    \zref@extract{chapend:#1}{abspage}%
    -\zref@extract{chap:#1}{abspage}%
    +1\relax
}
\makeatother
%    \end{macrocode}
%    \begin{macrocode}
\begin{document}
%    \end{macrocode}
%    As exception we use \cs{makeatletter} here, because
%    this is just an example file that also should show some of
%    programmer's interface.
%    \begin{macrocode}
\makeatletter

\frontmatter
\zlabel{documentstart}

\begin{itemize}
\item
  The frontmatter part has
  \number\numexpr\zref@extract{chap:first}{abspage}-1\relax
  ~pages.
\item
  Chapter \zref{chap:first} has \ChapterPages{first} page(s).
\item
  Section \zref{hello} is on the
  \ifcase\numexpr
    \zref@extractdefault{hello}{page}{0}%
    -\zref@extractdefault{chap:first}{page}{0}%
    +1\relax
    ??\or first\or second\or third\or forth\fi
  ~page inside its chapter.
\item
  The document has
  \zref[abspage]{LastPage} pages.
  This number is \ifodd\ztotpages odd\else even\fi.
\item
  The last page is labeled with \zpageref{LastPage}.
\item
  The title of chapter \zref{chap:next} %
  is ``\zref[chaptitle]{chap:next}''.
\end{itemize}

\tableofcontents

\mainmatter
\ChapterStart{first}{First chapter}

%    \end{macrocode}
%    The user level commands should protect babel shorthands
%    where possible. On the other side, expandable extracting
%    macros are useful in calculations, see above the
%    examples with \cs{numexpr}.
%    \begin{macrocode}
\section{Test}
\zlabel{a"o}
Section \zref{a"o} on page
\zref@wrapper@babel\zref@extract{a"o}{page}.

Text.
\newpage

\section{Hello World}
\zlabel{hello}

\ChapterStop

\ChapterStart{next}{Next chapter with \emph{umlauts}: "a"o"u"s}

%    \end{macrocode}
%
%    Here an example follows that makes use of \pdfTeX's ``savepos''
%    feature. The position on the page is not known before the
%    page is constructed and shipped out. Therefore the position
%    ist stored in references and are available for calculations
%    in the next \hologo{LaTeX} compile run.
%    \begin{macrocode}
The width of the first column is
  \the\dimexpr \zposx{secondcol}sp - \zposx{firstcol}sp\relax,\\
the height difference of the two baselines is
  \the\dimexpr \zposy{firstcol}sp - \zposy{secondline}sp\relax:\\
\begin{tabular}{ll}
  \zsavepos{firstcol}Hello&\zsavepos{secondcol}World\\
  \zsavepos{secondline}Second line&foobar\\
\end{tabular}

%    \end{macrocode}
%    With \cs{zrefused} \hologo{LaTeX} is notified, if the references
%    are not yet available and \hologo{LaTeX} can generate the rerun
%    hint.
%    \begin{macrocode}
\zrefused{firstcol}
\zrefused{secondcol}
\zrefused{secondline}

\ChapterStop
%    \end{macrocode}
%    Test for module \cs{dotfill}.
%    \begin{macrocode}
\ChapterStart{dotfill}{Test for dotfill feature}
\newcommand*{\dftest}[1]{%
  #1&
  [\makebox[{#1}]{\dotfill}]&
  [\makebox[{#1}]{\zdotfill}]\\
}
\begin{tabular}{rll}
& [\verb|\dotfill|] & [\verb|\zdotfill|]\\
\dftest{0.43em}
\dftest{0.44em}
\dftest{0.45em}
\dftest{0.87em}
\dftest{0.88em}
\dftest{0.89em}
\dftest{1.31em}
\dftest{1.32em}
\dftest{1.33em}
\end{tabular}
\ChapterStop
\end{document}
%    \end{macrocode}
%
%    \begin{macrocode}
%</example>
%    \end{macrocode}
%
% \StopEventually{
% }
%
% \section{Implementation}
%
% \subsection{Package \xpackage{zref}}
%
% \subsubsection{Identification}
%
%    \begin{macrocode}
%<*package>
\NeedsTeXFormat{LaTeX2e}
\ProvidesPackage{zref}
  [2020-07-03 v2.32 A new reference scheme for LaTeX (HO)]%
%    \end{macrocode}
%
% \subsubsection{Load basic module}
%
%    \begin{macrocode}
\RequirePackage{zref-base}[2019/11/29]
%    \end{macrocode}
%    Abort package loading if \xpackage{zref-base} could not be
%    loaded successfully.
%    \begin{macrocode}
\@ifundefined{ZREF@base@ok}{\endinput}{}
%    \end{macrocode}
%
% \subsubsection{Process options}
%
%    Known modules are loaded and the release date is checked.
%    \begin{macrocode}
\def\ZREF@temp#1{%
  \DeclareOption{#1}{%
    \AtEndOfPackage{%
      \RequirePackage{zref-#1}[2019/11/29]%
    }%
  }%
}
\ZREF@temp{abspage}
\ZREF@temp{counter}
\ZREF@temp{dotfill}
\ZREF@temp{hyperref}
\ZREF@temp{lastpage}
\ZREF@temp{marks}
\ZREF@temp{nextpage}
\ZREF@temp{pageattr}
\ZREF@temp{pagelayout}
\ZREF@temp{perpage}
\ZREF@temp{runs}
\ZREF@temp{savepos}
\ZREF@temp{thepage}
\ZREF@temp{titleref}
\ZREF@temp{totpages}
\ZREF@temp{user}
\ZREF@temp{xr}
%    \end{macrocode}
%    \begin{macrocode}
\ProcessOptions\relax
%</package>
%    \end{macrocode}
%
%
% \subsection{Module \xmodule{base}}
%
% \subsubsection{Prefixes}
%
%    This package uses the following prefixes for macro names:
%    \begin{description}
%    \item[\cs{zref@}:] Macros of the programmer's interface.
%    \item[\cs{ZREF@}:] Internal macros.
%    \item[\cs{Z@L@}\metacs{listname}:]
%      The properties of the list \meta{listname}.
%    \item[\cs{Z@D@}\metacs{propname}:]
%      The default value for property \meta{propname}.
%    \item[\cs{Z@E@}\metacs{propname}:]
%      Extract function for property \meta{propname}.
%    \item[\cs{Z@X@}\metacs{propname}:]
%      Information whether a property value for property
%      \meta{propname} is expanded
%      immediately or at shipout time.
%    \item[\cs{Z@C@}\metacs{propname}:]
%      Current value of the property \meta{propname}.
%    \item[\cs{Z@R@}\metacs{labelname}:]
%      Data for reference \meta{labelname}.
%    \item[\cs{ZREF@org@}:] Original versions of patched commands.
%    \item[\cs{z}:] For macros in user land, defined if
%      module \xmodule{user} is set.
%    \end{description}
%    The following family names are used for keys defined
%    according to the \xpackage{keyval} package:
%    \begin{description}
%    \item[\texttt{ZREF@TR}:] Setup for module \xmodule{titleref}.
%    \end{description}
%
% \subsubsection{Identification}
%
%    \begin{macrocode}
%<*base>
\NeedsTeXFormat{LaTeX2e}
\ProvidesPackage{zref-base}%
  [2020-07-03 v2.32 Module base for zref (HO)]%
%    \end{macrocode}
%
% \subsubsection{Utilities}
%
%    \begin{macrocode}
\providecommand\IfFormatAtLeastTF{\@ifl@t@r\fmtversion}
\RequirePackage{ltxcmds}[2010/12/02]
\RequirePackage{infwarerr}[2010/04/08]
\RequirePackage{kvsetkeys}[2010/03/01]
\RequirePackage{kvdefinekeys}[2010/03/01]
\RequirePackage{pdftexcmds}[2010/04/01]
%    \end{macrocode}
%
%    \begin{macro}{\ZREF@name}
%    Several times the package name is used, thus we store it
%    in \cmd{\ZREF@name}.
%    \begin{macrocode}
\def\ZREF@name{zref}
%    \end{macrocode}
%    \end{macro}
%
%    \begin{macrocode}
\ltx@IfUndefined{protected}{%
  \RequirePackage{makerobust}[2006/03/18]%
%    \end{macrocode}
%    \begin{macro}{\ZREF@Robust}
%    \begin{macrocode}
  \def\ZREF@Robust#1#2{%
    \def\ZREF@temp{\MakeRobustcommand#2}%
    \afterassignment\ZREF@temp
    #1#2%
  }%
%    \end{macrocode}
%    \end{macro}
%    \begin{macrocode}
}{%
%    \end{macrocode}
%    \begin{macro}{\ZREF@Robust}
%    \begin{macrocode}
  \def\ZREF@Robust#1{%
    \protected#1%
  }%
%    \end{macrocode}
%    \end{macro}
%    \begin{macrocode}
}
%    \end{macrocode}
%    \begin{macro}{\ZREF@IfDefinable}
%    \begin{macrocode}
\def\ZREF@IfDefinable#1#2#3{%
  \@ifdefinable{#1}{%
    \ZREF@Robust{#2}#1#3%
  }%
}
%    \end{macrocode}
%    \end{macro}
%
%    \begin{macro}{\ZREF@UpdatePdfTeX}
%    \cs{ZREF@UpdatePdfTeX} is used as help message text in
%    error messages.
%    \begin{macrocode}
\def\ZREF@UpdatePdfTeX{Update pdfTeX.}
%    \end{macrocode}
%    \end{macro}
%
%    \begin{macro}{\ifZREF@found}
%    The following switch is usded in list processing.
%    \begin{macrocode}
\newif\ifZREF@found
%    \end{macrocode}
%    \end{macro}
%
%    \begin{macro}{\ZREF@patch}
%    Macro \cs{ZREF@patch} first checks the existence of
%    the command and safes it.
%    \begin{macrocode}
\def\ZREF@patch#1{%
  \ltx@IfUndefined{#1}{%
    \ltx@gobble
  }{%
    \expandafter\let\csname ZREF@org@#1\expandafter\endcsname
    \csname #1\endcsname
    \ltx@firstofone
  }%
}
%    \end{macrocode}
%    \end{macro}
%
% \subsubsection{Check for \hologo{eTeX}}
%
%    The use of \hologo{eTeX} should be standard nowadays for \hologo{LaTeX}.
%    We test for \hologo{eTeX} in order to use its features later.
%    \begin{macrocode}
\ltx@IfUndefined{eTeXversion}{%
  \PackageError\ZREF@name{%
    Missing support for eTeX; package is abandoned%
  }{%
    Use a TeX compiler that support eTeX and enable eTeX %
    in the format.%
  }%
  \endinput
}{}%
%    \end{macrocode}
%
%    \begin{macrocode}
\RequirePackage{etexcmds}[2007/09/09]
\ifetex@unexpanded
\else
  \PackageError\ZREF@name{%
    Missing e-TeX's \string\unexpanded.\MessageBreak
    Add \string\RequirePackage\string{etexcmds\string} before %
    \string\documentclass%
  }{%
    Probably you are using some package (e.g. ConTeXt) that %
    redefines \string\unexpanded%
  }%
  \expandafter\endinput
\fi
%    \end{macrocode}
%
% \subsubsection{Auxiliary file stuff}
%
%    We are using some commands in the \xfile{.aux} files.
%    However sometimes these auxiliary files are interpreted
%    by \hologo{LaTeX} processes that haven't loaded this package
%    (e.g. package \xpackage{xr}).
%    Therefore we provide dummy definitions.
%    \begin{macrocode}
\RequirePackage{auxhook}
\AddLineBeginAux{%
  \string\providecommand\string\zref@newlabel[2]{}%
}
%    \end{macrocode}
%    \begin{macro}{\ZREF@RefPrefix}
%    \begin{macrocode}
\def\ZREF@RefPrefix{Z@R}
%    \end{macrocode}
%    \end{macro}
%    \begin{macro}{\zref@newlabel}
%    For the implementation of \cs{zref@newlabel} we call
%    the same internal macro \cs{@newl@bel} that is used
%    in \cs{newlabel}. Thus we have for free:
%    \begin{itemize}
%    \item \cs{Z@R@}\metacs{labelname} is defined.
%    \item \hologo{LaTeX}'s check for multiple references.
%    \item \hologo{LaTeX}'s check for changed references.
%    \end{itemize}
%    \begin{macrocode}
\ZREF@Robust\edef\zref@newlabel{%
  \noexpand\@newl@bel{\ZREF@RefPrefix}%
}
%    \end{macrocode}
%    \end{macro}
%
% \subsubsection{Property lists}
%
%    \begin{macro}{\zref@newlist}
%    Property lists are stored as list of property names
%    enclosed in curly braces. \cs{zref@newlist}
%    creates a new list as empty list. Assignments to
%    property lists are global.
%    \begin{macrocode}
\ZREF@Robust\def\zref@newlist#1{%
  \zref@iflistundefined{#1}{%
    \@ifdefinable{Z@L@#1}{%
     \global\expandafter\let\csname Z@L@#1\endcsname\ltx@empty
     \PackageInfo\ZREF@name{New property list: #1}%
    }%
  }{%
    \PackageError\ZREF@name{%
      Property list `#1' already exists%
    }\@ehc
  }%
}
%    \end{macrocode}
%    \end{macro}
%
%    \begin{macro}{\zref@iflistundefined}
%    \cs{zref@iflistundefined} checks the existence of the
%    property list \verb|#1|. If the property list is present,
%    then \verb|#2| is executed and \verb|#3| otherwise.
%    \begin{macrocode}
\def\zref@iflistundefined#1{%
  \ltx@ifundefined{Z@L@#1}%
}
%    \end{macrocode}
%    \end{macro}
%    \begin{macro}{\zref@listexists}
%    \cs{zref@listexists} only executes \verb|#2| if the
%    property list \verb|#1| exists and raises an error
%    message otherwise.
%    \begin{macrocode}
\ZREF@Robust\def\zref@listexists#1{%
  \zref@iflistundefined{#1}{%
    \PackageError\ZREF@name{%
      Property list `#1' does not exist%
    }\@ehc
  }%
}
%    \end{macrocode}
%    \end{macro}
%
%    \begin{macro}{\zref@iflistcontainsprop}
%    \cs{zref@iflistcontainsprop} checks, whether a property \verb|#2| is
%    already present in a property list \verb|#1|.
%    \begin{macrocode}
\ZREF@Robust\def\zref@iflistcontainsprop#1#2{%
  \zref@iflistundefined{#1}{%
    \ltx@secondoftwo
  }{%
    \begingroup\expandafter\endgroup
    \expandafter\in@
    \csname#2\expandafter\expandafter\expandafter\endcsname
    \expandafter\expandafter\expandafter{\csname Z@L@#1\endcsname}%
    \csname ltx@\ifin@ first\else second\fi oftwo\endcsname
  }%
}
%    \end{macrocode}
%    \end{macro}
%
%    \begin{macro}{\zref@listforloop}
%    \begin{macrocode}
\def\zref@listforloop#1#2{%
  \zref@listexists{#1}{%
    \expandafter\expandafter\expandafter\@tfor
    \expandafter\expandafter\expandafter\zref@prop
    \expandafter\expandafter\expandafter:%
    \expandafter\expandafter\expandafter=%
    \csname Z@L@#1\endcsname
    \do{%
      \begingroup
        \escapechar=-1 %
        \edef\x{\endgroup
          \def\noexpand\zref@prop{%
            \expandafter\string\zref@prop
          }%
        }%
      \x
      #2\zref@prop
    }%
  }%
}
%    \end{macrocode}
%    \end{macro}
%
%    \begin{macro}{\zref@addprops}
%    \cs{zref@addprop} adds the properties \verb|#2| to the property
%    list \verb|#1|, if the property is not already in the list.
%    Otherwise a warning is given.
%    \begin{macrocode}
\ZREF@Robust\def\zref@addprops#1#2{%
  \zref@listexists{#1}{%
    \comma@parse{#2}{%
      \zref@propexists\comma@entry{%
        \zref@iflistcontainsprop{#1}\comma@entry{%
          \PackageWarning\ZREF@name{%
            Property `\comma@entry' is already in list `#1'%
          }%
        }{%
          \begingroup\expandafter\endgroup
          \expandafter\g@addto@macro
          \csname Z@L@#1\expandafter\endcsname
          \expandafter{\csname\comma@entry\endcsname}%
        }%
      }%
      \ltx@gobble
    }%
  }%
}
%    \end{macrocode}
%    \end{macro}
%    \begin{macro}{\zref@addprop}
%    \cs{zref@addprop} adds the property \verb|#2| to the property
%    list \verb|#1|, if the property is not already in the list.
%    Otherwise a warning is given.
%    \begin{macrocode}
\ZREF@Robust\def\zref@addprop#1#2{%
  \zref@listexists{#1}{%
    \zref@propexists{#2}{%
      \zref@iflistcontainsprop{#1}{#2}{%
        \PackageWarning\ZREF@name{%
          Property `#2' is already in list `#1'%
        }%
      }{%
        \begingroup\expandafter\endgroup
        \expandafter\g@addto@macro
        \csname Z@L@#1\expandafter\endcsname
        \expandafter{\csname#2\endcsname}%
      }%
    }%
  }%
}
%    \end{macrocode}
%    \end{macro}
%
%    \begin{macro}{\zref@localaddprops}
%    \begin{macrocode}
\ZREF@Robust\def\zref@localaddprops#1#2{%
  \zref@listexists{#1}{%
    \comma@parse{#2}{%
      \zref@propexists\comma@entry{%
        \zref@iflistcontainsprop{#1}\comma@entry{%
          \PackageWarning\ZREF@name{%
            Property `\comma@entry' is already in list `#1'%
          }%
        }{%
          \begingroup\expandafter\endgroup
          \expandafter\ltx@LocalAppendToMacro
          \csname Z@L@#1\expandafter\endcsname
          \expandafter{\csname\comma@entry\endcsname}%
        }%
      }%
      \ltx@gobble
    }%
  }%
}
%    \end{macrocode}
%    \end{macro}
%    \begin{macro}{\zref@localaddprop}
%    \begin{macrocode}
\ZREF@Robust\def\zref@localaddprop#1#2{%
  \zref@listexists{#1}{%
    \zref@propexists{#2}{%
      \zref@iflistcontainsprop{#1}{#2}{%
        \PackageWarning\ZREF@name{%
          Property `#2' is already in list `#1'%
        }%
      }{%
        \begingroup\expandafter\endgroup
        \expandafter\ltx@LocalAppendToMacro
        \csname Z@L@#1\expandafter\endcsname
        \expandafter{\csname#2\endcsname}%
      }%
    }%
  }%
}
%    \end{macrocode}
%    \end{macro}
%
%    \begin{macrocode}
\ltx@IfUndefined{pdf@strcmp}{%
%    \end{macrocode}
%    \begin{macro}{\zref@delprop}
%    \begin{macrocode}
  \ZREF@Robust\def\zref@delprop{%
    \ZREF@delprop\gdef
  }%
%    \end{macrocode}
%    \end{macro}
%    \begin{macro}{\zref@localdelprop}
%    \begin{macrocode}
  \ZREF@Robust\def\zref@localdelprop{%
    \ZREF@delprop\def
  }%
%    \end{macrocode}
%    \end{macro}
%    \begin{macro}{\ZREF@delprop}
%    \begin{macrocode}
  \def\ZREF@delprop#1#2#3{%
    \zref@listexists{#2}{%
      \begingroup
        \escapechar=-1 %
        \def\ZREF@param{#3}%
        \@onelevel@sanitize\ZREF@param
        \toks@{}%
        \expandafter\expandafter\expandafter\ZREF@@delprop
        \csname Z@L@#2\endcsname!%
      \expandafter\endgroup
      \expandafter#1\csname Z@L@#2\expandafter\endcsname
      \expandafter{%
        \the\toks@
      }%
    }%
  }%
%    \end{macrocode}
%    \end{macro}
%    \begin{macro}{\ZREF@@delprop}
%    \begin{macrocode}
  \def\ZREF@@delprop#1{%
    \expandafter\ZREF@@@delprop\expandafter{\string#1}#1%
  }%
%    \end{macrocode}
%    \end{macro}
%    \begin{macro}{\ZREF@@@delprop}
%    \begin{macrocode}
  \def\ZREF@@@delprop#1#2{%
    \ifx#2!%
    \else
      \def\ZREF@temp{#1}%
      \@onelevel@sanitize\ZREF@temp
      \ifx\ZREF@param\ZREF@temp
      \else
        \toks@\expandafter{%
          \the\expandafter\toks@\csname#1\endcsname
        }%
      \fi
      \expandafter\ZREF@@delprop
    \fi
  }%
%    \end{macrocode}
%    \end{macro}
%    \begin{macrocode}
}{%
%    \end{macrocode}
%    \begin{macro}{\zref@delprop}
%    \begin{macrocode}
  \ZREF@Robust\def\zref@delprop{%
    \ZREF@delprop\xdef
  }%
%    \end{macrocode}
%    \end{macro}
%    \begin{macro}{\zref@localdelprop}
%    \begin{macrocode}
  \ZREF@Robust\def\zref@localdelprop{%
    \ZREF@delprop\edef
  }%
%    \end{macrocode}
%    \end{macro}
%    \begin{macro}{\ZREF@delprop}
%    \begin{macrocode}
  \def\ZREF@delprop#1#2#3{%
    \zref@listexists{#2}{%
      \def\ZREF@param{#3}%
      \edef\ZREF@SavedEscapechar{\the\escapechar}%
      \escapechar=-1 %
      \expandafter#1\csname Z@L@#2%
      \expandafter\expandafter\expandafter\endcsname{%
        \expandafter\expandafter\expandafter\ZREF@@delprop
        \csname Z@L@#2\endcsname!%
      }%
      \escapechar=\ZREF@SavedEscapechar\relax
    }%
  }%
%    \end{macrocode}
%    \end{macro}
%    \begin{macro}{\ZREF@@delprop}
%    Caution: |#1| might be an \cs{if} or similar token.
%    \begin{macrocode}
  \def\ZREF@@delprop#1{%
    \expandafter\ZREF@@@delprop\expandafter{\string#1}#1%
  }%
%    \end{macrocode}
%    \end{macro}
%    \begin{macro}{\ZREF@@@delprop}
%    \begin{macrocode}
  \def\ZREF@@@delprop#1#2{%
    \ifx#2!%
    \else
      \ifnum\pdf@strcmp{#1}{\ZREF@param}=\ltx@zero
      \else
        \expandafter\noexpand\csname#1\endcsname
      \fi
      \expandafter\ZREF@@delprop
    \fi
  }%
%    \end{macrocode}
%    \end{macro}
%    \begin{macrocode}
}
%    \end{macrocode}
%
% \subsubsection{Properties}
%
%    \begin{macro}{\zref@ifpropundefined}
%    \cs{zref@ifpropundefined} checks the existence of the
%    property \verb|#1|. If the property is present,
%    then \verb|#2| is executed and \verb|#3| otherwise.
%    \begin{macrocode}
\def\zref@ifpropundefined#1{%
  \ltx@ifundefined{Z@E@#1}%
}
%    \end{macrocode}
%    \end{macro}
%    \begin{macro}{\zref@propexists}
%    Some macros rely on the existence of a property.
%    \cs{zref@propexists} only executes \verb|#2| if the
%    property \verb|#1| exists and raises an error
%    message otherwise.
%    \begin{macrocode}
\ZREF@Robust\def\zref@propexists#1{%
  \zref@ifpropundefined{#1}{%
    \PackageError\ZREF@name{%
      Property `#1' does not exist%
    }\@ehc
  }%
}
%    \end{macrocode}
%    \end{macro}
%
%    \begin{macro}{\zref@newprop}
%    A new property is declared by \cs{zref@newprop}, the
%    property name \meta{propname} is given in \verb|#1|. The
%    property is created and configured. If the star form
%    is given, then the expansion of the property value
%    is delayed to page shipout time, when the reference
%    is written to the \xfile{.aux} file.
%    \begin{description}
%    \item[\cs{Z@D@}\metacs{propname}:]
%      Stores the default value for this property.
%    \item[\cs{Z@E@}\metacs{propname}:]
%      Extract function.
%    \item[\cs{Z@X@}\metacs{propname}:]
%      Information whether the expansion of the property value
%      is delayed to shipout time.
%    \item[\cs{Z@C@}\metacs{propname}:]
%      Current value of the property.
%    \end{description}
%    \begin{macrocode}
\ZREF@Robust\def\zref@newprop{%
  \@ifstar{%
    \let\ZREF@X\noexpand
    \ZREF@newprop
  }{%
    \let\ZREF@X\ltx@empty
    \ZREF@newprop
  }%
}
%    \end{macrocode}
%    \end{macro}
%    \begin{macro}{\ZREF@newprop}
%    \begin{macrocode}
\def\ZREF@newprop#1{%
  \edef\ZREF@P{#1}%
  \@onelevel@sanitize\ZREF@P
  \begingroup
  \ifx\ZREF@P\ZREF@par
    \@PackageError\ZREF@name{%
      Invalid property name `\ZREF@P'%
    }{%
      The property name `par' is not allowed %
      because of internal reasons.%
      \MessageBreak
      \@ehc
    }%
    \def\ZREF@@newprop[##1]##2{\endgroup}%
  \else
    \zref@ifpropundefined\ZREF@P{%
      \endgroup
      \PackageInfo\ZREF@name{%
        New property: \ZREF@P
      }%
    }{%
      \@PackageError\ZREF@name{%
        Property `\ZREF@P' already exists%
      }\@ehc
      \def\ZREF@@newprop[##1]##2{\endgroup}%
    }%
  \fi
  \@ifnextchar[\ZREF@@newprop{\ZREF@@newprop[\zref@default]}%
}
%    \end{macrocode}
%    \end{macro}
%    \begin{macro}{\ZREF@par}
%    \begin{macrocode}
\def\ZREF@par{par}
\@onelevel@sanitize\ZREF@par
%    \end{macrocode}
%    \end{macro}
%    \begin{macro}{\ZREF@@newprop}
%    \begin{macrocode}
\def\ZREF@@newprop[#1]{%
  \global\@namedef{Z@D@\ZREF@P}{#1}%
  \global\expandafter\let\csname Z@X@\ZREF@P\endcsname\ZREF@X
  \begingroup\expandafter\endgroup
  \expandafter\ZREF@@@newprop\csname\ZREF@P\endcsname
  \expandafter\gdef\csname Z@C@\ZREF@P\endcsname{}%
  \zref@setcurrent\ZREF@P
}
\def\ZREF@@@newprop#1{%
  \expandafter
  \gdef\csname Z@E@\ZREF@P\endcsname##1#1##2##3\ZREF@nil{##2}%
}
%    \end{macrocode}
%    \end{macro}
%
%    \begin{macro}{\zref@showprop}
%    \begin{macrocode}
\ZREF@Robust\def\zref@showprop#1{%
  \zref@ifpropundefined{#1}{%
    \@PackageInfoNoLine{\ZREF@name}{%
      Show property `#1': <undefined>%
    }%
  }{%
    \begingroup
      \toks@\expandafter\expandafter\expandafter{%
        \csname Z@C@#1\endcsname
      }%
      \edef\ZREF@value{\the\toks@}%
      \ltx@onelevel@sanitize\ZREF@value
      \toks@\expandafter\expandafter\expandafter{%
        \csname Z@D@#1\endcsname
      }%
      \edef\ZREF@default{\the\toks@}%
      \ltx@onelevel@sanitize\ZREF@default
      \@PackageInfoNoLine{\ZREF@name}{%
        Show property `#1':\MessageBreak
        \expandafter\ifx\csname Z@X@#1\endcsname\ltx@empty
          Immediate %
        \else
          Delayed %
        \fi
        value: [\ZREF@value]\MessageBreak
        Default: [\ZREF@default]%
      }%
    \endgroup
  }%
}
%    \end{macrocode}
%    \end{macro}
%
%    \begin{macro}{\zref@setcurrent}
%    \cs{zref@setcurrent} sets the current value for a property.
%    \begin{macrocode}
\ZREF@Robust\def\zref@setcurrent#1#2{%
  \zref@propexists{#1}{%
    \expandafter\def\csname Z@C@#1\endcsname{#2}%
  }%
}
%    \end{macrocode}
%    \end{macro}
%
%    \begin{macro}{\ZREF@getcurrent}
%    \cs{zref@getcurrent} gets the current value for a property.
%    \begin{macrocode}
\def\ZREF@getcurrent#1{%
  \romannumeral0%
  \ltx@ifundefined{Z@C@#1}{%
    \ltx@space
  }{%
    \expandafter\expandafter\expandafter\ltx@space
    \csname Z@C@#1\endcsname
  }%
}
%    \end{macrocode}
%    \end{macro}
%    \begin{macro}{\ZREF@u@getcurrent}
%    \begin{macrocode}
\def\ZREF@wu@getcurrent#1{%
  \etex@unexpanded\expandafter\expandafter\expandafter{%
    \ZREF@getcurrent{#1}%
  }%
}
%    \end{macrocode}
%    \end{macro}
%    \begin{macro}{\zref@getcurrent}
%    \begin{macrocode}
\let\zref@getcurrent\ZREF@getcurrent
%    \end{macrocode}
%    \end{macro}
%
% \subsubsection{Reference generation}
%
%    \begin{macro}{\zref@label}
%    Label macro that uses the main property list.
%    \begin{macrocode}
\ZREF@Robust\def\zref@label#1{%
  \zref@labelbylist{#1}\ZREF@mainlist
}
%    \end{macrocode}
%    \end{macro}
%    \begin{macro}{\zref@labelbylist}
%    Label macro that stores the properties, specified in the
%    property list \verb|#2|.
%    \begin{macrocode}
\ZREF@Robust\def\zref@labelbylist#1#2{%
  \@bsphack
    \zref@listexists{#2}{%
      \expandafter\expandafter\expandafter\ZREF@label
      \expandafter\expandafter\expandafter{%
        \csname Z@L@#2\endcsname
      }{#1}%
    }%
  \@esphack
}
%    \end{macrocode}
%    \end{macro}
%    \begin{macro}{\zref@labelbyprops}
%    The properties are directly specified in a comma separated list.
%    \begin{macrocode}
\ZREF@Robust\def\zref@labelbyprops#1#2{%
  \@bsphack
    \begingroup
      \toks@{}%
      \comma@parse{#2}{%
        \zref@ifpropundefined\comma@entry{%
          \PackageWarning\ZREF@name{%
            Property `\comma@entry' is not known%
          }%
        }{%
          \toks@\expandafter{%
            \the\expandafter\toks@\csname\comma@entry\endcsname
          }%
        }%
        \ltx@gobble
      }%
    \expandafter\endgroup
    \expandafter\ZREF@label\expandafter{\the\toks@}{#1}%
  \@esphack
}
%    \end{macrocode}
%    \end{macro}
%    \begin{macro}{\zref@labelbykv}
%    \begin{macrocode}
\ZREF@Robust\def\zref@labelbykv#1#2{%
  \@bsphack
    \begingroup
      \let\Z@L@ZREF@temp\ltx@empty
      \kvsetkeys{ZREF@LABEL}{#1}%
      \ifZREF@immediate
        \expandafter\zref@wrapper@immediate\expandafter{%
          \expandafter\ZREF@label\expandafter{\Z@L@ZREF@temp}{#2}%
        }%
      \else
        \expandafter\ZREF@label\expandafter{\Z@L@ZREF@temp}{#2}%
      \fi
    \endgroup
  \@esphack
}
%    \end{macrocode}
%    \end{macro}
%    \begin{macrocode}
\kv@define@key{ZREF@LABEL}{prop}{%
  \edef\ZREF@param{#1}%
  \zref@propexists\ZREF@param{%
    \zref@iflistcontainsprop{ZREF@temp}\ZREF@param{}{%
      \begingroup\expandafter\endgroup
      \expandafter\ltx@LocalAppendToMacro
      \expandafter\Z@L@ZREF@temp
      \expandafter{\csname\ZREF@param\endcsname}%
    }%
  }%
}
\kv@define@key{ZREF@LABEL}{list}{%
  \zref@listforloop{#1}{%
    \zref@iflistcontainsprop{ZREF@temp}\zref@prop{}{%
      \begingroup\expandafter\endgroup
      \expandafter\ltx@LocalAppendToMacro
      \expandafter\Z@L@ZREF@temp
      \expandafter{\csname\zref@prop\endcsname}%
    }%
    \ltx@gobble
  }%
}
\kv@define@key{ZREF@LABEL}{delprop}{%
  \zref@propexists{#1}{%
    \zref@localdelprop{ZREF@temp}{#1}%
  }%
}
\kv@define@key{ZREF@LABEL}{immediate}[true]{%
  \edef\ZREF@param{#1}%
  \ifx\ZREF@param\ZREF@true
    \ZREF@immediatetrue
  \else
    \ifx\ZREF@param\ZREF@false
      \ZREF@immediatefalse
    \else
      \PackageWarning\ZREF@name{%
        Option `immediate' expects `true' or `false'.\MessageBreak
        Ignoring invalid value `\ZREF@param'%
      }%
    \fi
  \fi
}
%    \end{macrocode}
%    \begin{macro}{\ZREF@false}
%    \begin{macrocode}
\def\ZREF@false{false}
%    \end{macrocode}
%    \end{macro}
%    \begin{macro}{\ZREF@true}
%    \begin{macrocode}
\def\ZREF@true{true}
%    \end{macrocode}
%    \end{macro}
%
%    \begin{macrocode}
\kv@define@key{ZREF@LABEL}{values}[]{%
  \kv@parse{#1}{%
    \ifx\kv@value\relax
      \@PackageWarning\ZREF@name{%
        Missing value for property `\kv@key'%
      }%
      \expandafter\ltx@gobbletwo
    \else
      \expandafter\zref@setcurrent
    \fi
  }%
}
%    \end{macrocode}
%
%    \begin{macro}{\ifZREF@immediate}
%    The switch \cs{ifZREF@immediate} tells us, whether the
%    label should be written immediately or at page shipout time.
%    \cs{ZREF@label} need to be notified about this, because
%    it must disable the deferred execution of property values,
%    if the label is written immediately.
%    \begin{macrocode}
\newif\ifZREF@immediate
%    \end{macrocode}
%    \end{macro}
%    \begin{macro}{\zref@wrapper@immediate}
%    The argument of \cs{zref@wrapper@immediate} is executed inside
%    a group where \cs{write} is redefined by adding
%    \cs{immediate} before its execution. Also \cs{ZREF@label}
%    is notified via the switch \cs{ifZREF@immediate}.
%    \begin{macrocode}
\ZREF@Robust{\long\def}\zref@wrapper@immediate#1{%
  \begingroup
    \ZREF@immediatetrue
    \let\ZREF@org@write\write
    \def\write{\immediate\ZREF@org@write}%
    #1%
  \endgroup
}
%    \end{macrocode}
%    \end{macro}
%
%    \begin{macro}{\ZREF@label}
%    \cs{ZREF@label} writes the data in the \xfile{.aux} file.
%    \verb|#1| contains the list of valid properties,
%    \verb|#2| the name of the reference. In case of immediate
%    writing, the deferred execution of property values is
%    disabled. Also \thepage is made expandable in this case.
%    \begin{macrocode}
\def\ZREF@label#1#2{%
  \if@filesw
    \begingroup
      \ifZREF@immediate
        \let\ZREF@org@thepage\thepage
      \fi
      \protected@write\@auxout{%
        \ifZREF@immediate
          \let\thepage\ZREF@org@thepage
        \fi
        \let\ZREF@temp\ltx@empty
        \@tfor\ZREF@P:=#1\do{%
          \begingroup
            \escapechar=-1 %
            \edef\x{\endgroup
              \def\noexpand\ZREF@P{%
                \expandafter\string\ZREF@P
              }%
            }%
          \x
          \expandafter\ifx
              \csname
                \ifZREF@immediate
                  relax%
                \else
                  Z@X@\ZREF@P%
                \fi
              \endcsname
              \noexpand
            \expandafter\let\csname Z@C@\ZREF@P\endcsname\relax
          \fi
          \toks@\expandafter{\ZREF@temp}%
          \edef\ZREF@temp{%
            \the\toks@
            \ltx@backslashchar\ZREF@P{%
              \expandafter\noexpand\csname Z@C@\ZREF@P\endcsname
            }%
          }%
        }%
      }{%
        \string\zref@newlabel{#2}{\ZREF@temp}%
      }%
    \endgroup
  \fi
}
\def\ZREF@addtoks#1{%
  \toks@\expandafter\expandafter\expandafter{%
    \expandafter\the\expandafter\toks@#1%
  }%
}
%    \end{macrocode}
%    \end{macro}
%
% \subsubsection{Reference querying and extracting}
%
%    Design goal for the extracting macros is that the extraction
%    process is full expandable. Thus these macros can be used
%    in expandable contexts. But there are problems that cannot
%    be solved by full expandable macros:
%    \begin{itemize}
%    \item In standard \hologo{LaTeX} undefined references sets a flag
%      and generate a warning. Both actions are not expandable.
%    \item Babel's support for its shorthand uses commands that
%      use non-expandable assignments. However currently there
%      is hope, that primitives are added to \pdfTeX\ that
%      allows the detection of contexts. Then the shorthand
%      can detect, if they are executed inside \cs{csname}
%      and protect themselves automatically.
%    \end{itemize}
%    \begin{macro}{\zref@ifrefundefined}
%    If a reference \verb|#1| is undefined, then macro
%    \cs{zref@ifrefundefined} calls \verb|#2| and \verb|#3| otherwise.
%    \begin{macrocode}
\def\zref@ifrefundefined#1{%
  \ltx@ifundefined{Z@R@#1}%
}
%    \end{macrocode}
%    \end{macro}
%
%    \begin{macro}{\zifrefundefined}
%    If a reference \verb|#1| is undefined, then macro
%    \cs{zref@ifrefundefined} calls \verb|#2| and \verb|#3| otherwise.
%    Also the reference is marked used.
%    \begin{macrocode}
\ZREF@IfDefinable\zifrefundefined\def{%
  #1{%
    \zref@wrapper@babel\ZREF@ifrefundefined{#1}%
  }%
}
%    \end{macrocode}
%    \end{macro}
%    \begin{macro}{\ZREF@ifrefundefined}
%    \begin{macrocode}
\def\ZREF@ifrefundefined#1{%
  \zref@refused{#1}%
  \zref@ifrefundefined{#1}%
}
%    \end{macrocode}
%    \end{macro}
%
%    \begin{macro}{\zref@refused}
%    The problem with undefined references is addressed by
%    the macro \cs{zref@refused}. This can be used outside
%    the expandable context. In case of an undefined reference
%    the flag is set to notify \hologo{LaTeX} and a warning is given.
%    \begin{macrocode}
\ZREF@Robust\def\zref@refused#1{%
  \zref@wrapper@babel\ZREF@refused{#1}%
}
%    \end{macrocode}
%    \end{macro}
%    \begin{macro}{\ZREF@refused}
%    \begin{macrocode}
\def\ZREF@refused#1{%
  \zref@ifrefundefined{#1}{%
    \protect\G@refundefinedtrue
    \@latex@warning{%
      Reference `#1' on page \thepage \space undefined%
    }%
  }{}%
}
%    \end{macrocode}
%    \end{macro}
%
%    \begin{macro}{\zref@ifrefcontainsprop}
%    \cs{zref@ifrefcontainsprop} looks, if the reference \verb|#1|
%    has the property \verb|#2| and calls then \verb|#3| and
%    \verb|#4| otherwise.
%    \begin{macrocode}
\def\zref@ifrefcontainsprop#1#2{%
  \zref@ifrefundefined{#1}{%
    \ltx@secondoftwo
  }{%
    \expandafter\ZREF@ifrefcontainsprop
    \csname Z@E@#2\expandafter\endcsname
    \csname#2\expandafter\expandafter\expandafter\endcsname
    \expandafter\expandafter\expandafter{%
      \csname Z@R@#1\endcsname
    }%
  }%
}
\def\ZREF@ifrefcontainsprop#1#2#3{%
  \expandafter\ifx\expandafter\ZREF@novalue
  #1#3#2\ZREF@novalue\ZREF@nil\ltx@empty
    \expandafter\ltx@secondoftwo
  \else
    \expandafter\ltx@firstoftwo
  \fi
}
\def\ZREF@novalue{\ZREF@NOVALUE}
%    \end{macrocode}
%    \end{macro}
%
%    \begin{macro}{\zref@extract}
%    \cs{zref@extract} is an abbreviation for the case
%    that the default of the property is used as default
%    value.
%    \begin{macrocode}
\def\ZREF@extract#1#2{%
  \romannumeral0%
  \ltx@ifundefined{Z@D@#2}{%
    \expandafter\ltx@space\zref@default
  }{%
    \expandafter\expandafter\expandafter\ZREF@@extract
    \expandafter\expandafter\expandafter{%
      \csname Z@D@#2\endcsname
    }{#1}{#2}%
  }%
}
%    \end{macrocode}
%    \end{macro}
%    \begin{macro}{\ZREF@@extract}
%    \begin{macrocode}
\def\ZREF@@extract#1#2#3{%
  \expandafter\expandafter\expandafter\ltx@space
  \zref@extractdefault{#2}{#3}{#1}%
}
%    \end{macrocode}
%    \end{macro}
%    \begin{macro}{\ZREF@wu@extract}
%    \begin{macrocode}
\def\ZREF@wu@extract#1#2{%
  \etex@unexpanded\expandafter\expandafter\expandafter{%
    \ZREF@extract{#1}{#2}%
  }%
}
%    \end{macrocode}
%    \end{macro}
%    \begin{macro}{\zref@extract}
%    \begin{macrocode}
\let\zref@extract\ZREF@extract
%    \end{macrocode}
%    \end{macro}
%
%    \begin{macro}{\ZREF@extractdefault}
%    The basic extracting macro is \cs{zref@extractdefault} with
%    the reference name in \verb|#1|, the property in \verb|#2| and
%    the default value in \verb|#3| in case for problems.
%    \begin{macrocode}
\def\ZREF@extractdefault#1#2#3{%
  \romannumeral0%
  \zref@ifrefundefined{#1}\ltx@firstoftwo{%
    \zref@ifpropundefined{#2}\ltx@firstoftwo\ltx@secondoftwo
  }{%
    \ltx@space
    #3%
  }{%
    \expandafter\expandafter\expandafter\ltx@space
    \csname Z@E@#2\expandafter\expandafter\expandafter\endcsname
    \csname Z@R@#1\expandafter\endcsname
    \csname#2\endcsname{#3}\ZREF@nil
  }%
}
%    \end{macrocode}
%    \end{macro}
%    \begin{macro}{\ZREF@wu@extractdefault}
%    \begin{macrocode}
\def\ZREF@wu@extractdefault#1#2#3{%
  \etex@unexpanded\expandafter\expandafter\expandafter{%
    \ZREF@extractdefault{#1}{#2}{#3}%
  }%
}
%    \end{macrocode}
%    \end{macro}
%    \begin{macro}{\zref@extractdefault}
%    \begin{macrocode}
\let\zref@extractdefault\ZREF@extractdefault
%    \end{macrocode}
%    \end{macro}
%
%    \begin{macro}{\zref@def@extract}
%    \begin{macrocode}
\ZREF@Robust\def\zref@def@extract#1{%
  \zref@wrapper@babel{\ZREF@def@extract{#1}}%
}
%    \end{macrocode}
%    \end{macro}
%    \begin{macro}{\ZREF@def@extract}
%    \begin{macrocode}
\def\ZREF@def@extract#1#2#3{%
  \zref@refused{#2}%
  \expandafter\expandafter\expandafter\def
  \expandafter\expandafter\expandafter#1%
  \expandafter\expandafter\expandafter{%
    \zref@extract{#2}{#3}%
  }%
}
%    \end{macrocode}
%    \end{macro}
%    \begin{macro}{\zref@def@extractdefault}
%    \begin{macrocode}
\ZREF@Robust\def\zref@def@extractdefault#1{%
  \zref@wrapper@babel{\ZREF@def@extractdefault{#1}}%
}
%    \end{macrocode}
%    \end{macro}
%    \begin{macro}{\ZREF@def@extractdefault}
%    \begin{macrocode}
\def\ZREF@def@extractdefault#1#2#3#4{%
  \zref@refused{#2}%
  \expandafter\expandafter\expandafter\def
  \expandafter\expandafter\expandafter#1%
  \expandafter\expandafter\expandafter{%
    \zref@extractdefault{#2}{#3}{#4}%
  }%
}
%    \end{macrocode}
%    \end{macro}
%
%    \begin{macro}{\ZREF@wrapper@unexpanded}
%    \begin{macrocode}
\ZREF@Robust{\long\def}\ZREF@wrapper@unexpanded#1{%
  \let\zref@wrapper@unexpanded\ltx@firstofone
  \let\zref@getcurrent\ZREF@wu@getcurrent
  \let\zref@extractdefault\ZREF@wu@extractdefault
  \let\zref@extract\ZREF@wu@extract
  #1%
  \let\zref@wrapper@unexpanded\ZREF@wrapper@unexpanded
  \let\zref@getcurrent\ZREF@getcurrent
  \let\zref@extractdefault\ZREF@extractdefault
  \let\zref@extract\ZREF@extract
}
%    \end{macrocode}
%    \end{macro}
%    \begin{macro}{\zref@wrapper@unexpanded}
%    \begin{macrocode}
\ltx@IfUndefined{etex@unexpanded}{%
  \let\zref@wrapper@unexpanded\ltx@firstofone
}{%
  \let\zref@wrapper@unexpanded\ZREF@wrapper@unexpanded
}
%    \end{macrocode}
%    \end{macro}
%
% \subsubsection{Compatibility with \xpackage{babel}}
%
%    \begin{macro}{\zref@wrapper@babel}
%    \begin{macrocode}
\ZREF@Robust{\long\def}\zref@wrapper@babel#1#2{%
  \ifcsname if@safe@actives\endcsname
    \expandafter\ltx@firstofone
  \else
    \expandafter\ltx@secondoftwo
  \fi
  {%
    \if@safe@actives
      \expandafter\ltx@secondoftwo
    \else
      \expandafter\ltx@firstoftwo
    \fi
    {%
      \begingroup
        \csname @safe@activestrue\endcsname
        \edef\x{#2}%
      \expandafter\endgroup
      \expandafter\ZREF@wrapper@babel\expandafter{\x}{#1}%
    }%
  }{%
    #1{#2}%
  }%
}
\long\def\ZREF@wrapper@babel#1#2{%
  #2{#1}%
}
%    \end{macrocode}
%    \end{macro}
%
% \subsubsection{Unique counter support}
%
%    \begin{macro}{\zref@require@unique}
%    Generate the counter |zref@unique| if the counter does not
%    already exist.
%    \begin{macrocode}
\ZREF@Robust\def\zref@require@unique{%
  \@ifundefined{c@zref@unique}{%
    \begingroup
      \let\@addtoreset\ltx@gobbletwo
      \newcounter{zref@unique}%
    \endgroup
%    \end{macrocode}
%    \end{macro}
%
%    \begin{macro}{\thezref@unique}
%    \cs{thezref@unique} is used for automatically generated unique
%    labelnames.
%    \begin{macrocode}
    \renewcommand*{\thezref@unique}{%
      zref@\number\c@zref@unique
    }%
  }{}%
}
%    \end{macrocode}
%    \end{macro}
%
% \subsubsection{Utilities}
%
%    \begin{macro}{\ZREF@number}
%    \begin{macrocode}
\ltx@IfUndefined{numexpr}{%
  \def\ZREF@number#1{\number#1}%
}{%
  \def\ZREF@number#1{\the\numexpr(#1)\relax}%
}
%    \end{macrocode}
%    \end{macro}
%
% \subsubsection{Setup}
%
%    \begin{macro}{\zref@setdefault}
%    Standard \hologo{LaTeX} prints ``??'' in bold face if a reference
%    is not known. \cs{zref@default} holds the text that is
%    printed in case of unknown references and is used, if
%    the default was not specified during the definition
%    of the new property by \cs{ref@newprop}.
%    The global default value can be set by \cs{zref@setdefault}.
%    \begin{macrocode}
\ZREF@Robust\def\zref@setdefault#1{%
  \def\zref@default{#1}%
}
%    \end{macrocode}
%    \end{macro}
%    \begin{macro}{\zref@default}
%    Now we initialize \cs{zref@default} with the same value
%    that \hologo{LaTeX} uses for its undefined references.
%    \begin{macrocode}
\zref@setdefault{%
  \nfss@text{\reset@font\bfseries ??}%
}
%    \end{macrocode}
%    \end{macro}
%
% \paragraph{Main property list.}
%
%    \begin{macro}{\zref@setmainlist}
%    The name of the default property list is stored in
%    \cs{ZREF@mainlist} and can be set by \cs{zref@setmainlist}.
%    \begin{macrocode}
\ZREF@Robust\def\zref@setmainlist#1{%
  \def\ZREF@mainlist{#1}%
}
\zref@setmainlist{main}
%    \end{macrocode}
%    \end{macro}
%    Now we create the list.
%    \begin{macrocode}
\zref@newlist\ZREF@mainlist
%    \end{macrocode}
%
% \paragraph{Main properties.}
%
%    The two properties \texttt{default} and \texttt{page}
%    are created and added to the main property list.
%    They store the data that
%    standard \hologo{LaTeX} uses in its references created by
%    \cs{label}.
%    \begin{description}
%    \item[\texttt{default}] the apperance of the latest
%      counter that is incremented by \cs{refstepcounter}
%    \item[\texttt{page}] the apperance of the page counter
%    \end{description}
%    \begin{macrocode}
\zref@newprop{default}{\@currentlabel}
\zref@newprop*{page}{\thepage}
\zref@addprops\ZREF@mainlist{default,page}
%    \end{macrocode}
%
% \paragraph{Properties}
%
%    \begin{macro}{\ZREF@NewPropAnchor}
%    \begin{macrocode}
\def\ZREF@NewPropAnchor{%
  \zref@newprop{anchor}{%
    \ltx@ifundefined{@currentHref}{}{\@currentHref}%
  }%
  \global\let\ZREF@NewPropAnchor\relax
}
%    \end{macrocode}
%    \end{macro}
%    \begin{macro}{\zref@titleref@current}
%    Later we will redefine the section and caption macros
%    to catch the current title and remember the value in
%    \cs{zref@titleref@current}.
%    \begin{macro}{\ZREF@NewPropTitle}
%    \begin{macrocode}
\def\ZREF@NewPropTitle{%
  \gdef\zref@titleref@current{}%
  \zref@newprop{title}{\zref@titleref@current}%
  \global\let\ZREF@NewPropTitle\relax
}
%    \end{macrocode}
%    \end{macro}
%    \end{macro}
%    \begin{macro}{\ZREF@NewPropTheotype}
%    \begin{macrocode}
\def\ZREF@NewPropTheotype{%
  \zref@newprop{theotype}{}%
  \global\let\ZREF@NewPropTheotype\relax
}
%    \end{macrocode}
%    \end{macro}
%    \begin{macro}{\ZREF@NewPropPageValue}
%    \begin{macrocode}
\def\ZREF@NewPropPageValue{%
  \zref@newprop*{pagevalue}[0]{\number\c@page}%
  \global\let\ZREF@NewPropPageValue\relax
}
%    \end{macrocode}
%    \end{macro}
%
% \paragraph{Mark successful loading}
%
%    \begin{macrocode}
\let\ZREF@base@ok=Y
%</base>
%    \end{macrocode}
%
%
% \subsection{Module \xmodule{user}}
%
%    \begin{macrocode}
%<*user>
\NeedsTeXFormat{LaTeX2e}
\ProvidesPackage{zref-user}%
  [2020-07-03 v2.32 Module user for zref (HO)]%
\RequirePackage{zref-base}[2019/11/29]
\ifx\ZREF@base@ok Y%
\else
  \expandafter\endinput
\fi
%    \end{macrocode}
%
%    Module \xmodule{user} enables a small user interface.
%    All macros are prefixed by \cs{z}.
%
%    First we define the pendants to the standard \hologo{LaTeX} referencing
%    commands \cs{label}, \cs{ref}, and \cs{pageref}.
%
%    \begin{macro}{\zlabel}
%    Similar to \cs{label} the macro \cs{zlabel} writes a reference
%    entry in the \xfile{.aux} file. The main property list is used.
%    Also we add the babel patch. The \cs{label} command can also be
%    used inside section titles, but it must not go into the
%    table of contents. Therefore we have to check this situation.
%    \begin{macrocode}
\newcommand*\zlabel{%
  \ifx\label\ltx@gobble
    \expandafter\ltx@gobble
  \else
    \expandafter\zref@wrapper@babel\expandafter\zref@label
  \fi
}%
%    \end{macrocode}
%    \end{macro}
%
%    \begin{macro}{\zkvlabel}
%    \begin{macrocode}
\newcommand*{\zkvlabel}[1]{%
  \ifx\label\ltx@gobble
    \expandafter\ltx@gobblethree
  \fi
  \zref@wrapper@babel{\zref@labelbykv{#1}}%
}%
%    \end{macrocode}
%    \end{macro}
%
%    \begin{macro}{\zref}
%    Macro \cs{zref} is the corresponding macro for \cs{ref}.
%    Also it provides an optional argument in order to
%    select another property.
%    \begin{macrocode}
\newcommand*{\zref}[2][default]{% robust because of optional argument
  \zref@propexists{#1}{%
    \zref@wrapper@babel\ZREF@zref{#2}{#1}%
  }%
}%
\def\ZREF@zref#1{%
  \zref@refused{#1}%
  \zref@extract{#1}%
}%
%    \end{macrocode}
%    \end{macro}
%
%    \begin{macro}{\zpageref}
%    For macro \cs{zpageref} we just call \cs{zref}
%    with property \verb|page|.
%    \begin{macrocode}
\ZREF@IfDefinable\zpageref\def{%
  {\zref[page]}%
}
%    \end{macrocode}
%    \end{macro}
%
%    \begin{macro}{\zrefused}
%    For the following expandible user macros \cs{zrefused}
%    should be used to notify \hologo{LaTeX} in case of undefined
%    references.
%    \begin{macrocode}
\ZREF@IfDefinable\zrefused\def{%
  {\zref@refused}%
}
%    \end{macrocode}
%    \end{macro}
%
%    \begin{macrocode}
%</user>
%    \end{macrocode}
%
%
% \subsection{Module \xmodule{abspage}}
%
%    \begin{macrocode}
%<*abspage>
\NeedsTeXFormat{LaTeX2e}
\ProvidesPackage{zref-abspage}%
  [2020-07-03 v2.32 Module abspage for zref (HO)]%
\RequirePackage{zref-base}[2019/11/29]
\ifx\ZREF@base@ok Y%
\else
  \expandafter\endinput
\fi
%    \end{macrocode}
%
%    Module \xmodule{abspage} adds a new property \verb|abspage|
%    to the \verb|main| property list for absolute page numbers.
%    These are recorded by the help of package \xpackage{atbegshi}.
%    \begin{macrocode}
\RequirePackage{atbegshi}[2011/10/05]%
%    \end{macrocode}
%    The counter \verb|abspage| must not go in the clear list
%    of \verb|@ckpt| that is used to set counters in
%    \xfile{.aux} files of included \hologo{TeX} files.
%    \begin{macrocode}
\begingroup
  \let\@addtoreset\ltx@gobbletwo
  \newcounter{abspage}%
\endgroup
\setcounter{abspage}{0}%
\AtBeginShipout{%
  \stepcounter{abspage}%
}%
\zref@newprop*{abspage}[0]{\the\c@abspage}%
\zref@addprop\ZREF@mainlist{abspage}%
%    \end{macrocode}
%    Note that counter \verb|abspage| shows the
%    previous page during page processing. Before
%    shipout the counter is incremented.
%    Thus the property is correctly written with
%    deferred writing. If the counter is written
%    using \cs{zref@wrapper@immediate}, then the number
%    is too small by one.
%
%    \begin{macrocode}
%</abspage>
%    \end{macrocode}
%
%
% \subsection{Module \xmodule{counter}}
%
%    \begin{macrocode}
%<*counter>
\NeedsTeXFormat{LaTeX2e}
\ProvidesPackage{zref-counter}%
  [2020-07-03 v2.32 Module counter for zref (HO)]%
\RequirePackage{zref-base}[2019/11/29]
\ifx\ZREF@base@ok Y%
\else
  \expandafter\endinput
\fi
%    \end{macrocode}
%
%    For features such as \xpackage{hyperref}'s \cs{autoref}
%    we need the name of the counter. The property
%    \verb|counter| is defined and added to the main property
%    list. Starting with \LaTeX\ 2020-10-01 the proper can use
%    \@currentcounter. In older formats
%    \cs{refstepcounter} has to be patched but this can fail in some
%    cases, see issue \#5.
%    \begin{macrocode}
\@ifl@t@r\fmtversion{2020-10-01}
 {
  \zref@newprop{counter}{\@currentcounter}
  \zref@addprop\ZREF@mainlist{counter}
 }
 {
  \zref@newprop{counter}{}
  \zref@addprop\ZREF@mainlist{counter}
  \AtBeginDocument{%
    \ZREF@patch{refstepcounter}{%
      \def\refstepcounter#1{%
        \zref@setcurrent{counter}{#1}%
        \ZREF@org@refstepcounter{#1}%
      }%
    }%
  }
 }
%    \end{macrocode}
%    \begin{macrocode}
%</counter>
%    \end{macrocode}
%
%
% \subsection{Module \xmodule{lastpage}}
%
%    \begin{macrocode}
%<*lastpage>
\NeedsTeXFormat{LaTeX2e}
\ProvidesPackage{zref-lastpage}%
  [2020-07-03 v2.32 Module lastpage for zref (HO)]%
\RequirePackage{zref-base}[2019/11/29]
\RequirePackage{zref-abspage}[2019/11/29]
\RequirePackage{atveryend}[2009/12/07]
\ifx\ZREF@base@ok Y%
\else
  \expandafter\endinput
\fi
%    \end{macrocode}
%
%    The module \xmodule{lastpage} implements the service
%    of package \xpackage{lastpage} by setting a reference
%    \verb|LastPage| at the end of the document.
%    If module \xmodule{abspage} is given, also the absolute
%    page number is available, because the properties of
%    the main property list are used.
%    \begin{macrocode}
\zref@newlist{LastPage}
\AfterLastShipout{%
  \if@filesw
    \begingroup
      \advance\c@page\m@ne
      \toks@\expandafter\expandafter\expandafter{%
        \expandafter\Z@L@main
        \Z@L@LastPage
      }%
      \expandafter\zref@wrapper@immediate\expandafter{%
        \expandafter\ZREF@label\expandafter{\the\toks@}{LastPage}%
      }%
    \endgroup
  \fi
}
%    \end{macrocode}
%    \begin{macro}{\zref@iflastpage}
%    \begin{macrocode}
\def\zref@iflastpage#1{%
  \ifnum\zref@extractdefault{#1}{abspage}{-1}=%
        \zref@extractdefault{LastPage}{abspage}{-2} %
    \expandafter\ltx@firstoftwo
  \else
    \expandafter\ltx@secondoftwo
  \fi
}
%    \end{macrocode}
%    \end{macro}
%    \begin{macro}{\ziflastpage}
%    \begin{macrocode}
\ZREF@IfDefinable\ziflastpage\def{%
  {\zref@wrapper@babel\ZREF@iflastpage}%
}
%    \end{macrocode}
%    \end{macro}
%    \begin{macro}{ZREF@iflastpage}
%    \begin{macrocode}
\def\ZREF@iflastpage#1{%
  \zref@refused{LastPage}%
  \zref@refused{#1}%
  \zref@iflastpage{#1}%
}
%    \end{macrocode}
%    \end{macro}
%    \begin{macrocode}
%</lastpage>
%    \end{macrocode}
%
%
% \subsection{Module \xmodule{thepage}}
%
%    \begin{macrocode}
%<*thepage>
\NeedsTeXFormat{LaTeX2e}
\ProvidesPackage{zref-thepage}%
  [2020-07-03 v2.32 Module thepage for zref (HO)]%
\RequirePackage{zref-base}[2019/11/29]
\ifx\ZREF@base@ok Y%
\else
  \expandafter\endinput
\fi
%    \end{macrocode}
%
%    \begin{macrocode}
\RequirePackage{atbegshi}[2011/10/05]
\RequirePackage{zref-abspage}[2019/11/29]
%    \end{macrocode}
%    \begin{macrocode}
\zref@newlist{thepage}
\zref@addprop{thepage}{page}
\ZREF@NewPropPageValue
%    \end{macrocode}
%    \begin{macro}{\zref@thepage@atbegshi@hook}
%    \begin{macrocode}
\let\zref@thepage@atbegshi@hook\ltx@empty
%    \end{macrocode}
%    \end{macro}
%    \begin{macrocode}
\zref@addprop{thepage}{pagevalue}
\AtBeginShipout{%
  \AtBeginShipoutAddToBox{%
    \zref@thepage@atbegshi@hook
    \zref@labelbylist{thepage\the\value{abspage}}{thepage}%
  }%
}
%    \end{macrocode}
%    \begin{macro}{\zref@thepage@name}
%    \begin{macrocode}
\ltx@IfUndefined{numexpr}{%
  \def\zref@thepage@name#1{thepage\number#1}%
}{%
  \def\zref@thepage@name#1{thepage\the\numexpr#1}%
}
%    \end{macrocode}
%    \end{macro}
%    \begin{macro}{\zref@thepage}
%    \begin{macrocode}
\def\zref@thepage#1{%
  \zref@extract{\zref@thepage@name{#1}}{page}%
}%
%    \end{macrocode}
%    \end{macro}
%    \begin{macro}{\zref@thepage@refused}
%    \begin{macrocode}
\ZREF@Robust\def\zref@thepage@refused#1{%
  \zref@refused{\zref@thepage@name{#1}}%
}%
%    \end{macrocode}
%    \end{macro}
%    \begin{macro}{\zthepage}
%    \begin{macrocode}
\ZREF@IfDefinable\zthepage\def{%
  #1{%
    \zref@thepage@refused{#1}%
    \zref@thepage{#1}%
  }%
}
%    \end{macrocode}
%    \end{macro}
%    \begin{macrocode}
%</thepage>
%    \end{macrocode}
%
%
% \subsection{Module \xmodule{nextpage}}
%
%    \begin{macrocode}
%<*nextpage>
\NeedsTeXFormat{LaTeX2e}
\ProvidesPackage{zref-nextpage}%
  [2020-07-03 v2.32 Module nextpage for zref (HO)]%
\RequirePackage{zref-base}[2019/11/29]
\ifx\ZREF@base@ok Y%
\else
  \expandafter\endinput
\fi
%    \end{macrocode}
%
%    \begin{macrocode}
\RequirePackage{zref-abspage}[2019/11/29]
\RequirePackage{zref-thepage}[2019/11/29]
\RequirePackage{zref-lastpage}[2019/11/29]
\RequirePackage{uniquecounter}[2009/12/18]
%    \end{macrocode}
%    \begin{macrocode}
\UniqueCounterNew{znextpage}

\newcommand*{\znextpagesetup}{%
  \afterassignment\ZREF@np@setup@i
  \def\ZREF@np@call@unknown##1%
}
\def\ZREF@np@setup@i{%
  \afterassignment\ZREF@np@setup@ii
  \def\ZREF@np@call@nonext##1%
}
\def\ZREF@np@setup@ii{%
  \def\ZREF@np@call@next##1%
}
\def\ZREF@np@call@unknown#1{#1}
\def\ZREF@np@call@nonext#1{#1}
\def\ZREF@np@call@next#1{#1}
\ZREF@IfDefinable\znextpage\def{%
  {\UniqueCounterCall{znextpage}{\ZREF@nextpage}}%
}%
\newcommand*{\znonextpagename}{}
\newcommand*{\zunknownnextpagename}{\Z@D@page}
\def\ZREF@nextpage#1{%
  \begingroup
    \def\ZREF@refname@this{zref@np#1}%
    \zref@labelbyprops\ZREF@refname@this{abspage}%
    \chardef\ZREF@call=0 % unknown
    \ZREF@ifrefundefined\ZREF@refname@this{%
    }{%
      \edef\ZREF@pagenum@this{%
        \zref@extractdefault\ZREF@refname@this{abspage}{0}%
      }%
      \edef\ZREF@refname@next{%
        \zref@thepage@name{%
          \the\numexpr\ZREF@pagenum@this+1%
        }%
      }%
      \ifnum\ZREF@pagenum@this>0 %
        \ZREF@ifrefundefined{LastPage}{%
          \zref@ifrefundefined\ZREF@refname@next{%
          }{%
            \chardef\ZREF@call=2 % next page
          }%
        }{%
          \edef\ZREF@pagenum@last{%
            \zref@extractdefault{LastPage}{abspage}{0}%
          }%
          \ifnum\ZREF@pagenum@this<\ZREF@pagenum@last\ltx@space
            \ZREF@ifrefundefined\ZREF@refname@next{%
            }{%
              \chardef\ZREF@call=2 % next page
            }%
          \else
            \ifnum\ZREF@pagenum@this=\ZREF@pagenum@this\ltx@space
              \chardef\ZREF@call=1 % no next page
            \fi
          \fi
        }%
      \fi
    }%
    \edef\x{%
      \endgroup
      \ifcase\ZREF@call
        \noexpand\ZREF@np@call@unknown{%
          \noexpand\zunknownnextpagename
        }%
      \or
        \noexpand\ZREF@np@call@nonext{%
          \noexpand\znonextpagename
        }%
      \else
        \noexpand\ZREF@np@call@next{%
          \noexpand\zref@extract{\ZREF@refname@next}{page}%
        }%
      \fi
    }%
  \x
}
%    \end{macrocode}
%    \begin{macrocode}
%</nextpage>
%    \end{macrocode}
%
%
% \subsection{Module \xmodule{totpages}}
%
%    \begin{macrocode}
%<*totpages>
\NeedsTeXFormat{LaTeX2e}
\ProvidesPackage{zref-totpages}%
  [2020-07-03 v2.32 Module totpages for zref (HO)]%
\RequirePackage{zref-base}[2019/11/29]
\ifx\ZREF@base@ok Y%
\else
  \expandafter\endinput
\fi
%    \end{macrocode}
%
%    The absolute page number of the last page is the total
%    page number.
%    \begin{macrocode}
\RequirePackage{zref-abspage}[2019/11/29]
\RequirePackage{zref-lastpage}[2019/11/29]
%    \end{macrocode}
%
%    \begin{macro}{\ztotpages}
%    Macro \cs{ztotpages} contains the number of pages. It can be
%    used inside expandable calculations. It expands to zero if
%    the reference is not yet available.
%    \begin{macrocode}
\newcommand*{\ztotpages}{%
  \zref@extractdefault{LastPage}{abspage}{0}%
}
%    \end{macrocode}
%    \end{macro}
%    Also we mark the reference |LastPage| as used:
%    \begin{macrocode}
\AtBeginDocument{%
  \zref@refused{LastPage}%
}
%    \end{macrocode}
%    \begin{macrocode}
%</totpages>
%    \end{macrocode}
%
%
% \subsection{Module \xmodule{pagelayout}}
%
%    \begin{macrocode}
%<*pagelayout>
\NeedsTeXFormat{LaTeX2e}
\ProvidesPackage{zref-pagelayout}%
  [2020-07-03 v2.32 Module pagelayout for zref (HO)]%
\RequirePackage{zref-base}[2019/11/29]
\ifx\ZREF@base@ok Y%
\else
  \expandafter\endinput
\fi
%    \end{macrocode}
%
%    \begin{macrocode}
\RequirePackage{zref-thepage}[2019/11/29]
\RequirePackage{iftex}[2019/11/07]%
\RequirePackage{atveryend}[2010/03/24]
%    \end{macrocode}
%
%
% \subsubsection{Define layout properties}
%
%    \begin{macrocode}
\def\ZREF@temp#1{%
  \begingroup
    \escapechar=-1 %
  \ltx@ifundefined{\string#1}{\endgroup}{%
    \edef\x{%
      \endgroup
      \noexpand\zref@newprop*{\string#1}%
                             [\noexpand\number\noexpand#1]% hash-ok
                             {\noexpand\number\noexpand#1}%
      \noexpand\zref@addprop{thepage}{\string#1}%
    }%
    \x
  }%
}
\ZREF@temp\mag
\ZREF@temp\paperwidth
\ZREF@temp\paperheight
\ZREF@temp\stockwidth % memoir.cls, crop.sty
\ZREF@temp\stockheight % memoir.cls, crop.sty
\ZREF@temp\mediawidth % VTeX
\ZREF@temp\mediaheight % VTeX
\ifluatex
\zref@newprop*{pdfvorigin}%
                             [\number\pdfvariable vorigin]% hash-ok
                             {\number\pdfvariable vorigin}%
\zref@addprop{thepage}{pdfvorigin}
\zref@newprop*{pdfhorigin}%
                             [\number\pdfvariable horigin]% hash-ok
                             {\number\pdfvariable horigin}%
\zref@addprop{thepage}{pdfhorigin}
\zref@newprop*{pdfpageheight}%
                             [\number\pageheight]% hash-ok
                             {\number\pageheight}%
\zref@addprop{thepage}{pdfpageheight}
\zref@newprop*{pdfpagewidth}%
                             [\number\pagewidth]% hash-ok
                             {\number\pagewidth}%
\zref@addprop{thepage}{pdfpagewidth}
\else
\ZREF@temp\pdfpagewidth
\ZREF@temp\pdfpageheight
\ZREF@temp\pdfhorigin
\ZREF@temp\pdfvorigin
\fi
\ZREF@temp\hoffset
\ZREF@temp\voffset
\ZREF@temp\topmargin
\ZREF@temp\oddsidemargin
\ZREF@temp\evensidemargin
\ZREF@temp\textwidth
\ZREF@temp\textheight
\ZREF@temp\headheight
\ZREF@temp\headsep
\ZREF@temp\footskip
\ZREF@temp\marginparwidth
\ZREF@temp\marginparsep
\ZREF@temp\columnwidth
\ZREF@temp\columnsep
\ZREF@temp\trimedge % memoir.cls
\ZREF@temp\spinemargin % memoir.cls
\ZREF@temp\foremargin % memoir.cls
\ZREF@temp\trimtop % memoir.cls
\ZREF@temp\uppermargin % memoir.cls
\ZREF@temp\headmargin % memoir.cls
%    \end{macrocode}
%
%    \begin{macrocode}
\IfFormatAtLeastTF{2020/10/01}
 {
   \zref@newprop*{outputboxwd}[0pt]{\ShipoutBoxWidth}
   \zref@newprop*{outputboxht}[0pt]{\ShipoutBoxHeight}
   \zref@newprop*{outputboxdp}[0pt]{\ShipoutBoxDepth}
 }
 {
   \zref@newprop*{outputboxwd}[0pt]{\AtBeginShipoutBoxWidth}
   \zref@newprop*{outputboxht}[0pt]{\AtBeginShipoutBoxHeight}
   \zref@newprop*{outputboxdp}[0pt]{\AtBeginShipoutBoxDepth}
 }
%    \end{macrocode}
%    \begin{macrocode}
\zref@addprops{thepage}{outputboxwd,outputboxht,outputboxdp}
%    \end{macrocode}
%
%    \begin{macro}{\ifZREF@pl@list}
%    \begin{macrocode}
\ltx@newif\ifZREF@pl@list
%    \end{macrocode}
%    \end{macro}
%    \begin{macro}{\zref@listpagelayout}
%    \begin{macrocode}
\ZREF@IfDefinable\zlistpagelayout\def{%
  {\global\ZREF@pl@listtrue}%
}
%    \end{macrocode}
%    \end{macro}
%    \begin{macro}{\ZREF@pl@AfterLastShipout}
%    \begin{macrocode}
\def\ZREF@pl@AfterLastShipout{%
  \ifZREF@pl@list
    \edef\ZREF@page@max{\the\value{abspage}}%
    \ltx@ifundefined{ZREF@org@testdef}{%
      \let\ZREF@org@testdef\@testdef
      \def\@testdef##1##2##3{%
        \ZREF@org@testdef{##1}{##2}{##3}%
        \def\ZREF@temp{##1}%
        \ifx\ZREF@temp\ZREF@RefPrefix
          \expandafter\gdef\csname##1@##2\endcsname{##3}%
        \fi
      }%
    }{}%
    \AtVeryEndDocument{\ZREF@pl@AtVeryEnd}%
  \fi
}
%    \end{macrocode}
%    \end{macro}
%    \begin{macro}{\ZREF@pl@AtVeryEnd}
%    \begin{macrocode}
\def\ZREF@pl@AtVeryEnd{%
  \begingroup
    \toks@{Page layout parameters:\MessageBreak}%
    \count@=1 %
    \ZREF@pl@ListPage
    \edef\x{\endgroup
      \noexpand\@PackageInfoNoLine{zref-pagelayout}{\the\toks@}%
    }%
  \x
}
%    \end{macrocode}
%    \end{macro}
%    \begin{macro}{\ZREF@pl@ListPage}
%    \begin{macrocode}
\def\ZREF@pl@ListPage{%
  \edef\x{%
    \toks@={%
      \the\toks@
      Page \the\count@:\noexpand\MessageBreak
      \zref@ifrefundefined{thepage\the\count@}{}{%
        \ltx@space\ltx@space mag = %
        \zref@extract{thepage\the\count@}{mag}%
        \noexpand\MessageBreak
        \ZREF@pl@ListEntry{paperwidth}%
        \ZREF@pl@ListEntry{paperheight}%
        \ZREF@pl@ListEntry{stockwidth}%
        \ZREF@pl@ListEntry{stockheight}%
        \ZREF@pl@ListEntry{mediawidth}%
        \ZREF@pl@ListEntry{mediaheight}%
        \ZREF@pl@ListEntry{pdfpagewidth}%
        \ZREF@pl@ListEntry{pdfpageheight}%
        \ZREF@pl@ListEntry{pdfhorigin}%
        \ZREF@pl@ListEntry{pdfvorigin}%
        \ZREF@pl@ListEntry{hoffset}%
        \ZREF@pl@ListEntry{voffset}%
        \ZREF@pl@ListEntry{topmargin}%
        \ZREF@pl@ListEntry{oddsidemargin}%
        \ZREF@pl@ListEntry{evensidemargin}%
        \ZREF@pl@ListEntry{textwidth}%
        \ZREF@pl@ListEntry{textheight}%
        \ZREF@pl@ListEntry{headheight}%
        \ZREF@pl@ListEntry{headsep}%
        \ZREF@pl@ListEntry{footskip}%
        \ZREF@pl@ListEntry{marginparwidth}%
        \ZREF@pl@ListEntry{marginparsep}%
        \ZREF@pl@ListEntry{columnwidth}%
        \ZREF@pl@ListEntry{columnsep}%
        \ZREF@pl@ListEntry{trimedge}%
        \ZREF@pl@ListEntry{spinemargin}%
        \ZREF@pl@ListEntry{foremargin}%
        \ZREF@pl@ListEntry{trimtop}%
        \ZREF@pl@ListEntry{uppermargin}%
        \ZREF@pl@ListEntry{headmargin}%
      }%
    }%
  }\x
  \ifnum\ZREF@page@max>\count@
    \advance\count@ by\ltx@one
  \else
    \expandafter\ltx@gobble
  \fi
  \ZREF@pl@ListPage
}
%    \end{macrocode}
%    \end{macro}
%    \begin{macro}{\ZREF@pl@ListEntry}
%    \begin{macrocode}
\def\ZREF@pl@ListEntry#1{%
  \zref@ifpropundefined{#1}{%
  }{%
    \zref@ifrefcontainsprop{thepage\the\count@}{#1}{%
      \ltx@space\ltx@space#1 = %
      \zref@extract{thepage\the\count@}{#1}sp = %
      \the\dimexpr\zref@extract{thepage\the\count@}{#1}sp\relax
      \noexpand\MessageBreak
    }{}%
  }%
}
%    \end{macrocode}
%    \end{macro}
%    \begin{macrocode}
\AfterLastShipout{%
  \ZREF@pl@AfterLastShipout
}
%    \end{macrocode}
%
%    \begin{macrocode}
%</pagelayout>
%    \end{macrocode}
%
%
% \subsection{Module \xmodule{pageattr}}
%
%    \begin{macrocode}
%<*pageattr>
\NeedsTeXFormat{LaTeX2e}
\ProvidesPackage{zref-pageattr}%
  [2020-07-03 v2.32 Module pageattr for zref (HO)]%
\RequirePackage{zref-base}[2019/11/29]
\ifx\ZREF@base@ok Y%
\else
  \expandafter\endinput
\fi
%    \end{macrocode}
%
%    \begin{macrocode}
\RequirePackage{iftex}[2019/11/07]%
%    \end{macrocode}
%
%
%    \begin{macrocode}
\let\ZREF@temp=N%
\ifluatex
\expandafter\@firstoftwo
\else
\expandafter\@secondoftwo
\fi
{%luatex
 \RequirePackage{zref-thepage}[2019/11/29]
 \RequirePackage{zref-lastpage}[2019/11/29]%
 \zref@newprop*{pdfpageattr}[]{\zref@hex{\the\pdfvariable pageattr}}%
 \zref@addprop{thepage}{pdfpageattr}%
 \zref@newprop*{pdfpagesattr}[]{\zref@hex{\the\pdfvariable pagesattr}}%
 \zref@addprop{LastPage}{pdfpagesattr}%
 \let\ZREF@temp=Y%
}
{%other
 \ltx@IfUndefined{pdfpageattr}{%
  \@PackageInfoNoLine{zref-pageattr}{%
    \string\pdfpageattr\space is not available%
  }%
  \def\zref@pdfpageattr#1{}%
  \def\zref@pdfpageattr@used#1{}%
 }{%
  \RequirePackage{zref-thepage}[2019/11/29]%
  \zref@newprop*{pdfpageattr}[]{\zref@hex{\the\pdfpageattr}}%
  \zref@addprop{thepage}{pdfpageattr}%
  \let\ZREF@temp=Y%
 }
 \ltx@IfUndefined{pdfpagesattr}{%
  \@PackageInfoNoLine{zref-pageattr}{%
    \string\pdfpagesattr\space is not available%
  }%
  \def\zref@pdfpagesattr{}%
  \def\zref@pdfpagesattr@used{}%
 }{%
  \RequirePackage{zref-lastpage}[2019/11/29]%
  \zref@newprop*{pdfpagesattr}[]{\zref@hex{\the\pdfpagesattr}}%
  \zref@addprop{LastPage}{pdfpagesattr}%
  \let\ZREF@temp=Y%
 }%
}%
%    \end{macrocode}
%    \begin{macrocode}
\ifx\ZREF@temp N%
  \expandafter\endinput
\fi
%    \end{macrocode}
%    \begin{macrocode}
\RequirePackage{zref-abspage}[2019/11/29]
\RequirePackage{atveryend}[2010/03/24]
\RequirePackage{pdftexcmds}[2010/04/01]
\let\ZREF@temp=Y%
\ltx@IfUndefined{pdf@escapehex}{\let\ZREF@temp=N}{}
\ltx@IfUndefined{pdf@unescapehex}{\let\ZREF@temp=N}{}
\ifx\ZREF@temp N%
  \let\zref@hex\ltx@firstofone
  \let\zref@unhex\ltx@firstofone
\else
  \let\zref@hex\pdf@escapehex
  \let\zref@unhex\pdf@unescapehex
\fi
%    \end{macrocode}
%
%    \begin{macro}{\ifZREF@pa@list}
%    \begin{macrocode}
\ltx@newif\ifZREF@pa@list
%    \end{macrocode}
%    \end{macro}
%    \begin{macro}{\zref@listpageattr}
%    \begin{macrocode}
\ZREF@IfDefinable\zlistpageattr\def{%
  {\ZREF@pa@listtrue}%
}
%    \end{macrocode}
%    \end{macro}
%    \begin{macro}{\ZREF@pa@AfterLastShipout}
%    \begin{macrocode}
\def\ZREF@pa@AfterLastShipout{%
  \ifZREF@pa@list
    \edef\ZREF@page@max{\the\value{abspage}}%
    \ltx@ifundefined{ZREF@org@testdef}{%
      \let\ZREF@org@testdef\@testdef
      \def\@testdef##1##2##3{%
        \ZREF@org@testdef{##1}{##2}{##3}%
        \def\ZREF@temp{##1}%
        \ifx\ZREF@temp\ZREF@RefPrefix
          \expandafter\xdef\csname##1@##2\endcsname{##3}%
        \fi
      }%
    }{}%
    \AtVeryEndDocument{\ZREF@pa@AtVeryEnd}%
  \fi
}
%    \end{macrocode}
%    \end{macro}
%    \begin{macro}{\ZREF@pa@AtVeryEnd}
%    \begin{macrocode}
\let\ZREF@temp=Y%
\ltx@IfUndefined{pdfpageattr}{}{\let\ZREF@temp=N}
\ifluatex \let\ZREF@temp=N \fi
\ifx\ZREF@temp Y
 \expandafter\@firstoftwo
\else
 \expandafter\@secondoftwo
\fi
 {%
  \def\ZREF@pa@AtVeryEnd{}%
 }
 {%
  \def\ZREF@pa@AtVeryEnd{%
    \begingroup
      \toks@{List of \ltx@backslashchar
             \ifluatex pdfvariable\else pdf\fi
             pdfpageattr:\MessageBreak}%
      \count@=1 %
      \ZREF@pa@ListPage
      \edef\x{\endgroup
        \noexpand\@PackageInfoNoLine{zref-pageattr}{%
          \the\toks@
        }%
      }%
    \x
  }%
%    \end{macrocode}
%    \end{macro}
%    \begin{macro}{\zref@pageattr}
%    \begin{macrocode}
\def\zref@pdfpageattr#1{%
  \zref@unhex{%
    \zref@extract{thepage\ZREF@number{#1}}{pdfpageattr}%
  }%
}
% compability, \zref@pageattr was defined in older versions
\let\zref@pageattr\zref@pdfpageattr
%    \end{macrocode}
%    \end{macro}
%    \begin{macro}{\zref@pageattr@used}
%    \begin{macrocode}
\ZREF@Robust\def\zref@pageattr@used#1{%
  \zref@refused{thepage\ZREF@number{#1}}%
}
%    \end{macrocode}
%    \end{macro}
%    \begin{macro}{\ZREF@pa@ListPage}
%    \begin{macrocode}
  \def\ZREF@pa@ListPage{%
    \edef\x{%
      \toks@={%
        \the\toks@
        Page \the\count@:%
        \noexpand\MessageBreak
        \zref@ifrefundefined{thepage\the\count@}{}{%
          <<\zref@pdfpageattr\count@>>%
          \noexpand\MessageBreak
        }%
      }%
    }\x
    \ifnum\ZREF@page@max>\count@
      \advance\count@ by\ltx@one
    \else
      \expandafter\ltx@gobble
    \fi
    \ZREF@pa@ListPage
  }%
}
%    \end{macrocode}
%    \end{macro}
%    \begin{macrocode}
\let\ZREF@temp=Y%
\ltx@IfUndefined{pdfpagesattr}{}{\let\ZREF@temp=N}
\ifluatex \let\ZREF@temp=N \fi
\ifx\ZREF@temp N
 \expandafter\@firstofone
\fi
 {%
%    \end{macrocode}
%    \begin{macro}{\zref@pdfpagesattr}
%    \begin{macrocode}
  \def\zref@pdfpagesattr{%
    \zref@unhex{%
      \zref@extract{LastPage}{pdfpagesattr}%
    }%
  }%
%    \end{macrocode}
%    \end{macro}
%    \begin{macro}{\zref@pdfpagesattr@used}
%    \begin{macrocode}
  \ZREF@Robust\def\zref@pdfpagesattr@used{%
    \zref@refused{LastPage}%
  }%
%    \end{macrocode}
%    \end{macro}
%    \begin{macrocode}
  \ltx@LocalAppendToMacro\ZREF@pa@AtVeryEnd{%
    \@PackageInfoNoLine{zref-pageattr}{%
      \ltx@backslashchar
      \ifluatex pdfvariable\else pdf\fi
      pagesattr:\MessageBreak
      <<\zref@pdfpagesattr>>%
      \MessageBreak
    }%
  }%
}
\AfterLastShipout{%
  \ZREF@pa@AfterLastShipout
}
%    \end{macrocode}
%    \begin{macrocode}
%</pageattr>
%    \end{macrocode}
%
%
% \subsection{Module \xmodule{marks}}
%
%    \begin{macrocode}
%<*marks>
\NeedsTeXFormat{LaTeX2e}
\ProvidesPackage{zref-marks}%
  [2020-07-03 v2.32 Module marks for zref (HO)]%
\RequirePackage{zref-base}[2019/11/29]
\ifx\ZREF@base@ok Y%
\else
  \expandafter\endinput
\fi
%    \end{macrocode}
%
%    \begin{macrocode}
\newcommand*{\zref@marks@register}[3][]{%
  \edef\ZREF@TempName{#1}%
  \edef\ZREF@TempNum{\ZREF@number{#2}}%
  \ifnum\ZREF@TempNum<\ltx@zero %
    \PackageError\ZREF@name{%
      \string\zref@marks@register\ltx@space is called with invalid%
      \MessageBreak
      marks register number (\ZREF@TempNum)%
    }{%
      Use `0' or the command, defined by \string\newmarks.\MessageBreak
      \@ehc
    }%
  \else
    \ifx\ZREF@TempName\ltx@empty
      \edef\ZREF@TempName{mark\romannumeral\ZREF@TempNum}%
    \else
      \edef\ZREF@TempName{marks\ZREF@TempName}%
    \fi
    \ZREF@MARKS@DefineProp{top}%
    \ZREF@MARKS@DefineProp{first}%
    \ZREF@MARKS@DefineProp{bot}%
    \kv@parse{#3}{%
      \ifx\kv@value\relax
        \def\kv@value{top,first,bot}%
      \fi
      \edef\ZREF@temp{\expandafter\ltx@car\kv@key X\@nil}%
      \ifx\ZREF@temp\ZREF@STAR
        \edef\kv@key{\expandafter\ltx@cdr\kv@key\@nil}%
        \zref@newlist\kv@key
      \fi
      \expandafter\comma@parse\expandafter{\kv@value}{%
        \ifcase0\ifx\comma@entry\ZREF@NAME@top 1\else
                \ifx\comma@entry\ZREF@NAME@first 1\else
                \ifx\comma@entry\ZREF@NAME@bot 1\fi\fi\fi\ltx@space
          \PackageWarning{zref-marks}{%
            Use `top', `first' or `bot' for the list values%
            \MessageBreak
            in the third argument of \string\zref@marks@register.%
            \MessageBreak
            Ignoring unkown value `\comma@entry'%
          }%
        \else
          \zref@addprop{\kv@key}{\comma@entry\ZREF@TempName}%
        \fi
        \ltx@gobble
      }%
      \ltx@gobbletwo
    }%
  \fi
}
\def\ZREF@STAR{*}
\def\ZREF@NAME@top{top}
\def\ZREF@NAME@first{first}
\def\ZREF@NAME@bot{bot}
\def\ZREF@MARKS@DefineProp#1{%
  \zref@ifpropundefined{#1\ZREF@TempName}{%
    \ifnum\ZREF@TempNum=\ltx@zero
      \begingroup
        \edef\x{\endgroup
          \noexpand\zref@newprop*{#1\ZREF@TempName}[]{%
            \expandafter\noexpand\csname#1mark\endcsname
          }%
        }%
      \x
    \else
      \begingroup
        \edef\x{\endgroup
          \noexpand\zref@newprop*{#1\ZREF@TempName}[]{%
            \expandafter\noexpand\csname#1marks\endcsname
            \ZREF@TempNum
          }%
        }%
      \x
    \fi
  }{%
    \PackageWarning{zref-marks}{%
      \string\zref@marks@register\ltx@space does not generate the%
      \MessageBreak
      new property `#1\ZREF@TempName', because\MessageBreak
      it is already defined%
    }%
  }%
}
%    \end{macrocode}
%    \begin{macrocode}
%</marks>
%    \end{macrocode}
%
% \subsection{Module \xmodule{runs}}
%
%    This module does not use the label-reference-system.
%    The reference changes with each \hologo{LaTeX}~run and would
%    force a rerun warning always.
%    \begin{macrocode}
%<*runs>
\NeedsTeXFormat{LaTeX2e}
\ProvidesPackage{zref-runs}%
  [2020-07-03 v2.32 Module runs for zref (HO)]%
%    \end{macrocode}
%
%    \begin{macro}{\zruns}
%    \begin{macrocode}
\providecommand*{\zruns}{0}%
\AtBeginDocument{%
  \edef\zruns{\number\numexpr\zruns+1}%
  \begingroup
    \def\on@line{}%
    \PackageInfo{zref-runs}{LaTeX runs: \zruns}%
    \if@filesw
      \immediate\write\@mainaux{%
        \string\gdef\string\zruns{\zruns}%
      }%
    \fi
  \endgroup
}
%    \end{macrocode}
%    \end{macro}
%
%    \begin{macrocode}
%</runs>
%    \end{macrocode}
%
%
% \subsection{Module \xmodule{perpage}}
%
%    \begin{macrocode}
%<*perpage>
\NeedsTeXFormat{LaTeX2e}
\ProvidesPackage{zref-perpage}%
  [2020-07-03 v2.32 Module perpage for zref (HO)]%
\RequirePackage{zref-base}[2019/11/29]
\ifx\ZREF@base@ok Y%
\else
  \expandafter\endinput
\fi
%    \end{macrocode}
%
%    This module resets a counter at page boundaries.
%    Because of the asynchronous output routine
%    page counter properties cannot be asked directly,
%    references are necessary.
%
%    For detecting changed pages module \xmodule{abspage}
%    is loaded.
%    \begin{macrocode}
\RequirePackage{zref-abspage}[2019/11/29]
%    \end{macrocode}
%
%    We group the properties for the needed references
%    in the property list |perpage|. The property |pagevalue|
%    records the correct value of the page counter.
%    \begin{macrocode}
\ZREF@NewPropPageValue
\zref@newlist{perpage}
\zref@addprops{perpage}{abspage,page,pagevalue}
%    \end{macrocode}
%    The page value, known by the reference mechanism, will be
%    stored in counter |zpage|.
%    \begin{macrocode}
\newcounter{zpage}
%    \end{macrocode}
%    Counter |zref@unique| helps in generating unique reference
%    names.
%    \begin{macrocode}
\zref@require@unique
%    \end{macrocode}
%    In order to be able to reset the counter, we hook here
%    into \cs{stepcounter}. In fact two nested hooks are used
%    to allow other packages to use the first hook at the beginning
%    of \cs{stepcounter}.
%    \begin{macrocode}
\let\ZREF@org@stepcounter\stepcounter
\def\stepcounter#1{%
  \ifcsname @stepcounterhook@#1\endcsname
    \csname @stepcounterhook@#1\endcsname
  \fi
  \ZREF@org@stepcounter{#1}%
}
%    \end{macrocode}
%    \cs{@stpelt} must be adapted due to the change in latex
%    2015-01, see https://github.com/ho-tex/zref/issues/26
%    \begin{macrocode}
\let\ZREF@org@@stpelt\@stpelt
\def\@stpelt#1{%
  \ifcsname ZREF@perpage@#1\endcsname
    \begingroup
      \let\stepcounter\ZREF@org@stepcounter
      \ZREF@org@@stpelt{#1}%
    \endgroup
    \expandafter\ltx@gobbletwo
  \fi
  \ZREF@org@@stpelt{#1}%
}
%    \end{macrocode}
%
%    \begin{macro}{\zmakeperpage}
%    Makro \cs{zmakeperpage} resets a counter at each page
%    break. It uses the same syntax and semantics as
%    \cs{MakePerPage} from package \xpackage{perpage} \cite{perpage}.
%    The initial start value can be given by the optional
%    argument. Default is one that means after the first
%    \cs{stepcounter} on a new page the counter starts with
%    one.
%    \begin{macrocode}
\ZREF@IfDefinable\zmakeperpage\def{%
  {%
    \@ifnextchar[\ZREF@makeperpage@opt{\ZREF@@makeperpage[\ltx@zero]}%
  }%
}
%    \end{macrocode}
%    \end{macro}
%    We hook before the counter is incremented in \cs{stepcounter},
%    package \xpackage{perpage} afterwards. Thus a little
%    calculation is necessary.
%    \begin{macrocode}
\def\ZREF@makeperpage@opt[#1]{%
  \begingroup
    \edef\x{\endgroup
      \noexpand\ZREF@@makeperpage[\number\numexpr#1-1\relax]%
    }%
  \x
}
%    \end{macrocode}
%    \begin{macrocode}
\def\ZREF@@makeperpage[#1]#2{%
  \@ifundefined{@stepcounterhook@#2}{%
    \expandafter\gdef\csname @stepcounterhook@#2\endcsname{}%
  }{}%
  \expandafter\gdef\csname ZREF@perpage@#2\endcsname{%
    \ZREF@@perpage@step{#2}{#1}%
  }%
  \expandafter\g@addto@macro\csname @stepcounterhook@#2\endcsname{%
    \ifcsname ZREF@perpage@#2\endcsname
      \csname ZREF@perpage@#2\endcsname
    \fi
  }%
}
%    \end{macrocode}
%    \begin{macro}{\ZREF@@perpage@step}
%    The heart of this module follows.
%    \begin{macrocode}
\def\ZREF@@perpage@step#1#2{%
%    \end{macrocode}
%    First the reference is generated.
%    \begin{macrocode}
  \global\advance\c@zref@unique\ltx@one
  \begingroup
    \expandafter
    \zref@labelbylist\expandafter{\thezref@unique}{perpage}%
%    \end{macrocode}
%    The \cs{expandafter} commands are necessary, because \cs{ZREF@temp}
%    is also used inside of \cs{zref@labelbylist}.
%
%    The evaluation of the reference follows. If the reference
%    is not yet kwown, we use the page counter as approximation.
%    \begin{macrocode}
    \zref@ifrefundefined\thezref@unique{%
      \global\c@zpage=\c@page
      \global\let\thezpage\thepage
      \expandafter\xdef\csname ZREF@abspage@#1\endcsname{%
        \number\c@abspage
      }%
    }{%
%    \end{macrocode}
%    The reference is used to set \cs{thezpage} and counter |zpage|.
%    \begin{macrocode}
      \global\c@zpage=\zref@extract\thezref@unique{pagevalue}\relax
      \xdef\thezpage{\noexpand\zref@extract{\thezref@unique}{page}}%
      \expandafter\xdef\csname ZREF@abspage@#1\endcsname{%
        \zref@extractdefault\thezref@unique
            {abspage}{\number\c@abspage}%
      }%
    }%
%    \end{macrocode}
%    Page changes are detected by a changed absolute page number.
%    \begin{macrocode}
    \expandafter\ifx\csname ZREF@abspage@#1\expandafter\endcsname
                    \csname ZREF@currentabspage@#1\endcsname
    \else
      \global\csname c@#1\endcsname=#2\relax
      \global\expandafter\let
          \csname ZREF@currentabspage@#1\expandafter\endcsname
          \csname ZREF@abspage@#1\endcsname
    \fi
  \endgroup
}
%    \end{macrocode}
%    \end{macro}
%    \begin{macro}{\zunmakeperpage}
%    Macro \cs{zunmakeperpage} cancels the effect of \cs{zmakeperpage}.
%    \begin{macrocode}
\ZREF@IfDefinable\zunmakeperpage\def{%
  #1{%
    \global\expandafter
    \let\csname ZREF@perpage@#1\endcsname\@undefined
  }%
}
%    \end{macrocode}
%    \end{macro}
%
%    \begin{macrocode}
%</perpage>
%    \end{macrocode}
%
%
% \subsection{Module \xmodule{titleref}}
%
%    \begin{macrocode}
%<*titleref>
\NeedsTeXFormat{LaTeX2e}
\ProvidesPackage{zref-titleref}%
  [2020-07-03 v2.32 Module titleref for zref (HO)]%
\RequirePackage{zref-base}[2019/11/29]
\ifx\ZREF@base@ok Y%
\else
  \expandafter\endinput
\fi
\RequirePackage{gettitlestring}[2009/12/08]
%    \end{macrocode}
%
% \subsubsection{Implementation}
%    \begin{macrocode}
\RequirePackage{keyval}
%    \end{macrocode}
%
%    This module makes section and caption titles available
%    for the reference system. It uses some of the ideas
%    of package \xpackage{nameref} and \xpackage{titleref}.
%
%    Now we can add the property \verb|title| is added to
%    the main property list.
%    \begin{macrocode}
\ZREF@NewPropTitle
\zref@addprop\ZREF@mainlist{title}%
%    \end{macrocode}
%
%    The title strings go into the \xfile{.aux} file, thus
%    they need some kind of protection.
%    Package \xpackage{titleref} uses a protected expansion
%    method. The advantage is that this can be used to
%    cleanup the string and to remove \cs{label}, \cs{index}
%    and other macros unwanted for referencing.
%    But there is the risk that fragile stuff can break.
%
%    Therefore package \xpackage{nameref} does not expand
%    the string. Thus the entries can safely be written to
%    the \xfile{.aux} file. But potentially dangerous
%    macros such as \cs{label} remain in the string and
%    can cause problems when using the string in references.
%
%    \begin{macro}{\ifzref@titleref@expand}
%    The switch \cs{ifzref@titleref@expand} distinguishes
%    between the both methods. Package \xpackage{nameref}'s
%    behaviour is achieved by setting the switch to false,
%    otherwise \xpackage{titleref}'s expansion is used.
%    Default is false.
%    \begin{macrocode}
\newif\ifzref@titleref@expand
%    \end{macrocode}
%    \end{macro}
%
%    \begin{macro}{\ZREF@titleref@hook}
%    The hook \cs{ZREF@titleref@hook} allows to extend
%    the cleanup for the expansion method. Thus
%    unnecessary macros can be removed or dangerous commands
%    removed. The hook is executed before the expansion of
%    \cs{zref@titleref@current}.
%    \begin{macrocode}
\let\ZREF@titleref@hook\ltx@empty
%    \end{macrocode}
%    \end{macro}
%    \begin{macro}{\zref@titleref@cleanup}
%    The hook should not be used directly, instead we provide the
%    macro \cs{zref@titleref@cleanup} to add stuff to the hook
%    and prevents that a previous non-empty content is not discarded
%    accidently.
%    \begin{macrocode}
\ZREF@Robust\def\zref@titleref@cleanup#1{%
   \begingroup
    \toks@\expandafter{%
      \ZREF@titleref@hook
      #1%
    }%
  \expandafter\endgroup
  \expandafter\def\expandafter\ZREF@titleref@hook\expandafter{%
    \the\toks@
  }%
}%
%    \end{macrocode}
%    \end{macro}
%
%    \begin{macro}{\ifzref@titleref@stripperiod}
%    Sometimes a title contains a period at the end.
%    Package \xpackage{nameref} removes this. This behaviour
%    is controlled by the switch \cs{ifzref@titleref@stripperiod}
%    and works regardless of the setting of option \xoption{expand}.
%    Period stripping is the default.
%    \begin{macrocode}
\newif\ifzref@titleref@stripperiod
\zref@titleref@stripperiodtrue
%    \end{macrocode}
%    \end{macro}
%
%    \begin{macro}{\zref@titleref@setcurrent}
%    Macro \cs{zref@titleref@setcurrent} sets a new current title
%    stored in \cs{zref@titleref@current}. Some cleanup and
%    expansion is performed that can be controlled by the
%    previous switches.
%    \begin{macrocode}
\ZREF@Robust\def\zref@titleref@setcurrent#1{%
  \ifzref@titleref@expand
    \GetTitleStringExpand{#1}%
  \else
    \GetTitleStringNonExpand{#1}%
  \fi
  \edef\zref@titleref@current{%
    \detokenize\expandafter{\GetTitleStringResult}%
  }%
  \ifzref@titleref@stripperiod
    \edef\zref@titleref@current{%
      \expandafter\ZREF@stripperiod\zref@titleref@current
      \ltx@empty.\ltx@empty\@nil
    }%
  \fi
}%
\GetTitleStringDisableCommands{%
  \ZREF@titleref@hook
}
%    \end{macrocode}
%    \end{macro}
%    \begin{macro}{\ZREF@stripperiod}
%    If \cs{ZREF@stripperiod} is called, the argument consists of
%    space tokens and tokens with catcode 12 (other), because
%    of \hologo{eTeX}'s \cs{detokenize}.
%    \begin{macrocode}
\def\ZREF@stripperiod#1.\ltx@empty#2\@nil{#1}%
%    \end{macrocode}
%    \end{macro}
%
% \subsubsection{User interface}
%
%    \begin{macro}{\ztitlerefsetup}
%    The behaviour of module \xmodule{titleref} is controlled
%    by switches and a hook. They can be set by \cs{ztitlerefsetup}
%    with a key value interface, provided by package \xpackage{keyval}.
%    Also the current title can be given explicitly by the key
%    \verb|title|.
%    \begin{macrocode}
\define@key{ZREF@TR}{expand}[true]{%
  \csname zref@titleref@expand#1\endcsname
}%
\define@key{ZREF@TR}{stripperiod}[true]{%
  \csname zref@titleref@stripperiod#1\endcsname
}%
\define@key{ZREF@TR}{cleanup}{%
  \zref@titleref@cleanup{#1}%
}%
\define@key{ZREF@TR}{title}{%
  \def\zref@titleref@current{#1}%
}%
\ZREF@IfDefinable\ztitlerefsetup\def{%
  {\kvsetkeys{ZREF@TR}}%
}%
%    \end{macrocode}
%    \end{macro}
%
%    \begin{macro}{\ztitleref}
%    The user command \cs{ztitleref} references the title.
%    For safety \cs{label} is disabled to prevent multiply
%    defined references.
%    \begin{macrocode}
\ZREF@IfDefinable\ztitleref\def{%
  {\zref@wrapper@babel\ZREF@titleref}%
}%
\def\ZREF@titleref#1{%
  \begingroup
    \zref@refused{#1}%
    \let\label\ltx@gobble
    \zref@extract{#1}{title}%
  \endgroup
}%
%    \end{macrocode}
%    \end{macro}
%
% \subsubsection{Patches for section and caption commands}
%
%    The section and caption macros are patched to extract the
%    title data.
%
%    Captions of figures and tables.
%    \begin{macrocode}
\AtBeginDocument{%
  \ZREF@patch{@caption}{%
    \long\def\@caption#1[#2]{%
      \zref@titleref@setcurrent{#2}%
      \ZREF@org@@caption{#1}[{#2}]%
    }%
  }%
%    \end{macrocode}
%
%    Section commands without star. The title version for the
%    table of contents is used because it is usually shorter and
%    more robust.
%    \begin{macrocode}
  \ZREF@patch{@part}{%
    \def\@part[#1]{%
      \zref@titleref@setcurrent{#1}%
      \ZREF@org@@part[{#1}]%
    }%
  }%
  \ZREF@patch{@chapter}{%
    \def\@chapter[#1]{%
      \zref@titleref@setcurrent{#1}%
      \ZREF@org@@chapter[{#1}]%
    }%
  }%
  \ZREF@patch{@sect}{%
    \def\@sect#1#2#3#4#5#6[#7]{%
      \zref@titleref@setcurrent{#7}%
      \ZREF@org@@sect{#1}{#2}{#3}{#4}{#5}{#6}[{#7}]%
    }%
  }%
%    \end{macrocode}
%    The star versions of the section commands.
%    \begin{macrocode}
  \ZREF@patch{@spart}{%
    \def\@spart#1{%
      \zref@titleref@setcurrent{#1}%
      \ZREF@org@@spart{#1}%
    }%
  }%
  \ZREF@patch{@schapter}{%
    \def\@schapter#1{%
      \zref@titleref@setcurrent{#1}%
      \ZREF@org@@schapter{#1}%
    }%
  }%
  \ZREF@patch{@ssect}{%
    \def\@ssect#1#2#3#4#5{%
      \zref@titleref@setcurrent{#5}%
      \ZREF@org@@ssect{#1}{#2}{#3}{#4}{#5}%
    }%
  }%
%    \end{macrocode}
%
% \subsubsection{Environment \textsf{description}}
%
%    \begin{macrocode}
  \ZREF@patch{descriptionlabel}{%
    \def\descriptionlabel#1{%
      \zref@titleref@setcurrent{#1}%
      \ZREF@org@descriptionlabel{#1}%
    }%
  }%
%    \end{macrocode}
%
% \subsubsection{Class \xclass{memoir}}
%
%    \begin{macrocode}
  \@ifclassloaded{memoir}{%
    \ltx@IfUndefined{ifheadnameref}{}{%
      \def\@chapter[#1]#2{%
        \ltx@IfUndefined{ch@pt@c}{%
          \zref@titleref@setcurrent{#1}%
        }{%
          \ifx\ch@pt@c\ltx@empty
            \zref@titleref@setcurrent{#2}%
          \else
            \def\NR@temp{#1}%
            \ifx\NR@temp\ltx@empty
              \expandafter\zref@titleref@setcurrent
              \expandafter{\ch@pt@c}%
            \else
              \ifheadnameref
                \zref@titleref@setcurrent{#1}%
              \else
                \expandafter\zref@titleref@setcurrent
                \expandafter{\ch@pt@c}%
              \fi
            \fi
          \fi
        }%
        \ZREF@org@@chapter[{#1}]{#2}%
      }%
      \ZREF@patch{M@sect}{%
        \def\M@sect#1#2#3#4#5#6[#7][#8]{%
          \ifheadnameref
            \zref@titleref@setcurrent{#8}%
          \else
            \zref@titleref@setcurrent{#7}%
          \fi
          \ZREF@org@M@sect{#1}{#2}{#3}{#4}{#5}{#6}[{#7}][{#8}]%
        }%
      }%
    }%
  }{}%
%    \end{macrocode}
%
% \subsubsection{Class \xclass{beamer}}
%
%    \begin{macrocode}
  \@ifclassloaded{beamer}{%
    \ZREF@patch{beamer@section}{%
      \long\def\beamer@section[#1]{%
        \zref@titleref@setcurrent{#1}%
        \ZREF@org@beamer@section[{#1}]%
      }%
    }%
    \ZREF@patch{beamer@subsection}{%
      \long\def\beamer@subsection[#1]{%
        \zref@titleref@setcurrent{#1}%
        \ZREF@org@beamer@subsection[{#1}]%
      }%
    }%
    \ZREF@patch{beamer@subsubsection}{%
      \long\def\beamer@subsubsection[#1]{%
        \zref@titleref@setcurrent{#1}%
        \ZREF@org@beamer@subsubsection[{#1}]%
      }%
    }%
  }{}%
%    \end{macrocode}
%
% \subsubsection{Package \xpackage{titlesec}}
%
%    \begin{macrocode}
  \@ifpackageloaded{titlesec}{%
    \ZREF@patch{ttl@sect@i}{%
      \def\ttl@sect@i#1#2[#3]#4{%
        \zref@titleref@setcurrent{#4}%
        \ZREF@org@ttl@sect@i{#1}{#2}[{#3}]{#4}%
      }%
    }%
    \ZREF@patch{ttl@straight@i}{%
      \def\ttl@straight@i#1[#2]#3{%
        \def\ZREF@temp{#2}%
        \ifx\ZREF@temp\ltx@empty
          \zref@titleref@setcurrent{#3}%
        \else
          \zref@titleref@setcurrent{#2}%
        \fi
        \ZREF@org@ttl@straight@i{#1}[{#2}]{#3}%
      }%
    }%
  }{}%
%    \end{macrocode}
%
% \subsubsection{Package \xpackage{longtable}}
%
%    Package \xpackage{longtable}: some support for its \cs{caption}.
%    However \cs{label} inside the caption is not supported.
%    \begin{macrocode}
  \@ifpackageloaded{longtable}{%
    \ZREF@patch{LT@c@ption}{%
      \def\LT@c@ption#1[#2]#3{%
        \ZREF@org@LT@c@ption{#1}[{#2}]{#3}%
        \zref@titleref@setcurrent{#2}%
      }%
    }%
  }{}%
%    \end{macrocode}
%
% \subsubsection{Package \xpackage{listings}}
%
%    Package \xpackage{listings}: support for its caption.
%    \begin{macrocode}
  \@ifpackageloaded{listings}{%
    \ZREF@patch{lst@MakeCaption}{%
      \def\lst@MakeCaption{%
        \ifx\lst@label\ltx@empty
        \else
          \expandafter\zref@titleref@setcurrent\expandafter{%
            \lst@@caption
          }%
        \fi
        \ZREF@org@lst@MakeCaption
      }%
    }%
  }{}%
%    \end{macrocode}
%
% \subsubsection{Theorems}
%
%    \begin{macrocode}
  \ZREF@patch{@opargbegintheorem}{%
    \def\@opargbegintheorem#1#2#3{%
      \zref@titleref@setcurrent{#3}%
      \ZREF@org@@opargbegintheorem{#1}{#2}{#3}%
    }%
  }%
  \@ifpackageloaded{amsthm}{%
    \begingroup
      \edef\x{macro:\string#1\string#2[\string#3]}%
      \@onelevel@sanitize\x
      \def\y#1->#2\@nil{#1}%
      \edef\z{\expandafter\y\meaning\@begintheorem->\@nil}%
      \@onelevel@sanitize\z
    \expandafter\endgroup
    \ifx\x\z
      \ZREF@patch{@begintheorem}{%
        \def\@begintheorem#1#2[#3]{%
          \zref@titleref@setcurrent{#3}%
          \ZREF@org@@begintheorem{#1}{#2}[{#3}]%
        }%
      }%
    \fi
  }{}%
%    \end{macrocode}
%
%    \begin{macrocode}
}
%    \end{macrocode}
%
%    \begin{macrocode}
%</titleref>
%    \end{macrocode}
%
%
% \subsection{Module \xmodule{xr}}
%
%    \begin{macrocode}
%<*xr>
\NeedsTeXFormat{LaTeX2e}
\ProvidesPackage{zref-xr}%
  [2020-07-03 v2.32 Module xr for zref (HO)]%
\RequirePackage{zref-base}[2019/11/29]
\ifx\ZREF@base@ok Y%
\else
  \expandafter\endinput
\fi
%    \end{macrocode}
%    \begin{macrocode}
\RequirePackage{keyval}
\RequirePackage{kvoptions}[2010/02/22]
%    \end{macrocode}
%
%    We declare property |url|, because this is added,
%    if a reference is imported and has not already set this field.
%    Or if |hyperref| is used, then this property can be asked.
%    \begin{macrocode}
\zref@newprop{url}{}%
\zref@newprop{urluse}{}%
%    \end{macrocode}
%    \begin{macrocode}
\zref@newprop{externaldocument}{}%
%    \end{macrocode}
%
%    Most code, especially the handling of the \xfile{.aux}
%    files are taken from David Carlisle's \xpackage{xr} package.
%    Therefore I drop the documentation for these macros here.
%
%    \begin{macro}{\zref@xr@ext}
%    If the URL is not specied, then assume
%    processed file with a guessed extension.
%    Use the setting of \xpackage{hyperref} if available.
%    \begin{macrocode}
\providecommand*{\zref@xr@ext}{%
  \ltx@ifundefined{XR@ext}{pdf}{\XR@ext}%
}%
%    \end{macrocode}
%    \end{macro}
%    \begin{macro}{\ifZREF@xr@zreflabel}
%    The use of the star form of \cs{zexternaldocument}
%    is remembered in the switch \cs{ifZREF@xr@zreflabel}.
%    \begin{macrocode}
\newif\ifZREF@xr@zreflabel
%    \end{macrocode}
%    \end{macro}
%    \begin{macrocode}
\SetupKeyvalOptions{%
  family=ZREF@XR,%
  prefix=ZREF@xr@%
}
\DeclareBoolOption[true]{tozreflabel}
\DeclareBoolOption[false]{toltxlabel}
\DeclareBoolOption{verbose}
\define@key{ZREF@XR}{ext}{%
  \def\zref@xr@{#1}%
}
\DeclareBoolOption[false]{urluse}
%    \end{macrocode}
%    \begin{macro}{\zxrsetup}
%    \begin{macrocode}
\newcommand*{\zxrsetup}{%
  \kvsetkeys{ZREF@XR}%
}%
%    \end{macrocode}
%    \end{macro}
%
%    \begin{macro}{\ZREF@xr@URL}
%    \begin{macrocode}
\newcount\ZREF@xr@URL
\ZREF@xr@URL=\ltx@zero
%    \end{macrocode}
%    \end{macro}
%    \begin{macro}{\ZREF@xr@AddURL}
%    \begin{macrocode}
\def\ZREF@xr@AddURL#1{%
  \begingroup
    \def\ZREF@temp{#1}%
    \count@=\ltx@one
    \ZREF@xr@@AddUrl
  \endgroup
}
%    \end{macrocode}
%    \end{macro}
%    \begin{macro}{\ZREF@xr@@AddUrl}
%    \begin{macrocode}
\def\ZREF@xr@@AddUrl{%
  \ifnum\count@>\ZREF@xr@URL
    \global\advance\ZREF@xr@URL by\ltx@one
    \xdef\ZREF@xr@theURL{\romannumeral\ZREF@xr@URL}%
    \global\expandafter\let
        \csname Z@U@\ZREF@xr@theURL\endcsname\ZREF@temp
    \@PackageInfo{zref-xr}{%
      \ltx@backslashchar Z@U@\ZREF@xr@theURL:\MessageBreak
      \ZREF@temp\MessageBreak
    }%
  \else
    \expandafter
    \ifx\csname Z@U@\romannumeral\count@\endcsname\ZREF@temp
      \xdef\ZREF@xr@theURL{\romannumeral\count@}%
    \else
      \expandafter\expandafter\expandafter\ZREF@xr@@AddUrl
    \fi
  \fi
}
%    \end{macrocode}
%    \end{macro}
%
%    \begin{macro}{\zexternaldocument}
%    In its star form it looks for \cs{newlabel},
%    otherwise for \cs{zref@newlabel}.
%     Later we
%    will read \xfile{.aux} files that expects
%    |@| to have catcode 11 (letter).
%    \begin{macrocode}
\ZREF@IfDefinable\zexternaldocument\def{%
  {%
    \ZREF@NewPropAnchor
    \ZREF@NewPropTitle
    \begingroup
      \csname @safe@actives@true\endcsname
      \makeatletter
      \@ifstar{%
        \ZREF@xr@zreflabelfalse
        \@testopt\ZREF@xr@externaldocument{}%
      }{%
        \ZREF@xr@zreflabeltrue
        \@testopt\ZREF@xr@externaldocument{}%
      }%
  }%
}%
%    \end{macrocode}
%    \end{macro}
%
%    If the \cs{include} featuer was used, there can be
%    several \xfile{.aux} files. These files are read
%    one after another, especially they are not recursively
%    read in order to save read registers. Thus it can
%    happen that the read order of the newlabel commands
%    differs from \hologo{LaTeX}'s order using \cs{input}.
%
%    \begin{macro}{\ZREF@xr@externaldocument}
%    It reads the remaining arguments. \cs{newcommand}
%    comes in handy for the optional argument.
%    \begin{macrocode}
\def\ZREF@xr@externaldocument[#1]#2{%
    \def\ZREF@xr@prefix{#1}%
    \let\ZREF@xr@filelist\ltx@empty
    \edef\ZREF@xr@externalfile{#2}%
    \edef\ZREF@xr@file{\ZREF@xr@externalfile.aux}%
    \filename@parse{#2}%
    \@testopt\ZREF@xr@graburl{#2.\zref@xr@ext}%
}%
\def\ZREF@xr@graburl[#1]{%
    \edef\ZREF@xr@url{#1}%
    \ifZREF@xr@urluse
      \expandafter\ZREF@xr@AddURL\expandafter{\ZREF@xr@url}%
      \expandafter\def\expandafter\ZREF@xr@url
      \expandafter{\csname Z@U@\ZREF@xr@theURL\endcsname}%
    \fi
    \ZREF@xr@checkfile
  \endgroup
}%
%    \end{macrocode}
%    \end{macro}
%    \begin{macro}{\ZREF@xr@processfile}
%    We follow \xpackage{xr} here, \cs{IfFileExists} offers
%    a nicer test, but we have to open the file anyway.
%    \begin{macrocode}
\def\ZREF@xr@checkfile{%
  \openin\@inputcheck\ZREF@xr@file\relax
  \ifeof\@inputcheck
    \PackageWarning{zref-xr}{%
      File `\ZREF@xr@file' not found or empty,\MessageBreak
      labels not imported%
    }%
  \else
    \PackageInfo{zref-xr}{%
      Label \ifZREF@xr@zreflabel (zref) \fi
      import from `\ZREF@xr@file'%
    }%
    \def\ZREF@xr@found{0}%
    \def\ZREF@xr@ignored@empty{0}%
    \def\ZREF@xr@ignored@zref{0}%
    \def\ZREF@xr@ignored@ltx{0}%
    \ZREF@xr@processfile
    \closein\@inputcheck
    \begingroup
      \let\on@line\ltx@empty
      \PackageInfo{zref-xr}{%
        Statistics for `\ZREF@xr@file':\MessageBreak
        \ZREF@xr@found\space
        \ifZREF@xr@zreflabel zref\else LaTeX\fi\space
        label(s) found%
        \ifnum\ZREF@xr@ignored@empty>0 %
          ,\MessageBreak
          \ZREF@xr@ignored@empty\space empty label(s) ignored%
        \fi
        \ifnum\ZREF@xr@ignored@zref>0 %
          ,\MessageBreak
          \ZREF@xr@ignored@zref\space
          duplicated zref label(s) ignored%
        \fi
        \ifnum\ZREF@xr@ignored@ltx>0 %
          ,\MessageBreak
          \ZREF@xr@ignored@ltx\space
          duplicated latex label(s) ignored%
        \fi
      }%
    \endgroup
  \fi
  \ifx\ZREF@xr@filelist\ltx@empty
  \else
    \edef\ZREF@xr@file{%
      \expandafter\ltx@car\ZREF@xr@filelist\@nil
    }%
    \edef\ZREF@xr@filelist{%
      \expandafter\ltx@cdr\ZREF@xr@filelist\ltx@empty\@nil
    }%
    \expandafter\ZREF@xr@checkfile
  \fi
}%
%    \end{macrocode}
%    \end{macro}
%
%    \begin{macro}{\ZREF@xr@processfile}
%    \begin{macrocode}
\def\ZREF@xr@processfile{%
  \read\@inputcheck to\ZREF@xr@line
  \expandafter\ZREF@xr@processline\ZREF@xr@line..\ZREF@nil
  \ifeof\@inputcheck
  \else
    \expandafter\ZREF@xr@processfile
  \fi
}%
%    \end{macrocode}
%    \end{macro}
%    \begin{macro}{\ZREF@xr@processline}
%    The most work must be done for analyzing the
%    arguments of \cs{newlabel}.
%    \begin{macrocode}
\long\def\ZREF@xr@processline#1#2#3\ZREF@nil{%
  \def\x{#1}%
  \toks@{#2}%
  \ifZREF@xr@zreflabel
    \ifx\x\ZREF@xr@zref@newlabel
      \expandafter
      \ZREF@xr@process@zreflabel\ZREF@xr@line...\ZREF@nil
    \fi
  \else
    \ifx\x\ZREF@xr@newlabel
      \expandafter
      \ZREF@xr@process@label\ZREF@xr@line...[]\ZREF@nil
    \fi
  \fi
  \ifx\x\ZREF@xr@@input
    \edef\ZREF@xr@filelist{%
      \etex@unexpanded\expandafter{\ZREF@xr@filelist}%
      {\filename@area\the\toks@}%
    }%
  \fi
}%
\def\ZREF@xr@process@zreflabel\zref@newlabel#1#2#3\ZREF@nil{%
  \edef\ZREF@xr@refname{Z@R@\ZREF@xr@prefix#1}%
  \edef\ZREF@xr@found{\the\numexpr\ZREF@xr@found+1\relax}%
  \def\x{#2}%
  \edef\ZREF@xr@tempname{$temp$}%
  \edef\ZREF@xr@temprefname{Z@R@\ZREF@xr@tempname}%
  \let\ZREF@xr@list\x
  \ifx\ZREF@xr@list\ltx@empty
    \PackageWarningNoLine{zref-xr}{%
      Label `#1' without properties ignored\MessageBreak
      in file `\ZREF@xr@file'%
    }%
    \edef\ZREF@xr@ignored@empty{%
      \the\numexpr\ZREF@xr@ignored@empty+1\relax
    }%
  \else
    \expandafter\ZREF@xr@checklist\x\ZREF@nil
    \expandafter\let\csname\ZREF@xr@temprefname\endcsname\x
    \expandafter\ltx@LocalAppendToMacro
    \csname\ZREF@xr@temprefname\expandafter\endcsname
    \expandafter{%
      \expandafter\externaldocument\expandafter{%
        \ZREF@xr@externalfile
      }%
    }%
    \ZREF@xr@urlcheck\ZREF@xr@tempname
    \ifZREF@xr@tozreflabel
      \@ifundefined{\ZREF@xr@refname}{%
        \ifZREF@xr@verbose
          \PackageInfo{zref-xr}{%
            Import to zref label `\ZREF@xr@tempname#1'%
          }%
        \fi
        \global\expandafter
        \let\csname\ZREF@xr@refname\expandafter\endcsname
        \csname\ZREF@xr@temprefname\endcsname
      }{%
        \ZREF@xr@zref@ignorewarning{\ZREF@xr@prefix#1}%
      }%
    \fi
    \ifZREF@xr@toltxlabel
      \ZREF@xr@tolabel{\ZREF@xr@tempname}{\ZREF@xr@prefix#1}%
    \fi
  \fi
}%
\def\ZREF@xr@process@label\newlabel#1#2#3[#4]#5\ZREF@nil{%
  \def\ZREF@xr@refname{Z@R@\ZREF@xr@prefix#1}%
  \edef\ZREF@xr@found{\the\numexpr\ZREF@xr@found+1\relax}%
  \def\x{#2}%
  \edef\ZREF@xr@tempname{$temp$}%
  \edef\ZREF@xr@temprefname{Z@R@\ZREF@xr@tempname}%
  \expandafter\ZREF@xr@scanparams
      \csname\ZREF@xr@temprefname\expandafter\endcsname
      \x{}{}{}{}{}\ZREF@nil
  \ifx\\#4\\%
  \else
    % ntheorem knows an optional argument at the end of \newlabel
    \ZREF@NewPropTheotype
    \expandafter\ltx@LocalAppendToMacro
        \csname\ZREF@xr@temprefname\endcsname{\theotype{#4}}%
  \fi
  \expandafter\ltx@LocalAppendToMacro
  \csname\ZREF@xr@temprefname\expandafter\endcsname\expandafter{%
    \expandafter\externaldocument\expandafter{%
      \ZREF@xr@externalfile
    }%
  }%
  \ZREF@xr@urlcheck\ZREF@xr@tempname
  \ifZREF@xr@tozreflabel
    \@ifundefined{\ZREF@xr@refname}{%
      \ifZREF@xr@verbose
        \PackageInfo{zref-xr}{%
          Import to zref label `\ZREF@xr@prefix#1'%
        }%
      \fi
      \global\expandafter
      \let\csname\ZREF@xr@refname\expandafter\endcsname
      \csname\ZREF@xr@temprefname\endcsname
    }{%
      \ZREF@xr@zref@ignorewarning{\ZREF@xr@prefix#1}%
    }%
  \fi
  \ifZREF@xr@toltxlabel
    \ZREF@xr@tolabel{\ZREF@xr@tempname}{\ZREF@xr@prefix#1}%
  \fi
}
\def\ZREF@xr@zref@newlabel{\zref@newlabel}%
\def\ZREF@xr@newlabel{\newlabel}%
\def\ZREF@xr@@input{\@input}%
\def\ZREF@xr@relax{\relax}%
%    \end{macrocode}
%    \end{macro}
%    \begin{macro}{\ZREF@xr@tolabel}
%    \begin{macrocode}
\def\ZREF@xr@tolabel#1#2{%
  \ifZREF@xr@verbose
    \PackageInfo{zref-xr}{%
      Import to LaTeX label `#2'%
    }%
  \fi
  \zref@wrapper@unexpanded{%
    \expandafter\xdef\csname r@#2\endcsname{%
      {%
        \ltx@ifundefined{M@TitleReference}{%
          \ltx@ifundefined{TR@TitleReference}{%
            \zref@extractdefault{#1}{default}{}%
          }{%
            \noexpand\TR@TitleReference
            {\zref@extractdefault{#1}{default}{}}%
            {\zref@extractdefault{#1}{title}{}}%
          }%
        }{%
          \noexpand\M@TitleReference
          {\zref@extractdefault{#1}{default}{}}%
          {\zref@extractdefault{#1}{title}{}}%
        }%
      }%
      {\zref@extractdefault{#1}{page}{}}%
      \ltx@ifpackageloaded{nameref}{%
        {\zref@extractdefault{#1}{title}{}}%
        {\zref@extractdefault{#1}{anchor}{}}%
        \zref@ifrefcontainsprop{#1}{urluse}{%
          {\zref@extractdefault{#1}{urluse}{}}%
        }{%
          {\zref@extractdefault{#1}{url}{}}%
        }%
      }{}%
    }%
  }%
}
%    \end{macrocode}
%    \end{macro}
%
%    \begin{macro}{\ZREF@xr@zref@ignorewarning}
%    \begin{macrocode}
\def\ZREF@xr@zref@ignorewarning#1{%
  \PackageWarningNoLine{zref-xr}{%
    Zref label `#1' is already in use\MessageBreak
    in file `\ZREF@xr@file'%
  }%
  \edef\ZREF@xr@ignored@zref{%
    \the\numexpr\ZREF@xr@ignored@zref+1%
  }%
}%
%    \end{macrocode}
%    \end{macro}
%
%    \begin{macro}{\ZREF@xr@ltx@ignorewarning}
%    \begin{macrocode}
\def\ZREF@xr@ltx@ignorewarning#1{%
  \PackageWarningNoLine{zref-xr}{%
    LaTeX label `#1' is already in use\MessageBreak
    in file `\ZREF@xr@file'%
  }%
  \edef\ZREF@xr@ignored@ltx{%
    \the\numexpr\ZREF@xr@ignored@ltx+1%
  }%
}%
%    \end{macrocode}
%    \end{macro}
%
%    \begin{macro}{\ZREF@xr@checklist}
%    \begin{macrocode}
\def\ZREF@xr@checklist#1#2#3\ZREF@nil{%
  \ifx\@undefined#1\relax
    \expandafter\ZREF@xr@checkkey\string#1\@nil
  \fi
  \ifx\\#3\\%
  \else
    \ltx@ReturnAfterFi{%
      \ZREF@xr@checklist#3\ZREF@nil
    }%
  \fi
}%
\def\ZREF@xr@checkkey#1#2\@nil{%
  \zref@ifpropundefined{#2}{%
    \zref@newprop{#2}{}%
  }{}%
}%
%    \end{macrocode}
%    \end{macro}
%
%    \begin{macro}{\ZREF@xr@scanparams}
%    \begin{macrocode}
\def\ZREF@xr@scanparams#1#2#3#4#5#6#7\ZREF@nil{%
  \let#1\ltx@empty
  \ZREF@foundfalse
  \ZREF@xr@scantitleref#1#2\TR@TitleReference{}{}\ZREF@nil
  \ifZREF@found
  \else
    \ltx@LocalAppendToMacro#1{\default{#2}}%
  \fi
  % page
  \ltx@LocalAppendToMacro#1{\page{#3}}%
  % nameref title
  \ifZREF@found
  \else
    \ifx\\#4\\%
    \else
      \def\ZREF@xr@temp{#4}%
      \ifx\ZREF@xr@temp\ZREF@xr@relax
      \else
        \ltx@LocalAppendToMacro#1{\title{#4}}%
      \fi
    \fi
  \fi
  % anchor
  \ifx\\#5\\%
  \else
    \ltx@LocalAppendToMacro#1{\anchor{#5}}%
  \fi
  \ifx\\#6\\%
  \else
    \ifZREF@xr@urluse
      \ZREF@xr@AddURL{#6}%
      \expandafter\ltx@LocalAppendToMacro\expandafter#1%
      \expandafter{%
        \expandafter\urluse\expandafter{%
          \csname Z@U@\ZREF@xr@theURL\endcsname
        }%
      }%
    \else
      \ltx@LocalAppendToMacro#1{\url{#6}}%
    \fi
  \fi
}%
%    \end{macrocode}
%    \end{macro}
%
%    \begin{macro}{\ZREF@xr@scantitleref}
%    \begin{macrocode}
\def\ZREF@xr@scantitleref#1#2\TR@TitleReference#3#4#5\ZREF@nil{%
  \ifx\\#5\\%
  \else
    \ltx@LocalAppendToMacro#1{%
      \default{#3}%
      \title{#4}%
    }%
    \ZREF@foundtrue
  \fi
}%
%    \end{macrocode}
%    \end{macro}
%
%    \begin{macro}{\ZREF@xr@urlcheck}%
%    \begin{macrocode}
\def\ZREF@xr@urlcheck#1{%
  \zref@ifrefcontainsprop{#1}{anchor}{%
    \zref@ifrefcontainsprop{#1}{url}{%
    }{%
      \expandafter
      \ltx@LocalAppendToMacro\csname Z@R@#1\expandafter\endcsname
      \expandafter{%
        \csname url\ifZREF@xr@urluse use\fi
        \expandafter\endcsname\expandafter{\ZREF@xr@url}%
      }%
    }%
  }{%
  }%
}%
%    \end{macrocode}
%    \end{macro}
%
%    \begin{macrocode}
%</xr>
%    \end{macrocode}
%
%
% \subsection{Module \xmodule{hyperref}}
%
%    UNFINISHED :-(
%
%    \begin{macrocode}
%<*hyperref>
\NeedsTeXFormat{LaTeX2e}
\ProvidesPackage{zref-hyperref}%
  [2020-07-03 v2.32 Module hyperref for zref (HO)]%
\RequirePackage{zref-base}[2019/11/29]
\ifx\ZREF@base@ok Y%
\else
  \expandafter\endinput
\fi
%    \end{macrocode}
%
%    \begin{macrocode}
\ZREF@NewPropAnchor
\zref@addprop\ZREF@mainlist{anchor}%
%    \end{macrocode}
%
%    \begin{macrocode}
%</hyperref>
%    \end{macrocode}
%
%
% \subsection{Module \xmodule{savepos}}
%
%    Module \xmodule{savepos} provides an interface for
%    \pdfTeX's \cmd{\pdfsavepos}, see the manual for
%    \pdfTeX.
%
% \subsubsection{Identification}
%
%    \begin{macrocode}
%<*savepos>
\NeedsTeXFormat{LaTeX2e}
\ProvidesPackage{zref-savepos}%
  [2020-07-03 v2.32 Module savepos for zref (HO)]%
\RequirePackage{zref-base}[2019/11/29]
\ifx\ZREF@base@ok Y%
\else
  \expandafter\endinput
\fi
%    \end{macrocode}
%
% \subsubsection{Availability}
%
%    First we check, whether the feature is available.
%    \begin{macrocode}
\ifx\directlua\@undefined
\ltx@IfUndefined{pdfsavepos}{%
  \PackageError\ZREF@name{%
    \string\pdfsavepos\space is not supported.\MessageBreak
    It is provided by pdfTeX (1.40) or XeTeX%
  }\ZREF@UpdatePdfTeX
  \endinput
}{}%
\fi
%    \end{macrocode}
%    In PDF mode we are done. However support for DVI mode
%    was added later in version 1.40.0. In earlier versions
%    \cs{pdfsavepos} is defined, but its execution raises
%    an error. Note that \hologo{XeTeX} also provides \cs{pdfsavepos}.
%    \begin{macrocode}
\ifpdf
\else
  \ltx@IfUndefined{pdftexversion}{%
  }{%
    \ifnum\pdftexversion<140 %
      \PackageError\ZREF@name{%
        \string\pdfsavepos\space is not supported in DVI mode%
        \MessageBreak
        of this pdfTeX version%
      }\ZREF@UpdatePdfTeX
      \expandafter\expandafter\expandafter\endinput
    \fi
  }%
\fi
%    \end{macrocode}
%
% \subsubsection{Setup}
%
%    \begin{macrocode}
\zref@newlist{savepos}
\ifx\directlua\@undefined
  \zref@newprop*{posx}[0]{\the\pdflastxpos}
  \zref@newprop*{posy}[0]{\the\pdflastypos}
\else
  \zref@newprop*{posx}[0]{\the\lastxpos}
  \zref@newprop*{posy}[0]{\the\lastypos}
\fi
\zref@addprops{savepos}{posx,posy}
%    \end{macrocode}
%
% \subsubsection{User macros}
%
%    \begin{macro}{\zref@savepos}
%    \begin{macrocode}
\ifx\directlua\@undefined
  \def\zref@savepos{%
    \if@filesw
      \pdfsavepos
    \fi
  }
\else
  \def\zref@savepos{%
    \if@filesw
      \savepos
    \fi
  }
\fi
%    \end{macrocode}
%    \end{macro}
%
%    \begin{macro}{\ZREF@zsavepos}
%    \begin{macrocode}
\def\ZREF@zsavepos#1#2#3{%
  \@bsphack
  \if@filesw
    \zref@savepos
    #1{#3}{#2}%
    \ltx@IfUndefined{TeXXeTstate}{%
    }{%
      \ifnum\TeXXeTstate=\ltx@zero
      \else
        \zref@savepos
      \fi
    }%
  \fi
  \@esphack
}
%    \end{macrocode}
%    \end{macro}
%
%    \begin{macro}{\zsavepos}
%    The current location is stored in a reference with the
%    given name.
%    \begin{macrocode}
\ZREF@IfDefinable\zsavepos\def{%
  {%
    \ZREF@zsavepos\zref@labelbylist{savepos}%
  }%
}
%    \end{macrocode}
%    \end{macro}
%
%    \begin{macro}{\zsaveposx}
%    \begin{macrocode}
\ZREF@IfDefinable\zsaveposx\def{%
  {%
    \ZREF@zsavepos\zref@labelbyprops{posx}%
  }%
}
%    \end{macrocode}
%    \end{macro}
%    \begin{macro}{\zsaveposy}
%    \begin{macrocode}
\ZREF@IfDefinable\zsaveposy\def{%
  {%
    \ZREF@zsavepos\zref@labelbyprops{posy}%
  }%
}
%    \end{macrocode}
%    \end{macro}
%
%    \begin{macro}{\zposx}
%    \begin{macro}{\zposy}
%    The horizontal and vertical position are available by
%    \cs{zposx} and \cs{zposy}. Do not rely on absolute
%    positions. They differ in DVI and PDF mode of \pdfTeX.
%    Use differences instead. The unit of the position numbers
%    is sp.
%    \begin{macrocode}
\newcommand*{\zposx}[1]{%
  \zref@extract{#1}{posx}%
}%
\newcommand*{\zposy}[1]{%
  \zref@extract{#1}{posy}%
}%
%    \end{macrocode}
%    \end{macro}
%    \end{macro}
%
%    Typically horizontal and vertical positions are used
%    inside calculations. Therefore the extracting macros
%    should be expandable and babel's patch is not applyable.
%
%    Also it is in the responsibility of the user to marked
%    used positions by \cs{zrefused}
%    in order to notify \hologo{LaTeX} about undefined references.
%
%    \begin{macro}{\ZREF@savepos@ok}
%    \begin{macrocode}
\let\ZREF@savepos@ok=Y
%    \end{macrocode}
%    \end{macro}
%
%    \begin{macrocode}
%</savepos>
%    \end{macrocode}
%
%
% \subsection{Module \xmodule{abspos}}
%
% \subsubsection{Identification}
%
%    \begin{macrocode}
%<*abspos>
\NeedsTeXFormat{LaTeX2e}
\ProvidesPackage{zref-abspos}%
  [2020-07-03 v2.32 Module abspos for zref (HO)]%
\RequirePackage{zref-base}[2019/11/29]
\ifx\ZREF@base@ok Y%
\else
  \expandafter\endinput
\fi
%    \end{macrocode}
%
%    \begin{macrocode}
\RequirePackage{zref-savepos}[2019/11/29]
\ifx\ZREF@savepos@ok Y%
\else
  \expandafter\endinput
\fi
%    \end{macrocode}
%    \begin{macrocode}
\RequirePackage{zref-pagelayout}[2019/11/29]
\zref@addprop{savepos}{abspage}
\zref@addprop{savepos}{pagevalue}
%    \end{macrocode}
%
%    \begin{macro}{\zref@absposx}
%    \begin{macrocode}
\newcommand*{\zref@absposx}[3]{%
  \number
    \expandafter\zref@absposnumx\expandafter{%
      \number\zref@extractdefault{#1}{abspage}{0}%
    }{#2}{#3}%
  \ltx@space
}
%    \end{macrocode}
%    \end{macro}
%
%    \begin{macro}{\zref@absposy}
%    \begin{macrocode}
\newcommand*{\zref@absposy}[3]{%
  \number
  \expandafter\zref@absposnumy\expandafter{%
    \number\zref@extractdefault{#1}{abspage}{0}%
  }{#2}{#3}%
  \ltx@space
}
%    \end{macrocode}
%    \end{macro}
%
%    \begin{macro}{\zref@absposnumx}
%    \begin{macrocode}
\newcommand*{\zref@absposnumx}[3]{%
  \number
%  \ifnum#1>\ltx@zero
%    \zref@ifrefundefined{thepage#1}{%
%      0%
%    }{%
      \numexpr\ZREF@absposnum{thepage#1}{#2}{x}{#3}\relax
%    }%
%  \else
%    0%
%  \fi
}
%    \end{macrocode}
%    \end{macro}
%
%    \begin{macro}{\zref@absposnumy}
%    \begin{macrocode}
\newcommand*{\zref@absposnumy}[3]{%
  \number
%  \ifnum#1>\ltx@zero
%    \zref@ifrefundefined{thepage#1}{%
%      0%
%    }{%
      \numexpr\ZREF@absposnum{thepage#1}{#2}{y}{#3}\relax
%    }%
%  \else
%    0%
%  \fi
}
%    \end{macrocode}
%    \end{macro}
%
%    \begin{macro}{\ZREF@absposnum}
%    \begin{macrocode}
\def\ZREF@absposnum#1#2#3#4{%
  \ltx@ifundefined{ZREF@abspos@#2@#3@#4}{%
    0%
  }{%
    \csname ZREF@abspos@#2@#3@#4\endcsname{#1}%
  }%
}
%    \end{macrocode}
%    \end{macro}
%
%    \begin{macro}{\zref@def@absposx}
%    \begin{macrocode}
\ZREF@Robust\def\zref@def@absposx#1{%
  \zref@wrapper@babel{\ZREF@def@abspos{#1}\zref@absposx}%
}
%    \end{macrocode}
%    \end{macro}
%
%    \begin{macro}{\zref@def@absposy}
%    \begin{macrocode}
\ZREF@Robust\def\zref@def@absposy#1{%
  \zref@wrapper@babel{\ZREF@def@abspos{#1}\zref@absposy}%
}
%    \end{macrocode}
%    \end{macro}
%
%    \begin{macro}{\zref@def@absposnumx}
%    \begin{macrocode}
\ZREF@Robust\def\zref@def@absposnumx#1{%
  \zref@wrapper@babel{\ZREF@def@abspos{#1}\zref@absposnumx}%
}
%    \end{macrocode}
%    \end{macro}
%
%    \begin{macro}{\zref@def@absposnumy}
%    \begin{macrocode}
\ZREF@Robust\def\zref@def@absposnumy#1{%
  \zref@wrapper@babel{\ZREF@def@abspos{#1}\zref@absposnumy}%
}
%    \end{macrocode}
%    \end{macro}
%    \begin{macro}{\ZREF@def@abspos}
%    \begin{macrocode}
\def\ZREF@def@absposnumy#1#2#3#4#5{%
  \edef#1{#2{#3}{#4}{#5}}%
}
%    \end{macrocode}
%    \end{macro}
%
%    \begin{macro}{\zref@absposused}
%    \begin{macrocode}
\ZREF@Robust\def\zref@absposused{%
  \zref@wrapper@babel\ZREF@abspos@used
}
%    \end{macrocode}
%    \end{macro}
%    \begin{macro}{\ZREF@abspos@used}
%    \begin{macrocode}
\def\ZREF@abspos@used#1{%
  \zref@refused{#1}%
  \zref@ifrefundefined{#1}{%
  }{%
    \begingroup
      \edef\ZREF@temp{%
        \zref@extractdefault{#1}{abspage}{0}%
      }%
      \ifnum\ZREF@temp>\ltx@zero
        \zref@refused{thepage\ZREF@temp}%
      \else
        \@PackageError{zref-abspos}{%
          \string\zref@pos@label@used\ltx@space
          needs property `abspage'\MessageBreak
          in label `#1'%
        }\@ehc
      \fi
    \endgroup
  }%
}
%    \end{macrocode}
%    \end{macro}
%
%    \begin{macro}{\zref@absposnumused}
%    \begin{macrocode}
\newcommand*{\zref@absposnumused}[1]{%
  \ifnum#1>\ltx@zero
    \zref@refused{thepage\number#1}%
  \else
    \@PackageError{zref-abspos}{%
      Invalid absolute page number (#1)\MessageBreak
      for \string\zref@pos@num@used.\MessageBreak
      A positive integer is expected%
    }\@ehc
  \fi
}
%    \end{macrocode}
%    \end{macro}
%
%    \begin{macro}{\zref@ifabsposundefined}
%    \begin{macrocode}
\def\zref@ifabsposundefined#1{%
  \zref@ifrefundefined{#1}\ltx@firsttwo{%
    \expandafter\zref@ifabsposnumundefined\expandafter{%
      \number\zref@extractdefault{#1}{abspage}{0}%
    }%
  }%
}
%    \end{macrocode}
%    \end{macro}
%
%    \begin{macro}{\zref@ifabsposnumundefined}
%    \begin{macrocode}
\def\zref@ifabsposnumundefined#1{%
  \ifnum\ZREF@number{#1}>\ltx@zero
    \zref@ifrefundefined{thepage#1}%
        \ltx@firstoftwo\ltx@secondoftwo
  \else
    \expandafter\ltx@firstoftwo
  \fi
}
%    \end{macrocode}
%    \end{macro}
%
% \subsubsection{Media}
%
%    \begin{macro}{\ZREF@abspos@media@width}
%    \begin{macrocode}
\edef\ZREF@abspos@media@width{%
  \ltx@ifundefined{pdfpagewidth}{%
    \ltx@ifundefined{mediawidth}{%
      \ltx@ifundefined{stockwidth}{%
        paperwidth%
      }{%
        stockwidth%
      }%
    }{%
      mediawidth%
    }%
  }{%
    pdfpagewidth%
  }%
}
\ifluatex
\def\ZREF@abspos@media@width{pdfpagewidth}%
\fi
%    \end{macrocode}
%    \end{macro}
%    \begin{macro}{\ZREF@abspos@media@height}
%    \begin{macrocode}
\edef\ZREF@abspos@media@height{%
  \ltx@ifundefined{pdfpageheight}{%
    \ltx@ifundefined{mediaheight}{%
      \ltx@ifundefined{stockheight}{%
        paperheight%
      }{%
        stockheight%
      }%
    }{%
      mediaheight%
    }%
  }{%
    \noexpand\ifcase\pdfpageheight
      \ltx@ifundefined{stockheight}{%
        paperheight%
      }{%
        stockheight%
      }%
    \noexpand\else
      pdfpageheight%
    \noexpand\fi
  }%
}
\ifluatex
\edef\ZREF@abspos@media@height{%
   \noexpand\ifcase\pageheight
      \ltx@ifundefined{stockheight}{%
        paperheight%
      }{%
        stockheight%
      }%
    \noexpand\else
      pdfpageheight%
    \noexpand\fi}%
\fi
%    \end{macrocode}
%    \end{macro}
%
%    \begin{macro}{\ZREF@abspos@media@x@left}
%    \begin{macrocode}
\def\ZREF@abspos@media@x@left#1{%
  0%
}
%    \end{macrocode}
%    \end{macro}
%    \begin{macro}{\ZREF@abspos@media@x@right}
%    \begin{macrocode}
\def\ZREF@abspos@media@x@right#1{%
  \zref@extract{#1}\ZREF@abspos@media@width
}
%    \end{macrocode}
%    \end{macro}
%    \begin{macro}{\ZREF@abspos@media@x@center}
%    \begin{macrocode}
\def\ZREF@abspos@media@x@center#1{%
  \ZREF@abspos@media@x@left{#1}%
  +\zref@extract{#1}\ZREF@abspos@media@width/2%
}
%    \end{macrocode}
%    \end{macro}
%
%    \begin{macro}{\ZREF@abspos@media@y@top}
%    \begin{macrocode}
\def\ZREF@abspos@media@y@top#1{%
  \zref@extract{#1}\ZREF@abspos@media@height
}
%    \end{macrocode}
%    \end{macro}
%    \begin{macro}{\ZREF@abspos@media@y@bottom}
%    \begin{macrocode}
\def\ZREF@abspos@media@y@bottom#1{%
  0%
}
%    \end{macrocode}
%    \end{macro}
%    \begin{macro}{\ZREF@abspos@media@y@center}
%    \begin{macrocode}
\def\ZREF@abspos@media@y@center#1{%
  \zref@extract{#1}\ZREF@abspos@media@height/2%
}
%    \end{macrocode}
%    \end{macro}
%
% \subsubsection{Paper}
%
%    \begin{macro}{\ZREF@abspos@paper@x@left}
%    \begin{macrocode}
\def\ZREF@abspos@paper@x@left#1{%
  0%
}
%    \end{macrocode}
%    \end{macro}
%    \begin{macro}{\ZREF@abspos@paper@x@right}
%    \begin{macrocode}
\def\ZREF@abspos@paper@x@right#1{%
  \zref@extract{#1}{paperwidth}%
}
%    \end{macrocode}
%    \end{macro}
%    \begin{macro}{\ZREF@abspos@paper@x@center}
%    \begin{macrocode}
\def\ZREF@abspos@paper@x@center#1{%
  \zref@extract{#1}{paperwidth}/2%
}
%    \end{macrocode}
%    \end{macro}
%
%    \begin{macro}{\ZREF@abspos@paper@y@top}
%    \begin{macrocode}
\let\ZREF@abspos@paper@y@top\ZREF@abspos@media@y@top
%    \end{macrocode}
%    \end{macro}
%    \begin{macro}{\ZREF@abspos@paper@y@bottom}
%    \begin{macrocode}
\def\ZREF@abspos@paper@y@bottom#1{%
  \ZREF@abspos@paper@y@top{#1}%
  -\zref@extract{#1}{paperheight}%
}
%    \end{macrocode}
%    \end{macro}
%    \begin{macro}{\ZREF@abspos@paper@y@center}
%    \begin{macrocode}
\def\ZREF@abspos@paper@y@center#1{%
  \ZREF@abspos@paper@y@top{#1}%
  -\zref@extract{#1}{paperheight}/2%
}
%    \end{macrocode}
%    \end{macro}
%
% \subsubsection{Origin}
%
% There doesn't seem a good reason to make these tests depend on pdf mode
%  in current engines, so comment out the \verb|\ifpdf| tests.
%    \begin{macro}{\ZREF@abspos@origin@x}
%    \begin{macrocode}
\let\ZREF@temp\ltx@two
\ltx@IfUndefined{pdfhorigin}{}{%
%  \ifpdf
    \let\ZREF@temp\ltx@zero
%  \fi
}
\ifluatex
% \ifpdf
  \let\ZREF@temp\ltx@zero
% \fi
\fi

\ifx\ZREF@temp\ltx@two
  \ifnum\mag=1000 %
    \let\ZREF@temp\ltx@one
  \fi
\fi
\ifcase\ZREF@temp
  \def\ZREF@abspos@origin@x#1{%
    \zref@extract{#1}{pdfhorigin}%
  }%
\or
  \def\ZREF@abspos@origin@x#1{%
    4736286%
  }%
\or
  \def\ZREF@abspos@origin@x#1{%
    \numexpr\mag/1000*\dimexpr 1truein\relax\relax
  }%
\fi
%    \end{macrocode}
%    \end{macro}
%    \begin{macro}{\ZREF@abspos@origin@y}
%    \begin{macrocode}
\let\ZREF@temp\ltx@two
\ltx@IfUndefined{pdfvorigin}{}{%
%  \ifpdf
    \let\ZREF@temp\ltx@zero
%  \fi
}
\ifluatex
%  \ifpdf
    \let\ZREF@temp\ltx@zero
%  \fi
\fi
\ifx\ZREF@temp\ltx@two
  \ifnum\mag=1000 %
    \let\ZREF@temp\ltx@one
  \fi
\fi
\ifcase\ZREF@temp
  \def\ZREF@abspos@origin@y#1{%
    \zref@extract{#1}{pdfvorigin}%
  }%
\or
  \def\ZREF@abspos@origin@y#1{%
    4736286%
  }%
\or
  \def\ZREF@abspos@origin@y#1{%
    \numexpr\mag/1000*\dimexpr 1truein\relax\relax
   }%
\fi
%    \end{macrocode}
%    \end{macro}
%
% \subsubsection{Header}
%
%    \begin{macro}{\ZREF@abspos@head@x@left}
%    \begin{macrocode}
\def\ZREF@abspos@head@x@left#1{%
  \ZREF@abspos@paper@x@left{#1}%
  +\ZREF@abspos@origin@x{#1}%
  +\zref@extract{#1}{hoffset}%
  +\ifodd\zref@extractdefault{#1}{pagevalue}{\number\c@page} %
     \zref@extract{#1}{oddsidemargin}%
   \else
     \zref@extract{#1}{evensidemargin}%
   \fi
}
%    \end{macrocode}
%    \end{macro}
%    \begin{macro}{\ZREF@abspos@head@x@right}
%    \begin{macrocode}
\def\ZREF@abspos@head@x@right#1{%
  \ZREF@abspos@head@x@left{#1}%
  +\zref@extract{#1}{textwidth}%
}
%    \end{macrocode}
%    \end{macro}
%    \begin{macro}{\ZREF@abspos@head@x@center}
%    \begin{macrocode}
\def\ZREF@abspos@head@x@center#1{%
  \ZREF@abspos@head@x@left{#1}%
  +\zref@extract{#1}{textwidth}/2%
}
%    \end{macrocode}
%    \end{macro}
%
%    \begin{macro}{\ZREF@abspos@head@y@top}
%    \begin{macrocode}
\def\ZREF@abspos@head@y@top#1{%
  \ZREF@abspos@paper@y@top{#1}%
  -\ZREF@abspos@origin@y{#1}%
  -\zref@extract{#1}{voffset}%
  -\zref@extract{#1}{topmargin}%
}
%    \end{macrocode}
%    \end{macro}
%    \begin{macro}{\ZREF@abspos@head@y@bottom}
%    \begin{macrocode}
\def\ZREF@abspos@head@y@bottom#1{%
  \ZREF@abspos@head@y@top{#1}%
  -\zref@extract{#1}{headheight}%
}
%    \end{macrocode}
%    \end{macro}
%    \begin{macro}{\ZREF@abspos@head@y@center}
%    \begin{macrocode}
\def\ZREF@abspos@head@y@center#1{%
  \ZREF@abspos@head@y@top{#1}%
  -\zref@extract{#1}{headheight}/2%
}
%    \end{macrocode}
%    \end{macro}
%
% \subsubsection{Body}
%
%    \begin{macro}{\ZREF@abspos@body@x@left}
%    \begin{macrocode}
\let\ZREF@abspos@body@x@left\ZREF@abspos@head@x@left
%    \end{macrocode}
%    \end{macro}
%    \begin{macro}{\ZREF@abspos@body@x@right}
%    \begin{macrocode}
\let\ZREF@abspos@body@x@right\ZREF@abspos@head@x@right
%    \end{macrocode}
%    \end{macro}
%    \begin{macro}{\ZREF@abspos@body@x@center}
%    \begin{macrocode}
\let\ZREF@abspos@body@x@center\ZREF@abspos@head@x@center
%    \end{macrocode}
%    \end{macro}
%
%    \begin{macro}{\ZREF@abspos@body@y@top}
%    \begin{macrocode}
\def\ZREF@abspos@body@y@top#1{%
  \ZREF@abspos@head@y@bottom{#1}%
  -\zref@extract{#1}{headsep}%
}
%    \end{macrocode}
%    \end{macro}
%    \begin{macro}{\ZREF@abspos@body@y@bottom}
%    \begin{macrocode}
\def\ZREF@abspos@body@y@bottom#1{%
  \ZREF@abspos@body@y@top{#1}%
  -\zref@extract{#1}{textheight}%
}
%    \end{macrocode}
%    \end{macro}
%    \begin{macro}{\ZREF@abspos@body@y@center}
%    \begin{macrocode}
\def\ZREF@abspos@body@y@center#1{%
  \ZREF@abspos@body@y@top{#1}%
  -\zref@extract{#1}{textheight}/2%
}
%    \end{macrocode}
%    \end{macro}
%
% \subsubsection{Footer}
%
%    \begin{macro}{\ZREF@abspos@foot@x@left}
%    \begin{macrocode}
\let\ZREF@abspos@foot@x@left\ZREF@abspos@head@x@left
%    \end{macrocode}
%    \end{macro}
%    \begin{macro}{\ZREF@abspos@foot@x@right}
%    \begin{macrocode}
\let\ZREF@abspos@foot@x@right\ZREF@abspos@head@x@right
%    \end{macrocode}
%    \end{macro}
%    \begin{macro}{\ZREF@abspos@foot@x@center}
%    \begin{macrocode}
\let\ZREF@abspos@foot@x@center\ZREF@abspos@head@x@center
%    \end{macrocode}
%    \end{macro}
%
%    \begin{macro}{\ZREF@abspos@foot@y@bottom}
%    \begin{macrocode}
\def\ZREF@abspos@foot@y@bottom#1{%
  \ZREF@abspos@body@y@bottom{#1}%
  -\zref@extract{#1}{footskip}%
}
%    \end{macrocode}
%    \end{macro}
%
% \subsubsection{Marginal notes}
%
%    \begin{macro}{\ZREF@abspos@marginpar@x@left}
%    \begin{macrocode}
\def\ZREF@abspos@marginpar@x@left#1{%
  \ifodd\zref@extractdefault{#1}{pagevalue}{\number\c@page} %
    \ZREF@abspos@body@x@right{#1}%
    +\zref@extract{#1}{marginparsep}%
  \else
    \ZREF@abspos@body@x@left{#1}%
    -\zref@extract{#1}{marginparsep}%
    -\zref@extract{#1}{marginparwidth}%
  \fi
}
%    \end{macrocode}
%    \end{macro}
%    \begin{macro}{\ZREF@abspos@marginpar@x@right}
%    \begin{macrocode}
\def\ZREF@abspos@marginpar@x@right#1{%
  \ZREF@abspos@marginpar@x@left{#1}%
  +\zref@extract{#1}{marginparwidth}%
}
%    \end{macrocode}
%    \end{macro}
%    \begin{macro}{\ZREF@abspos@marginpar@x@center}
%    \begin{macrocode}
\def\ZREF@abspos@marginpar@x@center#1{%
  \ZREF@abspos@marginpar@x@left{#1}%
  +\zref@extract{#1}{marginparwidth}/2%
}
%    \end{macrocode}
%    \end{macro}
%
%    \begin{macro}{\ZREF@abspos@marginpar@y@top}
%    \begin{macrocode}
\let\ZREF@abspos@marginpar@y@top\ZREF@abspos@body@y@top
%    \end{macrocode}
%    \end{macro}
%    \begin{macro}{\ZREF@abspos@marginpar@y@bottom}
%    \begin{macrocode}
\let\ZREF@abspos@marginpar@y@bottom\ZREF@abspos@body@y@bottom
%    \end{macrocode}
%    \end{macro}
%    \begin{macro}{\ZREF@abspos@marginpar@y@center}
%    \begin{macrocode}
\let\ZREF@abspos@marginpar@y@center\ZREF@abspos@body@y@center
%    \end{macrocode}
%    \end{macro}
%
% \subsubsection{Stock paper}
%
%    \begin{macro}{\ZREF@abspos@stock@x@left}
%    \begin{macrocode}
\let\ZREF@abspos@stock@x@left\ZREF@abspos@paper@x@left
%    \end{macrocode}
%    \end{macro}
%    \begin{macro}{\ZREF@abspos@stock@x@right}
%    \begin{macrocode}
\let\ZREF@abspos@stock@x@right\ZREF@abspos@paper@x@right
%    \end{macrocode}
%    \end{macro}
%    \begin{macro}{\ZREF@abspos@stock@x@center}
%    \begin{macrocode}
\let\ZREF@abspos@stock@x@center\ZREF@abspos@paper@x@center
%    \end{macrocode}
%    \end{macro}
%
%    \begin{macro}{\ZREF@abspos@stock@y@top}
%    \begin{macrocode}
\let\ZREF@abspos@stock@y@top\ZREF@abspos@paper@y@top
%    \end{macrocode}
%    \end{macro}
%    \begin{macro}{\ZREF@abspos@stock@y@bottom}
%    \begin{macrocode}
\let\ZREF@abspos@stock@y@bottom\ZREF@abspos@paper@y@bottom
%    \end{macrocode}
%    \end{macro}
%    \begin{macro}{\ZREF@abspos@stock@y@center}
%    \begin{macrocode}
\let\ZREF@abspos@stock@y@center\ZREF@abspos@paper@y@center
%    \end{macrocode}
%    \end{macro}
%
%    \begin{macrocode}
%</abspos>
%    \end{macrocode}
%
%
% \subsection{Module \xmodule{dotfill}}
%
%    \begin{macrocode}
%<*dotfill>
\NeedsTeXFormat{LaTeX2e}
\ProvidesPackage{zref-dotfill}%
  [2020-07-03 v2.32 Module dotfill for zref (HO)]%
\RequirePackage{zref-base}[2019/11/29]
\ifx\ZREF@base@ok Y%
\else
  \expandafter\endinput
\fi
%    \end{macrocode}
%
%    For measuring the width of \cs{zdotfill} we use the features
%    provided by module \xmodule{savepos}.
%    \begin{macrocode}
\RequirePackage{zref-savepos}[2019/11/29]
%    \end{macrocode}
%    For automatically generated label names we use the unique
%    counter of module \xmodule{base}.
%    \begin{macrocode}
\zref@require@unique
%    \end{macrocode}
%    Configuration is done by the key value interface of package
%    \xpackage{keyval}.
%    \begin{macrocode}
\RequirePackage{keyval}
%    \end{macrocode}
%    The definitions of the keys follow.
%    \begin{macrocode}
\define@key{ZREF@DF}{unit}{%
  \def\ZREF@df@unit{#1}%
}
\define@key{ZREF@DF}{min}{%
  \def\ZREF@df@min{#1}%
}
\define@key{ZREF@DF}{dot}{%
  \def\ZREF@df@dot{#1}%
}
%    \end{macrocode}
%    Defaults are set, see user interface.
%    \begin{macrocode}
\providecommand\ZREF@df@min{2}
\providecommand\ZREF@df@unit{.44em}
\providecommand\ZREF@df@dot{.}
%    \end{macrocode}
%    \begin{macro}{\zdotfillsetup}
%    Configuration of \cs{zdotfill} is done by \cs{zdotfillsetup}.
%    \begin{macrocode}
\newcommand*{\zdotfillsetup}{\kvsetkeys{ZREF@DF}}
%    \end{macrocode}
%    \end{macro}
%
%    \begin{macro}{\zdotfill}
%    \cs{zdotfill} sets labels at the left and the right to get
%    the horizontal position. \cs{zsavepos} is not used, because
%    we do not need the vertical position.
%    \begin{macrocode}
\ZREF@IfDefinable\zdotfill\def{%
  {%
    \leavevmode
    \global\advance\c@zref@unique\ltx@one
    \begingroup
      \def\ZREF@temp{zref@\number\c@zref@unique}%
      \pdfsavepos
      \zref@labelbyprops{\thezref@unique L}{posx}%
      \setlength{\dimen@}{\ZREF@df@unit}%
      \zref@ifrefundefined{\thezref@unique R}{%
        \ZREF@dotfill
      }{%
        \ifnum\numexpr\zposx{\thezref@unique R}%
                      -\zposx{\thezref@unique L}\relax
            <\dimexpr\ZREF@df@min\dimen@\relax
          \hfill
        \else
          \ZREF@dotfill
        \fi
      }%
      \pdfsavepos
      \zref@labelbyprops{\thezref@unique R}{posx}%
    \endgroup
    \kern\z@
  }%
}
%    \end{macrocode}
%    \end{macro}
%    \begin{macro}{\ZREF@dotfill}
%    Help macro that actually sets the dots.
%    \begin{macrocode}
\def\ZREF@dotfill{%
  \cleaders\hb@xt@\dimen@{\hss\ZREF@df@dot\hss}\hfill
}
%    \end{macrocode}
%    \end{macro}
%
%    \begin{macrocode}
%</dotfill>
%    \end{macrocode}
%
% \subsection{Module \xmodule{env}}
%
%    \begin{macrocode}
%<*env>
\NeedsTeXFormat{LaTeX2e}
\ProvidesPackage{zref-env}%
  [2020-07-03 v2.32 Module env for zref (HO)]%
\RequirePackage{zref-base}[2019/11/29]
\ifx\ZREF@base@ok Y%
\else
  \expandafter\endinput
\fi
%    \end{macrocode}
%
%    \begin{macrocode}
\zref@newprop{envname}[]{\@currenvir}
\zref@newprop{envline}[]{\zref@env@line}
%    \end{macrocode}
%
%    \begin{macro}{\zref@env@line}
%    Macro \cs{zref@env@line} extracts the line number
%    from \cs{@currenvline}.
%    \begin{macrocode}
\def\zref@env@line{%
  \ifx\@currenvline\ltx@empty
  \else
    \expandafter
    \ZREF@ENV@line\@currenvline\ltx@empty line \ltx@empty\@nil
  \fi
}
%    \end{macrocode}
%    \end{macro}
%    \begin{macro}{\ZREF@ENV@line}
%    \begin{macrocode}
\def\ZREF@ENV@line#1line #2\ltx@empty#3\@nil{#2}%
%    \end{macrocode}
%    \end{macro}
%
%    \begin{macrocode}
%</env>
%    \end{macrocode}
%
%% \section{Installation}
%
% \subsection{Download}
%
% \paragraph{Package.} This package is available on
% CTAN\footnote{\CTANpkg{zref}}:
% \begin{description}
% \item[\CTAN{macros/latex/contrib/zref/zref.dtx}] The source file.
% \item[\CTAN{macros/latex/contrib/zref/zref.pdf}] Documentation.
% \end{description}
%
%
% \paragraph{Bundle.} All the packages of the bundle `zref'
% are also available in a TDS compliant ZIP archive. There
% the packages are already unpacked and the documentation files
% are generated. The files and directories obey the TDS standard.
% \begin{description}
% \item[\CTANinstall{install/macros/latex/contrib/zref.tds.zip}]
% \end{description}
% \emph{TDS} refers to the standard ``A Directory Structure
% for \TeX\ Files'' (\CTANpkg{tds}). Directories
% with \xfile{texmf} in their name are usually organized this way.
%
% \subsection{Bundle installation}
%
% \paragraph{Unpacking.} Unpack the \xfile{zref.tds.zip} in the
% TDS tree (also known as \xfile{texmf} tree) of your choice.
% Example (linux):
% \begin{quote}
%   |unzip zref.tds.zip -d ~/texmf|
% \end{quote}
%
%
% \subsection{Package installation}
%
% \paragraph{Unpacking.} The \xfile{.dtx} file is a self-extracting
% \docstrip\ archive. The files are extracted by running the
% \xfile{.dtx} through \plainTeX:
% \begin{quote}
%   \verb|tex zref.dtx|
% \end{quote}
%
% \paragraph{TDS.} Now the different files must be moved into
% the different directories in your installation TDS tree
% (also known as \xfile{texmf} tree):
% \begin{quote}
% \def\t{^^A
% \begin{tabular}{@{}>{\ttfamily}l@{ $\rightarrow$ }>{\ttfamily}l@{}}
%   zref.sty & tex/latex/zref/zref.sty\\
%   zref-base.sty & tex/latex/zref/zref-base.sty\\
%   zref-abspage.sty & tex/latex/zref/zref-abspage.sty\\
%   zref-abspos.sty & tex/latex/zref/zref-abspos.sty\\
%   zref-counter.sty & tex/latex/zref/zref-counter.sty\\
%   zref-dotfill.sty & tex/latex/zref/zref-dotfill.sty\\
%   zref-env.sty & tex/latex/zref/zref-env.sty\\
%   zref-hyperref.sty & tex/latex/zref/zref-hyperref.sty\\
%   zref-lastpage.sty & tex/latex/zref/zref-lastpage.sty\\
%   zref-marks.sty & tex/latex/zref/zref-marks.sty\\
%   zref-nextpage.sty & tex/latex/zref/zref-nextpage.sty\\
%   zref-pageattr.sty & tex/latex/zref/zref-pageattr.sty\\
%   zref-pagelayout.sty & tex/latex/zref/zref-pagelayout.sty\\
%   zref-perpage.sty & tex/latex/zref/zref-perpage.sty\\
%   zref-runs.sty & tex/latex/zref/zref-runs.sty\\
%   zref-savepos.sty & tex/latex/zref/zref-savepos.sty\\
%   zref-thepage.sty & tex/latex/zref/zref-thepage.sty\\
%   zref-titleref.sty & tex/latex/zref/zref-titleref.sty\\
%   zref-totpages.sty & tex/latex/zref/zref-totpages.sty\\
%   zref-user.sty & tex/latex/zref/zref-user.sty\\
%   zref-xr.sty & tex/latex/zref/zref-xr.sty\\
%   zref.pdf & doc/latex/zref/zref.pdf\\
%   zref-example.tex & doc/latex/zref/zref-example.tex\\
%   zref-example-lastpage.tex & doc/latex/zref/zref-example-lastpage.tex\\
%   zref-example-nextpage.tex & doc/latex/zref/zref-example-nextpage.tex\\
%   zref.dtx & source/latex/zref/zref.dtx\\
% \end{tabular}^^A
% }^^A
% \sbox0{\t}^^A
% \ifdim\wd0>\linewidth
%   \begingroup
%     \advance\linewidth by\leftmargin
%     \advance\linewidth by\rightmargin
%   \edef\x{\endgroup
%     \def\noexpand\lw{\the\linewidth}^^A
%   }\x
%   \def\lwbox{^^A
%     \leavevmode
%     \hbox to \linewidth{^^A
%       \kern-\leftmargin\relax
%       \hss
%       \usebox0
%       \hss
%       \kern-\rightmargin\relax
%     }^^A
%   }^^A
%   \ifdim\wd0>\lw
%     \sbox0{\small\t}^^A
%     \ifdim\wd0>\linewidth
%       \ifdim\wd0>\lw
%         \sbox0{\footnotesize\t}^^A
%         \ifdim\wd0>\linewidth
%           \ifdim\wd0>\lw
%             \sbox0{\scriptsize\t}^^A
%             \ifdim\wd0>\linewidth
%               \ifdim\wd0>\lw
%                 \sbox0{\tiny\t}^^A
%                 \ifdim\wd0>\linewidth
%                   \lwbox
%                 \else
%                   \usebox0
%                 \fi
%               \else
%                 \lwbox
%               \fi
%             \else
%               \usebox0
%             \fi
%           \else
%             \lwbox
%           \fi
%         \else
%           \usebox0
%         \fi
%       \else
%         \lwbox
%       \fi
%     \else
%       \usebox0
%     \fi
%   \else
%     \lwbox
%   \fi
% \else
%   \usebox0
% \fi
% \end{quote}
% If you have a \xfile{docstrip.cfg} that configures and enables \docstrip's
% TDS installing feature, then some files can already be in the right
% place, see the documentation of \docstrip.
%
% \subsection{Refresh file name databases}
%
% If your \TeX~distribution
% (\TeX\,Live, \mikTeX, \dots) relies on file name databases, you must refresh
% these. For example, \TeX\,Live\ users run \verb|texhash| or
% \verb|mktexlsr|.
%
% \subsection{Some details for the interested}
%
% \paragraph{Unpacking with \LaTeX.}
% The \xfile{.dtx} chooses its action depending on the format:
% \begin{description}
% \item[\plainTeX:] Run \docstrip\ and extract the files.
% \item[\LaTeX:] Generate the documentation.
% \end{description}
% If you insist on using \LaTeX\ for \docstrip\ (really,
% \docstrip\ does not need \LaTeX), then inform the autodetect routine
% about your intention:
% \begin{quote}
%   \verb|latex \let\install=y\input{zref.dtx}|
% \end{quote}
% Do not forget to quote the argument according to the demands
% of your shell.
%
% \paragraph{Generating the documentation.}
% You can use both the \xfile{.dtx} or the \xfile{.drv} to generate
% the documentation. The process can be configured by the
% configuration file \xfile{ltxdoc.cfg}. For instance, put this
% line into this file, if you want to have A4 as paper format:
% \begin{quote}
%   \verb|\PassOptionsToClass{a4paper}{article}|
% \end{quote}
% An example follows how to generate the
% documentation with pdf\LaTeX:
% \begin{quote}
%\begin{verbatim}
%pdflatex zref.dtx
%makeindex -s gind.ist zref.idx
%pdflatex zref.dtx
%makeindex -s gind.ist zref.idx
%pdflatex zref.dtx
%\end{verbatim}
% \end{quote}
%
% \begin{thebibliography}{99}
% \bibpackage{footmisc}{Robin Fairbairns}{2004/01/23 v5.3a}
%   {\CTANpkg{footmisc}}
% \bibpackage{hyperref}{Sebastian Rahtz, Heiko Oberdiek}{2006/08/16 v6.75c}
%   {\CTANpkg{hyperref}}
% \bibpackage{lastpage}{Jeff Goldberg}{1994/06/25 v0.1b}
%   {\CTANpkg{lastpage}}
% \bibpackage{nameref}{Sebastian Rahtz, Heiko Oberdiek}{2006/02/12 v2.24}
%   {\CTANpkg{nameref}}
% \bibpackage{perpage}{David Kastrup}{2002/12/20 v1.0}
%   {\CTANpkg{perpage}}
% \bibpackage{titleref}{Donald Arsenau}{2001/04/05 v3.1}
%   {\CTANpkg{titleref}}
% \bibpackage{totpages}{Wilhelm M\"uller}{1999/07/14 v1.00}
%   {\CTANpkg{totpages}}
% \bibpackage{xr}{David Carlisle}{1994/05/28 v5.02}
%   {\CTANpkg{xr}}
% \bibpackage{xr-hyper}{David Carlisle}{2000/03/22 v6.00beta4}
%   {\CTANpkg{xr-hyper}}
% \end{thebibliography}
%
% \begin{History}
%   \begin{Version}{2006/02/20 v1.0}
%   \item
%     First version.
%   \end{Version}
%   \begin{Version}{2006/05/03 v1.1}
%   \item
%     Module \xmodule{perpage} added.
%   \item
%     Module redesign as packages.
%   \end{Version}
%   \begin{Version}{2006/05/25 v1.2}
%   \item
%     Module \xmodule{dotfillmin} added.
%   \item
%     Module \xmodule{base}: macros \cs{zref@require@uniqe}
%     and \cs{thezref@unique} added
%     (used by modules \xmodule{titleref} and \xmodule{dotfillmin}).
%   \end{Version}
%   \begin{Version}{2006/09/08 v1.3}
%   \item
%     Typo fixes and English cleanup by Per Starback.
%   \end{Version}
%   \begin{Version}{2007/01/23 v1.4}
%   \item
%     Typo in macro name fixed in documentation.
%   \end{Version}
%   \begin{Version}{2007/02/18 v1.5}
%   \item
%     \cs{zref@getcurrent} added (suggestion of Igor Akkerman).
%   \item
%     Module \xmodule{savepos} also supports \hologo{XeTeX}.
%   \end{Version}
%   \begin{Version}{2007/04/06 v1.6}
%   \item
%     Fix in modules \xmodule{abspage} and \xmodule{base}:
%     Now counter |abspage| and |zref@unique|
%     are not remembered by \cs{include}.
%   \item
%     Beamer support for module \xmodule{titleref}.
%   \end{Version}
%   \begin{Version}{2007/04/17 v1.7}
%   \item
%     Package \xpackage{atbegshi} replaces \xpackage{everyshi}.
%   \end{Version}
%   \begin{Version}{2007/04/22 v1.8}
%   \item
%     \cs{zref@wrapper@babel} and \cs{zref@refused} are now expandable
%     if babel is not used or \cs{if@safe@actives} is already set to true.
%     (Feature request of Josselin Noirel)
%   \end{Version}
%   \begin{Version}{2007/05/02 v1.9}
%   \item
%     Module \xmodule{titleref}: Some support for \cs{caption} of package
%     \xpackage{longtable}, but only if \cs{label} is given after
%     \cs{caption}.
%   \end{Version}
%   \begin{Version}{2007/05/06 v2.0}
%   \item
%     Uses package \xpackage{etexcmds} for accessing \hologo{eTeX}'s \cs{unexpanded}.
%   \end{Version}
%   \begin{Version}{2007/05/28 v2.1}
%   \item
%     Module \xmodule{titleref} supports caption of package \xpackage{listings}.
%   \item
%     Fixes in module \xmodule{titleref} for support
%     of packages \xpackage{titlesec} and \xpackage{longtable}.
%   \end{Version}
%   \begin{Version}{2008/09/21 v2.2}
%   \item
%     Module \xmodule{base}: \cs{zref@iflistcontainsprop} is documented, but
%     a broken \cs{zref@listcontainsprop} implemented. Name and
%     implementation fixed (thanks Ohad Kammar).
%   \end{Version}
%   \begin{Version}{2008/10/01 v2.3}
%   \item
%     \cs{zref@localaddprop} added (feature request of Ohad Kammar).
%   \item
%     Module \xmodule{lastpage}: list `LastPage' added. Label `LastPage' will use
%     the properties of this list (default is empty) along with the
%     properties of the main list.
%   \end{Version}
%   \begin{Version}{2009/08/07 v2.4}
%   \item
%     Module \xmodule{runs} added.
%   \end{Version}
%   \begin{Version}{2009/12/06 v2.5}
%   \item
%     Module \xmodule{lastpage}: Uses package \xpackage{atveryend}.
%   \item
%     Module \xmodule{titleref}: Further commands are disabled
%     during string expansion, imported from package \xpackage{nameref}.
%   \end{Version}
%   \begin{Version}{2009/12/07 v2.6}
%   \item
%     Version date added for package \xpackage{atveryend}.
%   \end{Version}
%   \begin{Version}{2009/12/08 v2.7}
%   \item
%     Module \xmodule{titleref}: Use of package \xpackage{gettitlestring}.
%   \end{Version}
%   \begin{Version}{2010/03/26 v2.8}
%   \item
%     \cs{zifrefundefined} added.
%   \item
%     Module \xmodule{lastpage}: Macros \cs{zref@iflastpage}
%     and \cs{ziflastpage} added.
%   \item
%     Module \xmodule{thepage} added.
%   \item
%     Module \xmodule{nextpage} added.
%   \end{Version}
%   \begin{Version}{2010/03/29 v2.9}
%   \item
%     Module \xmodule{marks} added (without documentation).
%   \item
%     \cs{zref@addprop} now adds expanded property to list.
%   \item
%     Useless \cs{ZREF@ErrorNoLine} removed.
%   \end{Version}
%   \begin{Version}{2010/04/08 v2.10}
%   \item
%     Module \xmodule{xr} remembers the external document name
%     in property `externaldocument'.
%   \end{Version}
%   \begin{Version}{2010/04/15 v2.11}
%   \item
%     Module \xmodule{titleref}: Better support of class \xclass{memoir}.
%   \item
%     Module \xmodule{titleref}: Support of theorems.
%   \end{Version}
%   \begin{Version}{2010/04/17 v2.12}
%   \item
%     Module \xmodule{base}: \cs{zref@newprop} ensures global
%     empty default.
%   \item
%     Module \xmodule{xr}: Setup options \xoption{tozreflabel}
%     and \xoption{toltxlabel} added.
%   \end{Version}
%   \begin{Version}{2010/04/19 v2.13}
%   \item
%     \cs{zref@setcurrent} throws an error if the property
%     does not exist (Florent Chervet).
%   \item
%     \cs{zref@getcurrent} the documentation is fixed (Florent Chervet).
%     Also it returns the empty string in case of errors.
%   \item
%     \cs{zref@addprop} and \cs{zref@localaddprop} now take a list of
%     property names (feature request of Florent Chervet).
%   \item
%     Example for \cs{zref@wrapper@unexpanded} corrected (Florent Chervet).
%   \end{Version}
%   \begin{Version}{2010/04/22 v2.14}
%   \item
%     Bug fix for \cs{zref@getcurrent} second argument wasn't eaten in
%     case of unknown property.
%   \item
%     \cs{zref@getcurrent} supports \cs{zref@wrapper@unexpanded}.
%   \item
%     \cs{zref@wrapper@unexpanded} added for \cs{ZREF@xr@tolabel}.
%   \item
%     \cs{zref@extract}, \cs{zref@extractdefault}, \cs{zref@getcurrent}
%     are expandable in exact two steps
%     except inside \cs{zref@wrapper@unexpanded}.
%   \end{Version}
%   \begin{Version}{2010/04/23 v2.15}
%   \item
%     \cs{zexternaldocument} fixed for property `url' when
%     importing \cs{new@label} (bug found by Victor Ivrii).
%   \item
%     Two expansion steps also in \cs{zref@wrapper@unexpanded}.
%   \item
%     Nested calls of \cs{zref@wrapper@unexpanded} possible.
%   \end{Version}
%   \begin{Version}{2010/04/28 v2.16}
%   \item
%     More consequent use of package `ltxcmds' and `hologo'.
%   \item
%     Module \xmodule{pagelayout} added.
%   \item
%     Module \xmodule{pageattr} added.
%   \item
%     Robustness introduced for non-expandable interface macros.
%   \item
%     Internal change of the data format of property lists
%     (suggestion of Florent Chervet).
%   \item
%     Module \xmodule{titleref}: Support of environment \textsf{description}.
%   \end{Version}
%   \begin{Version}{2010/05/01 v2.17}
%   \item
%     \cs{zref@newprop} throws an error if the property already exists.
%   \item
%     Module \xmodule{xr}: Bug fix for the case of several \xext{aux} files
%     (bug found by Victor Ivrii).
%   \item
%     Module \xmodule{xr}: Property `urluse' and option \xoption{urluse} added.
%   \end{Version}
%   \begin{Version}{2010/05/13 v2.18}
%   \item
%     Module \xmodule{env} added.
%   \item
%     Module \xmodule{savepos}: \cs{zref@savepos} added.
%   \end{Version}
%   \begin{Version}{2010/10/22 v2.19}
%   \item
%     \cs{zref@addprop} and \cs{zref@localaddprop} are limited
%     to one property only
%     (incompatibility to versions v2.13 to v2.18).
%   \item
%     \cs{zref@addprops} and \cs{zref@localaddprops} added.
%   \item
%     \cs{zref@delprop} and \cs{zref@localdelprop} added.
%   \item
%     \cs{zref@labelbykv} and \cs{zkvlabel} (module \xmodule{user})
%     with keys \xoption{prop}, \xoption{list}, \xoption{delprop},
%     \xoption{immediate}, \xoption{values} added.
%   \end{Version}
%   \begin{Version}{2011/02/12 v2.20}
%   \item
%     Fix for warning in \xpackage{zref-xr}.
%   \end{Version}
%   \begin{Version}{2011/03/18 v2.21}
%   \item
%     Fix in module \xmodule{pagelayout} for \cs{zlistpagelayout}.
%   \item
%     Fix for \cs{zref@localaddprop} (probably since v2.19).
%   \end{Version}
%   \begin{Version}{2011/10/05 v2.22}
%   \item
%     Documentation fixed for \cs{zref@(local)addprop(s)}.
%   \item
%     Module \xmodule{base}: \cs{zref@def@extract}, \cs{zref@def@extractdefault}
%     added.
%   \item
%     Fix in module \xmodule{pagelayout}: Because of missing \cs{noexpand}
%     commands the values of the pagelayout properties on all pages were
%     the values at package loading.
%   \item
%     Module \xmodule{base}: \cs{zref@showprop} added.
%   \end{Version}
%   \begin{Version}{2011/12/05 v2.23}
%   \item
%     Module \xmodule{savepos}: \cs{zsaveposx} and \cs{zsaveposy} added.
%   \end{Version}
%   \begin{Version}{2012/04/04 v2.24}
%   \item
%     Module \xmodule{titleref}, package \xpackage{titlesec}:
%     some support for class `straight' (\cs{ttl@straight@i}) added.
%   \end{Version}
%   \begin{Version}{2016/05/16 v2.25}
%   \item
%     Documentation updates.
%   \end{Version}
%   \begin{Version}{2016/05/21 v2.26}
%   \item
%     update zref-savepos for new luatex
%   \end{Version}
%   \begin{Version}{2018/11/21 v2.27}
%   \item
%     adapted zref-perpage, see issue https://github.com/ho-tex/zref/issues/2
%   \end{Version}
%   \begin{Version}{2019/11/29 v2.28}
%   \item
%     Documentation updates.
%   \item
%     Use \xpackage{iftex} directly.
%   \end{Version}
%   \begin{Version}{2020-03-03 v2.29}
%   \item
%     adapted in module zref-pagelayout the properties pdfhorigin, pdfvorigin,
%     pdfpagewidth, pdfpageheight for luatex to the right primitives.
%   \item
%     Removed no longer needed code for older lualatex versions.
%   \item added some documentation of the abspos module.
%   \item adapted the abspos module to the new luatex primitives.
%   \item adapted pageattr module to the new luatex primitives.
%   \item added short documentation for pageattr module
%   \item use luatex command names directly in zref-savepos rather than defining pdftex compatibility names.
%   \item allow zref-abspos to use \verb|\pdf[vh]origin| in dvi mode.
%   \end{Version}
%   \begin{Version}{2020-03-04 v2.30}
%   \item
%     add pagevalue property to savepos in the abspos module (issue 1)
%   \end{Version}
%   \begin{Version}{2020-05-28 v2.31}
%   \item Adapted module zref-counter to use \cs{@currentcounter} in the next
%   \LaTeX version (issue 5)
%   \end{Version}
%   \begin{Version}{2020-07-03 v2.32}
%   \item Changed in zref-pagelayout the names of the shipout box dimensions
%    to adapt to the new hook management.
%   \end{Version}
% \end{History}
%
% \PrintIndex
%
% \Finale
\endinput
|
% \end{quote}
% Do not forget to quote the argument according to the demands
% of your shell.
%
% \paragraph{Generating the documentation.}
% You can use both the \xfile{.dtx} or the \xfile{.drv} to generate
% the documentation. The process can be configured by the
% configuration file \xfile{ltxdoc.cfg}. For instance, put this
% line into this file, if you want to have A4 as paper format:
% \begin{quote}
%   \verb|\PassOptionsToClass{a4paper}{article}|
% \end{quote}
% An example follows how to generate the
% documentation with pdf\LaTeX:
% \begin{quote}
%\begin{verbatim}
%pdflatex zref.dtx
%makeindex -s gind.ist zref.idx
%pdflatex zref.dtx
%makeindex -s gind.ist zref.idx
%pdflatex zref.dtx
%\end{verbatim}
% \end{quote}
%
% \begin{thebibliography}{99}
% \bibpackage{footmisc}{Robin Fairbairns}{2004/01/23 v5.3a}
%   {\CTANpkg{footmisc}}
% \bibpackage{hyperref}{Sebastian Rahtz, Heiko Oberdiek}{2006/08/16 v6.75c}
%   {\CTANpkg{hyperref}}
% \bibpackage{lastpage}{Jeff Goldberg}{1994/06/25 v0.1b}
%   {\CTANpkg{lastpage}}
% \bibpackage{nameref}{Sebastian Rahtz, Heiko Oberdiek}{2006/02/12 v2.24}
%   {\CTANpkg{nameref}}
% \bibpackage{perpage}{David Kastrup}{2002/12/20 v1.0}
%   {\CTANpkg{perpage}}
% \bibpackage{titleref}{Donald Arsenau}{2001/04/05 v3.1}
%   {\CTANpkg{titleref}}
% \bibpackage{totpages}{Wilhelm M\"uller}{1999/07/14 v1.00}
%   {\CTANpkg{totpages}}
% \bibpackage{xr}{David Carlisle}{1994/05/28 v5.02}
%   {\CTANpkg{xr}}
% \bibpackage{xr-hyper}{David Carlisle}{2000/03/22 v6.00beta4}
%   {\CTANpkg{xr-hyper}}
% \end{thebibliography}
%
% \begin{History}
%   \begin{Version}{2006/02/20 v1.0}
%   \item
%     First version.
%   \end{Version}
%   \begin{Version}{2006/05/03 v1.1}
%   \item
%     Module \xmodule{perpage} added.
%   \item
%     Module redesign as packages.
%   \end{Version}
%   \begin{Version}{2006/05/25 v1.2}
%   \item
%     Module \xmodule{dotfillmin} added.
%   \item
%     Module \xmodule{base}: macros \cs{zref@require@uniqe}
%     and \cs{thezref@unique} added
%     (used by modules \xmodule{titleref} and \xmodule{dotfillmin}).
%   \end{Version}
%   \begin{Version}{2006/09/08 v1.3}
%   \item
%     Typo fixes and English cleanup by Per Starback.
%   \end{Version}
%   \begin{Version}{2007/01/23 v1.4}
%   \item
%     Typo in macro name fixed in documentation.
%   \end{Version}
%   \begin{Version}{2007/02/18 v1.5}
%   \item
%     \cs{zref@getcurrent} added (suggestion of Igor Akkerman).
%   \item
%     Module \xmodule{savepos} also supports \hologo{XeTeX}.
%   \end{Version}
%   \begin{Version}{2007/04/06 v1.6}
%   \item
%     Fix in modules \xmodule{abspage} and \xmodule{base}:
%     Now counter |abspage| and |zref@unique|
%     are not remembered by \cs{include}.
%   \item
%     Beamer support for module \xmodule{titleref}.
%   \end{Version}
%   \begin{Version}{2007/04/17 v1.7}
%   \item
%     Package \xpackage{atbegshi} replaces \xpackage{everyshi}.
%   \end{Version}
%   \begin{Version}{2007/04/22 v1.8}
%   \item
%     \cs{zref@wrapper@babel} and \cs{zref@refused} are now expandable
%     if babel is not used or \cs{if@safe@actives} is already set to true.
%     (Feature request of Josselin Noirel)
%   \end{Version}
%   \begin{Version}{2007/05/02 v1.9}
%   \item
%     Module \xmodule{titleref}: Some support for \cs{caption} of package
%     \xpackage{longtable}, but only if \cs{label} is given after
%     \cs{caption}.
%   \end{Version}
%   \begin{Version}{2007/05/06 v2.0}
%   \item
%     Uses package \xpackage{etexcmds} for accessing \hologo{eTeX}'s \cs{unexpanded}.
%   \end{Version}
%   \begin{Version}{2007/05/28 v2.1}
%   \item
%     Module \xmodule{titleref} supports caption of package \xpackage{listings}.
%   \item
%     Fixes in module \xmodule{titleref} for support
%     of packages \xpackage{titlesec} and \xpackage{longtable}.
%   \end{Version}
%   \begin{Version}{2008/09/21 v2.2}
%   \item
%     Module \xmodule{base}: \cs{zref@iflistcontainsprop} is documented, but
%     a broken \cs{zref@listcontainsprop} implemented. Name and
%     implementation fixed (thanks Ohad Kammar).
%   \end{Version}
%   \begin{Version}{2008/10/01 v2.3}
%   \item
%     \cs{zref@localaddprop} added (feature request of Ohad Kammar).
%   \item
%     Module \xmodule{lastpage}: list `LastPage' added. Label `LastPage' will use
%     the properties of this list (default is empty) along with the
%     properties of the main list.
%   \end{Version}
%   \begin{Version}{2009/08/07 v2.4}
%   \item
%     Module \xmodule{runs} added.
%   \end{Version}
%   \begin{Version}{2009/12/06 v2.5}
%   \item
%     Module \xmodule{lastpage}: Uses package \xpackage{atveryend}.
%   \item
%     Module \xmodule{titleref}: Further commands are disabled
%     during string expansion, imported from package \xpackage{nameref}.
%   \end{Version}
%   \begin{Version}{2009/12/07 v2.6}
%   \item
%     Version date added for package \xpackage{atveryend}.
%   \end{Version}
%   \begin{Version}{2009/12/08 v2.7}
%   \item
%     Module \xmodule{titleref}: Use of package \xpackage{gettitlestring}.
%   \end{Version}
%   \begin{Version}{2010/03/26 v2.8}
%   \item
%     \cs{zifrefundefined} added.
%   \item
%     Module \xmodule{lastpage}: Macros \cs{zref@iflastpage}
%     and \cs{ziflastpage} added.
%   \item
%     Module \xmodule{thepage} added.
%   \item
%     Module \xmodule{nextpage} added.
%   \end{Version}
%   \begin{Version}{2010/03/29 v2.9}
%   \item
%     Module \xmodule{marks} added (without documentation).
%   \item
%     \cs{zref@addprop} now adds expanded property to list.
%   \item
%     Useless \cs{ZREF@ErrorNoLine} removed.
%   \end{Version}
%   \begin{Version}{2010/04/08 v2.10}
%   \item
%     Module \xmodule{xr} remembers the external document name
%     in property `externaldocument'.
%   \end{Version}
%   \begin{Version}{2010/04/15 v2.11}
%   \item
%     Module \xmodule{titleref}: Better support of class \xclass{memoir}.
%   \item
%     Module \xmodule{titleref}: Support of theorems.
%   \end{Version}
%   \begin{Version}{2010/04/17 v2.12}
%   \item
%     Module \xmodule{base}: \cs{zref@newprop} ensures global
%     empty default.
%   \item
%     Module \xmodule{xr}: Setup options \xoption{tozreflabel}
%     and \xoption{toltxlabel} added.
%   \end{Version}
%   \begin{Version}{2010/04/19 v2.13}
%   \item
%     \cs{zref@setcurrent} throws an error if the property
%     does not exist (Florent Chervet).
%   \item
%     \cs{zref@getcurrent} the documentation is fixed (Florent Chervet).
%     Also it returns the empty string in case of errors.
%   \item
%     \cs{zref@addprop} and \cs{zref@localaddprop} now take a list of
%     property names (feature request of Florent Chervet).
%   \item
%     Example for \cs{zref@wrapper@unexpanded} corrected (Florent Chervet).
%   \end{Version}
%   \begin{Version}{2010/04/22 v2.14}
%   \item
%     Bug fix for \cs{zref@getcurrent} second argument wasn't eaten in
%     case of unknown property.
%   \item
%     \cs{zref@getcurrent} supports \cs{zref@wrapper@unexpanded}.
%   \item
%     \cs{zref@wrapper@unexpanded} added for \cs{ZREF@xr@tolabel}.
%   \item
%     \cs{zref@extract}, \cs{zref@extractdefault}, \cs{zref@getcurrent}
%     are expandable in exact two steps
%     except inside \cs{zref@wrapper@unexpanded}.
%   \end{Version}
%   \begin{Version}{2010/04/23 v2.15}
%   \item
%     \cs{zexternaldocument} fixed for property `url' when
%     importing \cs{new@label} (bug found by Victor Ivrii).
%   \item
%     Two expansion steps also in \cs{zref@wrapper@unexpanded}.
%   \item
%     Nested calls of \cs{zref@wrapper@unexpanded} possible.
%   \end{Version}
%   \begin{Version}{2010/04/28 v2.16}
%   \item
%     More consequent use of package `ltxcmds' and `hologo'.
%   \item
%     Module \xmodule{pagelayout} added.
%   \item
%     Module \xmodule{pageattr} added.
%   \item
%     Robustness introduced for non-expandable interface macros.
%   \item
%     Internal change of the data format of property lists
%     (suggestion of Florent Chervet).
%   \item
%     Module \xmodule{titleref}: Support of environment \textsf{description}.
%   \end{Version}
%   \begin{Version}{2010/05/01 v2.17}
%   \item
%     \cs{zref@newprop} throws an error if the property already exists.
%   \item
%     Module \xmodule{xr}: Bug fix for the case of several \xext{aux} files
%     (bug found by Victor Ivrii).
%   \item
%     Module \xmodule{xr}: Property `urluse' and option \xoption{urluse} added.
%   \end{Version}
%   \begin{Version}{2010/05/13 v2.18}
%   \item
%     Module \xmodule{env} added.
%   \item
%     Module \xmodule{savepos}: \cs{zref@savepos} added.
%   \end{Version}
%   \begin{Version}{2010/10/22 v2.19}
%   \item
%     \cs{zref@addprop} and \cs{zref@localaddprop} are limited
%     to one property only
%     (incompatibility to versions v2.13 to v2.18).
%   \item
%     \cs{zref@addprops} and \cs{zref@localaddprops} added.
%   \item
%     \cs{zref@delprop} and \cs{zref@localdelprop} added.
%   \item
%     \cs{zref@labelbykv} and \cs{zkvlabel} (module \xmodule{user})
%     with keys \xoption{prop}, \xoption{list}, \xoption{delprop},
%     \xoption{immediate}, \xoption{values} added.
%   \end{Version}
%   \begin{Version}{2011/02/12 v2.20}
%   \item
%     Fix for warning in \xpackage{zref-xr}.
%   \end{Version}
%   \begin{Version}{2011/03/18 v2.21}
%   \item
%     Fix in module \xmodule{pagelayout} for \cs{zlistpagelayout}.
%   \item
%     Fix for \cs{zref@localaddprop} (probably since v2.19).
%   \end{Version}
%   \begin{Version}{2011/10/05 v2.22}
%   \item
%     Documentation fixed for \cs{zref@(local)addprop(s)}.
%   \item
%     Module \xmodule{base}: \cs{zref@def@extract}, \cs{zref@def@extractdefault}
%     added.
%   \item
%     Fix in module \xmodule{pagelayout}: Because of missing \cs{noexpand}
%     commands the values of the pagelayout properties on all pages were
%     the values at package loading.
%   \item
%     Module \xmodule{base}: \cs{zref@showprop} added.
%   \end{Version}
%   \begin{Version}{2011/12/05 v2.23}
%   \item
%     Module \xmodule{savepos}: \cs{zsaveposx} and \cs{zsaveposy} added.
%   \end{Version}
%   \begin{Version}{2012/04/04 v2.24}
%   \item
%     Module \xmodule{titleref}, package \xpackage{titlesec}:
%     some support for class `straight' (\cs{ttl@straight@i}) added.
%   \end{Version}
%   \begin{Version}{2016/05/16 v2.25}
%   \item
%     Documentation updates.
%   \end{Version}
%   \begin{Version}{2016/05/21 v2.26}
%   \item
%     update zref-savepos for new luatex
%   \end{Version}
%   \begin{Version}{2018/11/21 v2.27}
%   \item
%     adapted zref-perpage, see issue https://github.com/ho-tex/zref/issues/2
%   \end{Version}
%   \begin{Version}{2019/11/29 v2.28}
%   \item
%     Documentation updates.
%   \item
%     Use \xpackage{iftex} directly.
%   \end{Version}
%   \begin{Version}{2020-03-03 v2.29}
%   \item
%     adapted in module zref-pagelayout the properties pdfhorigin, pdfvorigin,
%     pdfpagewidth, pdfpageheight for luatex to the right primitives.
%   \item
%     Removed no longer needed code for older lualatex versions.
%   \item added some documentation of the abspos module.
%   \item adapted the abspos module to the new luatex primitives.
%   \item adapted pageattr module to the new luatex primitives.
%   \item added short documentation for pageattr module
%   \item use luatex command names directly in zref-savepos rather than defining pdftex compatibility names.
%   \item allow zref-abspos to use \verb|\pdf[vh]origin| in dvi mode.
%   \end{Version}
%   \begin{Version}{2020-03-04 v2.30}
%   \item
%     add pagevalue property to savepos in the abspos module (issue 1)
%   \end{Version}
%   \begin{Version}{2020-05-28 v2.31}
%   \item Adapted module zref-counter to use \cs{@currentcounter} in the next
%   \LaTeX version (issue 5)
%   \end{Version}
%   \begin{Version}{2020-07-03 v2.32}
%   \item Changed in zref-pagelayout the names of the shipout box dimensions
%    to adapt to the new hook management.
%   \end{Version}
% \end{History}
%
% \PrintIndex
%
% \Finale
\endinput

%        (quote the arguments according to the demands of your shell)
%
% Documentation:
%    (a) If zref.drv is present:
%           latex zref.drv
%    (b) Without zref.drv:
%           latex zref.dtx; ...
%    The class ltxdoc loads the configuration file ltxdoc.cfg
%    if available. Here you can specify further options, e.g.
%    use A4 as paper format:
%       \PassOptionsToClass{a4paper}{article}
%
%    Programm calls to get the documentation (example):
%       pdflatex zref.dtx
%       makeindex -s gind.ist zref.idx
%       pdflatex zref.dtx
%       makeindex -s gind.ist zref.idx
%       pdflatex zref.dtx
%
% Installation:
%    TDS:tex/latex/zref/zref.sty
%    TDS:tex/latex/zref/zref-base.sty
%    TDS:tex/latex/zref/zref-abspage.sty
%    TDS:tex/latex/zref/zref-abspos.sty
%    TDS:tex/latex/zref/zref-counter.sty
%    TDS:tex/latex/zref/zref-dotfill.sty
%    TDS:tex/latex/zref/zref-env.sty
%    TDS:tex/latex/zref/zref-hyperref.sty
%    TDS:tex/latex/zref/zref-lastpage.sty
%    TDS:tex/latex/zref/zref-marks.sty
%    TDS:tex/latex/zref/zref-nextpage.sty
%    TDS:tex/latex/zref/zref-pageattr.sty
%    TDS:tex/latex/zref/zref-pagelayout.sty
%    TDS:tex/latex/zref/zref-perpage.sty
%    TDS:tex/latex/zref/zref-runs.sty
%    TDS:tex/latex/zref/zref-savepos.sty
%    TDS:tex/latex/zref/zref-thepage.sty
%    TDS:tex/latex/zref/zref-titleref.sty
%    TDS:tex/latex/zref/zref-totpages.sty
%    TDS:tex/latex/zref/zref-user.sty
%    TDS:tex/latex/zref/zref-xr.sty
%    TDS:doc/latex/zref/zref.pdf
%    TDS:doc/latex/zref/zref-example.tex
%    TDS:doc/latex/zref/zref-example-lastpage.tex
%    TDS:doc/latex/zref/zref-example-nextpage.tex
%    TDS:source/latex/zref/zref.dtx
%
%<*ignore>
\begingroup
  \catcode123=1 %
  \catcode125=2 %
  \def\x{LaTeX2e}%
\expandafter\endgroup
\ifcase 0\ifx\install y1\fi\expandafter
         \ifx\csname processbatchFile\endcsname\relax\else1\fi
         \ifx\fmtname\x\else 1\fi\relax
\else\csname fi\endcsname
%</ignore>
%<*install>
\input docstrip.tex
\Msg{************************************************************************}
\Msg{* Installation}
\Msg{* Package: zref 2020-07-03 v2.32 A new reference scheme for LaTeX (HO)}
\Msg{************************************************************************}

\keepsilent
\askforoverwritefalse

\let\MetaPrefix\relax
\preamble

This is a generated file.

Project: zref
Version: 2020-07-03 v2.32

Copyright (C)
  2006-2012 Heiko Oberdiek
  2016-2020 Oberdiek Package Support Group

This work may be distributed and/or modified under the
conditions of the LaTeX Project Public License, either
version 1.3c of this license or (at your option) any later
version. This version of this license is in
   https://www.latex-project.org/lppl/lppl-1-3c.txt
and the latest version of this license is in
   https://www.latex-project.org/lppl.txt
and version 1.3 or later is part of all distributions of
LaTeX version 2005/12/01 or later.

This work has the LPPL maintenance status "maintained".

The Current Maintainers of this work are
Heiko Oberdiek and the Oberdiek Package Support Group
https://github.com/ho-tex/zref/issues


This work consists of the main source file zref.dtx
and the derived files
   zref.sty, zref.pdf, zref.ins, zref.drv, zref-base.sty,
   zref-abspage.sty, zref-abspos.sty, zref-counter.sty,
   zref-dotfill.sty, zref-env.sty, zref-hyperref.sty,
   zref-lastpage.sty, zref-marks.sty, zref-nextpage.sty,
   zref-pageattr.sty, zref-pagelayout.sty, zref-perpage.sty,
   zref-runs.sty, zref-savepos.sty, zref-thepage.sty,
   zref-titleref.sty, zref-totpages.sty, zref-user.sty,
   zref-xr.sty, zref-example.tex, zref-example-lastpage.tex,
   zref-example-nextpage.tex, zref-test1.tex,
   zref-test-base.tex, zref-test-runs.tex,
   zref-test-titleref-memoir.tex.

\endpreamble
\let\MetaPrefix\DoubleperCent

\generate{%
  \file{zref.ins}{\from{zref.dtx}{install}}%
  \file{zref.drv}{\from{zref.dtx}{driver}}%
  \usedir{tex/latex/zref}%
  \file{zref.sty}{\from{zref.dtx}{package}}%
  \file{zref-base.sty}{\from{zref.dtx}{base}}%
  \file{zref-abspage.sty}{\from{zref.dtx}{abspage}}%
  \file{zref-abspos.sty}{\from{zref.dtx}{abspos}}%
  \file{zref-counter.sty}{\from{zref.dtx}{counter}}%
  \file{zref-dotfill.sty}{\from{zref.dtx}{dotfill}}%
  \file{zref-env.sty}{\from{zref.dtx}{env}}%
  \file{zref-hyperref.sty}{\from{zref.dtx}{hyperref}}%
  \file{zref-lastpage.sty}{\from{zref.dtx}{lastpage}}%
  \file{zref-marks.sty}{\from{zref.dtx}{marks}}%
  \file{zref-nextpage.sty}{\from{zref.dtx}{nextpage}}%
  \file{zref-pageattr.sty}{\from{zref.dtx}{pageattr}}%
  \file{zref-pagelayout.sty}{\from{zref.dtx}{pagelayout}}%
  \file{zref-perpage.sty}{\from{zref.dtx}{perpage}}%
  \file{zref-runs.sty}{\from{zref.dtx}{runs}}%
  \file{zref-savepos.sty}{\from{zref.dtx}{savepos}}%
  \file{zref-thepage.sty}{\from{zref.dtx}{thepage}}%
  \file{zref-titleref.sty}{\from{zref.dtx}{titleref}}%
  \file{zref-totpages.sty}{\from{zref.dtx}{totpages}}%
  \file{zref-user.sty}{\from{zref.dtx}{user}}%
  \file{zref-xr.sty}{\from{zref.dtx}{xr}}%
  \usedir{doc/latex/zref}%
  \file{zref-example.tex}{\from{zref.dtx}{example}}%
  \file{zref-example-lastpage.tex}{\from{zref.dtx}{example-lastpage}}%
  \file{zref-example-nextpage.tex}{\from{zref.dtx}{example-nextpage}}%
%  \usedir{doc/latex/zref/test}%
%  \file{zref-test1.tex}{\from{zref.dtx}{test1}}%
%  \file{zref-test-base.tex}{\from{zref.dtx}{test-base}}%
%  \file{zref-test-runs.tex}{\from{zref.dtx}{test-runs}}%
%  \file{zref-test-titleref-memoir.tex}{\from{zref.dtx}{test-titleref-memoir}}%
}

\catcode32=13\relax% active space
\let =\space%
\Msg{************************************************************************}
\Msg{*}
\Msg{* To finish the installation you have to move the following}
\Msg{* files into a directory searched by TeX:}
\Msg{*}
\Msg{*     zref.sty, zref-base.sty, zref-abspage.sty, zref-abspos.sty,}
\Msg{*     zref-counter.sty, zref-dotfill.sty, zref-env.sty,}
\Msg{*     zref-hyperref.sty, zref-lastpage.sty, zref-marks.sty,}
\Msg{*     zref-nextpage.sty, zref-pageattr.sty, zref-pagelayout.sty,}
\Msg{*     zref-perpage.sty, zref-runs.sty, zref-savepos.sty,}
\Msg{*     zref-thepage.sty, zref-titleref.sty, zref-totpages.sty,}
\Msg{*     zref-user.sty, zref-xr.sty}
\Msg{*}
\Msg{* To produce the documentation run the file `zref.drv'}
\Msg{* through LaTeX.}
\Msg{*}
\Msg{* Happy TeXing!}
\Msg{*}
\Msg{************************************************************************}

\endbatchfile
%</install>
%<*ignore>
\fi
%</ignore>
%<*driver>
\NeedsTeXFormat{LaTeX2e}
\ProvidesFile{zref.drv}%
  [2020-07-03 v2.32 A new reference scheme for LaTeX (HO)]%
\documentclass{ltxdoc}
\usepackage{holtxdoc}[2011/11/22]
\setlength{\emergencystretch}{1em}
\begin{document}
  \DocInput{zref.dtx}%
\end{document}
%</driver>
% \fi
%
%
%
% \GetFileInfo{zref.drv}
%
% \title{The \xpackage{zref} package}
% \date{2020-07-03 v2.32}
% \author{Heiko Oberdiek\thanks
% {Please report any issues at \url{https://github.com/ho-tex/zref/issues}}}
%
% \maketitle
%
% \begin{abstract}
% Package \xpackage{zref} tries to get rid of the restriction
% in \hologo{LaTeX}'s reference system that only two properties are
% supported. The package implements an extensible referencing
% system, where properties are handled in a more flexible way.
% It offers an interface for macro programmers for the access
% to the system and some applications that uses the new
% reference scheme.
% \end{abstract}
%
% \tableofcontents
%
% \newcommand*{\metacs}[1]{\texttt{\itshape #1}}
%
% \newcommand{\markexp}{^^A
%   \textsuperscript{exp}^^A
%   \space
% }
% \newcommand{\markexptwo}{^^A
%   \textsuperscript{exp2}^^A
%   \space
% }
% \newcommand{\markrobust}{^^A
%   \textsuperscript{robust}^^A
%   \space
% }
% \newcommand{\marknowarn}{^^A
%   \textsuperscript{nowarn}^^A
%   \space
% }
% \newcommand{\markbabel}{^^A
%   \textsuperscript{babel}^^A
%   \space
% }
%
% \section{Introduction}
%
% Standard \hologo{LaTeX}'s reference system with \cs{label}, \cs{ref},
% and \cs{pageref} supports two properties, the apperance of
% the counter that is last incremented by \cs{refstepcounter} and
% the page with the \cs{label} command.
%
% Unhappily \hologo{LaTeX} does not provide an interface for adding
% another properties. Packages such as \xpackage{hyperref},
% \xpackage{nameref}, or \xpackage{titleref} are forced to
% use ugly hacks to extend the reference system.
% These ugly hacks are one of the causes for \xpackage{hyperref}'s
% difficulty regarding compatibility with other packages.
%
% \subsection{Standard \hologo{LaTeX} behaviour}
%
% References are created by the \cs{label} command:
% \begin{quote}
%   |\chapter{Second chapter}|\\
%   |\section{First section on page 7} % section 2.1|\\
%   |\label{myref}|
% \end{quote}
% Now \hologo{LaTeX} records the section number |2.1| and the page |7|
% in the reference. Internally the reference is a list with
% two entries:
% \begin{quote}
%   |\r@myref| $\rightarrow$ |{2.1}{7}|
% \end{quote}
% The length of the list if fixed in the \hologo{LaTeX} kernel,
% An interface for adding new properties is missing.
%
% There are several tries to add new properties:
% \begin{description}
% \item[\xpackage{hyperref}] uses a list of five
%   properties instead of the standard list with two entries.
%   This causes many compatibility problems with \hologo{LaTeX} and
%   other packages.
% \item[\xpackage{titleref}] stores its title data into the
%   first entry in the list. \hologo{LaTeX} is happy because it
%   does only see its list with two entries. The situation
%   becomes more difficult, if more properties are added
%   this way. Then the macros form a nested structure
%   inside the first reference argument for the label.
%   Expandable extractions will then become painful.
% \end{description}
%
% \subsection{Basic idea}
%
% Some time ago Morten H\o gholm sent me an experimental
% cross referencing mechanism as ``expl3'' code. His
% idea is:
% \begin{quote}
%   |\g_xref_mylabel_plist| $\rightarrow$\\
%   \mbox{}\qquad|\xref_dance_key{salsa}\xref_name_key{Morten}|\dots
% \end{quote}
% The entries have the following format:
% \begin{quote}
%   |\xref_|\meta{your key}|_key{|\meta{some text}|}|
% \end{quote}
% This approach is much more flexible:
% \begin{itemize}
%   \item New properties can easily be added, just use a new key.
%   \item The length of the list is not fixed. A reference
%         can use a subset of the keys.
%   \item The order of the entries does not matter.
% \end{itemize}
% Unhappily I am not familiar with the experimental code
% for \hologo{LaTeX3} that will need some time before its first release.
% Thus I have implemented it as \hologo{LaTeXe} package without disturbing
% the existing \hologo{LaTeX} reference system.
%
% \subsection{Interfaces}
%
% The package provides a generic \emph{interface for programmers}.
% Commands of this interface are prefixed by \cs{zref@}.
%
% Option \xoption{user} enabels the \emph{user interface}.
% Here the commands are prefixed by \cs{z} to avoid name
% clashes with existing macros.
%
% Then the packages provides some \emph{modules}. They
% are applications for the reference system and can also be
% considered as examples how to use the reference system.
%
% The modules can be loaded as packages. The package
% name is prefixed with |zref-|, for example:
%\begin{quote}
%\begin{verbatim}
%\RequirePackage{zref-abspage}
%\end{verbatim}
%\end{quote}
% This is the preferred way if the package is loaded from
% within other packages to avoid option clashes.
%
% As alternative package |zref| can be used and the modules
% are given as options:
%\begin{quote}
%\begin{verbatim}
%\usepackage[perpage,user]{zref}
%\end{verbatim}
%\end{quote}
%
% \section{Interface for programmers}
%
% The user interface is described in the next section \ref{sec:user}.
%
% \subsection{Entities}
%
% \paragraph{Reference.}
% Internally a reference is a list of key value pairs:
% \begin{quote}
%   |\Z@R@myref| $\rightarrow$
%   |\default{2.1}\page{7}|
% \end{quote}
% The generic format of a entry is:
% \begin{quote}
%   |\Z@R@|\meta{refname} $\rightarrow$
%   |\|\meta{propname}|{|\meta{value}|}|
% \end{quote}
% \meta{refname} is the name that denoted references
% (the name used in \cs{label} and \cs{ref}). \meta{propname}
% is the name of the property or key. The property key macro
% is never executed, it is used in parameter text matching only.
%
% \paragraph{Property.}
% Because the name of a property is used in a macro name
% that must survive the \xfile{.aux} file, the name is
% restricted to letters and `@'.
%
% \paragraph{Property list.}
% Often references are used for special purposes. Thus it
% saves memory if just the properties are used in this
% reference that are necessary for its purpose.
%
% Therefore this package uses the concept of \emph{property lists}.
% A property list is a set of properties. The set of properties
% that is used by the default \cs{label} command is the
% \emph{main property list}.
%
% \subsection{Property list}
%
% \markexp means that the implementation of the
% marked macro is expandable. \markexptwo goes a step further
% and marks the macro expandable in exact two expansion steps.
%
% \begin{declcs}{zref@newlist}
%   \M{listname}
% \end{declcs}
% Declares a new empty property list.
%
% \begin{declcs}{zref@addprop}
%   \M{listname} \M{propname}\\
%   \cs{zref@localaddprop} \M{listname} \M{propname}
% \end{declcs}
% Adds the property \meta{propname}
% to the property list \meta{listname}.
% The property and list must exist.
% The addition is global by \cs{zref@addprop} and limited to
% local scope by \cs{zref@localaddprop}.
% Between 2010/04/19 v2.13 and 2010/10/22 v2.19 a comma separated list
% of properties could be used as argument \meta{propname}.
% Since 2010/10/22 v2.19 the addition of several properties
% at once is supported by \cs{zref@addprops}.
%
% \begin{declcs}{zref@addprops}
%   \M{listname} \M{propname list}\\
%   \cs{zref@localaddprops} \M{listname} \M{propname list}
% \end{declcs}
% These macros add a comma separated list of properties
% \meta{propname list} to list \meta{listname}.
% \cs{zref@addprops} works globally and \cs{zref@localaddprops}
% locally. Since 2010/10/22 v2.19.
%
% \begin{declcs}{zref@listexists}
%   \M{listname} \M{then}
% \end{declcs}
% Executes \meta{then} if the property list \meta{listname} exists
% or raise an error otherwise.
%
% \begin{declcs}{zref@iflistundefined}\markexp \M{listname}
%   \M{then} \M{else}
% \end{declcs}
% Executes \meta{then} if the list exists or \meta{else}
% otherwise.
%
% \begin{declcs}{zref@iflistcontainsprop}
%   \M{listname} \M{propname} \M{then} \M{else}
% \end{declcs}
% Executes \meta{then} if the property \meta{propname} is part
% of property list \meta{listname} or otherwise it
% runs the \meta{else} part.
%
% \subsection{Property}
%
% \begin{declcs}{zref@newprop}
%   \* \M{propname} |[|\meta{default}|]| \M{value}
% \end{declcs}
% This command declares and configures a new property with
% name \meta{propname}.
%
% In case of unknown references
% or the property does not exist in the reference, the
% \meta{default} is used as value. If it is not specified
% here, a global default is used, see \cs{zref@setdefault}.
%
% The correct values of some properties are not known
% immediately but at page shipout time. Prominent example
% is the page number. These properties are declared with
% the star form of the command.
%
% \begin{declcs}{zref@setcurrent}
%   \M{propname} \M{value}
% \end{declcs}
% This sets the current value of the property \meta{propname}.
% It is a generalization of setting \hologo{LaTeX}'s \cs{currentlabel}.
%
% \begin{declcs}{zref@getcurrent}\markexptwo
%   \M{propname}
% \end{declcs}
% This returns the current value of the property \meta{propname}.
% The value may not be correct, especially if the property
% is bound to a page (start form of \cs{zref@newprop}) and the
% right value is only known at shipout time (e.g. property `page').
% In case of errors (e.g. unknown property) the empty string is
% returned.
%
% Since version 2010/04/22 v2.14 \cs{zref@getcurrent} supports
% \cs{zref@wrapper@unexpanded}.
%
% \begin{declcs}{zref@propexists}
%   \M{propname} \M{then}
% \end{declcs}
% Calls \meta{then} if the property \meta{propname} is available
% or generates an error message otherwise.
%
% \begin{declcs}{zref@ifpropundefined}\markexp
%   \M{propname} \M{then} \M{else}
% \end{declcs}
% Calls \meta{then} or \meta{else} depending on the existence
% of property \meta{propname}.
%
% \subsection{Reference generation}
%
% \begin{declcs}{zref@label}
%   \M{refname}
% \end{declcs}
% This works similar to \cs{label}. The reference \meta{refname} is
% created and put into the \xfile{.aux} file with the
% properties of the main property list.
%
% \begin{declcs}{zref@labelbylist}
%   \M{refname} \M{listname}
% \end{declcs}
% Same as \cs{zref@label} except that the properties are
% taken from the specified property list \meta{listname}.
%
% \begin{declcs}{zref@labelbyprops}
%   \M{refname} |{|\meta{propnameA}|,|\meta{propnameB}|,|\dots|}|
% \end{declcs}
% Same as \cs{zref@label} except that these properties
% are used that are given as comma separated list in the
% second argument.
%
% \begin{declcs}{zref@newlabel}
%   \M{refname} |{|\dots|}|
% \end{declcs}
% This is the macro that is used in the \xfile{.aux} file.
% It is basically the same as \cs{newlabel} apart from
% the format of the data in the second argument.
%
% \subsection{Data extraction}
%
% \begin{declcs}{zref@extractdefault}\markexptwo
%   \M{refname} \M{propname} \M{default}
% \end{declcs}
% This is the basic command that refernces the value of a
% property \meta{propname} for the reference \meta{refname}.
% In case of errors such as undefined reference the \meta{default}
% is used instead.
%
% \begin{declcs}{zref@extract}\markexptwo
%   \M{refname} \M{propname}
% \end{declcs}
% The command is an abbreviation for \cs{zref@extractdefault}.
% As default the default of the property is taken, otherwise
% the global default.
%
% Example for page references:
% \begin{quote}
% \begin{tabular}{@{}ll}
% \hologo{LaTeX}:& |\pageref{foobar}|\\
% \xpackage{zref}:& |\zref@extract{foobar}{page}|
% \end{tabular}
% \end{quote}
% Both \cs{zref@extract} and \cs{zref@extractdefault} are
% expandable. That means, these macros can directly
% be used in expandable calculations, see the example file.
% On the other side, babel's shorthands are not supported,
% there are no warnings in case of undefined references.
%
% If an user interface doesn't need expandable macros
% then it can use \cs{zref@refused} and \cs{zref@wrapper@babel}
% for its user macros.
%
% \begin{declcs}{zref@refused}
%   \M{refname}
% \end{declcs}
% This command is not expandable. It causes the warnings if
% the reference \meta{refname} is not defined. Use the
% \cs{zref@extract} commands inside expandable contexts and
% mark their use outside by \cs{zref@refused}, see the example
% file.
%
% \begin{declcs}{zref@def@extract}%
%     \M{cmd} \M{refname} \M{propname}\\
%   \cs{zref@def@extractdefault}
%     \M{cmd} \M{refname} \M{propname} \M{default}
% \end{declcs}
% Both macros extract the property \meta{propname} from the reference
% \meta{refname} the same way as macros \cs{zref@extract} and
% \cs{zref@extractdefault}. The result is stored in macro \meta{cmd}.
% Also \cs{zref@refused} is called to notify \hologo{LaTeX} that
% the reference \meta{refname} is used. Added in 2011/10/04 v2.22.
%
% \begin{declcs}{zref@ifrefundefined}\markexp
%   \M{refname} \M{then} \M{else}
% \end{declcs}
% Macro \cs{zref@ifrefundefined} calls arguments \meta{then} or \meta{else}
% dependent on the existence of the reference \meta{refname}.
%
% \begin{declcs}{zifrefundefined}
%   \M{refname} \M{then} \M{else}
% \end{declcs}
% Macro \cs{zifrefundefined} calls \cs{ref@refused} before
% executing \cs{zref@ifrefundefined}. Babel shorthands are
% supported in \meta{refname}.
%
% \begin{declcs}{zref@ifrefcontainsprop}\markexp
%   \M{refname} \M{propname} \M{then} \M{else}
% \end{declcs}
% Test whether a reference provides a property.
%
% \subsection{Setup}
%
% \begin{declcs}{zref@default}
% \end{declcs}
% Holds the global default for unknown values.
%
% \begin{declcs}{zref@setdefault}
%   |{|\meta{value}|}|
% \end{declcs}
% Sets the global default for unknown values.
% The global default is used, if a property does not specify
% an own default and the value for a property
% cannot be extracted. This can happen if the reference is
% unknown or the reference does not have the property.
%
% \begin{declcs}{zref@setmainlist}
%   |{|\meta{value}|}|
% \end{declcs}
% Sets the name of the main property list. The package
% sets and uses |main|.
%
% \subsection{Declared properties}
%
% \begin{quote}
% \renewcommand*{\arraystretch}{1.05}
% \setlength{\tabcolsep}{2\tabcolsep}
% \def\VV{\vphantom{\large \"Ag}}
% \def\EMPTY{{\itshape\textless empty\textgreater}}
% \def\pl#1{&#1&thepage&\cs{number}\cs{#1}}
% \begin{tabular}{@{}>{\sffamily}l|>{\ttfamily}l|>{\ttfamily}l|>{\ttfamily}l@{}}
%   \rmfamily Module&
%   \rmfamily Property&
%   \rmfamily Property list&
%   \rmfamily Default\\
%   \hline
%   (base)
%     & default & main & \EMPTY\\
%     & page & main & \EMPTY\\
%   \hline
%   abspage
%     & abspage & main & 0\\
%   \hline
%   counter
%     & counter & main & \EMPTY\\
%   \hline
%   hyperref
%     & anchor & main & \EMPTY\\
%     & url    & & \EMPTY\\
%   \hline
%   pageattr
%     & pdfpageattr & thepage & \dots\\
%     & pdfpagesattr & LastPage & \dots\\
%   \hline
%   pagelayout\footnotemark
%     \pl{mag}\\
%     \pl{paperwidth}\\
%     \pl{paperheight}\\
%     \pl{stockwidth}\\
%     \pl{stockheight}\\
%     \pl{pdfpageheight}\\
%     \pl{pdfpagewidth}\\
%     \pl{pdfhorigin}\\
%     \pl{pdfvorigin}\\
%     \pl{hoffset}\\
%     \pl{voffset}\\
%     \pl{topmargin}\\
%     \pl{oddsidemargin}\\
%     \pl{evensidemargin}\\
%     \pl{textwidth}\\
%     \pl{textheight}\\
%     \pl{headheight}\\
%     \pl{headsep}\\
%     \pl{footskip}\\
%     \pl{marginparwidth}\\
%     \pl{marginparsep}\\
%     \pl{columnwidth}\\
%     \pl{columnsep}\\
%   \hline
%   perpage
%     & pagevalue & perpage & 0\\
%     & page & perpage & \EMPTY\\
%     & abspage & perpage & 0\\
%   \hline
%   savepos
%     & posx & savepos & 0\\
%     & posy & savepos & 0\\
%   \hline
%   titleref
%     & title & main & \EMPTY\\
%   \hline
%   xr
%     & anchor   & & \EMPTY\\
%     & externaldocument & & \EMPTY\\
%     & theotype & & \EMPTY\\
%     & title    & & \EMPTY\\
%     & url      & & \EMPTY\\
% \end{tabular}
% \footnotetext{Module \xmodule{pagelayout} only defines properties
% if the parameter exists.}%
% \end{quote}
%
% \subsection{Wrapper for advanced situations}
%
% \begin{declcs}{zref@wrapper@babel}
%   |{|\dots|}| \M{name}
% \end{declcs}
% This macro helps to add shorthand support. The second
% argument is protected, then the code of the first argument
% is called with the protected name appended. Examples
% are in the sources.
%
% \begin{declcs}{zref@wrapper@immediate}
%   |{|\dots|}|
% \end{declcs}
% There are situations where a label must be written instantly
% to the \xfile{.aux} file, for example after the last page.
% If the \cs{zlabel} or \cs{label} command is put inside this wrapper,
% immediate writing is enabled. See the implementation for
% module \xmodule{lastpage} for an example of its use.
%
% \begin{declcs}{zref@wrapper@unexpanded}
%   |{|\dots|}|
% \end{declcs}
% Assuming someone wants to extract a value for property |bar|
% and store the result in a macro |\foo| without traces of
% the expanding macros and without expanding the value.
% This (theoretical?) problem can be solved by this wrapper:
% \begin{quote}
%\begin{verbatim}
%\zref@wrapper@unexpanded{%
%  \edef\foo{%
%    \zref@extract{someref}{bar}%
%  }%
%}
%\end{verbatim}
% \end{quote}
% The \cs{edef} forces the expansion of \cs{zref@extract},
% but the extraction of the value is prevented by
% the wrapper that uses \hologo{eTeX}'\ \cs{unexpanded} for this
% purpose. Supported macros are \cs{zref@extract}, \cs{zref@extractdefault}
% and since version 2010/04/22 v2.14 macro \cs{zref@getcurrent}.
%
% \subsection{Counter for unique names}
%
% Some modules (\xmodule{titleref} and \xmodule{dotfillmin})
% need unique names for automatically generated label names.
%
% \begin{declcs}{zref@require@unique}
% \end{declcs}
% This command creates the unique counter |zref@unique| if the
% counter does not already exist.
%
% \begin{declcs}{thezref@unique}
% \end{declcs}
% This command is used to generate unique label names.
%
% \section{User interface}\label{sec:user}
%
% \subsection{Module \xmodule{user}}
%
% The user interface for this package and its modules
% is enabled by \xpackage{zref}'s package option \xmodule{user}
% or package \xpackage{zref-user}.
% The names of user commands are prefixed by |z| in order
% to avoid name clashes with existing macros of the same
% functionality. Thus the package does not disturb the
% traditional reference scheme, both can be used together.
%
% The syntax descriptions contain the following markers
% that are intended as hints for programmers:
% \begin{quote}
% \begin{tabular}{@{}ll@{}}
%   \markbabel\unskip  & Babel shorthands are allowed.\\
%   \markrobust\unskip & Robust macro.\\
%   \markexp\unskip    & Expandable version:\\
%                      & \textbullet\ robust,
%                        unless the extracted values are fragile,\\
%                      & \textbullet\ no babel shorthand suport.\\
%   \markexptwo\unskip & Expandable like \markexp and:\\
%                      & \textbullet\ expandable in exact two steps.\\
% \end{tabular}
% \end{quote}
%
% The basic user interface of the package without modules
% are commands that mimic the standard \hologo{LaTeX} behaviour
% of \cs{label}, \cs{ref}, and \cs{pageref}:
%
% \begin{declcs}{zlabel}
%   \M{refname}\markbabel
% \end{declcs}
% Similar to \cs{label}. It generates a label with name
% \meta{refname} in the new reference scheme.
%
% \begin{declcs}{zref}
%   |[|\meta{propname}|]| \M{refname}\markbabel
% \end{declcs}
% Without optional argument similar to \cs{ref}, it returns
% the default reference property. This property is named
% |default|:
% \[
%   |\zref{|x|}| \equiv |\zref[default]{|x|}|
% \]
%
% \begin{declcs}{zpageref}
%   \M{refname}\markbabel
% \end{declcs}
% Convenience macro, similar to \cs{pageref}.
% \[
%   |\zpageref{|x|}| \equiv |\zref[page]{|x|}|
% \]
%
% \begin{declcs}{zrefused}
%   \M{refname}\markbabel
% \end{declcs}
% Some of the user commands in the modules are expandable.
% The use of such commands do not cause any undefined
% reference warnings, because inside of expandable
% contexts this is not possible. However, if there is
% a place outside of expandable contexts, \cs{refused}
% is strongly recommended. The reference \meta{refname}
% is marked as used, undefined ones will generate
% warnings.
%
% \subsection{Module \xmodule{abspage}}
%
% With the help of package \xpackage{atbegshi}
% a new counter |abspage| with absolute page numbers is provided.
% For technical and historical reasons the counter itself is zero based:
% if you use it directly on the first page, e.g
% with |\arabic{abspage}| you will get 0 as value. When using |\zref|
% the first page will be page 1 as expected.
% Also a new property |abspage| is defined and added to the
% main property list. Thus you can reference the absolute
% page number:
% \begin{quote}
%   |Section \zref{foo} is on page \zpageref{foo}.|\\
%   |This is page \zref[abspage]{foo}|\\
%   |of \zref[abspage]{LastPage}.|
% \end{quote}
% The example also makes use of module \xmodule{lastpage}.
%
% \subsection{Module \xmodule{lastpage}}
%
% Provides the functionality of package \xpackage{lastpage}
% \cite{lastpage} in the new reference scheme.
% The label |LastPage| is put at the end of the document.
% You can refer the last page number with:
% \begin{quote}
%   |\zref@extract{LastPage}{page}| (+ |\zref@refused{LastPage}|)\\
% \end{quote}
% or
% \begin{quote}
%   |\zpageref{LastPage}| (module \xmodule{user})
% \end{quote}
%
% Since version 2008/10/01 v2.3 the module defines the list |LastPage|.
% In addition to the properties of the main list label |LastPage| also
% stores the properties of this list |LastPage|. The default of this
% list is empty. The list can be used by the user to add additional
% properties for label |LastPage|.
%
% \subsubsection{Tests for last page}
%
% Since version 2010/03/26 v2.8 the macros \cs{zref@iflastpage}
% and \cs{ziflastpage} were added. They test the reference,
% whether it is a reference of the last page.
%
% \begin{declcs}{zref@iflastpage}\markexp \M{refname} \M{then} \M{else}
% \end{declcs}
% Macro \cs{zref@iflastpage} compares the references \meta{refname}
% with \meta{LastPage}. Basis of the comparison is the value of
% property |abspage|, because the values are different for different
% pages. This is not ensured by property |page|. Therefore module
% \xmodule{abspage} is loaded by module \xmodule{lastpage}.
% If both values of property |abspage| are present and match,
% then \meta{then} is executed, otherwise code \meta{else} is called.
% If one or both references are undefined or lack the property |abspage|,
% then \meta{else} is executed.
%
% Macro \cs{zref@iflastpage} is expandable, therefore \cs{zref@refused}
% should be called on \meta{refname} and \meta{LastPage}.
%
% \begin{declcs}{ziflastpage} \M{refname} \M{then} \M{else}
% \end{declcs}
% Macro \cs{ziflastpage} has the same function as \cs{zref@iflastpage},
% but adds support for babel shorthands in \meta{refname} and
% calls \cs{zref@refused}. However macro \cs{ziflastpage} is not
% expandable.
%
% \subsubsection{Example}
%
%    \begin{macrocode}
%<*example-lastpage>
%<<END_EXAMPLE
\NeedsTeXFormat{LaTeX2e}
\documentclass{report}

\newcounter{foo}
\renewcommand*{\thefoo}{\Alph{foo}}

\usepackage{zref-lastpage,zref-user}[2019/11/29]

\makeatletter
\zref@newprop{thefoo}{\thefoo}
\zref@newprop{valuefoo}{\the\value{foo}}
\zref@newprop{chapter}{\thechapter}
\zref@addprops{LastPage}{thefoo,valuefoo,chapter}
\makeatother

\newcommand*{\foo}{%
  \stepcounter{foo}%
  [Current foo: \thefoo]%
}

\begin{document}
  \chapter{First chapter}
  Last page is \zref{LastPage}.\\
  Last chapter is \zref[chapter]{LastPage}.\\
  Last foo is \zref[thefoo]{LastPage}.\\
  Last value of foo is \zref[valuefoo]{LastPage}.\\
  \foo
  \chapter{Second chapter}
  \foo\foo\foo
  \chapter{Last chapter}
  \foo
\end{document}
%END_EXAMPLE
%</example-lastpage>
%    \end{macrocode}
%
% \subsection{Module \xmodule{thepage}}
%
% This module \xmodule{thepage} loads module \xmodule{abspage},
% constructs a reference name using the absolute page number
% and remembers property |page|. Other properties can be added
% by adding them to the property list |thepage|.
%
% \begin{declcs}{zthepage} \M{absolute page number}\\
% \end{declcs}
% Macro \cs{zthepage} is basically a \cs{zpageref}. The reference
% name is yield by the \meta{absolute page number}. If the reference
% is not defined, then the default for property |page| is used.
%
% \begin{declcs}{zref@thepage@name}\markexp \M{absolute page number}
% \end{declcs}
% Macro \cs{zref@thepage@name} returns the internal reference
% name that is constructed using the \meta{absolute page number}.
% The internal reference name should not be used directly, because
% it might change in future versions.
%
% \begin{declcs}{zref@thepage}\markexp \M{absolute page number}\\
%   \cs{zref@thepage@refused} \M{absolute page number}
% \end{declcs}
% Macro \cs{zref@thepage} returns the page number (\cs{thepage})
% of \meta{absolute page number}. Because this macro is expandable,
% \cs{zref@thepage@refused} is used outside an expandable context
% to mark the reference as used.
%
%
% \subsection{Module \xmodule{nextpage}}
%
% \begin{declcs}{znextpage}
% \end{declcs}
% Macro \cs{znextpage} prints \cs{thepage} of the following page.
% It gets the current absolute page number by using a label.
% There are three cases for the next page:
% \begin{enumerate}
% \item The next page is not known yet because of undefined references.
%   Then \cs{zunknownnextpagename} is used instead. The default for
%   this macro is the default of property |page|.
% \item This page is the last page. Then \cs{znonextpagename}
%   is used. Its default is empty.
% \item The next page is known, then \cs{thepage} of the next
%   page is used (the value of property |page| of the next page).
% \end{enumerate}
%
% \subsubsection{Configuration}
%
% The behaviour can be configured by the following macros.
%
% \begin{declcs}{zunknownnextpagename}\\
%   \cs{znonextpagename}
% \end{declcs}
% If the next page is not known or available, then
% \cs{znextpage} uses these name macros as default.
% \cs{zunknownnextpagename} is used in case of undefined
% references. Default is the value of property |page| of the
% next page (\cs{thepage}). Module \xmodule{thepage} is used.
%
% Macro \cs{znonextpagename} is used, if the next page does
% not exists. That means that the current page is last page.
% The default is empty.
%
% \begin{declcs}{znextpagesetup} \M{unknown} \M{no next}
%   \M{next}
% \end{declcs}
% Acording to the case (see \cs{znextpage}) macro \cs{znextpage}
% calls an internal macro with an argument. The argument is
% either \cs{thepage} of the next page or one of \cs{zunknownnextpagename}
% or \cs{znonextpagename}. These internal macro can be changed by
% \cs{znextpagesetup}. It expects the definition texts for these
% three cases of a macro with one argument.
% The default is
% \begin{quote}
%   |\znextpagesetup{#1}{#1}{#1}|
% \end{quote}
%
% \subsubsection{Example}
%
%    \begin{macrocode}
%<*example-nextpage>
%<<END_EXAMPLE
\documentclass{book}

\usepackage{zref-nextpage}[2019/11/29]
\znextpagesetup
  {\thepage}% next page is unknown
  {\thepage\ (#1)}% this page is last page
  {\thepage\ $\rightarrow$ #1}% next page is known
\renewcommand*{\znonextpagename}{last page}

\usepackage{fancyhdr}
\pagestyle{fancy}
\fancyhf{}
\fancyhead[LE,RO]{\znextpage}
\fancypagestyle{plain}{%
  \fancyhf{}%
  \fancyhead[LE,RO]{\znextpage}%
}

\begin{document}
\frontmatter
  \tableofcontents
\mainmatter
  \chapter{Hello World}
  \clearpage
  \section{Last section}
\end{document}
%END_EXAMPLE
%</example-nextpage>
%    \end{macrocode}
%
%
% \subsection{Module \xmodule{totpages}}
%
% For the total number of pages of a document you need to
% know the absolute page number of the last page.
% Both modules \xmodule{abspage} and \xmodule{lastpage}
% are necessary and automatically enabled.
%
% \begin{declcs}{ztotpages}\markexp
% \end{declcs}
% Prints the total number of pages or |0| if this
% number is not yet known. It expands to an explicit number
% and can also used even in expandable calculations (\cs{numexpr})
% or counter assignments.
%
% \subsection{Module \xmodule{pagelayout}}
%
% The module defines additional properties for each parameter
% of the page layout that is effective during page shipout.
% The value of length parameters is given in sp without the
% unit as plain number.
%
% Some parameters are specific for a class (e.g.\@ \xoption{stockwidth}
% and \xoption{stockheight} for class \xclass{memoir}) or the
% \hologo{TeX} engine like \hologo{pdfTeX}. If the parameter is not
% available, then the property will not be defined. The default value
% of the property is the current setting of the parameter.
%
% The module \xmodule{thepage} is loaded that generates a label
% for each page. The properties of module \xmodule{pagelayout}
% are added to the property list |thepage| of module \xmodule{thepage}.
%
% List of properties:
% \begin{quote}
%   \def\x#1{\cs{#1}&#1}%
%   \begin{tabular}{l>{\sffamily}ll}
%     \textrm{parameter} & \textrm{property} & remarks\\
%     \hline
%     \x{mag}\\
%     \x{paperwidth}\\
%     \x{paperheight}\\
%     \x{stockwidth}& class \xclass{memoir}\\
%     \x{stockheight}& class \xclass{memoir}\\
%     \x{pdfpagewidth}& \hologo{pdfTeX}, \hologo{LuaTeX}\\
%     \x{pdfpageheight}& \hologo{pdfTeX}, \hologo{LuaTeX}\\
%     \x{pdfhorigin}& \hologo{pdfTeX}, \hologo{LuaTeX}\\
%     \x{pdfvorigin}& \hologo{pdfTeX}, \hologo{LuaTeX}\\
%     \x{hoffset}\\
%     \x{voffset}\\
%     \x{topmargin}\\
%     \x{oddsidemargin}\\
%     \x{evensidemargin}\\
%     \x{textwidth}\\
%     \x{textheight}\\
%     \x{headheight}\\
%     \x{headsep}\\
%     \x{footskip}\\
%     \x{marginparwidth}\\
%     \x{marginparsep}\\
%     \x{columnwidth}\\
%     \x{columnsep}\\
%   \end{tabular}
% \end{quote}
%
% \begin{declcs}{zlistpagelayout}
% \end{declcs}
% At the end of document the page layout parameter for each page are
% printed into the \xext{log} file if macro \cs{zlistpagelayout}
% is called before |\end{document}| (preamble is a good place).
%
% \subsection{Module \xmodule{marks}}
%
% ToDo.
%
% \subsection{Module \xmodule{runs}}
%
% Module \xmodule{runs} counts the \hologo{LaTeX} runs since last
% \xext{aux} file creation and prints the number in the
% \xext{log} file.
%
% \begin{declcs}{zruns}\markexp
% \end{declcs}
% Prints the the total number of \hologo{LaTeX} runs including
% the current one. It expands to an explicit number.
% Before |begin{document}| the value is zero meaning
% the \xext{aux} file is not read yet. If a previous
% \xext{aux} file exists, the value found there increased
% by one is the new number. Otherwise \cs{zruns} is set to one.
% \hologo{LaTeX}~runs where the \xext{aux} files are not rewritten
% are not counted (see \cs{nofiles}).
%
% \subsection{Module \xmodule{perpage}}
%
% With \cs{@addtoreset} or \cs{numberwithin} a counter can be
% reset if another counter is incremented. This do not work
% well if the other counter is the page counter. The page counter is
% incremented in the output routine that is often called asynchronous
% somewhere on the next page. A reference mechanism costs at least
% two \hologo{LaTeX} runs, but ensures correct page counter values.
%
% \begin{declcs}{zmakeperpage}
%   |[|\meta{reset}|]| \M{counter}
% \end{declcs}
% At the of a new page counter \meta{counter} starts
% counting with value \meta{reset} (default is \texttt{1}).
% The macro has the same
% syntax and semantics as \cs{MakePerPage} of package
% \xpackage{perpage} \cite{perpage}.
% Also \xoption{perpage} of package \xpackage{footmisc} \cite{footmisc}
% can easily be simulated by
% \begin{quote}
% |\zmakeperpage{footnote}| \itshape |% \usepackage[perpage]{footmisc}|
%\end{quote}
% If footnote symbols are used, some people dislike the
% first symbol \ensuremath{\dagger}. It can easily be skipped:
% \begin{quote}
% |\zmakeperpage[2]{footnote}|
% \end{quote}
%
% \begin{declcs}{thezpage}\\
%   counter |zpage|
% \end{declcs}
% If the formatted counter value of the counter that is reset at
% a new page contains the page value, then you can use \cs{thezpage},
% the page number of the current page. Or counter |zpage| can be
% used, if the page number should be formatted differently from
% the current page number. Example:
%\begin{quote}
%\begin{verbatim}
%\newcounter{foobar}
%\zmakeperpage{foobar}
%\renewcommand*{\thefoobar}{\thezpage-\arabic{foobar}}
% % or
%\renewcommand*{\thefoobar}{\roman{zpage}-\arabic{foobar}}
%\end{verbatim}
%\end{quote}
%
% \begin{declcs}{zunmakeperpage}
%   \M{counter}
% \end{declcs}
% The reset mechanism for this counter is deactivated.
%
%
% \subsection{Module \xmodule{counter}}
%
% This option just add the property |counter| to the main
% property list. The property stores the counter name,
% that was responsible for the reference. This is the
% property \xpackage{hyperref}'s \cs{autoref} feature
% uses. Thus this property |counter| may be useful
% for a reimplementation of the autoref feature,
% see the section \ref{sec:todo} with the todo list.
%
% \subsection{Module \xmodule{titleref}}
%
% This option makes section and caption titles available
% to the reference system similar to packages \xpackage{titleref}
% or \xpackage{nameref}.
%
% \begin{declcs}{ztitleref}
%   \M{refname}\markbabel
% \end{declcs}
% Print the section or caption title of reference \meta{refname},
% similar to \cs{nameref} or \cs{titleref}.
%
% \begin{declcs}{ztitlerefsetup}
%   |{|\textit{key$_1$=value$_1$, key$_2$=value$_2$, \dots}|}|
% \end{declcs}
% This command allows to configure the behaviour of module
% \xmodule{titleref}. The following keys are available:
% \begin{description}
% \let\bfseries\relax
% \item[\texttt{title=}\meta{value}]\mbox{}\\*
%   Sets the current title.
% \item[\texttt{stripperiod=true\string|false}]\mbox{}\\*
%   Follow package \xpackage{nameref} that
%   removes a last period. Default: |true|.
% \item[\texttt{expand=true\string|false}]\mbox{}\\*
%   Package \cs{titleref} expands the title first. This way
%   garbage and dangerous commands can be removed, e.g.
%   \cs{label}, \cs{index}\dots. See implementation section
%   for more details. Default is |false|.
% \item[\texttt{cleanup=\{}\textrm{\dots}\texttt{\}}]\mbox{}\\*
%   Hook to add own cleanup code, if method |expand| is used.
%   See implementation section for more details.
% \end{description}
%
% \subsection{Module \xmodule{savepos}}
%
% This option supports a feature that \pdfTeX\ provides
% (and \hologo{XeTeX}).
% \pdfTeX\ is able to tell the current position on the
% page. The page position is not instantly known. First the
% page must be constructed by \hologo{TeX}'s asynchronous output
% routine. Thus the time where the position is known is
% the page shipout time. Thus a reference system where the
% information is recorded in the first run and made available
% for use in the second run comes in handy.
%
% \begin{declcs}{zsavepos}
%   \M{refname}
% \end{declcs}
% It generates a reference with name \meta{refname}.
% The reference stores the location where \cs{zsavepos}
% is executed in properties |posx| and |posy|.
%
% \begin{declcs}{zsaveposx}
%   \M{refname}\\
%   \cs{zsaveposy}
%   \M{refname}
% \end{declcs}
% Same as \cs{zsavepos} except that only the |x| or |y| component
% of the position is stored. Since 2011/12/05 v2.23.
%
% \begin{declcs}{zposx}\markexp
%   \M{refname}\\
%   \SpecialUsageIndex{\zposy}\cs{zposy}\markexp \M{refname}
% \end{declcs}
% Get the position as number. Unit is sp. Horizontal
% positions by \cs{zposx} increase from left to right.
% Vertical positions by \cs{zposy} from bottom to top.
%
% Do not rely on absolute page numbers. Because of problems
% with the origin the numbers may differ in DVI or PDF mode
% of \pdfTeX. Therefore work with relative values by comparisons.
%
% Both \cs{zposx} and \cs{zposy} are expandable and can be
% used inside calculations (\cs{setcounter}, \cs{addtocounter},
% package \xpackage{calc}, \cs{numexpr}). However this property
% prevents from notifying \hologo{LaTeX} that the reference is actually
% used (the notifying is not expandable). Therefore you should
% mark the reference as used by \cs{zrefused}.
%
% This module uses \pdfTeX's \cs{pdfsavepos}, \cs{pdflastxpos},
% and \cs{pdflastypos}. They are available in PDF mode and
% since version 1.40.0 also in DVI mode.
%
% \begin{declcs}{zref@savepos}
% \end{declcs}
% Macro \cs{zref@savepos} performs the first part of \cs{zsavepos}
% by calling \cs{pdfsavepos} (if \xext{aux} files are writable).
%
% Thus \cs{zsavepos} is basically \cs{zref@savepos} followed
% by |\zref@labelbylist{|\texttt{\meta{refname}}|}{savepos}|.
% If \cs{TeXXeTstate} is detected and enabled, \cs{savepos}
% also adds \cs{zref@savepos} at the end to support \cs{beginR}
% where the whatits are processed in reverse order.
% The property list |savepos| contains the properties |posx|
% and |posy|.
%
% \subsection{Module \xmodule{abspos}}
% Module \xmodule{abspos} allows to get various values of the page layout. There is no user
% command, only a number of internal commands. For example:
% \begin{quote}
%   |\zref@absposx{|\meta{label}|}{|\meta{value}|}{|\meta{position}|}|\\
%   |\zref@absposy{|\meta{label}|}{|\meta{value}|}{|\meta{position}|}|
% \end{quote}
%
% The return value is like in the module \xmodule{savepos} a number representing a length in sp.
% The length are measured from the bottom left of the page.
%
% \meta{label} is a label set with \cs{zlabel} or \cs{zsavepos} that allows to retrieve the
% absolute page number.
%
% \meta{position} is for the x-command one of |left|, |right| or |center|. For the y-command
%  it is one of |top|, |bottom|, |center|.
%
% The possible content of \meta{value} can be seen in the following table. Be aware that the code
% makes some assumptions which are perhaps not always true -- for example that the left of
% the head is identical to the left of the body.
%
% \medskip
% \begin{tabular}{lll}
%  \textbf{value} & \textbf{axis} & \textbf{comments}\\\hline
%   media         & x             &   left=0, right=\cs{pdfpagewidth} \\
%   paper         & x             &   left=0, right=\cs{paperwidth}\\
%   stock         & x             &   derived from paper\\
%   media         & y             & bottom=0, top=\cs{pdfpageheigh}\\
%   paper         & y             & top=\cs{pdfpageheight}, bottom=top-\cs{paperheight}\\
%   stock         & y             & top  derived from paper\\
%   head          & x             & calculated with hoffset, horigin, etc\\
%   head          & y             & calculated\\
%   body          & x             & = head value\\%
%   body          & y             & = head bottom - \cs{headsep}\\
%   foot          & x             & = head\\
%   foot          & y             & calculated from body bottom and \cs{footskip}\\
%   marginpar     & x             &different on odd/even pages! \\
%   marginpar     & y             & = body %
% \end{tabular}
%
% \subsection{Module \xmodule{dotfill}}
%
% \begin{declcs}{zdotfill}
% \end{declcs}
% This package provides the command \cs{zdotfill} that works
% similar to \cs{dotfill}, but can be configured. Especially
% it suppresses the dots if a minimum number of dots cannot be set.
%
% \begin{declcs}{zdotfillsetup}
%   |{|\textit{key$_1$=value$_1$, key$_2$=value$_2$, \dots}|}|
% \end{declcs}
% This command allows to configure the behaviour of \cs{zdotfill}.
% The following keys are available:
% \begin{description}
% \let\bfseries\relax
% \item[\texttt{min=}\meta{count value}]\mbox{}\\*
%   If the actual number of dots are smaller than \meta{count value},
%   then the dots are suppressed. Default: |2|.
% \item[\texttt{unit=}\meta{dimen value}]\mbox{}\\*
%   The width of a dot unit is given by \meta{dimen value}.
%   Default: |0.44em| (same as the unit in \cs{dotfill}).
% \item[\texttt{dot=}\meta{value}]\mbox{}\\*
%   The dot itself is given by \meta{value}. Default: |.| (dot, same as
%   the dot in \cs{dotfill}).
% \end{description}
%
% \subsection{Module \xmodule{env}}
%
% This module defines two properties |envname| and |envline|. They
% remember the name of the environment and the line number at
% the start of the environment.
%
% \subsection{Module \xmodule{xr}}
%
% This package provides the functionality of package \xpackage{xr},
% see \cite{xr}. It also supports the syntax of \xpackage{xr-hyper}.
%
% \begin{declcs}{zexternaldocument}
%   \* |[|\meta{prefix}|]|\markbabel \M{external document}
%   |[|\meta{url}|]|
% \end{declcs}
% See \cs{externaldocument} for a description of this option.
% The found labels also get a property |externaldocument| that
% remembers \meta{external document}.
% The standard reference scheme and the scheme of this
% package use different name spaces for reference names.
% If the external document uses both systems. Then one
% import statement would put the names in one namespace
% and probably causing problems with multiple references
% of the same name. Thus the star form only looks for
% \cs{newlabel} in the \xfile{.aux} files, whereas without
% star only \cs{zref@newlabel}s are used.
%
% In the star form it tries to detect labels from
% \xpackage{hyperref}, \xpackage{titleref}, and \xpackage{ntheorem}.
% If such an extended property from the packages before cannot
% be found or are empty, they are not included in the
% imported reference.
%
% Warnings are given if a reference name is already in use
% and the item is ignored. Unknown properties will automatically
% be declared.
%
% If the external references contain |anchor| properties,
% then we need also a url to be able to address the external
% file. As default the filename is taken with a default
% extension.
%
% \begin{declcs}{zxrsetup}
%   |{|\textit{key$_1$=value$_1$, key$_2$=value$_2$, \dots}|}|
% \end{declcs}
% The following setup options are available:
% \begin{description}
% \item[\xoption{ext}:] It sets the default extension.
% \item[\xoption{tozreflabel}:] Boolean option. The found references
%   are imported as zref labels. This is enabled by default.
% \item[\xoption{toltxlabel}:] Boolean option. The found references
%   are imported as \hologo{LaTeX} labels. Packages \xpackage{nameref},
%   \xpackage{titleref} and class \xclass{memoir} are supported.
% \item[\xoption{urluse}:] Boolean option. If enabled, then a URL
%   is stored in a macro and the macro is put in property `urluse'.
%   The URL is not put in property `url'.
%   The purpose is to save \hologo{TeX} memory.
% \item[\xoption{verbose}:] Boolean option. List the imported labels in the
%   \xext{log} file. Default is |false|.
% \end{description}
%
% \begin{declcs}{zref@xr@ext}
% \end{declcs}
% If the \meta{url} is not specified in \cs{zref@externaldocument},
% then the url will be constructed with the file name and
% this macro as extension. \cs{XR@ext} is used if
% \xpackage{hyperref} is loaded, otherwise |pdf|.
%
% \subsection{Module \xmodule{pageattr}}
%
% This module allows to recover the content of the register
% \cs{pdfpageattr} and \cs{pdfpagesattr} in pdftex and the equivalent register
% in luatex. There is no user command. Programmer commands are
%
% \begin{quote}
% |\zref@pdfpageattr{|\meta{absolute page number}|}|
%
% |\zref@pdfpagesattr{|\meta{absolute page number}|}|
% \end{quote}
% \section{ToDo}\label{sec:todo}
%
% Among other things the following issues are left for future work:
% \begin{itemize}
% \item Other applications: autoref, hyperref, \dots
% \end{itemize}
%
% \section{Example}
%
%    \begin{macrocode}
%<*example>
%    \end{macrocode}
%
%    \begin{macrocode}
\documentclass{book}

\usepackage[ngerman]{babel}%

\usepackage[savepos,totpages,titleref,dotfill,counter,user]{zref}

%    \end{macrocode}
%
%    Chapters are wrapped inside \cs{ChapterStart} and
%    \cs{ChapterStop}. The first argument \verb|#1| of \cs{ChapterStart}
%    is used to form a label id \verb|chap:#1|. At the end of
%    the chapter another label is set by \cs{zref@wrapper@immediate}, because
%    otherwise at the end of document a deferred write would not
%    be written, because there is no page for shipout.
%
%    Also this example shows how chapter titles can be recorded.
%    A new property \verb|chaptitle| is declared and added to the
%    main property list. In \cs{ChapterStart} the current value
%    of the property is updated.
%    \begin{macrocode}
\makeatletter
\zref@newprop{chaptitle}{}
\zref@addprop{main}{chaptitle}

\newcommand*{\ChapterStart}[2]{%
  \cleardoublepage
  \def\current@chapid{#1}%
  \zref@setcurrent{chaptitle}{#2}%
  \chapter{#2}%
  \zlabel{chap:#1}%
}
\newcommand*{\ChapterStop}{%
  \cleardoublepage
  \zref@wrapper@immediate{%
    \zref@labelbyprops{chapend:\current@chapid}{abspage}%
  }%
}
%    \end{macrocode}
%    \cs{ChapterPages} calculates and returns the number of pages of the
%    referenced chapter.
%    \begin{macrocode}
\newcommand*{\ChapterPages}[1]{%
  \zrefused{chap:#1}%
  \zrefused{chapend:#1}%
  \number\numexpr
    \zref@extract{chapend:#1}{abspage}%
    -\zref@extract{chap:#1}{abspage}%
    +1\relax
}
\makeatother
%    \end{macrocode}
%    \begin{macrocode}
\begin{document}
%    \end{macrocode}
%    As exception we use \cs{makeatletter} here, because
%    this is just an example file that also should show some of
%    programmer's interface.
%    \begin{macrocode}
\makeatletter

\frontmatter
\zlabel{documentstart}

\begin{itemize}
\item
  The frontmatter part has
  \number\numexpr\zref@extract{chap:first}{abspage}-1\relax
  ~pages.
\item
  Chapter \zref{chap:first} has \ChapterPages{first} page(s).
\item
  Section \zref{hello} is on the
  \ifcase\numexpr
    \zref@extractdefault{hello}{page}{0}%
    -\zref@extractdefault{chap:first}{page}{0}%
    +1\relax
    ??\or first\or second\or third\or forth\fi
  ~page inside its chapter.
\item
  The document has
  \zref[abspage]{LastPage} pages.
  This number is \ifodd\ztotpages odd\else even\fi.
\item
  The last page is labeled with \zpageref{LastPage}.
\item
  The title of chapter \zref{chap:next} %
  is ``\zref[chaptitle]{chap:next}''.
\end{itemize}

\tableofcontents

\mainmatter
\ChapterStart{first}{First chapter}

%    \end{macrocode}
%    The user level commands should protect babel shorthands
%    where possible. On the other side, expandable extracting
%    macros are useful in calculations, see above the
%    examples with \cs{numexpr}.
%    \begin{macrocode}
\section{Test}
\zlabel{a"o}
Section \zref{a"o} on page
\zref@wrapper@babel\zref@extract{a"o}{page}.

Text.
\newpage

\section{Hello World}
\zlabel{hello}

\ChapterStop

\ChapterStart{next}{Next chapter with \emph{umlauts}: "a"o"u"s}

%    \end{macrocode}
%
%    Here an example follows that makes use of \pdfTeX's ``savepos''
%    feature. The position on the page is not known before the
%    page is constructed and shipped out. Therefore the position
%    ist stored in references and are available for calculations
%    in the next \hologo{LaTeX} compile run.
%    \begin{macrocode}
The width of the first column is
  \the\dimexpr \zposx{secondcol}sp - \zposx{firstcol}sp\relax,\\
the height difference of the two baselines is
  \the\dimexpr \zposy{firstcol}sp - \zposy{secondline}sp\relax:\\
\begin{tabular}{ll}
  \zsavepos{firstcol}Hello&\zsavepos{secondcol}World\\
  \zsavepos{secondline}Second line&foobar\\
\end{tabular}

%    \end{macrocode}
%    With \cs{zrefused} \hologo{LaTeX} is notified, if the references
%    are not yet available and \hologo{LaTeX} can generate the rerun
%    hint.
%    \begin{macrocode}
\zrefused{firstcol}
\zrefused{secondcol}
\zrefused{secondline}

\ChapterStop
%    \end{macrocode}
%    Test for module \cs{dotfill}.
%    \begin{macrocode}
\ChapterStart{dotfill}{Test for dotfill feature}
\newcommand*{\dftest}[1]{%
  #1&
  [\makebox[{#1}]{\dotfill}]&
  [\makebox[{#1}]{\zdotfill}]\\
}
\begin{tabular}{rll}
& [\verb|\dotfill|] & [\verb|\zdotfill|]\\
\dftest{0.43em}
\dftest{0.44em}
\dftest{0.45em}
\dftest{0.87em}
\dftest{0.88em}
\dftest{0.89em}
\dftest{1.31em}
\dftest{1.32em}
\dftest{1.33em}
\end{tabular}
\ChapterStop
\end{document}
%    \end{macrocode}
%
%    \begin{macrocode}
%</example>
%    \end{macrocode}
%
% \StopEventually{
% }
%
% \section{Implementation}
%
% \subsection{Package \xpackage{zref}}
%
% \subsubsection{Identification}
%
%    \begin{macrocode}
%<*package>
\NeedsTeXFormat{LaTeX2e}
\ProvidesPackage{zref}
  [2020-07-03 v2.32 A new reference scheme for LaTeX (HO)]%
%    \end{macrocode}
%
% \subsubsection{Load basic module}
%
%    \begin{macrocode}
\RequirePackage{zref-base}[2019/11/29]
%    \end{macrocode}
%    Abort package loading if \xpackage{zref-base} could not be
%    loaded successfully.
%    \begin{macrocode}
\@ifundefined{ZREF@base@ok}{\endinput}{}
%    \end{macrocode}
%
% \subsubsection{Process options}
%
%    Known modules are loaded and the release date is checked.
%    \begin{macrocode}
\def\ZREF@temp#1{%
  \DeclareOption{#1}{%
    \AtEndOfPackage{%
      \RequirePackage{zref-#1}[2019/11/29]%
    }%
  }%
}
\ZREF@temp{abspage}
\ZREF@temp{counter}
\ZREF@temp{dotfill}
\ZREF@temp{hyperref}
\ZREF@temp{lastpage}
\ZREF@temp{marks}
\ZREF@temp{nextpage}
\ZREF@temp{pageattr}
\ZREF@temp{pagelayout}
\ZREF@temp{perpage}
\ZREF@temp{runs}
\ZREF@temp{savepos}
\ZREF@temp{thepage}
\ZREF@temp{titleref}
\ZREF@temp{totpages}
\ZREF@temp{user}
\ZREF@temp{xr}
%    \end{macrocode}
%    \begin{macrocode}
\ProcessOptions\relax
%</package>
%    \end{macrocode}
%
%
% \subsection{Module \xmodule{base}}
%
% \subsubsection{Prefixes}
%
%    This package uses the following prefixes for macro names:
%    \begin{description}
%    \item[\cs{zref@}:] Macros of the programmer's interface.
%    \item[\cs{ZREF@}:] Internal macros.
%    \item[\cs{Z@L@}\metacs{listname}:]
%      The properties of the list \meta{listname}.
%    \item[\cs{Z@D@}\metacs{propname}:]
%      The default value for property \meta{propname}.
%    \item[\cs{Z@E@}\metacs{propname}:]
%      Extract function for property \meta{propname}.
%    \item[\cs{Z@X@}\metacs{propname}:]
%      Information whether a property value for property
%      \meta{propname} is expanded
%      immediately or at shipout time.
%    \item[\cs{Z@C@}\metacs{propname}:]
%      Current value of the property \meta{propname}.
%    \item[\cs{Z@R@}\metacs{labelname}:]
%      Data for reference \meta{labelname}.
%    \item[\cs{ZREF@org@}:] Original versions of patched commands.
%    \item[\cs{z}:] For macros in user land, defined if
%      module \xmodule{user} is set.
%    \end{description}
%    The following family names are used for keys defined
%    according to the \xpackage{keyval} package:
%    \begin{description}
%    \item[\texttt{ZREF@TR}:] Setup for module \xmodule{titleref}.
%    \end{description}
%
% \subsubsection{Identification}
%
%    \begin{macrocode}
%<*base>
\NeedsTeXFormat{LaTeX2e}
\ProvidesPackage{zref-base}%
  [2020-07-03 v2.32 Module base for zref (HO)]%
%    \end{macrocode}
%
% \subsubsection{Utilities}
%
%    \begin{macrocode}
\providecommand\IfFormatAtLeastTF{\@ifl@t@r\fmtversion}
\RequirePackage{ltxcmds}[2010/12/02]
\RequirePackage{infwarerr}[2010/04/08]
\RequirePackage{kvsetkeys}[2010/03/01]
\RequirePackage{kvdefinekeys}[2010/03/01]
\RequirePackage{pdftexcmds}[2010/04/01]
%    \end{macrocode}
%
%    \begin{macro}{\ZREF@name}
%    Several times the package name is used, thus we store it
%    in \cmd{\ZREF@name}.
%    \begin{macrocode}
\def\ZREF@name{zref}
%    \end{macrocode}
%    \end{macro}
%
%    \begin{macrocode}
\ltx@IfUndefined{protected}{%
  \RequirePackage{makerobust}[2006/03/18]%
%    \end{macrocode}
%    \begin{macro}{\ZREF@Robust}
%    \begin{macrocode}
  \def\ZREF@Robust#1#2{%
    \def\ZREF@temp{\MakeRobustcommand#2}%
    \afterassignment\ZREF@temp
    #1#2%
  }%
%    \end{macrocode}
%    \end{macro}
%    \begin{macrocode}
}{%
%    \end{macrocode}
%    \begin{macro}{\ZREF@Robust}
%    \begin{macrocode}
  \def\ZREF@Robust#1{%
    \protected#1%
  }%
%    \end{macrocode}
%    \end{macro}
%    \begin{macrocode}
}
%    \end{macrocode}
%    \begin{macro}{\ZREF@IfDefinable}
%    \begin{macrocode}
\def\ZREF@IfDefinable#1#2#3{%
  \@ifdefinable{#1}{%
    \ZREF@Robust{#2}#1#3%
  }%
}
%    \end{macrocode}
%    \end{macro}
%
%    \begin{macro}{\ZREF@UpdatePdfTeX}
%    \cs{ZREF@UpdatePdfTeX} is used as help message text in
%    error messages.
%    \begin{macrocode}
\def\ZREF@UpdatePdfTeX{Update pdfTeX.}
%    \end{macrocode}
%    \end{macro}
%
%    \begin{macro}{\ifZREF@found}
%    The following switch is usded in list processing.
%    \begin{macrocode}
\newif\ifZREF@found
%    \end{macrocode}
%    \end{macro}
%
%    \begin{macro}{\ZREF@patch}
%    Macro \cs{ZREF@patch} first checks the existence of
%    the command and safes it.
%    \begin{macrocode}
\def\ZREF@patch#1{%
  \ltx@IfUndefined{#1}{%
    \ltx@gobble
  }{%
    \expandafter\let\csname ZREF@org@#1\expandafter\endcsname
    \csname #1\endcsname
    \ltx@firstofone
  }%
}
%    \end{macrocode}
%    \end{macro}
%
% \subsubsection{Check for \hologo{eTeX}}
%
%    The use of \hologo{eTeX} should be standard nowadays for \hologo{LaTeX}.
%    We test for \hologo{eTeX} in order to use its features later.
%    \begin{macrocode}
\ltx@IfUndefined{eTeXversion}{%
  \PackageError\ZREF@name{%
    Missing support for eTeX; package is abandoned%
  }{%
    Use a TeX compiler that support eTeX and enable eTeX %
    in the format.%
  }%
  \endinput
}{}%
%    \end{macrocode}
%
%    \begin{macrocode}
\RequirePackage{etexcmds}[2007/09/09]
\ifetex@unexpanded
\else
  \PackageError\ZREF@name{%
    Missing e-TeX's \string\unexpanded.\MessageBreak
    Add \string\RequirePackage\string{etexcmds\string} before %
    \string\documentclass%
  }{%
    Probably you are using some package (e.g. ConTeXt) that %
    redefines \string\unexpanded%
  }%
  \expandafter\endinput
\fi
%    \end{macrocode}
%
% \subsubsection{Auxiliary file stuff}
%
%    We are using some commands in the \xfile{.aux} files.
%    However sometimes these auxiliary files are interpreted
%    by \hologo{LaTeX} processes that haven't loaded this package
%    (e.g. package \xpackage{xr}).
%    Therefore we provide dummy definitions.
%    \begin{macrocode}
\RequirePackage{auxhook}
\AddLineBeginAux{%
  \string\providecommand\string\zref@newlabel[2]{}%
}
%    \end{macrocode}
%    \begin{macro}{\ZREF@RefPrefix}
%    \begin{macrocode}
\def\ZREF@RefPrefix{Z@R}
%    \end{macrocode}
%    \end{macro}
%    \begin{macro}{\zref@newlabel}
%    For the implementation of \cs{zref@newlabel} we call
%    the same internal macro \cs{@newl@bel} that is used
%    in \cs{newlabel}. Thus we have for free:
%    \begin{itemize}
%    \item \cs{Z@R@}\metacs{labelname} is defined.
%    \item \hologo{LaTeX}'s check for multiple references.
%    \item \hologo{LaTeX}'s check for changed references.
%    \end{itemize}
%    \begin{macrocode}
\ZREF@Robust\edef\zref@newlabel{%
  \noexpand\@newl@bel{\ZREF@RefPrefix}%
}
%    \end{macrocode}
%    \end{macro}
%
% \subsubsection{Property lists}
%
%    \begin{macro}{\zref@newlist}
%    Property lists are stored as list of property names
%    enclosed in curly braces. \cs{zref@newlist}
%    creates a new list as empty list. Assignments to
%    property lists are global.
%    \begin{macrocode}
\ZREF@Robust\def\zref@newlist#1{%
  \zref@iflistundefined{#1}{%
    \@ifdefinable{Z@L@#1}{%
     \global\expandafter\let\csname Z@L@#1\endcsname\ltx@empty
     \PackageInfo\ZREF@name{New property list: #1}%
    }%
  }{%
    \PackageError\ZREF@name{%
      Property list `#1' already exists%
    }\@ehc
  }%
}
%    \end{macrocode}
%    \end{macro}
%
%    \begin{macro}{\zref@iflistundefined}
%    \cs{zref@iflistundefined} checks the existence of the
%    property list \verb|#1|. If the property list is present,
%    then \verb|#2| is executed and \verb|#3| otherwise.
%    \begin{macrocode}
\def\zref@iflistundefined#1{%
  \ltx@ifundefined{Z@L@#1}%
}
%    \end{macrocode}
%    \end{macro}
%    \begin{macro}{\zref@listexists}
%    \cs{zref@listexists} only executes \verb|#2| if the
%    property list \verb|#1| exists and raises an error
%    message otherwise.
%    \begin{macrocode}
\ZREF@Robust\def\zref@listexists#1{%
  \zref@iflistundefined{#1}{%
    \PackageError\ZREF@name{%
      Property list `#1' does not exist%
    }\@ehc
  }%
}
%    \end{macrocode}
%    \end{macro}
%
%    \begin{macro}{\zref@iflistcontainsprop}
%    \cs{zref@iflistcontainsprop} checks, whether a property \verb|#2| is
%    already present in a property list \verb|#1|.
%    \begin{macrocode}
\ZREF@Robust\def\zref@iflistcontainsprop#1#2{%
  \zref@iflistundefined{#1}{%
    \ltx@secondoftwo
  }{%
    \begingroup\expandafter\endgroup
    \expandafter\in@
    \csname#2\expandafter\expandafter\expandafter\endcsname
    \expandafter\expandafter\expandafter{\csname Z@L@#1\endcsname}%
    \csname ltx@\ifin@ first\else second\fi oftwo\endcsname
  }%
}
%    \end{macrocode}
%    \end{macro}
%
%    \begin{macro}{\zref@listforloop}
%    \begin{macrocode}
\def\zref@listforloop#1#2{%
  \zref@listexists{#1}{%
    \expandafter\expandafter\expandafter\@tfor
    \expandafter\expandafter\expandafter\zref@prop
    \expandafter\expandafter\expandafter:%
    \expandafter\expandafter\expandafter=%
    \csname Z@L@#1\endcsname
    \do{%
      \begingroup
        \escapechar=-1 %
        \edef\x{\endgroup
          \def\noexpand\zref@prop{%
            \expandafter\string\zref@prop
          }%
        }%
      \x
      #2\zref@prop
    }%
  }%
}
%    \end{macrocode}
%    \end{macro}
%
%    \begin{macro}{\zref@addprops}
%    \cs{zref@addprop} adds the properties \verb|#2| to the property
%    list \verb|#1|, if the property is not already in the list.
%    Otherwise a warning is given.
%    \begin{macrocode}
\ZREF@Robust\def\zref@addprops#1#2{%
  \zref@listexists{#1}{%
    \comma@parse{#2}{%
      \zref@propexists\comma@entry{%
        \zref@iflistcontainsprop{#1}\comma@entry{%
          \PackageWarning\ZREF@name{%
            Property `\comma@entry' is already in list `#1'%
          }%
        }{%
          \begingroup\expandafter\endgroup
          \expandafter\g@addto@macro
          \csname Z@L@#1\expandafter\endcsname
          \expandafter{\csname\comma@entry\endcsname}%
        }%
      }%
      \ltx@gobble
    }%
  }%
}
%    \end{macrocode}
%    \end{macro}
%    \begin{macro}{\zref@addprop}
%    \cs{zref@addprop} adds the property \verb|#2| to the property
%    list \verb|#1|, if the property is not already in the list.
%    Otherwise a warning is given.
%    \begin{macrocode}
\ZREF@Robust\def\zref@addprop#1#2{%
  \zref@listexists{#1}{%
    \zref@propexists{#2}{%
      \zref@iflistcontainsprop{#1}{#2}{%
        \PackageWarning\ZREF@name{%
          Property `#2' is already in list `#1'%
        }%
      }{%
        \begingroup\expandafter\endgroup
        \expandafter\g@addto@macro
        \csname Z@L@#1\expandafter\endcsname
        \expandafter{\csname#2\endcsname}%
      }%
    }%
  }%
}
%    \end{macrocode}
%    \end{macro}
%
%    \begin{macro}{\zref@localaddprops}
%    \begin{macrocode}
\ZREF@Robust\def\zref@localaddprops#1#2{%
  \zref@listexists{#1}{%
    \comma@parse{#2}{%
      \zref@propexists\comma@entry{%
        \zref@iflistcontainsprop{#1}\comma@entry{%
          \PackageWarning\ZREF@name{%
            Property `\comma@entry' is already in list `#1'%
          }%
        }{%
          \begingroup\expandafter\endgroup
          \expandafter\ltx@LocalAppendToMacro
          \csname Z@L@#1\expandafter\endcsname
          \expandafter{\csname\comma@entry\endcsname}%
        }%
      }%
      \ltx@gobble
    }%
  }%
}
%    \end{macrocode}
%    \end{macro}
%    \begin{macro}{\zref@localaddprop}
%    \begin{macrocode}
\ZREF@Robust\def\zref@localaddprop#1#2{%
  \zref@listexists{#1}{%
    \zref@propexists{#2}{%
      \zref@iflistcontainsprop{#1}{#2}{%
        \PackageWarning\ZREF@name{%
          Property `#2' is already in list `#1'%
        }%
      }{%
        \begingroup\expandafter\endgroup
        \expandafter\ltx@LocalAppendToMacro
        \csname Z@L@#1\expandafter\endcsname
        \expandafter{\csname#2\endcsname}%
      }%
    }%
  }%
}
%    \end{macrocode}
%    \end{macro}
%
%    \begin{macrocode}
\ltx@IfUndefined{pdf@strcmp}{%
%    \end{macrocode}
%    \begin{macro}{\zref@delprop}
%    \begin{macrocode}
  \ZREF@Robust\def\zref@delprop{%
    \ZREF@delprop\gdef
  }%
%    \end{macrocode}
%    \end{macro}
%    \begin{macro}{\zref@localdelprop}
%    \begin{macrocode}
  \ZREF@Robust\def\zref@localdelprop{%
    \ZREF@delprop\def
  }%
%    \end{macrocode}
%    \end{macro}
%    \begin{macro}{\ZREF@delprop}
%    \begin{macrocode}
  \def\ZREF@delprop#1#2#3{%
    \zref@listexists{#2}{%
      \begingroup
        \escapechar=-1 %
        \def\ZREF@param{#3}%
        \@onelevel@sanitize\ZREF@param
        \toks@{}%
        \expandafter\expandafter\expandafter\ZREF@@delprop
        \csname Z@L@#2\endcsname!%
      \expandafter\endgroup
      \expandafter#1\csname Z@L@#2\expandafter\endcsname
      \expandafter{%
        \the\toks@
      }%
    }%
  }%
%    \end{macrocode}
%    \end{macro}
%    \begin{macro}{\ZREF@@delprop}
%    \begin{macrocode}
  \def\ZREF@@delprop#1{%
    \expandafter\ZREF@@@delprop\expandafter{\string#1}#1%
  }%
%    \end{macrocode}
%    \end{macro}
%    \begin{macro}{\ZREF@@@delprop}
%    \begin{macrocode}
  \def\ZREF@@@delprop#1#2{%
    \ifx#2!%
    \else
      \def\ZREF@temp{#1}%
      \@onelevel@sanitize\ZREF@temp
      \ifx\ZREF@param\ZREF@temp
      \else
        \toks@\expandafter{%
          \the\expandafter\toks@\csname#1\endcsname
        }%
      \fi
      \expandafter\ZREF@@delprop
    \fi
  }%
%    \end{macrocode}
%    \end{macro}
%    \begin{macrocode}
}{%
%    \end{macrocode}
%    \begin{macro}{\zref@delprop}
%    \begin{macrocode}
  \ZREF@Robust\def\zref@delprop{%
    \ZREF@delprop\xdef
  }%
%    \end{macrocode}
%    \end{macro}
%    \begin{macro}{\zref@localdelprop}
%    \begin{macrocode}
  \ZREF@Robust\def\zref@localdelprop{%
    \ZREF@delprop\edef
  }%
%    \end{macrocode}
%    \end{macro}
%    \begin{macro}{\ZREF@delprop}
%    \begin{macrocode}
  \def\ZREF@delprop#1#2#3{%
    \zref@listexists{#2}{%
      \def\ZREF@param{#3}%
      \edef\ZREF@SavedEscapechar{\the\escapechar}%
      \escapechar=-1 %
      \expandafter#1\csname Z@L@#2%
      \expandafter\expandafter\expandafter\endcsname{%
        \expandafter\expandafter\expandafter\ZREF@@delprop
        \csname Z@L@#2\endcsname!%
      }%
      \escapechar=\ZREF@SavedEscapechar\relax
    }%
  }%
%    \end{macrocode}
%    \end{macro}
%    \begin{macro}{\ZREF@@delprop}
%    Caution: |#1| might be an \cs{if} or similar token.
%    \begin{macrocode}
  \def\ZREF@@delprop#1{%
    \expandafter\ZREF@@@delprop\expandafter{\string#1}#1%
  }%
%    \end{macrocode}
%    \end{macro}
%    \begin{macro}{\ZREF@@@delprop}
%    \begin{macrocode}
  \def\ZREF@@@delprop#1#2{%
    \ifx#2!%
    \else
      \ifnum\pdf@strcmp{#1}{\ZREF@param}=\ltx@zero
      \else
        \expandafter\noexpand\csname#1\endcsname
      \fi
      \expandafter\ZREF@@delprop
    \fi
  }%
%    \end{macrocode}
%    \end{macro}
%    \begin{macrocode}
}
%    \end{macrocode}
%
% \subsubsection{Properties}
%
%    \begin{macro}{\zref@ifpropundefined}
%    \cs{zref@ifpropundefined} checks the existence of the
%    property \verb|#1|. If the property is present,
%    then \verb|#2| is executed and \verb|#3| otherwise.
%    \begin{macrocode}
\def\zref@ifpropundefined#1{%
  \ltx@ifundefined{Z@E@#1}%
}
%    \end{macrocode}
%    \end{macro}
%    \begin{macro}{\zref@propexists}
%    Some macros rely on the existence of a property.
%    \cs{zref@propexists} only executes \verb|#2| if the
%    property \verb|#1| exists and raises an error
%    message otherwise.
%    \begin{macrocode}
\ZREF@Robust\def\zref@propexists#1{%
  \zref@ifpropundefined{#1}{%
    \PackageError\ZREF@name{%
      Property `#1' does not exist%
    }\@ehc
  }%
}
%    \end{macrocode}
%    \end{macro}
%
%    \begin{macro}{\zref@newprop}
%    A new property is declared by \cs{zref@newprop}, the
%    property name \meta{propname} is given in \verb|#1|. The
%    property is created and configured. If the star form
%    is given, then the expansion of the property value
%    is delayed to page shipout time, when the reference
%    is written to the \xfile{.aux} file.
%    \begin{description}
%    \item[\cs{Z@D@}\metacs{propname}:]
%      Stores the default value for this property.
%    \item[\cs{Z@E@}\metacs{propname}:]
%      Extract function.
%    \item[\cs{Z@X@}\metacs{propname}:]
%      Information whether the expansion of the property value
%      is delayed to shipout time.
%    \item[\cs{Z@C@}\metacs{propname}:]
%      Current value of the property.
%    \end{description}
%    \begin{macrocode}
\ZREF@Robust\def\zref@newprop{%
  \@ifstar{%
    \let\ZREF@X\noexpand
    \ZREF@newprop
  }{%
    \let\ZREF@X\ltx@empty
    \ZREF@newprop
  }%
}
%    \end{macrocode}
%    \end{macro}
%    \begin{macro}{\ZREF@newprop}
%    \begin{macrocode}
\def\ZREF@newprop#1{%
  \edef\ZREF@P{#1}%
  \@onelevel@sanitize\ZREF@P
  \begingroup
  \ifx\ZREF@P\ZREF@par
    \@PackageError\ZREF@name{%
      Invalid property name `\ZREF@P'%
    }{%
      The property name `par' is not allowed %
      because of internal reasons.%
      \MessageBreak
      \@ehc
    }%
    \def\ZREF@@newprop[##1]##2{\endgroup}%
  \else
    \zref@ifpropundefined\ZREF@P{%
      \endgroup
      \PackageInfo\ZREF@name{%
        New property: \ZREF@P
      }%
    }{%
      \@PackageError\ZREF@name{%
        Property `\ZREF@P' already exists%
      }\@ehc
      \def\ZREF@@newprop[##1]##2{\endgroup}%
    }%
  \fi
  \@ifnextchar[\ZREF@@newprop{\ZREF@@newprop[\zref@default]}%
}
%    \end{macrocode}
%    \end{macro}
%    \begin{macro}{\ZREF@par}
%    \begin{macrocode}
\def\ZREF@par{par}
\@onelevel@sanitize\ZREF@par
%    \end{macrocode}
%    \end{macro}
%    \begin{macro}{\ZREF@@newprop}
%    \begin{macrocode}
\def\ZREF@@newprop[#1]{%
  \global\@namedef{Z@D@\ZREF@P}{#1}%
  \global\expandafter\let\csname Z@X@\ZREF@P\endcsname\ZREF@X
  \begingroup\expandafter\endgroup
  \expandafter\ZREF@@@newprop\csname\ZREF@P\endcsname
  \expandafter\gdef\csname Z@C@\ZREF@P\endcsname{}%
  \zref@setcurrent\ZREF@P
}
\def\ZREF@@@newprop#1{%
  \expandafter
  \gdef\csname Z@E@\ZREF@P\endcsname##1#1##2##3\ZREF@nil{##2}%
}
%    \end{macrocode}
%    \end{macro}
%
%    \begin{macro}{\zref@showprop}
%    \begin{macrocode}
\ZREF@Robust\def\zref@showprop#1{%
  \zref@ifpropundefined{#1}{%
    \@PackageInfoNoLine{\ZREF@name}{%
      Show property `#1': <undefined>%
    }%
  }{%
    \begingroup
      \toks@\expandafter\expandafter\expandafter{%
        \csname Z@C@#1\endcsname
      }%
      \edef\ZREF@value{\the\toks@}%
      \ltx@onelevel@sanitize\ZREF@value
      \toks@\expandafter\expandafter\expandafter{%
        \csname Z@D@#1\endcsname
      }%
      \edef\ZREF@default{\the\toks@}%
      \ltx@onelevel@sanitize\ZREF@default
      \@PackageInfoNoLine{\ZREF@name}{%
        Show property `#1':\MessageBreak
        \expandafter\ifx\csname Z@X@#1\endcsname\ltx@empty
          Immediate %
        \else
          Delayed %
        \fi
        value: [\ZREF@value]\MessageBreak
        Default: [\ZREF@default]%
      }%
    \endgroup
  }%
}
%    \end{macrocode}
%    \end{macro}
%
%    \begin{macro}{\zref@setcurrent}
%    \cs{zref@setcurrent} sets the current value for a property.
%    \begin{macrocode}
\ZREF@Robust\def\zref@setcurrent#1#2{%
  \zref@propexists{#1}{%
    \expandafter\def\csname Z@C@#1\endcsname{#2}%
  }%
}
%    \end{macrocode}
%    \end{macro}
%
%    \begin{macro}{\ZREF@getcurrent}
%    \cs{zref@getcurrent} gets the current value for a property.
%    \begin{macrocode}
\def\ZREF@getcurrent#1{%
  \romannumeral0%
  \ltx@ifundefined{Z@C@#1}{%
    \ltx@space
  }{%
    \expandafter\expandafter\expandafter\ltx@space
    \csname Z@C@#1\endcsname
  }%
}
%    \end{macrocode}
%    \end{macro}
%    \begin{macro}{\ZREF@u@getcurrent}
%    \begin{macrocode}
\def\ZREF@wu@getcurrent#1{%
  \etex@unexpanded\expandafter\expandafter\expandafter{%
    \ZREF@getcurrent{#1}%
  }%
}
%    \end{macrocode}
%    \end{macro}
%    \begin{macro}{\zref@getcurrent}
%    \begin{macrocode}
\let\zref@getcurrent\ZREF@getcurrent
%    \end{macrocode}
%    \end{macro}
%
% \subsubsection{Reference generation}
%
%    \begin{macro}{\zref@label}
%    Label macro that uses the main property list.
%    \begin{macrocode}
\ZREF@Robust\def\zref@label#1{%
  \zref@labelbylist{#1}\ZREF@mainlist
}
%    \end{macrocode}
%    \end{macro}
%    \begin{macro}{\zref@labelbylist}
%    Label macro that stores the properties, specified in the
%    property list \verb|#2|.
%    \begin{macrocode}
\ZREF@Robust\def\zref@labelbylist#1#2{%
  \@bsphack
    \zref@listexists{#2}{%
      \expandafter\expandafter\expandafter\ZREF@label
      \expandafter\expandafter\expandafter{%
        \csname Z@L@#2\endcsname
      }{#1}%
    }%
  \@esphack
}
%    \end{macrocode}
%    \end{macro}
%    \begin{macro}{\zref@labelbyprops}
%    The properties are directly specified in a comma separated list.
%    \begin{macrocode}
\ZREF@Robust\def\zref@labelbyprops#1#2{%
  \@bsphack
    \begingroup
      \toks@{}%
      \comma@parse{#2}{%
        \zref@ifpropundefined\comma@entry{%
          \PackageWarning\ZREF@name{%
            Property `\comma@entry' is not known%
          }%
        }{%
          \toks@\expandafter{%
            \the\expandafter\toks@\csname\comma@entry\endcsname
          }%
        }%
        \ltx@gobble
      }%
    \expandafter\endgroup
    \expandafter\ZREF@label\expandafter{\the\toks@}{#1}%
  \@esphack
}
%    \end{macrocode}
%    \end{macro}
%    \begin{macro}{\zref@labelbykv}
%    \begin{macrocode}
\ZREF@Robust\def\zref@labelbykv#1#2{%
  \@bsphack
    \begingroup
      \let\Z@L@ZREF@temp\ltx@empty
      \kvsetkeys{ZREF@LABEL}{#1}%
      \ifZREF@immediate
        \expandafter\zref@wrapper@immediate\expandafter{%
          \expandafter\ZREF@label\expandafter{\Z@L@ZREF@temp}{#2}%
        }%
      \else
        \expandafter\ZREF@label\expandafter{\Z@L@ZREF@temp}{#2}%
      \fi
    \endgroup
  \@esphack
}
%    \end{macrocode}
%    \end{macro}
%    \begin{macrocode}
\kv@define@key{ZREF@LABEL}{prop}{%
  \edef\ZREF@param{#1}%
  \zref@propexists\ZREF@param{%
    \zref@iflistcontainsprop{ZREF@temp}\ZREF@param{}{%
      \begingroup\expandafter\endgroup
      \expandafter\ltx@LocalAppendToMacro
      \expandafter\Z@L@ZREF@temp
      \expandafter{\csname\ZREF@param\endcsname}%
    }%
  }%
}
\kv@define@key{ZREF@LABEL}{list}{%
  \zref@listforloop{#1}{%
    \zref@iflistcontainsprop{ZREF@temp}\zref@prop{}{%
      \begingroup\expandafter\endgroup
      \expandafter\ltx@LocalAppendToMacro
      \expandafter\Z@L@ZREF@temp
      \expandafter{\csname\zref@prop\endcsname}%
    }%
    \ltx@gobble
  }%
}
\kv@define@key{ZREF@LABEL}{delprop}{%
  \zref@propexists{#1}{%
    \zref@localdelprop{ZREF@temp}{#1}%
  }%
}
\kv@define@key{ZREF@LABEL}{immediate}[true]{%
  \edef\ZREF@param{#1}%
  \ifx\ZREF@param\ZREF@true
    \ZREF@immediatetrue
  \else
    \ifx\ZREF@param\ZREF@false
      \ZREF@immediatefalse
    \else
      \PackageWarning\ZREF@name{%
        Option `immediate' expects `true' or `false'.\MessageBreak
        Ignoring invalid value `\ZREF@param'%
      }%
    \fi
  \fi
}
%    \end{macrocode}
%    \begin{macro}{\ZREF@false}
%    \begin{macrocode}
\def\ZREF@false{false}
%    \end{macrocode}
%    \end{macro}
%    \begin{macro}{\ZREF@true}
%    \begin{macrocode}
\def\ZREF@true{true}
%    \end{macrocode}
%    \end{macro}
%
%    \begin{macrocode}
\kv@define@key{ZREF@LABEL}{values}[]{%
  \kv@parse{#1}{%
    \ifx\kv@value\relax
      \@PackageWarning\ZREF@name{%
        Missing value for property `\kv@key'%
      }%
      \expandafter\ltx@gobbletwo
    \else
      \expandafter\zref@setcurrent
    \fi
  }%
}
%    \end{macrocode}
%
%    \begin{macro}{\ifZREF@immediate}
%    The switch \cs{ifZREF@immediate} tells us, whether the
%    label should be written immediately or at page shipout time.
%    \cs{ZREF@label} need to be notified about this, because
%    it must disable the deferred execution of property values,
%    if the label is written immediately.
%    \begin{macrocode}
\newif\ifZREF@immediate
%    \end{macrocode}
%    \end{macro}
%    \begin{macro}{\zref@wrapper@immediate}
%    The argument of \cs{zref@wrapper@immediate} is executed inside
%    a group where \cs{write} is redefined by adding
%    \cs{immediate} before its execution. Also \cs{ZREF@label}
%    is notified via the switch \cs{ifZREF@immediate}.
%    \begin{macrocode}
\ZREF@Robust{\long\def}\zref@wrapper@immediate#1{%
  \begingroup
    \ZREF@immediatetrue
    \let\ZREF@org@write\write
    \def\write{\immediate\ZREF@org@write}%
    #1%
  \endgroup
}
%    \end{macrocode}
%    \end{macro}
%
%    \begin{macro}{\ZREF@label}
%    \cs{ZREF@label} writes the data in the \xfile{.aux} file.
%    \verb|#1| contains the list of valid properties,
%    \verb|#2| the name of the reference. In case of immediate
%    writing, the deferred execution of property values is
%    disabled. Also \thepage is made expandable in this case.
%    \begin{macrocode}
\def\ZREF@label#1#2{%
  \if@filesw
    \begingroup
      \ifZREF@immediate
        \let\ZREF@org@thepage\thepage
      \fi
      \protected@write\@auxout{%
        \ifZREF@immediate
          \let\thepage\ZREF@org@thepage
        \fi
        \let\ZREF@temp\ltx@empty
        \@tfor\ZREF@P:=#1\do{%
          \begingroup
            \escapechar=-1 %
            \edef\x{\endgroup
              \def\noexpand\ZREF@P{%
                \expandafter\string\ZREF@P
              }%
            }%
          \x
          \expandafter\ifx
              \csname
                \ifZREF@immediate
                  relax%
                \else
                  Z@X@\ZREF@P%
                \fi
              \endcsname
              \noexpand
            \expandafter\let\csname Z@C@\ZREF@P\endcsname\relax
          \fi
          \toks@\expandafter{\ZREF@temp}%
          \edef\ZREF@temp{%
            \the\toks@
            \ltx@backslashchar\ZREF@P{%
              \expandafter\noexpand\csname Z@C@\ZREF@P\endcsname
            }%
          }%
        }%
      }{%
        \string\zref@newlabel{#2}{\ZREF@temp}%
      }%
    \endgroup
  \fi
}
\def\ZREF@addtoks#1{%
  \toks@\expandafter\expandafter\expandafter{%
    \expandafter\the\expandafter\toks@#1%
  }%
}
%    \end{macrocode}
%    \end{macro}
%
% \subsubsection{Reference querying and extracting}
%
%    Design goal for the extracting macros is that the extraction
%    process is full expandable. Thus these macros can be used
%    in expandable contexts. But there are problems that cannot
%    be solved by full expandable macros:
%    \begin{itemize}
%    \item In standard \hologo{LaTeX} undefined references sets a flag
%      and generate a warning. Both actions are not expandable.
%    \item Babel's support for its shorthand uses commands that
%      use non-expandable assignments. However currently there
%      is hope, that primitives are added to \pdfTeX\ that
%      allows the detection of contexts. Then the shorthand
%      can detect, if they are executed inside \cs{csname}
%      and protect themselves automatically.
%    \end{itemize}
%    \begin{macro}{\zref@ifrefundefined}
%    If a reference \verb|#1| is undefined, then macro
%    \cs{zref@ifrefundefined} calls \verb|#2| and \verb|#3| otherwise.
%    \begin{macrocode}
\def\zref@ifrefundefined#1{%
  \ltx@ifundefined{Z@R@#1}%
}
%    \end{macrocode}
%    \end{macro}
%
%    \begin{macro}{\zifrefundefined}
%    If a reference \verb|#1| is undefined, then macro
%    \cs{zref@ifrefundefined} calls \verb|#2| and \verb|#3| otherwise.
%    Also the reference is marked used.
%    \begin{macrocode}
\ZREF@IfDefinable\zifrefundefined\def{%
  #1{%
    \zref@wrapper@babel\ZREF@ifrefundefined{#1}%
  }%
}
%    \end{macrocode}
%    \end{macro}
%    \begin{macro}{\ZREF@ifrefundefined}
%    \begin{macrocode}
\def\ZREF@ifrefundefined#1{%
  \zref@refused{#1}%
  \zref@ifrefundefined{#1}%
}
%    \end{macrocode}
%    \end{macro}
%
%    \begin{macro}{\zref@refused}
%    The problem with undefined references is addressed by
%    the macro \cs{zref@refused}. This can be used outside
%    the expandable context. In case of an undefined reference
%    the flag is set to notify \hologo{LaTeX} and a warning is given.
%    \begin{macrocode}
\ZREF@Robust\def\zref@refused#1{%
  \zref@wrapper@babel\ZREF@refused{#1}%
}
%    \end{macrocode}
%    \end{macro}
%    \begin{macro}{\ZREF@refused}
%    \begin{macrocode}
\def\ZREF@refused#1{%
  \zref@ifrefundefined{#1}{%
    \protect\G@refundefinedtrue
    \@latex@warning{%
      Reference `#1' on page \thepage \space undefined%
    }%
  }{}%
}
%    \end{macrocode}
%    \end{macro}
%
%    \begin{macro}{\zref@ifrefcontainsprop}
%    \cs{zref@ifrefcontainsprop} looks, if the reference \verb|#1|
%    has the property \verb|#2| and calls then \verb|#3| and
%    \verb|#4| otherwise.
%    \begin{macrocode}
\def\zref@ifrefcontainsprop#1#2{%
  \zref@ifrefundefined{#1}{%
    \ltx@secondoftwo
  }{%
    \expandafter\ZREF@ifrefcontainsprop
    \csname Z@E@#2\expandafter\endcsname
    \csname#2\expandafter\expandafter\expandafter\endcsname
    \expandafter\expandafter\expandafter{%
      \csname Z@R@#1\endcsname
    }%
  }%
}
\def\ZREF@ifrefcontainsprop#1#2#3{%
  \expandafter\ifx\expandafter\ZREF@novalue
  #1#3#2\ZREF@novalue\ZREF@nil\ltx@empty
    \expandafter\ltx@secondoftwo
  \else
    \expandafter\ltx@firstoftwo
  \fi
}
\def\ZREF@novalue{\ZREF@NOVALUE}
%    \end{macrocode}
%    \end{macro}
%
%    \begin{macro}{\zref@extract}
%    \cs{zref@extract} is an abbreviation for the case
%    that the default of the property is used as default
%    value.
%    \begin{macrocode}
\def\ZREF@extract#1#2{%
  \romannumeral0%
  \ltx@ifundefined{Z@D@#2}{%
    \expandafter\ltx@space\zref@default
  }{%
    \expandafter\expandafter\expandafter\ZREF@@extract
    \expandafter\expandafter\expandafter{%
      \csname Z@D@#2\endcsname
    }{#1}{#2}%
  }%
}
%    \end{macrocode}
%    \end{macro}
%    \begin{macro}{\ZREF@@extract}
%    \begin{macrocode}
\def\ZREF@@extract#1#2#3{%
  \expandafter\expandafter\expandafter\ltx@space
  \zref@extractdefault{#2}{#3}{#1}%
}
%    \end{macrocode}
%    \end{macro}
%    \begin{macro}{\ZREF@wu@extract}
%    \begin{macrocode}
\def\ZREF@wu@extract#1#2{%
  \etex@unexpanded\expandafter\expandafter\expandafter{%
    \ZREF@extract{#1}{#2}%
  }%
}
%    \end{macrocode}
%    \end{macro}
%    \begin{macro}{\zref@extract}
%    \begin{macrocode}
\let\zref@extract\ZREF@extract
%    \end{macrocode}
%    \end{macro}
%
%    \begin{macro}{\ZREF@extractdefault}
%    The basic extracting macro is \cs{zref@extractdefault} with
%    the reference name in \verb|#1|, the property in \verb|#2| and
%    the default value in \verb|#3| in case for problems.
%    \begin{macrocode}
\def\ZREF@extractdefault#1#2#3{%
  \romannumeral0%
  \zref@ifrefundefined{#1}\ltx@firstoftwo{%
    \zref@ifpropundefined{#2}\ltx@firstoftwo\ltx@secondoftwo
  }{%
    \ltx@space
    #3%
  }{%
    \expandafter\expandafter\expandafter\ltx@space
    \csname Z@E@#2\expandafter\expandafter\expandafter\endcsname
    \csname Z@R@#1\expandafter\endcsname
    \csname#2\endcsname{#3}\ZREF@nil
  }%
}
%    \end{macrocode}
%    \end{macro}
%    \begin{macro}{\ZREF@wu@extractdefault}
%    \begin{macrocode}
\def\ZREF@wu@extractdefault#1#2#3{%
  \etex@unexpanded\expandafter\expandafter\expandafter{%
    \ZREF@extractdefault{#1}{#2}{#3}%
  }%
}
%    \end{macrocode}
%    \end{macro}
%    \begin{macro}{\zref@extractdefault}
%    \begin{macrocode}
\let\zref@extractdefault\ZREF@extractdefault
%    \end{macrocode}
%    \end{macro}
%
%    \begin{macro}{\zref@def@extract}
%    \begin{macrocode}
\ZREF@Robust\def\zref@def@extract#1{%
  \zref@wrapper@babel{\ZREF@def@extract{#1}}%
}
%    \end{macrocode}
%    \end{macro}
%    \begin{macro}{\ZREF@def@extract}
%    \begin{macrocode}
\def\ZREF@def@extract#1#2#3{%
  \zref@refused{#2}%
  \expandafter\expandafter\expandafter\def
  \expandafter\expandafter\expandafter#1%
  \expandafter\expandafter\expandafter{%
    \zref@extract{#2}{#3}%
  }%
}
%    \end{macrocode}
%    \end{macro}
%    \begin{macro}{\zref@def@extractdefault}
%    \begin{macrocode}
\ZREF@Robust\def\zref@def@extractdefault#1{%
  \zref@wrapper@babel{\ZREF@def@extractdefault{#1}}%
}
%    \end{macrocode}
%    \end{macro}
%    \begin{macro}{\ZREF@def@extractdefault}
%    \begin{macrocode}
\def\ZREF@def@extractdefault#1#2#3#4{%
  \zref@refused{#2}%
  \expandafter\expandafter\expandafter\def
  \expandafter\expandafter\expandafter#1%
  \expandafter\expandafter\expandafter{%
    \zref@extractdefault{#2}{#3}{#4}%
  }%
}
%    \end{macrocode}
%    \end{macro}
%
%    \begin{macro}{\ZREF@wrapper@unexpanded}
%    \begin{macrocode}
\ZREF@Robust{\long\def}\ZREF@wrapper@unexpanded#1{%
  \let\zref@wrapper@unexpanded\ltx@firstofone
  \let\zref@getcurrent\ZREF@wu@getcurrent
  \let\zref@extractdefault\ZREF@wu@extractdefault
  \let\zref@extract\ZREF@wu@extract
  #1%
  \let\zref@wrapper@unexpanded\ZREF@wrapper@unexpanded
  \let\zref@getcurrent\ZREF@getcurrent
  \let\zref@extractdefault\ZREF@extractdefault
  \let\zref@extract\ZREF@extract
}
%    \end{macrocode}
%    \end{macro}
%    \begin{macro}{\zref@wrapper@unexpanded}
%    \begin{macrocode}
\ltx@IfUndefined{etex@unexpanded}{%
  \let\zref@wrapper@unexpanded\ltx@firstofone
}{%
  \let\zref@wrapper@unexpanded\ZREF@wrapper@unexpanded
}
%    \end{macrocode}
%    \end{macro}
%
% \subsubsection{Compatibility with \xpackage{babel}}
%
%    \begin{macro}{\zref@wrapper@babel}
%    \begin{macrocode}
\ZREF@Robust{\long\def}\zref@wrapper@babel#1#2{%
  \ifcsname if@safe@actives\endcsname
    \expandafter\ltx@firstofone
  \else
    \expandafter\ltx@secondoftwo
  \fi
  {%
    \if@safe@actives
      \expandafter\ltx@secondoftwo
    \else
      \expandafter\ltx@firstoftwo
    \fi
    {%
      \begingroup
        \csname @safe@activestrue\endcsname
        \edef\x{#2}%
      \expandafter\endgroup
      \expandafter\ZREF@wrapper@babel\expandafter{\x}{#1}%
    }%
  }{%
    #1{#2}%
  }%
}
\long\def\ZREF@wrapper@babel#1#2{%
  #2{#1}%
}
%    \end{macrocode}
%    \end{macro}
%
% \subsubsection{Unique counter support}
%
%    \begin{macro}{\zref@require@unique}
%    Generate the counter |zref@unique| if the counter does not
%    already exist.
%    \begin{macrocode}
\ZREF@Robust\def\zref@require@unique{%
  \@ifundefined{c@zref@unique}{%
    \begingroup
      \let\@addtoreset\ltx@gobbletwo
      \newcounter{zref@unique}%
    \endgroup
%    \end{macrocode}
%    \end{macro}
%
%    \begin{macro}{\thezref@unique}
%    \cs{thezref@unique} is used for automatically generated unique
%    labelnames.
%    \begin{macrocode}
    \renewcommand*{\thezref@unique}{%
      zref@\number\c@zref@unique
    }%
  }{}%
}
%    \end{macrocode}
%    \end{macro}
%
% \subsubsection{Utilities}
%
%    \begin{macro}{\ZREF@number}
%    \begin{macrocode}
\ltx@IfUndefined{numexpr}{%
  \def\ZREF@number#1{\number#1}%
}{%
  \def\ZREF@number#1{\the\numexpr(#1)\relax}%
}
%    \end{macrocode}
%    \end{macro}
%
% \subsubsection{Setup}
%
%    \begin{macro}{\zref@setdefault}
%    Standard \hologo{LaTeX} prints ``??'' in bold face if a reference
%    is not known. \cs{zref@default} holds the text that is
%    printed in case of unknown references and is used, if
%    the default was not specified during the definition
%    of the new property by \cs{ref@newprop}.
%    The global default value can be set by \cs{zref@setdefault}.
%    \begin{macrocode}
\ZREF@Robust\def\zref@setdefault#1{%
  \def\zref@default{#1}%
}
%    \end{macrocode}
%    \end{macro}
%    \begin{macro}{\zref@default}
%    Now we initialize \cs{zref@default} with the same value
%    that \hologo{LaTeX} uses for its undefined references.
%    \begin{macrocode}
\zref@setdefault{%
  \nfss@text{\reset@font\bfseries ??}%
}
%    \end{macrocode}
%    \end{macro}
%
% \paragraph{Main property list.}
%
%    \begin{macro}{\zref@setmainlist}
%    The name of the default property list is stored in
%    \cs{ZREF@mainlist} and can be set by \cs{zref@setmainlist}.
%    \begin{macrocode}
\ZREF@Robust\def\zref@setmainlist#1{%
  \def\ZREF@mainlist{#1}%
}
\zref@setmainlist{main}
%    \end{macrocode}
%    \end{macro}
%    Now we create the list.
%    \begin{macrocode}
\zref@newlist\ZREF@mainlist
%    \end{macrocode}
%
% \paragraph{Main properties.}
%
%    The two properties \texttt{default} and \texttt{page}
%    are created and added to the main property list.
%    They store the data that
%    standard \hologo{LaTeX} uses in its references created by
%    \cs{label}.
%    \begin{description}
%    \item[\texttt{default}] the apperance of the latest
%      counter that is incremented by \cs{refstepcounter}
%    \item[\texttt{page}] the apperance of the page counter
%    \end{description}
%    \begin{macrocode}
\zref@newprop{default}{\@currentlabel}
\zref@newprop*{page}{\thepage}
\zref@addprops\ZREF@mainlist{default,page}
%    \end{macrocode}
%
% \paragraph{Properties}
%
%    \begin{macro}{\ZREF@NewPropAnchor}
%    \begin{macrocode}
\def\ZREF@NewPropAnchor{%
  \zref@newprop{anchor}{%
    \ltx@ifundefined{@currentHref}{}{\@currentHref}%
  }%
  \global\let\ZREF@NewPropAnchor\relax
}
%    \end{macrocode}
%    \end{macro}
%    \begin{macro}{\zref@titleref@current}
%    Later we will redefine the section and caption macros
%    to catch the current title and remember the value in
%    \cs{zref@titleref@current}.
%    \begin{macro}{\ZREF@NewPropTitle}
%    \begin{macrocode}
\def\ZREF@NewPropTitle{%
  \gdef\zref@titleref@current{}%
  \zref@newprop{title}{\zref@titleref@current}%
  \global\let\ZREF@NewPropTitle\relax
}
%    \end{macrocode}
%    \end{macro}
%    \end{macro}
%    \begin{macro}{\ZREF@NewPropTheotype}
%    \begin{macrocode}
\def\ZREF@NewPropTheotype{%
  \zref@newprop{theotype}{}%
  \global\let\ZREF@NewPropTheotype\relax
}
%    \end{macrocode}
%    \end{macro}
%    \begin{macro}{\ZREF@NewPropPageValue}
%    \begin{macrocode}
\def\ZREF@NewPropPageValue{%
  \zref@newprop*{pagevalue}[0]{\number\c@page}%
  \global\let\ZREF@NewPropPageValue\relax
}
%    \end{macrocode}
%    \end{macro}
%
% \paragraph{Mark successful loading}
%
%    \begin{macrocode}
\let\ZREF@base@ok=Y
%</base>
%    \end{macrocode}
%
%
% \subsection{Module \xmodule{user}}
%
%    \begin{macrocode}
%<*user>
\NeedsTeXFormat{LaTeX2e}
\ProvidesPackage{zref-user}%
  [2020-07-03 v2.32 Module user for zref (HO)]%
\RequirePackage{zref-base}[2019/11/29]
\ifx\ZREF@base@ok Y%
\else
  \expandafter\endinput
\fi
%    \end{macrocode}
%
%    Module \xmodule{user} enables a small user interface.
%    All macros are prefixed by \cs{z}.
%
%    First we define the pendants to the standard \hologo{LaTeX} referencing
%    commands \cs{label}, \cs{ref}, and \cs{pageref}.
%
%    \begin{macro}{\zlabel}
%    Similar to \cs{label} the macro \cs{zlabel} writes a reference
%    entry in the \xfile{.aux} file. The main property list is used.
%    Also we add the babel patch. The \cs{label} command can also be
%    used inside section titles, but it must not go into the
%    table of contents. Therefore we have to check this situation.
%    \begin{macrocode}
\newcommand*\zlabel{%
  \ifx\label\ltx@gobble
    \expandafter\ltx@gobble
  \else
    \expandafter\zref@wrapper@babel\expandafter\zref@label
  \fi
}%
%    \end{macrocode}
%    \end{macro}
%
%    \begin{macro}{\zkvlabel}
%    \begin{macrocode}
\newcommand*{\zkvlabel}[1]{%
  \ifx\label\ltx@gobble
    \expandafter\ltx@gobblethree
  \fi
  \zref@wrapper@babel{\zref@labelbykv{#1}}%
}%
%    \end{macrocode}
%    \end{macro}
%
%    \begin{macro}{\zref}
%    Macro \cs{zref} is the corresponding macro for \cs{ref}.
%    Also it provides an optional argument in order to
%    select another property.
%    \begin{macrocode}
\newcommand*{\zref}[2][default]{% robust because of optional argument
  \zref@propexists{#1}{%
    \zref@wrapper@babel\ZREF@zref{#2}{#1}%
  }%
}%
\def\ZREF@zref#1{%
  \zref@refused{#1}%
  \zref@extract{#1}%
}%
%    \end{macrocode}
%    \end{macro}
%
%    \begin{macro}{\zpageref}
%    For macro \cs{zpageref} we just call \cs{zref}
%    with property \verb|page|.
%    \begin{macrocode}
\ZREF@IfDefinable\zpageref\def{%
  {\zref[page]}%
}
%    \end{macrocode}
%    \end{macro}
%
%    \begin{macro}{\zrefused}
%    For the following expandible user macros \cs{zrefused}
%    should be used to notify \hologo{LaTeX} in case of undefined
%    references.
%    \begin{macrocode}
\ZREF@IfDefinable\zrefused\def{%
  {\zref@refused}%
}
%    \end{macrocode}
%    \end{macro}
%
%    \begin{macrocode}
%</user>
%    \end{macrocode}
%
%
% \subsection{Module \xmodule{abspage}}
%
%    \begin{macrocode}
%<*abspage>
\NeedsTeXFormat{LaTeX2e}
\ProvidesPackage{zref-abspage}%
  [2020-07-03 v2.32 Module abspage for zref (HO)]%
\RequirePackage{zref-base}[2019/11/29]
\ifx\ZREF@base@ok Y%
\else
  \expandafter\endinput
\fi
%    \end{macrocode}
%
%    Module \xmodule{abspage} adds a new property \verb|abspage|
%    to the \verb|main| property list for absolute page numbers.
%    These are recorded by the help of package \xpackage{atbegshi}.
%    \begin{macrocode}
\RequirePackage{atbegshi}[2011/10/05]%
%    \end{macrocode}
%    The counter \verb|abspage| must not go in the clear list
%    of \verb|@ckpt| that is used to set counters in
%    \xfile{.aux} files of included \hologo{TeX} files.
%    \begin{macrocode}
\begingroup
  \let\@addtoreset\ltx@gobbletwo
  \newcounter{abspage}%
\endgroup
\setcounter{abspage}{0}%
\AtBeginShipout{%
  \stepcounter{abspage}%
}%
\zref@newprop*{abspage}[0]{\the\c@abspage}%
\zref@addprop\ZREF@mainlist{abspage}%
%    \end{macrocode}
%    Note that counter \verb|abspage| shows the
%    previous page during page processing. Before
%    shipout the counter is incremented.
%    Thus the property is correctly written with
%    deferred writing. If the counter is written
%    using \cs{zref@wrapper@immediate}, then the number
%    is too small by one.
%
%    \begin{macrocode}
%</abspage>
%    \end{macrocode}
%
%
% \subsection{Module \xmodule{counter}}
%
%    \begin{macrocode}
%<*counter>
\NeedsTeXFormat{LaTeX2e}
\ProvidesPackage{zref-counter}%
  [2020-07-03 v2.32 Module counter for zref (HO)]%
\RequirePackage{zref-base}[2019/11/29]
\ifx\ZREF@base@ok Y%
\else
  \expandafter\endinput
\fi
%    \end{macrocode}
%
%    For features such as \xpackage{hyperref}'s \cs{autoref}
%    we need the name of the counter. The property
%    \verb|counter| is defined and added to the main property
%    list. Starting with \LaTeX\ 2020-10-01 the proper can use
%    \@currentcounter. In older formats
%    \cs{refstepcounter} has to be patched but this can fail in some
%    cases, see issue \#5.
%    \begin{macrocode}
\@ifl@t@r\fmtversion{2020-10-01}
 {
  \zref@newprop{counter}{\@currentcounter}
  \zref@addprop\ZREF@mainlist{counter}
 }
 {
  \zref@newprop{counter}{}
  \zref@addprop\ZREF@mainlist{counter}
  \AtBeginDocument{%
    \ZREF@patch{refstepcounter}{%
      \def\refstepcounter#1{%
        \zref@setcurrent{counter}{#1}%
        \ZREF@org@refstepcounter{#1}%
      }%
    }%
  }
 }
%    \end{macrocode}
%    \begin{macrocode}
%</counter>
%    \end{macrocode}
%
%
% \subsection{Module \xmodule{lastpage}}
%
%    \begin{macrocode}
%<*lastpage>
\NeedsTeXFormat{LaTeX2e}
\ProvidesPackage{zref-lastpage}%
  [2020-07-03 v2.32 Module lastpage for zref (HO)]%
\RequirePackage{zref-base}[2019/11/29]
\RequirePackage{zref-abspage}[2019/11/29]
\RequirePackage{atveryend}[2009/12/07]
\ifx\ZREF@base@ok Y%
\else
  \expandafter\endinput
\fi
%    \end{macrocode}
%
%    The module \xmodule{lastpage} implements the service
%    of package \xpackage{lastpage} by setting a reference
%    \verb|LastPage| at the end of the document.
%    If module \xmodule{abspage} is given, also the absolute
%    page number is available, because the properties of
%    the main property list are used.
%    \begin{macrocode}
\zref@newlist{LastPage}
\AfterLastShipout{%
  \if@filesw
    \begingroup
      \advance\c@page\m@ne
      \toks@\expandafter\expandafter\expandafter{%
        \expandafter\Z@L@main
        \Z@L@LastPage
      }%
      \expandafter\zref@wrapper@immediate\expandafter{%
        \expandafter\ZREF@label\expandafter{\the\toks@}{LastPage}%
      }%
    \endgroup
  \fi
}
%    \end{macrocode}
%    \begin{macro}{\zref@iflastpage}
%    \begin{macrocode}
\def\zref@iflastpage#1{%
  \ifnum\zref@extractdefault{#1}{abspage}{-1}=%
        \zref@extractdefault{LastPage}{abspage}{-2} %
    \expandafter\ltx@firstoftwo
  \else
    \expandafter\ltx@secondoftwo
  \fi
}
%    \end{macrocode}
%    \end{macro}
%    \begin{macro}{\ziflastpage}
%    \begin{macrocode}
\ZREF@IfDefinable\ziflastpage\def{%
  {\zref@wrapper@babel\ZREF@iflastpage}%
}
%    \end{macrocode}
%    \end{macro}
%    \begin{macro}{ZREF@iflastpage}
%    \begin{macrocode}
\def\ZREF@iflastpage#1{%
  \zref@refused{LastPage}%
  \zref@refused{#1}%
  \zref@iflastpage{#1}%
}
%    \end{macrocode}
%    \end{macro}
%    \begin{macrocode}
%</lastpage>
%    \end{macrocode}
%
%
% \subsection{Module \xmodule{thepage}}
%
%    \begin{macrocode}
%<*thepage>
\NeedsTeXFormat{LaTeX2e}
\ProvidesPackage{zref-thepage}%
  [2020-07-03 v2.32 Module thepage for zref (HO)]%
\RequirePackage{zref-base}[2019/11/29]
\ifx\ZREF@base@ok Y%
\else
  \expandafter\endinput
\fi
%    \end{macrocode}
%
%    \begin{macrocode}
\RequirePackage{atbegshi}[2011/10/05]
\RequirePackage{zref-abspage}[2019/11/29]
%    \end{macrocode}
%    \begin{macrocode}
\zref@newlist{thepage}
\zref@addprop{thepage}{page}
\ZREF@NewPropPageValue
%    \end{macrocode}
%    \begin{macro}{\zref@thepage@atbegshi@hook}
%    \begin{macrocode}
\let\zref@thepage@atbegshi@hook\ltx@empty
%    \end{macrocode}
%    \end{macro}
%    \begin{macrocode}
\zref@addprop{thepage}{pagevalue}
\AtBeginShipout{%
  \AtBeginShipoutAddToBox{%
    \zref@thepage@atbegshi@hook
    \zref@labelbylist{thepage\the\value{abspage}}{thepage}%
  }%
}
%    \end{macrocode}
%    \begin{macro}{\zref@thepage@name}
%    \begin{macrocode}
\ltx@IfUndefined{numexpr}{%
  \def\zref@thepage@name#1{thepage\number#1}%
}{%
  \def\zref@thepage@name#1{thepage\the\numexpr#1}%
}
%    \end{macrocode}
%    \end{macro}
%    \begin{macro}{\zref@thepage}
%    \begin{macrocode}
\def\zref@thepage#1{%
  \zref@extract{\zref@thepage@name{#1}}{page}%
}%
%    \end{macrocode}
%    \end{macro}
%    \begin{macro}{\zref@thepage@refused}
%    \begin{macrocode}
\ZREF@Robust\def\zref@thepage@refused#1{%
  \zref@refused{\zref@thepage@name{#1}}%
}%
%    \end{macrocode}
%    \end{macro}
%    \begin{macro}{\zthepage}
%    \begin{macrocode}
\ZREF@IfDefinable\zthepage\def{%
  #1{%
    \zref@thepage@refused{#1}%
    \zref@thepage{#1}%
  }%
}
%    \end{macrocode}
%    \end{macro}
%    \begin{macrocode}
%</thepage>
%    \end{macrocode}
%
%
% \subsection{Module \xmodule{nextpage}}
%
%    \begin{macrocode}
%<*nextpage>
\NeedsTeXFormat{LaTeX2e}
\ProvidesPackage{zref-nextpage}%
  [2020-07-03 v2.32 Module nextpage for zref (HO)]%
\RequirePackage{zref-base}[2019/11/29]
\ifx\ZREF@base@ok Y%
\else
  \expandafter\endinput
\fi
%    \end{macrocode}
%
%    \begin{macrocode}
\RequirePackage{zref-abspage}[2019/11/29]
\RequirePackage{zref-thepage}[2019/11/29]
\RequirePackage{zref-lastpage}[2019/11/29]
\RequirePackage{uniquecounter}[2009/12/18]
%    \end{macrocode}
%    \begin{macrocode}
\UniqueCounterNew{znextpage}

\newcommand*{\znextpagesetup}{%
  \afterassignment\ZREF@np@setup@i
  \def\ZREF@np@call@unknown##1%
}
\def\ZREF@np@setup@i{%
  \afterassignment\ZREF@np@setup@ii
  \def\ZREF@np@call@nonext##1%
}
\def\ZREF@np@setup@ii{%
  \def\ZREF@np@call@next##1%
}
\def\ZREF@np@call@unknown#1{#1}
\def\ZREF@np@call@nonext#1{#1}
\def\ZREF@np@call@next#1{#1}
\ZREF@IfDefinable\znextpage\def{%
  {\UniqueCounterCall{znextpage}{\ZREF@nextpage}}%
}%
\newcommand*{\znonextpagename}{}
\newcommand*{\zunknownnextpagename}{\Z@D@page}
\def\ZREF@nextpage#1{%
  \begingroup
    \def\ZREF@refname@this{zref@np#1}%
    \zref@labelbyprops\ZREF@refname@this{abspage}%
    \chardef\ZREF@call=0 % unknown
    \ZREF@ifrefundefined\ZREF@refname@this{%
    }{%
      \edef\ZREF@pagenum@this{%
        \zref@extractdefault\ZREF@refname@this{abspage}{0}%
      }%
      \edef\ZREF@refname@next{%
        \zref@thepage@name{%
          \the\numexpr\ZREF@pagenum@this+1%
        }%
      }%
      \ifnum\ZREF@pagenum@this>0 %
        \ZREF@ifrefundefined{LastPage}{%
          \zref@ifrefundefined\ZREF@refname@next{%
          }{%
            \chardef\ZREF@call=2 % next page
          }%
        }{%
          \edef\ZREF@pagenum@last{%
            \zref@extractdefault{LastPage}{abspage}{0}%
          }%
          \ifnum\ZREF@pagenum@this<\ZREF@pagenum@last\ltx@space
            \ZREF@ifrefundefined\ZREF@refname@next{%
            }{%
              \chardef\ZREF@call=2 % next page
            }%
          \else
            \ifnum\ZREF@pagenum@this=\ZREF@pagenum@this\ltx@space
              \chardef\ZREF@call=1 % no next page
            \fi
          \fi
        }%
      \fi
    }%
    \edef\x{%
      \endgroup
      \ifcase\ZREF@call
        \noexpand\ZREF@np@call@unknown{%
          \noexpand\zunknownnextpagename
        }%
      \or
        \noexpand\ZREF@np@call@nonext{%
          \noexpand\znonextpagename
        }%
      \else
        \noexpand\ZREF@np@call@next{%
          \noexpand\zref@extract{\ZREF@refname@next}{page}%
        }%
      \fi
    }%
  \x
}
%    \end{macrocode}
%    \begin{macrocode}
%</nextpage>
%    \end{macrocode}
%
%
% \subsection{Module \xmodule{totpages}}
%
%    \begin{macrocode}
%<*totpages>
\NeedsTeXFormat{LaTeX2e}
\ProvidesPackage{zref-totpages}%
  [2020-07-03 v2.32 Module totpages for zref (HO)]%
\RequirePackage{zref-base}[2019/11/29]
\ifx\ZREF@base@ok Y%
\else
  \expandafter\endinput
\fi
%    \end{macrocode}
%
%    The absolute page number of the last page is the total
%    page number.
%    \begin{macrocode}
\RequirePackage{zref-abspage}[2019/11/29]
\RequirePackage{zref-lastpage}[2019/11/29]
%    \end{macrocode}
%
%    \begin{macro}{\ztotpages}
%    Macro \cs{ztotpages} contains the number of pages. It can be
%    used inside expandable calculations. It expands to zero if
%    the reference is not yet available.
%    \begin{macrocode}
\newcommand*{\ztotpages}{%
  \zref@extractdefault{LastPage}{abspage}{0}%
}
%    \end{macrocode}
%    \end{macro}
%    Also we mark the reference |LastPage| as used:
%    \begin{macrocode}
\AtBeginDocument{%
  \zref@refused{LastPage}%
}
%    \end{macrocode}
%    \begin{macrocode}
%</totpages>
%    \end{macrocode}
%
%
% \subsection{Module \xmodule{pagelayout}}
%
%    \begin{macrocode}
%<*pagelayout>
\NeedsTeXFormat{LaTeX2e}
\ProvidesPackage{zref-pagelayout}%
  [2020-07-03 v2.32 Module pagelayout for zref (HO)]%
\RequirePackage{zref-base}[2019/11/29]
\ifx\ZREF@base@ok Y%
\else
  \expandafter\endinput
\fi
%    \end{macrocode}
%
%    \begin{macrocode}
\RequirePackage{zref-thepage}[2019/11/29]
\RequirePackage{iftex}[2019/11/07]%
\RequirePackage{atveryend}[2010/03/24]
%    \end{macrocode}
%
%
% \subsubsection{Define layout properties}
%
%    \begin{macrocode}
\def\ZREF@temp#1{%
  \begingroup
    \escapechar=-1 %
  \ltx@ifundefined{\string#1}{\endgroup}{%
    \edef\x{%
      \endgroup
      \noexpand\zref@newprop*{\string#1}%
                             [\noexpand\number\noexpand#1]% hash-ok
                             {\noexpand\number\noexpand#1}%
      \noexpand\zref@addprop{thepage}{\string#1}%
    }%
    \x
  }%
}
\ZREF@temp\mag
\ZREF@temp\paperwidth
\ZREF@temp\paperheight
\ZREF@temp\stockwidth % memoir.cls, crop.sty
\ZREF@temp\stockheight % memoir.cls, crop.sty
\ZREF@temp\mediawidth % VTeX
\ZREF@temp\mediaheight % VTeX
\ifluatex
\zref@newprop*{pdfvorigin}%
                             [\number\pdfvariable vorigin]% hash-ok
                             {\number\pdfvariable vorigin}%
\zref@addprop{thepage}{pdfvorigin}
\zref@newprop*{pdfhorigin}%
                             [\number\pdfvariable horigin]% hash-ok
                             {\number\pdfvariable horigin}%
\zref@addprop{thepage}{pdfhorigin}
\zref@newprop*{pdfpageheight}%
                             [\number\pageheight]% hash-ok
                             {\number\pageheight}%
\zref@addprop{thepage}{pdfpageheight}
\zref@newprop*{pdfpagewidth}%
                             [\number\pagewidth]% hash-ok
                             {\number\pagewidth}%
\zref@addprop{thepage}{pdfpagewidth}
\else
\ZREF@temp\pdfpagewidth
\ZREF@temp\pdfpageheight
\ZREF@temp\pdfhorigin
\ZREF@temp\pdfvorigin
\fi
\ZREF@temp\hoffset
\ZREF@temp\voffset
\ZREF@temp\topmargin
\ZREF@temp\oddsidemargin
\ZREF@temp\evensidemargin
\ZREF@temp\textwidth
\ZREF@temp\textheight
\ZREF@temp\headheight
\ZREF@temp\headsep
\ZREF@temp\footskip
\ZREF@temp\marginparwidth
\ZREF@temp\marginparsep
\ZREF@temp\columnwidth
\ZREF@temp\columnsep
\ZREF@temp\trimedge % memoir.cls
\ZREF@temp\spinemargin % memoir.cls
\ZREF@temp\foremargin % memoir.cls
\ZREF@temp\trimtop % memoir.cls
\ZREF@temp\uppermargin % memoir.cls
\ZREF@temp\headmargin % memoir.cls
%    \end{macrocode}
%
%    \begin{macrocode}
\IfFormatAtLeastTF{2020/10/01}
 {
   \zref@newprop*{outputboxwd}[0pt]{\ShipoutBoxWidth}
   \zref@newprop*{outputboxht}[0pt]{\ShipoutBoxHeight}
   \zref@newprop*{outputboxdp}[0pt]{\ShipoutBoxDepth}
 }
 {
   \zref@newprop*{outputboxwd}[0pt]{\AtBeginShipoutBoxWidth}
   \zref@newprop*{outputboxht}[0pt]{\AtBeginShipoutBoxHeight}
   \zref@newprop*{outputboxdp}[0pt]{\AtBeginShipoutBoxDepth}
 }
%    \end{macrocode}
%    \begin{macrocode}
\zref@addprops{thepage}{outputboxwd,outputboxht,outputboxdp}
%    \end{macrocode}
%
%    \begin{macro}{\ifZREF@pl@list}
%    \begin{macrocode}
\ltx@newif\ifZREF@pl@list
%    \end{macrocode}
%    \end{macro}
%    \begin{macro}{\zref@listpagelayout}
%    \begin{macrocode}
\ZREF@IfDefinable\zlistpagelayout\def{%
  {\global\ZREF@pl@listtrue}%
}
%    \end{macrocode}
%    \end{macro}
%    \begin{macro}{\ZREF@pl@AfterLastShipout}
%    \begin{macrocode}
\def\ZREF@pl@AfterLastShipout{%
  \ifZREF@pl@list
    \edef\ZREF@page@max{\the\value{abspage}}%
    \ltx@ifundefined{ZREF@org@testdef}{%
      \let\ZREF@org@testdef\@testdef
      \def\@testdef##1##2##3{%
        \ZREF@org@testdef{##1}{##2}{##3}%
        \def\ZREF@temp{##1}%
        \ifx\ZREF@temp\ZREF@RefPrefix
          \expandafter\gdef\csname##1@##2\endcsname{##3}%
        \fi
      }%
    }{}%
    \AtVeryEndDocument{\ZREF@pl@AtVeryEnd}%
  \fi
}
%    \end{macrocode}
%    \end{macro}
%    \begin{macro}{\ZREF@pl@AtVeryEnd}
%    \begin{macrocode}
\def\ZREF@pl@AtVeryEnd{%
  \begingroup
    \toks@{Page layout parameters:\MessageBreak}%
    \count@=1 %
    \ZREF@pl@ListPage
    \edef\x{\endgroup
      \noexpand\@PackageInfoNoLine{zref-pagelayout}{\the\toks@}%
    }%
  \x
}
%    \end{macrocode}
%    \end{macro}
%    \begin{macro}{\ZREF@pl@ListPage}
%    \begin{macrocode}
\def\ZREF@pl@ListPage{%
  \edef\x{%
    \toks@={%
      \the\toks@
      Page \the\count@:\noexpand\MessageBreak
      \zref@ifrefundefined{thepage\the\count@}{}{%
        \ltx@space\ltx@space mag = %
        \zref@extract{thepage\the\count@}{mag}%
        \noexpand\MessageBreak
        \ZREF@pl@ListEntry{paperwidth}%
        \ZREF@pl@ListEntry{paperheight}%
        \ZREF@pl@ListEntry{stockwidth}%
        \ZREF@pl@ListEntry{stockheight}%
        \ZREF@pl@ListEntry{mediawidth}%
        \ZREF@pl@ListEntry{mediaheight}%
        \ZREF@pl@ListEntry{pdfpagewidth}%
        \ZREF@pl@ListEntry{pdfpageheight}%
        \ZREF@pl@ListEntry{pdfhorigin}%
        \ZREF@pl@ListEntry{pdfvorigin}%
        \ZREF@pl@ListEntry{hoffset}%
        \ZREF@pl@ListEntry{voffset}%
        \ZREF@pl@ListEntry{topmargin}%
        \ZREF@pl@ListEntry{oddsidemargin}%
        \ZREF@pl@ListEntry{evensidemargin}%
        \ZREF@pl@ListEntry{textwidth}%
        \ZREF@pl@ListEntry{textheight}%
        \ZREF@pl@ListEntry{headheight}%
        \ZREF@pl@ListEntry{headsep}%
        \ZREF@pl@ListEntry{footskip}%
        \ZREF@pl@ListEntry{marginparwidth}%
        \ZREF@pl@ListEntry{marginparsep}%
        \ZREF@pl@ListEntry{columnwidth}%
        \ZREF@pl@ListEntry{columnsep}%
        \ZREF@pl@ListEntry{trimedge}%
        \ZREF@pl@ListEntry{spinemargin}%
        \ZREF@pl@ListEntry{foremargin}%
        \ZREF@pl@ListEntry{trimtop}%
        \ZREF@pl@ListEntry{uppermargin}%
        \ZREF@pl@ListEntry{headmargin}%
      }%
    }%
  }\x
  \ifnum\ZREF@page@max>\count@
    \advance\count@ by\ltx@one
  \else
    \expandafter\ltx@gobble
  \fi
  \ZREF@pl@ListPage
}
%    \end{macrocode}
%    \end{macro}
%    \begin{macro}{\ZREF@pl@ListEntry}
%    \begin{macrocode}
\def\ZREF@pl@ListEntry#1{%
  \zref@ifpropundefined{#1}{%
  }{%
    \zref@ifrefcontainsprop{thepage\the\count@}{#1}{%
      \ltx@space\ltx@space#1 = %
      \zref@extract{thepage\the\count@}{#1}sp = %
      \the\dimexpr\zref@extract{thepage\the\count@}{#1}sp\relax
      \noexpand\MessageBreak
    }{}%
  }%
}
%    \end{macrocode}
%    \end{macro}
%    \begin{macrocode}
\AfterLastShipout{%
  \ZREF@pl@AfterLastShipout
}
%    \end{macrocode}
%
%    \begin{macrocode}
%</pagelayout>
%    \end{macrocode}
%
%
% \subsection{Module \xmodule{pageattr}}
%
%    \begin{macrocode}
%<*pageattr>
\NeedsTeXFormat{LaTeX2e}
\ProvidesPackage{zref-pageattr}%
  [2020-07-03 v2.32 Module pageattr for zref (HO)]%
\RequirePackage{zref-base}[2019/11/29]
\ifx\ZREF@base@ok Y%
\else
  \expandafter\endinput
\fi
%    \end{macrocode}
%
%    \begin{macrocode}
\RequirePackage{iftex}[2019/11/07]%
%    \end{macrocode}
%
%
%    \begin{macrocode}
\let\ZREF@temp=N%
\ifluatex
\expandafter\@firstoftwo
\else
\expandafter\@secondoftwo
\fi
{%luatex
 \RequirePackage{zref-thepage}[2019/11/29]
 \RequirePackage{zref-lastpage}[2019/11/29]%
 \zref@newprop*{pdfpageattr}[]{\zref@hex{\the\pdfvariable pageattr}}%
 \zref@addprop{thepage}{pdfpageattr}%
 \zref@newprop*{pdfpagesattr}[]{\zref@hex{\the\pdfvariable pagesattr}}%
 \zref@addprop{LastPage}{pdfpagesattr}%
 \let\ZREF@temp=Y%
}
{%other
 \ltx@IfUndefined{pdfpageattr}{%
  \@PackageInfoNoLine{zref-pageattr}{%
    \string\pdfpageattr\space is not available%
  }%
  \def\zref@pdfpageattr#1{}%
  \def\zref@pdfpageattr@used#1{}%
 }{%
  \RequirePackage{zref-thepage}[2019/11/29]%
  \zref@newprop*{pdfpageattr}[]{\zref@hex{\the\pdfpageattr}}%
  \zref@addprop{thepage}{pdfpageattr}%
  \let\ZREF@temp=Y%
 }
 \ltx@IfUndefined{pdfpagesattr}{%
  \@PackageInfoNoLine{zref-pageattr}{%
    \string\pdfpagesattr\space is not available%
  }%
  \def\zref@pdfpagesattr{}%
  \def\zref@pdfpagesattr@used{}%
 }{%
  \RequirePackage{zref-lastpage}[2019/11/29]%
  \zref@newprop*{pdfpagesattr}[]{\zref@hex{\the\pdfpagesattr}}%
  \zref@addprop{LastPage}{pdfpagesattr}%
  \let\ZREF@temp=Y%
 }%
}%
%    \end{macrocode}
%    \begin{macrocode}
\ifx\ZREF@temp N%
  \expandafter\endinput
\fi
%    \end{macrocode}
%    \begin{macrocode}
\RequirePackage{zref-abspage}[2019/11/29]
\RequirePackage{atveryend}[2010/03/24]
\RequirePackage{pdftexcmds}[2010/04/01]
\let\ZREF@temp=Y%
\ltx@IfUndefined{pdf@escapehex}{\let\ZREF@temp=N}{}
\ltx@IfUndefined{pdf@unescapehex}{\let\ZREF@temp=N}{}
\ifx\ZREF@temp N%
  \let\zref@hex\ltx@firstofone
  \let\zref@unhex\ltx@firstofone
\else
  \let\zref@hex\pdf@escapehex
  \let\zref@unhex\pdf@unescapehex
\fi
%    \end{macrocode}
%
%    \begin{macro}{\ifZREF@pa@list}
%    \begin{macrocode}
\ltx@newif\ifZREF@pa@list
%    \end{macrocode}
%    \end{macro}
%    \begin{macro}{\zref@listpageattr}
%    \begin{macrocode}
\ZREF@IfDefinable\zlistpageattr\def{%
  {\ZREF@pa@listtrue}%
}
%    \end{macrocode}
%    \end{macro}
%    \begin{macro}{\ZREF@pa@AfterLastShipout}
%    \begin{macrocode}
\def\ZREF@pa@AfterLastShipout{%
  \ifZREF@pa@list
    \edef\ZREF@page@max{\the\value{abspage}}%
    \ltx@ifundefined{ZREF@org@testdef}{%
      \let\ZREF@org@testdef\@testdef
      \def\@testdef##1##2##3{%
        \ZREF@org@testdef{##1}{##2}{##3}%
        \def\ZREF@temp{##1}%
        \ifx\ZREF@temp\ZREF@RefPrefix
          \expandafter\xdef\csname##1@##2\endcsname{##3}%
        \fi
      }%
    }{}%
    \AtVeryEndDocument{\ZREF@pa@AtVeryEnd}%
  \fi
}
%    \end{macrocode}
%    \end{macro}
%    \begin{macro}{\ZREF@pa@AtVeryEnd}
%    \begin{macrocode}
\let\ZREF@temp=Y%
\ltx@IfUndefined{pdfpageattr}{}{\let\ZREF@temp=N}
\ifluatex \let\ZREF@temp=N \fi
\ifx\ZREF@temp Y
 \expandafter\@firstoftwo
\else
 \expandafter\@secondoftwo
\fi
 {%
  \def\ZREF@pa@AtVeryEnd{}%
 }
 {%
  \def\ZREF@pa@AtVeryEnd{%
    \begingroup
      \toks@{List of \ltx@backslashchar
             \ifluatex pdfvariable\else pdf\fi
             pdfpageattr:\MessageBreak}%
      \count@=1 %
      \ZREF@pa@ListPage
      \edef\x{\endgroup
        \noexpand\@PackageInfoNoLine{zref-pageattr}{%
          \the\toks@
        }%
      }%
    \x
  }%
%    \end{macrocode}
%    \end{macro}
%    \begin{macro}{\zref@pageattr}
%    \begin{macrocode}
\def\zref@pdfpageattr#1{%
  \zref@unhex{%
    \zref@extract{thepage\ZREF@number{#1}}{pdfpageattr}%
  }%
}
% compability, \zref@pageattr was defined in older versions
\let\zref@pageattr\zref@pdfpageattr
%    \end{macrocode}
%    \end{macro}
%    \begin{macro}{\zref@pageattr@used}
%    \begin{macrocode}
\ZREF@Robust\def\zref@pageattr@used#1{%
  \zref@refused{thepage\ZREF@number{#1}}%
}
%    \end{macrocode}
%    \end{macro}
%    \begin{macro}{\ZREF@pa@ListPage}
%    \begin{macrocode}
  \def\ZREF@pa@ListPage{%
    \edef\x{%
      \toks@={%
        \the\toks@
        Page \the\count@:%
        \noexpand\MessageBreak
        \zref@ifrefundefined{thepage\the\count@}{}{%
          <<\zref@pdfpageattr\count@>>%
          \noexpand\MessageBreak
        }%
      }%
    }\x
    \ifnum\ZREF@page@max>\count@
      \advance\count@ by\ltx@one
    \else
      \expandafter\ltx@gobble
    \fi
    \ZREF@pa@ListPage
  }%
}
%    \end{macrocode}
%    \end{macro}
%    \begin{macrocode}
\let\ZREF@temp=Y%
\ltx@IfUndefined{pdfpagesattr}{}{\let\ZREF@temp=N}
\ifluatex \let\ZREF@temp=N \fi
\ifx\ZREF@temp N
 \expandafter\@firstofone
\fi
 {%
%    \end{macrocode}
%    \begin{macro}{\zref@pdfpagesattr}
%    \begin{macrocode}
  \def\zref@pdfpagesattr{%
    \zref@unhex{%
      \zref@extract{LastPage}{pdfpagesattr}%
    }%
  }%
%    \end{macrocode}
%    \end{macro}
%    \begin{macro}{\zref@pdfpagesattr@used}
%    \begin{macrocode}
  \ZREF@Robust\def\zref@pdfpagesattr@used{%
    \zref@refused{LastPage}%
  }%
%    \end{macrocode}
%    \end{macro}
%    \begin{macrocode}
  \ltx@LocalAppendToMacro\ZREF@pa@AtVeryEnd{%
    \@PackageInfoNoLine{zref-pageattr}{%
      \ltx@backslashchar
      \ifluatex pdfvariable\else pdf\fi
      pagesattr:\MessageBreak
      <<\zref@pdfpagesattr>>%
      \MessageBreak
    }%
  }%
}
\AfterLastShipout{%
  \ZREF@pa@AfterLastShipout
}
%    \end{macrocode}
%    \begin{macrocode}
%</pageattr>
%    \end{macrocode}
%
%
% \subsection{Module \xmodule{marks}}
%
%    \begin{macrocode}
%<*marks>
\NeedsTeXFormat{LaTeX2e}
\ProvidesPackage{zref-marks}%
  [2020-07-03 v2.32 Module marks for zref (HO)]%
\RequirePackage{zref-base}[2019/11/29]
\ifx\ZREF@base@ok Y%
\else
  \expandafter\endinput
\fi
%    \end{macrocode}
%
%    \begin{macrocode}
\newcommand*{\zref@marks@register}[3][]{%
  \edef\ZREF@TempName{#1}%
  \edef\ZREF@TempNum{\ZREF@number{#2}}%
  \ifnum\ZREF@TempNum<\ltx@zero %
    \PackageError\ZREF@name{%
      \string\zref@marks@register\ltx@space is called with invalid%
      \MessageBreak
      marks register number (\ZREF@TempNum)%
    }{%
      Use `0' or the command, defined by \string\newmarks.\MessageBreak
      \@ehc
    }%
  \else
    \ifx\ZREF@TempName\ltx@empty
      \edef\ZREF@TempName{mark\romannumeral\ZREF@TempNum}%
    \else
      \edef\ZREF@TempName{marks\ZREF@TempName}%
    \fi
    \ZREF@MARKS@DefineProp{top}%
    \ZREF@MARKS@DefineProp{first}%
    \ZREF@MARKS@DefineProp{bot}%
    \kv@parse{#3}{%
      \ifx\kv@value\relax
        \def\kv@value{top,first,bot}%
      \fi
      \edef\ZREF@temp{\expandafter\ltx@car\kv@key X\@nil}%
      \ifx\ZREF@temp\ZREF@STAR
        \edef\kv@key{\expandafter\ltx@cdr\kv@key\@nil}%
        \zref@newlist\kv@key
      \fi
      \expandafter\comma@parse\expandafter{\kv@value}{%
        \ifcase0\ifx\comma@entry\ZREF@NAME@top 1\else
                \ifx\comma@entry\ZREF@NAME@first 1\else
                \ifx\comma@entry\ZREF@NAME@bot 1\fi\fi\fi\ltx@space
          \PackageWarning{zref-marks}{%
            Use `top', `first' or `bot' for the list values%
            \MessageBreak
            in the third argument of \string\zref@marks@register.%
            \MessageBreak
            Ignoring unkown value `\comma@entry'%
          }%
        \else
          \zref@addprop{\kv@key}{\comma@entry\ZREF@TempName}%
        \fi
        \ltx@gobble
      }%
      \ltx@gobbletwo
    }%
  \fi
}
\def\ZREF@STAR{*}
\def\ZREF@NAME@top{top}
\def\ZREF@NAME@first{first}
\def\ZREF@NAME@bot{bot}
\def\ZREF@MARKS@DefineProp#1{%
  \zref@ifpropundefined{#1\ZREF@TempName}{%
    \ifnum\ZREF@TempNum=\ltx@zero
      \begingroup
        \edef\x{\endgroup
          \noexpand\zref@newprop*{#1\ZREF@TempName}[]{%
            \expandafter\noexpand\csname#1mark\endcsname
          }%
        }%
      \x
    \else
      \begingroup
        \edef\x{\endgroup
          \noexpand\zref@newprop*{#1\ZREF@TempName}[]{%
            \expandafter\noexpand\csname#1marks\endcsname
            \ZREF@TempNum
          }%
        }%
      \x
    \fi
  }{%
    \PackageWarning{zref-marks}{%
      \string\zref@marks@register\ltx@space does not generate the%
      \MessageBreak
      new property `#1\ZREF@TempName', because\MessageBreak
      it is already defined%
    }%
  }%
}
%    \end{macrocode}
%    \begin{macrocode}
%</marks>
%    \end{macrocode}
%
% \subsection{Module \xmodule{runs}}
%
%    This module does not use the label-reference-system.
%    The reference changes with each \hologo{LaTeX}~run and would
%    force a rerun warning always.
%    \begin{macrocode}
%<*runs>
\NeedsTeXFormat{LaTeX2e}
\ProvidesPackage{zref-runs}%
  [2020-07-03 v2.32 Module runs for zref (HO)]%
%    \end{macrocode}
%
%    \begin{macro}{\zruns}
%    \begin{macrocode}
\providecommand*{\zruns}{0}%
\AtBeginDocument{%
  \edef\zruns{\number\numexpr\zruns+1}%
  \begingroup
    \def\on@line{}%
    \PackageInfo{zref-runs}{LaTeX runs: \zruns}%
    \if@filesw
      \immediate\write\@mainaux{%
        \string\gdef\string\zruns{\zruns}%
      }%
    \fi
  \endgroup
}
%    \end{macrocode}
%    \end{macro}
%
%    \begin{macrocode}
%</runs>
%    \end{macrocode}
%
%
% \subsection{Module \xmodule{perpage}}
%
%    \begin{macrocode}
%<*perpage>
\NeedsTeXFormat{LaTeX2e}
\ProvidesPackage{zref-perpage}%
  [2020-07-03 v2.32 Module perpage for zref (HO)]%
\RequirePackage{zref-base}[2019/11/29]
\ifx\ZREF@base@ok Y%
\else
  \expandafter\endinput
\fi
%    \end{macrocode}
%
%    This module resets a counter at page boundaries.
%    Because of the asynchronous output routine
%    page counter properties cannot be asked directly,
%    references are necessary.
%
%    For detecting changed pages module \xmodule{abspage}
%    is loaded.
%    \begin{macrocode}
\RequirePackage{zref-abspage}[2019/11/29]
%    \end{macrocode}
%
%    We group the properties for the needed references
%    in the property list |perpage|. The property |pagevalue|
%    records the correct value of the page counter.
%    \begin{macrocode}
\ZREF@NewPropPageValue
\zref@newlist{perpage}
\zref@addprops{perpage}{abspage,page,pagevalue}
%    \end{macrocode}
%    The page value, known by the reference mechanism, will be
%    stored in counter |zpage|.
%    \begin{macrocode}
\newcounter{zpage}
%    \end{macrocode}
%    Counter |zref@unique| helps in generating unique reference
%    names.
%    \begin{macrocode}
\zref@require@unique
%    \end{macrocode}
%    In order to be able to reset the counter, we hook here
%    into \cs{stepcounter}. In fact two nested hooks are used
%    to allow other packages to use the first hook at the beginning
%    of \cs{stepcounter}.
%    \begin{macrocode}
\let\ZREF@org@stepcounter\stepcounter
\def\stepcounter#1{%
  \ifcsname @stepcounterhook@#1\endcsname
    \csname @stepcounterhook@#1\endcsname
  \fi
  \ZREF@org@stepcounter{#1}%
}
%    \end{macrocode}
%    \cs{@stpelt} must be adapted due to the change in latex
%    2015-01, see https://github.com/ho-tex/zref/issues/26
%    \begin{macrocode}
\let\ZREF@org@@stpelt\@stpelt
\def\@stpelt#1{%
  \ifcsname ZREF@perpage@#1\endcsname
    \begingroup
      \let\stepcounter\ZREF@org@stepcounter
      \ZREF@org@@stpelt{#1}%
    \endgroup
    \expandafter\ltx@gobbletwo
  \fi
  \ZREF@org@@stpelt{#1}%
}
%    \end{macrocode}
%
%    \begin{macro}{\zmakeperpage}
%    Makro \cs{zmakeperpage} resets a counter at each page
%    break. It uses the same syntax and semantics as
%    \cs{MakePerPage} from package \xpackage{perpage} \cite{perpage}.
%    The initial start value can be given by the optional
%    argument. Default is one that means after the first
%    \cs{stepcounter} on a new page the counter starts with
%    one.
%    \begin{macrocode}
\ZREF@IfDefinable\zmakeperpage\def{%
  {%
    \@ifnextchar[\ZREF@makeperpage@opt{\ZREF@@makeperpage[\ltx@zero]}%
  }%
}
%    \end{macrocode}
%    \end{macro}
%    We hook before the counter is incremented in \cs{stepcounter},
%    package \xpackage{perpage} afterwards. Thus a little
%    calculation is necessary.
%    \begin{macrocode}
\def\ZREF@makeperpage@opt[#1]{%
  \begingroup
    \edef\x{\endgroup
      \noexpand\ZREF@@makeperpage[\number\numexpr#1-1\relax]%
    }%
  \x
}
%    \end{macrocode}
%    \begin{macrocode}
\def\ZREF@@makeperpage[#1]#2{%
  \@ifundefined{@stepcounterhook@#2}{%
    \expandafter\gdef\csname @stepcounterhook@#2\endcsname{}%
  }{}%
  \expandafter\gdef\csname ZREF@perpage@#2\endcsname{%
    \ZREF@@perpage@step{#2}{#1}%
  }%
  \expandafter\g@addto@macro\csname @stepcounterhook@#2\endcsname{%
    \ifcsname ZREF@perpage@#2\endcsname
      \csname ZREF@perpage@#2\endcsname
    \fi
  }%
}
%    \end{macrocode}
%    \begin{macro}{\ZREF@@perpage@step}
%    The heart of this module follows.
%    \begin{macrocode}
\def\ZREF@@perpage@step#1#2{%
%    \end{macrocode}
%    First the reference is generated.
%    \begin{macrocode}
  \global\advance\c@zref@unique\ltx@one
  \begingroup
    \expandafter
    \zref@labelbylist\expandafter{\thezref@unique}{perpage}%
%    \end{macrocode}
%    The \cs{expandafter} commands are necessary, because \cs{ZREF@temp}
%    is also used inside of \cs{zref@labelbylist}.
%
%    The evaluation of the reference follows. If the reference
%    is not yet kwown, we use the page counter as approximation.
%    \begin{macrocode}
    \zref@ifrefundefined\thezref@unique{%
      \global\c@zpage=\c@page
      \global\let\thezpage\thepage
      \expandafter\xdef\csname ZREF@abspage@#1\endcsname{%
        \number\c@abspage
      }%
    }{%
%    \end{macrocode}
%    The reference is used to set \cs{thezpage} and counter |zpage|.
%    \begin{macrocode}
      \global\c@zpage=\zref@extract\thezref@unique{pagevalue}\relax
      \xdef\thezpage{\noexpand\zref@extract{\thezref@unique}{page}}%
      \expandafter\xdef\csname ZREF@abspage@#1\endcsname{%
        \zref@extractdefault\thezref@unique
            {abspage}{\number\c@abspage}%
      }%
    }%
%    \end{macrocode}
%    Page changes are detected by a changed absolute page number.
%    \begin{macrocode}
    \expandafter\ifx\csname ZREF@abspage@#1\expandafter\endcsname
                    \csname ZREF@currentabspage@#1\endcsname
    \else
      \global\csname c@#1\endcsname=#2\relax
      \global\expandafter\let
          \csname ZREF@currentabspage@#1\expandafter\endcsname
          \csname ZREF@abspage@#1\endcsname
    \fi
  \endgroup
}
%    \end{macrocode}
%    \end{macro}
%    \begin{macro}{\zunmakeperpage}
%    Macro \cs{zunmakeperpage} cancels the effect of \cs{zmakeperpage}.
%    \begin{macrocode}
\ZREF@IfDefinable\zunmakeperpage\def{%
  #1{%
    \global\expandafter
    \let\csname ZREF@perpage@#1\endcsname\@undefined
  }%
}
%    \end{macrocode}
%    \end{macro}
%
%    \begin{macrocode}
%</perpage>
%    \end{macrocode}
%
%
% \subsection{Module \xmodule{titleref}}
%
%    \begin{macrocode}
%<*titleref>
\NeedsTeXFormat{LaTeX2e}
\ProvidesPackage{zref-titleref}%
  [2020-07-03 v2.32 Module titleref for zref (HO)]%
\RequirePackage{zref-base}[2019/11/29]
\ifx\ZREF@base@ok Y%
\else
  \expandafter\endinput
\fi
\RequirePackage{gettitlestring}[2009/12/08]
%    \end{macrocode}
%
% \subsubsection{Implementation}
%    \begin{macrocode}
\RequirePackage{keyval}
%    \end{macrocode}
%
%    This module makes section and caption titles available
%    for the reference system. It uses some of the ideas
%    of package \xpackage{nameref} and \xpackage{titleref}.
%
%    Now we can add the property \verb|title| is added to
%    the main property list.
%    \begin{macrocode}
\ZREF@NewPropTitle
\zref@addprop\ZREF@mainlist{title}%
%    \end{macrocode}
%
%    The title strings go into the \xfile{.aux} file, thus
%    they need some kind of protection.
%    Package \xpackage{titleref} uses a protected expansion
%    method. The advantage is that this can be used to
%    cleanup the string and to remove \cs{label}, \cs{index}
%    and other macros unwanted for referencing.
%    But there is the risk that fragile stuff can break.
%
%    Therefore package \xpackage{nameref} does not expand
%    the string. Thus the entries can safely be written to
%    the \xfile{.aux} file. But potentially dangerous
%    macros such as \cs{label} remain in the string and
%    can cause problems when using the string in references.
%
%    \begin{macro}{\ifzref@titleref@expand}
%    The switch \cs{ifzref@titleref@expand} distinguishes
%    between the both methods. Package \xpackage{nameref}'s
%    behaviour is achieved by setting the switch to false,
%    otherwise \xpackage{titleref}'s expansion is used.
%    Default is false.
%    \begin{macrocode}
\newif\ifzref@titleref@expand
%    \end{macrocode}
%    \end{macro}
%
%    \begin{macro}{\ZREF@titleref@hook}
%    The hook \cs{ZREF@titleref@hook} allows to extend
%    the cleanup for the expansion method. Thus
%    unnecessary macros can be removed or dangerous commands
%    removed. The hook is executed before the expansion of
%    \cs{zref@titleref@current}.
%    \begin{macrocode}
\let\ZREF@titleref@hook\ltx@empty
%    \end{macrocode}
%    \end{macro}
%    \begin{macro}{\zref@titleref@cleanup}
%    The hook should not be used directly, instead we provide the
%    macro \cs{zref@titleref@cleanup} to add stuff to the hook
%    and prevents that a previous non-empty content is not discarded
%    accidently.
%    \begin{macrocode}
\ZREF@Robust\def\zref@titleref@cleanup#1{%
   \begingroup
    \toks@\expandafter{%
      \ZREF@titleref@hook
      #1%
    }%
  \expandafter\endgroup
  \expandafter\def\expandafter\ZREF@titleref@hook\expandafter{%
    \the\toks@
  }%
}%
%    \end{macrocode}
%    \end{macro}
%
%    \begin{macro}{\ifzref@titleref@stripperiod}
%    Sometimes a title contains a period at the end.
%    Package \xpackage{nameref} removes this. This behaviour
%    is controlled by the switch \cs{ifzref@titleref@stripperiod}
%    and works regardless of the setting of option \xoption{expand}.
%    Period stripping is the default.
%    \begin{macrocode}
\newif\ifzref@titleref@stripperiod
\zref@titleref@stripperiodtrue
%    \end{macrocode}
%    \end{macro}
%
%    \begin{macro}{\zref@titleref@setcurrent}
%    Macro \cs{zref@titleref@setcurrent} sets a new current title
%    stored in \cs{zref@titleref@current}. Some cleanup and
%    expansion is performed that can be controlled by the
%    previous switches.
%    \begin{macrocode}
\ZREF@Robust\def\zref@titleref@setcurrent#1{%
  \ifzref@titleref@expand
    \GetTitleStringExpand{#1}%
  \else
    \GetTitleStringNonExpand{#1}%
  \fi
  \edef\zref@titleref@current{%
    \detokenize\expandafter{\GetTitleStringResult}%
  }%
  \ifzref@titleref@stripperiod
    \edef\zref@titleref@current{%
      \expandafter\ZREF@stripperiod\zref@titleref@current
      \ltx@empty.\ltx@empty\@nil
    }%
  \fi
}%
\GetTitleStringDisableCommands{%
  \ZREF@titleref@hook
}
%    \end{macrocode}
%    \end{macro}
%    \begin{macro}{\ZREF@stripperiod}
%    If \cs{ZREF@stripperiod} is called, the argument consists of
%    space tokens and tokens with catcode 12 (other), because
%    of \hologo{eTeX}'s \cs{detokenize}.
%    \begin{macrocode}
\def\ZREF@stripperiod#1.\ltx@empty#2\@nil{#1}%
%    \end{macrocode}
%    \end{macro}
%
% \subsubsection{User interface}
%
%    \begin{macro}{\ztitlerefsetup}
%    The behaviour of module \xmodule{titleref} is controlled
%    by switches and a hook. They can be set by \cs{ztitlerefsetup}
%    with a key value interface, provided by package \xpackage{keyval}.
%    Also the current title can be given explicitly by the key
%    \verb|title|.
%    \begin{macrocode}
\define@key{ZREF@TR}{expand}[true]{%
  \csname zref@titleref@expand#1\endcsname
}%
\define@key{ZREF@TR}{stripperiod}[true]{%
  \csname zref@titleref@stripperiod#1\endcsname
}%
\define@key{ZREF@TR}{cleanup}{%
  \zref@titleref@cleanup{#1}%
}%
\define@key{ZREF@TR}{title}{%
  \def\zref@titleref@current{#1}%
}%
\ZREF@IfDefinable\ztitlerefsetup\def{%
  {\kvsetkeys{ZREF@TR}}%
}%
%    \end{macrocode}
%    \end{macro}
%
%    \begin{macro}{\ztitleref}
%    The user command \cs{ztitleref} references the title.
%    For safety \cs{label} is disabled to prevent multiply
%    defined references.
%    \begin{macrocode}
\ZREF@IfDefinable\ztitleref\def{%
  {\zref@wrapper@babel\ZREF@titleref}%
}%
\def\ZREF@titleref#1{%
  \begingroup
    \zref@refused{#1}%
    \let\label\ltx@gobble
    \zref@extract{#1}{title}%
  \endgroup
}%
%    \end{macrocode}
%    \end{macro}
%
% \subsubsection{Patches for section and caption commands}
%
%    The section and caption macros are patched to extract the
%    title data.
%
%    Captions of figures and tables.
%    \begin{macrocode}
\AtBeginDocument{%
  \ZREF@patch{@caption}{%
    \long\def\@caption#1[#2]{%
      \zref@titleref@setcurrent{#2}%
      \ZREF@org@@caption{#1}[{#2}]%
    }%
  }%
%    \end{macrocode}
%
%    Section commands without star. The title version for the
%    table of contents is used because it is usually shorter and
%    more robust.
%    \begin{macrocode}
  \ZREF@patch{@part}{%
    \def\@part[#1]{%
      \zref@titleref@setcurrent{#1}%
      \ZREF@org@@part[{#1}]%
    }%
  }%
  \ZREF@patch{@chapter}{%
    \def\@chapter[#1]{%
      \zref@titleref@setcurrent{#1}%
      \ZREF@org@@chapter[{#1}]%
    }%
  }%
  \ZREF@patch{@sect}{%
    \def\@sect#1#2#3#4#5#6[#7]{%
      \zref@titleref@setcurrent{#7}%
      \ZREF@org@@sect{#1}{#2}{#3}{#4}{#5}{#6}[{#7}]%
    }%
  }%
%    \end{macrocode}
%    The star versions of the section commands.
%    \begin{macrocode}
  \ZREF@patch{@spart}{%
    \def\@spart#1{%
      \zref@titleref@setcurrent{#1}%
      \ZREF@org@@spart{#1}%
    }%
  }%
  \ZREF@patch{@schapter}{%
    \def\@schapter#1{%
      \zref@titleref@setcurrent{#1}%
      \ZREF@org@@schapter{#1}%
    }%
  }%
  \ZREF@patch{@ssect}{%
    \def\@ssect#1#2#3#4#5{%
      \zref@titleref@setcurrent{#5}%
      \ZREF@org@@ssect{#1}{#2}{#3}{#4}{#5}%
    }%
  }%
%    \end{macrocode}
%
% \subsubsection{Environment \textsf{description}}
%
%    \begin{macrocode}
  \ZREF@patch{descriptionlabel}{%
    \def\descriptionlabel#1{%
      \zref@titleref@setcurrent{#1}%
      \ZREF@org@descriptionlabel{#1}%
    }%
  }%
%    \end{macrocode}
%
% \subsubsection{Class \xclass{memoir}}
%
%    \begin{macrocode}
  \@ifclassloaded{memoir}{%
    \ltx@IfUndefined{ifheadnameref}{}{%
      \def\@chapter[#1]#2{%
        \ltx@IfUndefined{ch@pt@c}{%
          \zref@titleref@setcurrent{#1}%
        }{%
          \ifx\ch@pt@c\ltx@empty
            \zref@titleref@setcurrent{#2}%
          \else
            \def\NR@temp{#1}%
            \ifx\NR@temp\ltx@empty
              \expandafter\zref@titleref@setcurrent
              \expandafter{\ch@pt@c}%
            \else
              \ifheadnameref
                \zref@titleref@setcurrent{#1}%
              \else
                \expandafter\zref@titleref@setcurrent
                \expandafter{\ch@pt@c}%
              \fi
            \fi
          \fi
        }%
        \ZREF@org@@chapter[{#1}]{#2}%
      }%
      \ZREF@patch{M@sect}{%
        \def\M@sect#1#2#3#4#5#6[#7][#8]{%
          \ifheadnameref
            \zref@titleref@setcurrent{#8}%
          \else
            \zref@titleref@setcurrent{#7}%
          \fi
          \ZREF@org@M@sect{#1}{#2}{#3}{#4}{#5}{#6}[{#7}][{#8}]%
        }%
      }%
    }%
  }{}%
%    \end{macrocode}
%
% \subsubsection{Class \xclass{beamer}}
%
%    \begin{macrocode}
  \@ifclassloaded{beamer}{%
    \ZREF@patch{beamer@section}{%
      \long\def\beamer@section[#1]{%
        \zref@titleref@setcurrent{#1}%
        \ZREF@org@beamer@section[{#1}]%
      }%
    }%
    \ZREF@patch{beamer@subsection}{%
      \long\def\beamer@subsection[#1]{%
        \zref@titleref@setcurrent{#1}%
        \ZREF@org@beamer@subsection[{#1}]%
      }%
    }%
    \ZREF@patch{beamer@subsubsection}{%
      \long\def\beamer@subsubsection[#1]{%
        \zref@titleref@setcurrent{#1}%
        \ZREF@org@beamer@subsubsection[{#1}]%
      }%
    }%
  }{}%
%    \end{macrocode}
%
% \subsubsection{Package \xpackage{titlesec}}
%
%    \begin{macrocode}
  \@ifpackageloaded{titlesec}{%
    \ZREF@patch{ttl@sect@i}{%
      \def\ttl@sect@i#1#2[#3]#4{%
        \zref@titleref@setcurrent{#4}%
        \ZREF@org@ttl@sect@i{#1}{#2}[{#3}]{#4}%
      }%
    }%
    \ZREF@patch{ttl@straight@i}{%
      \def\ttl@straight@i#1[#2]#3{%
        \def\ZREF@temp{#2}%
        \ifx\ZREF@temp\ltx@empty
          \zref@titleref@setcurrent{#3}%
        \else
          \zref@titleref@setcurrent{#2}%
        \fi
        \ZREF@org@ttl@straight@i{#1}[{#2}]{#3}%
      }%
    }%
  }{}%
%    \end{macrocode}
%
% \subsubsection{Package \xpackage{longtable}}
%
%    Package \xpackage{longtable}: some support for its \cs{caption}.
%    However \cs{label} inside the caption is not supported.
%    \begin{macrocode}
  \@ifpackageloaded{longtable}{%
    \ZREF@patch{LT@c@ption}{%
      \def\LT@c@ption#1[#2]#3{%
        \ZREF@org@LT@c@ption{#1}[{#2}]{#3}%
        \zref@titleref@setcurrent{#2}%
      }%
    }%
  }{}%
%    \end{macrocode}
%
% \subsubsection{Package \xpackage{listings}}
%
%    Package \xpackage{listings}: support for its caption.
%    \begin{macrocode}
  \@ifpackageloaded{listings}{%
    \ZREF@patch{lst@MakeCaption}{%
      \def\lst@MakeCaption{%
        \ifx\lst@label\ltx@empty
        \else
          \expandafter\zref@titleref@setcurrent\expandafter{%
            \lst@@caption
          }%
        \fi
        \ZREF@org@lst@MakeCaption
      }%
    }%
  }{}%
%    \end{macrocode}
%
% \subsubsection{Theorems}
%
%    \begin{macrocode}
  \ZREF@patch{@opargbegintheorem}{%
    \def\@opargbegintheorem#1#2#3{%
      \zref@titleref@setcurrent{#3}%
      \ZREF@org@@opargbegintheorem{#1}{#2}{#3}%
    }%
  }%
  \@ifpackageloaded{amsthm}{%
    \begingroup
      \edef\x{macro:\string#1\string#2[\string#3]}%
      \@onelevel@sanitize\x
      \def\y#1->#2\@nil{#1}%
      \edef\z{\expandafter\y\meaning\@begintheorem->\@nil}%
      \@onelevel@sanitize\z
    \expandafter\endgroup
    \ifx\x\z
      \ZREF@patch{@begintheorem}{%
        \def\@begintheorem#1#2[#3]{%
          \zref@titleref@setcurrent{#3}%
          \ZREF@org@@begintheorem{#1}{#2}[{#3}]%
        }%
      }%
    \fi
  }{}%
%    \end{macrocode}
%
%    \begin{macrocode}
}
%    \end{macrocode}
%
%    \begin{macrocode}
%</titleref>
%    \end{macrocode}
%
%
% \subsection{Module \xmodule{xr}}
%
%    \begin{macrocode}
%<*xr>
\NeedsTeXFormat{LaTeX2e}
\ProvidesPackage{zref-xr}%
  [2020-07-03 v2.32 Module xr for zref (HO)]%
\RequirePackage{zref-base}[2019/11/29]
\ifx\ZREF@base@ok Y%
\else
  \expandafter\endinput
\fi
%    \end{macrocode}
%    \begin{macrocode}
\RequirePackage{keyval}
\RequirePackage{kvoptions}[2010/02/22]
%    \end{macrocode}
%
%    We declare property |url|, because this is added,
%    if a reference is imported and has not already set this field.
%    Or if |hyperref| is used, then this property can be asked.
%    \begin{macrocode}
\zref@newprop{url}{}%
\zref@newprop{urluse}{}%
%    \end{macrocode}
%    \begin{macrocode}
\zref@newprop{externaldocument}{}%
%    \end{macrocode}
%
%    Most code, especially the handling of the \xfile{.aux}
%    files are taken from David Carlisle's \xpackage{xr} package.
%    Therefore I drop the documentation for these macros here.
%
%    \begin{macro}{\zref@xr@ext}
%    If the URL is not specied, then assume
%    processed file with a guessed extension.
%    Use the setting of \xpackage{hyperref} if available.
%    \begin{macrocode}
\providecommand*{\zref@xr@ext}{%
  \ltx@ifundefined{XR@ext}{pdf}{\XR@ext}%
}%
%    \end{macrocode}
%    \end{macro}
%    \begin{macro}{\ifZREF@xr@zreflabel}
%    The use of the star form of \cs{zexternaldocument}
%    is remembered in the switch \cs{ifZREF@xr@zreflabel}.
%    \begin{macrocode}
\newif\ifZREF@xr@zreflabel
%    \end{macrocode}
%    \end{macro}
%    \begin{macrocode}
\SetupKeyvalOptions{%
  family=ZREF@XR,%
  prefix=ZREF@xr@%
}
\DeclareBoolOption[true]{tozreflabel}
\DeclareBoolOption[false]{toltxlabel}
\DeclareBoolOption{verbose}
\define@key{ZREF@XR}{ext}{%
  \def\zref@xr@{#1}%
}
\DeclareBoolOption[false]{urluse}
%    \end{macrocode}
%    \begin{macro}{\zxrsetup}
%    \begin{macrocode}
\newcommand*{\zxrsetup}{%
  \kvsetkeys{ZREF@XR}%
}%
%    \end{macrocode}
%    \end{macro}
%
%    \begin{macro}{\ZREF@xr@URL}
%    \begin{macrocode}
\newcount\ZREF@xr@URL
\ZREF@xr@URL=\ltx@zero
%    \end{macrocode}
%    \end{macro}
%    \begin{macro}{\ZREF@xr@AddURL}
%    \begin{macrocode}
\def\ZREF@xr@AddURL#1{%
  \begingroup
    \def\ZREF@temp{#1}%
    \count@=\ltx@one
    \ZREF@xr@@AddUrl
  \endgroup
}
%    \end{macrocode}
%    \end{macro}
%    \begin{macro}{\ZREF@xr@@AddUrl}
%    \begin{macrocode}
\def\ZREF@xr@@AddUrl{%
  \ifnum\count@>\ZREF@xr@URL
    \global\advance\ZREF@xr@URL by\ltx@one
    \xdef\ZREF@xr@theURL{\romannumeral\ZREF@xr@URL}%
    \global\expandafter\let
        \csname Z@U@\ZREF@xr@theURL\endcsname\ZREF@temp
    \@PackageInfo{zref-xr}{%
      \ltx@backslashchar Z@U@\ZREF@xr@theURL:\MessageBreak
      \ZREF@temp\MessageBreak
    }%
  \else
    \expandafter
    \ifx\csname Z@U@\romannumeral\count@\endcsname\ZREF@temp
      \xdef\ZREF@xr@theURL{\romannumeral\count@}%
    \else
      \expandafter\expandafter\expandafter\ZREF@xr@@AddUrl
    \fi
  \fi
}
%    \end{macrocode}
%    \end{macro}
%
%    \begin{macro}{\zexternaldocument}
%    In its star form it looks for \cs{newlabel},
%    otherwise for \cs{zref@newlabel}.
%     Later we
%    will read \xfile{.aux} files that expects
%    |@| to have catcode 11 (letter).
%    \begin{macrocode}
\ZREF@IfDefinable\zexternaldocument\def{%
  {%
    \ZREF@NewPropAnchor
    \ZREF@NewPropTitle
    \begingroup
      \csname @safe@actives@true\endcsname
      \makeatletter
      \@ifstar{%
        \ZREF@xr@zreflabelfalse
        \@testopt\ZREF@xr@externaldocument{}%
      }{%
        \ZREF@xr@zreflabeltrue
        \@testopt\ZREF@xr@externaldocument{}%
      }%
  }%
}%
%    \end{macrocode}
%    \end{macro}
%
%    If the \cs{include} featuer was used, there can be
%    several \xfile{.aux} files. These files are read
%    one after another, especially they are not recursively
%    read in order to save read registers. Thus it can
%    happen that the read order of the newlabel commands
%    differs from \hologo{LaTeX}'s order using \cs{input}.
%
%    \begin{macro}{\ZREF@xr@externaldocument}
%    It reads the remaining arguments. \cs{newcommand}
%    comes in handy for the optional argument.
%    \begin{macrocode}
\def\ZREF@xr@externaldocument[#1]#2{%
    \def\ZREF@xr@prefix{#1}%
    \let\ZREF@xr@filelist\ltx@empty
    \edef\ZREF@xr@externalfile{#2}%
    \edef\ZREF@xr@file{\ZREF@xr@externalfile.aux}%
    \filename@parse{#2}%
    \@testopt\ZREF@xr@graburl{#2.\zref@xr@ext}%
}%
\def\ZREF@xr@graburl[#1]{%
    \edef\ZREF@xr@url{#1}%
    \ifZREF@xr@urluse
      \expandafter\ZREF@xr@AddURL\expandafter{\ZREF@xr@url}%
      \expandafter\def\expandafter\ZREF@xr@url
      \expandafter{\csname Z@U@\ZREF@xr@theURL\endcsname}%
    \fi
    \ZREF@xr@checkfile
  \endgroup
}%
%    \end{macrocode}
%    \end{macro}
%    \begin{macro}{\ZREF@xr@processfile}
%    We follow \xpackage{xr} here, \cs{IfFileExists} offers
%    a nicer test, but we have to open the file anyway.
%    \begin{macrocode}
\def\ZREF@xr@checkfile{%
  \openin\@inputcheck\ZREF@xr@file\relax
  \ifeof\@inputcheck
    \PackageWarning{zref-xr}{%
      File `\ZREF@xr@file' not found or empty,\MessageBreak
      labels not imported%
    }%
  \else
    \PackageInfo{zref-xr}{%
      Label \ifZREF@xr@zreflabel (zref) \fi
      import from `\ZREF@xr@file'%
    }%
    \def\ZREF@xr@found{0}%
    \def\ZREF@xr@ignored@empty{0}%
    \def\ZREF@xr@ignored@zref{0}%
    \def\ZREF@xr@ignored@ltx{0}%
    \ZREF@xr@processfile
    \closein\@inputcheck
    \begingroup
      \let\on@line\ltx@empty
      \PackageInfo{zref-xr}{%
        Statistics for `\ZREF@xr@file':\MessageBreak
        \ZREF@xr@found\space
        \ifZREF@xr@zreflabel zref\else LaTeX\fi\space
        label(s) found%
        \ifnum\ZREF@xr@ignored@empty>0 %
          ,\MessageBreak
          \ZREF@xr@ignored@empty\space empty label(s) ignored%
        \fi
        \ifnum\ZREF@xr@ignored@zref>0 %
          ,\MessageBreak
          \ZREF@xr@ignored@zref\space
          duplicated zref label(s) ignored%
        \fi
        \ifnum\ZREF@xr@ignored@ltx>0 %
          ,\MessageBreak
          \ZREF@xr@ignored@ltx\space
          duplicated latex label(s) ignored%
        \fi
      }%
    \endgroup
  \fi
  \ifx\ZREF@xr@filelist\ltx@empty
  \else
    \edef\ZREF@xr@file{%
      \expandafter\ltx@car\ZREF@xr@filelist\@nil
    }%
    \edef\ZREF@xr@filelist{%
      \expandafter\ltx@cdr\ZREF@xr@filelist\ltx@empty\@nil
    }%
    \expandafter\ZREF@xr@checkfile
  \fi
}%
%    \end{macrocode}
%    \end{macro}
%
%    \begin{macro}{\ZREF@xr@processfile}
%    \begin{macrocode}
\def\ZREF@xr@processfile{%
  \read\@inputcheck to\ZREF@xr@line
  \expandafter\ZREF@xr@processline\ZREF@xr@line..\ZREF@nil
  \ifeof\@inputcheck
  \else
    \expandafter\ZREF@xr@processfile
  \fi
}%
%    \end{macrocode}
%    \end{macro}
%    \begin{macro}{\ZREF@xr@processline}
%    The most work must be done for analyzing the
%    arguments of \cs{newlabel}.
%    \begin{macrocode}
\long\def\ZREF@xr@processline#1#2#3\ZREF@nil{%
  \def\x{#1}%
  \toks@{#2}%
  \ifZREF@xr@zreflabel
    \ifx\x\ZREF@xr@zref@newlabel
      \expandafter
      \ZREF@xr@process@zreflabel\ZREF@xr@line...\ZREF@nil
    \fi
  \else
    \ifx\x\ZREF@xr@newlabel
      \expandafter
      \ZREF@xr@process@label\ZREF@xr@line...[]\ZREF@nil
    \fi
  \fi
  \ifx\x\ZREF@xr@@input
    \edef\ZREF@xr@filelist{%
      \etex@unexpanded\expandafter{\ZREF@xr@filelist}%
      {\filename@area\the\toks@}%
    }%
  \fi
}%
\def\ZREF@xr@process@zreflabel\zref@newlabel#1#2#3\ZREF@nil{%
  \edef\ZREF@xr@refname{Z@R@\ZREF@xr@prefix#1}%
  \edef\ZREF@xr@found{\the\numexpr\ZREF@xr@found+1\relax}%
  \def\x{#2}%
  \edef\ZREF@xr@tempname{$temp$}%
  \edef\ZREF@xr@temprefname{Z@R@\ZREF@xr@tempname}%
  \let\ZREF@xr@list\x
  \ifx\ZREF@xr@list\ltx@empty
    \PackageWarningNoLine{zref-xr}{%
      Label `#1' without properties ignored\MessageBreak
      in file `\ZREF@xr@file'%
    }%
    \edef\ZREF@xr@ignored@empty{%
      \the\numexpr\ZREF@xr@ignored@empty+1\relax
    }%
  \else
    \expandafter\ZREF@xr@checklist\x\ZREF@nil
    \expandafter\let\csname\ZREF@xr@temprefname\endcsname\x
    \expandafter\ltx@LocalAppendToMacro
    \csname\ZREF@xr@temprefname\expandafter\endcsname
    \expandafter{%
      \expandafter\externaldocument\expandafter{%
        \ZREF@xr@externalfile
      }%
    }%
    \ZREF@xr@urlcheck\ZREF@xr@tempname
    \ifZREF@xr@tozreflabel
      \@ifundefined{\ZREF@xr@refname}{%
        \ifZREF@xr@verbose
          \PackageInfo{zref-xr}{%
            Import to zref label `\ZREF@xr@tempname#1'%
          }%
        \fi
        \global\expandafter
        \let\csname\ZREF@xr@refname\expandafter\endcsname
        \csname\ZREF@xr@temprefname\endcsname
      }{%
        \ZREF@xr@zref@ignorewarning{\ZREF@xr@prefix#1}%
      }%
    \fi
    \ifZREF@xr@toltxlabel
      \ZREF@xr@tolabel{\ZREF@xr@tempname}{\ZREF@xr@prefix#1}%
    \fi
  \fi
}%
\def\ZREF@xr@process@label\newlabel#1#2#3[#4]#5\ZREF@nil{%
  \def\ZREF@xr@refname{Z@R@\ZREF@xr@prefix#1}%
  \edef\ZREF@xr@found{\the\numexpr\ZREF@xr@found+1\relax}%
  \def\x{#2}%
  \edef\ZREF@xr@tempname{$temp$}%
  \edef\ZREF@xr@temprefname{Z@R@\ZREF@xr@tempname}%
  \expandafter\ZREF@xr@scanparams
      \csname\ZREF@xr@temprefname\expandafter\endcsname
      \x{}{}{}{}{}\ZREF@nil
  \ifx\\#4\\%
  \else
    % ntheorem knows an optional argument at the end of \newlabel
    \ZREF@NewPropTheotype
    \expandafter\ltx@LocalAppendToMacro
        \csname\ZREF@xr@temprefname\endcsname{\theotype{#4}}%
  \fi
  \expandafter\ltx@LocalAppendToMacro
  \csname\ZREF@xr@temprefname\expandafter\endcsname\expandafter{%
    \expandafter\externaldocument\expandafter{%
      \ZREF@xr@externalfile
    }%
  }%
  \ZREF@xr@urlcheck\ZREF@xr@tempname
  \ifZREF@xr@tozreflabel
    \@ifundefined{\ZREF@xr@refname}{%
      \ifZREF@xr@verbose
        \PackageInfo{zref-xr}{%
          Import to zref label `\ZREF@xr@prefix#1'%
        }%
      \fi
      \global\expandafter
      \let\csname\ZREF@xr@refname\expandafter\endcsname
      \csname\ZREF@xr@temprefname\endcsname
    }{%
      \ZREF@xr@zref@ignorewarning{\ZREF@xr@prefix#1}%
    }%
  \fi
  \ifZREF@xr@toltxlabel
    \ZREF@xr@tolabel{\ZREF@xr@tempname}{\ZREF@xr@prefix#1}%
  \fi
}
\def\ZREF@xr@zref@newlabel{\zref@newlabel}%
\def\ZREF@xr@newlabel{\newlabel}%
\def\ZREF@xr@@input{\@input}%
\def\ZREF@xr@relax{\relax}%
%    \end{macrocode}
%    \end{macro}
%    \begin{macro}{\ZREF@xr@tolabel}
%    \begin{macrocode}
\def\ZREF@xr@tolabel#1#2{%
  \ifZREF@xr@verbose
    \PackageInfo{zref-xr}{%
      Import to LaTeX label `#2'%
    }%
  \fi
  \zref@wrapper@unexpanded{%
    \expandafter\xdef\csname r@#2\endcsname{%
      {%
        \ltx@ifundefined{M@TitleReference}{%
          \ltx@ifundefined{TR@TitleReference}{%
            \zref@extractdefault{#1}{default}{}%
          }{%
            \noexpand\TR@TitleReference
            {\zref@extractdefault{#1}{default}{}}%
            {\zref@extractdefault{#1}{title}{}}%
          }%
        }{%
          \noexpand\M@TitleReference
          {\zref@extractdefault{#1}{default}{}}%
          {\zref@extractdefault{#1}{title}{}}%
        }%
      }%
      {\zref@extractdefault{#1}{page}{}}%
      \ltx@ifpackageloaded{nameref}{%
        {\zref@extractdefault{#1}{title}{}}%
        {\zref@extractdefault{#1}{anchor}{}}%
        \zref@ifrefcontainsprop{#1}{urluse}{%
          {\zref@extractdefault{#1}{urluse}{}}%
        }{%
          {\zref@extractdefault{#1}{url}{}}%
        }%
      }{}%
    }%
  }%
}
%    \end{macrocode}
%    \end{macro}
%
%    \begin{macro}{\ZREF@xr@zref@ignorewarning}
%    \begin{macrocode}
\def\ZREF@xr@zref@ignorewarning#1{%
  \PackageWarningNoLine{zref-xr}{%
    Zref label `#1' is already in use\MessageBreak
    in file `\ZREF@xr@file'%
  }%
  \edef\ZREF@xr@ignored@zref{%
    \the\numexpr\ZREF@xr@ignored@zref+1%
  }%
}%
%    \end{macrocode}
%    \end{macro}
%
%    \begin{macro}{\ZREF@xr@ltx@ignorewarning}
%    \begin{macrocode}
\def\ZREF@xr@ltx@ignorewarning#1{%
  \PackageWarningNoLine{zref-xr}{%
    LaTeX label `#1' is already in use\MessageBreak
    in file `\ZREF@xr@file'%
  }%
  \edef\ZREF@xr@ignored@ltx{%
    \the\numexpr\ZREF@xr@ignored@ltx+1%
  }%
}%
%    \end{macrocode}
%    \end{macro}
%
%    \begin{macro}{\ZREF@xr@checklist}
%    \begin{macrocode}
\def\ZREF@xr@checklist#1#2#3\ZREF@nil{%
  \ifx\@undefined#1\relax
    \expandafter\ZREF@xr@checkkey\string#1\@nil
  \fi
  \ifx\\#3\\%
  \else
    \ltx@ReturnAfterFi{%
      \ZREF@xr@checklist#3\ZREF@nil
    }%
  \fi
}%
\def\ZREF@xr@checkkey#1#2\@nil{%
  \zref@ifpropundefined{#2}{%
    \zref@newprop{#2}{}%
  }{}%
}%
%    \end{macrocode}
%    \end{macro}
%
%    \begin{macro}{\ZREF@xr@scanparams}
%    \begin{macrocode}
\def\ZREF@xr@scanparams#1#2#3#4#5#6#7\ZREF@nil{%
  \let#1\ltx@empty
  \ZREF@foundfalse
  \ZREF@xr@scantitleref#1#2\TR@TitleReference{}{}\ZREF@nil
  \ifZREF@found
  \else
    \ltx@LocalAppendToMacro#1{\default{#2}}%
  \fi
  % page
  \ltx@LocalAppendToMacro#1{\page{#3}}%
  % nameref title
  \ifZREF@found
  \else
    \ifx\\#4\\%
    \else
      \def\ZREF@xr@temp{#4}%
      \ifx\ZREF@xr@temp\ZREF@xr@relax
      \else
        \ltx@LocalAppendToMacro#1{\title{#4}}%
      \fi
    \fi
  \fi
  % anchor
  \ifx\\#5\\%
  \else
    \ltx@LocalAppendToMacro#1{\anchor{#5}}%
  \fi
  \ifx\\#6\\%
  \else
    \ifZREF@xr@urluse
      \ZREF@xr@AddURL{#6}%
      \expandafter\ltx@LocalAppendToMacro\expandafter#1%
      \expandafter{%
        \expandafter\urluse\expandafter{%
          \csname Z@U@\ZREF@xr@theURL\endcsname
        }%
      }%
    \else
      \ltx@LocalAppendToMacro#1{\url{#6}}%
    \fi
  \fi
}%
%    \end{macrocode}
%    \end{macro}
%
%    \begin{macro}{\ZREF@xr@scantitleref}
%    \begin{macrocode}
\def\ZREF@xr@scantitleref#1#2\TR@TitleReference#3#4#5\ZREF@nil{%
  \ifx\\#5\\%
  \else
    \ltx@LocalAppendToMacro#1{%
      \default{#3}%
      \title{#4}%
    }%
    \ZREF@foundtrue
  \fi
}%
%    \end{macrocode}
%    \end{macro}
%
%    \begin{macro}{\ZREF@xr@urlcheck}%
%    \begin{macrocode}
\def\ZREF@xr@urlcheck#1{%
  \zref@ifrefcontainsprop{#1}{anchor}{%
    \zref@ifrefcontainsprop{#1}{url}{%
    }{%
      \expandafter
      \ltx@LocalAppendToMacro\csname Z@R@#1\expandafter\endcsname
      \expandafter{%
        \csname url\ifZREF@xr@urluse use\fi
        \expandafter\endcsname\expandafter{\ZREF@xr@url}%
      }%
    }%
  }{%
  }%
}%
%    \end{macrocode}
%    \end{macro}
%
%    \begin{macrocode}
%</xr>
%    \end{macrocode}
%
%
% \subsection{Module \xmodule{hyperref}}
%
%    UNFINISHED :-(
%
%    \begin{macrocode}
%<*hyperref>
\NeedsTeXFormat{LaTeX2e}
\ProvidesPackage{zref-hyperref}%
  [2020-07-03 v2.32 Module hyperref for zref (HO)]%
\RequirePackage{zref-base}[2019/11/29]
\ifx\ZREF@base@ok Y%
\else
  \expandafter\endinput
\fi
%    \end{macrocode}
%
%    \begin{macrocode}
\ZREF@NewPropAnchor
\zref@addprop\ZREF@mainlist{anchor}%
%    \end{macrocode}
%
%    \begin{macrocode}
%</hyperref>
%    \end{macrocode}
%
%
% \subsection{Module \xmodule{savepos}}
%
%    Module \xmodule{savepos} provides an interface for
%    \pdfTeX's \cmd{\pdfsavepos}, see the manual for
%    \pdfTeX.
%
% \subsubsection{Identification}
%
%    \begin{macrocode}
%<*savepos>
\NeedsTeXFormat{LaTeX2e}
\ProvidesPackage{zref-savepos}%
  [2020-07-03 v2.32 Module savepos for zref (HO)]%
\RequirePackage{zref-base}[2019/11/29]
\ifx\ZREF@base@ok Y%
\else
  \expandafter\endinput
\fi
%    \end{macrocode}
%
% \subsubsection{Availability}
%
%    First we check, whether the feature is available.
%    \begin{macrocode}
\ifx\directlua\@undefined
\ltx@IfUndefined{pdfsavepos}{%
  \PackageError\ZREF@name{%
    \string\pdfsavepos\space is not supported.\MessageBreak
    It is provided by pdfTeX (1.40) or XeTeX%
  }\ZREF@UpdatePdfTeX
  \endinput
}{}%
\fi
%    \end{macrocode}
%    In PDF mode we are done. However support for DVI mode
%    was added later in version 1.40.0. In earlier versions
%    \cs{pdfsavepos} is defined, but its execution raises
%    an error. Note that \hologo{XeTeX} also provides \cs{pdfsavepos}.
%    \begin{macrocode}
\ifpdf
\else
  \ltx@IfUndefined{pdftexversion}{%
  }{%
    \ifnum\pdftexversion<140 %
      \PackageError\ZREF@name{%
        \string\pdfsavepos\space is not supported in DVI mode%
        \MessageBreak
        of this pdfTeX version%
      }\ZREF@UpdatePdfTeX
      \expandafter\expandafter\expandafter\endinput
    \fi
  }%
\fi
%    \end{macrocode}
%
% \subsubsection{Setup}
%
%    \begin{macrocode}
\zref@newlist{savepos}
\ifx\directlua\@undefined
  \zref@newprop*{posx}[0]{\the\pdflastxpos}
  \zref@newprop*{posy}[0]{\the\pdflastypos}
\else
  \zref@newprop*{posx}[0]{\the\lastxpos}
  \zref@newprop*{posy}[0]{\the\lastypos}
\fi
\zref@addprops{savepos}{posx,posy}
%    \end{macrocode}
%
% \subsubsection{User macros}
%
%    \begin{macro}{\zref@savepos}
%    \begin{macrocode}
\ifx\directlua\@undefined
  \def\zref@savepos{%
    \if@filesw
      \pdfsavepos
    \fi
  }
\else
  \def\zref@savepos{%
    \if@filesw
      \savepos
    \fi
  }
\fi
%    \end{macrocode}
%    \end{macro}
%
%    \begin{macro}{\ZREF@zsavepos}
%    \begin{macrocode}
\def\ZREF@zsavepos#1#2#3{%
  \@bsphack
  \if@filesw
    \zref@savepos
    #1{#3}{#2}%
    \ltx@IfUndefined{TeXXeTstate}{%
    }{%
      \ifnum\TeXXeTstate=\ltx@zero
      \else
        \zref@savepos
      \fi
    }%
  \fi
  \@esphack
}
%    \end{macrocode}
%    \end{macro}
%
%    \begin{macro}{\zsavepos}
%    The current location is stored in a reference with the
%    given name.
%    \begin{macrocode}
\ZREF@IfDefinable\zsavepos\def{%
  {%
    \ZREF@zsavepos\zref@labelbylist{savepos}%
  }%
}
%    \end{macrocode}
%    \end{macro}
%
%    \begin{macro}{\zsaveposx}
%    \begin{macrocode}
\ZREF@IfDefinable\zsaveposx\def{%
  {%
    \ZREF@zsavepos\zref@labelbyprops{posx}%
  }%
}
%    \end{macrocode}
%    \end{macro}
%    \begin{macro}{\zsaveposy}
%    \begin{macrocode}
\ZREF@IfDefinable\zsaveposy\def{%
  {%
    \ZREF@zsavepos\zref@labelbyprops{posy}%
  }%
}
%    \end{macrocode}
%    \end{macro}
%
%    \begin{macro}{\zposx}
%    \begin{macro}{\zposy}
%    The horizontal and vertical position are available by
%    \cs{zposx} and \cs{zposy}. Do not rely on absolute
%    positions. They differ in DVI and PDF mode of \pdfTeX.
%    Use differences instead. The unit of the position numbers
%    is sp.
%    \begin{macrocode}
\newcommand*{\zposx}[1]{%
  \zref@extract{#1}{posx}%
}%
\newcommand*{\zposy}[1]{%
  \zref@extract{#1}{posy}%
}%
%    \end{macrocode}
%    \end{macro}
%    \end{macro}
%
%    Typically horizontal and vertical positions are used
%    inside calculations. Therefore the extracting macros
%    should be expandable and babel's patch is not applyable.
%
%    Also it is in the responsibility of the user to marked
%    used positions by \cs{zrefused}
%    in order to notify \hologo{LaTeX} about undefined references.
%
%    \begin{macro}{\ZREF@savepos@ok}
%    \begin{macrocode}
\let\ZREF@savepos@ok=Y
%    \end{macrocode}
%    \end{macro}
%
%    \begin{macrocode}
%</savepos>
%    \end{macrocode}
%
%
% \subsection{Module \xmodule{abspos}}
%
% \subsubsection{Identification}
%
%    \begin{macrocode}
%<*abspos>
\NeedsTeXFormat{LaTeX2e}
\ProvidesPackage{zref-abspos}%
  [2020-07-03 v2.32 Module abspos for zref (HO)]%
\RequirePackage{zref-base}[2019/11/29]
\ifx\ZREF@base@ok Y%
\else
  \expandafter\endinput
\fi
%    \end{macrocode}
%
%    \begin{macrocode}
\RequirePackage{zref-savepos}[2019/11/29]
\ifx\ZREF@savepos@ok Y%
\else
  \expandafter\endinput
\fi
%    \end{macrocode}
%    \begin{macrocode}
\RequirePackage{zref-pagelayout}[2019/11/29]
\zref@addprop{savepos}{abspage}
\zref@addprop{savepos}{pagevalue}
%    \end{macrocode}
%
%    \begin{macro}{\zref@absposx}
%    \begin{macrocode}
\newcommand*{\zref@absposx}[3]{%
  \number
    \expandafter\zref@absposnumx\expandafter{%
      \number\zref@extractdefault{#1}{abspage}{0}%
    }{#2}{#3}%
  \ltx@space
}
%    \end{macrocode}
%    \end{macro}
%
%    \begin{macro}{\zref@absposy}
%    \begin{macrocode}
\newcommand*{\zref@absposy}[3]{%
  \number
  \expandafter\zref@absposnumy\expandafter{%
    \number\zref@extractdefault{#1}{abspage}{0}%
  }{#2}{#3}%
  \ltx@space
}
%    \end{macrocode}
%    \end{macro}
%
%    \begin{macro}{\zref@absposnumx}
%    \begin{macrocode}
\newcommand*{\zref@absposnumx}[3]{%
  \number
%  \ifnum#1>\ltx@zero
%    \zref@ifrefundefined{thepage#1}{%
%      0%
%    }{%
      \numexpr\ZREF@absposnum{thepage#1}{#2}{x}{#3}\relax
%    }%
%  \else
%    0%
%  \fi
}
%    \end{macrocode}
%    \end{macro}
%
%    \begin{macro}{\zref@absposnumy}
%    \begin{macrocode}
\newcommand*{\zref@absposnumy}[3]{%
  \number
%  \ifnum#1>\ltx@zero
%    \zref@ifrefundefined{thepage#1}{%
%      0%
%    }{%
      \numexpr\ZREF@absposnum{thepage#1}{#2}{y}{#3}\relax
%    }%
%  \else
%    0%
%  \fi
}
%    \end{macrocode}
%    \end{macro}
%
%    \begin{macro}{\ZREF@absposnum}
%    \begin{macrocode}
\def\ZREF@absposnum#1#2#3#4{%
  \ltx@ifundefined{ZREF@abspos@#2@#3@#4}{%
    0%
  }{%
    \csname ZREF@abspos@#2@#3@#4\endcsname{#1}%
  }%
}
%    \end{macrocode}
%    \end{macro}
%
%    \begin{macro}{\zref@def@absposx}
%    \begin{macrocode}
\ZREF@Robust\def\zref@def@absposx#1{%
  \zref@wrapper@babel{\ZREF@def@abspos{#1}\zref@absposx}%
}
%    \end{macrocode}
%    \end{macro}
%
%    \begin{macro}{\zref@def@absposy}
%    \begin{macrocode}
\ZREF@Robust\def\zref@def@absposy#1{%
  \zref@wrapper@babel{\ZREF@def@abspos{#1}\zref@absposy}%
}
%    \end{macrocode}
%    \end{macro}
%
%    \begin{macro}{\zref@def@absposnumx}
%    \begin{macrocode}
\ZREF@Robust\def\zref@def@absposnumx#1{%
  \zref@wrapper@babel{\ZREF@def@abspos{#1}\zref@absposnumx}%
}
%    \end{macrocode}
%    \end{macro}
%
%    \begin{macro}{\zref@def@absposnumy}
%    \begin{macrocode}
\ZREF@Robust\def\zref@def@absposnumy#1{%
  \zref@wrapper@babel{\ZREF@def@abspos{#1}\zref@absposnumy}%
}
%    \end{macrocode}
%    \end{macro}
%    \begin{macro}{\ZREF@def@abspos}
%    \begin{macrocode}
\def\ZREF@def@absposnumy#1#2#3#4#5{%
  \edef#1{#2{#3}{#4}{#5}}%
}
%    \end{macrocode}
%    \end{macro}
%
%    \begin{macro}{\zref@absposused}
%    \begin{macrocode}
\ZREF@Robust\def\zref@absposused{%
  \zref@wrapper@babel\ZREF@abspos@used
}
%    \end{macrocode}
%    \end{macro}
%    \begin{macro}{\ZREF@abspos@used}
%    \begin{macrocode}
\def\ZREF@abspos@used#1{%
  \zref@refused{#1}%
  \zref@ifrefundefined{#1}{%
  }{%
    \begingroup
      \edef\ZREF@temp{%
        \zref@extractdefault{#1}{abspage}{0}%
      }%
      \ifnum\ZREF@temp>\ltx@zero
        \zref@refused{thepage\ZREF@temp}%
      \else
        \@PackageError{zref-abspos}{%
          \string\zref@pos@label@used\ltx@space
          needs property `abspage'\MessageBreak
          in label `#1'%
        }\@ehc
      \fi
    \endgroup
  }%
}
%    \end{macrocode}
%    \end{macro}
%
%    \begin{macro}{\zref@absposnumused}
%    \begin{macrocode}
\newcommand*{\zref@absposnumused}[1]{%
  \ifnum#1>\ltx@zero
    \zref@refused{thepage\number#1}%
  \else
    \@PackageError{zref-abspos}{%
      Invalid absolute page number (#1)\MessageBreak
      for \string\zref@pos@num@used.\MessageBreak
      A positive integer is expected%
    }\@ehc
  \fi
}
%    \end{macrocode}
%    \end{macro}
%
%    \begin{macro}{\zref@ifabsposundefined}
%    \begin{macrocode}
\def\zref@ifabsposundefined#1{%
  \zref@ifrefundefined{#1}\ltx@firsttwo{%
    \expandafter\zref@ifabsposnumundefined\expandafter{%
      \number\zref@extractdefault{#1}{abspage}{0}%
    }%
  }%
}
%    \end{macrocode}
%    \end{macro}
%
%    \begin{macro}{\zref@ifabsposnumundefined}
%    \begin{macrocode}
\def\zref@ifabsposnumundefined#1{%
  \ifnum\ZREF@number{#1}>\ltx@zero
    \zref@ifrefundefined{thepage#1}%
        \ltx@firstoftwo\ltx@secondoftwo
  \else
    \expandafter\ltx@firstoftwo
  \fi
}
%    \end{macrocode}
%    \end{macro}
%
% \subsubsection{Media}
%
%    \begin{macro}{\ZREF@abspos@media@width}
%    \begin{macrocode}
\edef\ZREF@abspos@media@width{%
  \ltx@ifundefined{pdfpagewidth}{%
    \ltx@ifundefined{mediawidth}{%
      \ltx@ifundefined{stockwidth}{%
        paperwidth%
      }{%
        stockwidth%
      }%
    }{%
      mediawidth%
    }%
  }{%
    pdfpagewidth%
  }%
}
\ifluatex
\def\ZREF@abspos@media@width{pdfpagewidth}%
\fi
%    \end{macrocode}
%    \end{macro}
%    \begin{macro}{\ZREF@abspos@media@height}
%    \begin{macrocode}
\edef\ZREF@abspos@media@height{%
  \ltx@ifundefined{pdfpageheight}{%
    \ltx@ifundefined{mediaheight}{%
      \ltx@ifundefined{stockheight}{%
        paperheight%
      }{%
        stockheight%
      }%
    }{%
      mediaheight%
    }%
  }{%
    \noexpand\ifcase\pdfpageheight
      \ltx@ifundefined{stockheight}{%
        paperheight%
      }{%
        stockheight%
      }%
    \noexpand\else
      pdfpageheight%
    \noexpand\fi
  }%
}
\ifluatex
\edef\ZREF@abspos@media@height{%
   \noexpand\ifcase\pageheight
      \ltx@ifundefined{stockheight}{%
        paperheight%
      }{%
        stockheight%
      }%
    \noexpand\else
      pdfpageheight%
    \noexpand\fi}%
\fi
%    \end{macrocode}
%    \end{macro}
%
%    \begin{macro}{\ZREF@abspos@media@x@left}
%    \begin{macrocode}
\def\ZREF@abspos@media@x@left#1{%
  0%
}
%    \end{macrocode}
%    \end{macro}
%    \begin{macro}{\ZREF@abspos@media@x@right}
%    \begin{macrocode}
\def\ZREF@abspos@media@x@right#1{%
  \zref@extract{#1}\ZREF@abspos@media@width
}
%    \end{macrocode}
%    \end{macro}
%    \begin{macro}{\ZREF@abspos@media@x@center}
%    \begin{macrocode}
\def\ZREF@abspos@media@x@center#1{%
  \ZREF@abspos@media@x@left{#1}%
  +\zref@extract{#1}\ZREF@abspos@media@width/2%
}
%    \end{macrocode}
%    \end{macro}
%
%    \begin{macro}{\ZREF@abspos@media@y@top}
%    \begin{macrocode}
\def\ZREF@abspos@media@y@top#1{%
  \zref@extract{#1}\ZREF@abspos@media@height
}
%    \end{macrocode}
%    \end{macro}
%    \begin{macro}{\ZREF@abspos@media@y@bottom}
%    \begin{macrocode}
\def\ZREF@abspos@media@y@bottom#1{%
  0%
}
%    \end{macrocode}
%    \end{macro}
%    \begin{macro}{\ZREF@abspos@media@y@center}
%    \begin{macrocode}
\def\ZREF@abspos@media@y@center#1{%
  \zref@extract{#1}\ZREF@abspos@media@height/2%
}
%    \end{macrocode}
%    \end{macro}
%
% \subsubsection{Paper}
%
%    \begin{macro}{\ZREF@abspos@paper@x@left}
%    \begin{macrocode}
\def\ZREF@abspos@paper@x@left#1{%
  0%
}
%    \end{macrocode}
%    \end{macro}
%    \begin{macro}{\ZREF@abspos@paper@x@right}
%    \begin{macrocode}
\def\ZREF@abspos@paper@x@right#1{%
  \zref@extract{#1}{paperwidth}%
}
%    \end{macrocode}
%    \end{macro}
%    \begin{macro}{\ZREF@abspos@paper@x@center}
%    \begin{macrocode}
\def\ZREF@abspos@paper@x@center#1{%
  \zref@extract{#1}{paperwidth}/2%
}
%    \end{macrocode}
%    \end{macro}
%
%    \begin{macro}{\ZREF@abspos@paper@y@top}
%    \begin{macrocode}
\let\ZREF@abspos@paper@y@top\ZREF@abspos@media@y@top
%    \end{macrocode}
%    \end{macro}
%    \begin{macro}{\ZREF@abspos@paper@y@bottom}
%    \begin{macrocode}
\def\ZREF@abspos@paper@y@bottom#1{%
  \ZREF@abspos@paper@y@top{#1}%
  -\zref@extract{#1}{paperheight}%
}
%    \end{macrocode}
%    \end{macro}
%    \begin{macro}{\ZREF@abspos@paper@y@center}
%    \begin{macrocode}
\def\ZREF@abspos@paper@y@center#1{%
  \ZREF@abspos@paper@y@top{#1}%
  -\zref@extract{#1}{paperheight}/2%
}
%    \end{macrocode}
%    \end{macro}
%
% \subsubsection{Origin}
%
% There doesn't seem a good reason to make these tests depend on pdf mode
%  in current engines, so comment out the \verb|\ifpdf| tests.
%    \begin{macro}{\ZREF@abspos@origin@x}
%    \begin{macrocode}
\let\ZREF@temp\ltx@two
\ltx@IfUndefined{pdfhorigin}{}{%
%  \ifpdf
    \let\ZREF@temp\ltx@zero
%  \fi
}
\ifluatex
% \ifpdf
  \let\ZREF@temp\ltx@zero
% \fi
\fi

\ifx\ZREF@temp\ltx@two
  \ifnum\mag=1000 %
    \let\ZREF@temp\ltx@one
  \fi
\fi
\ifcase\ZREF@temp
  \def\ZREF@abspos@origin@x#1{%
    \zref@extract{#1}{pdfhorigin}%
  }%
\or
  \def\ZREF@abspos@origin@x#1{%
    4736286%
  }%
\or
  \def\ZREF@abspos@origin@x#1{%
    \numexpr\mag/1000*\dimexpr 1truein\relax\relax
  }%
\fi
%    \end{macrocode}
%    \end{macro}
%    \begin{macro}{\ZREF@abspos@origin@y}
%    \begin{macrocode}
\let\ZREF@temp\ltx@two
\ltx@IfUndefined{pdfvorigin}{}{%
%  \ifpdf
    \let\ZREF@temp\ltx@zero
%  \fi
}
\ifluatex
%  \ifpdf
    \let\ZREF@temp\ltx@zero
%  \fi
\fi
\ifx\ZREF@temp\ltx@two
  \ifnum\mag=1000 %
    \let\ZREF@temp\ltx@one
  \fi
\fi
\ifcase\ZREF@temp
  \def\ZREF@abspos@origin@y#1{%
    \zref@extract{#1}{pdfvorigin}%
  }%
\or
  \def\ZREF@abspos@origin@y#1{%
    4736286%
  }%
\or
  \def\ZREF@abspos@origin@y#1{%
    \numexpr\mag/1000*\dimexpr 1truein\relax\relax
   }%
\fi
%    \end{macrocode}
%    \end{macro}
%
% \subsubsection{Header}
%
%    \begin{macro}{\ZREF@abspos@head@x@left}
%    \begin{macrocode}
\def\ZREF@abspos@head@x@left#1{%
  \ZREF@abspos@paper@x@left{#1}%
  +\ZREF@abspos@origin@x{#1}%
  +\zref@extract{#1}{hoffset}%
  +\ifodd\zref@extractdefault{#1}{pagevalue}{\number\c@page} %
     \zref@extract{#1}{oddsidemargin}%
   \else
     \zref@extract{#1}{evensidemargin}%
   \fi
}
%    \end{macrocode}
%    \end{macro}
%    \begin{macro}{\ZREF@abspos@head@x@right}
%    \begin{macrocode}
\def\ZREF@abspos@head@x@right#1{%
  \ZREF@abspos@head@x@left{#1}%
  +\zref@extract{#1}{textwidth}%
}
%    \end{macrocode}
%    \end{macro}
%    \begin{macro}{\ZREF@abspos@head@x@center}
%    \begin{macrocode}
\def\ZREF@abspos@head@x@center#1{%
  \ZREF@abspos@head@x@left{#1}%
  +\zref@extract{#1}{textwidth}/2%
}
%    \end{macrocode}
%    \end{macro}
%
%    \begin{macro}{\ZREF@abspos@head@y@top}
%    \begin{macrocode}
\def\ZREF@abspos@head@y@top#1{%
  \ZREF@abspos@paper@y@top{#1}%
  -\ZREF@abspos@origin@y{#1}%
  -\zref@extract{#1}{voffset}%
  -\zref@extract{#1}{topmargin}%
}
%    \end{macrocode}
%    \end{macro}
%    \begin{macro}{\ZREF@abspos@head@y@bottom}
%    \begin{macrocode}
\def\ZREF@abspos@head@y@bottom#1{%
  \ZREF@abspos@head@y@top{#1}%
  -\zref@extract{#1}{headheight}%
}
%    \end{macrocode}
%    \end{macro}
%    \begin{macro}{\ZREF@abspos@head@y@center}
%    \begin{macrocode}
\def\ZREF@abspos@head@y@center#1{%
  \ZREF@abspos@head@y@top{#1}%
  -\zref@extract{#1}{headheight}/2%
}
%    \end{macrocode}
%    \end{macro}
%
% \subsubsection{Body}
%
%    \begin{macro}{\ZREF@abspos@body@x@left}
%    \begin{macrocode}
\let\ZREF@abspos@body@x@left\ZREF@abspos@head@x@left
%    \end{macrocode}
%    \end{macro}
%    \begin{macro}{\ZREF@abspos@body@x@right}
%    \begin{macrocode}
\let\ZREF@abspos@body@x@right\ZREF@abspos@head@x@right
%    \end{macrocode}
%    \end{macro}
%    \begin{macro}{\ZREF@abspos@body@x@center}
%    \begin{macrocode}
\let\ZREF@abspos@body@x@center\ZREF@abspos@head@x@center
%    \end{macrocode}
%    \end{macro}
%
%    \begin{macro}{\ZREF@abspos@body@y@top}
%    \begin{macrocode}
\def\ZREF@abspos@body@y@top#1{%
  \ZREF@abspos@head@y@bottom{#1}%
  -\zref@extract{#1}{headsep}%
}
%    \end{macrocode}
%    \end{macro}
%    \begin{macro}{\ZREF@abspos@body@y@bottom}
%    \begin{macrocode}
\def\ZREF@abspos@body@y@bottom#1{%
  \ZREF@abspos@body@y@top{#1}%
  -\zref@extract{#1}{textheight}%
}
%    \end{macrocode}
%    \end{macro}
%    \begin{macro}{\ZREF@abspos@body@y@center}
%    \begin{macrocode}
\def\ZREF@abspos@body@y@center#1{%
  \ZREF@abspos@body@y@top{#1}%
  -\zref@extract{#1}{textheight}/2%
}
%    \end{macrocode}
%    \end{macro}
%
% \subsubsection{Footer}
%
%    \begin{macro}{\ZREF@abspos@foot@x@left}
%    \begin{macrocode}
\let\ZREF@abspos@foot@x@left\ZREF@abspos@head@x@left
%    \end{macrocode}
%    \end{macro}
%    \begin{macro}{\ZREF@abspos@foot@x@right}
%    \begin{macrocode}
\let\ZREF@abspos@foot@x@right\ZREF@abspos@head@x@right
%    \end{macrocode}
%    \end{macro}
%    \begin{macro}{\ZREF@abspos@foot@x@center}
%    \begin{macrocode}
\let\ZREF@abspos@foot@x@center\ZREF@abspos@head@x@center
%    \end{macrocode}
%    \end{macro}
%
%    \begin{macro}{\ZREF@abspos@foot@y@bottom}
%    \begin{macrocode}
\def\ZREF@abspos@foot@y@bottom#1{%
  \ZREF@abspos@body@y@bottom{#1}%
  -\zref@extract{#1}{footskip}%
}
%    \end{macrocode}
%    \end{macro}
%
% \subsubsection{Marginal notes}
%
%    \begin{macro}{\ZREF@abspos@marginpar@x@left}
%    \begin{macrocode}
\def\ZREF@abspos@marginpar@x@left#1{%
  \ifodd\zref@extractdefault{#1}{pagevalue}{\number\c@page} %
    \ZREF@abspos@body@x@right{#1}%
    +\zref@extract{#1}{marginparsep}%
  \else
    \ZREF@abspos@body@x@left{#1}%
    -\zref@extract{#1}{marginparsep}%
    -\zref@extract{#1}{marginparwidth}%
  \fi
}
%    \end{macrocode}
%    \end{macro}
%    \begin{macro}{\ZREF@abspos@marginpar@x@right}
%    \begin{macrocode}
\def\ZREF@abspos@marginpar@x@right#1{%
  \ZREF@abspos@marginpar@x@left{#1}%
  +\zref@extract{#1}{marginparwidth}%
}
%    \end{macrocode}
%    \end{macro}
%    \begin{macro}{\ZREF@abspos@marginpar@x@center}
%    \begin{macrocode}
\def\ZREF@abspos@marginpar@x@center#1{%
  \ZREF@abspos@marginpar@x@left{#1}%
  +\zref@extract{#1}{marginparwidth}/2%
}
%    \end{macrocode}
%    \end{macro}
%
%    \begin{macro}{\ZREF@abspos@marginpar@y@top}
%    \begin{macrocode}
\let\ZREF@abspos@marginpar@y@top\ZREF@abspos@body@y@top
%    \end{macrocode}
%    \end{macro}
%    \begin{macro}{\ZREF@abspos@marginpar@y@bottom}
%    \begin{macrocode}
\let\ZREF@abspos@marginpar@y@bottom\ZREF@abspos@body@y@bottom
%    \end{macrocode}
%    \end{macro}
%    \begin{macro}{\ZREF@abspos@marginpar@y@center}
%    \begin{macrocode}
\let\ZREF@abspos@marginpar@y@center\ZREF@abspos@body@y@center
%    \end{macrocode}
%    \end{macro}
%
% \subsubsection{Stock paper}
%
%    \begin{macro}{\ZREF@abspos@stock@x@left}
%    \begin{macrocode}
\let\ZREF@abspos@stock@x@left\ZREF@abspos@paper@x@left
%    \end{macrocode}
%    \end{macro}
%    \begin{macro}{\ZREF@abspos@stock@x@right}
%    \begin{macrocode}
\let\ZREF@abspos@stock@x@right\ZREF@abspos@paper@x@right
%    \end{macrocode}
%    \end{macro}
%    \begin{macro}{\ZREF@abspos@stock@x@center}
%    \begin{macrocode}
\let\ZREF@abspos@stock@x@center\ZREF@abspos@paper@x@center
%    \end{macrocode}
%    \end{macro}
%
%    \begin{macro}{\ZREF@abspos@stock@y@top}
%    \begin{macrocode}
\let\ZREF@abspos@stock@y@top\ZREF@abspos@paper@y@top
%    \end{macrocode}
%    \end{macro}
%    \begin{macro}{\ZREF@abspos@stock@y@bottom}
%    \begin{macrocode}
\let\ZREF@abspos@stock@y@bottom\ZREF@abspos@paper@y@bottom
%    \end{macrocode}
%    \end{macro}
%    \begin{macro}{\ZREF@abspos@stock@y@center}
%    \begin{macrocode}
\let\ZREF@abspos@stock@y@center\ZREF@abspos@paper@y@center
%    \end{macrocode}
%    \end{macro}
%
%    \begin{macrocode}
%</abspos>
%    \end{macrocode}
%
%
% \subsection{Module \xmodule{dotfill}}
%
%    \begin{macrocode}
%<*dotfill>
\NeedsTeXFormat{LaTeX2e}
\ProvidesPackage{zref-dotfill}%
  [2020-07-03 v2.32 Module dotfill for zref (HO)]%
\RequirePackage{zref-base}[2019/11/29]
\ifx\ZREF@base@ok Y%
\else
  \expandafter\endinput
\fi
%    \end{macrocode}
%
%    For measuring the width of \cs{zdotfill} we use the features
%    provided by module \xmodule{savepos}.
%    \begin{macrocode}
\RequirePackage{zref-savepos}[2019/11/29]
%    \end{macrocode}
%    For automatically generated label names we use the unique
%    counter of module \xmodule{base}.
%    \begin{macrocode}
\zref@require@unique
%    \end{macrocode}
%    Configuration is done by the key value interface of package
%    \xpackage{keyval}.
%    \begin{macrocode}
\RequirePackage{keyval}
%    \end{macrocode}
%    The definitions of the keys follow.
%    \begin{macrocode}
\define@key{ZREF@DF}{unit}{%
  \def\ZREF@df@unit{#1}%
}
\define@key{ZREF@DF}{min}{%
  \def\ZREF@df@min{#1}%
}
\define@key{ZREF@DF}{dot}{%
  \def\ZREF@df@dot{#1}%
}
%    \end{macrocode}
%    Defaults are set, see user interface.
%    \begin{macrocode}
\providecommand\ZREF@df@min{2}
\providecommand\ZREF@df@unit{.44em}
\providecommand\ZREF@df@dot{.}
%    \end{macrocode}
%    \begin{macro}{\zdotfillsetup}
%    Configuration of \cs{zdotfill} is done by \cs{zdotfillsetup}.
%    \begin{macrocode}
\newcommand*{\zdotfillsetup}{\kvsetkeys{ZREF@DF}}
%    \end{macrocode}
%    \end{macro}
%
%    \begin{macro}{\zdotfill}
%    \cs{zdotfill} sets labels at the left and the right to get
%    the horizontal position. \cs{zsavepos} is not used, because
%    we do not need the vertical position.
%    \begin{macrocode}
\ZREF@IfDefinable\zdotfill\def{%
  {%
    \leavevmode
    \global\advance\c@zref@unique\ltx@one
    \begingroup
      \def\ZREF@temp{zref@\number\c@zref@unique}%
      \pdfsavepos
      \zref@labelbyprops{\thezref@unique L}{posx}%
      \setlength{\dimen@}{\ZREF@df@unit}%
      \zref@ifrefundefined{\thezref@unique R}{%
        \ZREF@dotfill
      }{%
        \ifnum\numexpr\zposx{\thezref@unique R}%
                      -\zposx{\thezref@unique L}\relax
            <\dimexpr\ZREF@df@min\dimen@\relax
          \hfill
        \else
          \ZREF@dotfill
        \fi
      }%
      \pdfsavepos
      \zref@labelbyprops{\thezref@unique R}{posx}%
    \endgroup
    \kern\z@
  }%
}
%    \end{macrocode}
%    \end{macro}
%    \begin{macro}{\ZREF@dotfill}
%    Help macro that actually sets the dots.
%    \begin{macrocode}
\def\ZREF@dotfill{%
  \cleaders\hb@xt@\dimen@{\hss\ZREF@df@dot\hss}\hfill
}
%    \end{macrocode}
%    \end{macro}
%
%    \begin{macrocode}
%</dotfill>
%    \end{macrocode}
%
% \subsection{Module \xmodule{env}}
%
%    \begin{macrocode}
%<*env>
\NeedsTeXFormat{LaTeX2e}
\ProvidesPackage{zref-env}%
  [2020-07-03 v2.32 Module env for zref (HO)]%
\RequirePackage{zref-base}[2019/11/29]
\ifx\ZREF@base@ok Y%
\else
  \expandafter\endinput
\fi
%    \end{macrocode}
%
%    \begin{macrocode}
\zref@newprop{envname}[]{\@currenvir}
\zref@newprop{envline}[]{\zref@env@line}
%    \end{macrocode}
%
%    \begin{macro}{\zref@env@line}
%    Macro \cs{zref@env@line} extracts the line number
%    from \cs{@currenvline}.
%    \begin{macrocode}
\def\zref@env@line{%
  \ifx\@currenvline\ltx@empty
  \else
    \expandafter
    \ZREF@ENV@line\@currenvline\ltx@empty line \ltx@empty\@nil
  \fi
}
%    \end{macrocode}
%    \end{macro}
%    \begin{macro}{\ZREF@ENV@line}
%    \begin{macrocode}
\def\ZREF@ENV@line#1line #2\ltx@empty#3\@nil{#2}%
%    \end{macrocode}
%    \end{macro}
%
%    \begin{macrocode}
%</env>
%    \end{macrocode}
%
%% \section{Installation}
%
% \subsection{Download}
%
% \paragraph{Package.} This package is available on
% CTAN\footnote{\CTANpkg{zref}}:
% \begin{description}
% \item[\CTAN{macros/latex/contrib/zref/zref.dtx}] The source file.
% \item[\CTAN{macros/latex/contrib/zref/zref.pdf}] Documentation.
% \end{description}
%
%
% \paragraph{Bundle.} All the packages of the bundle `zref'
% are also available in a TDS compliant ZIP archive. There
% the packages are already unpacked and the documentation files
% are generated. The files and directories obey the TDS standard.
% \begin{description}
% \item[\CTANinstall{install/macros/latex/contrib/zref.tds.zip}]
% \end{description}
% \emph{TDS} refers to the standard ``A Directory Structure
% for \TeX\ Files'' (\CTANpkg{tds}). Directories
% with \xfile{texmf} in their name are usually organized this way.
%
% \subsection{Bundle installation}
%
% \paragraph{Unpacking.} Unpack the \xfile{zref.tds.zip} in the
% TDS tree (also known as \xfile{texmf} tree) of your choice.
% Example (linux):
% \begin{quote}
%   |unzip zref.tds.zip -d ~/texmf|
% \end{quote}
%
%
% \subsection{Package installation}
%
% \paragraph{Unpacking.} The \xfile{.dtx} file is a self-extracting
% \docstrip\ archive. The files are extracted by running the
% \xfile{.dtx} through \plainTeX:
% \begin{quote}
%   \verb|tex zref.dtx|
% \end{quote}
%
% \paragraph{TDS.} Now the different files must be moved into
% the different directories in your installation TDS tree
% (also known as \xfile{texmf} tree):
% \begin{quote}
% \def\t{^^A
% \begin{tabular}{@{}>{\ttfamily}l@{ $\rightarrow$ }>{\ttfamily}l@{}}
%   zref.sty & tex/latex/zref/zref.sty\\
%   zref-base.sty & tex/latex/zref/zref-base.sty\\
%   zref-abspage.sty & tex/latex/zref/zref-abspage.sty\\
%   zref-abspos.sty & tex/latex/zref/zref-abspos.sty\\
%   zref-counter.sty & tex/latex/zref/zref-counter.sty\\
%   zref-dotfill.sty & tex/latex/zref/zref-dotfill.sty\\
%   zref-env.sty & tex/latex/zref/zref-env.sty\\
%   zref-hyperref.sty & tex/latex/zref/zref-hyperref.sty\\
%   zref-lastpage.sty & tex/latex/zref/zref-lastpage.sty\\
%   zref-marks.sty & tex/latex/zref/zref-marks.sty\\
%   zref-nextpage.sty & tex/latex/zref/zref-nextpage.sty\\
%   zref-pageattr.sty & tex/latex/zref/zref-pageattr.sty\\
%   zref-pagelayout.sty & tex/latex/zref/zref-pagelayout.sty\\
%   zref-perpage.sty & tex/latex/zref/zref-perpage.sty\\
%   zref-runs.sty & tex/latex/zref/zref-runs.sty\\
%   zref-savepos.sty & tex/latex/zref/zref-savepos.sty\\
%   zref-thepage.sty & tex/latex/zref/zref-thepage.sty\\
%   zref-titleref.sty & tex/latex/zref/zref-titleref.sty\\
%   zref-totpages.sty & tex/latex/zref/zref-totpages.sty\\
%   zref-user.sty & tex/latex/zref/zref-user.sty\\
%   zref-xr.sty & tex/latex/zref/zref-xr.sty\\
%   zref.pdf & doc/latex/zref/zref.pdf\\
%   zref-example.tex & doc/latex/zref/zref-example.tex\\
%   zref-example-lastpage.tex & doc/latex/zref/zref-example-lastpage.tex\\
%   zref-example-nextpage.tex & doc/latex/zref/zref-example-nextpage.tex\\
%   zref.dtx & source/latex/zref/zref.dtx\\
% \end{tabular}^^A
% }^^A
% \sbox0{\t}^^A
% \ifdim\wd0>\linewidth
%   \begingroup
%     \advance\linewidth by\leftmargin
%     \advance\linewidth by\rightmargin
%   \edef\x{\endgroup
%     \def\noexpand\lw{\the\linewidth}^^A
%   }\x
%   \def\lwbox{^^A
%     \leavevmode
%     \hbox to \linewidth{^^A
%       \kern-\leftmargin\relax
%       \hss
%       \usebox0
%       \hss
%       \kern-\rightmargin\relax
%     }^^A
%   }^^A
%   \ifdim\wd0>\lw
%     \sbox0{\small\t}^^A
%     \ifdim\wd0>\linewidth
%       \ifdim\wd0>\lw
%         \sbox0{\footnotesize\t}^^A
%         \ifdim\wd0>\linewidth
%           \ifdim\wd0>\lw
%             \sbox0{\scriptsize\t}^^A
%             \ifdim\wd0>\linewidth
%               \ifdim\wd0>\lw
%                 \sbox0{\tiny\t}^^A
%                 \ifdim\wd0>\linewidth
%                   \lwbox
%                 \else
%                   \usebox0
%                 \fi
%               \else
%                 \lwbox
%               \fi
%             \else
%               \usebox0
%             \fi
%           \else
%             \lwbox
%           \fi
%         \else
%           \usebox0
%         \fi
%       \else
%         \lwbox
%       \fi
%     \else
%       \usebox0
%     \fi
%   \else
%     \lwbox
%   \fi
% \else
%   \usebox0
% \fi
% \end{quote}
% If you have a \xfile{docstrip.cfg} that configures and enables \docstrip's
% TDS installing feature, then some files can already be in the right
% place, see the documentation of \docstrip.
%
% \subsection{Refresh file name databases}
%
% If your \TeX~distribution
% (\TeX\,Live, \mikTeX, \dots) relies on file name databases, you must refresh
% these. For example, \TeX\,Live\ users run \verb|texhash| or
% \verb|mktexlsr|.
%
% \subsection{Some details for the interested}
%
% \paragraph{Unpacking with \LaTeX.}
% The \xfile{.dtx} chooses its action depending on the format:
% \begin{description}
% \item[\plainTeX:] Run \docstrip\ and extract the files.
% \item[\LaTeX:] Generate the documentation.
% \end{description}
% If you insist on using \LaTeX\ for \docstrip\ (really,
% \docstrip\ does not need \LaTeX), then inform the autodetect routine
% about your intention:
% \begin{quote}
%   \verb|latex \let\install=y% \iffalse meta-comment
%
% File: zref.dtx
% Version: 2020-07-03 v2.32
% Info: A new reference scheme for LaTeX
%
% Copyright (C)
%    2006-2012  Heiko Oberdiek
%    2016-2020  Oberdiek Package Support Group
%    https://github.com/ho-tex/zref/issues
%
% This work may be distributed and/or modified under the
% conditions of the LaTeX Project Public License, either
% version 1.3c of this license or (at your option) any later
% version. This version of this license is in
%    https://www.latex-project.org/lppl/lppl-1-3c.txt
% and the latest version of this license is in
%    https://www.latex-project.org/lppl.txt
% and version 1.3 or later is part of all distributions of
% LaTeX version 2005/12/01 or later.
%
% This work has the LPPL maintenance status "maintained".
%
% The Current Maintainers of this work are
% Heiko Oberdiek and the Oberdiek Package Support Group
% https://github.com/ho-tex/zref/issues
%
% This work consists of the main source file zref.dtx
% and the derived files
%    zref.sty, zref.pdf, zref.ins, zref.drv, zref-base.sty,
%    zref-abspage.sty, zref-abspos.sty, zref-counter.sty,
%    zref-dotfill.sty, zref-env.sty, zref-hyperref.sty,
%    zref-lastpage.sty, zref-marks.sty, zref-nextpage.sty,
%    zref-pageattr.sty, zref-pagelayout.sty, zref-perpage.sty,
%    zref-runs.sty, zref-savepos.sty, zref-thepage.sty,
%    zref-titleref.sty, zref-totpages.sty, zref-user.sty,
%    zref-xr.sty, zref-example.tex, zref-example-lastpage.tex,
%    zref-example-nextpage.tex, zref-test1.tex,
%    zref-test-base.tex, zref-test-runs.tex,
%    zref-test-titleref-memoir.tex.
%
% Distribution:
%    CTAN:macros/latex/contrib/zref/zref.dtx
%    CTAN:macros/latex/contrib/zref/zref.pdf
%
% Unpacking:
%    (a) If zref.ins is present:
%           tex zref.ins
%    (b) Without zref.ins:
%           tex zref.dtx
%    (c) If you insist on using LaTeX
%           latex \let\install=y% \iffalse meta-comment
%
% File: zref.dtx
% Version: 2020-07-03 v2.32
% Info: A new reference scheme for LaTeX
%
% Copyright (C)
%    2006-2012  Heiko Oberdiek
%    2016-2020  Oberdiek Package Support Group
%    https://github.com/ho-tex/zref/issues
%
% This work may be distributed and/or modified under the
% conditions of the LaTeX Project Public License, either
% version 1.3c of this license or (at your option) any later
% version. This version of this license is in
%    https://www.latex-project.org/lppl/lppl-1-3c.txt
% and the latest version of this license is in
%    https://www.latex-project.org/lppl.txt
% and version 1.3 or later is part of all distributions of
% LaTeX version 2005/12/01 or later.
%
% This work has the LPPL maintenance status "maintained".
%
% The Current Maintainers of this work are
% Heiko Oberdiek and the Oberdiek Package Support Group
% https://github.com/ho-tex/zref/issues
%
% This work consists of the main source file zref.dtx
% and the derived files
%    zref.sty, zref.pdf, zref.ins, zref.drv, zref-base.sty,
%    zref-abspage.sty, zref-abspos.sty, zref-counter.sty,
%    zref-dotfill.sty, zref-env.sty, zref-hyperref.sty,
%    zref-lastpage.sty, zref-marks.sty, zref-nextpage.sty,
%    zref-pageattr.sty, zref-pagelayout.sty, zref-perpage.sty,
%    zref-runs.sty, zref-savepos.sty, zref-thepage.sty,
%    zref-titleref.sty, zref-totpages.sty, zref-user.sty,
%    zref-xr.sty, zref-example.tex, zref-example-lastpage.tex,
%    zref-example-nextpage.tex, zref-test1.tex,
%    zref-test-base.tex, zref-test-runs.tex,
%    zref-test-titleref-memoir.tex.
%
% Distribution:
%    CTAN:macros/latex/contrib/zref/zref.dtx
%    CTAN:macros/latex/contrib/zref/zref.pdf
%
% Unpacking:
%    (a) If zref.ins is present:
%           tex zref.ins
%    (b) Without zref.ins:
%           tex zref.dtx
%    (c) If you insist on using LaTeX
%           latex \let\install=y\input{zref.dtx}
%        (quote the arguments according to the demands of your shell)
%
% Documentation:
%    (a) If zref.drv is present:
%           latex zref.drv
%    (b) Without zref.drv:
%           latex zref.dtx; ...
%    The class ltxdoc loads the configuration file ltxdoc.cfg
%    if available. Here you can specify further options, e.g.
%    use A4 as paper format:
%       \PassOptionsToClass{a4paper}{article}
%
%    Programm calls to get the documentation (example):
%       pdflatex zref.dtx
%       makeindex -s gind.ist zref.idx
%       pdflatex zref.dtx
%       makeindex -s gind.ist zref.idx
%       pdflatex zref.dtx
%
% Installation:
%    TDS:tex/latex/zref/zref.sty
%    TDS:tex/latex/zref/zref-base.sty
%    TDS:tex/latex/zref/zref-abspage.sty
%    TDS:tex/latex/zref/zref-abspos.sty
%    TDS:tex/latex/zref/zref-counter.sty
%    TDS:tex/latex/zref/zref-dotfill.sty
%    TDS:tex/latex/zref/zref-env.sty
%    TDS:tex/latex/zref/zref-hyperref.sty
%    TDS:tex/latex/zref/zref-lastpage.sty
%    TDS:tex/latex/zref/zref-marks.sty
%    TDS:tex/latex/zref/zref-nextpage.sty
%    TDS:tex/latex/zref/zref-pageattr.sty
%    TDS:tex/latex/zref/zref-pagelayout.sty
%    TDS:tex/latex/zref/zref-perpage.sty
%    TDS:tex/latex/zref/zref-runs.sty
%    TDS:tex/latex/zref/zref-savepos.sty
%    TDS:tex/latex/zref/zref-thepage.sty
%    TDS:tex/latex/zref/zref-titleref.sty
%    TDS:tex/latex/zref/zref-totpages.sty
%    TDS:tex/latex/zref/zref-user.sty
%    TDS:tex/latex/zref/zref-xr.sty
%    TDS:doc/latex/zref/zref.pdf
%    TDS:doc/latex/zref/zref-example.tex
%    TDS:doc/latex/zref/zref-example-lastpage.tex
%    TDS:doc/latex/zref/zref-example-nextpage.tex
%    TDS:source/latex/zref/zref.dtx
%
%<*ignore>
\begingroup
  \catcode123=1 %
  \catcode125=2 %
  \def\x{LaTeX2e}%
\expandafter\endgroup
\ifcase 0\ifx\install y1\fi\expandafter
         \ifx\csname processbatchFile\endcsname\relax\else1\fi
         \ifx\fmtname\x\else 1\fi\relax
\else\csname fi\endcsname
%</ignore>
%<*install>
\input docstrip.tex
\Msg{************************************************************************}
\Msg{* Installation}
\Msg{* Package: zref 2020-07-03 v2.32 A new reference scheme for LaTeX (HO)}
\Msg{************************************************************************}

\keepsilent
\askforoverwritefalse

\let\MetaPrefix\relax
\preamble

This is a generated file.

Project: zref
Version: 2020-07-03 v2.32

Copyright (C)
  2006-2012 Heiko Oberdiek
  2016-2020 Oberdiek Package Support Group

This work may be distributed and/or modified under the
conditions of the LaTeX Project Public License, either
version 1.3c of this license or (at your option) any later
version. This version of this license is in
   https://www.latex-project.org/lppl/lppl-1-3c.txt
and the latest version of this license is in
   https://www.latex-project.org/lppl.txt
and version 1.3 or later is part of all distributions of
LaTeX version 2005/12/01 or later.

This work has the LPPL maintenance status "maintained".

The Current Maintainers of this work are
Heiko Oberdiek and the Oberdiek Package Support Group
https://github.com/ho-tex/zref/issues


This work consists of the main source file zref.dtx
and the derived files
   zref.sty, zref.pdf, zref.ins, zref.drv, zref-base.sty,
   zref-abspage.sty, zref-abspos.sty, zref-counter.sty,
   zref-dotfill.sty, zref-env.sty, zref-hyperref.sty,
   zref-lastpage.sty, zref-marks.sty, zref-nextpage.sty,
   zref-pageattr.sty, zref-pagelayout.sty, zref-perpage.sty,
   zref-runs.sty, zref-savepos.sty, zref-thepage.sty,
   zref-titleref.sty, zref-totpages.sty, zref-user.sty,
   zref-xr.sty, zref-example.tex, zref-example-lastpage.tex,
   zref-example-nextpage.tex, zref-test1.tex,
   zref-test-base.tex, zref-test-runs.tex,
   zref-test-titleref-memoir.tex.

\endpreamble
\let\MetaPrefix\DoubleperCent

\generate{%
  \file{zref.ins}{\from{zref.dtx}{install}}%
  \file{zref.drv}{\from{zref.dtx}{driver}}%
  \usedir{tex/latex/zref}%
  \file{zref.sty}{\from{zref.dtx}{package}}%
  \file{zref-base.sty}{\from{zref.dtx}{base}}%
  \file{zref-abspage.sty}{\from{zref.dtx}{abspage}}%
  \file{zref-abspos.sty}{\from{zref.dtx}{abspos}}%
  \file{zref-counter.sty}{\from{zref.dtx}{counter}}%
  \file{zref-dotfill.sty}{\from{zref.dtx}{dotfill}}%
  \file{zref-env.sty}{\from{zref.dtx}{env}}%
  \file{zref-hyperref.sty}{\from{zref.dtx}{hyperref}}%
  \file{zref-lastpage.sty}{\from{zref.dtx}{lastpage}}%
  \file{zref-marks.sty}{\from{zref.dtx}{marks}}%
  \file{zref-nextpage.sty}{\from{zref.dtx}{nextpage}}%
  \file{zref-pageattr.sty}{\from{zref.dtx}{pageattr}}%
  \file{zref-pagelayout.sty}{\from{zref.dtx}{pagelayout}}%
  \file{zref-perpage.sty}{\from{zref.dtx}{perpage}}%
  \file{zref-runs.sty}{\from{zref.dtx}{runs}}%
  \file{zref-savepos.sty}{\from{zref.dtx}{savepos}}%
  \file{zref-thepage.sty}{\from{zref.dtx}{thepage}}%
  \file{zref-titleref.sty}{\from{zref.dtx}{titleref}}%
  \file{zref-totpages.sty}{\from{zref.dtx}{totpages}}%
  \file{zref-user.sty}{\from{zref.dtx}{user}}%
  \file{zref-xr.sty}{\from{zref.dtx}{xr}}%
  \usedir{doc/latex/zref}%
  \file{zref-example.tex}{\from{zref.dtx}{example}}%
  \file{zref-example-lastpage.tex}{\from{zref.dtx}{example-lastpage}}%
  \file{zref-example-nextpage.tex}{\from{zref.dtx}{example-nextpage}}%
%  \usedir{doc/latex/zref/test}%
%  \file{zref-test1.tex}{\from{zref.dtx}{test1}}%
%  \file{zref-test-base.tex}{\from{zref.dtx}{test-base}}%
%  \file{zref-test-runs.tex}{\from{zref.dtx}{test-runs}}%
%  \file{zref-test-titleref-memoir.tex}{\from{zref.dtx}{test-titleref-memoir}}%
}

\catcode32=13\relax% active space
\let =\space%
\Msg{************************************************************************}
\Msg{*}
\Msg{* To finish the installation you have to move the following}
\Msg{* files into a directory searched by TeX:}
\Msg{*}
\Msg{*     zref.sty, zref-base.sty, zref-abspage.sty, zref-abspos.sty,}
\Msg{*     zref-counter.sty, zref-dotfill.sty, zref-env.sty,}
\Msg{*     zref-hyperref.sty, zref-lastpage.sty, zref-marks.sty,}
\Msg{*     zref-nextpage.sty, zref-pageattr.sty, zref-pagelayout.sty,}
\Msg{*     zref-perpage.sty, zref-runs.sty, zref-savepos.sty,}
\Msg{*     zref-thepage.sty, zref-titleref.sty, zref-totpages.sty,}
\Msg{*     zref-user.sty, zref-xr.sty}
\Msg{*}
\Msg{* To produce the documentation run the file `zref.drv'}
\Msg{* through LaTeX.}
\Msg{*}
\Msg{* Happy TeXing!}
\Msg{*}
\Msg{************************************************************************}

\endbatchfile
%</install>
%<*ignore>
\fi
%</ignore>
%<*driver>
\NeedsTeXFormat{LaTeX2e}
\ProvidesFile{zref.drv}%
  [2020-07-03 v2.32 A new reference scheme for LaTeX (HO)]%
\documentclass{ltxdoc}
\usepackage{holtxdoc}[2011/11/22]
\setlength{\emergencystretch}{1em}
\begin{document}
  \DocInput{zref.dtx}%
\end{document}
%</driver>
% \fi
%
%
%
% \GetFileInfo{zref.drv}
%
% \title{The \xpackage{zref} package}
% \date{2020-07-03 v2.32}
% \author{Heiko Oberdiek\thanks
% {Please report any issues at \url{https://github.com/ho-tex/zref/issues}}}
%
% \maketitle
%
% \begin{abstract}
% Package \xpackage{zref} tries to get rid of the restriction
% in \hologo{LaTeX}'s reference system that only two properties are
% supported. The package implements an extensible referencing
% system, where properties are handled in a more flexible way.
% It offers an interface for macro programmers for the access
% to the system and some applications that uses the new
% reference scheme.
% \end{abstract}
%
% \tableofcontents
%
% \newcommand*{\metacs}[1]{\texttt{\itshape #1}}
%
% \newcommand{\markexp}{^^A
%   \textsuperscript{exp}^^A
%   \space
% }
% \newcommand{\markexptwo}{^^A
%   \textsuperscript{exp2}^^A
%   \space
% }
% \newcommand{\markrobust}{^^A
%   \textsuperscript{robust}^^A
%   \space
% }
% \newcommand{\marknowarn}{^^A
%   \textsuperscript{nowarn}^^A
%   \space
% }
% \newcommand{\markbabel}{^^A
%   \textsuperscript{babel}^^A
%   \space
% }
%
% \section{Introduction}
%
% Standard \hologo{LaTeX}'s reference system with \cs{label}, \cs{ref},
% and \cs{pageref} supports two properties, the apperance of
% the counter that is last incremented by \cs{refstepcounter} and
% the page with the \cs{label} command.
%
% Unhappily \hologo{LaTeX} does not provide an interface for adding
% another properties. Packages such as \xpackage{hyperref},
% \xpackage{nameref}, or \xpackage{titleref} are forced to
% use ugly hacks to extend the reference system.
% These ugly hacks are one of the causes for \xpackage{hyperref}'s
% difficulty regarding compatibility with other packages.
%
% \subsection{Standard \hologo{LaTeX} behaviour}
%
% References are created by the \cs{label} command:
% \begin{quote}
%   |\chapter{Second chapter}|\\
%   |\section{First section on page 7} % section 2.1|\\
%   |\label{myref}|
% \end{quote}
% Now \hologo{LaTeX} records the section number |2.1| and the page |7|
% in the reference. Internally the reference is a list with
% two entries:
% \begin{quote}
%   |\r@myref| $\rightarrow$ |{2.1}{7}|
% \end{quote}
% The length of the list if fixed in the \hologo{LaTeX} kernel,
% An interface for adding new properties is missing.
%
% There are several tries to add new properties:
% \begin{description}
% \item[\xpackage{hyperref}] uses a list of five
%   properties instead of the standard list with two entries.
%   This causes many compatibility problems with \hologo{LaTeX} and
%   other packages.
% \item[\xpackage{titleref}] stores its title data into the
%   first entry in the list. \hologo{LaTeX} is happy because it
%   does only see its list with two entries. The situation
%   becomes more difficult, if more properties are added
%   this way. Then the macros form a nested structure
%   inside the first reference argument for the label.
%   Expandable extractions will then become painful.
% \end{description}
%
% \subsection{Basic idea}
%
% Some time ago Morten H\o gholm sent me an experimental
% cross referencing mechanism as ``expl3'' code. His
% idea is:
% \begin{quote}
%   |\g_xref_mylabel_plist| $\rightarrow$\\
%   \mbox{}\qquad|\xref_dance_key{salsa}\xref_name_key{Morten}|\dots
% \end{quote}
% The entries have the following format:
% \begin{quote}
%   |\xref_|\meta{your key}|_key{|\meta{some text}|}|
% \end{quote}
% This approach is much more flexible:
% \begin{itemize}
%   \item New properties can easily be added, just use a new key.
%   \item The length of the list is not fixed. A reference
%         can use a subset of the keys.
%   \item The order of the entries does not matter.
% \end{itemize}
% Unhappily I am not familiar with the experimental code
% for \hologo{LaTeX3} that will need some time before its first release.
% Thus I have implemented it as \hologo{LaTeXe} package without disturbing
% the existing \hologo{LaTeX} reference system.
%
% \subsection{Interfaces}
%
% The package provides a generic \emph{interface for programmers}.
% Commands of this interface are prefixed by \cs{zref@}.
%
% Option \xoption{user} enabels the \emph{user interface}.
% Here the commands are prefixed by \cs{z} to avoid name
% clashes with existing macros.
%
% Then the packages provides some \emph{modules}. They
% are applications for the reference system and can also be
% considered as examples how to use the reference system.
%
% The modules can be loaded as packages. The package
% name is prefixed with |zref-|, for example:
%\begin{quote}
%\begin{verbatim}
%\RequirePackage{zref-abspage}
%\end{verbatim}
%\end{quote}
% This is the preferred way if the package is loaded from
% within other packages to avoid option clashes.
%
% As alternative package |zref| can be used and the modules
% are given as options:
%\begin{quote}
%\begin{verbatim}
%\usepackage[perpage,user]{zref}
%\end{verbatim}
%\end{quote}
%
% \section{Interface for programmers}
%
% The user interface is described in the next section \ref{sec:user}.
%
% \subsection{Entities}
%
% \paragraph{Reference.}
% Internally a reference is a list of key value pairs:
% \begin{quote}
%   |\Z@R@myref| $\rightarrow$
%   |\default{2.1}\page{7}|
% \end{quote}
% The generic format of a entry is:
% \begin{quote}
%   |\Z@R@|\meta{refname} $\rightarrow$
%   |\|\meta{propname}|{|\meta{value}|}|
% \end{quote}
% \meta{refname} is the name that denoted references
% (the name used in \cs{label} and \cs{ref}). \meta{propname}
% is the name of the property or key. The property key macro
% is never executed, it is used in parameter text matching only.
%
% \paragraph{Property.}
% Because the name of a property is used in a macro name
% that must survive the \xfile{.aux} file, the name is
% restricted to letters and `@'.
%
% \paragraph{Property list.}
% Often references are used for special purposes. Thus it
% saves memory if just the properties are used in this
% reference that are necessary for its purpose.
%
% Therefore this package uses the concept of \emph{property lists}.
% A property list is a set of properties. The set of properties
% that is used by the default \cs{label} command is the
% \emph{main property list}.
%
% \subsection{Property list}
%
% \markexp means that the implementation of the
% marked macro is expandable. \markexptwo goes a step further
% and marks the macro expandable in exact two expansion steps.
%
% \begin{declcs}{zref@newlist}
%   \M{listname}
% \end{declcs}
% Declares a new empty property list.
%
% \begin{declcs}{zref@addprop}
%   \M{listname} \M{propname}\\
%   \cs{zref@localaddprop} \M{listname} \M{propname}
% \end{declcs}
% Adds the property \meta{propname}
% to the property list \meta{listname}.
% The property and list must exist.
% The addition is global by \cs{zref@addprop} and limited to
% local scope by \cs{zref@localaddprop}.
% Between 2010/04/19 v2.13 and 2010/10/22 v2.19 a comma separated list
% of properties could be used as argument \meta{propname}.
% Since 2010/10/22 v2.19 the addition of several properties
% at once is supported by \cs{zref@addprops}.
%
% \begin{declcs}{zref@addprops}
%   \M{listname} \M{propname list}\\
%   \cs{zref@localaddprops} \M{listname} \M{propname list}
% \end{declcs}
% These macros add a comma separated list of properties
% \meta{propname list} to list \meta{listname}.
% \cs{zref@addprops} works globally and \cs{zref@localaddprops}
% locally. Since 2010/10/22 v2.19.
%
% \begin{declcs}{zref@listexists}
%   \M{listname} \M{then}
% \end{declcs}
% Executes \meta{then} if the property list \meta{listname} exists
% or raise an error otherwise.
%
% \begin{declcs}{zref@iflistundefined}\markexp \M{listname}
%   \M{then} \M{else}
% \end{declcs}
% Executes \meta{then} if the list exists or \meta{else}
% otherwise.
%
% \begin{declcs}{zref@iflistcontainsprop}
%   \M{listname} \M{propname} \M{then} \M{else}
% \end{declcs}
% Executes \meta{then} if the property \meta{propname} is part
% of property list \meta{listname} or otherwise it
% runs the \meta{else} part.
%
% \subsection{Property}
%
% \begin{declcs}{zref@newprop}
%   \* \M{propname} |[|\meta{default}|]| \M{value}
% \end{declcs}
% This command declares and configures a new property with
% name \meta{propname}.
%
% In case of unknown references
% or the property does not exist in the reference, the
% \meta{default} is used as value. If it is not specified
% here, a global default is used, see \cs{zref@setdefault}.
%
% The correct values of some properties are not known
% immediately but at page shipout time. Prominent example
% is the page number. These properties are declared with
% the star form of the command.
%
% \begin{declcs}{zref@setcurrent}
%   \M{propname} \M{value}
% \end{declcs}
% This sets the current value of the property \meta{propname}.
% It is a generalization of setting \hologo{LaTeX}'s \cs{currentlabel}.
%
% \begin{declcs}{zref@getcurrent}\markexptwo
%   \M{propname}
% \end{declcs}
% This returns the current value of the property \meta{propname}.
% The value may not be correct, especially if the property
% is bound to a page (start form of \cs{zref@newprop}) and the
% right value is only known at shipout time (e.g. property `page').
% In case of errors (e.g. unknown property) the empty string is
% returned.
%
% Since version 2010/04/22 v2.14 \cs{zref@getcurrent} supports
% \cs{zref@wrapper@unexpanded}.
%
% \begin{declcs}{zref@propexists}
%   \M{propname} \M{then}
% \end{declcs}
% Calls \meta{then} if the property \meta{propname} is available
% or generates an error message otherwise.
%
% \begin{declcs}{zref@ifpropundefined}\markexp
%   \M{propname} \M{then} \M{else}
% \end{declcs}
% Calls \meta{then} or \meta{else} depending on the existence
% of property \meta{propname}.
%
% \subsection{Reference generation}
%
% \begin{declcs}{zref@label}
%   \M{refname}
% \end{declcs}
% This works similar to \cs{label}. The reference \meta{refname} is
% created and put into the \xfile{.aux} file with the
% properties of the main property list.
%
% \begin{declcs}{zref@labelbylist}
%   \M{refname} \M{listname}
% \end{declcs}
% Same as \cs{zref@label} except that the properties are
% taken from the specified property list \meta{listname}.
%
% \begin{declcs}{zref@labelbyprops}
%   \M{refname} |{|\meta{propnameA}|,|\meta{propnameB}|,|\dots|}|
% \end{declcs}
% Same as \cs{zref@label} except that these properties
% are used that are given as comma separated list in the
% second argument.
%
% \begin{declcs}{zref@newlabel}
%   \M{refname} |{|\dots|}|
% \end{declcs}
% This is the macro that is used in the \xfile{.aux} file.
% It is basically the same as \cs{newlabel} apart from
% the format of the data in the second argument.
%
% \subsection{Data extraction}
%
% \begin{declcs}{zref@extractdefault}\markexptwo
%   \M{refname} \M{propname} \M{default}
% \end{declcs}
% This is the basic command that refernces the value of a
% property \meta{propname} for the reference \meta{refname}.
% In case of errors such as undefined reference the \meta{default}
% is used instead.
%
% \begin{declcs}{zref@extract}\markexptwo
%   \M{refname} \M{propname}
% \end{declcs}
% The command is an abbreviation for \cs{zref@extractdefault}.
% As default the default of the property is taken, otherwise
% the global default.
%
% Example for page references:
% \begin{quote}
% \begin{tabular}{@{}ll}
% \hologo{LaTeX}:& |\pageref{foobar}|\\
% \xpackage{zref}:& |\zref@extract{foobar}{page}|
% \end{tabular}
% \end{quote}
% Both \cs{zref@extract} and \cs{zref@extractdefault} are
% expandable. That means, these macros can directly
% be used in expandable calculations, see the example file.
% On the other side, babel's shorthands are not supported,
% there are no warnings in case of undefined references.
%
% If an user interface doesn't need expandable macros
% then it can use \cs{zref@refused} and \cs{zref@wrapper@babel}
% for its user macros.
%
% \begin{declcs}{zref@refused}
%   \M{refname}
% \end{declcs}
% This command is not expandable. It causes the warnings if
% the reference \meta{refname} is not defined. Use the
% \cs{zref@extract} commands inside expandable contexts and
% mark their use outside by \cs{zref@refused}, see the example
% file.
%
% \begin{declcs}{zref@def@extract}%
%     \M{cmd} \M{refname} \M{propname}\\
%   \cs{zref@def@extractdefault}
%     \M{cmd} \M{refname} \M{propname} \M{default}
% \end{declcs}
% Both macros extract the property \meta{propname} from the reference
% \meta{refname} the same way as macros \cs{zref@extract} and
% \cs{zref@extractdefault}. The result is stored in macro \meta{cmd}.
% Also \cs{zref@refused} is called to notify \hologo{LaTeX} that
% the reference \meta{refname} is used. Added in 2011/10/04 v2.22.
%
% \begin{declcs}{zref@ifrefundefined}\markexp
%   \M{refname} \M{then} \M{else}
% \end{declcs}
% Macro \cs{zref@ifrefundefined} calls arguments \meta{then} or \meta{else}
% dependent on the existence of the reference \meta{refname}.
%
% \begin{declcs}{zifrefundefined}
%   \M{refname} \M{then} \M{else}
% \end{declcs}
% Macro \cs{zifrefundefined} calls \cs{ref@refused} before
% executing \cs{zref@ifrefundefined}. Babel shorthands are
% supported in \meta{refname}.
%
% \begin{declcs}{zref@ifrefcontainsprop}\markexp
%   \M{refname} \M{propname} \M{then} \M{else}
% \end{declcs}
% Test whether a reference provides a property.
%
% \subsection{Setup}
%
% \begin{declcs}{zref@default}
% \end{declcs}
% Holds the global default for unknown values.
%
% \begin{declcs}{zref@setdefault}
%   |{|\meta{value}|}|
% \end{declcs}
% Sets the global default for unknown values.
% The global default is used, if a property does not specify
% an own default and the value for a property
% cannot be extracted. This can happen if the reference is
% unknown or the reference does not have the property.
%
% \begin{declcs}{zref@setmainlist}
%   |{|\meta{value}|}|
% \end{declcs}
% Sets the name of the main property list. The package
% sets and uses |main|.
%
% \subsection{Declared properties}
%
% \begin{quote}
% \renewcommand*{\arraystretch}{1.05}
% \setlength{\tabcolsep}{2\tabcolsep}
% \def\VV{\vphantom{\large \"Ag}}
% \def\EMPTY{{\itshape\textless empty\textgreater}}
% \def\pl#1{&#1&thepage&\cs{number}\cs{#1}}
% \begin{tabular}{@{}>{\sffamily}l|>{\ttfamily}l|>{\ttfamily}l|>{\ttfamily}l@{}}
%   \rmfamily Module&
%   \rmfamily Property&
%   \rmfamily Property list&
%   \rmfamily Default\\
%   \hline
%   (base)
%     & default & main & \EMPTY\\
%     & page & main & \EMPTY\\
%   \hline
%   abspage
%     & abspage & main & 0\\
%   \hline
%   counter
%     & counter & main & \EMPTY\\
%   \hline
%   hyperref
%     & anchor & main & \EMPTY\\
%     & url    & & \EMPTY\\
%   \hline
%   pageattr
%     & pdfpageattr & thepage & \dots\\
%     & pdfpagesattr & LastPage & \dots\\
%   \hline
%   pagelayout\footnotemark
%     \pl{mag}\\
%     \pl{paperwidth}\\
%     \pl{paperheight}\\
%     \pl{stockwidth}\\
%     \pl{stockheight}\\
%     \pl{pdfpageheight}\\
%     \pl{pdfpagewidth}\\
%     \pl{pdfhorigin}\\
%     \pl{pdfvorigin}\\
%     \pl{hoffset}\\
%     \pl{voffset}\\
%     \pl{topmargin}\\
%     \pl{oddsidemargin}\\
%     \pl{evensidemargin}\\
%     \pl{textwidth}\\
%     \pl{textheight}\\
%     \pl{headheight}\\
%     \pl{headsep}\\
%     \pl{footskip}\\
%     \pl{marginparwidth}\\
%     \pl{marginparsep}\\
%     \pl{columnwidth}\\
%     \pl{columnsep}\\
%   \hline
%   perpage
%     & pagevalue & perpage & 0\\
%     & page & perpage & \EMPTY\\
%     & abspage & perpage & 0\\
%   \hline
%   savepos
%     & posx & savepos & 0\\
%     & posy & savepos & 0\\
%   \hline
%   titleref
%     & title & main & \EMPTY\\
%   \hline
%   xr
%     & anchor   & & \EMPTY\\
%     & externaldocument & & \EMPTY\\
%     & theotype & & \EMPTY\\
%     & title    & & \EMPTY\\
%     & url      & & \EMPTY\\
% \end{tabular}
% \footnotetext{Module \xmodule{pagelayout} only defines properties
% if the parameter exists.}%
% \end{quote}
%
% \subsection{Wrapper for advanced situations}
%
% \begin{declcs}{zref@wrapper@babel}
%   |{|\dots|}| \M{name}
% \end{declcs}
% This macro helps to add shorthand support. The second
% argument is protected, then the code of the first argument
% is called with the protected name appended. Examples
% are in the sources.
%
% \begin{declcs}{zref@wrapper@immediate}
%   |{|\dots|}|
% \end{declcs}
% There are situations where a label must be written instantly
% to the \xfile{.aux} file, for example after the last page.
% If the \cs{zlabel} or \cs{label} command is put inside this wrapper,
% immediate writing is enabled. See the implementation for
% module \xmodule{lastpage} for an example of its use.
%
% \begin{declcs}{zref@wrapper@unexpanded}
%   |{|\dots|}|
% \end{declcs}
% Assuming someone wants to extract a value for property |bar|
% and store the result in a macro |\foo| without traces of
% the expanding macros and without expanding the value.
% This (theoretical?) problem can be solved by this wrapper:
% \begin{quote}
%\begin{verbatim}
%\zref@wrapper@unexpanded{%
%  \edef\foo{%
%    \zref@extract{someref}{bar}%
%  }%
%}
%\end{verbatim}
% \end{quote}
% The \cs{edef} forces the expansion of \cs{zref@extract},
% but the extraction of the value is prevented by
% the wrapper that uses \hologo{eTeX}'\ \cs{unexpanded} for this
% purpose. Supported macros are \cs{zref@extract}, \cs{zref@extractdefault}
% and since version 2010/04/22 v2.14 macro \cs{zref@getcurrent}.
%
% \subsection{Counter for unique names}
%
% Some modules (\xmodule{titleref} and \xmodule{dotfillmin})
% need unique names for automatically generated label names.
%
% \begin{declcs}{zref@require@unique}
% \end{declcs}
% This command creates the unique counter |zref@unique| if the
% counter does not already exist.
%
% \begin{declcs}{thezref@unique}
% \end{declcs}
% This command is used to generate unique label names.
%
% \section{User interface}\label{sec:user}
%
% \subsection{Module \xmodule{user}}
%
% The user interface for this package and its modules
% is enabled by \xpackage{zref}'s package option \xmodule{user}
% or package \xpackage{zref-user}.
% The names of user commands are prefixed by |z| in order
% to avoid name clashes with existing macros of the same
% functionality. Thus the package does not disturb the
% traditional reference scheme, both can be used together.
%
% The syntax descriptions contain the following markers
% that are intended as hints for programmers:
% \begin{quote}
% \begin{tabular}{@{}ll@{}}
%   \markbabel\unskip  & Babel shorthands are allowed.\\
%   \markrobust\unskip & Robust macro.\\
%   \markexp\unskip    & Expandable version:\\
%                      & \textbullet\ robust,
%                        unless the extracted values are fragile,\\
%                      & \textbullet\ no babel shorthand suport.\\
%   \markexptwo\unskip & Expandable like \markexp and:\\
%                      & \textbullet\ expandable in exact two steps.\\
% \end{tabular}
% \end{quote}
%
% The basic user interface of the package without modules
% are commands that mimic the standard \hologo{LaTeX} behaviour
% of \cs{label}, \cs{ref}, and \cs{pageref}:
%
% \begin{declcs}{zlabel}
%   \M{refname}\markbabel
% \end{declcs}
% Similar to \cs{label}. It generates a label with name
% \meta{refname} in the new reference scheme.
%
% \begin{declcs}{zref}
%   |[|\meta{propname}|]| \M{refname}\markbabel
% \end{declcs}
% Without optional argument similar to \cs{ref}, it returns
% the default reference property. This property is named
% |default|:
% \[
%   |\zref{|x|}| \equiv |\zref[default]{|x|}|
% \]
%
% \begin{declcs}{zpageref}
%   \M{refname}\markbabel
% \end{declcs}
% Convenience macro, similar to \cs{pageref}.
% \[
%   |\zpageref{|x|}| \equiv |\zref[page]{|x|}|
% \]
%
% \begin{declcs}{zrefused}
%   \M{refname}\markbabel
% \end{declcs}
% Some of the user commands in the modules are expandable.
% The use of such commands do not cause any undefined
% reference warnings, because inside of expandable
% contexts this is not possible. However, if there is
% a place outside of expandable contexts, \cs{refused}
% is strongly recommended. The reference \meta{refname}
% is marked as used, undefined ones will generate
% warnings.
%
% \subsection{Module \xmodule{abspage}}
%
% With the help of package \xpackage{atbegshi}
% a new counter |abspage| with absolute page numbers is provided.
% For technical and historical reasons the counter itself is zero based:
% if you use it directly on the first page, e.g
% with |\arabic{abspage}| you will get 0 as value. When using |\zref|
% the first page will be page 1 as expected.
% Also a new property |abspage| is defined and added to the
% main property list. Thus you can reference the absolute
% page number:
% \begin{quote}
%   |Section \zref{foo} is on page \zpageref{foo}.|\\
%   |This is page \zref[abspage]{foo}|\\
%   |of \zref[abspage]{LastPage}.|
% \end{quote}
% The example also makes use of module \xmodule{lastpage}.
%
% \subsection{Module \xmodule{lastpage}}
%
% Provides the functionality of package \xpackage{lastpage}
% \cite{lastpage} in the new reference scheme.
% The label |LastPage| is put at the end of the document.
% You can refer the last page number with:
% \begin{quote}
%   |\zref@extract{LastPage}{page}| (+ |\zref@refused{LastPage}|)\\
% \end{quote}
% or
% \begin{quote}
%   |\zpageref{LastPage}| (module \xmodule{user})
% \end{quote}
%
% Since version 2008/10/01 v2.3 the module defines the list |LastPage|.
% In addition to the properties of the main list label |LastPage| also
% stores the properties of this list |LastPage|. The default of this
% list is empty. The list can be used by the user to add additional
% properties for label |LastPage|.
%
% \subsubsection{Tests for last page}
%
% Since version 2010/03/26 v2.8 the macros \cs{zref@iflastpage}
% and \cs{ziflastpage} were added. They test the reference,
% whether it is a reference of the last page.
%
% \begin{declcs}{zref@iflastpage}\markexp \M{refname} \M{then} \M{else}
% \end{declcs}
% Macro \cs{zref@iflastpage} compares the references \meta{refname}
% with \meta{LastPage}. Basis of the comparison is the value of
% property |abspage|, because the values are different for different
% pages. This is not ensured by property |page|. Therefore module
% \xmodule{abspage} is loaded by module \xmodule{lastpage}.
% If both values of property |abspage| are present and match,
% then \meta{then} is executed, otherwise code \meta{else} is called.
% If one or both references are undefined or lack the property |abspage|,
% then \meta{else} is executed.
%
% Macro \cs{zref@iflastpage} is expandable, therefore \cs{zref@refused}
% should be called on \meta{refname} and \meta{LastPage}.
%
% \begin{declcs}{ziflastpage} \M{refname} \M{then} \M{else}
% \end{declcs}
% Macro \cs{ziflastpage} has the same function as \cs{zref@iflastpage},
% but adds support for babel shorthands in \meta{refname} and
% calls \cs{zref@refused}. However macro \cs{ziflastpage} is not
% expandable.
%
% \subsubsection{Example}
%
%    \begin{macrocode}
%<*example-lastpage>
%<<END_EXAMPLE
\NeedsTeXFormat{LaTeX2e}
\documentclass{report}

\newcounter{foo}
\renewcommand*{\thefoo}{\Alph{foo}}

\usepackage{zref-lastpage,zref-user}[2019/11/29]

\makeatletter
\zref@newprop{thefoo}{\thefoo}
\zref@newprop{valuefoo}{\the\value{foo}}
\zref@newprop{chapter}{\thechapter}
\zref@addprops{LastPage}{thefoo,valuefoo,chapter}
\makeatother

\newcommand*{\foo}{%
  \stepcounter{foo}%
  [Current foo: \thefoo]%
}

\begin{document}
  \chapter{First chapter}
  Last page is \zref{LastPage}.\\
  Last chapter is \zref[chapter]{LastPage}.\\
  Last foo is \zref[thefoo]{LastPage}.\\
  Last value of foo is \zref[valuefoo]{LastPage}.\\
  \foo
  \chapter{Second chapter}
  \foo\foo\foo
  \chapter{Last chapter}
  \foo
\end{document}
%END_EXAMPLE
%</example-lastpage>
%    \end{macrocode}
%
% \subsection{Module \xmodule{thepage}}
%
% This module \xmodule{thepage} loads module \xmodule{abspage},
% constructs a reference name using the absolute page number
% and remembers property |page|. Other properties can be added
% by adding them to the property list |thepage|.
%
% \begin{declcs}{zthepage} \M{absolute page number}\\
% \end{declcs}
% Macro \cs{zthepage} is basically a \cs{zpageref}. The reference
% name is yield by the \meta{absolute page number}. If the reference
% is not defined, then the default for property |page| is used.
%
% \begin{declcs}{zref@thepage@name}\markexp \M{absolute page number}
% \end{declcs}
% Macro \cs{zref@thepage@name} returns the internal reference
% name that is constructed using the \meta{absolute page number}.
% The internal reference name should not be used directly, because
% it might change in future versions.
%
% \begin{declcs}{zref@thepage}\markexp \M{absolute page number}\\
%   \cs{zref@thepage@refused} \M{absolute page number}
% \end{declcs}
% Macro \cs{zref@thepage} returns the page number (\cs{thepage})
% of \meta{absolute page number}. Because this macro is expandable,
% \cs{zref@thepage@refused} is used outside an expandable context
% to mark the reference as used.
%
%
% \subsection{Module \xmodule{nextpage}}
%
% \begin{declcs}{znextpage}
% \end{declcs}
% Macro \cs{znextpage} prints \cs{thepage} of the following page.
% It gets the current absolute page number by using a label.
% There are three cases for the next page:
% \begin{enumerate}
% \item The next page is not known yet because of undefined references.
%   Then \cs{zunknownnextpagename} is used instead. The default for
%   this macro is the default of property |page|.
% \item This page is the last page. Then \cs{znonextpagename}
%   is used. Its default is empty.
% \item The next page is known, then \cs{thepage} of the next
%   page is used (the value of property |page| of the next page).
% \end{enumerate}
%
% \subsubsection{Configuration}
%
% The behaviour can be configured by the following macros.
%
% \begin{declcs}{zunknownnextpagename}\\
%   \cs{znonextpagename}
% \end{declcs}
% If the next page is not known or available, then
% \cs{znextpage} uses these name macros as default.
% \cs{zunknownnextpagename} is used in case of undefined
% references. Default is the value of property |page| of the
% next page (\cs{thepage}). Module \xmodule{thepage} is used.
%
% Macro \cs{znonextpagename} is used, if the next page does
% not exists. That means that the current page is last page.
% The default is empty.
%
% \begin{declcs}{znextpagesetup} \M{unknown} \M{no next}
%   \M{next}
% \end{declcs}
% Acording to the case (see \cs{znextpage}) macro \cs{znextpage}
% calls an internal macro with an argument. The argument is
% either \cs{thepage} of the next page or one of \cs{zunknownnextpagename}
% or \cs{znonextpagename}. These internal macro can be changed by
% \cs{znextpagesetup}. It expects the definition texts for these
% three cases of a macro with one argument.
% The default is
% \begin{quote}
%   |\znextpagesetup{#1}{#1}{#1}|
% \end{quote}
%
% \subsubsection{Example}
%
%    \begin{macrocode}
%<*example-nextpage>
%<<END_EXAMPLE
\documentclass{book}

\usepackage{zref-nextpage}[2019/11/29]
\znextpagesetup
  {\thepage}% next page is unknown
  {\thepage\ (#1)}% this page is last page
  {\thepage\ $\rightarrow$ #1}% next page is known
\renewcommand*{\znonextpagename}{last page}

\usepackage{fancyhdr}
\pagestyle{fancy}
\fancyhf{}
\fancyhead[LE,RO]{\znextpage}
\fancypagestyle{plain}{%
  \fancyhf{}%
  \fancyhead[LE,RO]{\znextpage}%
}

\begin{document}
\frontmatter
  \tableofcontents
\mainmatter
  \chapter{Hello World}
  \clearpage
  \section{Last section}
\end{document}
%END_EXAMPLE
%</example-nextpage>
%    \end{macrocode}
%
%
% \subsection{Module \xmodule{totpages}}
%
% For the total number of pages of a document you need to
% know the absolute page number of the last page.
% Both modules \xmodule{abspage} and \xmodule{lastpage}
% are necessary and automatically enabled.
%
% \begin{declcs}{ztotpages}\markexp
% \end{declcs}
% Prints the total number of pages or |0| if this
% number is not yet known. It expands to an explicit number
% and can also used even in expandable calculations (\cs{numexpr})
% or counter assignments.
%
% \subsection{Module \xmodule{pagelayout}}
%
% The module defines additional properties for each parameter
% of the page layout that is effective during page shipout.
% The value of length parameters is given in sp without the
% unit as plain number.
%
% Some parameters are specific for a class (e.g.\@ \xoption{stockwidth}
% and \xoption{stockheight} for class \xclass{memoir}) or the
% \hologo{TeX} engine like \hologo{pdfTeX}. If the parameter is not
% available, then the property will not be defined. The default value
% of the property is the current setting of the parameter.
%
% The module \xmodule{thepage} is loaded that generates a label
% for each page. The properties of module \xmodule{pagelayout}
% are added to the property list |thepage| of module \xmodule{thepage}.
%
% List of properties:
% \begin{quote}
%   \def\x#1{\cs{#1}&#1}%
%   \begin{tabular}{l>{\sffamily}ll}
%     \textrm{parameter} & \textrm{property} & remarks\\
%     \hline
%     \x{mag}\\
%     \x{paperwidth}\\
%     \x{paperheight}\\
%     \x{stockwidth}& class \xclass{memoir}\\
%     \x{stockheight}& class \xclass{memoir}\\
%     \x{pdfpagewidth}& \hologo{pdfTeX}, \hologo{LuaTeX}\\
%     \x{pdfpageheight}& \hologo{pdfTeX}, \hologo{LuaTeX}\\
%     \x{pdfhorigin}& \hologo{pdfTeX}, \hologo{LuaTeX}\\
%     \x{pdfvorigin}& \hologo{pdfTeX}, \hologo{LuaTeX}\\
%     \x{hoffset}\\
%     \x{voffset}\\
%     \x{topmargin}\\
%     \x{oddsidemargin}\\
%     \x{evensidemargin}\\
%     \x{textwidth}\\
%     \x{textheight}\\
%     \x{headheight}\\
%     \x{headsep}\\
%     \x{footskip}\\
%     \x{marginparwidth}\\
%     \x{marginparsep}\\
%     \x{columnwidth}\\
%     \x{columnsep}\\
%   \end{tabular}
% \end{quote}
%
% \begin{declcs}{zlistpagelayout}
% \end{declcs}
% At the end of document the page layout parameter for each page are
% printed into the \xext{log} file if macro \cs{zlistpagelayout}
% is called before |\end{document}| (preamble is a good place).
%
% \subsection{Module \xmodule{marks}}
%
% ToDo.
%
% \subsection{Module \xmodule{runs}}
%
% Module \xmodule{runs} counts the \hologo{LaTeX} runs since last
% \xext{aux} file creation and prints the number in the
% \xext{log} file.
%
% \begin{declcs}{zruns}\markexp
% \end{declcs}
% Prints the the total number of \hologo{LaTeX} runs including
% the current one. It expands to an explicit number.
% Before |begin{document}| the value is zero meaning
% the \xext{aux} file is not read yet. If a previous
% \xext{aux} file exists, the value found there increased
% by one is the new number. Otherwise \cs{zruns} is set to one.
% \hologo{LaTeX}~runs where the \xext{aux} files are not rewritten
% are not counted (see \cs{nofiles}).
%
% \subsection{Module \xmodule{perpage}}
%
% With \cs{@addtoreset} or \cs{numberwithin} a counter can be
% reset if another counter is incremented. This do not work
% well if the other counter is the page counter. The page counter is
% incremented in the output routine that is often called asynchronous
% somewhere on the next page. A reference mechanism costs at least
% two \hologo{LaTeX} runs, but ensures correct page counter values.
%
% \begin{declcs}{zmakeperpage}
%   |[|\meta{reset}|]| \M{counter}
% \end{declcs}
% At the of a new page counter \meta{counter} starts
% counting with value \meta{reset} (default is \texttt{1}).
% The macro has the same
% syntax and semantics as \cs{MakePerPage} of package
% \xpackage{perpage} \cite{perpage}.
% Also \xoption{perpage} of package \xpackage{footmisc} \cite{footmisc}
% can easily be simulated by
% \begin{quote}
% |\zmakeperpage{footnote}| \itshape |% \usepackage[perpage]{footmisc}|
%\end{quote}
% If footnote symbols are used, some people dislike the
% first symbol \ensuremath{\dagger}. It can easily be skipped:
% \begin{quote}
% |\zmakeperpage[2]{footnote}|
% \end{quote}
%
% \begin{declcs}{thezpage}\\
%   counter |zpage|
% \end{declcs}
% If the formatted counter value of the counter that is reset at
% a new page contains the page value, then you can use \cs{thezpage},
% the page number of the current page. Or counter |zpage| can be
% used, if the page number should be formatted differently from
% the current page number. Example:
%\begin{quote}
%\begin{verbatim}
%\newcounter{foobar}
%\zmakeperpage{foobar}
%\renewcommand*{\thefoobar}{\thezpage-\arabic{foobar}}
% % or
%\renewcommand*{\thefoobar}{\roman{zpage}-\arabic{foobar}}
%\end{verbatim}
%\end{quote}
%
% \begin{declcs}{zunmakeperpage}
%   \M{counter}
% \end{declcs}
% The reset mechanism for this counter is deactivated.
%
%
% \subsection{Module \xmodule{counter}}
%
% This option just add the property |counter| to the main
% property list. The property stores the counter name,
% that was responsible for the reference. This is the
% property \xpackage{hyperref}'s \cs{autoref} feature
% uses. Thus this property |counter| may be useful
% for a reimplementation of the autoref feature,
% see the section \ref{sec:todo} with the todo list.
%
% \subsection{Module \xmodule{titleref}}
%
% This option makes section and caption titles available
% to the reference system similar to packages \xpackage{titleref}
% or \xpackage{nameref}.
%
% \begin{declcs}{ztitleref}
%   \M{refname}\markbabel
% \end{declcs}
% Print the section or caption title of reference \meta{refname},
% similar to \cs{nameref} or \cs{titleref}.
%
% \begin{declcs}{ztitlerefsetup}
%   |{|\textit{key$_1$=value$_1$, key$_2$=value$_2$, \dots}|}|
% \end{declcs}
% This command allows to configure the behaviour of module
% \xmodule{titleref}. The following keys are available:
% \begin{description}
% \let\bfseries\relax
% \item[\texttt{title=}\meta{value}]\mbox{}\\*
%   Sets the current title.
% \item[\texttt{stripperiod=true\string|false}]\mbox{}\\*
%   Follow package \xpackage{nameref} that
%   removes a last period. Default: |true|.
% \item[\texttt{expand=true\string|false}]\mbox{}\\*
%   Package \cs{titleref} expands the title first. This way
%   garbage and dangerous commands can be removed, e.g.
%   \cs{label}, \cs{index}\dots. See implementation section
%   for more details. Default is |false|.
% \item[\texttt{cleanup=\{}\textrm{\dots}\texttt{\}}]\mbox{}\\*
%   Hook to add own cleanup code, if method |expand| is used.
%   See implementation section for more details.
% \end{description}
%
% \subsection{Module \xmodule{savepos}}
%
% This option supports a feature that \pdfTeX\ provides
% (and \hologo{XeTeX}).
% \pdfTeX\ is able to tell the current position on the
% page. The page position is not instantly known. First the
% page must be constructed by \hologo{TeX}'s asynchronous output
% routine. Thus the time where the position is known is
% the page shipout time. Thus a reference system where the
% information is recorded in the first run and made available
% for use in the second run comes in handy.
%
% \begin{declcs}{zsavepos}
%   \M{refname}
% \end{declcs}
% It generates a reference with name \meta{refname}.
% The reference stores the location where \cs{zsavepos}
% is executed in properties |posx| and |posy|.
%
% \begin{declcs}{zsaveposx}
%   \M{refname}\\
%   \cs{zsaveposy}
%   \M{refname}
% \end{declcs}
% Same as \cs{zsavepos} except that only the |x| or |y| component
% of the position is stored. Since 2011/12/05 v2.23.
%
% \begin{declcs}{zposx}\markexp
%   \M{refname}\\
%   \SpecialUsageIndex{\zposy}\cs{zposy}\markexp \M{refname}
% \end{declcs}
% Get the position as number. Unit is sp. Horizontal
% positions by \cs{zposx} increase from left to right.
% Vertical positions by \cs{zposy} from bottom to top.
%
% Do not rely on absolute page numbers. Because of problems
% with the origin the numbers may differ in DVI or PDF mode
% of \pdfTeX. Therefore work with relative values by comparisons.
%
% Both \cs{zposx} and \cs{zposy} are expandable and can be
% used inside calculations (\cs{setcounter}, \cs{addtocounter},
% package \xpackage{calc}, \cs{numexpr}). However this property
% prevents from notifying \hologo{LaTeX} that the reference is actually
% used (the notifying is not expandable). Therefore you should
% mark the reference as used by \cs{zrefused}.
%
% This module uses \pdfTeX's \cs{pdfsavepos}, \cs{pdflastxpos},
% and \cs{pdflastypos}. They are available in PDF mode and
% since version 1.40.0 also in DVI mode.
%
% \begin{declcs}{zref@savepos}
% \end{declcs}
% Macro \cs{zref@savepos} performs the first part of \cs{zsavepos}
% by calling \cs{pdfsavepos} (if \xext{aux} files are writable).
%
% Thus \cs{zsavepos} is basically \cs{zref@savepos} followed
% by |\zref@labelbylist{|\texttt{\meta{refname}}|}{savepos}|.
% If \cs{TeXXeTstate} is detected and enabled, \cs{savepos}
% also adds \cs{zref@savepos} at the end to support \cs{beginR}
% where the whatits are processed in reverse order.
% The property list |savepos| contains the properties |posx|
% and |posy|.
%
% \subsection{Module \xmodule{abspos}}
% Module \xmodule{abspos} allows to get various values of the page layout. There is no user
% command, only a number of internal commands. For example:
% \begin{quote}
%   |\zref@absposx{|\meta{label}|}{|\meta{value}|}{|\meta{position}|}|\\
%   |\zref@absposy{|\meta{label}|}{|\meta{value}|}{|\meta{position}|}|
% \end{quote}
%
% The return value is like in the module \xmodule{savepos} a number representing a length in sp.
% The length are measured from the bottom left of the page.
%
% \meta{label} is a label set with \cs{zlabel} or \cs{zsavepos} that allows to retrieve the
% absolute page number.
%
% \meta{position} is for the x-command one of |left|, |right| or |center|. For the y-command
%  it is one of |top|, |bottom|, |center|.
%
% The possible content of \meta{value} can be seen in the following table. Be aware that the code
% makes some assumptions which are perhaps not always true -- for example that the left of
% the head is identical to the left of the body.
%
% \medskip
% \begin{tabular}{lll}
%  \textbf{value} & \textbf{axis} & \textbf{comments}\\\hline
%   media         & x             &   left=0, right=\cs{pdfpagewidth} \\
%   paper         & x             &   left=0, right=\cs{paperwidth}\\
%   stock         & x             &   derived from paper\\
%   media         & y             & bottom=0, top=\cs{pdfpageheigh}\\
%   paper         & y             & top=\cs{pdfpageheight}, bottom=top-\cs{paperheight}\\
%   stock         & y             & top  derived from paper\\
%   head          & x             & calculated with hoffset, horigin, etc\\
%   head          & y             & calculated\\
%   body          & x             & = head value\\%
%   body          & y             & = head bottom - \cs{headsep}\\
%   foot          & x             & = head\\
%   foot          & y             & calculated from body bottom and \cs{footskip}\\
%   marginpar     & x             &different on odd/even pages! \\
%   marginpar     & y             & = body %
% \end{tabular}
%
% \subsection{Module \xmodule{dotfill}}
%
% \begin{declcs}{zdotfill}
% \end{declcs}
% This package provides the command \cs{zdotfill} that works
% similar to \cs{dotfill}, but can be configured. Especially
% it suppresses the dots if a minimum number of dots cannot be set.
%
% \begin{declcs}{zdotfillsetup}
%   |{|\textit{key$_1$=value$_1$, key$_2$=value$_2$, \dots}|}|
% \end{declcs}
% This command allows to configure the behaviour of \cs{zdotfill}.
% The following keys are available:
% \begin{description}
% \let\bfseries\relax
% \item[\texttt{min=}\meta{count value}]\mbox{}\\*
%   If the actual number of dots are smaller than \meta{count value},
%   then the dots are suppressed. Default: |2|.
% \item[\texttt{unit=}\meta{dimen value}]\mbox{}\\*
%   The width of a dot unit is given by \meta{dimen value}.
%   Default: |0.44em| (same as the unit in \cs{dotfill}).
% \item[\texttt{dot=}\meta{value}]\mbox{}\\*
%   The dot itself is given by \meta{value}. Default: |.| (dot, same as
%   the dot in \cs{dotfill}).
% \end{description}
%
% \subsection{Module \xmodule{env}}
%
% This module defines two properties |envname| and |envline|. They
% remember the name of the environment and the line number at
% the start of the environment.
%
% \subsection{Module \xmodule{xr}}
%
% This package provides the functionality of package \xpackage{xr},
% see \cite{xr}. It also supports the syntax of \xpackage{xr-hyper}.
%
% \begin{declcs}{zexternaldocument}
%   \* |[|\meta{prefix}|]|\markbabel \M{external document}
%   |[|\meta{url}|]|
% \end{declcs}
% See \cs{externaldocument} for a description of this option.
% The found labels also get a property |externaldocument| that
% remembers \meta{external document}.
% The standard reference scheme and the scheme of this
% package use different name spaces for reference names.
% If the external document uses both systems. Then one
% import statement would put the names in one namespace
% and probably causing problems with multiple references
% of the same name. Thus the star form only looks for
% \cs{newlabel} in the \xfile{.aux} files, whereas without
% star only \cs{zref@newlabel}s are used.
%
% In the star form it tries to detect labels from
% \xpackage{hyperref}, \xpackage{titleref}, and \xpackage{ntheorem}.
% If such an extended property from the packages before cannot
% be found or are empty, they are not included in the
% imported reference.
%
% Warnings are given if a reference name is already in use
% and the item is ignored. Unknown properties will automatically
% be declared.
%
% If the external references contain |anchor| properties,
% then we need also a url to be able to address the external
% file. As default the filename is taken with a default
% extension.
%
% \begin{declcs}{zxrsetup}
%   |{|\textit{key$_1$=value$_1$, key$_2$=value$_2$, \dots}|}|
% \end{declcs}
% The following setup options are available:
% \begin{description}
% \item[\xoption{ext}:] It sets the default extension.
% \item[\xoption{tozreflabel}:] Boolean option. The found references
%   are imported as zref labels. This is enabled by default.
% \item[\xoption{toltxlabel}:] Boolean option. The found references
%   are imported as \hologo{LaTeX} labels. Packages \xpackage{nameref},
%   \xpackage{titleref} and class \xclass{memoir} are supported.
% \item[\xoption{urluse}:] Boolean option. If enabled, then a URL
%   is stored in a macro and the macro is put in property `urluse'.
%   The URL is not put in property `url'.
%   The purpose is to save \hologo{TeX} memory.
% \item[\xoption{verbose}:] Boolean option. List the imported labels in the
%   \xext{log} file. Default is |false|.
% \end{description}
%
% \begin{declcs}{zref@xr@ext}
% \end{declcs}
% If the \meta{url} is not specified in \cs{zref@externaldocument},
% then the url will be constructed with the file name and
% this macro as extension. \cs{XR@ext} is used if
% \xpackage{hyperref} is loaded, otherwise |pdf|.
%
% \subsection{Module \xmodule{pageattr}}
%
% This module allows to recover the content of the register
% \cs{pdfpageattr} and \cs{pdfpagesattr} in pdftex and the equivalent register
% in luatex. There is no user command. Programmer commands are
%
% \begin{quote}
% |\zref@pdfpageattr{|\meta{absolute page number}|}|
%
% |\zref@pdfpagesattr{|\meta{absolute page number}|}|
% \end{quote}
% \section{ToDo}\label{sec:todo}
%
% Among other things the following issues are left for future work:
% \begin{itemize}
% \item Other applications: autoref, hyperref, \dots
% \end{itemize}
%
% \section{Example}
%
%    \begin{macrocode}
%<*example>
%    \end{macrocode}
%
%    \begin{macrocode}
\documentclass{book}

\usepackage[ngerman]{babel}%

\usepackage[savepos,totpages,titleref,dotfill,counter,user]{zref}

%    \end{macrocode}
%
%    Chapters are wrapped inside \cs{ChapterStart} and
%    \cs{ChapterStop}. The first argument \verb|#1| of \cs{ChapterStart}
%    is used to form a label id \verb|chap:#1|. At the end of
%    the chapter another label is set by \cs{zref@wrapper@immediate}, because
%    otherwise at the end of document a deferred write would not
%    be written, because there is no page for shipout.
%
%    Also this example shows how chapter titles can be recorded.
%    A new property \verb|chaptitle| is declared and added to the
%    main property list. In \cs{ChapterStart} the current value
%    of the property is updated.
%    \begin{macrocode}
\makeatletter
\zref@newprop{chaptitle}{}
\zref@addprop{main}{chaptitle}

\newcommand*{\ChapterStart}[2]{%
  \cleardoublepage
  \def\current@chapid{#1}%
  \zref@setcurrent{chaptitle}{#2}%
  \chapter{#2}%
  \zlabel{chap:#1}%
}
\newcommand*{\ChapterStop}{%
  \cleardoublepage
  \zref@wrapper@immediate{%
    \zref@labelbyprops{chapend:\current@chapid}{abspage}%
  }%
}
%    \end{macrocode}
%    \cs{ChapterPages} calculates and returns the number of pages of the
%    referenced chapter.
%    \begin{macrocode}
\newcommand*{\ChapterPages}[1]{%
  \zrefused{chap:#1}%
  \zrefused{chapend:#1}%
  \number\numexpr
    \zref@extract{chapend:#1}{abspage}%
    -\zref@extract{chap:#1}{abspage}%
    +1\relax
}
\makeatother
%    \end{macrocode}
%    \begin{macrocode}
\begin{document}
%    \end{macrocode}
%    As exception we use \cs{makeatletter} here, because
%    this is just an example file that also should show some of
%    programmer's interface.
%    \begin{macrocode}
\makeatletter

\frontmatter
\zlabel{documentstart}

\begin{itemize}
\item
  The frontmatter part has
  \number\numexpr\zref@extract{chap:first}{abspage}-1\relax
  ~pages.
\item
  Chapter \zref{chap:first} has \ChapterPages{first} page(s).
\item
  Section \zref{hello} is on the
  \ifcase\numexpr
    \zref@extractdefault{hello}{page}{0}%
    -\zref@extractdefault{chap:first}{page}{0}%
    +1\relax
    ??\or first\or second\or third\or forth\fi
  ~page inside its chapter.
\item
  The document has
  \zref[abspage]{LastPage} pages.
  This number is \ifodd\ztotpages odd\else even\fi.
\item
  The last page is labeled with \zpageref{LastPage}.
\item
  The title of chapter \zref{chap:next} %
  is ``\zref[chaptitle]{chap:next}''.
\end{itemize}

\tableofcontents

\mainmatter
\ChapterStart{first}{First chapter}

%    \end{macrocode}
%    The user level commands should protect babel shorthands
%    where possible. On the other side, expandable extracting
%    macros are useful in calculations, see above the
%    examples with \cs{numexpr}.
%    \begin{macrocode}
\section{Test}
\zlabel{a"o}
Section \zref{a"o} on page
\zref@wrapper@babel\zref@extract{a"o}{page}.

Text.
\newpage

\section{Hello World}
\zlabel{hello}

\ChapterStop

\ChapterStart{next}{Next chapter with \emph{umlauts}: "a"o"u"s}

%    \end{macrocode}
%
%    Here an example follows that makes use of \pdfTeX's ``savepos''
%    feature. The position on the page is not known before the
%    page is constructed and shipped out. Therefore the position
%    ist stored in references and are available for calculations
%    in the next \hologo{LaTeX} compile run.
%    \begin{macrocode}
The width of the first column is
  \the\dimexpr \zposx{secondcol}sp - \zposx{firstcol}sp\relax,\\
the height difference of the two baselines is
  \the\dimexpr \zposy{firstcol}sp - \zposy{secondline}sp\relax:\\
\begin{tabular}{ll}
  \zsavepos{firstcol}Hello&\zsavepos{secondcol}World\\
  \zsavepos{secondline}Second line&foobar\\
\end{tabular}

%    \end{macrocode}
%    With \cs{zrefused} \hologo{LaTeX} is notified, if the references
%    are not yet available and \hologo{LaTeX} can generate the rerun
%    hint.
%    \begin{macrocode}
\zrefused{firstcol}
\zrefused{secondcol}
\zrefused{secondline}

\ChapterStop
%    \end{macrocode}
%    Test for module \cs{dotfill}.
%    \begin{macrocode}
\ChapterStart{dotfill}{Test for dotfill feature}
\newcommand*{\dftest}[1]{%
  #1&
  [\makebox[{#1}]{\dotfill}]&
  [\makebox[{#1}]{\zdotfill}]\\
}
\begin{tabular}{rll}
& [\verb|\dotfill|] & [\verb|\zdotfill|]\\
\dftest{0.43em}
\dftest{0.44em}
\dftest{0.45em}
\dftest{0.87em}
\dftest{0.88em}
\dftest{0.89em}
\dftest{1.31em}
\dftest{1.32em}
\dftest{1.33em}
\end{tabular}
\ChapterStop
\end{document}
%    \end{macrocode}
%
%    \begin{macrocode}
%</example>
%    \end{macrocode}
%
% \StopEventually{
% }
%
% \section{Implementation}
%
% \subsection{Package \xpackage{zref}}
%
% \subsubsection{Identification}
%
%    \begin{macrocode}
%<*package>
\NeedsTeXFormat{LaTeX2e}
\ProvidesPackage{zref}
  [2020-07-03 v2.32 A new reference scheme for LaTeX (HO)]%
%    \end{macrocode}
%
% \subsubsection{Load basic module}
%
%    \begin{macrocode}
\RequirePackage{zref-base}[2019/11/29]
%    \end{macrocode}
%    Abort package loading if \xpackage{zref-base} could not be
%    loaded successfully.
%    \begin{macrocode}
\@ifundefined{ZREF@base@ok}{\endinput}{}
%    \end{macrocode}
%
% \subsubsection{Process options}
%
%    Known modules are loaded and the release date is checked.
%    \begin{macrocode}
\def\ZREF@temp#1{%
  \DeclareOption{#1}{%
    \AtEndOfPackage{%
      \RequirePackage{zref-#1}[2019/11/29]%
    }%
  }%
}
\ZREF@temp{abspage}
\ZREF@temp{counter}
\ZREF@temp{dotfill}
\ZREF@temp{hyperref}
\ZREF@temp{lastpage}
\ZREF@temp{marks}
\ZREF@temp{nextpage}
\ZREF@temp{pageattr}
\ZREF@temp{pagelayout}
\ZREF@temp{perpage}
\ZREF@temp{runs}
\ZREF@temp{savepos}
\ZREF@temp{thepage}
\ZREF@temp{titleref}
\ZREF@temp{totpages}
\ZREF@temp{user}
\ZREF@temp{xr}
%    \end{macrocode}
%    \begin{macrocode}
\ProcessOptions\relax
%</package>
%    \end{macrocode}
%
%
% \subsection{Module \xmodule{base}}
%
% \subsubsection{Prefixes}
%
%    This package uses the following prefixes for macro names:
%    \begin{description}
%    \item[\cs{zref@}:] Macros of the programmer's interface.
%    \item[\cs{ZREF@}:] Internal macros.
%    \item[\cs{Z@L@}\metacs{listname}:]
%      The properties of the list \meta{listname}.
%    \item[\cs{Z@D@}\metacs{propname}:]
%      The default value for property \meta{propname}.
%    \item[\cs{Z@E@}\metacs{propname}:]
%      Extract function for property \meta{propname}.
%    \item[\cs{Z@X@}\metacs{propname}:]
%      Information whether a property value for property
%      \meta{propname} is expanded
%      immediately or at shipout time.
%    \item[\cs{Z@C@}\metacs{propname}:]
%      Current value of the property \meta{propname}.
%    \item[\cs{Z@R@}\metacs{labelname}:]
%      Data for reference \meta{labelname}.
%    \item[\cs{ZREF@org@}:] Original versions of patched commands.
%    \item[\cs{z}:] For macros in user land, defined if
%      module \xmodule{user} is set.
%    \end{description}
%    The following family names are used for keys defined
%    according to the \xpackage{keyval} package:
%    \begin{description}
%    \item[\texttt{ZREF@TR}:] Setup for module \xmodule{titleref}.
%    \end{description}
%
% \subsubsection{Identification}
%
%    \begin{macrocode}
%<*base>
\NeedsTeXFormat{LaTeX2e}
\ProvidesPackage{zref-base}%
  [2020-07-03 v2.32 Module base for zref (HO)]%
%    \end{macrocode}
%
% \subsubsection{Utilities}
%
%    \begin{macrocode}
\providecommand\IfFormatAtLeastTF{\@ifl@t@r\fmtversion}
\RequirePackage{ltxcmds}[2010/12/02]
\RequirePackage{infwarerr}[2010/04/08]
\RequirePackage{kvsetkeys}[2010/03/01]
\RequirePackage{kvdefinekeys}[2010/03/01]
\RequirePackage{pdftexcmds}[2010/04/01]
%    \end{macrocode}
%
%    \begin{macro}{\ZREF@name}
%    Several times the package name is used, thus we store it
%    in \cmd{\ZREF@name}.
%    \begin{macrocode}
\def\ZREF@name{zref}
%    \end{macrocode}
%    \end{macro}
%
%    \begin{macrocode}
\ltx@IfUndefined{protected}{%
  \RequirePackage{makerobust}[2006/03/18]%
%    \end{macrocode}
%    \begin{macro}{\ZREF@Robust}
%    \begin{macrocode}
  \def\ZREF@Robust#1#2{%
    \def\ZREF@temp{\MakeRobustcommand#2}%
    \afterassignment\ZREF@temp
    #1#2%
  }%
%    \end{macrocode}
%    \end{macro}
%    \begin{macrocode}
}{%
%    \end{macrocode}
%    \begin{macro}{\ZREF@Robust}
%    \begin{macrocode}
  \def\ZREF@Robust#1{%
    \protected#1%
  }%
%    \end{macrocode}
%    \end{macro}
%    \begin{macrocode}
}
%    \end{macrocode}
%    \begin{macro}{\ZREF@IfDefinable}
%    \begin{macrocode}
\def\ZREF@IfDefinable#1#2#3{%
  \@ifdefinable{#1}{%
    \ZREF@Robust{#2}#1#3%
  }%
}
%    \end{macrocode}
%    \end{macro}
%
%    \begin{macro}{\ZREF@UpdatePdfTeX}
%    \cs{ZREF@UpdatePdfTeX} is used as help message text in
%    error messages.
%    \begin{macrocode}
\def\ZREF@UpdatePdfTeX{Update pdfTeX.}
%    \end{macrocode}
%    \end{macro}
%
%    \begin{macro}{\ifZREF@found}
%    The following switch is usded in list processing.
%    \begin{macrocode}
\newif\ifZREF@found
%    \end{macrocode}
%    \end{macro}
%
%    \begin{macro}{\ZREF@patch}
%    Macro \cs{ZREF@patch} first checks the existence of
%    the command and safes it.
%    \begin{macrocode}
\def\ZREF@patch#1{%
  \ltx@IfUndefined{#1}{%
    \ltx@gobble
  }{%
    \expandafter\let\csname ZREF@org@#1\expandafter\endcsname
    \csname #1\endcsname
    \ltx@firstofone
  }%
}
%    \end{macrocode}
%    \end{macro}
%
% \subsubsection{Check for \hologo{eTeX}}
%
%    The use of \hologo{eTeX} should be standard nowadays for \hologo{LaTeX}.
%    We test for \hologo{eTeX} in order to use its features later.
%    \begin{macrocode}
\ltx@IfUndefined{eTeXversion}{%
  \PackageError\ZREF@name{%
    Missing support for eTeX; package is abandoned%
  }{%
    Use a TeX compiler that support eTeX and enable eTeX %
    in the format.%
  }%
  \endinput
}{}%
%    \end{macrocode}
%
%    \begin{macrocode}
\RequirePackage{etexcmds}[2007/09/09]
\ifetex@unexpanded
\else
  \PackageError\ZREF@name{%
    Missing e-TeX's \string\unexpanded.\MessageBreak
    Add \string\RequirePackage\string{etexcmds\string} before %
    \string\documentclass%
  }{%
    Probably you are using some package (e.g. ConTeXt) that %
    redefines \string\unexpanded%
  }%
  \expandafter\endinput
\fi
%    \end{macrocode}
%
% \subsubsection{Auxiliary file stuff}
%
%    We are using some commands in the \xfile{.aux} files.
%    However sometimes these auxiliary files are interpreted
%    by \hologo{LaTeX} processes that haven't loaded this package
%    (e.g. package \xpackage{xr}).
%    Therefore we provide dummy definitions.
%    \begin{macrocode}
\RequirePackage{auxhook}
\AddLineBeginAux{%
  \string\providecommand\string\zref@newlabel[2]{}%
}
%    \end{macrocode}
%    \begin{macro}{\ZREF@RefPrefix}
%    \begin{macrocode}
\def\ZREF@RefPrefix{Z@R}
%    \end{macrocode}
%    \end{macro}
%    \begin{macro}{\zref@newlabel}
%    For the implementation of \cs{zref@newlabel} we call
%    the same internal macro \cs{@newl@bel} that is used
%    in \cs{newlabel}. Thus we have for free:
%    \begin{itemize}
%    \item \cs{Z@R@}\metacs{labelname} is defined.
%    \item \hologo{LaTeX}'s check for multiple references.
%    \item \hologo{LaTeX}'s check for changed references.
%    \end{itemize}
%    \begin{macrocode}
\ZREF@Robust\edef\zref@newlabel{%
  \noexpand\@newl@bel{\ZREF@RefPrefix}%
}
%    \end{macrocode}
%    \end{macro}
%
% \subsubsection{Property lists}
%
%    \begin{macro}{\zref@newlist}
%    Property lists are stored as list of property names
%    enclosed in curly braces. \cs{zref@newlist}
%    creates a new list as empty list. Assignments to
%    property lists are global.
%    \begin{macrocode}
\ZREF@Robust\def\zref@newlist#1{%
  \zref@iflistundefined{#1}{%
    \@ifdefinable{Z@L@#1}{%
     \global\expandafter\let\csname Z@L@#1\endcsname\ltx@empty
     \PackageInfo\ZREF@name{New property list: #1}%
    }%
  }{%
    \PackageError\ZREF@name{%
      Property list `#1' already exists%
    }\@ehc
  }%
}
%    \end{macrocode}
%    \end{macro}
%
%    \begin{macro}{\zref@iflistundefined}
%    \cs{zref@iflistundefined} checks the existence of the
%    property list \verb|#1|. If the property list is present,
%    then \verb|#2| is executed and \verb|#3| otherwise.
%    \begin{macrocode}
\def\zref@iflistundefined#1{%
  \ltx@ifundefined{Z@L@#1}%
}
%    \end{macrocode}
%    \end{macro}
%    \begin{macro}{\zref@listexists}
%    \cs{zref@listexists} only executes \verb|#2| if the
%    property list \verb|#1| exists and raises an error
%    message otherwise.
%    \begin{macrocode}
\ZREF@Robust\def\zref@listexists#1{%
  \zref@iflistundefined{#1}{%
    \PackageError\ZREF@name{%
      Property list `#1' does not exist%
    }\@ehc
  }%
}
%    \end{macrocode}
%    \end{macro}
%
%    \begin{macro}{\zref@iflistcontainsprop}
%    \cs{zref@iflistcontainsprop} checks, whether a property \verb|#2| is
%    already present in a property list \verb|#1|.
%    \begin{macrocode}
\ZREF@Robust\def\zref@iflistcontainsprop#1#2{%
  \zref@iflistundefined{#1}{%
    \ltx@secondoftwo
  }{%
    \begingroup\expandafter\endgroup
    \expandafter\in@
    \csname#2\expandafter\expandafter\expandafter\endcsname
    \expandafter\expandafter\expandafter{\csname Z@L@#1\endcsname}%
    \csname ltx@\ifin@ first\else second\fi oftwo\endcsname
  }%
}
%    \end{macrocode}
%    \end{macro}
%
%    \begin{macro}{\zref@listforloop}
%    \begin{macrocode}
\def\zref@listforloop#1#2{%
  \zref@listexists{#1}{%
    \expandafter\expandafter\expandafter\@tfor
    \expandafter\expandafter\expandafter\zref@prop
    \expandafter\expandafter\expandafter:%
    \expandafter\expandafter\expandafter=%
    \csname Z@L@#1\endcsname
    \do{%
      \begingroup
        \escapechar=-1 %
        \edef\x{\endgroup
          \def\noexpand\zref@prop{%
            \expandafter\string\zref@prop
          }%
        }%
      \x
      #2\zref@prop
    }%
  }%
}
%    \end{macrocode}
%    \end{macro}
%
%    \begin{macro}{\zref@addprops}
%    \cs{zref@addprop} adds the properties \verb|#2| to the property
%    list \verb|#1|, if the property is not already in the list.
%    Otherwise a warning is given.
%    \begin{macrocode}
\ZREF@Robust\def\zref@addprops#1#2{%
  \zref@listexists{#1}{%
    \comma@parse{#2}{%
      \zref@propexists\comma@entry{%
        \zref@iflistcontainsprop{#1}\comma@entry{%
          \PackageWarning\ZREF@name{%
            Property `\comma@entry' is already in list `#1'%
          }%
        }{%
          \begingroup\expandafter\endgroup
          \expandafter\g@addto@macro
          \csname Z@L@#1\expandafter\endcsname
          \expandafter{\csname\comma@entry\endcsname}%
        }%
      }%
      \ltx@gobble
    }%
  }%
}
%    \end{macrocode}
%    \end{macro}
%    \begin{macro}{\zref@addprop}
%    \cs{zref@addprop} adds the property \verb|#2| to the property
%    list \verb|#1|, if the property is not already in the list.
%    Otherwise a warning is given.
%    \begin{macrocode}
\ZREF@Robust\def\zref@addprop#1#2{%
  \zref@listexists{#1}{%
    \zref@propexists{#2}{%
      \zref@iflistcontainsprop{#1}{#2}{%
        \PackageWarning\ZREF@name{%
          Property `#2' is already in list `#1'%
        }%
      }{%
        \begingroup\expandafter\endgroup
        \expandafter\g@addto@macro
        \csname Z@L@#1\expandafter\endcsname
        \expandafter{\csname#2\endcsname}%
      }%
    }%
  }%
}
%    \end{macrocode}
%    \end{macro}
%
%    \begin{macro}{\zref@localaddprops}
%    \begin{macrocode}
\ZREF@Robust\def\zref@localaddprops#1#2{%
  \zref@listexists{#1}{%
    \comma@parse{#2}{%
      \zref@propexists\comma@entry{%
        \zref@iflistcontainsprop{#1}\comma@entry{%
          \PackageWarning\ZREF@name{%
            Property `\comma@entry' is already in list `#1'%
          }%
        }{%
          \begingroup\expandafter\endgroup
          \expandafter\ltx@LocalAppendToMacro
          \csname Z@L@#1\expandafter\endcsname
          \expandafter{\csname\comma@entry\endcsname}%
        }%
      }%
      \ltx@gobble
    }%
  }%
}
%    \end{macrocode}
%    \end{macro}
%    \begin{macro}{\zref@localaddprop}
%    \begin{macrocode}
\ZREF@Robust\def\zref@localaddprop#1#2{%
  \zref@listexists{#1}{%
    \zref@propexists{#2}{%
      \zref@iflistcontainsprop{#1}{#2}{%
        \PackageWarning\ZREF@name{%
          Property `#2' is already in list `#1'%
        }%
      }{%
        \begingroup\expandafter\endgroup
        \expandafter\ltx@LocalAppendToMacro
        \csname Z@L@#1\expandafter\endcsname
        \expandafter{\csname#2\endcsname}%
      }%
    }%
  }%
}
%    \end{macrocode}
%    \end{macro}
%
%    \begin{macrocode}
\ltx@IfUndefined{pdf@strcmp}{%
%    \end{macrocode}
%    \begin{macro}{\zref@delprop}
%    \begin{macrocode}
  \ZREF@Robust\def\zref@delprop{%
    \ZREF@delprop\gdef
  }%
%    \end{macrocode}
%    \end{macro}
%    \begin{macro}{\zref@localdelprop}
%    \begin{macrocode}
  \ZREF@Robust\def\zref@localdelprop{%
    \ZREF@delprop\def
  }%
%    \end{macrocode}
%    \end{macro}
%    \begin{macro}{\ZREF@delprop}
%    \begin{macrocode}
  \def\ZREF@delprop#1#2#3{%
    \zref@listexists{#2}{%
      \begingroup
        \escapechar=-1 %
        \def\ZREF@param{#3}%
        \@onelevel@sanitize\ZREF@param
        \toks@{}%
        \expandafter\expandafter\expandafter\ZREF@@delprop
        \csname Z@L@#2\endcsname!%
      \expandafter\endgroup
      \expandafter#1\csname Z@L@#2\expandafter\endcsname
      \expandafter{%
        \the\toks@
      }%
    }%
  }%
%    \end{macrocode}
%    \end{macro}
%    \begin{macro}{\ZREF@@delprop}
%    \begin{macrocode}
  \def\ZREF@@delprop#1{%
    \expandafter\ZREF@@@delprop\expandafter{\string#1}#1%
  }%
%    \end{macrocode}
%    \end{macro}
%    \begin{macro}{\ZREF@@@delprop}
%    \begin{macrocode}
  \def\ZREF@@@delprop#1#2{%
    \ifx#2!%
    \else
      \def\ZREF@temp{#1}%
      \@onelevel@sanitize\ZREF@temp
      \ifx\ZREF@param\ZREF@temp
      \else
        \toks@\expandafter{%
          \the\expandafter\toks@\csname#1\endcsname
        }%
      \fi
      \expandafter\ZREF@@delprop
    \fi
  }%
%    \end{macrocode}
%    \end{macro}
%    \begin{macrocode}
}{%
%    \end{macrocode}
%    \begin{macro}{\zref@delprop}
%    \begin{macrocode}
  \ZREF@Robust\def\zref@delprop{%
    \ZREF@delprop\xdef
  }%
%    \end{macrocode}
%    \end{macro}
%    \begin{macro}{\zref@localdelprop}
%    \begin{macrocode}
  \ZREF@Robust\def\zref@localdelprop{%
    \ZREF@delprop\edef
  }%
%    \end{macrocode}
%    \end{macro}
%    \begin{macro}{\ZREF@delprop}
%    \begin{macrocode}
  \def\ZREF@delprop#1#2#3{%
    \zref@listexists{#2}{%
      \def\ZREF@param{#3}%
      \edef\ZREF@SavedEscapechar{\the\escapechar}%
      \escapechar=-1 %
      \expandafter#1\csname Z@L@#2%
      \expandafter\expandafter\expandafter\endcsname{%
        \expandafter\expandafter\expandafter\ZREF@@delprop
        \csname Z@L@#2\endcsname!%
      }%
      \escapechar=\ZREF@SavedEscapechar\relax
    }%
  }%
%    \end{macrocode}
%    \end{macro}
%    \begin{macro}{\ZREF@@delprop}
%    Caution: |#1| might be an \cs{if} or similar token.
%    \begin{macrocode}
  \def\ZREF@@delprop#1{%
    \expandafter\ZREF@@@delprop\expandafter{\string#1}#1%
  }%
%    \end{macrocode}
%    \end{macro}
%    \begin{macro}{\ZREF@@@delprop}
%    \begin{macrocode}
  \def\ZREF@@@delprop#1#2{%
    \ifx#2!%
    \else
      \ifnum\pdf@strcmp{#1}{\ZREF@param}=\ltx@zero
      \else
        \expandafter\noexpand\csname#1\endcsname
      \fi
      \expandafter\ZREF@@delprop
    \fi
  }%
%    \end{macrocode}
%    \end{macro}
%    \begin{macrocode}
}
%    \end{macrocode}
%
% \subsubsection{Properties}
%
%    \begin{macro}{\zref@ifpropundefined}
%    \cs{zref@ifpropundefined} checks the existence of the
%    property \verb|#1|. If the property is present,
%    then \verb|#2| is executed and \verb|#3| otherwise.
%    \begin{macrocode}
\def\zref@ifpropundefined#1{%
  \ltx@ifundefined{Z@E@#1}%
}
%    \end{macrocode}
%    \end{macro}
%    \begin{macro}{\zref@propexists}
%    Some macros rely on the existence of a property.
%    \cs{zref@propexists} only executes \verb|#2| if the
%    property \verb|#1| exists and raises an error
%    message otherwise.
%    \begin{macrocode}
\ZREF@Robust\def\zref@propexists#1{%
  \zref@ifpropundefined{#1}{%
    \PackageError\ZREF@name{%
      Property `#1' does not exist%
    }\@ehc
  }%
}
%    \end{macrocode}
%    \end{macro}
%
%    \begin{macro}{\zref@newprop}
%    A new property is declared by \cs{zref@newprop}, the
%    property name \meta{propname} is given in \verb|#1|. The
%    property is created and configured. If the star form
%    is given, then the expansion of the property value
%    is delayed to page shipout time, when the reference
%    is written to the \xfile{.aux} file.
%    \begin{description}
%    \item[\cs{Z@D@}\metacs{propname}:]
%      Stores the default value for this property.
%    \item[\cs{Z@E@}\metacs{propname}:]
%      Extract function.
%    \item[\cs{Z@X@}\metacs{propname}:]
%      Information whether the expansion of the property value
%      is delayed to shipout time.
%    \item[\cs{Z@C@}\metacs{propname}:]
%      Current value of the property.
%    \end{description}
%    \begin{macrocode}
\ZREF@Robust\def\zref@newprop{%
  \@ifstar{%
    \let\ZREF@X\noexpand
    \ZREF@newprop
  }{%
    \let\ZREF@X\ltx@empty
    \ZREF@newprop
  }%
}
%    \end{macrocode}
%    \end{macro}
%    \begin{macro}{\ZREF@newprop}
%    \begin{macrocode}
\def\ZREF@newprop#1{%
  \edef\ZREF@P{#1}%
  \@onelevel@sanitize\ZREF@P
  \begingroup
  \ifx\ZREF@P\ZREF@par
    \@PackageError\ZREF@name{%
      Invalid property name `\ZREF@P'%
    }{%
      The property name `par' is not allowed %
      because of internal reasons.%
      \MessageBreak
      \@ehc
    }%
    \def\ZREF@@newprop[##1]##2{\endgroup}%
  \else
    \zref@ifpropundefined\ZREF@P{%
      \endgroup
      \PackageInfo\ZREF@name{%
        New property: \ZREF@P
      }%
    }{%
      \@PackageError\ZREF@name{%
        Property `\ZREF@P' already exists%
      }\@ehc
      \def\ZREF@@newprop[##1]##2{\endgroup}%
    }%
  \fi
  \@ifnextchar[\ZREF@@newprop{\ZREF@@newprop[\zref@default]}%
}
%    \end{macrocode}
%    \end{macro}
%    \begin{macro}{\ZREF@par}
%    \begin{macrocode}
\def\ZREF@par{par}
\@onelevel@sanitize\ZREF@par
%    \end{macrocode}
%    \end{macro}
%    \begin{macro}{\ZREF@@newprop}
%    \begin{macrocode}
\def\ZREF@@newprop[#1]{%
  \global\@namedef{Z@D@\ZREF@P}{#1}%
  \global\expandafter\let\csname Z@X@\ZREF@P\endcsname\ZREF@X
  \begingroup\expandafter\endgroup
  \expandafter\ZREF@@@newprop\csname\ZREF@P\endcsname
  \expandafter\gdef\csname Z@C@\ZREF@P\endcsname{}%
  \zref@setcurrent\ZREF@P
}
\def\ZREF@@@newprop#1{%
  \expandafter
  \gdef\csname Z@E@\ZREF@P\endcsname##1#1##2##3\ZREF@nil{##2}%
}
%    \end{macrocode}
%    \end{macro}
%
%    \begin{macro}{\zref@showprop}
%    \begin{macrocode}
\ZREF@Robust\def\zref@showprop#1{%
  \zref@ifpropundefined{#1}{%
    \@PackageInfoNoLine{\ZREF@name}{%
      Show property `#1': <undefined>%
    }%
  }{%
    \begingroup
      \toks@\expandafter\expandafter\expandafter{%
        \csname Z@C@#1\endcsname
      }%
      \edef\ZREF@value{\the\toks@}%
      \ltx@onelevel@sanitize\ZREF@value
      \toks@\expandafter\expandafter\expandafter{%
        \csname Z@D@#1\endcsname
      }%
      \edef\ZREF@default{\the\toks@}%
      \ltx@onelevel@sanitize\ZREF@default
      \@PackageInfoNoLine{\ZREF@name}{%
        Show property `#1':\MessageBreak
        \expandafter\ifx\csname Z@X@#1\endcsname\ltx@empty
          Immediate %
        \else
          Delayed %
        \fi
        value: [\ZREF@value]\MessageBreak
        Default: [\ZREF@default]%
      }%
    \endgroup
  }%
}
%    \end{macrocode}
%    \end{macro}
%
%    \begin{macro}{\zref@setcurrent}
%    \cs{zref@setcurrent} sets the current value for a property.
%    \begin{macrocode}
\ZREF@Robust\def\zref@setcurrent#1#2{%
  \zref@propexists{#1}{%
    \expandafter\def\csname Z@C@#1\endcsname{#2}%
  }%
}
%    \end{macrocode}
%    \end{macro}
%
%    \begin{macro}{\ZREF@getcurrent}
%    \cs{zref@getcurrent} gets the current value for a property.
%    \begin{macrocode}
\def\ZREF@getcurrent#1{%
  \romannumeral0%
  \ltx@ifundefined{Z@C@#1}{%
    \ltx@space
  }{%
    \expandafter\expandafter\expandafter\ltx@space
    \csname Z@C@#1\endcsname
  }%
}
%    \end{macrocode}
%    \end{macro}
%    \begin{macro}{\ZREF@u@getcurrent}
%    \begin{macrocode}
\def\ZREF@wu@getcurrent#1{%
  \etex@unexpanded\expandafter\expandafter\expandafter{%
    \ZREF@getcurrent{#1}%
  }%
}
%    \end{macrocode}
%    \end{macro}
%    \begin{macro}{\zref@getcurrent}
%    \begin{macrocode}
\let\zref@getcurrent\ZREF@getcurrent
%    \end{macrocode}
%    \end{macro}
%
% \subsubsection{Reference generation}
%
%    \begin{macro}{\zref@label}
%    Label macro that uses the main property list.
%    \begin{macrocode}
\ZREF@Robust\def\zref@label#1{%
  \zref@labelbylist{#1}\ZREF@mainlist
}
%    \end{macrocode}
%    \end{macro}
%    \begin{macro}{\zref@labelbylist}
%    Label macro that stores the properties, specified in the
%    property list \verb|#2|.
%    \begin{macrocode}
\ZREF@Robust\def\zref@labelbylist#1#2{%
  \@bsphack
    \zref@listexists{#2}{%
      \expandafter\expandafter\expandafter\ZREF@label
      \expandafter\expandafter\expandafter{%
        \csname Z@L@#2\endcsname
      }{#1}%
    }%
  \@esphack
}
%    \end{macrocode}
%    \end{macro}
%    \begin{macro}{\zref@labelbyprops}
%    The properties are directly specified in a comma separated list.
%    \begin{macrocode}
\ZREF@Robust\def\zref@labelbyprops#1#2{%
  \@bsphack
    \begingroup
      \toks@{}%
      \comma@parse{#2}{%
        \zref@ifpropundefined\comma@entry{%
          \PackageWarning\ZREF@name{%
            Property `\comma@entry' is not known%
          }%
        }{%
          \toks@\expandafter{%
            \the\expandafter\toks@\csname\comma@entry\endcsname
          }%
        }%
        \ltx@gobble
      }%
    \expandafter\endgroup
    \expandafter\ZREF@label\expandafter{\the\toks@}{#1}%
  \@esphack
}
%    \end{macrocode}
%    \end{macro}
%    \begin{macro}{\zref@labelbykv}
%    \begin{macrocode}
\ZREF@Robust\def\zref@labelbykv#1#2{%
  \@bsphack
    \begingroup
      \let\Z@L@ZREF@temp\ltx@empty
      \kvsetkeys{ZREF@LABEL}{#1}%
      \ifZREF@immediate
        \expandafter\zref@wrapper@immediate\expandafter{%
          \expandafter\ZREF@label\expandafter{\Z@L@ZREF@temp}{#2}%
        }%
      \else
        \expandafter\ZREF@label\expandafter{\Z@L@ZREF@temp}{#2}%
      \fi
    \endgroup
  \@esphack
}
%    \end{macrocode}
%    \end{macro}
%    \begin{macrocode}
\kv@define@key{ZREF@LABEL}{prop}{%
  \edef\ZREF@param{#1}%
  \zref@propexists\ZREF@param{%
    \zref@iflistcontainsprop{ZREF@temp}\ZREF@param{}{%
      \begingroup\expandafter\endgroup
      \expandafter\ltx@LocalAppendToMacro
      \expandafter\Z@L@ZREF@temp
      \expandafter{\csname\ZREF@param\endcsname}%
    }%
  }%
}
\kv@define@key{ZREF@LABEL}{list}{%
  \zref@listforloop{#1}{%
    \zref@iflistcontainsprop{ZREF@temp}\zref@prop{}{%
      \begingroup\expandafter\endgroup
      \expandafter\ltx@LocalAppendToMacro
      \expandafter\Z@L@ZREF@temp
      \expandafter{\csname\zref@prop\endcsname}%
    }%
    \ltx@gobble
  }%
}
\kv@define@key{ZREF@LABEL}{delprop}{%
  \zref@propexists{#1}{%
    \zref@localdelprop{ZREF@temp}{#1}%
  }%
}
\kv@define@key{ZREF@LABEL}{immediate}[true]{%
  \edef\ZREF@param{#1}%
  \ifx\ZREF@param\ZREF@true
    \ZREF@immediatetrue
  \else
    \ifx\ZREF@param\ZREF@false
      \ZREF@immediatefalse
    \else
      \PackageWarning\ZREF@name{%
        Option `immediate' expects `true' or `false'.\MessageBreak
        Ignoring invalid value `\ZREF@param'%
      }%
    \fi
  \fi
}
%    \end{macrocode}
%    \begin{macro}{\ZREF@false}
%    \begin{macrocode}
\def\ZREF@false{false}
%    \end{macrocode}
%    \end{macro}
%    \begin{macro}{\ZREF@true}
%    \begin{macrocode}
\def\ZREF@true{true}
%    \end{macrocode}
%    \end{macro}
%
%    \begin{macrocode}
\kv@define@key{ZREF@LABEL}{values}[]{%
  \kv@parse{#1}{%
    \ifx\kv@value\relax
      \@PackageWarning\ZREF@name{%
        Missing value for property `\kv@key'%
      }%
      \expandafter\ltx@gobbletwo
    \else
      \expandafter\zref@setcurrent
    \fi
  }%
}
%    \end{macrocode}
%
%    \begin{macro}{\ifZREF@immediate}
%    The switch \cs{ifZREF@immediate} tells us, whether the
%    label should be written immediately or at page shipout time.
%    \cs{ZREF@label} need to be notified about this, because
%    it must disable the deferred execution of property values,
%    if the label is written immediately.
%    \begin{macrocode}
\newif\ifZREF@immediate
%    \end{macrocode}
%    \end{macro}
%    \begin{macro}{\zref@wrapper@immediate}
%    The argument of \cs{zref@wrapper@immediate} is executed inside
%    a group where \cs{write} is redefined by adding
%    \cs{immediate} before its execution. Also \cs{ZREF@label}
%    is notified via the switch \cs{ifZREF@immediate}.
%    \begin{macrocode}
\ZREF@Robust{\long\def}\zref@wrapper@immediate#1{%
  \begingroup
    \ZREF@immediatetrue
    \let\ZREF@org@write\write
    \def\write{\immediate\ZREF@org@write}%
    #1%
  \endgroup
}
%    \end{macrocode}
%    \end{macro}
%
%    \begin{macro}{\ZREF@label}
%    \cs{ZREF@label} writes the data in the \xfile{.aux} file.
%    \verb|#1| contains the list of valid properties,
%    \verb|#2| the name of the reference. In case of immediate
%    writing, the deferred execution of property values is
%    disabled. Also \thepage is made expandable in this case.
%    \begin{macrocode}
\def\ZREF@label#1#2{%
  \if@filesw
    \begingroup
      \ifZREF@immediate
        \let\ZREF@org@thepage\thepage
      \fi
      \protected@write\@auxout{%
        \ifZREF@immediate
          \let\thepage\ZREF@org@thepage
        \fi
        \let\ZREF@temp\ltx@empty
        \@tfor\ZREF@P:=#1\do{%
          \begingroup
            \escapechar=-1 %
            \edef\x{\endgroup
              \def\noexpand\ZREF@P{%
                \expandafter\string\ZREF@P
              }%
            }%
          \x
          \expandafter\ifx
              \csname
                \ifZREF@immediate
                  relax%
                \else
                  Z@X@\ZREF@P%
                \fi
              \endcsname
              \noexpand
            \expandafter\let\csname Z@C@\ZREF@P\endcsname\relax
          \fi
          \toks@\expandafter{\ZREF@temp}%
          \edef\ZREF@temp{%
            \the\toks@
            \ltx@backslashchar\ZREF@P{%
              \expandafter\noexpand\csname Z@C@\ZREF@P\endcsname
            }%
          }%
        }%
      }{%
        \string\zref@newlabel{#2}{\ZREF@temp}%
      }%
    \endgroup
  \fi
}
\def\ZREF@addtoks#1{%
  \toks@\expandafter\expandafter\expandafter{%
    \expandafter\the\expandafter\toks@#1%
  }%
}
%    \end{macrocode}
%    \end{macro}
%
% \subsubsection{Reference querying and extracting}
%
%    Design goal for the extracting macros is that the extraction
%    process is full expandable. Thus these macros can be used
%    in expandable contexts. But there are problems that cannot
%    be solved by full expandable macros:
%    \begin{itemize}
%    \item In standard \hologo{LaTeX} undefined references sets a flag
%      and generate a warning. Both actions are not expandable.
%    \item Babel's support for its shorthand uses commands that
%      use non-expandable assignments. However currently there
%      is hope, that primitives are added to \pdfTeX\ that
%      allows the detection of contexts. Then the shorthand
%      can detect, if they are executed inside \cs{csname}
%      and protect themselves automatically.
%    \end{itemize}
%    \begin{macro}{\zref@ifrefundefined}
%    If a reference \verb|#1| is undefined, then macro
%    \cs{zref@ifrefundefined} calls \verb|#2| and \verb|#3| otherwise.
%    \begin{macrocode}
\def\zref@ifrefundefined#1{%
  \ltx@ifundefined{Z@R@#1}%
}
%    \end{macrocode}
%    \end{macro}
%
%    \begin{macro}{\zifrefundefined}
%    If a reference \verb|#1| is undefined, then macro
%    \cs{zref@ifrefundefined} calls \verb|#2| and \verb|#3| otherwise.
%    Also the reference is marked used.
%    \begin{macrocode}
\ZREF@IfDefinable\zifrefundefined\def{%
  #1{%
    \zref@wrapper@babel\ZREF@ifrefundefined{#1}%
  }%
}
%    \end{macrocode}
%    \end{macro}
%    \begin{macro}{\ZREF@ifrefundefined}
%    \begin{macrocode}
\def\ZREF@ifrefundefined#1{%
  \zref@refused{#1}%
  \zref@ifrefundefined{#1}%
}
%    \end{macrocode}
%    \end{macro}
%
%    \begin{macro}{\zref@refused}
%    The problem with undefined references is addressed by
%    the macro \cs{zref@refused}. This can be used outside
%    the expandable context. In case of an undefined reference
%    the flag is set to notify \hologo{LaTeX} and a warning is given.
%    \begin{macrocode}
\ZREF@Robust\def\zref@refused#1{%
  \zref@wrapper@babel\ZREF@refused{#1}%
}
%    \end{macrocode}
%    \end{macro}
%    \begin{macro}{\ZREF@refused}
%    \begin{macrocode}
\def\ZREF@refused#1{%
  \zref@ifrefundefined{#1}{%
    \protect\G@refundefinedtrue
    \@latex@warning{%
      Reference `#1' on page \thepage \space undefined%
    }%
  }{}%
}
%    \end{macrocode}
%    \end{macro}
%
%    \begin{macro}{\zref@ifrefcontainsprop}
%    \cs{zref@ifrefcontainsprop} looks, if the reference \verb|#1|
%    has the property \verb|#2| and calls then \verb|#3| and
%    \verb|#4| otherwise.
%    \begin{macrocode}
\def\zref@ifrefcontainsprop#1#2{%
  \zref@ifrefundefined{#1}{%
    \ltx@secondoftwo
  }{%
    \expandafter\ZREF@ifrefcontainsprop
    \csname Z@E@#2\expandafter\endcsname
    \csname#2\expandafter\expandafter\expandafter\endcsname
    \expandafter\expandafter\expandafter{%
      \csname Z@R@#1\endcsname
    }%
  }%
}
\def\ZREF@ifrefcontainsprop#1#2#3{%
  \expandafter\ifx\expandafter\ZREF@novalue
  #1#3#2\ZREF@novalue\ZREF@nil\ltx@empty
    \expandafter\ltx@secondoftwo
  \else
    \expandafter\ltx@firstoftwo
  \fi
}
\def\ZREF@novalue{\ZREF@NOVALUE}
%    \end{macrocode}
%    \end{macro}
%
%    \begin{macro}{\zref@extract}
%    \cs{zref@extract} is an abbreviation for the case
%    that the default of the property is used as default
%    value.
%    \begin{macrocode}
\def\ZREF@extract#1#2{%
  \romannumeral0%
  \ltx@ifundefined{Z@D@#2}{%
    \expandafter\ltx@space\zref@default
  }{%
    \expandafter\expandafter\expandafter\ZREF@@extract
    \expandafter\expandafter\expandafter{%
      \csname Z@D@#2\endcsname
    }{#1}{#2}%
  }%
}
%    \end{macrocode}
%    \end{macro}
%    \begin{macro}{\ZREF@@extract}
%    \begin{macrocode}
\def\ZREF@@extract#1#2#3{%
  \expandafter\expandafter\expandafter\ltx@space
  \zref@extractdefault{#2}{#3}{#1}%
}
%    \end{macrocode}
%    \end{macro}
%    \begin{macro}{\ZREF@wu@extract}
%    \begin{macrocode}
\def\ZREF@wu@extract#1#2{%
  \etex@unexpanded\expandafter\expandafter\expandafter{%
    \ZREF@extract{#1}{#2}%
  }%
}
%    \end{macrocode}
%    \end{macro}
%    \begin{macro}{\zref@extract}
%    \begin{macrocode}
\let\zref@extract\ZREF@extract
%    \end{macrocode}
%    \end{macro}
%
%    \begin{macro}{\ZREF@extractdefault}
%    The basic extracting macro is \cs{zref@extractdefault} with
%    the reference name in \verb|#1|, the property in \verb|#2| and
%    the default value in \verb|#3| in case for problems.
%    \begin{macrocode}
\def\ZREF@extractdefault#1#2#3{%
  \romannumeral0%
  \zref@ifrefundefined{#1}\ltx@firstoftwo{%
    \zref@ifpropundefined{#2}\ltx@firstoftwo\ltx@secondoftwo
  }{%
    \ltx@space
    #3%
  }{%
    \expandafter\expandafter\expandafter\ltx@space
    \csname Z@E@#2\expandafter\expandafter\expandafter\endcsname
    \csname Z@R@#1\expandafter\endcsname
    \csname#2\endcsname{#3}\ZREF@nil
  }%
}
%    \end{macrocode}
%    \end{macro}
%    \begin{macro}{\ZREF@wu@extractdefault}
%    \begin{macrocode}
\def\ZREF@wu@extractdefault#1#2#3{%
  \etex@unexpanded\expandafter\expandafter\expandafter{%
    \ZREF@extractdefault{#1}{#2}{#3}%
  }%
}
%    \end{macrocode}
%    \end{macro}
%    \begin{macro}{\zref@extractdefault}
%    \begin{macrocode}
\let\zref@extractdefault\ZREF@extractdefault
%    \end{macrocode}
%    \end{macro}
%
%    \begin{macro}{\zref@def@extract}
%    \begin{macrocode}
\ZREF@Robust\def\zref@def@extract#1{%
  \zref@wrapper@babel{\ZREF@def@extract{#1}}%
}
%    \end{macrocode}
%    \end{macro}
%    \begin{macro}{\ZREF@def@extract}
%    \begin{macrocode}
\def\ZREF@def@extract#1#2#3{%
  \zref@refused{#2}%
  \expandafter\expandafter\expandafter\def
  \expandafter\expandafter\expandafter#1%
  \expandafter\expandafter\expandafter{%
    \zref@extract{#2}{#3}%
  }%
}
%    \end{macrocode}
%    \end{macro}
%    \begin{macro}{\zref@def@extractdefault}
%    \begin{macrocode}
\ZREF@Robust\def\zref@def@extractdefault#1{%
  \zref@wrapper@babel{\ZREF@def@extractdefault{#1}}%
}
%    \end{macrocode}
%    \end{macro}
%    \begin{macro}{\ZREF@def@extractdefault}
%    \begin{macrocode}
\def\ZREF@def@extractdefault#1#2#3#4{%
  \zref@refused{#2}%
  \expandafter\expandafter\expandafter\def
  \expandafter\expandafter\expandafter#1%
  \expandafter\expandafter\expandafter{%
    \zref@extractdefault{#2}{#3}{#4}%
  }%
}
%    \end{macrocode}
%    \end{macro}
%
%    \begin{macro}{\ZREF@wrapper@unexpanded}
%    \begin{macrocode}
\ZREF@Robust{\long\def}\ZREF@wrapper@unexpanded#1{%
  \let\zref@wrapper@unexpanded\ltx@firstofone
  \let\zref@getcurrent\ZREF@wu@getcurrent
  \let\zref@extractdefault\ZREF@wu@extractdefault
  \let\zref@extract\ZREF@wu@extract
  #1%
  \let\zref@wrapper@unexpanded\ZREF@wrapper@unexpanded
  \let\zref@getcurrent\ZREF@getcurrent
  \let\zref@extractdefault\ZREF@extractdefault
  \let\zref@extract\ZREF@extract
}
%    \end{macrocode}
%    \end{macro}
%    \begin{macro}{\zref@wrapper@unexpanded}
%    \begin{macrocode}
\ltx@IfUndefined{etex@unexpanded}{%
  \let\zref@wrapper@unexpanded\ltx@firstofone
}{%
  \let\zref@wrapper@unexpanded\ZREF@wrapper@unexpanded
}
%    \end{macrocode}
%    \end{macro}
%
% \subsubsection{Compatibility with \xpackage{babel}}
%
%    \begin{macro}{\zref@wrapper@babel}
%    \begin{macrocode}
\ZREF@Robust{\long\def}\zref@wrapper@babel#1#2{%
  \ifcsname if@safe@actives\endcsname
    \expandafter\ltx@firstofone
  \else
    \expandafter\ltx@secondoftwo
  \fi
  {%
    \if@safe@actives
      \expandafter\ltx@secondoftwo
    \else
      \expandafter\ltx@firstoftwo
    \fi
    {%
      \begingroup
        \csname @safe@activestrue\endcsname
        \edef\x{#2}%
      \expandafter\endgroup
      \expandafter\ZREF@wrapper@babel\expandafter{\x}{#1}%
    }%
  }{%
    #1{#2}%
  }%
}
\long\def\ZREF@wrapper@babel#1#2{%
  #2{#1}%
}
%    \end{macrocode}
%    \end{macro}
%
% \subsubsection{Unique counter support}
%
%    \begin{macro}{\zref@require@unique}
%    Generate the counter |zref@unique| if the counter does not
%    already exist.
%    \begin{macrocode}
\ZREF@Robust\def\zref@require@unique{%
  \@ifundefined{c@zref@unique}{%
    \begingroup
      \let\@addtoreset\ltx@gobbletwo
      \newcounter{zref@unique}%
    \endgroup
%    \end{macrocode}
%    \end{macro}
%
%    \begin{macro}{\thezref@unique}
%    \cs{thezref@unique} is used for automatically generated unique
%    labelnames.
%    \begin{macrocode}
    \renewcommand*{\thezref@unique}{%
      zref@\number\c@zref@unique
    }%
  }{}%
}
%    \end{macrocode}
%    \end{macro}
%
% \subsubsection{Utilities}
%
%    \begin{macro}{\ZREF@number}
%    \begin{macrocode}
\ltx@IfUndefined{numexpr}{%
  \def\ZREF@number#1{\number#1}%
}{%
  \def\ZREF@number#1{\the\numexpr(#1)\relax}%
}
%    \end{macrocode}
%    \end{macro}
%
% \subsubsection{Setup}
%
%    \begin{macro}{\zref@setdefault}
%    Standard \hologo{LaTeX} prints ``??'' in bold face if a reference
%    is not known. \cs{zref@default} holds the text that is
%    printed in case of unknown references and is used, if
%    the default was not specified during the definition
%    of the new property by \cs{ref@newprop}.
%    The global default value can be set by \cs{zref@setdefault}.
%    \begin{macrocode}
\ZREF@Robust\def\zref@setdefault#1{%
  \def\zref@default{#1}%
}
%    \end{macrocode}
%    \end{macro}
%    \begin{macro}{\zref@default}
%    Now we initialize \cs{zref@default} with the same value
%    that \hologo{LaTeX} uses for its undefined references.
%    \begin{macrocode}
\zref@setdefault{%
  \nfss@text{\reset@font\bfseries ??}%
}
%    \end{macrocode}
%    \end{macro}
%
% \paragraph{Main property list.}
%
%    \begin{macro}{\zref@setmainlist}
%    The name of the default property list is stored in
%    \cs{ZREF@mainlist} and can be set by \cs{zref@setmainlist}.
%    \begin{macrocode}
\ZREF@Robust\def\zref@setmainlist#1{%
  \def\ZREF@mainlist{#1}%
}
\zref@setmainlist{main}
%    \end{macrocode}
%    \end{macro}
%    Now we create the list.
%    \begin{macrocode}
\zref@newlist\ZREF@mainlist
%    \end{macrocode}
%
% \paragraph{Main properties.}
%
%    The two properties \texttt{default} and \texttt{page}
%    are created and added to the main property list.
%    They store the data that
%    standard \hologo{LaTeX} uses in its references created by
%    \cs{label}.
%    \begin{description}
%    \item[\texttt{default}] the apperance of the latest
%      counter that is incremented by \cs{refstepcounter}
%    \item[\texttt{page}] the apperance of the page counter
%    \end{description}
%    \begin{macrocode}
\zref@newprop{default}{\@currentlabel}
\zref@newprop*{page}{\thepage}
\zref@addprops\ZREF@mainlist{default,page}
%    \end{macrocode}
%
% \paragraph{Properties}
%
%    \begin{macro}{\ZREF@NewPropAnchor}
%    \begin{macrocode}
\def\ZREF@NewPropAnchor{%
  \zref@newprop{anchor}{%
    \ltx@ifundefined{@currentHref}{}{\@currentHref}%
  }%
  \global\let\ZREF@NewPropAnchor\relax
}
%    \end{macrocode}
%    \end{macro}
%    \begin{macro}{\zref@titleref@current}
%    Later we will redefine the section and caption macros
%    to catch the current title and remember the value in
%    \cs{zref@titleref@current}.
%    \begin{macro}{\ZREF@NewPropTitle}
%    \begin{macrocode}
\def\ZREF@NewPropTitle{%
  \gdef\zref@titleref@current{}%
  \zref@newprop{title}{\zref@titleref@current}%
  \global\let\ZREF@NewPropTitle\relax
}
%    \end{macrocode}
%    \end{macro}
%    \end{macro}
%    \begin{macro}{\ZREF@NewPropTheotype}
%    \begin{macrocode}
\def\ZREF@NewPropTheotype{%
  \zref@newprop{theotype}{}%
  \global\let\ZREF@NewPropTheotype\relax
}
%    \end{macrocode}
%    \end{macro}
%    \begin{macro}{\ZREF@NewPropPageValue}
%    \begin{macrocode}
\def\ZREF@NewPropPageValue{%
  \zref@newprop*{pagevalue}[0]{\number\c@page}%
  \global\let\ZREF@NewPropPageValue\relax
}
%    \end{macrocode}
%    \end{macro}
%
% \paragraph{Mark successful loading}
%
%    \begin{macrocode}
\let\ZREF@base@ok=Y
%</base>
%    \end{macrocode}
%
%
% \subsection{Module \xmodule{user}}
%
%    \begin{macrocode}
%<*user>
\NeedsTeXFormat{LaTeX2e}
\ProvidesPackage{zref-user}%
  [2020-07-03 v2.32 Module user for zref (HO)]%
\RequirePackage{zref-base}[2019/11/29]
\ifx\ZREF@base@ok Y%
\else
  \expandafter\endinput
\fi
%    \end{macrocode}
%
%    Module \xmodule{user} enables a small user interface.
%    All macros are prefixed by \cs{z}.
%
%    First we define the pendants to the standard \hologo{LaTeX} referencing
%    commands \cs{label}, \cs{ref}, and \cs{pageref}.
%
%    \begin{macro}{\zlabel}
%    Similar to \cs{label} the macro \cs{zlabel} writes a reference
%    entry in the \xfile{.aux} file. The main property list is used.
%    Also we add the babel patch. The \cs{label} command can also be
%    used inside section titles, but it must not go into the
%    table of contents. Therefore we have to check this situation.
%    \begin{macrocode}
\newcommand*\zlabel{%
  \ifx\label\ltx@gobble
    \expandafter\ltx@gobble
  \else
    \expandafter\zref@wrapper@babel\expandafter\zref@label
  \fi
}%
%    \end{macrocode}
%    \end{macro}
%
%    \begin{macro}{\zkvlabel}
%    \begin{macrocode}
\newcommand*{\zkvlabel}[1]{%
  \ifx\label\ltx@gobble
    \expandafter\ltx@gobblethree
  \fi
  \zref@wrapper@babel{\zref@labelbykv{#1}}%
}%
%    \end{macrocode}
%    \end{macro}
%
%    \begin{macro}{\zref}
%    Macro \cs{zref} is the corresponding macro for \cs{ref}.
%    Also it provides an optional argument in order to
%    select another property.
%    \begin{macrocode}
\newcommand*{\zref}[2][default]{% robust because of optional argument
  \zref@propexists{#1}{%
    \zref@wrapper@babel\ZREF@zref{#2}{#1}%
  }%
}%
\def\ZREF@zref#1{%
  \zref@refused{#1}%
  \zref@extract{#1}%
}%
%    \end{macrocode}
%    \end{macro}
%
%    \begin{macro}{\zpageref}
%    For macro \cs{zpageref} we just call \cs{zref}
%    with property \verb|page|.
%    \begin{macrocode}
\ZREF@IfDefinable\zpageref\def{%
  {\zref[page]}%
}
%    \end{macrocode}
%    \end{macro}
%
%    \begin{macro}{\zrefused}
%    For the following expandible user macros \cs{zrefused}
%    should be used to notify \hologo{LaTeX} in case of undefined
%    references.
%    \begin{macrocode}
\ZREF@IfDefinable\zrefused\def{%
  {\zref@refused}%
}
%    \end{macrocode}
%    \end{macro}
%
%    \begin{macrocode}
%</user>
%    \end{macrocode}
%
%
% \subsection{Module \xmodule{abspage}}
%
%    \begin{macrocode}
%<*abspage>
\NeedsTeXFormat{LaTeX2e}
\ProvidesPackage{zref-abspage}%
  [2020-07-03 v2.32 Module abspage for zref (HO)]%
\RequirePackage{zref-base}[2019/11/29]
\ifx\ZREF@base@ok Y%
\else
  \expandafter\endinput
\fi
%    \end{macrocode}
%
%    Module \xmodule{abspage} adds a new property \verb|abspage|
%    to the \verb|main| property list for absolute page numbers.
%    These are recorded by the help of package \xpackage{atbegshi}.
%    \begin{macrocode}
\RequirePackage{atbegshi}[2011/10/05]%
%    \end{macrocode}
%    The counter \verb|abspage| must not go in the clear list
%    of \verb|@ckpt| that is used to set counters in
%    \xfile{.aux} files of included \hologo{TeX} files.
%    \begin{macrocode}
\begingroup
  \let\@addtoreset\ltx@gobbletwo
  \newcounter{abspage}%
\endgroup
\setcounter{abspage}{0}%
\AtBeginShipout{%
  \stepcounter{abspage}%
}%
\zref@newprop*{abspage}[0]{\the\c@abspage}%
\zref@addprop\ZREF@mainlist{abspage}%
%    \end{macrocode}
%    Note that counter \verb|abspage| shows the
%    previous page during page processing. Before
%    shipout the counter is incremented.
%    Thus the property is correctly written with
%    deferred writing. If the counter is written
%    using \cs{zref@wrapper@immediate}, then the number
%    is too small by one.
%
%    \begin{macrocode}
%</abspage>
%    \end{macrocode}
%
%
% \subsection{Module \xmodule{counter}}
%
%    \begin{macrocode}
%<*counter>
\NeedsTeXFormat{LaTeX2e}
\ProvidesPackage{zref-counter}%
  [2020-07-03 v2.32 Module counter for zref (HO)]%
\RequirePackage{zref-base}[2019/11/29]
\ifx\ZREF@base@ok Y%
\else
  \expandafter\endinput
\fi
%    \end{macrocode}
%
%    For features such as \xpackage{hyperref}'s \cs{autoref}
%    we need the name of the counter. The property
%    \verb|counter| is defined and added to the main property
%    list. Starting with \LaTeX\ 2020-10-01 the proper can use
%    \@currentcounter. In older formats
%    \cs{refstepcounter} has to be patched but this can fail in some
%    cases, see issue \#5.
%    \begin{macrocode}
\@ifl@t@r\fmtversion{2020-10-01}
 {
  \zref@newprop{counter}{\@currentcounter}
  \zref@addprop\ZREF@mainlist{counter}
 }
 {
  \zref@newprop{counter}{}
  \zref@addprop\ZREF@mainlist{counter}
  \AtBeginDocument{%
    \ZREF@patch{refstepcounter}{%
      \def\refstepcounter#1{%
        \zref@setcurrent{counter}{#1}%
        \ZREF@org@refstepcounter{#1}%
      }%
    }%
  }
 }
%    \end{macrocode}
%    \begin{macrocode}
%</counter>
%    \end{macrocode}
%
%
% \subsection{Module \xmodule{lastpage}}
%
%    \begin{macrocode}
%<*lastpage>
\NeedsTeXFormat{LaTeX2e}
\ProvidesPackage{zref-lastpage}%
  [2020-07-03 v2.32 Module lastpage for zref (HO)]%
\RequirePackage{zref-base}[2019/11/29]
\RequirePackage{zref-abspage}[2019/11/29]
\RequirePackage{atveryend}[2009/12/07]
\ifx\ZREF@base@ok Y%
\else
  \expandafter\endinput
\fi
%    \end{macrocode}
%
%    The module \xmodule{lastpage} implements the service
%    of package \xpackage{lastpage} by setting a reference
%    \verb|LastPage| at the end of the document.
%    If module \xmodule{abspage} is given, also the absolute
%    page number is available, because the properties of
%    the main property list are used.
%    \begin{macrocode}
\zref@newlist{LastPage}
\AfterLastShipout{%
  \if@filesw
    \begingroup
      \advance\c@page\m@ne
      \toks@\expandafter\expandafter\expandafter{%
        \expandafter\Z@L@main
        \Z@L@LastPage
      }%
      \expandafter\zref@wrapper@immediate\expandafter{%
        \expandafter\ZREF@label\expandafter{\the\toks@}{LastPage}%
      }%
    \endgroup
  \fi
}
%    \end{macrocode}
%    \begin{macro}{\zref@iflastpage}
%    \begin{macrocode}
\def\zref@iflastpage#1{%
  \ifnum\zref@extractdefault{#1}{abspage}{-1}=%
        \zref@extractdefault{LastPage}{abspage}{-2} %
    \expandafter\ltx@firstoftwo
  \else
    \expandafter\ltx@secondoftwo
  \fi
}
%    \end{macrocode}
%    \end{macro}
%    \begin{macro}{\ziflastpage}
%    \begin{macrocode}
\ZREF@IfDefinable\ziflastpage\def{%
  {\zref@wrapper@babel\ZREF@iflastpage}%
}
%    \end{macrocode}
%    \end{macro}
%    \begin{macro}{ZREF@iflastpage}
%    \begin{macrocode}
\def\ZREF@iflastpage#1{%
  \zref@refused{LastPage}%
  \zref@refused{#1}%
  \zref@iflastpage{#1}%
}
%    \end{macrocode}
%    \end{macro}
%    \begin{macrocode}
%</lastpage>
%    \end{macrocode}
%
%
% \subsection{Module \xmodule{thepage}}
%
%    \begin{macrocode}
%<*thepage>
\NeedsTeXFormat{LaTeX2e}
\ProvidesPackage{zref-thepage}%
  [2020-07-03 v2.32 Module thepage for zref (HO)]%
\RequirePackage{zref-base}[2019/11/29]
\ifx\ZREF@base@ok Y%
\else
  \expandafter\endinput
\fi
%    \end{macrocode}
%
%    \begin{macrocode}
\RequirePackage{atbegshi}[2011/10/05]
\RequirePackage{zref-abspage}[2019/11/29]
%    \end{macrocode}
%    \begin{macrocode}
\zref@newlist{thepage}
\zref@addprop{thepage}{page}
\ZREF@NewPropPageValue
%    \end{macrocode}
%    \begin{macro}{\zref@thepage@atbegshi@hook}
%    \begin{macrocode}
\let\zref@thepage@atbegshi@hook\ltx@empty
%    \end{macrocode}
%    \end{macro}
%    \begin{macrocode}
\zref@addprop{thepage}{pagevalue}
\AtBeginShipout{%
  \AtBeginShipoutAddToBox{%
    \zref@thepage@atbegshi@hook
    \zref@labelbylist{thepage\the\value{abspage}}{thepage}%
  }%
}
%    \end{macrocode}
%    \begin{macro}{\zref@thepage@name}
%    \begin{macrocode}
\ltx@IfUndefined{numexpr}{%
  \def\zref@thepage@name#1{thepage\number#1}%
}{%
  \def\zref@thepage@name#1{thepage\the\numexpr#1}%
}
%    \end{macrocode}
%    \end{macro}
%    \begin{macro}{\zref@thepage}
%    \begin{macrocode}
\def\zref@thepage#1{%
  \zref@extract{\zref@thepage@name{#1}}{page}%
}%
%    \end{macrocode}
%    \end{macro}
%    \begin{macro}{\zref@thepage@refused}
%    \begin{macrocode}
\ZREF@Robust\def\zref@thepage@refused#1{%
  \zref@refused{\zref@thepage@name{#1}}%
}%
%    \end{macrocode}
%    \end{macro}
%    \begin{macro}{\zthepage}
%    \begin{macrocode}
\ZREF@IfDefinable\zthepage\def{%
  #1{%
    \zref@thepage@refused{#1}%
    \zref@thepage{#1}%
  }%
}
%    \end{macrocode}
%    \end{macro}
%    \begin{macrocode}
%</thepage>
%    \end{macrocode}
%
%
% \subsection{Module \xmodule{nextpage}}
%
%    \begin{macrocode}
%<*nextpage>
\NeedsTeXFormat{LaTeX2e}
\ProvidesPackage{zref-nextpage}%
  [2020-07-03 v2.32 Module nextpage for zref (HO)]%
\RequirePackage{zref-base}[2019/11/29]
\ifx\ZREF@base@ok Y%
\else
  \expandafter\endinput
\fi
%    \end{macrocode}
%
%    \begin{macrocode}
\RequirePackage{zref-abspage}[2019/11/29]
\RequirePackage{zref-thepage}[2019/11/29]
\RequirePackage{zref-lastpage}[2019/11/29]
\RequirePackage{uniquecounter}[2009/12/18]
%    \end{macrocode}
%    \begin{macrocode}
\UniqueCounterNew{znextpage}

\newcommand*{\znextpagesetup}{%
  \afterassignment\ZREF@np@setup@i
  \def\ZREF@np@call@unknown##1%
}
\def\ZREF@np@setup@i{%
  \afterassignment\ZREF@np@setup@ii
  \def\ZREF@np@call@nonext##1%
}
\def\ZREF@np@setup@ii{%
  \def\ZREF@np@call@next##1%
}
\def\ZREF@np@call@unknown#1{#1}
\def\ZREF@np@call@nonext#1{#1}
\def\ZREF@np@call@next#1{#1}
\ZREF@IfDefinable\znextpage\def{%
  {\UniqueCounterCall{znextpage}{\ZREF@nextpage}}%
}%
\newcommand*{\znonextpagename}{}
\newcommand*{\zunknownnextpagename}{\Z@D@page}
\def\ZREF@nextpage#1{%
  \begingroup
    \def\ZREF@refname@this{zref@np#1}%
    \zref@labelbyprops\ZREF@refname@this{abspage}%
    \chardef\ZREF@call=0 % unknown
    \ZREF@ifrefundefined\ZREF@refname@this{%
    }{%
      \edef\ZREF@pagenum@this{%
        \zref@extractdefault\ZREF@refname@this{abspage}{0}%
      }%
      \edef\ZREF@refname@next{%
        \zref@thepage@name{%
          \the\numexpr\ZREF@pagenum@this+1%
        }%
      }%
      \ifnum\ZREF@pagenum@this>0 %
        \ZREF@ifrefundefined{LastPage}{%
          \zref@ifrefundefined\ZREF@refname@next{%
          }{%
            \chardef\ZREF@call=2 % next page
          }%
        }{%
          \edef\ZREF@pagenum@last{%
            \zref@extractdefault{LastPage}{abspage}{0}%
          }%
          \ifnum\ZREF@pagenum@this<\ZREF@pagenum@last\ltx@space
            \ZREF@ifrefundefined\ZREF@refname@next{%
            }{%
              \chardef\ZREF@call=2 % next page
            }%
          \else
            \ifnum\ZREF@pagenum@this=\ZREF@pagenum@this\ltx@space
              \chardef\ZREF@call=1 % no next page
            \fi
          \fi
        }%
      \fi
    }%
    \edef\x{%
      \endgroup
      \ifcase\ZREF@call
        \noexpand\ZREF@np@call@unknown{%
          \noexpand\zunknownnextpagename
        }%
      \or
        \noexpand\ZREF@np@call@nonext{%
          \noexpand\znonextpagename
        }%
      \else
        \noexpand\ZREF@np@call@next{%
          \noexpand\zref@extract{\ZREF@refname@next}{page}%
        }%
      \fi
    }%
  \x
}
%    \end{macrocode}
%    \begin{macrocode}
%</nextpage>
%    \end{macrocode}
%
%
% \subsection{Module \xmodule{totpages}}
%
%    \begin{macrocode}
%<*totpages>
\NeedsTeXFormat{LaTeX2e}
\ProvidesPackage{zref-totpages}%
  [2020-07-03 v2.32 Module totpages for zref (HO)]%
\RequirePackage{zref-base}[2019/11/29]
\ifx\ZREF@base@ok Y%
\else
  \expandafter\endinput
\fi
%    \end{macrocode}
%
%    The absolute page number of the last page is the total
%    page number.
%    \begin{macrocode}
\RequirePackage{zref-abspage}[2019/11/29]
\RequirePackage{zref-lastpage}[2019/11/29]
%    \end{macrocode}
%
%    \begin{macro}{\ztotpages}
%    Macro \cs{ztotpages} contains the number of pages. It can be
%    used inside expandable calculations. It expands to zero if
%    the reference is not yet available.
%    \begin{macrocode}
\newcommand*{\ztotpages}{%
  \zref@extractdefault{LastPage}{abspage}{0}%
}
%    \end{macrocode}
%    \end{macro}
%    Also we mark the reference |LastPage| as used:
%    \begin{macrocode}
\AtBeginDocument{%
  \zref@refused{LastPage}%
}
%    \end{macrocode}
%    \begin{macrocode}
%</totpages>
%    \end{macrocode}
%
%
% \subsection{Module \xmodule{pagelayout}}
%
%    \begin{macrocode}
%<*pagelayout>
\NeedsTeXFormat{LaTeX2e}
\ProvidesPackage{zref-pagelayout}%
  [2020-07-03 v2.32 Module pagelayout for zref (HO)]%
\RequirePackage{zref-base}[2019/11/29]
\ifx\ZREF@base@ok Y%
\else
  \expandafter\endinput
\fi
%    \end{macrocode}
%
%    \begin{macrocode}
\RequirePackage{zref-thepage}[2019/11/29]
\RequirePackage{iftex}[2019/11/07]%
\RequirePackage{atveryend}[2010/03/24]
%    \end{macrocode}
%
%
% \subsubsection{Define layout properties}
%
%    \begin{macrocode}
\def\ZREF@temp#1{%
  \begingroup
    \escapechar=-1 %
  \ltx@ifundefined{\string#1}{\endgroup}{%
    \edef\x{%
      \endgroup
      \noexpand\zref@newprop*{\string#1}%
                             [\noexpand\number\noexpand#1]% hash-ok
                             {\noexpand\number\noexpand#1}%
      \noexpand\zref@addprop{thepage}{\string#1}%
    }%
    \x
  }%
}
\ZREF@temp\mag
\ZREF@temp\paperwidth
\ZREF@temp\paperheight
\ZREF@temp\stockwidth % memoir.cls, crop.sty
\ZREF@temp\stockheight % memoir.cls, crop.sty
\ZREF@temp\mediawidth % VTeX
\ZREF@temp\mediaheight % VTeX
\ifluatex
\zref@newprop*{pdfvorigin}%
                             [\number\pdfvariable vorigin]% hash-ok
                             {\number\pdfvariable vorigin}%
\zref@addprop{thepage}{pdfvorigin}
\zref@newprop*{pdfhorigin}%
                             [\number\pdfvariable horigin]% hash-ok
                             {\number\pdfvariable horigin}%
\zref@addprop{thepage}{pdfhorigin}
\zref@newprop*{pdfpageheight}%
                             [\number\pageheight]% hash-ok
                             {\number\pageheight}%
\zref@addprop{thepage}{pdfpageheight}
\zref@newprop*{pdfpagewidth}%
                             [\number\pagewidth]% hash-ok
                             {\number\pagewidth}%
\zref@addprop{thepage}{pdfpagewidth}
\else
\ZREF@temp\pdfpagewidth
\ZREF@temp\pdfpageheight
\ZREF@temp\pdfhorigin
\ZREF@temp\pdfvorigin
\fi
\ZREF@temp\hoffset
\ZREF@temp\voffset
\ZREF@temp\topmargin
\ZREF@temp\oddsidemargin
\ZREF@temp\evensidemargin
\ZREF@temp\textwidth
\ZREF@temp\textheight
\ZREF@temp\headheight
\ZREF@temp\headsep
\ZREF@temp\footskip
\ZREF@temp\marginparwidth
\ZREF@temp\marginparsep
\ZREF@temp\columnwidth
\ZREF@temp\columnsep
\ZREF@temp\trimedge % memoir.cls
\ZREF@temp\spinemargin % memoir.cls
\ZREF@temp\foremargin % memoir.cls
\ZREF@temp\trimtop % memoir.cls
\ZREF@temp\uppermargin % memoir.cls
\ZREF@temp\headmargin % memoir.cls
%    \end{macrocode}
%
%    \begin{macrocode}
\IfFormatAtLeastTF{2020/10/01}
 {
   \zref@newprop*{outputboxwd}[0pt]{\ShipoutBoxWidth}
   \zref@newprop*{outputboxht}[0pt]{\ShipoutBoxHeight}
   \zref@newprop*{outputboxdp}[0pt]{\ShipoutBoxDepth}
 }
 {
   \zref@newprop*{outputboxwd}[0pt]{\AtBeginShipoutBoxWidth}
   \zref@newprop*{outputboxht}[0pt]{\AtBeginShipoutBoxHeight}
   \zref@newprop*{outputboxdp}[0pt]{\AtBeginShipoutBoxDepth}
 }
%    \end{macrocode}
%    \begin{macrocode}
\zref@addprops{thepage}{outputboxwd,outputboxht,outputboxdp}
%    \end{macrocode}
%
%    \begin{macro}{\ifZREF@pl@list}
%    \begin{macrocode}
\ltx@newif\ifZREF@pl@list
%    \end{macrocode}
%    \end{macro}
%    \begin{macro}{\zref@listpagelayout}
%    \begin{macrocode}
\ZREF@IfDefinable\zlistpagelayout\def{%
  {\global\ZREF@pl@listtrue}%
}
%    \end{macrocode}
%    \end{macro}
%    \begin{macro}{\ZREF@pl@AfterLastShipout}
%    \begin{macrocode}
\def\ZREF@pl@AfterLastShipout{%
  \ifZREF@pl@list
    \edef\ZREF@page@max{\the\value{abspage}}%
    \ltx@ifundefined{ZREF@org@testdef}{%
      \let\ZREF@org@testdef\@testdef
      \def\@testdef##1##2##3{%
        \ZREF@org@testdef{##1}{##2}{##3}%
        \def\ZREF@temp{##1}%
        \ifx\ZREF@temp\ZREF@RefPrefix
          \expandafter\gdef\csname##1@##2\endcsname{##3}%
        \fi
      }%
    }{}%
    \AtVeryEndDocument{\ZREF@pl@AtVeryEnd}%
  \fi
}
%    \end{macrocode}
%    \end{macro}
%    \begin{macro}{\ZREF@pl@AtVeryEnd}
%    \begin{macrocode}
\def\ZREF@pl@AtVeryEnd{%
  \begingroup
    \toks@{Page layout parameters:\MessageBreak}%
    \count@=1 %
    \ZREF@pl@ListPage
    \edef\x{\endgroup
      \noexpand\@PackageInfoNoLine{zref-pagelayout}{\the\toks@}%
    }%
  \x
}
%    \end{macrocode}
%    \end{macro}
%    \begin{macro}{\ZREF@pl@ListPage}
%    \begin{macrocode}
\def\ZREF@pl@ListPage{%
  \edef\x{%
    \toks@={%
      \the\toks@
      Page \the\count@:\noexpand\MessageBreak
      \zref@ifrefundefined{thepage\the\count@}{}{%
        \ltx@space\ltx@space mag = %
        \zref@extract{thepage\the\count@}{mag}%
        \noexpand\MessageBreak
        \ZREF@pl@ListEntry{paperwidth}%
        \ZREF@pl@ListEntry{paperheight}%
        \ZREF@pl@ListEntry{stockwidth}%
        \ZREF@pl@ListEntry{stockheight}%
        \ZREF@pl@ListEntry{mediawidth}%
        \ZREF@pl@ListEntry{mediaheight}%
        \ZREF@pl@ListEntry{pdfpagewidth}%
        \ZREF@pl@ListEntry{pdfpageheight}%
        \ZREF@pl@ListEntry{pdfhorigin}%
        \ZREF@pl@ListEntry{pdfvorigin}%
        \ZREF@pl@ListEntry{hoffset}%
        \ZREF@pl@ListEntry{voffset}%
        \ZREF@pl@ListEntry{topmargin}%
        \ZREF@pl@ListEntry{oddsidemargin}%
        \ZREF@pl@ListEntry{evensidemargin}%
        \ZREF@pl@ListEntry{textwidth}%
        \ZREF@pl@ListEntry{textheight}%
        \ZREF@pl@ListEntry{headheight}%
        \ZREF@pl@ListEntry{headsep}%
        \ZREF@pl@ListEntry{footskip}%
        \ZREF@pl@ListEntry{marginparwidth}%
        \ZREF@pl@ListEntry{marginparsep}%
        \ZREF@pl@ListEntry{columnwidth}%
        \ZREF@pl@ListEntry{columnsep}%
        \ZREF@pl@ListEntry{trimedge}%
        \ZREF@pl@ListEntry{spinemargin}%
        \ZREF@pl@ListEntry{foremargin}%
        \ZREF@pl@ListEntry{trimtop}%
        \ZREF@pl@ListEntry{uppermargin}%
        \ZREF@pl@ListEntry{headmargin}%
      }%
    }%
  }\x
  \ifnum\ZREF@page@max>\count@
    \advance\count@ by\ltx@one
  \else
    \expandafter\ltx@gobble
  \fi
  \ZREF@pl@ListPage
}
%    \end{macrocode}
%    \end{macro}
%    \begin{macro}{\ZREF@pl@ListEntry}
%    \begin{macrocode}
\def\ZREF@pl@ListEntry#1{%
  \zref@ifpropundefined{#1}{%
  }{%
    \zref@ifrefcontainsprop{thepage\the\count@}{#1}{%
      \ltx@space\ltx@space#1 = %
      \zref@extract{thepage\the\count@}{#1}sp = %
      \the\dimexpr\zref@extract{thepage\the\count@}{#1}sp\relax
      \noexpand\MessageBreak
    }{}%
  }%
}
%    \end{macrocode}
%    \end{macro}
%    \begin{macrocode}
\AfterLastShipout{%
  \ZREF@pl@AfterLastShipout
}
%    \end{macrocode}
%
%    \begin{macrocode}
%</pagelayout>
%    \end{macrocode}
%
%
% \subsection{Module \xmodule{pageattr}}
%
%    \begin{macrocode}
%<*pageattr>
\NeedsTeXFormat{LaTeX2e}
\ProvidesPackage{zref-pageattr}%
  [2020-07-03 v2.32 Module pageattr for zref (HO)]%
\RequirePackage{zref-base}[2019/11/29]
\ifx\ZREF@base@ok Y%
\else
  \expandafter\endinput
\fi
%    \end{macrocode}
%
%    \begin{macrocode}
\RequirePackage{iftex}[2019/11/07]%
%    \end{macrocode}
%
%
%    \begin{macrocode}
\let\ZREF@temp=N%
\ifluatex
\expandafter\@firstoftwo
\else
\expandafter\@secondoftwo
\fi
{%luatex
 \RequirePackage{zref-thepage}[2019/11/29]
 \RequirePackage{zref-lastpage}[2019/11/29]%
 \zref@newprop*{pdfpageattr}[]{\zref@hex{\the\pdfvariable pageattr}}%
 \zref@addprop{thepage}{pdfpageattr}%
 \zref@newprop*{pdfpagesattr}[]{\zref@hex{\the\pdfvariable pagesattr}}%
 \zref@addprop{LastPage}{pdfpagesattr}%
 \let\ZREF@temp=Y%
}
{%other
 \ltx@IfUndefined{pdfpageattr}{%
  \@PackageInfoNoLine{zref-pageattr}{%
    \string\pdfpageattr\space is not available%
  }%
  \def\zref@pdfpageattr#1{}%
  \def\zref@pdfpageattr@used#1{}%
 }{%
  \RequirePackage{zref-thepage}[2019/11/29]%
  \zref@newprop*{pdfpageattr}[]{\zref@hex{\the\pdfpageattr}}%
  \zref@addprop{thepage}{pdfpageattr}%
  \let\ZREF@temp=Y%
 }
 \ltx@IfUndefined{pdfpagesattr}{%
  \@PackageInfoNoLine{zref-pageattr}{%
    \string\pdfpagesattr\space is not available%
  }%
  \def\zref@pdfpagesattr{}%
  \def\zref@pdfpagesattr@used{}%
 }{%
  \RequirePackage{zref-lastpage}[2019/11/29]%
  \zref@newprop*{pdfpagesattr}[]{\zref@hex{\the\pdfpagesattr}}%
  \zref@addprop{LastPage}{pdfpagesattr}%
  \let\ZREF@temp=Y%
 }%
}%
%    \end{macrocode}
%    \begin{macrocode}
\ifx\ZREF@temp N%
  \expandafter\endinput
\fi
%    \end{macrocode}
%    \begin{macrocode}
\RequirePackage{zref-abspage}[2019/11/29]
\RequirePackage{atveryend}[2010/03/24]
\RequirePackage{pdftexcmds}[2010/04/01]
\let\ZREF@temp=Y%
\ltx@IfUndefined{pdf@escapehex}{\let\ZREF@temp=N}{}
\ltx@IfUndefined{pdf@unescapehex}{\let\ZREF@temp=N}{}
\ifx\ZREF@temp N%
  \let\zref@hex\ltx@firstofone
  \let\zref@unhex\ltx@firstofone
\else
  \let\zref@hex\pdf@escapehex
  \let\zref@unhex\pdf@unescapehex
\fi
%    \end{macrocode}
%
%    \begin{macro}{\ifZREF@pa@list}
%    \begin{macrocode}
\ltx@newif\ifZREF@pa@list
%    \end{macrocode}
%    \end{macro}
%    \begin{macro}{\zref@listpageattr}
%    \begin{macrocode}
\ZREF@IfDefinable\zlistpageattr\def{%
  {\ZREF@pa@listtrue}%
}
%    \end{macrocode}
%    \end{macro}
%    \begin{macro}{\ZREF@pa@AfterLastShipout}
%    \begin{macrocode}
\def\ZREF@pa@AfterLastShipout{%
  \ifZREF@pa@list
    \edef\ZREF@page@max{\the\value{abspage}}%
    \ltx@ifundefined{ZREF@org@testdef}{%
      \let\ZREF@org@testdef\@testdef
      \def\@testdef##1##2##3{%
        \ZREF@org@testdef{##1}{##2}{##3}%
        \def\ZREF@temp{##1}%
        \ifx\ZREF@temp\ZREF@RefPrefix
          \expandafter\xdef\csname##1@##2\endcsname{##3}%
        \fi
      }%
    }{}%
    \AtVeryEndDocument{\ZREF@pa@AtVeryEnd}%
  \fi
}
%    \end{macrocode}
%    \end{macro}
%    \begin{macro}{\ZREF@pa@AtVeryEnd}
%    \begin{macrocode}
\let\ZREF@temp=Y%
\ltx@IfUndefined{pdfpageattr}{}{\let\ZREF@temp=N}
\ifluatex \let\ZREF@temp=N \fi
\ifx\ZREF@temp Y
 \expandafter\@firstoftwo
\else
 \expandafter\@secondoftwo
\fi
 {%
  \def\ZREF@pa@AtVeryEnd{}%
 }
 {%
  \def\ZREF@pa@AtVeryEnd{%
    \begingroup
      \toks@{List of \ltx@backslashchar
             \ifluatex pdfvariable\else pdf\fi
             pdfpageattr:\MessageBreak}%
      \count@=1 %
      \ZREF@pa@ListPage
      \edef\x{\endgroup
        \noexpand\@PackageInfoNoLine{zref-pageattr}{%
          \the\toks@
        }%
      }%
    \x
  }%
%    \end{macrocode}
%    \end{macro}
%    \begin{macro}{\zref@pageattr}
%    \begin{macrocode}
\def\zref@pdfpageattr#1{%
  \zref@unhex{%
    \zref@extract{thepage\ZREF@number{#1}}{pdfpageattr}%
  }%
}
% compability, \zref@pageattr was defined in older versions
\let\zref@pageattr\zref@pdfpageattr
%    \end{macrocode}
%    \end{macro}
%    \begin{macro}{\zref@pageattr@used}
%    \begin{macrocode}
\ZREF@Robust\def\zref@pageattr@used#1{%
  \zref@refused{thepage\ZREF@number{#1}}%
}
%    \end{macrocode}
%    \end{macro}
%    \begin{macro}{\ZREF@pa@ListPage}
%    \begin{macrocode}
  \def\ZREF@pa@ListPage{%
    \edef\x{%
      \toks@={%
        \the\toks@
        Page \the\count@:%
        \noexpand\MessageBreak
        \zref@ifrefundefined{thepage\the\count@}{}{%
          <<\zref@pdfpageattr\count@>>%
          \noexpand\MessageBreak
        }%
      }%
    }\x
    \ifnum\ZREF@page@max>\count@
      \advance\count@ by\ltx@one
    \else
      \expandafter\ltx@gobble
    \fi
    \ZREF@pa@ListPage
  }%
}
%    \end{macrocode}
%    \end{macro}
%    \begin{macrocode}
\let\ZREF@temp=Y%
\ltx@IfUndefined{pdfpagesattr}{}{\let\ZREF@temp=N}
\ifluatex \let\ZREF@temp=N \fi
\ifx\ZREF@temp N
 \expandafter\@firstofone
\fi
 {%
%    \end{macrocode}
%    \begin{macro}{\zref@pdfpagesattr}
%    \begin{macrocode}
  \def\zref@pdfpagesattr{%
    \zref@unhex{%
      \zref@extract{LastPage}{pdfpagesattr}%
    }%
  }%
%    \end{macrocode}
%    \end{macro}
%    \begin{macro}{\zref@pdfpagesattr@used}
%    \begin{macrocode}
  \ZREF@Robust\def\zref@pdfpagesattr@used{%
    \zref@refused{LastPage}%
  }%
%    \end{macrocode}
%    \end{macro}
%    \begin{macrocode}
  \ltx@LocalAppendToMacro\ZREF@pa@AtVeryEnd{%
    \@PackageInfoNoLine{zref-pageattr}{%
      \ltx@backslashchar
      \ifluatex pdfvariable\else pdf\fi
      pagesattr:\MessageBreak
      <<\zref@pdfpagesattr>>%
      \MessageBreak
    }%
  }%
}
\AfterLastShipout{%
  \ZREF@pa@AfterLastShipout
}
%    \end{macrocode}
%    \begin{macrocode}
%</pageattr>
%    \end{macrocode}
%
%
% \subsection{Module \xmodule{marks}}
%
%    \begin{macrocode}
%<*marks>
\NeedsTeXFormat{LaTeX2e}
\ProvidesPackage{zref-marks}%
  [2020-07-03 v2.32 Module marks for zref (HO)]%
\RequirePackage{zref-base}[2019/11/29]
\ifx\ZREF@base@ok Y%
\else
  \expandafter\endinput
\fi
%    \end{macrocode}
%
%    \begin{macrocode}
\newcommand*{\zref@marks@register}[3][]{%
  \edef\ZREF@TempName{#1}%
  \edef\ZREF@TempNum{\ZREF@number{#2}}%
  \ifnum\ZREF@TempNum<\ltx@zero %
    \PackageError\ZREF@name{%
      \string\zref@marks@register\ltx@space is called with invalid%
      \MessageBreak
      marks register number (\ZREF@TempNum)%
    }{%
      Use `0' or the command, defined by \string\newmarks.\MessageBreak
      \@ehc
    }%
  \else
    \ifx\ZREF@TempName\ltx@empty
      \edef\ZREF@TempName{mark\romannumeral\ZREF@TempNum}%
    \else
      \edef\ZREF@TempName{marks\ZREF@TempName}%
    \fi
    \ZREF@MARKS@DefineProp{top}%
    \ZREF@MARKS@DefineProp{first}%
    \ZREF@MARKS@DefineProp{bot}%
    \kv@parse{#3}{%
      \ifx\kv@value\relax
        \def\kv@value{top,first,bot}%
      \fi
      \edef\ZREF@temp{\expandafter\ltx@car\kv@key X\@nil}%
      \ifx\ZREF@temp\ZREF@STAR
        \edef\kv@key{\expandafter\ltx@cdr\kv@key\@nil}%
        \zref@newlist\kv@key
      \fi
      \expandafter\comma@parse\expandafter{\kv@value}{%
        \ifcase0\ifx\comma@entry\ZREF@NAME@top 1\else
                \ifx\comma@entry\ZREF@NAME@first 1\else
                \ifx\comma@entry\ZREF@NAME@bot 1\fi\fi\fi\ltx@space
          \PackageWarning{zref-marks}{%
            Use `top', `first' or `bot' for the list values%
            \MessageBreak
            in the third argument of \string\zref@marks@register.%
            \MessageBreak
            Ignoring unkown value `\comma@entry'%
          }%
        \else
          \zref@addprop{\kv@key}{\comma@entry\ZREF@TempName}%
        \fi
        \ltx@gobble
      }%
      \ltx@gobbletwo
    }%
  \fi
}
\def\ZREF@STAR{*}
\def\ZREF@NAME@top{top}
\def\ZREF@NAME@first{first}
\def\ZREF@NAME@bot{bot}
\def\ZREF@MARKS@DefineProp#1{%
  \zref@ifpropundefined{#1\ZREF@TempName}{%
    \ifnum\ZREF@TempNum=\ltx@zero
      \begingroup
        \edef\x{\endgroup
          \noexpand\zref@newprop*{#1\ZREF@TempName}[]{%
            \expandafter\noexpand\csname#1mark\endcsname
          }%
        }%
      \x
    \else
      \begingroup
        \edef\x{\endgroup
          \noexpand\zref@newprop*{#1\ZREF@TempName}[]{%
            \expandafter\noexpand\csname#1marks\endcsname
            \ZREF@TempNum
          }%
        }%
      \x
    \fi
  }{%
    \PackageWarning{zref-marks}{%
      \string\zref@marks@register\ltx@space does not generate the%
      \MessageBreak
      new property `#1\ZREF@TempName', because\MessageBreak
      it is already defined%
    }%
  }%
}
%    \end{macrocode}
%    \begin{macrocode}
%</marks>
%    \end{macrocode}
%
% \subsection{Module \xmodule{runs}}
%
%    This module does not use the label-reference-system.
%    The reference changes with each \hologo{LaTeX}~run and would
%    force a rerun warning always.
%    \begin{macrocode}
%<*runs>
\NeedsTeXFormat{LaTeX2e}
\ProvidesPackage{zref-runs}%
  [2020-07-03 v2.32 Module runs for zref (HO)]%
%    \end{macrocode}
%
%    \begin{macro}{\zruns}
%    \begin{macrocode}
\providecommand*{\zruns}{0}%
\AtBeginDocument{%
  \edef\zruns{\number\numexpr\zruns+1}%
  \begingroup
    \def\on@line{}%
    \PackageInfo{zref-runs}{LaTeX runs: \zruns}%
    \if@filesw
      \immediate\write\@mainaux{%
        \string\gdef\string\zruns{\zruns}%
      }%
    \fi
  \endgroup
}
%    \end{macrocode}
%    \end{macro}
%
%    \begin{macrocode}
%</runs>
%    \end{macrocode}
%
%
% \subsection{Module \xmodule{perpage}}
%
%    \begin{macrocode}
%<*perpage>
\NeedsTeXFormat{LaTeX2e}
\ProvidesPackage{zref-perpage}%
  [2020-07-03 v2.32 Module perpage for zref (HO)]%
\RequirePackage{zref-base}[2019/11/29]
\ifx\ZREF@base@ok Y%
\else
  \expandafter\endinput
\fi
%    \end{macrocode}
%
%    This module resets a counter at page boundaries.
%    Because of the asynchronous output routine
%    page counter properties cannot be asked directly,
%    references are necessary.
%
%    For detecting changed pages module \xmodule{abspage}
%    is loaded.
%    \begin{macrocode}
\RequirePackage{zref-abspage}[2019/11/29]
%    \end{macrocode}
%
%    We group the properties for the needed references
%    in the property list |perpage|. The property |pagevalue|
%    records the correct value of the page counter.
%    \begin{macrocode}
\ZREF@NewPropPageValue
\zref@newlist{perpage}
\zref@addprops{perpage}{abspage,page,pagevalue}
%    \end{macrocode}
%    The page value, known by the reference mechanism, will be
%    stored in counter |zpage|.
%    \begin{macrocode}
\newcounter{zpage}
%    \end{macrocode}
%    Counter |zref@unique| helps in generating unique reference
%    names.
%    \begin{macrocode}
\zref@require@unique
%    \end{macrocode}
%    In order to be able to reset the counter, we hook here
%    into \cs{stepcounter}. In fact two nested hooks are used
%    to allow other packages to use the first hook at the beginning
%    of \cs{stepcounter}.
%    \begin{macrocode}
\let\ZREF@org@stepcounter\stepcounter
\def\stepcounter#1{%
  \ifcsname @stepcounterhook@#1\endcsname
    \csname @stepcounterhook@#1\endcsname
  \fi
  \ZREF@org@stepcounter{#1}%
}
%    \end{macrocode}
%    \cs{@stpelt} must be adapted due to the change in latex
%    2015-01, see https://github.com/ho-tex/zref/issues/26
%    \begin{macrocode}
\let\ZREF@org@@stpelt\@stpelt
\def\@stpelt#1{%
  \ifcsname ZREF@perpage@#1\endcsname
    \begingroup
      \let\stepcounter\ZREF@org@stepcounter
      \ZREF@org@@stpelt{#1}%
    \endgroup
    \expandafter\ltx@gobbletwo
  \fi
  \ZREF@org@@stpelt{#1}%
}
%    \end{macrocode}
%
%    \begin{macro}{\zmakeperpage}
%    Makro \cs{zmakeperpage} resets a counter at each page
%    break. It uses the same syntax and semantics as
%    \cs{MakePerPage} from package \xpackage{perpage} \cite{perpage}.
%    The initial start value can be given by the optional
%    argument. Default is one that means after the first
%    \cs{stepcounter} on a new page the counter starts with
%    one.
%    \begin{macrocode}
\ZREF@IfDefinable\zmakeperpage\def{%
  {%
    \@ifnextchar[\ZREF@makeperpage@opt{\ZREF@@makeperpage[\ltx@zero]}%
  }%
}
%    \end{macrocode}
%    \end{macro}
%    We hook before the counter is incremented in \cs{stepcounter},
%    package \xpackage{perpage} afterwards. Thus a little
%    calculation is necessary.
%    \begin{macrocode}
\def\ZREF@makeperpage@opt[#1]{%
  \begingroup
    \edef\x{\endgroup
      \noexpand\ZREF@@makeperpage[\number\numexpr#1-1\relax]%
    }%
  \x
}
%    \end{macrocode}
%    \begin{macrocode}
\def\ZREF@@makeperpage[#1]#2{%
  \@ifundefined{@stepcounterhook@#2}{%
    \expandafter\gdef\csname @stepcounterhook@#2\endcsname{}%
  }{}%
  \expandafter\gdef\csname ZREF@perpage@#2\endcsname{%
    \ZREF@@perpage@step{#2}{#1}%
  }%
  \expandafter\g@addto@macro\csname @stepcounterhook@#2\endcsname{%
    \ifcsname ZREF@perpage@#2\endcsname
      \csname ZREF@perpage@#2\endcsname
    \fi
  }%
}
%    \end{macrocode}
%    \begin{macro}{\ZREF@@perpage@step}
%    The heart of this module follows.
%    \begin{macrocode}
\def\ZREF@@perpage@step#1#2{%
%    \end{macrocode}
%    First the reference is generated.
%    \begin{macrocode}
  \global\advance\c@zref@unique\ltx@one
  \begingroup
    \expandafter
    \zref@labelbylist\expandafter{\thezref@unique}{perpage}%
%    \end{macrocode}
%    The \cs{expandafter} commands are necessary, because \cs{ZREF@temp}
%    is also used inside of \cs{zref@labelbylist}.
%
%    The evaluation of the reference follows. If the reference
%    is not yet kwown, we use the page counter as approximation.
%    \begin{macrocode}
    \zref@ifrefundefined\thezref@unique{%
      \global\c@zpage=\c@page
      \global\let\thezpage\thepage
      \expandafter\xdef\csname ZREF@abspage@#1\endcsname{%
        \number\c@abspage
      }%
    }{%
%    \end{macrocode}
%    The reference is used to set \cs{thezpage} and counter |zpage|.
%    \begin{macrocode}
      \global\c@zpage=\zref@extract\thezref@unique{pagevalue}\relax
      \xdef\thezpage{\noexpand\zref@extract{\thezref@unique}{page}}%
      \expandafter\xdef\csname ZREF@abspage@#1\endcsname{%
        \zref@extractdefault\thezref@unique
            {abspage}{\number\c@abspage}%
      }%
    }%
%    \end{macrocode}
%    Page changes are detected by a changed absolute page number.
%    \begin{macrocode}
    \expandafter\ifx\csname ZREF@abspage@#1\expandafter\endcsname
                    \csname ZREF@currentabspage@#1\endcsname
    \else
      \global\csname c@#1\endcsname=#2\relax
      \global\expandafter\let
          \csname ZREF@currentabspage@#1\expandafter\endcsname
          \csname ZREF@abspage@#1\endcsname
    \fi
  \endgroup
}
%    \end{macrocode}
%    \end{macro}
%    \begin{macro}{\zunmakeperpage}
%    Macro \cs{zunmakeperpage} cancels the effect of \cs{zmakeperpage}.
%    \begin{macrocode}
\ZREF@IfDefinable\zunmakeperpage\def{%
  #1{%
    \global\expandafter
    \let\csname ZREF@perpage@#1\endcsname\@undefined
  }%
}
%    \end{macrocode}
%    \end{macro}
%
%    \begin{macrocode}
%</perpage>
%    \end{macrocode}
%
%
% \subsection{Module \xmodule{titleref}}
%
%    \begin{macrocode}
%<*titleref>
\NeedsTeXFormat{LaTeX2e}
\ProvidesPackage{zref-titleref}%
  [2020-07-03 v2.32 Module titleref for zref (HO)]%
\RequirePackage{zref-base}[2019/11/29]
\ifx\ZREF@base@ok Y%
\else
  \expandafter\endinput
\fi
\RequirePackage{gettitlestring}[2009/12/08]
%    \end{macrocode}
%
% \subsubsection{Implementation}
%    \begin{macrocode}
\RequirePackage{keyval}
%    \end{macrocode}
%
%    This module makes section and caption titles available
%    for the reference system. It uses some of the ideas
%    of package \xpackage{nameref} and \xpackage{titleref}.
%
%    Now we can add the property \verb|title| is added to
%    the main property list.
%    \begin{macrocode}
\ZREF@NewPropTitle
\zref@addprop\ZREF@mainlist{title}%
%    \end{macrocode}
%
%    The title strings go into the \xfile{.aux} file, thus
%    they need some kind of protection.
%    Package \xpackage{titleref} uses a protected expansion
%    method. The advantage is that this can be used to
%    cleanup the string and to remove \cs{label}, \cs{index}
%    and other macros unwanted for referencing.
%    But there is the risk that fragile stuff can break.
%
%    Therefore package \xpackage{nameref} does not expand
%    the string. Thus the entries can safely be written to
%    the \xfile{.aux} file. But potentially dangerous
%    macros such as \cs{label} remain in the string and
%    can cause problems when using the string in references.
%
%    \begin{macro}{\ifzref@titleref@expand}
%    The switch \cs{ifzref@titleref@expand} distinguishes
%    between the both methods. Package \xpackage{nameref}'s
%    behaviour is achieved by setting the switch to false,
%    otherwise \xpackage{titleref}'s expansion is used.
%    Default is false.
%    \begin{macrocode}
\newif\ifzref@titleref@expand
%    \end{macrocode}
%    \end{macro}
%
%    \begin{macro}{\ZREF@titleref@hook}
%    The hook \cs{ZREF@titleref@hook} allows to extend
%    the cleanup for the expansion method. Thus
%    unnecessary macros can be removed or dangerous commands
%    removed. The hook is executed before the expansion of
%    \cs{zref@titleref@current}.
%    \begin{macrocode}
\let\ZREF@titleref@hook\ltx@empty
%    \end{macrocode}
%    \end{macro}
%    \begin{macro}{\zref@titleref@cleanup}
%    The hook should not be used directly, instead we provide the
%    macro \cs{zref@titleref@cleanup} to add stuff to the hook
%    and prevents that a previous non-empty content is not discarded
%    accidently.
%    \begin{macrocode}
\ZREF@Robust\def\zref@titleref@cleanup#1{%
   \begingroup
    \toks@\expandafter{%
      \ZREF@titleref@hook
      #1%
    }%
  \expandafter\endgroup
  \expandafter\def\expandafter\ZREF@titleref@hook\expandafter{%
    \the\toks@
  }%
}%
%    \end{macrocode}
%    \end{macro}
%
%    \begin{macro}{\ifzref@titleref@stripperiod}
%    Sometimes a title contains a period at the end.
%    Package \xpackage{nameref} removes this. This behaviour
%    is controlled by the switch \cs{ifzref@titleref@stripperiod}
%    and works regardless of the setting of option \xoption{expand}.
%    Period stripping is the default.
%    \begin{macrocode}
\newif\ifzref@titleref@stripperiod
\zref@titleref@stripperiodtrue
%    \end{macrocode}
%    \end{macro}
%
%    \begin{macro}{\zref@titleref@setcurrent}
%    Macro \cs{zref@titleref@setcurrent} sets a new current title
%    stored in \cs{zref@titleref@current}. Some cleanup and
%    expansion is performed that can be controlled by the
%    previous switches.
%    \begin{macrocode}
\ZREF@Robust\def\zref@titleref@setcurrent#1{%
  \ifzref@titleref@expand
    \GetTitleStringExpand{#1}%
  \else
    \GetTitleStringNonExpand{#1}%
  \fi
  \edef\zref@titleref@current{%
    \detokenize\expandafter{\GetTitleStringResult}%
  }%
  \ifzref@titleref@stripperiod
    \edef\zref@titleref@current{%
      \expandafter\ZREF@stripperiod\zref@titleref@current
      \ltx@empty.\ltx@empty\@nil
    }%
  \fi
}%
\GetTitleStringDisableCommands{%
  \ZREF@titleref@hook
}
%    \end{macrocode}
%    \end{macro}
%    \begin{macro}{\ZREF@stripperiod}
%    If \cs{ZREF@stripperiod} is called, the argument consists of
%    space tokens and tokens with catcode 12 (other), because
%    of \hologo{eTeX}'s \cs{detokenize}.
%    \begin{macrocode}
\def\ZREF@stripperiod#1.\ltx@empty#2\@nil{#1}%
%    \end{macrocode}
%    \end{macro}
%
% \subsubsection{User interface}
%
%    \begin{macro}{\ztitlerefsetup}
%    The behaviour of module \xmodule{titleref} is controlled
%    by switches and a hook. They can be set by \cs{ztitlerefsetup}
%    with a key value interface, provided by package \xpackage{keyval}.
%    Also the current title can be given explicitly by the key
%    \verb|title|.
%    \begin{macrocode}
\define@key{ZREF@TR}{expand}[true]{%
  \csname zref@titleref@expand#1\endcsname
}%
\define@key{ZREF@TR}{stripperiod}[true]{%
  \csname zref@titleref@stripperiod#1\endcsname
}%
\define@key{ZREF@TR}{cleanup}{%
  \zref@titleref@cleanup{#1}%
}%
\define@key{ZREF@TR}{title}{%
  \def\zref@titleref@current{#1}%
}%
\ZREF@IfDefinable\ztitlerefsetup\def{%
  {\kvsetkeys{ZREF@TR}}%
}%
%    \end{macrocode}
%    \end{macro}
%
%    \begin{macro}{\ztitleref}
%    The user command \cs{ztitleref} references the title.
%    For safety \cs{label} is disabled to prevent multiply
%    defined references.
%    \begin{macrocode}
\ZREF@IfDefinable\ztitleref\def{%
  {\zref@wrapper@babel\ZREF@titleref}%
}%
\def\ZREF@titleref#1{%
  \begingroup
    \zref@refused{#1}%
    \let\label\ltx@gobble
    \zref@extract{#1}{title}%
  \endgroup
}%
%    \end{macrocode}
%    \end{macro}
%
% \subsubsection{Patches for section and caption commands}
%
%    The section and caption macros are patched to extract the
%    title data.
%
%    Captions of figures and tables.
%    \begin{macrocode}
\AtBeginDocument{%
  \ZREF@patch{@caption}{%
    \long\def\@caption#1[#2]{%
      \zref@titleref@setcurrent{#2}%
      \ZREF@org@@caption{#1}[{#2}]%
    }%
  }%
%    \end{macrocode}
%
%    Section commands without star. The title version for the
%    table of contents is used because it is usually shorter and
%    more robust.
%    \begin{macrocode}
  \ZREF@patch{@part}{%
    \def\@part[#1]{%
      \zref@titleref@setcurrent{#1}%
      \ZREF@org@@part[{#1}]%
    }%
  }%
  \ZREF@patch{@chapter}{%
    \def\@chapter[#1]{%
      \zref@titleref@setcurrent{#1}%
      \ZREF@org@@chapter[{#1}]%
    }%
  }%
  \ZREF@patch{@sect}{%
    \def\@sect#1#2#3#4#5#6[#7]{%
      \zref@titleref@setcurrent{#7}%
      \ZREF@org@@sect{#1}{#2}{#3}{#4}{#5}{#6}[{#7}]%
    }%
  }%
%    \end{macrocode}
%    The star versions of the section commands.
%    \begin{macrocode}
  \ZREF@patch{@spart}{%
    \def\@spart#1{%
      \zref@titleref@setcurrent{#1}%
      \ZREF@org@@spart{#1}%
    }%
  }%
  \ZREF@patch{@schapter}{%
    \def\@schapter#1{%
      \zref@titleref@setcurrent{#1}%
      \ZREF@org@@schapter{#1}%
    }%
  }%
  \ZREF@patch{@ssect}{%
    \def\@ssect#1#2#3#4#5{%
      \zref@titleref@setcurrent{#5}%
      \ZREF@org@@ssect{#1}{#2}{#3}{#4}{#5}%
    }%
  }%
%    \end{macrocode}
%
% \subsubsection{Environment \textsf{description}}
%
%    \begin{macrocode}
  \ZREF@patch{descriptionlabel}{%
    \def\descriptionlabel#1{%
      \zref@titleref@setcurrent{#1}%
      \ZREF@org@descriptionlabel{#1}%
    }%
  }%
%    \end{macrocode}
%
% \subsubsection{Class \xclass{memoir}}
%
%    \begin{macrocode}
  \@ifclassloaded{memoir}{%
    \ltx@IfUndefined{ifheadnameref}{}{%
      \def\@chapter[#1]#2{%
        \ltx@IfUndefined{ch@pt@c}{%
          \zref@titleref@setcurrent{#1}%
        }{%
          \ifx\ch@pt@c\ltx@empty
            \zref@titleref@setcurrent{#2}%
          \else
            \def\NR@temp{#1}%
            \ifx\NR@temp\ltx@empty
              \expandafter\zref@titleref@setcurrent
              \expandafter{\ch@pt@c}%
            \else
              \ifheadnameref
                \zref@titleref@setcurrent{#1}%
              \else
                \expandafter\zref@titleref@setcurrent
                \expandafter{\ch@pt@c}%
              \fi
            \fi
          \fi
        }%
        \ZREF@org@@chapter[{#1}]{#2}%
      }%
      \ZREF@patch{M@sect}{%
        \def\M@sect#1#2#3#4#5#6[#7][#8]{%
          \ifheadnameref
            \zref@titleref@setcurrent{#8}%
          \else
            \zref@titleref@setcurrent{#7}%
          \fi
          \ZREF@org@M@sect{#1}{#2}{#3}{#4}{#5}{#6}[{#7}][{#8}]%
        }%
      }%
    }%
  }{}%
%    \end{macrocode}
%
% \subsubsection{Class \xclass{beamer}}
%
%    \begin{macrocode}
  \@ifclassloaded{beamer}{%
    \ZREF@patch{beamer@section}{%
      \long\def\beamer@section[#1]{%
        \zref@titleref@setcurrent{#1}%
        \ZREF@org@beamer@section[{#1}]%
      }%
    }%
    \ZREF@patch{beamer@subsection}{%
      \long\def\beamer@subsection[#1]{%
        \zref@titleref@setcurrent{#1}%
        \ZREF@org@beamer@subsection[{#1}]%
      }%
    }%
    \ZREF@patch{beamer@subsubsection}{%
      \long\def\beamer@subsubsection[#1]{%
        \zref@titleref@setcurrent{#1}%
        \ZREF@org@beamer@subsubsection[{#1}]%
      }%
    }%
  }{}%
%    \end{macrocode}
%
% \subsubsection{Package \xpackage{titlesec}}
%
%    \begin{macrocode}
  \@ifpackageloaded{titlesec}{%
    \ZREF@patch{ttl@sect@i}{%
      \def\ttl@sect@i#1#2[#3]#4{%
        \zref@titleref@setcurrent{#4}%
        \ZREF@org@ttl@sect@i{#1}{#2}[{#3}]{#4}%
      }%
    }%
    \ZREF@patch{ttl@straight@i}{%
      \def\ttl@straight@i#1[#2]#3{%
        \def\ZREF@temp{#2}%
        \ifx\ZREF@temp\ltx@empty
          \zref@titleref@setcurrent{#3}%
        \else
          \zref@titleref@setcurrent{#2}%
        \fi
        \ZREF@org@ttl@straight@i{#1}[{#2}]{#3}%
      }%
    }%
  }{}%
%    \end{macrocode}
%
% \subsubsection{Package \xpackage{longtable}}
%
%    Package \xpackage{longtable}: some support for its \cs{caption}.
%    However \cs{label} inside the caption is not supported.
%    \begin{macrocode}
  \@ifpackageloaded{longtable}{%
    \ZREF@patch{LT@c@ption}{%
      \def\LT@c@ption#1[#2]#3{%
        \ZREF@org@LT@c@ption{#1}[{#2}]{#3}%
        \zref@titleref@setcurrent{#2}%
      }%
    }%
  }{}%
%    \end{macrocode}
%
% \subsubsection{Package \xpackage{listings}}
%
%    Package \xpackage{listings}: support for its caption.
%    \begin{macrocode}
  \@ifpackageloaded{listings}{%
    \ZREF@patch{lst@MakeCaption}{%
      \def\lst@MakeCaption{%
        \ifx\lst@label\ltx@empty
        \else
          \expandafter\zref@titleref@setcurrent\expandafter{%
            \lst@@caption
          }%
        \fi
        \ZREF@org@lst@MakeCaption
      }%
    }%
  }{}%
%    \end{macrocode}
%
% \subsubsection{Theorems}
%
%    \begin{macrocode}
  \ZREF@patch{@opargbegintheorem}{%
    \def\@opargbegintheorem#1#2#3{%
      \zref@titleref@setcurrent{#3}%
      \ZREF@org@@opargbegintheorem{#1}{#2}{#3}%
    }%
  }%
  \@ifpackageloaded{amsthm}{%
    \begingroup
      \edef\x{macro:\string#1\string#2[\string#3]}%
      \@onelevel@sanitize\x
      \def\y#1->#2\@nil{#1}%
      \edef\z{\expandafter\y\meaning\@begintheorem->\@nil}%
      \@onelevel@sanitize\z
    \expandafter\endgroup
    \ifx\x\z
      \ZREF@patch{@begintheorem}{%
        \def\@begintheorem#1#2[#3]{%
          \zref@titleref@setcurrent{#3}%
          \ZREF@org@@begintheorem{#1}{#2}[{#3}]%
        }%
      }%
    \fi
  }{}%
%    \end{macrocode}
%
%    \begin{macrocode}
}
%    \end{macrocode}
%
%    \begin{macrocode}
%</titleref>
%    \end{macrocode}
%
%
% \subsection{Module \xmodule{xr}}
%
%    \begin{macrocode}
%<*xr>
\NeedsTeXFormat{LaTeX2e}
\ProvidesPackage{zref-xr}%
  [2020-07-03 v2.32 Module xr for zref (HO)]%
\RequirePackage{zref-base}[2019/11/29]
\ifx\ZREF@base@ok Y%
\else
  \expandafter\endinput
\fi
%    \end{macrocode}
%    \begin{macrocode}
\RequirePackage{keyval}
\RequirePackage{kvoptions}[2010/02/22]
%    \end{macrocode}
%
%    We declare property |url|, because this is added,
%    if a reference is imported and has not already set this field.
%    Or if |hyperref| is used, then this property can be asked.
%    \begin{macrocode}
\zref@newprop{url}{}%
\zref@newprop{urluse}{}%
%    \end{macrocode}
%    \begin{macrocode}
\zref@newprop{externaldocument}{}%
%    \end{macrocode}
%
%    Most code, especially the handling of the \xfile{.aux}
%    files are taken from David Carlisle's \xpackage{xr} package.
%    Therefore I drop the documentation for these macros here.
%
%    \begin{macro}{\zref@xr@ext}
%    If the URL is not specied, then assume
%    processed file with a guessed extension.
%    Use the setting of \xpackage{hyperref} if available.
%    \begin{macrocode}
\providecommand*{\zref@xr@ext}{%
  \ltx@ifundefined{XR@ext}{pdf}{\XR@ext}%
}%
%    \end{macrocode}
%    \end{macro}
%    \begin{macro}{\ifZREF@xr@zreflabel}
%    The use of the star form of \cs{zexternaldocument}
%    is remembered in the switch \cs{ifZREF@xr@zreflabel}.
%    \begin{macrocode}
\newif\ifZREF@xr@zreflabel
%    \end{macrocode}
%    \end{macro}
%    \begin{macrocode}
\SetupKeyvalOptions{%
  family=ZREF@XR,%
  prefix=ZREF@xr@%
}
\DeclareBoolOption[true]{tozreflabel}
\DeclareBoolOption[false]{toltxlabel}
\DeclareBoolOption{verbose}
\define@key{ZREF@XR}{ext}{%
  \def\zref@xr@{#1}%
}
\DeclareBoolOption[false]{urluse}
%    \end{macrocode}
%    \begin{macro}{\zxrsetup}
%    \begin{macrocode}
\newcommand*{\zxrsetup}{%
  \kvsetkeys{ZREF@XR}%
}%
%    \end{macrocode}
%    \end{macro}
%
%    \begin{macro}{\ZREF@xr@URL}
%    \begin{macrocode}
\newcount\ZREF@xr@URL
\ZREF@xr@URL=\ltx@zero
%    \end{macrocode}
%    \end{macro}
%    \begin{macro}{\ZREF@xr@AddURL}
%    \begin{macrocode}
\def\ZREF@xr@AddURL#1{%
  \begingroup
    \def\ZREF@temp{#1}%
    \count@=\ltx@one
    \ZREF@xr@@AddUrl
  \endgroup
}
%    \end{macrocode}
%    \end{macro}
%    \begin{macro}{\ZREF@xr@@AddUrl}
%    \begin{macrocode}
\def\ZREF@xr@@AddUrl{%
  \ifnum\count@>\ZREF@xr@URL
    \global\advance\ZREF@xr@URL by\ltx@one
    \xdef\ZREF@xr@theURL{\romannumeral\ZREF@xr@URL}%
    \global\expandafter\let
        \csname Z@U@\ZREF@xr@theURL\endcsname\ZREF@temp
    \@PackageInfo{zref-xr}{%
      \ltx@backslashchar Z@U@\ZREF@xr@theURL:\MessageBreak
      \ZREF@temp\MessageBreak
    }%
  \else
    \expandafter
    \ifx\csname Z@U@\romannumeral\count@\endcsname\ZREF@temp
      \xdef\ZREF@xr@theURL{\romannumeral\count@}%
    \else
      \expandafter\expandafter\expandafter\ZREF@xr@@AddUrl
    \fi
  \fi
}
%    \end{macrocode}
%    \end{macro}
%
%    \begin{macro}{\zexternaldocument}
%    In its star form it looks for \cs{newlabel},
%    otherwise for \cs{zref@newlabel}.
%     Later we
%    will read \xfile{.aux} files that expects
%    |@| to have catcode 11 (letter).
%    \begin{macrocode}
\ZREF@IfDefinable\zexternaldocument\def{%
  {%
    \ZREF@NewPropAnchor
    \ZREF@NewPropTitle
    \begingroup
      \csname @safe@actives@true\endcsname
      \makeatletter
      \@ifstar{%
        \ZREF@xr@zreflabelfalse
        \@testopt\ZREF@xr@externaldocument{}%
      }{%
        \ZREF@xr@zreflabeltrue
        \@testopt\ZREF@xr@externaldocument{}%
      }%
  }%
}%
%    \end{macrocode}
%    \end{macro}
%
%    If the \cs{include} featuer was used, there can be
%    several \xfile{.aux} files. These files are read
%    one after another, especially they are not recursively
%    read in order to save read registers. Thus it can
%    happen that the read order of the newlabel commands
%    differs from \hologo{LaTeX}'s order using \cs{input}.
%
%    \begin{macro}{\ZREF@xr@externaldocument}
%    It reads the remaining arguments. \cs{newcommand}
%    comes in handy for the optional argument.
%    \begin{macrocode}
\def\ZREF@xr@externaldocument[#1]#2{%
    \def\ZREF@xr@prefix{#1}%
    \let\ZREF@xr@filelist\ltx@empty
    \edef\ZREF@xr@externalfile{#2}%
    \edef\ZREF@xr@file{\ZREF@xr@externalfile.aux}%
    \filename@parse{#2}%
    \@testopt\ZREF@xr@graburl{#2.\zref@xr@ext}%
}%
\def\ZREF@xr@graburl[#1]{%
    \edef\ZREF@xr@url{#1}%
    \ifZREF@xr@urluse
      \expandafter\ZREF@xr@AddURL\expandafter{\ZREF@xr@url}%
      \expandafter\def\expandafter\ZREF@xr@url
      \expandafter{\csname Z@U@\ZREF@xr@theURL\endcsname}%
    \fi
    \ZREF@xr@checkfile
  \endgroup
}%
%    \end{macrocode}
%    \end{macro}
%    \begin{macro}{\ZREF@xr@processfile}
%    We follow \xpackage{xr} here, \cs{IfFileExists} offers
%    a nicer test, but we have to open the file anyway.
%    \begin{macrocode}
\def\ZREF@xr@checkfile{%
  \openin\@inputcheck\ZREF@xr@file\relax
  \ifeof\@inputcheck
    \PackageWarning{zref-xr}{%
      File `\ZREF@xr@file' not found or empty,\MessageBreak
      labels not imported%
    }%
  \else
    \PackageInfo{zref-xr}{%
      Label \ifZREF@xr@zreflabel (zref) \fi
      import from `\ZREF@xr@file'%
    }%
    \def\ZREF@xr@found{0}%
    \def\ZREF@xr@ignored@empty{0}%
    \def\ZREF@xr@ignored@zref{0}%
    \def\ZREF@xr@ignored@ltx{0}%
    \ZREF@xr@processfile
    \closein\@inputcheck
    \begingroup
      \let\on@line\ltx@empty
      \PackageInfo{zref-xr}{%
        Statistics for `\ZREF@xr@file':\MessageBreak
        \ZREF@xr@found\space
        \ifZREF@xr@zreflabel zref\else LaTeX\fi\space
        label(s) found%
        \ifnum\ZREF@xr@ignored@empty>0 %
          ,\MessageBreak
          \ZREF@xr@ignored@empty\space empty label(s) ignored%
        \fi
        \ifnum\ZREF@xr@ignored@zref>0 %
          ,\MessageBreak
          \ZREF@xr@ignored@zref\space
          duplicated zref label(s) ignored%
        \fi
        \ifnum\ZREF@xr@ignored@ltx>0 %
          ,\MessageBreak
          \ZREF@xr@ignored@ltx\space
          duplicated latex label(s) ignored%
        \fi
      }%
    \endgroup
  \fi
  \ifx\ZREF@xr@filelist\ltx@empty
  \else
    \edef\ZREF@xr@file{%
      \expandafter\ltx@car\ZREF@xr@filelist\@nil
    }%
    \edef\ZREF@xr@filelist{%
      \expandafter\ltx@cdr\ZREF@xr@filelist\ltx@empty\@nil
    }%
    \expandafter\ZREF@xr@checkfile
  \fi
}%
%    \end{macrocode}
%    \end{macro}
%
%    \begin{macro}{\ZREF@xr@processfile}
%    \begin{macrocode}
\def\ZREF@xr@processfile{%
  \read\@inputcheck to\ZREF@xr@line
  \expandafter\ZREF@xr@processline\ZREF@xr@line..\ZREF@nil
  \ifeof\@inputcheck
  \else
    \expandafter\ZREF@xr@processfile
  \fi
}%
%    \end{macrocode}
%    \end{macro}
%    \begin{macro}{\ZREF@xr@processline}
%    The most work must be done for analyzing the
%    arguments of \cs{newlabel}.
%    \begin{macrocode}
\long\def\ZREF@xr@processline#1#2#3\ZREF@nil{%
  \def\x{#1}%
  \toks@{#2}%
  \ifZREF@xr@zreflabel
    \ifx\x\ZREF@xr@zref@newlabel
      \expandafter
      \ZREF@xr@process@zreflabel\ZREF@xr@line...\ZREF@nil
    \fi
  \else
    \ifx\x\ZREF@xr@newlabel
      \expandafter
      \ZREF@xr@process@label\ZREF@xr@line...[]\ZREF@nil
    \fi
  \fi
  \ifx\x\ZREF@xr@@input
    \edef\ZREF@xr@filelist{%
      \etex@unexpanded\expandafter{\ZREF@xr@filelist}%
      {\filename@area\the\toks@}%
    }%
  \fi
}%
\def\ZREF@xr@process@zreflabel\zref@newlabel#1#2#3\ZREF@nil{%
  \edef\ZREF@xr@refname{Z@R@\ZREF@xr@prefix#1}%
  \edef\ZREF@xr@found{\the\numexpr\ZREF@xr@found+1\relax}%
  \def\x{#2}%
  \edef\ZREF@xr@tempname{$temp$}%
  \edef\ZREF@xr@temprefname{Z@R@\ZREF@xr@tempname}%
  \let\ZREF@xr@list\x
  \ifx\ZREF@xr@list\ltx@empty
    \PackageWarningNoLine{zref-xr}{%
      Label `#1' without properties ignored\MessageBreak
      in file `\ZREF@xr@file'%
    }%
    \edef\ZREF@xr@ignored@empty{%
      \the\numexpr\ZREF@xr@ignored@empty+1\relax
    }%
  \else
    \expandafter\ZREF@xr@checklist\x\ZREF@nil
    \expandafter\let\csname\ZREF@xr@temprefname\endcsname\x
    \expandafter\ltx@LocalAppendToMacro
    \csname\ZREF@xr@temprefname\expandafter\endcsname
    \expandafter{%
      \expandafter\externaldocument\expandafter{%
        \ZREF@xr@externalfile
      }%
    }%
    \ZREF@xr@urlcheck\ZREF@xr@tempname
    \ifZREF@xr@tozreflabel
      \@ifundefined{\ZREF@xr@refname}{%
        \ifZREF@xr@verbose
          \PackageInfo{zref-xr}{%
            Import to zref label `\ZREF@xr@tempname#1'%
          }%
        \fi
        \global\expandafter
        \let\csname\ZREF@xr@refname\expandafter\endcsname
        \csname\ZREF@xr@temprefname\endcsname
      }{%
        \ZREF@xr@zref@ignorewarning{\ZREF@xr@prefix#1}%
      }%
    \fi
    \ifZREF@xr@toltxlabel
      \ZREF@xr@tolabel{\ZREF@xr@tempname}{\ZREF@xr@prefix#1}%
    \fi
  \fi
}%
\def\ZREF@xr@process@label\newlabel#1#2#3[#4]#5\ZREF@nil{%
  \def\ZREF@xr@refname{Z@R@\ZREF@xr@prefix#1}%
  \edef\ZREF@xr@found{\the\numexpr\ZREF@xr@found+1\relax}%
  \def\x{#2}%
  \edef\ZREF@xr@tempname{$temp$}%
  \edef\ZREF@xr@temprefname{Z@R@\ZREF@xr@tempname}%
  \expandafter\ZREF@xr@scanparams
      \csname\ZREF@xr@temprefname\expandafter\endcsname
      \x{}{}{}{}{}\ZREF@nil
  \ifx\\#4\\%
  \else
    % ntheorem knows an optional argument at the end of \newlabel
    \ZREF@NewPropTheotype
    \expandafter\ltx@LocalAppendToMacro
        \csname\ZREF@xr@temprefname\endcsname{\theotype{#4}}%
  \fi
  \expandafter\ltx@LocalAppendToMacro
  \csname\ZREF@xr@temprefname\expandafter\endcsname\expandafter{%
    \expandafter\externaldocument\expandafter{%
      \ZREF@xr@externalfile
    }%
  }%
  \ZREF@xr@urlcheck\ZREF@xr@tempname
  \ifZREF@xr@tozreflabel
    \@ifundefined{\ZREF@xr@refname}{%
      \ifZREF@xr@verbose
        \PackageInfo{zref-xr}{%
          Import to zref label `\ZREF@xr@prefix#1'%
        }%
      \fi
      \global\expandafter
      \let\csname\ZREF@xr@refname\expandafter\endcsname
      \csname\ZREF@xr@temprefname\endcsname
    }{%
      \ZREF@xr@zref@ignorewarning{\ZREF@xr@prefix#1}%
    }%
  \fi
  \ifZREF@xr@toltxlabel
    \ZREF@xr@tolabel{\ZREF@xr@tempname}{\ZREF@xr@prefix#1}%
  \fi
}
\def\ZREF@xr@zref@newlabel{\zref@newlabel}%
\def\ZREF@xr@newlabel{\newlabel}%
\def\ZREF@xr@@input{\@input}%
\def\ZREF@xr@relax{\relax}%
%    \end{macrocode}
%    \end{macro}
%    \begin{macro}{\ZREF@xr@tolabel}
%    \begin{macrocode}
\def\ZREF@xr@tolabel#1#2{%
  \ifZREF@xr@verbose
    \PackageInfo{zref-xr}{%
      Import to LaTeX label `#2'%
    }%
  \fi
  \zref@wrapper@unexpanded{%
    \expandafter\xdef\csname r@#2\endcsname{%
      {%
        \ltx@ifundefined{M@TitleReference}{%
          \ltx@ifundefined{TR@TitleReference}{%
            \zref@extractdefault{#1}{default}{}%
          }{%
            \noexpand\TR@TitleReference
            {\zref@extractdefault{#1}{default}{}}%
            {\zref@extractdefault{#1}{title}{}}%
          }%
        }{%
          \noexpand\M@TitleReference
          {\zref@extractdefault{#1}{default}{}}%
          {\zref@extractdefault{#1}{title}{}}%
        }%
      }%
      {\zref@extractdefault{#1}{page}{}}%
      \ltx@ifpackageloaded{nameref}{%
        {\zref@extractdefault{#1}{title}{}}%
        {\zref@extractdefault{#1}{anchor}{}}%
        \zref@ifrefcontainsprop{#1}{urluse}{%
          {\zref@extractdefault{#1}{urluse}{}}%
        }{%
          {\zref@extractdefault{#1}{url}{}}%
        }%
      }{}%
    }%
  }%
}
%    \end{macrocode}
%    \end{macro}
%
%    \begin{macro}{\ZREF@xr@zref@ignorewarning}
%    \begin{macrocode}
\def\ZREF@xr@zref@ignorewarning#1{%
  \PackageWarningNoLine{zref-xr}{%
    Zref label `#1' is already in use\MessageBreak
    in file `\ZREF@xr@file'%
  }%
  \edef\ZREF@xr@ignored@zref{%
    \the\numexpr\ZREF@xr@ignored@zref+1%
  }%
}%
%    \end{macrocode}
%    \end{macro}
%
%    \begin{macro}{\ZREF@xr@ltx@ignorewarning}
%    \begin{macrocode}
\def\ZREF@xr@ltx@ignorewarning#1{%
  \PackageWarningNoLine{zref-xr}{%
    LaTeX label `#1' is already in use\MessageBreak
    in file `\ZREF@xr@file'%
  }%
  \edef\ZREF@xr@ignored@ltx{%
    \the\numexpr\ZREF@xr@ignored@ltx+1%
  }%
}%
%    \end{macrocode}
%    \end{macro}
%
%    \begin{macro}{\ZREF@xr@checklist}
%    \begin{macrocode}
\def\ZREF@xr@checklist#1#2#3\ZREF@nil{%
  \ifx\@undefined#1\relax
    \expandafter\ZREF@xr@checkkey\string#1\@nil
  \fi
  \ifx\\#3\\%
  \else
    \ltx@ReturnAfterFi{%
      \ZREF@xr@checklist#3\ZREF@nil
    }%
  \fi
}%
\def\ZREF@xr@checkkey#1#2\@nil{%
  \zref@ifpropundefined{#2}{%
    \zref@newprop{#2}{}%
  }{}%
}%
%    \end{macrocode}
%    \end{macro}
%
%    \begin{macro}{\ZREF@xr@scanparams}
%    \begin{macrocode}
\def\ZREF@xr@scanparams#1#2#3#4#5#6#7\ZREF@nil{%
  \let#1\ltx@empty
  \ZREF@foundfalse
  \ZREF@xr@scantitleref#1#2\TR@TitleReference{}{}\ZREF@nil
  \ifZREF@found
  \else
    \ltx@LocalAppendToMacro#1{\default{#2}}%
  \fi
  % page
  \ltx@LocalAppendToMacro#1{\page{#3}}%
  % nameref title
  \ifZREF@found
  \else
    \ifx\\#4\\%
    \else
      \def\ZREF@xr@temp{#4}%
      \ifx\ZREF@xr@temp\ZREF@xr@relax
      \else
        \ltx@LocalAppendToMacro#1{\title{#4}}%
      \fi
    \fi
  \fi
  % anchor
  \ifx\\#5\\%
  \else
    \ltx@LocalAppendToMacro#1{\anchor{#5}}%
  \fi
  \ifx\\#6\\%
  \else
    \ifZREF@xr@urluse
      \ZREF@xr@AddURL{#6}%
      \expandafter\ltx@LocalAppendToMacro\expandafter#1%
      \expandafter{%
        \expandafter\urluse\expandafter{%
          \csname Z@U@\ZREF@xr@theURL\endcsname
        }%
      }%
    \else
      \ltx@LocalAppendToMacro#1{\url{#6}}%
    \fi
  \fi
}%
%    \end{macrocode}
%    \end{macro}
%
%    \begin{macro}{\ZREF@xr@scantitleref}
%    \begin{macrocode}
\def\ZREF@xr@scantitleref#1#2\TR@TitleReference#3#4#5\ZREF@nil{%
  \ifx\\#5\\%
  \else
    \ltx@LocalAppendToMacro#1{%
      \default{#3}%
      \title{#4}%
    }%
    \ZREF@foundtrue
  \fi
}%
%    \end{macrocode}
%    \end{macro}
%
%    \begin{macro}{\ZREF@xr@urlcheck}%
%    \begin{macrocode}
\def\ZREF@xr@urlcheck#1{%
  \zref@ifrefcontainsprop{#1}{anchor}{%
    \zref@ifrefcontainsprop{#1}{url}{%
    }{%
      \expandafter
      \ltx@LocalAppendToMacro\csname Z@R@#1\expandafter\endcsname
      \expandafter{%
        \csname url\ifZREF@xr@urluse use\fi
        \expandafter\endcsname\expandafter{\ZREF@xr@url}%
      }%
    }%
  }{%
  }%
}%
%    \end{macrocode}
%    \end{macro}
%
%    \begin{macrocode}
%</xr>
%    \end{macrocode}
%
%
% \subsection{Module \xmodule{hyperref}}
%
%    UNFINISHED :-(
%
%    \begin{macrocode}
%<*hyperref>
\NeedsTeXFormat{LaTeX2e}
\ProvidesPackage{zref-hyperref}%
  [2020-07-03 v2.32 Module hyperref for zref (HO)]%
\RequirePackage{zref-base}[2019/11/29]
\ifx\ZREF@base@ok Y%
\else
  \expandafter\endinput
\fi
%    \end{macrocode}
%
%    \begin{macrocode}
\ZREF@NewPropAnchor
\zref@addprop\ZREF@mainlist{anchor}%
%    \end{macrocode}
%
%    \begin{macrocode}
%</hyperref>
%    \end{macrocode}
%
%
% \subsection{Module \xmodule{savepos}}
%
%    Module \xmodule{savepos} provides an interface for
%    \pdfTeX's \cmd{\pdfsavepos}, see the manual for
%    \pdfTeX.
%
% \subsubsection{Identification}
%
%    \begin{macrocode}
%<*savepos>
\NeedsTeXFormat{LaTeX2e}
\ProvidesPackage{zref-savepos}%
  [2020-07-03 v2.32 Module savepos for zref (HO)]%
\RequirePackage{zref-base}[2019/11/29]
\ifx\ZREF@base@ok Y%
\else
  \expandafter\endinput
\fi
%    \end{macrocode}
%
% \subsubsection{Availability}
%
%    First we check, whether the feature is available.
%    \begin{macrocode}
\ifx\directlua\@undefined
\ltx@IfUndefined{pdfsavepos}{%
  \PackageError\ZREF@name{%
    \string\pdfsavepos\space is not supported.\MessageBreak
    It is provided by pdfTeX (1.40) or XeTeX%
  }\ZREF@UpdatePdfTeX
  \endinput
}{}%
\fi
%    \end{macrocode}
%    In PDF mode we are done. However support for DVI mode
%    was added later in version 1.40.0. In earlier versions
%    \cs{pdfsavepos} is defined, but its execution raises
%    an error. Note that \hologo{XeTeX} also provides \cs{pdfsavepos}.
%    \begin{macrocode}
\ifpdf
\else
  \ltx@IfUndefined{pdftexversion}{%
  }{%
    \ifnum\pdftexversion<140 %
      \PackageError\ZREF@name{%
        \string\pdfsavepos\space is not supported in DVI mode%
        \MessageBreak
        of this pdfTeX version%
      }\ZREF@UpdatePdfTeX
      \expandafter\expandafter\expandafter\endinput
    \fi
  }%
\fi
%    \end{macrocode}
%
% \subsubsection{Setup}
%
%    \begin{macrocode}
\zref@newlist{savepos}
\ifx\directlua\@undefined
  \zref@newprop*{posx}[0]{\the\pdflastxpos}
  \zref@newprop*{posy}[0]{\the\pdflastypos}
\else
  \zref@newprop*{posx}[0]{\the\lastxpos}
  \zref@newprop*{posy}[0]{\the\lastypos}
\fi
\zref@addprops{savepos}{posx,posy}
%    \end{macrocode}
%
% \subsubsection{User macros}
%
%    \begin{macro}{\zref@savepos}
%    \begin{macrocode}
\ifx\directlua\@undefined
  \def\zref@savepos{%
    \if@filesw
      \pdfsavepos
    \fi
  }
\else
  \def\zref@savepos{%
    \if@filesw
      \savepos
    \fi
  }
\fi
%    \end{macrocode}
%    \end{macro}
%
%    \begin{macro}{\ZREF@zsavepos}
%    \begin{macrocode}
\def\ZREF@zsavepos#1#2#3{%
  \@bsphack
  \if@filesw
    \zref@savepos
    #1{#3}{#2}%
    \ltx@IfUndefined{TeXXeTstate}{%
    }{%
      \ifnum\TeXXeTstate=\ltx@zero
      \else
        \zref@savepos
      \fi
    }%
  \fi
  \@esphack
}
%    \end{macrocode}
%    \end{macro}
%
%    \begin{macro}{\zsavepos}
%    The current location is stored in a reference with the
%    given name.
%    \begin{macrocode}
\ZREF@IfDefinable\zsavepos\def{%
  {%
    \ZREF@zsavepos\zref@labelbylist{savepos}%
  }%
}
%    \end{macrocode}
%    \end{macro}
%
%    \begin{macro}{\zsaveposx}
%    \begin{macrocode}
\ZREF@IfDefinable\zsaveposx\def{%
  {%
    \ZREF@zsavepos\zref@labelbyprops{posx}%
  }%
}
%    \end{macrocode}
%    \end{macro}
%    \begin{macro}{\zsaveposy}
%    \begin{macrocode}
\ZREF@IfDefinable\zsaveposy\def{%
  {%
    \ZREF@zsavepos\zref@labelbyprops{posy}%
  }%
}
%    \end{macrocode}
%    \end{macro}
%
%    \begin{macro}{\zposx}
%    \begin{macro}{\zposy}
%    The horizontal and vertical position are available by
%    \cs{zposx} and \cs{zposy}. Do not rely on absolute
%    positions. They differ in DVI and PDF mode of \pdfTeX.
%    Use differences instead. The unit of the position numbers
%    is sp.
%    \begin{macrocode}
\newcommand*{\zposx}[1]{%
  \zref@extract{#1}{posx}%
}%
\newcommand*{\zposy}[1]{%
  \zref@extract{#1}{posy}%
}%
%    \end{macrocode}
%    \end{macro}
%    \end{macro}
%
%    Typically horizontal and vertical positions are used
%    inside calculations. Therefore the extracting macros
%    should be expandable and babel's patch is not applyable.
%
%    Also it is in the responsibility of the user to marked
%    used positions by \cs{zrefused}
%    in order to notify \hologo{LaTeX} about undefined references.
%
%    \begin{macro}{\ZREF@savepos@ok}
%    \begin{macrocode}
\let\ZREF@savepos@ok=Y
%    \end{macrocode}
%    \end{macro}
%
%    \begin{macrocode}
%</savepos>
%    \end{macrocode}
%
%
% \subsection{Module \xmodule{abspos}}
%
% \subsubsection{Identification}
%
%    \begin{macrocode}
%<*abspos>
\NeedsTeXFormat{LaTeX2e}
\ProvidesPackage{zref-abspos}%
  [2020-07-03 v2.32 Module abspos for zref (HO)]%
\RequirePackage{zref-base}[2019/11/29]
\ifx\ZREF@base@ok Y%
\else
  \expandafter\endinput
\fi
%    \end{macrocode}
%
%    \begin{macrocode}
\RequirePackage{zref-savepos}[2019/11/29]
\ifx\ZREF@savepos@ok Y%
\else
  \expandafter\endinput
\fi
%    \end{macrocode}
%    \begin{macrocode}
\RequirePackage{zref-pagelayout}[2019/11/29]
\zref@addprop{savepos}{abspage}
\zref@addprop{savepos}{pagevalue}
%    \end{macrocode}
%
%    \begin{macro}{\zref@absposx}
%    \begin{macrocode}
\newcommand*{\zref@absposx}[3]{%
  \number
    \expandafter\zref@absposnumx\expandafter{%
      \number\zref@extractdefault{#1}{abspage}{0}%
    }{#2}{#3}%
  \ltx@space
}
%    \end{macrocode}
%    \end{macro}
%
%    \begin{macro}{\zref@absposy}
%    \begin{macrocode}
\newcommand*{\zref@absposy}[3]{%
  \number
  \expandafter\zref@absposnumy\expandafter{%
    \number\zref@extractdefault{#1}{abspage}{0}%
  }{#2}{#3}%
  \ltx@space
}
%    \end{macrocode}
%    \end{macro}
%
%    \begin{macro}{\zref@absposnumx}
%    \begin{macrocode}
\newcommand*{\zref@absposnumx}[3]{%
  \number
%  \ifnum#1>\ltx@zero
%    \zref@ifrefundefined{thepage#1}{%
%      0%
%    }{%
      \numexpr\ZREF@absposnum{thepage#1}{#2}{x}{#3}\relax
%    }%
%  \else
%    0%
%  \fi
}
%    \end{macrocode}
%    \end{macro}
%
%    \begin{macro}{\zref@absposnumy}
%    \begin{macrocode}
\newcommand*{\zref@absposnumy}[3]{%
  \number
%  \ifnum#1>\ltx@zero
%    \zref@ifrefundefined{thepage#1}{%
%      0%
%    }{%
      \numexpr\ZREF@absposnum{thepage#1}{#2}{y}{#3}\relax
%    }%
%  \else
%    0%
%  \fi
}
%    \end{macrocode}
%    \end{macro}
%
%    \begin{macro}{\ZREF@absposnum}
%    \begin{macrocode}
\def\ZREF@absposnum#1#2#3#4{%
  \ltx@ifundefined{ZREF@abspos@#2@#3@#4}{%
    0%
  }{%
    \csname ZREF@abspos@#2@#3@#4\endcsname{#1}%
  }%
}
%    \end{macrocode}
%    \end{macro}
%
%    \begin{macro}{\zref@def@absposx}
%    \begin{macrocode}
\ZREF@Robust\def\zref@def@absposx#1{%
  \zref@wrapper@babel{\ZREF@def@abspos{#1}\zref@absposx}%
}
%    \end{macrocode}
%    \end{macro}
%
%    \begin{macro}{\zref@def@absposy}
%    \begin{macrocode}
\ZREF@Robust\def\zref@def@absposy#1{%
  \zref@wrapper@babel{\ZREF@def@abspos{#1}\zref@absposy}%
}
%    \end{macrocode}
%    \end{macro}
%
%    \begin{macro}{\zref@def@absposnumx}
%    \begin{macrocode}
\ZREF@Robust\def\zref@def@absposnumx#1{%
  \zref@wrapper@babel{\ZREF@def@abspos{#1}\zref@absposnumx}%
}
%    \end{macrocode}
%    \end{macro}
%
%    \begin{macro}{\zref@def@absposnumy}
%    \begin{macrocode}
\ZREF@Robust\def\zref@def@absposnumy#1{%
  \zref@wrapper@babel{\ZREF@def@abspos{#1}\zref@absposnumy}%
}
%    \end{macrocode}
%    \end{macro}
%    \begin{macro}{\ZREF@def@abspos}
%    \begin{macrocode}
\def\ZREF@def@absposnumy#1#2#3#4#5{%
  \edef#1{#2{#3}{#4}{#5}}%
}
%    \end{macrocode}
%    \end{macro}
%
%    \begin{macro}{\zref@absposused}
%    \begin{macrocode}
\ZREF@Robust\def\zref@absposused{%
  \zref@wrapper@babel\ZREF@abspos@used
}
%    \end{macrocode}
%    \end{macro}
%    \begin{macro}{\ZREF@abspos@used}
%    \begin{macrocode}
\def\ZREF@abspos@used#1{%
  \zref@refused{#1}%
  \zref@ifrefundefined{#1}{%
  }{%
    \begingroup
      \edef\ZREF@temp{%
        \zref@extractdefault{#1}{abspage}{0}%
      }%
      \ifnum\ZREF@temp>\ltx@zero
        \zref@refused{thepage\ZREF@temp}%
      \else
        \@PackageError{zref-abspos}{%
          \string\zref@pos@label@used\ltx@space
          needs property `abspage'\MessageBreak
          in label `#1'%
        }\@ehc
      \fi
    \endgroup
  }%
}
%    \end{macrocode}
%    \end{macro}
%
%    \begin{macro}{\zref@absposnumused}
%    \begin{macrocode}
\newcommand*{\zref@absposnumused}[1]{%
  \ifnum#1>\ltx@zero
    \zref@refused{thepage\number#1}%
  \else
    \@PackageError{zref-abspos}{%
      Invalid absolute page number (#1)\MessageBreak
      for \string\zref@pos@num@used.\MessageBreak
      A positive integer is expected%
    }\@ehc
  \fi
}
%    \end{macrocode}
%    \end{macro}
%
%    \begin{macro}{\zref@ifabsposundefined}
%    \begin{macrocode}
\def\zref@ifabsposundefined#1{%
  \zref@ifrefundefined{#1}\ltx@firsttwo{%
    \expandafter\zref@ifabsposnumundefined\expandafter{%
      \number\zref@extractdefault{#1}{abspage}{0}%
    }%
  }%
}
%    \end{macrocode}
%    \end{macro}
%
%    \begin{macro}{\zref@ifabsposnumundefined}
%    \begin{macrocode}
\def\zref@ifabsposnumundefined#1{%
  \ifnum\ZREF@number{#1}>\ltx@zero
    \zref@ifrefundefined{thepage#1}%
        \ltx@firstoftwo\ltx@secondoftwo
  \else
    \expandafter\ltx@firstoftwo
  \fi
}
%    \end{macrocode}
%    \end{macro}
%
% \subsubsection{Media}
%
%    \begin{macro}{\ZREF@abspos@media@width}
%    \begin{macrocode}
\edef\ZREF@abspos@media@width{%
  \ltx@ifundefined{pdfpagewidth}{%
    \ltx@ifundefined{mediawidth}{%
      \ltx@ifundefined{stockwidth}{%
        paperwidth%
      }{%
        stockwidth%
      }%
    }{%
      mediawidth%
    }%
  }{%
    pdfpagewidth%
  }%
}
\ifluatex
\def\ZREF@abspos@media@width{pdfpagewidth}%
\fi
%    \end{macrocode}
%    \end{macro}
%    \begin{macro}{\ZREF@abspos@media@height}
%    \begin{macrocode}
\edef\ZREF@abspos@media@height{%
  \ltx@ifundefined{pdfpageheight}{%
    \ltx@ifundefined{mediaheight}{%
      \ltx@ifundefined{stockheight}{%
        paperheight%
      }{%
        stockheight%
      }%
    }{%
      mediaheight%
    }%
  }{%
    \noexpand\ifcase\pdfpageheight
      \ltx@ifundefined{stockheight}{%
        paperheight%
      }{%
        stockheight%
      }%
    \noexpand\else
      pdfpageheight%
    \noexpand\fi
  }%
}
\ifluatex
\edef\ZREF@abspos@media@height{%
   \noexpand\ifcase\pageheight
      \ltx@ifundefined{stockheight}{%
        paperheight%
      }{%
        stockheight%
      }%
    \noexpand\else
      pdfpageheight%
    \noexpand\fi}%
\fi
%    \end{macrocode}
%    \end{macro}
%
%    \begin{macro}{\ZREF@abspos@media@x@left}
%    \begin{macrocode}
\def\ZREF@abspos@media@x@left#1{%
  0%
}
%    \end{macrocode}
%    \end{macro}
%    \begin{macro}{\ZREF@abspos@media@x@right}
%    \begin{macrocode}
\def\ZREF@abspos@media@x@right#1{%
  \zref@extract{#1}\ZREF@abspos@media@width
}
%    \end{macrocode}
%    \end{macro}
%    \begin{macro}{\ZREF@abspos@media@x@center}
%    \begin{macrocode}
\def\ZREF@abspos@media@x@center#1{%
  \ZREF@abspos@media@x@left{#1}%
  +\zref@extract{#1}\ZREF@abspos@media@width/2%
}
%    \end{macrocode}
%    \end{macro}
%
%    \begin{macro}{\ZREF@abspos@media@y@top}
%    \begin{macrocode}
\def\ZREF@abspos@media@y@top#1{%
  \zref@extract{#1}\ZREF@abspos@media@height
}
%    \end{macrocode}
%    \end{macro}
%    \begin{macro}{\ZREF@abspos@media@y@bottom}
%    \begin{macrocode}
\def\ZREF@abspos@media@y@bottom#1{%
  0%
}
%    \end{macrocode}
%    \end{macro}
%    \begin{macro}{\ZREF@abspos@media@y@center}
%    \begin{macrocode}
\def\ZREF@abspos@media@y@center#1{%
  \zref@extract{#1}\ZREF@abspos@media@height/2%
}
%    \end{macrocode}
%    \end{macro}
%
% \subsubsection{Paper}
%
%    \begin{macro}{\ZREF@abspos@paper@x@left}
%    \begin{macrocode}
\def\ZREF@abspos@paper@x@left#1{%
  0%
}
%    \end{macrocode}
%    \end{macro}
%    \begin{macro}{\ZREF@abspos@paper@x@right}
%    \begin{macrocode}
\def\ZREF@abspos@paper@x@right#1{%
  \zref@extract{#1}{paperwidth}%
}
%    \end{macrocode}
%    \end{macro}
%    \begin{macro}{\ZREF@abspos@paper@x@center}
%    \begin{macrocode}
\def\ZREF@abspos@paper@x@center#1{%
  \zref@extract{#1}{paperwidth}/2%
}
%    \end{macrocode}
%    \end{macro}
%
%    \begin{macro}{\ZREF@abspos@paper@y@top}
%    \begin{macrocode}
\let\ZREF@abspos@paper@y@top\ZREF@abspos@media@y@top
%    \end{macrocode}
%    \end{macro}
%    \begin{macro}{\ZREF@abspos@paper@y@bottom}
%    \begin{macrocode}
\def\ZREF@abspos@paper@y@bottom#1{%
  \ZREF@abspos@paper@y@top{#1}%
  -\zref@extract{#1}{paperheight}%
}
%    \end{macrocode}
%    \end{macro}
%    \begin{macro}{\ZREF@abspos@paper@y@center}
%    \begin{macrocode}
\def\ZREF@abspos@paper@y@center#1{%
  \ZREF@abspos@paper@y@top{#1}%
  -\zref@extract{#1}{paperheight}/2%
}
%    \end{macrocode}
%    \end{macro}
%
% \subsubsection{Origin}
%
% There doesn't seem a good reason to make these tests depend on pdf mode
%  in current engines, so comment out the \verb|\ifpdf| tests.
%    \begin{macro}{\ZREF@abspos@origin@x}
%    \begin{macrocode}
\let\ZREF@temp\ltx@two
\ltx@IfUndefined{pdfhorigin}{}{%
%  \ifpdf
    \let\ZREF@temp\ltx@zero
%  \fi
}
\ifluatex
% \ifpdf
  \let\ZREF@temp\ltx@zero
% \fi
\fi

\ifx\ZREF@temp\ltx@two
  \ifnum\mag=1000 %
    \let\ZREF@temp\ltx@one
  \fi
\fi
\ifcase\ZREF@temp
  \def\ZREF@abspos@origin@x#1{%
    \zref@extract{#1}{pdfhorigin}%
  }%
\or
  \def\ZREF@abspos@origin@x#1{%
    4736286%
  }%
\or
  \def\ZREF@abspos@origin@x#1{%
    \numexpr\mag/1000*\dimexpr 1truein\relax\relax
  }%
\fi
%    \end{macrocode}
%    \end{macro}
%    \begin{macro}{\ZREF@abspos@origin@y}
%    \begin{macrocode}
\let\ZREF@temp\ltx@two
\ltx@IfUndefined{pdfvorigin}{}{%
%  \ifpdf
    \let\ZREF@temp\ltx@zero
%  \fi
}
\ifluatex
%  \ifpdf
    \let\ZREF@temp\ltx@zero
%  \fi
\fi
\ifx\ZREF@temp\ltx@two
  \ifnum\mag=1000 %
    \let\ZREF@temp\ltx@one
  \fi
\fi
\ifcase\ZREF@temp
  \def\ZREF@abspos@origin@y#1{%
    \zref@extract{#1}{pdfvorigin}%
  }%
\or
  \def\ZREF@abspos@origin@y#1{%
    4736286%
  }%
\or
  \def\ZREF@abspos@origin@y#1{%
    \numexpr\mag/1000*\dimexpr 1truein\relax\relax
   }%
\fi
%    \end{macrocode}
%    \end{macro}
%
% \subsubsection{Header}
%
%    \begin{macro}{\ZREF@abspos@head@x@left}
%    \begin{macrocode}
\def\ZREF@abspos@head@x@left#1{%
  \ZREF@abspos@paper@x@left{#1}%
  +\ZREF@abspos@origin@x{#1}%
  +\zref@extract{#1}{hoffset}%
  +\ifodd\zref@extractdefault{#1}{pagevalue}{\number\c@page} %
     \zref@extract{#1}{oddsidemargin}%
   \else
     \zref@extract{#1}{evensidemargin}%
   \fi
}
%    \end{macrocode}
%    \end{macro}
%    \begin{macro}{\ZREF@abspos@head@x@right}
%    \begin{macrocode}
\def\ZREF@abspos@head@x@right#1{%
  \ZREF@abspos@head@x@left{#1}%
  +\zref@extract{#1}{textwidth}%
}
%    \end{macrocode}
%    \end{macro}
%    \begin{macro}{\ZREF@abspos@head@x@center}
%    \begin{macrocode}
\def\ZREF@abspos@head@x@center#1{%
  \ZREF@abspos@head@x@left{#1}%
  +\zref@extract{#1}{textwidth}/2%
}
%    \end{macrocode}
%    \end{macro}
%
%    \begin{macro}{\ZREF@abspos@head@y@top}
%    \begin{macrocode}
\def\ZREF@abspos@head@y@top#1{%
  \ZREF@abspos@paper@y@top{#1}%
  -\ZREF@abspos@origin@y{#1}%
  -\zref@extract{#1}{voffset}%
  -\zref@extract{#1}{topmargin}%
}
%    \end{macrocode}
%    \end{macro}
%    \begin{macro}{\ZREF@abspos@head@y@bottom}
%    \begin{macrocode}
\def\ZREF@abspos@head@y@bottom#1{%
  \ZREF@abspos@head@y@top{#1}%
  -\zref@extract{#1}{headheight}%
}
%    \end{macrocode}
%    \end{macro}
%    \begin{macro}{\ZREF@abspos@head@y@center}
%    \begin{macrocode}
\def\ZREF@abspos@head@y@center#1{%
  \ZREF@abspos@head@y@top{#1}%
  -\zref@extract{#1}{headheight}/2%
}
%    \end{macrocode}
%    \end{macro}
%
% \subsubsection{Body}
%
%    \begin{macro}{\ZREF@abspos@body@x@left}
%    \begin{macrocode}
\let\ZREF@abspos@body@x@left\ZREF@abspos@head@x@left
%    \end{macrocode}
%    \end{macro}
%    \begin{macro}{\ZREF@abspos@body@x@right}
%    \begin{macrocode}
\let\ZREF@abspos@body@x@right\ZREF@abspos@head@x@right
%    \end{macrocode}
%    \end{macro}
%    \begin{macro}{\ZREF@abspos@body@x@center}
%    \begin{macrocode}
\let\ZREF@abspos@body@x@center\ZREF@abspos@head@x@center
%    \end{macrocode}
%    \end{macro}
%
%    \begin{macro}{\ZREF@abspos@body@y@top}
%    \begin{macrocode}
\def\ZREF@abspos@body@y@top#1{%
  \ZREF@abspos@head@y@bottom{#1}%
  -\zref@extract{#1}{headsep}%
}
%    \end{macrocode}
%    \end{macro}
%    \begin{macro}{\ZREF@abspos@body@y@bottom}
%    \begin{macrocode}
\def\ZREF@abspos@body@y@bottom#1{%
  \ZREF@abspos@body@y@top{#1}%
  -\zref@extract{#1}{textheight}%
}
%    \end{macrocode}
%    \end{macro}
%    \begin{macro}{\ZREF@abspos@body@y@center}
%    \begin{macrocode}
\def\ZREF@abspos@body@y@center#1{%
  \ZREF@abspos@body@y@top{#1}%
  -\zref@extract{#1}{textheight}/2%
}
%    \end{macrocode}
%    \end{macro}
%
% \subsubsection{Footer}
%
%    \begin{macro}{\ZREF@abspos@foot@x@left}
%    \begin{macrocode}
\let\ZREF@abspos@foot@x@left\ZREF@abspos@head@x@left
%    \end{macrocode}
%    \end{macro}
%    \begin{macro}{\ZREF@abspos@foot@x@right}
%    \begin{macrocode}
\let\ZREF@abspos@foot@x@right\ZREF@abspos@head@x@right
%    \end{macrocode}
%    \end{macro}
%    \begin{macro}{\ZREF@abspos@foot@x@center}
%    \begin{macrocode}
\let\ZREF@abspos@foot@x@center\ZREF@abspos@head@x@center
%    \end{macrocode}
%    \end{macro}
%
%    \begin{macro}{\ZREF@abspos@foot@y@bottom}
%    \begin{macrocode}
\def\ZREF@abspos@foot@y@bottom#1{%
  \ZREF@abspos@body@y@bottom{#1}%
  -\zref@extract{#1}{footskip}%
}
%    \end{macrocode}
%    \end{macro}
%
% \subsubsection{Marginal notes}
%
%    \begin{macro}{\ZREF@abspos@marginpar@x@left}
%    \begin{macrocode}
\def\ZREF@abspos@marginpar@x@left#1{%
  \ifodd\zref@extractdefault{#1}{pagevalue}{\number\c@page} %
    \ZREF@abspos@body@x@right{#1}%
    +\zref@extract{#1}{marginparsep}%
  \else
    \ZREF@abspos@body@x@left{#1}%
    -\zref@extract{#1}{marginparsep}%
    -\zref@extract{#1}{marginparwidth}%
  \fi
}
%    \end{macrocode}
%    \end{macro}
%    \begin{macro}{\ZREF@abspos@marginpar@x@right}
%    \begin{macrocode}
\def\ZREF@abspos@marginpar@x@right#1{%
  \ZREF@abspos@marginpar@x@left{#1}%
  +\zref@extract{#1}{marginparwidth}%
}
%    \end{macrocode}
%    \end{macro}
%    \begin{macro}{\ZREF@abspos@marginpar@x@center}
%    \begin{macrocode}
\def\ZREF@abspos@marginpar@x@center#1{%
  \ZREF@abspos@marginpar@x@left{#1}%
  +\zref@extract{#1}{marginparwidth}/2%
}
%    \end{macrocode}
%    \end{macro}
%
%    \begin{macro}{\ZREF@abspos@marginpar@y@top}
%    \begin{macrocode}
\let\ZREF@abspos@marginpar@y@top\ZREF@abspos@body@y@top
%    \end{macrocode}
%    \end{macro}
%    \begin{macro}{\ZREF@abspos@marginpar@y@bottom}
%    \begin{macrocode}
\let\ZREF@abspos@marginpar@y@bottom\ZREF@abspos@body@y@bottom
%    \end{macrocode}
%    \end{macro}
%    \begin{macro}{\ZREF@abspos@marginpar@y@center}
%    \begin{macrocode}
\let\ZREF@abspos@marginpar@y@center\ZREF@abspos@body@y@center
%    \end{macrocode}
%    \end{macro}
%
% \subsubsection{Stock paper}
%
%    \begin{macro}{\ZREF@abspos@stock@x@left}
%    \begin{macrocode}
\let\ZREF@abspos@stock@x@left\ZREF@abspos@paper@x@left
%    \end{macrocode}
%    \end{macro}
%    \begin{macro}{\ZREF@abspos@stock@x@right}
%    \begin{macrocode}
\let\ZREF@abspos@stock@x@right\ZREF@abspos@paper@x@right
%    \end{macrocode}
%    \end{macro}
%    \begin{macro}{\ZREF@abspos@stock@x@center}
%    \begin{macrocode}
\let\ZREF@abspos@stock@x@center\ZREF@abspos@paper@x@center
%    \end{macrocode}
%    \end{macro}
%
%    \begin{macro}{\ZREF@abspos@stock@y@top}
%    \begin{macrocode}
\let\ZREF@abspos@stock@y@top\ZREF@abspos@paper@y@top
%    \end{macrocode}
%    \end{macro}
%    \begin{macro}{\ZREF@abspos@stock@y@bottom}
%    \begin{macrocode}
\let\ZREF@abspos@stock@y@bottom\ZREF@abspos@paper@y@bottom
%    \end{macrocode}
%    \end{macro}
%    \begin{macro}{\ZREF@abspos@stock@y@center}
%    \begin{macrocode}
\let\ZREF@abspos@stock@y@center\ZREF@abspos@paper@y@center
%    \end{macrocode}
%    \end{macro}
%
%    \begin{macrocode}
%</abspos>
%    \end{macrocode}
%
%
% \subsection{Module \xmodule{dotfill}}
%
%    \begin{macrocode}
%<*dotfill>
\NeedsTeXFormat{LaTeX2e}
\ProvidesPackage{zref-dotfill}%
  [2020-07-03 v2.32 Module dotfill for zref (HO)]%
\RequirePackage{zref-base}[2019/11/29]
\ifx\ZREF@base@ok Y%
\else
  \expandafter\endinput
\fi
%    \end{macrocode}
%
%    For measuring the width of \cs{zdotfill} we use the features
%    provided by module \xmodule{savepos}.
%    \begin{macrocode}
\RequirePackage{zref-savepos}[2019/11/29]
%    \end{macrocode}
%    For automatically generated label names we use the unique
%    counter of module \xmodule{base}.
%    \begin{macrocode}
\zref@require@unique
%    \end{macrocode}
%    Configuration is done by the key value interface of package
%    \xpackage{keyval}.
%    \begin{macrocode}
\RequirePackage{keyval}
%    \end{macrocode}
%    The definitions of the keys follow.
%    \begin{macrocode}
\define@key{ZREF@DF}{unit}{%
  \def\ZREF@df@unit{#1}%
}
\define@key{ZREF@DF}{min}{%
  \def\ZREF@df@min{#1}%
}
\define@key{ZREF@DF}{dot}{%
  \def\ZREF@df@dot{#1}%
}
%    \end{macrocode}
%    Defaults are set, see user interface.
%    \begin{macrocode}
\providecommand\ZREF@df@min{2}
\providecommand\ZREF@df@unit{.44em}
\providecommand\ZREF@df@dot{.}
%    \end{macrocode}
%    \begin{macro}{\zdotfillsetup}
%    Configuration of \cs{zdotfill} is done by \cs{zdotfillsetup}.
%    \begin{macrocode}
\newcommand*{\zdotfillsetup}{\kvsetkeys{ZREF@DF}}
%    \end{macrocode}
%    \end{macro}
%
%    \begin{macro}{\zdotfill}
%    \cs{zdotfill} sets labels at the left and the right to get
%    the horizontal position. \cs{zsavepos} is not used, because
%    we do not need the vertical position.
%    \begin{macrocode}
\ZREF@IfDefinable\zdotfill\def{%
  {%
    \leavevmode
    \global\advance\c@zref@unique\ltx@one
    \begingroup
      \def\ZREF@temp{zref@\number\c@zref@unique}%
      \pdfsavepos
      \zref@labelbyprops{\thezref@unique L}{posx}%
      \setlength{\dimen@}{\ZREF@df@unit}%
      \zref@ifrefundefined{\thezref@unique R}{%
        \ZREF@dotfill
      }{%
        \ifnum\numexpr\zposx{\thezref@unique R}%
                      -\zposx{\thezref@unique L}\relax
            <\dimexpr\ZREF@df@min\dimen@\relax
          \hfill
        \else
          \ZREF@dotfill
        \fi
      }%
      \pdfsavepos
      \zref@labelbyprops{\thezref@unique R}{posx}%
    \endgroup
    \kern\z@
  }%
}
%    \end{macrocode}
%    \end{macro}
%    \begin{macro}{\ZREF@dotfill}
%    Help macro that actually sets the dots.
%    \begin{macrocode}
\def\ZREF@dotfill{%
  \cleaders\hb@xt@\dimen@{\hss\ZREF@df@dot\hss}\hfill
}
%    \end{macrocode}
%    \end{macro}
%
%    \begin{macrocode}
%</dotfill>
%    \end{macrocode}
%
% \subsection{Module \xmodule{env}}
%
%    \begin{macrocode}
%<*env>
\NeedsTeXFormat{LaTeX2e}
\ProvidesPackage{zref-env}%
  [2020-07-03 v2.32 Module env for zref (HO)]%
\RequirePackage{zref-base}[2019/11/29]
\ifx\ZREF@base@ok Y%
\else
  \expandafter\endinput
\fi
%    \end{macrocode}
%
%    \begin{macrocode}
\zref@newprop{envname}[]{\@currenvir}
\zref@newprop{envline}[]{\zref@env@line}
%    \end{macrocode}
%
%    \begin{macro}{\zref@env@line}
%    Macro \cs{zref@env@line} extracts the line number
%    from \cs{@currenvline}.
%    \begin{macrocode}
\def\zref@env@line{%
  \ifx\@currenvline\ltx@empty
  \else
    \expandafter
    \ZREF@ENV@line\@currenvline\ltx@empty line \ltx@empty\@nil
  \fi
}
%    \end{macrocode}
%    \end{macro}
%    \begin{macro}{\ZREF@ENV@line}
%    \begin{macrocode}
\def\ZREF@ENV@line#1line #2\ltx@empty#3\@nil{#2}%
%    \end{macrocode}
%    \end{macro}
%
%    \begin{macrocode}
%</env>
%    \end{macrocode}
%
%% \section{Installation}
%
% \subsection{Download}
%
% \paragraph{Package.} This package is available on
% CTAN\footnote{\CTANpkg{zref}}:
% \begin{description}
% \item[\CTAN{macros/latex/contrib/zref/zref.dtx}] The source file.
% \item[\CTAN{macros/latex/contrib/zref/zref.pdf}] Documentation.
% \end{description}
%
%
% \paragraph{Bundle.} All the packages of the bundle `zref'
% are also available in a TDS compliant ZIP archive. There
% the packages are already unpacked and the documentation files
% are generated. The files and directories obey the TDS standard.
% \begin{description}
% \item[\CTANinstall{install/macros/latex/contrib/zref.tds.zip}]
% \end{description}
% \emph{TDS} refers to the standard ``A Directory Structure
% for \TeX\ Files'' (\CTANpkg{tds}). Directories
% with \xfile{texmf} in their name are usually organized this way.
%
% \subsection{Bundle installation}
%
% \paragraph{Unpacking.} Unpack the \xfile{zref.tds.zip} in the
% TDS tree (also known as \xfile{texmf} tree) of your choice.
% Example (linux):
% \begin{quote}
%   |unzip zref.tds.zip -d ~/texmf|
% \end{quote}
%
%
% \subsection{Package installation}
%
% \paragraph{Unpacking.} The \xfile{.dtx} file is a self-extracting
% \docstrip\ archive. The files are extracted by running the
% \xfile{.dtx} through \plainTeX:
% \begin{quote}
%   \verb|tex zref.dtx|
% \end{quote}
%
% \paragraph{TDS.} Now the different files must be moved into
% the different directories in your installation TDS tree
% (also known as \xfile{texmf} tree):
% \begin{quote}
% \def\t{^^A
% \begin{tabular}{@{}>{\ttfamily}l@{ $\rightarrow$ }>{\ttfamily}l@{}}
%   zref.sty & tex/latex/zref/zref.sty\\
%   zref-base.sty & tex/latex/zref/zref-base.sty\\
%   zref-abspage.sty & tex/latex/zref/zref-abspage.sty\\
%   zref-abspos.sty & tex/latex/zref/zref-abspos.sty\\
%   zref-counter.sty & tex/latex/zref/zref-counter.sty\\
%   zref-dotfill.sty & tex/latex/zref/zref-dotfill.sty\\
%   zref-env.sty & tex/latex/zref/zref-env.sty\\
%   zref-hyperref.sty & tex/latex/zref/zref-hyperref.sty\\
%   zref-lastpage.sty & tex/latex/zref/zref-lastpage.sty\\
%   zref-marks.sty & tex/latex/zref/zref-marks.sty\\
%   zref-nextpage.sty & tex/latex/zref/zref-nextpage.sty\\
%   zref-pageattr.sty & tex/latex/zref/zref-pageattr.sty\\
%   zref-pagelayout.sty & tex/latex/zref/zref-pagelayout.sty\\
%   zref-perpage.sty & tex/latex/zref/zref-perpage.sty\\
%   zref-runs.sty & tex/latex/zref/zref-runs.sty\\
%   zref-savepos.sty & tex/latex/zref/zref-savepos.sty\\
%   zref-thepage.sty & tex/latex/zref/zref-thepage.sty\\
%   zref-titleref.sty & tex/latex/zref/zref-titleref.sty\\
%   zref-totpages.sty & tex/latex/zref/zref-totpages.sty\\
%   zref-user.sty & tex/latex/zref/zref-user.sty\\
%   zref-xr.sty & tex/latex/zref/zref-xr.sty\\
%   zref.pdf & doc/latex/zref/zref.pdf\\
%   zref-example.tex & doc/latex/zref/zref-example.tex\\
%   zref-example-lastpage.tex & doc/latex/zref/zref-example-lastpage.tex\\
%   zref-example-nextpage.tex & doc/latex/zref/zref-example-nextpage.tex\\
%   zref.dtx & source/latex/zref/zref.dtx\\
% \end{tabular}^^A
% }^^A
% \sbox0{\t}^^A
% \ifdim\wd0>\linewidth
%   \begingroup
%     \advance\linewidth by\leftmargin
%     \advance\linewidth by\rightmargin
%   \edef\x{\endgroup
%     \def\noexpand\lw{\the\linewidth}^^A
%   }\x
%   \def\lwbox{^^A
%     \leavevmode
%     \hbox to \linewidth{^^A
%       \kern-\leftmargin\relax
%       \hss
%       \usebox0
%       \hss
%       \kern-\rightmargin\relax
%     }^^A
%   }^^A
%   \ifdim\wd0>\lw
%     \sbox0{\small\t}^^A
%     \ifdim\wd0>\linewidth
%       \ifdim\wd0>\lw
%         \sbox0{\footnotesize\t}^^A
%         \ifdim\wd0>\linewidth
%           \ifdim\wd0>\lw
%             \sbox0{\scriptsize\t}^^A
%             \ifdim\wd0>\linewidth
%               \ifdim\wd0>\lw
%                 \sbox0{\tiny\t}^^A
%                 \ifdim\wd0>\linewidth
%                   \lwbox
%                 \else
%                   \usebox0
%                 \fi
%               \else
%                 \lwbox
%               \fi
%             \else
%               \usebox0
%             \fi
%           \else
%             \lwbox
%           \fi
%         \else
%           \usebox0
%         \fi
%       \else
%         \lwbox
%       \fi
%     \else
%       \usebox0
%     \fi
%   \else
%     \lwbox
%   \fi
% \else
%   \usebox0
% \fi
% \end{quote}
% If you have a \xfile{docstrip.cfg} that configures and enables \docstrip's
% TDS installing feature, then some files can already be in the right
% place, see the documentation of \docstrip.
%
% \subsection{Refresh file name databases}
%
% If your \TeX~distribution
% (\TeX\,Live, \mikTeX, \dots) relies on file name databases, you must refresh
% these. For example, \TeX\,Live\ users run \verb|texhash| or
% \verb|mktexlsr|.
%
% \subsection{Some details for the interested}
%
% \paragraph{Unpacking with \LaTeX.}
% The \xfile{.dtx} chooses its action depending on the format:
% \begin{description}
% \item[\plainTeX:] Run \docstrip\ and extract the files.
% \item[\LaTeX:] Generate the documentation.
% \end{description}
% If you insist on using \LaTeX\ for \docstrip\ (really,
% \docstrip\ does not need \LaTeX), then inform the autodetect routine
% about your intention:
% \begin{quote}
%   \verb|latex \let\install=y\input{zref.dtx}|
% \end{quote}
% Do not forget to quote the argument according to the demands
% of your shell.
%
% \paragraph{Generating the documentation.}
% You can use both the \xfile{.dtx} or the \xfile{.drv} to generate
% the documentation. The process can be configured by the
% configuration file \xfile{ltxdoc.cfg}. For instance, put this
% line into this file, if you want to have A4 as paper format:
% \begin{quote}
%   \verb|\PassOptionsToClass{a4paper}{article}|
% \end{quote}
% An example follows how to generate the
% documentation with pdf\LaTeX:
% \begin{quote}
%\begin{verbatim}
%pdflatex zref.dtx
%makeindex -s gind.ist zref.idx
%pdflatex zref.dtx
%makeindex -s gind.ist zref.idx
%pdflatex zref.dtx
%\end{verbatim}
% \end{quote}
%
% \begin{thebibliography}{99}
% \bibpackage{footmisc}{Robin Fairbairns}{2004/01/23 v5.3a}
%   {\CTANpkg{footmisc}}
% \bibpackage{hyperref}{Sebastian Rahtz, Heiko Oberdiek}{2006/08/16 v6.75c}
%   {\CTANpkg{hyperref}}
% \bibpackage{lastpage}{Jeff Goldberg}{1994/06/25 v0.1b}
%   {\CTANpkg{lastpage}}
% \bibpackage{nameref}{Sebastian Rahtz, Heiko Oberdiek}{2006/02/12 v2.24}
%   {\CTANpkg{nameref}}
% \bibpackage{perpage}{David Kastrup}{2002/12/20 v1.0}
%   {\CTANpkg{perpage}}
% \bibpackage{titleref}{Donald Arsenau}{2001/04/05 v3.1}
%   {\CTANpkg{titleref}}
% \bibpackage{totpages}{Wilhelm M\"uller}{1999/07/14 v1.00}
%   {\CTANpkg{totpages}}
% \bibpackage{xr}{David Carlisle}{1994/05/28 v5.02}
%   {\CTANpkg{xr}}
% \bibpackage{xr-hyper}{David Carlisle}{2000/03/22 v6.00beta4}
%   {\CTANpkg{xr-hyper}}
% \end{thebibliography}
%
% \begin{History}
%   \begin{Version}{2006/02/20 v1.0}
%   \item
%     First version.
%   \end{Version}
%   \begin{Version}{2006/05/03 v1.1}
%   \item
%     Module \xmodule{perpage} added.
%   \item
%     Module redesign as packages.
%   \end{Version}
%   \begin{Version}{2006/05/25 v1.2}
%   \item
%     Module \xmodule{dotfillmin} added.
%   \item
%     Module \xmodule{base}: macros \cs{zref@require@uniqe}
%     and \cs{thezref@unique} added
%     (used by modules \xmodule{titleref} and \xmodule{dotfillmin}).
%   \end{Version}
%   \begin{Version}{2006/09/08 v1.3}
%   \item
%     Typo fixes and English cleanup by Per Starback.
%   \end{Version}
%   \begin{Version}{2007/01/23 v1.4}
%   \item
%     Typo in macro name fixed in documentation.
%   \end{Version}
%   \begin{Version}{2007/02/18 v1.5}
%   \item
%     \cs{zref@getcurrent} added (suggestion of Igor Akkerman).
%   \item
%     Module \xmodule{savepos} also supports \hologo{XeTeX}.
%   \end{Version}
%   \begin{Version}{2007/04/06 v1.6}
%   \item
%     Fix in modules \xmodule{abspage} and \xmodule{base}:
%     Now counter |abspage| and |zref@unique|
%     are not remembered by \cs{include}.
%   \item
%     Beamer support for module \xmodule{titleref}.
%   \end{Version}
%   \begin{Version}{2007/04/17 v1.7}
%   \item
%     Package \xpackage{atbegshi} replaces \xpackage{everyshi}.
%   \end{Version}
%   \begin{Version}{2007/04/22 v1.8}
%   \item
%     \cs{zref@wrapper@babel} and \cs{zref@refused} are now expandable
%     if babel is not used or \cs{if@safe@actives} is already set to true.
%     (Feature request of Josselin Noirel)
%   \end{Version}
%   \begin{Version}{2007/05/02 v1.9}
%   \item
%     Module \xmodule{titleref}: Some support for \cs{caption} of package
%     \xpackage{longtable}, but only if \cs{label} is given after
%     \cs{caption}.
%   \end{Version}
%   \begin{Version}{2007/05/06 v2.0}
%   \item
%     Uses package \xpackage{etexcmds} for accessing \hologo{eTeX}'s \cs{unexpanded}.
%   \end{Version}
%   \begin{Version}{2007/05/28 v2.1}
%   \item
%     Module \xmodule{titleref} supports caption of package \xpackage{listings}.
%   \item
%     Fixes in module \xmodule{titleref} for support
%     of packages \xpackage{titlesec} and \xpackage{longtable}.
%   \end{Version}
%   \begin{Version}{2008/09/21 v2.2}
%   \item
%     Module \xmodule{base}: \cs{zref@iflistcontainsprop} is documented, but
%     a broken \cs{zref@listcontainsprop} implemented. Name and
%     implementation fixed (thanks Ohad Kammar).
%   \end{Version}
%   \begin{Version}{2008/10/01 v2.3}
%   \item
%     \cs{zref@localaddprop} added (feature request of Ohad Kammar).
%   \item
%     Module \xmodule{lastpage}: list `LastPage' added. Label `LastPage' will use
%     the properties of this list (default is empty) along with the
%     properties of the main list.
%   \end{Version}
%   \begin{Version}{2009/08/07 v2.4}
%   \item
%     Module \xmodule{runs} added.
%   \end{Version}
%   \begin{Version}{2009/12/06 v2.5}
%   \item
%     Module \xmodule{lastpage}: Uses package \xpackage{atveryend}.
%   \item
%     Module \xmodule{titleref}: Further commands are disabled
%     during string expansion, imported from package \xpackage{nameref}.
%   \end{Version}
%   \begin{Version}{2009/12/07 v2.6}
%   \item
%     Version date added for package \xpackage{atveryend}.
%   \end{Version}
%   \begin{Version}{2009/12/08 v2.7}
%   \item
%     Module \xmodule{titleref}: Use of package \xpackage{gettitlestring}.
%   \end{Version}
%   \begin{Version}{2010/03/26 v2.8}
%   \item
%     \cs{zifrefundefined} added.
%   \item
%     Module \xmodule{lastpage}: Macros \cs{zref@iflastpage}
%     and \cs{ziflastpage} added.
%   \item
%     Module \xmodule{thepage} added.
%   \item
%     Module \xmodule{nextpage} added.
%   \end{Version}
%   \begin{Version}{2010/03/29 v2.9}
%   \item
%     Module \xmodule{marks} added (without documentation).
%   \item
%     \cs{zref@addprop} now adds expanded property to list.
%   \item
%     Useless \cs{ZREF@ErrorNoLine} removed.
%   \end{Version}
%   \begin{Version}{2010/04/08 v2.10}
%   \item
%     Module \xmodule{xr} remembers the external document name
%     in property `externaldocument'.
%   \end{Version}
%   \begin{Version}{2010/04/15 v2.11}
%   \item
%     Module \xmodule{titleref}: Better support of class \xclass{memoir}.
%   \item
%     Module \xmodule{titleref}: Support of theorems.
%   \end{Version}
%   \begin{Version}{2010/04/17 v2.12}
%   \item
%     Module \xmodule{base}: \cs{zref@newprop} ensures global
%     empty default.
%   \item
%     Module \xmodule{xr}: Setup options \xoption{tozreflabel}
%     and \xoption{toltxlabel} added.
%   \end{Version}
%   \begin{Version}{2010/04/19 v2.13}
%   \item
%     \cs{zref@setcurrent} throws an error if the property
%     does not exist (Florent Chervet).
%   \item
%     \cs{zref@getcurrent} the documentation is fixed (Florent Chervet).
%     Also it returns the empty string in case of errors.
%   \item
%     \cs{zref@addprop} and \cs{zref@localaddprop} now take a list of
%     property names (feature request of Florent Chervet).
%   \item
%     Example for \cs{zref@wrapper@unexpanded} corrected (Florent Chervet).
%   \end{Version}
%   \begin{Version}{2010/04/22 v2.14}
%   \item
%     Bug fix for \cs{zref@getcurrent} second argument wasn't eaten in
%     case of unknown property.
%   \item
%     \cs{zref@getcurrent} supports \cs{zref@wrapper@unexpanded}.
%   \item
%     \cs{zref@wrapper@unexpanded} added for \cs{ZREF@xr@tolabel}.
%   \item
%     \cs{zref@extract}, \cs{zref@extractdefault}, \cs{zref@getcurrent}
%     are expandable in exact two steps
%     except inside \cs{zref@wrapper@unexpanded}.
%   \end{Version}
%   \begin{Version}{2010/04/23 v2.15}
%   \item
%     \cs{zexternaldocument} fixed for property `url' when
%     importing \cs{new@label} (bug found by Victor Ivrii).
%   \item
%     Two expansion steps also in \cs{zref@wrapper@unexpanded}.
%   \item
%     Nested calls of \cs{zref@wrapper@unexpanded} possible.
%   \end{Version}
%   \begin{Version}{2010/04/28 v2.16}
%   \item
%     More consequent use of package `ltxcmds' and `hologo'.
%   \item
%     Module \xmodule{pagelayout} added.
%   \item
%     Module \xmodule{pageattr} added.
%   \item
%     Robustness introduced for non-expandable interface macros.
%   \item
%     Internal change of the data format of property lists
%     (suggestion of Florent Chervet).
%   \item
%     Module \xmodule{titleref}: Support of environment \textsf{description}.
%   \end{Version}
%   \begin{Version}{2010/05/01 v2.17}
%   \item
%     \cs{zref@newprop} throws an error if the property already exists.
%   \item
%     Module \xmodule{xr}: Bug fix for the case of several \xext{aux} files
%     (bug found by Victor Ivrii).
%   \item
%     Module \xmodule{xr}: Property `urluse' and option \xoption{urluse} added.
%   \end{Version}
%   \begin{Version}{2010/05/13 v2.18}
%   \item
%     Module \xmodule{env} added.
%   \item
%     Module \xmodule{savepos}: \cs{zref@savepos} added.
%   \end{Version}
%   \begin{Version}{2010/10/22 v2.19}
%   \item
%     \cs{zref@addprop} and \cs{zref@localaddprop} are limited
%     to one property only
%     (incompatibility to versions v2.13 to v2.18).
%   \item
%     \cs{zref@addprops} and \cs{zref@localaddprops} added.
%   \item
%     \cs{zref@delprop} and \cs{zref@localdelprop} added.
%   \item
%     \cs{zref@labelbykv} and \cs{zkvlabel} (module \xmodule{user})
%     with keys \xoption{prop}, \xoption{list}, \xoption{delprop},
%     \xoption{immediate}, \xoption{values} added.
%   \end{Version}
%   \begin{Version}{2011/02/12 v2.20}
%   \item
%     Fix for warning in \xpackage{zref-xr}.
%   \end{Version}
%   \begin{Version}{2011/03/18 v2.21}
%   \item
%     Fix in module \xmodule{pagelayout} for \cs{zlistpagelayout}.
%   \item
%     Fix for \cs{zref@localaddprop} (probably since v2.19).
%   \end{Version}
%   \begin{Version}{2011/10/05 v2.22}
%   \item
%     Documentation fixed for \cs{zref@(local)addprop(s)}.
%   \item
%     Module \xmodule{base}: \cs{zref@def@extract}, \cs{zref@def@extractdefault}
%     added.
%   \item
%     Fix in module \xmodule{pagelayout}: Because of missing \cs{noexpand}
%     commands the values of the pagelayout properties on all pages were
%     the values at package loading.
%   \item
%     Module \xmodule{base}: \cs{zref@showprop} added.
%   \end{Version}
%   \begin{Version}{2011/12/05 v2.23}
%   \item
%     Module \xmodule{savepos}: \cs{zsaveposx} and \cs{zsaveposy} added.
%   \end{Version}
%   \begin{Version}{2012/04/04 v2.24}
%   \item
%     Module \xmodule{titleref}, package \xpackage{titlesec}:
%     some support for class `straight' (\cs{ttl@straight@i}) added.
%   \end{Version}
%   \begin{Version}{2016/05/16 v2.25}
%   \item
%     Documentation updates.
%   \end{Version}
%   \begin{Version}{2016/05/21 v2.26}
%   \item
%     update zref-savepos for new luatex
%   \end{Version}
%   \begin{Version}{2018/11/21 v2.27}
%   \item
%     adapted zref-perpage, see issue https://github.com/ho-tex/zref/issues/2
%   \end{Version}
%   \begin{Version}{2019/11/29 v2.28}
%   \item
%     Documentation updates.
%   \item
%     Use \xpackage{iftex} directly.
%   \end{Version}
%   \begin{Version}{2020-03-03 v2.29}
%   \item
%     adapted in module zref-pagelayout the properties pdfhorigin, pdfvorigin,
%     pdfpagewidth, pdfpageheight for luatex to the right primitives.
%   \item
%     Removed no longer needed code for older lualatex versions.
%   \item added some documentation of the abspos module.
%   \item adapted the abspos module to the new luatex primitives.
%   \item adapted pageattr module to the new luatex primitives.
%   \item added short documentation for pageattr module
%   \item use luatex command names directly in zref-savepos rather than defining pdftex compatibility names.
%   \item allow zref-abspos to use \verb|\pdf[vh]origin| in dvi mode.
%   \end{Version}
%   \begin{Version}{2020-03-04 v2.30}
%   \item
%     add pagevalue property to savepos in the abspos module (issue 1)
%   \end{Version}
%   \begin{Version}{2020-05-28 v2.31}
%   \item Adapted module zref-counter to use \cs{@currentcounter} in the next
%   \LaTeX version (issue 5)
%   \end{Version}
%   \begin{Version}{2020-07-03 v2.32}
%   \item Changed in zref-pagelayout the names of the shipout box dimensions
%    to adapt to the new hook management.
%   \end{Version}
% \end{History}
%
% \PrintIndex
%
% \Finale
\endinput

%        (quote the arguments according to the demands of your shell)
%
% Documentation:
%    (a) If zref.drv is present:
%           latex zref.drv
%    (b) Without zref.drv:
%           latex zref.dtx; ...
%    The class ltxdoc loads the configuration file ltxdoc.cfg
%    if available. Here you can specify further options, e.g.
%    use A4 as paper format:
%       \PassOptionsToClass{a4paper}{article}
%
%    Programm calls to get the documentation (example):
%       pdflatex zref.dtx
%       makeindex -s gind.ist zref.idx
%       pdflatex zref.dtx
%       makeindex -s gind.ist zref.idx
%       pdflatex zref.dtx
%
% Installation:
%    TDS:tex/latex/zref/zref.sty
%    TDS:tex/latex/zref/zref-base.sty
%    TDS:tex/latex/zref/zref-abspage.sty
%    TDS:tex/latex/zref/zref-abspos.sty
%    TDS:tex/latex/zref/zref-counter.sty
%    TDS:tex/latex/zref/zref-dotfill.sty
%    TDS:tex/latex/zref/zref-env.sty
%    TDS:tex/latex/zref/zref-hyperref.sty
%    TDS:tex/latex/zref/zref-lastpage.sty
%    TDS:tex/latex/zref/zref-marks.sty
%    TDS:tex/latex/zref/zref-nextpage.sty
%    TDS:tex/latex/zref/zref-pageattr.sty
%    TDS:tex/latex/zref/zref-pagelayout.sty
%    TDS:tex/latex/zref/zref-perpage.sty
%    TDS:tex/latex/zref/zref-runs.sty
%    TDS:tex/latex/zref/zref-savepos.sty
%    TDS:tex/latex/zref/zref-thepage.sty
%    TDS:tex/latex/zref/zref-titleref.sty
%    TDS:tex/latex/zref/zref-totpages.sty
%    TDS:tex/latex/zref/zref-user.sty
%    TDS:tex/latex/zref/zref-xr.sty
%    TDS:doc/latex/zref/zref.pdf
%    TDS:doc/latex/zref/zref-example.tex
%    TDS:doc/latex/zref/zref-example-lastpage.tex
%    TDS:doc/latex/zref/zref-example-nextpage.tex
%    TDS:source/latex/zref/zref.dtx
%
%<*ignore>
\begingroup
  \catcode123=1 %
  \catcode125=2 %
  \def\x{LaTeX2e}%
\expandafter\endgroup
\ifcase 0\ifx\install y1\fi\expandafter
         \ifx\csname processbatchFile\endcsname\relax\else1\fi
         \ifx\fmtname\x\else 1\fi\relax
\else\csname fi\endcsname
%</ignore>
%<*install>
\input docstrip.tex
\Msg{************************************************************************}
\Msg{* Installation}
\Msg{* Package: zref 2020-07-03 v2.32 A new reference scheme for LaTeX (HO)}
\Msg{************************************************************************}

\keepsilent
\askforoverwritefalse

\let\MetaPrefix\relax
\preamble

This is a generated file.

Project: zref
Version: 2020-07-03 v2.32

Copyright (C)
  2006-2012 Heiko Oberdiek
  2016-2020 Oberdiek Package Support Group

This work may be distributed and/or modified under the
conditions of the LaTeX Project Public License, either
version 1.3c of this license or (at your option) any later
version. This version of this license is in
   https://www.latex-project.org/lppl/lppl-1-3c.txt
and the latest version of this license is in
   https://www.latex-project.org/lppl.txt
and version 1.3 or later is part of all distributions of
LaTeX version 2005/12/01 or later.

This work has the LPPL maintenance status "maintained".

The Current Maintainers of this work are
Heiko Oberdiek and the Oberdiek Package Support Group
https://github.com/ho-tex/zref/issues


This work consists of the main source file zref.dtx
and the derived files
   zref.sty, zref.pdf, zref.ins, zref.drv, zref-base.sty,
   zref-abspage.sty, zref-abspos.sty, zref-counter.sty,
   zref-dotfill.sty, zref-env.sty, zref-hyperref.sty,
   zref-lastpage.sty, zref-marks.sty, zref-nextpage.sty,
   zref-pageattr.sty, zref-pagelayout.sty, zref-perpage.sty,
   zref-runs.sty, zref-savepos.sty, zref-thepage.sty,
   zref-titleref.sty, zref-totpages.sty, zref-user.sty,
   zref-xr.sty, zref-example.tex, zref-example-lastpage.tex,
   zref-example-nextpage.tex, zref-test1.tex,
   zref-test-base.tex, zref-test-runs.tex,
   zref-test-titleref-memoir.tex.

\endpreamble
\let\MetaPrefix\DoubleperCent

\generate{%
  \file{zref.ins}{\from{zref.dtx}{install}}%
  \file{zref.drv}{\from{zref.dtx}{driver}}%
  \usedir{tex/latex/zref}%
  \file{zref.sty}{\from{zref.dtx}{package}}%
  \file{zref-base.sty}{\from{zref.dtx}{base}}%
  \file{zref-abspage.sty}{\from{zref.dtx}{abspage}}%
  \file{zref-abspos.sty}{\from{zref.dtx}{abspos}}%
  \file{zref-counter.sty}{\from{zref.dtx}{counter}}%
  \file{zref-dotfill.sty}{\from{zref.dtx}{dotfill}}%
  \file{zref-env.sty}{\from{zref.dtx}{env}}%
  \file{zref-hyperref.sty}{\from{zref.dtx}{hyperref}}%
  \file{zref-lastpage.sty}{\from{zref.dtx}{lastpage}}%
  \file{zref-marks.sty}{\from{zref.dtx}{marks}}%
  \file{zref-nextpage.sty}{\from{zref.dtx}{nextpage}}%
  \file{zref-pageattr.sty}{\from{zref.dtx}{pageattr}}%
  \file{zref-pagelayout.sty}{\from{zref.dtx}{pagelayout}}%
  \file{zref-perpage.sty}{\from{zref.dtx}{perpage}}%
  \file{zref-runs.sty}{\from{zref.dtx}{runs}}%
  \file{zref-savepos.sty}{\from{zref.dtx}{savepos}}%
  \file{zref-thepage.sty}{\from{zref.dtx}{thepage}}%
  \file{zref-titleref.sty}{\from{zref.dtx}{titleref}}%
  \file{zref-totpages.sty}{\from{zref.dtx}{totpages}}%
  \file{zref-user.sty}{\from{zref.dtx}{user}}%
  \file{zref-xr.sty}{\from{zref.dtx}{xr}}%
  \usedir{doc/latex/zref}%
  \file{zref-example.tex}{\from{zref.dtx}{example}}%
  \file{zref-example-lastpage.tex}{\from{zref.dtx}{example-lastpage}}%
  \file{zref-example-nextpage.tex}{\from{zref.dtx}{example-nextpage}}%
%  \usedir{doc/latex/zref/test}%
%  \file{zref-test1.tex}{\from{zref.dtx}{test1}}%
%  \file{zref-test-base.tex}{\from{zref.dtx}{test-base}}%
%  \file{zref-test-runs.tex}{\from{zref.dtx}{test-runs}}%
%  \file{zref-test-titleref-memoir.tex}{\from{zref.dtx}{test-titleref-memoir}}%
}

\catcode32=13\relax% active space
\let =\space%
\Msg{************************************************************************}
\Msg{*}
\Msg{* To finish the installation you have to move the following}
\Msg{* files into a directory searched by TeX:}
\Msg{*}
\Msg{*     zref.sty, zref-base.sty, zref-abspage.sty, zref-abspos.sty,}
\Msg{*     zref-counter.sty, zref-dotfill.sty, zref-env.sty,}
\Msg{*     zref-hyperref.sty, zref-lastpage.sty, zref-marks.sty,}
\Msg{*     zref-nextpage.sty, zref-pageattr.sty, zref-pagelayout.sty,}
\Msg{*     zref-perpage.sty, zref-runs.sty, zref-savepos.sty,}
\Msg{*     zref-thepage.sty, zref-titleref.sty, zref-totpages.sty,}
\Msg{*     zref-user.sty, zref-xr.sty}
\Msg{*}
\Msg{* To produce the documentation run the file `zref.drv'}
\Msg{* through LaTeX.}
\Msg{*}
\Msg{* Happy TeXing!}
\Msg{*}
\Msg{************************************************************************}

\endbatchfile
%</install>
%<*ignore>
\fi
%</ignore>
%<*driver>
\NeedsTeXFormat{LaTeX2e}
\ProvidesFile{zref.drv}%
  [2020-07-03 v2.32 A new reference scheme for LaTeX (HO)]%
\documentclass{ltxdoc}
\usepackage{holtxdoc}[2011/11/22]
\setlength{\emergencystretch}{1em}
\begin{document}
  \DocInput{zref.dtx}%
\end{document}
%</driver>
% \fi
%
%
%
% \GetFileInfo{zref.drv}
%
% \title{The \xpackage{zref} package}
% \date{2020-07-03 v2.32}
% \author{Heiko Oberdiek\thanks
% {Please report any issues at \url{https://github.com/ho-tex/zref/issues}}}
%
% \maketitle
%
% \begin{abstract}
% Package \xpackage{zref} tries to get rid of the restriction
% in \hologo{LaTeX}'s reference system that only two properties are
% supported. The package implements an extensible referencing
% system, where properties are handled in a more flexible way.
% It offers an interface for macro programmers for the access
% to the system and some applications that uses the new
% reference scheme.
% \end{abstract}
%
% \tableofcontents
%
% \newcommand*{\metacs}[1]{\texttt{\itshape #1}}
%
% \newcommand{\markexp}{^^A
%   \textsuperscript{exp}^^A
%   \space
% }
% \newcommand{\markexptwo}{^^A
%   \textsuperscript{exp2}^^A
%   \space
% }
% \newcommand{\markrobust}{^^A
%   \textsuperscript{robust}^^A
%   \space
% }
% \newcommand{\marknowarn}{^^A
%   \textsuperscript{nowarn}^^A
%   \space
% }
% \newcommand{\markbabel}{^^A
%   \textsuperscript{babel}^^A
%   \space
% }
%
% \section{Introduction}
%
% Standard \hologo{LaTeX}'s reference system with \cs{label}, \cs{ref},
% and \cs{pageref} supports two properties, the apperance of
% the counter that is last incremented by \cs{refstepcounter} and
% the page with the \cs{label} command.
%
% Unhappily \hologo{LaTeX} does not provide an interface for adding
% another properties. Packages such as \xpackage{hyperref},
% \xpackage{nameref}, or \xpackage{titleref} are forced to
% use ugly hacks to extend the reference system.
% These ugly hacks are one of the causes for \xpackage{hyperref}'s
% difficulty regarding compatibility with other packages.
%
% \subsection{Standard \hologo{LaTeX} behaviour}
%
% References are created by the \cs{label} command:
% \begin{quote}
%   |\chapter{Second chapter}|\\
%   |\section{First section on page 7} % section 2.1|\\
%   |\label{myref}|
% \end{quote}
% Now \hologo{LaTeX} records the section number |2.1| and the page |7|
% in the reference. Internally the reference is a list with
% two entries:
% \begin{quote}
%   |\r@myref| $\rightarrow$ |{2.1}{7}|
% \end{quote}
% The length of the list if fixed in the \hologo{LaTeX} kernel,
% An interface for adding new properties is missing.
%
% There are several tries to add new properties:
% \begin{description}
% \item[\xpackage{hyperref}] uses a list of five
%   properties instead of the standard list with two entries.
%   This causes many compatibility problems with \hologo{LaTeX} and
%   other packages.
% \item[\xpackage{titleref}] stores its title data into the
%   first entry in the list. \hologo{LaTeX} is happy because it
%   does only see its list with two entries. The situation
%   becomes more difficult, if more properties are added
%   this way. Then the macros form a nested structure
%   inside the first reference argument for the label.
%   Expandable extractions will then become painful.
% \end{description}
%
% \subsection{Basic idea}
%
% Some time ago Morten H\o gholm sent me an experimental
% cross referencing mechanism as ``expl3'' code. His
% idea is:
% \begin{quote}
%   |\g_xref_mylabel_plist| $\rightarrow$\\
%   \mbox{}\qquad|\xref_dance_key{salsa}\xref_name_key{Morten}|\dots
% \end{quote}
% The entries have the following format:
% \begin{quote}
%   |\xref_|\meta{your key}|_key{|\meta{some text}|}|
% \end{quote}
% This approach is much more flexible:
% \begin{itemize}
%   \item New properties can easily be added, just use a new key.
%   \item The length of the list is not fixed. A reference
%         can use a subset of the keys.
%   \item The order of the entries does not matter.
% \end{itemize}
% Unhappily I am not familiar with the experimental code
% for \hologo{LaTeX3} that will need some time before its first release.
% Thus I have implemented it as \hologo{LaTeXe} package without disturbing
% the existing \hologo{LaTeX} reference system.
%
% \subsection{Interfaces}
%
% The package provides a generic \emph{interface for programmers}.
% Commands of this interface are prefixed by \cs{zref@}.
%
% Option \xoption{user} enabels the \emph{user interface}.
% Here the commands are prefixed by \cs{z} to avoid name
% clashes with existing macros.
%
% Then the packages provides some \emph{modules}. They
% are applications for the reference system and can also be
% considered as examples how to use the reference system.
%
% The modules can be loaded as packages. The package
% name is prefixed with |zref-|, for example:
%\begin{quote}
%\begin{verbatim}
%\RequirePackage{zref-abspage}
%\end{verbatim}
%\end{quote}
% This is the preferred way if the package is loaded from
% within other packages to avoid option clashes.
%
% As alternative package |zref| can be used and the modules
% are given as options:
%\begin{quote}
%\begin{verbatim}
%\usepackage[perpage,user]{zref}
%\end{verbatim}
%\end{quote}
%
% \section{Interface for programmers}
%
% The user interface is described in the next section \ref{sec:user}.
%
% \subsection{Entities}
%
% \paragraph{Reference.}
% Internally a reference is a list of key value pairs:
% \begin{quote}
%   |\Z@R@myref| $\rightarrow$
%   |\default{2.1}\page{7}|
% \end{quote}
% The generic format of a entry is:
% \begin{quote}
%   |\Z@R@|\meta{refname} $\rightarrow$
%   |\|\meta{propname}|{|\meta{value}|}|
% \end{quote}
% \meta{refname} is the name that denoted references
% (the name used in \cs{label} and \cs{ref}). \meta{propname}
% is the name of the property or key. The property key macro
% is never executed, it is used in parameter text matching only.
%
% \paragraph{Property.}
% Because the name of a property is used in a macro name
% that must survive the \xfile{.aux} file, the name is
% restricted to letters and `@'.
%
% \paragraph{Property list.}
% Often references are used for special purposes. Thus it
% saves memory if just the properties are used in this
% reference that are necessary for its purpose.
%
% Therefore this package uses the concept of \emph{property lists}.
% A property list is a set of properties. The set of properties
% that is used by the default \cs{label} command is the
% \emph{main property list}.
%
% \subsection{Property list}
%
% \markexp means that the implementation of the
% marked macro is expandable. \markexptwo goes a step further
% and marks the macro expandable in exact two expansion steps.
%
% \begin{declcs}{zref@newlist}
%   \M{listname}
% \end{declcs}
% Declares a new empty property list.
%
% \begin{declcs}{zref@addprop}
%   \M{listname} \M{propname}\\
%   \cs{zref@localaddprop} \M{listname} \M{propname}
% \end{declcs}
% Adds the property \meta{propname}
% to the property list \meta{listname}.
% The property and list must exist.
% The addition is global by \cs{zref@addprop} and limited to
% local scope by \cs{zref@localaddprop}.
% Between 2010/04/19 v2.13 and 2010/10/22 v2.19 a comma separated list
% of properties could be used as argument \meta{propname}.
% Since 2010/10/22 v2.19 the addition of several properties
% at once is supported by \cs{zref@addprops}.
%
% \begin{declcs}{zref@addprops}
%   \M{listname} \M{propname list}\\
%   \cs{zref@localaddprops} \M{listname} \M{propname list}
% \end{declcs}
% These macros add a comma separated list of properties
% \meta{propname list} to list \meta{listname}.
% \cs{zref@addprops} works globally and \cs{zref@localaddprops}
% locally. Since 2010/10/22 v2.19.
%
% \begin{declcs}{zref@listexists}
%   \M{listname} \M{then}
% \end{declcs}
% Executes \meta{then} if the property list \meta{listname} exists
% or raise an error otherwise.
%
% \begin{declcs}{zref@iflistundefined}\markexp \M{listname}
%   \M{then} \M{else}
% \end{declcs}
% Executes \meta{then} if the list exists or \meta{else}
% otherwise.
%
% \begin{declcs}{zref@iflistcontainsprop}
%   \M{listname} \M{propname} \M{then} \M{else}
% \end{declcs}
% Executes \meta{then} if the property \meta{propname} is part
% of property list \meta{listname} or otherwise it
% runs the \meta{else} part.
%
% \subsection{Property}
%
% \begin{declcs}{zref@newprop}
%   \* \M{propname} |[|\meta{default}|]| \M{value}
% \end{declcs}
% This command declares and configures a new property with
% name \meta{propname}.
%
% In case of unknown references
% or the property does not exist in the reference, the
% \meta{default} is used as value. If it is not specified
% here, a global default is used, see \cs{zref@setdefault}.
%
% The correct values of some properties are not known
% immediately but at page shipout time. Prominent example
% is the page number. These properties are declared with
% the star form of the command.
%
% \begin{declcs}{zref@setcurrent}
%   \M{propname} \M{value}
% \end{declcs}
% This sets the current value of the property \meta{propname}.
% It is a generalization of setting \hologo{LaTeX}'s \cs{currentlabel}.
%
% \begin{declcs}{zref@getcurrent}\markexptwo
%   \M{propname}
% \end{declcs}
% This returns the current value of the property \meta{propname}.
% The value may not be correct, especially if the property
% is bound to a page (start form of \cs{zref@newprop}) and the
% right value is only known at shipout time (e.g. property `page').
% In case of errors (e.g. unknown property) the empty string is
% returned.
%
% Since version 2010/04/22 v2.14 \cs{zref@getcurrent} supports
% \cs{zref@wrapper@unexpanded}.
%
% \begin{declcs}{zref@propexists}
%   \M{propname} \M{then}
% \end{declcs}
% Calls \meta{then} if the property \meta{propname} is available
% or generates an error message otherwise.
%
% \begin{declcs}{zref@ifpropundefined}\markexp
%   \M{propname} \M{then} \M{else}
% \end{declcs}
% Calls \meta{then} or \meta{else} depending on the existence
% of property \meta{propname}.
%
% \subsection{Reference generation}
%
% \begin{declcs}{zref@label}
%   \M{refname}
% \end{declcs}
% This works similar to \cs{label}. The reference \meta{refname} is
% created and put into the \xfile{.aux} file with the
% properties of the main property list.
%
% \begin{declcs}{zref@labelbylist}
%   \M{refname} \M{listname}
% \end{declcs}
% Same as \cs{zref@label} except that the properties are
% taken from the specified property list \meta{listname}.
%
% \begin{declcs}{zref@labelbyprops}
%   \M{refname} |{|\meta{propnameA}|,|\meta{propnameB}|,|\dots|}|
% \end{declcs}
% Same as \cs{zref@label} except that these properties
% are used that are given as comma separated list in the
% second argument.
%
% \begin{declcs}{zref@newlabel}
%   \M{refname} |{|\dots|}|
% \end{declcs}
% This is the macro that is used in the \xfile{.aux} file.
% It is basically the same as \cs{newlabel} apart from
% the format of the data in the second argument.
%
% \subsection{Data extraction}
%
% \begin{declcs}{zref@extractdefault}\markexptwo
%   \M{refname} \M{propname} \M{default}
% \end{declcs}
% This is the basic command that refernces the value of a
% property \meta{propname} for the reference \meta{refname}.
% In case of errors such as undefined reference the \meta{default}
% is used instead.
%
% \begin{declcs}{zref@extract}\markexptwo
%   \M{refname} \M{propname}
% \end{declcs}
% The command is an abbreviation for \cs{zref@extractdefault}.
% As default the default of the property is taken, otherwise
% the global default.
%
% Example for page references:
% \begin{quote}
% \begin{tabular}{@{}ll}
% \hologo{LaTeX}:& |\pageref{foobar}|\\
% \xpackage{zref}:& |\zref@extract{foobar}{page}|
% \end{tabular}
% \end{quote}
% Both \cs{zref@extract} and \cs{zref@extractdefault} are
% expandable. That means, these macros can directly
% be used in expandable calculations, see the example file.
% On the other side, babel's shorthands are not supported,
% there are no warnings in case of undefined references.
%
% If an user interface doesn't need expandable macros
% then it can use \cs{zref@refused} and \cs{zref@wrapper@babel}
% for its user macros.
%
% \begin{declcs}{zref@refused}
%   \M{refname}
% \end{declcs}
% This command is not expandable. It causes the warnings if
% the reference \meta{refname} is not defined. Use the
% \cs{zref@extract} commands inside expandable contexts and
% mark their use outside by \cs{zref@refused}, see the example
% file.
%
% \begin{declcs}{zref@def@extract}%
%     \M{cmd} \M{refname} \M{propname}\\
%   \cs{zref@def@extractdefault}
%     \M{cmd} \M{refname} \M{propname} \M{default}
% \end{declcs}
% Both macros extract the property \meta{propname} from the reference
% \meta{refname} the same way as macros \cs{zref@extract} and
% \cs{zref@extractdefault}. The result is stored in macro \meta{cmd}.
% Also \cs{zref@refused} is called to notify \hologo{LaTeX} that
% the reference \meta{refname} is used. Added in 2011/10/04 v2.22.
%
% \begin{declcs}{zref@ifrefundefined}\markexp
%   \M{refname} \M{then} \M{else}
% \end{declcs}
% Macro \cs{zref@ifrefundefined} calls arguments \meta{then} or \meta{else}
% dependent on the existence of the reference \meta{refname}.
%
% \begin{declcs}{zifrefundefined}
%   \M{refname} \M{then} \M{else}
% \end{declcs}
% Macro \cs{zifrefundefined} calls \cs{ref@refused} before
% executing \cs{zref@ifrefundefined}. Babel shorthands are
% supported in \meta{refname}.
%
% \begin{declcs}{zref@ifrefcontainsprop}\markexp
%   \M{refname} \M{propname} \M{then} \M{else}
% \end{declcs}
% Test whether a reference provides a property.
%
% \subsection{Setup}
%
% \begin{declcs}{zref@default}
% \end{declcs}
% Holds the global default for unknown values.
%
% \begin{declcs}{zref@setdefault}
%   |{|\meta{value}|}|
% \end{declcs}
% Sets the global default for unknown values.
% The global default is used, if a property does not specify
% an own default and the value for a property
% cannot be extracted. This can happen if the reference is
% unknown or the reference does not have the property.
%
% \begin{declcs}{zref@setmainlist}
%   |{|\meta{value}|}|
% \end{declcs}
% Sets the name of the main property list. The package
% sets and uses |main|.
%
% \subsection{Declared properties}
%
% \begin{quote}
% \renewcommand*{\arraystretch}{1.05}
% \setlength{\tabcolsep}{2\tabcolsep}
% \def\VV{\vphantom{\large \"Ag}}
% \def\EMPTY{{\itshape\textless empty\textgreater}}
% \def\pl#1{&#1&thepage&\cs{number}\cs{#1}}
% \begin{tabular}{@{}>{\sffamily}l|>{\ttfamily}l|>{\ttfamily}l|>{\ttfamily}l@{}}
%   \rmfamily Module&
%   \rmfamily Property&
%   \rmfamily Property list&
%   \rmfamily Default\\
%   \hline
%   (base)
%     & default & main & \EMPTY\\
%     & page & main & \EMPTY\\
%   \hline
%   abspage
%     & abspage & main & 0\\
%   \hline
%   counter
%     & counter & main & \EMPTY\\
%   \hline
%   hyperref
%     & anchor & main & \EMPTY\\
%     & url    & & \EMPTY\\
%   \hline
%   pageattr
%     & pdfpageattr & thepage & \dots\\
%     & pdfpagesattr & LastPage & \dots\\
%   \hline
%   pagelayout\footnotemark
%     \pl{mag}\\
%     \pl{paperwidth}\\
%     \pl{paperheight}\\
%     \pl{stockwidth}\\
%     \pl{stockheight}\\
%     \pl{pdfpageheight}\\
%     \pl{pdfpagewidth}\\
%     \pl{pdfhorigin}\\
%     \pl{pdfvorigin}\\
%     \pl{hoffset}\\
%     \pl{voffset}\\
%     \pl{topmargin}\\
%     \pl{oddsidemargin}\\
%     \pl{evensidemargin}\\
%     \pl{textwidth}\\
%     \pl{textheight}\\
%     \pl{headheight}\\
%     \pl{headsep}\\
%     \pl{footskip}\\
%     \pl{marginparwidth}\\
%     \pl{marginparsep}\\
%     \pl{columnwidth}\\
%     \pl{columnsep}\\
%   \hline
%   perpage
%     & pagevalue & perpage & 0\\
%     & page & perpage & \EMPTY\\
%     & abspage & perpage & 0\\
%   \hline
%   savepos
%     & posx & savepos & 0\\
%     & posy & savepos & 0\\
%   \hline
%   titleref
%     & title & main & \EMPTY\\
%   \hline
%   xr
%     & anchor   & & \EMPTY\\
%     & externaldocument & & \EMPTY\\
%     & theotype & & \EMPTY\\
%     & title    & & \EMPTY\\
%     & url      & & \EMPTY\\
% \end{tabular}
% \footnotetext{Module \xmodule{pagelayout} only defines properties
% if the parameter exists.}%
% \end{quote}
%
% \subsection{Wrapper for advanced situations}
%
% \begin{declcs}{zref@wrapper@babel}
%   |{|\dots|}| \M{name}
% \end{declcs}
% This macro helps to add shorthand support. The second
% argument is protected, then the code of the first argument
% is called with the protected name appended. Examples
% are in the sources.
%
% \begin{declcs}{zref@wrapper@immediate}
%   |{|\dots|}|
% \end{declcs}
% There are situations where a label must be written instantly
% to the \xfile{.aux} file, for example after the last page.
% If the \cs{zlabel} or \cs{label} command is put inside this wrapper,
% immediate writing is enabled. See the implementation for
% module \xmodule{lastpage} for an example of its use.
%
% \begin{declcs}{zref@wrapper@unexpanded}
%   |{|\dots|}|
% \end{declcs}
% Assuming someone wants to extract a value for property |bar|
% and store the result in a macro |\foo| without traces of
% the expanding macros and without expanding the value.
% This (theoretical?) problem can be solved by this wrapper:
% \begin{quote}
%\begin{verbatim}
%\zref@wrapper@unexpanded{%
%  \edef\foo{%
%    \zref@extract{someref}{bar}%
%  }%
%}
%\end{verbatim}
% \end{quote}
% The \cs{edef} forces the expansion of \cs{zref@extract},
% but the extraction of the value is prevented by
% the wrapper that uses \hologo{eTeX}'\ \cs{unexpanded} for this
% purpose. Supported macros are \cs{zref@extract}, \cs{zref@extractdefault}
% and since version 2010/04/22 v2.14 macro \cs{zref@getcurrent}.
%
% \subsection{Counter for unique names}
%
% Some modules (\xmodule{titleref} and \xmodule{dotfillmin})
% need unique names for automatically generated label names.
%
% \begin{declcs}{zref@require@unique}
% \end{declcs}
% This command creates the unique counter |zref@unique| if the
% counter does not already exist.
%
% \begin{declcs}{thezref@unique}
% \end{declcs}
% This command is used to generate unique label names.
%
% \section{User interface}\label{sec:user}
%
% \subsection{Module \xmodule{user}}
%
% The user interface for this package and its modules
% is enabled by \xpackage{zref}'s package option \xmodule{user}
% or package \xpackage{zref-user}.
% The names of user commands are prefixed by |z| in order
% to avoid name clashes with existing macros of the same
% functionality. Thus the package does not disturb the
% traditional reference scheme, both can be used together.
%
% The syntax descriptions contain the following markers
% that are intended as hints for programmers:
% \begin{quote}
% \begin{tabular}{@{}ll@{}}
%   \markbabel\unskip  & Babel shorthands are allowed.\\
%   \markrobust\unskip & Robust macro.\\
%   \markexp\unskip    & Expandable version:\\
%                      & \textbullet\ robust,
%                        unless the extracted values are fragile,\\
%                      & \textbullet\ no babel shorthand suport.\\
%   \markexptwo\unskip & Expandable like \markexp and:\\
%                      & \textbullet\ expandable in exact two steps.\\
% \end{tabular}
% \end{quote}
%
% The basic user interface of the package without modules
% are commands that mimic the standard \hologo{LaTeX} behaviour
% of \cs{label}, \cs{ref}, and \cs{pageref}:
%
% \begin{declcs}{zlabel}
%   \M{refname}\markbabel
% \end{declcs}
% Similar to \cs{label}. It generates a label with name
% \meta{refname} in the new reference scheme.
%
% \begin{declcs}{zref}
%   |[|\meta{propname}|]| \M{refname}\markbabel
% \end{declcs}
% Without optional argument similar to \cs{ref}, it returns
% the default reference property. This property is named
% |default|:
% \[
%   |\zref{|x|}| \equiv |\zref[default]{|x|}|
% \]
%
% \begin{declcs}{zpageref}
%   \M{refname}\markbabel
% \end{declcs}
% Convenience macro, similar to \cs{pageref}.
% \[
%   |\zpageref{|x|}| \equiv |\zref[page]{|x|}|
% \]
%
% \begin{declcs}{zrefused}
%   \M{refname}\markbabel
% \end{declcs}
% Some of the user commands in the modules are expandable.
% The use of such commands do not cause any undefined
% reference warnings, because inside of expandable
% contexts this is not possible. However, if there is
% a place outside of expandable contexts, \cs{refused}
% is strongly recommended. The reference \meta{refname}
% is marked as used, undefined ones will generate
% warnings.
%
% \subsection{Module \xmodule{abspage}}
%
% With the help of package \xpackage{atbegshi}
% a new counter |abspage| with absolute page numbers is provided.
% For technical and historical reasons the counter itself is zero based:
% if you use it directly on the first page, e.g
% with |\arabic{abspage}| you will get 0 as value. When using |\zref|
% the first page will be page 1 as expected.
% Also a new property |abspage| is defined and added to the
% main property list. Thus you can reference the absolute
% page number:
% \begin{quote}
%   |Section \zref{foo} is on page \zpageref{foo}.|\\
%   |This is page \zref[abspage]{foo}|\\
%   |of \zref[abspage]{LastPage}.|
% \end{quote}
% The example also makes use of module \xmodule{lastpage}.
%
% \subsection{Module \xmodule{lastpage}}
%
% Provides the functionality of package \xpackage{lastpage}
% \cite{lastpage} in the new reference scheme.
% The label |LastPage| is put at the end of the document.
% You can refer the last page number with:
% \begin{quote}
%   |\zref@extract{LastPage}{page}| (+ |\zref@refused{LastPage}|)\\
% \end{quote}
% or
% \begin{quote}
%   |\zpageref{LastPage}| (module \xmodule{user})
% \end{quote}
%
% Since version 2008/10/01 v2.3 the module defines the list |LastPage|.
% In addition to the properties of the main list label |LastPage| also
% stores the properties of this list |LastPage|. The default of this
% list is empty. The list can be used by the user to add additional
% properties for label |LastPage|.
%
% \subsubsection{Tests for last page}
%
% Since version 2010/03/26 v2.8 the macros \cs{zref@iflastpage}
% and \cs{ziflastpage} were added. They test the reference,
% whether it is a reference of the last page.
%
% \begin{declcs}{zref@iflastpage}\markexp \M{refname} \M{then} \M{else}
% \end{declcs}
% Macro \cs{zref@iflastpage} compares the references \meta{refname}
% with \meta{LastPage}. Basis of the comparison is the value of
% property |abspage|, because the values are different for different
% pages. This is not ensured by property |page|. Therefore module
% \xmodule{abspage} is loaded by module \xmodule{lastpage}.
% If both values of property |abspage| are present and match,
% then \meta{then} is executed, otherwise code \meta{else} is called.
% If one or both references are undefined or lack the property |abspage|,
% then \meta{else} is executed.
%
% Macro \cs{zref@iflastpage} is expandable, therefore \cs{zref@refused}
% should be called on \meta{refname} and \meta{LastPage}.
%
% \begin{declcs}{ziflastpage} \M{refname} \M{then} \M{else}
% \end{declcs}
% Macro \cs{ziflastpage} has the same function as \cs{zref@iflastpage},
% but adds support for babel shorthands in \meta{refname} and
% calls \cs{zref@refused}. However macro \cs{ziflastpage} is not
% expandable.
%
% \subsubsection{Example}
%
%    \begin{macrocode}
%<*example-lastpage>
%<<END_EXAMPLE
\NeedsTeXFormat{LaTeX2e}
\documentclass{report}

\newcounter{foo}
\renewcommand*{\thefoo}{\Alph{foo}}

\usepackage{zref-lastpage,zref-user}[2019/11/29]

\makeatletter
\zref@newprop{thefoo}{\thefoo}
\zref@newprop{valuefoo}{\the\value{foo}}
\zref@newprop{chapter}{\thechapter}
\zref@addprops{LastPage}{thefoo,valuefoo,chapter}
\makeatother

\newcommand*{\foo}{%
  \stepcounter{foo}%
  [Current foo: \thefoo]%
}

\begin{document}
  \chapter{First chapter}
  Last page is \zref{LastPage}.\\
  Last chapter is \zref[chapter]{LastPage}.\\
  Last foo is \zref[thefoo]{LastPage}.\\
  Last value of foo is \zref[valuefoo]{LastPage}.\\
  \foo
  \chapter{Second chapter}
  \foo\foo\foo
  \chapter{Last chapter}
  \foo
\end{document}
%END_EXAMPLE
%</example-lastpage>
%    \end{macrocode}
%
% \subsection{Module \xmodule{thepage}}
%
% This module \xmodule{thepage} loads module \xmodule{abspage},
% constructs a reference name using the absolute page number
% and remembers property |page|. Other properties can be added
% by adding them to the property list |thepage|.
%
% \begin{declcs}{zthepage} \M{absolute page number}\\
% \end{declcs}
% Macro \cs{zthepage} is basically a \cs{zpageref}. The reference
% name is yield by the \meta{absolute page number}. If the reference
% is not defined, then the default for property |page| is used.
%
% \begin{declcs}{zref@thepage@name}\markexp \M{absolute page number}
% \end{declcs}
% Macro \cs{zref@thepage@name} returns the internal reference
% name that is constructed using the \meta{absolute page number}.
% The internal reference name should not be used directly, because
% it might change in future versions.
%
% \begin{declcs}{zref@thepage}\markexp \M{absolute page number}\\
%   \cs{zref@thepage@refused} \M{absolute page number}
% \end{declcs}
% Macro \cs{zref@thepage} returns the page number (\cs{thepage})
% of \meta{absolute page number}. Because this macro is expandable,
% \cs{zref@thepage@refused} is used outside an expandable context
% to mark the reference as used.
%
%
% \subsection{Module \xmodule{nextpage}}
%
% \begin{declcs}{znextpage}
% \end{declcs}
% Macro \cs{znextpage} prints \cs{thepage} of the following page.
% It gets the current absolute page number by using a label.
% There are three cases for the next page:
% \begin{enumerate}
% \item The next page is not known yet because of undefined references.
%   Then \cs{zunknownnextpagename} is used instead. The default for
%   this macro is the default of property |page|.
% \item This page is the last page. Then \cs{znonextpagename}
%   is used. Its default is empty.
% \item The next page is known, then \cs{thepage} of the next
%   page is used (the value of property |page| of the next page).
% \end{enumerate}
%
% \subsubsection{Configuration}
%
% The behaviour can be configured by the following macros.
%
% \begin{declcs}{zunknownnextpagename}\\
%   \cs{znonextpagename}
% \end{declcs}
% If the next page is not known or available, then
% \cs{znextpage} uses these name macros as default.
% \cs{zunknownnextpagename} is used in case of undefined
% references. Default is the value of property |page| of the
% next page (\cs{thepage}). Module \xmodule{thepage} is used.
%
% Macro \cs{znonextpagename} is used, if the next page does
% not exists. That means that the current page is last page.
% The default is empty.
%
% \begin{declcs}{znextpagesetup} \M{unknown} \M{no next}
%   \M{next}
% \end{declcs}
% Acording to the case (see \cs{znextpage}) macro \cs{znextpage}
% calls an internal macro with an argument. The argument is
% either \cs{thepage} of the next page or one of \cs{zunknownnextpagename}
% or \cs{znonextpagename}. These internal macro can be changed by
% \cs{znextpagesetup}. It expects the definition texts for these
% three cases of a macro with one argument.
% The default is
% \begin{quote}
%   |\znextpagesetup{#1}{#1}{#1}|
% \end{quote}
%
% \subsubsection{Example}
%
%    \begin{macrocode}
%<*example-nextpage>
%<<END_EXAMPLE
\documentclass{book}

\usepackage{zref-nextpage}[2019/11/29]
\znextpagesetup
  {\thepage}% next page is unknown
  {\thepage\ (#1)}% this page is last page
  {\thepage\ $\rightarrow$ #1}% next page is known
\renewcommand*{\znonextpagename}{last page}

\usepackage{fancyhdr}
\pagestyle{fancy}
\fancyhf{}
\fancyhead[LE,RO]{\znextpage}
\fancypagestyle{plain}{%
  \fancyhf{}%
  \fancyhead[LE,RO]{\znextpage}%
}

\begin{document}
\frontmatter
  \tableofcontents
\mainmatter
  \chapter{Hello World}
  \clearpage
  \section{Last section}
\end{document}
%END_EXAMPLE
%</example-nextpage>
%    \end{macrocode}
%
%
% \subsection{Module \xmodule{totpages}}
%
% For the total number of pages of a document you need to
% know the absolute page number of the last page.
% Both modules \xmodule{abspage} and \xmodule{lastpage}
% are necessary and automatically enabled.
%
% \begin{declcs}{ztotpages}\markexp
% \end{declcs}
% Prints the total number of pages or |0| if this
% number is not yet known. It expands to an explicit number
% and can also used even in expandable calculations (\cs{numexpr})
% or counter assignments.
%
% \subsection{Module \xmodule{pagelayout}}
%
% The module defines additional properties for each parameter
% of the page layout that is effective during page shipout.
% The value of length parameters is given in sp without the
% unit as plain number.
%
% Some parameters are specific for a class (e.g.\@ \xoption{stockwidth}
% and \xoption{stockheight} for class \xclass{memoir}) or the
% \hologo{TeX} engine like \hologo{pdfTeX}. If the parameter is not
% available, then the property will not be defined. The default value
% of the property is the current setting of the parameter.
%
% The module \xmodule{thepage} is loaded that generates a label
% for each page. The properties of module \xmodule{pagelayout}
% are added to the property list |thepage| of module \xmodule{thepage}.
%
% List of properties:
% \begin{quote}
%   \def\x#1{\cs{#1}&#1}%
%   \begin{tabular}{l>{\sffamily}ll}
%     \textrm{parameter} & \textrm{property} & remarks\\
%     \hline
%     \x{mag}\\
%     \x{paperwidth}\\
%     \x{paperheight}\\
%     \x{stockwidth}& class \xclass{memoir}\\
%     \x{stockheight}& class \xclass{memoir}\\
%     \x{pdfpagewidth}& \hologo{pdfTeX}, \hologo{LuaTeX}\\
%     \x{pdfpageheight}& \hologo{pdfTeX}, \hologo{LuaTeX}\\
%     \x{pdfhorigin}& \hologo{pdfTeX}, \hologo{LuaTeX}\\
%     \x{pdfvorigin}& \hologo{pdfTeX}, \hologo{LuaTeX}\\
%     \x{hoffset}\\
%     \x{voffset}\\
%     \x{topmargin}\\
%     \x{oddsidemargin}\\
%     \x{evensidemargin}\\
%     \x{textwidth}\\
%     \x{textheight}\\
%     \x{headheight}\\
%     \x{headsep}\\
%     \x{footskip}\\
%     \x{marginparwidth}\\
%     \x{marginparsep}\\
%     \x{columnwidth}\\
%     \x{columnsep}\\
%   \end{tabular}
% \end{quote}
%
% \begin{declcs}{zlistpagelayout}
% \end{declcs}
% At the end of document the page layout parameter for each page are
% printed into the \xext{log} file if macro \cs{zlistpagelayout}
% is called before |\end{document}| (preamble is a good place).
%
% \subsection{Module \xmodule{marks}}
%
% ToDo.
%
% \subsection{Module \xmodule{runs}}
%
% Module \xmodule{runs} counts the \hologo{LaTeX} runs since last
% \xext{aux} file creation and prints the number in the
% \xext{log} file.
%
% \begin{declcs}{zruns}\markexp
% \end{declcs}
% Prints the the total number of \hologo{LaTeX} runs including
% the current one. It expands to an explicit number.
% Before |begin{document}| the value is zero meaning
% the \xext{aux} file is not read yet. If a previous
% \xext{aux} file exists, the value found there increased
% by one is the new number. Otherwise \cs{zruns} is set to one.
% \hologo{LaTeX}~runs where the \xext{aux} files are not rewritten
% are not counted (see \cs{nofiles}).
%
% \subsection{Module \xmodule{perpage}}
%
% With \cs{@addtoreset} or \cs{numberwithin} a counter can be
% reset if another counter is incremented. This do not work
% well if the other counter is the page counter. The page counter is
% incremented in the output routine that is often called asynchronous
% somewhere on the next page. A reference mechanism costs at least
% two \hologo{LaTeX} runs, but ensures correct page counter values.
%
% \begin{declcs}{zmakeperpage}
%   |[|\meta{reset}|]| \M{counter}
% \end{declcs}
% At the of a new page counter \meta{counter} starts
% counting with value \meta{reset} (default is \texttt{1}).
% The macro has the same
% syntax and semantics as \cs{MakePerPage} of package
% \xpackage{perpage} \cite{perpage}.
% Also \xoption{perpage} of package \xpackage{footmisc} \cite{footmisc}
% can easily be simulated by
% \begin{quote}
% |\zmakeperpage{footnote}| \itshape |% \usepackage[perpage]{footmisc}|
%\end{quote}
% If footnote symbols are used, some people dislike the
% first symbol \ensuremath{\dagger}. It can easily be skipped:
% \begin{quote}
% |\zmakeperpage[2]{footnote}|
% \end{quote}
%
% \begin{declcs}{thezpage}\\
%   counter |zpage|
% \end{declcs}
% If the formatted counter value of the counter that is reset at
% a new page contains the page value, then you can use \cs{thezpage},
% the page number of the current page. Or counter |zpage| can be
% used, if the page number should be formatted differently from
% the current page number. Example:
%\begin{quote}
%\begin{verbatim}
%\newcounter{foobar}
%\zmakeperpage{foobar}
%\renewcommand*{\thefoobar}{\thezpage-\arabic{foobar}}
% % or
%\renewcommand*{\thefoobar}{\roman{zpage}-\arabic{foobar}}
%\end{verbatim}
%\end{quote}
%
% \begin{declcs}{zunmakeperpage}
%   \M{counter}
% \end{declcs}
% The reset mechanism for this counter is deactivated.
%
%
% \subsection{Module \xmodule{counter}}
%
% This option just add the property |counter| to the main
% property list. The property stores the counter name,
% that was responsible for the reference. This is the
% property \xpackage{hyperref}'s \cs{autoref} feature
% uses. Thus this property |counter| may be useful
% for a reimplementation of the autoref feature,
% see the section \ref{sec:todo} with the todo list.
%
% \subsection{Module \xmodule{titleref}}
%
% This option makes section and caption titles available
% to the reference system similar to packages \xpackage{titleref}
% or \xpackage{nameref}.
%
% \begin{declcs}{ztitleref}
%   \M{refname}\markbabel
% \end{declcs}
% Print the section or caption title of reference \meta{refname},
% similar to \cs{nameref} or \cs{titleref}.
%
% \begin{declcs}{ztitlerefsetup}
%   |{|\textit{key$_1$=value$_1$, key$_2$=value$_2$, \dots}|}|
% \end{declcs}
% This command allows to configure the behaviour of module
% \xmodule{titleref}. The following keys are available:
% \begin{description}
% \let\bfseries\relax
% \item[\texttt{title=}\meta{value}]\mbox{}\\*
%   Sets the current title.
% \item[\texttt{stripperiod=true\string|false}]\mbox{}\\*
%   Follow package \xpackage{nameref} that
%   removes a last period. Default: |true|.
% \item[\texttt{expand=true\string|false}]\mbox{}\\*
%   Package \cs{titleref} expands the title first. This way
%   garbage and dangerous commands can be removed, e.g.
%   \cs{label}, \cs{index}\dots. See implementation section
%   for more details. Default is |false|.
% \item[\texttt{cleanup=\{}\textrm{\dots}\texttt{\}}]\mbox{}\\*
%   Hook to add own cleanup code, if method |expand| is used.
%   See implementation section for more details.
% \end{description}
%
% \subsection{Module \xmodule{savepos}}
%
% This option supports a feature that \pdfTeX\ provides
% (and \hologo{XeTeX}).
% \pdfTeX\ is able to tell the current position on the
% page. The page position is not instantly known. First the
% page must be constructed by \hologo{TeX}'s asynchronous output
% routine. Thus the time where the position is known is
% the page shipout time. Thus a reference system where the
% information is recorded in the first run and made available
% for use in the second run comes in handy.
%
% \begin{declcs}{zsavepos}
%   \M{refname}
% \end{declcs}
% It generates a reference with name \meta{refname}.
% The reference stores the location where \cs{zsavepos}
% is executed in properties |posx| and |posy|.
%
% \begin{declcs}{zsaveposx}
%   \M{refname}\\
%   \cs{zsaveposy}
%   \M{refname}
% \end{declcs}
% Same as \cs{zsavepos} except that only the |x| or |y| component
% of the position is stored. Since 2011/12/05 v2.23.
%
% \begin{declcs}{zposx}\markexp
%   \M{refname}\\
%   \SpecialUsageIndex{\zposy}\cs{zposy}\markexp \M{refname}
% \end{declcs}
% Get the position as number. Unit is sp. Horizontal
% positions by \cs{zposx} increase from left to right.
% Vertical positions by \cs{zposy} from bottom to top.
%
% Do not rely on absolute page numbers. Because of problems
% with the origin the numbers may differ in DVI or PDF mode
% of \pdfTeX. Therefore work with relative values by comparisons.
%
% Both \cs{zposx} and \cs{zposy} are expandable and can be
% used inside calculations (\cs{setcounter}, \cs{addtocounter},
% package \xpackage{calc}, \cs{numexpr}). However this property
% prevents from notifying \hologo{LaTeX} that the reference is actually
% used (the notifying is not expandable). Therefore you should
% mark the reference as used by \cs{zrefused}.
%
% This module uses \pdfTeX's \cs{pdfsavepos}, \cs{pdflastxpos},
% and \cs{pdflastypos}. They are available in PDF mode and
% since version 1.40.0 also in DVI mode.
%
% \begin{declcs}{zref@savepos}
% \end{declcs}
% Macro \cs{zref@savepos} performs the first part of \cs{zsavepos}
% by calling \cs{pdfsavepos} (if \xext{aux} files are writable).
%
% Thus \cs{zsavepos} is basically \cs{zref@savepos} followed
% by |\zref@labelbylist{|\texttt{\meta{refname}}|}{savepos}|.
% If \cs{TeXXeTstate} is detected and enabled, \cs{savepos}
% also adds \cs{zref@savepos} at the end to support \cs{beginR}
% where the whatits are processed in reverse order.
% The property list |savepos| contains the properties |posx|
% and |posy|.
%
% \subsection{Module \xmodule{abspos}}
% Module \xmodule{abspos} allows to get various values of the page layout. There is no user
% command, only a number of internal commands. For example:
% \begin{quote}
%   |\zref@absposx{|\meta{label}|}{|\meta{value}|}{|\meta{position}|}|\\
%   |\zref@absposy{|\meta{label}|}{|\meta{value}|}{|\meta{position}|}|
% \end{quote}
%
% The return value is like in the module \xmodule{savepos} a number representing a length in sp.
% The length are measured from the bottom left of the page.
%
% \meta{label} is a label set with \cs{zlabel} or \cs{zsavepos} that allows to retrieve the
% absolute page number.
%
% \meta{position} is for the x-command one of |left|, |right| or |center|. For the y-command
%  it is one of |top|, |bottom|, |center|.
%
% The possible content of \meta{value} can be seen in the following table. Be aware that the code
% makes some assumptions which are perhaps not always true -- for example that the left of
% the head is identical to the left of the body.
%
% \medskip
% \begin{tabular}{lll}
%  \textbf{value} & \textbf{axis} & \textbf{comments}\\\hline
%   media         & x             &   left=0, right=\cs{pdfpagewidth} \\
%   paper         & x             &   left=0, right=\cs{paperwidth}\\
%   stock         & x             &   derived from paper\\
%   media         & y             & bottom=0, top=\cs{pdfpageheigh}\\
%   paper         & y             & top=\cs{pdfpageheight}, bottom=top-\cs{paperheight}\\
%   stock         & y             & top  derived from paper\\
%   head          & x             & calculated with hoffset, horigin, etc\\
%   head          & y             & calculated\\
%   body          & x             & = head value\\%
%   body          & y             & = head bottom - \cs{headsep}\\
%   foot          & x             & = head\\
%   foot          & y             & calculated from body bottom and \cs{footskip}\\
%   marginpar     & x             &different on odd/even pages! \\
%   marginpar     & y             & = body %
% \end{tabular}
%
% \subsection{Module \xmodule{dotfill}}
%
% \begin{declcs}{zdotfill}
% \end{declcs}
% This package provides the command \cs{zdotfill} that works
% similar to \cs{dotfill}, but can be configured. Especially
% it suppresses the dots if a minimum number of dots cannot be set.
%
% \begin{declcs}{zdotfillsetup}
%   |{|\textit{key$_1$=value$_1$, key$_2$=value$_2$, \dots}|}|
% \end{declcs}
% This command allows to configure the behaviour of \cs{zdotfill}.
% The following keys are available:
% \begin{description}
% \let\bfseries\relax
% \item[\texttt{min=}\meta{count value}]\mbox{}\\*
%   If the actual number of dots are smaller than \meta{count value},
%   then the dots are suppressed. Default: |2|.
% \item[\texttt{unit=}\meta{dimen value}]\mbox{}\\*
%   The width of a dot unit is given by \meta{dimen value}.
%   Default: |0.44em| (same as the unit in \cs{dotfill}).
% \item[\texttt{dot=}\meta{value}]\mbox{}\\*
%   The dot itself is given by \meta{value}. Default: |.| (dot, same as
%   the dot in \cs{dotfill}).
% \end{description}
%
% \subsection{Module \xmodule{env}}
%
% This module defines two properties |envname| and |envline|. They
% remember the name of the environment and the line number at
% the start of the environment.
%
% \subsection{Module \xmodule{xr}}
%
% This package provides the functionality of package \xpackage{xr},
% see \cite{xr}. It also supports the syntax of \xpackage{xr-hyper}.
%
% \begin{declcs}{zexternaldocument}
%   \* |[|\meta{prefix}|]|\markbabel \M{external document}
%   |[|\meta{url}|]|
% \end{declcs}
% See \cs{externaldocument} for a description of this option.
% The found labels also get a property |externaldocument| that
% remembers \meta{external document}.
% The standard reference scheme and the scheme of this
% package use different name spaces for reference names.
% If the external document uses both systems. Then one
% import statement would put the names in one namespace
% and probably causing problems with multiple references
% of the same name. Thus the star form only looks for
% \cs{newlabel} in the \xfile{.aux} files, whereas without
% star only \cs{zref@newlabel}s are used.
%
% In the star form it tries to detect labels from
% \xpackage{hyperref}, \xpackage{titleref}, and \xpackage{ntheorem}.
% If such an extended property from the packages before cannot
% be found or are empty, they are not included in the
% imported reference.
%
% Warnings are given if a reference name is already in use
% and the item is ignored. Unknown properties will automatically
% be declared.
%
% If the external references contain |anchor| properties,
% then we need also a url to be able to address the external
% file. As default the filename is taken with a default
% extension.
%
% \begin{declcs}{zxrsetup}
%   |{|\textit{key$_1$=value$_1$, key$_2$=value$_2$, \dots}|}|
% \end{declcs}
% The following setup options are available:
% \begin{description}
% \item[\xoption{ext}:] It sets the default extension.
% \item[\xoption{tozreflabel}:] Boolean option. The found references
%   are imported as zref labels. This is enabled by default.
% \item[\xoption{toltxlabel}:] Boolean option. The found references
%   are imported as \hologo{LaTeX} labels. Packages \xpackage{nameref},
%   \xpackage{titleref} and class \xclass{memoir} are supported.
% \item[\xoption{urluse}:] Boolean option. If enabled, then a URL
%   is stored in a macro and the macro is put in property `urluse'.
%   The URL is not put in property `url'.
%   The purpose is to save \hologo{TeX} memory.
% \item[\xoption{verbose}:] Boolean option. List the imported labels in the
%   \xext{log} file. Default is |false|.
% \end{description}
%
% \begin{declcs}{zref@xr@ext}
% \end{declcs}
% If the \meta{url} is not specified in \cs{zref@externaldocument},
% then the url will be constructed with the file name and
% this macro as extension. \cs{XR@ext} is used if
% \xpackage{hyperref} is loaded, otherwise |pdf|.
%
% \subsection{Module \xmodule{pageattr}}
%
% This module allows to recover the content of the register
% \cs{pdfpageattr} and \cs{pdfpagesattr} in pdftex and the equivalent register
% in luatex. There is no user command. Programmer commands are
%
% \begin{quote}
% |\zref@pdfpageattr{|\meta{absolute page number}|}|
%
% |\zref@pdfpagesattr{|\meta{absolute page number}|}|
% \end{quote}
% \section{ToDo}\label{sec:todo}
%
% Among other things the following issues are left for future work:
% \begin{itemize}
% \item Other applications: autoref, hyperref, \dots
% \end{itemize}
%
% \section{Example}
%
%    \begin{macrocode}
%<*example>
%    \end{macrocode}
%
%    \begin{macrocode}
\documentclass{book}

\usepackage[ngerman]{babel}%

\usepackage[savepos,totpages,titleref,dotfill,counter,user]{zref}

%    \end{macrocode}
%
%    Chapters are wrapped inside \cs{ChapterStart} and
%    \cs{ChapterStop}. The first argument \verb|#1| of \cs{ChapterStart}
%    is used to form a label id \verb|chap:#1|. At the end of
%    the chapter another label is set by \cs{zref@wrapper@immediate}, because
%    otherwise at the end of document a deferred write would not
%    be written, because there is no page for shipout.
%
%    Also this example shows how chapter titles can be recorded.
%    A new property \verb|chaptitle| is declared and added to the
%    main property list. In \cs{ChapterStart} the current value
%    of the property is updated.
%    \begin{macrocode}
\makeatletter
\zref@newprop{chaptitle}{}
\zref@addprop{main}{chaptitle}

\newcommand*{\ChapterStart}[2]{%
  \cleardoublepage
  \def\current@chapid{#1}%
  \zref@setcurrent{chaptitle}{#2}%
  \chapter{#2}%
  \zlabel{chap:#1}%
}
\newcommand*{\ChapterStop}{%
  \cleardoublepage
  \zref@wrapper@immediate{%
    \zref@labelbyprops{chapend:\current@chapid}{abspage}%
  }%
}
%    \end{macrocode}
%    \cs{ChapterPages} calculates and returns the number of pages of the
%    referenced chapter.
%    \begin{macrocode}
\newcommand*{\ChapterPages}[1]{%
  \zrefused{chap:#1}%
  \zrefused{chapend:#1}%
  \number\numexpr
    \zref@extract{chapend:#1}{abspage}%
    -\zref@extract{chap:#1}{abspage}%
    +1\relax
}
\makeatother
%    \end{macrocode}
%    \begin{macrocode}
\begin{document}
%    \end{macrocode}
%    As exception we use \cs{makeatletter} here, because
%    this is just an example file that also should show some of
%    programmer's interface.
%    \begin{macrocode}
\makeatletter

\frontmatter
\zlabel{documentstart}

\begin{itemize}
\item
  The frontmatter part has
  \number\numexpr\zref@extract{chap:first}{abspage}-1\relax
  ~pages.
\item
  Chapter \zref{chap:first} has \ChapterPages{first} page(s).
\item
  Section \zref{hello} is on the
  \ifcase\numexpr
    \zref@extractdefault{hello}{page}{0}%
    -\zref@extractdefault{chap:first}{page}{0}%
    +1\relax
    ??\or first\or second\or third\or forth\fi
  ~page inside its chapter.
\item
  The document has
  \zref[abspage]{LastPage} pages.
  This number is \ifodd\ztotpages odd\else even\fi.
\item
  The last page is labeled with \zpageref{LastPage}.
\item
  The title of chapter \zref{chap:next} %
  is ``\zref[chaptitle]{chap:next}''.
\end{itemize}

\tableofcontents

\mainmatter
\ChapterStart{first}{First chapter}

%    \end{macrocode}
%    The user level commands should protect babel shorthands
%    where possible. On the other side, expandable extracting
%    macros are useful in calculations, see above the
%    examples with \cs{numexpr}.
%    \begin{macrocode}
\section{Test}
\zlabel{a"o}
Section \zref{a"o} on page
\zref@wrapper@babel\zref@extract{a"o}{page}.

Text.
\newpage

\section{Hello World}
\zlabel{hello}

\ChapterStop

\ChapterStart{next}{Next chapter with \emph{umlauts}: "a"o"u"s}

%    \end{macrocode}
%
%    Here an example follows that makes use of \pdfTeX's ``savepos''
%    feature. The position on the page is not known before the
%    page is constructed and shipped out. Therefore the position
%    ist stored in references and are available for calculations
%    in the next \hologo{LaTeX} compile run.
%    \begin{macrocode}
The width of the first column is
  \the\dimexpr \zposx{secondcol}sp - \zposx{firstcol}sp\relax,\\
the height difference of the two baselines is
  \the\dimexpr \zposy{firstcol}sp - \zposy{secondline}sp\relax:\\
\begin{tabular}{ll}
  \zsavepos{firstcol}Hello&\zsavepos{secondcol}World\\
  \zsavepos{secondline}Second line&foobar\\
\end{tabular}

%    \end{macrocode}
%    With \cs{zrefused} \hologo{LaTeX} is notified, if the references
%    are not yet available and \hologo{LaTeX} can generate the rerun
%    hint.
%    \begin{macrocode}
\zrefused{firstcol}
\zrefused{secondcol}
\zrefused{secondline}

\ChapterStop
%    \end{macrocode}
%    Test for module \cs{dotfill}.
%    \begin{macrocode}
\ChapterStart{dotfill}{Test for dotfill feature}
\newcommand*{\dftest}[1]{%
  #1&
  [\makebox[{#1}]{\dotfill}]&
  [\makebox[{#1}]{\zdotfill}]\\
}
\begin{tabular}{rll}
& [\verb|\dotfill|] & [\verb|\zdotfill|]\\
\dftest{0.43em}
\dftest{0.44em}
\dftest{0.45em}
\dftest{0.87em}
\dftest{0.88em}
\dftest{0.89em}
\dftest{1.31em}
\dftest{1.32em}
\dftest{1.33em}
\end{tabular}
\ChapterStop
\end{document}
%    \end{macrocode}
%
%    \begin{macrocode}
%</example>
%    \end{macrocode}
%
% \StopEventually{
% }
%
% \section{Implementation}
%
% \subsection{Package \xpackage{zref}}
%
% \subsubsection{Identification}
%
%    \begin{macrocode}
%<*package>
\NeedsTeXFormat{LaTeX2e}
\ProvidesPackage{zref}
  [2020-07-03 v2.32 A new reference scheme for LaTeX (HO)]%
%    \end{macrocode}
%
% \subsubsection{Load basic module}
%
%    \begin{macrocode}
\RequirePackage{zref-base}[2019/11/29]
%    \end{macrocode}
%    Abort package loading if \xpackage{zref-base} could not be
%    loaded successfully.
%    \begin{macrocode}
\@ifundefined{ZREF@base@ok}{\endinput}{}
%    \end{macrocode}
%
% \subsubsection{Process options}
%
%    Known modules are loaded and the release date is checked.
%    \begin{macrocode}
\def\ZREF@temp#1{%
  \DeclareOption{#1}{%
    \AtEndOfPackage{%
      \RequirePackage{zref-#1}[2019/11/29]%
    }%
  }%
}
\ZREF@temp{abspage}
\ZREF@temp{counter}
\ZREF@temp{dotfill}
\ZREF@temp{hyperref}
\ZREF@temp{lastpage}
\ZREF@temp{marks}
\ZREF@temp{nextpage}
\ZREF@temp{pageattr}
\ZREF@temp{pagelayout}
\ZREF@temp{perpage}
\ZREF@temp{runs}
\ZREF@temp{savepos}
\ZREF@temp{thepage}
\ZREF@temp{titleref}
\ZREF@temp{totpages}
\ZREF@temp{user}
\ZREF@temp{xr}
%    \end{macrocode}
%    \begin{macrocode}
\ProcessOptions\relax
%</package>
%    \end{macrocode}
%
%
% \subsection{Module \xmodule{base}}
%
% \subsubsection{Prefixes}
%
%    This package uses the following prefixes for macro names:
%    \begin{description}
%    \item[\cs{zref@}:] Macros of the programmer's interface.
%    \item[\cs{ZREF@}:] Internal macros.
%    \item[\cs{Z@L@}\metacs{listname}:]
%      The properties of the list \meta{listname}.
%    \item[\cs{Z@D@}\metacs{propname}:]
%      The default value for property \meta{propname}.
%    \item[\cs{Z@E@}\metacs{propname}:]
%      Extract function for property \meta{propname}.
%    \item[\cs{Z@X@}\metacs{propname}:]
%      Information whether a property value for property
%      \meta{propname} is expanded
%      immediately or at shipout time.
%    \item[\cs{Z@C@}\metacs{propname}:]
%      Current value of the property \meta{propname}.
%    \item[\cs{Z@R@}\metacs{labelname}:]
%      Data for reference \meta{labelname}.
%    \item[\cs{ZREF@org@}:] Original versions of patched commands.
%    \item[\cs{z}:] For macros in user land, defined if
%      module \xmodule{user} is set.
%    \end{description}
%    The following family names are used for keys defined
%    according to the \xpackage{keyval} package:
%    \begin{description}
%    \item[\texttt{ZREF@TR}:] Setup for module \xmodule{titleref}.
%    \end{description}
%
% \subsubsection{Identification}
%
%    \begin{macrocode}
%<*base>
\NeedsTeXFormat{LaTeX2e}
\ProvidesPackage{zref-base}%
  [2020-07-03 v2.32 Module base for zref (HO)]%
%    \end{macrocode}
%
% \subsubsection{Utilities}
%
%    \begin{macrocode}
\providecommand\IfFormatAtLeastTF{\@ifl@t@r\fmtversion}
\RequirePackage{ltxcmds}[2010/12/02]
\RequirePackage{infwarerr}[2010/04/08]
\RequirePackage{kvsetkeys}[2010/03/01]
\RequirePackage{kvdefinekeys}[2010/03/01]
\RequirePackage{pdftexcmds}[2010/04/01]
%    \end{macrocode}
%
%    \begin{macro}{\ZREF@name}
%    Several times the package name is used, thus we store it
%    in \cmd{\ZREF@name}.
%    \begin{macrocode}
\def\ZREF@name{zref}
%    \end{macrocode}
%    \end{macro}
%
%    \begin{macrocode}
\ltx@IfUndefined{protected}{%
  \RequirePackage{makerobust}[2006/03/18]%
%    \end{macrocode}
%    \begin{macro}{\ZREF@Robust}
%    \begin{macrocode}
  \def\ZREF@Robust#1#2{%
    \def\ZREF@temp{\MakeRobustcommand#2}%
    \afterassignment\ZREF@temp
    #1#2%
  }%
%    \end{macrocode}
%    \end{macro}
%    \begin{macrocode}
}{%
%    \end{macrocode}
%    \begin{macro}{\ZREF@Robust}
%    \begin{macrocode}
  \def\ZREF@Robust#1{%
    \protected#1%
  }%
%    \end{macrocode}
%    \end{macro}
%    \begin{macrocode}
}
%    \end{macrocode}
%    \begin{macro}{\ZREF@IfDefinable}
%    \begin{macrocode}
\def\ZREF@IfDefinable#1#2#3{%
  \@ifdefinable{#1}{%
    \ZREF@Robust{#2}#1#3%
  }%
}
%    \end{macrocode}
%    \end{macro}
%
%    \begin{macro}{\ZREF@UpdatePdfTeX}
%    \cs{ZREF@UpdatePdfTeX} is used as help message text in
%    error messages.
%    \begin{macrocode}
\def\ZREF@UpdatePdfTeX{Update pdfTeX.}
%    \end{macrocode}
%    \end{macro}
%
%    \begin{macro}{\ifZREF@found}
%    The following switch is usded in list processing.
%    \begin{macrocode}
\newif\ifZREF@found
%    \end{macrocode}
%    \end{macro}
%
%    \begin{macro}{\ZREF@patch}
%    Macro \cs{ZREF@patch} first checks the existence of
%    the command and safes it.
%    \begin{macrocode}
\def\ZREF@patch#1{%
  \ltx@IfUndefined{#1}{%
    \ltx@gobble
  }{%
    \expandafter\let\csname ZREF@org@#1\expandafter\endcsname
    \csname #1\endcsname
    \ltx@firstofone
  }%
}
%    \end{macrocode}
%    \end{macro}
%
% \subsubsection{Check for \hologo{eTeX}}
%
%    The use of \hologo{eTeX} should be standard nowadays for \hologo{LaTeX}.
%    We test for \hologo{eTeX} in order to use its features later.
%    \begin{macrocode}
\ltx@IfUndefined{eTeXversion}{%
  \PackageError\ZREF@name{%
    Missing support for eTeX; package is abandoned%
  }{%
    Use a TeX compiler that support eTeX and enable eTeX %
    in the format.%
  }%
  \endinput
}{}%
%    \end{macrocode}
%
%    \begin{macrocode}
\RequirePackage{etexcmds}[2007/09/09]
\ifetex@unexpanded
\else
  \PackageError\ZREF@name{%
    Missing e-TeX's \string\unexpanded.\MessageBreak
    Add \string\RequirePackage\string{etexcmds\string} before %
    \string\documentclass%
  }{%
    Probably you are using some package (e.g. ConTeXt) that %
    redefines \string\unexpanded%
  }%
  \expandafter\endinput
\fi
%    \end{macrocode}
%
% \subsubsection{Auxiliary file stuff}
%
%    We are using some commands in the \xfile{.aux} files.
%    However sometimes these auxiliary files are interpreted
%    by \hologo{LaTeX} processes that haven't loaded this package
%    (e.g. package \xpackage{xr}).
%    Therefore we provide dummy definitions.
%    \begin{macrocode}
\RequirePackage{auxhook}
\AddLineBeginAux{%
  \string\providecommand\string\zref@newlabel[2]{}%
}
%    \end{macrocode}
%    \begin{macro}{\ZREF@RefPrefix}
%    \begin{macrocode}
\def\ZREF@RefPrefix{Z@R}
%    \end{macrocode}
%    \end{macro}
%    \begin{macro}{\zref@newlabel}
%    For the implementation of \cs{zref@newlabel} we call
%    the same internal macro \cs{@newl@bel} that is used
%    in \cs{newlabel}. Thus we have for free:
%    \begin{itemize}
%    \item \cs{Z@R@}\metacs{labelname} is defined.
%    \item \hologo{LaTeX}'s check for multiple references.
%    \item \hologo{LaTeX}'s check for changed references.
%    \end{itemize}
%    \begin{macrocode}
\ZREF@Robust\edef\zref@newlabel{%
  \noexpand\@newl@bel{\ZREF@RefPrefix}%
}
%    \end{macrocode}
%    \end{macro}
%
% \subsubsection{Property lists}
%
%    \begin{macro}{\zref@newlist}
%    Property lists are stored as list of property names
%    enclosed in curly braces. \cs{zref@newlist}
%    creates a new list as empty list. Assignments to
%    property lists are global.
%    \begin{macrocode}
\ZREF@Robust\def\zref@newlist#1{%
  \zref@iflistundefined{#1}{%
    \@ifdefinable{Z@L@#1}{%
     \global\expandafter\let\csname Z@L@#1\endcsname\ltx@empty
     \PackageInfo\ZREF@name{New property list: #1}%
    }%
  }{%
    \PackageError\ZREF@name{%
      Property list `#1' already exists%
    }\@ehc
  }%
}
%    \end{macrocode}
%    \end{macro}
%
%    \begin{macro}{\zref@iflistundefined}
%    \cs{zref@iflistundefined} checks the existence of the
%    property list \verb|#1|. If the property list is present,
%    then \verb|#2| is executed and \verb|#3| otherwise.
%    \begin{macrocode}
\def\zref@iflistundefined#1{%
  \ltx@ifundefined{Z@L@#1}%
}
%    \end{macrocode}
%    \end{macro}
%    \begin{macro}{\zref@listexists}
%    \cs{zref@listexists} only executes \verb|#2| if the
%    property list \verb|#1| exists and raises an error
%    message otherwise.
%    \begin{macrocode}
\ZREF@Robust\def\zref@listexists#1{%
  \zref@iflistundefined{#1}{%
    \PackageError\ZREF@name{%
      Property list `#1' does not exist%
    }\@ehc
  }%
}
%    \end{macrocode}
%    \end{macro}
%
%    \begin{macro}{\zref@iflistcontainsprop}
%    \cs{zref@iflistcontainsprop} checks, whether a property \verb|#2| is
%    already present in a property list \verb|#1|.
%    \begin{macrocode}
\ZREF@Robust\def\zref@iflistcontainsprop#1#2{%
  \zref@iflistundefined{#1}{%
    \ltx@secondoftwo
  }{%
    \begingroup\expandafter\endgroup
    \expandafter\in@
    \csname#2\expandafter\expandafter\expandafter\endcsname
    \expandafter\expandafter\expandafter{\csname Z@L@#1\endcsname}%
    \csname ltx@\ifin@ first\else second\fi oftwo\endcsname
  }%
}
%    \end{macrocode}
%    \end{macro}
%
%    \begin{macro}{\zref@listforloop}
%    \begin{macrocode}
\def\zref@listforloop#1#2{%
  \zref@listexists{#1}{%
    \expandafter\expandafter\expandafter\@tfor
    \expandafter\expandafter\expandafter\zref@prop
    \expandafter\expandafter\expandafter:%
    \expandafter\expandafter\expandafter=%
    \csname Z@L@#1\endcsname
    \do{%
      \begingroup
        \escapechar=-1 %
        \edef\x{\endgroup
          \def\noexpand\zref@prop{%
            \expandafter\string\zref@prop
          }%
        }%
      \x
      #2\zref@prop
    }%
  }%
}
%    \end{macrocode}
%    \end{macro}
%
%    \begin{macro}{\zref@addprops}
%    \cs{zref@addprop} adds the properties \verb|#2| to the property
%    list \verb|#1|, if the property is not already in the list.
%    Otherwise a warning is given.
%    \begin{macrocode}
\ZREF@Robust\def\zref@addprops#1#2{%
  \zref@listexists{#1}{%
    \comma@parse{#2}{%
      \zref@propexists\comma@entry{%
        \zref@iflistcontainsprop{#1}\comma@entry{%
          \PackageWarning\ZREF@name{%
            Property `\comma@entry' is already in list `#1'%
          }%
        }{%
          \begingroup\expandafter\endgroup
          \expandafter\g@addto@macro
          \csname Z@L@#1\expandafter\endcsname
          \expandafter{\csname\comma@entry\endcsname}%
        }%
      }%
      \ltx@gobble
    }%
  }%
}
%    \end{macrocode}
%    \end{macro}
%    \begin{macro}{\zref@addprop}
%    \cs{zref@addprop} adds the property \verb|#2| to the property
%    list \verb|#1|, if the property is not already in the list.
%    Otherwise a warning is given.
%    \begin{macrocode}
\ZREF@Robust\def\zref@addprop#1#2{%
  \zref@listexists{#1}{%
    \zref@propexists{#2}{%
      \zref@iflistcontainsprop{#1}{#2}{%
        \PackageWarning\ZREF@name{%
          Property `#2' is already in list `#1'%
        }%
      }{%
        \begingroup\expandafter\endgroup
        \expandafter\g@addto@macro
        \csname Z@L@#1\expandafter\endcsname
        \expandafter{\csname#2\endcsname}%
      }%
    }%
  }%
}
%    \end{macrocode}
%    \end{macro}
%
%    \begin{macro}{\zref@localaddprops}
%    \begin{macrocode}
\ZREF@Robust\def\zref@localaddprops#1#2{%
  \zref@listexists{#1}{%
    \comma@parse{#2}{%
      \zref@propexists\comma@entry{%
        \zref@iflistcontainsprop{#1}\comma@entry{%
          \PackageWarning\ZREF@name{%
            Property `\comma@entry' is already in list `#1'%
          }%
        }{%
          \begingroup\expandafter\endgroup
          \expandafter\ltx@LocalAppendToMacro
          \csname Z@L@#1\expandafter\endcsname
          \expandafter{\csname\comma@entry\endcsname}%
        }%
      }%
      \ltx@gobble
    }%
  }%
}
%    \end{macrocode}
%    \end{macro}
%    \begin{macro}{\zref@localaddprop}
%    \begin{macrocode}
\ZREF@Robust\def\zref@localaddprop#1#2{%
  \zref@listexists{#1}{%
    \zref@propexists{#2}{%
      \zref@iflistcontainsprop{#1}{#2}{%
        \PackageWarning\ZREF@name{%
          Property `#2' is already in list `#1'%
        }%
      }{%
        \begingroup\expandafter\endgroup
        \expandafter\ltx@LocalAppendToMacro
        \csname Z@L@#1\expandafter\endcsname
        \expandafter{\csname#2\endcsname}%
      }%
    }%
  }%
}
%    \end{macrocode}
%    \end{macro}
%
%    \begin{macrocode}
\ltx@IfUndefined{pdf@strcmp}{%
%    \end{macrocode}
%    \begin{macro}{\zref@delprop}
%    \begin{macrocode}
  \ZREF@Robust\def\zref@delprop{%
    \ZREF@delprop\gdef
  }%
%    \end{macrocode}
%    \end{macro}
%    \begin{macro}{\zref@localdelprop}
%    \begin{macrocode}
  \ZREF@Robust\def\zref@localdelprop{%
    \ZREF@delprop\def
  }%
%    \end{macrocode}
%    \end{macro}
%    \begin{macro}{\ZREF@delprop}
%    \begin{macrocode}
  \def\ZREF@delprop#1#2#3{%
    \zref@listexists{#2}{%
      \begingroup
        \escapechar=-1 %
        \def\ZREF@param{#3}%
        \@onelevel@sanitize\ZREF@param
        \toks@{}%
        \expandafter\expandafter\expandafter\ZREF@@delprop
        \csname Z@L@#2\endcsname!%
      \expandafter\endgroup
      \expandafter#1\csname Z@L@#2\expandafter\endcsname
      \expandafter{%
        \the\toks@
      }%
    }%
  }%
%    \end{macrocode}
%    \end{macro}
%    \begin{macro}{\ZREF@@delprop}
%    \begin{macrocode}
  \def\ZREF@@delprop#1{%
    \expandafter\ZREF@@@delprop\expandafter{\string#1}#1%
  }%
%    \end{macrocode}
%    \end{macro}
%    \begin{macro}{\ZREF@@@delprop}
%    \begin{macrocode}
  \def\ZREF@@@delprop#1#2{%
    \ifx#2!%
    \else
      \def\ZREF@temp{#1}%
      \@onelevel@sanitize\ZREF@temp
      \ifx\ZREF@param\ZREF@temp
      \else
        \toks@\expandafter{%
          \the\expandafter\toks@\csname#1\endcsname
        }%
      \fi
      \expandafter\ZREF@@delprop
    \fi
  }%
%    \end{macrocode}
%    \end{macro}
%    \begin{macrocode}
}{%
%    \end{macrocode}
%    \begin{macro}{\zref@delprop}
%    \begin{macrocode}
  \ZREF@Robust\def\zref@delprop{%
    \ZREF@delprop\xdef
  }%
%    \end{macrocode}
%    \end{macro}
%    \begin{macro}{\zref@localdelprop}
%    \begin{macrocode}
  \ZREF@Robust\def\zref@localdelprop{%
    \ZREF@delprop\edef
  }%
%    \end{macrocode}
%    \end{macro}
%    \begin{macro}{\ZREF@delprop}
%    \begin{macrocode}
  \def\ZREF@delprop#1#2#3{%
    \zref@listexists{#2}{%
      \def\ZREF@param{#3}%
      \edef\ZREF@SavedEscapechar{\the\escapechar}%
      \escapechar=-1 %
      \expandafter#1\csname Z@L@#2%
      \expandafter\expandafter\expandafter\endcsname{%
        \expandafter\expandafter\expandafter\ZREF@@delprop
        \csname Z@L@#2\endcsname!%
      }%
      \escapechar=\ZREF@SavedEscapechar\relax
    }%
  }%
%    \end{macrocode}
%    \end{macro}
%    \begin{macro}{\ZREF@@delprop}
%    Caution: |#1| might be an \cs{if} or similar token.
%    \begin{macrocode}
  \def\ZREF@@delprop#1{%
    \expandafter\ZREF@@@delprop\expandafter{\string#1}#1%
  }%
%    \end{macrocode}
%    \end{macro}
%    \begin{macro}{\ZREF@@@delprop}
%    \begin{macrocode}
  \def\ZREF@@@delprop#1#2{%
    \ifx#2!%
    \else
      \ifnum\pdf@strcmp{#1}{\ZREF@param}=\ltx@zero
      \else
        \expandafter\noexpand\csname#1\endcsname
      \fi
      \expandafter\ZREF@@delprop
    \fi
  }%
%    \end{macrocode}
%    \end{macro}
%    \begin{macrocode}
}
%    \end{macrocode}
%
% \subsubsection{Properties}
%
%    \begin{macro}{\zref@ifpropundefined}
%    \cs{zref@ifpropundefined} checks the existence of the
%    property \verb|#1|. If the property is present,
%    then \verb|#2| is executed and \verb|#3| otherwise.
%    \begin{macrocode}
\def\zref@ifpropundefined#1{%
  \ltx@ifundefined{Z@E@#1}%
}
%    \end{macrocode}
%    \end{macro}
%    \begin{macro}{\zref@propexists}
%    Some macros rely on the existence of a property.
%    \cs{zref@propexists} only executes \verb|#2| if the
%    property \verb|#1| exists and raises an error
%    message otherwise.
%    \begin{macrocode}
\ZREF@Robust\def\zref@propexists#1{%
  \zref@ifpropundefined{#1}{%
    \PackageError\ZREF@name{%
      Property `#1' does not exist%
    }\@ehc
  }%
}
%    \end{macrocode}
%    \end{macro}
%
%    \begin{macro}{\zref@newprop}
%    A new property is declared by \cs{zref@newprop}, the
%    property name \meta{propname} is given in \verb|#1|. The
%    property is created and configured. If the star form
%    is given, then the expansion of the property value
%    is delayed to page shipout time, when the reference
%    is written to the \xfile{.aux} file.
%    \begin{description}
%    \item[\cs{Z@D@}\metacs{propname}:]
%      Stores the default value for this property.
%    \item[\cs{Z@E@}\metacs{propname}:]
%      Extract function.
%    \item[\cs{Z@X@}\metacs{propname}:]
%      Information whether the expansion of the property value
%      is delayed to shipout time.
%    \item[\cs{Z@C@}\metacs{propname}:]
%      Current value of the property.
%    \end{description}
%    \begin{macrocode}
\ZREF@Robust\def\zref@newprop{%
  \@ifstar{%
    \let\ZREF@X\noexpand
    \ZREF@newprop
  }{%
    \let\ZREF@X\ltx@empty
    \ZREF@newprop
  }%
}
%    \end{macrocode}
%    \end{macro}
%    \begin{macro}{\ZREF@newprop}
%    \begin{macrocode}
\def\ZREF@newprop#1{%
  \edef\ZREF@P{#1}%
  \@onelevel@sanitize\ZREF@P
  \begingroup
  \ifx\ZREF@P\ZREF@par
    \@PackageError\ZREF@name{%
      Invalid property name `\ZREF@P'%
    }{%
      The property name `par' is not allowed %
      because of internal reasons.%
      \MessageBreak
      \@ehc
    }%
    \def\ZREF@@newprop[##1]##2{\endgroup}%
  \else
    \zref@ifpropundefined\ZREF@P{%
      \endgroup
      \PackageInfo\ZREF@name{%
        New property: \ZREF@P
      }%
    }{%
      \@PackageError\ZREF@name{%
        Property `\ZREF@P' already exists%
      }\@ehc
      \def\ZREF@@newprop[##1]##2{\endgroup}%
    }%
  \fi
  \@ifnextchar[\ZREF@@newprop{\ZREF@@newprop[\zref@default]}%
}
%    \end{macrocode}
%    \end{macro}
%    \begin{macro}{\ZREF@par}
%    \begin{macrocode}
\def\ZREF@par{par}
\@onelevel@sanitize\ZREF@par
%    \end{macrocode}
%    \end{macro}
%    \begin{macro}{\ZREF@@newprop}
%    \begin{macrocode}
\def\ZREF@@newprop[#1]{%
  \global\@namedef{Z@D@\ZREF@P}{#1}%
  \global\expandafter\let\csname Z@X@\ZREF@P\endcsname\ZREF@X
  \begingroup\expandafter\endgroup
  \expandafter\ZREF@@@newprop\csname\ZREF@P\endcsname
  \expandafter\gdef\csname Z@C@\ZREF@P\endcsname{}%
  \zref@setcurrent\ZREF@P
}
\def\ZREF@@@newprop#1{%
  \expandafter
  \gdef\csname Z@E@\ZREF@P\endcsname##1#1##2##3\ZREF@nil{##2}%
}
%    \end{macrocode}
%    \end{macro}
%
%    \begin{macro}{\zref@showprop}
%    \begin{macrocode}
\ZREF@Robust\def\zref@showprop#1{%
  \zref@ifpropundefined{#1}{%
    \@PackageInfoNoLine{\ZREF@name}{%
      Show property `#1': <undefined>%
    }%
  }{%
    \begingroup
      \toks@\expandafter\expandafter\expandafter{%
        \csname Z@C@#1\endcsname
      }%
      \edef\ZREF@value{\the\toks@}%
      \ltx@onelevel@sanitize\ZREF@value
      \toks@\expandafter\expandafter\expandafter{%
        \csname Z@D@#1\endcsname
      }%
      \edef\ZREF@default{\the\toks@}%
      \ltx@onelevel@sanitize\ZREF@default
      \@PackageInfoNoLine{\ZREF@name}{%
        Show property `#1':\MessageBreak
        \expandafter\ifx\csname Z@X@#1\endcsname\ltx@empty
          Immediate %
        \else
          Delayed %
        \fi
        value: [\ZREF@value]\MessageBreak
        Default: [\ZREF@default]%
      }%
    \endgroup
  }%
}
%    \end{macrocode}
%    \end{macro}
%
%    \begin{macro}{\zref@setcurrent}
%    \cs{zref@setcurrent} sets the current value for a property.
%    \begin{macrocode}
\ZREF@Robust\def\zref@setcurrent#1#2{%
  \zref@propexists{#1}{%
    \expandafter\def\csname Z@C@#1\endcsname{#2}%
  }%
}
%    \end{macrocode}
%    \end{macro}
%
%    \begin{macro}{\ZREF@getcurrent}
%    \cs{zref@getcurrent} gets the current value for a property.
%    \begin{macrocode}
\def\ZREF@getcurrent#1{%
  \romannumeral0%
  \ltx@ifundefined{Z@C@#1}{%
    \ltx@space
  }{%
    \expandafter\expandafter\expandafter\ltx@space
    \csname Z@C@#1\endcsname
  }%
}
%    \end{macrocode}
%    \end{macro}
%    \begin{macro}{\ZREF@u@getcurrent}
%    \begin{macrocode}
\def\ZREF@wu@getcurrent#1{%
  \etex@unexpanded\expandafter\expandafter\expandafter{%
    \ZREF@getcurrent{#1}%
  }%
}
%    \end{macrocode}
%    \end{macro}
%    \begin{macro}{\zref@getcurrent}
%    \begin{macrocode}
\let\zref@getcurrent\ZREF@getcurrent
%    \end{macrocode}
%    \end{macro}
%
% \subsubsection{Reference generation}
%
%    \begin{macro}{\zref@label}
%    Label macro that uses the main property list.
%    \begin{macrocode}
\ZREF@Robust\def\zref@label#1{%
  \zref@labelbylist{#1}\ZREF@mainlist
}
%    \end{macrocode}
%    \end{macro}
%    \begin{macro}{\zref@labelbylist}
%    Label macro that stores the properties, specified in the
%    property list \verb|#2|.
%    \begin{macrocode}
\ZREF@Robust\def\zref@labelbylist#1#2{%
  \@bsphack
    \zref@listexists{#2}{%
      \expandafter\expandafter\expandafter\ZREF@label
      \expandafter\expandafter\expandafter{%
        \csname Z@L@#2\endcsname
      }{#1}%
    }%
  \@esphack
}
%    \end{macrocode}
%    \end{macro}
%    \begin{macro}{\zref@labelbyprops}
%    The properties are directly specified in a comma separated list.
%    \begin{macrocode}
\ZREF@Robust\def\zref@labelbyprops#1#2{%
  \@bsphack
    \begingroup
      \toks@{}%
      \comma@parse{#2}{%
        \zref@ifpropundefined\comma@entry{%
          \PackageWarning\ZREF@name{%
            Property `\comma@entry' is not known%
          }%
        }{%
          \toks@\expandafter{%
            \the\expandafter\toks@\csname\comma@entry\endcsname
          }%
        }%
        \ltx@gobble
      }%
    \expandafter\endgroup
    \expandafter\ZREF@label\expandafter{\the\toks@}{#1}%
  \@esphack
}
%    \end{macrocode}
%    \end{macro}
%    \begin{macro}{\zref@labelbykv}
%    \begin{macrocode}
\ZREF@Robust\def\zref@labelbykv#1#2{%
  \@bsphack
    \begingroup
      \let\Z@L@ZREF@temp\ltx@empty
      \kvsetkeys{ZREF@LABEL}{#1}%
      \ifZREF@immediate
        \expandafter\zref@wrapper@immediate\expandafter{%
          \expandafter\ZREF@label\expandafter{\Z@L@ZREF@temp}{#2}%
        }%
      \else
        \expandafter\ZREF@label\expandafter{\Z@L@ZREF@temp}{#2}%
      \fi
    \endgroup
  \@esphack
}
%    \end{macrocode}
%    \end{macro}
%    \begin{macrocode}
\kv@define@key{ZREF@LABEL}{prop}{%
  \edef\ZREF@param{#1}%
  \zref@propexists\ZREF@param{%
    \zref@iflistcontainsprop{ZREF@temp}\ZREF@param{}{%
      \begingroup\expandafter\endgroup
      \expandafter\ltx@LocalAppendToMacro
      \expandafter\Z@L@ZREF@temp
      \expandafter{\csname\ZREF@param\endcsname}%
    }%
  }%
}
\kv@define@key{ZREF@LABEL}{list}{%
  \zref@listforloop{#1}{%
    \zref@iflistcontainsprop{ZREF@temp}\zref@prop{}{%
      \begingroup\expandafter\endgroup
      \expandafter\ltx@LocalAppendToMacro
      \expandafter\Z@L@ZREF@temp
      \expandafter{\csname\zref@prop\endcsname}%
    }%
    \ltx@gobble
  }%
}
\kv@define@key{ZREF@LABEL}{delprop}{%
  \zref@propexists{#1}{%
    \zref@localdelprop{ZREF@temp}{#1}%
  }%
}
\kv@define@key{ZREF@LABEL}{immediate}[true]{%
  \edef\ZREF@param{#1}%
  \ifx\ZREF@param\ZREF@true
    \ZREF@immediatetrue
  \else
    \ifx\ZREF@param\ZREF@false
      \ZREF@immediatefalse
    \else
      \PackageWarning\ZREF@name{%
        Option `immediate' expects `true' or `false'.\MessageBreak
        Ignoring invalid value `\ZREF@param'%
      }%
    \fi
  \fi
}
%    \end{macrocode}
%    \begin{macro}{\ZREF@false}
%    \begin{macrocode}
\def\ZREF@false{false}
%    \end{macrocode}
%    \end{macro}
%    \begin{macro}{\ZREF@true}
%    \begin{macrocode}
\def\ZREF@true{true}
%    \end{macrocode}
%    \end{macro}
%
%    \begin{macrocode}
\kv@define@key{ZREF@LABEL}{values}[]{%
  \kv@parse{#1}{%
    \ifx\kv@value\relax
      \@PackageWarning\ZREF@name{%
        Missing value for property `\kv@key'%
      }%
      \expandafter\ltx@gobbletwo
    \else
      \expandafter\zref@setcurrent
    \fi
  }%
}
%    \end{macrocode}
%
%    \begin{macro}{\ifZREF@immediate}
%    The switch \cs{ifZREF@immediate} tells us, whether the
%    label should be written immediately or at page shipout time.
%    \cs{ZREF@label} need to be notified about this, because
%    it must disable the deferred execution of property values,
%    if the label is written immediately.
%    \begin{macrocode}
\newif\ifZREF@immediate
%    \end{macrocode}
%    \end{macro}
%    \begin{macro}{\zref@wrapper@immediate}
%    The argument of \cs{zref@wrapper@immediate} is executed inside
%    a group where \cs{write} is redefined by adding
%    \cs{immediate} before its execution. Also \cs{ZREF@label}
%    is notified via the switch \cs{ifZREF@immediate}.
%    \begin{macrocode}
\ZREF@Robust{\long\def}\zref@wrapper@immediate#1{%
  \begingroup
    \ZREF@immediatetrue
    \let\ZREF@org@write\write
    \def\write{\immediate\ZREF@org@write}%
    #1%
  \endgroup
}
%    \end{macrocode}
%    \end{macro}
%
%    \begin{macro}{\ZREF@label}
%    \cs{ZREF@label} writes the data in the \xfile{.aux} file.
%    \verb|#1| contains the list of valid properties,
%    \verb|#2| the name of the reference. In case of immediate
%    writing, the deferred execution of property values is
%    disabled. Also \thepage is made expandable in this case.
%    \begin{macrocode}
\def\ZREF@label#1#2{%
  \if@filesw
    \begingroup
      \ifZREF@immediate
        \let\ZREF@org@thepage\thepage
      \fi
      \protected@write\@auxout{%
        \ifZREF@immediate
          \let\thepage\ZREF@org@thepage
        \fi
        \let\ZREF@temp\ltx@empty
        \@tfor\ZREF@P:=#1\do{%
          \begingroup
            \escapechar=-1 %
            \edef\x{\endgroup
              \def\noexpand\ZREF@P{%
                \expandafter\string\ZREF@P
              }%
            }%
          \x
          \expandafter\ifx
              \csname
                \ifZREF@immediate
                  relax%
                \else
                  Z@X@\ZREF@P%
                \fi
              \endcsname
              \noexpand
            \expandafter\let\csname Z@C@\ZREF@P\endcsname\relax
          \fi
          \toks@\expandafter{\ZREF@temp}%
          \edef\ZREF@temp{%
            \the\toks@
            \ltx@backslashchar\ZREF@P{%
              \expandafter\noexpand\csname Z@C@\ZREF@P\endcsname
            }%
          }%
        }%
      }{%
        \string\zref@newlabel{#2}{\ZREF@temp}%
      }%
    \endgroup
  \fi
}
\def\ZREF@addtoks#1{%
  \toks@\expandafter\expandafter\expandafter{%
    \expandafter\the\expandafter\toks@#1%
  }%
}
%    \end{macrocode}
%    \end{macro}
%
% \subsubsection{Reference querying and extracting}
%
%    Design goal for the extracting macros is that the extraction
%    process is full expandable. Thus these macros can be used
%    in expandable contexts. But there are problems that cannot
%    be solved by full expandable macros:
%    \begin{itemize}
%    \item In standard \hologo{LaTeX} undefined references sets a flag
%      and generate a warning. Both actions are not expandable.
%    \item Babel's support for its shorthand uses commands that
%      use non-expandable assignments. However currently there
%      is hope, that primitives are added to \pdfTeX\ that
%      allows the detection of contexts. Then the shorthand
%      can detect, if they are executed inside \cs{csname}
%      and protect themselves automatically.
%    \end{itemize}
%    \begin{macro}{\zref@ifrefundefined}
%    If a reference \verb|#1| is undefined, then macro
%    \cs{zref@ifrefundefined} calls \verb|#2| and \verb|#3| otherwise.
%    \begin{macrocode}
\def\zref@ifrefundefined#1{%
  \ltx@ifundefined{Z@R@#1}%
}
%    \end{macrocode}
%    \end{macro}
%
%    \begin{macro}{\zifrefundefined}
%    If a reference \verb|#1| is undefined, then macro
%    \cs{zref@ifrefundefined} calls \verb|#2| and \verb|#3| otherwise.
%    Also the reference is marked used.
%    \begin{macrocode}
\ZREF@IfDefinable\zifrefundefined\def{%
  #1{%
    \zref@wrapper@babel\ZREF@ifrefundefined{#1}%
  }%
}
%    \end{macrocode}
%    \end{macro}
%    \begin{macro}{\ZREF@ifrefundefined}
%    \begin{macrocode}
\def\ZREF@ifrefundefined#1{%
  \zref@refused{#1}%
  \zref@ifrefundefined{#1}%
}
%    \end{macrocode}
%    \end{macro}
%
%    \begin{macro}{\zref@refused}
%    The problem with undefined references is addressed by
%    the macro \cs{zref@refused}. This can be used outside
%    the expandable context. In case of an undefined reference
%    the flag is set to notify \hologo{LaTeX} and a warning is given.
%    \begin{macrocode}
\ZREF@Robust\def\zref@refused#1{%
  \zref@wrapper@babel\ZREF@refused{#1}%
}
%    \end{macrocode}
%    \end{macro}
%    \begin{macro}{\ZREF@refused}
%    \begin{macrocode}
\def\ZREF@refused#1{%
  \zref@ifrefundefined{#1}{%
    \protect\G@refundefinedtrue
    \@latex@warning{%
      Reference `#1' on page \thepage \space undefined%
    }%
  }{}%
}
%    \end{macrocode}
%    \end{macro}
%
%    \begin{macro}{\zref@ifrefcontainsprop}
%    \cs{zref@ifrefcontainsprop} looks, if the reference \verb|#1|
%    has the property \verb|#2| and calls then \verb|#3| and
%    \verb|#4| otherwise.
%    \begin{macrocode}
\def\zref@ifrefcontainsprop#1#2{%
  \zref@ifrefundefined{#1}{%
    \ltx@secondoftwo
  }{%
    \expandafter\ZREF@ifrefcontainsprop
    \csname Z@E@#2\expandafter\endcsname
    \csname#2\expandafter\expandafter\expandafter\endcsname
    \expandafter\expandafter\expandafter{%
      \csname Z@R@#1\endcsname
    }%
  }%
}
\def\ZREF@ifrefcontainsprop#1#2#3{%
  \expandafter\ifx\expandafter\ZREF@novalue
  #1#3#2\ZREF@novalue\ZREF@nil\ltx@empty
    \expandafter\ltx@secondoftwo
  \else
    \expandafter\ltx@firstoftwo
  \fi
}
\def\ZREF@novalue{\ZREF@NOVALUE}
%    \end{macrocode}
%    \end{macro}
%
%    \begin{macro}{\zref@extract}
%    \cs{zref@extract} is an abbreviation for the case
%    that the default of the property is used as default
%    value.
%    \begin{macrocode}
\def\ZREF@extract#1#2{%
  \romannumeral0%
  \ltx@ifundefined{Z@D@#2}{%
    \expandafter\ltx@space\zref@default
  }{%
    \expandafter\expandafter\expandafter\ZREF@@extract
    \expandafter\expandafter\expandafter{%
      \csname Z@D@#2\endcsname
    }{#1}{#2}%
  }%
}
%    \end{macrocode}
%    \end{macro}
%    \begin{macro}{\ZREF@@extract}
%    \begin{macrocode}
\def\ZREF@@extract#1#2#3{%
  \expandafter\expandafter\expandafter\ltx@space
  \zref@extractdefault{#2}{#3}{#1}%
}
%    \end{macrocode}
%    \end{macro}
%    \begin{macro}{\ZREF@wu@extract}
%    \begin{macrocode}
\def\ZREF@wu@extract#1#2{%
  \etex@unexpanded\expandafter\expandafter\expandafter{%
    \ZREF@extract{#1}{#2}%
  }%
}
%    \end{macrocode}
%    \end{macro}
%    \begin{macro}{\zref@extract}
%    \begin{macrocode}
\let\zref@extract\ZREF@extract
%    \end{macrocode}
%    \end{macro}
%
%    \begin{macro}{\ZREF@extractdefault}
%    The basic extracting macro is \cs{zref@extractdefault} with
%    the reference name in \verb|#1|, the property in \verb|#2| and
%    the default value in \verb|#3| in case for problems.
%    \begin{macrocode}
\def\ZREF@extractdefault#1#2#3{%
  \romannumeral0%
  \zref@ifrefundefined{#1}\ltx@firstoftwo{%
    \zref@ifpropundefined{#2}\ltx@firstoftwo\ltx@secondoftwo
  }{%
    \ltx@space
    #3%
  }{%
    \expandafter\expandafter\expandafter\ltx@space
    \csname Z@E@#2\expandafter\expandafter\expandafter\endcsname
    \csname Z@R@#1\expandafter\endcsname
    \csname#2\endcsname{#3}\ZREF@nil
  }%
}
%    \end{macrocode}
%    \end{macro}
%    \begin{macro}{\ZREF@wu@extractdefault}
%    \begin{macrocode}
\def\ZREF@wu@extractdefault#1#2#3{%
  \etex@unexpanded\expandafter\expandafter\expandafter{%
    \ZREF@extractdefault{#1}{#2}{#3}%
  }%
}
%    \end{macrocode}
%    \end{macro}
%    \begin{macro}{\zref@extractdefault}
%    \begin{macrocode}
\let\zref@extractdefault\ZREF@extractdefault
%    \end{macrocode}
%    \end{macro}
%
%    \begin{macro}{\zref@def@extract}
%    \begin{macrocode}
\ZREF@Robust\def\zref@def@extract#1{%
  \zref@wrapper@babel{\ZREF@def@extract{#1}}%
}
%    \end{macrocode}
%    \end{macro}
%    \begin{macro}{\ZREF@def@extract}
%    \begin{macrocode}
\def\ZREF@def@extract#1#2#3{%
  \zref@refused{#2}%
  \expandafter\expandafter\expandafter\def
  \expandafter\expandafter\expandafter#1%
  \expandafter\expandafter\expandafter{%
    \zref@extract{#2}{#3}%
  }%
}
%    \end{macrocode}
%    \end{macro}
%    \begin{macro}{\zref@def@extractdefault}
%    \begin{macrocode}
\ZREF@Robust\def\zref@def@extractdefault#1{%
  \zref@wrapper@babel{\ZREF@def@extractdefault{#1}}%
}
%    \end{macrocode}
%    \end{macro}
%    \begin{macro}{\ZREF@def@extractdefault}
%    \begin{macrocode}
\def\ZREF@def@extractdefault#1#2#3#4{%
  \zref@refused{#2}%
  \expandafter\expandafter\expandafter\def
  \expandafter\expandafter\expandafter#1%
  \expandafter\expandafter\expandafter{%
    \zref@extractdefault{#2}{#3}{#4}%
  }%
}
%    \end{macrocode}
%    \end{macro}
%
%    \begin{macro}{\ZREF@wrapper@unexpanded}
%    \begin{macrocode}
\ZREF@Robust{\long\def}\ZREF@wrapper@unexpanded#1{%
  \let\zref@wrapper@unexpanded\ltx@firstofone
  \let\zref@getcurrent\ZREF@wu@getcurrent
  \let\zref@extractdefault\ZREF@wu@extractdefault
  \let\zref@extract\ZREF@wu@extract
  #1%
  \let\zref@wrapper@unexpanded\ZREF@wrapper@unexpanded
  \let\zref@getcurrent\ZREF@getcurrent
  \let\zref@extractdefault\ZREF@extractdefault
  \let\zref@extract\ZREF@extract
}
%    \end{macrocode}
%    \end{macro}
%    \begin{macro}{\zref@wrapper@unexpanded}
%    \begin{macrocode}
\ltx@IfUndefined{etex@unexpanded}{%
  \let\zref@wrapper@unexpanded\ltx@firstofone
}{%
  \let\zref@wrapper@unexpanded\ZREF@wrapper@unexpanded
}
%    \end{macrocode}
%    \end{macro}
%
% \subsubsection{Compatibility with \xpackage{babel}}
%
%    \begin{macro}{\zref@wrapper@babel}
%    \begin{macrocode}
\ZREF@Robust{\long\def}\zref@wrapper@babel#1#2{%
  \ifcsname if@safe@actives\endcsname
    \expandafter\ltx@firstofone
  \else
    \expandafter\ltx@secondoftwo
  \fi
  {%
    \if@safe@actives
      \expandafter\ltx@secondoftwo
    \else
      \expandafter\ltx@firstoftwo
    \fi
    {%
      \begingroup
        \csname @safe@activestrue\endcsname
        \edef\x{#2}%
      \expandafter\endgroup
      \expandafter\ZREF@wrapper@babel\expandafter{\x}{#1}%
    }%
  }{%
    #1{#2}%
  }%
}
\long\def\ZREF@wrapper@babel#1#2{%
  #2{#1}%
}
%    \end{macrocode}
%    \end{macro}
%
% \subsubsection{Unique counter support}
%
%    \begin{macro}{\zref@require@unique}
%    Generate the counter |zref@unique| if the counter does not
%    already exist.
%    \begin{macrocode}
\ZREF@Robust\def\zref@require@unique{%
  \@ifundefined{c@zref@unique}{%
    \begingroup
      \let\@addtoreset\ltx@gobbletwo
      \newcounter{zref@unique}%
    \endgroup
%    \end{macrocode}
%    \end{macro}
%
%    \begin{macro}{\thezref@unique}
%    \cs{thezref@unique} is used for automatically generated unique
%    labelnames.
%    \begin{macrocode}
    \renewcommand*{\thezref@unique}{%
      zref@\number\c@zref@unique
    }%
  }{}%
}
%    \end{macrocode}
%    \end{macro}
%
% \subsubsection{Utilities}
%
%    \begin{macro}{\ZREF@number}
%    \begin{macrocode}
\ltx@IfUndefined{numexpr}{%
  \def\ZREF@number#1{\number#1}%
}{%
  \def\ZREF@number#1{\the\numexpr(#1)\relax}%
}
%    \end{macrocode}
%    \end{macro}
%
% \subsubsection{Setup}
%
%    \begin{macro}{\zref@setdefault}
%    Standard \hologo{LaTeX} prints ``??'' in bold face if a reference
%    is not known. \cs{zref@default} holds the text that is
%    printed in case of unknown references and is used, if
%    the default was not specified during the definition
%    of the new property by \cs{ref@newprop}.
%    The global default value can be set by \cs{zref@setdefault}.
%    \begin{macrocode}
\ZREF@Robust\def\zref@setdefault#1{%
  \def\zref@default{#1}%
}
%    \end{macrocode}
%    \end{macro}
%    \begin{macro}{\zref@default}
%    Now we initialize \cs{zref@default} with the same value
%    that \hologo{LaTeX} uses for its undefined references.
%    \begin{macrocode}
\zref@setdefault{%
  \nfss@text{\reset@font\bfseries ??}%
}
%    \end{macrocode}
%    \end{macro}
%
% \paragraph{Main property list.}
%
%    \begin{macro}{\zref@setmainlist}
%    The name of the default property list is stored in
%    \cs{ZREF@mainlist} and can be set by \cs{zref@setmainlist}.
%    \begin{macrocode}
\ZREF@Robust\def\zref@setmainlist#1{%
  \def\ZREF@mainlist{#1}%
}
\zref@setmainlist{main}
%    \end{macrocode}
%    \end{macro}
%    Now we create the list.
%    \begin{macrocode}
\zref@newlist\ZREF@mainlist
%    \end{macrocode}
%
% \paragraph{Main properties.}
%
%    The two properties \texttt{default} and \texttt{page}
%    are created and added to the main property list.
%    They store the data that
%    standard \hologo{LaTeX} uses in its references created by
%    \cs{label}.
%    \begin{description}
%    \item[\texttt{default}] the apperance of the latest
%      counter that is incremented by \cs{refstepcounter}
%    \item[\texttt{page}] the apperance of the page counter
%    \end{description}
%    \begin{macrocode}
\zref@newprop{default}{\@currentlabel}
\zref@newprop*{page}{\thepage}
\zref@addprops\ZREF@mainlist{default,page}
%    \end{macrocode}
%
% \paragraph{Properties}
%
%    \begin{macro}{\ZREF@NewPropAnchor}
%    \begin{macrocode}
\def\ZREF@NewPropAnchor{%
  \zref@newprop{anchor}{%
    \ltx@ifundefined{@currentHref}{}{\@currentHref}%
  }%
  \global\let\ZREF@NewPropAnchor\relax
}
%    \end{macrocode}
%    \end{macro}
%    \begin{macro}{\zref@titleref@current}
%    Later we will redefine the section and caption macros
%    to catch the current title and remember the value in
%    \cs{zref@titleref@current}.
%    \begin{macro}{\ZREF@NewPropTitle}
%    \begin{macrocode}
\def\ZREF@NewPropTitle{%
  \gdef\zref@titleref@current{}%
  \zref@newprop{title}{\zref@titleref@current}%
  \global\let\ZREF@NewPropTitle\relax
}
%    \end{macrocode}
%    \end{macro}
%    \end{macro}
%    \begin{macro}{\ZREF@NewPropTheotype}
%    \begin{macrocode}
\def\ZREF@NewPropTheotype{%
  \zref@newprop{theotype}{}%
  \global\let\ZREF@NewPropTheotype\relax
}
%    \end{macrocode}
%    \end{macro}
%    \begin{macro}{\ZREF@NewPropPageValue}
%    \begin{macrocode}
\def\ZREF@NewPropPageValue{%
  \zref@newprop*{pagevalue}[0]{\number\c@page}%
  \global\let\ZREF@NewPropPageValue\relax
}
%    \end{macrocode}
%    \end{macro}
%
% \paragraph{Mark successful loading}
%
%    \begin{macrocode}
\let\ZREF@base@ok=Y
%</base>
%    \end{macrocode}
%
%
% \subsection{Module \xmodule{user}}
%
%    \begin{macrocode}
%<*user>
\NeedsTeXFormat{LaTeX2e}
\ProvidesPackage{zref-user}%
  [2020-07-03 v2.32 Module user for zref (HO)]%
\RequirePackage{zref-base}[2019/11/29]
\ifx\ZREF@base@ok Y%
\else
  \expandafter\endinput
\fi
%    \end{macrocode}
%
%    Module \xmodule{user} enables a small user interface.
%    All macros are prefixed by \cs{z}.
%
%    First we define the pendants to the standard \hologo{LaTeX} referencing
%    commands \cs{label}, \cs{ref}, and \cs{pageref}.
%
%    \begin{macro}{\zlabel}
%    Similar to \cs{label} the macro \cs{zlabel} writes a reference
%    entry in the \xfile{.aux} file. The main property list is used.
%    Also we add the babel patch. The \cs{label} command can also be
%    used inside section titles, but it must not go into the
%    table of contents. Therefore we have to check this situation.
%    \begin{macrocode}
\newcommand*\zlabel{%
  \ifx\label\ltx@gobble
    \expandafter\ltx@gobble
  \else
    \expandafter\zref@wrapper@babel\expandafter\zref@label
  \fi
}%
%    \end{macrocode}
%    \end{macro}
%
%    \begin{macro}{\zkvlabel}
%    \begin{macrocode}
\newcommand*{\zkvlabel}[1]{%
  \ifx\label\ltx@gobble
    \expandafter\ltx@gobblethree
  \fi
  \zref@wrapper@babel{\zref@labelbykv{#1}}%
}%
%    \end{macrocode}
%    \end{macro}
%
%    \begin{macro}{\zref}
%    Macro \cs{zref} is the corresponding macro for \cs{ref}.
%    Also it provides an optional argument in order to
%    select another property.
%    \begin{macrocode}
\newcommand*{\zref}[2][default]{% robust because of optional argument
  \zref@propexists{#1}{%
    \zref@wrapper@babel\ZREF@zref{#2}{#1}%
  }%
}%
\def\ZREF@zref#1{%
  \zref@refused{#1}%
  \zref@extract{#1}%
}%
%    \end{macrocode}
%    \end{macro}
%
%    \begin{macro}{\zpageref}
%    For macro \cs{zpageref} we just call \cs{zref}
%    with property \verb|page|.
%    \begin{macrocode}
\ZREF@IfDefinable\zpageref\def{%
  {\zref[page]}%
}
%    \end{macrocode}
%    \end{macro}
%
%    \begin{macro}{\zrefused}
%    For the following expandible user macros \cs{zrefused}
%    should be used to notify \hologo{LaTeX} in case of undefined
%    references.
%    \begin{macrocode}
\ZREF@IfDefinable\zrefused\def{%
  {\zref@refused}%
}
%    \end{macrocode}
%    \end{macro}
%
%    \begin{macrocode}
%</user>
%    \end{macrocode}
%
%
% \subsection{Module \xmodule{abspage}}
%
%    \begin{macrocode}
%<*abspage>
\NeedsTeXFormat{LaTeX2e}
\ProvidesPackage{zref-abspage}%
  [2020-07-03 v2.32 Module abspage for zref (HO)]%
\RequirePackage{zref-base}[2019/11/29]
\ifx\ZREF@base@ok Y%
\else
  \expandafter\endinput
\fi
%    \end{macrocode}
%
%    Module \xmodule{abspage} adds a new property \verb|abspage|
%    to the \verb|main| property list for absolute page numbers.
%    These are recorded by the help of package \xpackage{atbegshi}.
%    \begin{macrocode}
\RequirePackage{atbegshi}[2011/10/05]%
%    \end{macrocode}
%    The counter \verb|abspage| must not go in the clear list
%    of \verb|@ckpt| that is used to set counters in
%    \xfile{.aux} files of included \hologo{TeX} files.
%    \begin{macrocode}
\begingroup
  \let\@addtoreset\ltx@gobbletwo
  \newcounter{abspage}%
\endgroup
\setcounter{abspage}{0}%
\AtBeginShipout{%
  \stepcounter{abspage}%
}%
\zref@newprop*{abspage}[0]{\the\c@abspage}%
\zref@addprop\ZREF@mainlist{abspage}%
%    \end{macrocode}
%    Note that counter \verb|abspage| shows the
%    previous page during page processing. Before
%    shipout the counter is incremented.
%    Thus the property is correctly written with
%    deferred writing. If the counter is written
%    using \cs{zref@wrapper@immediate}, then the number
%    is too small by one.
%
%    \begin{macrocode}
%</abspage>
%    \end{macrocode}
%
%
% \subsection{Module \xmodule{counter}}
%
%    \begin{macrocode}
%<*counter>
\NeedsTeXFormat{LaTeX2e}
\ProvidesPackage{zref-counter}%
  [2020-07-03 v2.32 Module counter for zref (HO)]%
\RequirePackage{zref-base}[2019/11/29]
\ifx\ZREF@base@ok Y%
\else
  \expandafter\endinput
\fi
%    \end{macrocode}
%
%    For features such as \xpackage{hyperref}'s \cs{autoref}
%    we need the name of the counter. The property
%    \verb|counter| is defined and added to the main property
%    list. Starting with \LaTeX\ 2020-10-01 the proper can use
%    \@currentcounter. In older formats
%    \cs{refstepcounter} has to be patched but this can fail in some
%    cases, see issue \#5.
%    \begin{macrocode}
\@ifl@t@r\fmtversion{2020-10-01}
 {
  \zref@newprop{counter}{\@currentcounter}
  \zref@addprop\ZREF@mainlist{counter}
 }
 {
  \zref@newprop{counter}{}
  \zref@addprop\ZREF@mainlist{counter}
  \AtBeginDocument{%
    \ZREF@patch{refstepcounter}{%
      \def\refstepcounter#1{%
        \zref@setcurrent{counter}{#1}%
        \ZREF@org@refstepcounter{#1}%
      }%
    }%
  }
 }
%    \end{macrocode}
%    \begin{macrocode}
%</counter>
%    \end{macrocode}
%
%
% \subsection{Module \xmodule{lastpage}}
%
%    \begin{macrocode}
%<*lastpage>
\NeedsTeXFormat{LaTeX2e}
\ProvidesPackage{zref-lastpage}%
  [2020-07-03 v2.32 Module lastpage for zref (HO)]%
\RequirePackage{zref-base}[2019/11/29]
\RequirePackage{zref-abspage}[2019/11/29]
\RequirePackage{atveryend}[2009/12/07]
\ifx\ZREF@base@ok Y%
\else
  \expandafter\endinput
\fi
%    \end{macrocode}
%
%    The module \xmodule{lastpage} implements the service
%    of package \xpackage{lastpage} by setting a reference
%    \verb|LastPage| at the end of the document.
%    If module \xmodule{abspage} is given, also the absolute
%    page number is available, because the properties of
%    the main property list are used.
%    \begin{macrocode}
\zref@newlist{LastPage}
\AfterLastShipout{%
  \if@filesw
    \begingroup
      \advance\c@page\m@ne
      \toks@\expandafter\expandafter\expandafter{%
        \expandafter\Z@L@main
        \Z@L@LastPage
      }%
      \expandafter\zref@wrapper@immediate\expandafter{%
        \expandafter\ZREF@label\expandafter{\the\toks@}{LastPage}%
      }%
    \endgroup
  \fi
}
%    \end{macrocode}
%    \begin{macro}{\zref@iflastpage}
%    \begin{macrocode}
\def\zref@iflastpage#1{%
  \ifnum\zref@extractdefault{#1}{abspage}{-1}=%
        \zref@extractdefault{LastPage}{abspage}{-2} %
    \expandafter\ltx@firstoftwo
  \else
    \expandafter\ltx@secondoftwo
  \fi
}
%    \end{macrocode}
%    \end{macro}
%    \begin{macro}{\ziflastpage}
%    \begin{macrocode}
\ZREF@IfDefinable\ziflastpage\def{%
  {\zref@wrapper@babel\ZREF@iflastpage}%
}
%    \end{macrocode}
%    \end{macro}
%    \begin{macro}{ZREF@iflastpage}
%    \begin{macrocode}
\def\ZREF@iflastpage#1{%
  \zref@refused{LastPage}%
  \zref@refused{#1}%
  \zref@iflastpage{#1}%
}
%    \end{macrocode}
%    \end{macro}
%    \begin{macrocode}
%</lastpage>
%    \end{macrocode}
%
%
% \subsection{Module \xmodule{thepage}}
%
%    \begin{macrocode}
%<*thepage>
\NeedsTeXFormat{LaTeX2e}
\ProvidesPackage{zref-thepage}%
  [2020-07-03 v2.32 Module thepage for zref (HO)]%
\RequirePackage{zref-base}[2019/11/29]
\ifx\ZREF@base@ok Y%
\else
  \expandafter\endinput
\fi
%    \end{macrocode}
%
%    \begin{macrocode}
\RequirePackage{atbegshi}[2011/10/05]
\RequirePackage{zref-abspage}[2019/11/29]
%    \end{macrocode}
%    \begin{macrocode}
\zref@newlist{thepage}
\zref@addprop{thepage}{page}
\ZREF@NewPropPageValue
%    \end{macrocode}
%    \begin{macro}{\zref@thepage@atbegshi@hook}
%    \begin{macrocode}
\let\zref@thepage@atbegshi@hook\ltx@empty
%    \end{macrocode}
%    \end{macro}
%    \begin{macrocode}
\zref@addprop{thepage}{pagevalue}
\AtBeginShipout{%
  \AtBeginShipoutAddToBox{%
    \zref@thepage@atbegshi@hook
    \zref@labelbylist{thepage\the\value{abspage}}{thepage}%
  }%
}
%    \end{macrocode}
%    \begin{macro}{\zref@thepage@name}
%    \begin{macrocode}
\ltx@IfUndefined{numexpr}{%
  \def\zref@thepage@name#1{thepage\number#1}%
}{%
  \def\zref@thepage@name#1{thepage\the\numexpr#1}%
}
%    \end{macrocode}
%    \end{macro}
%    \begin{macro}{\zref@thepage}
%    \begin{macrocode}
\def\zref@thepage#1{%
  \zref@extract{\zref@thepage@name{#1}}{page}%
}%
%    \end{macrocode}
%    \end{macro}
%    \begin{macro}{\zref@thepage@refused}
%    \begin{macrocode}
\ZREF@Robust\def\zref@thepage@refused#1{%
  \zref@refused{\zref@thepage@name{#1}}%
}%
%    \end{macrocode}
%    \end{macro}
%    \begin{macro}{\zthepage}
%    \begin{macrocode}
\ZREF@IfDefinable\zthepage\def{%
  #1{%
    \zref@thepage@refused{#1}%
    \zref@thepage{#1}%
  }%
}
%    \end{macrocode}
%    \end{macro}
%    \begin{macrocode}
%</thepage>
%    \end{macrocode}
%
%
% \subsection{Module \xmodule{nextpage}}
%
%    \begin{macrocode}
%<*nextpage>
\NeedsTeXFormat{LaTeX2e}
\ProvidesPackage{zref-nextpage}%
  [2020-07-03 v2.32 Module nextpage for zref (HO)]%
\RequirePackage{zref-base}[2019/11/29]
\ifx\ZREF@base@ok Y%
\else
  \expandafter\endinput
\fi
%    \end{macrocode}
%
%    \begin{macrocode}
\RequirePackage{zref-abspage}[2019/11/29]
\RequirePackage{zref-thepage}[2019/11/29]
\RequirePackage{zref-lastpage}[2019/11/29]
\RequirePackage{uniquecounter}[2009/12/18]
%    \end{macrocode}
%    \begin{macrocode}
\UniqueCounterNew{znextpage}

\newcommand*{\znextpagesetup}{%
  \afterassignment\ZREF@np@setup@i
  \def\ZREF@np@call@unknown##1%
}
\def\ZREF@np@setup@i{%
  \afterassignment\ZREF@np@setup@ii
  \def\ZREF@np@call@nonext##1%
}
\def\ZREF@np@setup@ii{%
  \def\ZREF@np@call@next##1%
}
\def\ZREF@np@call@unknown#1{#1}
\def\ZREF@np@call@nonext#1{#1}
\def\ZREF@np@call@next#1{#1}
\ZREF@IfDefinable\znextpage\def{%
  {\UniqueCounterCall{znextpage}{\ZREF@nextpage}}%
}%
\newcommand*{\znonextpagename}{}
\newcommand*{\zunknownnextpagename}{\Z@D@page}
\def\ZREF@nextpage#1{%
  \begingroup
    \def\ZREF@refname@this{zref@np#1}%
    \zref@labelbyprops\ZREF@refname@this{abspage}%
    \chardef\ZREF@call=0 % unknown
    \ZREF@ifrefundefined\ZREF@refname@this{%
    }{%
      \edef\ZREF@pagenum@this{%
        \zref@extractdefault\ZREF@refname@this{abspage}{0}%
      }%
      \edef\ZREF@refname@next{%
        \zref@thepage@name{%
          \the\numexpr\ZREF@pagenum@this+1%
        }%
      }%
      \ifnum\ZREF@pagenum@this>0 %
        \ZREF@ifrefundefined{LastPage}{%
          \zref@ifrefundefined\ZREF@refname@next{%
          }{%
            \chardef\ZREF@call=2 % next page
          }%
        }{%
          \edef\ZREF@pagenum@last{%
            \zref@extractdefault{LastPage}{abspage}{0}%
          }%
          \ifnum\ZREF@pagenum@this<\ZREF@pagenum@last\ltx@space
            \ZREF@ifrefundefined\ZREF@refname@next{%
            }{%
              \chardef\ZREF@call=2 % next page
            }%
          \else
            \ifnum\ZREF@pagenum@this=\ZREF@pagenum@this\ltx@space
              \chardef\ZREF@call=1 % no next page
            \fi
          \fi
        }%
      \fi
    }%
    \edef\x{%
      \endgroup
      \ifcase\ZREF@call
        \noexpand\ZREF@np@call@unknown{%
          \noexpand\zunknownnextpagename
        }%
      \or
        \noexpand\ZREF@np@call@nonext{%
          \noexpand\znonextpagename
        }%
      \else
        \noexpand\ZREF@np@call@next{%
          \noexpand\zref@extract{\ZREF@refname@next}{page}%
        }%
      \fi
    }%
  \x
}
%    \end{macrocode}
%    \begin{macrocode}
%</nextpage>
%    \end{macrocode}
%
%
% \subsection{Module \xmodule{totpages}}
%
%    \begin{macrocode}
%<*totpages>
\NeedsTeXFormat{LaTeX2e}
\ProvidesPackage{zref-totpages}%
  [2020-07-03 v2.32 Module totpages for zref (HO)]%
\RequirePackage{zref-base}[2019/11/29]
\ifx\ZREF@base@ok Y%
\else
  \expandafter\endinput
\fi
%    \end{macrocode}
%
%    The absolute page number of the last page is the total
%    page number.
%    \begin{macrocode}
\RequirePackage{zref-abspage}[2019/11/29]
\RequirePackage{zref-lastpage}[2019/11/29]
%    \end{macrocode}
%
%    \begin{macro}{\ztotpages}
%    Macro \cs{ztotpages} contains the number of pages. It can be
%    used inside expandable calculations. It expands to zero if
%    the reference is not yet available.
%    \begin{macrocode}
\newcommand*{\ztotpages}{%
  \zref@extractdefault{LastPage}{abspage}{0}%
}
%    \end{macrocode}
%    \end{macro}
%    Also we mark the reference |LastPage| as used:
%    \begin{macrocode}
\AtBeginDocument{%
  \zref@refused{LastPage}%
}
%    \end{macrocode}
%    \begin{macrocode}
%</totpages>
%    \end{macrocode}
%
%
% \subsection{Module \xmodule{pagelayout}}
%
%    \begin{macrocode}
%<*pagelayout>
\NeedsTeXFormat{LaTeX2e}
\ProvidesPackage{zref-pagelayout}%
  [2020-07-03 v2.32 Module pagelayout for zref (HO)]%
\RequirePackage{zref-base}[2019/11/29]
\ifx\ZREF@base@ok Y%
\else
  \expandafter\endinput
\fi
%    \end{macrocode}
%
%    \begin{macrocode}
\RequirePackage{zref-thepage}[2019/11/29]
\RequirePackage{iftex}[2019/11/07]%
\RequirePackage{atveryend}[2010/03/24]
%    \end{macrocode}
%
%
% \subsubsection{Define layout properties}
%
%    \begin{macrocode}
\def\ZREF@temp#1{%
  \begingroup
    \escapechar=-1 %
  \ltx@ifundefined{\string#1}{\endgroup}{%
    \edef\x{%
      \endgroup
      \noexpand\zref@newprop*{\string#1}%
                             [\noexpand\number\noexpand#1]% hash-ok
                             {\noexpand\number\noexpand#1}%
      \noexpand\zref@addprop{thepage}{\string#1}%
    }%
    \x
  }%
}
\ZREF@temp\mag
\ZREF@temp\paperwidth
\ZREF@temp\paperheight
\ZREF@temp\stockwidth % memoir.cls, crop.sty
\ZREF@temp\stockheight % memoir.cls, crop.sty
\ZREF@temp\mediawidth % VTeX
\ZREF@temp\mediaheight % VTeX
\ifluatex
\zref@newprop*{pdfvorigin}%
                             [\number\pdfvariable vorigin]% hash-ok
                             {\number\pdfvariable vorigin}%
\zref@addprop{thepage}{pdfvorigin}
\zref@newprop*{pdfhorigin}%
                             [\number\pdfvariable horigin]% hash-ok
                             {\number\pdfvariable horigin}%
\zref@addprop{thepage}{pdfhorigin}
\zref@newprop*{pdfpageheight}%
                             [\number\pageheight]% hash-ok
                             {\number\pageheight}%
\zref@addprop{thepage}{pdfpageheight}
\zref@newprop*{pdfpagewidth}%
                             [\number\pagewidth]% hash-ok
                             {\number\pagewidth}%
\zref@addprop{thepage}{pdfpagewidth}
\else
\ZREF@temp\pdfpagewidth
\ZREF@temp\pdfpageheight
\ZREF@temp\pdfhorigin
\ZREF@temp\pdfvorigin
\fi
\ZREF@temp\hoffset
\ZREF@temp\voffset
\ZREF@temp\topmargin
\ZREF@temp\oddsidemargin
\ZREF@temp\evensidemargin
\ZREF@temp\textwidth
\ZREF@temp\textheight
\ZREF@temp\headheight
\ZREF@temp\headsep
\ZREF@temp\footskip
\ZREF@temp\marginparwidth
\ZREF@temp\marginparsep
\ZREF@temp\columnwidth
\ZREF@temp\columnsep
\ZREF@temp\trimedge % memoir.cls
\ZREF@temp\spinemargin % memoir.cls
\ZREF@temp\foremargin % memoir.cls
\ZREF@temp\trimtop % memoir.cls
\ZREF@temp\uppermargin % memoir.cls
\ZREF@temp\headmargin % memoir.cls
%    \end{macrocode}
%
%    \begin{macrocode}
\IfFormatAtLeastTF{2020/10/01}
 {
   \zref@newprop*{outputboxwd}[0pt]{\ShipoutBoxWidth}
   \zref@newprop*{outputboxht}[0pt]{\ShipoutBoxHeight}
   \zref@newprop*{outputboxdp}[0pt]{\ShipoutBoxDepth}
 }
 {
   \zref@newprop*{outputboxwd}[0pt]{\AtBeginShipoutBoxWidth}
   \zref@newprop*{outputboxht}[0pt]{\AtBeginShipoutBoxHeight}
   \zref@newprop*{outputboxdp}[0pt]{\AtBeginShipoutBoxDepth}
 }
%    \end{macrocode}
%    \begin{macrocode}
\zref@addprops{thepage}{outputboxwd,outputboxht,outputboxdp}
%    \end{macrocode}
%
%    \begin{macro}{\ifZREF@pl@list}
%    \begin{macrocode}
\ltx@newif\ifZREF@pl@list
%    \end{macrocode}
%    \end{macro}
%    \begin{macro}{\zref@listpagelayout}
%    \begin{macrocode}
\ZREF@IfDefinable\zlistpagelayout\def{%
  {\global\ZREF@pl@listtrue}%
}
%    \end{macrocode}
%    \end{macro}
%    \begin{macro}{\ZREF@pl@AfterLastShipout}
%    \begin{macrocode}
\def\ZREF@pl@AfterLastShipout{%
  \ifZREF@pl@list
    \edef\ZREF@page@max{\the\value{abspage}}%
    \ltx@ifundefined{ZREF@org@testdef}{%
      \let\ZREF@org@testdef\@testdef
      \def\@testdef##1##2##3{%
        \ZREF@org@testdef{##1}{##2}{##3}%
        \def\ZREF@temp{##1}%
        \ifx\ZREF@temp\ZREF@RefPrefix
          \expandafter\gdef\csname##1@##2\endcsname{##3}%
        \fi
      }%
    }{}%
    \AtVeryEndDocument{\ZREF@pl@AtVeryEnd}%
  \fi
}
%    \end{macrocode}
%    \end{macro}
%    \begin{macro}{\ZREF@pl@AtVeryEnd}
%    \begin{macrocode}
\def\ZREF@pl@AtVeryEnd{%
  \begingroup
    \toks@{Page layout parameters:\MessageBreak}%
    \count@=1 %
    \ZREF@pl@ListPage
    \edef\x{\endgroup
      \noexpand\@PackageInfoNoLine{zref-pagelayout}{\the\toks@}%
    }%
  \x
}
%    \end{macrocode}
%    \end{macro}
%    \begin{macro}{\ZREF@pl@ListPage}
%    \begin{macrocode}
\def\ZREF@pl@ListPage{%
  \edef\x{%
    \toks@={%
      \the\toks@
      Page \the\count@:\noexpand\MessageBreak
      \zref@ifrefundefined{thepage\the\count@}{}{%
        \ltx@space\ltx@space mag = %
        \zref@extract{thepage\the\count@}{mag}%
        \noexpand\MessageBreak
        \ZREF@pl@ListEntry{paperwidth}%
        \ZREF@pl@ListEntry{paperheight}%
        \ZREF@pl@ListEntry{stockwidth}%
        \ZREF@pl@ListEntry{stockheight}%
        \ZREF@pl@ListEntry{mediawidth}%
        \ZREF@pl@ListEntry{mediaheight}%
        \ZREF@pl@ListEntry{pdfpagewidth}%
        \ZREF@pl@ListEntry{pdfpageheight}%
        \ZREF@pl@ListEntry{pdfhorigin}%
        \ZREF@pl@ListEntry{pdfvorigin}%
        \ZREF@pl@ListEntry{hoffset}%
        \ZREF@pl@ListEntry{voffset}%
        \ZREF@pl@ListEntry{topmargin}%
        \ZREF@pl@ListEntry{oddsidemargin}%
        \ZREF@pl@ListEntry{evensidemargin}%
        \ZREF@pl@ListEntry{textwidth}%
        \ZREF@pl@ListEntry{textheight}%
        \ZREF@pl@ListEntry{headheight}%
        \ZREF@pl@ListEntry{headsep}%
        \ZREF@pl@ListEntry{footskip}%
        \ZREF@pl@ListEntry{marginparwidth}%
        \ZREF@pl@ListEntry{marginparsep}%
        \ZREF@pl@ListEntry{columnwidth}%
        \ZREF@pl@ListEntry{columnsep}%
        \ZREF@pl@ListEntry{trimedge}%
        \ZREF@pl@ListEntry{spinemargin}%
        \ZREF@pl@ListEntry{foremargin}%
        \ZREF@pl@ListEntry{trimtop}%
        \ZREF@pl@ListEntry{uppermargin}%
        \ZREF@pl@ListEntry{headmargin}%
      }%
    }%
  }\x
  \ifnum\ZREF@page@max>\count@
    \advance\count@ by\ltx@one
  \else
    \expandafter\ltx@gobble
  \fi
  \ZREF@pl@ListPage
}
%    \end{macrocode}
%    \end{macro}
%    \begin{macro}{\ZREF@pl@ListEntry}
%    \begin{macrocode}
\def\ZREF@pl@ListEntry#1{%
  \zref@ifpropundefined{#1}{%
  }{%
    \zref@ifrefcontainsprop{thepage\the\count@}{#1}{%
      \ltx@space\ltx@space#1 = %
      \zref@extract{thepage\the\count@}{#1}sp = %
      \the\dimexpr\zref@extract{thepage\the\count@}{#1}sp\relax
      \noexpand\MessageBreak
    }{}%
  }%
}
%    \end{macrocode}
%    \end{macro}
%    \begin{macrocode}
\AfterLastShipout{%
  \ZREF@pl@AfterLastShipout
}
%    \end{macrocode}
%
%    \begin{macrocode}
%</pagelayout>
%    \end{macrocode}
%
%
% \subsection{Module \xmodule{pageattr}}
%
%    \begin{macrocode}
%<*pageattr>
\NeedsTeXFormat{LaTeX2e}
\ProvidesPackage{zref-pageattr}%
  [2020-07-03 v2.32 Module pageattr for zref (HO)]%
\RequirePackage{zref-base}[2019/11/29]
\ifx\ZREF@base@ok Y%
\else
  \expandafter\endinput
\fi
%    \end{macrocode}
%
%    \begin{macrocode}
\RequirePackage{iftex}[2019/11/07]%
%    \end{macrocode}
%
%
%    \begin{macrocode}
\let\ZREF@temp=N%
\ifluatex
\expandafter\@firstoftwo
\else
\expandafter\@secondoftwo
\fi
{%luatex
 \RequirePackage{zref-thepage}[2019/11/29]
 \RequirePackage{zref-lastpage}[2019/11/29]%
 \zref@newprop*{pdfpageattr}[]{\zref@hex{\the\pdfvariable pageattr}}%
 \zref@addprop{thepage}{pdfpageattr}%
 \zref@newprop*{pdfpagesattr}[]{\zref@hex{\the\pdfvariable pagesattr}}%
 \zref@addprop{LastPage}{pdfpagesattr}%
 \let\ZREF@temp=Y%
}
{%other
 \ltx@IfUndefined{pdfpageattr}{%
  \@PackageInfoNoLine{zref-pageattr}{%
    \string\pdfpageattr\space is not available%
  }%
  \def\zref@pdfpageattr#1{}%
  \def\zref@pdfpageattr@used#1{}%
 }{%
  \RequirePackage{zref-thepage}[2019/11/29]%
  \zref@newprop*{pdfpageattr}[]{\zref@hex{\the\pdfpageattr}}%
  \zref@addprop{thepage}{pdfpageattr}%
  \let\ZREF@temp=Y%
 }
 \ltx@IfUndefined{pdfpagesattr}{%
  \@PackageInfoNoLine{zref-pageattr}{%
    \string\pdfpagesattr\space is not available%
  }%
  \def\zref@pdfpagesattr{}%
  \def\zref@pdfpagesattr@used{}%
 }{%
  \RequirePackage{zref-lastpage}[2019/11/29]%
  \zref@newprop*{pdfpagesattr}[]{\zref@hex{\the\pdfpagesattr}}%
  \zref@addprop{LastPage}{pdfpagesattr}%
  \let\ZREF@temp=Y%
 }%
}%
%    \end{macrocode}
%    \begin{macrocode}
\ifx\ZREF@temp N%
  \expandafter\endinput
\fi
%    \end{macrocode}
%    \begin{macrocode}
\RequirePackage{zref-abspage}[2019/11/29]
\RequirePackage{atveryend}[2010/03/24]
\RequirePackage{pdftexcmds}[2010/04/01]
\let\ZREF@temp=Y%
\ltx@IfUndefined{pdf@escapehex}{\let\ZREF@temp=N}{}
\ltx@IfUndefined{pdf@unescapehex}{\let\ZREF@temp=N}{}
\ifx\ZREF@temp N%
  \let\zref@hex\ltx@firstofone
  \let\zref@unhex\ltx@firstofone
\else
  \let\zref@hex\pdf@escapehex
  \let\zref@unhex\pdf@unescapehex
\fi
%    \end{macrocode}
%
%    \begin{macro}{\ifZREF@pa@list}
%    \begin{macrocode}
\ltx@newif\ifZREF@pa@list
%    \end{macrocode}
%    \end{macro}
%    \begin{macro}{\zref@listpageattr}
%    \begin{macrocode}
\ZREF@IfDefinable\zlistpageattr\def{%
  {\ZREF@pa@listtrue}%
}
%    \end{macrocode}
%    \end{macro}
%    \begin{macro}{\ZREF@pa@AfterLastShipout}
%    \begin{macrocode}
\def\ZREF@pa@AfterLastShipout{%
  \ifZREF@pa@list
    \edef\ZREF@page@max{\the\value{abspage}}%
    \ltx@ifundefined{ZREF@org@testdef}{%
      \let\ZREF@org@testdef\@testdef
      \def\@testdef##1##2##3{%
        \ZREF@org@testdef{##1}{##2}{##3}%
        \def\ZREF@temp{##1}%
        \ifx\ZREF@temp\ZREF@RefPrefix
          \expandafter\xdef\csname##1@##2\endcsname{##3}%
        \fi
      }%
    }{}%
    \AtVeryEndDocument{\ZREF@pa@AtVeryEnd}%
  \fi
}
%    \end{macrocode}
%    \end{macro}
%    \begin{macro}{\ZREF@pa@AtVeryEnd}
%    \begin{macrocode}
\let\ZREF@temp=Y%
\ltx@IfUndefined{pdfpageattr}{}{\let\ZREF@temp=N}
\ifluatex \let\ZREF@temp=N \fi
\ifx\ZREF@temp Y
 \expandafter\@firstoftwo
\else
 \expandafter\@secondoftwo
\fi
 {%
  \def\ZREF@pa@AtVeryEnd{}%
 }
 {%
  \def\ZREF@pa@AtVeryEnd{%
    \begingroup
      \toks@{List of \ltx@backslashchar
             \ifluatex pdfvariable\else pdf\fi
             pdfpageattr:\MessageBreak}%
      \count@=1 %
      \ZREF@pa@ListPage
      \edef\x{\endgroup
        \noexpand\@PackageInfoNoLine{zref-pageattr}{%
          \the\toks@
        }%
      }%
    \x
  }%
%    \end{macrocode}
%    \end{macro}
%    \begin{macro}{\zref@pageattr}
%    \begin{macrocode}
\def\zref@pdfpageattr#1{%
  \zref@unhex{%
    \zref@extract{thepage\ZREF@number{#1}}{pdfpageattr}%
  }%
}
% compability, \zref@pageattr was defined in older versions
\let\zref@pageattr\zref@pdfpageattr
%    \end{macrocode}
%    \end{macro}
%    \begin{macro}{\zref@pageattr@used}
%    \begin{macrocode}
\ZREF@Robust\def\zref@pageattr@used#1{%
  \zref@refused{thepage\ZREF@number{#1}}%
}
%    \end{macrocode}
%    \end{macro}
%    \begin{macro}{\ZREF@pa@ListPage}
%    \begin{macrocode}
  \def\ZREF@pa@ListPage{%
    \edef\x{%
      \toks@={%
        \the\toks@
        Page \the\count@:%
        \noexpand\MessageBreak
        \zref@ifrefundefined{thepage\the\count@}{}{%
          <<\zref@pdfpageattr\count@>>%
          \noexpand\MessageBreak
        }%
      }%
    }\x
    \ifnum\ZREF@page@max>\count@
      \advance\count@ by\ltx@one
    \else
      \expandafter\ltx@gobble
    \fi
    \ZREF@pa@ListPage
  }%
}
%    \end{macrocode}
%    \end{macro}
%    \begin{macrocode}
\let\ZREF@temp=Y%
\ltx@IfUndefined{pdfpagesattr}{}{\let\ZREF@temp=N}
\ifluatex \let\ZREF@temp=N \fi
\ifx\ZREF@temp N
 \expandafter\@firstofone
\fi
 {%
%    \end{macrocode}
%    \begin{macro}{\zref@pdfpagesattr}
%    \begin{macrocode}
  \def\zref@pdfpagesattr{%
    \zref@unhex{%
      \zref@extract{LastPage}{pdfpagesattr}%
    }%
  }%
%    \end{macrocode}
%    \end{macro}
%    \begin{macro}{\zref@pdfpagesattr@used}
%    \begin{macrocode}
  \ZREF@Robust\def\zref@pdfpagesattr@used{%
    \zref@refused{LastPage}%
  }%
%    \end{macrocode}
%    \end{macro}
%    \begin{macrocode}
  \ltx@LocalAppendToMacro\ZREF@pa@AtVeryEnd{%
    \@PackageInfoNoLine{zref-pageattr}{%
      \ltx@backslashchar
      \ifluatex pdfvariable\else pdf\fi
      pagesattr:\MessageBreak
      <<\zref@pdfpagesattr>>%
      \MessageBreak
    }%
  }%
}
\AfterLastShipout{%
  \ZREF@pa@AfterLastShipout
}
%    \end{macrocode}
%    \begin{macrocode}
%</pageattr>
%    \end{macrocode}
%
%
% \subsection{Module \xmodule{marks}}
%
%    \begin{macrocode}
%<*marks>
\NeedsTeXFormat{LaTeX2e}
\ProvidesPackage{zref-marks}%
  [2020-07-03 v2.32 Module marks for zref (HO)]%
\RequirePackage{zref-base}[2019/11/29]
\ifx\ZREF@base@ok Y%
\else
  \expandafter\endinput
\fi
%    \end{macrocode}
%
%    \begin{macrocode}
\newcommand*{\zref@marks@register}[3][]{%
  \edef\ZREF@TempName{#1}%
  \edef\ZREF@TempNum{\ZREF@number{#2}}%
  \ifnum\ZREF@TempNum<\ltx@zero %
    \PackageError\ZREF@name{%
      \string\zref@marks@register\ltx@space is called with invalid%
      \MessageBreak
      marks register number (\ZREF@TempNum)%
    }{%
      Use `0' or the command, defined by \string\newmarks.\MessageBreak
      \@ehc
    }%
  \else
    \ifx\ZREF@TempName\ltx@empty
      \edef\ZREF@TempName{mark\romannumeral\ZREF@TempNum}%
    \else
      \edef\ZREF@TempName{marks\ZREF@TempName}%
    \fi
    \ZREF@MARKS@DefineProp{top}%
    \ZREF@MARKS@DefineProp{first}%
    \ZREF@MARKS@DefineProp{bot}%
    \kv@parse{#3}{%
      \ifx\kv@value\relax
        \def\kv@value{top,first,bot}%
      \fi
      \edef\ZREF@temp{\expandafter\ltx@car\kv@key X\@nil}%
      \ifx\ZREF@temp\ZREF@STAR
        \edef\kv@key{\expandafter\ltx@cdr\kv@key\@nil}%
        \zref@newlist\kv@key
      \fi
      \expandafter\comma@parse\expandafter{\kv@value}{%
        \ifcase0\ifx\comma@entry\ZREF@NAME@top 1\else
                \ifx\comma@entry\ZREF@NAME@first 1\else
                \ifx\comma@entry\ZREF@NAME@bot 1\fi\fi\fi\ltx@space
          \PackageWarning{zref-marks}{%
            Use `top', `first' or `bot' for the list values%
            \MessageBreak
            in the third argument of \string\zref@marks@register.%
            \MessageBreak
            Ignoring unkown value `\comma@entry'%
          }%
        \else
          \zref@addprop{\kv@key}{\comma@entry\ZREF@TempName}%
        \fi
        \ltx@gobble
      }%
      \ltx@gobbletwo
    }%
  \fi
}
\def\ZREF@STAR{*}
\def\ZREF@NAME@top{top}
\def\ZREF@NAME@first{first}
\def\ZREF@NAME@bot{bot}
\def\ZREF@MARKS@DefineProp#1{%
  \zref@ifpropundefined{#1\ZREF@TempName}{%
    \ifnum\ZREF@TempNum=\ltx@zero
      \begingroup
        \edef\x{\endgroup
          \noexpand\zref@newprop*{#1\ZREF@TempName}[]{%
            \expandafter\noexpand\csname#1mark\endcsname
          }%
        }%
      \x
    \else
      \begingroup
        \edef\x{\endgroup
          \noexpand\zref@newprop*{#1\ZREF@TempName}[]{%
            \expandafter\noexpand\csname#1marks\endcsname
            \ZREF@TempNum
          }%
        }%
      \x
    \fi
  }{%
    \PackageWarning{zref-marks}{%
      \string\zref@marks@register\ltx@space does not generate the%
      \MessageBreak
      new property `#1\ZREF@TempName', because\MessageBreak
      it is already defined%
    }%
  }%
}
%    \end{macrocode}
%    \begin{macrocode}
%</marks>
%    \end{macrocode}
%
% \subsection{Module \xmodule{runs}}
%
%    This module does not use the label-reference-system.
%    The reference changes with each \hologo{LaTeX}~run and would
%    force a rerun warning always.
%    \begin{macrocode}
%<*runs>
\NeedsTeXFormat{LaTeX2e}
\ProvidesPackage{zref-runs}%
  [2020-07-03 v2.32 Module runs for zref (HO)]%
%    \end{macrocode}
%
%    \begin{macro}{\zruns}
%    \begin{macrocode}
\providecommand*{\zruns}{0}%
\AtBeginDocument{%
  \edef\zruns{\number\numexpr\zruns+1}%
  \begingroup
    \def\on@line{}%
    \PackageInfo{zref-runs}{LaTeX runs: \zruns}%
    \if@filesw
      \immediate\write\@mainaux{%
        \string\gdef\string\zruns{\zruns}%
      }%
    \fi
  \endgroup
}
%    \end{macrocode}
%    \end{macro}
%
%    \begin{macrocode}
%</runs>
%    \end{macrocode}
%
%
% \subsection{Module \xmodule{perpage}}
%
%    \begin{macrocode}
%<*perpage>
\NeedsTeXFormat{LaTeX2e}
\ProvidesPackage{zref-perpage}%
  [2020-07-03 v2.32 Module perpage for zref (HO)]%
\RequirePackage{zref-base}[2019/11/29]
\ifx\ZREF@base@ok Y%
\else
  \expandafter\endinput
\fi
%    \end{macrocode}
%
%    This module resets a counter at page boundaries.
%    Because of the asynchronous output routine
%    page counter properties cannot be asked directly,
%    references are necessary.
%
%    For detecting changed pages module \xmodule{abspage}
%    is loaded.
%    \begin{macrocode}
\RequirePackage{zref-abspage}[2019/11/29]
%    \end{macrocode}
%
%    We group the properties for the needed references
%    in the property list |perpage|. The property |pagevalue|
%    records the correct value of the page counter.
%    \begin{macrocode}
\ZREF@NewPropPageValue
\zref@newlist{perpage}
\zref@addprops{perpage}{abspage,page,pagevalue}
%    \end{macrocode}
%    The page value, known by the reference mechanism, will be
%    stored in counter |zpage|.
%    \begin{macrocode}
\newcounter{zpage}
%    \end{macrocode}
%    Counter |zref@unique| helps in generating unique reference
%    names.
%    \begin{macrocode}
\zref@require@unique
%    \end{macrocode}
%    In order to be able to reset the counter, we hook here
%    into \cs{stepcounter}. In fact two nested hooks are used
%    to allow other packages to use the first hook at the beginning
%    of \cs{stepcounter}.
%    \begin{macrocode}
\let\ZREF@org@stepcounter\stepcounter
\def\stepcounter#1{%
  \ifcsname @stepcounterhook@#1\endcsname
    \csname @stepcounterhook@#1\endcsname
  \fi
  \ZREF@org@stepcounter{#1}%
}
%    \end{macrocode}
%    \cs{@stpelt} must be adapted due to the change in latex
%    2015-01, see https://github.com/ho-tex/zref/issues/26
%    \begin{macrocode}
\let\ZREF@org@@stpelt\@stpelt
\def\@stpelt#1{%
  \ifcsname ZREF@perpage@#1\endcsname
    \begingroup
      \let\stepcounter\ZREF@org@stepcounter
      \ZREF@org@@stpelt{#1}%
    \endgroup
    \expandafter\ltx@gobbletwo
  \fi
  \ZREF@org@@stpelt{#1}%
}
%    \end{macrocode}
%
%    \begin{macro}{\zmakeperpage}
%    Makro \cs{zmakeperpage} resets a counter at each page
%    break. It uses the same syntax and semantics as
%    \cs{MakePerPage} from package \xpackage{perpage} \cite{perpage}.
%    The initial start value can be given by the optional
%    argument. Default is one that means after the first
%    \cs{stepcounter} on a new page the counter starts with
%    one.
%    \begin{macrocode}
\ZREF@IfDefinable\zmakeperpage\def{%
  {%
    \@ifnextchar[\ZREF@makeperpage@opt{\ZREF@@makeperpage[\ltx@zero]}%
  }%
}
%    \end{macrocode}
%    \end{macro}
%    We hook before the counter is incremented in \cs{stepcounter},
%    package \xpackage{perpage} afterwards. Thus a little
%    calculation is necessary.
%    \begin{macrocode}
\def\ZREF@makeperpage@opt[#1]{%
  \begingroup
    \edef\x{\endgroup
      \noexpand\ZREF@@makeperpage[\number\numexpr#1-1\relax]%
    }%
  \x
}
%    \end{macrocode}
%    \begin{macrocode}
\def\ZREF@@makeperpage[#1]#2{%
  \@ifundefined{@stepcounterhook@#2}{%
    \expandafter\gdef\csname @stepcounterhook@#2\endcsname{}%
  }{}%
  \expandafter\gdef\csname ZREF@perpage@#2\endcsname{%
    \ZREF@@perpage@step{#2}{#1}%
  }%
  \expandafter\g@addto@macro\csname @stepcounterhook@#2\endcsname{%
    \ifcsname ZREF@perpage@#2\endcsname
      \csname ZREF@perpage@#2\endcsname
    \fi
  }%
}
%    \end{macrocode}
%    \begin{macro}{\ZREF@@perpage@step}
%    The heart of this module follows.
%    \begin{macrocode}
\def\ZREF@@perpage@step#1#2{%
%    \end{macrocode}
%    First the reference is generated.
%    \begin{macrocode}
  \global\advance\c@zref@unique\ltx@one
  \begingroup
    \expandafter
    \zref@labelbylist\expandafter{\thezref@unique}{perpage}%
%    \end{macrocode}
%    The \cs{expandafter} commands are necessary, because \cs{ZREF@temp}
%    is also used inside of \cs{zref@labelbylist}.
%
%    The evaluation of the reference follows. If the reference
%    is not yet kwown, we use the page counter as approximation.
%    \begin{macrocode}
    \zref@ifrefundefined\thezref@unique{%
      \global\c@zpage=\c@page
      \global\let\thezpage\thepage
      \expandafter\xdef\csname ZREF@abspage@#1\endcsname{%
        \number\c@abspage
      }%
    }{%
%    \end{macrocode}
%    The reference is used to set \cs{thezpage} and counter |zpage|.
%    \begin{macrocode}
      \global\c@zpage=\zref@extract\thezref@unique{pagevalue}\relax
      \xdef\thezpage{\noexpand\zref@extract{\thezref@unique}{page}}%
      \expandafter\xdef\csname ZREF@abspage@#1\endcsname{%
        \zref@extractdefault\thezref@unique
            {abspage}{\number\c@abspage}%
      }%
    }%
%    \end{macrocode}
%    Page changes are detected by a changed absolute page number.
%    \begin{macrocode}
    \expandafter\ifx\csname ZREF@abspage@#1\expandafter\endcsname
                    \csname ZREF@currentabspage@#1\endcsname
    \else
      \global\csname c@#1\endcsname=#2\relax
      \global\expandafter\let
          \csname ZREF@currentabspage@#1\expandafter\endcsname
          \csname ZREF@abspage@#1\endcsname
    \fi
  \endgroup
}
%    \end{macrocode}
%    \end{macro}
%    \begin{macro}{\zunmakeperpage}
%    Macro \cs{zunmakeperpage} cancels the effect of \cs{zmakeperpage}.
%    \begin{macrocode}
\ZREF@IfDefinable\zunmakeperpage\def{%
  #1{%
    \global\expandafter
    \let\csname ZREF@perpage@#1\endcsname\@undefined
  }%
}
%    \end{macrocode}
%    \end{macro}
%
%    \begin{macrocode}
%</perpage>
%    \end{macrocode}
%
%
% \subsection{Module \xmodule{titleref}}
%
%    \begin{macrocode}
%<*titleref>
\NeedsTeXFormat{LaTeX2e}
\ProvidesPackage{zref-titleref}%
  [2020-07-03 v2.32 Module titleref for zref (HO)]%
\RequirePackage{zref-base}[2019/11/29]
\ifx\ZREF@base@ok Y%
\else
  \expandafter\endinput
\fi
\RequirePackage{gettitlestring}[2009/12/08]
%    \end{macrocode}
%
% \subsubsection{Implementation}
%    \begin{macrocode}
\RequirePackage{keyval}
%    \end{macrocode}
%
%    This module makes section and caption titles available
%    for the reference system. It uses some of the ideas
%    of package \xpackage{nameref} and \xpackage{titleref}.
%
%    Now we can add the property \verb|title| is added to
%    the main property list.
%    \begin{macrocode}
\ZREF@NewPropTitle
\zref@addprop\ZREF@mainlist{title}%
%    \end{macrocode}
%
%    The title strings go into the \xfile{.aux} file, thus
%    they need some kind of protection.
%    Package \xpackage{titleref} uses a protected expansion
%    method. The advantage is that this can be used to
%    cleanup the string and to remove \cs{label}, \cs{index}
%    and other macros unwanted for referencing.
%    But there is the risk that fragile stuff can break.
%
%    Therefore package \xpackage{nameref} does not expand
%    the string. Thus the entries can safely be written to
%    the \xfile{.aux} file. But potentially dangerous
%    macros such as \cs{label} remain in the string and
%    can cause problems when using the string in references.
%
%    \begin{macro}{\ifzref@titleref@expand}
%    The switch \cs{ifzref@titleref@expand} distinguishes
%    between the both methods. Package \xpackage{nameref}'s
%    behaviour is achieved by setting the switch to false,
%    otherwise \xpackage{titleref}'s expansion is used.
%    Default is false.
%    \begin{macrocode}
\newif\ifzref@titleref@expand
%    \end{macrocode}
%    \end{macro}
%
%    \begin{macro}{\ZREF@titleref@hook}
%    The hook \cs{ZREF@titleref@hook} allows to extend
%    the cleanup for the expansion method. Thus
%    unnecessary macros can be removed or dangerous commands
%    removed. The hook is executed before the expansion of
%    \cs{zref@titleref@current}.
%    \begin{macrocode}
\let\ZREF@titleref@hook\ltx@empty
%    \end{macrocode}
%    \end{macro}
%    \begin{macro}{\zref@titleref@cleanup}
%    The hook should not be used directly, instead we provide the
%    macro \cs{zref@titleref@cleanup} to add stuff to the hook
%    and prevents that a previous non-empty content is not discarded
%    accidently.
%    \begin{macrocode}
\ZREF@Robust\def\zref@titleref@cleanup#1{%
   \begingroup
    \toks@\expandafter{%
      \ZREF@titleref@hook
      #1%
    }%
  \expandafter\endgroup
  \expandafter\def\expandafter\ZREF@titleref@hook\expandafter{%
    \the\toks@
  }%
}%
%    \end{macrocode}
%    \end{macro}
%
%    \begin{macro}{\ifzref@titleref@stripperiod}
%    Sometimes a title contains a period at the end.
%    Package \xpackage{nameref} removes this. This behaviour
%    is controlled by the switch \cs{ifzref@titleref@stripperiod}
%    and works regardless of the setting of option \xoption{expand}.
%    Period stripping is the default.
%    \begin{macrocode}
\newif\ifzref@titleref@stripperiod
\zref@titleref@stripperiodtrue
%    \end{macrocode}
%    \end{macro}
%
%    \begin{macro}{\zref@titleref@setcurrent}
%    Macro \cs{zref@titleref@setcurrent} sets a new current title
%    stored in \cs{zref@titleref@current}. Some cleanup and
%    expansion is performed that can be controlled by the
%    previous switches.
%    \begin{macrocode}
\ZREF@Robust\def\zref@titleref@setcurrent#1{%
  \ifzref@titleref@expand
    \GetTitleStringExpand{#1}%
  \else
    \GetTitleStringNonExpand{#1}%
  \fi
  \edef\zref@titleref@current{%
    \detokenize\expandafter{\GetTitleStringResult}%
  }%
  \ifzref@titleref@stripperiod
    \edef\zref@titleref@current{%
      \expandafter\ZREF@stripperiod\zref@titleref@current
      \ltx@empty.\ltx@empty\@nil
    }%
  \fi
}%
\GetTitleStringDisableCommands{%
  \ZREF@titleref@hook
}
%    \end{macrocode}
%    \end{macro}
%    \begin{macro}{\ZREF@stripperiod}
%    If \cs{ZREF@stripperiod} is called, the argument consists of
%    space tokens and tokens with catcode 12 (other), because
%    of \hologo{eTeX}'s \cs{detokenize}.
%    \begin{macrocode}
\def\ZREF@stripperiod#1.\ltx@empty#2\@nil{#1}%
%    \end{macrocode}
%    \end{macro}
%
% \subsubsection{User interface}
%
%    \begin{macro}{\ztitlerefsetup}
%    The behaviour of module \xmodule{titleref} is controlled
%    by switches and a hook. They can be set by \cs{ztitlerefsetup}
%    with a key value interface, provided by package \xpackage{keyval}.
%    Also the current title can be given explicitly by the key
%    \verb|title|.
%    \begin{macrocode}
\define@key{ZREF@TR}{expand}[true]{%
  \csname zref@titleref@expand#1\endcsname
}%
\define@key{ZREF@TR}{stripperiod}[true]{%
  \csname zref@titleref@stripperiod#1\endcsname
}%
\define@key{ZREF@TR}{cleanup}{%
  \zref@titleref@cleanup{#1}%
}%
\define@key{ZREF@TR}{title}{%
  \def\zref@titleref@current{#1}%
}%
\ZREF@IfDefinable\ztitlerefsetup\def{%
  {\kvsetkeys{ZREF@TR}}%
}%
%    \end{macrocode}
%    \end{macro}
%
%    \begin{macro}{\ztitleref}
%    The user command \cs{ztitleref} references the title.
%    For safety \cs{label} is disabled to prevent multiply
%    defined references.
%    \begin{macrocode}
\ZREF@IfDefinable\ztitleref\def{%
  {\zref@wrapper@babel\ZREF@titleref}%
}%
\def\ZREF@titleref#1{%
  \begingroup
    \zref@refused{#1}%
    \let\label\ltx@gobble
    \zref@extract{#1}{title}%
  \endgroup
}%
%    \end{macrocode}
%    \end{macro}
%
% \subsubsection{Patches for section and caption commands}
%
%    The section and caption macros are patched to extract the
%    title data.
%
%    Captions of figures and tables.
%    \begin{macrocode}
\AtBeginDocument{%
  \ZREF@patch{@caption}{%
    \long\def\@caption#1[#2]{%
      \zref@titleref@setcurrent{#2}%
      \ZREF@org@@caption{#1}[{#2}]%
    }%
  }%
%    \end{macrocode}
%
%    Section commands without star. The title version for the
%    table of contents is used because it is usually shorter and
%    more robust.
%    \begin{macrocode}
  \ZREF@patch{@part}{%
    \def\@part[#1]{%
      \zref@titleref@setcurrent{#1}%
      \ZREF@org@@part[{#1}]%
    }%
  }%
  \ZREF@patch{@chapter}{%
    \def\@chapter[#1]{%
      \zref@titleref@setcurrent{#1}%
      \ZREF@org@@chapter[{#1}]%
    }%
  }%
  \ZREF@patch{@sect}{%
    \def\@sect#1#2#3#4#5#6[#7]{%
      \zref@titleref@setcurrent{#7}%
      \ZREF@org@@sect{#1}{#2}{#3}{#4}{#5}{#6}[{#7}]%
    }%
  }%
%    \end{macrocode}
%    The star versions of the section commands.
%    \begin{macrocode}
  \ZREF@patch{@spart}{%
    \def\@spart#1{%
      \zref@titleref@setcurrent{#1}%
      \ZREF@org@@spart{#1}%
    }%
  }%
  \ZREF@patch{@schapter}{%
    \def\@schapter#1{%
      \zref@titleref@setcurrent{#1}%
      \ZREF@org@@schapter{#1}%
    }%
  }%
  \ZREF@patch{@ssect}{%
    \def\@ssect#1#2#3#4#5{%
      \zref@titleref@setcurrent{#5}%
      \ZREF@org@@ssect{#1}{#2}{#3}{#4}{#5}%
    }%
  }%
%    \end{macrocode}
%
% \subsubsection{Environment \textsf{description}}
%
%    \begin{macrocode}
  \ZREF@patch{descriptionlabel}{%
    \def\descriptionlabel#1{%
      \zref@titleref@setcurrent{#1}%
      \ZREF@org@descriptionlabel{#1}%
    }%
  }%
%    \end{macrocode}
%
% \subsubsection{Class \xclass{memoir}}
%
%    \begin{macrocode}
  \@ifclassloaded{memoir}{%
    \ltx@IfUndefined{ifheadnameref}{}{%
      \def\@chapter[#1]#2{%
        \ltx@IfUndefined{ch@pt@c}{%
          \zref@titleref@setcurrent{#1}%
        }{%
          \ifx\ch@pt@c\ltx@empty
            \zref@titleref@setcurrent{#2}%
          \else
            \def\NR@temp{#1}%
            \ifx\NR@temp\ltx@empty
              \expandafter\zref@titleref@setcurrent
              \expandafter{\ch@pt@c}%
            \else
              \ifheadnameref
                \zref@titleref@setcurrent{#1}%
              \else
                \expandafter\zref@titleref@setcurrent
                \expandafter{\ch@pt@c}%
              \fi
            \fi
          \fi
        }%
        \ZREF@org@@chapter[{#1}]{#2}%
      }%
      \ZREF@patch{M@sect}{%
        \def\M@sect#1#2#3#4#5#6[#7][#8]{%
          \ifheadnameref
            \zref@titleref@setcurrent{#8}%
          \else
            \zref@titleref@setcurrent{#7}%
          \fi
          \ZREF@org@M@sect{#1}{#2}{#3}{#4}{#5}{#6}[{#7}][{#8}]%
        }%
      }%
    }%
  }{}%
%    \end{macrocode}
%
% \subsubsection{Class \xclass{beamer}}
%
%    \begin{macrocode}
  \@ifclassloaded{beamer}{%
    \ZREF@patch{beamer@section}{%
      \long\def\beamer@section[#1]{%
        \zref@titleref@setcurrent{#1}%
        \ZREF@org@beamer@section[{#1}]%
      }%
    }%
    \ZREF@patch{beamer@subsection}{%
      \long\def\beamer@subsection[#1]{%
        \zref@titleref@setcurrent{#1}%
        \ZREF@org@beamer@subsection[{#1}]%
      }%
    }%
    \ZREF@patch{beamer@subsubsection}{%
      \long\def\beamer@subsubsection[#1]{%
        \zref@titleref@setcurrent{#1}%
        \ZREF@org@beamer@subsubsection[{#1}]%
      }%
    }%
  }{}%
%    \end{macrocode}
%
% \subsubsection{Package \xpackage{titlesec}}
%
%    \begin{macrocode}
  \@ifpackageloaded{titlesec}{%
    \ZREF@patch{ttl@sect@i}{%
      \def\ttl@sect@i#1#2[#3]#4{%
        \zref@titleref@setcurrent{#4}%
        \ZREF@org@ttl@sect@i{#1}{#2}[{#3}]{#4}%
      }%
    }%
    \ZREF@patch{ttl@straight@i}{%
      \def\ttl@straight@i#1[#2]#3{%
        \def\ZREF@temp{#2}%
        \ifx\ZREF@temp\ltx@empty
          \zref@titleref@setcurrent{#3}%
        \else
          \zref@titleref@setcurrent{#2}%
        \fi
        \ZREF@org@ttl@straight@i{#1}[{#2}]{#3}%
      }%
    }%
  }{}%
%    \end{macrocode}
%
% \subsubsection{Package \xpackage{longtable}}
%
%    Package \xpackage{longtable}: some support for its \cs{caption}.
%    However \cs{label} inside the caption is not supported.
%    \begin{macrocode}
  \@ifpackageloaded{longtable}{%
    \ZREF@patch{LT@c@ption}{%
      \def\LT@c@ption#1[#2]#3{%
        \ZREF@org@LT@c@ption{#1}[{#2}]{#3}%
        \zref@titleref@setcurrent{#2}%
      }%
    }%
  }{}%
%    \end{macrocode}
%
% \subsubsection{Package \xpackage{listings}}
%
%    Package \xpackage{listings}: support for its caption.
%    \begin{macrocode}
  \@ifpackageloaded{listings}{%
    \ZREF@patch{lst@MakeCaption}{%
      \def\lst@MakeCaption{%
        \ifx\lst@label\ltx@empty
        \else
          \expandafter\zref@titleref@setcurrent\expandafter{%
            \lst@@caption
          }%
        \fi
        \ZREF@org@lst@MakeCaption
      }%
    }%
  }{}%
%    \end{macrocode}
%
% \subsubsection{Theorems}
%
%    \begin{macrocode}
  \ZREF@patch{@opargbegintheorem}{%
    \def\@opargbegintheorem#1#2#3{%
      \zref@titleref@setcurrent{#3}%
      \ZREF@org@@opargbegintheorem{#1}{#2}{#3}%
    }%
  }%
  \@ifpackageloaded{amsthm}{%
    \begingroup
      \edef\x{macro:\string#1\string#2[\string#3]}%
      \@onelevel@sanitize\x
      \def\y#1->#2\@nil{#1}%
      \edef\z{\expandafter\y\meaning\@begintheorem->\@nil}%
      \@onelevel@sanitize\z
    \expandafter\endgroup
    \ifx\x\z
      \ZREF@patch{@begintheorem}{%
        \def\@begintheorem#1#2[#3]{%
          \zref@titleref@setcurrent{#3}%
          \ZREF@org@@begintheorem{#1}{#2}[{#3}]%
        }%
      }%
    \fi
  }{}%
%    \end{macrocode}
%
%    \begin{macrocode}
}
%    \end{macrocode}
%
%    \begin{macrocode}
%</titleref>
%    \end{macrocode}
%
%
% \subsection{Module \xmodule{xr}}
%
%    \begin{macrocode}
%<*xr>
\NeedsTeXFormat{LaTeX2e}
\ProvidesPackage{zref-xr}%
  [2020-07-03 v2.32 Module xr for zref (HO)]%
\RequirePackage{zref-base}[2019/11/29]
\ifx\ZREF@base@ok Y%
\else
  \expandafter\endinput
\fi
%    \end{macrocode}
%    \begin{macrocode}
\RequirePackage{keyval}
\RequirePackage{kvoptions}[2010/02/22]
%    \end{macrocode}
%
%    We declare property |url|, because this is added,
%    if a reference is imported and has not already set this field.
%    Or if |hyperref| is used, then this property can be asked.
%    \begin{macrocode}
\zref@newprop{url}{}%
\zref@newprop{urluse}{}%
%    \end{macrocode}
%    \begin{macrocode}
\zref@newprop{externaldocument}{}%
%    \end{macrocode}
%
%    Most code, especially the handling of the \xfile{.aux}
%    files are taken from David Carlisle's \xpackage{xr} package.
%    Therefore I drop the documentation for these macros here.
%
%    \begin{macro}{\zref@xr@ext}
%    If the URL is not specied, then assume
%    processed file with a guessed extension.
%    Use the setting of \xpackage{hyperref} if available.
%    \begin{macrocode}
\providecommand*{\zref@xr@ext}{%
  \ltx@ifundefined{XR@ext}{pdf}{\XR@ext}%
}%
%    \end{macrocode}
%    \end{macro}
%    \begin{macro}{\ifZREF@xr@zreflabel}
%    The use of the star form of \cs{zexternaldocument}
%    is remembered in the switch \cs{ifZREF@xr@zreflabel}.
%    \begin{macrocode}
\newif\ifZREF@xr@zreflabel
%    \end{macrocode}
%    \end{macro}
%    \begin{macrocode}
\SetupKeyvalOptions{%
  family=ZREF@XR,%
  prefix=ZREF@xr@%
}
\DeclareBoolOption[true]{tozreflabel}
\DeclareBoolOption[false]{toltxlabel}
\DeclareBoolOption{verbose}
\define@key{ZREF@XR}{ext}{%
  \def\zref@xr@{#1}%
}
\DeclareBoolOption[false]{urluse}
%    \end{macrocode}
%    \begin{macro}{\zxrsetup}
%    \begin{macrocode}
\newcommand*{\zxrsetup}{%
  \kvsetkeys{ZREF@XR}%
}%
%    \end{macrocode}
%    \end{macro}
%
%    \begin{macro}{\ZREF@xr@URL}
%    \begin{macrocode}
\newcount\ZREF@xr@URL
\ZREF@xr@URL=\ltx@zero
%    \end{macrocode}
%    \end{macro}
%    \begin{macro}{\ZREF@xr@AddURL}
%    \begin{macrocode}
\def\ZREF@xr@AddURL#1{%
  \begingroup
    \def\ZREF@temp{#1}%
    \count@=\ltx@one
    \ZREF@xr@@AddUrl
  \endgroup
}
%    \end{macrocode}
%    \end{macro}
%    \begin{macro}{\ZREF@xr@@AddUrl}
%    \begin{macrocode}
\def\ZREF@xr@@AddUrl{%
  \ifnum\count@>\ZREF@xr@URL
    \global\advance\ZREF@xr@URL by\ltx@one
    \xdef\ZREF@xr@theURL{\romannumeral\ZREF@xr@URL}%
    \global\expandafter\let
        \csname Z@U@\ZREF@xr@theURL\endcsname\ZREF@temp
    \@PackageInfo{zref-xr}{%
      \ltx@backslashchar Z@U@\ZREF@xr@theURL:\MessageBreak
      \ZREF@temp\MessageBreak
    }%
  \else
    \expandafter
    \ifx\csname Z@U@\romannumeral\count@\endcsname\ZREF@temp
      \xdef\ZREF@xr@theURL{\romannumeral\count@}%
    \else
      \expandafter\expandafter\expandafter\ZREF@xr@@AddUrl
    \fi
  \fi
}
%    \end{macrocode}
%    \end{macro}
%
%    \begin{macro}{\zexternaldocument}
%    In its star form it looks for \cs{newlabel},
%    otherwise for \cs{zref@newlabel}.
%     Later we
%    will read \xfile{.aux} files that expects
%    |@| to have catcode 11 (letter).
%    \begin{macrocode}
\ZREF@IfDefinable\zexternaldocument\def{%
  {%
    \ZREF@NewPropAnchor
    \ZREF@NewPropTitle
    \begingroup
      \csname @safe@actives@true\endcsname
      \makeatletter
      \@ifstar{%
        \ZREF@xr@zreflabelfalse
        \@testopt\ZREF@xr@externaldocument{}%
      }{%
        \ZREF@xr@zreflabeltrue
        \@testopt\ZREF@xr@externaldocument{}%
      }%
  }%
}%
%    \end{macrocode}
%    \end{macro}
%
%    If the \cs{include} featuer was used, there can be
%    several \xfile{.aux} files. These files are read
%    one after another, especially they are not recursively
%    read in order to save read registers. Thus it can
%    happen that the read order of the newlabel commands
%    differs from \hologo{LaTeX}'s order using \cs{input}.
%
%    \begin{macro}{\ZREF@xr@externaldocument}
%    It reads the remaining arguments. \cs{newcommand}
%    comes in handy for the optional argument.
%    \begin{macrocode}
\def\ZREF@xr@externaldocument[#1]#2{%
    \def\ZREF@xr@prefix{#1}%
    \let\ZREF@xr@filelist\ltx@empty
    \edef\ZREF@xr@externalfile{#2}%
    \edef\ZREF@xr@file{\ZREF@xr@externalfile.aux}%
    \filename@parse{#2}%
    \@testopt\ZREF@xr@graburl{#2.\zref@xr@ext}%
}%
\def\ZREF@xr@graburl[#1]{%
    \edef\ZREF@xr@url{#1}%
    \ifZREF@xr@urluse
      \expandafter\ZREF@xr@AddURL\expandafter{\ZREF@xr@url}%
      \expandafter\def\expandafter\ZREF@xr@url
      \expandafter{\csname Z@U@\ZREF@xr@theURL\endcsname}%
    \fi
    \ZREF@xr@checkfile
  \endgroup
}%
%    \end{macrocode}
%    \end{macro}
%    \begin{macro}{\ZREF@xr@processfile}
%    We follow \xpackage{xr} here, \cs{IfFileExists} offers
%    a nicer test, but we have to open the file anyway.
%    \begin{macrocode}
\def\ZREF@xr@checkfile{%
  \openin\@inputcheck\ZREF@xr@file\relax
  \ifeof\@inputcheck
    \PackageWarning{zref-xr}{%
      File `\ZREF@xr@file' not found or empty,\MessageBreak
      labels not imported%
    }%
  \else
    \PackageInfo{zref-xr}{%
      Label \ifZREF@xr@zreflabel (zref) \fi
      import from `\ZREF@xr@file'%
    }%
    \def\ZREF@xr@found{0}%
    \def\ZREF@xr@ignored@empty{0}%
    \def\ZREF@xr@ignored@zref{0}%
    \def\ZREF@xr@ignored@ltx{0}%
    \ZREF@xr@processfile
    \closein\@inputcheck
    \begingroup
      \let\on@line\ltx@empty
      \PackageInfo{zref-xr}{%
        Statistics for `\ZREF@xr@file':\MessageBreak
        \ZREF@xr@found\space
        \ifZREF@xr@zreflabel zref\else LaTeX\fi\space
        label(s) found%
        \ifnum\ZREF@xr@ignored@empty>0 %
          ,\MessageBreak
          \ZREF@xr@ignored@empty\space empty label(s) ignored%
        \fi
        \ifnum\ZREF@xr@ignored@zref>0 %
          ,\MessageBreak
          \ZREF@xr@ignored@zref\space
          duplicated zref label(s) ignored%
        \fi
        \ifnum\ZREF@xr@ignored@ltx>0 %
          ,\MessageBreak
          \ZREF@xr@ignored@ltx\space
          duplicated latex label(s) ignored%
        \fi
      }%
    \endgroup
  \fi
  \ifx\ZREF@xr@filelist\ltx@empty
  \else
    \edef\ZREF@xr@file{%
      \expandafter\ltx@car\ZREF@xr@filelist\@nil
    }%
    \edef\ZREF@xr@filelist{%
      \expandafter\ltx@cdr\ZREF@xr@filelist\ltx@empty\@nil
    }%
    \expandafter\ZREF@xr@checkfile
  \fi
}%
%    \end{macrocode}
%    \end{macro}
%
%    \begin{macro}{\ZREF@xr@processfile}
%    \begin{macrocode}
\def\ZREF@xr@processfile{%
  \read\@inputcheck to\ZREF@xr@line
  \expandafter\ZREF@xr@processline\ZREF@xr@line..\ZREF@nil
  \ifeof\@inputcheck
  \else
    \expandafter\ZREF@xr@processfile
  \fi
}%
%    \end{macrocode}
%    \end{macro}
%    \begin{macro}{\ZREF@xr@processline}
%    The most work must be done for analyzing the
%    arguments of \cs{newlabel}.
%    \begin{macrocode}
\long\def\ZREF@xr@processline#1#2#3\ZREF@nil{%
  \def\x{#1}%
  \toks@{#2}%
  \ifZREF@xr@zreflabel
    \ifx\x\ZREF@xr@zref@newlabel
      \expandafter
      \ZREF@xr@process@zreflabel\ZREF@xr@line...\ZREF@nil
    \fi
  \else
    \ifx\x\ZREF@xr@newlabel
      \expandafter
      \ZREF@xr@process@label\ZREF@xr@line...[]\ZREF@nil
    \fi
  \fi
  \ifx\x\ZREF@xr@@input
    \edef\ZREF@xr@filelist{%
      \etex@unexpanded\expandafter{\ZREF@xr@filelist}%
      {\filename@area\the\toks@}%
    }%
  \fi
}%
\def\ZREF@xr@process@zreflabel\zref@newlabel#1#2#3\ZREF@nil{%
  \edef\ZREF@xr@refname{Z@R@\ZREF@xr@prefix#1}%
  \edef\ZREF@xr@found{\the\numexpr\ZREF@xr@found+1\relax}%
  \def\x{#2}%
  \edef\ZREF@xr@tempname{$temp$}%
  \edef\ZREF@xr@temprefname{Z@R@\ZREF@xr@tempname}%
  \let\ZREF@xr@list\x
  \ifx\ZREF@xr@list\ltx@empty
    \PackageWarningNoLine{zref-xr}{%
      Label `#1' without properties ignored\MessageBreak
      in file `\ZREF@xr@file'%
    }%
    \edef\ZREF@xr@ignored@empty{%
      \the\numexpr\ZREF@xr@ignored@empty+1\relax
    }%
  \else
    \expandafter\ZREF@xr@checklist\x\ZREF@nil
    \expandafter\let\csname\ZREF@xr@temprefname\endcsname\x
    \expandafter\ltx@LocalAppendToMacro
    \csname\ZREF@xr@temprefname\expandafter\endcsname
    \expandafter{%
      \expandafter\externaldocument\expandafter{%
        \ZREF@xr@externalfile
      }%
    }%
    \ZREF@xr@urlcheck\ZREF@xr@tempname
    \ifZREF@xr@tozreflabel
      \@ifundefined{\ZREF@xr@refname}{%
        \ifZREF@xr@verbose
          \PackageInfo{zref-xr}{%
            Import to zref label `\ZREF@xr@tempname#1'%
          }%
        \fi
        \global\expandafter
        \let\csname\ZREF@xr@refname\expandafter\endcsname
        \csname\ZREF@xr@temprefname\endcsname
      }{%
        \ZREF@xr@zref@ignorewarning{\ZREF@xr@prefix#1}%
      }%
    \fi
    \ifZREF@xr@toltxlabel
      \ZREF@xr@tolabel{\ZREF@xr@tempname}{\ZREF@xr@prefix#1}%
    \fi
  \fi
}%
\def\ZREF@xr@process@label\newlabel#1#2#3[#4]#5\ZREF@nil{%
  \def\ZREF@xr@refname{Z@R@\ZREF@xr@prefix#1}%
  \edef\ZREF@xr@found{\the\numexpr\ZREF@xr@found+1\relax}%
  \def\x{#2}%
  \edef\ZREF@xr@tempname{$temp$}%
  \edef\ZREF@xr@temprefname{Z@R@\ZREF@xr@tempname}%
  \expandafter\ZREF@xr@scanparams
      \csname\ZREF@xr@temprefname\expandafter\endcsname
      \x{}{}{}{}{}\ZREF@nil
  \ifx\\#4\\%
  \else
    % ntheorem knows an optional argument at the end of \newlabel
    \ZREF@NewPropTheotype
    \expandafter\ltx@LocalAppendToMacro
        \csname\ZREF@xr@temprefname\endcsname{\theotype{#4}}%
  \fi
  \expandafter\ltx@LocalAppendToMacro
  \csname\ZREF@xr@temprefname\expandafter\endcsname\expandafter{%
    \expandafter\externaldocument\expandafter{%
      \ZREF@xr@externalfile
    }%
  }%
  \ZREF@xr@urlcheck\ZREF@xr@tempname
  \ifZREF@xr@tozreflabel
    \@ifundefined{\ZREF@xr@refname}{%
      \ifZREF@xr@verbose
        \PackageInfo{zref-xr}{%
          Import to zref label `\ZREF@xr@prefix#1'%
        }%
      \fi
      \global\expandafter
      \let\csname\ZREF@xr@refname\expandafter\endcsname
      \csname\ZREF@xr@temprefname\endcsname
    }{%
      \ZREF@xr@zref@ignorewarning{\ZREF@xr@prefix#1}%
    }%
  \fi
  \ifZREF@xr@toltxlabel
    \ZREF@xr@tolabel{\ZREF@xr@tempname}{\ZREF@xr@prefix#1}%
  \fi
}
\def\ZREF@xr@zref@newlabel{\zref@newlabel}%
\def\ZREF@xr@newlabel{\newlabel}%
\def\ZREF@xr@@input{\@input}%
\def\ZREF@xr@relax{\relax}%
%    \end{macrocode}
%    \end{macro}
%    \begin{macro}{\ZREF@xr@tolabel}
%    \begin{macrocode}
\def\ZREF@xr@tolabel#1#2{%
  \ifZREF@xr@verbose
    \PackageInfo{zref-xr}{%
      Import to LaTeX label `#2'%
    }%
  \fi
  \zref@wrapper@unexpanded{%
    \expandafter\xdef\csname r@#2\endcsname{%
      {%
        \ltx@ifundefined{M@TitleReference}{%
          \ltx@ifundefined{TR@TitleReference}{%
            \zref@extractdefault{#1}{default}{}%
          }{%
            \noexpand\TR@TitleReference
            {\zref@extractdefault{#1}{default}{}}%
            {\zref@extractdefault{#1}{title}{}}%
          }%
        }{%
          \noexpand\M@TitleReference
          {\zref@extractdefault{#1}{default}{}}%
          {\zref@extractdefault{#1}{title}{}}%
        }%
      }%
      {\zref@extractdefault{#1}{page}{}}%
      \ltx@ifpackageloaded{nameref}{%
        {\zref@extractdefault{#1}{title}{}}%
        {\zref@extractdefault{#1}{anchor}{}}%
        \zref@ifrefcontainsprop{#1}{urluse}{%
          {\zref@extractdefault{#1}{urluse}{}}%
        }{%
          {\zref@extractdefault{#1}{url}{}}%
        }%
      }{}%
    }%
  }%
}
%    \end{macrocode}
%    \end{macro}
%
%    \begin{macro}{\ZREF@xr@zref@ignorewarning}
%    \begin{macrocode}
\def\ZREF@xr@zref@ignorewarning#1{%
  \PackageWarningNoLine{zref-xr}{%
    Zref label `#1' is already in use\MessageBreak
    in file `\ZREF@xr@file'%
  }%
  \edef\ZREF@xr@ignored@zref{%
    \the\numexpr\ZREF@xr@ignored@zref+1%
  }%
}%
%    \end{macrocode}
%    \end{macro}
%
%    \begin{macro}{\ZREF@xr@ltx@ignorewarning}
%    \begin{macrocode}
\def\ZREF@xr@ltx@ignorewarning#1{%
  \PackageWarningNoLine{zref-xr}{%
    LaTeX label `#1' is already in use\MessageBreak
    in file `\ZREF@xr@file'%
  }%
  \edef\ZREF@xr@ignored@ltx{%
    \the\numexpr\ZREF@xr@ignored@ltx+1%
  }%
}%
%    \end{macrocode}
%    \end{macro}
%
%    \begin{macro}{\ZREF@xr@checklist}
%    \begin{macrocode}
\def\ZREF@xr@checklist#1#2#3\ZREF@nil{%
  \ifx\@undefined#1\relax
    \expandafter\ZREF@xr@checkkey\string#1\@nil
  \fi
  \ifx\\#3\\%
  \else
    \ltx@ReturnAfterFi{%
      \ZREF@xr@checklist#3\ZREF@nil
    }%
  \fi
}%
\def\ZREF@xr@checkkey#1#2\@nil{%
  \zref@ifpropundefined{#2}{%
    \zref@newprop{#2}{}%
  }{}%
}%
%    \end{macrocode}
%    \end{macro}
%
%    \begin{macro}{\ZREF@xr@scanparams}
%    \begin{macrocode}
\def\ZREF@xr@scanparams#1#2#3#4#5#6#7\ZREF@nil{%
  \let#1\ltx@empty
  \ZREF@foundfalse
  \ZREF@xr@scantitleref#1#2\TR@TitleReference{}{}\ZREF@nil
  \ifZREF@found
  \else
    \ltx@LocalAppendToMacro#1{\default{#2}}%
  \fi
  % page
  \ltx@LocalAppendToMacro#1{\page{#3}}%
  % nameref title
  \ifZREF@found
  \else
    \ifx\\#4\\%
    \else
      \def\ZREF@xr@temp{#4}%
      \ifx\ZREF@xr@temp\ZREF@xr@relax
      \else
        \ltx@LocalAppendToMacro#1{\title{#4}}%
      \fi
    \fi
  \fi
  % anchor
  \ifx\\#5\\%
  \else
    \ltx@LocalAppendToMacro#1{\anchor{#5}}%
  \fi
  \ifx\\#6\\%
  \else
    \ifZREF@xr@urluse
      \ZREF@xr@AddURL{#6}%
      \expandafter\ltx@LocalAppendToMacro\expandafter#1%
      \expandafter{%
        \expandafter\urluse\expandafter{%
          \csname Z@U@\ZREF@xr@theURL\endcsname
        }%
      }%
    \else
      \ltx@LocalAppendToMacro#1{\url{#6}}%
    \fi
  \fi
}%
%    \end{macrocode}
%    \end{macro}
%
%    \begin{macro}{\ZREF@xr@scantitleref}
%    \begin{macrocode}
\def\ZREF@xr@scantitleref#1#2\TR@TitleReference#3#4#5\ZREF@nil{%
  \ifx\\#5\\%
  \else
    \ltx@LocalAppendToMacro#1{%
      \default{#3}%
      \title{#4}%
    }%
    \ZREF@foundtrue
  \fi
}%
%    \end{macrocode}
%    \end{macro}
%
%    \begin{macro}{\ZREF@xr@urlcheck}%
%    \begin{macrocode}
\def\ZREF@xr@urlcheck#1{%
  \zref@ifrefcontainsprop{#1}{anchor}{%
    \zref@ifrefcontainsprop{#1}{url}{%
    }{%
      \expandafter
      \ltx@LocalAppendToMacro\csname Z@R@#1\expandafter\endcsname
      \expandafter{%
        \csname url\ifZREF@xr@urluse use\fi
        \expandafter\endcsname\expandafter{\ZREF@xr@url}%
      }%
    }%
  }{%
  }%
}%
%    \end{macrocode}
%    \end{macro}
%
%    \begin{macrocode}
%</xr>
%    \end{macrocode}
%
%
% \subsection{Module \xmodule{hyperref}}
%
%    UNFINISHED :-(
%
%    \begin{macrocode}
%<*hyperref>
\NeedsTeXFormat{LaTeX2e}
\ProvidesPackage{zref-hyperref}%
  [2020-07-03 v2.32 Module hyperref for zref (HO)]%
\RequirePackage{zref-base}[2019/11/29]
\ifx\ZREF@base@ok Y%
\else
  \expandafter\endinput
\fi
%    \end{macrocode}
%
%    \begin{macrocode}
\ZREF@NewPropAnchor
\zref@addprop\ZREF@mainlist{anchor}%
%    \end{macrocode}
%
%    \begin{macrocode}
%</hyperref>
%    \end{macrocode}
%
%
% \subsection{Module \xmodule{savepos}}
%
%    Module \xmodule{savepos} provides an interface for
%    \pdfTeX's \cmd{\pdfsavepos}, see the manual for
%    \pdfTeX.
%
% \subsubsection{Identification}
%
%    \begin{macrocode}
%<*savepos>
\NeedsTeXFormat{LaTeX2e}
\ProvidesPackage{zref-savepos}%
  [2020-07-03 v2.32 Module savepos for zref (HO)]%
\RequirePackage{zref-base}[2019/11/29]
\ifx\ZREF@base@ok Y%
\else
  \expandafter\endinput
\fi
%    \end{macrocode}
%
% \subsubsection{Availability}
%
%    First we check, whether the feature is available.
%    \begin{macrocode}
\ifx\directlua\@undefined
\ltx@IfUndefined{pdfsavepos}{%
  \PackageError\ZREF@name{%
    \string\pdfsavepos\space is not supported.\MessageBreak
    It is provided by pdfTeX (1.40) or XeTeX%
  }\ZREF@UpdatePdfTeX
  \endinput
}{}%
\fi
%    \end{macrocode}
%    In PDF mode we are done. However support for DVI mode
%    was added later in version 1.40.0. In earlier versions
%    \cs{pdfsavepos} is defined, but its execution raises
%    an error. Note that \hologo{XeTeX} also provides \cs{pdfsavepos}.
%    \begin{macrocode}
\ifpdf
\else
  \ltx@IfUndefined{pdftexversion}{%
  }{%
    \ifnum\pdftexversion<140 %
      \PackageError\ZREF@name{%
        \string\pdfsavepos\space is not supported in DVI mode%
        \MessageBreak
        of this pdfTeX version%
      }\ZREF@UpdatePdfTeX
      \expandafter\expandafter\expandafter\endinput
    \fi
  }%
\fi
%    \end{macrocode}
%
% \subsubsection{Setup}
%
%    \begin{macrocode}
\zref@newlist{savepos}
\ifx\directlua\@undefined
  \zref@newprop*{posx}[0]{\the\pdflastxpos}
  \zref@newprop*{posy}[0]{\the\pdflastypos}
\else
  \zref@newprop*{posx}[0]{\the\lastxpos}
  \zref@newprop*{posy}[0]{\the\lastypos}
\fi
\zref@addprops{savepos}{posx,posy}
%    \end{macrocode}
%
% \subsubsection{User macros}
%
%    \begin{macro}{\zref@savepos}
%    \begin{macrocode}
\ifx\directlua\@undefined
  \def\zref@savepos{%
    \if@filesw
      \pdfsavepos
    \fi
  }
\else
  \def\zref@savepos{%
    \if@filesw
      \savepos
    \fi
  }
\fi
%    \end{macrocode}
%    \end{macro}
%
%    \begin{macro}{\ZREF@zsavepos}
%    \begin{macrocode}
\def\ZREF@zsavepos#1#2#3{%
  \@bsphack
  \if@filesw
    \zref@savepos
    #1{#3}{#2}%
    \ltx@IfUndefined{TeXXeTstate}{%
    }{%
      \ifnum\TeXXeTstate=\ltx@zero
      \else
        \zref@savepos
      \fi
    }%
  \fi
  \@esphack
}
%    \end{macrocode}
%    \end{macro}
%
%    \begin{macro}{\zsavepos}
%    The current location is stored in a reference with the
%    given name.
%    \begin{macrocode}
\ZREF@IfDefinable\zsavepos\def{%
  {%
    \ZREF@zsavepos\zref@labelbylist{savepos}%
  }%
}
%    \end{macrocode}
%    \end{macro}
%
%    \begin{macro}{\zsaveposx}
%    \begin{macrocode}
\ZREF@IfDefinable\zsaveposx\def{%
  {%
    \ZREF@zsavepos\zref@labelbyprops{posx}%
  }%
}
%    \end{macrocode}
%    \end{macro}
%    \begin{macro}{\zsaveposy}
%    \begin{macrocode}
\ZREF@IfDefinable\zsaveposy\def{%
  {%
    \ZREF@zsavepos\zref@labelbyprops{posy}%
  }%
}
%    \end{macrocode}
%    \end{macro}
%
%    \begin{macro}{\zposx}
%    \begin{macro}{\zposy}
%    The horizontal and vertical position are available by
%    \cs{zposx} and \cs{zposy}. Do not rely on absolute
%    positions. They differ in DVI and PDF mode of \pdfTeX.
%    Use differences instead. The unit of the position numbers
%    is sp.
%    \begin{macrocode}
\newcommand*{\zposx}[1]{%
  \zref@extract{#1}{posx}%
}%
\newcommand*{\zposy}[1]{%
  \zref@extract{#1}{posy}%
}%
%    \end{macrocode}
%    \end{macro}
%    \end{macro}
%
%    Typically horizontal and vertical positions are used
%    inside calculations. Therefore the extracting macros
%    should be expandable and babel's patch is not applyable.
%
%    Also it is in the responsibility of the user to marked
%    used positions by \cs{zrefused}
%    in order to notify \hologo{LaTeX} about undefined references.
%
%    \begin{macro}{\ZREF@savepos@ok}
%    \begin{macrocode}
\let\ZREF@savepos@ok=Y
%    \end{macrocode}
%    \end{macro}
%
%    \begin{macrocode}
%</savepos>
%    \end{macrocode}
%
%
% \subsection{Module \xmodule{abspos}}
%
% \subsubsection{Identification}
%
%    \begin{macrocode}
%<*abspos>
\NeedsTeXFormat{LaTeX2e}
\ProvidesPackage{zref-abspos}%
  [2020-07-03 v2.32 Module abspos for zref (HO)]%
\RequirePackage{zref-base}[2019/11/29]
\ifx\ZREF@base@ok Y%
\else
  \expandafter\endinput
\fi
%    \end{macrocode}
%
%    \begin{macrocode}
\RequirePackage{zref-savepos}[2019/11/29]
\ifx\ZREF@savepos@ok Y%
\else
  \expandafter\endinput
\fi
%    \end{macrocode}
%    \begin{macrocode}
\RequirePackage{zref-pagelayout}[2019/11/29]
\zref@addprop{savepos}{abspage}
\zref@addprop{savepos}{pagevalue}
%    \end{macrocode}
%
%    \begin{macro}{\zref@absposx}
%    \begin{macrocode}
\newcommand*{\zref@absposx}[3]{%
  \number
    \expandafter\zref@absposnumx\expandafter{%
      \number\zref@extractdefault{#1}{abspage}{0}%
    }{#2}{#3}%
  \ltx@space
}
%    \end{macrocode}
%    \end{macro}
%
%    \begin{macro}{\zref@absposy}
%    \begin{macrocode}
\newcommand*{\zref@absposy}[3]{%
  \number
  \expandafter\zref@absposnumy\expandafter{%
    \number\zref@extractdefault{#1}{abspage}{0}%
  }{#2}{#3}%
  \ltx@space
}
%    \end{macrocode}
%    \end{macro}
%
%    \begin{macro}{\zref@absposnumx}
%    \begin{macrocode}
\newcommand*{\zref@absposnumx}[3]{%
  \number
%  \ifnum#1>\ltx@zero
%    \zref@ifrefundefined{thepage#1}{%
%      0%
%    }{%
      \numexpr\ZREF@absposnum{thepage#1}{#2}{x}{#3}\relax
%    }%
%  \else
%    0%
%  \fi
}
%    \end{macrocode}
%    \end{macro}
%
%    \begin{macro}{\zref@absposnumy}
%    \begin{macrocode}
\newcommand*{\zref@absposnumy}[3]{%
  \number
%  \ifnum#1>\ltx@zero
%    \zref@ifrefundefined{thepage#1}{%
%      0%
%    }{%
      \numexpr\ZREF@absposnum{thepage#1}{#2}{y}{#3}\relax
%    }%
%  \else
%    0%
%  \fi
}
%    \end{macrocode}
%    \end{macro}
%
%    \begin{macro}{\ZREF@absposnum}
%    \begin{macrocode}
\def\ZREF@absposnum#1#2#3#4{%
  \ltx@ifundefined{ZREF@abspos@#2@#3@#4}{%
    0%
  }{%
    \csname ZREF@abspos@#2@#3@#4\endcsname{#1}%
  }%
}
%    \end{macrocode}
%    \end{macro}
%
%    \begin{macro}{\zref@def@absposx}
%    \begin{macrocode}
\ZREF@Robust\def\zref@def@absposx#1{%
  \zref@wrapper@babel{\ZREF@def@abspos{#1}\zref@absposx}%
}
%    \end{macrocode}
%    \end{macro}
%
%    \begin{macro}{\zref@def@absposy}
%    \begin{macrocode}
\ZREF@Robust\def\zref@def@absposy#1{%
  \zref@wrapper@babel{\ZREF@def@abspos{#1}\zref@absposy}%
}
%    \end{macrocode}
%    \end{macro}
%
%    \begin{macro}{\zref@def@absposnumx}
%    \begin{macrocode}
\ZREF@Robust\def\zref@def@absposnumx#1{%
  \zref@wrapper@babel{\ZREF@def@abspos{#1}\zref@absposnumx}%
}
%    \end{macrocode}
%    \end{macro}
%
%    \begin{macro}{\zref@def@absposnumy}
%    \begin{macrocode}
\ZREF@Robust\def\zref@def@absposnumy#1{%
  \zref@wrapper@babel{\ZREF@def@abspos{#1}\zref@absposnumy}%
}
%    \end{macrocode}
%    \end{macro}
%    \begin{macro}{\ZREF@def@abspos}
%    \begin{macrocode}
\def\ZREF@def@absposnumy#1#2#3#4#5{%
  \edef#1{#2{#3}{#4}{#5}}%
}
%    \end{macrocode}
%    \end{macro}
%
%    \begin{macro}{\zref@absposused}
%    \begin{macrocode}
\ZREF@Robust\def\zref@absposused{%
  \zref@wrapper@babel\ZREF@abspos@used
}
%    \end{macrocode}
%    \end{macro}
%    \begin{macro}{\ZREF@abspos@used}
%    \begin{macrocode}
\def\ZREF@abspos@used#1{%
  \zref@refused{#1}%
  \zref@ifrefundefined{#1}{%
  }{%
    \begingroup
      \edef\ZREF@temp{%
        \zref@extractdefault{#1}{abspage}{0}%
      }%
      \ifnum\ZREF@temp>\ltx@zero
        \zref@refused{thepage\ZREF@temp}%
      \else
        \@PackageError{zref-abspos}{%
          \string\zref@pos@label@used\ltx@space
          needs property `abspage'\MessageBreak
          in label `#1'%
        }\@ehc
      \fi
    \endgroup
  }%
}
%    \end{macrocode}
%    \end{macro}
%
%    \begin{macro}{\zref@absposnumused}
%    \begin{macrocode}
\newcommand*{\zref@absposnumused}[1]{%
  \ifnum#1>\ltx@zero
    \zref@refused{thepage\number#1}%
  \else
    \@PackageError{zref-abspos}{%
      Invalid absolute page number (#1)\MessageBreak
      for \string\zref@pos@num@used.\MessageBreak
      A positive integer is expected%
    }\@ehc
  \fi
}
%    \end{macrocode}
%    \end{macro}
%
%    \begin{macro}{\zref@ifabsposundefined}
%    \begin{macrocode}
\def\zref@ifabsposundefined#1{%
  \zref@ifrefundefined{#1}\ltx@firsttwo{%
    \expandafter\zref@ifabsposnumundefined\expandafter{%
      \number\zref@extractdefault{#1}{abspage}{0}%
    }%
  }%
}
%    \end{macrocode}
%    \end{macro}
%
%    \begin{macro}{\zref@ifabsposnumundefined}
%    \begin{macrocode}
\def\zref@ifabsposnumundefined#1{%
  \ifnum\ZREF@number{#1}>\ltx@zero
    \zref@ifrefundefined{thepage#1}%
        \ltx@firstoftwo\ltx@secondoftwo
  \else
    \expandafter\ltx@firstoftwo
  \fi
}
%    \end{macrocode}
%    \end{macro}
%
% \subsubsection{Media}
%
%    \begin{macro}{\ZREF@abspos@media@width}
%    \begin{macrocode}
\edef\ZREF@abspos@media@width{%
  \ltx@ifundefined{pdfpagewidth}{%
    \ltx@ifundefined{mediawidth}{%
      \ltx@ifundefined{stockwidth}{%
        paperwidth%
      }{%
        stockwidth%
      }%
    }{%
      mediawidth%
    }%
  }{%
    pdfpagewidth%
  }%
}
\ifluatex
\def\ZREF@abspos@media@width{pdfpagewidth}%
\fi
%    \end{macrocode}
%    \end{macro}
%    \begin{macro}{\ZREF@abspos@media@height}
%    \begin{macrocode}
\edef\ZREF@abspos@media@height{%
  \ltx@ifundefined{pdfpageheight}{%
    \ltx@ifundefined{mediaheight}{%
      \ltx@ifundefined{stockheight}{%
        paperheight%
      }{%
        stockheight%
      }%
    }{%
      mediaheight%
    }%
  }{%
    \noexpand\ifcase\pdfpageheight
      \ltx@ifundefined{stockheight}{%
        paperheight%
      }{%
        stockheight%
      }%
    \noexpand\else
      pdfpageheight%
    \noexpand\fi
  }%
}
\ifluatex
\edef\ZREF@abspos@media@height{%
   \noexpand\ifcase\pageheight
      \ltx@ifundefined{stockheight}{%
        paperheight%
      }{%
        stockheight%
      }%
    \noexpand\else
      pdfpageheight%
    \noexpand\fi}%
\fi
%    \end{macrocode}
%    \end{macro}
%
%    \begin{macro}{\ZREF@abspos@media@x@left}
%    \begin{macrocode}
\def\ZREF@abspos@media@x@left#1{%
  0%
}
%    \end{macrocode}
%    \end{macro}
%    \begin{macro}{\ZREF@abspos@media@x@right}
%    \begin{macrocode}
\def\ZREF@abspos@media@x@right#1{%
  \zref@extract{#1}\ZREF@abspos@media@width
}
%    \end{macrocode}
%    \end{macro}
%    \begin{macro}{\ZREF@abspos@media@x@center}
%    \begin{macrocode}
\def\ZREF@abspos@media@x@center#1{%
  \ZREF@abspos@media@x@left{#1}%
  +\zref@extract{#1}\ZREF@abspos@media@width/2%
}
%    \end{macrocode}
%    \end{macro}
%
%    \begin{macro}{\ZREF@abspos@media@y@top}
%    \begin{macrocode}
\def\ZREF@abspos@media@y@top#1{%
  \zref@extract{#1}\ZREF@abspos@media@height
}
%    \end{macrocode}
%    \end{macro}
%    \begin{macro}{\ZREF@abspos@media@y@bottom}
%    \begin{macrocode}
\def\ZREF@abspos@media@y@bottom#1{%
  0%
}
%    \end{macrocode}
%    \end{macro}
%    \begin{macro}{\ZREF@abspos@media@y@center}
%    \begin{macrocode}
\def\ZREF@abspos@media@y@center#1{%
  \zref@extract{#1}\ZREF@abspos@media@height/2%
}
%    \end{macrocode}
%    \end{macro}
%
% \subsubsection{Paper}
%
%    \begin{macro}{\ZREF@abspos@paper@x@left}
%    \begin{macrocode}
\def\ZREF@abspos@paper@x@left#1{%
  0%
}
%    \end{macrocode}
%    \end{macro}
%    \begin{macro}{\ZREF@abspos@paper@x@right}
%    \begin{macrocode}
\def\ZREF@abspos@paper@x@right#1{%
  \zref@extract{#1}{paperwidth}%
}
%    \end{macrocode}
%    \end{macro}
%    \begin{macro}{\ZREF@abspos@paper@x@center}
%    \begin{macrocode}
\def\ZREF@abspos@paper@x@center#1{%
  \zref@extract{#1}{paperwidth}/2%
}
%    \end{macrocode}
%    \end{macro}
%
%    \begin{macro}{\ZREF@abspos@paper@y@top}
%    \begin{macrocode}
\let\ZREF@abspos@paper@y@top\ZREF@abspos@media@y@top
%    \end{macrocode}
%    \end{macro}
%    \begin{macro}{\ZREF@abspos@paper@y@bottom}
%    \begin{macrocode}
\def\ZREF@abspos@paper@y@bottom#1{%
  \ZREF@abspos@paper@y@top{#1}%
  -\zref@extract{#1}{paperheight}%
}
%    \end{macrocode}
%    \end{macro}
%    \begin{macro}{\ZREF@abspos@paper@y@center}
%    \begin{macrocode}
\def\ZREF@abspos@paper@y@center#1{%
  \ZREF@abspos@paper@y@top{#1}%
  -\zref@extract{#1}{paperheight}/2%
}
%    \end{macrocode}
%    \end{macro}
%
% \subsubsection{Origin}
%
% There doesn't seem a good reason to make these tests depend on pdf mode
%  in current engines, so comment out the \verb|\ifpdf| tests.
%    \begin{macro}{\ZREF@abspos@origin@x}
%    \begin{macrocode}
\let\ZREF@temp\ltx@two
\ltx@IfUndefined{pdfhorigin}{}{%
%  \ifpdf
    \let\ZREF@temp\ltx@zero
%  \fi
}
\ifluatex
% \ifpdf
  \let\ZREF@temp\ltx@zero
% \fi
\fi

\ifx\ZREF@temp\ltx@two
  \ifnum\mag=1000 %
    \let\ZREF@temp\ltx@one
  \fi
\fi
\ifcase\ZREF@temp
  \def\ZREF@abspos@origin@x#1{%
    \zref@extract{#1}{pdfhorigin}%
  }%
\or
  \def\ZREF@abspos@origin@x#1{%
    4736286%
  }%
\or
  \def\ZREF@abspos@origin@x#1{%
    \numexpr\mag/1000*\dimexpr 1truein\relax\relax
  }%
\fi
%    \end{macrocode}
%    \end{macro}
%    \begin{macro}{\ZREF@abspos@origin@y}
%    \begin{macrocode}
\let\ZREF@temp\ltx@two
\ltx@IfUndefined{pdfvorigin}{}{%
%  \ifpdf
    \let\ZREF@temp\ltx@zero
%  \fi
}
\ifluatex
%  \ifpdf
    \let\ZREF@temp\ltx@zero
%  \fi
\fi
\ifx\ZREF@temp\ltx@two
  \ifnum\mag=1000 %
    \let\ZREF@temp\ltx@one
  \fi
\fi
\ifcase\ZREF@temp
  \def\ZREF@abspos@origin@y#1{%
    \zref@extract{#1}{pdfvorigin}%
  }%
\or
  \def\ZREF@abspos@origin@y#1{%
    4736286%
  }%
\or
  \def\ZREF@abspos@origin@y#1{%
    \numexpr\mag/1000*\dimexpr 1truein\relax\relax
   }%
\fi
%    \end{macrocode}
%    \end{macro}
%
% \subsubsection{Header}
%
%    \begin{macro}{\ZREF@abspos@head@x@left}
%    \begin{macrocode}
\def\ZREF@abspos@head@x@left#1{%
  \ZREF@abspos@paper@x@left{#1}%
  +\ZREF@abspos@origin@x{#1}%
  +\zref@extract{#1}{hoffset}%
  +\ifodd\zref@extractdefault{#1}{pagevalue}{\number\c@page} %
     \zref@extract{#1}{oddsidemargin}%
   \else
     \zref@extract{#1}{evensidemargin}%
   \fi
}
%    \end{macrocode}
%    \end{macro}
%    \begin{macro}{\ZREF@abspos@head@x@right}
%    \begin{macrocode}
\def\ZREF@abspos@head@x@right#1{%
  \ZREF@abspos@head@x@left{#1}%
  +\zref@extract{#1}{textwidth}%
}
%    \end{macrocode}
%    \end{macro}
%    \begin{macro}{\ZREF@abspos@head@x@center}
%    \begin{macrocode}
\def\ZREF@abspos@head@x@center#1{%
  \ZREF@abspos@head@x@left{#1}%
  +\zref@extract{#1}{textwidth}/2%
}
%    \end{macrocode}
%    \end{macro}
%
%    \begin{macro}{\ZREF@abspos@head@y@top}
%    \begin{macrocode}
\def\ZREF@abspos@head@y@top#1{%
  \ZREF@abspos@paper@y@top{#1}%
  -\ZREF@abspos@origin@y{#1}%
  -\zref@extract{#1}{voffset}%
  -\zref@extract{#1}{topmargin}%
}
%    \end{macrocode}
%    \end{macro}
%    \begin{macro}{\ZREF@abspos@head@y@bottom}
%    \begin{macrocode}
\def\ZREF@abspos@head@y@bottom#1{%
  \ZREF@abspos@head@y@top{#1}%
  -\zref@extract{#1}{headheight}%
}
%    \end{macrocode}
%    \end{macro}
%    \begin{macro}{\ZREF@abspos@head@y@center}
%    \begin{macrocode}
\def\ZREF@abspos@head@y@center#1{%
  \ZREF@abspos@head@y@top{#1}%
  -\zref@extract{#1}{headheight}/2%
}
%    \end{macrocode}
%    \end{macro}
%
% \subsubsection{Body}
%
%    \begin{macro}{\ZREF@abspos@body@x@left}
%    \begin{macrocode}
\let\ZREF@abspos@body@x@left\ZREF@abspos@head@x@left
%    \end{macrocode}
%    \end{macro}
%    \begin{macro}{\ZREF@abspos@body@x@right}
%    \begin{macrocode}
\let\ZREF@abspos@body@x@right\ZREF@abspos@head@x@right
%    \end{macrocode}
%    \end{macro}
%    \begin{macro}{\ZREF@abspos@body@x@center}
%    \begin{macrocode}
\let\ZREF@abspos@body@x@center\ZREF@abspos@head@x@center
%    \end{macrocode}
%    \end{macro}
%
%    \begin{macro}{\ZREF@abspos@body@y@top}
%    \begin{macrocode}
\def\ZREF@abspos@body@y@top#1{%
  \ZREF@abspos@head@y@bottom{#1}%
  -\zref@extract{#1}{headsep}%
}
%    \end{macrocode}
%    \end{macro}
%    \begin{macro}{\ZREF@abspos@body@y@bottom}
%    \begin{macrocode}
\def\ZREF@abspos@body@y@bottom#1{%
  \ZREF@abspos@body@y@top{#1}%
  -\zref@extract{#1}{textheight}%
}
%    \end{macrocode}
%    \end{macro}
%    \begin{macro}{\ZREF@abspos@body@y@center}
%    \begin{macrocode}
\def\ZREF@abspos@body@y@center#1{%
  \ZREF@abspos@body@y@top{#1}%
  -\zref@extract{#1}{textheight}/2%
}
%    \end{macrocode}
%    \end{macro}
%
% \subsubsection{Footer}
%
%    \begin{macro}{\ZREF@abspos@foot@x@left}
%    \begin{macrocode}
\let\ZREF@abspos@foot@x@left\ZREF@abspos@head@x@left
%    \end{macrocode}
%    \end{macro}
%    \begin{macro}{\ZREF@abspos@foot@x@right}
%    \begin{macrocode}
\let\ZREF@abspos@foot@x@right\ZREF@abspos@head@x@right
%    \end{macrocode}
%    \end{macro}
%    \begin{macro}{\ZREF@abspos@foot@x@center}
%    \begin{macrocode}
\let\ZREF@abspos@foot@x@center\ZREF@abspos@head@x@center
%    \end{macrocode}
%    \end{macro}
%
%    \begin{macro}{\ZREF@abspos@foot@y@bottom}
%    \begin{macrocode}
\def\ZREF@abspos@foot@y@bottom#1{%
  \ZREF@abspos@body@y@bottom{#1}%
  -\zref@extract{#1}{footskip}%
}
%    \end{macrocode}
%    \end{macro}
%
% \subsubsection{Marginal notes}
%
%    \begin{macro}{\ZREF@abspos@marginpar@x@left}
%    \begin{macrocode}
\def\ZREF@abspos@marginpar@x@left#1{%
  \ifodd\zref@extractdefault{#1}{pagevalue}{\number\c@page} %
    \ZREF@abspos@body@x@right{#1}%
    +\zref@extract{#1}{marginparsep}%
  \else
    \ZREF@abspos@body@x@left{#1}%
    -\zref@extract{#1}{marginparsep}%
    -\zref@extract{#1}{marginparwidth}%
  \fi
}
%    \end{macrocode}
%    \end{macro}
%    \begin{macro}{\ZREF@abspos@marginpar@x@right}
%    \begin{macrocode}
\def\ZREF@abspos@marginpar@x@right#1{%
  \ZREF@abspos@marginpar@x@left{#1}%
  +\zref@extract{#1}{marginparwidth}%
}
%    \end{macrocode}
%    \end{macro}
%    \begin{macro}{\ZREF@abspos@marginpar@x@center}
%    \begin{macrocode}
\def\ZREF@abspos@marginpar@x@center#1{%
  \ZREF@abspos@marginpar@x@left{#1}%
  +\zref@extract{#1}{marginparwidth}/2%
}
%    \end{macrocode}
%    \end{macro}
%
%    \begin{macro}{\ZREF@abspos@marginpar@y@top}
%    \begin{macrocode}
\let\ZREF@abspos@marginpar@y@top\ZREF@abspos@body@y@top
%    \end{macrocode}
%    \end{macro}
%    \begin{macro}{\ZREF@abspos@marginpar@y@bottom}
%    \begin{macrocode}
\let\ZREF@abspos@marginpar@y@bottom\ZREF@abspos@body@y@bottom
%    \end{macrocode}
%    \end{macro}
%    \begin{macro}{\ZREF@abspos@marginpar@y@center}
%    \begin{macrocode}
\let\ZREF@abspos@marginpar@y@center\ZREF@abspos@body@y@center
%    \end{macrocode}
%    \end{macro}
%
% \subsubsection{Stock paper}
%
%    \begin{macro}{\ZREF@abspos@stock@x@left}
%    \begin{macrocode}
\let\ZREF@abspos@stock@x@left\ZREF@abspos@paper@x@left
%    \end{macrocode}
%    \end{macro}
%    \begin{macro}{\ZREF@abspos@stock@x@right}
%    \begin{macrocode}
\let\ZREF@abspos@stock@x@right\ZREF@abspos@paper@x@right
%    \end{macrocode}
%    \end{macro}
%    \begin{macro}{\ZREF@abspos@stock@x@center}
%    \begin{macrocode}
\let\ZREF@abspos@stock@x@center\ZREF@abspos@paper@x@center
%    \end{macrocode}
%    \end{macro}
%
%    \begin{macro}{\ZREF@abspos@stock@y@top}
%    \begin{macrocode}
\let\ZREF@abspos@stock@y@top\ZREF@abspos@paper@y@top
%    \end{macrocode}
%    \end{macro}
%    \begin{macro}{\ZREF@abspos@stock@y@bottom}
%    \begin{macrocode}
\let\ZREF@abspos@stock@y@bottom\ZREF@abspos@paper@y@bottom
%    \end{macrocode}
%    \end{macro}
%    \begin{macro}{\ZREF@abspos@stock@y@center}
%    \begin{macrocode}
\let\ZREF@abspos@stock@y@center\ZREF@abspos@paper@y@center
%    \end{macrocode}
%    \end{macro}
%
%    \begin{macrocode}
%</abspos>
%    \end{macrocode}
%
%
% \subsection{Module \xmodule{dotfill}}
%
%    \begin{macrocode}
%<*dotfill>
\NeedsTeXFormat{LaTeX2e}
\ProvidesPackage{zref-dotfill}%
  [2020-07-03 v2.32 Module dotfill for zref (HO)]%
\RequirePackage{zref-base}[2019/11/29]
\ifx\ZREF@base@ok Y%
\else
  \expandafter\endinput
\fi
%    \end{macrocode}
%
%    For measuring the width of \cs{zdotfill} we use the features
%    provided by module \xmodule{savepos}.
%    \begin{macrocode}
\RequirePackage{zref-savepos}[2019/11/29]
%    \end{macrocode}
%    For automatically generated label names we use the unique
%    counter of module \xmodule{base}.
%    \begin{macrocode}
\zref@require@unique
%    \end{macrocode}
%    Configuration is done by the key value interface of package
%    \xpackage{keyval}.
%    \begin{macrocode}
\RequirePackage{keyval}
%    \end{macrocode}
%    The definitions of the keys follow.
%    \begin{macrocode}
\define@key{ZREF@DF}{unit}{%
  \def\ZREF@df@unit{#1}%
}
\define@key{ZREF@DF}{min}{%
  \def\ZREF@df@min{#1}%
}
\define@key{ZREF@DF}{dot}{%
  \def\ZREF@df@dot{#1}%
}
%    \end{macrocode}
%    Defaults are set, see user interface.
%    \begin{macrocode}
\providecommand\ZREF@df@min{2}
\providecommand\ZREF@df@unit{.44em}
\providecommand\ZREF@df@dot{.}
%    \end{macrocode}
%    \begin{macro}{\zdotfillsetup}
%    Configuration of \cs{zdotfill} is done by \cs{zdotfillsetup}.
%    \begin{macrocode}
\newcommand*{\zdotfillsetup}{\kvsetkeys{ZREF@DF}}
%    \end{macrocode}
%    \end{macro}
%
%    \begin{macro}{\zdotfill}
%    \cs{zdotfill} sets labels at the left and the right to get
%    the horizontal position. \cs{zsavepos} is not used, because
%    we do not need the vertical position.
%    \begin{macrocode}
\ZREF@IfDefinable\zdotfill\def{%
  {%
    \leavevmode
    \global\advance\c@zref@unique\ltx@one
    \begingroup
      \def\ZREF@temp{zref@\number\c@zref@unique}%
      \pdfsavepos
      \zref@labelbyprops{\thezref@unique L}{posx}%
      \setlength{\dimen@}{\ZREF@df@unit}%
      \zref@ifrefundefined{\thezref@unique R}{%
        \ZREF@dotfill
      }{%
        \ifnum\numexpr\zposx{\thezref@unique R}%
                      -\zposx{\thezref@unique L}\relax
            <\dimexpr\ZREF@df@min\dimen@\relax
          \hfill
        \else
          \ZREF@dotfill
        \fi
      }%
      \pdfsavepos
      \zref@labelbyprops{\thezref@unique R}{posx}%
    \endgroup
    \kern\z@
  }%
}
%    \end{macrocode}
%    \end{macro}
%    \begin{macro}{\ZREF@dotfill}
%    Help macro that actually sets the dots.
%    \begin{macrocode}
\def\ZREF@dotfill{%
  \cleaders\hb@xt@\dimen@{\hss\ZREF@df@dot\hss}\hfill
}
%    \end{macrocode}
%    \end{macro}
%
%    \begin{macrocode}
%</dotfill>
%    \end{macrocode}
%
% \subsection{Module \xmodule{env}}
%
%    \begin{macrocode}
%<*env>
\NeedsTeXFormat{LaTeX2e}
\ProvidesPackage{zref-env}%
  [2020-07-03 v2.32 Module env for zref (HO)]%
\RequirePackage{zref-base}[2019/11/29]
\ifx\ZREF@base@ok Y%
\else
  \expandafter\endinput
\fi
%    \end{macrocode}
%
%    \begin{macrocode}
\zref@newprop{envname}[]{\@currenvir}
\zref@newprop{envline}[]{\zref@env@line}
%    \end{macrocode}
%
%    \begin{macro}{\zref@env@line}
%    Macro \cs{zref@env@line} extracts the line number
%    from \cs{@currenvline}.
%    \begin{macrocode}
\def\zref@env@line{%
  \ifx\@currenvline\ltx@empty
  \else
    \expandafter
    \ZREF@ENV@line\@currenvline\ltx@empty line \ltx@empty\@nil
  \fi
}
%    \end{macrocode}
%    \end{macro}
%    \begin{macro}{\ZREF@ENV@line}
%    \begin{macrocode}
\def\ZREF@ENV@line#1line #2\ltx@empty#3\@nil{#2}%
%    \end{macrocode}
%    \end{macro}
%
%    \begin{macrocode}
%</env>
%    \end{macrocode}
%
%% \section{Installation}
%
% \subsection{Download}
%
% \paragraph{Package.} This package is available on
% CTAN\footnote{\CTANpkg{zref}}:
% \begin{description}
% \item[\CTAN{macros/latex/contrib/zref/zref.dtx}] The source file.
% \item[\CTAN{macros/latex/contrib/zref/zref.pdf}] Documentation.
% \end{description}
%
%
% \paragraph{Bundle.} All the packages of the bundle `zref'
% are also available in a TDS compliant ZIP archive. There
% the packages are already unpacked and the documentation files
% are generated. The files and directories obey the TDS standard.
% \begin{description}
% \item[\CTANinstall{install/macros/latex/contrib/zref.tds.zip}]
% \end{description}
% \emph{TDS} refers to the standard ``A Directory Structure
% for \TeX\ Files'' (\CTANpkg{tds}). Directories
% with \xfile{texmf} in their name are usually organized this way.
%
% \subsection{Bundle installation}
%
% \paragraph{Unpacking.} Unpack the \xfile{zref.tds.zip} in the
% TDS tree (also known as \xfile{texmf} tree) of your choice.
% Example (linux):
% \begin{quote}
%   |unzip zref.tds.zip -d ~/texmf|
% \end{quote}
%
%
% \subsection{Package installation}
%
% \paragraph{Unpacking.} The \xfile{.dtx} file is a self-extracting
% \docstrip\ archive. The files are extracted by running the
% \xfile{.dtx} through \plainTeX:
% \begin{quote}
%   \verb|tex zref.dtx|
% \end{quote}
%
% \paragraph{TDS.} Now the different files must be moved into
% the different directories in your installation TDS tree
% (also known as \xfile{texmf} tree):
% \begin{quote}
% \def\t{^^A
% \begin{tabular}{@{}>{\ttfamily}l@{ $\rightarrow$ }>{\ttfamily}l@{}}
%   zref.sty & tex/latex/zref/zref.sty\\
%   zref-base.sty & tex/latex/zref/zref-base.sty\\
%   zref-abspage.sty & tex/latex/zref/zref-abspage.sty\\
%   zref-abspos.sty & tex/latex/zref/zref-abspos.sty\\
%   zref-counter.sty & tex/latex/zref/zref-counter.sty\\
%   zref-dotfill.sty & tex/latex/zref/zref-dotfill.sty\\
%   zref-env.sty & tex/latex/zref/zref-env.sty\\
%   zref-hyperref.sty & tex/latex/zref/zref-hyperref.sty\\
%   zref-lastpage.sty & tex/latex/zref/zref-lastpage.sty\\
%   zref-marks.sty & tex/latex/zref/zref-marks.sty\\
%   zref-nextpage.sty & tex/latex/zref/zref-nextpage.sty\\
%   zref-pageattr.sty & tex/latex/zref/zref-pageattr.sty\\
%   zref-pagelayout.sty & tex/latex/zref/zref-pagelayout.sty\\
%   zref-perpage.sty & tex/latex/zref/zref-perpage.sty\\
%   zref-runs.sty & tex/latex/zref/zref-runs.sty\\
%   zref-savepos.sty & tex/latex/zref/zref-savepos.sty\\
%   zref-thepage.sty & tex/latex/zref/zref-thepage.sty\\
%   zref-titleref.sty & tex/latex/zref/zref-titleref.sty\\
%   zref-totpages.sty & tex/latex/zref/zref-totpages.sty\\
%   zref-user.sty & tex/latex/zref/zref-user.sty\\
%   zref-xr.sty & tex/latex/zref/zref-xr.sty\\
%   zref.pdf & doc/latex/zref/zref.pdf\\
%   zref-example.tex & doc/latex/zref/zref-example.tex\\
%   zref-example-lastpage.tex & doc/latex/zref/zref-example-lastpage.tex\\
%   zref-example-nextpage.tex & doc/latex/zref/zref-example-nextpage.tex\\
%   zref.dtx & source/latex/zref/zref.dtx\\
% \end{tabular}^^A
% }^^A
% \sbox0{\t}^^A
% \ifdim\wd0>\linewidth
%   \begingroup
%     \advance\linewidth by\leftmargin
%     \advance\linewidth by\rightmargin
%   \edef\x{\endgroup
%     \def\noexpand\lw{\the\linewidth}^^A
%   }\x
%   \def\lwbox{^^A
%     \leavevmode
%     \hbox to \linewidth{^^A
%       \kern-\leftmargin\relax
%       \hss
%       \usebox0
%       \hss
%       \kern-\rightmargin\relax
%     }^^A
%   }^^A
%   \ifdim\wd0>\lw
%     \sbox0{\small\t}^^A
%     \ifdim\wd0>\linewidth
%       \ifdim\wd0>\lw
%         \sbox0{\footnotesize\t}^^A
%         \ifdim\wd0>\linewidth
%           \ifdim\wd0>\lw
%             \sbox0{\scriptsize\t}^^A
%             \ifdim\wd0>\linewidth
%               \ifdim\wd0>\lw
%                 \sbox0{\tiny\t}^^A
%                 \ifdim\wd0>\linewidth
%                   \lwbox
%                 \else
%                   \usebox0
%                 \fi
%               \else
%                 \lwbox
%               \fi
%             \else
%               \usebox0
%             \fi
%           \else
%             \lwbox
%           \fi
%         \else
%           \usebox0
%         \fi
%       \else
%         \lwbox
%       \fi
%     \else
%       \usebox0
%     \fi
%   \else
%     \lwbox
%   \fi
% \else
%   \usebox0
% \fi
% \end{quote}
% If you have a \xfile{docstrip.cfg} that configures and enables \docstrip's
% TDS installing feature, then some files can already be in the right
% place, see the documentation of \docstrip.
%
% \subsection{Refresh file name databases}
%
% If your \TeX~distribution
% (\TeX\,Live, \mikTeX, \dots) relies on file name databases, you must refresh
% these. For example, \TeX\,Live\ users run \verb|texhash| or
% \verb|mktexlsr|.
%
% \subsection{Some details for the interested}
%
% \paragraph{Unpacking with \LaTeX.}
% The \xfile{.dtx} chooses its action depending on the format:
% \begin{description}
% \item[\plainTeX:] Run \docstrip\ and extract the files.
% \item[\LaTeX:] Generate the documentation.
% \end{description}
% If you insist on using \LaTeX\ for \docstrip\ (really,
% \docstrip\ does not need \LaTeX), then inform the autodetect routine
% about your intention:
% \begin{quote}
%   \verb|latex \let\install=y% \iffalse meta-comment
%
% File: zref.dtx
% Version: 2020-07-03 v2.32
% Info: A new reference scheme for LaTeX
%
% Copyright (C)
%    2006-2012  Heiko Oberdiek
%    2016-2020  Oberdiek Package Support Group
%    https://github.com/ho-tex/zref/issues
%
% This work may be distributed and/or modified under the
% conditions of the LaTeX Project Public License, either
% version 1.3c of this license or (at your option) any later
% version. This version of this license is in
%    https://www.latex-project.org/lppl/lppl-1-3c.txt
% and the latest version of this license is in
%    https://www.latex-project.org/lppl.txt
% and version 1.3 or later is part of all distributions of
% LaTeX version 2005/12/01 or later.
%
% This work has the LPPL maintenance status "maintained".
%
% The Current Maintainers of this work are
% Heiko Oberdiek and the Oberdiek Package Support Group
% https://github.com/ho-tex/zref/issues
%
% This work consists of the main source file zref.dtx
% and the derived files
%    zref.sty, zref.pdf, zref.ins, zref.drv, zref-base.sty,
%    zref-abspage.sty, zref-abspos.sty, zref-counter.sty,
%    zref-dotfill.sty, zref-env.sty, zref-hyperref.sty,
%    zref-lastpage.sty, zref-marks.sty, zref-nextpage.sty,
%    zref-pageattr.sty, zref-pagelayout.sty, zref-perpage.sty,
%    zref-runs.sty, zref-savepos.sty, zref-thepage.sty,
%    zref-titleref.sty, zref-totpages.sty, zref-user.sty,
%    zref-xr.sty, zref-example.tex, zref-example-lastpage.tex,
%    zref-example-nextpage.tex, zref-test1.tex,
%    zref-test-base.tex, zref-test-runs.tex,
%    zref-test-titleref-memoir.tex.
%
% Distribution:
%    CTAN:macros/latex/contrib/zref/zref.dtx
%    CTAN:macros/latex/contrib/zref/zref.pdf
%
% Unpacking:
%    (a) If zref.ins is present:
%           tex zref.ins
%    (b) Without zref.ins:
%           tex zref.dtx
%    (c) If you insist on using LaTeX
%           latex \let\install=y\input{zref.dtx}
%        (quote the arguments according to the demands of your shell)
%
% Documentation:
%    (a) If zref.drv is present:
%           latex zref.drv
%    (b) Without zref.drv:
%           latex zref.dtx; ...
%    The class ltxdoc loads the configuration file ltxdoc.cfg
%    if available. Here you can specify further options, e.g.
%    use A4 as paper format:
%       \PassOptionsToClass{a4paper}{article}
%
%    Programm calls to get the documentation (example):
%       pdflatex zref.dtx
%       makeindex -s gind.ist zref.idx
%       pdflatex zref.dtx
%       makeindex -s gind.ist zref.idx
%       pdflatex zref.dtx
%
% Installation:
%    TDS:tex/latex/zref/zref.sty
%    TDS:tex/latex/zref/zref-base.sty
%    TDS:tex/latex/zref/zref-abspage.sty
%    TDS:tex/latex/zref/zref-abspos.sty
%    TDS:tex/latex/zref/zref-counter.sty
%    TDS:tex/latex/zref/zref-dotfill.sty
%    TDS:tex/latex/zref/zref-env.sty
%    TDS:tex/latex/zref/zref-hyperref.sty
%    TDS:tex/latex/zref/zref-lastpage.sty
%    TDS:tex/latex/zref/zref-marks.sty
%    TDS:tex/latex/zref/zref-nextpage.sty
%    TDS:tex/latex/zref/zref-pageattr.sty
%    TDS:tex/latex/zref/zref-pagelayout.sty
%    TDS:tex/latex/zref/zref-perpage.sty
%    TDS:tex/latex/zref/zref-runs.sty
%    TDS:tex/latex/zref/zref-savepos.sty
%    TDS:tex/latex/zref/zref-thepage.sty
%    TDS:tex/latex/zref/zref-titleref.sty
%    TDS:tex/latex/zref/zref-totpages.sty
%    TDS:tex/latex/zref/zref-user.sty
%    TDS:tex/latex/zref/zref-xr.sty
%    TDS:doc/latex/zref/zref.pdf
%    TDS:doc/latex/zref/zref-example.tex
%    TDS:doc/latex/zref/zref-example-lastpage.tex
%    TDS:doc/latex/zref/zref-example-nextpage.tex
%    TDS:source/latex/zref/zref.dtx
%
%<*ignore>
\begingroup
  \catcode123=1 %
  \catcode125=2 %
  \def\x{LaTeX2e}%
\expandafter\endgroup
\ifcase 0\ifx\install y1\fi\expandafter
         \ifx\csname processbatchFile\endcsname\relax\else1\fi
         \ifx\fmtname\x\else 1\fi\relax
\else\csname fi\endcsname
%</ignore>
%<*install>
\input docstrip.tex
\Msg{************************************************************************}
\Msg{* Installation}
\Msg{* Package: zref 2020-07-03 v2.32 A new reference scheme for LaTeX (HO)}
\Msg{************************************************************************}

\keepsilent
\askforoverwritefalse

\let\MetaPrefix\relax
\preamble

This is a generated file.

Project: zref
Version: 2020-07-03 v2.32

Copyright (C)
  2006-2012 Heiko Oberdiek
  2016-2020 Oberdiek Package Support Group

This work may be distributed and/or modified under the
conditions of the LaTeX Project Public License, either
version 1.3c of this license or (at your option) any later
version. This version of this license is in
   https://www.latex-project.org/lppl/lppl-1-3c.txt
and the latest version of this license is in
   https://www.latex-project.org/lppl.txt
and version 1.3 or later is part of all distributions of
LaTeX version 2005/12/01 or later.

This work has the LPPL maintenance status "maintained".

The Current Maintainers of this work are
Heiko Oberdiek and the Oberdiek Package Support Group
https://github.com/ho-tex/zref/issues


This work consists of the main source file zref.dtx
and the derived files
   zref.sty, zref.pdf, zref.ins, zref.drv, zref-base.sty,
   zref-abspage.sty, zref-abspos.sty, zref-counter.sty,
   zref-dotfill.sty, zref-env.sty, zref-hyperref.sty,
   zref-lastpage.sty, zref-marks.sty, zref-nextpage.sty,
   zref-pageattr.sty, zref-pagelayout.sty, zref-perpage.sty,
   zref-runs.sty, zref-savepos.sty, zref-thepage.sty,
   zref-titleref.sty, zref-totpages.sty, zref-user.sty,
   zref-xr.sty, zref-example.tex, zref-example-lastpage.tex,
   zref-example-nextpage.tex, zref-test1.tex,
   zref-test-base.tex, zref-test-runs.tex,
   zref-test-titleref-memoir.tex.

\endpreamble
\let\MetaPrefix\DoubleperCent

\generate{%
  \file{zref.ins}{\from{zref.dtx}{install}}%
  \file{zref.drv}{\from{zref.dtx}{driver}}%
  \usedir{tex/latex/zref}%
  \file{zref.sty}{\from{zref.dtx}{package}}%
  \file{zref-base.sty}{\from{zref.dtx}{base}}%
  \file{zref-abspage.sty}{\from{zref.dtx}{abspage}}%
  \file{zref-abspos.sty}{\from{zref.dtx}{abspos}}%
  \file{zref-counter.sty}{\from{zref.dtx}{counter}}%
  \file{zref-dotfill.sty}{\from{zref.dtx}{dotfill}}%
  \file{zref-env.sty}{\from{zref.dtx}{env}}%
  \file{zref-hyperref.sty}{\from{zref.dtx}{hyperref}}%
  \file{zref-lastpage.sty}{\from{zref.dtx}{lastpage}}%
  \file{zref-marks.sty}{\from{zref.dtx}{marks}}%
  \file{zref-nextpage.sty}{\from{zref.dtx}{nextpage}}%
  \file{zref-pageattr.sty}{\from{zref.dtx}{pageattr}}%
  \file{zref-pagelayout.sty}{\from{zref.dtx}{pagelayout}}%
  \file{zref-perpage.sty}{\from{zref.dtx}{perpage}}%
  \file{zref-runs.sty}{\from{zref.dtx}{runs}}%
  \file{zref-savepos.sty}{\from{zref.dtx}{savepos}}%
  \file{zref-thepage.sty}{\from{zref.dtx}{thepage}}%
  \file{zref-titleref.sty}{\from{zref.dtx}{titleref}}%
  \file{zref-totpages.sty}{\from{zref.dtx}{totpages}}%
  \file{zref-user.sty}{\from{zref.dtx}{user}}%
  \file{zref-xr.sty}{\from{zref.dtx}{xr}}%
  \usedir{doc/latex/zref}%
  \file{zref-example.tex}{\from{zref.dtx}{example}}%
  \file{zref-example-lastpage.tex}{\from{zref.dtx}{example-lastpage}}%
  \file{zref-example-nextpage.tex}{\from{zref.dtx}{example-nextpage}}%
%  \usedir{doc/latex/zref/test}%
%  \file{zref-test1.tex}{\from{zref.dtx}{test1}}%
%  \file{zref-test-base.tex}{\from{zref.dtx}{test-base}}%
%  \file{zref-test-runs.tex}{\from{zref.dtx}{test-runs}}%
%  \file{zref-test-titleref-memoir.tex}{\from{zref.dtx}{test-titleref-memoir}}%
}

\catcode32=13\relax% active space
\let =\space%
\Msg{************************************************************************}
\Msg{*}
\Msg{* To finish the installation you have to move the following}
\Msg{* files into a directory searched by TeX:}
\Msg{*}
\Msg{*     zref.sty, zref-base.sty, zref-abspage.sty, zref-abspos.sty,}
\Msg{*     zref-counter.sty, zref-dotfill.sty, zref-env.sty,}
\Msg{*     zref-hyperref.sty, zref-lastpage.sty, zref-marks.sty,}
\Msg{*     zref-nextpage.sty, zref-pageattr.sty, zref-pagelayout.sty,}
\Msg{*     zref-perpage.sty, zref-runs.sty, zref-savepos.sty,}
\Msg{*     zref-thepage.sty, zref-titleref.sty, zref-totpages.sty,}
\Msg{*     zref-user.sty, zref-xr.sty}
\Msg{*}
\Msg{* To produce the documentation run the file `zref.drv'}
\Msg{* through LaTeX.}
\Msg{*}
\Msg{* Happy TeXing!}
\Msg{*}
\Msg{************************************************************************}

\endbatchfile
%</install>
%<*ignore>
\fi
%</ignore>
%<*driver>
\NeedsTeXFormat{LaTeX2e}
\ProvidesFile{zref.drv}%
  [2020-07-03 v2.32 A new reference scheme for LaTeX (HO)]%
\documentclass{ltxdoc}
\usepackage{holtxdoc}[2011/11/22]
\setlength{\emergencystretch}{1em}
\begin{document}
  \DocInput{zref.dtx}%
\end{document}
%</driver>
% \fi
%
%
%
% \GetFileInfo{zref.drv}
%
% \title{The \xpackage{zref} package}
% \date{2020-07-03 v2.32}
% \author{Heiko Oberdiek\thanks
% {Please report any issues at \url{https://github.com/ho-tex/zref/issues}}}
%
% \maketitle
%
% \begin{abstract}
% Package \xpackage{zref} tries to get rid of the restriction
% in \hologo{LaTeX}'s reference system that only two properties are
% supported. The package implements an extensible referencing
% system, where properties are handled in a more flexible way.
% It offers an interface for macro programmers for the access
% to the system and some applications that uses the new
% reference scheme.
% \end{abstract}
%
% \tableofcontents
%
% \newcommand*{\metacs}[1]{\texttt{\itshape #1}}
%
% \newcommand{\markexp}{^^A
%   \textsuperscript{exp}^^A
%   \space
% }
% \newcommand{\markexptwo}{^^A
%   \textsuperscript{exp2}^^A
%   \space
% }
% \newcommand{\markrobust}{^^A
%   \textsuperscript{robust}^^A
%   \space
% }
% \newcommand{\marknowarn}{^^A
%   \textsuperscript{nowarn}^^A
%   \space
% }
% \newcommand{\markbabel}{^^A
%   \textsuperscript{babel}^^A
%   \space
% }
%
% \section{Introduction}
%
% Standard \hologo{LaTeX}'s reference system with \cs{label}, \cs{ref},
% and \cs{pageref} supports two properties, the apperance of
% the counter that is last incremented by \cs{refstepcounter} and
% the page with the \cs{label} command.
%
% Unhappily \hologo{LaTeX} does not provide an interface for adding
% another properties. Packages such as \xpackage{hyperref},
% \xpackage{nameref}, or \xpackage{titleref} are forced to
% use ugly hacks to extend the reference system.
% These ugly hacks are one of the causes for \xpackage{hyperref}'s
% difficulty regarding compatibility with other packages.
%
% \subsection{Standard \hologo{LaTeX} behaviour}
%
% References are created by the \cs{label} command:
% \begin{quote}
%   |\chapter{Second chapter}|\\
%   |\section{First section on page 7} % section 2.1|\\
%   |\label{myref}|
% \end{quote}
% Now \hologo{LaTeX} records the section number |2.1| and the page |7|
% in the reference. Internally the reference is a list with
% two entries:
% \begin{quote}
%   |\r@myref| $\rightarrow$ |{2.1}{7}|
% \end{quote}
% The length of the list if fixed in the \hologo{LaTeX} kernel,
% An interface for adding new properties is missing.
%
% There are several tries to add new properties:
% \begin{description}
% \item[\xpackage{hyperref}] uses a list of five
%   properties instead of the standard list with two entries.
%   This causes many compatibility problems with \hologo{LaTeX} and
%   other packages.
% \item[\xpackage{titleref}] stores its title data into the
%   first entry in the list. \hologo{LaTeX} is happy because it
%   does only see its list with two entries. The situation
%   becomes more difficult, if more properties are added
%   this way. Then the macros form a nested structure
%   inside the first reference argument for the label.
%   Expandable extractions will then become painful.
% \end{description}
%
% \subsection{Basic idea}
%
% Some time ago Morten H\o gholm sent me an experimental
% cross referencing mechanism as ``expl3'' code. His
% idea is:
% \begin{quote}
%   |\g_xref_mylabel_plist| $\rightarrow$\\
%   \mbox{}\qquad|\xref_dance_key{salsa}\xref_name_key{Morten}|\dots
% \end{quote}
% The entries have the following format:
% \begin{quote}
%   |\xref_|\meta{your key}|_key{|\meta{some text}|}|
% \end{quote}
% This approach is much more flexible:
% \begin{itemize}
%   \item New properties can easily be added, just use a new key.
%   \item The length of the list is not fixed. A reference
%         can use a subset of the keys.
%   \item The order of the entries does not matter.
% \end{itemize}
% Unhappily I am not familiar with the experimental code
% for \hologo{LaTeX3} that will need some time before its first release.
% Thus I have implemented it as \hologo{LaTeXe} package without disturbing
% the existing \hologo{LaTeX} reference system.
%
% \subsection{Interfaces}
%
% The package provides a generic \emph{interface for programmers}.
% Commands of this interface are prefixed by \cs{zref@}.
%
% Option \xoption{user} enabels the \emph{user interface}.
% Here the commands are prefixed by \cs{z} to avoid name
% clashes with existing macros.
%
% Then the packages provides some \emph{modules}. They
% are applications for the reference system and can also be
% considered as examples how to use the reference system.
%
% The modules can be loaded as packages. The package
% name is prefixed with |zref-|, for example:
%\begin{quote}
%\begin{verbatim}
%\RequirePackage{zref-abspage}
%\end{verbatim}
%\end{quote}
% This is the preferred way if the package is loaded from
% within other packages to avoid option clashes.
%
% As alternative package |zref| can be used and the modules
% are given as options:
%\begin{quote}
%\begin{verbatim}
%\usepackage[perpage,user]{zref}
%\end{verbatim}
%\end{quote}
%
% \section{Interface for programmers}
%
% The user interface is described in the next section \ref{sec:user}.
%
% \subsection{Entities}
%
% \paragraph{Reference.}
% Internally a reference is a list of key value pairs:
% \begin{quote}
%   |\Z@R@myref| $\rightarrow$
%   |\default{2.1}\page{7}|
% \end{quote}
% The generic format of a entry is:
% \begin{quote}
%   |\Z@R@|\meta{refname} $\rightarrow$
%   |\|\meta{propname}|{|\meta{value}|}|
% \end{quote}
% \meta{refname} is the name that denoted references
% (the name used in \cs{label} and \cs{ref}). \meta{propname}
% is the name of the property or key. The property key macro
% is never executed, it is used in parameter text matching only.
%
% \paragraph{Property.}
% Because the name of a property is used in a macro name
% that must survive the \xfile{.aux} file, the name is
% restricted to letters and `@'.
%
% \paragraph{Property list.}
% Often references are used for special purposes. Thus it
% saves memory if just the properties are used in this
% reference that are necessary for its purpose.
%
% Therefore this package uses the concept of \emph{property lists}.
% A property list is a set of properties. The set of properties
% that is used by the default \cs{label} command is the
% \emph{main property list}.
%
% \subsection{Property list}
%
% \markexp means that the implementation of the
% marked macro is expandable. \markexptwo goes a step further
% and marks the macro expandable in exact two expansion steps.
%
% \begin{declcs}{zref@newlist}
%   \M{listname}
% \end{declcs}
% Declares a new empty property list.
%
% \begin{declcs}{zref@addprop}
%   \M{listname} \M{propname}\\
%   \cs{zref@localaddprop} \M{listname} \M{propname}
% \end{declcs}
% Adds the property \meta{propname}
% to the property list \meta{listname}.
% The property and list must exist.
% The addition is global by \cs{zref@addprop} and limited to
% local scope by \cs{zref@localaddprop}.
% Between 2010/04/19 v2.13 and 2010/10/22 v2.19 a comma separated list
% of properties could be used as argument \meta{propname}.
% Since 2010/10/22 v2.19 the addition of several properties
% at once is supported by \cs{zref@addprops}.
%
% \begin{declcs}{zref@addprops}
%   \M{listname} \M{propname list}\\
%   \cs{zref@localaddprops} \M{listname} \M{propname list}
% \end{declcs}
% These macros add a comma separated list of properties
% \meta{propname list} to list \meta{listname}.
% \cs{zref@addprops} works globally and \cs{zref@localaddprops}
% locally. Since 2010/10/22 v2.19.
%
% \begin{declcs}{zref@listexists}
%   \M{listname} \M{then}
% \end{declcs}
% Executes \meta{then} if the property list \meta{listname} exists
% or raise an error otherwise.
%
% \begin{declcs}{zref@iflistundefined}\markexp \M{listname}
%   \M{then} \M{else}
% \end{declcs}
% Executes \meta{then} if the list exists or \meta{else}
% otherwise.
%
% \begin{declcs}{zref@iflistcontainsprop}
%   \M{listname} \M{propname} \M{then} \M{else}
% \end{declcs}
% Executes \meta{then} if the property \meta{propname} is part
% of property list \meta{listname} or otherwise it
% runs the \meta{else} part.
%
% \subsection{Property}
%
% \begin{declcs}{zref@newprop}
%   \* \M{propname} |[|\meta{default}|]| \M{value}
% \end{declcs}
% This command declares and configures a new property with
% name \meta{propname}.
%
% In case of unknown references
% or the property does not exist in the reference, the
% \meta{default} is used as value. If it is not specified
% here, a global default is used, see \cs{zref@setdefault}.
%
% The correct values of some properties are not known
% immediately but at page shipout time. Prominent example
% is the page number. These properties are declared with
% the star form of the command.
%
% \begin{declcs}{zref@setcurrent}
%   \M{propname} \M{value}
% \end{declcs}
% This sets the current value of the property \meta{propname}.
% It is a generalization of setting \hologo{LaTeX}'s \cs{currentlabel}.
%
% \begin{declcs}{zref@getcurrent}\markexptwo
%   \M{propname}
% \end{declcs}
% This returns the current value of the property \meta{propname}.
% The value may not be correct, especially if the property
% is bound to a page (start form of \cs{zref@newprop}) and the
% right value is only known at shipout time (e.g. property `page').
% In case of errors (e.g. unknown property) the empty string is
% returned.
%
% Since version 2010/04/22 v2.14 \cs{zref@getcurrent} supports
% \cs{zref@wrapper@unexpanded}.
%
% \begin{declcs}{zref@propexists}
%   \M{propname} \M{then}
% \end{declcs}
% Calls \meta{then} if the property \meta{propname} is available
% or generates an error message otherwise.
%
% \begin{declcs}{zref@ifpropundefined}\markexp
%   \M{propname} \M{then} \M{else}
% \end{declcs}
% Calls \meta{then} or \meta{else} depending on the existence
% of property \meta{propname}.
%
% \subsection{Reference generation}
%
% \begin{declcs}{zref@label}
%   \M{refname}
% \end{declcs}
% This works similar to \cs{label}. The reference \meta{refname} is
% created and put into the \xfile{.aux} file with the
% properties of the main property list.
%
% \begin{declcs}{zref@labelbylist}
%   \M{refname} \M{listname}
% \end{declcs}
% Same as \cs{zref@label} except that the properties are
% taken from the specified property list \meta{listname}.
%
% \begin{declcs}{zref@labelbyprops}
%   \M{refname} |{|\meta{propnameA}|,|\meta{propnameB}|,|\dots|}|
% \end{declcs}
% Same as \cs{zref@label} except that these properties
% are used that are given as comma separated list in the
% second argument.
%
% \begin{declcs}{zref@newlabel}
%   \M{refname} |{|\dots|}|
% \end{declcs}
% This is the macro that is used in the \xfile{.aux} file.
% It is basically the same as \cs{newlabel} apart from
% the format of the data in the second argument.
%
% \subsection{Data extraction}
%
% \begin{declcs}{zref@extractdefault}\markexptwo
%   \M{refname} \M{propname} \M{default}
% \end{declcs}
% This is the basic command that refernces the value of a
% property \meta{propname} for the reference \meta{refname}.
% In case of errors such as undefined reference the \meta{default}
% is used instead.
%
% \begin{declcs}{zref@extract}\markexptwo
%   \M{refname} \M{propname}
% \end{declcs}
% The command is an abbreviation for \cs{zref@extractdefault}.
% As default the default of the property is taken, otherwise
% the global default.
%
% Example for page references:
% \begin{quote}
% \begin{tabular}{@{}ll}
% \hologo{LaTeX}:& |\pageref{foobar}|\\
% \xpackage{zref}:& |\zref@extract{foobar}{page}|
% \end{tabular}
% \end{quote}
% Both \cs{zref@extract} and \cs{zref@extractdefault} are
% expandable. That means, these macros can directly
% be used in expandable calculations, see the example file.
% On the other side, babel's shorthands are not supported,
% there are no warnings in case of undefined references.
%
% If an user interface doesn't need expandable macros
% then it can use \cs{zref@refused} and \cs{zref@wrapper@babel}
% for its user macros.
%
% \begin{declcs}{zref@refused}
%   \M{refname}
% \end{declcs}
% This command is not expandable. It causes the warnings if
% the reference \meta{refname} is not defined. Use the
% \cs{zref@extract} commands inside expandable contexts and
% mark their use outside by \cs{zref@refused}, see the example
% file.
%
% \begin{declcs}{zref@def@extract}%
%     \M{cmd} \M{refname} \M{propname}\\
%   \cs{zref@def@extractdefault}
%     \M{cmd} \M{refname} \M{propname} \M{default}
% \end{declcs}
% Both macros extract the property \meta{propname} from the reference
% \meta{refname} the same way as macros \cs{zref@extract} and
% \cs{zref@extractdefault}. The result is stored in macro \meta{cmd}.
% Also \cs{zref@refused} is called to notify \hologo{LaTeX} that
% the reference \meta{refname} is used. Added in 2011/10/04 v2.22.
%
% \begin{declcs}{zref@ifrefundefined}\markexp
%   \M{refname} \M{then} \M{else}
% \end{declcs}
% Macro \cs{zref@ifrefundefined} calls arguments \meta{then} or \meta{else}
% dependent on the existence of the reference \meta{refname}.
%
% \begin{declcs}{zifrefundefined}
%   \M{refname} \M{then} \M{else}
% \end{declcs}
% Macro \cs{zifrefundefined} calls \cs{ref@refused} before
% executing \cs{zref@ifrefundefined}. Babel shorthands are
% supported in \meta{refname}.
%
% \begin{declcs}{zref@ifrefcontainsprop}\markexp
%   \M{refname} \M{propname} \M{then} \M{else}
% \end{declcs}
% Test whether a reference provides a property.
%
% \subsection{Setup}
%
% \begin{declcs}{zref@default}
% \end{declcs}
% Holds the global default for unknown values.
%
% \begin{declcs}{zref@setdefault}
%   |{|\meta{value}|}|
% \end{declcs}
% Sets the global default for unknown values.
% The global default is used, if a property does not specify
% an own default and the value for a property
% cannot be extracted. This can happen if the reference is
% unknown or the reference does not have the property.
%
% \begin{declcs}{zref@setmainlist}
%   |{|\meta{value}|}|
% \end{declcs}
% Sets the name of the main property list. The package
% sets and uses |main|.
%
% \subsection{Declared properties}
%
% \begin{quote}
% \renewcommand*{\arraystretch}{1.05}
% \setlength{\tabcolsep}{2\tabcolsep}
% \def\VV{\vphantom{\large \"Ag}}
% \def\EMPTY{{\itshape\textless empty\textgreater}}
% \def\pl#1{&#1&thepage&\cs{number}\cs{#1}}
% \begin{tabular}{@{}>{\sffamily}l|>{\ttfamily}l|>{\ttfamily}l|>{\ttfamily}l@{}}
%   \rmfamily Module&
%   \rmfamily Property&
%   \rmfamily Property list&
%   \rmfamily Default\\
%   \hline
%   (base)
%     & default & main & \EMPTY\\
%     & page & main & \EMPTY\\
%   \hline
%   abspage
%     & abspage & main & 0\\
%   \hline
%   counter
%     & counter & main & \EMPTY\\
%   \hline
%   hyperref
%     & anchor & main & \EMPTY\\
%     & url    & & \EMPTY\\
%   \hline
%   pageattr
%     & pdfpageattr & thepage & \dots\\
%     & pdfpagesattr & LastPage & \dots\\
%   \hline
%   pagelayout\footnotemark
%     \pl{mag}\\
%     \pl{paperwidth}\\
%     \pl{paperheight}\\
%     \pl{stockwidth}\\
%     \pl{stockheight}\\
%     \pl{pdfpageheight}\\
%     \pl{pdfpagewidth}\\
%     \pl{pdfhorigin}\\
%     \pl{pdfvorigin}\\
%     \pl{hoffset}\\
%     \pl{voffset}\\
%     \pl{topmargin}\\
%     \pl{oddsidemargin}\\
%     \pl{evensidemargin}\\
%     \pl{textwidth}\\
%     \pl{textheight}\\
%     \pl{headheight}\\
%     \pl{headsep}\\
%     \pl{footskip}\\
%     \pl{marginparwidth}\\
%     \pl{marginparsep}\\
%     \pl{columnwidth}\\
%     \pl{columnsep}\\
%   \hline
%   perpage
%     & pagevalue & perpage & 0\\
%     & page & perpage & \EMPTY\\
%     & abspage & perpage & 0\\
%   \hline
%   savepos
%     & posx & savepos & 0\\
%     & posy & savepos & 0\\
%   \hline
%   titleref
%     & title & main & \EMPTY\\
%   \hline
%   xr
%     & anchor   & & \EMPTY\\
%     & externaldocument & & \EMPTY\\
%     & theotype & & \EMPTY\\
%     & title    & & \EMPTY\\
%     & url      & & \EMPTY\\
% \end{tabular}
% \footnotetext{Module \xmodule{pagelayout} only defines properties
% if the parameter exists.}%
% \end{quote}
%
% \subsection{Wrapper for advanced situations}
%
% \begin{declcs}{zref@wrapper@babel}
%   |{|\dots|}| \M{name}
% \end{declcs}
% This macro helps to add shorthand support. The second
% argument is protected, then the code of the first argument
% is called with the protected name appended. Examples
% are in the sources.
%
% \begin{declcs}{zref@wrapper@immediate}
%   |{|\dots|}|
% \end{declcs}
% There are situations where a label must be written instantly
% to the \xfile{.aux} file, for example after the last page.
% If the \cs{zlabel} or \cs{label} command is put inside this wrapper,
% immediate writing is enabled. See the implementation for
% module \xmodule{lastpage} for an example of its use.
%
% \begin{declcs}{zref@wrapper@unexpanded}
%   |{|\dots|}|
% \end{declcs}
% Assuming someone wants to extract a value for property |bar|
% and store the result in a macro |\foo| without traces of
% the expanding macros and without expanding the value.
% This (theoretical?) problem can be solved by this wrapper:
% \begin{quote}
%\begin{verbatim}
%\zref@wrapper@unexpanded{%
%  \edef\foo{%
%    \zref@extract{someref}{bar}%
%  }%
%}
%\end{verbatim}
% \end{quote}
% The \cs{edef} forces the expansion of \cs{zref@extract},
% but the extraction of the value is prevented by
% the wrapper that uses \hologo{eTeX}'\ \cs{unexpanded} for this
% purpose. Supported macros are \cs{zref@extract}, \cs{zref@extractdefault}
% and since version 2010/04/22 v2.14 macro \cs{zref@getcurrent}.
%
% \subsection{Counter for unique names}
%
% Some modules (\xmodule{titleref} and \xmodule{dotfillmin})
% need unique names for automatically generated label names.
%
% \begin{declcs}{zref@require@unique}
% \end{declcs}
% This command creates the unique counter |zref@unique| if the
% counter does not already exist.
%
% \begin{declcs}{thezref@unique}
% \end{declcs}
% This command is used to generate unique label names.
%
% \section{User interface}\label{sec:user}
%
% \subsection{Module \xmodule{user}}
%
% The user interface for this package and its modules
% is enabled by \xpackage{zref}'s package option \xmodule{user}
% or package \xpackage{zref-user}.
% The names of user commands are prefixed by |z| in order
% to avoid name clashes with existing macros of the same
% functionality. Thus the package does not disturb the
% traditional reference scheme, both can be used together.
%
% The syntax descriptions contain the following markers
% that are intended as hints for programmers:
% \begin{quote}
% \begin{tabular}{@{}ll@{}}
%   \markbabel\unskip  & Babel shorthands are allowed.\\
%   \markrobust\unskip & Robust macro.\\
%   \markexp\unskip    & Expandable version:\\
%                      & \textbullet\ robust,
%                        unless the extracted values are fragile,\\
%                      & \textbullet\ no babel shorthand suport.\\
%   \markexptwo\unskip & Expandable like \markexp and:\\
%                      & \textbullet\ expandable in exact two steps.\\
% \end{tabular}
% \end{quote}
%
% The basic user interface of the package without modules
% are commands that mimic the standard \hologo{LaTeX} behaviour
% of \cs{label}, \cs{ref}, and \cs{pageref}:
%
% \begin{declcs}{zlabel}
%   \M{refname}\markbabel
% \end{declcs}
% Similar to \cs{label}. It generates a label with name
% \meta{refname} in the new reference scheme.
%
% \begin{declcs}{zref}
%   |[|\meta{propname}|]| \M{refname}\markbabel
% \end{declcs}
% Without optional argument similar to \cs{ref}, it returns
% the default reference property. This property is named
% |default|:
% \[
%   |\zref{|x|}| \equiv |\zref[default]{|x|}|
% \]
%
% \begin{declcs}{zpageref}
%   \M{refname}\markbabel
% \end{declcs}
% Convenience macro, similar to \cs{pageref}.
% \[
%   |\zpageref{|x|}| \equiv |\zref[page]{|x|}|
% \]
%
% \begin{declcs}{zrefused}
%   \M{refname}\markbabel
% \end{declcs}
% Some of the user commands in the modules are expandable.
% The use of such commands do not cause any undefined
% reference warnings, because inside of expandable
% contexts this is not possible. However, if there is
% a place outside of expandable contexts, \cs{refused}
% is strongly recommended. The reference \meta{refname}
% is marked as used, undefined ones will generate
% warnings.
%
% \subsection{Module \xmodule{abspage}}
%
% With the help of package \xpackage{atbegshi}
% a new counter |abspage| with absolute page numbers is provided.
% For technical and historical reasons the counter itself is zero based:
% if you use it directly on the first page, e.g
% with |\arabic{abspage}| you will get 0 as value. When using |\zref|
% the first page will be page 1 as expected.
% Also a new property |abspage| is defined and added to the
% main property list. Thus you can reference the absolute
% page number:
% \begin{quote}
%   |Section \zref{foo} is on page \zpageref{foo}.|\\
%   |This is page \zref[abspage]{foo}|\\
%   |of \zref[abspage]{LastPage}.|
% \end{quote}
% The example also makes use of module \xmodule{lastpage}.
%
% \subsection{Module \xmodule{lastpage}}
%
% Provides the functionality of package \xpackage{lastpage}
% \cite{lastpage} in the new reference scheme.
% The label |LastPage| is put at the end of the document.
% You can refer the last page number with:
% \begin{quote}
%   |\zref@extract{LastPage}{page}| (+ |\zref@refused{LastPage}|)\\
% \end{quote}
% or
% \begin{quote}
%   |\zpageref{LastPage}| (module \xmodule{user})
% \end{quote}
%
% Since version 2008/10/01 v2.3 the module defines the list |LastPage|.
% In addition to the properties of the main list label |LastPage| also
% stores the properties of this list |LastPage|. The default of this
% list is empty. The list can be used by the user to add additional
% properties for label |LastPage|.
%
% \subsubsection{Tests for last page}
%
% Since version 2010/03/26 v2.8 the macros \cs{zref@iflastpage}
% and \cs{ziflastpage} were added. They test the reference,
% whether it is a reference of the last page.
%
% \begin{declcs}{zref@iflastpage}\markexp \M{refname} \M{then} \M{else}
% \end{declcs}
% Macro \cs{zref@iflastpage} compares the references \meta{refname}
% with \meta{LastPage}. Basis of the comparison is the value of
% property |abspage|, because the values are different for different
% pages. This is not ensured by property |page|. Therefore module
% \xmodule{abspage} is loaded by module \xmodule{lastpage}.
% If both values of property |abspage| are present and match,
% then \meta{then} is executed, otherwise code \meta{else} is called.
% If one or both references are undefined or lack the property |abspage|,
% then \meta{else} is executed.
%
% Macro \cs{zref@iflastpage} is expandable, therefore \cs{zref@refused}
% should be called on \meta{refname} and \meta{LastPage}.
%
% \begin{declcs}{ziflastpage} \M{refname} \M{then} \M{else}
% \end{declcs}
% Macro \cs{ziflastpage} has the same function as \cs{zref@iflastpage},
% but adds support for babel shorthands in \meta{refname} and
% calls \cs{zref@refused}. However macro \cs{ziflastpage} is not
% expandable.
%
% \subsubsection{Example}
%
%    \begin{macrocode}
%<*example-lastpage>
%<<END_EXAMPLE
\NeedsTeXFormat{LaTeX2e}
\documentclass{report}

\newcounter{foo}
\renewcommand*{\thefoo}{\Alph{foo}}

\usepackage{zref-lastpage,zref-user}[2019/11/29]

\makeatletter
\zref@newprop{thefoo}{\thefoo}
\zref@newprop{valuefoo}{\the\value{foo}}
\zref@newprop{chapter}{\thechapter}
\zref@addprops{LastPage}{thefoo,valuefoo,chapter}
\makeatother

\newcommand*{\foo}{%
  \stepcounter{foo}%
  [Current foo: \thefoo]%
}

\begin{document}
  \chapter{First chapter}
  Last page is \zref{LastPage}.\\
  Last chapter is \zref[chapter]{LastPage}.\\
  Last foo is \zref[thefoo]{LastPage}.\\
  Last value of foo is \zref[valuefoo]{LastPage}.\\
  \foo
  \chapter{Second chapter}
  \foo\foo\foo
  \chapter{Last chapter}
  \foo
\end{document}
%END_EXAMPLE
%</example-lastpage>
%    \end{macrocode}
%
% \subsection{Module \xmodule{thepage}}
%
% This module \xmodule{thepage} loads module \xmodule{abspage},
% constructs a reference name using the absolute page number
% and remembers property |page|. Other properties can be added
% by adding them to the property list |thepage|.
%
% \begin{declcs}{zthepage} \M{absolute page number}\\
% \end{declcs}
% Macro \cs{zthepage} is basically a \cs{zpageref}. The reference
% name is yield by the \meta{absolute page number}. If the reference
% is not defined, then the default for property |page| is used.
%
% \begin{declcs}{zref@thepage@name}\markexp \M{absolute page number}
% \end{declcs}
% Macro \cs{zref@thepage@name} returns the internal reference
% name that is constructed using the \meta{absolute page number}.
% The internal reference name should not be used directly, because
% it might change in future versions.
%
% \begin{declcs}{zref@thepage}\markexp \M{absolute page number}\\
%   \cs{zref@thepage@refused} \M{absolute page number}
% \end{declcs}
% Macro \cs{zref@thepage} returns the page number (\cs{thepage})
% of \meta{absolute page number}. Because this macro is expandable,
% \cs{zref@thepage@refused} is used outside an expandable context
% to mark the reference as used.
%
%
% \subsection{Module \xmodule{nextpage}}
%
% \begin{declcs}{znextpage}
% \end{declcs}
% Macro \cs{znextpage} prints \cs{thepage} of the following page.
% It gets the current absolute page number by using a label.
% There are three cases for the next page:
% \begin{enumerate}
% \item The next page is not known yet because of undefined references.
%   Then \cs{zunknownnextpagename} is used instead. The default for
%   this macro is the default of property |page|.
% \item This page is the last page. Then \cs{znonextpagename}
%   is used. Its default is empty.
% \item The next page is known, then \cs{thepage} of the next
%   page is used (the value of property |page| of the next page).
% \end{enumerate}
%
% \subsubsection{Configuration}
%
% The behaviour can be configured by the following macros.
%
% \begin{declcs}{zunknownnextpagename}\\
%   \cs{znonextpagename}
% \end{declcs}
% If the next page is not known or available, then
% \cs{znextpage} uses these name macros as default.
% \cs{zunknownnextpagename} is used in case of undefined
% references. Default is the value of property |page| of the
% next page (\cs{thepage}). Module \xmodule{thepage} is used.
%
% Macro \cs{znonextpagename} is used, if the next page does
% not exists. That means that the current page is last page.
% The default is empty.
%
% \begin{declcs}{znextpagesetup} \M{unknown} \M{no next}
%   \M{next}
% \end{declcs}
% Acording to the case (see \cs{znextpage}) macro \cs{znextpage}
% calls an internal macro with an argument. The argument is
% either \cs{thepage} of the next page or one of \cs{zunknownnextpagename}
% or \cs{znonextpagename}. These internal macro can be changed by
% \cs{znextpagesetup}. It expects the definition texts for these
% three cases of a macro with one argument.
% The default is
% \begin{quote}
%   |\znextpagesetup{#1}{#1}{#1}|
% \end{quote}
%
% \subsubsection{Example}
%
%    \begin{macrocode}
%<*example-nextpage>
%<<END_EXAMPLE
\documentclass{book}

\usepackage{zref-nextpage}[2019/11/29]
\znextpagesetup
  {\thepage}% next page is unknown
  {\thepage\ (#1)}% this page is last page
  {\thepage\ $\rightarrow$ #1}% next page is known
\renewcommand*{\znonextpagename}{last page}

\usepackage{fancyhdr}
\pagestyle{fancy}
\fancyhf{}
\fancyhead[LE,RO]{\znextpage}
\fancypagestyle{plain}{%
  \fancyhf{}%
  \fancyhead[LE,RO]{\znextpage}%
}

\begin{document}
\frontmatter
  \tableofcontents
\mainmatter
  \chapter{Hello World}
  \clearpage
  \section{Last section}
\end{document}
%END_EXAMPLE
%</example-nextpage>
%    \end{macrocode}
%
%
% \subsection{Module \xmodule{totpages}}
%
% For the total number of pages of a document you need to
% know the absolute page number of the last page.
% Both modules \xmodule{abspage} and \xmodule{lastpage}
% are necessary and automatically enabled.
%
% \begin{declcs}{ztotpages}\markexp
% \end{declcs}
% Prints the total number of pages or |0| if this
% number is not yet known. It expands to an explicit number
% and can also used even in expandable calculations (\cs{numexpr})
% or counter assignments.
%
% \subsection{Module \xmodule{pagelayout}}
%
% The module defines additional properties for each parameter
% of the page layout that is effective during page shipout.
% The value of length parameters is given in sp without the
% unit as plain number.
%
% Some parameters are specific for a class (e.g.\@ \xoption{stockwidth}
% and \xoption{stockheight} for class \xclass{memoir}) or the
% \hologo{TeX} engine like \hologo{pdfTeX}. If the parameter is not
% available, then the property will not be defined. The default value
% of the property is the current setting of the parameter.
%
% The module \xmodule{thepage} is loaded that generates a label
% for each page. The properties of module \xmodule{pagelayout}
% are added to the property list |thepage| of module \xmodule{thepage}.
%
% List of properties:
% \begin{quote}
%   \def\x#1{\cs{#1}&#1}%
%   \begin{tabular}{l>{\sffamily}ll}
%     \textrm{parameter} & \textrm{property} & remarks\\
%     \hline
%     \x{mag}\\
%     \x{paperwidth}\\
%     \x{paperheight}\\
%     \x{stockwidth}& class \xclass{memoir}\\
%     \x{stockheight}& class \xclass{memoir}\\
%     \x{pdfpagewidth}& \hologo{pdfTeX}, \hologo{LuaTeX}\\
%     \x{pdfpageheight}& \hologo{pdfTeX}, \hologo{LuaTeX}\\
%     \x{pdfhorigin}& \hologo{pdfTeX}, \hologo{LuaTeX}\\
%     \x{pdfvorigin}& \hologo{pdfTeX}, \hologo{LuaTeX}\\
%     \x{hoffset}\\
%     \x{voffset}\\
%     \x{topmargin}\\
%     \x{oddsidemargin}\\
%     \x{evensidemargin}\\
%     \x{textwidth}\\
%     \x{textheight}\\
%     \x{headheight}\\
%     \x{headsep}\\
%     \x{footskip}\\
%     \x{marginparwidth}\\
%     \x{marginparsep}\\
%     \x{columnwidth}\\
%     \x{columnsep}\\
%   \end{tabular}
% \end{quote}
%
% \begin{declcs}{zlistpagelayout}
% \end{declcs}
% At the end of document the page layout parameter for each page are
% printed into the \xext{log} file if macro \cs{zlistpagelayout}
% is called before |\end{document}| (preamble is a good place).
%
% \subsection{Module \xmodule{marks}}
%
% ToDo.
%
% \subsection{Module \xmodule{runs}}
%
% Module \xmodule{runs} counts the \hologo{LaTeX} runs since last
% \xext{aux} file creation and prints the number in the
% \xext{log} file.
%
% \begin{declcs}{zruns}\markexp
% \end{declcs}
% Prints the the total number of \hologo{LaTeX} runs including
% the current one. It expands to an explicit number.
% Before |begin{document}| the value is zero meaning
% the \xext{aux} file is not read yet. If a previous
% \xext{aux} file exists, the value found there increased
% by one is the new number. Otherwise \cs{zruns} is set to one.
% \hologo{LaTeX}~runs where the \xext{aux} files are not rewritten
% are not counted (see \cs{nofiles}).
%
% \subsection{Module \xmodule{perpage}}
%
% With \cs{@addtoreset} or \cs{numberwithin} a counter can be
% reset if another counter is incremented. This do not work
% well if the other counter is the page counter. The page counter is
% incremented in the output routine that is often called asynchronous
% somewhere on the next page. A reference mechanism costs at least
% two \hologo{LaTeX} runs, but ensures correct page counter values.
%
% \begin{declcs}{zmakeperpage}
%   |[|\meta{reset}|]| \M{counter}
% \end{declcs}
% At the of a new page counter \meta{counter} starts
% counting with value \meta{reset} (default is \texttt{1}).
% The macro has the same
% syntax and semantics as \cs{MakePerPage} of package
% \xpackage{perpage} \cite{perpage}.
% Also \xoption{perpage} of package \xpackage{footmisc} \cite{footmisc}
% can easily be simulated by
% \begin{quote}
% |\zmakeperpage{footnote}| \itshape |% \usepackage[perpage]{footmisc}|
%\end{quote}
% If footnote symbols are used, some people dislike the
% first symbol \ensuremath{\dagger}. It can easily be skipped:
% \begin{quote}
% |\zmakeperpage[2]{footnote}|
% \end{quote}
%
% \begin{declcs}{thezpage}\\
%   counter |zpage|
% \end{declcs}
% If the formatted counter value of the counter that is reset at
% a new page contains the page value, then you can use \cs{thezpage},
% the page number of the current page. Or counter |zpage| can be
% used, if the page number should be formatted differently from
% the current page number. Example:
%\begin{quote}
%\begin{verbatim}
%\newcounter{foobar}
%\zmakeperpage{foobar}
%\renewcommand*{\thefoobar}{\thezpage-\arabic{foobar}}
% % or
%\renewcommand*{\thefoobar}{\roman{zpage}-\arabic{foobar}}
%\end{verbatim}
%\end{quote}
%
% \begin{declcs}{zunmakeperpage}
%   \M{counter}
% \end{declcs}
% The reset mechanism for this counter is deactivated.
%
%
% \subsection{Module \xmodule{counter}}
%
% This option just add the property |counter| to the main
% property list. The property stores the counter name,
% that was responsible for the reference. This is the
% property \xpackage{hyperref}'s \cs{autoref} feature
% uses. Thus this property |counter| may be useful
% for a reimplementation of the autoref feature,
% see the section \ref{sec:todo} with the todo list.
%
% \subsection{Module \xmodule{titleref}}
%
% This option makes section and caption titles available
% to the reference system similar to packages \xpackage{titleref}
% or \xpackage{nameref}.
%
% \begin{declcs}{ztitleref}
%   \M{refname}\markbabel
% \end{declcs}
% Print the section or caption title of reference \meta{refname},
% similar to \cs{nameref} or \cs{titleref}.
%
% \begin{declcs}{ztitlerefsetup}
%   |{|\textit{key$_1$=value$_1$, key$_2$=value$_2$, \dots}|}|
% \end{declcs}
% This command allows to configure the behaviour of module
% \xmodule{titleref}. The following keys are available:
% \begin{description}
% \let\bfseries\relax
% \item[\texttt{title=}\meta{value}]\mbox{}\\*
%   Sets the current title.
% \item[\texttt{stripperiod=true\string|false}]\mbox{}\\*
%   Follow package \xpackage{nameref} that
%   removes a last period. Default: |true|.
% \item[\texttt{expand=true\string|false}]\mbox{}\\*
%   Package \cs{titleref} expands the title first. This way
%   garbage and dangerous commands can be removed, e.g.
%   \cs{label}, \cs{index}\dots. See implementation section
%   for more details. Default is |false|.
% \item[\texttt{cleanup=\{}\textrm{\dots}\texttt{\}}]\mbox{}\\*
%   Hook to add own cleanup code, if method |expand| is used.
%   See implementation section for more details.
% \end{description}
%
% \subsection{Module \xmodule{savepos}}
%
% This option supports a feature that \pdfTeX\ provides
% (and \hologo{XeTeX}).
% \pdfTeX\ is able to tell the current position on the
% page. The page position is not instantly known. First the
% page must be constructed by \hologo{TeX}'s asynchronous output
% routine. Thus the time where the position is known is
% the page shipout time. Thus a reference system where the
% information is recorded in the first run and made available
% for use in the second run comes in handy.
%
% \begin{declcs}{zsavepos}
%   \M{refname}
% \end{declcs}
% It generates a reference with name \meta{refname}.
% The reference stores the location where \cs{zsavepos}
% is executed in properties |posx| and |posy|.
%
% \begin{declcs}{zsaveposx}
%   \M{refname}\\
%   \cs{zsaveposy}
%   \M{refname}
% \end{declcs}
% Same as \cs{zsavepos} except that only the |x| or |y| component
% of the position is stored. Since 2011/12/05 v2.23.
%
% \begin{declcs}{zposx}\markexp
%   \M{refname}\\
%   \SpecialUsageIndex{\zposy}\cs{zposy}\markexp \M{refname}
% \end{declcs}
% Get the position as number. Unit is sp. Horizontal
% positions by \cs{zposx} increase from left to right.
% Vertical positions by \cs{zposy} from bottom to top.
%
% Do not rely on absolute page numbers. Because of problems
% with the origin the numbers may differ in DVI or PDF mode
% of \pdfTeX. Therefore work with relative values by comparisons.
%
% Both \cs{zposx} and \cs{zposy} are expandable and can be
% used inside calculations (\cs{setcounter}, \cs{addtocounter},
% package \xpackage{calc}, \cs{numexpr}). However this property
% prevents from notifying \hologo{LaTeX} that the reference is actually
% used (the notifying is not expandable). Therefore you should
% mark the reference as used by \cs{zrefused}.
%
% This module uses \pdfTeX's \cs{pdfsavepos}, \cs{pdflastxpos},
% and \cs{pdflastypos}. They are available in PDF mode and
% since version 1.40.0 also in DVI mode.
%
% \begin{declcs}{zref@savepos}
% \end{declcs}
% Macro \cs{zref@savepos} performs the first part of \cs{zsavepos}
% by calling \cs{pdfsavepos} (if \xext{aux} files are writable).
%
% Thus \cs{zsavepos} is basically \cs{zref@savepos} followed
% by |\zref@labelbylist{|\texttt{\meta{refname}}|}{savepos}|.
% If \cs{TeXXeTstate} is detected and enabled, \cs{savepos}
% also adds \cs{zref@savepos} at the end to support \cs{beginR}
% where the whatits are processed in reverse order.
% The property list |savepos| contains the properties |posx|
% and |posy|.
%
% \subsection{Module \xmodule{abspos}}
% Module \xmodule{abspos} allows to get various values of the page layout. There is no user
% command, only a number of internal commands. For example:
% \begin{quote}
%   |\zref@absposx{|\meta{label}|}{|\meta{value}|}{|\meta{position}|}|\\
%   |\zref@absposy{|\meta{label}|}{|\meta{value}|}{|\meta{position}|}|
% \end{quote}
%
% The return value is like in the module \xmodule{savepos} a number representing a length in sp.
% The length are measured from the bottom left of the page.
%
% \meta{label} is a label set with \cs{zlabel} or \cs{zsavepos} that allows to retrieve the
% absolute page number.
%
% \meta{position} is for the x-command one of |left|, |right| or |center|. For the y-command
%  it is one of |top|, |bottom|, |center|.
%
% The possible content of \meta{value} can be seen in the following table. Be aware that the code
% makes some assumptions which are perhaps not always true -- for example that the left of
% the head is identical to the left of the body.
%
% \medskip
% \begin{tabular}{lll}
%  \textbf{value} & \textbf{axis} & \textbf{comments}\\\hline
%   media         & x             &   left=0, right=\cs{pdfpagewidth} \\
%   paper         & x             &   left=0, right=\cs{paperwidth}\\
%   stock         & x             &   derived from paper\\
%   media         & y             & bottom=0, top=\cs{pdfpageheigh}\\
%   paper         & y             & top=\cs{pdfpageheight}, bottom=top-\cs{paperheight}\\
%   stock         & y             & top  derived from paper\\
%   head          & x             & calculated with hoffset, horigin, etc\\
%   head          & y             & calculated\\
%   body          & x             & = head value\\%
%   body          & y             & = head bottom - \cs{headsep}\\
%   foot          & x             & = head\\
%   foot          & y             & calculated from body bottom and \cs{footskip}\\
%   marginpar     & x             &different on odd/even pages! \\
%   marginpar     & y             & = body %
% \end{tabular}
%
% \subsection{Module \xmodule{dotfill}}
%
% \begin{declcs}{zdotfill}
% \end{declcs}
% This package provides the command \cs{zdotfill} that works
% similar to \cs{dotfill}, but can be configured. Especially
% it suppresses the dots if a minimum number of dots cannot be set.
%
% \begin{declcs}{zdotfillsetup}
%   |{|\textit{key$_1$=value$_1$, key$_2$=value$_2$, \dots}|}|
% \end{declcs}
% This command allows to configure the behaviour of \cs{zdotfill}.
% The following keys are available:
% \begin{description}
% \let\bfseries\relax
% \item[\texttt{min=}\meta{count value}]\mbox{}\\*
%   If the actual number of dots are smaller than \meta{count value},
%   then the dots are suppressed. Default: |2|.
% \item[\texttt{unit=}\meta{dimen value}]\mbox{}\\*
%   The width of a dot unit is given by \meta{dimen value}.
%   Default: |0.44em| (same as the unit in \cs{dotfill}).
% \item[\texttt{dot=}\meta{value}]\mbox{}\\*
%   The dot itself is given by \meta{value}. Default: |.| (dot, same as
%   the dot in \cs{dotfill}).
% \end{description}
%
% \subsection{Module \xmodule{env}}
%
% This module defines two properties |envname| and |envline|. They
% remember the name of the environment and the line number at
% the start of the environment.
%
% \subsection{Module \xmodule{xr}}
%
% This package provides the functionality of package \xpackage{xr},
% see \cite{xr}. It also supports the syntax of \xpackage{xr-hyper}.
%
% \begin{declcs}{zexternaldocument}
%   \* |[|\meta{prefix}|]|\markbabel \M{external document}
%   |[|\meta{url}|]|
% \end{declcs}
% See \cs{externaldocument} for a description of this option.
% The found labels also get a property |externaldocument| that
% remembers \meta{external document}.
% The standard reference scheme and the scheme of this
% package use different name spaces for reference names.
% If the external document uses both systems. Then one
% import statement would put the names in one namespace
% and probably causing problems with multiple references
% of the same name. Thus the star form only looks for
% \cs{newlabel} in the \xfile{.aux} files, whereas without
% star only \cs{zref@newlabel}s are used.
%
% In the star form it tries to detect labels from
% \xpackage{hyperref}, \xpackage{titleref}, and \xpackage{ntheorem}.
% If such an extended property from the packages before cannot
% be found or are empty, they are not included in the
% imported reference.
%
% Warnings are given if a reference name is already in use
% and the item is ignored. Unknown properties will automatically
% be declared.
%
% If the external references contain |anchor| properties,
% then we need also a url to be able to address the external
% file. As default the filename is taken with a default
% extension.
%
% \begin{declcs}{zxrsetup}
%   |{|\textit{key$_1$=value$_1$, key$_2$=value$_2$, \dots}|}|
% \end{declcs}
% The following setup options are available:
% \begin{description}
% \item[\xoption{ext}:] It sets the default extension.
% \item[\xoption{tozreflabel}:] Boolean option. The found references
%   are imported as zref labels. This is enabled by default.
% \item[\xoption{toltxlabel}:] Boolean option. The found references
%   are imported as \hologo{LaTeX} labels. Packages \xpackage{nameref},
%   \xpackage{titleref} and class \xclass{memoir} are supported.
% \item[\xoption{urluse}:] Boolean option. If enabled, then a URL
%   is stored in a macro and the macro is put in property `urluse'.
%   The URL is not put in property `url'.
%   The purpose is to save \hologo{TeX} memory.
% \item[\xoption{verbose}:] Boolean option. List the imported labels in the
%   \xext{log} file. Default is |false|.
% \end{description}
%
% \begin{declcs}{zref@xr@ext}
% \end{declcs}
% If the \meta{url} is not specified in \cs{zref@externaldocument},
% then the url will be constructed with the file name and
% this macro as extension. \cs{XR@ext} is used if
% \xpackage{hyperref} is loaded, otherwise |pdf|.
%
% \subsection{Module \xmodule{pageattr}}
%
% This module allows to recover the content of the register
% \cs{pdfpageattr} and \cs{pdfpagesattr} in pdftex and the equivalent register
% in luatex. There is no user command. Programmer commands are
%
% \begin{quote}
% |\zref@pdfpageattr{|\meta{absolute page number}|}|
%
% |\zref@pdfpagesattr{|\meta{absolute page number}|}|
% \end{quote}
% \section{ToDo}\label{sec:todo}
%
% Among other things the following issues are left for future work:
% \begin{itemize}
% \item Other applications: autoref, hyperref, \dots
% \end{itemize}
%
% \section{Example}
%
%    \begin{macrocode}
%<*example>
%    \end{macrocode}
%
%    \begin{macrocode}
\documentclass{book}

\usepackage[ngerman]{babel}%

\usepackage[savepos,totpages,titleref,dotfill,counter,user]{zref}

%    \end{macrocode}
%
%    Chapters are wrapped inside \cs{ChapterStart} and
%    \cs{ChapterStop}. The first argument \verb|#1| of \cs{ChapterStart}
%    is used to form a label id \verb|chap:#1|. At the end of
%    the chapter another label is set by \cs{zref@wrapper@immediate}, because
%    otherwise at the end of document a deferred write would not
%    be written, because there is no page for shipout.
%
%    Also this example shows how chapter titles can be recorded.
%    A new property \verb|chaptitle| is declared and added to the
%    main property list. In \cs{ChapterStart} the current value
%    of the property is updated.
%    \begin{macrocode}
\makeatletter
\zref@newprop{chaptitle}{}
\zref@addprop{main}{chaptitle}

\newcommand*{\ChapterStart}[2]{%
  \cleardoublepage
  \def\current@chapid{#1}%
  \zref@setcurrent{chaptitle}{#2}%
  \chapter{#2}%
  \zlabel{chap:#1}%
}
\newcommand*{\ChapterStop}{%
  \cleardoublepage
  \zref@wrapper@immediate{%
    \zref@labelbyprops{chapend:\current@chapid}{abspage}%
  }%
}
%    \end{macrocode}
%    \cs{ChapterPages} calculates and returns the number of pages of the
%    referenced chapter.
%    \begin{macrocode}
\newcommand*{\ChapterPages}[1]{%
  \zrefused{chap:#1}%
  \zrefused{chapend:#1}%
  \number\numexpr
    \zref@extract{chapend:#1}{abspage}%
    -\zref@extract{chap:#1}{abspage}%
    +1\relax
}
\makeatother
%    \end{macrocode}
%    \begin{macrocode}
\begin{document}
%    \end{macrocode}
%    As exception we use \cs{makeatletter} here, because
%    this is just an example file that also should show some of
%    programmer's interface.
%    \begin{macrocode}
\makeatletter

\frontmatter
\zlabel{documentstart}

\begin{itemize}
\item
  The frontmatter part has
  \number\numexpr\zref@extract{chap:first}{abspage}-1\relax
  ~pages.
\item
  Chapter \zref{chap:first} has \ChapterPages{first} page(s).
\item
  Section \zref{hello} is on the
  \ifcase\numexpr
    \zref@extractdefault{hello}{page}{0}%
    -\zref@extractdefault{chap:first}{page}{0}%
    +1\relax
    ??\or first\or second\or third\or forth\fi
  ~page inside its chapter.
\item
  The document has
  \zref[abspage]{LastPage} pages.
  This number is \ifodd\ztotpages odd\else even\fi.
\item
  The last page is labeled with \zpageref{LastPage}.
\item
  The title of chapter \zref{chap:next} %
  is ``\zref[chaptitle]{chap:next}''.
\end{itemize}

\tableofcontents

\mainmatter
\ChapterStart{first}{First chapter}

%    \end{macrocode}
%    The user level commands should protect babel shorthands
%    where possible. On the other side, expandable extracting
%    macros are useful in calculations, see above the
%    examples with \cs{numexpr}.
%    \begin{macrocode}
\section{Test}
\zlabel{a"o}
Section \zref{a"o} on page
\zref@wrapper@babel\zref@extract{a"o}{page}.

Text.
\newpage

\section{Hello World}
\zlabel{hello}

\ChapterStop

\ChapterStart{next}{Next chapter with \emph{umlauts}: "a"o"u"s}

%    \end{macrocode}
%
%    Here an example follows that makes use of \pdfTeX's ``savepos''
%    feature. The position on the page is not known before the
%    page is constructed and shipped out. Therefore the position
%    ist stored in references and are available for calculations
%    in the next \hologo{LaTeX} compile run.
%    \begin{macrocode}
The width of the first column is
  \the\dimexpr \zposx{secondcol}sp - \zposx{firstcol}sp\relax,\\
the height difference of the two baselines is
  \the\dimexpr \zposy{firstcol}sp - \zposy{secondline}sp\relax:\\
\begin{tabular}{ll}
  \zsavepos{firstcol}Hello&\zsavepos{secondcol}World\\
  \zsavepos{secondline}Second line&foobar\\
\end{tabular}

%    \end{macrocode}
%    With \cs{zrefused} \hologo{LaTeX} is notified, if the references
%    are not yet available and \hologo{LaTeX} can generate the rerun
%    hint.
%    \begin{macrocode}
\zrefused{firstcol}
\zrefused{secondcol}
\zrefused{secondline}

\ChapterStop
%    \end{macrocode}
%    Test for module \cs{dotfill}.
%    \begin{macrocode}
\ChapterStart{dotfill}{Test for dotfill feature}
\newcommand*{\dftest}[1]{%
  #1&
  [\makebox[{#1}]{\dotfill}]&
  [\makebox[{#1}]{\zdotfill}]\\
}
\begin{tabular}{rll}
& [\verb|\dotfill|] & [\verb|\zdotfill|]\\
\dftest{0.43em}
\dftest{0.44em}
\dftest{0.45em}
\dftest{0.87em}
\dftest{0.88em}
\dftest{0.89em}
\dftest{1.31em}
\dftest{1.32em}
\dftest{1.33em}
\end{tabular}
\ChapterStop
\end{document}
%    \end{macrocode}
%
%    \begin{macrocode}
%</example>
%    \end{macrocode}
%
% \StopEventually{
% }
%
% \section{Implementation}
%
% \subsection{Package \xpackage{zref}}
%
% \subsubsection{Identification}
%
%    \begin{macrocode}
%<*package>
\NeedsTeXFormat{LaTeX2e}
\ProvidesPackage{zref}
  [2020-07-03 v2.32 A new reference scheme for LaTeX (HO)]%
%    \end{macrocode}
%
% \subsubsection{Load basic module}
%
%    \begin{macrocode}
\RequirePackage{zref-base}[2019/11/29]
%    \end{macrocode}
%    Abort package loading if \xpackage{zref-base} could not be
%    loaded successfully.
%    \begin{macrocode}
\@ifundefined{ZREF@base@ok}{\endinput}{}
%    \end{macrocode}
%
% \subsubsection{Process options}
%
%    Known modules are loaded and the release date is checked.
%    \begin{macrocode}
\def\ZREF@temp#1{%
  \DeclareOption{#1}{%
    \AtEndOfPackage{%
      \RequirePackage{zref-#1}[2019/11/29]%
    }%
  }%
}
\ZREF@temp{abspage}
\ZREF@temp{counter}
\ZREF@temp{dotfill}
\ZREF@temp{hyperref}
\ZREF@temp{lastpage}
\ZREF@temp{marks}
\ZREF@temp{nextpage}
\ZREF@temp{pageattr}
\ZREF@temp{pagelayout}
\ZREF@temp{perpage}
\ZREF@temp{runs}
\ZREF@temp{savepos}
\ZREF@temp{thepage}
\ZREF@temp{titleref}
\ZREF@temp{totpages}
\ZREF@temp{user}
\ZREF@temp{xr}
%    \end{macrocode}
%    \begin{macrocode}
\ProcessOptions\relax
%</package>
%    \end{macrocode}
%
%
% \subsection{Module \xmodule{base}}
%
% \subsubsection{Prefixes}
%
%    This package uses the following prefixes for macro names:
%    \begin{description}
%    \item[\cs{zref@}:] Macros of the programmer's interface.
%    \item[\cs{ZREF@}:] Internal macros.
%    \item[\cs{Z@L@}\metacs{listname}:]
%      The properties of the list \meta{listname}.
%    \item[\cs{Z@D@}\metacs{propname}:]
%      The default value for property \meta{propname}.
%    \item[\cs{Z@E@}\metacs{propname}:]
%      Extract function for property \meta{propname}.
%    \item[\cs{Z@X@}\metacs{propname}:]
%      Information whether a property value for property
%      \meta{propname} is expanded
%      immediately or at shipout time.
%    \item[\cs{Z@C@}\metacs{propname}:]
%      Current value of the property \meta{propname}.
%    \item[\cs{Z@R@}\metacs{labelname}:]
%      Data for reference \meta{labelname}.
%    \item[\cs{ZREF@org@}:] Original versions of patched commands.
%    \item[\cs{z}:] For macros in user land, defined if
%      module \xmodule{user} is set.
%    \end{description}
%    The following family names are used for keys defined
%    according to the \xpackage{keyval} package:
%    \begin{description}
%    \item[\texttt{ZREF@TR}:] Setup for module \xmodule{titleref}.
%    \end{description}
%
% \subsubsection{Identification}
%
%    \begin{macrocode}
%<*base>
\NeedsTeXFormat{LaTeX2e}
\ProvidesPackage{zref-base}%
  [2020-07-03 v2.32 Module base for zref (HO)]%
%    \end{macrocode}
%
% \subsubsection{Utilities}
%
%    \begin{macrocode}
\providecommand\IfFormatAtLeastTF{\@ifl@t@r\fmtversion}
\RequirePackage{ltxcmds}[2010/12/02]
\RequirePackage{infwarerr}[2010/04/08]
\RequirePackage{kvsetkeys}[2010/03/01]
\RequirePackage{kvdefinekeys}[2010/03/01]
\RequirePackage{pdftexcmds}[2010/04/01]
%    \end{macrocode}
%
%    \begin{macro}{\ZREF@name}
%    Several times the package name is used, thus we store it
%    in \cmd{\ZREF@name}.
%    \begin{macrocode}
\def\ZREF@name{zref}
%    \end{macrocode}
%    \end{macro}
%
%    \begin{macrocode}
\ltx@IfUndefined{protected}{%
  \RequirePackage{makerobust}[2006/03/18]%
%    \end{macrocode}
%    \begin{macro}{\ZREF@Robust}
%    \begin{macrocode}
  \def\ZREF@Robust#1#2{%
    \def\ZREF@temp{\MakeRobustcommand#2}%
    \afterassignment\ZREF@temp
    #1#2%
  }%
%    \end{macrocode}
%    \end{macro}
%    \begin{macrocode}
}{%
%    \end{macrocode}
%    \begin{macro}{\ZREF@Robust}
%    \begin{macrocode}
  \def\ZREF@Robust#1{%
    \protected#1%
  }%
%    \end{macrocode}
%    \end{macro}
%    \begin{macrocode}
}
%    \end{macrocode}
%    \begin{macro}{\ZREF@IfDefinable}
%    \begin{macrocode}
\def\ZREF@IfDefinable#1#2#3{%
  \@ifdefinable{#1}{%
    \ZREF@Robust{#2}#1#3%
  }%
}
%    \end{macrocode}
%    \end{macro}
%
%    \begin{macro}{\ZREF@UpdatePdfTeX}
%    \cs{ZREF@UpdatePdfTeX} is used as help message text in
%    error messages.
%    \begin{macrocode}
\def\ZREF@UpdatePdfTeX{Update pdfTeX.}
%    \end{macrocode}
%    \end{macro}
%
%    \begin{macro}{\ifZREF@found}
%    The following switch is usded in list processing.
%    \begin{macrocode}
\newif\ifZREF@found
%    \end{macrocode}
%    \end{macro}
%
%    \begin{macro}{\ZREF@patch}
%    Macro \cs{ZREF@patch} first checks the existence of
%    the command and safes it.
%    \begin{macrocode}
\def\ZREF@patch#1{%
  \ltx@IfUndefined{#1}{%
    \ltx@gobble
  }{%
    \expandafter\let\csname ZREF@org@#1\expandafter\endcsname
    \csname #1\endcsname
    \ltx@firstofone
  }%
}
%    \end{macrocode}
%    \end{macro}
%
% \subsubsection{Check for \hologo{eTeX}}
%
%    The use of \hologo{eTeX} should be standard nowadays for \hologo{LaTeX}.
%    We test for \hologo{eTeX} in order to use its features later.
%    \begin{macrocode}
\ltx@IfUndefined{eTeXversion}{%
  \PackageError\ZREF@name{%
    Missing support for eTeX; package is abandoned%
  }{%
    Use a TeX compiler that support eTeX and enable eTeX %
    in the format.%
  }%
  \endinput
}{}%
%    \end{macrocode}
%
%    \begin{macrocode}
\RequirePackage{etexcmds}[2007/09/09]
\ifetex@unexpanded
\else
  \PackageError\ZREF@name{%
    Missing e-TeX's \string\unexpanded.\MessageBreak
    Add \string\RequirePackage\string{etexcmds\string} before %
    \string\documentclass%
  }{%
    Probably you are using some package (e.g. ConTeXt) that %
    redefines \string\unexpanded%
  }%
  \expandafter\endinput
\fi
%    \end{macrocode}
%
% \subsubsection{Auxiliary file stuff}
%
%    We are using some commands in the \xfile{.aux} files.
%    However sometimes these auxiliary files are interpreted
%    by \hologo{LaTeX} processes that haven't loaded this package
%    (e.g. package \xpackage{xr}).
%    Therefore we provide dummy definitions.
%    \begin{macrocode}
\RequirePackage{auxhook}
\AddLineBeginAux{%
  \string\providecommand\string\zref@newlabel[2]{}%
}
%    \end{macrocode}
%    \begin{macro}{\ZREF@RefPrefix}
%    \begin{macrocode}
\def\ZREF@RefPrefix{Z@R}
%    \end{macrocode}
%    \end{macro}
%    \begin{macro}{\zref@newlabel}
%    For the implementation of \cs{zref@newlabel} we call
%    the same internal macro \cs{@newl@bel} that is used
%    in \cs{newlabel}. Thus we have for free:
%    \begin{itemize}
%    \item \cs{Z@R@}\metacs{labelname} is defined.
%    \item \hologo{LaTeX}'s check for multiple references.
%    \item \hologo{LaTeX}'s check for changed references.
%    \end{itemize}
%    \begin{macrocode}
\ZREF@Robust\edef\zref@newlabel{%
  \noexpand\@newl@bel{\ZREF@RefPrefix}%
}
%    \end{macrocode}
%    \end{macro}
%
% \subsubsection{Property lists}
%
%    \begin{macro}{\zref@newlist}
%    Property lists are stored as list of property names
%    enclosed in curly braces. \cs{zref@newlist}
%    creates a new list as empty list. Assignments to
%    property lists are global.
%    \begin{macrocode}
\ZREF@Robust\def\zref@newlist#1{%
  \zref@iflistundefined{#1}{%
    \@ifdefinable{Z@L@#1}{%
     \global\expandafter\let\csname Z@L@#1\endcsname\ltx@empty
     \PackageInfo\ZREF@name{New property list: #1}%
    }%
  }{%
    \PackageError\ZREF@name{%
      Property list `#1' already exists%
    }\@ehc
  }%
}
%    \end{macrocode}
%    \end{macro}
%
%    \begin{macro}{\zref@iflistundefined}
%    \cs{zref@iflistundefined} checks the existence of the
%    property list \verb|#1|. If the property list is present,
%    then \verb|#2| is executed and \verb|#3| otherwise.
%    \begin{macrocode}
\def\zref@iflistundefined#1{%
  \ltx@ifundefined{Z@L@#1}%
}
%    \end{macrocode}
%    \end{macro}
%    \begin{macro}{\zref@listexists}
%    \cs{zref@listexists} only executes \verb|#2| if the
%    property list \verb|#1| exists and raises an error
%    message otherwise.
%    \begin{macrocode}
\ZREF@Robust\def\zref@listexists#1{%
  \zref@iflistundefined{#1}{%
    \PackageError\ZREF@name{%
      Property list `#1' does not exist%
    }\@ehc
  }%
}
%    \end{macrocode}
%    \end{macro}
%
%    \begin{macro}{\zref@iflistcontainsprop}
%    \cs{zref@iflistcontainsprop} checks, whether a property \verb|#2| is
%    already present in a property list \verb|#1|.
%    \begin{macrocode}
\ZREF@Robust\def\zref@iflistcontainsprop#1#2{%
  \zref@iflistundefined{#1}{%
    \ltx@secondoftwo
  }{%
    \begingroup\expandafter\endgroup
    \expandafter\in@
    \csname#2\expandafter\expandafter\expandafter\endcsname
    \expandafter\expandafter\expandafter{\csname Z@L@#1\endcsname}%
    \csname ltx@\ifin@ first\else second\fi oftwo\endcsname
  }%
}
%    \end{macrocode}
%    \end{macro}
%
%    \begin{macro}{\zref@listforloop}
%    \begin{macrocode}
\def\zref@listforloop#1#2{%
  \zref@listexists{#1}{%
    \expandafter\expandafter\expandafter\@tfor
    \expandafter\expandafter\expandafter\zref@prop
    \expandafter\expandafter\expandafter:%
    \expandafter\expandafter\expandafter=%
    \csname Z@L@#1\endcsname
    \do{%
      \begingroup
        \escapechar=-1 %
        \edef\x{\endgroup
          \def\noexpand\zref@prop{%
            \expandafter\string\zref@prop
          }%
        }%
      \x
      #2\zref@prop
    }%
  }%
}
%    \end{macrocode}
%    \end{macro}
%
%    \begin{macro}{\zref@addprops}
%    \cs{zref@addprop} adds the properties \verb|#2| to the property
%    list \verb|#1|, if the property is not already in the list.
%    Otherwise a warning is given.
%    \begin{macrocode}
\ZREF@Robust\def\zref@addprops#1#2{%
  \zref@listexists{#1}{%
    \comma@parse{#2}{%
      \zref@propexists\comma@entry{%
        \zref@iflistcontainsprop{#1}\comma@entry{%
          \PackageWarning\ZREF@name{%
            Property `\comma@entry' is already in list `#1'%
          }%
        }{%
          \begingroup\expandafter\endgroup
          \expandafter\g@addto@macro
          \csname Z@L@#1\expandafter\endcsname
          \expandafter{\csname\comma@entry\endcsname}%
        }%
      }%
      \ltx@gobble
    }%
  }%
}
%    \end{macrocode}
%    \end{macro}
%    \begin{macro}{\zref@addprop}
%    \cs{zref@addprop} adds the property \verb|#2| to the property
%    list \verb|#1|, if the property is not already in the list.
%    Otherwise a warning is given.
%    \begin{macrocode}
\ZREF@Robust\def\zref@addprop#1#2{%
  \zref@listexists{#1}{%
    \zref@propexists{#2}{%
      \zref@iflistcontainsprop{#1}{#2}{%
        \PackageWarning\ZREF@name{%
          Property `#2' is already in list `#1'%
        }%
      }{%
        \begingroup\expandafter\endgroup
        \expandafter\g@addto@macro
        \csname Z@L@#1\expandafter\endcsname
        \expandafter{\csname#2\endcsname}%
      }%
    }%
  }%
}
%    \end{macrocode}
%    \end{macro}
%
%    \begin{macro}{\zref@localaddprops}
%    \begin{macrocode}
\ZREF@Robust\def\zref@localaddprops#1#2{%
  \zref@listexists{#1}{%
    \comma@parse{#2}{%
      \zref@propexists\comma@entry{%
        \zref@iflistcontainsprop{#1}\comma@entry{%
          \PackageWarning\ZREF@name{%
            Property `\comma@entry' is already in list `#1'%
          }%
        }{%
          \begingroup\expandafter\endgroup
          \expandafter\ltx@LocalAppendToMacro
          \csname Z@L@#1\expandafter\endcsname
          \expandafter{\csname\comma@entry\endcsname}%
        }%
      }%
      \ltx@gobble
    }%
  }%
}
%    \end{macrocode}
%    \end{macro}
%    \begin{macro}{\zref@localaddprop}
%    \begin{macrocode}
\ZREF@Robust\def\zref@localaddprop#1#2{%
  \zref@listexists{#1}{%
    \zref@propexists{#2}{%
      \zref@iflistcontainsprop{#1}{#2}{%
        \PackageWarning\ZREF@name{%
          Property `#2' is already in list `#1'%
        }%
      }{%
        \begingroup\expandafter\endgroup
        \expandafter\ltx@LocalAppendToMacro
        \csname Z@L@#1\expandafter\endcsname
        \expandafter{\csname#2\endcsname}%
      }%
    }%
  }%
}
%    \end{macrocode}
%    \end{macro}
%
%    \begin{macrocode}
\ltx@IfUndefined{pdf@strcmp}{%
%    \end{macrocode}
%    \begin{macro}{\zref@delprop}
%    \begin{macrocode}
  \ZREF@Robust\def\zref@delprop{%
    \ZREF@delprop\gdef
  }%
%    \end{macrocode}
%    \end{macro}
%    \begin{macro}{\zref@localdelprop}
%    \begin{macrocode}
  \ZREF@Robust\def\zref@localdelprop{%
    \ZREF@delprop\def
  }%
%    \end{macrocode}
%    \end{macro}
%    \begin{macro}{\ZREF@delprop}
%    \begin{macrocode}
  \def\ZREF@delprop#1#2#3{%
    \zref@listexists{#2}{%
      \begingroup
        \escapechar=-1 %
        \def\ZREF@param{#3}%
        \@onelevel@sanitize\ZREF@param
        \toks@{}%
        \expandafter\expandafter\expandafter\ZREF@@delprop
        \csname Z@L@#2\endcsname!%
      \expandafter\endgroup
      \expandafter#1\csname Z@L@#2\expandafter\endcsname
      \expandafter{%
        \the\toks@
      }%
    }%
  }%
%    \end{macrocode}
%    \end{macro}
%    \begin{macro}{\ZREF@@delprop}
%    \begin{macrocode}
  \def\ZREF@@delprop#1{%
    \expandafter\ZREF@@@delprop\expandafter{\string#1}#1%
  }%
%    \end{macrocode}
%    \end{macro}
%    \begin{macro}{\ZREF@@@delprop}
%    \begin{macrocode}
  \def\ZREF@@@delprop#1#2{%
    \ifx#2!%
    \else
      \def\ZREF@temp{#1}%
      \@onelevel@sanitize\ZREF@temp
      \ifx\ZREF@param\ZREF@temp
      \else
        \toks@\expandafter{%
          \the\expandafter\toks@\csname#1\endcsname
        }%
      \fi
      \expandafter\ZREF@@delprop
    \fi
  }%
%    \end{macrocode}
%    \end{macro}
%    \begin{macrocode}
}{%
%    \end{macrocode}
%    \begin{macro}{\zref@delprop}
%    \begin{macrocode}
  \ZREF@Robust\def\zref@delprop{%
    \ZREF@delprop\xdef
  }%
%    \end{macrocode}
%    \end{macro}
%    \begin{macro}{\zref@localdelprop}
%    \begin{macrocode}
  \ZREF@Robust\def\zref@localdelprop{%
    \ZREF@delprop\edef
  }%
%    \end{macrocode}
%    \end{macro}
%    \begin{macro}{\ZREF@delprop}
%    \begin{macrocode}
  \def\ZREF@delprop#1#2#3{%
    \zref@listexists{#2}{%
      \def\ZREF@param{#3}%
      \edef\ZREF@SavedEscapechar{\the\escapechar}%
      \escapechar=-1 %
      \expandafter#1\csname Z@L@#2%
      \expandafter\expandafter\expandafter\endcsname{%
        \expandafter\expandafter\expandafter\ZREF@@delprop
        \csname Z@L@#2\endcsname!%
      }%
      \escapechar=\ZREF@SavedEscapechar\relax
    }%
  }%
%    \end{macrocode}
%    \end{macro}
%    \begin{macro}{\ZREF@@delprop}
%    Caution: |#1| might be an \cs{if} or similar token.
%    \begin{macrocode}
  \def\ZREF@@delprop#1{%
    \expandafter\ZREF@@@delprop\expandafter{\string#1}#1%
  }%
%    \end{macrocode}
%    \end{macro}
%    \begin{macro}{\ZREF@@@delprop}
%    \begin{macrocode}
  \def\ZREF@@@delprop#1#2{%
    \ifx#2!%
    \else
      \ifnum\pdf@strcmp{#1}{\ZREF@param}=\ltx@zero
      \else
        \expandafter\noexpand\csname#1\endcsname
      \fi
      \expandafter\ZREF@@delprop
    \fi
  }%
%    \end{macrocode}
%    \end{macro}
%    \begin{macrocode}
}
%    \end{macrocode}
%
% \subsubsection{Properties}
%
%    \begin{macro}{\zref@ifpropundefined}
%    \cs{zref@ifpropundefined} checks the existence of the
%    property \verb|#1|. If the property is present,
%    then \verb|#2| is executed and \verb|#3| otherwise.
%    \begin{macrocode}
\def\zref@ifpropundefined#1{%
  \ltx@ifundefined{Z@E@#1}%
}
%    \end{macrocode}
%    \end{macro}
%    \begin{macro}{\zref@propexists}
%    Some macros rely on the existence of a property.
%    \cs{zref@propexists} only executes \verb|#2| if the
%    property \verb|#1| exists and raises an error
%    message otherwise.
%    \begin{macrocode}
\ZREF@Robust\def\zref@propexists#1{%
  \zref@ifpropundefined{#1}{%
    \PackageError\ZREF@name{%
      Property `#1' does not exist%
    }\@ehc
  }%
}
%    \end{macrocode}
%    \end{macro}
%
%    \begin{macro}{\zref@newprop}
%    A new property is declared by \cs{zref@newprop}, the
%    property name \meta{propname} is given in \verb|#1|. The
%    property is created and configured. If the star form
%    is given, then the expansion of the property value
%    is delayed to page shipout time, when the reference
%    is written to the \xfile{.aux} file.
%    \begin{description}
%    \item[\cs{Z@D@}\metacs{propname}:]
%      Stores the default value for this property.
%    \item[\cs{Z@E@}\metacs{propname}:]
%      Extract function.
%    \item[\cs{Z@X@}\metacs{propname}:]
%      Information whether the expansion of the property value
%      is delayed to shipout time.
%    \item[\cs{Z@C@}\metacs{propname}:]
%      Current value of the property.
%    \end{description}
%    \begin{macrocode}
\ZREF@Robust\def\zref@newprop{%
  \@ifstar{%
    \let\ZREF@X\noexpand
    \ZREF@newprop
  }{%
    \let\ZREF@X\ltx@empty
    \ZREF@newprop
  }%
}
%    \end{macrocode}
%    \end{macro}
%    \begin{macro}{\ZREF@newprop}
%    \begin{macrocode}
\def\ZREF@newprop#1{%
  \edef\ZREF@P{#1}%
  \@onelevel@sanitize\ZREF@P
  \begingroup
  \ifx\ZREF@P\ZREF@par
    \@PackageError\ZREF@name{%
      Invalid property name `\ZREF@P'%
    }{%
      The property name `par' is not allowed %
      because of internal reasons.%
      \MessageBreak
      \@ehc
    }%
    \def\ZREF@@newprop[##1]##2{\endgroup}%
  \else
    \zref@ifpropundefined\ZREF@P{%
      \endgroup
      \PackageInfo\ZREF@name{%
        New property: \ZREF@P
      }%
    }{%
      \@PackageError\ZREF@name{%
        Property `\ZREF@P' already exists%
      }\@ehc
      \def\ZREF@@newprop[##1]##2{\endgroup}%
    }%
  \fi
  \@ifnextchar[\ZREF@@newprop{\ZREF@@newprop[\zref@default]}%
}
%    \end{macrocode}
%    \end{macro}
%    \begin{macro}{\ZREF@par}
%    \begin{macrocode}
\def\ZREF@par{par}
\@onelevel@sanitize\ZREF@par
%    \end{macrocode}
%    \end{macro}
%    \begin{macro}{\ZREF@@newprop}
%    \begin{macrocode}
\def\ZREF@@newprop[#1]{%
  \global\@namedef{Z@D@\ZREF@P}{#1}%
  \global\expandafter\let\csname Z@X@\ZREF@P\endcsname\ZREF@X
  \begingroup\expandafter\endgroup
  \expandafter\ZREF@@@newprop\csname\ZREF@P\endcsname
  \expandafter\gdef\csname Z@C@\ZREF@P\endcsname{}%
  \zref@setcurrent\ZREF@P
}
\def\ZREF@@@newprop#1{%
  \expandafter
  \gdef\csname Z@E@\ZREF@P\endcsname##1#1##2##3\ZREF@nil{##2}%
}
%    \end{macrocode}
%    \end{macro}
%
%    \begin{macro}{\zref@showprop}
%    \begin{macrocode}
\ZREF@Robust\def\zref@showprop#1{%
  \zref@ifpropundefined{#1}{%
    \@PackageInfoNoLine{\ZREF@name}{%
      Show property `#1': <undefined>%
    }%
  }{%
    \begingroup
      \toks@\expandafter\expandafter\expandafter{%
        \csname Z@C@#1\endcsname
      }%
      \edef\ZREF@value{\the\toks@}%
      \ltx@onelevel@sanitize\ZREF@value
      \toks@\expandafter\expandafter\expandafter{%
        \csname Z@D@#1\endcsname
      }%
      \edef\ZREF@default{\the\toks@}%
      \ltx@onelevel@sanitize\ZREF@default
      \@PackageInfoNoLine{\ZREF@name}{%
        Show property `#1':\MessageBreak
        \expandafter\ifx\csname Z@X@#1\endcsname\ltx@empty
          Immediate %
        \else
          Delayed %
        \fi
        value: [\ZREF@value]\MessageBreak
        Default: [\ZREF@default]%
      }%
    \endgroup
  }%
}
%    \end{macrocode}
%    \end{macro}
%
%    \begin{macro}{\zref@setcurrent}
%    \cs{zref@setcurrent} sets the current value for a property.
%    \begin{macrocode}
\ZREF@Robust\def\zref@setcurrent#1#2{%
  \zref@propexists{#1}{%
    \expandafter\def\csname Z@C@#1\endcsname{#2}%
  }%
}
%    \end{macrocode}
%    \end{macro}
%
%    \begin{macro}{\ZREF@getcurrent}
%    \cs{zref@getcurrent} gets the current value for a property.
%    \begin{macrocode}
\def\ZREF@getcurrent#1{%
  \romannumeral0%
  \ltx@ifundefined{Z@C@#1}{%
    \ltx@space
  }{%
    \expandafter\expandafter\expandafter\ltx@space
    \csname Z@C@#1\endcsname
  }%
}
%    \end{macrocode}
%    \end{macro}
%    \begin{macro}{\ZREF@u@getcurrent}
%    \begin{macrocode}
\def\ZREF@wu@getcurrent#1{%
  \etex@unexpanded\expandafter\expandafter\expandafter{%
    \ZREF@getcurrent{#1}%
  }%
}
%    \end{macrocode}
%    \end{macro}
%    \begin{macro}{\zref@getcurrent}
%    \begin{macrocode}
\let\zref@getcurrent\ZREF@getcurrent
%    \end{macrocode}
%    \end{macro}
%
% \subsubsection{Reference generation}
%
%    \begin{macro}{\zref@label}
%    Label macro that uses the main property list.
%    \begin{macrocode}
\ZREF@Robust\def\zref@label#1{%
  \zref@labelbylist{#1}\ZREF@mainlist
}
%    \end{macrocode}
%    \end{macro}
%    \begin{macro}{\zref@labelbylist}
%    Label macro that stores the properties, specified in the
%    property list \verb|#2|.
%    \begin{macrocode}
\ZREF@Robust\def\zref@labelbylist#1#2{%
  \@bsphack
    \zref@listexists{#2}{%
      \expandafter\expandafter\expandafter\ZREF@label
      \expandafter\expandafter\expandafter{%
        \csname Z@L@#2\endcsname
      }{#1}%
    }%
  \@esphack
}
%    \end{macrocode}
%    \end{macro}
%    \begin{macro}{\zref@labelbyprops}
%    The properties are directly specified in a comma separated list.
%    \begin{macrocode}
\ZREF@Robust\def\zref@labelbyprops#1#2{%
  \@bsphack
    \begingroup
      \toks@{}%
      \comma@parse{#2}{%
        \zref@ifpropundefined\comma@entry{%
          \PackageWarning\ZREF@name{%
            Property `\comma@entry' is not known%
          }%
        }{%
          \toks@\expandafter{%
            \the\expandafter\toks@\csname\comma@entry\endcsname
          }%
        }%
        \ltx@gobble
      }%
    \expandafter\endgroup
    \expandafter\ZREF@label\expandafter{\the\toks@}{#1}%
  \@esphack
}
%    \end{macrocode}
%    \end{macro}
%    \begin{macro}{\zref@labelbykv}
%    \begin{macrocode}
\ZREF@Robust\def\zref@labelbykv#1#2{%
  \@bsphack
    \begingroup
      \let\Z@L@ZREF@temp\ltx@empty
      \kvsetkeys{ZREF@LABEL}{#1}%
      \ifZREF@immediate
        \expandafter\zref@wrapper@immediate\expandafter{%
          \expandafter\ZREF@label\expandafter{\Z@L@ZREF@temp}{#2}%
        }%
      \else
        \expandafter\ZREF@label\expandafter{\Z@L@ZREF@temp}{#2}%
      \fi
    \endgroup
  \@esphack
}
%    \end{macrocode}
%    \end{macro}
%    \begin{macrocode}
\kv@define@key{ZREF@LABEL}{prop}{%
  \edef\ZREF@param{#1}%
  \zref@propexists\ZREF@param{%
    \zref@iflistcontainsprop{ZREF@temp}\ZREF@param{}{%
      \begingroup\expandafter\endgroup
      \expandafter\ltx@LocalAppendToMacro
      \expandafter\Z@L@ZREF@temp
      \expandafter{\csname\ZREF@param\endcsname}%
    }%
  }%
}
\kv@define@key{ZREF@LABEL}{list}{%
  \zref@listforloop{#1}{%
    \zref@iflistcontainsprop{ZREF@temp}\zref@prop{}{%
      \begingroup\expandafter\endgroup
      \expandafter\ltx@LocalAppendToMacro
      \expandafter\Z@L@ZREF@temp
      \expandafter{\csname\zref@prop\endcsname}%
    }%
    \ltx@gobble
  }%
}
\kv@define@key{ZREF@LABEL}{delprop}{%
  \zref@propexists{#1}{%
    \zref@localdelprop{ZREF@temp}{#1}%
  }%
}
\kv@define@key{ZREF@LABEL}{immediate}[true]{%
  \edef\ZREF@param{#1}%
  \ifx\ZREF@param\ZREF@true
    \ZREF@immediatetrue
  \else
    \ifx\ZREF@param\ZREF@false
      \ZREF@immediatefalse
    \else
      \PackageWarning\ZREF@name{%
        Option `immediate' expects `true' or `false'.\MessageBreak
        Ignoring invalid value `\ZREF@param'%
      }%
    \fi
  \fi
}
%    \end{macrocode}
%    \begin{macro}{\ZREF@false}
%    \begin{macrocode}
\def\ZREF@false{false}
%    \end{macrocode}
%    \end{macro}
%    \begin{macro}{\ZREF@true}
%    \begin{macrocode}
\def\ZREF@true{true}
%    \end{macrocode}
%    \end{macro}
%
%    \begin{macrocode}
\kv@define@key{ZREF@LABEL}{values}[]{%
  \kv@parse{#1}{%
    \ifx\kv@value\relax
      \@PackageWarning\ZREF@name{%
        Missing value for property `\kv@key'%
      }%
      \expandafter\ltx@gobbletwo
    \else
      \expandafter\zref@setcurrent
    \fi
  }%
}
%    \end{macrocode}
%
%    \begin{macro}{\ifZREF@immediate}
%    The switch \cs{ifZREF@immediate} tells us, whether the
%    label should be written immediately or at page shipout time.
%    \cs{ZREF@label} need to be notified about this, because
%    it must disable the deferred execution of property values,
%    if the label is written immediately.
%    \begin{macrocode}
\newif\ifZREF@immediate
%    \end{macrocode}
%    \end{macro}
%    \begin{macro}{\zref@wrapper@immediate}
%    The argument of \cs{zref@wrapper@immediate} is executed inside
%    a group where \cs{write} is redefined by adding
%    \cs{immediate} before its execution. Also \cs{ZREF@label}
%    is notified via the switch \cs{ifZREF@immediate}.
%    \begin{macrocode}
\ZREF@Robust{\long\def}\zref@wrapper@immediate#1{%
  \begingroup
    \ZREF@immediatetrue
    \let\ZREF@org@write\write
    \def\write{\immediate\ZREF@org@write}%
    #1%
  \endgroup
}
%    \end{macrocode}
%    \end{macro}
%
%    \begin{macro}{\ZREF@label}
%    \cs{ZREF@label} writes the data in the \xfile{.aux} file.
%    \verb|#1| contains the list of valid properties,
%    \verb|#2| the name of the reference. In case of immediate
%    writing, the deferred execution of property values is
%    disabled. Also \thepage is made expandable in this case.
%    \begin{macrocode}
\def\ZREF@label#1#2{%
  \if@filesw
    \begingroup
      \ifZREF@immediate
        \let\ZREF@org@thepage\thepage
      \fi
      \protected@write\@auxout{%
        \ifZREF@immediate
          \let\thepage\ZREF@org@thepage
        \fi
        \let\ZREF@temp\ltx@empty
        \@tfor\ZREF@P:=#1\do{%
          \begingroup
            \escapechar=-1 %
            \edef\x{\endgroup
              \def\noexpand\ZREF@P{%
                \expandafter\string\ZREF@P
              }%
            }%
          \x
          \expandafter\ifx
              \csname
                \ifZREF@immediate
                  relax%
                \else
                  Z@X@\ZREF@P%
                \fi
              \endcsname
              \noexpand
            \expandafter\let\csname Z@C@\ZREF@P\endcsname\relax
          \fi
          \toks@\expandafter{\ZREF@temp}%
          \edef\ZREF@temp{%
            \the\toks@
            \ltx@backslashchar\ZREF@P{%
              \expandafter\noexpand\csname Z@C@\ZREF@P\endcsname
            }%
          }%
        }%
      }{%
        \string\zref@newlabel{#2}{\ZREF@temp}%
      }%
    \endgroup
  \fi
}
\def\ZREF@addtoks#1{%
  \toks@\expandafter\expandafter\expandafter{%
    \expandafter\the\expandafter\toks@#1%
  }%
}
%    \end{macrocode}
%    \end{macro}
%
% \subsubsection{Reference querying and extracting}
%
%    Design goal for the extracting macros is that the extraction
%    process is full expandable. Thus these macros can be used
%    in expandable contexts. But there are problems that cannot
%    be solved by full expandable macros:
%    \begin{itemize}
%    \item In standard \hologo{LaTeX} undefined references sets a flag
%      and generate a warning. Both actions are not expandable.
%    \item Babel's support for its shorthand uses commands that
%      use non-expandable assignments. However currently there
%      is hope, that primitives are added to \pdfTeX\ that
%      allows the detection of contexts. Then the shorthand
%      can detect, if they are executed inside \cs{csname}
%      and protect themselves automatically.
%    \end{itemize}
%    \begin{macro}{\zref@ifrefundefined}
%    If a reference \verb|#1| is undefined, then macro
%    \cs{zref@ifrefundefined} calls \verb|#2| and \verb|#3| otherwise.
%    \begin{macrocode}
\def\zref@ifrefundefined#1{%
  \ltx@ifundefined{Z@R@#1}%
}
%    \end{macrocode}
%    \end{macro}
%
%    \begin{macro}{\zifrefundefined}
%    If a reference \verb|#1| is undefined, then macro
%    \cs{zref@ifrefundefined} calls \verb|#2| and \verb|#3| otherwise.
%    Also the reference is marked used.
%    \begin{macrocode}
\ZREF@IfDefinable\zifrefundefined\def{%
  #1{%
    \zref@wrapper@babel\ZREF@ifrefundefined{#1}%
  }%
}
%    \end{macrocode}
%    \end{macro}
%    \begin{macro}{\ZREF@ifrefundefined}
%    \begin{macrocode}
\def\ZREF@ifrefundefined#1{%
  \zref@refused{#1}%
  \zref@ifrefundefined{#1}%
}
%    \end{macrocode}
%    \end{macro}
%
%    \begin{macro}{\zref@refused}
%    The problem with undefined references is addressed by
%    the macro \cs{zref@refused}. This can be used outside
%    the expandable context. In case of an undefined reference
%    the flag is set to notify \hologo{LaTeX} and a warning is given.
%    \begin{macrocode}
\ZREF@Robust\def\zref@refused#1{%
  \zref@wrapper@babel\ZREF@refused{#1}%
}
%    \end{macrocode}
%    \end{macro}
%    \begin{macro}{\ZREF@refused}
%    \begin{macrocode}
\def\ZREF@refused#1{%
  \zref@ifrefundefined{#1}{%
    \protect\G@refundefinedtrue
    \@latex@warning{%
      Reference `#1' on page \thepage \space undefined%
    }%
  }{}%
}
%    \end{macrocode}
%    \end{macro}
%
%    \begin{macro}{\zref@ifrefcontainsprop}
%    \cs{zref@ifrefcontainsprop} looks, if the reference \verb|#1|
%    has the property \verb|#2| and calls then \verb|#3| and
%    \verb|#4| otherwise.
%    \begin{macrocode}
\def\zref@ifrefcontainsprop#1#2{%
  \zref@ifrefundefined{#1}{%
    \ltx@secondoftwo
  }{%
    \expandafter\ZREF@ifrefcontainsprop
    \csname Z@E@#2\expandafter\endcsname
    \csname#2\expandafter\expandafter\expandafter\endcsname
    \expandafter\expandafter\expandafter{%
      \csname Z@R@#1\endcsname
    }%
  }%
}
\def\ZREF@ifrefcontainsprop#1#2#3{%
  \expandafter\ifx\expandafter\ZREF@novalue
  #1#3#2\ZREF@novalue\ZREF@nil\ltx@empty
    \expandafter\ltx@secondoftwo
  \else
    \expandafter\ltx@firstoftwo
  \fi
}
\def\ZREF@novalue{\ZREF@NOVALUE}
%    \end{macrocode}
%    \end{macro}
%
%    \begin{macro}{\zref@extract}
%    \cs{zref@extract} is an abbreviation for the case
%    that the default of the property is used as default
%    value.
%    \begin{macrocode}
\def\ZREF@extract#1#2{%
  \romannumeral0%
  \ltx@ifundefined{Z@D@#2}{%
    \expandafter\ltx@space\zref@default
  }{%
    \expandafter\expandafter\expandafter\ZREF@@extract
    \expandafter\expandafter\expandafter{%
      \csname Z@D@#2\endcsname
    }{#1}{#2}%
  }%
}
%    \end{macrocode}
%    \end{macro}
%    \begin{macro}{\ZREF@@extract}
%    \begin{macrocode}
\def\ZREF@@extract#1#2#3{%
  \expandafter\expandafter\expandafter\ltx@space
  \zref@extractdefault{#2}{#3}{#1}%
}
%    \end{macrocode}
%    \end{macro}
%    \begin{macro}{\ZREF@wu@extract}
%    \begin{macrocode}
\def\ZREF@wu@extract#1#2{%
  \etex@unexpanded\expandafter\expandafter\expandafter{%
    \ZREF@extract{#1}{#2}%
  }%
}
%    \end{macrocode}
%    \end{macro}
%    \begin{macro}{\zref@extract}
%    \begin{macrocode}
\let\zref@extract\ZREF@extract
%    \end{macrocode}
%    \end{macro}
%
%    \begin{macro}{\ZREF@extractdefault}
%    The basic extracting macro is \cs{zref@extractdefault} with
%    the reference name in \verb|#1|, the property in \verb|#2| and
%    the default value in \verb|#3| in case for problems.
%    \begin{macrocode}
\def\ZREF@extractdefault#1#2#3{%
  \romannumeral0%
  \zref@ifrefundefined{#1}\ltx@firstoftwo{%
    \zref@ifpropundefined{#2}\ltx@firstoftwo\ltx@secondoftwo
  }{%
    \ltx@space
    #3%
  }{%
    \expandafter\expandafter\expandafter\ltx@space
    \csname Z@E@#2\expandafter\expandafter\expandafter\endcsname
    \csname Z@R@#1\expandafter\endcsname
    \csname#2\endcsname{#3}\ZREF@nil
  }%
}
%    \end{macrocode}
%    \end{macro}
%    \begin{macro}{\ZREF@wu@extractdefault}
%    \begin{macrocode}
\def\ZREF@wu@extractdefault#1#2#3{%
  \etex@unexpanded\expandafter\expandafter\expandafter{%
    \ZREF@extractdefault{#1}{#2}{#3}%
  }%
}
%    \end{macrocode}
%    \end{macro}
%    \begin{macro}{\zref@extractdefault}
%    \begin{macrocode}
\let\zref@extractdefault\ZREF@extractdefault
%    \end{macrocode}
%    \end{macro}
%
%    \begin{macro}{\zref@def@extract}
%    \begin{macrocode}
\ZREF@Robust\def\zref@def@extract#1{%
  \zref@wrapper@babel{\ZREF@def@extract{#1}}%
}
%    \end{macrocode}
%    \end{macro}
%    \begin{macro}{\ZREF@def@extract}
%    \begin{macrocode}
\def\ZREF@def@extract#1#2#3{%
  \zref@refused{#2}%
  \expandafter\expandafter\expandafter\def
  \expandafter\expandafter\expandafter#1%
  \expandafter\expandafter\expandafter{%
    \zref@extract{#2}{#3}%
  }%
}
%    \end{macrocode}
%    \end{macro}
%    \begin{macro}{\zref@def@extractdefault}
%    \begin{macrocode}
\ZREF@Robust\def\zref@def@extractdefault#1{%
  \zref@wrapper@babel{\ZREF@def@extractdefault{#1}}%
}
%    \end{macrocode}
%    \end{macro}
%    \begin{macro}{\ZREF@def@extractdefault}
%    \begin{macrocode}
\def\ZREF@def@extractdefault#1#2#3#4{%
  \zref@refused{#2}%
  \expandafter\expandafter\expandafter\def
  \expandafter\expandafter\expandafter#1%
  \expandafter\expandafter\expandafter{%
    \zref@extractdefault{#2}{#3}{#4}%
  }%
}
%    \end{macrocode}
%    \end{macro}
%
%    \begin{macro}{\ZREF@wrapper@unexpanded}
%    \begin{macrocode}
\ZREF@Robust{\long\def}\ZREF@wrapper@unexpanded#1{%
  \let\zref@wrapper@unexpanded\ltx@firstofone
  \let\zref@getcurrent\ZREF@wu@getcurrent
  \let\zref@extractdefault\ZREF@wu@extractdefault
  \let\zref@extract\ZREF@wu@extract
  #1%
  \let\zref@wrapper@unexpanded\ZREF@wrapper@unexpanded
  \let\zref@getcurrent\ZREF@getcurrent
  \let\zref@extractdefault\ZREF@extractdefault
  \let\zref@extract\ZREF@extract
}
%    \end{macrocode}
%    \end{macro}
%    \begin{macro}{\zref@wrapper@unexpanded}
%    \begin{macrocode}
\ltx@IfUndefined{etex@unexpanded}{%
  \let\zref@wrapper@unexpanded\ltx@firstofone
}{%
  \let\zref@wrapper@unexpanded\ZREF@wrapper@unexpanded
}
%    \end{macrocode}
%    \end{macro}
%
% \subsubsection{Compatibility with \xpackage{babel}}
%
%    \begin{macro}{\zref@wrapper@babel}
%    \begin{macrocode}
\ZREF@Robust{\long\def}\zref@wrapper@babel#1#2{%
  \ifcsname if@safe@actives\endcsname
    \expandafter\ltx@firstofone
  \else
    \expandafter\ltx@secondoftwo
  \fi
  {%
    \if@safe@actives
      \expandafter\ltx@secondoftwo
    \else
      \expandafter\ltx@firstoftwo
    \fi
    {%
      \begingroup
        \csname @safe@activestrue\endcsname
        \edef\x{#2}%
      \expandafter\endgroup
      \expandafter\ZREF@wrapper@babel\expandafter{\x}{#1}%
    }%
  }{%
    #1{#2}%
  }%
}
\long\def\ZREF@wrapper@babel#1#2{%
  #2{#1}%
}
%    \end{macrocode}
%    \end{macro}
%
% \subsubsection{Unique counter support}
%
%    \begin{macro}{\zref@require@unique}
%    Generate the counter |zref@unique| if the counter does not
%    already exist.
%    \begin{macrocode}
\ZREF@Robust\def\zref@require@unique{%
  \@ifundefined{c@zref@unique}{%
    \begingroup
      \let\@addtoreset\ltx@gobbletwo
      \newcounter{zref@unique}%
    \endgroup
%    \end{macrocode}
%    \end{macro}
%
%    \begin{macro}{\thezref@unique}
%    \cs{thezref@unique} is used for automatically generated unique
%    labelnames.
%    \begin{macrocode}
    \renewcommand*{\thezref@unique}{%
      zref@\number\c@zref@unique
    }%
  }{}%
}
%    \end{macrocode}
%    \end{macro}
%
% \subsubsection{Utilities}
%
%    \begin{macro}{\ZREF@number}
%    \begin{macrocode}
\ltx@IfUndefined{numexpr}{%
  \def\ZREF@number#1{\number#1}%
}{%
  \def\ZREF@number#1{\the\numexpr(#1)\relax}%
}
%    \end{macrocode}
%    \end{macro}
%
% \subsubsection{Setup}
%
%    \begin{macro}{\zref@setdefault}
%    Standard \hologo{LaTeX} prints ``??'' in bold face if a reference
%    is not known. \cs{zref@default} holds the text that is
%    printed in case of unknown references and is used, if
%    the default was not specified during the definition
%    of the new property by \cs{ref@newprop}.
%    The global default value can be set by \cs{zref@setdefault}.
%    \begin{macrocode}
\ZREF@Robust\def\zref@setdefault#1{%
  \def\zref@default{#1}%
}
%    \end{macrocode}
%    \end{macro}
%    \begin{macro}{\zref@default}
%    Now we initialize \cs{zref@default} with the same value
%    that \hologo{LaTeX} uses for its undefined references.
%    \begin{macrocode}
\zref@setdefault{%
  \nfss@text{\reset@font\bfseries ??}%
}
%    \end{macrocode}
%    \end{macro}
%
% \paragraph{Main property list.}
%
%    \begin{macro}{\zref@setmainlist}
%    The name of the default property list is stored in
%    \cs{ZREF@mainlist} and can be set by \cs{zref@setmainlist}.
%    \begin{macrocode}
\ZREF@Robust\def\zref@setmainlist#1{%
  \def\ZREF@mainlist{#1}%
}
\zref@setmainlist{main}
%    \end{macrocode}
%    \end{macro}
%    Now we create the list.
%    \begin{macrocode}
\zref@newlist\ZREF@mainlist
%    \end{macrocode}
%
% \paragraph{Main properties.}
%
%    The two properties \texttt{default} and \texttt{page}
%    are created and added to the main property list.
%    They store the data that
%    standard \hologo{LaTeX} uses in its references created by
%    \cs{label}.
%    \begin{description}
%    \item[\texttt{default}] the apperance of the latest
%      counter that is incremented by \cs{refstepcounter}
%    \item[\texttt{page}] the apperance of the page counter
%    \end{description}
%    \begin{macrocode}
\zref@newprop{default}{\@currentlabel}
\zref@newprop*{page}{\thepage}
\zref@addprops\ZREF@mainlist{default,page}
%    \end{macrocode}
%
% \paragraph{Properties}
%
%    \begin{macro}{\ZREF@NewPropAnchor}
%    \begin{macrocode}
\def\ZREF@NewPropAnchor{%
  \zref@newprop{anchor}{%
    \ltx@ifundefined{@currentHref}{}{\@currentHref}%
  }%
  \global\let\ZREF@NewPropAnchor\relax
}
%    \end{macrocode}
%    \end{macro}
%    \begin{macro}{\zref@titleref@current}
%    Later we will redefine the section and caption macros
%    to catch the current title and remember the value in
%    \cs{zref@titleref@current}.
%    \begin{macro}{\ZREF@NewPropTitle}
%    \begin{macrocode}
\def\ZREF@NewPropTitle{%
  \gdef\zref@titleref@current{}%
  \zref@newprop{title}{\zref@titleref@current}%
  \global\let\ZREF@NewPropTitle\relax
}
%    \end{macrocode}
%    \end{macro}
%    \end{macro}
%    \begin{macro}{\ZREF@NewPropTheotype}
%    \begin{macrocode}
\def\ZREF@NewPropTheotype{%
  \zref@newprop{theotype}{}%
  \global\let\ZREF@NewPropTheotype\relax
}
%    \end{macrocode}
%    \end{macro}
%    \begin{macro}{\ZREF@NewPropPageValue}
%    \begin{macrocode}
\def\ZREF@NewPropPageValue{%
  \zref@newprop*{pagevalue}[0]{\number\c@page}%
  \global\let\ZREF@NewPropPageValue\relax
}
%    \end{macrocode}
%    \end{macro}
%
% \paragraph{Mark successful loading}
%
%    \begin{macrocode}
\let\ZREF@base@ok=Y
%</base>
%    \end{macrocode}
%
%
% \subsection{Module \xmodule{user}}
%
%    \begin{macrocode}
%<*user>
\NeedsTeXFormat{LaTeX2e}
\ProvidesPackage{zref-user}%
  [2020-07-03 v2.32 Module user for zref (HO)]%
\RequirePackage{zref-base}[2019/11/29]
\ifx\ZREF@base@ok Y%
\else
  \expandafter\endinput
\fi
%    \end{macrocode}
%
%    Module \xmodule{user} enables a small user interface.
%    All macros are prefixed by \cs{z}.
%
%    First we define the pendants to the standard \hologo{LaTeX} referencing
%    commands \cs{label}, \cs{ref}, and \cs{pageref}.
%
%    \begin{macro}{\zlabel}
%    Similar to \cs{label} the macro \cs{zlabel} writes a reference
%    entry in the \xfile{.aux} file. The main property list is used.
%    Also we add the babel patch. The \cs{label} command can also be
%    used inside section titles, but it must not go into the
%    table of contents. Therefore we have to check this situation.
%    \begin{macrocode}
\newcommand*\zlabel{%
  \ifx\label\ltx@gobble
    \expandafter\ltx@gobble
  \else
    \expandafter\zref@wrapper@babel\expandafter\zref@label
  \fi
}%
%    \end{macrocode}
%    \end{macro}
%
%    \begin{macro}{\zkvlabel}
%    \begin{macrocode}
\newcommand*{\zkvlabel}[1]{%
  \ifx\label\ltx@gobble
    \expandafter\ltx@gobblethree
  \fi
  \zref@wrapper@babel{\zref@labelbykv{#1}}%
}%
%    \end{macrocode}
%    \end{macro}
%
%    \begin{macro}{\zref}
%    Macro \cs{zref} is the corresponding macro for \cs{ref}.
%    Also it provides an optional argument in order to
%    select another property.
%    \begin{macrocode}
\newcommand*{\zref}[2][default]{% robust because of optional argument
  \zref@propexists{#1}{%
    \zref@wrapper@babel\ZREF@zref{#2}{#1}%
  }%
}%
\def\ZREF@zref#1{%
  \zref@refused{#1}%
  \zref@extract{#1}%
}%
%    \end{macrocode}
%    \end{macro}
%
%    \begin{macro}{\zpageref}
%    For macro \cs{zpageref} we just call \cs{zref}
%    with property \verb|page|.
%    \begin{macrocode}
\ZREF@IfDefinable\zpageref\def{%
  {\zref[page]}%
}
%    \end{macrocode}
%    \end{macro}
%
%    \begin{macro}{\zrefused}
%    For the following expandible user macros \cs{zrefused}
%    should be used to notify \hologo{LaTeX} in case of undefined
%    references.
%    \begin{macrocode}
\ZREF@IfDefinable\zrefused\def{%
  {\zref@refused}%
}
%    \end{macrocode}
%    \end{macro}
%
%    \begin{macrocode}
%</user>
%    \end{macrocode}
%
%
% \subsection{Module \xmodule{abspage}}
%
%    \begin{macrocode}
%<*abspage>
\NeedsTeXFormat{LaTeX2e}
\ProvidesPackage{zref-abspage}%
  [2020-07-03 v2.32 Module abspage for zref (HO)]%
\RequirePackage{zref-base}[2019/11/29]
\ifx\ZREF@base@ok Y%
\else
  \expandafter\endinput
\fi
%    \end{macrocode}
%
%    Module \xmodule{abspage} adds a new property \verb|abspage|
%    to the \verb|main| property list for absolute page numbers.
%    These are recorded by the help of package \xpackage{atbegshi}.
%    \begin{macrocode}
\RequirePackage{atbegshi}[2011/10/05]%
%    \end{macrocode}
%    The counter \verb|abspage| must not go in the clear list
%    of \verb|@ckpt| that is used to set counters in
%    \xfile{.aux} files of included \hologo{TeX} files.
%    \begin{macrocode}
\begingroup
  \let\@addtoreset\ltx@gobbletwo
  \newcounter{abspage}%
\endgroup
\setcounter{abspage}{0}%
\AtBeginShipout{%
  \stepcounter{abspage}%
}%
\zref@newprop*{abspage}[0]{\the\c@abspage}%
\zref@addprop\ZREF@mainlist{abspage}%
%    \end{macrocode}
%    Note that counter \verb|abspage| shows the
%    previous page during page processing. Before
%    shipout the counter is incremented.
%    Thus the property is correctly written with
%    deferred writing. If the counter is written
%    using \cs{zref@wrapper@immediate}, then the number
%    is too small by one.
%
%    \begin{macrocode}
%</abspage>
%    \end{macrocode}
%
%
% \subsection{Module \xmodule{counter}}
%
%    \begin{macrocode}
%<*counter>
\NeedsTeXFormat{LaTeX2e}
\ProvidesPackage{zref-counter}%
  [2020-07-03 v2.32 Module counter for zref (HO)]%
\RequirePackage{zref-base}[2019/11/29]
\ifx\ZREF@base@ok Y%
\else
  \expandafter\endinput
\fi
%    \end{macrocode}
%
%    For features such as \xpackage{hyperref}'s \cs{autoref}
%    we need the name of the counter. The property
%    \verb|counter| is defined and added to the main property
%    list. Starting with \LaTeX\ 2020-10-01 the proper can use
%    \@currentcounter. In older formats
%    \cs{refstepcounter} has to be patched but this can fail in some
%    cases, see issue \#5.
%    \begin{macrocode}
\@ifl@t@r\fmtversion{2020-10-01}
 {
  \zref@newprop{counter}{\@currentcounter}
  \zref@addprop\ZREF@mainlist{counter}
 }
 {
  \zref@newprop{counter}{}
  \zref@addprop\ZREF@mainlist{counter}
  \AtBeginDocument{%
    \ZREF@patch{refstepcounter}{%
      \def\refstepcounter#1{%
        \zref@setcurrent{counter}{#1}%
        \ZREF@org@refstepcounter{#1}%
      }%
    }%
  }
 }
%    \end{macrocode}
%    \begin{macrocode}
%</counter>
%    \end{macrocode}
%
%
% \subsection{Module \xmodule{lastpage}}
%
%    \begin{macrocode}
%<*lastpage>
\NeedsTeXFormat{LaTeX2e}
\ProvidesPackage{zref-lastpage}%
  [2020-07-03 v2.32 Module lastpage for zref (HO)]%
\RequirePackage{zref-base}[2019/11/29]
\RequirePackage{zref-abspage}[2019/11/29]
\RequirePackage{atveryend}[2009/12/07]
\ifx\ZREF@base@ok Y%
\else
  \expandafter\endinput
\fi
%    \end{macrocode}
%
%    The module \xmodule{lastpage} implements the service
%    of package \xpackage{lastpage} by setting a reference
%    \verb|LastPage| at the end of the document.
%    If module \xmodule{abspage} is given, also the absolute
%    page number is available, because the properties of
%    the main property list are used.
%    \begin{macrocode}
\zref@newlist{LastPage}
\AfterLastShipout{%
  \if@filesw
    \begingroup
      \advance\c@page\m@ne
      \toks@\expandafter\expandafter\expandafter{%
        \expandafter\Z@L@main
        \Z@L@LastPage
      }%
      \expandafter\zref@wrapper@immediate\expandafter{%
        \expandafter\ZREF@label\expandafter{\the\toks@}{LastPage}%
      }%
    \endgroup
  \fi
}
%    \end{macrocode}
%    \begin{macro}{\zref@iflastpage}
%    \begin{macrocode}
\def\zref@iflastpage#1{%
  \ifnum\zref@extractdefault{#1}{abspage}{-1}=%
        \zref@extractdefault{LastPage}{abspage}{-2} %
    \expandafter\ltx@firstoftwo
  \else
    \expandafter\ltx@secondoftwo
  \fi
}
%    \end{macrocode}
%    \end{macro}
%    \begin{macro}{\ziflastpage}
%    \begin{macrocode}
\ZREF@IfDefinable\ziflastpage\def{%
  {\zref@wrapper@babel\ZREF@iflastpage}%
}
%    \end{macrocode}
%    \end{macro}
%    \begin{macro}{ZREF@iflastpage}
%    \begin{macrocode}
\def\ZREF@iflastpage#1{%
  \zref@refused{LastPage}%
  \zref@refused{#1}%
  \zref@iflastpage{#1}%
}
%    \end{macrocode}
%    \end{macro}
%    \begin{macrocode}
%</lastpage>
%    \end{macrocode}
%
%
% \subsection{Module \xmodule{thepage}}
%
%    \begin{macrocode}
%<*thepage>
\NeedsTeXFormat{LaTeX2e}
\ProvidesPackage{zref-thepage}%
  [2020-07-03 v2.32 Module thepage for zref (HO)]%
\RequirePackage{zref-base}[2019/11/29]
\ifx\ZREF@base@ok Y%
\else
  \expandafter\endinput
\fi
%    \end{macrocode}
%
%    \begin{macrocode}
\RequirePackage{atbegshi}[2011/10/05]
\RequirePackage{zref-abspage}[2019/11/29]
%    \end{macrocode}
%    \begin{macrocode}
\zref@newlist{thepage}
\zref@addprop{thepage}{page}
\ZREF@NewPropPageValue
%    \end{macrocode}
%    \begin{macro}{\zref@thepage@atbegshi@hook}
%    \begin{macrocode}
\let\zref@thepage@atbegshi@hook\ltx@empty
%    \end{macrocode}
%    \end{macro}
%    \begin{macrocode}
\zref@addprop{thepage}{pagevalue}
\AtBeginShipout{%
  \AtBeginShipoutAddToBox{%
    \zref@thepage@atbegshi@hook
    \zref@labelbylist{thepage\the\value{abspage}}{thepage}%
  }%
}
%    \end{macrocode}
%    \begin{macro}{\zref@thepage@name}
%    \begin{macrocode}
\ltx@IfUndefined{numexpr}{%
  \def\zref@thepage@name#1{thepage\number#1}%
}{%
  \def\zref@thepage@name#1{thepage\the\numexpr#1}%
}
%    \end{macrocode}
%    \end{macro}
%    \begin{macro}{\zref@thepage}
%    \begin{macrocode}
\def\zref@thepage#1{%
  \zref@extract{\zref@thepage@name{#1}}{page}%
}%
%    \end{macrocode}
%    \end{macro}
%    \begin{macro}{\zref@thepage@refused}
%    \begin{macrocode}
\ZREF@Robust\def\zref@thepage@refused#1{%
  \zref@refused{\zref@thepage@name{#1}}%
}%
%    \end{macrocode}
%    \end{macro}
%    \begin{macro}{\zthepage}
%    \begin{macrocode}
\ZREF@IfDefinable\zthepage\def{%
  #1{%
    \zref@thepage@refused{#1}%
    \zref@thepage{#1}%
  }%
}
%    \end{macrocode}
%    \end{macro}
%    \begin{macrocode}
%</thepage>
%    \end{macrocode}
%
%
% \subsection{Module \xmodule{nextpage}}
%
%    \begin{macrocode}
%<*nextpage>
\NeedsTeXFormat{LaTeX2e}
\ProvidesPackage{zref-nextpage}%
  [2020-07-03 v2.32 Module nextpage for zref (HO)]%
\RequirePackage{zref-base}[2019/11/29]
\ifx\ZREF@base@ok Y%
\else
  \expandafter\endinput
\fi
%    \end{macrocode}
%
%    \begin{macrocode}
\RequirePackage{zref-abspage}[2019/11/29]
\RequirePackage{zref-thepage}[2019/11/29]
\RequirePackage{zref-lastpage}[2019/11/29]
\RequirePackage{uniquecounter}[2009/12/18]
%    \end{macrocode}
%    \begin{macrocode}
\UniqueCounterNew{znextpage}

\newcommand*{\znextpagesetup}{%
  \afterassignment\ZREF@np@setup@i
  \def\ZREF@np@call@unknown##1%
}
\def\ZREF@np@setup@i{%
  \afterassignment\ZREF@np@setup@ii
  \def\ZREF@np@call@nonext##1%
}
\def\ZREF@np@setup@ii{%
  \def\ZREF@np@call@next##1%
}
\def\ZREF@np@call@unknown#1{#1}
\def\ZREF@np@call@nonext#1{#1}
\def\ZREF@np@call@next#1{#1}
\ZREF@IfDefinable\znextpage\def{%
  {\UniqueCounterCall{znextpage}{\ZREF@nextpage}}%
}%
\newcommand*{\znonextpagename}{}
\newcommand*{\zunknownnextpagename}{\Z@D@page}
\def\ZREF@nextpage#1{%
  \begingroup
    \def\ZREF@refname@this{zref@np#1}%
    \zref@labelbyprops\ZREF@refname@this{abspage}%
    \chardef\ZREF@call=0 % unknown
    \ZREF@ifrefundefined\ZREF@refname@this{%
    }{%
      \edef\ZREF@pagenum@this{%
        \zref@extractdefault\ZREF@refname@this{abspage}{0}%
      }%
      \edef\ZREF@refname@next{%
        \zref@thepage@name{%
          \the\numexpr\ZREF@pagenum@this+1%
        }%
      }%
      \ifnum\ZREF@pagenum@this>0 %
        \ZREF@ifrefundefined{LastPage}{%
          \zref@ifrefundefined\ZREF@refname@next{%
          }{%
            \chardef\ZREF@call=2 % next page
          }%
        }{%
          \edef\ZREF@pagenum@last{%
            \zref@extractdefault{LastPage}{abspage}{0}%
          }%
          \ifnum\ZREF@pagenum@this<\ZREF@pagenum@last\ltx@space
            \ZREF@ifrefundefined\ZREF@refname@next{%
            }{%
              \chardef\ZREF@call=2 % next page
            }%
          \else
            \ifnum\ZREF@pagenum@this=\ZREF@pagenum@this\ltx@space
              \chardef\ZREF@call=1 % no next page
            \fi
          \fi
        }%
      \fi
    }%
    \edef\x{%
      \endgroup
      \ifcase\ZREF@call
        \noexpand\ZREF@np@call@unknown{%
          \noexpand\zunknownnextpagename
        }%
      \or
        \noexpand\ZREF@np@call@nonext{%
          \noexpand\znonextpagename
        }%
      \else
        \noexpand\ZREF@np@call@next{%
          \noexpand\zref@extract{\ZREF@refname@next}{page}%
        }%
      \fi
    }%
  \x
}
%    \end{macrocode}
%    \begin{macrocode}
%</nextpage>
%    \end{macrocode}
%
%
% \subsection{Module \xmodule{totpages}}
%
%    \begin{macrocode}
%<*totpages>
\NeedsTeXFormat{LaTeX2e}
\ProvidesPackage{zref-totpages}%
  [2020-07-03 v2.32 Module totpages for zref (HO)]%
\RequirePackage{zref-base}[2019/11/29]
\ifx\ZREF@base@ok Y%
\else
  \expandafter\endinput
\fi
%    \end{macrocode}
%
%    The absolute page number of the last page is the total
%    page number.
%    \begin{macrocode}
\RequirePackage{zref-abspage}[2019/11/29]
\RequirePackage{zref-lastpage}[2019/11/29]
%    \end{macrocode}
%
%    \begin{macro}{\ztotpages}
%    Macro \cs{ztotpages} contains the number of pages. It can be
%    used inside expandable calculations. It expands to zero if
%    the reference is not yet available.
%    \begin{macrocode}
\newcommand*{\ztotpages}{%
  \zref@extractdefault{LastPage}{abspage}{0}%
}
%    \end{macrocode}
%    \end{macro}
%    Also we mark the reference |LastPage| as used:
%    \begin{macrocode}
\AtBeginDocument{%
  \zref@refused{LastPage}%
}
%    \end{macrocode}
%    \begin{macrocode}
%</totpages>
%    \end{macrocode}
%
%
% \subsection{Module \xmodule{pagelayout}}
%
%    \begin{macrocode}
%<*pagelayout>
\NeedsTeXFormat{LaTeX2e}
\ProvidesPackage{zref-pagelayout}%
  [2020-07-03 v2.32 Module pagelayout for zref (HO)]%
\RequirePackage{zref-base}[2019/11/29]
\ifx\ZREF@base@ok Y%
\else
  \expandafter\endinput
\fi
%    \end{macrocode}
%
%    \begin{macrocode}
\RequirePackage{zref-thepage}[2019/11/29]
\RequirePackage{iftex}[2019/11/07]%
\RequirePackage{atveryend}[2010/03/24]
%    \end{macrocode}
%
%
% \subsubsection{Define layout properties}
%
%    \begin{macrocode}
\def\ZREF@temp#1{%
  \begingroup
    \escapechar=-1 %
  \ltx@ifundefined{\string#1}{\endgroup}{%
    \edef\x{%
      \endgroup
      \noexpand\zref@newprop*{\string#1}%
                             [\noexpand\number\noexpand#1]% hash-ok
                             {\noexpand\number\noexpand#1}%
      \noexpand\zref@addprop{thepage}{\string#1}%
    }%
    \x
  }%
}
\ZREF@temp\mag
\ZREF@temp\paperwidth
\ZREF@temp\paperheight
\ZREF@temp\stockwidth % memoir.cls, crop.sty
\ZREF@temp\stockheight % memoir.cls, crop.sty
\ZREF@temp\mediawidth % VTeX
\ZREF@temp\mediaheight % VTeX
\ifluatex
\zref@newprop*{pdfvorigin}%
                             [\number\pdfvariable vorigin]% hash-ok
                             {\number\pdfvariable vorigin}%
\zref@addprop{thepage}{pdfvorigin}
\zref@newprop*{pdfhorigin}%
                             [\number\pdfvariable horigin]% hash-ok
                             {\number\pdfvariable horigin}%
\zref@addprop{thepage}{pdfhorigin}
\zref@newprop*{pdfpageheight}%
                             [\number\pageheight]% hash-ok
                             {\number\pageheight}%
\zref@addprop{thepage}{pdfpageheight}
\zref@newprop*{pdfpagewidth}%
                             [\number\pagewidth]% hash-ok
                             {\number\pagewidth}%
\zref@addprop{thepage}{pdfpagewidth}
\else
\ZREF@temp\pdfpagewidth
\ZREF@temp\pdfpageheight
\ZREF@temp\pdfhorigin
\ZREF@temp\pdfvorigin
\fi
\ZREF@temp\hoffset
\ZREF@temp\voffset
\ZREF@temp\topmargin
\ZREF@temp\oddsidemargin
\ZREF@temp\evensidemargin
\ZREF@temp\textwidth
\ZREF@temp\textheight
\ZREF@temp\headheight
\ZREF@temp\headsep
\ZREF@temp\footskip
\ZREF@temp\marginparwidth
\ZREF@temp\marginparsep
\ZREF@temp\columnwidth
\ZREF@temp\columnsep
\ZREF@temp\trimedge % memoir.cls
\ZREF@temp\spinemargin % memoir.cls
\ZREF@temp\foremargin % memoir.cls
\ZREF@temp\trimtop % memoir.cls
\ZREF@temp\uppermargin % memoir.cls
\ZREF@temp\headmargin % memoir.cls
%    \end{macrocode}
%
%    \begin{macrocode}
\IfFormatAtLeastTF{2020/10/01}
 {
   \zref@newprop*{outputboxwd}[0pt]{\ShipoutBoxWidth}
   \zref@newprop*{outputboxht}[0pt]{\ShipoutBoxHeight}
   \zref@newprop*{outputboxdp}[0pt]{\ShipoutBoxDepth}
 }
 {
   \zref@newprop*{outputboxwd}[0pt]{\AtBeginShipoutBoxWidth}
   \zref@newprop*{outputboxht}[0pt]{\AtBeginShipoutBoxHeight}
   \zref@newprop*{outputboxdp}[0pt]{\AtBeginShipoutBoxDepth}
 }
%    \end{macrocode}
%    \begin{macrocode}
\zref@addprops{thepage}{outputboxwd,outputboxht,outputboxdp}
%    \end{macrocode}
%
%    \begin{macro}{\ifZREF@pl@list}
%    \begin{macrocode}
\ltx@newif\ifZREF@pl@list
%    \end{macrocode}
%    \end{macro}
%    \begin{macro}{\zref@listpagelayout}
%    \begin{macrocode}
\ZREF@IfDefinable\zlistpagelayout\def{%
  {\global\ZREF@pl@listtrue}%
}
%    \end{macrocode}
%    \end{macro}
%    \begin{macro}{\ZREF@pl@AfterLastShipout}
%    \begin{macrocode}
\def\ZREF@pl@AfterLastShipout{%
  \ifZREF@pl@list
    \edef\ZREF@page@max{\the\value{abspage}}%
    \ltx@ifundefined{ZREF@org@testdef}{%
      \let\ZREF@org@testdef\@testdef
      \def\@testdef##1##2##3{%
        \ZREF@org@testdef{##1}{##2}{##3}%
        \def\ZREF@temp{##1}%
        \ifx\ZREF@temp\ZREF@RefPrefix
          \expandafter\gdef\csname##1@##2\endcsname{##3}%
        \fi
      }%
    }{}%
    \AtVeryEndDocument{\ZREF@pl@AtVeryEnd}%
  \fi
}
%    \end{macrocode}
%    \end{macro}
%    \begin{macro}{\ZREF@pl@AtVeryEnd}
%    \begin{macrocode}
\def\ZREF@pl@AtVeryEnd{%
  \begingroup
    \toks@{Page layout parameters:\MessageBreak}%
    \count@=1 %
    \ZREF@pl@ListPage
    \edef\x{\endgroup
      \noexpand\@PackageInfoNoLine{zref-pagelayout}{\the\toks@}%
    }%
  \x
}
%    \end{macrocode}
%    \end{macro}
%    \begin{macro}{\ZREF@pl@ListPage}
%    \begin{macrocode}
\def\ZREF@pl@ListPage{%
  \edef\x{%
    \toks@={%
      \the\toks@
      Page \the\count@:\noexpand\MessageBreak
      \zref@ifrefundefined{thepage\the\count@}{}{%
        \ltx@space\ltx@space mag = %
        \zref@extract{thepage\the\count@}{mag}%
        \noexpand\MessageBreak
        \ZREF@pl@ListEntry{paperwidth}%
        \ZREF@pl@ListEntry{paperheight}%
        \ZREF@pl@ListEntry{stockwidth}%
        \ZREF@pl@ListEntry{stockheight}%
        \ZREF@pl@ListEntry{mediawidth}%
        \ZREF@pl@ListEntry{mediaheight}%
        \ZREF@pl@ListEntry{pdfpagewidth}%
        \ZREF@pl@ListEntry{pdfpageheight}%
        \ZREF@pl@ListEntry{pdfhorigin}%
        \ZREF@pl@ListEntry{pdfvorigin}%
        \ZREF@pl@ListEntry{hoffset}%
        \ZREF@pl@ListEntry{voffset}%
        \ZREF@pl@ListEntry{topmargin}%
        \ZREF@pl@ListEntry{oddsidemargin}%
        \ZREF@pl@ListEntry{evensidemargin}%
        \ZREF@pl@ListEntry{textwidth}%
        \ZREF@pl@ListEntry{textheight}%
        \ZREF@pl@ListEntry{headheight}%
        \ZREF@pl@ListEntry{headsep}%
        \ZREF@pl@ListEntry{footskip}%
        \ZREF@pl@ListEntry{marginparwidth}%
        \ZREF@pl@ListEntry{marginparsep}%
        \ZREF@pl@ListEntry{columnwidth}%
        \ZREF@pl@ListEntry{columnsep}%
        \ZREF@pl@ListEntry{trimedge}%
        \ZREF@pl@ListEntry{spinemargin}%
        \ZREF@pl@ListEntry{foremargin}%
        \ZREF@pl@ListEntry{trimtop}%
        \ZREF@pl@ListEntry{uppermargin}%
        \ZREF@pl@ListEntry{headmargin}%
      }%
    }%
  }\x
  \ifnum\ZREF@page@max>\count@
    \advance\count@ by\ltx@one
  \else
    \expandafter\ltx@gobble
  \fi
  \ZREF@pl@ListPage
}
%    \end{macrocode}
%    \end{macro}
%    \begin{macro}{\ZREF@pl@ListEntry}
%    \begin{macrocode}
\def\ZREF@pl@ListEntry#1{%
  \zref@ifpropundefined{#1}{%
  }{%
    \zref@ifrefcontainsprop{thepage\the\count@}{#1}{%
      \ltx@space\ltx@space#1 = %
      \zref@extract{thepage\the\count@}{#1}sp = %
      \the\dimexpr\zref@extract{thepage\the\count@}{#1}sp\relax
      \noexpand\MessageBreak
    }{}%
  }%
}
%    \end{macrocode}
%    \end{macro}
%    \begin{macrocode}
\AfterLastShipout{%
  \ZREF@pl@AfterLastShipout
}
%    \end{macrocode}
%
%    \begin{macrocode}
%</pagelayout>
%    \end{macrocode}
%
%
% \subsection{Module \xmodule{pageattr}}
%
%    \begin{macrocode}
%<*pageattr>
\NeedsTeXFormat{LaTeX2e}
\ProvidesPackage{zref-pageattr}%
  [2020-07-03 v2.32 Module pageattr for zref (HO)]%
\RequirePackage{zref-base}[2019/11/29]
\ifx\ZREF@base@ok Y%
\else
  \expandafter\endinput
\fi
%    \end{macrocode}
%
%    \begin{macrocode}
\RequirePackage{iftex}[2019/11/07]%
%    \end{macrocode}
%
%
%    \begin{macrocode}
\let\ZREF@temp=N%
\ifluatex
\expandafter\@firstoftwo
\else
\expandafter\@secondoftwo
\fi
{%luatex
 \RequirePackage{zref-thepage}[2019/11/29]
 \RequirePackage{zref-lastpage}[2019/11/29]%
 \zref@newprop*{pdfpageattr}[]{\zref@hex{\the\pdfvariable pageattr}}%
 \zref@addprop{thepage}{pdfpageattr}%
 \zref@newprop*{pdfpagesattr}[]{\zref@hex{\the\pdfvariable pagesattr}}%
 \zref@addprop{LastPage}{pdfpagesattr}%
 \let\ZREF@temp=Y%
}
{%other
 \ltx@IfUndefined{pdfpageattr}{%
  \@PackageInfoNoLine{zref-pageattr}{%
    \string\pdfpageattr\space is not available%
  }%
  \def\zref@pdfpageattr#1{}%
  \def\zref@pdfpageattr@used#1{}%
 }{%
  \RequirePackage{zref-thepage}[2019/11/29]%
  \zref@newprop*{pdfpageattr}[]{\zref@hex{\the\pdfpageattr}}%
  \zref@addprop{thepage}{pdfpageattr}%
  \let\ZREF@temp=Y%
 }
 \ltx@IfUndefined{pdfpagesattr}{%
  \@PackageInfoNoLine{zref-pageattr}{%
    \string\pdfpagesattr\space is not available%
  }%
  \def\zref@pdfpagesattr{}%
  \def\zref@pdfpagesattr@used{}%
 }{%
  \RequirePackage{zref-lastpage}[2019/11/29]%
  \zref@newprop*{pdfpagesattr}[]{\zref@hex{\the\pdfpagesattr}}%
  \zref@addprop{LastPage}{pdfpagesattr}%
  \let\ZREF@temp=Y%
 }%
}%
%    \end{macrocode}
%    \begin{macrocode}
\ifx\ZREF@temp N%
  \expandafter\endinput
\fi
%    \end{macrocode}
%    \begin{macrocode}
\RequirePackage{zref-abspage}[2019/11/29]
\RequirePackage{atveryend}[2010/03/24]
\RequirePackage{pdftexcmds}[2010/04/01]
\let\ZREF@temp=Y%
\ltx@IfUndefined{pdf@escapehex}{\let\ZREF@temp=N}{}
\ltx@IfUndefined{pdf@unescapehex}{\let\ZREF@temp=N}{}
\ifx\ZREF@temp N%
  \let\zref@hex\ltx@firstofone
  \let\zref@unhex\ltx@firstofone
\else
  \let\zref@hex\pdf@escapehex
  \let\zref@unhex\pdf@unescapehex
\fi
%    \end{macrocode}
%
%    \begin{macro}{\ifZREF@pa@list}
%    \begin{macrocode}
\ltx@newif\ifZREF@pa@list
%    \end{macrocode}
%    \end{macro}
%    \begin{macro}{\zref@listpageattr}
%    \begin{macrocode}
\ZREF@IfDefinable\zlistpageattr\def{%
  {\ZREF@pa@listtrue}%
}
%    \end{macrocode}
%    \end{macro}
%    \begin{macro}{\ZREF@pa@AfterLastShipout}
%    \begin{macrocode}
\def\ZREF@pa@AfterLastShipout{%
  \ifZREF@pa@list
    \edef\ZREF@page@max{\the\value{abspage}}%
    \ltx@ifundefined{ZREF@org@testdef}{%
      \let\ZREF@org@testdef\@testdef
      \def\@testdef##1##2##3{%
        \ZREF@org@testdef{##1}{##2}{##3}%
        \def\ZREF@temp{##1}%
        \ifx\ZREF@temp\ZREF@RefPrefix
          \expandafter\xdef\csname##1@##2\endcsname{##3}%
        \fi
      }%
    }{}%
    \AtVeryEndDocument{\ZREF@pa@AtVeryEnd}%
  \fi
}
%    \end{macrocode}
%    \end{macro}
%    \begin{macro}{\ZREF@pa@AtVeryEnd}
%    \begin{macrocode}
\let\ZREF@temp=Y%
\ltx@IfUndefined{pdfpageattr}{}{\let\ZREF@temp=N}
\ifluatex \let\ZREF@temp=N \fi
\ifx\ZREF@temp Y
 \expandafter\@firstoftwo
\else
 \expandafter\@secondoftwo
\fi
 {%
  \def\ZREF@pa@AtVeryEnd{}%
 }
 {%
  \def\ZREF@pa@AtVeryEnd{%
    \begingroup
      \toks@{List of \ltx@backslashchar
             \ifluatex pdfvariable\else pdf\fi
             pdfpageattr:\MessageBreak}%
      \count@=1 %
      \ZREF@pa@ListPage
      \edef\x{\endgroup
        \noexpand\@PackageInfoNoLine{zref-pageattr}{%
          \the\toks@
        }%
      }%
    \x
  }%
%    \end{macrocode}
%    \end{macro}
%    \begin{macro}{\zref@pageattr}
%    \begin{macrocode}
\def\zref@pdfpageattr#1{%
  \zref@unhex{%
    \zref@extract{thepage\ZREF@number{#1}}{pdfpageattr}%
  }%
}
% compability, \zref@pageattr was defined in older versions
\let\zref@pageattr\zref@pdfpageattr
%    \end{macrocode}
%    \end{macro}
%    \begin{macro}{\zref@pageattr@used}
%    \begin{macrocode}
\ZREF@Robust\def\zref@pageattr@used#1{%
  \zref@refused{thepage\ZREF@number{#1}}%
}
%    \end{macrocode}
%    \end{macro}
%    \begin{macro}{\ZREF@pa@ListPage}
%    \begin{macrocode}
  \def\ZREF@pa@ListPage{%
    \edef\x{%
      \toks@={%
        \the\toks@
        Page \the\count@:%
        \noexpand\MessageBreak
        \zref@ifrefundefined{thepage\the\count@}{}{%
          <<\zref@pdfpageattr\count@>>%
          \noexpand\MessageBreak
        }%
      }%
    }\x
    \ifnum\ZREF@page@max>\count@
      \advance\count@ by\ltx@one
    \else
      \expandafter\ltx@gobble
    \fi
    \ZREF@pa@ListPage
  }%
}
%    \end{macrocode}
%    \end{macro}
%    \begin{macrocode}
\let\ZREF@temp=Y%
\ltx@IfUndefined{pdfpagesattr}{}{\let\ZREF@temp=N}
\ifluatex \let\ZREF@temp=N \fi
\ifx\ZREF@temp N
 \expandafter\@firstofone
\fi
 {%
%    \end{macrocode}
%    \begin{macro}{\zref@pdfpagesattr}
%    \begin{macrocode}
  \def\zref@pdfpagesattr{%
    \zref@unhex{%
      \zref@extract{LastPage}{pdfpagesattr}%
    }%
  }%
%    \end{macrocode}
%    \end{macro}
%    \begin{macro}{\zref@pdfpagesattr@used}
%    \begin{macrocode}
  \ZREF@Robust\def\zref@pdfpagesattr@used{%
    \zref@refused{LastPage}%
  }%
%    \end{macrocode}
%    \end{macro}
%    \begin{macrocode}
  \ltx@LocalAppendToMacro\ZREF@pa@AtVeryEnd{%
    \@PackageInfoNoLine{zref-pageattr}{%
      \ltx@backslashchar
      \ifluatex pdfvariable\else pdf\fi
      pagesattr:\MessageBreak
      <<\zref@pdfpagesattr>>%
      \MessageBreak
    }%
  }%
}
\AfterLastShipout{%
  \ZREF@pa@AfterLastShipout
}
%    \end{macrocode}
%    \begin{macrocode}
%</pageattr>
%    \end{macrocode}
%
%
% \subsection{Module \xmodule{marks}}
%
%    \begin{macrocode}
%<*marks>
\NeedsTeXFormat{LaTeX2e}
\ProvidesPackage{zref-marks}%
  [2020-07-03 v2.32 Module marks for zref (HO)]%
\RequirePackage{zref-base}[2019/11/29]
\ifx\ZREF@base@ok Y%
\else
  \expandafter\endinput
\fi
%    \end{macrocode}
%
%    \begin{macrocode}
\newcommand*{\zref@marks@register}[3][]{%
  \edef\ZREF@TempName{#1}%
  \edef\ZREF@TempNum{\ZREF@number{#2}}%
  \ifnum\ZREF@TempNum<\ltx@zero %
    \PackageError\ZREF@name{%
      \string\zref@marks@register\ltx@space is called with invalid%
      \MessageBreak
      marks register number (\ZREF@TempNum)%
    }{%
      Use `0' or the command, defined by \string\newmarks.\MessageBreak
      \@ehc
    }%
  \else
    \ifx\ZREF@TempName\ltx@empty
      \edef\ZREF@TempName{mark\romannumeral\ZREF@TempNum}%
    \else
      \edef\ZREF@TempName{marks\ZREF@TempName}%
    \fi
    \ZREF@MARKS@DefineProp{top}%
    \ZREF@MARKS@DefineProp{first}%
    \ZREF@MARKS@DefineProp{bot}%
    \kv@parse{#3}{%
      \ifx\kv@value\relax
        \def\kv@value{top,first,bot}%
      \fi
      \edef\ZREF@temp{\expandafter\ltx@car\kv@key X\@nil}%
      \ifx\ZREF@temp\ZREF@STAR
        \edef\kv@key{\expandafter\ltx@cdr\kv@key\@nil}%
        \zref@newlist\kv@key
      \fi
      \expandafter\comma@parse\expandafter{\kv@value}{%
        \ifcase0\ifx\comma@entry\ZREF@NAME@top 1\else
                \ifx\comma@entry\ZREF@NAME@first 1\else
                \ifx\comma@entry\ZREF@NAME@bot 1\fi\fi\fi\ltx@space
          \PackageWarning{zref-marks}{%
            Use `top', `first' or `bot' for the list values%
            \MessageBreak
            in the third argument of \string\zref@marks@register.%
            \MessageBreak
            Ignoring unkown value `\comma@entry'%
          }%
        \else
          \zref@addprop{\kv@key}{\comma@entry\ZREF@TempName}%
        \fi
        \ltx@gobble
      }%
      \ltx@gobbletwo
    }%
  \fi
}
\def\ZREF@STAR{*}
\def\ZREF@NAME@top{top}
\def\ZREF@NAME@first{first}
\def\ZREF@NAME@bot{bot}
\def\ZREF@MARKS@DefineProp#1{%
  \zref@ifpropundefined{#1\ZREF@TempName}{%
    \ifnum\ZREF@TempNum=\ltx@zero
      \begingroup
        \edef\x{\endgroup
          \noexpand\zref@newprop*{#1\ZREF@TempName}[]{%
            \expandafter\noexpand\csname#1mark\endcsname
          }%
        }%
      \x
    \else
      \begingroup
        \edef\x{\endgroup
          \noexpand\zref@newprop*{#1\ZREF@TempName}[]{%
            \expandafter\noexpand\csname#1marks\endcsname
            \ZREF@TempNum
          }%
        }%
      \x
    \fi
  }{%
    \PackageWarning{zref-marks}{%
      \string\zref@marks@register\ltx@space does not generate the%
      \MessageBreak
      new property `#1\ZREF@TempName', because\MessageBreak
      it is already defined%
    }%
  }%
}
%    \end{macrocode}
%    \begin{macrocode}
%</marks>
%    \end{macrocode}
%
% \subsection{Module \xmodule{runs}}
%
%    This module does not use the label-reference-system.
%    The reference changes with each \hologo{LaTeX}~run and would
%    force a rerun warning always.
%    \begin{macrocode}
%<*runs>
\NeedsTeXFormat{LaTeX2e}
\ProvidesPackage{zref-runs}%
  [2020-07-03 v2.32 Module runs for zref (HO)]%
%    \end{macrocode}
%
%    \begin{macro}{\zruns}
%    \begin{macrocode}
\providecommand*{\zruns}{0}%
\AtBeginDocument{%
  \edef\zruns{\number\numexpr\zruns+1}%
  \begingroup
    \def\on@line{}%
    \PackageInfo{zref-runs}{LaTeX runs: \zruns}%
    \if@filesw
      \immediate\write\@mainaux{%
        \string\gdef\string\zruns{\zruns}%
      }%
    \fi
  \endgroup
}
%    \end{macrocode}
%    \end{macro}
%
%    \begin{macrocode}
%</runs>
%    \end{macrocode}
%
%
% \subsection{Module \xmodule{perpage}}
%
%    \begin{macrocode}
%<*perpage>
\NeedsTeXFormat{LaTeX2e}
\ProvidesPackage{zref-perpage}%
  [2020-07-03 v2.32 Module perpage for zref (HO)]%
\RequirePackage{zref-base}[2019/11/29]
\ifx\ZREF@base@ok Y%
\else
  \expandafter\endinput
\fi
%    \end{macrocode}
%
%    This module resets a counter at page boundaries.
%    Because of the asynchronous output routine
%    page counter properties cannot be asked directly,
%    references are necessary.
%
%    For detecting changed pages module \xmodule{abspage}
%    is loaded.
%    \begin{macrocode}
\RequirePackage{zref-abspage}[2019/11/29]
%    \end{macrocode}
%
%    We group the properties for the needed references
%    in the property list |perpage|. The property |pagevalue|
%    records the correct value of the page counter.
%    \begin{macrocode}
\ZREF@NewPropPageValue
\zref@newlist{perpage}
\zref@addprops{perpage}{abspage,page,pagevalue}
%    \end{macrocode}
%    The page value, known by the reference mechanism, will be
%    stored in counter |zpage|.
%    \begin{macrocode}
\newcounter{zpage}
%    \end{macrocode}
%    Counter |zref@unique| helps in generating unique reference
%    names.
%    \begin{macrocode}
\zref@require@unique
%    \end{macrocode}
%    In order to be able to reset the counter, we hook here
%    into \cs{stepcounter}. In fact two nested hooks are used
%    to allow other packages to use the first hook at the beginning
%    of \cs{stepcounter}.
%    \begin{macrocode}
\let\ZREF@org@stepcounter\stepcounter
\def\stepcounter#1{%
  \ifcsname @stepcounterhook@#1\endcsname
    \csname @stepcounterhook@#1\endcsname
  \fi
  \ZREF@org@stepcounter{#1}%
}
%    \end{macrocode}
%    \cs{@stpelt} must be adapted due to the change in latex
%    2015-01, see https://github.com/ho-tex/zref/issues/26
%    \begin{macrocode}
\let\ZREF@org@@stpelt\@stpelt
\def\@stpelt#1{%
  \ifcsname ZREF@perpage@#1\endcsname
    \begingroup
      \let\stepcounter\ZREF@org@stepcounter
      \ZREF@org@@stpelt{#1}%
    \endgroup
    \expandafter\ltx@gobbletwo
  \fi
  \ZREF@org@@stpelt{#1}%
}
%    \end{macrocode}
%
%    \begin{macro}{\zmakeperpage}
%    Makro \cs{zmakeperpage} resets a counter at each page
%    break. It uses the same syntax and semantics as
%    \cs{MakePerPage} from package \xpackage{perpage} \cite{perpage}.
%    The initial start value can be given by the optional
%    argument. Default is one that means after the first
%    \cs{stepcounter} on a new page the counter starts with
%    one.
%    \begin{macrocode}
\ZREF@IfDefinable\zmakeperpage\def{%
  {%
    \@ifnextchar[\ZREF@makeperpage@opt{\ZREF@@makeperpage[\ltx@zero]}%
  }%
}
%    \end{macrocode}
%    \end{macro}
%    We hook before the counter is incremented in \cs{stepcounter},
%    package \xpackage{perpage} afterwards. Thus a little
%    calculation is necessary.
%    \begin{macrocode}
\def\ZREF@makeperpage@opt[#1]{%
  \begingroup
    \edef\x{\endgroup
      \noexpand\ZREF@@makeperpage[\number\numexpr#1-1\relax]%
    }%
  \x
}
%    \end{macrocode}
%    \begin{macrocode}
\def\ZREF@@makeperpage[#1]#2{%
  \@ifundefined{@stepcounterhook@#2}{%
    \expandafter\gdef\csname @stepcounterhook@#2\endcsname{}%
  }{}%
  \expandafter\gdef\csname ZREF@perpage@#2\endcsname{%
    \ZREF@@perpage@step{#2}{#1}%
  }%
  \expandafter\g@addto@macro\csname @stepcounterhook@#2\endcsname{%
    \ifcsname ZREF@perpage@#2\endcsname
      \csname ZREF@perpage@#2\endcsname
    \fi
  }%
}
%    \end{macrocode}
%    \begin{macro}{\ZREF@@perpage@step}
%    The heart of this module follows.
%    \begin{macrocode}
\def\ZREF@@perpage@step#1#2{%
%    \end{macrocode}
%    First the reference is generated.
%    \begin{macrocode}
  \global\advance\c@zref@unique\ltx@one
  \begingroup
    \expandafter
    \zref@labelbylist\expandafter{\thezref@unique}{perpage}%
%    \end{macrocode}
%    The \cs{expandafter} commands are necessary, because \cs{ZREF@temp}
%    is also used inside of \cs{zref@labelbylist}.
%
%    The evaluation of the reference follows. If the reference
%    is not yet kwown, we use the page counter as approximation.
%    \begin{macrocode}
    \zref@ifrefundefined\thezref@unique{%
      \global\c@zpage=\c@page
      \global\let\thezpage\thepage
      \expandafter\xdef\csname ZREF@abspage@#1\endcsname{%
        \number\c@abspage
      }%
    }{%
%    \end{macrocode}
%    The reference is used to set \cs{thezpage} and counter |zpage|.
%    \begin{macrocode}
      \global\c@zpage=\zref@extract\thezref@unique{pagevalue}\relax
      \xdef\thezpage{\noexpand\zref@extract{\thezref@unique}{page}}%
      \expandafter\xdef\csname ZREF@abspage@#1\endcsname{%
        \zref@extractdefault\thezref@unique
            {abspage}{\number\c@abspage}%
      }%
    }%
%    \end{macrocode}
%    Page changes are detected by a changed absolute page number.
%    \begin{macrocode}
    \expandafter\ifx\csname ZREF@abspage@#1\expandafter\endcsname
                    \csname ZREF@currentabspage@#1\endcsname
    \else
      \global\csname c@#1\endcsname=#2\relax
      \global\expandafter\let
          \csname ZREF@currentabspage@#1\expandafter\endcsname
          \csname ZREF@abspage@#1\endcsname
    \fi
  \endgroup
}
%    \end{macrocode}
%    \end{macro}
%    \begin{macro}{\zunmakeperpage}
%    Macro \cs{zunmakeperpage} cancels the effect of \cs{zmakeperpage}.
%    \begin{macrocode}
\ZREF@IfDefinable\zunmakeperpage\def{%
  #1{%
    \global\expandafter
    \let\csname ZREF@perpage@#1\endcsname\@undefined
  }%
}
%    \end{macrocode}
%    \end{macro}
%
%    \begin{macrocode}
%</perpage>
%    \end{macrocode}
%
%
% \subsection{Module \xmodule{titleref}}
%
%    \begin{macrocode}
%<*titleref>
\NeedsTeXFormat{LaTeX2e}
\ProvidesPackage{zref-titleref}%
  [2020-07-03 v2.32 Module titleref for zref (HO)]%
\RequirePackage{zref-base}[2019/11/29]
\ifx\ZREF@base@ok Y%
\else
  \expandafter\endinput
\fi
\RequirePackage{gettitlestring}[2009/12/08]
%    \end{macrocode}
%
% \subsubsection{Implementation}
%    \begin{macrocode}
\RequirePackage{keyval}
%    \end{macrocode}
%
%    This module makes section and caption titles available
%    for the reference system. It uses some of the ideas
%    of package \xpackage{nameref} and \xpackage{titleref}.
%
%    Now we can add the property \verb|title| is added to
%    the main property list.
%    \begin{macrocode}
\ZREF@NewPropTitle
\zref@addprop\ZREF@mainlist{title}%
%    \end{macrocode}
%
%    The title strings go into the \xfile{.aux} file, thus
%    they need some kind of protection.
%    Package \xpackage{titleref} uses a protected expansion
%    method. The advantage is that this can be used to
%    cleanup the string and to remove \cs{label}, \cs{index}
%    and other macros unwanted for referencing.
%    But there is the risk that fragile stuff can break.
%
%    Therefore package \xpackage{nameref} does not expand
%    the string. Thus the entries can safely be written to
%    the \xfile{.aux} file. But potentially dangerous
%    macros such as \cs{label} remain in the string and
%    can cause problems when using the string in references.
%
%    \begin{macro}{\ifzref@titleref@expand}
%    The switch \cs{ifzref@titleref@expand} distinguishes
%    between the both methods. Package \xpackage{nameref}'s
%    behaviour is achieved by setting the switch to false,
%    otherwise \xpackage{titleref}'s expansion is used.
%    Default is false.
%    \begin{macrocode}
\newif\ifzref@titleref@expand
%    \end{macrocode}
%    \end{macro}
%
%    \begin{macro}{\ZREF@titleref@hook}
%    The hook \cs{ZREF@titleref@hook} allows to extend
%    the cleanup for the expansion method. Thus
%    unnecessary macros can be removed or dangerous commands
%    removed. The hook is executed before the expansion of
%    \cs{zref@titleref@current}.
%    \begin{macrocode}
\let\ZREF@titleref@hook\ltx@empty
%    \end{macrocode}
%    \end{macro}
%    \begin{macro}{\zref@titleref@cleanup}
%    The hook should not be used directly, instead we provide the
%    macro \cs{zref@titleref@cleanup} to add stuff to the hook
%    and prevents that a previous non-empty content is not discarded
%    accidently.
%    \begin{macrocode}
\ZREF@Robust\def\zref@titleref@cleanup#1{%
   \begingroup
    \toks@\expandafter{%
      \ZREF@titleref@hook
      #1%
    }%
  \expandafter\endgroup
  \expandafter\def\expandafter\ZREF@titleref@hook\expandafter{%
    \the\toks@
  }%
}%
%    \end{macrocode}
%    \end{macro}
%
%    \begin{macro}{\ifzref@titleref@stripperiod}
%    Sometimes a title contains a period at the end.
%    Package \xpackage{nameref} removes this. This behaviour
%    is controlled by the switch \cs{ifzref@titleref@stripperiod}
%    and works regardless of the setting of option \xoption{expand}.
%    Period stripping is the default.
%    \begin{macrocode}
\newif\ifzref@titleref@stripperiod
\zref@titleref@stripperiodtrue
%    \end{macrocode}
%    \end{macro}
%
%    \begin{macro}{\zref@titleref@setcurrent}
%    Macro \cs{zref@titleref@setcurrent} sets a new current title
%    stored in \cs{zref@titleref@current}. Some cleanup and
%    expansion is performed that can be controlled by the
%    previous switches.
%    \begin{macrocode}
\ZREF@Robust\def\zref@titleref@setcurrent#1{%
  \ifzref@titleref@expand
    \GetTitleStringExpand{#1}%
  \else
    \GetTitleStringNonExpand{#1}%
  \fi
  \edef\zref@titleref@current{%
    \detokenize\expandafter{\GetTitleStringResult}%
  }%
  \ifzref@titleref@stripperiod
    \edef\zref@titleref@current{%
      \expandafter\ZREF@stripperiod\zref@titleref@current
      \ltx@empty.\ltx@empty\@nil
    }%
  \fi
}%
\GetTitleStringDisableCommands{%
  \ZREF@titleref@hook
}
%    \end{macrocode}
%    \end{macro}
%    \begin{macro}{\ZREF@stripperiod}
%    If \cs{ZREF@stripperiod} is called, the argument consists of
%    space tokens and tokens with catcode 12 (other), because
%    of \hologo{eTeX}'s \cs{detokenize}.
%    \begin{macrocode}
\def\ZREF@stripperiod#1.\ltx@empty#2\@nil{#1}%
%    \end{macrocode}
%    \end{macro}
%
% \subsubsection{User interface}
%
%    \begin{macro}{\ztitlerefsetup}
%    The behaviour of module \xmodule{titleref} is controlled
%    by switches and a hook. They can be set by \cs{ztitlerefsetup}
%    with a key value interface, provided by package \xpackage{keyval}.
%    Also the current title can be given explicitly by the key
%    \verb|title|.
%    \begin{macrocode}
\define@key{ZREF@TR}{expand}[true]{%
  \csname zref@titleref@expand#1\endcsname
}%
\define@key{ZREF@TR}{stripperiod}[true]{%
  \csname zref@titleref@stripperiod#1\endcsname
}%
\define@key{ZREF@TR}{cleanup}{%
  \zref@titleref@cleanup{#1}%
}%
\define@key{ZREF@TR}{title}{%
  \def\zref@titleref@current{#1}%
}%
\ZREF@IfDefinable\ztitlerefsetup\def{%
  {\kvsetkeys{ZREF@TR}}%
}%
%    \end{macrocode}
%    \end{macro}
%
%    \begin{macro}{\ztitleref}
%    The user command \cs{ztitleref} references the title.
%    For safety \cs{label} is disabled to prevent multiply
%    defined references.
%    \begin{macrocode}
\ZREF@IfDefinable\ztitleref\def{%
  {\zref@wrapper@babel\ZREF@titleref}%
}%
\def\ZREF@titleref#1{%
  \begingroup
    \zref@refused{#1}%
    \let\label\ltx@gobble
    \zref@extract{#1}{title}%
  \endgroup
}%
%    \end{macrocode}
%    \end{macro}
%
% \subsubsection{Patches for section and caption commands}
%
%    The section and caption macros are patched to extract the
%    title data.
%
%    Captions of figures and tables.
%    \begin{macrocode}
\AtBeginDocument{%
  \ZREF@patch{@caption}{%
    \long\def\@caption#1[#2]{%
      \zref@titleref@setcurrent{#2}%
      \ZREF@org@@caption{#1}[{#2}]%
    }%
  }%
%    \end{macrocode}
%
%    Section commands without star. The title version for the
%    table of contents is used because it is usually shorter and
%    more robust.
%    \begin{macrocode}
  \ZREF@patch{@part}{%
    \def\@part[#1]{%
      \zref@titleref@setcurrent{#1}%
      \ZREF@org@@part[{#1}]%
    }%
  }%
  \ZREF@patch{@chapter}{%
    \def\@chapter[#1]{%
      \zref@titleref@setcurrent{#1}%
      \ZREF@org@@chapter[{#1}]%
    }%
  }%
  \ZREF@patch{@sect}{%
    \def\@sect#1#2#3#4#5#6[#7]{%
      \zref@titleref@setcurrent{#7}%
      \ZREF@org@@sect{#1}{#2}{#3}{#4}{#5}{#6}[{#7}]%
    }%
  }%
%    \end{macrocode}
%    The star versions of the section commands.
%    \begin{macrocode}
  \ZREF@patch{@spart}{%
    \def\@spart#1{%
      \zref@titleref@setcurrent{#1}%
      \ZREF@org@@spart{#1}%
    }%
  }%
  \ZREF@patch{@schapter}{%
    \def\@schapter#1{%
      \zref@titleref@setcurrent{#1}%
      \ZREF@org@@schapter{#1}%
    }%
  }%
  \ZREF@patch{@ssect}{%
    \def\@ssect#1#2#3#4#5{%
      \zref@titleref@setcurrent{#5}%
      \ZREF@org@@ssect{#1}{#2}{#3}{#4}{#5}%
    }%
  }%
%    \end{macrocode}
%
% \subsubsection{Environment \textsf{description}}
%
%    \begin{macrocode}
  \ZREF@patch{descriptionlabel}{%
    \def\descriptionlabel#1{%
      \zref@titleref@setcurrent{#1}%
      \ZREF@org@descriptionlabel{#1}%
    }%
  }%
%    \end{macrocode}
%
% \subsubsection{Class \xclass{memoir}}
%
%    \begin{macrocode}
  \@ifclassloaded{memoir}{%
    \ltx@IfUndefined{ifheadnameref}{}{%
      \def\@chapter[#1]#2{%
        \ltx@IfUndefined{ch@pt@c}{%
          \zref@titleref@setcurrent{#1}%
        }{%
          \ifx\ch@pt@c\ltx@empty
            \zref@titleref@setcurrent{#2}%
          \else
            \def\NR@temp{#1}%
            \ifx\NR@temp\ltx@empty
              \expandafter\zref@titleref@setcurrent
              \expandafter{\ch@pt@c}%
            \else
              \ifheadnameref
                \zref@titleref@setcurrent{#1}%
              \else
                \expandafter\zref@titleref@setcurrent
                \expandafter{\ch@pt@c}%
              \fi
            \fi
          \fi
        }%
        \ZREF@org@@chapter[{#1}]{#2}%
      }%
      \ZREF@patch{M@sect}{%
        \def\M@sect#1#2#3#4#5#6[#7][#8]{%
          \ifheadnameref
            \zref@titleref@setcurrent{#8}%
          \else
            \zref@titleref@setcurrent{#7}%
          \fi
          \ZREF@org@M@sect{#1}{#2}{#3}{#4}{#5}{#6}[{#7}][{#8}]%
        }%
      }%
    }%
  }{}%
%    \end{macrocode}
%
% \subsubsection{Class \xclass{beamer}}
%
%    \begin{macrocode}
  \@ifclassloaded{beamer}{%
    \ZREF@patch{beamer@section}{%
      \long\def\beamer@section[#1]{%
        \zref@titleref@setcurrent{#1}%
        \ZREF@org@beamer@section[{#1}]%
      }%
    }%
    \ZREF@patch{beamer@subsection}{%
      \long\def\beamer@subsection[#1]{%
        \zref@titleref@setcurrent{#1}%
        \ZREF@org@beamer@subsection[{#1}]%
      }%
    }%
    \ZREF@patch{beamer@subsubsection}{%
      \long\def\beamer@subsubsection[#1]{%
        \zref@titleref@setcurrent{#1}%
        \ZREF@org@beamer@subsubsection[{#1}]%
      }%
    }%
  }{}%
%    \end{macrocode}
%
% \subsubsection{Package \xpackage{titlesec}}
%
%    \begin{macrocode}
  \@ifpackageloaded{titlesec}{%
    \ZREF@patch{ttl@sect@i}{%
      \def\ttl@sect@i#1#2[#3]#4{%
        \zref@titleref@setcurrent{#4}%
        \ZREF@org@ttl@sect@i{#1}{#2}[{#3}]{#4}%
      }%
    }%
    \ZREF@patch{ttl@straight@i}{%
      \def\ttl@straight@i#1[#2]#3{%
        \def\ZREF@temp{#2}%
        \ifx\ZREF@temp\ltx@empty
          \zref@titleref@setcurrent{#3}%
        \else
          \zref@titleref@setcurrent{#2}%
        \fi
        \ZREF@org@ttl@straight@i{#1}[{#2}]{#3}%
      }%
    }%
  }{}%
%    \end{macrocode}
%
% \subsubsection{Package \xpackage{longtable}}
%
%    Package \xpackage{longtable}: some support for its \cs{caption}.
%    However \cs{label} inside the caption is not supported.
%    \begin{macrocode}
  \@ifpackageloaded{longtable}{%
    \ZREF@patch{LT@c@ption}{%
      \def\LT@c@ption#1[#2]#3{%
        \ZREF@org@LT@c@ption{#1}[{#2}]{#3}%
        \zref@titleref@setcurrent{#2}%
      }%
    }%
  }{}%
%    \end{macrocode}
%
% \subsubsection{Package \xpackage{listings}}
%
%    Package \xpackage{listings}: support for its caption.
%    \begin{macrocode}
  \@ifpackageloaded{listings}{%
    \ZREF@patch{lst@MakeCaption}{%
      \def\lst@MakeCaption{%
        \ifx\lst@label\ltx@empty
        \else
          \expandafter\zref@titleref@setcurrent\expandafter{%
            \lst@@caption
          }%
        \fi
        \ZREF@org@lst@MakeCaption
      }%
    }%
  }{}%
%    \end{macrocode}
%
% \subsubsection{Theorems}
%
%    \begin{macrocode}
  \ZREF@patch{@opargbegintheorem}{%
    \def\@opargbegintheorem#1#2#3{%
      \zref@titleref@setcurrent{#3}%
      \ZREF@org@@opargbegintheorem{#1}{#2}{#3}%
    }%
  }%
  \@ifpackageloaded{amsthm}{%
    \begingroup
      \edef\x{macro:\string#1\string#2[\string#3]}%
      \@onelevel@sanitize\x
      \def\y#1->#2\@nil{#1}%
      \edef\z{\expandafter\y\meaning\@begintheorem->\@nil}%
      \@onelevel@sanitize\z
    \expandafter\endgroup
    \ifx\x\z
      \ZREF@patch{@begintheorem}{%
        \def\@begintheorem#1#2[#3]{%
          \zref@titleref@setcurrent{#3}%
          \ZREF@org@@begintheorem{#1}{#2}[{#3}]%
        }%
      }%
    \fi
  }{}%
%    \end{macrocode}
%
%    \begin{macrocode}
}
%    \end{macrocode}
%
%    \begin{macrocode}
%</titleref>
%    \end{macrocode}
%
%
% \subsection{Module \xmodule{xr}}
%
%    \begin{macrocode}
%<*xr>
\NeedsTeXFormat{LaTeX2e}
\ProvidesPackage{zref-xr}%
  [2020-07-03 v2.32 Module xr for zref (HO)]%
\RequirePackage{zref-base}[2019/11/29]
\ifx\ZREF@base@ok Y%
\else
  \expandafter\endinput
\fi
%    \end{macrocode}
%    \begin{macrocode}
\RequirePackage{keyval}
\RequirePackage{kvoptions}[2010/02/22]
%    \end{macrocode}
%
%    We declare property |url|, because this is added,
%    if a reference is imported and has not already set this field.
%    Or if |hyperref| is used, then this property can be asked.
%    \begin{macrocode}
\zref@newprop{url}{}%
\zref@newprop{urluse}{}%
%    \end{macrocode}
%    \begin{macrocode}
\zref@newprop{externaldocument}{}%
%    \end{macrocode}
%
%    Most code, especially the handling of the \xfile{.aux}
%    files are taken from David Carlisle's \xpackage{xr} package.
%    Therefore I drop the documentation for these macros here.
%
%    \begin{macro}{\zref@xr@ext}
%    If the URL is not specied, then assume
%    processed file with a guessed extension.
%    Use the setting of \xpackage{hyperref} if available.
%    \begin{macrocode}
\providecommand*{\zref@xr@ext}{%
  \ltx@ifundefined{XR@ext}{pdf}{\XR@ext}%
}%
%    \end{macrocode}
%    \end{macro}
%    \begin{macro}{\ifZREF@xr@zreflabel}
%    The use of the star form of \cs{zexternaldocument}
%    is remembered in the switch \cs{ifZREF@xr@zreflabel}.
%    \begin{macrocode}
\newif\ifZREF@xr@zreflabel
%    \end{macrocode}
%    \end{macro}
%    \begin{macrocode}
\SetupKeyvalOptions{%
  family=ZREF@XR,%
  prefix=ZREF@xr@%
}
\DeclareBoolOption[true]{tozreflabel}
\DeclareBoolOption[false]{toltxlabel}
\DeclareBoolOption{verbose}
\define@key{ZREF@XR}{ext}{%
  \def\zref@xr@{#1}%
}
\DeclareBoolOption[false]{urluse}
%    \end{macrocode}
%    \begin{macro}{\zxrsetup}
%    \begin{macrocode}
\newcommand*{\zxrsetup}{%
  \kvsetkeys{ZREF@XR}%
}%
%    \end{macrocode}
%    \end{macro}
%
%    \begin{macro}{\ZREF@xr@URL}
%    \begin{macrocode}
\newcount\ZREF@xr@URL
\ZREF@xr@URL=\ltx@zero
%    \end{macrocode}
%    \end{macro}
%    \begin{macro}{\ZREF@xr@AddURL}
%    \begin{macrocode}
\def\ZREF@xr@AddURL#1{%
  \begingroup
    \def\ZREF@temp{#1}%
    \count@=\ltx@one
    \ZREF@xr@@AddUrl
  \endgroup
}
%    \end{macrocode}
%    \end{macro}
%    \begin{macro}{\ZREF@xr@@AddUrl}
%    \begin{macrocode}
\def\ZREF@xr@@AddUrl{%
  \ifnum\count@>\ZREF@xr@URL
    \global\advance\ZREF@xr@URL by\ltx@one
    \xdef\ZREF@xr@theURL{\romannumeral\ZREF@xr@URL}%
    \global\expandafter\let
        \csname Z@U@\ZREF@xr@theURL\endcsname\ZREF@temp
    \@PackageInfo{zref-xr}{%
      \ltx@backslashchar Z@U@\ZREF@xr@theURL:\MessageBreak
      \ZREF@temp\MessageBreak
    }%
  \else
    \expandafter
    \ifx\csname Z@U@\romannumeral\count@\endcsname\ZREF@temp
      \xdef\ZREF@xr@theURL{\romannumeral\count@}%
    \else
      \expandafter\expandafter\expandafter\ZREF@xr@@AddUrl
    \fi
  \fi
}
%    \end{macrocode}
%    \end{macro}
%
%    \begin{macro}{\zexternaldocument}
%    In its star form it looks for \cs{newlabel},
%    otherwise for \cs{zref@newlabel}.
%     Later we
%    will read \xfile{.aux} files that expects
%    |@| to have catcode 11 (letter).
%    \begin{macrocode}
\ZREF@IfDefinable\zexternaldocument\def{%
  {%
    \ZREF@NewPropAnchor
    \ZREF@NewPropTitle
    \begingroup
      \csname @safe@actives@true\endcsname
      \makeatletter
      \@ifstar{%
        \ZREF@xr@zreflabelfalse
        \@testopt\ZREF@xr@externaldocument{}%
      }{%
        \ZREF@xr@zreflabeltrue
        \@testopt\ZREF@xr@externaldocument{}%
      }%
  }%
}%
%    \end{macrocode}
%    \end{macro}
%
%    If the \cs{include} featuer was used, there can be
%    several \xfile{.aux} files. These files are read
%    one after another, especially they are not recursively
%    read in order to save read registers. Thus it can
%    happen that the read order of the newlabel commands
%    differs from \hologo{LaTeX}'s order using \cs{input}.
%
%    \begin{macro}{\ZREF@xr@externaldocument}
%    It reads the remaining arguments. \cs{newcommand}
%    comes in handy for the optional argument.
%    \begin{macrocode}
\def\ZREF@xr@externaldocument[#1]#2{%
    \def\ZREF@xr@prefix{#1}%
    \let\ZREF@xr@filelist\ltx@empty
    \edef\ZREF@xr@externalfile{#2}%
    \edef\ZREF@xr@file{\ZREF@xr@externalfile.aux}%
    \filename@parse{#2}%
    \@testopt\ZREF@xr@graburl{#2.\zref@xr@ext}%
}%
\def\ZREF@xr@graburl[#1]{%
    \edef\ZREF@xr@url{#1}%
    \ifZREF@xr@urluse
      \expandafter\ZREF@xr@AddURL\expandafter{\ZREF@xr@url}%
      \expandafter\def\expandafter\ZREF@xr@url
      \expandafter{\csname Z@U@\ZREF@xr@theURL\endcsname}%
    \fi
    \ZREF@xr@checkfile
  \endgroup
}%
%    \end{macrocode}
%    \end{macro}
%    \begin{macro}{\ZREF@xr@processfile}
%    We follow \xpackage{xr} here, \cs{IfFileExists} offers
%    a nicer test, but we have to open the file anyway.
%    \begin{macrocode}
\def\ZREF@xr@checkfile{%
  \openin\@inputcheck\ZREF@xr@file\relax
  \ifeof\@inputcheck
    \PackageWarning{zref-xr}{%
      File `\ZREF@xr@file' not found or empty,\MessageBreak
      labels not imported%
    }%
  \else
    \PackageInfo{zref-xr}{%
      Label \ifZREF@xr@zreflabel (zref) \fi
      import from `\ZREF@xr@file'%
    }%
    \def\ZREF@xr@found{0}%
    \def\ZREF@xr@ignored@empty{0}%
    \def\ZREF@xr@ignored@zref{0}%
    \def\ZREF@xr@ignored@ltx{0}%
    \ZREF@xr@processfile
    \closein\@inputcheck
    \begingroup
      \let\on@line\ltx@empty
      \PackageInfo{zref-xr}{%
        Statistics for `\ZREF@xr@file':\MessageBreak
        \ZREF@xr@found\space
        \ifZREF@xr@zreflabel zref\else LaTeX\fi\space
        label(s) found%
        \ifnum\ZREF@xr@ignored@empty>0 %
          ,\MessageBreak
          \ZREF@xr@ignored@empty\space empty label(s) ignored%
        \fi
        \ifnum\ZREF@xr@ignored@zref>0 %
          ,\MessageBreak
          \ZREF@xr@ignored@zref\space
          duplicated zref label(s) ignored%
        \fi
        \ifnum\ZREF@xr@ignored@ltx>0 %
          ,\MessageBreak
          \ZREF@xr@ignored@ltx\space
          duplicated latex label(s) ignored%
        \fi
      }%
    \endgroup
  \fi
  \ifx\ZREF@xr@filelist\ltx@empty
  \else
    \edef\ZREF@xr@file{%
      \expandafter\ltx@car\ZREF@xr@filelist\@nil
    }%
    \edef\ZREF@xr@filelist{%
      \expandafter\ltx@cdr\ZREF@xr@filelist\ltx@empty\@nil
    }%
    \expandafter\ZREF@xr@checkfile
  \fi
}%
%    \end{macrocode}
%    \end{macro}
%
%    \begin{macro}{\ZREF@xr@processfile}
%    \begin{macrocode}
\def\ZREF@xr@processfile{%
  \read\@inputcheck to\ZREF@xr@line
  \expandafter\ZREF@xr@processline\ZREF@xr@line..\ZREF@nil
  \ifeof\@inputcheck
  \else
    \expandafter\ZREF@xr@processfile
  \fi
}%
%    \end{macrocode}
%    \end{macro}
%    \begin{macro}{\ZREF@xr@processline}
%    The most work must be done for analyzing the
%    arguments of \cs{newlabel}.
%    \begin{macrocode}
\long\def\ZREF@xr@processline#1#2#3\ZREF@nil{%
  \def\x{#1}%
  \toks@{#2}%
  \ifZREF@xr@zreflabel
    \ifx\x\ZREF@xr@zref@newlabel
      \expandafter
      \ZREF@xr@process@zreflabel\ZREF@xr@line...\ZREF@nil
    \fi
  \else
    \ifx\x\ZREF@xr@newlabel
      \expandafter
      \ZREF@xr@process@label\ZREF@xr@line...[]\ZREF@nil
    \fi
  \fi
  \ifx\x\ZREF@xr@@input
    \edef\ZREF@xr@filelist{%
      \etex@unexpanded\expandafter{\ZREF@xr@filelist}%
      {\filename@area\the\toks@}%
    }%
  \fi
}%
\def\ZREF@xr@process@zreflabel\zref@newlabel#1#2#3\ZREF@nil{%
  \edef\ZREF@xr@refname{Z@R@\ZREF@xr@prefix#1}%
  \edef\ZREF@xr@found{\the\numexpr\ZREF@xr@found+1\relax}%
  \def\x{#2}%
  \edef\ZREF@xr@tempname{$temp$}%
  \edef\ZREF@xr@temprefname{Z@R@\ZREF@xr@tempname}%
  \let\ZREF@xr@list\x
  \ifx\ZREF@xr@list\ltx@empty
    \PackageWarningNoLine{zref-xr}{%
      Label `#1' without properties ignored\MessageBreak
      in file `\ZREF@xr@file'%
    }%
    \edef\ZREF@xr@ignored@empty{%
      \the\numexpr\ZREF@xr@ignored@empty+1\relax
    }%
  \else
    \expandafter\ZREF@xr@checklist\x\ZREF@nil
    \expandafter\let\csname\ZREF@xr@temprefname\endcsname\x
    \expandafter\ltx@LocalAppendToMacro
    \csname\ZREF@xr@temprefname\expandafter\endcsname
    \expandafter{%
      \expandafter\externaldocument\expandafter{%
        \ZREF@xr@externalfile
      }%
    }%
    \ZREF@xr@urlcheck\ZREF@xr@tempname
    \ifZREF@xr@tozreflabel
      \@ifundefined{\ZREF@xr@refname}{%
        \ifZREF@xr@verbose
          \PackageInfo{zref-xr}{%
            Import to zref label `\ZREF@xr@tempname#1'%
          }%
        \fi
        \global\expandafter
        \let\csname\ZREF@xr@refname\expandafter\endcsname
        \csname\ZREF@xr@temprefname\endcsname
      }{%
        \ZREF@xr@zref@ignorewarning{\ZREF@xr@prefix#1}%
      }%
    \fi
    \ifZREF@xr@toltxlabel
      \ZREF@xr@tolabel{\ZREF@xr@tempname}{\ZREF@xr@prefix#1}%
    \fi
  \fi
}%
\def\ZREF@xr@process@label\newlabel#1#2#3[#4]#5\ZREF@nil{%
  \def\ZREF@xr@refname{Z@R@\ZREF@xr@prefix#1}%
  \edef\ZREF@xr@found{\the\numexpr\ZREF@xr@found+1\relax}%
  \def\x{#2}%
  \edef\ZREF@xr@tempname{$temp$}%
  \edef\ZREF@xr@temprefname{Z@R@\ZREF@xr@tempname}%
  \expandafter\ZREF@xr@scanparams
      \csname\ZREF@xr@temprefname\expandafter\endcsname
      \x{}{}{}{}{}\ZREF@nil
  \ifx\\#4\\%
  \else
    % ntheorem knows an optional argument at the end of \newlabel
    \ZREF@NewPropTheotype
    \expandafter\ltx@LocalAppendToMacro
        \csname\ZREF@xr@temprefname\endcsname{\theotype{#4}}%
  \fi
  \expandafter\ltx@LocalAppendToMacro
  \csname\ZREF@xr@temprefname\expandafter\endcsname\expandafter{%
    \expandafter\externaldocument\expandafter{%
      \ZREF@xr@externalfile
    }%
  }%
  \ZREF@xr@urlcheck\ZREF@xr@tempname
  \ifZREF@xr@tozreflabel
    \@ifundefined{\ZREF@xr@refname}{%
      \ifZREF@xr@verbose
        \PackageInfo{zref-xr}{%
          Import to zref label `\ZREF@xr@prefix#1'%
        }%
      \fi
      \global\expandafter
      \let\csname\ZREF@xr@refname\expandafter\endcsname
      \csname\ZREF@xr@temprefname\endcsname
    }{%
      \ZREF@xr@zref@ignorewarning{\ZREF@xr@prefix#1}%
    }%
  \fi
  \ifZREF@xr@toltxlabel
    \ZREF@xr@tolabel{\ZREF@xr@tempname}{\ZREF@xr@prefix#1}%
  \fi
}
\def\ZREF@xr@zref@newlabel{\zref@newlabel}%
\def\ZREF@xr@newlabel{\newlabel}%
\def\ZREF@xr@@input{\@input}%
\def\ZREF@xr@relax{\relax}%
%    \end{macrocode}
%    \end{macro}
%    \begin{macro}{\ZREF@xr@tolabel}
%    \begin{macrocode}
\def\ZREF@xr@tolabel#1#2{%
  \ifZREF@xr@verbose
    \PackageInfo{zref-xr}{%
      Import to LaTeX label `#2'%
    }%
  \fi
  \zref@wrapper@unexpanded{%
    \expandafter\xdef\csname r@#2\endcsname{%
      {%
        \ltx@ifundefined{M@TitleReference}{%
          \ltx@ifundefined{TR@TitleReference}{%
            \zref@extractdefault{#1}{default}{}%
          }{%
            \noexpand\TR@TitleReference
            {\zref@extractdefault{#1}{default}{}}%
            {\zref@extractdefault{#1}{title}{}}%
          }%
        }{%
          \noexpand\M@TitleReference
          {\zref@extractdefault{#1}{default}{}}%
          {\zref@extractdefault{#1}{title}{}}%
        }%
      }%
      {\zref@extractdefault{#1}{page}{}}%
      \ltx@ifpackageloaded{nameref}{%
        {\zref@extractdefault{#1}{title}{}}%
        {\zref@extractdefault{#1}{anchor}{}}%
        \zref@ifrefcontainsprop{#1}{urluse}{%
          {\zref@extractdefault{#1}{urluse}{}}%
        }{%
          {\zref@extractdefault{#1}{url}{}}%
        }%
      }{}%
    }%
  }%
}
%    \end{macrocode}
%    \end{macro}
%
%    \begin{macro}{\ZREF@xr@zref@ignorewarning}
%    \begin{macrocode}
\def\ZREF@xr@zref@ignorewarning#1{%
  \PackageWarningNoLine{zref-xr}{%
    Zref label `#1' is already in use\MessageBreak
    in file `\ZREF@xr@file'%
  }%
  \edef\ZREF@xr@ignored@zref{%
    \the\numexpr\ZREF@xr@ignored@zref+1%
  }%
}%
%    \end{macrocode}
%    \end{macro}
%
%    \begin{macro}{\ZREF@xr@ltx@ignorewarning}
%    \begin{macrocode}
\def\ZREF@xr@ltx@ignorewarning#1{%
  \PackageWarningNoLine{zref-xr}{%
    LaTeX label `#1' is already in use\MessageBreak
    in file `\ZREF@xr@file'%
  }%
  \edef\ZREF@xr@ignored@ltx{%
    \the\numexpr\ZREF@xr@ignored@ltx+1%
  }%
}%
%    \end{macrocode}
%    \end{macro}
%
%    \begin{macro}{\ZREF@xr@checklist}
%    \begin{macrocode}
\def\ZREF@xr@checklist#1#2#3\ZREF@nil{%
  \ifx\@undefined#1\relax
    \expandafter\ZREF@xr@checkkey\string#1\@nil
  \fi
  \ifx\\#3\\%
  \else
    \ltx@ReturnAfterFi{%
      \ZREF@xr@checklist#3\ZREF@nil
    }%
  \fi
}%
\def\ZREF@xr@checkkey#1#2\@nil{%
  \zref@ifpropundefined{#2}{%
    \zref@newprop{#2}{}%
  }{}%
}%
%    \end{macrocode}
%    \end{macro}
%
%    \begin{macro}{\ZREF@xr@scanparams}
%    \begin{macrocode}
\def\ZREF@xr@scanparams#1#2#3#4#5#6#7\ZREF@nil{%
  \let#1\ltx@empty
  \ZREF@foundfalse
  \ZREF@xr@scantitleref#1#2\TR@TitleReference{}{}\ZREF@nil
  \ifZREF@found
  \else
    \ltx@LocalAppendToMacro#1{\default{#2}}%
  \fi
  % page
  \ltx@LocalAppendToMacro#1{\page{#3}}%
  % nameref title
  \ifZREF@found
  \else
    \ifx\\#4\\%
    \else
      \def\ZREF@xr@temp{#4}%
      \ifx\ZREF@xr@temp\ZREF@xr@relax
      \else
        \ltx@LocalAppendToMacro#1{\title{#4}}%
      \fi
    \fi
  \fi
  % anchor
  \ifx\\#5\\%
  \else
    \ltx@LocalAppendToMacro#1{\anchor{#5}}%
  \fi
  \ifx\\#6\\%
  \else
    \ifZREF@xr@urluse
      \ZREF@xr@AddURL{#6}%
      \expandafter\ltx@LocalAppendToMacro\expandafter#1%
      \expandafter{%
        \expandafter\urluse\expandafter{%
          \csname Z@U@\ZREF@xr@theURL\endcsname
        }%
      }%
    \else
      \ltx@LocalAppendToMacro#1{\url{#6}}%
    \fi
  \fi
}%
%    \end{macrocode}
%    \end{macro}
%
%    \begin{macro}{\ZREF@xr@scantitleref}
%    \begin{macrocode}
\def\ZREF@xr@scantitleref#1#2\TR@TitleReference#3#4#5\ZREF@nil{%
  \ifx\\#5\\%
  \else
    \ltx@LocalAppendToMacro#1{%
      \default{#3}%
      \title{#4}%
    }%
    \ZREF@foundtrue
  \fi
}%
%    \end{macrocode}
%    \end{macro}
%
%    \begin{macro}{\ZREF@xr@urlcheck}%
%    \begin{macrocode}
\def\ZREF@xr@urlcheck#1{%
  \zref@ifrefcontainsprop{#1}{anchor}{%
    \zref@ifrefcontainsprop{#1}{url}{%
    }{%
      \expandafter
      \ltx@LocalAppendToMacro\csname Z@R@#1\expandafter\endcsname
      \expandafter{%
        \csname url\ifZREF@xr@urluse use\fi
        \expandafter\endcsname\expandafter{\ZREF@xr@url}%
      }%
    }%
  }{%
  }%
}%
%    \end{macrocode}
%    \end{macro}
%
%    \begin{macrocode}
%</xr>
%    \end{macrocode}
%
%
% \subsection{Module \xmodule{hyperref}}
%
%    UNFINISHED :-(
%
%    \begin{macrocode}
%<*hyperref>
\NeedsTeXFormat{LaTeX2e}
\ProvidesPackage{zref-hyperref}%
  [2020-07-03 v2.32 Module hyperref for zref (HO)]%
\RequirePackage{zref-base}[2019/11/29]
\ifx\ZREF@base@ok Y%
\else
  \expandafter\endinput
\fi
%    \end{macrocode}
%
%    \begin{macrocode}
\ZREF@NewPropAnchor
\zref@addprop\ZREF@mainlist{anchor}%
%    \end{macrocode}
%
%    \begin{macrocode}
%</hyperref>
%    \end{macrocode}
%
%
% \subsection{Module \xmodule{savepos}}
%
%    Module \xmodule{savepos} provides an interface for
%    \pdfTeX's \cmd{\pdfsavepos}, see the manual for
%    \pdfTeX.
%
% \subsubsection{Identification}
%
%    \begin{macrocode}
%<*savepos>
\NeedsTeXFormat{LaTeX2e}
\ProvidesPackage{zref-savepos}%
  [2020-07-03 v2.32 Module savepos for zref (HO)]%
\RequirePackage{zref-base}[2019/11/29]
\ifx\ZREF@base@ok Y%
\else
  \expandafter\endinput
\fi
%    \end{macrocode}
%
% \subsubsection{Availability}
%
%    First we check, whether the feature is available.
%    \begin{macrocode}
\ifx\directlua\@undefined
\ltx@IfUndefined{pdfsavepos}{%
  \PackageError\ZREF@name{%
    \string\pdfsavepos\space is not supported.\MessageBreak
    It is provided by pdfTeX (1.40) or XeTeX%
  }\ZREF@UpdatePdfTeX
  \endinput
}{}%
\fi
%    \end{macrocode}
%    In PDF mode we are done. However support for DVI mode
%    was added later in version 1.40.0. In earlier versions
%    \cs{pdfsavepos} is defined, but its execution raises
%    an error. Note that \hologo{XeTeX} also provides \cs{pdfsavepos}.
%    \begin{macrocode}
\ifpdf
\else
  \ltx@IfUndefined{pdftexversion}{%
  }{%
    \ifnum\pdftexversion<140 %
      \PackageError\ZREF@name{%
        \string\pdfsavepos\space is not supported in DVI mode%
        \MessageBreak
        of this pdfTeX version%
      }\ZREF@UpdatePdfTeX
      \expandafter\expandafter\expandafter\endinput
    \fi
  }%
\fi
%    \end{macrocode}
%
% \subsubsection{Setup}
%
%    \begin{macrocode}
\zref@newlist{savepos}
\ifx\directlua\@undefined
  \zref@newprop*{posx}[0]{\the\pdflastxpos}
  \zref@newprop*{posy}[0]{\the\pdflastypos}
\else
  \zref@newprop*{posx}[0]{\the\lastxpos}
  \zref@newprop*{posy}[0]{\the\lastypos}
\fi
\zref@addprops{savepos}{posx,posy}
%    \end{macrocode}
%
% \subsubsection{User macros}
%
%    \begin{macro}{\zref@savepos}
%    \begin{macrocode}
\ifx\directlua\@undefined
  \def\zref@savepos{%
    \if@filesw
      \pdfsavepos
    \fi
  }
\else
  \def\zref@savepos{%
    \if@filesw
      \savepos
    \fi
  }
\fi
%    \end{macrocode}
%    \end{macro}
%
%    \begin{macro}{\ZREF@zsavepos}
%    \begin{macrocode}
\def\ZREF@zsavepos#1#2#3{%
  \@bsphack
  \if@filesw
    \zref@savepos
    #1{#3}{#2}%
    \ltx@IfUndefined{TeXXeTstate}{%
    }{%
      \ifnum\TeXXeTstate=\ltx@zero
      \else
        \zref@savepos
      \fi
    }%
  \fi
  \@esphack
}
%    \end{macrocode}
%    \end{macro}
%
%    \begin{macro}{\zsavepos}
%    The current location is stored in a reference with the
%    given name.
%    \begin{macrocode}
\ZREF@IfDefinable\zsavepos\def{%
  {%
    \ZREF@zsavepos\zref@labelbylist{savepos}%
  }%
}
%    \end{macrocode}
%    \end{macro}
%
%    \begin{macro}{\zsaveposx}
%    \begin{macrocode}
\ZREF@IfDefinable\zsaveposx\def{%
  {%
    \ZREF@zsavepos\zref@labelbyprops{posx}%
  }%
}
%    \end{macrocode}
%    \end{macro}
%    \begin{macro}{\zsaveposy}
%    \begin{macrocode}
\ZREF@IfDefinable\zsaveposy\def{%
  {%
    \ZREF@zsavepos\zref@labelbyprops{posy}%
  }%
}
%    \end{macrocode}
%    \end{macro}
%
%    \begin{macro}{\zposx}
%    \begin{macro}{\zposy}
%    The horizontal and vertical position are available by
%    \cs{zposx} and \cs{zposy}. Do not rely on absolute
%    positions. They differ in DVI and PDF mode of \pdfTeX.
%    Use differences instead. The unit of the position numbers
%    is sp.
%    \begin{macrocode}
\newcommand*{\zposx}[1]{%
  \zref@extract{#1}{posx}%
}%
\newcommand*{\zposy}[1]{%
  \zref@extract{#1}{posy}%
}%
%    \end{macrocode}
%    \end{macro}
%    \end{macro}
%
%    Typically horizontal and vertical positions are used
%    inside calculations. Therefore the extracting macros
%    should be expandable and babel's patch is not applyable.
%
%    Also it is in the responsibility of the user to marked
%    used positions by \cs{zrefused}
%    in order to notify \hologo{LaTeX} about undefined references.
%
%    \begin{macro}{\ZREF@savepos@ok}
%    \begin{macrocode}
\let\ZREF@savepos@ok=Y
%    \end{macrocode}
%    \end{macro}
%
%    \begin{macrocode}
%</savepos>
%    \end{macrocode}
%
%
% \subsection{Module \xmodule{abspos}}
%
% \subsubsection{Identification}
%
%    \begin{macrocode}
%<*abspos>
\NeedsTeXFormat{LaTeX2e}
\ProvidesPackage{zref-abspos}%
  [2020-07-03 v2.32 Module abspos for zref (HO)]%
\RequirePackage{zref-base}[2019/11/29]
\ifx\ZREF@base@ok Y%
\else
  \expandafter\endinput
\fi
%    \end{macrocode}
%
%    \begin{macrocode}
\RequirePackage{zref-savepos}[2019/11/29]
\ifx\ZREF@savepos@ok Y%
\else
  \expandafter\endinput
\fi
%    \end{macrocode}
%    \begin{macrocode}
\RequirePackage{zref-pagelayout}[2019/11/29]
\zref@addprop{savepos}{abspage}
\zref@addprop{savepos}{pagevalue}
%    \end{macrocode}
%
%    \begin{macro}{\zref@absposx}
%    \begin{macrocode}
\newcommand*{\zref@absposx}[3]{%
  \number
    \expandafter\zref@absposnumx\expandafter{%
      \number\zref@extractdefault{#1}{abspage}{0}%
    }{#2}{#3}%
  \ltx@space
}
%    \end{macrocode}
%    \end{macro}
%
%    \begin{macro}{\zref@absposy}
%    \begin{macrocode}
\newcommand*{\zref@absposy}[3]{%
  \number
  \expandafter\zref@absposnumy\expandafter{%
    \number\zref@extractdefault{#1}{abspage}{0}%
  }{#2}{#3}%
  \ltx@space
}
%    \end{macrocode}
%    \end{macro}
%
%    \begin{macro}{\zref@absposnumx}
%    \begin{macrocode}
\newcommand*{\zref@absposnumx}[3]{%
  \number
%  \ifnum#1>\ltx@zero
%    \zref@ifrefundefined{thepage#1}{%
%      0%
%    }{%
      \numexpr\ZREF@absposnum{thepage#1}{#2}{x}{#3}\relax
%    }%
%  \else
%    0%
%  \fi
}
%    \end{macrocode}
%    \end{macro}
%
%    \begin{macro}{\zref@absposnumy}
%    \begin{macrocode}
\newcommand*{\zref@absposnumy}[3]{%
  \number
%  \ifnum#1>\ltx@zero
%    \zref@ifrefundefined{thepage#1}{%
%      0%
%    }{%
      \numexpr\ZREF@absposnum{thepage#1}{#2}{y}{#3}\relax
%    }%
%  \else
%    0%
%  \fi
}
%    \end{macrocode}
%    \end{macro}
%
%    \begin{macro}{\ZREF@absposnum}
%    \begin{macrocode}
\def\ZREF@absposnum#1#2#3#4{%
  \ltx@ifundefined{ZREF@abspos@#2@#3@#4}{%
    0%
  }{%
    \csname ZREF@abspos@#2@#3@#4\endcsname{#1}%
  }%
}
%    \end{macrocode}
%    \end{macro}
%
%    \begin{macro}{\zref@def@absposx}
%    \begin{macrocode}
\ZREF@Robust\def\zref@def@absposx#1{%
  \zref@wrapper@babel{\ZREF@def@abspos{#1}\zref@absposx}%
}
%    \end{macrocode}
%    \end{macro}
%
%    \begin{macro}{\zref@def@absposy}
%    \begin{macrocode}
\ZREF@Robust\def\zref@def@absposy#1{%
  \zref@wrapper@babel{\ZREF@def@abspos{#1}\zref@absposy}%
}
%    \end{macrocode}
%    \end{macro}
%
%    \begin{macro}{\zref@def@absposnumx}
%    \begin{macrocode}
\ZREF@Robust\def\zref@def@absposnumx#1{%
  \zref@wrapper@babel{\ZREF@def@abspos{#1}\zref@absposnumx}%
}
%    \end{macrocode}
%    \end{macro}
%
%    \begin{macro}{\zref@def@absposnumy}
%    \begin{macrocode}
\ZREF@Robust\def\zref@def@absposnumy#1{%
  \zref@wrapper@babel{\ZREF@def@abspos{#1}\zref@absposnumy}%
}
%    \end{macrocode}
%    \end{macro}
%    \begin{macro}{\ZREF@def@abspos}
%    \begin{macrocode}
\def\ZREF@def@absposnumy#1#2#3#4#5{%
  \edef#1{#2{#3}{#4}{#5}}%
}
%    \end{macrocode}
%    \end{macro}
%
%    \begin{macro}{\zref@absposused}
%    \begin{macrocode}
\ZREF@Robust\def\zref@absposused{%
  \zref@wrapper@babel\ZREF@abspos@used
}
%    \end{macrocode}
%    \end{macro}
%    \begin{macro}{\ZREF@abspos@used}
%    \begin{macrocode}
\def\ZREF@abspos@used#1{%
  \zref@refused{#1}%
  \zref@ifrefundefined{#1}{%
  }{%
    \begingroup
      \edef\ZREF@temp{%
        \zref@extractdefault{#1}{abspage}{0}%
      }%
      \ifnum\ZREF@temp>\ltx@zero
        \zref@refused{thepage\ZREF@temp}%
      \else
        \@PackageError{zref-abspos}{%
          \string\zref@pos@label@used\ltx@space
          needs property `abspage'\MessageBreak
          in label `#1'%
        }\@ehc
      \fi
    \endgroup
  }%
}
%    \end{macrocode}
%    \end{macro}
%
%    \begin{macro}{\zref@absposnumused}
%    \begin{macrocode}
\newcommand*{\zref@absposnumused}[1]{%
  \ifnum#1>\ltx@zero
    \zref@refused{thepage\number#1}%
  \else
    \@PackageError{zref-abspos}{%
      Invalid absolute page number (#1)\MessageBreak
      for \string\zref@pos@num@used.\MessageBreak
      A positive integer is expected%
    }\@ehc
  \fi
}
%    \end{macrocode}
%    \end{macro}
%
%    \begin{macro}{\zref@ifabsposundefined}
%    \begin{macrocode}
\def\zref@ifabsposundefined#1{%
  \zref@ifrefundefined{#1}\ltx@firsttwo{%
    \expandafter\zref@ifabsposnumundefined\expandafter{%
      \number\zref@extractdefault{#1}{abspage}{0}%
    }%
  }%
}
%    \end{macrocode}
%    \end{macro}
%
%    \begin{macro}{\zref@ifabsposnumundefined}
%    \begin{macrocode}
\def\zref@ifabsposnumundefined#1{%
  \ifnum\ZREF@number{#1}>\ltx@zero
    \zref@ifrefundefined{thepage#1}%
        \ltx@firstoftwo\ltx@secondoftwo
  \else
    \expandafter\ltx@firstoftwo
  \fi
}
%    \end{macrocode}
%    \end{macro}
%
% \subsubsection{Media}
%
%    \begin{macro}{\ZREF@abspos@media@width}
%    \begin{macrocode}
\edef\ZREF@abspos@media@width{%
  \ltx@ifundefined{pdfpagewidth}{%
    \ltx@ifundefined{mediawidth}{%
      \ltx@ifundefined{stockwidth}{%
        paperwidth%
      }{%
        stockwidth%
      }%
    }{%
      mediawidth%
    }%
  }{%
    pdfpagewidth%
  }%
}
\ifluatex
\def\ZREF@abspos@media@width{pdfpagewidth}%
\fi
%    \end{macrocode}
%    \end{macro}
%    \begin{macro}{\ZREF@abspos@media@height}
%    \begin{macrocode}
\edef\ZREF@abspos@media@height{%
  \ltx@ifundefined{pdfpageheight}{%
    \ltx@ifundefined{mediaheight}{%
      \ltx@ifundefined{stockheight}{%
        paperheight%
      }{%
        stockheight%
      }%
    }{%
      mediaheight%
    }%
  }{%
    \noexpand\ifcase\pdfpageheight
      \ltx@ifundefined{stockheight}{%
        paperheight%
      }{%
        stockheight%
      }%
    \noexpand\else
      pdfpageheight%
    \noexpand\fi
  }%
}
\ifluatex
\edef\ZREF@abspos@media@height{%
   \noexpand\ifcase\pageheight
      \ltx@ifundefined{stockheight}{%
        paperheight%
      }{%
        stockheight%
      }%
    \noexpand\else
      pdfpageheight%
    \noexpand\fi}%
\fi
%    \end{macrocode}
%    \end{macro}
%
%    \begin{macro}{\ZREF@abspos@media@x@left}
%    \begin{macrocode}
\def\ZREF@abspos@media@x@left#1{%
  0%
}
%    \end{macrocode}
%    \end{macro}
%    \begin{macro}{\ZREF@abspos@media@x@right}
%    \begin{macrocode}
\def\ZREF@abspos@media@x@right#1{%
  \zref@extract{#1}\ZREF@abspos@media@width
}
%    \end{macrocode}
%    \end{macro}
%    \begin{macro}{\ZREF@abspos@media@x@center}
%    \begin{macrocode}
\def\ZREF@abspos@media@x@center#1{%
  \ZREF@abspos@media@x@left{#1}%
  +\zref@extract{#1}\ZREF@abspos@media@width/2%
}
%    \end{macrocode}
%    \end{macro}
%
%    \begin{macro}{\ZREF@abspos@media@y@top}
%    \begin{macrocode}
\def\ZREF@abspos@media@y@top#1{%
  \zref@extract{#1}\ZREF@abspos@media@height
}
%    \end{macrocode}
%    \end{macro}
%    \begin{macro}{\ZREF@abspos@media@y@bottom}
%    \begin{macrocode}
\def\ZREF@abspos@media@y@bottom#1{%
  0%
}
%    \end{macrocode}
%    \end{macro}
%    \begin{macro}{\ZREF@abspos@media@y@center}
%    \begin{macrocode}
\def\ZREF@abspos@media@y@center#1{%
  \zref@extract{#1}\ZREF@abspos@media@height/2%
}
%    \end{macrocode}
%    \end{macro}
%
% \subsubsection{Paper}
%
%    \begin{macro}{\ZREF@abspos@paper@x@left}
%    \begin{macrocode}
\def\ZREF@abspos@paper@x@left#1{%
  0%
}
%    \end{macrocode}
%    \end{macro}
%    \begin{macro}{\ZREF@abspos@paper@x@right}
%    \begin{macrocode}
\def\ZREF@abspos@paper@x@right#1{%
  \zref@extract{#1}{paperwidth}%
}
%    \end{macrocode}
%    \end{macro}
%    \begin{macro}{\ZREF@abspos@paper@x@center}
%    \begin{macrocode}
\def\ZREF@abspos@paper@x@center#1{%
  \zref@extract{#1}{paperwidth}/2%
}
%    \end{macrocode}
%    \end{macro}
%
%    \begin{macro}{\ZREF@abspos@paper@y@top}
%    \begin{macrocode}
\let\ZREF@abspos@paper@y@top\ZREF@abspos@media@y@top
%    \end{macrocode}
%    \end{macro}
%    \begin{macro}{\ZREF@abspos@paper@y@bottom}
%    \begin{macrocode}
\def\ZREF@abspos@paper@y@bottom#1{%
  \ZREF@abspos@paper@y@top{#1}%
  -\zref@extract{#1}{paperheight}%
}
%    \end{macrocode}
%    \end{macro}
%    \begin{macro}{\ZREF@abspos@paper@y@center}
%    \begin{macrocode}
\def\ZREF@abspos@paper@y@center#1{%
  \ZREF@abspos@paper@y@top{#1}%
  -\zref@extract{#1}{paperheight}/2%
}
%    \end{macrocode}
%    \end{macro}
%
% \subsubsection{Origin}
%
% There doesn't seem a good reason to make these tests depend on pdf mode
%  in current engines, so comment out the \verb|\ifpdf| tests.
%    \begin{macro}{\ZREF@abspos@origin@x}
%    \begin{macrocode}
\let\ZREF@temp\ltx@two
\ltx@IfUndefined{pdfhorigin}{}{%
%  \ifpdf
    \let\ZREF@temp\ltx@zero
%  \fi
}
\ifluatex
% \ifpdf
  \let\ZREF@temp\ltx@zero
% \fi
\fi

\ifx\ZREF@temp\ltx@two
  \ifnum\mag=1000 %
    \let\ZREF@temp\ltx@one
  \fi
\fi
\ifcase\ZREF@temp
  \def\ZREF@abspos@origin@x#1{%
    \zref@extract{#1}{pdfhorigin}%
  }%
\or
  \def\ZREF@abspos@origin@x#1{%
    4736286%
  }%
\or
  \def\ZREF@abspos@origin@x#1{%
    \numexpr\mag/1000*\dimexpr 1truein\relax\relax
  }%
\fi
%    \end{macrocode}
%    \end{macro}
%    \begin{macro}{\ZREF@abspos@origin@y}
%    \begin{macrocode}
\let\ZREF@temp\ltx@two
\ltx@IfUndefined{pdfvorigin}{}{%
%  \ifpdf
    \let\ZREF@temp\ltx@zero
%  \fi
}
\ifluatex
%  \ifpdf
    \let\ZREF@temp\ltx@zero
%  \fi
\fi
\ifx\ZREF@temp\ltx@two
  \ifnum\mag=1000 %
    \let\ZREF@temp\ltx@one
  \fi
\fi
\ifcase\ZREF@temp
  \def\ZREF@abspos@origin@y#1{%
    \zref@extract{#1}{pdfvorigin}%
  }%
\or
  \def\ZREF@abspos@origin@y#1{%
    4736286%
  }%
\or
  \def\ZREF@abspos@origin@y#1{%
    \numexpr\mag/1000*\dimexpr 1truein\relax\relax
   }%
\fi
%    \end{macrocode}
%    \end{macro}
%
% \subsubsection{Header}
%
%    \begin{macro}{\ZREF@abspos@head@x@left}
%    \begin{macrocode}
\def\ZREF@abspos@head@x@left#1{%
  \ZREF@abspos@paper@x@left{#1}%
  +\ZREF@abspos@origin@x{#1}%
  +\zref@extract{#1}{hoffset}%
  +\ifodd\zref@extractdefault{#1}{pagevalue}{\number\c@page} %
     \zref@extract{#1}{oddsidemargin}%
   \else
     \zref@extract{#1}{evensidemargin}%
   \fi
}
%    \end{macrocode}
%    \end{macro}
%    \begin{macro}{\ZREF@abspos@head@x@right}
%    \begin{macrocode}
\def\ZREF@abspos@head@x@right#1{%
  \ZREF@abspos@head@x@left{#1}%
  +\zref@extract{#1}{textwidth}%
}
%    \end{macrocode}
%    \end{macro}
%    \begin{macro}{\ZREF@abspos@head@x@center}
%    \begin{macrocode}
\def\ZREF@abspos@head@x@center#1{%
  \ZREF@abspos@head@x@left{#1}%
  +\zref@extract{#1}{textwidth}/2%
}
%    \end{macrocode}
%    \end{macro}
%
%    \begin{macro}{\ZREF@abspos@head@y@top}
%    \begin{macrocode}
\def\ZREF@abspos@head@y@top#1{%
  \ZREF@abspos@paper@y@top{#1}%
  -\ZREF@abspos@origin@y{#1}%
  -\zref@extract{#1}{voffset}%
  -\zref@extract{#1}{topmargin}%
}
%    \end{macrocode}
%    \end{macro}
%    \begin{macro}{\ZREF@abspos@head@y@bottom}
%    \begin{macrocode}
\def\ZREF@abspos@head@y@bottom#1{%
  \ZREF@abspos@head@y@top{#1}%
  -\zref@extract{#1}{headheight}%
}
%    \end{macrocode}
%    \end{macro}
%    \begin{macro}{\ZREF@abspos@head@y@center}
%    \begin{macrocode}
\def\ZREF@abspos@head@y@center#1{%
  \ZREF@abspos@head@y@top{#1}%
  -\zref@extract{#1}{headheight}/2%
}
%    \end{macrocode}
%    \end{macro}
%
% \subsubsection{Body}
%
%    \begin{macro}{\ZREF@abspos@body@x@left}
%    \begin{macrocode}
\let\ZREF@abspos@body@x@left\ZREF@abspos@head@x@left
%    \end{macrocode}
%    \end{macro}
%    \begin{macro}{\ZREF@abspos@body@x@right}
%    \begin{macrocode}
\let\ZREF@abspos@body@x@right\ZREF@abspos@head@x@right
%    \end{macrocode}
%    \end{macro}
%    \begin{macro}{\ZREF@abspos@body@x@center}
%    \begin{macrocode}
\let\ZREF@abspos@body@x@center\ZREF@abspos@head@x@center
%    \end{macrocode}
%    \end{macro}
%
%    \begin{macro}{\ZREF@abspos@body@y@top}
%    \begin{macrocode}
\def\ZREF@abspos@body@y@top#1{%
  \ZREF@abspos@head@y@bottom{#1}%
  -\zref@extract{#1}{headsep}%
}
%    \end{macrocode}
%    \end{macro}
%    \begin{macro}{\ZREF@abspos@body@y@bottom}
%    \begin{macrocode}
\def\ZREF@abspos@body@y@bottom#1{%
  \ZREF@abspos@body@y@top{#1}%
  -\zref@extract{#1}{textheight}%
}
%    \end{macrocode}
%    \end{macro}
%    \begin{macro}{\ZREF@abspos@body@y@center}
%    \begin{macrocode}
\def\ZREF@abspos@body@y@center#1{%
  \ZREF@abspos@body@y@top{#1}%
  -\zref@extract{#1}{textheight}/2%
}
%    \end{macrocode}
%    \end{macro}
%
% \subsubsection{Footer}
%
%    \begin{macro}{\ZREF@abspos@foot@x@left}
%    \begin{macrocode}
\let\ZREF@abspos@foot@x@left\ZREF@abspos@head@x@left
%    \end{macrocode}
%    \end{macro}
%    \begin{macro}{\ZREF@abspos@foot@x@right}
%    \begin{macrocode}
\let\ZREF@abspos@foot@x@right\ZREF@abspos@head@x@right
%    \end{macrocode}
%    \end{macro}
%    \begin{macro}{\ZREF@abspos@foot@x@center}
%    \begin{macrocode}
\let\ZREF@abspos@foot@x@center\ZREF@abspos@head@x@center
%    \end{macrocode}
%    \end{macro}
%
%    \begin{macro}{\ZREF@abspos@foot@y@bottom}
%    \begin{macrocode}
\def\ZREF@abspos@foot@y@bottom#1{%
  \ZREF@abspos@body@y@bottom{#1}%
  -\zref@extract{#1}{footskip}%
}
%    \end{macrocode}
%    \end{macro}
%
% \subsubsection{Marginal notes}
%
%    \begin{macro}{\ZREF@abspos@marginpar@x@left}
%    \begin{macrocode}
\def\ZREF@abspos@marginpar@x@left#1{%
  \ifodd\zref@extractdefault{#1}{pagevalue}{\number\c@page} %
    \ZREF@abspos@body@x@right{#1}%
    +\zref@extract{#1}{marginparsep}%
  \else
    \ZREF@abspos@body@x@left{#1}%
    -\zref@extract{#1}{marginparsep}%
    -\zref@extract{#1}{marginparwidth}%
  \fi
}
%    \end{macrocode}
%    \end{macro}
%    \begin{macro}{\ZREF@abspos@marginpar@x@right}
%    \begin{macrocode}
\def\ZREF@abspos@marginpar@x@right#1{%
  \ZREF@abspos@marginpar@x@left{#1}%
  +\zref@extract{#1}{marginparwidth}%
}
%    \end{macrocode}
%    \end{macro}
%    \begin{macro}{\ZREF@abspos@marginpar@x@center}
%    \begin{macrocode}
\def\ZREF@abspos@marginpar@x@center#1{%
  \ZREF@abspos@marginpar@x@left{#1}%
  +\zref@extract{#1}{marginparwidth}/2%
}
%    \end{macrocode}
%    \end{macro}
%
%    \begin{macro}{\ZREF@abspos@marginpar@y@top}
%    \begin{macrocode}
\let\ZREF@abspos@marginpar@y@top\ZREF@abspos@body@y@top
%    \end{macrocode}
%    \end{macro}
%    \begin{macro}{\ZREF@abspos@marginpar@y@bottom}
%    \begin{macrocode}
\let\ZREF@abspos@marginpar@y@bottom\ZREF@abspos@body@y@bottom
%    \end{macrocode}
%    \end{macro}
%    \begin{macro}{\ZREF@abspos@marginpar@y@center}
%    \begin{macrocode}
\let\ZREF@abspos@marginpar@y@center\ZREF@abspos@body@y@center
%    \end{macrocode}
%    \end{macro}
%
% \subsubsection{Stock paper}
%
%    \begin{macro}{\ZREF@abspos@stock@x@left}
%    \begin{macrocode}
\let\ZREF@abspos@stock@x@left\ZREF@abspos@paper@x@left
%    \end{macrocode}
%    \end{macro}
%    \begin{macro}{\ZREF@abspos@stock@x@right}
%    \begin{macrocode}
\let\ZREF@abspos@stock@x@right\ZREF@abspos@paper@x@right
%    \end{macrocode}
%    \end{macro}
%    \begin{macro}{\ZREF@abspos@stock@x@center}
%    \begin{macrocode}
\let\ZREF@abspos@stock@x@center\ZREF@abspos@paper@x@center
%    \end{macrocode}
%    \end{macro}
%
%    \begin{macro}{\ZREF@abspos@stock@y@top}
%    \begin{macrocode}
\let\ZREF@abspos@stock@y@top\ZREF@abspos@paper@y@top
%    \end{macrocode}
%    \end{macro}
%    \begin{macro}{\ZREF@abspos@stock@y@bottom}
%    \begin{macrocode}
\let\ZREF@abspos@stock@y@bottom\ZREF@abspos@paper@y@bottom
%    \end{macrocode}
%    \end{macro}
%    \begin{macro}{\ZREF@abspos@stock@y@center}
%    \begin{macrocode}
\let\ZREF@abspos@stock@y@center\ZREF@abspos@paper@y@center
%    \end{macrocode}
%    \end{macro}
%
%    \begin{macrocode}
%</abspos>
%    \end{macrocode}
%
%
% \subsection{Module \xmodule{dotfill}}
%
%    \begin{macrocode}
%<*dotfill>
\NeedsTeXFormat{LaTeX2e}
\ProvidesPackage{zref-dotfill}%
  [2020-07-03 v2.32 Module dotfill for zref (HO)]%
\RequirePackage{zref-base}[2019/11/29]
\ifx\ZREF@base@ok Y%
\else
  \expandafter\endinput
\fi
%    \end{macrocode}
%
%    For measuring the width of \cs{zdotfill} we use the features
%    provided by module \xmodule{savepos}.
%    \begin{macrocode}
\RequirePackage{zref-savepos}[2019/11/29]
%    \end{macrocode}
%    For automatically generated label names we use the unique
%    counter of module \xmodule{base}.
%    \begin{macrocode}
\zref@require@unique
%    \end{macrocode}
%    Configuration is done by the key value interface of package
%    \xpackage{keyval}.
%    \begin{macrocode}
\RequirePackage{keyval}
%    \end{macrocode}
%    The definitions of the keys follow.
%    \begin{macrocode}
\define@key{ZREF@DF}{unit}{%
  \def\ZREF@df@unit{#1}%
}
\define@key{ZREF@DF}{min}{%
  \def\ZREF@df@min{#1}%
}
\define@key{ZREF@DF}{dot}{%
  \def\ZREF@df@dot{#1}%
}
%    \end{macrocode}
%    Defaults are set, see user interface.
%    \begin{macrocode}
\providecommand\ZREF@df@min{2}
\providecommand\ZREF@df@unit{.44em}
\providecommand\ZREF@df@dot{.}
%    \end{macrocode}
%    \begin{macro}{\zdotfillsetup}
%    Configuration of \cs{zdotfill} is done by \cs{zdotfillsetup}.
%    \begin{macrocode}
\newcommand*{\zdotfillsetup}{\kvsetkeys{ZREF@DF}}
%    \end{macrocode}
%    \end{macro}
%
%    \begin{macro}{\zdotfill}
%    \cs{zdotfill} sets labels at the left and the right to get
%    the horizontal position. \cs{zsavepos} is not used, because
%    we do not need the vertical position.
%    \begin{macrocode}
\ZREF@IfDefinable\zdotfill\def{%
  {%
    \leavevmode
    \global\advance\c@zref@unique\ltx@one
    \begingroup
      \def\ZREF@temp{zref@\number\c@zref@unique}%
      \pdfsavepos
      \zref@labelbyprops{\thezref@unique L}{posx}%
      \setlength{\dimen@}{\ZREF@df@unit}%
      \zref@ifrefundefined{\thezref@unique R}{%
        \ZREF@dotfill
      }{%
        \ifnum\numexpr\zposx{\thezref@unique R}%
                      -\zposx{\thezref@unique L}\relax
            <\dimexpr\ZREF@df@min\dimen@\relax
          \hfill
        \else
          \ZREF@dotfill
        \fi
      }%
      \pdfsavepos
      \zref@labelbyprops{\thezref@unique R}{posx}%
    \endgroup
    \kern\z@
  }%
}
%    \end{macrocode}
%    \end{macro}
%    \begin{macro}{\ZREF@dotfill}
%    Help macro that actually sets the dots.
%    \begin{macrocode}
\def\ZREF@dotfill{%
  \cleaders\hb@xt@\dimen@{\hss\ZREF@df@dot\hss}\hfill
}
%    \end{macrocode}
%    \end{macro}
%
%    \begin{macrocode}
%</dotfill>
%    \end{macrocode}
%
% \subsection{Module \xmodule{env}}
%
%    \begin{macrocode}
%<*env>
\NeedsTeXFormat{LaTeX2e}
\ProvidesPackage{zref-env}%
  [2020-07-03 v2.32 Module env for zref (HO)]%
\RequirePackage{zref-base}[2019/11/29]
\ifx\ZREF@base@ok Y%
\else
  \expandafter\endinput
\fi
%    \end{macrocode}
%
%    \begin{macrocode}
\zref@newprop{envname}[]{\@currenvir}
\zref@newprop{envline}[]{\zref@env@line}
%    \end{macrocode}
%
%    \begin{macro}{\zref@env@line}
%    Macro \cs{zref@env@line} extracts the line number
%    from \cs{@currenvline}.
%    \begin{macrocode}
\def\zref@env@line{%
  \ifx\@currenvline\ltx@empty
  \else
    \expandafter
    \ZREF@ENV@line\@currenvline\ltx@empty line \ltx@empty\@nil
  \fi
}
%    \end{macrocode}
%    \end{macro}
%    \begin{macro}{\ZREF@ENV@line}
%    \begin{macrocode}
\def\ZREF@ENV@line#1line #2\ltx@empty#3\@nil{#2}%
%    \end{macrocode}
%    \end{macro}
%
%    \begin{macrocode}
%</env>
%    \end{macrocode}
%
%% \section{Installation}
%
% \subsection{Download}
%
% \paragraph{Package.} This package is available on
% CTAN\footnote{\CTANpkg{zref}}:
% \begin{description}
% \item[\CTAN{macros/latex/contrib/zref/zref.dtx}] The source file.
% \item[\CTAN{macros/latex/contrib/zref/zref.pdf}] Documentation.
% \end{description}
%
%
% \paragraph{Bundle.} All the packages of the bundle `zref'
% are also available in a TDS compliant ZIP archive. There
% the packages are already unpacked and the documentation files
% are generated. The files and directories obey the TDS standard.
% \begin{description}
% \item[\CTANinstall{install/macros/latex/contrib/zref.tds.zip}]
% \end{description}
% \emph{TDS} refers to the standard ``A Directory Structure
% for \TeX\ Files'' (\CTANpkg{tds}). Directories
% with \xfile{texmf} in their name are usually organized this way.
%
% \subsection{Bundle installation}
%
% \paragraph{Unpacking.} Unpack the \xfile{zref.tds.zip} in the
% TDS tree (also known as \xfile{texmf} tree) of your choice.
% Example (linux):
% \begin{quote}
%   |unzip zref.tds.zip -d ~/texmf|
% \end{quote}
%
%
% \subsection{Package installation}
%
% \paragraph{Unpacking.} The \xfile{.dtx} file is a self-extracting
% \docstrip\ archive. The files are extracted by running the
% \xfile{.dtx} through \plainTeX:
% \begin{quote}
%   \verb|tex zref.dtx|
% \end{quote}
%
% \paragraph{TDS.} Now the different files must be moved into
% the different directories in your installation TDS tree
% (also known as \xfile{texmf} tree):
% \begin{quote}
% \def\t{^^A
% \begin{tabular}{@{}>{\ttfamily}l@{ $\rightarrow$ }>{\ttfamily}l@{}}
%   zref.sty & tex/latex/zref/zref.sty\\
%   zref-base.sty & tex/latex/zref/zref-base.sty\\
%   zref-abspage.sty & tex/latex/zref/zref-abspage.sty\\
%   zref-abspos.sty & tex/latex/zref/zref-abspos.sty\\
%   zref-counter.sty & tex/latex/zref/zref-counter.sty\\
%   zref-dotfill.sty & tex/latex/zref/zref-dotfill.sty\\
%   zref-env.sty & tex/latex/zref/zref-env.sty\\
%   zref-hyperref.sty & tex/latex/zref/zref-hyperref.sty\\
%   zref-lastpage.sty & tex/latex/zref/zref-lastpage.sty\\
%   zref-marks.sty & tex/latex/zref/zref-marks.sty\\
%   zref-nextpage.sty & tex/latex/zref/zref-nextpage.sty\\
%   zref-pageattr.sty & tex/latex/zref/zref-pageattr.sty\\
%   zref-pagelayout.sty & tex/latex/zref/zref-pagelayout.sty\\
%   zref-perpage.sty & tex/latex/zref/zref-perpage.sty\\
%   zref-runs.sty & tex/latex/zref/zref-runs.sty\\
%   zref-savepos.sty & tex/latex/zref/zref-savepos.sty\\
%   zref-thepage.sty & tex/latex/zref/zref-thepage.sty\\
%   zref-titleref.sty & tex/latex/zref/zref-titleref.sty\\
%   zref-totpages.sty & tex/latex/zref/zref-totpages.sty\\
%   zref-user.sty & tex/latex/zref/zref-user.sty\\
%   zref-xr.sty & tex/latex/zref/zref-xr.sty\\
%   zref.pdf & doc/latex/zref/zref.pdf\\
%   zref-example.tex & doc/latex/zref/zref-example.tex\\
%   zref-example-lastpage.tex & doc/latex/zref/zref-example-lastpage.tex\\
%   zref-example-nextpage.tex & doc/latex/zref/zref-example-nextpage.tex\\
%   zref.dtx & source/latex/zref/zref.dtx\\
% \end{tabular}^^A
% }^^A
% \sbox0{\t}^^A
% \ifdim\wd0>\linewidth
%   \begingroup
%     \advance\linewidth by\leftmargin
%     \advance\linewidth by\rightmargin
%   \edef\x{\endgroup
%     \def\noexpand\lw{\the\linewidth}^^A
%   }\x
%   \def\lwbox{^^A
%     \leavevmode
%     \hbox to \linewidth{^^A
%       \kern-\leftmargin\relax
%       \hss
%       \usebox0
%       \hss
%       \kern-\rightmargin\relax
%     }^^A
%   }^^A
%   \ifdim\wd0>\lw
%     \sbox0{\small\t}^^A
%     \ifdim\wd0>\linewidth
%       \ifdim\wd0>\lw
%         \sbox0{\footnotesize\t}^^A
%         \ifdim\wd0>\linewidth
%           \ifdim\wd0>\lw
%             \sbox0{\scriptsize\t}^^A
%             \ifdim\wd0>\linewidth
%               \ifdim\wd0>\lw
%                 \sbox0{\tiny\t}^^A
%                 \ifdim\wd0>\linewidth
%                   \lwbox
%                 \else
%                   \usebox0
%                 \fi
%               \else
%                 \lwbox
%               \fi
%             \else
%               \usebox0
%             \fi
%           \else
%             \lwbox
%           \fi
%         \else
%           \usebox0
%         \fi
%       \else
%         \lwbox
%       \fi
%     \else
%       \usebox0
%     \fi
%   \else
%     \lwbox
%   \fi
% \else
%   \usebox0
% \fi
% \end{quote}
% If you have a \xfile{docstrip.cfg} that configures and enables \docstrip's
% TDS installing feature, then some files can already be in the right
% place, see the documentation of \docstrip.
%
% \subsection{Refresh file name databases}
%
% If your \TeX~distribution
% (\TeX\,Live, \mikTeX, \dots) relies on file name databases, you must refresh
% these. For example, \TeX\,Live\ users run \verb|texhash| or
% \verb|mktexlsr|.
%
% \subsection{Some details for the interested}
%
% \paragraph{Unpacking with \LaTeX.}
% The \xfile{.dtx} chooses its action depending on the format:
% \begin{description}
% \item[\plainTeX:] Run \docstrip\ and extract the files.
% \item[\LaTeX:] Generate the documentation.
% \end{description}
% If you insist on using \LaTeX\ for \docstrip\ (really,
% \docstrip\ does not need \LaTeX), then inform the autodetect routine
% about your intention:
% \begin{quote}
%   \verb|latex \let\install=y\input{zref.dtx}|
% \end{quote}
% Do not forget to quote the argument according to the demands
% of your shell.
%
% \paragraph{Generating the documentation.}
% You can use both the \xfile{.dtx} or the \xfile{.drv} to generate
% the documentation. The process can be configured by the
% configuration file \xfile{ltxdoc.cfg}. For instance, put this
% line into this file, if you want to have A4 as paper format:
% \begin{quote}
%   \verb|\PassOptionsToClass{a4paper}{article}|
% \end{quote}
% An example follows how to generate the
% documentation with pdf\LaTeX:
% \begin{quote}
%\begin{verbatim}
%pdflatex zref.dtx
%makeindex -s gind.ist zref.idx
%pdflatex zref.dtx
%makeindex -s gind.ist zref.idx
%pdflatex zref.dtx
%\end{verbatim}
% \end{quote}
%
% \begin{thebibliography}{99}
% \bibpackage{footmisc}{Robin Fairbairns}{2004/01/23 v5.3a}
%   {\CTANpkg{footmisc}}
% \bibpackage{hyperref}{Sebastian Rahtz, Heiko Oberdiek}{2006/08/16 v6.75c}
%   {\CTANpkg{hyperref}}
% \bibpackage{lastpage}{Jeff Goldberg}{1994/06/25 v0.1b}
%   {\CTANpkg{lastpage}}
% \bibpackage{nameref}{Sebastian Rahtz, Heiko Oberdiek}{2006/02/12 v2.24}
%   {\CTANpkg{nameref}}
% \bibpackage{perpage}{David Kastrup}{2002/12/20 v1.0}
%   {\CTANpkg{perpage}}
% \bibpackage{titleref}{Donald Arsenau}{2001/04/05 v3.1}
%   {\CTANpkg{titleref}}
% \bibpackage{totpages}{Wilhelm M\"uller}{1999/07/14 v1.00}
%   {\CTANpkg{totpages}}
% \bibpackage{xr}{David Carlisle}{1994/05/28 v5.02}
%   {\CTANpkg{xr}}
% \bibpackage{xr-hyper}{David Carlisle}{2000/03/22 v6.00beta4}
%   {\CTANpkg{xr-hyper}}
% \end{thebibliography}
%
% \begin{History}
%   \begin{Version}{2006/02/20 v1.0}
%   \item
%     First version.
%   \end{Version}
%   \begin{Version}{2006/05/03 v1.1}
%   \item
%     Module \xmodule{perpage} added.
%   \item
%     Module redesign as packages.
%   \end{Version}
%   \begin{Version}{2006/05/25 v1.2}
%   \item
%     Module \xmodule{dotfillmin} added.
%   \item
%     Module \xmodule{base}: macros \cs{zref@require@uniqe}
%     and \cs{thezref@unique} added
%     (used by modules \xmodule{titleref} and \xmodule{dotfillmin}).
%   \end{Version}
%   \begin{Version}{2006/09/08 v1.3}
%   \item
%     Typo fixes and English cleanup by Per Starback.
%   \end{Version}
%   \begin{Version}{2007/01/23 v1.4}
%   \item
%     Typo in macro name fixed in documentation.
%   \end{Version}
%   \begin{Version}{2007/02/18 v1.5}
%   \item
%     \cs{zref@getcurrent} added (suggestion of Igor Akkerman).
%   \item
%     Module \xmodule{savepos} also supports \hologo{XeTeX}.
%   \end{Version}
%   \begin{Version}{2007/04/06 v1.6}
%   \item
%     Fix in modules \xmodule{abspage} and \xmodule{base}:
%     Now counter |abspage| and |zref@unique|
%     are not remembered by \cs{include}.
%   \item
%     Beamer support for module \xmodule{titleref}.
%   \end{Version}
%   \begin{Version}{2007/04/17 v1.7}
%   \item
%     Package \xpackage{atbegshi} replaces \xpackage{everyshi}.
%   \end{Version}
%   \begin{Version}{2007/04/22 v1.8}
%   \item
%     \cs{zref@wrapper@babel} and \cs{zref@refused} are now expandable
%     if babel is not used or \cs{if@safe@actives} is already set to true.
%     (Feature request of Josselin Noirel)
%   \end{Version}
%   \begin{Version}{2007/05/02 v1.9}
%   \item
%     Module \xmodule{titleref}: Some support for \cs{caption} of package
%     \xpackage{longtable}, but only if \cs{label} is given after
%     \cs{caption}.
%   \end{Version}
%   \begin{Version}{2007/05/06 v2.0}
%   \item
%     Uses package \xpackage{etexcmds} for accessing \hologo{eTeX}'s \cs{unexpanded}.
%   \end{Version}
%   \begin{Version}{2007/05/28 v2.1}
%   \item
%     Module \xmodule{titleref} supports caption of package \xpackage{listings}.
%   \item
%     Fixes in module \xmodule{titleref} for support
%     of packages \xpackage{titlesec} and \xpackage{longtable}.
%   \end{Version}
%   \begin{Version}{2008/09/21 v2.2}
%   \item
%     Module \xmodule{base}: \cs{zref@iflistcontainsprop} is documented, but
%     a broken \cs{zref@listcontainsprop} implemented. Name and
%     implementation fixed (thanks Ohad Kammar).
%   \end{Version}
%   \begin{Version}{2008/10/01 v2.3}
%   \item
%     \cs{zref@localaddprop} added (feature request of Ohad Kammar).
%   \item
%     Module \xmodule{lastpage}: list `LastPage' added. Label `LastPage' will use
%     the properties of this list (default is empty) along with the
%     properties of the main list.
%   \end{Version}
%   \begin{Version}{2009/08/07 v2.4}
%   \item
%     Module \xmodule{runs} added.
%   \end{Version}
%   \begin{Version}{2009/12/06 v2.5}
%   \item
%     Module \xmodule{lastpage}: Uses package \xpackage{atveryend}.
%   \item
%     Module \xmodule{titleref}: Further commands are disabled
%     during string expansion, imported from package \xpackage{nameref}.
%   \end{Version}
%   \begin{Version}{2009/12/07 v2.6}
%   \item
%     Version date added for package \xpackage{atveryend}.
%   \end{Version}
%   \begin{Version}{2009/12/08 v2.7}
%   \item
%     Module \xmodule{titleref}: Use of package \xpackage{gettitlestring}.
%   \end{Version}
%   \begin{Version}{2010/03/26 v2.8}
%   \item
%     \cs{zifrefundefined} added.
%   \item
%     Module \xmodule{lastpage}: Macros \cs{zref@iflastpage}
%     and \cs{ziflastpage} added.
%   \item
%     Module \xmodule{thepage} added.
%   \item
%     Module \xmodule{nextpage} added.
%   \end{Version}
%   \begin{Version}{2010/03/29 v2.9}
%   \item
%     Module \xmodule{marks} added (without documentation).
%   \item
%     \cs{zref@addprop} now adds expanded property to list.
%   \item
%     Useless \cs{ZREF@ErrorNoLine} removed.
%   \end{Version}
%   \begin{Version}{2010/04/08 v2.10}
%   \item
%     Module \xmodule{xr} remembers the external document name
%     in property `externaldocument'.
%   \end{Version}
%   \begin{Version}{2010/04/15 v2.11}
%   \item
%     Module \xmodule{titleref}: Better support of class \xclass{memoir}.
%   \item
%     Module \xmodule{titleref}: Support of theorems.
%   \end{Version}
%   \begin{Version}{2010/04/17 v2.12}
%   \item
%     Module \xmodule{base}: \cs{zref@newprop} ensures global
%     empty default.
%   \item
%     Module \xmodule{xr}: Setup options \xoption{tozreflabel}
%     and \xoption{toltxlabel} added.
%   \end{Version}
%   \begin{Version}{2010/04/19 v2.13}
%   \item
%     \cs{zref@setcurrent} throws an error if the property
%     does not exist (Florent Chervet).
%   \item
%     \cs{zref@getcurrent} the documentation is fixed (Florent Chervet).
%     Also it returns the empty string in case of errors.
%   \item
%     \cs{zref@addprop} and \cs{zref@localaddprop} now take a list of
%     property names (feature request of Florent Chervet).
%   \item
%     Example for \cs{zref@wrapper@unexpanded} corrected (Florent Chervet).
%   \end{Version}
%   \begin{Version}{2010/04/22 v2.14}
%   \item
%     Bug fix for \cs{zref@getcurrent} second argument wasn't eaten in
%     case of unknown property.
%   \item
%     \cs{zref@getcurrent} supports \cs{zref@wrapper@unexpanded}.
%   \item
%     \cs{zref@wrapper@unexpanded} added for \cs{ZREF@xr@tolabel}.
%   \item
%     \cs{zref@extract}, \cs{zref@extractdefault}, \cs{zref@getcurrent}
%     are expandable in exact two steps
%     except inside \cs{zref@wrapper@unexpanded}.
%   \end{Version}
%   \begin{Version}{2010/04/23 v2.15}
%   \item
%     \cs{zexternaldocument} fixed for property `url' when
%     importing \cs{new@label} (bug found by Victor Ivrii).
%   \item
%     Two expansion steps also in \cs{zref@wrapper@unexpanded}.
%   \item
%     Nested calls of \cs{zref@wrapper@unexpanded} possible.
%   \end{Version}
%   \begin{Version}{2010/04/28 v2.16}
%   \item
%     More consequent use of package `ltxcmds' and `hologo'.
%   \item
%     Module \xmodule{pagelayout} added.
%   \item
%     Module \xmodule{pageattr} added.
%   \item
%     Robustness introduced for non-expandable interface macros.
%   \item
%     Internal change of the data format of property lists
%     (suggestion of Florent Chervet).
%   \item
%     Module \xmodule{titleref}: Support of environment \textsf{description}.
%   \end{Version}
%   \begin{Version}{2010/05/01 v2.17}
%   \item
%     \cs{zref@newprop} throws an error if the property already exists.
%   \item
%     Module \xmodule{xr}: Bug fix for the case of several \xext{aux} files
%     (bug found by Victor Ivrii).
%   \item
%     Module \xmodule{xr}: Property `urluse' and option \xoption{urluse} added.
%   \end{Version}
%   \begin{Version}{2010/05/13 v2.18}
%   \item
%     Module \xmodule{env} added.
%   \item
%     Module \xmodule{savepos}: \cs{zref@savepos} added.
%   \end{Version}
%   \begin{Version}{2010/10/22 v2.19}
%   \item
%     \cs{zref@addprop} and \cs{zref@localaddprop} are limited
%     to one property only
%     (incompatibility to versions v2.13 to v2.18).
%   \item
%     \cs{zref@addprops} and \cs{zref@localaddprops} added.
%   \item
%     \cs{zref@delprop} and \cs{zref@localdelprop} added.
%   \item
%     \cs{zref@labelbykv} and \cs{zkvlabel} (module \xmodule{user})
%     with keys \xoption{prop}, \xoption{list}, \xoption{delprop},
%     \xoption{immediate}, \xoption{values} added.
%   \end{Version}
%   \begin{Version}{2011/02/12 v2.20}
%   \item
%     Fix for warning in \xpackage{zref-xr}.
%   \end{Version}
%   \begin{Version}{2011/03/18 v2.21}
%   \item
%     Fix in module \xmodule{pagelayout} for \cs{zlistpagelayout}.
%   \item
%     Fix for \cs{zref@localaddprop} (probably since v2.19).
%   \end{Version}
%   \begin{Version}{2011/10/05 v2.22}
%   \item
%     Documentation fixed for \cs{zref@(local)addprop(s)}.
%   \item
%     Module \xmodule{base}: \cs{zref@def@extract}, \cs{zref@def@extractdefault}
%     added.
%   \item
%     Fix in module \xmodule{pagelayout}: Because of missing \cs{noexpand}
%     commands the values of the pagelayout properties on all pages were
%     the values at package loading.
%   \item
%     Module \xmodule{base}: \cs{zref@showprop} added.
%   \end{Version}
%   \begin{Version}{2011/12/05 v2.23}
%   \item
%     Module \xmodule{savepos}: \cs{zsaveposx} and \cs{zsaveposy} added.
%   \end{Version}
%   \begin{Version}{2012/04/04 v2.24}
%   \item
%     Module \xmodule{titleref}, package \xpackage{titlesec}:
%     some support for class `straight' (\cs{ttl@straight@i}) added.
%   \end{Version}
%   \begin{Version}{2016/05/16 v2.25}
%   \item
%     Documentation updates.
%   \end{Version}
%   \begin{Version}{2016/05/21 v2.26}
%   \item
%     update zref-savepos for new luatex
%   \end{Version}
%   \begin{Version}{2018/11/21 v2.27}
%   \item
%     adapted zref-perpage, see issue https://github.com/ho-tex/zref/issues/2
%   \end{Version}
%   \begin{Version}{2019/11/29 v2.28}
%   \item
%     Documentation updates.
%   \item
%     Use \xpackage{iftex} directly.
%   \end{Version}
%   \begin{Version}{2020-03-03 v2.29}
%   \item
%     adapted in module zref-pagelayout the properties pdfhorigin, pdfvorigin,
%     pdfpagewidth, pdfpageheight for luatex to the right primitives.
%   \item
%     Removed no longer needed code for older lualatex versions.
%   \item added some documentation of the abspos module.
%   \item adapted the abspos module to the new luatex primitives.
%   \item adapted pageattr module to the new luatex primitives.
%   \item added short documentation for pageattr module
%   \item use luatex command names directly in zref-savepos rather than defining pdftex compatibility names.
%   \item allow zref-abspos to use \verb|\pdf[vh]origin| in dvi mode.
%   \end{Version}
%   \begin{Version}{2020-03-04 v2.30}
%   \item
%     add pagevalue property to savepos in the abspos module (issue 1)
%   \end{Version}
%   \begin{Version}{2020-05-28 v2.31}
%   \item Adapted module zref-counter to use \cs{@currentcounter} in the next
%   \LaTeX version (issue 5)
%   \end{Version}
%   \begin{Version}{2020-07-03 v2.32}
%   \item Changed in zref-pagelayout the names of the shipout box dimensions
%    to adapt to the new hook management.
%   \end{Version}
% \end{History}
%
% \PrintIndex
%
% \Finale
\endinput
|
% \end{quote}
% Do not forget to quote the argument according to the demands
% of your shell.
%
% \paragraph{Generating the documentation.}
% You can use both the \xfile{.dtx} or the \xfile{.drv} to generate
% the documentation. The process can be configured by the
% configuration file \xfile{ltxdoc.cfg}. For instance, put this
% line into this file, if you want to have A4 as paper format:
% \begin{quote}
%   \verb|\PassOptionsToClass{a4paper}{article}|
% \end{quote}
% An example follows how to generate the
% documentation with pdf\LaTeX:
% \begin{quote}
%\begin{verbatim}
%pdflatex zref.dtx
%makeindex -s gind.ist zref.idx
%pdflatex zref.dtx
%makeindex -s gind.ist zref.idx
%pdflatex zref.dtx
%\end{verbatim}
% \end{quote}
%
% \begin{thebibliography}{99}
% \bibpackage{footmisc}{Robin Fairbairns}{2004/01/23 v5.3a}
%   {\CTANpkg{footmisc}}
% \bibpackage{hyperref}{Sebastian Rahtz, Heiko Oberdiek}{2006/08/16 v6.75c}
%   {\CTANpkg{hyperref}}
% \bibpackage{lastpage}{Jeff Goldberg}{1994/06/25 v0.1b}
%   {\CTANpkg{lastpage}}
% \bibpackage{nameref}{Sebastian Rahtz, Heiko Oberdiek}{2006/02/12 v2.24}
%   {\CTANpkg{nameref}}
% \bibpackage{perpage}{David Kastrup}{2002/12/20 v1.0}
%   {\CTANpkg{perpage}}
% \bibpackage{titleref}{Donald Arsenau}{2001/04/05 v3.1}
%   {\CTANpkg{titleref}}
% \bibpackage{totpages}{Wilhelm M\"uller}{1999/07/14 v1.00}
%   {\CTANpkg{totpages}}
% \bibpackage{xr}{David Carlisle}{1994/05/28 v5.02}
%   {\CTANpkg{xr}}
% \bibpackage{xr-hyper}{David Carlisle}{2000/03/22 v6.00beta4}
%   {\CTANpkg{xr-hyper}}
% \end{thebibliography}
%
% \begin{History}
%   \begin{Version}{2006/02/20 v1.0}
%   \item
%     First version.
%   \end{Version}
%   \begin{Version}{2006/05/03 v1.1}
%   \item
%     Module \xmodule{perpage} added.
%   \item
%     Module redesign as packages.
%   \end{Version}
%   \begin{Version}{2006/05/25 v1.2}
%   \item
%     Module \xmodule{dotfillmin} added.
%   \item
%     Module \xmodule{base}: macros \cs{zref@require@uniqe}
%     and \cs{thezref@unique} added
%     (used by modules \xmodule{titleref} and \xmodule{dotfillmin}).
%   \end{Version}
%   \begin{Version}{2006/09/08 v1.3}
%   \item
%     Typo fixes and English cleanup by Per Starback.
%   \end{Version}
%   \begin{Version}{2007/01/23 v1.4}
%   \item
%     Typo in macro name fixed in documentation.
%   \end{Version}
%   \begin{Version}{2007/02/18 v1.5}
%   \item
%     \cs{zref@getcurrent} added (suggestion of Igor Akkerman).
%   \item
%     Module \xmodule{savepos} also supports \hologo{XeTeX}.
%   \end{Version}
%   \begin{Version}{2007/04/06 v1.6}
%   \item
%     Fix in modules \xmodule{abspage} and \xmodule{base}:
%     Now counter |abspage| and |zref@unique|
%     are not remembered by \cs{include}.
%   \item
%     Beamer support for module \xmodule{titleref}.
%   \end{Version}
%   \begin{Version}{2007/04/17 v1.7}
%   \item
%     Package \xpackage{atbegshi} replaces \xpackage{everyshi}.
%   \end{Version}
%   \begin{Version}{2007/04/22 v1.8}
%   \item
%     \cs{zref@wrapper@babel} and \cs{zref@refused} are now expandable
%     if babel is not used or \cs{if@safe@actives} is already set to true.
%     (Feature request of Josselin Noirel)
%   \end{Version}
%   \begin{Version}{2007/05/02 v1.9}
%   \item
%     Module \xmodule{titleref}: Some support for \cs{caption} of package
%     \xpackage{longtable}, but only if \cs{label} is given after
%     \cs{caption}.
%   \end{Version}
%   \begin{Version}{2007/05/06 v2.0}
%   \item
%     Uses package \xpackage{etexcmds} for accessing \hologo{eTeX}'s \cs{unexpanded}.
%   \end{Version}
%   \begin{Version}{2007/05/28 v2.1}
%   \item
%     Module \xmodule{titleref} supports caption of package \xpackage{listings}.
%   \item
%     Fixes in module \xmodule{titleref} for support
%     of packages \xpackage{titlesec} and \xpackage{longtable}.
%   \end{Version}
%   \begin{Version}{2008/09/21 v2.2}
%   \item
%     Module \xmodule{base}: \cs{zref@iflistcontainsprop} is documented, but
%     a broken \cs{zref@listcontainsprop} implemented. Name and
%     implementation fixed (thanks Ohad Kammar).
%   \end{Version}
%   \begin{Version}{2008/10/01 v2.3}
%   \item
%     \cs{zref@localaddprop} added (feature request of Ohad Kammar).
%   \item
%     Module \xmodule{lastpage}: list `LastPage' added. Label `LastPage' will use
%     the properties of this list (default is empty) along with the
%     properties of the main list.
%   \end{Version}
%   \begin{Version}{2009/08/07 v2.4}
%   \item
%     Module \xmodule{runs} added.
%   \end{Version}
%   \begin{Version}{2009/12/06 v2.5}
%   \item
%     Module \xmodule{lastpage}: Uses package \xpackage{atveryend}.
%   \item
%     Module \xmodule{titleref}: Further commands are disabled
%     during string expansion, imported from package \xpackage{nameref}.
%   \end{Version}
%   \begin{Version}{2009/12/07 v2.6}
%   \item
%     Version date added for package \xpackage{atveryend}.
%   \end{Version}
%   \begin{Version}{2009/12/08 v2.7}
%   \item
%     Module \xmodule{titleref}: Use of package \xpackage{gettitlestring}.
%   \end{Version}
%   \begin{Version}{2010/03/26 v2.8}
%   \item
%     \cs{zifrefundefined} added.
%   \item
%     Module \xmodule{lastpage}: Macros \cs{zref@iflastpage}
%     and \cs{ziflastpage} added.
%   \item
%     Module \xmodule{thepage} added.
%   \item
%     Module \xmodule{nextpage} added.
%   \end{Version}
%   \begin{Version}{2010/03/29 v2.9}
%   \item
%     Module \xmodule{marks} added (without documentation).
%   \item
%     \cs{zref@addprop} now adds expanded property to list.
%   \item
%     Useless \cs{ZREF@ErrorNoLine} removed.
%   \end{Version}
%   \begin{Version}{2010/04/08 v2.10}
%   \item
%     Module \xmodule{xr} remembers the external document name
%     in property `externaldocument'.
%   \end{Version}
%   \begin{Version}{2010/04/15 v2.11}
%   \item
%     Module \xmodule{titleref}: Better support of class \xclass{memoir}.
%   \item
%     Module \xmodule{titleref}: Support of theorems.
%   \end{Version}
%   \begin{Version}{2010/04/17 v2.12}
%   \item
%     Module \xmodule{base}: \cs{zref@newprop} ensures global
%     empty default.
%   \item
%     Module \xmodule{xr}: Setup options \xoption{tozreflabel}
%     and \xoption{toltxlabel} added.
%   \end{Version}
%   \begin{Version}{2010/04/19 v2.13}
%   \item
%     \cs{zref@setcurrent} throws an error if the property
%     does not exist (Florent Chervet).
%   \item
%     \cs{zref@getcurrent} the documentation is fixed (Florent Chervet).
%     Also it returns the empty string in case of errors.
%   \item
%     \cs{zref@addprop} and \cs{zref@localaddprop} now take a list of
%     property names (feature request of Florent Chervet).
%   \item
%     Example for \cs{zref@wrapper@unexpanded} corrected (Florent Chervet).
%   \end{Version}
%   \begin{Version}{2010/04/22 v2.14}
%   \item
%     Bug fix for \cs{zref@getcurrent} second argument wasn't eaten in
%     case of unknown property.
%   \item
%     \cs{zref@getcurrent} supports \cs{zref@wrapper@unexpanded}.
%   \item
%     \cs{zref@wrapper@unexpanded} added for \cs{ZREF@xr@tolabel}.
%   \item
%     \cs{zref@extract}, \cs{zref@extractdefault}, \cs{zref@getcurrent}
%     are expandable in exact two steps
%     except inside \cs{zref@wrapper@unexpanded}.
%   \end{Version}
%   \begin{Version}{2010/04/23 v2.15}
%   \item
%     \cs{zexternaldocument} fixed for property `url' when
%     importing \cs{new@label} (bug found by Victor Ivrii).
%   \item
%     Two expansion steps also in \cs{zref@wrapper@unexpanded}.
%   \item
%     Nested calls of \cs{zref@wrapper@unexpanded} possible.
%   \end{Version}
%   \begin{Version}{2010/04/28 v2.16}
%   \item
%     More consequent use of package `ltxcmds' and `hologo'.
%   \item
%     Module \xmodule{pagelayout} added.
%   \item
%     Module \xmodule{pageattr} added.
%   \item
%     Robustness introduced for non-expandable interface macros.
%   \item
%     Internal change of the data format of property lists
%     (suggestion of Florent Chervet).
%   \item
%     Module \xmodule{titleref}: Support of environment \textsf{description}.
%   \end{Version}
%   \begin{Version}{2010/05/01 v2.17}
%   \item
%     \cs{zref@newprop} throws an error if the property already exists.
%   \item
%     Module \xmodule{xr}: Bug fix for the case of several \xext{aux} files
%     (bug found by Victor Ivrii).
%   \item
%     Module \xmodule{xr}: Property `urluse' and option \xoption{urluse} added.
%   \end{Version}
%   \begin{Version}{2010/05/13 v2.18}
%   \item
%     Module \xmodule{env} added.
%   \item
%     Module \xmodule{savepos}: \cs{zref@savepos} added.
%   \end{Version}
%   \begin{Version}{2010/10/22 v2.19}
%   \item
%     \cs{zref@addprop} and \cs{zref@localaddprop} are limited
%     to one property only
%     (incompatibility to versions v2.13 to v2.18).
%   \item
%     \cs{zref@addprops} and \cs{zref@localaddprops} added.
%   \item
%     \cs{zref@delprop} and \cs{zref@localdelprop} added.
%   \item
%     \cs{zref@labelbykv} and \cs{zkvlabel} (module \xmodule{user})
%     with keys \xoption{prop}, \xoption{list}, \xoption{delprop},
%     \xoption{immediate}, \xoption{values} added.
%   \end{Version}
%   \begin{Version}{2011/02/12 v2.20}
%   \item
%     Fix for warning in \xpackage{zref-xr}.
%   \end{Version}
%   \begin{Version}{2011/03/18 v2.21}
%   \item
%     Fix in module \xmodule{pagelayout} for \cs{zlistpagelayout}.
%   \item
%     Fix for \cs{zref@localaddprop} (probably since v2.19).
%   \end{Version}
%   \begin{Version}{2011/10/05 v2.22}
%   \item
%     Documentation fixed for \cs{zref@(local)addprop(s)}.
%   \item
%     Module \xmodule{base}: \cs{zref@def@extract}, \cs{zref@def@extractdefault}
%     added.
%   \item
%     Fix in module \xmodule{pagelayout}: Because of missing \cs{noexpand}
%     commands the values of the pagelayout properties on all pages were
%     the values at package loading.
%   \item
%     Module \xmodule{base}: \cs{zref@showprop} added.
%   \end{Version}
%   \begin{Version}{2011/12/05 v2.23}
%   \item
%     Module \xmodule{savepos}: \cs{zsaveposx} and \cs{zsaveposy} added.
%   \end{Version}
%   \begin{Version}{2012/04/04 v2.24}
%   \item
%     Module \xmodule{titleref}, package \xpackage{titlesec}:
%     some support for class `straight' (\cs{ttl@straight@i}) added.
%   \end{Version}
%   \begin{Version}{2016/05/16 v2.25}
%   \item
%     Documentation updates.
%   \end{Version}
%   \begin{Version}{2016/05/21 v2.26}
%   \item
%     update zref-savepos for new luatex
%   \end{Version}
%   \begin{Version}{2018/11/21 v2.27}
%   \item
%     adapted zref-perpage, see issue https://github.com/ho-tex/zref/issues/2
%   \end{Version}
%   \begin{Version}{2019/11/29 v2.28}
%   \item
%     Documentation updates.
%   \item
%     Use \xpackage{iftex} directly.
%   \end{Version}
%   \begin{Version}{2020-03-03 v2.29}
%   \item
%     adapted in module zref-pagelayout the properties pdfhorigin, pdfvorigin,
%     pdfpagewidth, pdfpageheight for luatex to the right primitives.
%   \item
%     Removed no longer needed code for older lualatex versions.
%   \item added some documentation of the abspos module.
%   \item adapted the abspos module to the new luatex primitives.
%   \item adapted pageattr module to the new luatex primitives.
%   \item added short documentation for pageattr module
%   \item use luatex command names directly in zref-savepos rather than defining pdftex compatibility names.
%   \item allow zref-abspos to use \verb|\pdf[vh]origin| in dvi mode.
%   \end{Version}
%   \begin{Version}{2020-03-04 v2.30}
%   \item
%     add pagevalue property to savepos in the abspos module (issue 1)
%   \end{Version}
%   \begin{Version}{2020-05-28 v2.31}
%   \item Adapted module zref-counter to use \cs{@currentcounter} in the next
%   \LaTeX version (issue 5)
%   \end{Version}
%   \begin{Version}{2020-07-03 v2.32}
%   \item Changed in zref-pagelayout the names of the shipout box dimensions
%    to adapt to the new hook management.
%   \end{Version}
% \end{History}
%
% \PrintIndex
%
% \Finale
\endinput
|
% \end{quote}
% Do not forget to quote the argument according to the demands
% of your shell.
%
% \paragraph{Generating the documentation.}
% You can use both the \xfile{.dtx} or the \xfile{.drv} to generate
% the documentation. The process can be configured by the
% configuration file \xfile{ltxdoc.cfg}. For instance, put this
% line into this file, if you want to have A4 as paper format:
% \begin{quote}
%   \verb|\PassOptionsToClass{a4paper}{article}|
% \end{quote}
% An example follows how to generate the
% documentation with pdf\LaTeX:
% \begin{quote}
%\begin{verbatim}
%pdflatex zref.dtx
%makeindex -s gind.ist zref.idx
%pdflatex zref.dtx
%makeindex -s gind.ist zref.idx
%pdflatex zref.dtx
%\end{verbatim}
% \end{quote}
%
% \begin{thebibliography}{99}
% \bibpackage{footmisc}{Robin Fairbairns}{2004/01/23 v5.3a}
%   {\CTANpkg{footmisc}}
% \bibpackage{hyperref}{Sebastian Rahtz, Heiko Oberdiek}{2006/08/16 v6.75c}
%   {\CTANpkg{hyperref}}
% \bibpackage{lastpage}{Jeff Goldberg}{1994/06/25 v0.1b}
%   {\CTANpkg{lastpage}}
% \bibpackage{nameref}{Sebastian Rahtz, Heiko Oberdiek}{2006/02/12 v2.24}
%   {\CTANpkg{nameref}}
% \bibpackage{perpage}{David Kastrup}{2002/12/20 v1.0}
%   {\CTANpkg{perpage}}
% \bibpackage{titleref}{Donald Arsenau}{2001/04/05 v3.1}
%   {\CTANpkg{titleref}}
% \bibpackage{totpages}{Wilhelm M\"uller}{1999/07/14 v1.00}
%   {\CTANpkg{totpages}}
% \bibpackage{xr}{David Carlisle}{1994/05/28 v5.02}
%   {\CTANpkg{xr}}
% \bibpackage{xr-hyper}{David Carlisle}{2000/03/22 v6.00beta4}
%   {\CTANpkg{xr-hyper}}
% \end{thebibliography}
%
% \begin{History}
%   \begin{Version}{2006/02/20 v1.0}
%   \item
%     First version.
%   \end{Version}
%   \begin{Version}{2006/05/03 v1.1}
%   \item
%     Module \xmodule{perpage} added.
%   \item
%     Module redesign as packages.
%   \end{Version}
%   \begin{Version}{2006/05/25 v1.2}
%   \item
%     Module \xmodule{dotfillmin} added.
%   \item
%     Module \xmodule{base}: macros \cs{zref@require@uniqe}
%     and \cs{thezref@unique} added
%     (used by modules \xmodule{titleref} and \xmodule{dotfillmin}).
%   \end{Version}
%   \begin{Version}{2006/09/08 v1.3}
%   \item
%     Typo fixes and English cleanup by Per Starback.
%   \end{Version}
%   \begin{Version}{2007/01/23 v1.4}
%   \item
%     Typo in macro name fixed in documentation.
%   \end{Version}
%   \begin{Version}{2007/02/18 v1.5}
%   \item
%     \cs{zref@getcurrent} added (suggestion of Igor Akkerman).
%   \item
%     Module \xmodule{savepos} also supports \hologo{XeTeX}.
%   \end{Version}
%   \begin{Version}{2007/04/06 v1.6}
%   \item
%     Fix in modules \xmodule{abspage} and \xmodule{base}:
%     Now counter |abspage| and |zref@unique|
%     are not remembered by \cs{include}.
%   \item
%     Beamer support for module \xmodule{titleref}.
%   \end{Version}
%   \begin{Version}{2007/04/17 v1.7}
%   \item
%     Package \xpackage{atbegshi} replaces \xpackage{everyshi}.
%   \end{Version}
%   \begin{Version}{2007/04/22 v1.8}
%   \item
%     \cs{zref@wrapper@babel} and \cs{zref@refused} are now expandable
%     if babel is not used or \cs{if@safe@actives} is already set to true.
%     (Feature request of Josselin Noirel)
%   \end{Version}
%   \begin{Version}{2007/05/02 v1.9}
%   \item
%     Module \xmodule{titleref}: Some support for \cs{caption} of package
%     \xpackage{longtable}, but only if \cs{label} is given after
%     \cs{caption}.
%   \end{Version}
%   \begin{Version}{2007/05/06 v2.0}
%   \item
%     Uses package \xpackage{etexcmds} for accessing \hologo{eTeX}'s \cs{unexpanded}.
%   \end{Version}
%   \begin{Version}{2007/05/28 v2.1}
%   \item
%     Module \xmodule{titleref} supports caption of package \xpackage{listings}.
%   \item
%     Fixes in module \xmodule{titleref} for support
%     of packages \xpackage{titlesec} and \xpackage{longtable}.
%   \end{Version}
%   \begin{Version}{2008/09/21 v2.2}
%   \item
%     Module \xmodule{base}: \cs{zref@iflistcontainsprop} is documented, but
%     a broken \cs{zref@listcontainsprop} implemented. Name and
%     implementation fixed (thanks Ohad Kammar).
%   \end{Version}
%   \begin{Version}{2008/10/01 v2.3}
%   \item
%     \cs{zref@localaddprop} added (feature request of Ohad Kammar).
%   \item
%     Module \xmodule{lastpage}: list `LastPage' added. Label `LastPage' will use
%     the properties of this list (default is empty) along with the
%     properties of the main list.
%   \end{Version}
%   \begin{Version}{2009/08/07 v2.4}
%   \item
%     Module \xmodule{runs} added.
%   \end{Version}
%   \begin{Version}{2009/12/06 v2.5}
%   \item
%     Module \xmodule{lastpage}: Uses package \xpackage{atveryend}.
%   \item
%     Module \xmodule{titleref}: Further commands are disabled
%     during string expansion, imported from package \xpackage{nameref}.
%   \end{Version}
%   \begin{Version}{2009/12/07 v2.6}
%   \item
%     Version date added for package \xpackage{atveryend}.
%   \end{Version}
%   \begin{Version}{2009/12/08 v2.7}
%   \item
%     Module \xmodule{titleref}: Use of package \xpackage{gettitlestring}.
%   \end{Version}
%   \begin{Version}{2010/03/26 v2.8}
%   \item
%     \cs{zifrefundefined} added.
%   \item
%     Module \xmodule{lastpage}: Macros \cs{zref@iflastpage}
%     and \cs{ziflastpage} added.
%   \item
%     Module \xmodule{thepage} added.
%   \item
%     Module \xmodule{nextpage} added.
%   \end{Version}
%   \begin{Version}{2010/03/29 v2.9}
%   \item
%     Module \xmodule{marks} added (without documentation).
%   \item
%     \cs{zref@addprop} now adds expanded property to list.
%   \item
%     Useless \cs{ZREF@ErrorNoLine} removed.
%   \end{Version}
%   \begin{Version}{2010/04/08 v2.10}
%   \item
%     Module \xmodule{xr} remembers the external document name
%     in property `externaldocument'.
%   \end{Version}
%   \begin{Version}{2010/04/15 v2.11}
%   \item
%     Module \xmodule{titleref}: Better support of class \xclass{memoir}.
%   \item
%     Module \xmodule{titleref}: Support of theorems.
%   \end{Version}
%   \begin{Version}{2010/04/17 v2.12}
%   \item
%     Module \xmodule{base}: \cs{zref@newprop} ensures global
%     empty default.
%   \item
%     Module \xmodule{xr}: Setup options \xoption{tozreflabel}
%     and \xoption{toltxlabel} added.
%   \end{Version}
%   \begin{Version}{2010/04/19 v2.13}
%   \item
%     \cs{zref@setcurrent} throws an error if the property
%     does not exist (Florent Chervet).
%   \item
%     \cs{zref@getcurrent} the documentation is fixed (Florent Chervet).
%     Also it returns the empty string in case of errors.
%   \item
%     \cs{zref@addprop} and \cs{zref@localaddprop} now take a list of
%     property names (feature request of Florent Chervet).
%   \item
%     Example for \cs{zref@wrapper@unexpanded} corrected (Florent Chervet).
%   \end{Version}
%   \begin{Version}{2010/04/22 v2.14}
%   \item
%     Bug fix for \cs{zref@getcurrent} second argument wasn't eaten in
%     case of unknown property.
%   \item
%     \cs{zref@getcurrent} supports \cs{zref@wrapper@unexpanded}.
%   \item
%     \cs{zref@wrapper@unexpanded} added for \cs{ZREF@xr@tolabel}.
%   \item
%     \cs{zref@extract}, \cs{zref@extractdefault}, \cs{zref@getcurrent}
%     are expandable in exact two steps
%     except inside \cs{zref@wrapper@unexpanded}.
%   \end{Version}
%   \begin{Version}{2010/04/23 v2.15}
%   \item
%     \cs{zexternaldocument} fixed for property `url' when
%     importing \cs{new@label} (bug found by Victor Ivrii).
%   \item
%     Two expansion steps also in \cs{zref@wrapper@unexpanded}.
%   \item
%     Nested calls of \cs{zref@wrapper@unexpanded} possible.
%   \end{Version}
%   \begin{Version}{2010/04/28 v2.16}
%   \item
%     More consequent use of package `ltxcmds' and `hologo'.
%   \item
%     Module \xmodule{pagelayout} added.
%   \item
%     Module \xmodule{pageattr} added.
%   \item
%     Robustness introduced for non-expandable interface macros.
%   \item
%     Internal change of the data format of property lists
%     (suggestion of Florent Chervet).
%   \item
%     Module \xmodule{titleref}: Support of environment \textsf{description}.
%   \end{Version}
%   \begin{Version}{2010/05/01 v2.17}
%   \item
%     \cs{zref@newprop} throws an error if the property already exists.
%   \item
%     Module \xmodule{xr}: Bug fix for the case of several \xext{aux} files
%     (bug found by Victor Ivrii).
%   \item
%     Module \xmodule{xr}: Property `urluse' and option \xoption{urluse} added.
%   \end{Version}
%   \begin{Version}{2010/05/13 v2.18}
%   \item
%     Module \xmodule{env} added.
%   \item
%     Module \xmodule{savepos}: \cs{zref@savepos} added.
%   \end{Version}
%   \begin{Version}{2010/10/22 v2.19}
%   \item
%     \cs{zref@addprop} and \cs{zref@localaddprop} are limited
%     to one property only
%     (incompatibility to versions v2.13 to v2.18).
%   \item
%     \cs{zref@addprops} and \cs{zref@localaddprops} added.
%   \item
%     \cs{zref@delprop} and \cs{zref@localdelprop} added.
%   \item
%     \cs{zref@labelbykv} and \cs{zkvlabel} (module \xmodule{user})
%     with keys \xoption{prop}, \xoption{list}, \xoption{delprop},
%     \xoption{immediate}, \xoption{values} added.
%   \end{Version}
%   \begin{Version}{2011/02/12 v2.20}
%   \item
%     Fix for warning in \xpackage{zref-xr}.
%   \end{Version}
%   \begin{Version}{2011/03/18 v2.21}
%   \item
%     Fix in module \xmodule{pagelayout} for \cs{zlistpagelayout}.
%   \item
%     Fix for \cs{zref@localaddprop} (probably since v2.19).
%   \end{Version}
%   \begin{Version}{2011/10/05 v2.22}
%   \item
%     Documentation fixed for \cs{zref@(local)addprop(s)}.
%   \item
%     Module \xmodule{base}: \cs{zref@def@extract}, \cs{zref@def@extractdefault}
%     added.
%   \item
%     Fix in module \xmodule{pagelayout}: Because of missing \cs{noexpand}
%     commands the values of the pagelayout properties on all pages were
%     the values at package loading.
%   \item
%     Module \xmodule{base}: \cs{zref@showprop} added.
%   \end{Version}
%   \begin{Version}{2011/12/05 v2.23}
%   \item
%     Module \xmodule{savepos}: \cs{zsaveposx} and \cs{zsaveposy} added.
%   \end{Version}
%   \begin{Version}{2012/04/04 v2.24}
%   \item
%     Module \xmodule{titleref}, package \xpackage{titlesec}:
%     some support for class `straight' (\cs{ttl@straight@i}) added.
%   \end{Version}
%   \begin{Version}{2016/05/16 v2.25}
%   \item
%     Documentation updates.
%   \end{Version}
%   \begin{Version}{2016/05/21 v2.26}
%   \item
%     update zref-savepos for new luatex
%   \end{Version}
%   \begin{Version}{2018/11/21 v2.27}
%   \item
%     adapted zref-perpage, see issue https://github.com/ho-tex/zref/issues/2
%   \end{Version}
%   \begin{Version}{2019/11/29 v2.28}
%   \item
%     Documentation updates.
%   \item
%     Use \xpackage{iftex} directly.
%   \end{Version}
%   \begin{Version}{2020-03-03 v2.29}
%   \item
%     adapted in module zref-pagelayout the properties pdfhorigin, pdfvorigin,
%     pdfpagewidth, pdfpageheight for luatex to the right primitives.
%   \item
%     Removed no longer needed code for older lualatex versions.
%   \item added some documentation of the abspos module.
%   \item adapted the abspos module to the new luatex primitives.
%   \item adapted pageattr module to the new luatex primitives.
%   \item added short documentation for pageattr module
%   \item use luatex command names directly in zref-savepos rather than defining pdftex compatibility names.
%   \item allow zref-abspos to use \verb|\pdf[vh]origin| in dvi mode.
%   \end{Version}
%   \begin{Version}{2020-03-04 v2.30}
%   \item
%     add pagevalue property to savepos in the abspos module (issue 1)
%   \end{Version}
%   \begin{Version}{2020-05-28 v2.31}
%   \item Adapted module zref-counter to use \cs{@currentcounter} in the next
%   \LaTeX version (issue 5)
%   \end{Version}
%   \begin{Version}{2020-07-03 v2.32}
%   \item Changed in zref-pagelayout the names of the shipout box dimensions
%    to adapt to the new hook management.
%   \end{Version}
% \end{History}
%
% \PrintIndex
%
% \Finale
\endinput

%        (quote the arguments according to the demands of your shell)
%
% Documentation:
%    (a) If zref.drv is present:
%           latex zref.drv
%    (b) Without zref.drv:
%           latex zref.dtx; ...
%    The class ltxdoc loads the configuration file ltxdoc.cfg
%    if available. Here you can specify further options, e.g.
%    use A4 as paper format:
%       \PassOptionsToClass{a4paper}{article}
%
%    Programm calls to get the documentation (example):
%       pdflatex zref.dtx
%       makeindex -s gind.ist zref.idx
%       pdflatex zref.dtx
%       makeindex -s gind.ist zref.idx
%       pdflatex zref.dtx
%
% Installation:
%    TDS:tex/latex/zref/zref.sty
%    TDS:tex/latex/zref/zref-base.sty
%    TDS:tex/latex/zref/zref-abspage.sty
%    TDS:tex/latex/zref/zref-abspos.sty
%    TDS:tex/latex/zref/zref-counter.sty
%    TDS:tex/latex/zref/zref-dotfill.sty
%    TDS:tex/latex/zref/zref-env.sty
%    TDS:tex/latex/zref/zref-hyperref.sty
%    TDS:tex/latex/zref/zref-lastpage.sty
%    TDS:tex/latex/zref/zref-marks.sty
%    TDS:tex/latex/zref/zref-nextpage.sty
%    TDS:tex/latex/zref/zref-pageattr.sty
%    TDS:tex/latex/zref/zref-pagelayout.sty
%    TDS:tex/latex/zref/zref-perpage.sty
%    TDS:tex/latex/zref/zref-runs.sty
%    TDS:tex/latex/zref/zref-savepos.sty
%    TDS:tex/latex/zref/zref-thepage.sty
%    TDS:tex/latex/zref/zref-titleref.sty
%    TDS:tex/latex/zref/zref-totpages.sty
%    TDS:tex/latex/zref/zref-user.sty
%    TDS:tex/latex/zref/zref-xr.sty
%    TDS:doc/latex/zref/zref.pdf
%    TDS:doc/latex/zref/zref-example.tex
%    TDS:doc/latex/zref/zref-example-lastpage.tex
%    TDS:doc/latex/zref/zref-example-nextpage.tex
%    TDS:source/latex/zref/zref.dtx
%
%<*ignore>
\begingroup
  \catcode123=1 %
  \catcode125=2 %
  \def\x{LaTeX2e}%
\expandafter\endgroup
\ifcase 0\ifx\install y1\fi\expandafter
         \ifx\csname processbatchFile\endcsname\relax\else1\fi
         \ifx\fmtname\x\else 1\fi\relax
\else\csname fi\endcsname
%</ignore>
%<*install>
\input docstrip.tex
\Msg{************************************************************************}
\Msg{* Installation}
\Msg{* Package: zref 2020-07-03 v2.32 A new reference scheme for LaTeX (HO)}
\Msg{************************************************************************}

\keepsilent
\askforoverwritefalse

\let\MetaPrefix\relax
\preamble

This is a generated file.

Project: zref
Version: 2020-07-03 v2.32

Copyright (C)
  2006-2012 Heiko Oberdiek
  2016-2020 Oberdiek Package Support Group

This work may be distributed and/or modified under the
conditions of the LaTeX Project Public License, either
version 1.3c of this license or (at your option) any later
version. This version of this license is in
   https://www.latex-project.org/lppl/lppl-1-3c.txt
and the latest version of this license is in
   https://www.latex-project.org/lppl.txt
and version 1.3 or later is part of all distributions of
LaTeX version 2005/12/01 or later.

This work has the LPPL maintenance status "maintained".

The Current Maintainers of this work are
Heiko Oberdiek and the Oberdiek Package Support Group
https://github.com/ho-tex/zref/issues


This work consists of the main source file zref.dtx
and the derived files
   zref.sty, zref.pdf, zref.ins, zref.drv, zref-base.sty,
   zref-abspage.sty, zref-abspos.sty, zref-counter.sty,
   zref-dotfill.sty, zref-env.sty, zref-hyperref.sty,
   zref-lastpage.sty, zref-marks.sty, zref-nextpage.sty,
   zref-pageattr.sty, zref-pagelayout.sty, zref-perpage.sty,
   zref-runs.sty, zref-savepos.sty, zref-thepage.sty,
   zref-titleref.sty, zref-totpages.sty, zref-user.sty,
   zref-xr.sty, zref-example.tex, zref-example-lastpage.tex,
   zref-example-nextpage.tex, zref-test1.tex,
   zref-test-base.tex, zref-test-runs.tex,
   zref-test-titleref-memoir.tex.

\endpreamble
\let\MetaPrefix\DoubleperCent

\generate{%
  \file{zref.ins}{\from{zref.dtx}{install}}%
  \file{zref.drv}{\from{zref.dtx}{driver}}%
  \usedir{tex/latex/zref}%
  \file{zref.sty}{\from{zref.dtx}{package}}%
  \file{zref-base.sty}{\from{zref.dtx}{base}}%
  \file{zref-abspage.sty}{\from{zref.dtx}{abspage}}%
  \file{zref-abspos.sty}{\from{zref.dtx}{abspos}}%
  \file{zref-counter.sty}{\from{zref.dtx}{counter}}%
  \file{zref-dotfill.sty}{\from{zref.dtx}{dotfill}}%
  \file{zref-env.sty}{\from{zref.dtx}{env}}%
  \file{zref-hyperref.sty}{\from{zref.dtx}{hyperref}}%
  \file{zref-lastpage.sty}{\from{zref.dtx}{lastpage}}%
  \file{zref-marks.sty}{\from{zref.dtx}{marks}}%
  \file{zref-nextpage.sty}{\from{zref.dtx}{nextpage}}%
  \file{zref-pageattr.sty}{\from{zref.dtx}{pageattr}}%
  \file{zref-pagelayout.sty}{\from{zref.dtx}{pagelayout}}%
  \file{zref-perpage.sty}{\from{zref.dtx}{perpage}}%
  \file{zref-runs.sty}{\from{zref.dtx}{runs}}%
  \file{zref-savepos.sty}{\from{zref.dtx}{savepos}}%
  \file{zref-thepage.sty}{\from{zref.dtx}{thepage}}%
  \file{zref-titleref.sty}{\from{zref.dtx}{titleref}}%
  \file{zref-totpages.sty}{\from{zref.dtx}{totpages}}%
  \file{zref-user.sty}{\from{zref.dtx}{user}}%
  \file{zref-xr.sty}{\from{zref.dtx}{xr}}%
  \usedir{doc/latex/zref}%
  \file{zref-example.tex}{\from{zref.dtx}{example}}%
  \file{zref-example-lastpage.tex}{\from{zref.dtx}{example-lastpage}}%
  \file{zref-example-nextpage.tex}{\from{zref.dtx}{example-nextpage}}%
%  \usedir{doc/latex/zref/test}%
%  \file{zref-test1.tex}{\from{zref.dtx}{test1}}%
%  \file{zref-test-base.tex}{\from{zref.dtx}{test-base}}%
%  \file{zref-test-runs.tex}{\from{zref.dtx}{test-runs}}%
%  \file{zref-test-titleref-memoir.tex}{\from{zref.dtx}{test-titleref-memoir}}%
}

\catcode32=13\relax% active space
\let =\space%
\Msg{************************************************************************}
\Msg{*}
\Msg{* To finish the installation you have to move the following}
\Msg{* files into a directory searched by TeX:}
\Msg{*}
\Msg{*     zref.sty, zref-base.sty, zref-abspage.sty, zref-abspos.sty,}
\Msg{*     zref-counter.sty, zref-dotfill.sty, zref-env.sty,}
\Msg{*     zref-hyperref.sty, zref-lastpage.sty, zref-marks.sty,}
\Msg{*     zref-nextpage.sty, zref-pageattr.sty, zref-pagelayout.sty,}
\Msg{*     zref-perpage.sty, zref-runs.sty, zref-savepos.sty,}
\Msg{*     zref-thepage.sty, zref-titleref.sty, zref-totpages.sty,}
\Msg{*     zref-user.sty, zref-xr.sty}
\Msg{*}
\Msg{* To produce the documentation run the file `zref.drv'}
\Msg{* through LaTeX.}
\Msg{*}
\Msg{* Happy TeXing!}
\Msg{*}
\Msg{************************************************************************}

\endbatchfile
%</install>
%<*ignore>
\fi
%</ignore>
%<*driver>
\NeedsTeXFormat{LaTeX2e}
\ProvidesFile{zref.drv}%
  [2020-07-03 v2.32 A new reference scheme for LaTeX (HO)]%
\documentclass{ltxdoc}
\usepackage{holtxdoc}[2011/11/22]
\setlength{\emergencystretch}{1em}
\begin{document}
  \DocInput{zref.dtx}%
\end{document}
%</driver>
% \fi
%
%
%
% \GetFileInfo{zref.drv}
%
% \title{The \xpackage{zref} package}
% \date{2020-07-03 v2.32}
% \author{Heiko Oberdiek\thanks
% {Please report any issues at \url{https://github.com/ho-tex/zref/issues}}}
%
% \maketitle
%
% \begin{abstract}
% Package \xpackage{zref} tries to get rid of the restriction
% in \hologo{LaTeX}'s reference system that only two properties are
% supported. The package implements an extensible referencing
% system, where properties are handled in a more flexible way.
% It offers an interface for macro programmers for the access
% to the system and some applications that uses the new
% reference scheme.
% \end{abstract}
%
% \tableofcontents
%
% \newcommand*{\metacs}[1]{\texttt{\itshape #1}}
%
% \newcommand{\markexp}{^^A
%   \textsuperscript{exp}^^A
%   \space
% }
% \newcommand{\markexptwo}{^^A
%   \textsuperscript{exp2}^^A
%   \space
% }
% \newcommand{\markrobust}{^^A
%   \textsuperscript{robust}^^A
%   \space
% }
% \newcommand{\marknowarn}{^^A
%   \textsuperscript{nowarn}^^A
%   \space
% }
% \newcommand{\markbabel}{^^A
%   \textsuperscript{babel}^^A
%   \space
% }
%
% \section{Introduction}
%
% Standard \hologo{LaTeX}'s reference system with \cs{label}, \cs{ref},
% and \cs{pageref} supports two properties, the apperance of
% the counter that is last incremented by \cs{refstepcounter} and
% the page with the \cs{label} command.
%
% Unhappily \hologo{LaTeX} does not provide an interface for adding
% another properties. Packages such as \xpackage{hyperref},
% \xpackage{nameref}, or \xpackage{titleref} are forced to
% use ugly hacks to extend the reference system.
% These ugly hacks are one of the causes for \xpackage{hyperref}'s
% difficulty regarding compatibility with other packages.
%
% \subsection{Standard \hologo{LaTeX} behaviour}
%
% References are created by the \cs{label} command:
% \begin{quote}
%   |\chapter{Second chapter}|\\
%   |\section{First section on page 7} % section 2.1|\\
%   |\label{myref}|
% \end{quote}
% Now \hologo{LaTeX} records the section number |2.1| and the page |7|
% in the reference. Internally the reference is a list with
% two entries:
% \begin{quote}
%   |\r@myref| $\rightarrow$ |{2.1}{7}|
% \end{quote}
% The length of the list if fixed in the \hologo{LaTeX} kernel,
% An interface for adding new properties is missing.
%
% There are several tries to add new properties:
% \begin{description}
% \item[\xpackage{hyperref}] uses a list of five
%   properties instead of the standard list with two entries.
%   This causes many compatibility problems with \hologo{LaTeX} and
%   other packages.
% \item[\xpackage{titleref}] stores its title data into the
%   first entry in the list. \hologo{LaTeX} is happy because it
%   does only see its list with two entries. The situation
%   becomes more difficult, if more properties are added
%   this way. Then the macros form a nested structure
%   inside the first reference argument for the label.
%   Expandable extractions will then become painful.
% \end{description}
%
% \subsection{Basic idea}
%
% Some time ago Morten H\o gholm sent me an experimental
% cross referencing mechanism as ``expl3'' code. His
% idea is:
% \begin{quote}
%   |\g_xref_mylabel_plist| $\rightarrow$\\
%   \mbox{}\qquad|\xref_dance_key{salsa}\xref_name_key{Morten}|\dots
% \end{quote}
% The entries have the following format:
% \begin{quote}
%   |\xref_|\meta{your key}|_key{|\meta{some text}|}|
% \end{quote}
% This approach is much more flexible:
% \begin{itemize}
%   \item New properties can easily be added, just use a new key.
%   \item The length of the list is not fixed. A reference
%         can use a subset of the keys.
%   \item The order of the entries does not matter.
% \end{itemize}
% Unhappily I am not familiar with the experimental code
% for \hologo{LaTeX3} that will need some time before its first release.
% Thus I have implemented it as \hologo{LaTeXe} package without disturbing
% the existing \hologo{LaTeX} reference system.
%
% \subsection{Interfaces}
%
% The package provides a generic \emph{interface for programmers}.
% Commands of this interface are prefixed by \cs{zref@}.
%
% Option \xoption{user} enabels the \emph{user interface}.
% Here the commands are prefixed by \cs{z} to avoid name
% clashes with existing macros.
%
% Then the packages provides some \emph{modules}. They
% are applications for the reference system and can also be
% considered as examples how to use the reference system.
%
% The modules can be loaded as packages. The package
% name is prefixed with |zref-|, for example:
%\begin{quote}
%\begin{verbatim}
%\RequirePackage{zref-abspage}
%\end{verbatim}
%\end{quote}
% This is the preferred way if the package is loaded from
% within other packages to avoid option clashes.
%
% As alternative package |zref| can be used and the modules
% are given as options:
%\begin{quote}
%\begin{verbatim}
%\usepackage[perpage,user]{zref}
%\end{verbatim}
%\end{quote}
%
% \section{Interface for programmers}
%
% The user interface is described in the next section \ref{sec:user}.
%
% \subsection{Entities}
%
% \paragraph{Reference.}
% Internally a reference is a list of key value pairs:
% \begin{quote}
%   |\Z@R@myref| $\rightarrow$
%   |\default{2.1}\page{7}|
% \end{quote}
% The generic format of a entry is:
% \begin{quote}
%   |\Z@R@|\meta{refname} $\rightarrow$
%   |\|\meta{propname}|{|\meta{value}|}|
% \end{quote}
% \meta{refname} is the name that denoted references
% (the name used in \cs{label} and \cs{ref}). \meta{propname}
% is the name of the property or key. The property key macro
% is never executed, it is used in parameter text matching only.
%
% \paragraph{Property.}
% Because the name of a property is used in a macro name
% that must survive the \xfile{.aux} file, the name is
% restricted to letters and `@'.
%
% \paragraph{Property list.}
% Often references are used for special purposes. Thus it
% saves memory if just the properties are used in this
% reference that are necessary for its purpose.
%
% Therefore this package uses the concept of \emph{property lists}.
% A property list is a set of properties. The set of properties
% that is used by the default \cs{label} command is the
% \emph{main property list}.
%
% \subsection{Property list}
%
% \markexp means that the implementation of the
% marked macro is expandable. \markexptwo goes a step further
% and marks the macro expandable in exact two expansion steps.
%
% \begin{declcs}{zref@newlist}
%   \M{listname}
% \end{declcs}
% Declares a new empty property list.
%
% \begin{declcs}{zref@addprop}
%   \M{listname} \M{propname}\\
%   \cs{zref@localaddprop} \M{listname} \M{propname}
% \end{declcs}
% Adds the property \meta{propname}
% to the property list \meta{listname}.
% The property and list must exist.
% The addition is global by \cs{zref@addprop} and limited to
% local scope by \cs{zref@localaddprop}.
% Between 2010/04/19 v2.13 and 2010/10/22 v2.19 a comma separated list
% of properties could be used as argument \meta{propname}.
% Since 2010/10/22 v2.19 the addition of several properties
% at once is supported by \cs{zref@addprops}.
%
% \begin{declcs}{zref@addprops}
%   \M{listname} \M{propname list}\\
%   \cs{zref@localaddprops} \M{listname} \M{propname list}
% \end{declcs}
% These macros add a comma separated list of properties
% \meta{propname list} to list \meta{listname}.
% \cs{zref@addprops} works globally and \cs{zref@localaddprops}
% locally. Since 2010/10/22 v2.19.
%
% \begin{declcs}{zref@listexists}
%   \M{listname} \M{then}
% \end{declcs}
% Executes \meta{then} if the property list \meta{listname} exists
% or raise an error otherwise.
%
% \begin{declcs}{zref@iflistundefined}\markexp \M{listname}
%   \M{then} \M{else}
% \end{declcs}
% Executes \meta{then} if the list exists or \meta{else}
% otherwise.
%
% \begin{declcs}{zref@iflistcontainsprop}
%   \M{listname} \M{propname} \M{then} \M{else}
% \end{declcs}
% Executes \meta{then} if the property \meta{propname} is part
% of property list \meta{listname} or otherwise it
% runs the \meta{else} part.
%
% \subsection{Property}
%
% \begin{declcs}{zref@newprop}
%   \* \M{propname} |[|\meta{default}|]| \M{value}
% \end{declcs}
% This command declares and configures a new property with
% name \meta{propname}.
%
% In case of unknown references
% or the property does not exist in the reference, the
% \meta{default} is used as value. If it is not specified
% here, a global default is used, see \cs{zref@setdefault}.
%
% The correct values of some properties are not known
% immediately but at page shipout time. Prominent example
% is the page number. These properties are declared with
% the star form of the command.
%
% \begin{declcs}{zref@setcurrent}
%   \M{propname} \M{value}
% \end{declcs}
% This sets the current value of the property \meta{propname}.
% It is a generalization of setting \hologo{LaTeX}'s \cs{currentlabel}.
%
% \begin{declcs}{zref@getcurrent}\markexptwo
%   \M{propname}
% \end{declcs}
% This returns the current value of the property \meta{propname}.
% The value may not be correct, especially if the property
% is bound to a page (start form of \cs{zref@newprop}) and the
% right value is only known at shipout time (e.g. property `page').
% In case of errors (e.g. unknown property) the empty string is
% returned.
%
% Since version 2010/04/22 v2.14 \cs{zref@getcurrent} supports
% \cs{zref@wrapper@unexpanded}.
%
% \begin{declcs}{zref@propexists}
%   \M{propname} \M{then}
% \end{declcs}
% Calls \meta{then} if the property \meta{propname} is available
% or generates an error message otherwise.
%
% \begin{declcs}{zref@ifpropundefined}\markexp
%   \M{propname} \M{then} \M{else}
% \end{declcs}
% Calls \meta{then} or \meta{else} depending on the existence
% of property \meta{propname}.
%
% \subsection{Reference generation}
%
% \begin{declcs}{zref@label}
%   \M{refname}
% \end{declcs}
% This works similar to \cs{label}. The reference \meta{refname} is
% created and put into the \xfile{.aux} file with the
% properties of the main property list.
%
% \begin{declcs}{zref@labelbylist}
%   \M{refname} \M{listname}
% \end{declcs}
% Same as \cs{zref@label} except that the properties are
% taken from the specified property list \meta{listname}.
%
% \begin{declcs}{zref@labelbyprops}
%   \M{refname} |{|\meta{propnameA}|,|\meta{propnameB}|,|\dots|}|
% \end{declcs}
% Same as \cs{zref@label} except that these properties
% are used that are given as comma separated list in the
% second argument.
%
% \begin{declcs}{zref@newlabel}
%   \M{refname} |{|\dots|}|
% \end{declcs}
% This is the macro that is used in the \xfile{.aux} file.
% It is basically the same as \cs{newlabel} apart from
% the format of the data in the second argument.
%
% \subsection{Data extraction}
%
% \begin{declcs}{zref@extractdefault}\markexptwo
%   \M{refname} \M{propname} \M{default}
% \end{declcs}
% This is the basic command that refernces the value of a
% property \meta{propname} for the reference \meta{refname}.
% In case of errors such as undefined reference the \meta{default}
% is used instead.
%
% \begin{declcs}{zref@extract}\markexptwo
%   \M{refname} \M{propname}
% \end{declcs}
% The command is an abbreviation for \cs{zref@extractdefault}.
% As default the default of the property is taken, otherwise
% the global default.
%
% Example for page references:
% \begin{quote}
% \begin{tabular}{@{}ll}
% \hologo{LaTeX}:& |\pageref{foobar}|\\
% \xpackage{zref}:& |\zref@extract{foobar}{page}|
% \end{tabular}
% \end{quote}
% Both \cs{zref@extract} and \cs{zref@extractdefault} are
% expandable. That means, these macros can directly
% be used in expandable calculations, see the example file.
% On the other side, babel's shorthands are not supported,
% there are no warnings in case of undefined references.
%
% If an user interface doesn't need expandable macros
% then it can use \cs{zref@refused} and \cs{zref@wrapper@babel}
% for its user macros.
%
% \begin{declcs}{zref@refused}
%   \M{refname}
% \end{declcs}
% This command is not expandable. It causes the warnings if
% the reference \meta{refname} is not defined. Use the
% \cs{zref@extract} commands inside expandable contexts and
% mark their use outside by \cs{zref@refused}, see the example
% file.
%
% \begin{declcs}{zref@def@extract}%
%     \M{cmd} \M{refname} \M{propname}\\
%   \cs{zref@def@extractdefault}
%     \M{cmd} \M{refname} \M{propname} \M{default}
% \end{declcs}
% Both macros extract the property \meta{propname} from the reference
% \meta{refname} the same way as macros \cs{zref@extract} and
% \cs{zref@extractdefault}. The result is stored in macro \meta{cmd}.
% Also \cs{zref@refused} is called to notify \hologo{LaTeX} that
% the reference \meta{refname} is used. Added in 2011/10/04 v2.22.
%
% \begin{declcs}{zref@ifrefundefined}\markexp
%   \M{refname} \M{then} \M{else}
% \end{declcs}
% Macro \cs{zref@ifrefundefined} calls arguments \meta{then} or \meta{else}
% dependent on the existence of the reference \meta{refname}.
%
% \begin{declcs}{zifrefundefined}
%   \M{refname} \M{then} \M{else}
% \end{declcs}
% Macro \cs{zifrefundefined} calls \cs{ref@refused} before
% executing \cs{zref@ifrefundefined}. Babel shorthands are
% supported in \meta{refname}.
%
% \begin{declcs}{zref@ifrefcontainsprop}\markexp
%   \M{refname} \M{propname} \M{then} \M{else}
% \end{declcs}
% Test whether a reference provides a property.
%
% \subsection{Setup}
%
% \begin{declcs}{zref@default}
% \end{declcs}
% Holds the global default for unknown values.
%
% \begin{declcs}{zref@setdefault}
%   |{|\meta{value}|}|
% \end{declcs}
% Sets the global default for unknown values.
% The global default is used, if a property does not specify
% an own default and the value for a property
% cannot be extracted. This can happen if the reference is
% unknown or the reference does not have the property.
%
% \begin{declcs}{zref@setmainlist}
%   |{|\meta{value}|}|
% \end{declcs}
% Sets the name of the main property list. The package
% sets and uses |main|.
%
% \subsection{Declared properties}
%
% \begin{quote}
% \renewcommand*{\arraystretch}{1.05}
% \setlength{\tabcolsep}{2\tabcolsep}
% \def\VV{\vphantom{\large \"Ag}}
% \def\EMPTY{{\itshape\textless empty\textgreater}}
% \def\pl#1{&#1&thepage&\cs{number}\cs{#1}}
% \begin{tabular}{@{}>{\sffamily}l|>{\ttfamily}l|>{\ttfamily}l|>{\ttfamily}l@{}}
%   \rmfamily Module&
%   \rmfamily Property&
%   \rmfamily Property list&
%   \rmfamily Default\\
%   \hline
%   (base)
%     & default & main & \EMPTY\\
%     & page & main & \EMPTY\\
%   \hline
%   abspage
%     & abspage & main & 0\\
%   \hline
%   counter
%     & counter & main & \EMPTY\\
%   \hline
%   hyperref
%     & anchor & main & \EMPTY\\
%     & url    & & \EMPTY\\
%   \hline
%   pageattr
%     & pdfpageattr & thepage & \dots\\
%     & pdfpagesattr & LastPage & \dots\\
%   \hline
%   pagelayout\footnotemark
%     \pl{mag}\\
%     \pl{paperwidth}\\
%     \pl{paperheight}\\
%     \pl{stockwidth}\\
%     \pl{stockheight}\\
%     \pl{pdfpageheight}\\
%     \pl{pdfpagewidth}\\
%     \pl{pdfhorigin}\\
%     \pl{pdfvorigin}\\
%     \pl{hoffset}\\
%     \pl{voffset}\\
%     \pl{topmargin}\\
%     \pl{oddsidemargin}\\
%     \pl{evensidemargin}\\
%     \pl{textwidth}\\
%     \pl{textheight}\\
%     \pl{headheight}\\
%     \pl{headsep}\\
%     \pl{footskip}\\
%     \pl{marginparwidth}\\
%     \pl{marginparsep}\\
%     \pl{columnwidth}\\
%     \pl{columnsep}\\
%   \hline
%   perpage
%     & pagevalue & perpage & 0\\
%     & page & perpage & \EMPTY\\
%     & abspage & perpage & 0\\
%   \hline
%   savepos
%     & posx & savepos & 0\\
%     & posy & savepos & 0\\
%   \hline
%   titleref
%     & title & main & \EMPTY\\
%   \hline
%   xr
%     & anchor   & & \EMPTY\\
%     & externaldocument & & \EMPTY\\
%     & theotype & & \EMPTY\\
%     & title    & & \EMPTY\\
%     & url      & & \EMPTY\\
% \end{tabular}
% \footnotetext{Module \xmodule{pagelayout} only defines properties
% if the parameter exists.}%
% \end{quote}
%
% \subsection{Wrapper for advanced situations}
%
% \begin{declcs}{zref@wrapper@babel}
%   |{|\dots|}| \M{name}
% \end{declcs}
% This macro helps to add shorthand support. The second
% argument is protected, then the code of the first argument
% is called with the protected name appended. Examples
% are in the sources.
%
% \begin{declcs}{zref@wrapper@immediate}
%   |{|\dots|}|
% \end{declcs}
% There are situations where a label must be written instantly
% to the \xfile{.aux} file, for example after the last page.
% If the \cs{zlabel} or \cs{label} command is put inside this wrapper,
% immediate writing is enabled. See the implementation for
% module \xmodule{lastpage} for an example of its use.
%
% \begin{declcs}{zref@wrapper@unexpanded}
%   |{|\dots|}|
% \end{declcs}
% Assuming someone wants to extract a value for property |bar|
% and store the result in a macro |\foo| without traces of
% the expanding macros and without expanding the value.
% This (theoretical?) problem can be solved by this wrapper:
% \begin{quote}
%\begin{verbatim}
%\zref@wrapper@unexpanded{%
%  \edef\foo{%
%    \zref@extract{someref}{bar}%
%  }%
%}
%\end{verbatim}
% \end{quote}
% The \cs{edef} forces the expansion of \cs{zref@extract},
% but the extraction of the value is prevented by
% the wrapper that uses \hologo{eTeX}'\ \cs{unexpanded} for this
% purpose. Supported macros are \cs{zref@extract}, \cs{zref@extractdefault}
% and since version 2010/04/22 v2.14 macro \cs{zref@getcurrent}.
%
% \subsection{Counter for unique names}
%
% Some modules (\xmodule{titleref} and \xmodule{dotfillmin})
% need unique names for automatically generated label names.
%
% \begin{declcs}{zref@require@unique}
% \end{declcs}
% This command creates the unique counter |zref@unique| if the
% counter does not already exist.
%
% \begin{declcs}{thezref@unique}
% \end{declcs}
% This command is used to generate unique label names.
%
% \section{User interface}\label{sec:user}
%
% \subsection{Module \xmodule{user}}
%
% The user interface for this package and its modules
% is enabled by \xpackage{zref}'s package option \xmodule{user}
% or package \xpackage{zref-user}.
% The names of user commands are prefixed by |z| in order
% to avoid name clashes with existing macros of the same
% functionality. Thus the package does not disturb the
% traditional reference scheme, both can be used together.
%
% The syntax descriptions contain the following markers
% that are intended as hints for programmers:
% \begin{quote}
% \begin{tabular}{@{}ll@{}}
%   \markbabel\unskip  & Babel shorthands are allowed.\\
%   \markrobust\unskip & Robust macro.\\
%   \markexp\unskip    & Expandable version:\\
%                      & \textbullet\ robust,
%                        unless the extracted values are fragile,\\
%                      & \textbullet\ no babel shorthand suport.\\
%   \markexptwo\unskip & Expandable like \markexp and:\\
%                      & \textbullet\ expandable in exact two steps.\\
% \end{tabular}
% \end{quote}
%
% The basic user interface of the package without modules
% are commands that mimic the standard \hologo{LaTeX} behaviour
% of \cs{label}, \cs{ref}, and \cs{pageref}:
%
% \begin{declcs}{zlabel}
%   \M{refname}\markbabel
% \end{declcs}
% Similar to \cs{label}. It generates a label with name
% \meta{refname} in the new reference scheme.
%
% \begin{declcs}{zref}
%   |[|\meta{propname}|]| \M{refname}\markbabel
% \end{declcs}
% Without optional argument similar to \cs{ref}, it returns
% the default reference property. This property is named
% |default|:
% \[
%   |\zref{|x|}| \equiv |\zref[default]{|x|}|
% \]
%
% \begin{declcs}{zpageref}
%   \M{refname}\markbabel
% \end{declcs}
% Convenience macro, similar to \cs{pageref}.
% \[
%   |\zpageref{|x|}| \equiv |\zref[page]{|x|}|
% \]
%
% \begin{declcs}{zrefused}
%   \M{refname}\markbabel
% \end{declcs}
% Some of the user commands in the modules are expandable.
% The use of such commands do not cause any undefined
% reference warnings, because inside of expandable
% contexts this is not possible. However, if there is
% a place outside of expandable contexts, \cs{refused}
% is strongly recommended. The reference \meta{refname}
% is marked as used, undefined ones will generate
% warnings.
%
% \subsection{Module \xmodule{abspage}}
%
% With the help of package \xpackage{atbegshi}
% a new counter |abspage| with absolute page numbers is provided.
% For technical and historical reasons the counter itself is zero based:
% if you use it directly on the first page, e.g
% with |\arabic{abspage}| you will get 0 as value. When using |\zref|
% the first page will be page 1 as expected.
% Also a new property |abspage| is defined and added to the
% main property list. Thus you can reference the absolute
% page number:
% \begin{quote}
%   |Section \zref{foo} is on page \zpageref{foo}.|\\
%   |This is page \zref[abspage]{foo}|\\
%   |of \zref[abspage]{LastPage}.|
% \end{quote}
% The example also makes use of module \xmodule{lastpage}.
%
% \subsection{Module \xmodule{lastpage}}
%
% Provides the functionality of package \xpackage{lastpage}
% \cite{lastpage} in the new reference scheme.
% The label |LastPage| is put at the end of the document.
% You can refer the last page number with:
% \begin{quote}
%   |\zref@extract{LastPage}{page}| (+ |\zref@refused{LastPage}|)\\
% \end{quote}
% or
% \begin{quote}
%   |\zpageref{LastPage}| (module \xmodule{user})
% \end{quote}
%
% Since version 2008/10/01 v2.3 the module defines the list |LastPage|.
% In addition to the properties of the main list label |LastPage| also
% stores the properties of this list |LastPage|. The default of this
% list is empty. The list can be used by the user to add additional
% properties for label |LastPage|.
%
% \subsubsection{Tests for last page}
%
% Since version 2010/03/26 v2.8 the macros \cs{zref@iflastpage}
% and \cs{ziflastpage} were added. They test the reference,
% whether it is a reference of the last page.
%
% \begin{declcs}{zref@iflastpage}\markexp \M{refname} \M{then} \M{else}
% \end{declcs}
% Macro \cs{zref@iflastpage} compares the references \meta{refname}
% with \meta{LastPage}. Basis of the comparison is the value of
% property |abspage|, because the values are different for different
% pages. This is not ensured by property |page|. Therefore module
% \xmodule{abspage} is loaded by module \xmodule{lastpage}.
% If both values of property |abspage| are present and match,
% then \meta{then} is executed, otherwise code \meta{else} is called.
% If one or both references are undefined or lack the property |abspage|,
% then \meta{else} is executed.
%
% Macro \cs{zref@iflastpage} is expandable, therefore \cs{zref@refused}
% should be called on \meta{refname} and \meta{LastPage}.
%
% \begin{declcs}{ziflastpage} \M{refname} \M{then} \M{else}
% \end{declcs}
% Macro \cs{ziflastpage} has the same function as \cs{zref@iflastpage},
% but adds support for babel shorthands in \meta{refname} and
% calls \cs{zref@refused}. However macro \cs{ziflastpage} is not
% expandable.
%
% \subsubsection{Example}
%
%    \begin{macrocode}
%<*example-lastpage>
%<<END_EXAMPLE
\NeedsTeXFormat{LaTeX2e}
\documentclass{report}

\newcounter{foo}
\renewcommand*{\thefoo}{\Alph{foo}}

\usepackage{zref-lastpage,zref-user}[2019/11/29]

\makeatletter
\zref@newprop{thefoo}{\thefoo}
\zref@newprop{valuefoo}{\the\value{foo}}
\zref@newprop{chapter}{\thechapter}
\zref@addprops{LastPage}{thefoo,valuefoo,chapter}
\makeatother

\newcommand*{\foo}{%
  \stepcounter{foo}%
  [Current foo: \thefoo]%
}

\begin{document}
  \chapter{First chapter}
  Last page is \zref{LastPage}.\\
  Last chapter is \zref[chapter]{LastPage}.\\
  Last foo is \zref[thefoo]{LastPage}.\\
  Last value of foo is \zref[valuefoo]{LastPage}.\\
  \foo
  \chapter{Second chapter}
  \foo\foo\foo
  \chapter{Last chapter}
  \foo
\end{document}
%END_EXAMPLE
%</example-lastpage>
%    \end{macrocode}
%
% \subsection{Module \xmodule{thepage}}
%
% This module \xmodule{thepage} loads module \xmodule{abspage},
% constructs a reference name using the absolute page number
% and remembers property |page|. Other properties can be added
% by adding them to the property list |thepage|.
%
% \begin{declcs}{zthepage} \M{absolute page number}\\
% \end{declcs}
% Macro \cs{zthepage} is basically a \cs{zpageref}. The reference
% name is yield by the \meta{absolute page number}. If the reference
% is not defined, then the default for property |page| is used.
%
% \begin{declcs}{zref@thepage@name}\markexp \M{absolute page number}
% \end{declcs}
% Macro \cs{zref@thepage@name} returns the internal reference
% name that is constructed using the \meta{absolute page number}.
% The internal reference name should not be used directly, because
% it might change in future versions.
%
% \begin{declcs}{zref@thepage}\markexp \M{absolute page number}\\
%   \cs{zref@thepage@refused} \M{absolute page number}
% \end{declcs}
% Macro \cs{zref@thepage} returns the page number (\cs{thepage})
% of \meta{absolute page number}. Because this macro is expandable,
% \cs{zref@thepage@refused} is used outside an expandable context
% to mark the reference as used.
%
%
% \subsection{Module \xmodule{nextpage}}
%
% \begin{declcs}{znextpage}
% \end{declcs}
% Macro \cs{znextpage} prints \cs{thepage} of the following page.
% It gets the current absolute page number by using a label.
% There are three cases for the next page:
% \begin{enumerate}
% \item The next page is not known yet because of undefined references.
%   Then \cs{zunknownnextpagename} is used instead. The default for
%   this macro is the default of property |page|.
% \item This page is the last page. Then \cs{znonextpagename}
%   is used. Its default is empty.
% \item The next page is known, then \cs{thepage} of the next
%   page is used (the value of property |page| of the next page).
% \end{enumerate}
%
% \subsubsection{Configuration}
%
% The behaviour can be configured by the following macros.
%
% \begin{declcs}{zunknownnextpagename}\\
%   \cs{znonextpagename}
% \end{declcs}
% If the next page is not known or available, then
% \cs{znextpage} uses these name macros as default.
% \cs{zunknownnextpagename} is used in case of undefined
% references. Default is the value of property |page| of the
% next page (\cs{thepage}). Module \xmodule{thepage} is used.
%
% Macro \cs{znonextpagename} is used, if the next page does
% not exists. That means that the current page is last page.
% The default is empty.
%
% \begin{declcs}{znextpagesetup} \M{unknown} \M{no next}
%   \M{next}
% \end{declcs}
% Acording to the case (see \cs{znextpage}) macro \cs{znextpage}
% calls an internal macro with an argument. The argument is
% either \cs{thepage} of the next page or one of \cs{zunknownnextpagename}
% or \cs{znonextpagename}. These internal macro can be changed by
% \cs{znextpagesetup}. It expects the definition texts for these
% three cases of a macro with one argument.
% The default is
% \begin{quote}
%   |\znextpagesetup{#1}{#1}{#1}|
% \end{quote}
%
% \subsubsection{Example}
%
%    \begin{macrocode}
%<*example-nextpage>
%<<END_EXAMPLE
\documentclass{book}

\usepackage{zref-nextpage}[2019/11/29]
\znextpagesetup
  {\thepage}% next page is unknown
  {\thepage\ (#1)}% this page is last page
  {\thepage\ $\rightarrow$ #1}% next page is known
\renewcommand*{\znonextpagename}{last page}

\usepackage{fancyhdr}
\pagestyle{fancy}
\fancyhf{}
\fancyhead[LE,RO]{\znextpage}
\fancypagestyle{plain}{%
  \fancyhf{}%
  \fancyhead[LE,RO]{\znextpage}%
}

\begin{document}
\frontmatter
  \tableofcontents
\mainmatter
  \chapter{Hello World}
  \clearpage
  \section{Last section}
\end{document}
%END_EXAMPLE
%</example-nextpage>
%    \end{macrocode}
%
%
% \subsection{Module \xmodule{totpages}}
%
% For the total number of pages of a document you need to
% know the absolute page number of the last page.
% Both modules \xmodule{abspage} and \xmodule{lastpage}
% are necessary and automatically enabled.
%
% \begin{declcs}{ztotpages}\markexp
% \end{declcs}
% Prints the total number of pages or |0| if this
% number is not yet known. It expands to an explicit number
% and can also used even in expandable calculations (\cs{numexpr})
% or counter assignments.
%
% \subsection{Module \xmodule{pagelayout}}
%
% The module defines additional properties for each parameter
% of the page layout that is effective during page shipout.
% The value of length parameters is given in sp without the
% unit as plain number.
%
% Some parameters are specific for a class (e.g.\@ \xoption{stockwidth}
% and \xoption{stockheight} for class \xclass{memoir}) or the
% \hologo{TeX} engine like \hologo{pdfTeX}. If the parameter is not
% available, then the property will not be defined. The default value
% of the property is the current setting of the parameter.
%
% The module \xmodule{thepage} is loaded that generates a label
% for each page. The properties of module \xmodule{pagelayout}
% are added to the property list |thepage| of module \xmodule{thepage}.
%
% List of properties:
% \begin{quote}
%   \def\x#1{\cs{#1}&#1}%
%   \begin{tabular}{l>{\sffamily}ll}
%     \textrm{parameter} & \textrm{property} & remarks\\
%     \hline
%     \x{mag}\\
%     \x{paperwidth}\\
%     \x{paperheight}\\
%     \x{stockwidth}& class \xclass{memoir}\\
%     \x{stockheight}& class \xclass{memoir}\\
%     \x{pdfpagewidth}& \hologo{pdfTeX}, \hologo{LuaTeX}\\
%     \x{pdfpageheight}& \hologo{pdfTeX}, \hologo{LuaTeX}\\
%     \x{pdfhorigin}& \hologo{pdfTeX}, \hologo{LuaTeX}\\
%     \x{pdfvorigin}& \hologo{pdfTeX}, \hologo{LuaTeX}\\
%     \x{hoffset}\\
%     \x{voffset}\\
%     \x{topmargin}\\
%     \x{oddsidemargin}\\
%     \x{evensidemargin}\\
%     \x{textwidth}\\
%     \x{textheight}\\
%     \x{headheight}\\
%     \x{headsep}\\
%     \x{footskip}\\
%     \x{marginparwidth}\\
%     \x{marginparsep}\\
%     \x{columnwidth}\\
%     \x{columnsep}\\
%   \end{tabular}
% \end{quote}
%
% \begin{declcs}{zlistpagelayout}
% \end{declcs}
% At the end of document the page layout parameter for each page are
% printed into the \xext{log} file if macro \cs{zlistpagelayout}
% is called before |\end{document}| (preamble is a good place).
%
% \subsection{Module \xmodule{marks}}
%
% ToDo.
%
% \subsection{Module \xmodule{runs}}
%
% Module \xmodule{runs} counts the \hologo{LaTeX} runs since last
% \xext{aux} file creation and prints the number in the
% \xext{log} file.
%
% \begin{declcs}{zruns}\markexp
% \end{declcs}
% Prints the the total number of \hologo{LaTeX} runs including
% the current one. It expands to an explicit number.
% Before |begin{document}| the value is zero meaning
% the \xext{aux} file is not read yet. If a previous
% \xext{aux} file exists, the value found there increased
% by one is the new number. Otherwise \cs{zruns} is set to one.
% \hologo{LaTeX}~runs where the \xext{aux} files are not rewritten
% are not counted (see \cs{nofiles}).
%
% \subsection{Module \xmodule{perpage}}
%
% With \cs{@addtoreset} or \cs{numberwithin} a counter can be
% reset if another counter is incremented. This do not work
% well if the other counter is the page counter. The page counter is
% incremented in the output routine that is often called asynchronous
% somewhere on the next page. A reference mechanism costs at least
% two \hologo{LaTeX} runs, but ensures correct page counter values.
%
% \begin{declcs}{zmakeperpage}
%   |[|\meta{reset}|]| \M{counter}
% \end{declcs}
% At the of a new page counter \meta{counter} starts
% counting with value \meta{reset} (default is \texttt{1}).
% The macro has the same
% syntax and semantics as \cs{MakePerPage} of package
% \xpackage{perpage} \cite{perpage}.
% Also \xoption{perpage} of package \xpackage{footmisc} \cite{footmisc}
% can easily be simulated by
% \begin{quote}
% |\zmakeperpage{footnote}| \itshape |% \usepackage[perpage]{footmisc}|
%\end{quote}
% If footnote symbols are used, some people dislike the
% first symbol \ensuremath{\dagger}. It can easily be skipped:
% \begin{quote}
% |\zmakeperpage[2]{footnote}|
% \end{quote}
%
% \begin{declcs}{thezpage}\\
%   counter |zpage|
% \end{declcs}
% If the formatted counter value of the counter that is reset at
% a new page contains the page value, then you can use \cs{thezpage},
% the page number of the current page. Or counter |zpage| can be
% used, if the page number should be formatted differently from
% the current page number. Example:
%\begin{quote}
%\begin{verbatim}
%\newcounter{foobar}
%\zmakeperpage{foobar}
%\renewcommand*{\thefoobar}{\thezpage-\arabic{foobar}}
% % or
%\renewcommand*{\thefoobar}{\roman{zpage}-\arabic{foobar}}
%\end{verbatim}
%\end{quote}
%
% \begin{declcs}{zunmakeperpage}
%   \M{counter}
% \end{declcs}
% The reset mechanism for this counter is deactivated.
%
%
% \subsection{Module \xmodule{counter}}
%
% This option just add the property |counter| to the main
% property list. The property stores the counter name,
% that was responsible for the reference. This is the
% property \xpackage{hyperref}'s \cs{autoref} feature
% uses. Thus this property |counter| may be useful
% for a reimplementation of the autoref feature,
% see the section \ref{sec:todo} with the todo list.
%
% \subsection{Module \xmodule{titleref}}
%
% This option makes section and caption titles available
% to the reference system similar to packages \xpackage{titleref}
% or \xpackage{nameref}.
%
% \begin{declcs}{ztitleref}
%   \M{refname}\markbabel
% \end{declcs}
% Print the section or caption title of reference \meta{refname},
% similar to \cs{nameref} or \cs{titleref}.
%
% \begin{declcs}{ztitlerefsetup}
%   |{|\textit{key$_1$=value$_1$, key$_2$=value$_2$, \dots}|}|
% \end{declcs}
% This command allows to configure the behaviour of module
% \xmodule{titleref}. The following keys are available:
% \begin{description}
% \let\bfseries\relax
% \item[\texttt{title=}\meta{value}]\mbox{}\\*
%   Sets the current title.
% \item[\texttt{stripperiod=true\string|false}]\mbox{}\\*
%   Follow package \xpackage{nameref} that
%   removes a last period. Default: |true|.
% \item[\texttt{expand=true\string|false}]\mbox{}\\*
%   Package \cs{titleref} expands the title first. This way
%   garbage and dangerous commands can be removed, e.g.
%   \cs{label}, \cs{index}\dots. See implementation section
%   for more details. Default is |false|.
% \item[\texttt{cleanup=\{}\textrm{\dots}\texttt{\}}]\mbox{}\\*
%   Hook to add own cleanup code, if method |expand| is used.
%   See implementation section for more details.
% \end{description}
%
% \subsection{Module \xmodule{savepos}}
%
% This option supports a feature that \pdfTeX\ provides
% (and \hologo{XeTeX}).
% \pdfTeX\ is able to tell the current position on the
% page. The page position is not instantly known. First the
% page must be constructed by \hologo{TeX}'s asynchronous output
% routine. Thus the time where the position is known is
% the page shipout time. Thus a reference system where the
% information is recorded in the first run and made available
% for use in the second run comes in handy.
%
% \begin{declcs}{zsavepos}
%   \M{refname}
% \end{declcs}
% It generates a reference with name \meta{refname}.
% The reference stores the location where \cs{zsavepos}
% is executed in properties |posx| and |posy|.
%
% \begin{declcs}{zsaveposx}
%   \M{refname}\\
%   \cs{zsaveposy}
%   \M{refname}
% \end{declcs}
% Same as \cs{zsavepos} except that only the |x| or |y| component
% of the position is stored. Since 2011/12/05 v2.23.
%
% \begin{declcs}{zposx}\markexp
%   \M{refname}\\
%   \SpecialUsageIndex{\zposy}\cs{zposy}\markexp \M{refname}
% \end{declcs}
% Get the position as number. Unit is sp. Horizontal
% positions by \cs{zposx} increase from left to right.
% Vertical positions by \cs{zposy} from bottom to top.
%
% Do not rely on absolute page numbers. Because of problems
% with the origin the numbers may differ in DVI or PDF mode
% of \pdfTeX. Therefore work with relative values by comparisons.
%
% Both \cs{zposx} and \cs{zposy} are expandable and can be
% used inside calculations (\cs{setcounter}, \cs{addtocounter},
% package \xpackage{calc}, \cs{numexpr}). However this property
% prevents from notifying \hologo{LaTeX} that the reference is actually
% used (the notifying is not expandable). Therefore you should
% mark the reference as used by \cs{zrefused}.
%
% This module uses \pdfTeX's \cs{pdfsavepos}, \cs{pdflastxpos},
% and \cs{pdflastypos}. They are available in PDF mode and
% since version 1.40.0 also in DVI mode.
%
% \begin{declcs}{zref@savepos}
% \end{declcs}
% Macro \cs{zref@savepos} performs the first part of \cs{zsavepos}
% by calling \cs{pdfsavepos} (if \xext{aux} files are writable).
%
% Thus \cs{zsavepos} is basically \cs{zref@savepos} followed
% by |\zref@labelbylist{|\texttt{\meta{refname}}|}{savepos}|.
% If \cs{TeXXeTstate} is detected and enabled, \cs{savepos}
% also adds \cs{zref@savepos} at the end to support \cs{beginR}
% where the whatits are processed in reverse order.
% The property list |savepos| contains the properties |posx|
% and |posy|.
%
% \subsection{Module \xmodule{abspos}}
% Module \xmodule{abspos} allows to get various values of the page layout. There is no user
% command, only a number of internal commands. For example:
% \begin{quote}
%   |\zref@absposx{|\meta{label}|}{|\meta{value}|}{|\meta{position}|}|\\
%   |\zref@absposy{|\meta{label}|}{|\meta{value}|}{|\meta{position}|}|
% \end{quote}
%
% The return value is like in the module \xmodule{savepos} a number representing a length in sp.
% The length are measured from the bottom left of the page.
%
% \meta{label} is a label set with \cs{zlabel} or \cs{zsavepos} that allows to retrieve the
% absolute page number.
%
% \meta{position} is for the x-command one of |left|, |right| or |center|. For the y-command
%  it is one of |top|, |bottom|, |center|.
%
% The possible content of \meta{value} can be seen in the following table. Be aware that the code
% makes some assumptions which are perhaps not always true -- for example that the left of
% the head is identical to the left of the body.
%
% \medskip
% \begin{tabular}{lll}
%  \textbf{value} & \textbf{axis} & \textbf{comments}\\\hline
%   media         & x             &   left=0, right=\cs{pdfpagewidth} \\
%   paper         & x             &   left=0, right=\cs{paperwidth}\\
%   stock         & x             &   derived from paper\\
%   media         & y             & bottom=0, top=\cs{pdfpageheigh}\\
%   paper         & y             & top=\cs{pdfpageheight}, bottom=top-\cs{paperheight}\\
%   stock         & y             & top  derived from paper\\
%   head          & x             & calculated with hoffset, horigin, etc\\
%   head          & y             & calculated\\
%   body          & x             & = head value\\%
%   body          & y             & = head bottom - \cs{headsep}\\
%   foot          & x             & = head\\
%   foot          & y             & calculated from body bottom and \cs{footskip}\\
%   marginpar     & x             &different on odd/even pages! \\
%   marginpar     & y             & = body %
% \end{tabular}
%
% \subsection{Module \xmodule{dotfill}}
%
% \begin{declcs}{zdotfill}
% \end{declcs}
% This package provides the command \cs{zdotfill} that works
% similar to \cs{dotfill}, but can be configured. Especially
% it suppresses the dots if a minimum number of dots cannot be set.
%
% \begin{declcs}{zdotfillsetup}
%   |{|\textit{key$_1$=value$_1$, key$_2$=value$_2$, \dots}|}|
% \end{declcs}
% This command allows to configure the behaviour of \cs{zdotfill}.
% The following keys are available:
% \begin{description}
% \let\bfseries\relax
% \item[\texttt{min=}\meta{count value}]\mbox{}\\*
%   If the actual number of dots are smaller than \meta{count value},
%   then the dots are suppressed. Default: |2|.
% \item[\texttt{unit=}\meta{dimen value}]\mbox{}\\*
%   The width of a dot unit is given by \meta{dimen value}.
%   Default: |0.44em| (same as the unit in \cs{dotfill}).
% \item[\texttt{dot=}\meta{value}]\mbox{}\\*
%   The dot itself is given by \meta{value}. Default: |.| (dot, same as
%   the dot in \cs{dotfill}).
% \end{description}
%
% \subsection{Module \xmodule{env}}
%
% This module defines two properties |envname| and |envline|. They
% remember the name of the environment and the line number at
% the start of the environment.
%
% \subsection{Module \xmodule{xr}}
%
% This package provides the functionality of package \xpackage{xr},
% see \cite{xr}. It also supports the syntax of \xpackage{xr-hyper}.
%
% \begin{declcs}{zexternaldocument}
%   \* |[|\meta{prefix}|]|\markbabel \M{external document}
%   |[|\meta{url}|]|
% \end{declcs}
% See \cs{externaldocument} for a description of this option.
% The found labels also get a property |externaldocument| that
% remembers \meta{external document}.
% The standard reference scheme and the scheme of this
% package use different name spaces for reference names.
% If the external document uses both systems. Then one
% import statement would put the names in one namespace
% and probably causing problems with multiple references
% of the same name. Thus the star form only looks for
% \cs{newlabel} in the \xfile{.aux} files, whereas without
% star only \cs{zref@newlabel}s are used.
%
% In the star form it tries to detect labels from
% \xpackage{hyperref}, \xpackage{titleref}, and \xpackage{ntheorem}.
% If such an extended property from the packages before cannot
% be found or are empty, they are not included in the
% imported reference.
%
% Warnings are given if a reference name is already in use
% and the item is ignored. Unknown properties will automatically
% be declared.
%
% If the external references contain |anchor| properties,
% then we need also a url to be able to address the external
% file. As default the filename is taken with a default
% extension.
%
% \begin{declcs}{zxrsetup}
%   |{|\textit{key$_1$=value$_1$, key$_2$=value$_2$, \dots}|}|
% \end{declcs}
% The following setup options are available:
% \begin{description}
% \item[\xoption{ext}:] It sets the default extension.
% \item[\xoption{tozreflabel}:] Boolean option. The found references
%   are imported as zref labels. This is enabled by default.
% \item[\xoption{toltxlabel}:] Boolean option. The found references
%   are imported as \hologo{LaTeX} labels. Packages \xpackage{nameref},
%   \xpackage{titleref} and class \xclass{memoir} are supported.
% \item[\xoption{urluse}:] Boolean option. If enabled, then a URL
%   is stored in a macro and the macro is put in property `urluse'.
%   The URL is not put in property `url'.
%   The purpose is to save \hologo{TeX} memory.
% \item[\xoption{verbose}:] Boolean option. List the imported labels in the
%   \xext{log} file. Default is |false|.
% \end{description}
%
% \begin{declcs}{zref@xr@ext}
% \end{declcs}
% If the \meta{url} is not specified in \cs{zref@externaldocument},
% then the url will be constructed with the file name and
% this macro as extension. \cs{XR@ext} is used if
% \xpackage{hyperref} is loaded, otherwise |pdf|.
%
% \subsection{Module \xmodule{pageattr}}
%
% This module allows to recover the content of the register
% \cs{pdfpageattr} and \cs{pdfpagesattr} in pdftex and the equivalent register
% in luatex. There is no user command. Programmer commands are
%
% \begin{quote}
% |\zref@pdfpageattr{|\meta{absolute page number}|}|
%
% |\zref@pdfpagesattr{|\meta{absolute page number}|}|
% \end{quote}
% \section{ToDo}\label{sec:todo}
%
% Among other things the following issues are left for future work:
% \begin{itemize}
% \item Other applications: autoref, hyperref, \dots
% \end{itemize}
%
% \section{Example}
%
%    \begin{macrocode}
%<*example>
%    \end{macrocode}
%
%    \begin{macrocode}
\documentclass{book}

\usepackage[ngerman]{babel}%

\usepackage[savepos,totpages,titleref,dotfill,counter,user]{zref}

%    \end{macrocode}
%
%    Chapters are wrapped inside \cs{ChapterStart} and
%    \cs{ChapterStop}. The first argument \verb|#1| of \cs{ChapterStart}
%    is used to form a label id \verb|chap:#1|. At the end of
%    the chapter another label is set by \cs{zref@wrapper@immediate}, because
%    otherwise at the end of document a deferred write would not
%    be written, because there is no page for shipout.
%
%    Also this example shows how chapter titles can be recorded.
%    A new property \verb|chaptitle| is declared and added to the
%    main property list. In \cs{ChapterStart} the current value
%    of the property is updated.
%    \begin{macrocode}
\makeatletter
\zref@newprop{chaptitle}{}
\zref@addprop{main}{chaptitle}

\newcommand*{\ChapterStart}[2]{%
  \cleardoublepage
  \def\current@chapid{#1}%
  \zref@setcurrent{chaptitle}{#2}%
  \chapter{#2}%
  \zlabel{chap:#1}%
}
\newcommand*{\ChapterStop}{%
  \cleardoublepage
  \zref@wrapper@immediate{%
    \zref@labelbyprops{chapend:\current@chapid}{abspage}%
  }%
}
%    \end{macrocode}
%    \cs{ChapterPages} calculates and returns the number of pages of the
%    referenced chapter.
%    \begin{macrocode}
\newcommand*{\ChapterPages}[1]{%
  \zrefused{chap:#1}%
  \zrefused{chapend:#1}%
  \number\numexpr
    \zref@extract{chapend:#1}{abspage}%
    -\zref@extract{chap:#1}{abspage}%
    +1\relax
}
\makeatother
%    \end{macrocode}
%    \begin{macrocode}
\begin{document}
%    \end{macrocode}
%    As exception we use \cs{makeatletter} here, because
%    this is just an example file that also should show some of
%    programmer's interface.
%    \begin{macrocode}
\makeatletter

\frontmatter
\zlabel{documentstart}

\begin{itemize}
\item
  The frontmatter part has
  \number\numexpr\zref@extract{chap:first}{abspage}-1\relax
  ~pages.
\item
  Chapter \zref{chap:first} has \ChapterPages{first} page(s).
\item
  Section \zref{hello} is on the
  \ifcase\numexpr
    \zref@extractdefault{hello}{page}{0}%
    -\zref@extractdefault{chap:first}{page}{0}%
    +1\relax
    ??\or first\or second\or third\or forth\fi
  ~page inside its chapter.
\item
  The document has
  \zref[abspage]{LastPage} pages.
  This number is \ifodd\ztotpages odd\else even\fi.
\item
  The last page is labeled with \zpageref{LastPage}.
\item
  The title of chapter \zref{chap:next} %
  is ``\zref[chaptitle]{chap:next}''.
\end{itemize}

\tableofcontents

\mainmatter
\ChapterStart{first}{First chapter}

%    \end{macrocode}
%    The user level commands should protect babel shorthands
%    where possible. On the other side, expandable extracting
%    macros are useful in calculations, see above the
%    examples with \cs{numexpr}.
%    \begin{macrocode}
\section{Test}
\zlabel{a"o}
Section \zref{a"o} on page
\zref@wrapper@babel\zref@extract{a"o}{page}.

Text.
\newpage

\section{Hello World}
\zlabel{hello}

\ChapterStop

\ChapterStart{next}{Next chapter with \emph{umlauts}: "a"o"u"s}

%    \end{macrocode}
%
%    Here an example follows that makes use of \pdfTeX's ``savepos''
%    feature. The position on the page is not known before the
%    page is constructed and shipped out. Therefore the position
%    ist stored in references and are available for calculations
%    in the next \hologo{LaTeX} compile run.
%    \begin{macrocode}
The width of the first column is
  \the\dimexpr \zposx{secondcol}sp - \zposx{firstcol}sp\relax,\\
the height difference of the two baselines is
  \the\dimexpr \zposy{firstcol}sp - \zposy{secondline}sp\relax:\\
\begin{tabular}{ll}
  \zsavepos{firstcol}Hello&\zsavepos{secondcol}World\\
  \zsavepos{secondline}Second line&foobar\\
\end{tabular}

%    \end{macrocode}
%    With \cs{zrefused} \hologo{LaTeX} is notified, if the references
%    are not yet available and \hologo{LaTeX} can generate the rerun
%    hint.
%    \begin{macrocode}
\zrefused{firstcol}
\zrefused{secondcol}
\zrefused{secondline}

\ChapterStop
%    \end{macrocode}
%    Test for module \cs{dotfill}.
%    \begin{macrocode}
\ChapterStart{dotfill}{Test for dotfill feature}
\newcommand*{\dftest}[1]{%
  #1&
  [\makebox[{#1}]{\dotfill}]&
  [\makebox[{#1}]{\zdotfill}]\\
}
\begin{tabular}{rll}
& [\verb|\dotfill|] & [\verb|\zdotfill|]\\
\dftest{0.43em}
\dftest{0.44em}
\dftest{0.45em}
\dftest{0.87em}
\dftest{0.88em}
\dftest{0.89em}
\dftest{1.31em}
\dftest{1.32em}
\dftest{1.33em}
\end{tabular}
\ChapterStop
\end{document}
%    \end{macrocode}
%
%    \begin{macrocode}
%</example>
%    \end{macrocode}
%
% \StopEventually{
% }
%
% \section{Implementation}
%
% \subsection{Package \xpackage{zref}}
%
% \subsubsection{Identification}
%
%    \begin{macrocode}
%<*package>
\NeedsTeXFormat{LaTeX2e}
\ProvidesPackage{zref}
  [2020-07-03 v2.32 A new reference scheme for LaTeX (HO)]%
%    \end{macrocode}
%
% \subsubsection{Load basic module}
%
%    \begin{macrocode}
\RequirePackage{zref-base}[2019/11/29]
%    \end{macrocode}
%    Abort package loading if \xpackage{zref-base} could not be
%    loaded successfully.
%    \begin{macrocode}
\@ifundefined{ZREF@base@ok}{\endinput}{}
%    \end{macrocode}
%
% \subsubsection{Process options}
%
%    Known modules are loaded and the release date is checked.
%    \begin{macrocode}
\def\ZREF@temp#1{%
  \DeclareOption{#1}{%
    \AtEndOfPackage{%
      \RequirePackage{zref-#1}[2019/11/29]%
    }%
  }%
}
\ZREF@temp{abspage}
\ZREF@temp{counter}
\ZREF@temp{dotfill}
\ZREF@temp{hyperref}
\ZREF@temp{lastpage}
\ZREF@temp{marks}
\ZREF@temp{nextpage}
\ZREF@temp{pageattr}
\ZREF@temp{pagelayout}
\ZREF@temp{perpage}
\ZREF@temp{runs}
\ZREF@temp{savepos}
\ZREF@temp{thepage}
\ZREF@temp{titleref}
\ZREF@temp{totpages}
\ZREF@temp{user}
\ZREF@temp{xr}
%    \end{macrocode}
%    \begin{macrocode}
\ProcessOptions\relax
%</package>
%    \end{macrocode}
%
%
% \subsection{Module \xmodule{base}}
%
% \subsubsection{Prefixes}
%
%    This package uses the following prefixes for macro names:
%    \begin{description}
%    \item[\cs{zref@}:] Macros of the programmer's interface.
%    \item[\cs{ZREF@}:] Internal macros.
%    \item[\cs{Z@L@}\metacs{listname}:]
%      The properties of the list \meta{listname}.
%    \item[\cs{Z@D@}\metacs{propname}:]
%      The default value for property \meta{propname}.
%    \item[\cs{Z@E@}\metacs{propname}:]
%      Extract function for property \meta{propname}.
%    \item[\cs{Z@X@}\metacs{propname}:]
%      Information whether a property value for property
%      \meta{propname} is expanded
%      immediately or at shipout time.
%    \item[\cs{Z@C@}\metacs{propname}:]
%      Current value of the property \meta{propname}.
%    \item[\cs{Z@R@}\metacs{labelname}:]
%      Data for reference \meta{labelname}.
%    \item[\cs{ZREF@org@}:] Original versions of patched commands.
%    \item[\cs{z}:] For macros in user land, defined if
%      module \xmodule{user} is set.
%    \end{description}
%    The following family names are used for keys defined
%    according to the \xpackage{keyval} package:
%    \begin{description}
%    \item[\texttt{ZREF@TR}:] Setup for module \xmodule{titleref}.
%    \end{description}
%
% \subsubsection{Identification}
%
%    \begin{macrocode}
%<*base>
\NeedsTeXFormat{LaTeX2e}
\ProvidesPackage{zref-base}%
  [2020-07-03 v2.32 Module base for zref (HO)]%
%    \end{macrocode}
%
% \subsubsection{Utilities}
%
%    \begin{macrocode}
\providecommand\IfFormatAtLeastTF{\@ifl@t@r\fmtversion}
\RequirePackage{ltxcmds}[2010/12/02]
\RequirePackage{infwarerr}[2010/04/08]
\RequirePackage{kvsetkeys}[2010/03/01]
\RequirePackage{kvdefinekeys}[2010/03/01]
\RequirePackage{pdftexcmds}[2010/04/01]
%    \end{macrocode}
%
%    \begin{macro}{\ZREF@name}
%    Several times the package name is used, thus we store it
%    in \cmd{\ZREF@name}.
%    \begin{macrocode}
\def\ZREF@name{zref}
%    \end{macrocode}
%    \end{macro}
%
%    \begin{macrocode}
\ltx@IfUndefined{protected}{%
  \RequirePackage{makerobust}[2006/03/18]%
%    \end{macrocode}
%    \begin{macro}{\ZREF@Robust}
%    \begin{macrocode}
  \def\ZREF@Robust#1#2{%
    \def\ZREF@temp{\MakeRobustcommand#2}%
    \afterassignment\ZREF@temp
    #1#2%
  }%
%    \end{macrocode}
%    \end{macro}
%    \begin{macrocode}
}{%
%    \end{macrocode}
%    \begin{macro}{\ZREF@Robust}
%    \begin{macrocode}
  \def\ZREF@Robust#1{%
    \protected#1%
  }%
%    \end{macrocode}
%    \end{macro}
%    \begin{macrocode}
}
%    \end{macrocode}
%    \begin{macro}{\ZREF@IfDefinable}
%    \begin{macrocode}
\def\ZREF@IfDefinable#1#2#3{%
  \@ifdefinable{#1}{%
    \ZREF@Robust{#2}#1#3%
  }%
}
%    \end{macrocode}
%    \end{macro}
%
%    \begin{macro}{\ZREF@UpdatePdfTeX}
%    \cs{ZREF@UpdatePdfTeX} is used as help message text in
%    error messages.
%    \begin{macrocode}
\def\ZREF@UpdatePdfTeX{Update pdfTeX.}
%    \end{macrocode}
%    \end{macro}
%
%    \begin{macro}{\ifZREF@found}
%    The following switch is usded in list processing.
%    \begin{macrocode}
\newif\ifZREF@found
%    \end{macrocode}
%    \end{macro}
%
%    \begin{macro}{\ZREF@patch}
%    Macro \cs{ZREF@patch} first checks the existence of
%    the command and safes it.
%    \begin{macrocode}
\def\ZREF@patch#1{%
  \ltx@IfUndefined{#1}{%
    \ltx@gobble
  }{%
    \expandafter\let\csname ZREF@org@#1\expandafter\endcsname
    \csname #1\endcsname
    \ltx@firstofone
  }%
}
%    \end{macrocode}
%    \end{macro}
%
% \subsubsection{Check for \hologo{eTeX}}
%
%    The use of \hologo{eTeX} should be standard nowadays for \hologo{LaTeX}.
%    We test for \hologo{eTeX} in order to use its features later.
%    \begin{macrocode}
\ltx@IfUndefined{eTeXversion}{%
  \PackageError\ZREF@name{%
    Missing support for eTeX; package is abandoned%
  }{%
    Use a TeX compiler that support eTeX and enable eTeX %
    in the format.%
  }%
  \endinput
}{}%
%    \end{macrocode}
%
%    \begin{macrocode}
\RequirePackage{etexcmds}[2007/09/09]
\ifetex@unexpanded
\else
  \PackageError\ZREF@name{%
    Missing e-TeX's \string\unexpanded.\MessageBreak
    Add \string\RequirePackage\string{etexcmds\string} before %
    \string\documentclass%
  }{%
    Probably you are using some package (e.g. ConTeXt) that %
    redefines \string\unexpanded%
  }%
  \expandafter\endinput
\fi
%    \end{macrocode}
%
% \subsubsection{Auxiliary file stuff}
%
%    We are using some commands in the \xfile{.aux} files.
%    However sometimes these auxiliary files are interpreted
%    by \hologo{LaTeX} processes that haven't loaded this package
%    (e.g. package \xpackage{xr}).
%    Therefore we provide dummy definitions.
%    \begin{macrocode}
\RequirePackage{auxhook}
\AddLineBeginAux{%
  \string\providecommand\string\zref@newlabel[2]{}%
}
%    \end{macrocode}
%    \begin{macro}{\ZREF@RefPrefix}
%    \begin{macrocode}
\def\ZREF@RefPrefix{Z@R}
%    \end{macrocode}
%    \end{macro}
%    \begin{macro}{\zref@newlabel}
%    For the implementation of \cs{zref@newlabel} we call
%    the same internal macro \cs{@newl@bel} that is used
%    in \cs{newlabel}. Thus we have for free:
%    \begin{itemize}
%    \item \cs{Z@R@}\metacs{labelname} is defined.
%    \item \hologo{LaTeX}'s check for multiple references.
%    \item \hologo{LaTeX}'s check for changed references.
%    \end{itemize}
%    \begin{macrocode}
\ZREF@Robust\edef\zref@newlabel{%
  \noexpand\@newl@bel{\ZREF@RefPrefix}%
}
%    \end{macrocode}
%    \end{macro}
%
% \subsubsection{Property lists}
%
%    \begin{macro}{\zref@newlist}
%    Property lists are stored as list of property names
%    enclosed in curly braces. \cs{zref@newlist}
%    creates a new list as empty list. Assignments to
%    property lists are global.
%    \begin{macrocode}
\ZREF@Robust\def\zref@newlist#1{%
  \zref@iflistundefined{#1}{%
    \@ifdefinable{Z@L@#1}{%
     \global\expandafter\let\csname Z@L@#1\endcsname\ltx@empty
     \PackageInfo\ZREF@name{New property list: #1}%
    }%
  }{%
    \PackageError\ZREF@name{%
      Property list `#1' already exists%
    }\@ehc
  }%
}
%    \end{macrocode}
%    \end{macro}
%
%    \begin{macro}{\zref@iflistundefined}
%    \cs{zref@iflistundefined} checks the existence of the
%    property list \verb|#1|. If the property list is present,
%    then \verb|#2| is executed and \verb|#3| otherwise.
%    \begin{macrocode}
\def\zref@iflistundefined#1{%
  \ltx@ifundefined{Z@L@#1}%
}
%    \end{macrocode}
%    \end{macro}
%    \begin{macro}{\zref@listexists}
%    \cs{zref@listexists} only executes \verb|#2| if the
%    property list \verb|#1| exists and raises an error
%    message otherwise.
%    \begin{macrocode}
\ZREF@Robust\def\zref@listexists#1{%
  \zref@iflistundefined{#1}{%
    \PackageError\ZREF@name{%
      Property list `#1' does not exist%
    }\@ehc
  }%
}
%    \end{macrocode}
%    \end{macro}
%
%    \begin{macro}{\zref@iflistcontainsprop}
%    \cs{zref@iflistcontainsprop} checks, whether a property \verb|#2| is
%    already present in a property list \verb|#1|.
%    \begin{macrocode}
\ZREF@Robust\def\zref@iflistcontainsprop#1#2{%
  \zref@iflistundefined{#1}{%
    \ltx@secondoftwo
  }{%
    \begingroup\expandafter\endgroup
    \expandafter\in@
    \csname#2\expandafter\expandafter\expandafter\endcsname
    \expandafter\expandafter\expandafter{\csname Z@L@#1\endcsname}%
    \csname ltx@\ifin@ first\else second\fi oftwo\endcsname
  }%
}
%    \end{macrocode}
%    \end{macro}
%
%    \begin{macro}{\zref@listforloop}
%    \begin{macrocode}
\def\zref@listforloop#1#2{%
  \zref@listexists{#1}{%
    \expandafter\expandafter\expandafter\@tfor
    \expandafter\expandafter\expandafter\zref@prop
    \expandafter\expandafter\expandafter:%
    \expandafter\expandafter\expandafter=%
    \csname Z@L@#1\endcsname
    \do{%
      \begingroup
        \escapechar=-1 %
        \edef\x{\endgroup
          \def\noexpand\zref@prop{%
            \expandafter\string\zref@prop
          }%
        }%
      \x
      #2\zref@prop
    }%
  }%
}
%    \end{macrocode}
%    \end{macro}
%
%    \begin{macro}{\zref@addprops}
%    \cs{zref@addprop} adds the properties \verb|#2| to the property
%    list \verb|#1|, if the property is not already in the list.
%    Otherwise a warning is given.
%    \begin{macrocode}
\ZREF@Robust\def\zref@addprops#1#2{%
  \zref@listexists{#1}{%
    \comma@parse{#2}{%
      \zref@propexists\comma@entry{%
        \zref@iflistcontainsprop{#1}\comma@entry{%
          \PackageWarning\ZREF@name{%
            Property `\comma@entry' is already in list `#1'%
          }%
        }{%
          \begingroup\expandafter\endgroup
          \expandafter\g@addto@macro
          \csname Z@L@#1\expandafter\endcsname
          \expandafter{\csname\comma@entry\endcsname}%
        }%
      }%
      \ltx@gobble
    }%
  }%
}
%    \end{macrocode}
%    \end{macro}
%    \begin{macro}{\zref@addprop}
%    \cs{zref@addprop} adds the property \verb|#2| to the property
%    list \verb|#1|, if the property is not already in the list.
%    Otherwise a warning is given.
%    \begin{macrocode}
\ZREF@Robust\def\zref@addprop#1#2{%
  \zref@listexists{#1}{%
    \zref@propexists{#2}{%
      \zref@iflistcontainsprop{#1}{#2}{%
        \PackageWarning\ZREF@name{%
          Property `#2' is already in list `#1'%
        }%
      }{%
        \begingroup\expandafter\endgroup
        \expandafter\g@addto@macro
        \csname Z@L@#1\expandafter\endcsname
        \expandafter{\csname#2\endcsname}%
      }%
    }%
  }%
}
%    \end{macrocode}
%    \end{macro}
%
%    \begin{macro}{\zref@localaddprops}
%    \begin{macrocode}
\ZREF@Robust\def\zref@localaddprops#1#2{%
  \zref@listexists{#1}{%
    \comma@parse{#2}{%
      \zref@propexists\comma@entry{%
        \zref@iflistcontainsprop{#1}\comma@entry{%
          \PackageWarning\ZREF@name{%
            Property `\comma@entry' is already in list `#1'%
          }%
        }{%
          \begingroup\expandafter\endgroup
          \expandafter\ltx@LocalAppendToMacro
          \csname Z@L@#1\expandafter\endcsname
          \expandafter{\csname\comma@entry\endcsname}%
        }%
      }%
      \ltx@gobble
    }%
  }%
}
%    \end{macrocode}
%    \end{macro}
%    \begin{macro}{\zref@localaddprop}
%    \begin{macrocode}
\ZREF@Robust\def\zref@localaddprop#1#2{%
  \zref@listexists{#1}{%
    \zref@propexists{#2}{%
      \zref@iflistcontainsprop{#1}{#2}{%
        \PackageWarning\ZREF@name{%
          Property `#2' is already in list `#1'%
        }%
      }{%
        \begingroup\expandafter\endgroup
        \expandafter\ltx@LocalAppendToMacro
        \csname Z@L@#1\expandafter\endcsname
        \expandafter{\csname#2\endcsname}%
      }%
    }%
  }%
}
%    \end{macrocode}
%    \end{macro}
%
%    \begin{macrocode}
\ltx@IfUndefined{pdf@strcmp}{%
%    \end{macrocode}
%    \begin{macro}{\zref@delprop}
%    \begin{macrocode}
  \ZREF@Robust\def\zref@delprop{%
    \ZREF@delprop\gdef
  }%
%    \end{macrocode}
%    \end{macro}
%    \begin{macro}{\zref@localdelprop}
%    \begin{macrocode}
  \ZREF@Robust\def\zref@localdelprop{%
    \ZREF@delprop\def
  }%
%    \end{macrocode}
%    \end{macro}
%    \begin{macro}{\ZREF@delprop}
%    \begin{macrocode}
  \def\ZREF@delprop#1#2#3{%
    \zref@listexists{#2}{%
      \begingroup
        \escapechar=-1 %
        \def\ZREF@param{#3}%
        \@onelevel@sanitize\ZREF@param
        \toks@{}%
        \expandafter\expandafter\expandafter\ZREF@@delprop
        \csname Z@L@#2\endcsname!%
      \expandafter\endgroup
      \expandafter#1\csname Z@L@#2\expandafter\endcsname
      \expandafter{%
        \the\toks@
      }%
    }%
  }%
%    \end{macrocode}
%    \end{macro}
%    \begin{macro}{\ZREF@@delprop}
%    \begin{macrocode}
  \def\ZREF@@delprop#1{%
    \expandafter\ZREF@@@delprop\expandafter{\string#1}#1%
  }%
%    \end{macrocode}
%    \end{macro}
%    \begin{macro}{\ZREF@@@delprop}
%    \begin{macrocode}
  \def\ZREF@@@delprop#1#2{%
    \ifx#2!%
    \else
      \def\ZREF@temp{#1}%
      \@onelevel@sanitize\ZREF@temp
      \ifx\ZREF@param\ZREF@temp
      \else
        \toks@\expandafter{%
          \the\expandafter\toks@\csname#1\endcsname
        }%
      \fi
      \expandafter\ZREF@@delprop
    \fi
  }%
%    \end{macrocode}
%    \end{macro}
%    \begin{macrocode}
}{%
%    \end{macrocode}
%    \begin{macro}{\zref@delprop}
%    \begin{macrocode}
  \ZREF@Robust\def\zref@delprop{%
    \ZREF@delprop\xdef
  }%
%    \end{macrocode}
%    \end{macro}
%    \begin{macro}{\zref@localdelprop}
%    \begin{macrocode}
  \ZREF@Robust\def\zref@localdelprop{%
    \ZREF@delprop\edef
  }%
%    \end{macrocode}
%    \end{macro}
%    \begin{macro}{\ZREF@delprop}
%    \begin{macrocode}
  \def\ZREF@delprop#1#2#3{%
    \zref@listexists{#2}{%
      \def\ZREF@param{#3}%
      \edef\ZREF@SavedEscapechar{\the\escapechar}%
      \escapechar=-1 %
      \expandafter#1\csname Z@L@#2%
      \expandafter\expandafter\expandafter\endcsname{%
        \expandafter\expandafter\expandafter\ZREF@@delprop
        \csname Z@L@#2\endcsname!%
      }%
      \escapechar=\ZREF@SavedEscapechar\relax
    }%
  }%
%    \end{macrocode}
%    \end{macro}
%    \begin{macro}{\ZREF@@delprop}
%    Caution: |#1| might be an \cs{if} or similar token.
%    \begin{macrocode}
  \def\ZREF@@delprop#1{%
    \expandafter\ZREF@@@delprop\expandafter{\string#1}#1%
  }%
%    \end{macrocode}
%    \end{macro}
%    \begin{macro}{\ZREF@@@delprop}
%    \begin{macrocode}
  \def\ZREF@@@delprop#1#2{%
    \ifx#2!%
    \else
      \ifnum\pdf@strcmp{#1}{\ZREF@param}=\ltx@zero
      \else
        \expandafter\noexpand\csname#1\endcsname
      \fi
      \expandafter\ZREF@@delprop
    \fi
  }%
%    \end{macrocode}
%    \end{macro}
%    \begin{macrocode}
}
%    \end{macrocode}
%
% \subsubsection{Properties}
%
%    \begin{macro}{\zref@ifpropundefined}
%    \cs{zref@ifpropundefined} checks the existence of the
%    property \verb|#1|. If the property is present,
%    then \verb|#2| is executed and \verb|#3| otherwise.
%    \begin{macrocode}
\def\zref@ifpropundefined#1{%
  \ltx@ifundefined{Z@E@#1}%
}
%    \end{macrocode}
%    \end{macro}
%    \begin{macro}{\zref@propexists}
%    Some macros rely on the existence of a property.
%    \cs{zref@propexists} only executes \verb|#2| if the
%    property \verb|#1| exists and raises an error
%    message otherwise.
%    \begin{macrocode}
\ZREF@Robust\def\zref@propexists#1{%
  \zref@ifpropundefined{#1}{%
    \PackageError\ZREF@name{%
      Property `#1' does not exist%
    }\@ehc
  }%
}
%    \end{macrocode}
%    \end{macro}
%
%    \begin{macro}{\zref@newprop}
%    A new property is declared by \cs{zref@newprop}, the
%    property name \meta{propname} is given in \verb|#1|. The
%    property is created and configured. If the star form
%    is given, then the expansion of the property value
%    is delayed to page shipout time, when the reference
%    is written to the \xfile{.aux} file.
%    \begin{description}
%    \item[\cs{Z@D@}\metacs{propname}:]
%      Stores the default value for this property.
%    \item[\cs{Z@E@}\metacs{propname}:]
%      Extract function.
%    \item[\cs{Z@X@}\metacs{propname}:]
%      Information whether the expansion of the property value
%      is delayed to shipout time.
%    \item[\cs{Z@C@}\metacs{propname}:]
%      Current value of the property.
%    \end{description}
%    \begin{macrocode}
\ZREF@Robust\def\zref@newprop{%
  \@ifstar{%
    \let\ZREF@X\noexpand
    \ZREF@newprop
  }{%
    \let\ZREF@X\ltx@empty
    \ZREF@newprop
  }%
}
%    \end{macrocode}
%    \end{macro}
%    \begin{macro}{\ZREF@newprop}
%    \begin{macrocode}
\def\ZREF@newprop#1{%
  \edef\ZREF@P{#1}%
  \@onelevel@sanitize\ZREF@P
  \begingroup
  \ifx\ZREF@P\ZREF@par
    \@PackageError\ZREF@name{%
      Invalid property name `\ZREF@P'%
    }{%
      The property name `par' is not allowed %
      because of internal reasons.%
      \MessageBreak
      \@ehc
    }%
    \def\ZREF@@newprop[##1]##2{\endgroup}%
  \else
    \zref@ifpropundefined\ZREF@P{%
      \endgroup
      \PackageInfo\ZREF@name{%
        New property: \ZREF@P
      }%
    }{%
      \@PackageError\ZREF@name{%
        Property `\ZREF@P' already exists%
      }\@ehc
      \def\ZREF@@newprop[##1]##2{\endgroup}%
    }%
  \fi
  \@ifnextchar[\ZREF@@newprop{\ZREF@@newprop[\zref@default]}%
}
%    \end{macrocode}
%    \end{macro}
%    \begin{macro}{\ZREF@par}
%    \begin{macrocode}
\def\ZREF@par{par}
\@onelevel@sanitize\ZREF@par
%    \end{macrocode}
%    \end{macro}
%    \begin{macro}{\ZREF@@newprop}
%    \begin{macrocode}
\def\ZREF@@newprop[#1]{%
  \global\@namedef{Z@D@\ZREF@P}{#1}%
  \global\expandafter\let\csname Z@X@\ZREF@P\endcsname\ZREF@X
  \begingroup\expandafter\endgroup
  \expandafter\ZREF@@@newprop\csname\ZREF@P\endcsname
  \expandafter\gdef\csname Z@C@\ZREF@P\endcsname{}%
  \zref@setcurrent\ZREF@P
}
\def\ZREF@@@newprop#1{%
  \expandafter
  \gdef\csname Z@E@\ZREF@P\endcsname##1#1##2##3\ZREF@nil{##2}%
}
%    \end{macrocode}
%    \end{macro}
%
%    \begin{macro}{\zref@showprop}
%    \begin{macrocode}
\ZREF@Robust\def\zref@showprop#1{%
  \zref@ifpropundefined{#1}{%
    \@PackageInfoNoLine{\ZREF@name}{%
      Show property `#1': <undefined>%
    }%
  }{%
    \begingroup
      \toks@\expandafter\expandafter\expandafter{%
        \csname Z@C@#1\endcsname
      }%
      \edef\ZREF@value{\the\toks@}%
      \ltx@onelevel@sanitize\ZREF@value
      \toks@\expandafter\expandafter\expandafter{%
        \csname Z@D@#1\endcsname
      }%
      \edef\ZREF@default{\the\toks@}%
      \ltx@onelevel@sanitize\ZREF@default
      \@PackageInfoNoLine{\ZREF@name}{%
        Show property `#1':\MessageBreak
        \expandafter\ifx\csname Z@X@#1\endcsname\ltx@empty
          Immediate %
        \else
          Delayed %
        \fi
        value: [\ZREF@value]\MessageBreak
        Default: [\ZREF@default]%
      }%
    \endgroup
  }%
}
%    \end{macrocode}
%    \end{macro}
%
%    \begin{macro}{\zref@setcurrent}
%    \cs{zref@setcurrent} sets the current value for a property.
%    \begin{macrocode}
\ZREF@Robust\def\zref@setcurrent#1#2{%
  \zref@propexists{#1}{%
    \expandafter\def\csname Z@C@#1\endcsname{#2}%
  }%
}
%    \end{macrocode}
%    \end{macro}
%
%    \begin{macro}{\ZREF@getcurrent}
%    \cs{zref@getcurrent} gets the current value for a property.
%    \begin{macrocode}
\def\ZREF@getcurrent#1{%
  \romannumeral0%
  \ltx@ifundefined{Z@C@#1}{%
    \ltx@space
  }{%
    \expandafter\expandafter\expandafter\ltx@space
    \csname Z@C@#1\endcsname
  }%
}
%    \end{macrocode}
%    \end{macro}
%    \begin{macro}{\ZREF@u@getcurrent}
%    \begin{macrocode}
\def\ZREF@wu@getcurrent#1{%
  \etex@unexpanded\expandafter\expandafter\expandafter{%
    \ZREF@getcurrent{#1}%
  }%
}
%    \end{macrocode}
%    \end{macro}
%    \begin{macro}{\zref@getcurrent}
%    \begin{macrocode}
\let\zref@getcurrent\ZREF@getcurrent
%    \end{macrocode}
%    \end{macro}
%
% \subsubsection{Reference generation}
%
%    \begin{macro}{\zref@label}
%    Label macro that uses the main property list.
%    \begin{macrocode}
\ZREF@Robust\def\zref@label#1{%
  \zref@labelbylist{#1}\ZREF@mainlist
}
%    \end{macrocode}
%    \end{macro}
%    \begin{macro}{\zref@labelbylist}
%    Label macro that stores the properties, specified in the
%    property list \verb|#2|.
%    \begin{macrocode}
\ZREF@Robust\def\zref@labelbylist#1#2{%
  \@bsphack
    \zref@listexists{#2}{%
      \expandafter\expandafter\expandafter\ZREF@label
      \expandafter\expandafter\expandafter{%
        \csname Z@L@#2\endcsname
      }{#1}%
    }%
  \@esphack
}
%    \end{macrocode}
%    \end{macro}
%    \begin{macro}{\zref@labelbyprops}
%    The properties are directly specified in a comma separated list.
%    \begin{macrocode}
\ZREF@Robust\def\zref@labelbyprops#1#2{%
  \@bsphack
    \begingroup
      \toks@{}%
      \comma@parse{#2}{%
        \zref@ifpropundefined\comma@entry{%
          \PackageWarning\ZREF@name{%
            Property `\comma@entry' is not known%
          }%
        }{%
          \toks@\expandafter{%
            \the\expandafter\toks@\csname\comma@entry\endcsname
          }%
        }%
        \ltx@gobble
      }%
    \expandafter\endgroup
    \expandafter\ZREF@label\expandafter{\the\toks@}{#1}%
  \@esphack
}
%    \end{macrocode}
%    \end{macro}
%    \begin{macro}{\zref@labelbykv}
%    \begin{macrocode}
\ZREF@Robust\def\zref@labelbykv#1#2{%
  \@bsphack
    \begingroup
      \let\Z@L@ZREF@temp\ltx@empty
      \kvsetkeys{ZREF@LABEL}{#1}%
      \ifZREF@immediate
        \expandafter\zref@wrapper@immediate\expandafter{%
          \expandafter\ZREF@label\expandafter{\Z@L@ZREF@temp}{#2}%
        }%
      \else
        \expandafter\ZREF@label\expandafter{\Z@L@ZREF@temp}{#2}%
      \fi
    \endgroup
  \@esphack
}
%    \end{macrocode}
%    \end{macro}
%    \begin{macrocode}
\kv@define@key{ZREF@LABEL}{prop}{%
  \edef\ZREF@param{#1}%
  \zref@propexists\ZREF@param{%
    \zref@iflistcontainsprop{ZREF@temp}\ZREF@param{}{%
      \begingroup\expandafter\endgroup
      \expandafter\ltx@LocalAppendToMacro
      \expandafter\Z@L@ZREF@temp
      \expandafter{\csname\ZREF@param\endcsname}%
    }%
  }%
}
\kv@define@key{ZREF@LABEL}{list}{%
  \zref@listforloop{#1}{%
    \zref@iflistcontainsprop{ZREF@temp}\zref@prop{}{%
      \begingroup\expandafter\endgroup
      \expandafter\ltx@LocalAppendToMacro
      \expandafter\Z@L@ZREF@temp
      \expandafter{\csname\zref@prop\endcsname}%
    }%
    \ltx@gobble
  }%
}
\kv@define@key{ZREF@LABEL}{delprop}{%
  \zref@propexists{#1}{%
    \zref@localdelprop{ZREF@temp}{#1}%
  }%
}
\kv@define@key{ZREF@LABEL}{immediate}[true]{%
  \edef\ZREF@param{#1}%
  \ifx\ZREF@param\ZREF@true
    \ZREF@immediatetrue
  \else
    \ifx\ZREF@param\ZREF@false
      \ZREF@immediatefalse
    \else
      \PackageWarning\ZREF@name{%
        Option `immediate' expects `true' or `false'.\MessageBreak
        Ignoring invalid value `\ZREF@param'%
      }%
    \fi
  \fi
}
%    \end{macrocode}
%    \begin{macro}{\ZREF@false}
%    \begin{macrocode}
\def\ZREF@false{false}
%    \end{macrocode}
%    \end{macro}
%    \begin{macro}{\ZREF@true}
%    \begin{macrocode}
\def\ZREF@true{true}
%    \end{macrocode}
%    \end{macro}
%
%    \begin{macrocode}
\kv@define@key{ZREF@LABEL}{values}[]{%
  \kv@parse{#1}{%
    \ifx\kv@value\relax
      \@PackageWarning\ZREF@name{%
        Missing value for property `\kv@key'%
      }%
      \expandafter\ltx@gobbletwo
    \else
      \expandafter\zref@setcurrent
    \fi
  }%
}
%    \end{macrocode}
%
%    \begin{macro}{\ifZREF@immediate}
%    The switch \cs{ifZREF@immediate} tells us, whether the
%    label should be written immediately or at page shipout time.
%    \cs{ZREF@label} need to be notified about this, because
%    it must disable the deferred execution of property values,
%    if the label is written immediately.
%    \begin{macrocode}
\newif\ifZREF@immediate
%    \end{macrocode}
%    \end{macro}
%    \begin{macro}{\zref@wrapper@immediate}
%    The argument of \cs{zref@wrapper@immediate} is executed inside
%    a group where \cs{write} is redefined by adding
%    \cs{immediate} before its execution. Also \cs{ZREF@label}
%    is notified via the switch \cs{ifZREF@immediate}.
%    \begin{macrocode}
\ZREF@Robust{\long\def}\zref@wrapper@immediate#1{%
  \begingroup
    \ZREF@immediatetrue
    \let\ZREF@org@write\write
    \def\write{\immediate\ZREF@org@write}%
    #1%
  \endgroup
}
%    \end{macrocode}
%    \end{macro}
%
%    \begin{macro}{\ZREF@label}
%    \cs{ZREF@label} writes the data in the \xfile{.aux} file.
%    \verb|#1| contains the list of valid properties,
%    \verb|#2| the name of the reference. In case of immediate
%    writing, the deferred execution of property values is
%    disabled. Also \thepage is made expandable in this case.
%    \begin{macrocode}
\def\ZREF@label#1#2{%
  \if@filesw
    \begingroup
      \ifZREF@immediate
        \let\ZREF@org@thepage\thepage
      \fi
      \protected@write\@auxout{%
        \ifZREF@immediate
          \let\thepage\ZREF@org@thepage
        \fi
        \let\ZREF@temp\ltx@empty
        \@tfor\ZREF@P:=#1\do{%
          \begingroup
            \escapechar=-1 %
            \edef\x{\endgroup
              \def\noexpand\ZREF@P{%
                \expandafter\string\ZREF@P
              }%
            }%
          \x
          \expandafter\ifx
              \csname
                \ifZREF@immediate
                  relax%
                \else
                  Z@X@\ZREF@P%
                \fi
              \endcsname
              \noexpand
            \expandafter\let\csname Z@C@\ZREF@P\endcsname\relax
          \fi
          \toks@\expandafter{\ZREF@temp}%
          \edef\ZREF@temp{%
            \the\toks@
            \ltx@backslashchar\ZREF@P{%
              \expandafter\noexpand\csname Z@C@\ZREF@P\endcsname
            }%
          }%
        }%
      }{%
        \string\zref@newlabel{#2}{\ZREF@temp}%
      }%
    \endgroup
  \fi
}
\def\ZREF@addtoks#1{%
  \toks@\expandafter\expandafter\expandafter{%
    \expandafter\the\expandafter\toks@#1%
  }%
}
%    \end{macrocode}
%    \end{macro}
%
% \subsubsection{Reference querying and extracting}
%
%    Design goal for the extracting macros is that the extraction
%    process is full expandable. Thus these macros can be used
%    in expandable contexts. But there are problems that cannot
%    be solved by full expandable macros:
%    \begin{itemize}
%    \item In standard \hologo{LaTeX} undefined references sets a flag
%      and generate a warning. Both actions are not expandable.
%    \item Babel's support for its shorthand uses commands that
%      use non-expandable assignments. However currently there
%      is hope, that primitives are added to \pdfTeX\ that
%      allows the detection of contexts. Then the shorthand
%      can detect, if they are executed inside \cs{csname}
%      and protect themselves automatically.
%    \end{itemize}
%    \begin{macro}{\zref@ifrefundefined}
%    If a reference \verb|#1| is undefined, then macro
%    \cs{zref@ifrefundefined} calls \verb|#2| and \verb|#3| otherwise.
%    \begin{macrocode}
\def\zref@ifrefundefined#1{%
  \ltx@ifundefined{Z@R@#1}%
}
%    \end{macrocode}
%    \end{macro}
%
%    \begin{macro}{\zifrefundefined}
%    If a reference \verb|#1| is undefined, then macro
%    \cs{zref@ifrefundefined} calls \verb|#2| and \verb|#3| otherwise.
%    Also the reference is marked used.
%    \begin{macrocode}
\ZREF@IfDefinable\zifrefundefined\def{%
  #1{%
    \zref@wrapper@babel\ZREF@ifrefundefined{#1}%
  }%
}
%    \end{macrocode}
%    \end{macro}
%    \begin{macro}{\ZREF@ifrefundefined}
%    \begin{macrocode}
\def\ZREF@ifrefundefined#1{%
  \zref@refused{#1}%
  \zref@ifrefundefined{#1}%
}
%    \end{macrocode}
%    \end{macro}
%
%    \begin{macro}{\zref@refused}
%    The problem with undefined references is addressed by
%    the macro \cs{zref@refused}. This can be used outside
%    the expandable context. In case of an undefined reference
%    the flag is set to notify \hologo{LaTeX} and a warning is given.
%    \begin{macrocode}
\ZREF@Robust\def\zref@refused#1{%
  \zref@wrapper@babel\ZREF@refused{#1}%
}
%    \end{macrocode}
%    \end{macro}
%    \begin{macro}{\ZREF@refused}
%    \begin{macrocode}
\def\ZREF@refused#1{%
  \zref@ifrefundefined{#1}{%
    \protect\G@refundefinedtrue
    \@latex@warning{%
      Reference `#1' on page \thepage \space undefined%
    }%
  }{}%
}
%    \end{macrocode}
%    \end{macro}
%
%    \begin{macro}{\zref@ifrefcontainsprop}
%    \cs{zref@ifrefcontainsprop} looks, if the reference \verb|#1|
%    has the property \verb|#2| and calls then \verb|#3| and
%    \verb|#4| otherwise.
%    \begin{macrocode}
\def\zref@ifrefcontainsprop#1#2{%
  \zref@ifrefundefined{#1}{%
    \ltx@secondoftwo
  }{%
    \expandafter\ZREF@ifrefcontainsprop
    \csname Z@E@#2\expandafter\endcsname
    \csname#2\expandafter\expandafter\expandafter\endcsname
    \expandafter\expandafter\expandafter{%
      \csname Z@R@#1\endcsname
    }%
  }%
}
\def\ZREF@ifrefcontainsprop#1#2#3{%
  \expandafter\ifx\expandafter\ZREF@novalue
  #1#3#2\ZREF@novalue\ZREF@nil\ltx@empty
    \expandafter\ltx@secondoftwo
  \else
    \expandafter\ltx@firstoftwo
  \fi
}
\def\ZREF@novalue{\ZREF@NOVALUE}
%    \end{macrocode}
%    \end{macro}
%
%    \begin{macro}{\zref@extract}
%    \cs{zref@extract} is an abbreviation for the case
%    that the default of the property is used as default
%    value.
%    \begin{macrocode}
\def\ZREF@extract#1#2{%
  \romannumeral0%
  \ltx@ifundefined{Z@D@#2}{%
    \expandafter\ltx@space\zref@default
  }{%
    \expandafter\expandafter\expandafter\ZREF@@extract
    \expandafter\expandafter\expandafter{%
      \csname Z@D@#2\endcsname
    }{#1}{#2}%
  }%
}
%    \end{macrocode}
%    \end{macro}
%    \begin{macro}{\ZREF@@extract}
%    \begin{macrocode}
\def\ZREF@@extract#1#2#3{%
  \expandafter\expandafter\expandafter\ltx@space
  \zref@extractdefault{#2}{#3}{#1}%
}
%    \end{macrocode}
%    \end{macro}
%    \begin{macro}{\ZREF@wu@extract}
%    \begin{macrocode}
\def\ZREF@wu@extract#1#2{%
  \etex@unexpanded\expandafter\expandafter\expandafter{%
    \ZREF@extract{#1}{#2}%
  }%
}
%    \end{macrocode}
%    \end{macro}
%    \begin{macro}{\zref@extract}
%    \begin{macrocode}
\let\zref@extract\ZREF@extract
%    \end{macrocode}
%    \end{macro}
%
%    \begin{macro}{\ZREF@extractdefault}
%    The basic extracting macro is \cs{zref@extractdefault} with
%    the reference name in \verb|#1|, the property in \verb|#2| and
%    the default value in \verb|#3| in case for problems.
%    \begin{macrocode}
\def\ZREF@extractdefault#1#2#3{%
  \romannumeral0%
  \zref@ifrefundefined{#1}\ltx@firstoftwo{%
    \zref@ifpropundefined{#2}\ltx@firstoftwo\ltx@secondoftwo
  }{%
    \ltx@space
    #3%
  }{%
    \expandafter\expandafter\expandafter\ltx@space
    \csname Z@E@#2\expandafter\expandafter\expandafter\endcsname
    \csname Z@R@#1\expandafter\endcsname
    \csname#2\endcsname{#3}\ZREF@nil
  }%
}
%    \end{macrocode}
%    \end{macro}
%    \begin{macro}{\ZREF@wu@extractdefault}
%    \begin{macrocode}
\def\ZREF@wu@extractdefault#1#2#3{%
  \etex@unexpanded\expandafter\expandafter\expandafter{%
    \ZREF@extractdefault{#1}{#2}{#3}%
  }%
}
%    \end{macrocode}
%    \end{macro}
%    \begin{macro}{\zref@extractdefault}
%    \begin{macrocode}
\let\zref@extractdefault\ZREF@extractdefault
%    \end{macrocode}
%    \end{macro}
%
%    \begin{macro}{\zref@def@extract}
%    \begin{macrocode}
\ZREF@Robust\def\zref@def@extract#1{%
  \zref@wrapper@babel{\ZREF@def@extract{#1}}%
}
%    \end{macrocode}
%    \end{macro}
%    \begin{macro}{\ZREF@def@extract}
%    \begin{macrocode}
\def\ZREF@def@extract#1#2#3{%
  \zref@refused{#2}%
  \expandafter\expandafter\expandafter\def
  \expandafter\expandafter\expandafter#1%
  \expandafter\expandafter\expandafter{%
    \zref@extract{#2}{#3}%
  }%
}
%    \end{macrocode}
%    \end{macro}
%    \begin{macro}{\zref@def@extractdefault}
%    \begin{macrocode}
\ZREF@Robust\def\zref@def@extractdefault#1{%
  \zref@wrapper@babel{\ZREF@def@extractdefault{#1}}%
}
%    \end{macrocode}
%    \end{macro}
%    \begin{macro}{\ZREF@def@extractdefault}
%    \begin{macrocode}
\def\ZREF@def@extractdefault#1#2#3#4{%
  \zref@refused{#2}%
  \expandafter\expandafter\expandafter\def
  \expandafter\expandafter\expandafter#1%
  \expandafter\expandafter\expandafter{%
    \zref@extractdefault{#2}{#3}{#4}%
  }%
}
%    \end{macrocode}
%    \end{macro}
%
%    \begin{macro}{\ZREF@wrapper@unexpanded}
%    \begin{macrocode}
\ZREF@Robust{\long\def}\ZREF@wrapper@unexpanded#1{%
  \let\zref@wrapper@unexpanded\ltx@firstofone
  \let\zref@getcurrent\ZREF@wu@getcurrent
  \let\zref@extractdefault\ZREF@wu@extractdefault
  \let\zref@extract\ZREF@wu@extract
  #1%
  \let\zref@wrapper@unexpanded\ZREF@wrapper@unexpanded
  \let\zref@getcurrent\ZREF@getcurrent
  \let\zref@extractdefault\ZREF@extractdefault
  \let\zref@extract\ZREF@extract
}
%    \end{macrocode}
%    \end{macro}
%    \begin{macro}{\zref@wrapper@unexpanded}
%    \begin{macrocode}
\ltx@IfUndefined{etex@unexpanded}{%
  \let\zref@wrapper@unexpanded\ltx@firstofone
}{%
  \let\zref@wrapper@unexpanded\ZREF@wrapper@unexpanded
}
%    \end{macrocode}
%    \end{macro}
%
% \subsubsection{Compatibility with \xpackage{babel}}
%
%    \begin{macro}{\zref@wrapper@babel}
%    \begin{macrocode}
\ZREF@Robust{\long\def}\zref@wrapper@babel#1#2{%
  \ifcsname if@safe@actives\endcsname
    \expandafter\ltx@firstofone
  \else
    \expandafter\ltx@secondoftwo
  \fi
  {%
    \if@safe@actives
      \expandafter\ltx@secondoftwo
    \else
      \expandafter\ltx@firstoftwo
    \fi
    {%
      \begingroup
        \csname @safe@activestrue\endcsname
        \edef\x{#2}%
      \expandafter\endgroup
      \expandafter\ZREF@wrapper@babel\expandafter{\x}{#1}%
    }%
  }{%
    #1{#2}%
  }%
}
\long\def\ZREF@wrapper@babel#1#2{%
  #2{#1}%
}
%    \end{macrocode}
%    \end{macro}
%
% \subsubsection{Unique counter support}
%
%    \begin{macro}{\zref@require@unique}
%    Generate the counter |zref@unique| if the counter does not
%    already exist.
%    \begin{macrocode}
\ZREF@Robust\def\zref@require@unique{%
  \@ifundefined{c@zref@unique}{%
    \begingroup
      \let\@addtoreset\ltx@gobbletwo
      \newcounter{zref@unique}%
    \endgroup
%    \end{macrocode}
%    \end{macro}
%
%    \begin{macro}{\thezref@unique}
%    \cs{thezref@unique} is used for automatically generated unique
%    labelnames.
%    \begin{macrocode}
    \renewcommand*{\thezref@unique}{%
      zref@\number\c@zref@unique
    }%
  }{}%
}
%    \end{macrocode}
%    \end{macro}
%
% \subsubsection{Utilities}
%
%    \begin{macro}{\ZREF@number}
%    \begin{macrocode}
\ltx@IfUndefined{numexpr}{%
  \def\ZREF@number#1{\number#1}%
}{%
  \def\ZREF@number#1{\the\numexpr(#1)\relax}%
}
%    \end{macrocode}
%    \end{macro}
%
% \subsubsection{Setup}
%
%    \begin{macro}{\zref@setdefault}
%    Standard \hologo{LaTeX} prints ``??'' in bold face if a reference
%    is not known. \cs{zref@default} holds the text that is
%    printed in case of unknown references and is used, if
%    the default was not specified during the definition
%    of the new property by \cs{ref@newprop}.
%    The global default value can be set by \cs{zref@setdefault}.
%    \begin{macrocode}
\ZREF@Robust\def\zref@setdefault#1{%
  \def\zref@default{#1}%
}
%    \end{macrocode}
%    \end{macro}
%    \begin{macro}{\zref@default}
%    Now we initialize \cs{zref@default} with the same value
%    that \hologo{LaTeX} uses for its undefined references.
%    \begin{macrocode}
\zref@setdefault{%
  \nfss@text{\reset@font\bfseries ??}%
}
%    \end{macrocode}
%    \end{macro}
%
% \paragraph{Main property list.}
%
%    \begin{macro}{\zref@setmainlist}
%    The name of the default property list is stored in
%    \cs{ZREF@mainlist} and can be set by \cs{zref@setmainlist}.
%    \begin{macrocode}
\ZREF@Robust\def\zref@setmainlist#1{%
  \def\ZREF@mainlist{#1}%
}
\zref@setmainlist{main}
%    \end{macrocode}
%    \end{macro}
%    Now we create the list.
%    \begin{macrocode}
\zref@newlist\ZREF@mainlist
%    \end{macrocode}
%
% \paragraph{Main properties.}
%
%    The two properties \texttt{default} and \texttt{page}
%    are created and added to the main property list.
%    They store the data that
%    standard \hologo{LaTeX} uses in its references created by
%    \cs{label}.
%    \begin{description}
%    \item[\texttt{default}] the apperance of the latest
%      counter that is incremented by \cs{refstepcounter}
%    \item[\texttt{page}] the apperance of the page counter
%    \end{description}
%    \begin{macrocode}
\zref@newprop{default}{\@currentlabel}
\zref@newprop*{page}{\thepage}
\zref@addprops\ZREF@mainlist{default,page}
%    \end{macrocode}
%
% \paragraph{Properties}
%
%    \begin{macro}{\ZREF@NewPropAnchor}
%    \begin{macrocode}
\def\ZREF@NewPropAnchor{%
  \zref@newprop{anchor}{%
    \ltx@ifundefined{@currentHref}{}{\@currentHref}%
  }%
  \global\let\ZREF@NewPropAnchor\relax
}
%    \end{macrocode}
%    \end{macro}
%    \begin{macro}{\zref@titleref@current}
%    Later we will redefine the section and caption macros
%    to catch the current title and remember the value in
%    \cs{zref@titleref@current}.
%    \begin{macro}{\ZREF@NewPropTitle}
%    \begin{macrocode}
\def\ZREF@NewPropTitle{%
  \gdef\zref@titleref@current{}%
  \zref@newprop{title}{\zref@titleref@current}%
  \global\let\ZREF@NewPropTitle\relax
}
%    \end{macrocode}
%    \end{macro}
%    \end{macro}
%    \begin{macro}{\ZREF@NewPropTheotype}
%    \begin{macrocode}
\def\ZREF@NewPropTheotype{%
  \zref@newprop{theotype}{}%
  \global\let\ZREF@NewPropTheotype\relax
}
%    \end{macrocode}
%    \end{macro}
%    \begin{macro}{\ZREF@NewPropPageValue}
%    \begin{macrocode}
\def\ZREF@NewPropPageValue{%
  \zref@newprop*{pagevalue}[0]{\number\c@page}%
  \global\let\ZREF@NewPropPageValue\relax
}
%    \end{macrocode}
%    \end{macro}
%
% \paragraph{Mark successful loading}
%
%    \begin{macrocode}
\let\ZREF@base@ok=Y
%</base>
%    \end{macrocode}
%
%
% \subsection{Module \xmodule{user}}
%
%    \begin{macrocode}
%<*user>
\NeedsTeXFormat{LaTeX2e}
\ProvidesPackage{zref-user}%
  [2020-07-03 v2.32 Module user for zref (HO)]%
\RequirePackage{zref-base}[2019/11/29]
\ifx\ZREF@base@ok Y%
\else
  \expandafter\endinput
\fi
%    \end{macrocode}
%
%    Module \xmodule{user} enables a small user interface.
%    All macros are prefixed by \cs{z}.
%
%    First we define the pendants to the standard \hologo{LaTeX} referencing
%    commands \cs{label}, \cs{ref}, and \cs{pageref}.
%
%    \begin{macro}{\zlabel}
%    Similar to \cs{label} the macro \cs{zlabel} writes a reference
%    entry in the \xfile{.aux} file. The main property list is used.
%    Also we add the babel patch. The \cs{label} command can also be
%    used inside section titles, but it must not go into the
%    table of contents. Therefore we have to check this situation.
%    \begin{macrocode}
\newcommand*\zlabel{%
  \ifx\label\ltx@gobble
    \expandafter\ltx@gobble
  \else
    \expandafter\zref@wrapper@babel\expandafter\zref@label
  \fi
}%
%    \end{macrocode}
%    \end{macro}
%
%    \begin{macro}{\zkvlabel}
%    \begin{macrocode}
\newcommand*{\zkvlabel}[1]{%
  \ifx\label\ltx@gobble
    \expandafter\ltx@gobblethree
  \fi
  \zref@wrapper@babel{\zref@labelbykv{#1}}%
}%
%    \end{macrocode}
%    \end{macro}
%
%    \begin{macro}{\zref}
%    Macro \cs{zref} is the corresponding macro for \cs{ref}.
%    Also it provides an optional argument in order to
%    select another property.
%    \begin{macrocode}
\newcommand*{\zref}[2][default]{% robust because of optional argument
  \zref@propexists{#1}{%
    \zref@wrapper@babel\ZREF@zref{#2}{#1}%
  }%
}%
\def\ZREF@zref#1{%
  \zref@refused{#1}%
  \zref@extract{#1}%
}%
%    \end{macrocode}
%    \end{macro}
%
%    \begin{macro}{\zpageref}
%    For macro \cs{zpageref} we just call \cs{zref}
%    with property \verb|page|.
%    \begin{macrocode}
\ZREF@IfDefinable\zpageref\def{%
  {\zref[page]}%
}
%    \end{macrocode}
%    \end{macro}
%
%    \begin{macro}{\zrefused}
%    For the following expandible user macros \cs{zrefused}
%    should be used to notify \hologo{LaTeX} in case of undefined
%    references.
%    \begin{macrocode}
\ZREF@IfDefinable\zrefused\def{%
  {\zref@refused}%
}
%    \end{macrocode}
%    \end{macro}
%
%    \begin{macrocode}
%</user>
%    \end{macrocode}
%
%
% \subsection{Module \xmodule{abspage}}
%
%    \begin{macrocode}
%<*abspage>
\NeedsTeXFormat{LaTeX2e}
\ProvidesPackage{zref-abspage}%
  [2020-07-03 v2.32 Module abspage for zref (HO)]%
\RequirePackage{zref-base}[2019/11/29]
\ifx\ZREF@base@ok Y%
\else
  \expandafter\endinput
\fi
%    \end{macrocode}
%
%    Module \xmodule{abspage} adds a new property \verb|abspage|
%    to the \verb|main| property list for absolute page numbers.
%    These are recorded by the help of package \xpackage{atbegshi}.
%    \begin{macrocode}
\RequirePackage{atbegshi}[2011/10/05]%
%    \end{macrocode}
%    The counter \verb|abspage| must not go in the clear list
%    of \verb|@ckpt| that is used to set counters in
%    \xfile{.aux} files of included \hologo{TeX} files.
%    \begin{macrocode}
\begingroup
  \let\@addtoreset\ltx@gobbletwo
  \newcounter{abspage}%
\endgroup
\setcounter{abspage}{0}%
\AtBeginShipout{%
  \stepcounter{abspage}%
}%
\zref@newprop*{abspage}[0]{\the\c@abspage}%
\zref@addprop\ZREF@mainlist{abspage}%
%    \end{macrocode}
%    Note that counter \verb|abspage| shows the
%    previous page during page processing. Before
%    shipout the counter is incremented.
%    Thus the property is correctly written with
%    deferred writing. If the counter is written
%    using \cs{zref@wrapper@immediate}, then the number
%    is too small by one.
%
%    \begin{macrocode}
%</abspage>
%    \end{macrocode}
%
%
% \subsection{Module \xmodule{counter}}
%
%    \begin{macrocode}
%<*counter>
\NeedsTeXFormat{LaTeX2e}
\ProvidesPackage{zref-counter}%
  [2020-07-03 v2.32 Module counter for zref (HO)]%
\RequirePackage{zref-base}[2019/11/29]
\ifx\ZREF@base@ok Y%
\else
  \expandafter\endinput
\fi
%    \end{macrocode}
%
%    For features such as \xpackage{hyperref}'s \cs{autoref}
%    we need the name of the counter. The property
%    \verb|counter| is defined and added to the main property
%    list. Starting with \LaTeX\ 2020-10-01 the proper can use
%    \@currentcounter. In older formats
%    \cs{refstepcounter} has to be patched but this can fail in some
%    cases, see issue \#5.
%    \begin{macrocode}
\@ifl@t@r\fmtversion{2020-10-01}
 {
  \zref@newprop{counter}{\@currentcounter}
  \zref@addprop\ZREF@mainlist{counter}
 }
 {
  \zref@newprop{counter}{}
  \zref@addprop\ZREF@mainlist{counter}
  \AtBeginDocument{%
    \ZREF@patch{refstepcounter}{%
      \def\refstepcounter#1{%
        \zref@setcurrent{counter}{#1}%
        \ZREF@org@refstepcounter{#1}%
      }%
    }%
  }
 }
%    \end{macrocode}
%    \begin{macrocode}
%</counter>
%    \end{macrocode}
%
%
% \subsection{Module \xmodule{lastpage}}
%
%    \begin{macrocode}
%<*lastpage>
\NeedsTeXFormat{LaTeX2e}
\ProvidesPackage{zref-lastpage}%
  [2020-07-03 v2.32 Module lastpage for zref (HO)]%
\RequirePackage{zref-base}[2019/11/29]
\RequirePackage{zref-abspage}[2019/11/29]
\RequirePackage{atveryend}[2009/12/07]
\ifx\ZREF@base@ok Y%
\else
  \expandafter\endinput
\fi
%    \end{macrocode}
%
%    The module \xmodule{lastpage} implements the service
%    of package \xpackage{lastpage} by setting a reference
%    \verb|LastPage| at the end of the document.
%    If module \xmodule{abspage} is given, also the absolute
%    page number is available, because the properties of
%    the main property list are used.
%    \begin{macrocode}
\zref@newlist{LastPage}
\AfterLastShipout{%
  \if@filesw
    \begingroup
      \advance\c@page\m@ne
      \toks@\expandafter\expandafter\expandafter{%
        \expandafter\Z@L@main
        \Z@L@LastPage
      }%
      \expandafter\zref@wrapper@immediate\expandafter{%
        \expandafter\ZREF@label\expandafter{\the\toks@}{LastPage}%
      }%
    \endgroup
  \fi
}
%    \end{macrocode}
%    \begin{macro}{\zref@iflastpage}
%    \begin{macrocode}
\def\zref@iflastpage#1{%
  \ifnum\zref@extractdefault{#1}{abspage}{-1}=%
        \zref@extractdefault{LastPage}{abspage}{-2} %
    \expandafter\ltx@firstoftwo
  \else
    \expandafter\ltx@secondoftwo
  \fi
}
%    \end{macrocode}
%    \end{macro}
%    \begin{macro}{\ziflastpage}
%    \begin{macrocode}
\ZREF@IfDefinable\ziflastpage\def{%
  {\zref@wrapper@babel\ZREF@iflastpage}%
}
%    \end{macrocode}
%    \end{macro}
%    \begin{macro}{ZREF@iflastpage}
%    \begin{macrocode}
\def\ZREF@iflastpage#1{%
  \zref@refused{LastPage}%
  \zref@refused{#1}%
  \zref@iflastpage{#1}%
}
%    \end{macrocode}
%    \end{macro}
%    \begin{macrocode}
%</lastpage>
%    \end{macrocode}
%
%
% \subsection{Module \xmodule{thepage}}
%
%    \begin{macrocode}
%<*thepage>
\NeedsTeXFormat{LaTeX2e}
\ProvidesPackage{zref-thepage}%
  [2020-07-03 v2.32 Module thepage for zref (HO)]%
\RequirePackage{zref-base}[2019/11/29]
\ifx\ZREF@base@ok Y%
\else
  \expandafter\endinput
\fi
%    \end{macrocode}
%
%    \begin{macrocode}
\RequirePackage{atbegshi}[2011/10/05]
\RequirePackage{zref-abspage}[2019/11/29]
%    \end{macrocode}
%    \begin{macrocode}
\zref@newlist{thepage}
\zref@addprop{thepage}{page}
\ZREF@NewPropPageValue
%    \end{macrocode}
%    \begin{macro}{\zref@thepage@atbegshi@hook}
%    \begin{macrocode}
\let\zref@thepage@atbegshi@hook\ltx@empty
%    \end{macrocode}
%    \end{macro}
%    \begin{macrocode}
\zref@addprop{thepage}{pagevalue}
\AtBeginShipout{%
  \AtBeginShipoutAddToBox{%
    \zref@thepage@atbegshi@hook
    \zref@labelbylist{thepage\the\value{abspage}}{thepage}%
  }%
}
%    \end{macrocode}
%    \begin{macro}{\zref@thepage@name}
%    \begin{macrocode}
\ltx@IfUndefined{numexpr}{%
  \def\zref@thepage@name#1{thepage\number#1}%
}{%
  \def\zref@thepage@name#1{thepage\the\numexpr#1}%
}
%    \end{macrocode}
%    \end{macro}
%    \begin{macro}{\zref@thepage}
%    \begin{macrocode}
\def\zref@thepage#1{%
  \zref@extract{\zref@thepage@name{#1}}{page}%
}%
%    \end{macrocode}
%    \end{macro}
%    \begin{macro}{\zref@thepage@refused}
%    \begin{macrocode}
\ZREF@Robust\def\zref@thepage@refused#1{%
  \zref@refused{\zref@thepage@name{#1}}%
}%
%    \end{macrocode}
%    \end{macro}
%    \begin{macro}{\zthepage}
%    \begin{macrocode}
\ZREF@IfDefinable\zthepage\def{%
  #1{%
    \zref@thepage@refused{#1}%
    \zref@thepage{#1}%
  }%
}
%    \end{macrocode}
%    \end{macro}
%    \begin{macrocode}
%</thepage>
%    \end{macrocode}
%
%
% \subsection{Module \xmodule{nextpage}}
%
%    \begin{macrocode}
%<*nextpage>
\NeedsTeXFormat{LaTeX2e}
\ProvidesPackage{zref-nextpage}%
  [2020-07-03 v2.32 Module nextpage for zref (HO)]%
\RequirePackage{zref-base}[2019/11/29]
\ifx\ZREF@base@ok Y%
\else
  \expandafter\endinput
\fi
%    \end{macrocode}
%
%    \begin{macrocode}
\RequirePackage{zref-abspage}[2019/11/29]
\RequirePackage{zref-thepage}[2019/11/29]
\RequirePackage{zref-lastpage}[2019/11/29]
\RequirePackage{uniquecounter}[2009/12/18]
%    \end{macrocode}
%    \begin{macrocode}
\UniqueCounterNew{znextpage}

\newcommand*{\znextpagesetup}{%
  \afterassignment\ZREF@np@setup@i
  \def\ZREF@np@call@unknown##1%
}
\def\ZREF@np@setup@i{%
  \afterassignment\ZREF@np@setup@ii
  \def\ZREF@np@call@nonext##1%
}
\def\ZREF@np@setup@ii{%
  \def\ZREF@np@call@next##1%
}
\def\ZREF@np@call@unknown#1{#1}
\def\ZREF@np@call@nonext#1{#1}
\def\ZREF@np@call@next#1{#1}
\ZREF@IfDefinable\znextpage\def{%
  {\UniqueCounterCall{znextpage}{\ZREF@nextpage}}%
}%
\newcommand*{\znonextpagename}{}
\newcommand*{\zunknownnextpagename}{\Z@D@page}
\def\ZREF@nextpage#1{%
  \begingroup
    \def\ZREF@refname@this{zref@np#1}%
    \zref@labelbyprops\ZREF@refname@this{abspage}%
    \chardef\ZREF@call=0 % unknown
    \ZREF@ifrefundefined\ZREF@refname@this{%
    }{%
      \edef\ZREF@pagenum@this{%
        \zref@extractdefault\ZREF@refname@this{abspage}{0}%
      }%
      \edef\ZREF@refname@next{%
        \zref@thepage@name{%
          \the\numexpr\ZREF@pagenum@this+1%
        }%
      }%
      \ifnum\ZREF@pagenum@this>0 %
        \ZREF@ifrefundefined{LastPage}{%
          \zref@ifrefundefined\ZREF@refname@next{%
          }{%
            \chardef\ZREF@call=2 % next page
          }%
        }{%
          \edef\ZREF@pagenum@last{%
            \zref@extractdefault{LastPage}{abspage}{0}%
          }%
          \ifnum\ZREF@pagenum@this<\ZREF@pagenum@last\ltx@space
            \ZREF@ifrefundefined\ZREF@refname@next{%
            }{%
              \chardef\ZREF@call=2 % next page
            }%
          \else
            \ifnum\ZREF@pagenum@this=\ZREF@pagenum@this\ltx@space
              \chardef\ZREF@call=1 % no next page
            \fi
          \fi
        }%
      \fi
    }%
    \edef\x{%
      \endgroup
      \ifcase\ZREF@call
        \noexpand\ZREF@np@call@unknown{%
          \noexpand\zunknownnextpagename
        }%
      \or
        \noexpand\ZREF@np@call@nonext{%
          \noexpand\znonextpagename
        }%
      \else
        \noexpand\ZREF@np@call@next{%
          \noexpand\zref@extract{\ZREF@refname@next}{page}%
        }%
      \fi
    }%
  \x
}
%    \end{macrocode}
%    \begin{macrocode}
%</nextpage>
%    \end{macrocode}
%
%
% \subsection{Module \xmodule{totpages}}
%
%    \begin{macrocode}
%<*totpages>
\NeedsTeXFormat{LaTeX2e}
\ProvidesPackage{zref-totpages}%
  [2020-07-03 v2.32 Module totpages for zref (HO)]%
\RequirePackage{zref-base}[2019/11/29]
\ifx\ZREF@base@ok Y%
\else
  \expandafter\endinput
\fi
%    \end{macrocode}
%
%    The absolute page number of the last page is the total
%    page number.
%    \begin{macrocode}
\RequirePackage{zref-abspage}[2019/11/29]
\RequirePackage{zref-lastpage}[2019/11/29]
%    \end{macrocode}
%
%    \begin{macro}{\ztotpages}
%    Macro \cs{ztotpages} contains the number of pages. It can be
%    used inside expandable calculations. It expands to zero if
%    the reference is not yet available.
%    \begin{macrocode}
\newcommand*{\ztotpages}{%
  \zref@extractdefault{LastPage}{abspage}{0}%
}
%    \end{macrocode}
%    \end{macro}
%    Also we mark the reference |LastPage| as used:
%    \begin{macrocode}
\AtBeginDocument{%
  \zref@refused{LastPage}%
}
%    \end{macrocode}
%    \begin{macrocode}
%</totpages>
%    \end{macrocode}
%
%
% \subsection{Module \xmodule{pagelayout}}
%
%    \begin{macrocode}
%<*pagelayout>
\NeedsTeXFormat{LaTeX2e}
\ProvidesPackage{zref-pagelayout}%
  [2020-07-03 v2.32 Module pagelayout for zref (HO)]%
\RequirePackage{zref-base}[2019/11/29]
\ifx\ZREF@base@ok Y%
\else
  \expandafter\endinput
\fi
%    \end{macrocode}
%
%    \begin{macrocode}
\RequirePackage{zref-thepage}[2019/11/29]
\RequirePackage{iftex}[2019/11/07]%
\RequirePackage{atveryend}[2010/03/24]
%    \end{macrocode}
%
%
% \subsubsection{Define layout properties}
%
%    \begin{macrocode}
\def\ZREF@temp#1{%
  \begingroup
    \escapechar=-1 %
  \ltx@ifundefined{\string#1}{\endgroup}{%
    \edef\x{%
      \endgroup
      \noexpand\zref@newprop*{\string#1}%
                             [\noexpand\number\noexpand#1]% hash-ok
                             {\noexpand\number\noexpand#1}%
      \noexpand\zref@addprop{thepage}{\string#1}%
    }%
    \x
  }%
}
\ZREF@temp\mag
\ZREF@temp\paperwidth
\ZREF@temp\paperheight
\ZREF@temp\stockwidth % memoir.cls, crop.sty
\ZREF@temp\stockheight % memoir.cls, crop.sty
\ZREF@temp\mediawidth % VTeX
\ZREF@temp\mediaheight % VTeX
\ifluatex
\zref@newprop*{pdfvorigin}%
                             [\number\pdfvariable vorigin]% hash-ok
                             {\number\pdfvariable vorigin}%
\zref@addprop{thepage}{pdfvorigin}
\zref@newprop*{pdfhorigin}%
                             [\number\pdfvariable horigin]% hash-ok
                             {\number\pdfvariable horigin}%
\zref@addprop{thepage}{pdfhorigin}
\zref@newprop*{pdfpageheight}%
                             [\number\pageheight]% hash-ok
                             {\number\pageheight}%
\zref@addprop{thepage}{pdfpageheight}
\zref@newprop*{pdfpagewidth}%
                             [\number\pagewidth]% hash-ok
                             {\number\pagewidth}%
\zref@addprop{thepage}{pdfpagewidth}
\else
\ZREF@temp\pdfpagewidth
\ZREF@temp\pdfpageheight
\ZREF@temp\pdfhorigin
\ZREF@temp\pdfvorigin
\fi
\ZREF@temp\hoffset
\ZREF@temp\voffset
\ZREF@temp\topmargin
\ZREF@temp\oddsidemargin
\ZREF@temp\evensidemargin
\ZREF@temp\textwidth
\ZREF@temp\textheight
\ZREF@temp\headheight
\ZREF@temp\headsep
\ZREF@temp\footskip
\ZREF@temp\marginparwidth
\ZREF@temp\marginparsep
\ZREF@temp\columnwidth
\ZREF@temp\columnsep
\ZREF@temp\trimedge % memoir.cls
\ZREF@temp\spinemargin % memoir.cls
\ZREF@temp\foremargin % memoir.cls
\ZREF@temp\trimtop % memoir.cls
\ZREF@temp\uppermargin % memoir.cls
\ZREF@temp\headmargin % memoir.cls
%    \end{macrocode}
%
%    \begin{macrocode}
\IfFormatAtLeastTF{2020/10/01}
 {
   \zref@newprop*{outputboxwd}[0pt]{\ShipoutBoxWidth}
   \zref@newprop*{outputboxht}[0pt]{\ShipoutBoxHeight}
   \zref@newprop*{outputboxdp}[0pt]{\ShipoutBoxDepth}
 }
 {
   \zref@newprop*{outputboxwd}[0pt]{\AtBeginShipoutBoxWidth}
   \zref@newprop*{outputboxht}[0pt]{\AtBeginShipoutBoxHeight}
   \zref@newprop*{outputboxdp}[0pt]{\AtBeginShipoutBoxDepth}
 }
%    \end{macrocode}
%    \begin{macrocode}
\zref@addprops{thepage}{outputboxwd,outputboxht,outputboxdp}
%    \end{macrocode}
%
%    \begin{macro}{\ifZREF@pl@list}
%    \begin{macrocode}
\ltx@newif\ifZREF@pl@list
%    \end{macrocode}
%    \end{macro}
%    \begin{macro}{\zref@listpagelayout}
%    \begin{macrocode}
\ZREF@IfDefinable\zlistpagelayout\def{%
  {\global\ZREF@pl@listtrue}%
}
%    \end{macrocode}
%    \end{macro}
%    \begin{macro}{\ZREF@pl@AfterLastShipout}
%    \begin{macrocode}
\def\ZREF@pl@AfterLastShipout{%
  \ifZREF@pl@list
    \edef\ZREF@page@max{\the\value{abspage}}%
    \ltx@ifundefined{ZREF@org@testdef}{%
      \let\ZREF@org@testdef\@testdef
      \def\@testdef##1##2##3{%
        \ZREF@org@testdef{##1}{##2}{##3}%
        \def\ZREF@temp{##1}%
        \ifx\ZREF@temp\ZREF@RefPrefix
          \expandafter\gdef\csname##1@##2\endcsname{##3}%
        \fi
      }%
    }{}%
    \AtVeryEndDocument{\ZREF@pl@AtVeryEnd}%
  \fi
}
%    \end{macrocode}
%    \end{macro}
%    \begin{macro}{\ZREF@pl@AtVeryEnd}
%    \begin{macrocode}
\def\ZREF@pl@AtVeryEnd{%
  \begingroup
    \toks@{Page layout parameters:\MessageBreak}%
    \count@=1 %
    \ZREF@pl@ListPage
    \edef\x{\endgroup
      \noexpand\@PackageInfoNoLine{zref-pagelayout}{\the\toks@}%
    }%
  \x
}
%    \end{macrocode}
%    \end{macro}
%    \begin{macro}{\ZREF@pl@ListPage}
%    \begin{macrocode}
\def\ZREF@pl@ListPage{%
  \edef\x{%
    \toks@={%
      \the\toks@
      Page \the\count@:\noexpand\MessageBreak
      \zref@ifrefundefined{thepage\the\count@}{}{%
        \ltx@space\ltx@space mag = %
        \zref@extract{thepage\the\count@}{mag}%
        \noexpand\MessageBreak
        \ZREF@pl@ListEntry{paperwidth}%
        \ZREF@pl@ListEntry{paperheight}%
        \ZREF@pl@ListEntry{stockwidth}%
        \ZREF@pl@ListEntry{stockheight}%
        \ZREF@pl@ListEntry{mediawidth}%
        \ZREF@pl@ListEntry{mediaheight}%
        \ZREF@pl@ListEntry{pdfpagewidth}%
        \ZREF@pl@ListEntry{pdfpageheight}%
        \ZREF@pl@ListEntry{pdfhorigin}%
        \ZREF@pl@ListEntry{pdfvorigin}%
        \ZREF@pl@ListEntry{hoffset}%
        \ZREF@pl@ListEntry{voffset}%
        \ZREF@pl@ListEntry{topmargin}%
        \ZREF@pl@ListEntry{oddsidemargin}%
        \ZREF@pl@ListEntry{evensidemargin}%
        \ZREF@pl@ListEntry{textwidth}%
        \ZREF@pl@ListEntry{textheight}%
        \ZREF@pl@ListEntry{headheight}%
        \ZREF@pl@ListEntry{headsep}%
        \ZREF@pl@ListEntry{footskip}%
        \ZREF@pl@ListEntry{marginparwidth}%
        \ZREF@pl@ListEntry{marginparsep}%
        \ZREF@pl@ListEntry{columnwidth}%
        \ZREF@pl@ListEntry{columnsep}%
        \ZREF@pl@ListEntry{trimedge}%
        \ZREF@pl@ListEntry{spinemargin}%
        \ZREF@pl@ListEntry{foremargin}%
        \ZREF@pl@ListEntry{trimtop}%
        \ZREF@pl@ListEntry{uppermargin}%
        \ZREF@pl@ListEntry{headmargin}%
      }%
    }%
  }\x
  \ifnum\ZREF@page@max>\count@
    \advance\count@ by\ltx@one
  \else
    \expandafter\ltx@gobble
  \fi
  \ZREF@pl@ListPage
}
%    \end{macrocode}
%    \end{macro}
%    \begin{macro}{\ZREF@pl@ListEntry}
%    \begin{macrocode}
\def\ZREF@pl@ListEntry#1{%
  \zref@ifpropundefined{#1}{%
  }{%
    \zref@ifrefcontainsprop{thepage\the\count@}{#1}{%
      \ltx@space\ltx@space#1 = %
      \zref@extract{thepage\the\count@}{#1}sp = %
      \the\dimexpr\zref@extract{thepage\the\count@}{#1}sp\relax
      \noexpand\MessageBreak
    }{}%
  }%
}
%    \end{macrocode}
%    \end{macro}
%    \begin{macrocode}
\AfterLastShipout{%
  \ZREF@pl@AfterLastShipout
}
%    \end{macrocode}
%
%    \begin{macrocode}
%</pagelayout>
%    \end{macrocode}
%
%
% \subsection{Module \xmodule{pageattr}}
%
%    \begin{macrocode}
%<*pageattr>
\NeedsTeXFormat{LaTeX2e}
\ProvidesPackage{zref-pageattr}%
  [2020-07-03 v2.32 Module pageattr for zref (HO)]%
\RequirePackage{zref-base}[2019/11/29]
\ifx\ZREF@base@ok Y%
\else
  \expandafter\endinput
\fi
%    \end{macrocode}
%
%    \begin{macrocode}
\RequirePackage{iftex}[2019/11/07]%
%    \end{macrocode}
%
%
%    \begin{macrocode}
\let\ZREF@temp=N%
\ifluatex
\expandafter\@firstoftwo
\else
\expandafter\@secondoftwo
\fi
{%luatex
 \RequirePackage{zref-thepage}[2019/11/29]
 \RequirePackage{zref-lastpage}[2019/11/29]%
 \zref@newprop*{pdfpageattr}[]{\zref@hex{\the\pdfvariable pageattr}}%
 \zref@addprop{thepage}{pdfpageattr}%
 \zref@newprop*{pdfpagesattr}[]{\zref@hex{\the\pdfvariable pagesattr}}%
 \zref@addprop{LastPage}{pdfpagesattr}%
 \let\ZREF@temp=Y%
}
{%other
 \ltx@IfUndefined{pdfpageattr}{%
  \@PackageInfoNoLine{zref-pageattr}{%
    \string\pdfpageattr\space is not available%
  }%
  \def\zref@pdfpageattr#1{}%
  \def\zref@pdfpageattr@used#1{}%
 }{%
  \RequirePackage{zref-thepage}[2019/11/29]%
  \zref@newprop*{pdfpageattr}[]{\zref@hex{\the\pdfpageattr}}%
  \zref@addprop{thepage}{pdfpageattr}%
  \let\ZREF@temp=Y%
 }
 \ltx@IfUndefined{pdfpagesattr}{%
  \@PackageInfoNoLine{zref-pageattr}{%
    \string\pdfpagesattr\space is not available%
  }%
  \def\zref@pdfpagesattr{}%
  \def\zref@pdfpagesattr@used{}%
 }{%
  \RequirePackage{zref-lastpage}[2019/11/29]%
  \zref@newprop*{pdfpagesattr}[]{\zref@hex{\the\pdfpagesattr}}%
  \zref@addprop{LastPage}{pdfpagesattr}%
  \let\ZREF@temp=Y%
 }%
}%
%    \end{macrocode}
%    \begin{macrocode}
\ifx\ZREF@temp N%
  \expandafter\endinput
\fi
%    \end{macrocode}
%    \begin{macrocode}
\RequirePackage{zref-abspage}[2019/11/29]
\RequirePackage{atveryend}[2010/03/24]
\RequirePackage{pdftexcmds}[2010/04/01]
\let\ZREF@temp=Y%
\ltx@IfUndefined{pdf@escapehex}{\let\ZREF@temp=N}{}
\ltx@IfUndefined{pdf@unescapehex}{\let\ZREF@temp=N}{}
\ifx\ZREF@temp N%
  \let\zref@hex\ltx@firstofone
  \let\zref@unhex\ltx@firstofone
\else
  \let\zref@hex\pdf@escapehex
  \let\zref@unhex\pdf@unescapehex
\fi
%    \end{macrocode}
%
%    \begin{macro}{\ifZREF@pa@list}
%    \begin{macrocode}
\ltx@newif\ifZREF@pa@list
%    \end{macrocode}
%    \end{macro}
%    \begin{macro}{\zref@listpageattr}
%    \begin{macrocode}
\ZREF@IfDefinable\zlistpageattr\def{%
  {\ZREF@pa@listtrue}%
}
%    \end{macrocode}
%    \end{macro}
%    \begin{macro}{\ZREF@pa@AfterLastShipout}
%    \begin{macrocode}
\def\ZREF@pa@AfterLastShipout{%
  \ifZREF@pa@list
    \edef\ZREF@page@max{\the\value{abspage}}%
    \ltx@ifundefined{ZREF@org@testdef}{%
      \let\ZREF@org@testdef\@testdef
      \def\@testdef##1##2##3{%
        \ZREF@org@testdef{##1}{##2}{##3}%
        \def\ZREF@temp{##1}%
        \ifx\ZREF@temp\ZREF@RefPrefix
          \expandafter\xdef\csname##1@##2\endcsname{##3}%
        \fi
      }%
    }{}%
    \AtVeryEndDocument{\ZREF@pa@AtVeryEnd}%
  \fi
}
%    \end{macrocode}
%    \end{macro}
%    \begin{macro}{\ZREF@pa@AtVeryEnd}
%    \begin{macrocode}
\let\ZREF@temp=Y%
\ltx@IfUndefined{pdfpageattr}{}{\let\ZREF@temp=N}
\ifluatex \let\ZREF@temp=N \fi
\ifx\ZREF@temp Y
 \expandafter\@firstoftwo
\else
 \expandafter\@secondoftwo
\fi
 {%
  \def\ZREF@pa@AtVeryEnd{}%
 }
 {%
  \def\ZREF@pa@AtVeryEnd{%
    \begingroup
      \toks@{List of \ltx@backslashchar
             \ifluatex pdfvariable\else pdf\fi
             pdfpageattr:\MessageBreak}%
      \count@=1 %
      \ZREF@pa@ListPage
      \edef\x{\endgroup
        \noexpand\@PackageInfoNoLine{zref-pageattr}{%
          \the\toks@
        }%
      }%
    \x
  }%
%    \end{macrocode}
%    \end{macro}
%    \begin{macro}{\zref@pageattr}
%    \begin{macrocode}
\def\zref@pdfpageattr#1{%
  \zref@unhex{%
    \zref@extract{thepage\ZREF@number{#1}}{pdfpageattr}%
  }%
}
% compability, \zref@pageattr was defined in older versions
\let\zref@pageattr\zref@pdfpageattr
%    \end{macrocode}
%    \end{macro}
%    \begin{macro}{\zref@pageattr@used}
%    \begin{macrocode}
\ZREF@Robust\def\zref@pageattr@used#1{%
  \zref@refused{thepage\ZREF@number{#1}}%
}
%    \end{macrocode}
%    \end{macro}
%    \begin{macro}{\ZREF@pa@ListPage}
%    \begin{macrocode}
  \def\ZREF@pa@ListPage{%
    \edef\x{%
      \toks@={%
        \the\toks@
        Page \the\count@:%
        \noexpand\MessageBreak
        \zref@ifrefundefined{thepage\the\count@}{}{%
          <<\zref@pdfpageattr\count@>>%
          \noexpand\MessageBreak
        }%
      }%
    }\x
    \ifnum\ZREF@page@max>\count@
      \advance\count@ by\ltx@one
    \else
      \expandafter\ltx@gobble
    \fi
    \ZREF@pa@ListPage
  }%
}
%    \end{macrocode}
%    \end{macro}
%    \begin{macrocode}
\let\ZREF@temp=Y%
\ltx@IfUndefined{pdfpagesattr}{}{\let\ZREF@temp=N}
\ifluatex \let\ZREF@temp=N \fi
\ifx\ZREF@temp N
 \expandafter\@firstofone
\fi
 {%
%    \end{macrocode}
%    \begin{macro}{\zref@pdfpagesattr}
%    \begin{macrocode}
  \def\zref@pdfpagesattr{%
    \zref@unhex{%
      \zref@extract{LastPage}{pdfpagesattr}%
    }%
  }%
%    \end{macrocode}
%    \end{macro}
%    \begin{macro}{\zref@pdfpagesattr@used}
%    \begin{macrocode}
  \ZREF@Robust\def\zref@pdfpagesattr@used{%
    \zref@refused{LastPage}%
  }%
%    \end{macrocode}
%    \end{macro}
%    \begin{macrocode}
  \ltx@LocalAppendToMacro\ZREF@pa@AtVeryEnd{%
    \@PackageInfoNoLine{zref-pageattr}{%
      \ltx@backslashchar
      \ifluatex pdfvariable\else pdf\fi
      pagesattr:\MessageBreak
      <<\zref@pdfpagesattr>>%
      \MessageBreak
    }%
  }%
}
\AfterLastShipout{%
  \ZREF@pa@AfterLastShipout
}
%    \end{macrocode}
%    \begin{macrocode}
%</pageattr>
%    \end{macrocode}
%
%
% \subsection{Module \xmodule{marks}}
%
%    \begin{macrocode}
%<*marks>
\NeedsTeXFormat{LaTeX2e}
\ProvidesPackage{zref-marks}%
  [2020-07-03 v2.32 Module marks for zref (HO)]%
\RequirePackage{zref-base}[2019/11/29]
\ifx\ZREF@base@ok Y%
\else
  \expandafter\endinput
\fi
%    \end{macrocode}
%
%    \begin{macrocode}
\newcommand*{\zref@marks@register}[3][]{%
  \edef\ZREF@TempName{#1}%
  \edef\ZREF@TempNum{\ZREF@number{#2}}%
  \ifnum\ZREF@TempNum<\ltx@zero %
    \PackageError\ZREF@name{%
      \string\zref@marks@register\ltx@space is called with invalid%
      \MessageBreak
      marks register number (\ZREF@TempNum)%
    }{%
      Use `0' or the command, defined by \string\newmarks.\MessageBreak
      \@ehc
    }%
  \else
    \ifx\ZREF@TempName\ltx@empty
      \edef\ZREF@TempName{mark\romannumeral\ZREF@TempNum}%
    \else
      \edef\ZREF@TempName{marks\ZREF@TempName}%
    \fi
    \ZREF@MARKS@DefineProp{top}%
    \ZREF@MARKS@DefineProp{first}%
    \ZREF@MARKS@DefineProp{bot}%
    \kv@parse{#3}{%
      \ifx\kv@value\relax
        \def\kv@value{top,first,bot}%
      \fi
      \edef\ZREF@temp{\expandafter\ltx@car\kv@key X\@nil}%
      \ifx\ZREF@temp\ZREF@STAR
        \edef\kv@key{\expandafter\ltx@cdr\kv@key\@nil}%
        \zref@newlist\kv@key
      \fi
      \expandafter\comma@parse\expandafter{\kv@value}{%
        \ifcase0\ifx\comma@entry\ZREF@NAME@top 1\else
                \ifx\comma@entry\ZREF@NAME@first 1\else
                \ifx\comma@entry\ZREF@NAME@bot 1\fi\fi\fi\ltx@space
          \PackageWarning{zref-marks}{%
            Use `top', `first' or `bot' for the list values%
            \MessageBreak
            in the third argument of \string\zref@marks@register.%
            \MessageBreak
            Ignoring unkown value `\comma@entry'%
          }%
        \else
          \zref@addprop{\kv@key}{\comma@entry\ZREF@TempName}%
        \fi
        \ltx@gobble
      }%
      \ltx@gobbletwo
    }%
  \fi
}
\def\ZREF@STAR{*}
\def\ZREF@NAME@top{top}
\def\ZREF@NAME@first{first}
\def\ZREF@NAME@bot{bot}
\def\ZREF@MARKS@DefineProp#1{%
  \zref@ifpropundefined{#1\ZREF@TempName}{%
    \ifnum\ZREF@TempNum=\ltx@zero
      \begingroup
        \edef\x{\endgroup
          \noexpand\zref@newprop*{#1\ZREF@TempName}[]{%
            \expandafter\noexpand\csname#1mark\endcsname
          }%
        }%
      \x
    \else
      \begingroup
        \edef\x{\endgroup
          \noexpand\zref@newprop*{#1\ZREF@TempName}[]{%
            \expandafter\noexpand\csname#1marks\endcsname
            \ZREF@TempNum
          }%
        }%
      \x
    \fi
  }{%
    \PackageWarning{zref-marks}{%
      \string\zref@marks@register\ltx@space does not generate the%
      \MessageBreak
      new property `#1\ZREF@TempName', because\MessageBreak
      it is already defined%
    }%
  }%
}
%    \end{macrocode}
%    \begin{macrocode}
%</marks>
%    \end{macrocode}
%
% \subsection{Module \xmodule{runs}}
%
%    This module does not use the label-reference-system.
%    The reference changes with each \hologo{LaTeX}~run and would
%    force a rerun warning always.
%    \begin{macrocode}
%<*runs>
\NeedsTeXFormat{LaTeX2e}
\ProvidesPackage{zref-runs}%
  [2020-07-03 v2.32 Module runs for zref (HO)]%
%    \end{macrocode}
%
%    \begin{macro}{\zruns}
%    \begin{macrocode}
\providecommand*{\zruns}{0}%
\AtBeginDocument{%
  \edef\zruns{\number\numexpr\zruns+1}%
  \begingroup
    \def\on@line{}%
    \PackageInfo{zref-runs}{LaTeX runs: \zruns}%
    \if@filesw
      \immediate\write\@mainaux{%
        \string\gdef\string\zruns{\zruns}%
      }%
    \fi
  \endgroup
}
%    \end{macrocode}
%    \end{macro}
%
%    \begin{macrocode}
%</runs>
%    \end{macrocode}
%
%
% \subsection{Module \xmodule{perpage}}
%
%    \begin{macrocode}
%<*perpage>
\NeedsTeXFormat{LaTeX2e}
\ProvidesPackage{zref-perpage}%
  [2020-07-03 v2.32 Module perpage for zref (HO)]%
\RequirePackage{zref-base}[2019/11/29]
\ifx\ZREF@base@ok Y%
\else
  \expandafter\endinput
\fi
%    \end{macrocode}
%
%    This module resets a counter at page boundaries.
%    Because of the asynchronous output routine
%    page counter properties cannot be asked directly,
%    references are necessary.
%
%    For detecting changed pages module \xmodule{abspage}
%    is loaded.
%    \begin{macrocode}
\RequirePackage{zref-abspage}[2019/11/29]
%    \end{macrocode}
%
%    We group the properties for the needed references
%    in the property list |perpage|. The property |pagevalue|
%    records the correct value of the page counter.
%    \begin{macrocode}
\ZREF@NewPropPageValue
\zref@newlist{perpage}
\zref@addprops{perpage}{abspage,page,pagevalue}
%    \end{macrocode}
%    The page value, known by the reference mechanism, will be
%    stored in counter |zpage|.
%    \begin{macrocode}
\newcounter{zpage}
%    \end{macrocode}
%    Counter |zref@unique| helps in generating unique reference
%    names.
%    \begin{macrocode}
\zref@require@unique
%    \end{macrocode}
%    In order to be able to reset the counter, we hook here
%    into \cs{stepcounter}. In fact two nested hooks are used
%    to allow other packages to use the first hook at the beginning
%    of \cs{stepcounter}.
%    \begin{macrocode}
\let\ZREF@org@stepcounter\stepcounter
\def\stepcounter#1{%
  \ifcsname @stepcounterhook@#1\endcsname
    \csname @stepcounterhook@#1\endcsname
  \fi
  \ZREF@org@stepcounter{#1}%
}
%    \end{macrocode}
%    \cs{@stpelt} must be adapted due to the change in latex
%    2015-01, see https://github.com/ho-tex/zref/issues/26
%    \begin{macrocode}
\let\ZREF@org@@stpelt\@stpelt
\def\@stpelt#1{%
  \ifcsname ZREF@perpage@#1\endcsname
    \begingroup
      \let\stepcounter\ZREF@org@stepcounter
      \ZREF@org@@stpelt{#1}%
    \endgroup
    \expandafter\ltx@gobbletwo
  \fi
  \ZREF@org@@stpelt{#1}%
}
%    \end{macrocode}
%
%    \begin{macro}{\zmakeperpage}
%    Makro \cs{zmakeperpage} resets a counter at each page
%    break. It uses the same syntax and semantics as
%    \cs{MakePerPage} from package \xpackage{perpage} \cite{perpage}.
%    The initial start value can be given by the optional
%    argument. Default is one that means after the first
%    \cs{stepcounter} on a new page the counter starts with
%    one.
%    \begin{macrocode}
\ZREF@IfDefinable\zmakeperpage\def{%
  {%
    \@ifnextchar[\ZREF@makeperpage@opt{\ZREF@@makeperpage[\ltx@zero]}%
  }%
}
%    \end{macrocode}
%    \end{macro}
%    We hook before the counter is incremented in \cs{stepcounter},
%    package \xpackage{perpage} afterwards. Thus a little
%    calculation is necessary.
%    \begin{macrocode}
\def\ZREF@makeperpage@opt[#1]{%
  \begingroup
    \edef\x{\endgroup
      \noexpand\ZREF@@makeperpage[\number\numexpr#1-1\relax]%
    }%
  \x
}
%    \end{macrocode}
%    \begin{macrocode}
\def\ZREF@@makeperpage[#1]#2{%
  \@ifundefined{@stepcounterhook@#2}{%
    \expandafter\gdef\csname @stepcounterhook@#2\endcsname{}%
  }{}%
  \expandafter\gdef\csname ZREF@perpage@#2\endcsname{%
    \ZREF@@perpage@step{#2}{#1}%
  }%
  \expandafter\g@addto@macro\csname @stepcounterhook@#2\endcsname{%
    \ifcsname ZREF@perpage@#2\endcsname
      \csname ZREF@perpage@#2\endcsname
    \fi
  }%
}
%    \end{macrocode}
%    \begin{macro}{\ZREF@@perpage@step}
%    The heart of this module follows.
%    \begin{macrocode}
\def\ZREF@@perpage@step#1#2{%
%    \end{macrocode}
%    First the reference is generated.
%    \begin{macrocode}
  \global\advance\c@zref@unique\ltx@one
  \begingroup
    \expandafter
    \zref@labelbylist\expandafter{\thezref@unique}{perpage}%
%    \end{macrocode}
%    The \cs{expandafter} commands are necessary, because \cs{ZREF@temp}
%    is also used inside of \cs{zref@labelbylist}.
%
%    The evaluation of the reference follows. If the reference
%    is not yet kwown, we use the page counter as approximation.
%    \begin{macrocode}
    \zref@ifrefundefined\thezref@unique{%
      \global\c@zpage=\c@page
      \global\let\thezpage\thepage
      \expandafter\xdef\csname ZREF@abspage@#1\endcsname{%
        \number\c@abspage
      }%
    }{%
%    \end{macrocode}
%    The reference is used to set \cs{thezpage} and counter |zpage|.
%    \begin{macrocode}
      \global\c@zpage=\zref@extract\thezref@unique{pagevalue}\relax
      \xdef\thezpage{\noexpand\zref@extract{\thezref@unique}{page}}%
      \expandafter\xdef\csname ZREF@abspage@#1\endcsname{%
        \zref@extractdefault\thezref@unique
            {abspage}{\number\c@abspage}%
      }%
    }%
%    \end{macrocode}
%    Page changes are detected by a changed absolute page number.
%    \begin{macrocode}
    \expandafter\ifx\csname ZREF@abspage@#1\expandafter\endcsname
                    \csname ZREF@currentabspage@#1\endcsname
    \else
      \global\csname c@#1\endcsname=#2\relax
      \global\expandafter\let
          \csname ZREF@currentabspage@#1\expandafter\endcsname
          \csname ZREF@abspage@#1\endcsname
    \fi
  \endgroup
}
%    \end{macrocode}
%    \end{macro}
%    \begin{macro}{\zunmakeperpage}
%    Macro \cs{zunmakeperpage} cancels the effect of \cs{zmakeperpage}.
%    \begin{macrocode}
\ZREF@IfDefinable\zunmakeperpage\def{%
  #1{%
    \global\expandafter
    \let\csname ZREF@perpage@#1\endcsname\@undefined
  }%
}
%    \end{macrocode}
%    \end{macro}
%
%    \begin{macrocode}
%</perpage>
%    \end{macrocode}
%
%
% \subsection{Module \xmodule{titleref}}
%
%    \begin{macrocode}
%<*titleref>
\NeedsTeXFormat{LaTeX2e}
\ProvidesPackage{zref-titleref}%
  [2020-07-03 v2.32 Module titleref for zref (HO)]%
\RequirePackage{zref-base}[2019/11/29]
\ifx\ZREF@base@ok Y%
\else
  \expandafter\endinput
\fi
\RequirePackage{gettitlestring}[2009/12/08]
%    \end{macrocode}
%
% \subsubsection{Implementation}
%    \begin{macrocode}
\RequirePackage{keyval}
%    \end{macrocode}
%
%    This module makes section and caption titles available
%    for the reference system. It uses some of the ideas
%    of package \xpackage{nameref} and \xpackage{titleref}.
%
%    Now we can add the property \verb|title| is added to
%    the main property list.
%    \begin{macrocode}
\ZREF@NewPropTitle
\zref@addprop\ZREF@mainlist{title}%
%    \end{macrocode}
%
%    The title strings go into the \xfile{.aux} file, thus
%    they need some kind of protection.
%    Package \xpackage{titleref} uses a protected expansion
%    method. The advantage is that this can be used to
%    cleanup the string and to remove \cs{label}, \cs{index}
%    and other macros unwanted for referencing.
%    But there is the risk that fragile stuff can break.
%
%    Therefore package \xpackage{nameref} does not expand
%    the string. Thus the entries can safely be written to
%    the \xfile{.aux} file. But potentially dangerous
%    macros such as \cs{label} remain in the string and
%    can cause problems when using the string in references.
%
%    \begin{macro}{\ifzref@titleref@expand}
%    The switch \cs{ifzref@titleref@expand} distinguishes
%    between the both methods. Package \xpackage{nameref}'s
%    behaviour is achieved by setting the switch to false,
%    otherwise \xpackage{titleref}'s expansion is used.
%    Default is false.
%    \begin{macrocode}
\newif\ifzref@titleref@expand
%    \end{macrocode}
%    \end{macro}
%
%    \begin{macro}{\ZREF@titleref@hook}
%    The hook \cs{ZREF@titleref@hook} allows to extend
%    the cleanup for the expansion method. Thus
%    unnecessary macros can be removed or dangerous commands
%    removed. The hook is executed before the expansion of
%    \cs{zref@titleref@current}.
%    \begin{macrocode}
\let\ZREF@titleref@hook\ltx@empty
%    \end{macrocode}
%    \end{macro}
%    \begin{macro}{\zref@titleref@cleanup}
%    The hook should not be used directly, instead we provide the
%    macro \cs{zref@titleref@cleanup} to add stuff to the hook
%    and prevents that a previous non-empty content is not discarded
%    accidently.
%    \begin{macrocode}
\ZREF@Robust\def\zref@titleref@cleanup#1{%
   \begingroup
    \toks@\expandafter{%
      \ZREF@titleref@hook
      #1%
    }%
  \expandafter\endgroup
  \expandafter\def\expandafter\ZREF@titleref@hook\expandafter{%
    \the\toks@
  }%
}%
%    \end{macrocode}
%    \end{macro}
%
%    \begin{macro}{\ifzref@titleref@stripperiod}
%    Sometimes a title contains a period at the end.
%    Package \xpackage{nameref} removes this. This behaviour
%    is controlled by the switch \cs{ifzref@titleref@stripperiod}
%    and works regardless of the setting of option \xoption{expand}.
%    Period stripping is the default.
%    \begin{macrocode}
\newif\ifzref@titleref@stripperiod
\zref@titleref@stripperiodtrue
%    \end{macrocode}
%    \end{macro}
%
%    \begin{macro}{\zref@titleref@setcurrent}
%    Macro \cs{zref@titleref@setcurrent} sets a new current title
%    stored in \cs{zref@titleref@current}. Some cleanup and
%    expansion is performed that can be controlled by the
%    previous switches.
%    \begin{macrocode}
\ZREF@Robust\def\zref@titleref@setcurrent#1{%
  \ifzref@titleref@expand
    \GetTitleStringExpand{#1}%
  \else
    \GetTitleStringNonExpand{#1}%
  \fi
  \edef\zref@titleref@current{%
    \detokenize\expandafter{\GetTitleStringResult}%
  }%
  \ifzref@titleref@stripperiod
    \edef\zref@titleref@current{%
      \expandafter\ZREF@stripperiod\zref@titleref@current
      \ltx@empty.\ltx@empty\@nil
    }%
  \fi
}%
\GetTitleStringDisableCommands{%
  \ZREF@titleref@hook
}
%    \end{macrocode}
%    \end{macro}
%    \begin{macro}{\ZREF@stripperiod}
%    If \cs{ZREF@stripperiod} is called, the argument consists of
%    space tokens and tokens with catcode 12 (other), because
%    of \hologo{eTeX}'s \cs{detokenize}.
%    \begin{macrocode}
\def\ZREF@stripperiod#1.\ltx@empty#2\@nil{#1}%
%    \end{macrocode}
%    \end{macro}
%
% \subsubsection{User interface}
%
%    \begin{macro}{\ztitlerefsetup}
%    The behaviour of module \xmodule{titleref} is controlled
%    by switches and a hook. They can be set by \cs{ztitlerefsetup}
%    with a key value interface, provided by package \xpackage{keyval}.
%    Also the current title can be given explicitly by the key
%    \verb|title|.
%    \begin{macrocode}
\define@key{ZREF@TR}{expand}[true]{%
  \csname zref@titleref@expand#1\endcsname
}%
\define@key{ZREF@TR}{stripperiod}[true]{%
  \csname zref@titleref@stripperiod#1\endcsname
}%
\define@key{ZREF@TR}{cleanup}{%
  \zref@titleref@cleanup{#1}%
}%
\define@key{ZREF@TR}{title}{%
  \def\zref@titleref@current{#1}%
}%
\ZREF@IfDefinable\ztitlerefsetup\def{%
  {\kvsetkeys{ZREF@TR}}%
}%
%    \end{macrocode}
%    \end{macro}
%
%    \begin{macro}{\ztitleref}
%    The user command \cs{ztitleref} references the title.
%    For safety \cs{label} is disabled to prevent multiply
%    defined references.
%    \begin{macrocode}
\ZREF@IfDefinable\ztitleref\def{%
  {\zref@wrapper@babel\ZREF@titleref}%
}%
\def\ZREF@titleref#1{%
  \begingroup
    \zref@refused{#1}%
    \let\label\ltx@gobble
    \zref@extract{#1}{title}%
  \endgroup
}%
%    \end{macrocode}
%    \end{macro}
%
% \subsubsection{Patches for section and caption commands}
%
%    The section and caption macros are patched to extract the
%    title data.
%
%    Captions of figures and tables.
%    \begin{macrocode}
\AtBeginDocument{%
  \ZREF@patch{@caption}{%
    \long\def\@caption#1[#2]{%
      \zref@titleref@setcurrent{#2}%
      \ZREF@org@@caption{#1}[{#2}]%
    }%
  }%
%    \end{macrocode}
%
%    Section commands without star. The title version for the
%    table of contents is used because it is usually shorter and
%    more robust.
%    \begin{macrocode}
  \ZREF@patch{@part}{%
    \def\@part[#1]{%
      \zref@titleref@setcurrent{#1}%
      \ZREF@org@@part[{#1}]%
    }%
  }%
  \ZREF@patch{@chapter}{%
    \def\@chapter[#1]{%
      \zref@titleref@setcurrent{#1}%
      \ZREF@org@@chapter[{#1}]%
    }%
  }%
  \ZREF@patch{@sect}{%
    \def\@sect#1#2#3#4#5#6[#7]{%
      \zref@titleref@setcurrent{#7}%
      \ZREF@org@@sect{#1}{#2}{#3}{#4}{#5}{#6}[{#7}]%
    }%
  }%
%    \end{macrocode}
%    The star versions of the section commands.
%    \begin{macrocode}
  \ZREF@patch{@spart}{%
    \def\@spart#1{%
      \zref@titleref@setcurrent{#1}%
      \ZREF@org@@spart{#1}%
    }%
  }%
  \ZREF@patch{@schapter}{%
    \def\@schapter#1{%
      \zref@titleref@setcurrent{#1}%
      \ZREF@org@@schapter{#1}%
    }%
  }%
  \ZREF@patch{@ssect}{%
    \def\@ssect#1#2#3#4#5{%
      \zref@titleref@setcurrent{#5}%
      \ZREF@org@@ssect{#1}{#2}{#3}{#4}{#5}%
    }%
  }%
%    \end{macrocode}
%
% \subsubsection{Environment \textsf{description}}
%
%    \begin{macrocode}
  \ZREF@patch{descriptionlabel}{%
    \def\descriptionlabel#1{%
      \zref@titleref@setcurrent{#1}%
      \ZREF@org@descriptionlabel{#1}%
    }%
  }%
%    \end{macrocode}
%
% \subsubsection{Class \xclass{memoir}}
%
%    \begin{macrocode}
  \@ifclassloaded{memoir}{%
    \ltx@IfUndefined{ifheadnameref}{}{%
      \def\@chapter[#1]#2{%
        \ltx@IfUndefined{ch@pt@c}{%
          \zref@titleref@setcurrent{#1}%
        }{%
          \ifx\ch@pt@c\ltx@empty
            \zref@titleref@setcurrent{#2}%
          \else
            \def\NR@temp{#1}%
            \ifx\NR@temp\ltx@empty
              \expandafter\zref@titleref@setcurrent
              \expandafter{\ch@pt@c}%
            \else
              \ifheadnameref
                \zref@titleref@setcurrent{#1}%
              \else
                \expandafter\zref@titleref@setcurrent
                \expandafter{\ch@pt@c}%
              \fi
            \fi
          \fi
        }%
        \ZREF@org@@chapter[{#1}]{#2}%
      }%
      \ZREF@patch{M@sect}{%
        \def\M@sect#1#2#3#4#5#6[#7][#8]{%
          \ifheadnameref
            \zref@titleref@setcurrent{#8}%
          \else
            \zref@titleref@setcurrent{#7}%
          \fi
          \ZREF@org@M@sect{#1}{#2}{#3}{#4}{#5}{#6}[{#7}][{#8}]%
        }%
      }%
    }%
  }{}%
%    \end{macrocode}
%
% \subsubsection{Class \xclass{beamer}}
%
%    \begin{macrocode}
  \@ifclassloaded{beamer}{%
    \ZREF@patch{beamer@section}{%
      \long\def\beamer@section[#1]{%
        \zref@titleref@setcurrent{#1}%
        \ZREF@org@beamer@section[{#1}]%
      }%
    }%
    \ZREF@patch{beamer@subsection}{%
      \long\def\beamer@subsection[#1]{%
        \zref@titleref@setcurrent{#1}%
        \ZREF@org@beamer@subsection[{#1}]%
      }%
    }%
    \ZREF@patch{beamer@subsubsection}{%
      \long\def\beamer@subsubsection[#1]{%
        \zref@titleref@setcurrent{#1}%
        \ZREF@org@beamer@subsubsection[{#1}]%
      }%
    }%
  }{}%
%    \end{macrocode}
%
% \subsubsection{Package \xpackage{titlesec}}
%
%    \begin{macrocode}
  \@ifpackageloaded{titlesec}{%
    \ZREF@patch{ttl@sect@i}{%
      \def\ttl@sect@i#1#2[#3]#4{%
        \zref@titleref@setcurrent{#4}%
        \ZREF@org@ttl@sect@i{#1}{#2}[{#3}]{#4}%
      }%
    }%
    \ZREF@patch{ttl@straight@i}{%
      \def\ttl@straight@i#1[#2]#3{%
        \def\ZREF@temp{#2}%
        \ifx\ZREF@temp\ltx@empty
          \zref@titleref@setcurrent{#3}%
        \else
          \zref@titleref@setcurrent{#2}%
        \fi
        \ZREF@org@ttl@straight@i{#1}[{#2}]{#3}%
      }%
    }%
  }{}%
%    \end{macrocode}
%
% \subsubsection{Package \xpackage{longtable}}
%
%    Package \xpackage{longtable}: some support for its \cs{caption}.
%    However \cs{label} inside the caption is not supported.
%    \begin{macrocode}
  \@ifpackageloaded{longtable}{%
    \ZREF@patch{LT@c@ption}{%
      \def\LT@c@ption#1[#2]#3{%
        \ZREF@org@LT@c@ption{#1}[{#2}]{#3}%
        \zref@titleref@setcurrent{#2}%
      }%
    }%
  }{}%
%    \end{macrocode}
%
% \subsubsection{Package \xpackage{listings}}
%
%    Package \xpackage{listings}: support for its caption.
%    \begin{macrocode}
  \@ifpackageloaded{listings}{%
    \ZREF@patch{lst@MakeCaption}{%
      \def\lst@MakeCaption{%
        \ifx\lst@label\ltx@empty
        \else
          \expandafter\zref@titleref@setcurrent\expandafter{%
            \lst@@caption
          }%
        \fi
        \ZREF@org@lst@MakeCaption
      }%
    }%
  }{}%
%    \end{macrocode}
%
% \subsubsection{Theorems}
%
%    \begin{macrocode}
  \ZREF@patch{@opargbegintheorem}{%
    \def\@opargbegintheorem#1#2#3{%
      \zref@titleref@setcurrent{#3}%
      \ZREF@org@@opargbegintheorem{#1}{#2}{#3}%
    }%
  }%
  \@ifpackageloaded{amsthm}{%
    \begingroup
      \edef\x{macro:\string#1\string#2[\string#3]}%
      \@onelevel@sanitize\x
      \def\y#1->#2\@nil{#1}%
      \edef\z{\expandafter\y\meaning\@begintheorem->\@nil}%
      \@onelevel@sanitize\z
    \expandafter\endgroup
    \ifx\x\z
      \ZREF@patch{@begintheorem}{%
        \def\@begintheorem#1#2[#3]{%
          \zref@titleref@setcurrent{#3}%
          \ZREF@org@@begintheorem{#1}{#2}[{#3}]%
        }%
      }%
    \fi
  }{}%
%    \end{macrocode}
%
%    \begin{macrocode}
}
%    \end{macrocode}
%
%    \begin{macrocode}
%</titleref>
%    \end{macrocode}
%
%
% \subsection{Module \xmodule{xr}}
%
%    \begin{macrocode}
%<*xr>
\NeedsTeXFormat{LaTeX2e}
\ProvidesPackage{zref-xr}%
  [2020-07-03 v2.32 Module xr for zref (HO)]%
\RequirePackage{zref-base}[2019/11/29]
\ifx\ZREF@base@ok Y%
\else
  \expandafter\endinput
\fi
%    \end{macrocode}
%    \begin{macrocode}
\RequirePackage{keyval}
\RequirePackage{kvoptions}[2010/02/22]
%    \end{macrocode}
%
%    We declare property |url|, because this is added,
%    if a reference is imported and has not already set this field.
%    Or if |hyperref| is used, then this property can be asked.
%    \begin{macrocode}
\zref@newprop{url}{}%
\zref@newprop{urluse}{}%
%    \end{macrocode}
%    \begin{macrocode}
\zref@newprop{externaldocument}{}%
%    \end{macrocode}
%
%    Most code, especially the handling of the \xfile{.aux}
%    files are taken from David Carlisle's \xpackage{xr} package.
%    Therefore I drop the documentation for these macros here.
%
%    \begin{macro}{\zref@xr@ext}
%    If the URL is not specied, then assume
%    processed file with a guessed extension.
%    Use the setting of \xpackage{hyperref} if available.
%    \begin{macrocode}
\providecommand*{\zref@xr@ext}{%
  \ltx@ifundefined{XR@ext}{pdf}{\XR@ext}%
}%
%    \end{macrocode}
%    \end{macro}
%    \begin{macro}{\ifZREF@xr@zreflabel}
%    The use of the star form of \cs{zexternaldocument}
%    is remembered in the switch \cs{ifZREF@xr@zreflabel}.
%    \begin{macrocode}
\newif\ifZREF@xr@zreflabel
%    \end{macrocode}
%    \end{macro}
%    \begin{macrocode}
\SetupKeyvalOptions{%
  family=ZREF@XR,%
  prefix=ZREF@xr@%
}
\DeclareBoolOption[true]{tozreflabel}
\DeclareBoolOption[false]{toltxlabel}
\DeclareBoolOption{verbose}
\define@key{ZREF@XR}{ext}{%
  \def\zref@xr@{#1}%
}
\DeclareBoolOption[false]{urluse}
%    \end{macrocode}
%    \begin{macro}{\zxrsetup}
%    \begin{macrocode}
\newcommand*{\zxrsetup}{%
  \kvsetkeys{ZREF@XR}%
}%
%    \end{macrocode}
%    \end{macro}
%
%    \begin{macro}{\ZREF@xr@URL}
%    \begin{macrocode}
\newcount\ZREF@xr@URL
\ZREF@xr@URL=\ltx@zero
%    \end{macrocode}
%    \end{macro}
%    \begin{macro}{\ZREF@xr@AddURL}
%    \begin{macrocode}
\def\ZREF@xr@AddURL#1{%
  \begingroup
    \def\ZREF@temp{#1}%
    \count@=\ltx@one
    \ZREF@xr@@AddUrl
  \endgroup
}
%    \end{macrocode}
%    \end{macro}
%    \begin{macro}{\ZREF@xr@@AddUrl}
%    \begin{macrocode}
\def\ZREF@xr@@AddUrl{%
  \ifnum\count@>\ZREF@xr@URL
    \global\advance\ZREF@xr@URL by\ltx@one
    \xdef\ZREF@xr@theURL{\romannumeral\ZREF@xr@URL}%
    \global\expandafter\let
        \csname Z@U@\ZREF@xr@theURL\endcsname\ZREF@temp
    \@PackageInfo{zref-xr}{%
      \ltx@backslashchar Z@U@\ZREF@xr@theURL:\MessageBreak
      \ZREF@temp\MessageBreak
    }%
  \else
    \expandafter
    \ifx\csname Z@U@\romannumeral\count@\endcsname\ZREF@temp
      \xdef\ZREF@xr@theURL{\romannumeral\count@}%
    \else
      \expandafter\expandafter\expandafter\ZREF@xr@@AddUrl
    \fi
  \fi
}
%    \end{macrocode}
%    \end{macro}
%
%    \begin{macro}{\zexternaldocument}
%    In its star form it looks for \cs{newlabel},
%    otherwise for \cs{zref@newlabel}.
%     Later we
%    will read \xfile{.aux} files that expects
%    |@| to have catcode 11 (letter).
%    \begin{macrocode}
\ZREF@IfDefinable\zexternaldocument\def{%
  {%
    \ZREF@NewPropAnchor
    \ZREF@NewPropTitle
    \begingroup
      \csname @safe@actives@true\endcsname
      \makeatletter
      \@ifstar{%
        \ZREF@xr@zreflabelfalse
        \@testopt\ZREF@xr@externaldocument{}%
      }{%
        \ZREF@xr@zreflabeltrue
        \@testopt\ZREF@xr@externaldocument{}%
      }%
  }%
}%
%    \end{macrocode}
%    \end{macro}
%
%    If the \cs{include} featuer was used, there can be
%    several \xfile{.aux} files. These files are read
%    one after another, especially they are not recursively
%    read in order to save read registers. Thus it can
%    happen that the read order of the newlabel commands
%    differs from \hologo{LaTeX}'s order using \cs{input}.
%
%    \begin{macro}{\ZREF@xr@externaldocument}
%    It reads the remaining arguments. \cs{newcommand}
%    comes in handy for the optional argument.
%    \begin{macrocode}
\def\ZREF@xr@externaldocument[#1]#2{%
    \def\ZREF@xr@prefix{#1}%
    \let\ZREF@xr@filelist\ltx@empty
    \edef\ZREF@xr@externalfile{#2}%
    \edef\ZREF@xr@file{\ZREF@xr@externalfile.aux}%
    \filename@parse{#2}%
    \@testopt\ZREF@xr@graburl{#2.\zref@xr@ext}%
}%
\def\ZREF@xr@graburl[#1]{%
    \edef\ZREF@xr@url{#1}%
    \ifZREF@xr@urluse
      \expandafter\ZREF@xr@AddURL\expandafter{\ZREF@xr@url}%
      \expandafter\def\expandafter\ZREF@xr@url
      \expandafter{\csname Z@U@\ZREF@xr@theURL\endcsname}%
    \fi
    \ZREF@xr@checkfile
  \endgroup
}%
%    \end{macrocode}
%    \end{macro}
%    \begin{macro}{\ZREF@xr@processfile}
%    We follow \xpackage{xr} here, \cs{IfFileExists} offers
%    a nicer test, but we have to open the file anyway.
%    \begin{macrocode}
\def\ZREF@xr@checkfile{%
  \openin\@inputcheck\ZREF@xr@file\relax
  \ifeof\@inputcheck
    \PackageWarning{zref-xr}{%
      File `\ZREF@xr@file' not found or empty,\MessageBreak
      labels not imported%
    }%
  \else
    \PackageInfo{zref-xr}{%
      Label \ifZREF@xr@zreflabel (zref) \fi
      import from `\ZREF@xr@file'%
    }%
    \def\ZREF@xr@found{0}%
    \def\ZREF@xr@ignored@empty{0}%
    \def\ZREF@xr@ignored@zref{0}%
    \def\ZREF@xr@ignored@ltx{0}%
    \ZREF@xr@processfile
    \closein\@inputcheck
    \begingroup
      \let\on@line\ltx@empty
      \PackageInfo{zref-xr}{%
        Statistics for `\ZREF@xr@file':\MessageBreak
        \ZREF@xr@found\space
        \ifZREF@xr@zreflabel zref\else LaTeX\fi\space
        label(s) found%
        \ifnum\ZREF@xr@ignored@empty>0 %
          ,\MessageBreak
          \ZREF@xr@ignored@empty\space empty label(s) ignored%
        \fi
        \ifnum\ZREF@xr@ignored@zref>0 %
          ,\MessageBreak
          \ZREF@xr@ignored@zref\space
          duplicated zref label(s) ignored%
        \fi
        \ifnum\ZREF@xr@ignored@ltx>0 %
          ,\MessageBreak
          \ZREF@xr@ignored@ltx\space
          duplicated latex label(s) ignored%
        \fi
      }%
    \endgroup
  \fi
  \ifx\ZREF@xr@filelist\ltx@empty
  \else
    \edef\ZREF@xr@file{%
      \expandafter\ltx@car\ZREF@xr@filelist\@nil
    }%
    \edef\ZREF@xr@filelist{%
      \expandafter\ltx@cdr\ZREF@xr@filelist\ltx@empty\@nil
    }%
    \expandafter\ZREF@xr@checkfile
  \fi
}%
%    \end{macrocode}
%    \end{macro}
%
%    \begin{macro}{\ZREF@xr@processfile}
%    \begin{macrocode}
\def\ZREF@xr@processfile{%
  \read\@inputcheck to\ZREF@xr@line
  \expandafter\ZREF@xr@processline\ZREF@xr@line..\ZREF@nil
  \ifeof\@inputcheck
  \else
    \expandafter\ZREF@xr@processfile
  \fi
}%
%    \end{macrocode}
%    \end{macro}
%    \begin{macro}{\ZREF@xr@processline}
%    The most work must be done for analyzing the
%    arguments of \cs{newlabel}.
%    \begin{macrocode}
\long\def\ZREF@xr@processline#1#2#3\ZREF@nil{%
  \def\x{#1}%
  \toks@{#2}%
  \ifZREF@xr@zreflabel
    \ifx\x\ZREF@xr@zref@newlabel
      \expandafter
      \ZREF@xr@process@zreflabel\ZREF@xr@line...\ZREF@nil
    \fi
  \else
    \ifx\x\ZREF@xr@newlabel
      \expandafter
      \ZREF@xr@process@label\ZREF@xr@line...[]\ZREF@nil
    \fi
  \fi
  \ifx\x\ZREF@xr@@input
    \edef\ZREF@xr@filelist{%
      \etex@unexpanded\expandafter{\ZREF@xr@filelist}%
      {\filename@area\the\toks@}%
    }%
  \fi
}%
\def\ZREF@xr@process@zreflabel\zref@newlabel#1#2#3\ZREF@nil{%
  \edef\ZREF@xr@refname{Z@R@\ZREF@xr@prefix#1}%
  \edef\ZREF@xr@found{\the\numexpr\ZREF@xr@found+1\relax}%
  \def\x{#2}%
  \edef\ZREF@xr@tempname{$temp$}%
  \edef\ZREF@xr@temprefname{Z@R@\ZREF@xr@tempname}%
  \let\ZREF@xr@list\x
  \ifx\ZREF@xr@list\ltx@empty
    \PackageWarningNoLine{zref-xr}{%
      Label `#1' without properties ignored\MessageBreak
      in file `\ZREF@xr@file'%
    }%
    \edef\ZREF@xr@ignored@empty{%
      \the\numexpr\ZREF@xr@ignored@empty+1\relax
    }%
  \else
    \expandafter\ZREF@xr@checklist\x\ZREF@nil
    \expandafter\let\csname\ZREF@xr@temprefname\endcsname\x
    \expandafter\ltx@LocalAppendToMacro
    \csname\ZREF@xr@temprefname\expandafter\endcsname
    \expandafter{%
      \expandafter\externaldocument\expandafter{%
        \ZREF@xr@externalfile
      }%
    }%
    \ZREF@xr@urlcheck\ZREF@xr@tempname
    \ifZREF@xr@tozreflabel
      \@ifundefined{\ZREF@xr@refname}{%
        \ifZREF@xr@verbose
          \PackageInfo{zref-xr}{%
            Import to zref label `\ZREF@xr@tempname#1'%
          }%
        \fi
        \global\expandafter
        \let\csname\ZREF@xr@refname\expandafter\endcsname
        \csname\ZREF@xr@temprefname\endcsname
      }{%
        \ZREF@xr@zref@ignorewarning{\ZREF@xr@prefix#1}%
      }%
    \fi
    \ifZREF@xr@toltxlabel
      \ZREF@xr@tolabel{\ZREF@xr@tempname}{\ZREF@xr@prefix#1}%
    \fi
  \fi
}%
\def\ZREF@xr@process@label\newlabel#1#2#3[#4]#5\ZREF@nil{%
  \def\ZREF@xr@refname{Z@R@\ZREF@xr@prefix#1}%
  \edef\ZREF@xr@found{\the\numexpr\ZREF@xr@found+1\relax}%
  \def\x{#2}%
  \edef\ZREF@xr@tempname{$temp$}%
  \edef\ZREF@xr@temprefname{Z@R@\ZREF@xr@tempname}%
  \expandafter\ZREF@xr@scanparams
      \csname\ZREF@xr@temprefname\expandafter\endcsname
      \x{}{}{}{}{}\ZREF@nil
  \ifx\\#4\\%
  \else
    % ntheorem knows an optional argument at the end of \newlabel
    \ZREF@NewPropTheotype
    \expandafter\ltx@LocalAppendToMacro
        \csname\ZREF@xr@temprefname\endcsname{\theotype{#4}}%
  \fi
  \expandafter\ltx@LocalAppendToMacro
  \csname\ZREF@xr@temprefname\expandafter\endcsname\expandafter{%
    \expandafter\externaldocument\expandafter{%
      \ZREF@xr@externalfile
    }%
  }%
  \ZREF@xr@urlcheck\ZREF@xr@tempname
  \ifZREF@xr@tozreflabel
    \@ifundefined{\ZREF@xr@refname}{%
      \ifZREF@xr@verbose
        \PackageInfo{zref-xr}{%
          Import to zref label `\ZREF@xr@prefix#1'%
        }%
      \fi
      \global\expandafter
      \let\csname\ZREF@xr@refname\expandafter\endcsname
      \csname\ZREF@xr@temprefname\endcsname
    }{%
      \ZREF@xr@zref@ignorewarning{\ZREF@xr@prefix#1}%
    }%
  \fi
  \ifZREF@xr@toltxlabel
    \ZREF@xr@tolabel{\ZREF@xr@tempname}{\ZREF@xr@prefix#1}%
  \fi
}
\def\ZREF@xr@zref@newlabel{\zref@newlabel}%
\def\ZREF@xr@newlabel{\newlabel}%
\def\ZREF@xr@@input{\@input}%
\def\ZREF@xr@relax{\relax}%
%    \end{macrocode}
%    \end{macro}
%    \begin{macro}{\ZREF@xr@tolabel}
%    \begin{macrocode}
\def\ZREF@xr@tolabel#1#2{%
  \ifZREF@xr@verbose
    \PackageInfo{zref-xr}{%
      Import to LaTeX label `#2'%
    }%
  \fi
  \zref@wrapper@unexpanded{%
    \expandafter\xdef\csname r@#2\endcsname{%
      {%
        \ltx@ifundefined{M@TitleReference}{%
          \ltx@ifundefined{TR@TitleReference}{%
            \zref@extractdefault{#1}{default}{}%
          }{%
            \noexpand\TR@TitleReference
            {\zref@extractdefault{#1}{default}{}}%
            {\zref@extractdefault{#1}{title}{}}%
          }%
        }{%
          \noexpand\M@TitleReference
          {\zref@extractdefault{#1}{default}{}}%
          {\zref@extractdefault{#1}{title}{}}%
        }%
      }%
      {\zref@extractdefault{#1}{page}{}}%
      \ltx@ifpackageloaded{nameref}{%
        {\zref@extractdefault{#1}{title}{}}%
        {\zref@extractdefault{#1}{anchor}{}}%
        \zref@ifrefcontainsprop{#1}{urluse}{%
          {\zref@extractdefault{#1}{urluse}{}}%
        }{%
          {\zref@extractdefault{#1}{url}{}}%
        }%
      }{}%
    }%
  }%
}
%    \end{macrocode}
%    \end{macro}
%
%    \begin{macro}{\ZREF@xr@zref@ignorewarning}
%    \begin{macrocode}
\def\ZREF@xr@zref@ignorewarning#1{%
  \PackageWarningNoLine{zref-xr}{%
    Zref label `#1' is already in use\MessageBreak
    in file `\ZREF@xr@file'%
  }%
  \edef\ZREF@xr@ignored@zref{%
    \the\numexpr\ZREF@xr@ignored@zref+1%
  }%
}%
%    \end{macrocode}
%    \end{macro}
%
%    \begin{macro}{\ZREF@xr@ltx@ignorewarning}
%    \begin{macrocode}
\def\ZREF@xr@ltx@ignorewarning#1{%
  \PackageWarningNoLine{zref-xr}{%
    LaTeX label `#1' is already in use\MessageBreak
    in file `\ZREF@xr@file'%
  }%
  \edef\ZREF@xr@ignored@ltx{%
    \the\numexpr\ZREF@xr@ignored@ltx+1%
  }%
}%
%    \end{macrocode}
%    \end{macro}
%
%    \begin{macro}{\ZREF@xr@checklist}
%    \begin{macrocode}
\def\ZREF@xr@checklist#1#2#3\ZREF@nil{%
  \ifx\@undefined#1\relax
    \expandafter\ZREF@xr@checkkey\string#1\@nil
  \fi
  \ifx\\#3\\%
  \else
    \ltx@ReturnAfterFi{%
      \ZREF@xr@checklist#3\ZREF@nil
    }%
  \fi
}%
\def\ZREF@xr@checkkey#1#2\@nil{%
  \zref@ifpropundefined{#2}{%
    \zref@newprop{#2}{}%
  }{}%
}%
%    \end{macrocode}
%    \end{macro}
%
%    \begin{macro}{\ZREF@xr@scanparams}
%    \begin{macrocode}
\def\ZREF@xr@scanparams#1#2#3#4#5#6#7\ZREF@nil{%
  \let#1\ltx@empty
  \ZREF@foundfalse
  \ZREF@xr@scantitleref#1#2\TR@TitleReference{}{}\ZREF@nil
  \ifZREF@found
  \else
    \ltx@LocalAppendToMacro#1{\default{#2}}%
  \fi
  % page
  \ltx@LocalAppendToMacro#1{\page{#3}}%
  % nameref title
  \ifZREF@found
  \else
    \ifx\\#4\\%
    \else
      \def\ZREF@xr@temp{#4}%
      \ifx\ZREF@xr@temp\ZREF@xr@relax
      \else
        \ltx@LocalAppendToMacro#1{\title{#4}}%
      \fi
    \fi
  \fi
  % anchor
  \ifx\\#5\\%
  \else
    \ltx@LocalAppendToMacro#1{\anchor{#5}}%
  \fi
  \ifx\\#6\\%
  \else
    \ifZREF@xr@urluse
      \ZREF@xr@AddURL{#6}%
      \expandafter\ltx@LocalAppendToMacro\expandafter#1%
      \expandafter{%
        \expandafter\urluse\expandafter{%
          \csname Z@U@\ZREF@xr@theURL\endcsname
        }%
      }%
    \else
      \ltx@LocalAppendToMacro#1{\url{#6}}%
    \fi
  \fi
}%
%    \end{macrocode}
%    \end{macro}
%
%    \begin{macro}{\ZREF@xr@scantitleref}
%    \begin{macrocode}
\def\ZREF@xr@scantitleref#1#2\TR@TitleReference#3#4#5\ZREF@nil{%
  \ifx\\#5\\%
  \else
    \ltx@LocalAppendToMacro#1{%
      \default{#3}%
      \title{#4}%
    }%
    \ZREF@foundtrue
  \fi
}%
%    \end{macrocode}
%    \end{macro}
%
%    \begin{macro}{\ZREF@xr@urlcheck}%
%    \begin{macrocode}
\def\ZREF@xr@urlcheck#1{%
  \zref@ifrefcontainsprop{#1}{anchor}{%
    \zref@ifrefcontainsprop{#1}{url}{%
    }{%
      \expandafter
      \ltx@LocalAppendToMacro\csname Z@R@#1\expandafter\endcsname
      \expandafter{%
        \csname url\ifZREF@xr@urluse use\fi
        \expandafter\endcsname\expandafter{\ZREF@xr@url}%
      }%
    }%
  }{%
  }%
}%
%    \end{macrocode}
%    \end{macro}
%
%    \begin{macrocode}
%</xr>
%    \end{macrocode}
%
%
% \subsection{Module \xmodule{hyperref}}
%
%    UNFINISHED :-(
%
%    \begin{macrocode}
%<*hyperref>
\NeedsTeXFormat{LaTeX2e}
\ProvidesPackage{zref-hyperref}%
  [2020-07-03 v2.32 Module hyperref for zref (HO)]%
\RequirePackage{zref-base}[2019/11/29]
\ifx\ZREF@base@ok Y%
\else
  \expandafter\endinput
\fi
%    \end{macrocode}
%
%    \begin{macrocode}
\ZREF@NewPropAnchor
\zref@addprop\ZREF@mainlist{anchor}%
%    \end{macrocode}
%
%    \begin{macrocode}
%</hyperref>
%    \end{macrocode}
%
%
% \subsection{Module \xmodule{savepos}}
%
%    Module \xmodule{savepos} provides an interface for
%    \pdfTeX's \cmd{\pdfsavepos}, see the manual for
%    \pdfTeX.
%
% \subsubsection{Identification}
%
%    \begin{macrocode}
%<*savepos>
\NeedsTeXFormat{LaTeX2e}
\ProvidesPackage{zref-savepos}%
  [2020-07-03 v2.32 Module savepos for zref (HO)]%
\RequirePackage{zref-base}[2019/11/29]
\ifx\ZREF@base@ok Y%
\else
  \expandafter\endinput
\fi
%    \end{macrocode}
%
% \subsubsection{Availability}
%
%    First we check, whether the feature is available.
%    \begin{macrocode}
\ifx\directlua\@undefined
\ltx@IfUndefined{pdfsavepos}{%
  \PackageError\ZREF@name{%
    \string\pdfsavepos\space is not supported.\MessageBreak
    It is provided by pdfTeX (1.40) or XeTeX%
  }\ZREF@UpdatePdfTeX
  \endinput
}{}%
\fi
%    \end{macrocode}
%    In PDF mode we are done. However support for DVI mode
%    was added later in version 1.40.0. In earlier versions
%    \cs{pdfsavepos} is defined, but its execution raises
%    an error. Note that \hologo{XeTeX} also provides \cs{pdfsavepos}.
%    \begin{macrocode}
\ifpdf
\else
  \ltx@IfUndefined{pdftexversion}{%
  }{%
    \ifnum\pdftexversion<140 %
      \PackageError\ZREF@name{%
        \string\pdfsavepos\space is not supported in DVI mode%
        \MessageBreak
        of this pdfTeX version%
      }\ZREF@UpdatePdfTeX
      \expandafter\expandafter\expandafter\endinput
    \fi
  }%
\fi
%    \end{macrocode}
%
% \subsubsection{Setup}
%
%    \begin{macrocode}
\zref@newlist{savepos}
\ifx\directlua\@undefined
  \zref@newprop*{posx}[0]{\the\pdflastxpos}
  \zref@newprop*{posy}[0]{\the\pdflastypos}
\else
  \zref@newprop*{posx}[0]{\the\lastxpos}
  \zref@newprop*{posy}[0]{\the\lastypos}
\fi
\zref@addprops{savepos}{posx,posy}
%    \end{macrocode}
%
% \subsubsection{User macros}
%
%    \begin{macro}{\zref@savepos}
%    \begin{macrocode}
\ifx\directlua\@undefined
  \def\zref@savepos{%
    \if@filesw
      \pdfsavepos
    \fi
  }
\else
  \def\zref@savepos{%
    \if@filesw
      \savepos
    \fi
  }
\fi
%    \end{macrocode}
%    \end{macro}
%
%    \begin{macro}{\ZREF@zsavepos}
%    \begin{macrocode}
\def\ZREF@zsavepos#1#2#3{%
  \@bsphack
  \if@filesw
    \zref@savepos
    #1{#3}{#2}%
    \ltx@IfUndefined{TeXXeTstate}{%
    }{%
      \ifnum\TeXXeTstate=\ltx@zero
      \else
        \zref@savepos
      \fi
    }%
  \fi
  \@esphack
}
%    \end{macrocode}
%    \end{macro}
%
%    \begin{macro}{\zsavepos}
%    The current location is stored in a reference with the
%    given name.
%    \begin{macrocode}
\ZREF@IfDefinable\zsavepos\def{%
  {%
    \ZREF@zsavepos\zref@labelbylist{savepos}%
  }%
}
%    \end{macrocode}
%    \end{macro}
%
%    \begin{macro}{\zsaveposx}
%    \begin{macrocode}
\ZREF@IfDefinable\zsaveposx\def{%
  {%
    \ZREF@zsavepos\zref@labelbyprops{posx}%
  }%
}
%    \end{macrocode}
%    \end{macro}
%    \begin{macro}{\zsaveposy}
%    \begin{macrocode}
\ZREF@IfDefinable\zsaveposy\def{%
  {%
    \ZREF@zsavepos\zref@labelbyprops{posy}%
  }%
}
%    \end{macrocode}
%    \end{macro}
%
%    \begin{macro}{\zposx}
%    \begin{macro}{\zposy}
%    The horizontal and vertical position are available by
%    \cs{zposx} and \cs{zposy}. Do not rely on absolute
%    positions. They differ in DVI and PDF mode of \pdfTeX.
%    Use differences instead. The unit of the position numbers
%    is sp.
%    \begin{macrocode}
\newcommand*{\zposx}[1]{%
  \zref@extract{#1}{posx}%
}%
\newcommand*{\zposy}[1]{%
  \zref@extract{#1}{posy}%
}%
%    \end{macrocode}
%    \end{macro}
%    \end{macro}
%
%    Typically horizontal and vertical positions are used
%    inside calculations. Therefore the extracting macros
%    should be expandable and babel's patch is not applyable.
%
%    Also it is in the responsibility of the user to marked
%    used positions by \cs{zrefused}
%    in order to notify \hologo{LaTeX} about undefined references.
%
%    \begin{macro}{\ZREF@savepos@ok}
%    \begin{macrocode}
\let\ZREF@savepos@ok=Y
%    \end{macrocode}
%    \end{macro}
%
%    \begin{macrocode}
%</savepos>
%    \end{macrocode}
%
%
% \subsection{Module \xmodule{abspos}}
%
% \subsubsection{Identification}
%
%    \begin{macrocode}
%<*abspos>
\NeedsTeXFormat{LaTeX2e}
\ProvidesPackage{zref-abspos}%
  [2020-07-03 v2.32 Module abspos for zref (HO)]%
\RequirePackage{zref-base}[2019/11/29]
\ifx\ZREF@base@ok Y%
\else
  \expandafter\endinput
\fi
%    \end{macrocode}
%
%    \begin{macrocode}
\RequirePackage{zref-savepos}[2019/11/29]
\ifx\ZREF@savepos@ok Y%
\else
  \expandafter\endinput
\fi
%    \end{macrocode}
%    \begin{macrocode}
\RequirePackage{zref-pagelayout}[2019/11/29]
\zref@addprop{savepos}{abspage}
\zref@addprop{savepos}{pagevalue}
%    \end{macrocode}
%
%    \begin{macro}{\zref@absposx}
%    \begin{macrocode}
\newcommand*{\zref@absposx}[3]{%
  \number
    \expandafter\zref@absposnumx\expandafter{%
      \number\zref@extractdefault{#1}{abspage}{0}%
    }{#2}{#3}%
  \ltx@space
}
%    \end{macrocode}
%    \end{macro}
%
%    \begin{macro}{\zref@absposy}
%    \begin{macrocode}
\newcommand*{\zref@absposy}[3]{%
  \number
  \expandafter\zref@absposnumy\expandafter{%
    \number\zref@extractdefault{#1}{abspage}{0}%
  }{#2}{#3}%
  \ltx@space
}
%    \end{macrocode}
%    \end{macro}
%
%    \begin{macro}{\zref@absposnumx}
%    \begin{macrocode}
\newcommand*{\zref@absposnumx}[3]{%
  \number
%  \ifnum#1>\ltx@zero
%    \zref@ifrefundefined{thepage#1}{%
%      0%
%    }{%
      \numexpr\ZREF@absposnum{thepage#1}{#2}{x}{#3}\relax
%    }%
%  \else
%    0%
%  \fi
}
%    \end{macrocode}
%    \end{macro}
%
%    \begin{macro}{\zref@absposnumy}
%    \begin{macrocode}
\newcommand*{\zref@absposnumy}[3]{%
  \number
%  \ifnum#1>\ltx@zero
%    \zref@ifrefundefined{thepage#1}{%
%      0%
%    }{%
      \numexpr\ZREF@absposnum{thepage#1}{#2}{y}{#3}\relax
%    }%
%  \else
%    0%
%  \fi
}
%    \end{macrocode}
%    \end{macro}
%
%    \begin{macro}{\ZREF@absposnum}
%    \begin{macrocode}
\def\ZREF@absposnum#1#2#3#4{%
  \ltx@ifundefined{ZREF@abspos@#2@#3@#4}{%
    0%
  }{%
    \csname ZREF@abspos@#2@#3@#4\endcsname{#1}%
  }%
}
%    \end{macrocode}
%    \end{macro}
%
%    \begin{macro}{\zref@def@absposx}
%    \begin{macrocode}
\ZREF@Robust\def\zref@def@absposx#1{%
  \zref@wrapper@babel{\ZREF@def@abspos{#1}\zref@absposx}%
}
%    \end{macrocode}
%    \end{macro}
%
%    \begin{macro}{\zref@def@absposy}
%    \begin{macrocode}
\ZREF@Robust\def\zref@def@absposy#1{%
  \zref@wrapper@babel{\ZREF@def@abspos{#1}\zref@absposy}%
}
%    \end{macrocode}
%    \end{macro}
%
%    \begin{macro}{\zref@def@absposnumx}
%    \begin{macrocode}
\ZREF@Robust\def\zref@def@absposnumx#1{%
  \zref@wrapper@babel{\ZREF@def@abspos{#1}\zref@absposnumx}%
}
%    \end{macrocode}
%    \end{macro}
%
%    \begin{macro}{\zref@def@absposnumy}
%    \begin{macrocode}
\ZREF@Robust\def\zref@def@absposnumy#1{%
  \zref@wrapper@babel{\ZREF@def@abspos{#1}\zref@absposnumy}%
}
%    \end{macrocode}
%    \end{macro}
%    \begin{macro}{\ZREF@def@abspos}
%    \begin{macrocode}
\def\ZREF@def@absposnumy#1#2#3#4#5{%
  \edef#1{#2{#3}{#4}{#5}}%
}
%    \end{macrocode}
%    \end{macro}
%
%    \begin{macro}{\zref@absposused}
%    \begin{macrocode}
\ZREF@Robust\def\zref@absposused{%
  \zref@wrapper@babel\ZREF@abspos@used
}
%    \end{macrocode}
%    \end{macro}
%    \begin{macro}{\ZREF@abspos@used}
%    \begin{macrocode}
\def\ZREF@abspos@used#1{%
  \zref@refused{#1}%
  \zref@ifrefundefined{#1}{%
  }{%
    \begingroup
      \edef\ZREF@temp{%
        \zref@extractdefault{#1}{abspage}{0}%
      }%
      \ifnum\ZREF@temp>\ltx@zero
        \zref@refused{thepage\ZREF@temp}%
      \else
        \@PackageError{zref-abspos}{%
          \string\zref@pos@label@used\ltx@space
          needs property `abspage'\MessageBreak
          in label `#1'%
        }\@ehc
      \fi
    \endgroup
  }%
}
%    \end{macrocode}
%    \end{macro}
%
%    \begin{macro}{\zref@absposnumused}
%    \begin{macrocode}
\newcommand*{\zref@absposnumused}[1]{%
  \ifnum#1>\ltx@zero
    \zref@refused{thepage\number#1}%
  \else
    \@PackageError{zref-abspos}{%
      Invalid absolute page number (#1)\MessageBreak
      for \string\zref@pos@num@used.\MessageBreak
      A positive integer is expected%
    }\@ehc
  \fi
}
%    \end{macrocode}
%    \end{macro}
%
%    \begin{macro}{\zref@ifabsposundefined}
%    \begin{macrocode}
\def\zref@ifabsposundefined#1{%
  \zref@ifrefundefined{#1}\ltx@firsttwo{%
    \expandafter\zref@ifabsposnumundefined\expandafter{%
      \number\zref@extractdefault{#1}{abspage}{0}%
    }%
  }%
}
%    \end{macrocode}
%    \end{macro}
%
%    \begin{macro}{\zref@ifabsposnumundefined}
%    \begin{macrocode}
\def\zref@ifabsposnumundefined#1{%
  \ifnum\ZREF@number{#1}>\ltx@zero
    \zref@ifrefundefined{thepage#1}%
        \ltx@firstoftwo\ltx@secondoftwo
  \else
    \expandafter\ltx@firstoftwo
  \fi
}
%    \end{macrocode}
%    \end{macro}
%
% \subsubsection{Media}
%
%    \begin{macro}{\ZREF@abspos@media@width}
%    \begin{macrocode}
\edef\ZREF@abspos@media@width{%
  \ltx@ifundefined{pdfpagewidth}{%
    \ltx@ifundefined{mediawidth}{%
      \ltx@ifundefined{stockwidth}{%
        paperwidth%
      }{%
        stockwidth%
      }%
    }{%
      mediawidth%
    }%
  }{%
    pdfpagewidth%
  }%
}
\ifluatex
\def\ZREF@abspos@media@width{pdfpagewidth}%
\fi
%    \end{macrocode}
%    \end{macro}
%    \begin{macro}{\ZREF@abspos@media@height}
%    \begin{macrocode}
\edef\ZREF@abspos@media@height{%
  \ltx@ifundefined{pdfpageheight}{%
    \ltx@ifundefined{mediaheight}{%
      \ltx@ifundefined{stockheight}{%
        paperheight%
      }{%
        stockheight%
      }%
    }{%
      mediaheight%
    }%
  }{%
    \noexpand\ifcase\pdfpageheight
      \ltx@ifundefined{stockheight}{%
        paperheight%
      }{%
        stockheight%
      }%
    \noexpand\else
      pdfpageheight%
    \noexpand\fi
  }%
}
\ifluatex
\edef\ZREF@abspos@media@height{%
   \noexpand\ifcase\pageheight
      \ltx@ifundefined{stockheight}{%
        paperheight%
      }{%
        stockheight%
      }%
    \noexpand\else
      pdfpageheight%
    \noexpand\fi}%
\fi
%    \end{macrocode}
%    \end{macro}
%
%    \begin{macro}{\ZREF@abspos@media@x@left}
%    \begin{macrocode}
\def\ZREF@abspos@media@x@left#1{%
  0%
}
%    \end{macrocode}
%    \end{macro}
%    \begin{macro}{\ZREF@abspos@media@x@right}
%    \begin{macrocode}
\def\ZREF@abspos@media@x@right#1{%
  \zref@extract{#1}\ZREF@abspos@media@width
}
%    \end{macrocode}
%    \end{macro}
%    \begin{macro}{\ZREF@abspos@media@x@center}
%    \begin{macrocode}
\def\ZREF@abspos@media@x@center#1{%
  \ZREF@abspos@media@x@left{#1}%
  +\zref@extract{#1}\ZREF@abspos@media@width/2%
}
%    \end{macrocode}
%    \end{macro}
%
%    \begin{macro}{\ZREF@abspos@media@y@top}
%    \begin{macrocode}
\def\ZREF@abspos@media@y@top#1{%
  \zref@extract{#1}\ZREF@abspos@media@height
}
%    \end{macrocode}
%    \end{macro}
%    \begin{macro}{\ZREF@abspos@media@y@bottom}
%    \begin{macrocode}
\def\ZREF@abspos@media@y@bottom#1{%
  0%
}
%    \end{macrocode}
%    \end{macro}
%    \begin{macro}{\ZREF@abspos@media@y@center}
%    \begin{macrocode}
\def\ZREF@abspos@media@y@center#1{%
  \zref@extract{#1}\ZREF@abspos@media@height/2%
}
%    \end{macrocode}
%    \end{macro}
%
% \subsubsection{Paper}
%
%    \begin{macro}{\ZREF@abspos@paper@x@left}
%    \begin{macrocode}
\def\ZREF@abspos@paper@x@left#1{%
  0%
}
%    \end{macrocode}
%    \end{macro}
%    \begin{macro}{\ZREF@abspos@paper@x@right}
%    \begin{macrocode}
\def\ZREF@abspos@paper@x@right#1{%
  \zref@extract{#1}{paperwidth}%
}
%    \end{macrocode}
%    \end{macro}
%    \begin{macro}{\ZREF@abspos@paper@x@center}
%    \begin{macrocode}
\def\ZREF@abspos@paper@x@center#1{%
  \zref@extract{#1}{paperwidth}/2%
}
%    \end{macrocode}
%    \end{macro}
%
%    \begin{macro}{\ZREF@abspos@paper@y@top}
%    \begin{macrocode}
\let\ZREF@abspos@paper@y@top\ZREF@abspos@media@y@top
%    \end{macrocode}
%    \end{macro}
%    \begin{macro}{\ZREF@abspos@paper@y@bottom}
%    \begin{macrocode}
\def\ZREF@abspos@paper@y@bottom#1{%
  \ZREF@abspos@paper@y@top{#1}%
  -\zref@extract{#1}{paperheight}%
}
%    \end{macrocode}
%    \end{macro}
%    \begin{macro}{\ZREF@abspos@paper@y@center}
%    \begin{macrocode}
\def\ZREF@abspos@paper@y@center#1{%
  \ZREF@abspos@paper@y@top{#1}%
  -\zref@extract{#1}{paperheight}/2%
}
%    \end{macrocode}
%    \end{macro}
%
% \subsubsection{Origin}
%
% There doesn't seem a good reason to make these tests depend on pdf mode
%  in current engines, so comment out the \verb|\ifpdf| tests.
%    \begin{macro}{\ZREF@abspos@origin@x}
%    \begin{macrocode}
\let\ZREF@temp\ltx@two
\ltx@IfUndefined{pdfhorigin}{}{%
%  \ifpdf
    \let\ZREF@temp\ltx@zero
%  \fi
}
\ifluatex
% \ifpdf
  \let\ZREF@temp\ltx@zero
% \fi
\fi

\ifx\ZREF@temp\ltx@two
  \ifnum\mag=1000 %
    \let\ZREF@temp\ltx@one
  \fi
\fi
\ifcase\ZREF@temp
  \def\ZREF@abspos@origin@x#1{%
    \zref@extract{#1}{pdfhorigin}%
  }%
\or
  \def\ZREF@abspos@origin@x#1{%
    4736286%
  }%
\or
  \def\ZREF@abspos@origin@x#1{%
    \numexpr\mag/1000*\dimexpr 1truein\relax\relax
  }%
\fi
%    \end{macrocode}
%    \end{macro}
%    \begin{macro}{\ZREF@abspos@origin@y}
%    \begin{macrocode}
\let\ZREF@temp\ltx@two
\ltx@IfUndefined{pdfvorigin}{}{%
%  \ifpdf
    \let\ZREF@temp\ltx@zero
%  \fi
}
\ifluatex
%  \ifpdf
    \let\ZREF@temp\ltx@zero
%  \fi
\fi
\ifx\ZREF@temp\ltx@two
  \ifnum\mag=1000 %
    \let\ZREF@temp\ltx@one
  \fi
\fi
\ifcase\ZREF@temp
  \def\ZREF@abspos@origin@y#1{%
    \zref@extract{#1}{pdfvorigin}%
  }%
\or
  \def\ZREF@abspos@origin@y#1{%
    4736286%
  }%
\or
  \def\ZREF@abspos@origin@y#1{%
    \numexpr\mag/1000*\dimexpr 1truein\relax\relax
   }%
\fi
%    \end{macrocode}
%    \end{macro}
%
% \subsubsection{Header}
%
%    \begin{macro}{\ZREF@abspos@head@x@left}
%    \begin{macrocode}
\def\ZREF@abspos@head@x@left#1{%
  \ZREF@abspos@paper@x@left{#1}%
  +\ZREF@abspos@origin@x{#1}%
  +\zref@extract{#1}{hoffset}%
  +\ifodd\zref@extractdefault{#1}{pagevalue}{\number\c@page} %
     \zref@extract{#1}{oddsidemargin}%
   \else
     \zref@extract{#1}{evensidemargin}%
   \fi
}
%    \end{macrocode}
%    \end{macro}
%    \begin{macro}{\ZREF@abspos@head@x@right}
%    \begin{macrocode}
\def\ZREF@abspos@head@x@right#1{%
  \ZREF@abspos@head@x@left{#1}%
  +\zref@extract{#1}{textwidth}%
}
%    \end{macrocode}
%    \end{macro}
%    \begin{macro}{\ZREF@abspos@head@x@center}
%    \begin{macrocode}
\def\ZREF@abspos@head@x@center#1{%
  \ZREF@abspos@head@x@left{#1}%
  +\zref@extract{#1}{textwidth}/2%
}
%    \end{macrocode}
%    \end{macro}
%
%    \begin{macro}{\ZREF@abspos@head@y@top}
%    \begin{macrocode}
\def\ZREF@abspos@head@y@top#1{%
  \ZREF@abspos@paper@y@top{#1}%
  -\ZREF@abspos@origin@y{#1}%
  -\zref@extract{#1}{voffset}%
  -\zref@extract{#1}{topmargin}%
}
%    \end{macrocode}
%    \end{macro}
%    \begin{macro}{\ZREF@abspos@head@y@bottom}
%    \begin{macrocode}
\def\ZREF@abspos@head@y@bottom#1{%
  \ZREF@abspos@head@y@top{#1}%
  -\zref@extract{#1}{headheight}%
}
%    \end{macrocode}
%    \end{macro}
%    \begin{macro}{\ZREF@abspos@head@y@center}
%    \begin{macrocode}
\def\ZREF@abspos@head@y@center#1{%
  \ZREF@abspos@head@y@top{#1}%
  -\zref@extract{#1}{headheight}/2%
}
%    \end{macrocode}
%    \end{macro}
%
% \subsubsection{Body}
%
%    \begin{macro}{\ZREF@abspos@body@x@left}
%    \begin{macrocode}
\let\ZREF@abspos@body@x@left\ZREF@abspos@head@x@left
%    \end{macrocode}
%    \end{macro}
%    \begin{macro}{\ZREF@abspos@body@x@right}
%    \begin{macrocode}
\let\ZREF@abspos@body@x@right\ZREF@abspos@head@x@right
%    \end{macrocode}
%    \end{macro}
%    \begin{macro}{\ZREF@abspos@body@x@center}
%    \begin{macrocode}
\let\ZREF@abspos@body@x@center\ZREF@abspos@head@x@center
%    \end{macrocode}
%    \end{macro}
%
%    \begin{macro}{\ZREF@abspos@body@y@top}
%    \begin{macrocode}
\def\ZREF@abspos@body@y@top#1{%
  \ZREF@abspos@head@y@bottom{#1}%
  -\zref@extract{#1}{headsep}%
}
%    \end{macrocode}
%    \end{macro}
%    \begin{macro}{\ZREF@abspos@body@y@bottom}
%    \begin{macrocode}
\def\ZREF@abspos@body@y@bottom#1{%
  \ZREF@abspos@body@y@top{#1}%
  -\zref@extract{#1}{textheight}%
}
%    \end{macrocode}
%    \end{macro}
%    \begin{macro}{\ZREF@abspos@body@y@center}
%    \begin{macrocode}
\def\ZREF@abspos@body@y@center#1{%
  \ZREF@abspos@body@y@top{#1}%
  -\zref@extract{#1}{textheight}/2%
}
%    \end{macrocode}
%    \end{macro}
%
% \subsubsection{Footer}
%
%    \begin{macro}{\ZREF@abspos@foot@x@left}
%    \begin{macrocode}
\let\ZREF@abspos@foot@x@left\ZREF@abspos@head@x@left
%    \end{macrocode}
%    \end{macro}
%    \begin{macro}{\ZREF@abspos@foot@x@right}
%    \begin{macrocode}
\let\ZREF@abspos@foot@x@right\ZREF@abspos@head@x@right
%    \end{macrocode}
%    \end{macro}
%    \begin{macro}{\ZREF@abspos@foot@x@center}
%    \begin{macrocode}
\let\ZREF@abspos@foot@x@center\ZREF@abspos@head@x@center
%    \end{macrocode}
%    \end{macro}
%
%    \begin{macro}{\ZREF@abspos@foot@y@bottom}
%    \begin{macrocode}
\def\ZREF@abspos@foot@y@bottom#1{%
  \ZREF@abspos@body@y@bottom{#1}%
  -\zref@extract{#1}{footskip}%
}
%    \end{macrocode}
%    \end{macro}
%
% \subsubsection{Marginal notes}
%
%    \begin{macro}{\ZREF@abspos@marginpar@x@left}
%    \begin{macrocode}
\def\ZREF@abspos@marginpar@x@left#1{%
  \ifodd\zref@extractdefault{#1}{pagevalue}{\number\c@page} %
    \ZREF@abspos@body@x@right{#1}%
    +\zref@extract{#1}{marginparsep}%
  \else
    \ZREF@abspos@body@x@left{#1}%
    -\zref@extract{#1}{marginparsep}%
    -\zref@extract{#1}{marginparwidth}%
  \fi
}
%    \end{macrocode}
%    \end{macro}
%    \begin{macro}{\ZREF@abspos@marginpar@x@right}
%    \begin{macrocode}
\def\ZREF@abspos@marginpar@x@right#1{%
  \ZREF@abspos@marginpar@x@left{#1}%
  +\zref@extract{#1}{marginparwidth}%
}
%    \end{macrocode}
%    \end{macro}
%    \begin{macro}{\ZREF@abspos@marginpar@x@center}
%    \begin{macrocode}
\def\ZREF@abspos@marginpar@x@center#1{%
  \ZREF@abspos@marginpar@x@left{#1}%
  +\zref@extract{#1}{marginparwidth}/2%
}
%    \end{macrocode}
%    \end{macro}
%
%    \begin{macro}{\ZREF@abspos@marginpar@y@top}
%    \begin{macrocode}
\let\ZREF@abspos@marginpar@y@top\ZREF@abspos@body@y@top
%    \end{macrocode}
%    \end{macro}
%    \begin{macro}{\ZREF@abspos@marginpar@y@bottom}
%    \begin{macrocode}
\let\ZREF@abspos@marginpar@y@bottom\ZREF@abspos@body@y@bottom
%    \end{macrocode}
%    \end{macro}
%    \begin{macro}{\ZREF@abspos@marginpar@y@center}
%    \begin{macrocode}
\let\ZREF@abspos@marginpar@y@center\ZREF@abspos@body@y@center
%    \end{macrocode}
%    \end{macro}
%
% \subsubsection{Stock paper}
%
%    \begin{macro}{\ZREF@abspos@stock@x@left}
%    \begin{macrocode}
\let\ZREF@abspos@stock@x@left\ZREF@abspos@paper@x@left
%    \end{macrocode}
%    \end{macro}
%    \begin{macro}{\ZREF@abspos@stock@x@right}
%    \begin{macrocode}
\let\ZREF@abspos@stock@x@right\ZREF@abspos@paper@x@right
%    \end{macrocode}
%    \end{macro}
%    \begin{macro}{\ZREF@abspos@stock@x@center}
%    \begin{macrocode}
\let\ZREF@abspos@stock@x@center\ZREF@abspos@paper@x@center
%    \end{macrocode}
%    \end{macro}
%
%    \begin{macro}{\ZREF@abspos@stock@y@top}
%    \begin{macrocode}
\let\ZREF@abspos@stock@y@top\ZREF@abspos@paper@y@top
%    \end{macrocode}
%    \end{macro}
%    \begin{macro}{\ZREF@abspos@stock@y@bottom}
%    \begin{macrocode}
\let\ZREF@abspos@stock@y@bottom\ZREF@abspos@paper@y@bottom
%    \end{macrocode}
%    \end{macro}
%    \begin{macro}{\ZREF@abspos@stock@y@center}
%    \begin{macrocode}
\let\ZREF@abspos@stock@y@center\ZREF@abspos@paper@y@center
%    \end{macrocode}
%    \end{macro}
%
%    \begin{macrocode}
%</abspos>
%    \end{macrocode}
%
%
% \subsection{Module \xmodule{dotfill}}
%
%    \begin{macrocode}
%<*dotfill>
\NeedsTeXFormat{LaTeX2e}
\ProvidesPackage{zref-dotfill}%
  [2020-07-03 v2.32 Module dotfill for zref (HO)]%
\RequirePackage{zref-base}[2019/11/29]
\ifx\ZREF@base@ok Y%
\else
  \expandafter\endinput
\fi
%    \end{macrocode}
%
%    For measuring the width of \cs{zdotfill} we use the features
%    provided by module \xmodule{savepos}.
%    \begin{macrocode}
\RequirePackage{zref-savepos}[2019/11/29]
%    \end{macrocode}
%    For automatically generated label names we use the unique
%    counter of module \xmodule{base}.
%    \begin{macrocode}
\zref@require@unique
%    \end{macrocode}
%    Configuration is done by the key value interface of package
%    \xpackage{keyval}.
%    \begin{macrocode}
\RequirePackage{keyval}
%    \end{macrocode}
%    The definitions of the keys follow.
%    \begin{macrocode}
\define@key{ZREF@DF}{unit}{%
  \def\ZREF@df@unit{#1}%
}
\define@key{ZREF@DF}{min}{%
  \def\ZREF@df@min{#1}%
}
\define@key{ZREF@DF}{dot}{%
  \def\ZREF@df@dot{#1}%
}
%    \end{macrocode}
%    Defaults are set, see user interface.
%    \begin{macrocode}
\providecommand\ZREF@df@min{2}
\providecommand\ZREF@df@unit{.44em}
\providecommand\ZREF@df@dot{.}
%    \end{macrocode}
%    \begin{macro}{\zdotfillsetup}
%    Configuration of \cs{zdotfill} is done by \cs{zdotfillsetup}.
%    \begin{macrocode}
\newcommand*{\zdotfillsetup}{\kvsetkeys{ZREF@DF}}
%    \end{macrocode}
%    \end{macro}
%
%    \begin{macro}{\zdotfill}
%    \cs{zdotfill} sets labels at the left and the right to get
%    the horizontal position. \cs{zsavepos} is not used, because
%    we do not need the vertical position.
%    \begin{macrocode}
\ZREF@IfDefinable\zdotfill\def{%
  {%
    \leavevmode
    \global\advance\c@zref@unique\ltx@one
    \begingroup
      \def\ZREF@temp{zref@\number\c@zref@unique}%
      \pdfsavepos
      \zref@labelbyprops{\thezref@unique L}{posx}%
      \setlength{\dimen@}{\ZREF@df@unit}%
      \zref@ifrefundefined{\thezref@unique R}{%
        \ZREF@dotfill
      }{%
        \ifnum\numexpr\zposx{\thezref@unique R}%
                      -\zposx{\thezref@unique L}\relax
            <\dimexpr\ZREF@df@min\dimen@\relax
          \hfill
        \else
          \ZREF@dotfill
        \fi
      }%
      \pdfsavepos
      \zref@labelbyprops{\thezref@unique R}{posx}%
    \endgroup
    \kern\z@
  }%
}
%    \end{macrocode}
%    \end{macro}
%    \begin{macro}{\ZREF@dotfill}
%    Help macro that actually sets the dots.
%    \begin{macrocode}
\def\ZREF@dotfill{%
  \cleaders\hb@xt@\dimen@{\hss\ZREF@df@dot\hss}\hfill
}
%    \end{macrocode}
%    \end{macro}
%
%    \begin{macrocode}
%</dotfill>
%    \end{macrocode}
%
% \subsection{Module \xmodule{env}}
%
%    \begin{macrocode}
%<*env>
\NeedsTeXFormat{LaTeX2e}
\ProvidesPackage{zref-env}%
  [2020-07-03 v2.32 Module env for zref (HO)]%
\RequirePackage{zref-base}[2019/11/29]
\ifx\ZREF@base@ok Y%
\else
  \expandafter\endinput
\fi
%    \end{macrocode}
%
%    \begin{macrocode}
\zref@newprop{envname}[]{\@currenvir}
\zref@newprop{envline}[]{\zref@env@line}
%    \end{macrocode}
%
%    \begin{macro}{\zref@env@line}
%    Macro \cs{zref@env@line} extracts the line number
%    from \cs{@currenvline}.
%    \begin{macrocode}
\def\zref@env@line{%
  \ifx\@currenvline\ltx@empty
  \else
    \expandafter
    \ZREF@ENV@line\@currenvline\ltx@empty line \ltx@empty\@nil
  \fi
}
%    \end{macrocode}
%    \end{macro}
%    \begin{macro}{\ZREF@ENV@line}
%    \begin{macrocode}
\def\ZREF@ENV@line#1line #2\ltx@empty#3\@nil{#2}%
%    \end{macrocode}
%    \end{macro}
%
%    \begin{macrocode}
%</env>
%    \end{macrocode}
%
%% \section{Installation}
%
% \subsection{Download}
%
% \paragraph{Package.} This package is available on
% CTAN\footnote{\CTANpkg{zref}}:
% \begin{description}
% \item[\CTAN{macros/latex/contrib/zref/zref.dtx}] The source file.
% \item[\CTAN{macros/latex/contrib/zref/zref.pdf}] Documentation.
% \end{description}
%
%
% \paragraph{Bundle.} All the packages of the bundle `zref'
% are also available in a TDS compliant ZIP archive. There
% the packages are already unpacked and the documentation files
% are generated. The files and directories obey the TDS standard.
% \begin{description}
% \item[\CTANinstall{install/macros/latex/contrib/zref.tds.zip}]
% \end{description}
% \emph{TDS} refers to the standard ``A Directory Structure
% for \TeX\ Files'' (\CTANpkg{tds}). Directories
% with \xfile{texmf} in their name are usually organized this way.
%
% \subsection{Bundle installation}
%
% \paragraph{Unpacking.} Unpack the \xfile{zref.tds.zip} in the
% TDS tree (also known as \xfile{texmf} tree) of your choice.
% Example (linux):
% \begin{quote}
%   |unzip zref.tds.zip -d ~/texmf|
% \end{quote}
%
%
% \subsection{Package installation}
%
% \paragraph{Unpacking.} The \xfile{.dtx} file is a self-extracting
% \docstrip\ archive. The files are extracted by running the
% \xfile{.dtx} through \plainTeX:
% \begin{quote}
%   \verb|tex zref.dtx|
% \end{quote}
%
% \paragraph{TDS.} Now the different files must be moved into
% the different directories in your installation TDS tree
% (also known as \xfile{texmf} tree):
% \begin{quote}
% \def\t{^^A
% \begin{tabular}{@{}>{\ttfamily}l@{ $\rightarrow$ }>{\ttfamily}l@{}}
%   zref.sty & tex/latex/zref/zref.sty\\
%   zref-base.sty & tex/latex/zref/zref-base.sty\\
%   zref-abspage.sty & tex/latex/zref/zref-abspage.sty\\
%   zref-abspos.sty & tex/latex/zref/zref-abspos.sty\\
%   zref-counter.sty & tex/latex/zref/zref-counter.sty\\
%   zref-dotfill.sty & tex/latex/zref/zref-dotfill.sty\\
%   zref-env.sty & tex/latex/zref/zref-env.sty\\
%   zref-hyperref.sty & tex/latex/zref/zref-hyperref.sty\\
%   zref-lastpage.sty & tex/latex/zref/zref-lastpage.sty\\
%   zref-marks.sty & tex/latex/zref/zref-marks.sty\\
%   zref-nextpage.sty & tex/latex/zref/zref-nextpage.sty\\
%   zref-pageattr.sty & tex/latex/zref/zref-pageattr.sty\\
%   zref-pagelayout.sty & tex/latex/zref/zref-pagelayout.sty\\
%   zref-perpage.sty & tex/latex/zref/zref-perpage.sty\\
%   zref-runs.sty & tex/latex/zref/zref-runs.sty\\
%   zref-savepos.sty & tex/latex/zref/zref-savepos.sty\\
%   zref-thepage.sty & tex/latex/zref/zref-thepage.sty\\
%   zref-titleref.sty & tex/latex/zref/zref-titleref.sty\\
%   zref-totpages.sty & tex/latex/zref/zref-totpages.sty\\
%   zref-user.sty & tex/latex/zref/zref-user.sty\\
%   zref-xr.sty & tex/latex/zref/zref-xr.sty\\
%   zref.pdf & doc/latex/zref/zref.pdf\\
%   zref-example.tex & doc/latex/zref/zref-example.tex\\
%   zref-example-lastpage.tex & doc/latex/zref/zref-example-lastpage.tex\\
%   zref-example-nextpage.tex & doc/latex/zref/zref-example-nextpage.tex\\
%   zref.dtx & source/latex/zref/zref.dtx\\
% \end{tabular}^^A
% }^^A
% \sbox0{\t}^^A
% \ifdim\wd0>\linewidth
%   \begingroup
%     \advance\linewidth by\leftmargin
%     \advance\linewidth by\rightmargin
%   \edef\x{\endgroup
%     \def\noexpand\lw{\the\linewidth}^^A
%   }\x
%   \def\lwbox{^^A
%     \leavevmode
%     \hbox to \linewidth{^^A
%       \kern-\leftmargin\relax
%       \hss
%       \usebox0
%       \hss
%       \kern-\rightmargin\relax
%     }^^A
%   }^^A
%   \ifdim\wd0>\lw
%     \sbox0{\small\t}^^A
%     \ifdim\wd0>\linewidth
%       \ifdim\wd0>\lw
%         \sbox0{\footnotesize\t}^^A
%         \ifdim\wd0>\linewidth
%           \ifdim\wd0>\lw
%             \sbox0{\scriptsize\t}^^A
%             \ifdim\wd0>\linewidth
%               \ifdim\wd0>\lw
%                 \sbox0{\tiny\t}^^A
%                 \ifdim\wd0>\linewidth
%                   \lwbox
%                 \else
%                   \usebox0
%                 \fi
%               \else
%                 \lwbox
%               \fi
%             \else
%               \usebox0
%             \fi
%           \else
%             \lwbox
%           \fi
%         \else
%           \usebox0
%         \fi
%       \else
%         \lwbox
%       \fi
%     \else
%       \usebox0
%     \fi
%   \else
%     \lwbox
%   \fi
% \else
%   \usebox0
% \fi
% \end{quote}
% If you have a \xfile{docstrip.cfg} that configures and enables \docstrip's
% TDS installing feature, then some files can already be in the right
% place, see the documentation of \docstrip.
%
% \subsection{Refresh file name databases}
%
% If your \TeX~distribution
% (\TeX\,Live, \mikTeX, \dots) relies on file name databases, you must refresh
% these. For example, \TeX\,Live\ users run \verb|texhash| or
% \verb|mktexlsr|.
%
% \subsection{Some details for the interested}
%
% \paragraph{Unpacking with \LaTeX.}
% The \xfile{.dtx} chooses its action depending on the format:
% \begin{description}
% \item[\plainTeX:] Run \docstrip\ and extract the files.
% \item[\LaTeX:] Generate the documentation.
% \end{description}
% If you insist on using \LaTeX\ for \docstrip\ (really,
% \docstrip\ does not need \LaTeX), then inform the autodetect routine
% about your intention:
% \begin{quote}
%   \verb|latex \let\install=y% \iffalse meta-comment
%
% File: zref.dtx
% Version: 2020-07-03 v2.32
% Info: A new reference scheme for LaTeX
%
% Copyright (C)
%    2006-2012  Heiko Oberdiek
%    2016-2020  Oberdiek Package Support Group
%    https://github.com/ho-tex/zref/issues
%
% This work may be distributed and/or modified under the
% conditions of the LaTeX Project Public License, either
% version 1.3c of this license or (at your option) any later
% version. This version of this license is in
%    https://www.latex-project.org/lppl/lppl-1-3c.txt
% and the latest version of this license is in
%    https://www.latex-project.org/lppl.txt
% and version 1.3 or later is part of all distributions of
% LaTeX version 2005/12/01 or later.
%
% This work has the LPPL maintenance status "maintained".
%
% The Current Maintainers of this work are
% Heiko Oberdiek and the Oberdiek Package Support Group
% https://github.com/ho-tex/zref/issues
%
% This work consists of the main source file zref.dtx
% and the derived files
%    zref.sty, zref.pdf, zref.ins, zref.drv, zref-base.sty,
%    zref-abspage.sty, zref-abspos.sty, zref-counter.sty,
%    zref-dotfill.sty, zref-env.sty, zref-hyperref.sty,
%    zref-lastpage.sty, zref-marks.sty, zref-nextpage.sty,
%    zref-pageattr.sty, zref-pagelayout.sty, zref-perpage.sty,
%    zref-runs.sty, zref-savepos.sty, zref-thepage.sty,
%    zref-titleref.sty, zref-totpages.sty, zref-user.sty,
%    zref-xr.sty, zref-example.tex, zref-example-lastpage.tex,
%    zref-example-nextpage.tex, zref-test1.tex,
%    zref-test-base.tex, zref-test-runs.tex,
%    zref-test-titleref-memoir.tex.
%
% Distribution:
%    CTAN:macros/latex/contrib/zref/zref.dtx
%    CTAN:macros/latex/contrib/zref/zref.pdf
%
% Unpacking:
%    (a) If zref.ins is present:
%           tex zref.ins
%    (b) Without zref.ins:
%           tex zref.dtx
%    (c) If you insist on using LaTeX
%           latex \let\install=y% \iffalse meta-comment
%
% File: zref.dtx
% Version: 2020-07-03 v2.32
% Info: A new reference scheme for LaTeX
%
% Copyright (C)
%    2006-2012  Heiko Oberdiek
%    2016-2020  Oberdiek Package Support Group
%    https://github.com/ho-tex/zref/issues
%
% This work may be distributed and/or modified under the
% conditions of the LaTeX Project Public License, either
% version 1.3c of this license or (at your option) any later
% version. This version of this license is in
%    https://www.latex-project.org/lppl/lppl-1-3c.txt
% and the latest version of this license is in
%    https://www.latex-project.org/lppl.txt
% and version 1.3 or later is part of all distributions of
% LaTeX version 2005/12/01 or later.
%
% This work has the LPPL maintenance status "maintained".
%
% The Current Maintainers of this work are
% Heiko Oberdiek and the Oberdiek Package Support Group
% https://github.com/ho-tex/zref/issues
%
% This work consists of the main source file zref.dtx
% and the derived files
%    zref.sty, zref.pdf, zref.ins, zref.drv, zref-base.sty,
%    zref-abspage.sty, zref-abspos.sty, zref-counter.sty,
%    zref-dotfill.sty, zref-env.sty, zref-hyperref.sty,
%    zref-lastpage.sty, zref-marks.sty, zref-nextpage.sty,
%    zref-pageattr.sty, zref-pagelayout.sty, zref-perpage.sty,
%    zref-runs.sty, zref-savepos.sty, zref-thepage.sty,
%    zref-titleref.sty, zref-totpages.sty, zref-user.sty,
%    zref-xr.sty, zref-example.tex, zref-example-lastpage.tex,
%    zref-example-nextpage.tex, zref-test1.tex,
%    zref-test-base.tex, zref-test-runs.tex,
%    zref-test-titleref-memoir.tex.
%
% Distribution:
%    CTAN:macros/latex/contrib/zref/zref.dtx
%    CTAN:macros/latex/contrib/zref/zref.pdf
%
% Unpacking:
%    (a) If zref.ins is present:
%           tex zref.ins
%    (b) Without zref.ins:
%           tex zref.dtx
%    (c) If you insist on using LaTeX
%           latex \let\install=y% \iffalse meta-comment
%
% File: zref.dtx
% Version: 2020-07-03 v2.32
% Info: A new reference scheme for LaTeX
%
% Copyright (C)
%    2006-2012  Heiko Oberdiek
%    2016-2020  Oberdiek Package Support Group
%    https://github.com/ho-tex/zref/issues
%
% This work may be distributed and/or modified under the
% conditions of the LaTeX Project Public License, either
% version 1.3c of this license or (at your option) any later
% version. This version of this license is in
%    https://www.latex-project.org/lppl/lppl-1-3c.txt
% and the latest version of this license is in
%    https://www.latex-project.org/lppl.txt
% and version 1.3 or later is part of all distributions of
% LaTeX version 2005/12/01 or later.
%
% This work has the LPPL maintenance status "maintained".
%
% The Current Maintainers of this work are
% Heiko Oberdiek and the Oberdiek Package Support Group
% https://github.com/ho-tex/zref/issues
%
% This work consists of the main source file zref.dtx
% and the derived files
%    zref.sty, zref.pdf, zref.ins, zref.drv, zref-base.sty,
%    zref-abspage.sty, zref-abspos.sty, zref-counter.sty,
%    zref-dotfill.sty, zref-env.sty, zref-hyperref.sty,
%    zref-lastpage.sty, zref-marks.sty, zref-nextpage.sty,
%    zref-pageattr.sty, zref-pagelayout.sty, zref-perpage.sty,
%    zref-runs.sty, zref-savepos.sty, zref-thepage.sty,
%    zref-titleref.sty, zref-totpages.sty, zref-user.sty,
%    zref-xr.sty, zref-example.tex, zref-example-lastpage.tex,
%    zref-example-nextpage.tex, zref-test1.tex,
%    zref-test-base.tex, zref-test-runs.tex,
%    zref-test-titleref-memoir.tex.
%
% Distribution:
%    CTAN:macros/latex/contrib/zref/zref.dtx
%    CTAN:macros/latex/contrib/zref/zref.pdf
%
% Unpacking:
%    (a) If zref.ins is present:
%           tex zref.ins
%    (b) Without zref.ins:
%           tex zref.dtx
%    (c) If you insist on using LaTeX
%           latex \let\install=y\input{zref.dtx}
%        (quote the arguments according to the demands of your shell)
%
% Documentation:
%    (a) If zref.drv is present:
%           latex zref.drv
%    (b) Without zref.drv:
%           latex zref.dtx; ...
%    The class ltxdoc loads the configuration file ltxdoc.cfg
%    if available. Here you can specify further options, e.g.
%    use A4 as paper format:
%       \PassOptionsToClass{a4paper}{article}
%
%    Programm calls to get the documentation (example):
%       pdflatex zref.dtx
%       makeindex -s gind.ist zref.idx
%       pdflatex zref.dtx
%       makeindex -s gind.ist zref.idx
%       pdflatex zref.dtx
%
% Installation:
%    TDS:tex/latex/zref/zref.sty
%    TDS:tex/latex/zref/zref-base.sty
%    TDS:tex/latex/zref/zref-abspage.sty
%    TDS:tex/latex/zref/zref-abspos.sty
%    TDS:tex/latex/zref/zref-counter.sty
%    TDS:tex/latex/zref/zref-dotfill.sty
%    TDS:tex/latex/zref/zref-env.sty
%    TDS:tex/latex/zref/zref-hyperref.sty
%    TDS:tex/latex/zref/zref-lastpage.sty
%    TDS:tex/latex/zref/zref-marks.sty
%    TDS:tex/latex/zref/zref-nextpage.sty
%    TDS:tex/latex/zref/zref-pageattr.sty
%    TDS:tex/latex/zref/zref-pagelayout.sty
%    TDS:tex/latex/zref/zref-perpage.sty
%    TDS:tex/latex/zref/zref-runs.sty
%    TDS:tex/latex/zref/zref-savepos.sty
%    TDS:tex/latex/zref/zref-thepage.sty
%    TDS:tex/latex/zref/zref-titleref.sty
%    TDS:tex/latex/zref/zref-totpages.sty
%    TDS:tex/latex/zref/zref-user.sty
%    TDS:tex/latex/zref/zref-xr.sty
%    TDS:doc/latex/zref/zref.pdf
%    TDS:doc/latex/zref/zref-example.tex
%    TDS:doc/latex/zref/zref-example-lastpage.tex
%    TDS:doc/latex/zref/zref-example-nextpage.tex
%    TDS:source/latex/zref/zref.dtx
%
%<*ignore>
\begingroup
  \catcode123=1 %
  \catcode125=2 %
  \def\x{LaTeX2e}%
\expandafter\endgroup
\ifcase 0\ifx\install y1\fi\expandafter
         \ifx\csname processbatchFile\endcsname\relax\else1\fi
         \ifx\fmtname\x\else 1\fi\relax
\else\csname fi\endcsname
%</ignore>
%<*install>
\input docstrip.tex
\Msg{************************************************************************}
\Msg{* Installation}
\Msg{* Package: zref 2020-07-03 v2.32 A new reference scheme for LaTeX (HO)}
\Msg{************************************************************************}

\keepsilent
\askforoverwritefalse

\let\MetaPrefix\relax
\preamble

This is a generated file.

Project: zref
Version: 2020-07-03 v2.32

Copyright (C)
  2006-2012 Heiko Oberdiek
  2016-2020 Oberdiek Package Support Group

This work may be distributed and/or modified under the
conditions of the LaTeX Project Public License, either
version 1.3c of this license or (at your option) any later
version. This version of this license is in
   https://www.latex-project.org/lppl/lppl-1-3c.txt
and the latest version of this license is in
   https://www.latex-project.org/lppl.txt
and version 1.3 or later is part of all distributions of
LaTeX version 2005/12/01 or later.

This work has the LPPL maintenance status "maintained".

The Current Maintainers of this work are
Heiko Oberdiek and the Oberdiek Package Support Group
https://github.com/ho-tex/zref/issues


This work consists of the main source file zref.dtx
and the derived files
   zref.sty, zref.pdf, zref.ins, zref.drv, zref-base.sty,
   zref-abspage.sty, zref-abspos.sty, zref-counter.sty,
   zref-dotfill.sty, zref-env.sty, zref-hyperref.sty,
   zref-lastpage.sty, zref-marks.sty, zref-nextpage.sty,
   zref-pageattr.sty, zref-pagelayout.sty, zref-perpage.sty,
   zref-runs.sty, zref-savepos.sty, zref-thepage.sty,
   zref-titleref.sty, zref-totpages.sty, zref-user.sty,
   zref-xr.sty, zref-example.tex, zref-example-lastpage.tex,
   zref-example-nextpage.tex, zref-test1.tex,
   zref-test-base.tex, zref-test-runs.tex,
   zref-test-titleref-memoir.tex.

\endpreamble
\let\MetaPrefix\DoubleperCent

\generate{%
  \file{zref.ins}{\from{zref.dtx}{install}}%
  \file{zref.drv}{\from{zref.dtx}{driver}}%
  \usedir{tex/latex/zref}%
  \file{zref.sty}{\from{zref.dtx}{package}}%
  \file{zref-base.sty}{\from{zref.dtx}{base}}%
  \file{zref-abspage.sty}{\from{zref.dtx}{abspage}}%
  \file{zref-abspos.sty}{\from{zref.dtx}{abspos}}%
  \file{zref-counter.sty}{\from{zref.dtx}{counter}}%
  \file{zref-dotfill.sty}{\from{zref.dtx}{dotfill}}%
  \file{zref-env.sty}{\from{zref.dtx}{env}}%
  \file{zref-hyperref.sty}{\from{zref.dtx}{hyperref}}%
  \file{zref-lastpage.sty}{\from{zref.dtx}{lastpage}}%
  \file{zref-marks.sty}{\from{zref.dtx}{marks}}%
  \file{zref-nextpage.sty}{\from{zref.dtx}{nextpage}}%
  \file{zref-pageattr.sty}{\from{zref.dtx}{pageattr}}%
  \file{zref-pagelayout.sty}{\from{zref.dtx}{pagelayout}}%
  \file{zref-perpage.sty}{\from{zref.dtx}{perpage}}%
  \file{zref-runs.sty}{\from{zref.dtx}{runs}}%
  \file{zref-savepos.sty}{\from{zref.dtx}{savepos}}%
  \file{zref-thepage.sty}{\from{zref.dtx}{thepage}}%
  \file{zref-titleref.sty}{\from{zref.dtx}{titleref}}%
  \file{zref-totpages.sty}{\from{zref.dtx}{totpages}}%
  \file{zref-user.sty}{\from{zref.dtx}{user}}%
  \file{zref-xr.sty}{\from{zref.dtx}{xr}}%
  \usedir{doc/latex/zref}%
  \file{zref-example.tex}{\from{zref.dtx}{example}}%
  \file{zref-example-lastpage.tex}{\from{zref.dtx}{example-lastpage}}%
  \file{zref-example-nextpage.tex}{\from{zref.dtx}{example-nextpage}}%
%  \usedir{doc/latex/zref/test}%
%  \file{zref-test1.tex}{\from{zref.dtx}{test1}}%
%  \file{zref-test-base.tex}{\from{zref.dtx}{test-base}}%
%  \file{zref-test-runs.tex}{\from{zref.dtx}{test-runs}}%
%  \file{zref-test-titleref-memoir.tex}{\from{zref.dtx}{test-titleref-memoir}}%
}

\catcode32=13\relax% active space
\let =\space%
\Msg{************************************************************************}
\Msg{*}
\Msg{* To finish the installation you have to move the following}
\Msg{* files into a directory searched by TeX:}
\Msg{*}
\Msg{*     zref.sty, zref-base.sty, zref-abspage.sty, zref-abspos.sty,}
\Msg{*     zref-counter.sty, zref-dotfill.sty, zref-env.sty,}
\Msg{*     zref-hyperref.sty, zref-lastpage.sty, zref-marks.sty,}
\Msg{*     zref-nextpage.sty, zref-pageattr.sty, zref-pagelayout.sty,}
\Msg{*     zref-perpage.sty, zref-runs.sty, zref-savepos.sty,}
\Msg{*     zref-thepage.sty, zref-titleref.sty, zref-totpages.sty,}
\Msg{*     zref-user.sty, zref-xr.sty}
\Msg{*}
\Msg{* To produce the documentation run the file `zref.drv'}
\Msg{* through LaTeX.}
\Msg{*}
\Msg{* Happy TeXing!}
\Msg{*}
\Msg{************************************************************************}

\endbatchfile
%</install>
%<*ignore>
\fi
%</ignore>
%<*driver>
\NeedsTeXFormat{LaTeX2e}
\ProvidesFile{zref.drv}%
  [2020-07-03 v2.32 A new reference scheme for LaTeX (HO)]%
\documentclass{ltxdoc}
\usepackage{holtxdoc}[2011/11/22]
\setlength{\emergencystretch}{1em}
\begin{document}
  \DocInput{zref.dtx}%
\end{document}
%</driver>
% \fi
%
%
%
% \GetFileInfo{zref.drv}
%
% \title{The \xpackage{zref} package}
% \date{2020-07-03 v2.32}
% \author{Heiko Oberdiek\thanks
% {Please report any issues at \url{https://github.com/ho-tex/zref/issues}}}
%
% \maketitle
%
% \begin{abstract}
% Package \xpackage{zref} tries to get rid of the restriction
% in \hologo{LaTeX}'s reference system that only two properties are
% supported. The package implements an extensible referencing
% system, where properties are handled in a more flexible way.
% It offers an interface for macro programmers for the access
% to the system and some applications that uses the new
% reference scheme.
% \end{abstract}
%
% \tableofcontents
%
% \newcommand*{\metacs}[1]{\texttt{\itshape #1}}
%
% \newcommand{\markexp}{^^A
%   \textsuperscript{exp}^^A
%   \space
% }
% \newcommand{\markexptwo}{^^A
%   \textsuperscript{exp2}^^A
%   \space
% }
% \newcommand{\markrobust}{^^A
%   \textsuperscript{robust}^^A
%   \space
% }
% \newcommand{\marknowarn}{^^A
%   \textsuperscript{nowarn}^^A
%   \space
% }
% \newcommand{\markbabel}{^^A
%   \textsuperscript{babel}^^A
%   \space
% }
%
% \section{Introduction}
%
% Standard \hologo{LaTeX}'s reference system with \cs{label}, \cs{ref},
% and \cs{pageref} supports two properties, the apperance of
% the counter that is last incremented by \cs{refstepcounter} and
% the page with the \cs{label} command.
%
% Unhappily \hologo{LaTeX} does not provide an interface for adding
% another properties. Packages such as \xpackage{hyperref},
% \xpackage{nameref}, or \xpackage{titleref} are forced to
% use ugly hacks to extend the reference system.
% These ugly hacks are one of the causes for \xpackage{hyperref}'s
% difficulty regarding compatibility with other packages.
%
% \subsection{Standard \hologo{LaTeX} behaviour}
%
% References are created by the \cs{label} command:
% \begin{quote}
%   |\chapter{Second chapter}|\\
%   |\section{First section on page 7} % section 2.1|\\
%   |\label{myref}|
% \end{quote}
% Now \hologo{LaTeX} records the section number |2.1| and the page |7|
% in the reference. Internally the reference is a list with
% two entries:
% \begin{quote}
%   |\r@myref| $\rightarrow$ |{2.1}{7}|
% \end{quote}
% The length of the list if fixed in the \hologo{LaTeX} kernel,
% An interface for adding new properties is missing.
%
% There are several tries to add new properties:
% \begin{description}
% \item[\xpackage{hyperref}] uses a list of five
%   properties instead of the standard list with two entries.
%   This causes many compatibility problems with \hologo{LaTeX} and
%   other packages.
% \item[\xpackage{titleref}] stores its title data into the
%   first entry in the list. \hologo{LaTeX} is happy because it
%   does only see its list with two entries. The situation
%   becomes more difficult, if more properties are added
%   this way. Then the macros form a nested structure
%   inside the first reference argument for the label.
%   Expandable extractions will then become painful.
% \end{description}
%
% \subsection{Basic idea}
%
% Some time ago Morten H\o gholm sent me an experimental
% cross referencing mechanism as ``expl3'' code. His
% idea is:
% \begin{quote}
%   |\g_xref_mylabel_plist| $\rightarrow$\\
%   \mbox{}\qquad|\xref_dance_key{salsa}\xref_name_key{Morten}|\dots
% \end{quote}
% The entries have the following format:
% \begin{quote}
%   |\xref_|\meta{your key}|_key{|\meta{some text}|}|
% \end{quote}
% This approach is much more flexible:
% \begin{itemize}
%   \item New properties can easily be added, just use a new key.
%   \item The length of the list is not fixed. A reference
%         can use a subset of the keys.
%   \item The order of the entries does not matter.
% \end{itemize}
% Unhappily I am not familiar with the experimental code
% for \hologo{LaTeX3} that will need some time before its first release.
% Thus I have implemented it as \hologo{LaTeXe} package without disturbing
% the existing \hologo{LaTeX} reference system.
%
% \subsection{Interfaces}
%
% The package provides a generic \emph{interface for programmers}.
% Commands of this interface are prefixed by \cs{zref@}.
%
% Option \xoption{user} enabels the \emph{user interface}.
% Here the commands are prefixed by \cs{z} to avoid name
% clashes with existing macros.
%
% Then the packages provides some \emph{modules}. They
% are applications for the reference system and can also be
% considered as examples how to use the reference system.
%
% The modules can be loaded as packages. The package
% name is prefixed with |zref-|, for example:
%\begin{quote}
%\begin{verbatim}
%\RequirePackage{zref-abspage}
%\end{verbatim}
%\end{quote}
% This is the preferred way if the package is loaded from
% within other packages to avoid option clashes.
%
% As alternative package |zref| can be used and the modules
% are given as options:
%\begin{quote}
%\begin{verbatim}
%\usepackage[perpage,user]{zref}
%\end{verbatim}
%\end{quote}
%
% \section{Interface for programmers}
%
% The user interface is described in the next section \ref{sec:user}.
%
% \subsection{Entities}
%
% \paragraph{Reference.}
% Internally a reference is a list of key value pairs:
% \begin{quote}
%   |\Z@R@myref| $\rightarrow$
%   |\default{2.1}\page{7}|
% \end{quote}
% The generic format of a entry is:
% \begin{quote}
%   |\Z@R@|\meta{refname} $\rightarrow$
%   |\|\meta{propname}|{|\meta{value}|}|
% \end{quote}
% \meta{refname} is the name that denoted references
% (the name used in \cs{label} and \cs{ref}). \meta{propname}
% is the name of the property or key. The property key macro
% is never executed, it is used in parameter text matching only.
%
% \paragraph{Property.}
% Because the name of a property is used in a macro name
% that must survive the \xfile{.aux} file, the name is
% restricted to letters and `@'.
%
% \paragraph{Property list.}
% Often references are used for special purposes. Thus it
% saves memory if just the properties are used in this
% reference that are necessary for its purpose.
%
% Therefore this package uses the concept of \emph{property lists}.
% A property list is a set of properties. The set of properties
% that is used by the default \cs{label} command is the
% \emph{main property list}.
%
% \subsection{Property list}
%
% \markexp means that the implementation of the
% marked macro is expandable. \markexptwo goes a step further
% and marks the macro expandable in exact two expansion steps.
%
% \begin{declcs}{zref@newlist}
%   \M{listname}
% \end{declcs}
% Declares a new empty property list.
%
% \begin{declcs}{zref@addprop}
%   \M{listname} \M{propname}\\
%   \cs{zref@localaddprop} \M{listname} \M{propname}
% \end{declcs}
% Adds the property \meta{propname}
% to the property list \meta{listname}.
% The property and list must exist.
% The addition is global by \cs{zref@addprop} and limited to
% local scope by \cs{zref@localaddprop}.
% Between 2010/04/19 v2.13 and 2010/10/22 v2.19 a comma separated list
% of properties could be used as argument \meta{propname}.
% Since 2010/10/22 v2.19 the addition of several properties
% at once is supported by \cs{zref@addprops}.
%
% \begin{declcs}{zref@addprops}
%   \M{listname} \M{propname list}\\
%   \cs{zref@localaddprops} \M{listname} \M{propname list}
% \end{declcs}
% These macros add a comma separated list of properties
% \meta{propname list} to list \meta{listname}.
% \cs{zref@addprops} works globally and \cs{zref@localaddprops}
% locally. Since 2010/10/22 v2.19.
%
% \begin{declcs}{zref@listexists}
%   \M{listname} \M{then}
% \end{declcs}
% Executes \meta{then} if the property list \meta{listname} exists
% or raise an error otherwise.
%
% \begin{declcs}{zref@iflistundefined}\markexp \M{listname}
%   \M{then} \M{else}
% \end{declcs}
% Executes \meta{then} if the list exists or \meta{else}
% otherwise.
%
% \begin{declcs}{zref@iflistcontainsprop}
%   \M{listname} \M{propname} \M{then} \M{else}
% \end{declcs}
% Executes \meta{then} if the property \meta{propname} is part
% of property list \meta{listname} or otherwise it
% runs the \meta{else} part.
%
% \subsection{Property}
%
% \begin{declcs}{zref@newprop}
%   \* \M{propname} |[|\meta{default}|]| \M{value}
% \end{declcs}
% This command declares and configures a new property with
% name \meta{propname}.
%
% In case of unknown references
% or the property does not exist in the reference, the
% \meta{default} is used as value. If it is not specified
% here, a global default is used, see \cs{zref@setdefault}.
%
% The correct values of some properties are not known
% immediately but at page shipout time. Prominent example
% is the page number. These properties are declared with
% the star form of the command.
%
% \begin{declcs}{zref@setcurrent}
%   \M{propname} \M{value}
% \end{declcs}
% This sets the current value of the property \meta{propname}.
% It is a generalization of setting \hologo{LaTeX}'s \cs{currentlabel}.
%
% \begin{declcs}{zref@getcurrent}\markexptwo
%   \M{propname}
% \end{declcs}
% This returns the current value of the property \meta{propname}.
% The value may not be correct, especially if the property
% is bound to a page (start form of \cs{zref@newprop}) and the
% right value is only known at shipout time (e.g. property `page').
% In case of errors (e.g. unknown property) the empty string is
% returned.
%
% Since version 2010/04/22 v2.14 \cs{zref@getcurrent} supports
% \cs{zref@wrapper@unexpanded}.
%
% \begin{declcs}{zref@propexists}
%   \M{propname} \M{then}
% \end{declcs}
% Calls \meta{then} if the property \meta{propname} is available
% or generates an error message otherwise.
%
% \begin{declcs}{zref@ifpropundefined}\markexp
%   \M{propname} \M{then} \M{else}
% \end{declcs}
% Calls \meta{then} or \meta{else} depending on the existence
% of property \meta{propname}.
%
% \subsection{Reference generation}
%
% \begin{declcs}{zref@label}
%   \M{refname}
% \end{declcs}
% This works similar to \cs{label}. The reference \meta{refname} is
% created and put into the \xfile{.aux} file with the
% properties of the main property list.
%
% \begin{declcs}{zref@labelbylist}
%   \M{refname} \M{listname}
% \end{declcs}
% Same as \cs{zref@label} except that the properties are
% taken from the specified property list \meta{listname}.
%
% \begin{declcs}{zref@labelbyprops}
%   \M{refname} |{|\meta{propnameA}|,|\meta{propnameB}|,|\dots|}|
% \end{declcs}
% Same as \cs{zref@label} except that these properties
% are used that are given as comma separated list in the
% second argument.
%
% \begin{declcs}{zref@newlabel}
%   \M{refname} |{|\dots|}|
% \end{declcs}
% This is the macro that is used in the \xfile{.aux} file.
% It is basically the same as \cs{newlabel} apart from
% the format of the data in the second argument.
%
% \subsection{Data extraction}
%
% \begin{declcs}{zref@extractdefault}\markexptwo
%   \M{refname} \M{propname} \M{default}
% \end{declcs}
% This is the basic command that refernces the value of a
% property \meta{propname} for the reference \meta{refname}.
% In case of errors such as undefined reference the \meta{default}
% is used instead.
%
% \begin{declcs}{zref@extract}\markexptwo
%   \M{refname} \M{propname}
% \end{declcs}
% The command is an abbreviation for \cs{zref@extractdefault}.
% As default the default of the property is taken, otherwise
% the global default.
%
% Example for page references:
% \begin{quote}
% \begin{tabular}{@{}ll}
% \hologo{LaTeX}:& |\pageref{foobar}|\\
% \xpackage{zref}:& |\zref@extract{foobar}{page}|
% \end{tabular}
% \end{quote}
% Both \cs{zref@extract} and \cs{zref@extractdefault} are
% expandable. That means, these macros can directly
% be used in expandable calculations, see the example file.
% On the other side, babel's shorthands are not supported,
% there are no warnings in case of undefined references.
%
% If an user interface doesn't need expandable macros
% then it can use \cs{zref@refused} and \cs{zref@wrapper@babel}
% for its user macros.
%
% \begin{declcs}{zref@refused}
%   \M{refname}
% \end{declcs}
% This command is not expandable. It causes the warnings if
% the reference \meta{refname} is not defined. Use the
% \cs{zref@extract} commands inside expandable contexts and
% mark their use outside by \cs{zref@refused}, see the example
% file.
%
% \begin{declcs}{zref@def@extract}%
%     \M{cmd} \M{refname} \M{propname}\\
%   \cs{zref@def@extractdefault}
%     \M{cmd} \M{refname} \M{propname} \M{default}
% \end{declcs}
% Both macros extract the property \meta{propname} from the reference
% \meta{refname} the same way as macros \cs{zref@extract} and
% \cs{zref@extractdefault}. The result is stored in macro \meta{cmd}.
% Also \cs{zref@refused} is called to notify \hologo{LaTeX} that
% the reference \meta{refname} is used. Added in 2011/10/04 v2.22.
%
% \begin{declcs}{zref@ifrefundefined}\markexp
%   \M{refname} \M{then} \M{else}
% \end{declcs}
% Macro \cs{zref@ifrefundefined} calls arguments \meta{then} or \meta{else}
% dependent on the existence of the reference \meta{refname}.
%
% \begin{declcs}{zifrefundefined}
%   \M{refname} \M{then} \M{else}
% \end{declcs}
% Macro \cs{zifrefundefined} calls \cs{ref@refused} before
% executing \cs{zref@ifrefundefined}. Babel shorthands are
% supported in \meta{refname}.
%
% \begin{declcs}{zref@ifrefcontainsprop}\markexp
%   \M{refname} \M{propname} \M{then} \M{else}
% \end{declcs}
% Test whether a reference provides a property.
%
% \subsection{Setup}
%
% \begin{declcs}{zref@default}
% \end{declcs}
% Holds the global default for unknown values.
%
% \begin{declcs}{zref@setdefault}
%   |{|\meta{value}|}|
% \end{declcs}
% Sets the global default for unknown values.
% The global default is used, if a property does not specify
% an own default and the value for a property
% cannot be extracted. This can happen if the reference is
% unknown or the reference does not have the property.
%
% \begin{declcs}{zref@setmainlist}
%   |{|\meta{value}|}|
% \end{declcs}
% Sets the name of the main property list. The package
% sets and uses |main|.
%
% \subsection{Declared properties}
%
% \begin{quote}
% \renewcommand*{\arraystretch}{1.05}
% \setlength{\tabcolsep}{2\tabcolsep}
% \def\VV{\vphantom{\large \"Ag}}
% \def\EMPTY{{\itshape\textless empty\textgreater}}
% \def\pl#1{&#1&thepage&\cs{number}\cs{#1}}
% \begin{tabular}{@{}>{\sffamily}l|>{\ttfamily}l|>{\ttfamily}l|>{\ttfamily}l@{}}
%   \rmfamily Module&
%   \rmfamily Property&
%   \rmfamily Property list&
%   \rmfamily Default\\
%   \hline
%   (base)
%     & default & main & \EMPTY\\
%     & page & main & \EMPTY\\
%   \hline
%   abspage
%     & abspage & main & 0\\
%   \hline
%   counter
%     & counter & main & \EMPTY\\
%   \hline
%   hyperref
%     & anchor & main & \EMPTY\\
%     & url    & & \EMPTY\\
%   \hline
%   pageattr
%     & pdfpageattr & thepage & \dots\\
%     & pdfpagesattr & LastPage & \dots\\
%   \hline
%   pagelayout\footnotemark
%     \pl{mag}\\
%     \pl{paperwidth}\\
%     \pl{paperheight}\\
%     \pl{stockwidth}\\
%     \pl{stockheight}\\
%     \pl{pdfpageheight}\\
%     \pl{pdfpagewidth}\\
%     \pl{pdfhorigin}\\
%     \pl{pdfvorigin}\\
%     \pl{hoffset}\\
%     \pl{voffset}\\
%     \pl{topmargin}\\
%     \pl{oddsidemargin}\\
%     \pl{evensidemargin}\\
%     \pl{textwidth}\\
%     \pl{textheight}\\
%     \pl{headheight}\\
%     \pl{headsep}\\
%     \pl{footskip}\\
%     \pl{marginparwidth}\\
%     \pl{marginparsep}\\
%     \pl{columnwidth}\\
%     \pl{columnsep}\\
%   \hline
%   perpage
%     & pagevalue & perpage & 0\\
%     & page & perpage & \EMPTY\\
%     & abspage & perpage & 0\\
%   \hline
%   savepos
%     & posx & savepos & 0\\
%     & posy & savepos & 0\\
%   \hline
%   titleref
%     & title & main & \EMPTY\\
%   \hline
%   xr
%     & anchor   & & \EMPTY\\
%     & externaldocument & & \EMPTY\\
%     & theotype & & \EMPTY\\
%     & title    & & \EMPTY\\
%     & url      & & \EMPTY\\
% \end{tabular}
% \footnotetext{Module \xmodule{pagelayout} only defines properties
% if the parameter exists.}%
% \end{quote}
%
% \subsection{Wrapper for advanced situations}
%
% \begin{declcs}{zref@wrapper@babel}
%   |{|\dots|}| \M{name}
% \end{declcs}
% This macro helps to add shorthand support. The second
% argument is protected, then the code of the first argument
% is called with the protected name appended. Examples
% are in the sources.
%
% \begin{declcs}{zref@wrapper@immediate}
%   |{|\dots|}|
% \end{declcs}
% There are situations where a label must be written instantly
% to the \xfile{.aux} file, for example after the last page.
% If the \cs{zlabel} or \cs{label} command is put inside this wrapper,
% immediate writing is enabled. See the implementation for
% module \xmodule{lastpage} for an example of its use.
%
% \begin{declcs}{zref@wrapper@unexpanded}
%   |{|\dots|}|
% \end{declcs}
% Assuming someone wants to extract a value for property |bar|
% and store the result in a macro |\foo| without traces of
% the expanding macros and without expanding the value.
% This (theoretical?) problem can be solved by this wrapper:
% \begin{quote}
%\begin{verbatim}
%\zref@wrapper@unexpanded{%
%  \edef\foo{%
%    \zref@extract{someref}{bar}%
%  }%
%}
%\end{verbatim}
% \end{quote}
% The \cs{edef} forces the expansion of \cs{zref@extract},
% but the extraction of the value is prevented by
% the wrapper that uses \hologo{eTeX}'\ \cs{unexpanded} for this
% purpose. Supported macros are \cs{zref@extract}, \cs{zref@extractdefault}
% and since version 2010/04/22 v2.14 macro \cs{zref@getcurrent}.
%
% \subsection{Counter for unique names}
%
% Some modules (\xmodule{titleref} and \xmodule{dotfillmin})
% need unique names for automatically generated label names.
%
% \begin{declcs}{zref@require@unique}
% \end{declcs}
% This command creates the unique counter |zref@unique| if the
% counter does not already exist.
%
% \begin{declcs}{thezref@unique}
% \end{declcs}
% This command is used to generate unique label names.
%
% \section{User interface}\label{sec:user}
%
% \subsection{Module \xmodule{user}}
%
% The user interface for this package and its modules
% is enabled by \xpackage{zref}'s package option \xmodule{user}
% or package \xpackage{zref-user}.
% The names of user commands are prefixed by |z| in order
% to avoid name clashes with existing macros of the same
% functionality. Thus the package does not disturb the
% traditional reference scheme, both can be used together.
%
% The syntax descriptions contain the following markers
% that are intended as hints for programmers:
% \begin{quote}
% \begin{tabular}{@{}ll@{}}
%   \markbabel\unskip  & Babel shorthands are allowed.\\
%   \markrobust\unskip & Robust macro.\\
%   \markexp\unskip    & Expandable version:\\
%                      & \textbullet\ robust,
%                        unless the extracted values are fragile,\\
%                      & \textbullet\ no babel shorthand suport.\\
%   \markexptwo\unskip & Expandable like \markexp and:\\
%                      & \textbullet\ expandable in exact two steps.\\
% \end{tabular}
% \end{quote}
%
% The basic user interface of the package without modules
% are commands that mimic the standard \hologo{LaTeX} behaviour
% of \cs{label}, \cs{ref}, and \cs{pageref}:
%
% \begin{declcs}{zlabel}
%   \M{refname}\markbabel
% \end{declcs}
% Similar to \cs{label}. It generates a label with name
% \meta{refname} in the new reference scheme.
%
% \begin{declcs}{zref}
%   |[|\meta{propname}|]| \M{refname}\markbabel
% \end{declcs}
% Without optional argument similar to \cs{ref}, it returns
% the default reference property. This property is named
% |default|:
% \[
%   |\zref{|x|}| \equiv |\zref[default]{|x|}|
% \]
%
% \begin{declcs}{zpageref}
%   \M{refname}\markbabel
% \end{declcs}
% Convenience macro, similar to \cs{pageref}.
% \[
%   |\zpageref{|x|}| \equiv |\zref[page]{|x|}|
% \]
%
% \begin{declcs}{zrefused}
%   \M{refname}\markbabel
% \end{declcs}
% Some of the user commands in the modules are expandable.
% The use of such commands do not cause any undefined
% reference warnings, because inside of expandable
% contexts this is not possible. However, if there is
% a place outside of expandable contexts, \cs{refused}
% is strongly recommended. The reference \meta{refname}
% is marked as used, undefined ones will generate
% warnings.
%
% \subsection{Module \xmodule{abspage}}
%
% With the help of package \xpackage{atbegshi}
% a new counter |abspage| with absolute page numbers is provided.
% For technical and historical reasons the counter itself is zero based:
% if you use it directly on the first page, e.g
% with |\arabic{abspage}| you will get 0 as value. When using |\zref|
% the first page will be page 1 as expected.
% Also a new property |abspage| is defined and added to the
% main property list. Thus you can reference the absolute
% page number:
% \begin{quote}
%   |Section \zref{foo} is on page \zpageref{foo}.|\\
%   |This is page \zref[abspage]{foo}|\\
%   |of \zref[abspage]{LastPage}.|
% \end{quote}
% The example also makes use of module \xmodule{lastpage}.
%
% \subsection{Module \xmodule{lastpage}}
%
% Provides the functionality of package \xpackage{lastpage}
% \cite{lastpage} in the new reference scheme.
% The label |LastPage| is put at the end of the document.
% You can refer the last page number with:
% \begin{quote}
%   |\zref@extract{LastPage}{page}| (+ |\zref@refused{LastPage}|)\\
% \end{quote}
% or
% \begin{quote}
%   |\zpageref{LastPage}| (module \xmodule{user})
% \end{quote}
%
% Since version 2008/10/01 v2.3 the module defines the list |LastPage|.
% In addition to the properties of the main list label |LastPage| also
% stores the properties of this list |LastPage|. The default of this
% list is empty. The list can be used by the user to add additional
% properties for label |LastPage|.
%
% \subsubsection{Tests for last page}
%
% Since version 2010/03/26 v2.8 the macros \cs{zref@iflastpage}
% and \cs{ziflastpage} were added. They test the reference,
% whether it is a reference of the last page.
%
% \begin{declcs}{zref@iflastpage}\markexp \M{refname} \M{then} \M{else}
% \end{declcs}
% Macro \cs{zref@iflastpage} compares the references \meta{refname}
% with \meta{LastPage}. Basis of the comparison is the value of
% property |abspage|, because the values are different for different
% pages. This is not ensured by property |page|. Therefore module
% \xmodule{abspage} is loaded by module \xmodule{lastpage}.
% If both values of property |abspage| are present and match,
% then \meta{then} is executed, otherwise code \meta{else} is called.
% If one or both references are undefined or lack the property |abspage|,
% then \meta{else} is executed.
%
% Macro \cs{zref@iflastpage} is expandable, therefore \cs{zref@refused}
% should be called on \meta{refname} and \meta{LastPage}.
%
% \begin{declcs}{ziflastpage} \M{refname} \M{then} \M{else}
% \end{declcs}
% Macro \cs{ziflastpage} has the same function as \cs{zref@iflastpage},
% but adds support for babel shorthands in \meta{refname} and
% calls \cs{zref@refused}. However macro \cs{ziflastpage} is not
% expandable.
%
% \subsubsection{Example}
%
%    \begin{macrocode}
%<*example-lastpage>
%<<END_EXAMPLE
\NeedsTeXFormat{LaTeX2e}
\documentclass{report}

\newcounter{foo}
\renewcommand*{\thefoo}{\Alph{foo}}

\usepackage{zref-lastpage,zref-user}[2019/11/29]

\makeatletter
\zref@newprop{thefoo}{\thefoo}
\zref@newprop{valuefoo}{\the\value{foo}}
\zref@newprop{chapter}{\thechapter}
\zref@addprops{LastPage}{thefoo,valuefoo,chapter}
\makeatother

\newcommand*{\foo}{%
  \stepcounter{foo}%
  [Current foo: \thefoo]%
}

\begin{document}
  \chapter{First chapter}
  Last page is \zref{LastPage}.\\
  Last chapter is \zref[chapter]{LastPage}.\\
  Last foo is \zref[thefoo]{LastPage}.\\
  Last value of foo is \zref[valuefoo]{LastPage}.\\
  \foo
  \chapter{Second chapter}
  \foo\foo\foo
  \chapter{Last chapter}
  \foo
\end{document}
%END_EXAMPLE
%</example-lastpage>
%    \end{macrocode}
%
% \subsection{Module \xmodule{thepage}}
%
% This module \xmodule{thepage} loads module \xmodule{abspage},
% constructs a reference name using the absolute page number
% and remembers property |page|. Other properties can be added
% by adding them to the property list |thepage|.
%
% \begin{declcs}{zthepage} \M{absolute page number}\\
% \end{declcs}
% Macro \cs{zthepage} is basically a \cs{zpageref}. The reference
% name is yield by the \meta{absolute page number}. If the reference
% is not defined, then the default for property |page| is used.
%
% \begin{declcs}{zref@thepage@name}\markexp \M{absolute page number}
% \end{declcs}
% Macro \cs{zref@thepage@name} returns the internal reference
% name that is constructed using the \meta{absolute page number}.
% The internal reference name should not be used directly, because
% it might change in future versions.
%
% \begin{declcs}{zref@thepage}\markexp \M{absolute page number}\\
%   \cs{zref@thepage@refused} \M{absolute page number}
% \end{declcs}
% Macro \cs{zref@thepage} returns the page number (\cs{thepage})
% of \meta{absolute page number}. Because this macro is expandable,
% \cs{zref@thepage@refused} is used outside an expandable context
% to mark the reference as used.
%
%
% \subsection{Module \xmodule{nextpage}}
%
% \begin{declcs}{znextpage}
% \end{declcs}
% Macro \cs{znextpage} prints \cs{thepage} of the following page.
% It gets the current absolute page number by using a label.
% There are three cases for the next page:
% \begin{enumerate}
% \item The next page is not known yet because of undefined references.
%   Then \cs{zunknownnextpagename} is used instead. The default for
%   this macro is the default of property |page|.
% \item This page is the last page. Then \cs{znonextpagename}
%   is used. Its default is empty.
% \item The next page is known, then \cs{thepage} of the next
%   page is used (the value of property |page| of the next page).
% \end{enumerate}
%
% \subsubsection{Configuration}
%
% The behaviour can be configured by the following macros.
%
% \begin{declcs}{zunknownnextpagename}\\
%   \cs{znonextpagename}
% \end{declcs}
% If the next page is not known or available, then
% \cs{znextpage} uses these name macros as default.
% \cs{zunknownnextpagename} is used in case of undefined
% references. Default is the value of property |page| of the
% next page (\cs{thepage}). Module \xmodule{thepage} is used.
%
% Macro \cs{znonextpagename} is used, if the next page does
% not exists. That means that the current page is last page.
% The default is empty.
%
% \begin{declcs}{znextpagesetup} \M{unknown} \M{no next}
%   \M{next}
% \end{declcs}
% Acording to the case (see \cs{znextpage}) macro \cs{znextpage}
% calls an internal macro with an argument. The argument is
% either \cs{thepage} of the next page or one of \cs{zunknownnextpagename}
% or \cs{znonextpagename}. These internal macro can be changed by
% \cs{znextpagesetup}. It expects the definition texts for these
% three cases of a macro with one argument.
% The default is
% \begin{quote}
%   |\znextpagesetup{#1}{#1}{#1}|
% \end{quote}
%
% \subsubsection{Example}
%
%    \begin{macrocode}
%<*example-nextpage>
%<<END_EXAMPLE
\documentclass{book}

\usepackage{zref-nextpage}[2019/11/29]
\znextpagesetup
  {\thepage}% next page is unknown
  {\thepage\ (#1)}% this page is last page
  {\thepage\ $\rightarrow$ #1}% next page is known
\renewcommand*{\znonextpagename}{last page}

\usepackage{fancyhdr}
\pagestyle{fancy}
\fancyhf{}
\fancyhead[LE,RO]{\znextpage}
\fancypagestyle{plain}{%
  \fancyhf{}%
  \fancyhead[LE,RO]{\znextpage}%
}

\begin{document}
\frontmatter
  \tableofcontents
\mainmatter
  \chapter{Hello World}
  \clearpage
  \section{Last section}
\end{document}
%END_EXAMPLE
%</example-nextpage>
%    \end{macrocode}
%
%
% \subsection{Module \xmodule{totpages}}
%
% For the total number of pages of a document you need to
% know the absolute page number of the last page.
% Both modules \xmodule{abspage} and \xmodule{lastpage}
% are necessary and automatically enabled.
%
% \begin{declcs}{ztotpages}\markexp
% \end{declcs}
% Prints the total number of pages or |0| if this
% number is not yet known. It expands to an explicit number
% and can also used even in expandable calculations (\cs{numexpr})
% or counter assignments.
%
% \subsection{Module \xmodule{pagelayout}}
%
% The module defines additional properties for each parameter
% of the page layout that is effective during page shipout.
% The value of length parameters is given in sp without the
% unit as plain number.
%
% Some parameters are specific for a class (e.g.\@ \xoption{stockwidth}
% and \xoption{stockheight} for class \xclass{memoir}) or the
% \hologo{TeX} engine like \hologo{pdfTeX}. If the parameter is not
% available, then the property will not be defined. The default value
% of the property is the current setting of the parameter.
%
% The module \xmodule{thepage} is loaded that generates a label
% for each page. The properties of module \xmodule{pagelayout}
% are added to the property list |thepage| of module \xmodule{thepage}.
%
% List of properties:
% \begin{quote}
%   \def\x#1{\cs{#1}&#1}%
%   \begin{tabular}{l>{\sffamily}ll}
%     \textrm{parameter} & \textrm{property} & remarks\\
%     \hline
%     \x{mag}\\
%     \x{paperwidth}\\
%     \x{paperheight}\\
%     \x{stockwidth}& class \xclass{memoir}\\
%     \x{stockheight}& class \xclass{memoir}\\
%     \x{pdfpagewidth}& \hologo{pdfTeX}, \hologo{LuaTeX}\\
%     \x{pdfpageheight}& \hologo{pdfTeX}, \hologo{LuaTeX}\\
%     \x{pdfhorigin}& \hologo{pdfTeX}, \hologo{LuaTeX}\\
%     \x{pdfvorigin}& \hologo{pdfTeX}, \hologo{LuaTeX}\\
%     \x{hoffset}\\
%     \x{voffset}\\
%     \x{topmargin}\\
%     \x{oddsidemargin}\\
%     \x{evensidemargin}\\
%     \x{textwidth}\\
%     \x{textheight}\\
%     \x{headheight}\\
%     \x{headsep}\\
%     \x{footskip}\\
%     \x{marginparwidth}\\
%     \x{marginparsep}\\
%     \x{columnwidth}\\
%     \x{columnsep}\\
%   \end{tabular}
% \end{quote}
%
% \begin{declcs}{zlistpagelayout}
% \end{declcs}
% At the end of document the page layout parameter for each page are
% printed into the \xext{log} file if macro \cs{zlistpagelayout}
% is called before |\end{document}| (preamble is a good place).
%
% \subsection{Module \xmodule{marks}}
%
% ToDo.
%
% \subsection{Module \xmodule{runs}}
%
% Module \xmodule{runs} counts the \hologo{LaTeX} runs since last
% \xext{aux} file creation and prints the number in the
% \xext{log} file.
%
% \begin{declcs}{zruns}\markexp
% \end{declcs}
% Prints the the total number of \hologo{LaTeX} runs including
% the current one. It expands to an explicit number.
% Before |begin{document}| the value is zero meaning
% the \xext{aux} file is not read yet. If a previous
% \xext{aux} file exists, the value found there increased
% by one is the new number. Otherwise \cs{zruns} is set to one.
% \hologo{LaTeX}~runs where the \xext{aux} files are not rewritten
% are not counted (see \cs{nofiles}).
%
% \subsection{Module \xmodule{perpage}}
%
% With \cs{@addtoreset} or \cs{numberwithin} a counter can be
% reset if another counter is incremented. This do not work
% well if the other counter is the page counter. The page counter is
% incremented in the output routine that is often called asynchronous
% somewhere on the next page. A reference mechanism costs at least
% two \hologo{LaTeX} runs, but ensures correct page counter values.
%
% \begin{declcs}{zmakeperpage}
%   |[|\meta{reset}|]| \M{counter}
% \end{declcs}
% At the of a new page counter \meta{counter} starts
% counting with value \meta{reset} (default is \texttt{1}).
% The macro has the same
% syntax and semantics as \cs{MakePerPage} of package
% \xpackage{perpage} \cite{perpage}.
% Also \xoption{perpage} of package \xpackage{footmisc} \cite{footmisc}
% can easily be simulated by
% \begin{quote}
% |\zmakeperpage{footnote}| \itshape |% \usepackage[perpage]{footmisc}|
%\end{quote}
% If footnote symbols are used, some people dislike the
% first symbol \ensuremath{\dagger}. It can easily be skipped:
% \begin{quote}
% |\zmakeperpage[2]{footnote}|
% \end{quote}
%
% \begin{declcs}{thezpage}\\
%   counter |zpage|
% \end{declcs}
% If the formatted counter value of the counter that is reset at
% a new page contains the page value, then you can use \cs{thezpage},
% the page number of the current page. Or counter |zpage| can be
% used, if the page number should be formatted differently from
% the current page number. Example:
%\begin{quote}
%\begin{verbatim}
%\newcounter{foobar}
%\zmakeperpage{foobar}
%\renewcommand*{\thefoobar}{\thezpage-\arabic{foobar}}
% % or
%\renewcommand*{\thefoobar}{\roman{zpage}-\arabic{foobar}}
%\end{verbatim}
%\end{quote}
%
% \begin{declcs}{zunmakeperpage}
%   \M{counter}
% \end{declcs}
% The reset mechanism for this counter is deactivated.
%
%
% \subsection{Module \xmodule{counter}}
%
% This option just add the property |counter| to the main
% property list. The property stores the counter name,
% that was responsible for the reference. This is the
% property \xpackage{hyperref}'s \cs{autoref} feature
% uses. Thus this property |counter| may be useful
% for a reimplementation of the autoref feature,
% see the section \ref{sec:todo} with the todo list.
%
% \subsection{Module \xmodule{titleref}}
%
% This option makes section and caption titles available
% to the reference system similar to packages \xpackage{titleref}
% or \xpackage{nameref}.
%
% \begin{declcs}{ztitleref}
%   \M{refname}\markbabel
% \end{declcs}
% Print the section or caption title of reference \meta{refname},
% similar to \cs{nameref} or \cs{titleref}.
%
% \begin{declcs}{ztitlerefsetup}
%   |{|\textit{key$_1$=value$_1$, key$_2$=value$_2$, \dots}|}|
% \end{declcs}
% This command allows to configure the behaviour of module
% \xmodule{titleref}. The following keys are available:
% \begin{description}
% \let\bfseries\relax
% \item[\texttt{title=}\meta{value}]\mbox{}\\*
%   Sets the current title.
% \item[\texttt{stripperiod=true\string|false}]\mbox{}\\*
%   Follow package \xpackage{nameref} that
%   removes a last period. Default: |true|.
% \item[\texttt{expand=true\string|false}]\mbox{}\\*
%   Package \cs{titleref} expands the title first. This way
%   garbage and dangerous commands can be removed, e.g.
%   \cs{label}, \cs{index}\dots. See implementation section
%   for more details. Default is |false|.
% \item[\texttt{cleanup=\{}\textrm{\dots}\texttt{\}}]\mbox{}\\*
%   Hook to add own cleanup code, if method |expand| is used.
%   See implementation section for more details.
% \end{description}
%
% \subsection{Module \xmodule{savepos}}
%
% This option supports a feature that \pdfTeX\ provides
% (and \hologo{XeTeX}).
% \pdfTeX\ is able to tell the current position on the
% page. The page position is not instantly known. First the
% page must be constructed by \hologo{TeX}'s asynchronous output
% routine. Thus the time where the position is known is
% the page shipout time. Thus a reference system where the
% information is recorded in the first run and made available
% for use in the second run comes in handy.
%
% \begin{declcs}{zsavepos}
%   \M{refname}
% \end{declcs}
% It generates a reference with name \meta{refname}.
% The reference stores the location where \cs{zsavepos}
% is executed in properties |posx| and |posy|.
%
% \begin{declcs}{zsaveposx}
%   \M{refname}\\
%   \cs{zsaveposy}
%   \M{refname}
% \end{declcs}
% Same as \cs{zsavepos} except that only the |x| or |y| component
% of the position is stored. Since 2011/12/05 v2.23.
%
% \begin{declcs}{zposx}\markexp
%   \M{refname}\\
%   \SpecialUsageIndex{\zposy}\cs{zposy}\markexp \M{refname}
% \end{declcs}
% Get the position as number. Unit is sp. Horizontal
% positions by \cs{zposx} increase from left to right.
% Vertical positions by \cs{zposy} from bottom to top.
%
% Do not rely on absolute page numbers. Because of problems
% with the origin the numbers may differ in DVI or PDF mode
% of \pdfTeX. Therefore work with relative values by comparisons.
%
% Both \cs{zposx} and \cs{zposy} are expandable and can be
% used inside calculations (\cs{setcounter}, \cs{addtocounter},
% package \xpackage{calc}, \cs{numexpr}). However this property
% prevents from notifying \hologo{LaTeX} that the reference is actually
% used (the notifying is not expandable). Therefore you should
% mark the reference as used by \cs{zrefused}.
%
% This module uses \pdfTeX's \cs{pdfsavepos}, \cs{pdflastxpos},
% and \cs{pdflastypos}. They are available in PDF mode and
% since version 1.40.0 also in DVI mode.
%
% \begin{declcs}{zref@savepos}
% \end{declcs}
% Macro \cs{zref@savepos} performs the first part of \cs{zsavepos}
% by calling \cs{pdfsavepos} (if \xext{aux} files are writable).
%
% Thus \cs{zsavepos} is basically \cs{zref@savepos} followed
% by |\zref@labelbylist{|\texttt{\meta{refname}}|}{savepos}|.
% If \cs{TeXXeTstate} is detected and enabled, \cs{savepos}
% also adds \cs{zref@savepos} at the end to support \cs{beginR}
% where the whatits are processed in reverse order.
% The property list |savepos| contains the properties |posx|
% and |posy|.
%
% \subsection{Module \xmodule{abspos}}
% Module \xmodule{abspos} allows to get various values of the page layout. There is no user
% command, only a number of internal commands. For example:
% \begin{quote}
%   |\zref@absposx{|\meta{label}|}{|\meta{value}|}{|\meta{position}|}|\\
%   |\zref@absposy{|\meta{label}|}{|\meta{value}|}{|\meta{position}|}|
% \end{quote}
%
% The return value is like in the module \xmodule{savepos} a number representing a length in sp.
% The length are measured from the bottom left of the page.
%
% \meta{label} is a label set with \cs{zlabel} or \cs{zsavepos} that allows to retrieve the
% absolute page number.
%
% \meta{position} is for the x-command one of |left|, |right| or |center|. For the y-command
%  it is one of |top|, |bottom|, |center|.
%
% The possible content of \meta{value} can be seen in the following table. Be aware that the code
% makes some assumptions which are perhaps not always true -- for example that the left of
% the head is identical to the left of the body.
%
% \medskip
% \begin{tabular}{lll}
%  \textbf{value} & \textbf{axis} & \textbf{comments}\\\hline
%   media         & x             &   left=0, right=\cs{pdfpagewidth} \\
%   paper         & x             &   left=0, right=\cs{paperwidth}\\
%   stock         & x             &   derived from paper\\
%   media         & y             & bottom=0, top=\cs{pdfpageheigh}\\
%   paper         & y             & top=\cs{pdfpageheight}, bottom=top-\cs{paperheight}\\
%   stock         & y             & top  derived from paper\\
%   head          & x             & calculated with hoffset, horigin, etc\\
%   head          & y             & calculated\\
%   body          & x             & = head value\\%
%   body          & y             & = head bottom - \cs{headsep}\\
%   foot          & x             & = head\\
%   foot          & y             & calculated from body bottom and \cs{footskip}\\
%   marginpar     & x             &different on odd/even pages! \\
%   marginpar     & y             & = body %
% \end{tabular}
%
% \subsection{Module \xmodule{dotfill}}
%
% \begin{declcs}{zdotfill}
% \end{declcs}
% This package provides the command \cs{zdotfill} that works
% similar to \cs{dotfill}, but can be configured. Especially
% it suppresses the dots if a minimum number of dots cannot be set.
%
% \begin{declcs}{zdotfillsetup}
%   |{|\textit{key$_1$=value$_1$, key$_2$=value$_2$, \dots}|}|
% \end{declcs}
% This command allows to configure the behaviour of \cs{zdotfill}.
% The following keys are available:
% \begin{description}
% \let\bfseries\relax
% \item[\texttt{min=}\meta{count value}]\mbox{}\\*
%   If the actual number of dots are smaller than \meta{count value},
%   then the dots are suppressed. Default: |2|.
% \item[\texttt{unit=}\meta{dimen value}]\mbox{}\\*
%   The width of a dot unit is given by \meta{dimen value}.
%   Default: |0.44em| (same as the unit in \cs{dotfill}).
% \item[\texttt{dot=}\meta{value}]\mbox{}\\*
%   The dot itself is given by \meta{value}. Default: |.| (dot, same as
%   the dot in \cs{dotfill}).
% \end{description}
%
% \subsection{Module \xmodule{env}}
%
% This module defines two properties |envname| and |envline|. They
% remember the name of the environment and the line number at
% the start of the environment.
%
% \subsection{Module \xmodule{xr}}
%
% This package provides the functionality of package \xpackage{xr},
% see \cite{xr}. It also supports the syntax of \xpackage{xr-hyper}.
%
% \begin{declcs}{zexternaldocument}
%   \* |[|\meta{prefix}|]|\markbabel \M{external document}
%   |[|\meta{url}|]|
% \end{declcs}
% See \cs{externaldocument} for a description of this option.
% The found labels also get a property |externaldocument| that
% remembers \meta{external document}.
% The standard reference scheme and the scheme of this
% package use different name spaces for reference names.
% If the external document uses both systems. Then one
% import statement would put the names in one namespace
% and probably causing problems with multiple references
% of the same name. Thus the star form only looks for
% \cs{newlabel} in the \xfile{.aux} files, whereas without
% star only \cs{zref@newlabel}s are used.
%
% In the star form it tries to detect labels from
% \xpackage{hyperref}, \xpackage{titleref}, and \xpackage{ntheorem}.
% If such an extended property from the packages before cannot
% be found or are empty, they are not included in the
% imported reference.
%
% Warnings are given if a reference name is already in use
% and the item is ignored. Unknown properties will automatically
% be declared.
%
% If the external references contain |anchor| properties,
% then we need also a url to be able to address the external
% file. As default the filename is taken with a default
% extension.
%
% \begin{declcs}{zxrsetup}
%   |{|\textit{key$_1$=value$_1$, key$_2$=value$_2$, \dots}|}|
% \end{declcs}
% The following setup options are available:
% \begin{description}
% \item[\xoption{ext}:] It sets the default extension.
% \item[\xoption{tozreflabel}:] Boolean option. The found references
%   are imported as zref labels. This is enabled by default.
% \item[\xoption{toltxlabel}:] Boolean option. The found references
%   are imported as \hologo{LaTeX} labels. Packages \xpackage{nameref},
%   \xpackage{titleref} and class \xclass{memoir} are supported.
% \item[\xoption{urluse}:] Boolean option. If enabled, then a URL
%   is stored in a macro and the macro is put in property `urluse'.
%   The URL is not put in property `url'.
%   The purpose is to save \hologo{TeX} memory.
% \item[\xoption{verbose}:] Boolean option. List the imported labels in the
%   \xext{log} file. Default is |false|.
% \end{description}
%
% \begin{declcs}{zref@xr@ext}
% \end{declcs}
% If the \meta{url} is not specified in \cs{zref@externaldocument},
% then the url will be constructed with the file name and
% this macro as extension. \cs{XR@ext} is used if
% \xpackage{hyperref} is loaded, otherwise |pdf|.
%
% \subsection{Module \xmodule{pageattr}}
%
% This module allows to recover the content of the register
% \cs{pdfpageattr} and \cs{pdfpagesattr} in pdftex and the equivalent register
% in luatex. There is no user command. Programmer commands are
%
% \begin{quote}
% |\zref@pdfpageattr{|\meta{absolute page number}|}|
%
% |\zref@pdfpagesattr{|\meta{absolute page number}|}|
% \end{quote}
% \section{ToDo}\label{sec:todo}
%
% Among other things the following issues are left for future work:
% \begin{itemize}
% \item Other applications: autoref, hyperref, \dots
% \end{itemize}
%
% \section{Example}
%
%    \begin{macrocode}
%<*example>
%    \end{macrocode}
%
%    \begin{macrocode}
\documentclass{book}

\usepackage[ngerman]{babel}%

\usepackage[savepos,totpages,titleref,dotfill,counter,user]{zref}

%    \end{macrocode}
%
%    Chapters are wrapped inside \cs{ChapterStart} and
%    \cs{ChapterStop}. The first argument \verb|#1| of \cs{ChapterStart}
%    is used to form a label id \verb|chap:#1|. At the end of
%    the chapter another label is set by \cs{zref@wrapper@immediate}, because
%    otherwise at the end of document a deferred write would not
%    be written, because there is no page for shipout.
%
%    Also this example shows how chapter titles can be recorded.
%    A new property \verb|chaptitle| is declared and added to the
%    main property list. In \cs{ChapterStart} the current value
%    of the property is updated.
%    \begin{macrocode}
\makeatletter
\zref@newprop{chaptitle}{}
\zref@addprop{main}{chaptitle}

\newcommand*{\ChapterStart}[2]{%
  \cleardoublepage
  \def\current@chapid{#1}%
  \zref@setcurrent{chaptitle}{#2}%
  \chapter{#2}%
  \zlabel{chap:#1}%
}
\newcommand*{\ChapterStop}{%
  \cleardoublepage
  \zref@wrapper@immediate{%
    \zref@labelbyprops{chapend:\current@chapid}{abspage}%
  }%
}
%    \end{macrocode}
%    \cs{ChapterPages} calculates and returns the number of pages of the
%    referenced chapter.
%    \begin{macrocode}
\newcommand*{\ChapterPages}[1]{%
  \zrefused{chap:#1}%
  \zrefused{chapend:#1}%
  \number\numexpr
    \zref@extract{chapend:#1}{abspage}%
    -\zref@extract{chap:#1}{abspage}%
    +1\relax
}
\makeatother
%    \end{macrocode}
%    \begin{macrocode}
\begin{document}
%    \end{macrocode}
%    As exception we use \cs{makeatletter} here, because
%    this is just an example file that also should show some of
%    programmer's interface.
%    \begin{macrocode}
\makeatletter

\frontmatter
\zlabel{documentstart}

\begin{itemize}
\item
  The frontmatter part has
  \number\numexpr\zref@extract{chap:first}{abspage}-1\relax
  ~pages.
\item
  Chapter \zref{chap:first} has \ChapterPages{first} page(s).
\item
  Section \zref{hello} is on the
  \ifcase\numexpr
    \zref@extractdefault{hello}{page}{0}%
    -\zref@extractdefault{chap:first}{page}{0}%
    +1\relax
    ??\or first\or second\or third\or forth\fi
  ~page inside its chapter.
\item
  The document has
  \zref[abspage]{LastPage} pages.
  This number is \ifodd\ztotpages odd\else even\fi.
\item
  The last page is labeled with \zpageref{LastPage}.
\item
  The title of chapter \zref{chap:next} %
  is ``\zref[chaptitle]{chap:next}''.
\end{itemize}

\tableofcontents

\mainmatter
\ChapterStart{first}{First chapter}

%    \end{macrocode}
%    The user level commands should protect babel shorthands
%    where possible. On the other side, expandable extracting
%    macros are useful in calculations, see above the
%    examples with \cs{numexpr}.
%    \begin{macrocode}
\section{Test}
\zlabel{a"o}
Section \zref{a"o} on page
\zref@wrapper@babel\zref@extract{a"o}{page}.

Text.
\newpage

\section{Hello World}
\zlabel{hello}

\ChapterStop

\ChapterStart{next}{Next chapter with \emph{umlauts}: "a"o"u"s}

%    \end{macrocode}
%
%    Here an example follows that makes use of \pdfTeX's ``savepos''
%    feature. The position on the page is not known before the
%    page is constructed and shipped out. Therefore the position
%    ist stored in references and are available for calculations
%    in the next \hologo{LaTeX} compile run.
%    \begin{macrocode}
The width of the first column is
  \the\dimexpr \zposx{secondcol}sp - \zposx{firstcol}sp\relax,\\
the height difference of the two baselines is
  \the\dimexpr \zposy{firstcol}sp - \zposy{secondline}sp\relax:\\
\begin{tabular}{ll}
  \zsavepos{firstcol}Hello&\zsavepos{secondcol}World\\
  \zsavepos{secondline}Second line&foobar\\
\end{tabular}

%    \end{macrocode}
%    With \cs{zrefused} \hologo{LaTeX} is notified, if the references
%    are not yet available and \hologo{LaTeX} can generate the rerun
%    hint.
%    \begin{macrocode}
\zrefused{firstcol}
\zrefused{secondcol}
\zrefused{secondline}

\ChapterStop
%    \end{macrocode}
%    Test for module \cs{dotfill}.
%    \begin{macrocode}
\ChapterStart{dotfill}{Test for dotfill feature}
\newcommand*{\dftest}[1]{%
  #1&
  [\makebox[{#1}]{\dotfill}]&
  [\makebox[{#1}]{\zdotfill}]\\
}
\begin{tabular}{rll}
& [\verb|\dotfill|] & [\verb|\zdotfill|]\\
\dftest{0.43em}
\dftest{0.44em}
\dftest{0.45em}
\dftest{0.87em}
\dftest{0.88em}
\dftest{0.89em}
\dftest{1.31em}
\dftest{1.32em}
\dftest{1.33em}
\end{tabular}
\ChapterStop
\end{document}
%    \end{macrocode}
%
%    \begin{macrocode}
%</example>
%    \end{macrocode}
%
% \StopEventually{
% }
%
% \section{Implementation}
%
% \subsection{Package \xpackage{zref}}
%
% \subsubsection{Identification}
%
%    \begin{macrocode}
%<*package>
\NeedsTeXFormat{LaTeX2e}
\ProvidesPackage{zref}
  [2020-07-03 v2.32 A new reference scheme for LaTeX (HO)]%
%    \end{macrocode}
%
% \subsubsection{Load basic module}
%
%    \begin{macrocode}
\RequirePackage{zref-base}[2019/11/29]
%    \end{macrocode}
%    Abort package loading if \xpackage{zref-base} could not be
%    loaded successfully.
%    \begin{macrocode}
\@ifundefined{ZREF@base@ok}{\endinput}{}
%    \end{macrocode}
%
% \subsubsection{Process options}
%
%    Known modules are loaded and the release date is checked.
%    \begin{macrocode}
\def\ZREF@temp#1{%
  \DeclareOption{#1}{%
    \AtEndOfPackage{%
      \RequirePackage{zref-#1}[2019/11/29]%
    }%
  }%
}
\ZREF@temp{abspage}
\ZREF@temp{counter}
\ZREF@temp{dotfill}
\ZREF@temp{hyperref}
\ZREF@temp{lastpage}
\ZREF@temp{marks}
\ZREF@temp{nextpage}
\ZREF@temp{pageattr}
\ZREF@temp{pagelayout}
\ZREF@temp{perpage}
\ZREF@temp{runs}
\ZREF@temp{savepos}
\ZREF@temp{thepage}
\ZREF@temp{titleref}
\ZREF@temp{totpages}
\ZREF@temp{user}
\ZREF@temp{xr}
%    \end{macrocode}
%    \begin{macrocode}
\ProcessOptions\relax
%</package>
%    \end{macrocode}
%
%
% \subsection{Module \xmodule{base}}
%
% \subsubsection{Prefixes}
%
%    This package uses the following prefixes for macro names:
%    \begin{description}
%    \item[\cs{zref@}:] Macros of the programmer's interface.
%    \item[\cs{ZREF@}:] Internal macros.
%    \item[\cs{Z@L@}\metacs{listname}:]
%      The properties of the list \meta{listname}.
%    \item[\cs{Z@D@}\metacs{propname}:]
%      The default value for property \meta{propname}.
%    \item[\cs{Z@E@}\metacs{propname}:]
%      Extract function for property \meta{propname}.
%    \item[\cs{Z@X@}\metacs{propname}:]
%      Information whether a property value for property
%      \meta{propname} is expanded
%      immediately or at shipout time.
%    \item[\cs{Z@C@}\metacs{propname}:]
%      Current value of the property \meta{propname}.
%    \item[\cs{Z@R@}\metacs{labelname}:]
%      Data for reference \meta{labelname}.
%    \item[\cs{ZREF@org@}:] Original versions of patched commands.
%    \item[\cs{z}:] For macros in user land, defined if
%      module \xmodule{user} is set.
%    \end{description}
%    The following family names are used for keys defined
%    according to the \xpackage{keyval} package:
%    \begin{description}
%    \item[\texttt{ZREF@TR}:] Setup for module \xmodule{titleref}.
%    \end{description}
%
% \subsubsection{Identification}
%
%    \begin{macrocode}
%<*base>
\NeedsTeXFormat{LaTeX2e}
\ProvidesPackage{zref-base}%
  [2020-07-03 v2.32 Module base for zref (HO)]%
%    \end{macrocode}
%
% \subsubsection{Utilities}
%
%    \begin{macrocode}
\providecommand\IfFormatAtLeastTF{\@ifl@t@r\fmtversion}
\RequirePackage{ltxcmds}[2010/12/02]
\RequirePackage{infwarerr}[2010/04/08]
\RequirePackage{kvsetkeys}[2010/03/01]
\RequirePackage{kvdefinekeys}[2010/03/01]
\RequirePackage{pdftexcmds}[2010/04/01]
%    \end{macrocode}
%
%    \begin{macro}{\ZREF@name}
%    Several times the package name is used, thus we store it
%    in \cmd{\ZREF@name}.
%    \begin{macrocode}
\def\ZREF@name{zref}
%    \end{macrocode}
%    \end{macro}
%
%    \begin{macrocode}
\ltx@IfUndefined{protected}{%
  \RequirePackage{makerobust}[2006/03/18]%
%    \end{macrocode}
%    \begin{macro}{\ZREF@Robust}
%    \begin{macrocode}
  \def\ZREF@Robust#1#2{%
    \def\ZREF@temp{\MakeRobustcommand#2}%
    \afterassignment\ZREF@temp
    #1#2%
  }%
%    \end{macrocode}
%    \end{macro}
%    \begin{macrocode}
}{%
%    \end{macrocode}
%    \begin{macro}{\ZREF@Robust}
%    \begin{macrocode}
  \def\ZREF@Robust#1{%
    \protected#1%
  }%
%    \end{macrocode}
%    \end{macro}
%    \begin{macrocode}
}
%    \end{macrocode}
%    \begin{macro}{\ZREF@IfDefinable}
%    \begin{macrocode}
\def\ZREF@IfDefinable#1#2#3{%
  \@ifdefinable{#1}{%
    \ZREF@Robust{#2}#1#3%
  }%
}
%    \end{macrocode}
%    \end{macro}
%
%    \begin{macro}{\ZREF@UpdatePdfTeX}
%    \cs{ZREF@UpdatePdfTeX} is used as help message text in
%    error messages.
%    \begin{macrocode}
\def\ZREF@UpdatePdfTeX{Update pdfTeX.}
%    \end{macrocode}
%    \end{macro}
%
%    \begin{macro}{\ifZREF@found}
%    The following switch is usded in list processing.
%    \begin{macrocode}
\newif\ifZREF@found
%    \end{macrocode}
%    \end{macro}
%
%    \begin{macro}{\ZREF@patch}
%    Macro \cs{ZREF@patch} first checks the existence of
%    the command and safes it.
%    \begin{macrocode}
\def\ZREF@patch#1{%
  \ltx@IfUndefined{#1}{%
    \ltx@gobble
  }{%
    \expandafter\let\csname ZREF@org@#1\expandafter\endcsname
    \csname #1\endcsname
    \ltx@firstofone
  }%
}
%    \end{macrocode}
%    \end{macro}
%
% \subsubsection{Check for \hologo{eTeX}}
%
%    The use of \hologo{eTeX} should be standard nowadays for \hologo{LaTeX}.
%    We test for \hologo{eTeX} in order to use its features later.
%    \begin{macrocode}
\ltx@IfUndefined{eTeXversion}{%
  \PackageError\ZREF@name{%
    Missing support for eTeX; package is abandoned%
  }{%
    Use a TeX compiler that support eTeX and enable eTeX %
    in the format.%
  }%
  \endinput
}{}%
%    \end{macrocode}
%
%    \begin{macrocode}
\RequirePackage{etexcmds}[2007/09/09]
\ifetex@unexpanded
\else
  \PackageError\ZREF@name{%
    Missing e-TeX's \string\unexpanded.\MessageBreak
    Add \string\RequirePackage\string{etexcmds\string} before %
    \string\documentclass%
  }{%
    Probably you are using some package (e.g. ConTeXt) that %
    redefines \string\unexpanded%
  }%
  \expandafter\endinput
\fi
%    \end{macrocode}
%
% \subsubsection{Auxiliary file stuff}
%
%    We are using some commands in the \xfile{.aux} files.
%    However sometimes these auxiliary files are interpreted
%    by \hologo{LaTeX} processes that haven't loaded this package
%    (e.g. package \xpackage{xr}).
%    Therefore we provide dummy definitions.
%    \begin{macrocode}
\RequirePackage{auxhook}
\AddLineBeginAux{%
  \string\providecommand\string\zref@newlabel[2]{}%
}
%    \end{macrocode}
%    \begin{macro}{\ZREF@RefPrefix}
%    \begin{macrocode}
\def\ZREF@RefPrefix{Z@R}
%    \end{macrocode}
%    \end{macro}
%    \begin{macro}{\zref@newlabel}
%    For the implementation of \cs{zref@newlabel} we call
%    the same internal macro \cs{@newl@bel} that is used
%    in \cs{newlabel}. Thus we have for free:
%    \begin{itemize}
%    \item \cs{Z@R@}\metacs{labelname} is defined.
%    \item \hologo{LaTeX}'s check for multiple references.
%    \item \hologo{LaTeX}'s check for changed references.
%    \end{itemize}
%    \begin{macrocode}
\ZREF@Robust\edef\zref@newlabel{%
  \noexpand\@newl@bel{\ZREF@RefPrefix}%
}
%    \end{macrocode}
%    \end{macro}
%
% \subsubsection{Property lists}
%
%    \begin{macro}{\zref@newlist}
%    Property lists are stored as list of property names
%    enclosed in curly braces. \cs{zref@newlist}
%    creates a new list as empty list. Assignments to
%    property lists are global.
%    \begin{macrocode}
\ZREF@Robust\def\zref@newlist#1{%
  \zref@iflistundefined{#1}{%
    \@ifdefinable{Z@L@#1}{%
     \global\expandafter\let\csname Z@L@#1\endcsname\ltx@empty
     \PackageInfo\ZREF@name{New property list: #1}%
    }%
  }{%
    \PackageError\ZREF@name{%
      Property list `#1' already exists%
    }\@ehc
  }%
}
%    \end{macrocode}
%    \end{macro}
%
%    \begin{macro}{\zref@iflistundefined}
%    \cs{zref@iflistundefined} checks the existence of the
%    property list \verb|#1|. If the property list is present,
%    then \verb|#2| is executed and \verb|#3| otherwise.
%    \begin{macrocode}
\def\zref@iflistundefined#1{%
  \ltx@ifundefined{Z@L@#1}%
}
%    \end{macrocode}
%    \end{macro}
%    \begin{macro}{\zref@listexists}
%    \cs{zref@listexists} only executes \verb|#2| if the
%    property list \verb|#1| exists and raises an error
%    message otherwise.
%    \begin{macrocode}
\ZREF@Robust\def\zref@listexists#1{%
  \zref@iflistundefined{#1}{%
    \PackageError\ZREF@name{%
      Property list `#1' does not exist%
    }\@ehc
  }%
}
%    \end{macrocode}
%    \end{macro}
%
%    \begin{macro}{\zref@iflistcontainsprop}
%    \cs{zref@iflistcontainsprop} checks, whether a property \verb|#2| is
%    already present in a property list \verb|#1|.
%    \begin{macrocode}
\ZREF@Robust\def\zref@iflistcontainsprop#1#2{%
  \zref@iflistundefined{#1}{%
    \ltx@secondoftwo
  }{%
    \begingroup\expandafter\endgroup
    \expandafter\in@
    \csname#2\expandafter\expandafter\expandafter\endcsname
    \expandafter\expandafter\expandafter{\csname Z@L@#1\endcsname}%
    \csname ltx@\ifin@ first\else second\fi oftwo\endcsname
  }%
}
%    \end{macrocode}
%    \end{macro}
%
%    \begin{macro}{\zref@listforloop}
%    \begin{macrocode}
\def\zref@listforloop#1#2{%
  \zref@listexists{#1}{%
    \expandafter\expandafter\expandafter\@tfor
    \expandafter\expandafter\expandafter\zref@prop
    \expandafter\expandafter\expandafter:%
    \expandafter\expandafter\expandafter=%
    \csname Z@L@#1\endcsname
    \do{%
      \begingroup
        \escapechar=-1 %
        \edef\x{\endgroup
          \def\noexpand\zref@prop{%
            \expandafter\string\zref@prop
          }%
        }%
      \x
      #2\zref@prop
    }%
  }%
}
%    \end{macrocode}
%    \end{macro}
%
%    \begin{macro}{\zref@addprops}
%    \cs{zref@addprop} adds the properties \verb|#2| to the property
%    list \verb|#1|, if the property is not already in the list.
%    Otherwise a warning is given.
%    \begin{macrocode}
\ZREF@Robust\def\zref@addprops#1#2{%
  \zref@listexists{#1}{%
    \comma@parse{#2}{%
      \zref@propexists\comma@entry{%
        \zref@iflistcontainsprop{#1}\comma@entry{%
          \PackageWarning\ZREF@name{%
            Property `\comma@entry' is already in list `#1'%
          }%
        }{%
          \begingroup\expandafter\endgroup
          \expandafter\g@addto@macro
          \csname Z@L@#1\expandafter\endcsname
          \expandafter{\csname\comma@entry\endcsname}%
        }%
      }%
      \ltx@gobble
    }%
  }%
}
%    \end{macrocode}
%    \end{macro}
%    \begin{macro}{\zref@addprop}
%    \cs{zref@addprop} adds the property \verb|#2| to the property
%    list \verb|#1|, if the property is not already in the list.
%    Otherwise a warning is given.
%    \begin{macrocode}
\ZREF@Robust\def\zref@addprop#1#2{%
  \zref@listexists{#1}{%
    \zref@propexists{#2}{%
      \zref@iflistcontainsprop{#1}{#2}{%
        \PackageWarning\ZREF@name{%
          Property `#2' is already in list `#1'%
        }%
      }{%
        \begingroup\expandafter\endgroup
        \expandafter\g@addto@macro
        \csname Z@L@#1\expandafter\endcsname
        \expandafter{\csname#2\endcsname}%
      }%
    }%
  }%
}
%    \end{macrocode}
%    \end{macro}
%
%    \begin{macro}{\zref@localaddprops}
%    \begin{macrocode}
\ZREF@Robust\def\zref@localaddprops#1#2{%
  \zref@listexists{#1}{%
    \comma@parse{#2}{%
      \zref@propexists\comma@entry{%
        \zref@iflistcontainsprop{#1}\comma@entry{%
          \PackageWarning\ZREF@name{%
            Property `\comma@entry' is already in list `#1'%
          }%
        }{%
          \begingroup\expandafter\endgroup
          \expandafter\ltx@LocalAppendToMacro
          \csname Z@L@#1\expandafter\endcsname
          \expandafter{\csname\comma@entry\endcsname}%
        }%
      }%
      \ltx@gobble
    }%
  }%
}
%    \end{macrocode}
%    \end{macro}
%    \begin{macro}{\zref@localaddprop}
%    \begin{macrocode}
\ZREF@Robust\def\zref@localaddprop#1#2{%
  \zref@listexists{#1}{%
    \zref@propexists{#2}{%
      \zref@iflistcontainsprop{#1}{#2}{%
        \PackageWarning\ZREF@name{%
          Property `#2' is already in list `#1'%
        }%
      }{%
        \begingroup\expandafter\endgroup
        \expandafter\ltx@LocalAppendToMacro
        \csname Z@L@#1\expandafter\endcsname
        \expandafter{\csname#2\endcsname}%
      }%
    }%
  }%
}
%    \end{macrocode}
%    \end{macro}
%
%    \begin{macrocode}
\ltx@IfUndefined{pdf@strcmp}{%
%    \end{macrocode}
%    \begin{macro}{\zref@delprop}
%    \begin{macrocode}
  \ZREF@Robust\def\zref@delprop{%
    \ZREF@delprop\gdef
  }%
%    \end{macrocode}
%    \end{macro}
%    \begin{macro}{\zref@localdelprop}
%    \begin{macrocode}
  \ZREF@Robust\def\zref@localdelprop{%
    \ZREF@delprop\def
  }%
%    \end{macrocode}
%    \end{macro}
%    \begin{macro}{\ZREF@delprop}
%    \begin{macrocode}
  \def\ZREF@delprop#1#2#3{%
    \zref@listexists{#2}{%
      \begingroup
        \escapechar=-1 %
        \def\ZREF@param{#3}%
        \@onelevel@sanitize\ZREF@param
        \toks@{}%
        \expandafter\expandafter\expandafter\ZREF@@delprop
        \csname Z@L@#2\endcsname!%
      \expandafter\endgroup
      \expandafter#1\csname Z@L@#2\expandafter\endcsname
      \expandafter{%
        \the\toks@
      }%
    }%
  }%
%    \end{macrocode}
%    \end{macro}
%    \begin{macro}{\ZREF@@delprop}
%    \begin{macrocode}
  \def\ZREF@@delprop#1{%
    \expandafter\ZREF@@@delprop\expandafter{\string#1}#1%
  }%
%    \end{macrocode}
%    \end{macro}
%    \begin{macro}{\ZREF@@@delprop}
%    \begin{macrocode}
  \def\ZREF@@@delprop#1#2{%
    \ifx#2!%
    \else
      \def\ZREF@temp{#1}%
      \@onelevel@sanitize\ZREF@temp
      \ifx\ZREF@param\ZREF@temp
      \else
        \toks@\expandafter{%
          \the\expandafter\toks@\csname#1\endcsname
        }%
      \fi
      \expandafter\ZREF@@delprop
    \fi
  }%
%    \end{macrocode}
%    \end{macro}
%    \begin{macrocode}
}{%
%    \end{macrocode}
%    \begin{macro}{\zref@delprop}
%    \begin{macrocode}
  \ZREF@Robust\def\zref@delprop{%
    \ZREF@delprop\xdef
  }%
%    \end{macrocode}
%    \end{macro}
%    \begin{macro}{\zref@localdelprop}
%    \begin{macrocode}
  \ZREF@Robust\def\zref@localdelprop{%
    \ZREF@delprop\edef
  }%
%    \end{macrocode}
%    \end{macro}
%    \begin{macro}{\ZREF@delprop}
%    \begin{macrocode}
  \def\ZREF@delprop#1#2#3{%
    \zref@listexists{#2}{%
      \def\ZREF@param{#3}%
      \edef\ZREF@SavedEscapechar{\the\escapechar}%
      \escapechar=-1 %
      \expandafter#1\csname Z@L@#2%
      \expandafter\expandafter\expandafter\endcsname{%
        \expandafter\expandafter\expandafter\ZREF@@delprop
        \csname Z@L@#2\endcsname!%
      }%
      \escapechar=\ZREF@SavedEscapechar\relax
    }%
  }%
%    \end{macrocode}
%    \end{macro}
%    \begin{macro}{\ZREF@@delprop}
%    Caution: |#1| might be an \cs{if} or similar token.
%    \begin{macrocode}
  \def\ZREF@@delprop#1{%
    \expandafter\ZREF@@@delprop\expandafter{\string#1}#1%
  }%
%    \end{macrocode}
%    \end{macro}
%    \begin{macro}{\ZREF@@@delprop}
%    \begin{macrocode}
  \def\ZREF@@@delprop#1#2{%
    \ifx#2!%
    \else
      \ifnum\pdf@strcmp{#1}{\ZREF@param}=\ltx@zero
      \else
        \expandafter\noexpand\csname#1\endcsname
      \fi
      \expandafter\ZREF@@delprop
    \fi
  }%
%    \end{macrocode}
%    \end{macro}
%    \begin{macrocode}
}
%    \end{macrocode}
%
% \subsubsection{Properties}
%
%    \begin{macro}{\zref@ifpropundefined}
%    \cs{zref@ifpropundefined} checks the existence of the
%    property \verb|#1|. If the property is present,
%    then \verb|#2| is executed and \verb|#3| otherwise.
%    \begin{macrocode}
\def\zref@ifpropundefined#1{%
  \ltx@ifundefined{Z@E@#1}%
}
%    \end{macrocode}
%    \end{macro}
%    \begin{macro}{\zref@propexists}
%    Some macros rely on the existence of a property.
%    \cs{zref@propexists} only executes \verb|#2| if the
%    property \verb|#1| exists and raises an error
%    message otherwise.
%    \begin{macrocode}
\ZREF@Robust\def\zref@propexists#1{%
  \zref@ifpropundefined{#1}{%
    \PackageError\ZREF@name{%
      Property `#1' does not exist%
    }\@ehc
  }%
}
%    \end{macrocode}
%    \end{macro}
%
%    \begin{macro}{\zref@newprop}
%    A new property is declared by \cs{zref@newprop}, the
%    property name \meta{propname} is given in \verb|#1|. The
%    property is created and configured. If the star form
%    is given, then the expansion of the property value
%    is delayed to page shipout time, when the reference
%    is written to the \xfile{.aux} file.
%    \begin{description}
%    \item[\cs{Z@D@}\metacs{propname}:]
%      Stores the default value for this property.
%    \item[\cs{Z@E@}\metacs{propname}:]
%      Extract function.
%    \item[\cs{Z@X@}\metacs{propname}:]
%      Information whether the expansion of the property value
%      is delayed to shipout time.
%    \item[\cs{Z@C@}\metacs{propname}:]
%      Current value of the property.
%    \end{description}
%    \begin{macrocode}
\ZREF@Robust\def\zref@newprop{%
  \@ifstar{%
    \let\ZREF@X\noexpand
    \ZREF@newprop
  }{%
    \let\ZREF@X\ltx@empty
    \ZREF@newprop
  }%
}
%    \end{macrocode}
%    \end{macro}
%    \begin{macro}{\ZREF@newprop}
%    \begin{macrocode}
\def\ZREF@newprop#1{%
  \edef\ZREF@P{#1}%
  \@onelevel@sanitize\ZREF@P
  \begingroup
  \ifx\ZREF@P\ZREF@par
    \@PackageError\ZREF@name{%
      Invalid property name `\ZREF@P'%
    }{%
      The property name `par' is not allowed %
      because of internal reasons.%
      \MessageBreak
      \@ehc
    }%
    \def\ZREF@@newprop[##1]##2{\endgroup}%
  \else
    \zref@ifpropundefined\ZREF@P{%
      \endgroup
      \PackageInfo\ZREF@name{%
        New property: \ZREF@P
      }%
    }{%
      \@PackageError\ZREF@name{%
        Property `\ZREF@P' already exists%
      }\@ehc
      \def\ZREF@@newprop[##1]##2{\endgroup}%
    }%
  \fi
  \@ifnextchar[\ZREF@@newprop{\ZREF@@newprop[\zref@default]}%
}
%    \end{macrocode}
%    \end{macro}
%    \begin{macro}{\ZREF@par}
%    \begin{macrocode}
\def\ZREF@par{par}
\@onelevel@sanitize\ZREF@par
%    \end{macrocode}
%    \end{macro}
%    \begin{macro}{\ZREF@@newprop}
%    \begin{macrocode}
\def\ZREF@@newprop[#1]{%
  \global\@namedef{Z@D@\ZREF@P}{#1}%
  \global\expandafter\let\csname Z@X@\ZREF@P\endcsname\ZREF@X
  \begingroup\expandafter\endgroup
  \expandafter\ZREF@@@newprop\csname\ZREF@P\endcsname
  \expandafter\gdef\csname Z@C@\ZREF@P\endcsname{}%
  \zref@setcurrent\ZREF@P
}
\def\ZREF@@@newprop#1{%
  \expandafter
  \gdef\csname Z@E@\ZREF@P\endcsname##1#1##2##3\ZREF@nil{##2}%
}
%    \end{macrocode}
%    \end{macro}
%
%    \begin{macro}{\zref@showprop}
%    \begin{macrocode}
\ZREF@Robust\def\zref@showprop#1{%
  \zref@ifpropundefined{#1}{%
    \@PackageInfoNoLine{\ZREF@name}{%
      Show property `#1': <undefined>%
    }%
  }{%
    \begingroup
      \toks@\expandafter\expandafter\expandafter{%
        \csname Z@C@#1\endcsname
      }%
      \edef\ZREF@value{\the\toks@}%
      \ltx@onelevel@sanitize\ZREF@value
      \toks@\expandafter\expandafter\expandafter{%
        \csname Z@D@#1\endcsname
      }%
      \edef\ZREF@default{\the\toks@}%
      \ltx@onelevel@sanitize\ZREF@default
      \@PackageInfoNoLine{\ZREF@name}{%
        Show property `#1':\MessageBreak
        \expandafter\ifx\csname Z@X@#1\endcsname\ltx@empty
          Immediate %
        \else
          Delayed %
        \fi
        value: [\ZREF@value]\MessageBreak
        Default: [\ZREF@default]%
      }%
    \endgroup
  }%
}
%    \end{macrocode}
%    \end{macro}
%
%    \begin{macro}{\zref@setcurrent}
%    \cs{zref@setcurrent} sets the current value for a property.
%    \begin{macrocode}
\ZREF@Robust\def\zref@setcurrent#1#2{%
  \zref@propexists{#1}{%
    \expandafter\def\csname Z@C@#1\endcsname{#2}%
  }%
}
%    \end{macrocode}
%    \end{macro}
%
%    \begin{macro}{\ZREF@getcurrent}
%    \cs{zref@getcurrent} gets the current value for a property.
%    \begin{macrocode}
\def\ZREF@getcurrent#1{%
  \romannumeral0%
  \ltx@ifundefined{Z@C@#1}{%
    \ltx@space
  }{%
    \expandafter\expandafter\expandafter\ltx@space
    \csname Z@C@#1\endcsname
  }%
}
%    \end{macrocode}
%    \end{macro}
%    \begin{macro}{\ZREF@u@getcurrent}
%    \begin{macrocode}
\def\ZREF@wu@getcurrent#1{%
  \etex@unexpanded\expandafter\expandafter\expandafter{%
    \ZREF@getcurrent{#1}%
  }%
}
%    \end{macrocode}
%    \end{macro}
%    \begin{macro}{\zref@getcurrent}
%    \begin{macrocode}
\let\zref@getcurrent\ZREF@getcurrent
%    \end{macrocode}
%    \end{macro}
%
% \subsubsection{Reference generation}
%
%    \begin{macro}{\zref@label}
%    Label macro that uses the main property list.
%    \begin{macrocode}
\ZREF@Robust\def\zref@label#1{%
  \zref@labelbylist{#1}\ZREF@mainlist
}
%    \end{macrocode}
%    \end{macro}
%    \begin{macro}{\zref@labelbylist}
%    Label macro that stores the properties, specified in the
%    property list \verb|#2|.
%    \begin{macrocode}
\ZREF@Robust\def\zref@labelbylist#1#2{%
  \@bsphack
    \zref@listexists{#2}{%
      \expandafter\expandafter\expandafter\ZREF@label
      \expandafter\expandafter\expandafter{%
        \csname Z@L@#2\endcsname
      }{#1}%
    }%
  \@esphack
}
%    \end{macrocode}
%    \end{macro}
%    \begin{macro}{\zref@labelbyprops}
%    The properties are directly specified in a comma separated list.
%    \begin{macrocode}
\ZREF@Robust\def\zref@labelbyprops#1#2{%
  \@bsphack
    \begingroup
      \toks@{}%
      \comma@parse{#2}{%
        \zref@ifpropundefined\comma@entry{%
          \PackageWarning\ZREF@name{%
            Property `\comma@entry' is not known%
          }%
        }{%
          \toks@\expandafter{%
            \the\expandafter\toks@\csname\comma@entry\endcsname
          }%
        }%
        \ltx@gobble
      }%
    \expandafter\endgroup
    \expandafter\ZREF@label\expandafter{\the\toks@}{#1}%
  \@esphack
}
%    \end{macrocode}
%    \end{macro}
%    \begin{macro}{\zref@labelbykv}
%    \begin{macrocode}
\ZREF@Robust\def\zref@labelbykv#1#2{%
  \@bsphack
    \begingroup
      \let\Z@L@ZREF@temp\ltx@empty
      \kvsetkeys{ZREF@LABEL}{#1}%
      \ifZREF@immediate
        \expandafter\zref@wrapper@immediate\expandafter{%
          \expandafter\ZREF@label\expandafter{\Z@L@ZREF@temp}{#2}%
        }%
      \else
        \expandafter\ZREF@label\expandafter{\Z@L@ZREF@temp}{#2}%
      \fi
    \endgroup
  \@esphack
}
%    \end{macrocode}
%    \end{macro}
%    \begin{macrocode}
\kv@define@key{ZREF@LABEL}{prop}{%
  \edef\ZREF@param{#1}%
  \zref@propexists\ZREF@param{%
    \zref@iflistcontainsprop{ZREF@temp}\ZREF@param{}{%
      \begingroup\expandafter\endgroup
      \expandafter\ltx@LocalAppendToMacro
      \expandafter\Z@L@ZREF@temp
      \expandafter{\csname\ZREF@param\endcsname}%
    }%
  }%
}
\kv@define@key{ZREF@LABEL}{list}{%
  \zref@listforloop{#1}{%
    \zref@iflistcontainsprop{ZREF@temp}\zref@prop{}{%
      \begingroup\expandafter\endgroup
      \expandafter\ltx@LocalAppendToMacro
      \expandafter\Z@L@ZREF@temp
      \expandafter{\csname\zref@prop\endcsname}%
    }%
    \ltx@gobble
  }%
}
\kv@define@key{ZREF@LABEL}{delprop}{%
  \zref@propexists{#1}{%
    \zref@localdelprop{ZREF@temp}{#1}%
  }%
}
\kv@define@key{ZREF@LABEL}{immediate}[true]{%
  \edef\ZREF@param{#1}%
  \ifx\ZREF@param\ZREF@true
    \ZREF@immediatetrue
  \else
    \ifx\ZREF@param\ZREF@false
      \ZREF@immediatefalse
    \else
      \PackageWarning\ZREF@name{%
        Option `immediate' expects `true' or `false'.\MessageBreak
        Ignoring invalid value `\ZREF@param'%
      }%
    \fi
  \fi
}
%    \end{macrocode}
%    \begin{macro}{\ZREF@false}
%    \begin{macrocode}
\def\ZREF@false{false}
%    \end{macrocode}
%    \end{macro}
%    \begin{macro}{\ZREF@true}
%    \begin{macrocode}
\def\ZREF@true{true}
%    \end{macrocode}
%    \end{macro}
%
%    \begin{macrocode}
\kv@define@key{ZREF@LABEL}{values}[]{%
  \kv@parse{#1}{%
    \ifx\kv@value\relax
      \@PackageWarning\ZREF@name{%
        Missing value for property `\kv@key'%
      }%
      \expandafter\ltx@gobbletwo
    \else
      \expandafter\zref@setcurrent
    \fi
  }%
}
%    \end{macrocode}
%
%    \begin{macro}{\ifZREF@immediate}
%    The switch \cs{ifZREF@immediate} tells us, whether the
%    label should be written immediately or at page shipout time.
%    \cs{ZREF@label} need to be notified about this, because
%    it must disable the deferred execution of property values,
%    if the label is written immediately.
%    \begin{macrocode}
\newif\ifZREF@immediate
%    \end{macrocode}
%    \end{macro}
%    \begin{macro}{\zref@wrapper@immediate}
%    The argument of \cs{zref@wrapper@immediate} is executed inside
%    a group where \cs{write} is redefined by adding
%    \cs{immediate} before its execution. Also \cs{ZREF@label}
%    is notified via the switch \cs{ifZREF@immediate}.
%    \begin{macrocode}
\ZREF@Robust{\long\def}\zref@wrapper@immediate#1{%
  \begingroup
    \ZREF@immediatetrue
    \let\ZREF@org@write\write
    \def\write{\immediate\ZREF@org@write}%
    #1%
  \endgroup
}
%    \end{macrocode}
%    \end{macro}
%
%    \begin{macro}{\ZREF@label}
%    \cs{ZREF@label} writes the data in the \xfile{.aux} file.
%    \verb|#1| contains the list of valid properties,
%    \verb|#2| the name of the reference. In case of immediate
%    writing, the deferred execution of property values is
%    disabled. Also \thepage is made expandable in this case.
%    \begin{macrocode}
\def\ZREF@label#1#2{%
  \if@filesw
    \begingroup
      \ifZREF@immediate
        \let\ZREF@org@thepage\thepage
      \fi
      \protected@write\@auxout{%
        \ifZREF@immediate
          \let\thepage\ZREF@org@thepage
        \fi
        \let\ZREF@temp\ltx@empty
        \@tfor\ZREF@P:=#1\do{%
          \begingroup
            \escapechar=-1 %
            \edef\x{\endgroup
              \def\noexpand\ZREF@P{%
                \expandafter\string\ZREF@P
              }%
            }%
          \x
          \expandafter\ifx
              \csname
                \ifZREF@immediate
                  relax%
                \else
                  Z@X@\ZREF@P%
                \fi
              \endcsname
              \noexpand
            \expandafter\let\csname Z@C@\ZREF@P\endcsname\relax
          \fi
          \toks@\expandafter{\ZREF@temp}%
          \edef\ZREF@temp{%
            \the\toks@
            \ltx@backslashchar\ZREF@P{%
              \expandafter\noexpand\csname Z@C@\ZREF@P\endcsname
            }%
          }%
        }%
      }{%
        \string\zref@newlabel{#2}{\ZREF@temp}%
      }%
    \endgroup
  \fi
}
\def\ZREF@addtoks#1{%
  \toks@\expandafter\expandafter\expandafter{%
    \expandafter\the\expandafter\toks@#1%
  }%
}
%    \end{macrocode}
%    \end{macro}
%
% \subsubsection{Reference querying and extracting}
%
%    Design goal for the extracting macros is that the extraction
%    process is full expandable. Thus these macros can be used
%    in expandable contexts. But there are problems that cannot
%    be solved by full expandable macros:
%    \begin{itemize}
%    \item In standard \hologo{LaTeX} undefined references sets a flag
%      and generate a warning. Both actions are not expandable.
%    \item Babel's support for its shorthand uses commands that
%      use non-expandable assignments. However currently there
%      is hope, that primitives are added to \pdfTeX\ that
%      allows the detection of contexts. Then the shorthand
%      can detect, if they are executed inside \cs{csname}
%      and protect themselves automatically.
%    \end{itemize}
%    \begin{macro}{\zref@ifrefundefined}
%    If a reference \verb|#1| is undefined, then macro
%    \cs{zref@ifrefundefined} calls \verb|#2| and \verb|#3| otherwise.
%    \begin{macrocode}
\def\zref@ifrefundefined#1{%
  \ltx@ifundefined{Z@R@#1}%
}
%    \end{macrocode}
%    \end{macro}
%
%    \begin{macro}{\zifrefundefined}
%    If a reference \verb|#1| is undefined, then macro
%    \cs{zref@ifrefundefined} calls \verb|#2| and \verb|#3| otherwise.
%    Also the reference is marked used.
%    \begin{macrocode}
\ZREF@IfDefinable\zifrefundefined\def{%
  #1{%
    \zref@wrapper@babel\ZREF@ifrefundefined{#1}%
  }%
}
%    \end{macrocode}
%    \end{macro}
%    \begin{macro}{\ZREF@ifrefundefined}
%    \begin{macrocode}
\def\ZREF@ifrefundefined#1{%
  \zref@refused{#1}%
  \zref@ifrefundefined{#1}%
}
%    \end{macrocode}
%    \end{macro}
%
%    \begin{macro}{\zref@refused}
%    The problem with undefined references is addressed by
%    the macro \cs{zref@refused}. This can be used outside
%    the expandable context. In case of an undefined reference
%    the flag is set to notify \hologo{LaTeX} and a warning is given.
%    \begin{macrocode}
\ZREF@Robust\def\zref@refused#1{%
  \zref@wrapper@babel\ZREF@refused{#1}%
}
%    \end{macrocode}
%    \end{macro}
%    \begin{macro}{\ZREF@refused}
%    \begin{macrocode}
\def\ZREF@refused#1{%
  \zref@ifrefundefined{#1}{%
    \protect\G@refundefinedtrue
    \@latex@warning{%
      Reference `#1' on page \thepage \space undefined%
    }%
  }{}%
}
%    \end{macrocode}
%    \end{macro}
%
%    \begin{macro}{\zref@ifrefcontainsprop}
%    \cs{zref@ifrefcontainsprop} looks, if the reference \verb|#1|
%    has the property \verb|#2| and calls then \verb|#3| and
%    \verb|#4| otherwise.
%    \begin{macrocode}
\def\zref@ifrefcontainsprop#1#2{%
  \zref@ifrefundefined{#1}{%
    \ltx@secondoftwo
  }{%
    \expandafter\ZREF@ifrefcontainsprop
    \csname Z@E@#2\expandafter\endcsname
    \csname#2\expandafter\expandafter\expandafter\endcsname
    \expandafter\expandafter\expandafter{%
      \csname Z@R@#1\endcsname
    }%
  }%
}
\def\ZREF@ifrefcontainsprop#1#2#3{%
  \expandafter\ifx\expandafter\ZREF@novalue
  #1#3#2\ZREF@novalue\ZREF@nil\ltx@empty
    \expandafter\ltx@secondoftwo
  \else
    \expandafter\ltx@firstoftwo
  \fi
}
\def\ZREF@novalue{\ZREF@NOVALUE}
%    \end{macrocode}
%    \end{macro}
%
%    \begin{macro}{\zref@extract}
%    \cs{zref@extract} is an abbreviation for the case
%    that the default of the property is used as default
%    value.
%    \begin{macrocode}
\def\ZREF@extract#1#2{%
  \romannumeral0%
  \ltx@ifundefined{Z@D@#2}{%
    \expandafter\ltx@space\zref@default
  }{%
    \expandafter\expandafter\expandafter\ZREF@@extract
    \expandafter\expandafter\expandafter{%
      \csname Z@D@#2\endcsname
    }{#1}{#2}%
  }%
}
%    \end{macrocode}
%    \end{macro}
%    \begin{macro}{\ZREF@@extract}
%    \begin{macrocode}
\def\ZREF@@extract#1#2#3{%
  \expandafter\expandafter\expandafter\ltx@space
  \zref@extractdefault{#2}{#3}{#1}%
}
%    \end{macrocode}
%    \end{macro}
%    \begin{macro}{\ZREF@wu@extract}
%    \begin{macrocode}
\def\ZREF@wu@extract#1#2{%
  \etex@unexpanded\expandafter\expandafter\expandafter{%
    \ZREF@extract{#1}{#2}%
  }%
}
%    \end{macrocode}
%    \end{macro}
%    \begin{macro}{\zref@extract}
%    \begin{macrocode}
\let\zref@extract\ZREF@extract
%    \end{macrocode}
%    \end{macro}
%
%    \begin{macro}{\ZREF@extractdefault}
%    The basic extracting macro is \cs{zref@extractdefault} with
%    the reference name in \verb|#1|, the property in \verb|#2| and
%    the default value in \verb|#3| in case for problems.
%    \begin{macrocode}
\def\ZREF@extractdefault#1#2#3{%
  \romannumeral0%
  \zref@ifrefundefined{#1}\ltx@firstoftwo{%
    \zref@ifpropundefined{#2}\ltx@firstoftwo\ltx@secondoftwo
  }{%
    \ltx@space
    #3%
  }{%
    \expandafter\expandafter\expandafter\ltx@space
    \csname Z@E@#2\expandafter\expandafter\expandafter\endcsname
    \csname Z@R@#1\expandafter\endcsname
    \csname#2\endcsname{#3}\ZREF@nil
  }%
}
%    \end{macrocode}
%    \end{macro}
%    \begin{macro}{\ZREF@wu@extractdefault}
%    \begin{macrocode}
\def\ZREF@wu@extractdefault#1#2#3{%
  \etex@unexpanded\expandafter\expandafter\expandafter{%
    \ZREF@extractdefault{#1}{#2}{#3}%
  }%
}
%    \end{macrocode}
%    \end{macro}
%    \begin{macro}{\zref@extractdefault}
%    \begin{macrocode}
\let\zref@extractdefault\ZREF@extractdefault
%    \end{macrocode}
%    \end{macro}
%
%    \begin{macro}{\zref@def@extract}
%    \begin{macrocode}
\ZREF@Robust\def\zref@def@extract#1{%
  \zref@wrapper@babel{\ZREF@def@extract{#1}}%
}
%    \end{macrocode}
%    \end{macro}
%    \begin{macro}{\ZREF@def@extract}
%    \begin{macrocode}
\def\ZREF@def@extract#1#2#3{%
  \zref@refused{#2}%
  \expandafter\expandafter\expandafter\def
  \expandafter\expandafter\expandafter#1%
  \expandafter\expandafter\expandafter{%
    \zref@extract{#2}{#3}%
  }%
}
%    \end{macrocode}
%    \end{macro}
%    \begin{macro}{\zref@def@extractdefault}
%    \begin{macrocode}
\ZREF@Robust\def\zref@def@extractdefault#1{%
  \zref@wrapper@babel{\ZREF@def@extractdefault{#1}}%
}
%    \end{macrocode}
%    \end{macro}
%    \begin{macro}{\ZREF@def@extractdefault}
%    \begin{macrocode}
\def\ZREF@def@extractdefault#1#2#3#4{%
  \zref@refused{#2}%
  \expandafter\expandafter\expandafter\def
  \expandafter\expandafter\expandafter#1%
  \expandafter\expandafter\expandafter{%
    \zref@extractdefault{#2}{#3}{#4}%
  }%
}
%    \end{macrocode}
%    \end{macro}
%
%    \begin{macro}{\ZREF@wrapper@unexpanded}
%    \begin{macrocode}
\ZREF@Robust{\long\def}\ZREF@wrapper@unexpanded#1{%
  \let\zref@wrapper@unexpanded\ltx@firstofone
  \let\zref@getcurrent\ZREF@wu@getcurrent
  \let\zref@extractdefault\ZREF@wu@extractdefault
  \let\zref@extract\ZREF@wu@extract
  #1%
  \let\zref@wrapper@unexpanded\ZREF@wrapper@unexpanded
  \let\zref@getcurrent\ZREF@getcurrent
  \let\zref@extractdefault\ZREF@extractdefault
  \let\zref@extract\ZREF@extract
}
%    \end{macrocode}
%    \end{macro}
%    \begin{macro}{\zref@wrapper@unexpanded}
%    \begin{macrocode}
\ltx@IfUndefined{etex@unexpanded}{%
  \let\zref@wrapper@unexpanded\ltx@firstofone
}{%
  \let\zref@wrapper@unexpanded\ZREF@wrapper@unexpanded
}
%    \end{macrocode}
%    \end{macro}
%
% \subsubsection{Compatibility with \xpackage{babel}}
%
%    \begin{macro}{\zref@wrapper@babel}
%    \begin{macrocode}
\ZREF@Robust{\long\def}\zref@wrapper@babel#1#2{%
  \ifcsname if@safe@actives\endcsname
    \expandafter\ltx@firstofone
  \else
    \expandafter\ltx@secondoftwo
  \fi
  {%
    \if@safe@actives
      \expandafter\ltx@secondoftwo
    \else
      \expandafter\ltx@firstoftwo
    \fi
    {%
      \begingroup
        \csname @safe@activestrue\endcsname
        \edef\x{#2}%
      \expandafter\endgroup
      \expandafter\ZREF@wrapper@babel\expandafter{\x}{#1}%
    }%
  }{%
    #1{#2}%
  }%
}
\long\def\ZREF@wrapper@babel#1#2{%
  #2{#1}%
}
%    \end{macrocode}
%    \end{macro}
%
% \subsubsection{Unique counter support}
%
%    \begin{macro}{\zref@require@unique}
%    Generate the counter |zref@unique| if the counter does not
%    already exist.
%    \begin{macrocode}
\ZREF@Robust\def\zref@require@unique{%
  \@ifundefined{c@zref@unique}{%
    \begingroup
      \let\@addtoreset\ltx@gobbletwo
      \newcounter{zref@unique}%
    \endgroup
%    \end{macrocode}
%    \end{macro}
%
%    \begin{macro}{\thezref@unique}
%    \cs{thezref@unique} is used for automatically generated unique
%    labelnames.
%    \begin{macrocode}
    \renewcommand*{\thezref@unique}{%
      zref@\number\c@zref@unique
    }%
  }{}%
}
%    \end{macrocode}
%    \end{macro}
%
% \subsubsection{Utilities}
%
%    \begin{macro}{\ZREF@number}
%    \begin{macrocode}
\ltx@IfUndefined{numexpr}{%
  \def\ZREF@number#1{\number#1}%
}{%
  \def\ZREF@number#1{\the\numexpr(#1)\relax}%
}
%    \end{macrocode}
%    \end{macro}
%
% \subsubsection{Setup}
%
%    \begin{macro}{\zref@setdefault}
%    Standard \hologo{LaTeX} prints ``??'' in bold face if a reference
%    is not known. \cs{zref@default} holds the text that is
%    printed in case of unknown references and is used, if
%    the default was not specified during the definition
%    of the new property by \cs{ref@newprop}.
%    The global default value can be set by \cs{zref@setdefault}.
%    \begin{macrocode}
\ZREF@Robust\def\zref@setdefault#1{%
  \def\zref@default{#1}%
}
%    \end{macrocode}
%    \end{macro}
%    \begin{macro}{\zref@default}
%    Now we initialize \cs{zref@default} with the same value
%    that \hologo{LaTeX} uses for its undefined references.
%    \begin{macrocode}
\zref@setdefault{%
  \nfss@text{\reset@font\bfseries ??}%
}
%    \end{macrocode}
%    \end{macro}
%
% \paragraph{Main property list.}
%
%    \begin{macro}{\zref@setmainlist}
%    The name of the default property list is stored in
%    \cs{ZREF@mainlist} and can be set by \cs{zref@setmainlist}.
%    \begin{macrocode}
\ZREF@Robust\def\zref@setmainlist#1{%
  \def\ZREF@mainlist{#1}%
}
\zref@setmainlist{main}
%    \end{macrocode}
%    \end{macro}
%    Now we create the list.
%    \begin{macrocode}
\zref@newlist\ZREF@mainlist
%    \end{macrocode}
%
% \paragraph{Main properties.}
%
%    The two properties \texttt{default} and \texttt{page}
%    are created and added to the main property list.
%    They store the data that
%    standard \hologo{LaTeX} uses in its references created by
%    \cs{label}.
%    \begin{description}
%    \item[\texttt{default}] the apperance of the latest
%      counter that is incremented by \cs{refstepcounter}
%    \item[\texttt{page}] the apperance of the page counter
%    \end{description}
%    \begin{macrocode}
\zref@newprop{default}{\@currentlabel}
\zref@newprop*{page}{\thepage}
\zref@addprops\ZREF@mainlist{default,page}
%    \end{macrocode}
%
% \paragraph{Properties}
%
%    \begin{macro}{\ZREF@NewPropAnchor}
%    \begin{macrocode}
\def\ZREF@NewPropAnchor{%
  \zref@newprop{anchor}{%
    \ltx@ifundefined{@currentHref}{}{\@currentHref}%
  }%
  \global\let\ZREF@NewPropAnchor\relax
}
%    \end{macrocode}
%    \end{macro}
%    \begin{macro}{\zref@titleref@current}
%    Later we will redefine the section and caption macros
%    to catch the current title and remember the value in
%    \cs{zref@titleref@current}.
%    \begin{macro}{\ZREF@NewPropTitle}
%    \begin{macrocode}
\def\ZREF@NewPropTitle{%
  \gdef\zref@titleref@current{}%
  \zref@newprop{title}{\zref@titleref@current}%
  \global\let\ZREF@NewPropTitle\relax
}
%    \end{macrocode}
%    \end{macro}
%    \end{macro}
%    \begin{macro}{\ZREF@NewPropTheotype}
%    \begin{macrocode}
\def\ZREF@NewPropTheotype{%
  \zref@newprop{theotype}{}%
  \global\let\ZREF@NewPropTheotype\relax
}
%    \end{macrocode}
%    \end{macro}
%    \begin{macro}{\ZREF@NewPropPageValue}
%    \begin{macrocode}
\def\ZREF@NewPropPageValue{%
  \zref@newprop*{pagevalue}[0]{\number\c@page}%
  \global\let\ZREF@NewPropPageValue\relax
}
%    \end{macrocode}
%    \end{macro}
%
% \paragraph{Mark successful loading}
%
%    \begin{macrocode}
\let\ZREF@base@ok=Y
%</base>
%    \end{macrocode}
%
%
% \subsection{Module \xmodule{user}}
%
%    \begin{macrocode}
%<*user>
\NeedsTeXFormat{LaTeX2e}
\ProvidesPackage{zref-user}%
  [2020-07-03 v2.32 Module user for zref (HO)]%
\RequirePackage{zref-base}[2019/11/29]
\ifx\ZREF@base@ok Y%
\else
  \expandafter\endinput
\fi
%    \end{macrocode}
%
%    Module \xmodule{user} enables a small user interface.
%    All macros are prefixed by \cs{z}.
%
%    First we define the pendants to the standard \hologo{LaTeX} referencing
%    commands \cs{label}, \cs{ref}, and \cs{pageref}.
%
%    \begin{macro}{\zlabel}
%    Similar to \cs{label} the macro \cs{zlabel} writes a reference
%    entry in the \xfile{.aux} file. The main property list is used.
%    Also we add the babel patch. The \cs{label} command can also be
%    used inside section titles, but it must not go into the
%    table of contents. Therefore we have to check this situation.
%    \begin{macrocode}
\newcommand*\zlabel{%
  \ifx\label\ltx@gobble
    \expandafter\ltx@gobble
  \else
    \expandafter\zref@wrapper@babel\expandafter\zref@label
  \fi
}%
%    \end{macrocode}
%    \end{macro}
%
%    \begin{macro}{\zkvlabel}
%    \begin{macrocode}
\newcommand*{\zkvlabel}[1]{%
  \ifx\label\ltx@gobble
    \expandafter\ltx@gobblethree
  \fi
  \zref@wrapper@babel{\zref@labelbykv{#1}}%
}%
%    \end{macrocode}
%    \end{macro}
%
%    \begin{macro}{\zref}
%    Macro \cs{zref} is the corresponding macro for \cs{ref}.
%    Also it provides an optional argument in order to
%    select another property.
%    \begin{macrocode}
\newcommand*{\zref}[2][default]{% robust because of optional argument
  \zref@propexists{#1}{%
    \zref@wrapper@babel\ZREF@zref{#2}{#1}%
  }%
}%
\def\ZREF@zref#1{%
  \zref@refused{#1}%
  \zref@extract{#1}%
}%
%    \end{macrocode}
%    \end{macro}
%
%    \begin{macro}{\zpageref}
%    For macro \cs{zpageref} we just call \cs{zref}
%    with property \verb|page|.
%    \begin{macrocode}
\ZREF@IfDefinable\zpageref\def{%
  {\zref[page]}%
}
%    \end{macrocode}
%    \end{macro}
%
%    \begin{macro}{\zrefused}
%    For the following expandible user macros \cs{zrefused}
%    should be used to notify \hologo{LaTeX} in case of undefined
%    references.
%    \begin{macrocode}
\ZREF@IfDefinable\zrefused\def{%
  {\zref@refused}%
}
%    \end{macrocode}
%    \end{macro}
%
%    \begin{macrocode}
%</user>
%    \end{macrocode}
%
%
% \subsection{Module \xmodule{abspage}}
%
%    \begin{macrocode}
%<*abspage>
\NeedsTeXFormat{LaTeX2e}
\ProvidesPackage{zref-abspage}%
  [2020-07-03 v2.32 Module abspage for zref (HO)]%
\RequirePackage{zref-base}[2019/11/29]
\ifx\ZREF@base@ok Y%
\else
  \expandafter\endinput
\fi
%    \end{macrocode}
%
%    Module \xmodule{abspage} adds a new property \verb|abspage|
%    to the \verb|main| property list for absolute page numbers.
%    These are recorded by the help of package \xpackage{atbegshi}.
%    \begin{macrocode}
\RequirePackage{atbegshi}[2011/10/05]%
%    \end{macrocode}
%    The counter \verb|abspage| must not go in the clear list
%    of \verb|@ckpt| that is used to set counters in
%    \xfile{.aux} files of included \hologo{TeX} files.
%    \begin{macrocode}
\begingroup
  \let\@addtoreset\ltx@gobbletwo
  \newcounter{abspage}%
\endgroup
\setcounter{abspage}{0}%
\AtBeginShipout{%
  \stepcounter{abspage}%
}%
\zref@newprop*{abspage}[0]{\the\c@abspage}%
\zref@addprop\ZREF@mainlist{abspage}%
%    \end{macrocode}
%    Note that counter \verb|abspage| shows the
%    previous page during page processing. Before
%    shipout the counter is incremented.
%    Thus the property is correctly written with
%    deferred writing. If the counter is written
%    using \cs{zref@wrapper@immediate}, then the number
%    is too small by one.
%
%    \begin{macrocode}
%</abspage>
%    \end{macrocode}
%
%
% \subsection{Module \xmodule{counter}}
%
%    \begin{macrocode}
%<*counter>
\NeedsTeXFormat{LaTeX2e}
\ProvidesPackage{zref-counter}%
  [2020-07-03 v2.32 Module counter for zref (HO)]%
\RequirePackage{zref-base}[2019/11/29]
\ifx\ZREF@base@ok Y%
\else
  \expandafter\endinput
\fi
%    \end{macrocode}
%
%    For features such as \xpackage{hyperref}'s \cs{autoref}
%    we need the name of the counter. The property
%    \verb|counter| is defined and added to the main property
%    list. Starting with \LaTeX\ 2020-10-01 the proper can use
%    \@currentcounter. In older formats
%    \cs{refstepcounter} has to be patched but this can fail in some
%    cases, see issue \#5.
%    \begin{macrocode}
\@ifl@t@r\fmtversion{2020-10-01}
 {
  \zref@newprop{counter}{\@currentcounter}
  \zref@addprop\ZREF@mainlist{counter}
 }
 {
  \zref@newprop{counter}{}
  \zref@addprop\ZREF@mainlist{counter}
  \AtBeginDocument{%
    \ZREF@patch{refstepcounter}{%
      \def\refstepcounter#1{%
        \zref@setcurrent{counter}{#1}%
        \ZREF@org@refstepcounter{#1}%
      }%
    }%
  }
 }
%    \end{macrocode}
%    \begin{macrocode}
%</counter>
%    \end{macrocode}
%
%
% \subsection{Module \xmodule{lastpage}}
%
%    \begin{macrocode}
%<*lastpage>
\NeedsTeXFormat{LaTeX2e}
\ProvidesPackage{zref-lastpage}%
  [2020-07-03 v2.32 Module lastpage for zref (HO)]%
\RequirePackage{zref-base}[2019/11/29]
\RequirePackage{zref-abspage}[2019/11/29]
\RequirePackage{atveryend}[2009/12/07]
\ifx\ZREF@base@ok Y%
\else
  \expandafter\endinput
\fi
%    \end{macrocode}
%
%    The module \xmodule{lastpage} implements the service
%    of package \xpackage{lastpage} by setting a reference
%    \verb|LastPage| at the end of the document.
%    If module \xmodule{abspage} is given, also the absolute
%    page number is available, because the properties of
%    the main property list are used.
%    \begin{macrocode}
\zref@newlist{LastPage}
\AfterLastShipout{%
  \if@filesw
    \begingroup
      \advance\c@page\m@ne
      \toks@\expandafter\expandafter\expandafter{%
        \expandafter\Z@L@main
        \Z@L@LastPage
      }%
      \expandafter\zref@wrapper@immediate\expandafter{%
        \expandafter\ZREF@label\expandafter{\the\toks@}{LastPage}%
      }%
    \endgroup
  \fi
}
%    \end{macrocode}
%    \begin{macro}{\zref@iflastpage}
%    \begin{macrocode}
\def\zref@iflastpage#1{%
  \ifnum\zref@extractdefault{#1}{abspage}{-1}=%
        \zref@extractdefault{LastPage}{abspage}{-2} %
    \expandafter\ltx@firstoftwo
  \else
    \expandafter\ltx@secondoftwo
  \fi
}
%    \end{macrocode}
%    \end{macro}
%    \begin{macro}{\ziflastpage}
%    \begin{macrocode}
\ZREF@IfDefinable\ziflastpage\def{%
  {\zref@wrapper@babel\ZREF@iflastpage}%
}
%    \end{macrocode}
%    \end{macro}
%    \begin{macro}{ZREF@iflastpage}
%    \begin{macrocode}
\def\ZREF@iflastpage#1{%
  \zref@refused{LastPage}%
  \zref@refused{#1}%
  \zref@iflastpage{#1}%
}
%    \end{macrocode}
%    \end{macro}
%    \begin{macrocode}
%</lastpage>
%    \end{macrocode}
%
%
% \subsection{Module \xmodule{thepage}}
%
%    \begin{macrocode}
%<*thepage>
\NeedsTeXFormat{LaTeX2e}
\ProvidesPackage{zref-thepage}%
  [2020-07-03 v2.32 Module thepage for zref (HO)]%
\RequirePackage{zref-base}[2019/11/29]
\ifx\ZREF@base@ok Y%
\else
  \expandafter\endinput
\fi
%    \end{macrocode}
%
%    \begin{macrocode}
\RequirePackage{atbegshi}[2011/10/05]
\RequirePackage{zref-abspage}[2019/11/29]
%    \end{macrocode}
%    \begin{macrocode}
\zref@newlist{thepage}
\zref@addprop{thepage}{page}
\ZREF@NewPropPageValue
%    \end{macrocode}
%    \begin{macro}{\zref@thepage@atbegshi@hook}
%    \begin{macrocode}
\let\zref@thepage@atbegshi@hook\ltx@empty
%    \end{macrocode}
%    \end{macro}
%    \begin{macrocode}
\zref@addprop{thepage}{pagevalue}
\AtBeginShipout{%
  \AtBeginShipoutAddToBox{%
    \zref@thepage@atbegshi@hook
    \zref@labelbylist{thepage\the\value{abspage}}{thepage}%
  }%
}
%    \end{macrocode}
%    \begin{macro}{\zref@thepage@name}
%    \begin{macrocode}
\ltx@IfUndefined{numexpr}{%
  \def\zref@thepage@name#1{thepage\number#1}%
}{%
  \def\zref@thepage@name#1{thepage\the\numexpr#1}%
}
%    \end{macrocode}
%    \end{macro}
%    \begin{macro}{\zref@thepage}
%    \begin{macrocode}
\def\zref@thepage#1{%
  \zref@extract{\zref@thepage@name{#1}}{page}%
}%
%    \end{macrocode}
%    \end{macro}
%    \begin{macro}{\zref@thepage@refused}
%    \begin{macrocode}
\ZREF@Robust\def\zref@thepage@refused#1{%
  \zref@refused{\zref@thepage@name{#1}}%
}%
%    \end{macrocode}
%    \end{macro}
%    \begin{macro}{\zthepage}
%    \begin{macrocode}
\ZREF@IfDefinable\zthepage\def{%
  #1{%
    \zref@thepage@refused{#1}%
    \zref@thepage{#1}%
  }%
}
%    \end{macrocode}
%    \end{macro}
%    \begin{macrocode}
%</thepage>
%    \end{macrocode}
%
%
% \subsection{Module \xmodule{nextpage}}
%
%    \begin{macrocode}
%<*nextpage>
\NeedsTeXFormat{LaTeX2e}
\ProvidesPackage{zref-nextpage}%
  [2020-07-03 v2.32 Module nextpage for zref (HO)]%
\RequirePackage{zref-base}[2019/11/29]
\ifx\ZREF@base@ok Y%
\else
  \expandafter\endinput
\fi
%    \end{macrocode}
%
%    \begin{macrocode}
\RequirePackage{zref-abspage}[2019/11/29]
\RequirePackage{zref-thepage}[2019/11/29]
\RequirePackage{zref-lastpage}[2019/11/29]
\RequirePackage{uniquecounter}[2009/12/18]
%    \end{macrocode}
%    \begin{macrocode}
\UniqueCounterNew{znextpage}

\newcommand*{\znextpagesetup}{%
  \afterassignment\ZREF@np@setup@i
  \def\ZREF@np@call@unknown##1%
}
\def\ZREF@np@setup@i{%
  \afterassignment\ZREF@np@setup@ii
  \def\ZREF@np@call@nonext##1%
}
\def\ZREF@np@setup@ii{%
  \def\ZREF@np@call@next##1%
}
\def\ZREF@np@call@unknown#1{#1}
\def\ZREF@np@call@nonext#1{#1}
\def\ZREF@np@call@next#1{#1}
\ZREF@IfDefinable\znextpage\def{%
  {\UniqueCounterCall{znextpage}{\ZREF@nextpage}}%
}%
\newcommand*{\znonextpagename}{}
\newcommand*{\zunknownnextpagename}{\Z@D@page}
\def\ZREF@nextpage#1{%
  \begingroup
    \def\ZREF@refname@this{zref@np#1}%
    \zref@labelbyprops\ZREF@refname@this{abspage}%
    \chardef\ZREF@call=0 % unknown
    \ZREF@ifrefundefined\ZREF@refname@this{%
    }{%
      \edef\ZREF@pagenum@this{%
        \zref@extractdefault\ZREF@refname@this{abspage}{0}%
      }%
      \edef\ZREF@refname@next{%
        \zref@thepage@name{%
          \the\numexpr\ZREF@pagenum@this+1%
        }%
      }%
      \ifnum\ZREF@pagenum@this>0 %
        \ZREF@ifrefundefined{LastPage}{%
          \zref@ifrefundefined\ZREF@refname@next{%
          }{%
            \chardef\ZREF@call=2 % next page
          }%
        }{%
          \edef\ZREF@pagenum@last{%
            \zref@extractdefault{LastPage}{abspage}{0}%
          }%
          \ifnum\ZREF@pagenum@this<\ZREF@pagenum@last\ltx@space
            \ZREF@ifrefundefined\ZREF@refname@next{%
            }{%
              \chardef\ZREF@call=2 % next page
            }%
          \else
            \ifnum\ZREF@pagenum@this=\ZREF@pagenum@this\ltx@space
              \chardef\ZREF@call=1 % no next page
            \fi
          \fi
        }%
      \fi
    }%
    \edef\x{%
      \endgroup
      \ifcase\ZREF@call
        \noexpand\ZREF@np@call@unknown{%
          \noexpand\zunknownnextpagename
        }%
      \or
        \noexpand\ZREF@np@call@nonext{%
          \noexpand\znonextpagename
        }%
      \else
        \noexpand\ZREF@np@call@next{%
          \noexpand\zref@extract{\ZREF@refname@next}{page}%
        }%
      \fi
    }%
  \x
}
%    \end{macrocode}
%    \begin{macrocode}
%</nextpage>
%    \end{macrocode}
%
%
% \subsection{Module \xmodule{totpages}}
%
%    \begin{macrocode}
%<*totpages>
\NeedsTeXFormat{LaTeX2e}
\ProvidesPackage{zref-totpages}%
  [2020-07-03 v2.32 Module totpages for zref (HO)]%
\RequirePackage{zref-base}[2019/11/29]
\ifx\ZREF@base@ok Y%
\else
  \expandafter\endinput
\fi
%    \end{macrocode}
%
%    The absolute page number of the last page is the total
%    page number.
%    \begin{macrocode}
\RequirePackage{zref-abspage}[2019/11/29]
\RequirePackage{zref-lastpage}[2019/11/29]
%    \end{macrocode}
%
%    \begin{macro}{\ztotpages}
%    Macro \cs{ztotpages} contains the number of pages. It can be
%    used inside expandable calculations. It expands to zero if
%    the reference is not yet available.
%    \begin{macrocode}
\newcommand*{\ztotpages}{%
  \zref@extractdefault{LastPage}{abspage}{0}%
}
%    \end{macrocode}
%    \end{macro}
%    Also we mark the reference |LastPage| as used:
%    \begin{macrocode}
\AtBeginDocument{%
  \zref@refused{LastPage}%
}
%    \end{macrocode}
%    \begin{macrocode}
%</totpages>
%    \end{macrocode}
%
%
% \subsection{Module \xmodule{pagelayout}}
%
%    \begin{macrocode}
%<*pagelayout>
\NeedsTeXFormat{LaTeX2e}
\ProvidesPackage{zref-pagelayout}%
  [2020-07-03 v2.32 Module pagelayout for zref (HO)]%
\RequirePackage{zref-base}[2019/11/29]
\ifx\ZREF@base@ok Y%
\else
  \expandafter\endinput
\fi
%    \end{macrocode}
%
%    \begin{macrocode}
\RequirePackage{zref-thepage}[2019/11/29]
\RequirePackage{iftex}[2019/11/07]%
\RequirePackage{atveryend}[2010/03/24]
%    \end{macrocode}
%
%
% \subsubsection{Define layout properties}
%
%    \begin{macrocode}
\def\ZREF@temp#1{%
  \begingroup
    \escapechar=-1 %
  \ltx@ifundefined{\string#1}{\endgroup}{%
    \edef\x{%
      \endgroup
      \noexpand\zref@newprop*{\string#1}%
                             [\noexpand\number\noexpand#1]% hash-ok
                             {\noexpand\number\noexpand#1}%
      \noexpand\zref@addprop{thepage}{\string#1}%
    }%
    \x
  }%
}
\ZREF@temp\mag
\ZREF@temp\paperwidth
\ZREF@temp\paperheight
\ZREF@temp\stockwidth % memoir.cls, crop.sty
\ZREF@temp\stockheight % memoir.cls, crop.sty
\ZREF@temp\mediawidth % VTeX
\ZREF@temp\mediaheight % VTeX
\ifluatex
\zref@newprop*{pdfvorigin}%
                             [\number\pdfvariable vorigin]% hash-ok
                             {\number\pdfvariable vorigin}%
\zref@addprop{thepage}{pdfvorigin}
\zref@newprop*{pdfhorigin}%
                             [\number\pdfvariable horigin]% hash-ok
                             {\number\pdfvariable horigin}%
\zref@addprop{thepage}{pdfhorigin}
\zref@newprop*{pdfpageheight}%
                             [\number\pageheight]% hash-ok
                             {\number\pageheight}%
\zref@addprop{thepage}{pdfpageheight}
\zref@newprop*{pdfpagewidth}%
                             [\number\pagewidth]% hash-ok
                             {\number\pagewidth}%
\zref@addprop{thepage}{pdfpagewidth}
\else
\ZREF@temp\pdfpagewidth
\ZREF@temp\pdfpageheight
\ZREF@temp\pdfhorigin
\ZREF@temp\pdfvorigin
\fi
\ZREF@temp\hoffset
\ZREF@temp\voffset
\ZREF@temp\topmargin
\ZREF@temp\oddsidemargin
\ZREF@temp\evensidemargin
\ZREF@temp\textwidth
\ZREF@temp\textheight
\ZREF@temp\headheight
\ZREF@temp\headsep
\ZREF@temp\footskip
\ZREF@temp\marginparwidth
\ZREF@temp\marginparsep
\ZREF@temp\columnwidth
\ZREF@temp\columnsep
\ZREF@temp\trimedge % memoir.cls
\ZREF@temp\spinemargin % memoir.cls
\ZREF@temp\foremargin % memoir.cls
\ZREF@temp\trimtop % memoir.cls
\ZREF@temp\uppermargin % memoir.cls
\ZREF@temp\headmargin % memoir.cls
%    \end{macrocode}
%
%    \begin{macrocode}
\IfFormatAtLeastTF{2020/10/01}
 {
   \zref@newprop*{outputboxwd}[0pt]{\ShipoutBoxWidth}
   \zref@newprop*{outputboxht}[0pt]{\ShipoutBoxHeight}
   \zref@newprop*{outputboxdp}[0pt]{\ShipoutBoxDepth}
 }
 {
   \zref@newprop*{outputboxwd}[0pt]{\AtBeginShipoutBoxWidth}
   \zref@newprop*{outputboxht}[0pt]{\AtBeginShipoutBoxHeight}
   \zref@newprop*{outputboxdp}[0pt]{\AtBeginShipoutBoxDepth}
 }
%    \end{macrocode}
%    \begin{macrocode}
\zref@addprops{thepage}{outputboxwd,outputboxht,outputboxdp}
%    \end{macrocode}
%
%    \begin{macro}{\ifZREF@pl@list}
%    \begin{macrocode}
\ltx@newif\ifZREF@pl@list
%    \end{macrocode}
%    \end{macro}
%    \begin{macro}{\zref@listpagelayout}
%    \begin{macrocode}
\ZREF@IfDefinable\zlistpagelayout\def{%
  {\global\ZREF@pl@listtrue}%
}
%    \end{macrocode}
%    \end{macro}
%    \begin{macro}{\ZREF@pl@AfterLastShipout}
%    \begin{macrocode}
\def\ZREF@pl@AfterLastShipout{%
  \ifZREF@pl@list
    \edef\ZREF@page@max{\the\value{abspage}}%
    \ltx@ifundefined{ZREF@org@testdef}{%
      \let\ZREF@org@testdef\@testdef
      \def\@testdef##1##2##3{%
        \ZREF@org@testdef{##1}{##2}{##3}%
        \def\ZREF@temp{##1}%
        \ifx\ZREF@temp\ZREF@RefPrefix
          \expandafter\gdef\csname##1@##2\endcsname{##3}%
        \fi
      }%
    }{}%
    \AtVeryEndDocument{\ZREF@pl@AtVeryEnd}%
  \fi
}
%    \end{macrocode}
%    \end{macro}
%    \begin{macro}{\ZREF@pl@AtVeryEnd}
%    \begin{macrocode}
\def\ZREF@pl@AtVeryEnd{%
  \begingroup
    \toks@{Page layout parameters:\MessageBreak}%
    \count@=1 %
    \ZREF@pl@ListPage
    \edef\x{\endgroup
      \noexpand\@PackageInfoNoLine{zref-pagelayout}{\the\toks@}%
    }%
  \x
}
%    \end{macrocode}
%    \end{macro}
%    \begin{macro}{\ZREF@pl@ListPage}
%    \begin{macrocode}
\def\ZREF@pl@ListPage{%
  \edef\x{%
    \toks@={%
      \the\toks@
      Page \the\count@:\noexpand\MessageBreak
      \zref@ifrefundefined{thepage\the\count@}{}{%
        \ltx@space\ltx@space mag = %
        \zref@extract{thepage\the\count@}{mag}%
        \noexpand\MessageBreak
        \ZREF@pl@ListEntry{paperwidth}%
        \ZREF@pl@ListEntry{paperheight}%
        \ZREF@pl@ListEntry{stockwidth}%
        \ZREF@pl@ListEntry{stockheight}%
        \ZREF@pl@ListEntry{mediawidth}%
        \ZREF@pl@ListEntry{mediaheight}%
        \ZREF@pl@ListEntry{pdfpagewidth}%
        \ZREF@pl@ListEntry{pdfpageheight}%
        \ZREF@pl@ListEntry{pdfhorigin}%
        \ZREF@pl@ListEntry{pdfvorigin}%
        \ZREF@pl@ListEntry{hoffset}%
        \ZREF@pl@ListEntry{voffset}%
        \ZREF@pl@ListEntry{topmargin}%
        \ZREF@pl@ListEntry{oddsidemargin}%
        \ZREF@pl@ListEntry{evensidemargin}%
        \ZREF@pl@ListEntry{textwidth}%
        \ZREF@pl@ListEntry{textheight}%
        \ZREF@pl@ListEntry{headheight}%
        \ZREF@pl@ListEntry{headsep}%
        \ZREF@pl@ListEntry{footskip}%
        \ZREF@pl@ListEntry{marginparwidth}%
        \ZREF@pl@ListEntry{marginparsep}%
        \ZREF@pl@ListEntry{columnwidth}%
        \ZREF@pl@ListEntry{columnsep}%
        \ZREF@pl@ListEntry{trimedge}%
        \ZREF@pl@ListEntry{spinemargin}%
        \ZREF@pl@ListEntry{foremargin}%
        \ZREF@pl@ListEntry{trimtop}%
        \ZREF@pl@ListEntry{uppermargin}%
        \ZREF@pl@ListEntry{headmargin}%
      }%
    }%
  }\x
  \ifnum\ZREF@page@max>\count@
    \advance\count@ by\ltx@one
  \else
    \expandafter\ltx@gobble
  \fi
  \ZREF@pl@ListPage
}
%    \end{macrocode}
%    \end{macro}
%    \begin{macro}{\ZREF@pl@ListEntry}
%    \begin{macrocode}
\def\ZREF@pl@ListEntry#1{%
  \zref@ifpropundefined{#1}{%
  }{%
    \zref@ifrefcontainsprop{thepage\the\count@}{#1}{%
      \ltx@space\ltx@space#1 = %
      \zref@extract{thepage\the\count@}{#1}sp = %
      \the\dimexpr\zref@extract{thepage\the\count@}{#1}sp\relax
      \noexpand\MessageBreak
    }{}%
  }%
}
%    \end{macrocode}
%    \end{macro}
%    \begin{macrocode}
\AfterLastShipout{%
  \ZREF@pl@AfterLastShipout
}
%    \end{macrocode}
%
%    \begin{macrocode}
%</pagelayout>
%    \end{macrocode}
%
%
% \subsection{Module \xmodule{pageattr}}
%
%    \begin{macrocode}
%<*pageattr>
\NeedsTeXFormat{LaTeX2e}
\ProvidesPackage{zref-pageattr}%
  [2020-07-03 v2.32 Module pageattr for zref (HO)]%
\RequirePackage{zref-base}[2019/11/29]
\ifx\ZREF@base@ok Y%
\else
  \expandafter\endinput
\fi
%    \end{macrocode}
%
%    \begin{macrocode}
\RequirePackage{iftex}[2019/11/07]%
%    \end{macrocode}
%
%
%    \begin{macrocode}
\let\ZREF@temp=N%
\ifluatex
\expandafter\@firstoftwo
\else
\expandafter\@secondoftwo
\fi
{%luatex
 \RequirePackage{zref-thepage}[2019/11/29]
 \RequirePackage{zref-lastpage}[2019/11/29]%
 \zref@newprop*{pdfpageattr}[]{\zref@hex{\the\pdfvariable pageattr}}%
 \zref@addprop{thepage}{pdfpageattr}%
 \zref@newprop*{pdfpagesattr}[]{\zref@hex{\the\pdfvariable pagesattr}}%
 \zref@addprop{LastPage}{pdfpagesattr}%
 \let\ZREF@temp=Y%
}
{%other
 \ltx@IfUndefined{pdfpageattr}{%
  \@PackageInfoNoLine{zref-pageattr}{%
    \string\pdfpageattr\space is not available%
  }%
  \def\zref@pdfpageattr#1{}%
  \def\zref@pdfpageattr@used#1{}%
 }{%
  \RequirePackage{zref-thepage}[2019/11/29]%
  \zref@newprop*{pdfpageattr}[]{\zref@hex{\the\pdfpageattr}}%
  \zref@addprop{thepage}{pdfpageattr}%
  \let\ZREF@temp=Y%
 }
 \ltx@IfUndefined{pdfpagesattr}{%
  \@PackageInfoNoLine{zref-pageattr}{%
    \string\pdfpagesattr\space is not available%
  }%
  \def\zref@pdfpagesattr{}%
  \def\zref@pdfpagesattr@used{}%
 }{%
  \RequirePackage{zref-lastpage}[2019/11/29]%
  \zref@newprop*{pdfpagesattr}[]{\zref@hex{\the\pdfpagesattr}}%
  \zref@addprop{LastPage}{pdfpagesattr}%
  \let\ZREF@temp=Y%
 }%
}%
%    \end{macrocode}
%    \begin{macrocode}
\ifx\ZREF@temp N%
  \expandafter\endinput
\fi
%    \end{macrocode}
%    \begin{macrocode}
\RequirePackage{zref-abspage}[2019/11/29]
\RequirePackage{atveryend}[2010/03/24]
\RequirePackage{pdftexcmds}[2010/04/01]
\let\ZREF@temp=Y%
\ltx@IfUndefined{pdf@escapehex}{\let\ZREF@temp=N}{}
\ltx@IfUndefined{pdf@unescapehex}{\let\ZREF@temp=N}{}
\ifx\ZREF@temp N%
  \let\zref@hex\ltx@firstofone
  \let\zref@unhex\ltx@firstofone
\else
  \let\zref@hex\pdf@escapehex
  \let\zref@unhex\pdf@unescapehex
\fi
%    \end{macrocode}
%
%    \begin{macro}{\ifZREF@pa@list}
%    \begin{macrocode}
\ltx@newif\ifZREF@pa@list
%    \end{macrocode}
%    \end{macro}
%    \begin{macro}{\zref@listpageattr}
%    \begin{macrocode}
\ZREF@IfDefinable\zlistpageattr\def{%
  {\ZREF@pa@listtrue}%
}
%    \end{macrocode}
%    \end{macro}
%    \begin{macro}{\ZREF@pa@AfterLastShipout}
%    \begin{macrocode}
\def\ZREF@pa@AfterLastShipout{%
  \ifZREF@pa@list
    \edef\ZREF@page@max{\the\value{abspage}}%
    \ltx@ifundefined{ZREF@org@testdef}{%
      \let\ZREF@org@testdef\@testdef
      \def\@testdef##1##2##3{%
        \ZREF@org@testdef{##1}{##2}{##3}%
        \def\ZREF@temp{##1}%
        \ifx\ZREF@temp\ZREF@RefPrefix
          \expandafter\xdef\csname##1@##2\endcsname{##3}%
        \fi
      }%
    }{}%
    \AtVeryEndDocument{\ZREF@pa@AtVeryEnd}%
  \fi
}
%    \end{macrocode}
%    \end{macro}
%    \begin{macro}{\ZREF@pa@AtVeryEnd}
%    \begin{macrocode}
\let\ZREF@temp=Y%
\ltx@IfUndefined{pdfpageattr}{}{\let\ZREF@temp=N}
\ifluatex \let\ZREF@temp=N \fi
\ifx\ZREF@temp Y
 \expandafter\@firstoftwo
\else
 \expandafter\@secondoftwo
\fi
 {%
  \def\ZREF@pa@AtVeryEnd{}%
 }
 {%
  \def\ZREF@pa@AtVeryEnd{%
    \begingroup
      \toks@{List of \ltx@backslashchar
             \ifluatex pdfvariable\else pdf\fi
             pdfpageattr:\MessageBreak}%
      \count@=1 %
      \ZREF@pa@ListPage
      \edef\x{\endgroup
        \noexpand\@PackageInfoNoLine{zref-pageattr}{%
          \the\toks@
        }%
      }%
    \x
  }%
%    \end{macrocode}
%    \end{macro}
%    \begin{macro}{\zref@pageattr}
%    \begin{macrocode}
\def\zref@pdfpageattr#1{%
  \zref@unhex{%
    \zref@extract{thepage\ZREF@number{#1}}{pdfpageattr}%
  }%
}
% compability, \zref@pageattr was defined in older versions
\let\zref@pageattr\zref@pdfpageattr
%    \end{macrocode}
%    \end{macro}
%    \begin{macro}{\zref@pageattr@used}
%    \begin{macrocode}
\ZREF@Robust\def\zref@pageattr@used#1{%
  \zref@refused{thepage\ZREF@number{#1}}%
}
%    \end{macrocode}
%    \end{macro}
%    \begin{macro}{\ZREF@pa@ListPage}
%    \begin{macrocode}
  \def\ZREF@pa@ListPage{%
    \edef\x{%
      \toks@={%
        \the\toks@
        Page \the\count@:%
        \noexpand\MessageBreak
        \zref@ifrefundefined{thepage\the\count@}{}{%
          <<\zref@pdfpageattr\count@>>%
          \noexpand\MessageBreak
        }%
      }%
    }\x
    \ifnum\ZREF@page@max>\count@
      \advance\count@ by\ltx@one
    \else
      \expandafter\ltx@gobble
    \fi
    \ZREF@pa@ListPage
  }%
}
%    \end{macrocode}
%    \end{macro}
%    \begin{macrocode}
\let\ZREF@temp=Y%
\ltx@IfUndefined{pdfpagesattr}{}{\let\ZREF@temp=N}
\ifluatex \let\ZREF@temp=N \fi
\ifx\ZREF@temp N
 \expandafter\@firstofone
\fi
 {%
%    \end{macrocode}
%    \begin{macro}{\zref@pdfpagesattr}
%    \begin{macrocode}
  \def\zref@pdfpagesattr{%
    \zref@unhex{%
      \zref@extract{LastPage}{pdfpagesattr}%
    }%
  }%
%    \end{macrocode}
%    \end{macro}
%    \begin{macro}{\zref@pdfpagesattr@used}
%    \begin{macrocode}
  \ZREF@Robust\def\zref@pdfpagesattr@used{%
    \zref@refused{LastPage}%
  }%
%    \end{macrocode}
%    \end{macro}
%    \begin{macrocode}
  \ltx@LocalAppendToMacro\ZREF@pa@AtVeryEnd{%
    \@PackageInfoNoLine{zref-pageattr}{%
      \ltx@backslashchar
      \ifluatex pdfvariable\else pdf\fi
      pagesattr:\MessageBreak
      <<\zref@pdfpagesattr>>%
      \MessageBreak
    }%
  }%
}
\AfterLastShipout{%
  \ZREF@pa@AfterLastShipout
}
%    \end{macrocode}
%    \begin{macrocode}
%</pageattr>
%    \end{macrocode}
%
%
% \subsection{Module \xmodule{marks}}
%
%    \begin{macrocode}
%<*marks>
\NeedsTeXFormat{LaTeX2e}
\ProvidesPackage{zref-marks}%
  [2020-07-03 v2.32 Module marks for zref (HO)]%
\RequirePackage{zref-base}[2019/11/29]
\ifx\ZREF@base@ok Y%
\else
  \expandafter\endinput
\fi
%    \end{macrocode}
%
%    \begin{macrocode}
\newcommand*{\zref@marks@register}[3][]{%
  \edef\ZREF@TempName{#1}%
  \edef\ZREF@TempNum{\ZREF@number{#2}}%
  \ifnum\ZREF@TempNum<\ltx@zero %
    \PackageError\ZREF@name{%
      \string\zref@marks@register\ltx@space is called with invalid%
      \MessageBreak
      marks register number (\ZREF@TempNum)%
    }{%
      Use `0' or the command, defined by \string\newmarks.\MessageBreak
      \@ehc
    }%
  \else
    \ifx\ZREF@TempName\ltx@empty
      \edef\ZREF@TempName{mark\romannumeral\ZREF@TempNum}%
    \else
      \edef\ZREF@TempName{marks\ZREF@TempName}%
    \fi
    \ZREF@MARKS@DefineProp{top}%
    \ZREF@MARKS@DefineProp{first}%
    \ZREF@MARKS@DefineProp{bot}%
    \kv@parse{#3}{%
      \ifx\kv@value\relax
        \def\kv@value{top,first,bot}%
      \fi
      \edef\ZREF@temp{\expandafter\ltx@car\kv@key X\@nil}%
      \ifx\ZREF@temp\ZREF@STAR
        \edef\kv@key{\expandafter\ltx@cdr\kv@key\@nil}%
        \zref@newlist\kv@key
      \fi
      \expandafter\comma@parse\expandafter{\kv@value}{%
        \ifcase0\ifx\comma@entry\ZREF@NAME@top 1\else
                \ifx\comma@entry\ZREF@NAME@first 1\else
                \ifx\comma@entry\ZREF@NAME@bot 1\fi\fi\fi\ltx@space
          \PackageWarning{zref-marks}{%
            Use `top', `first' or `bot' for the list values%
            \MessageBreak
            in the third argument of \string\zref@marks@register.%
            \MessageBreak
            Ignoring unkown value `\comma@entry'%
          }%
        \else
          \zref@addprop{\kv@key}{\comma@entry\ZREF@TempName}%
        \fi
        \ltx@gobble
      }%
      \ltx@gobbletwo
    }%
  \fi
}
\def\ZREF@STAR{*}
\def\ZREF@NAME@top{top}
\def\ZREF@NAME@first{first}
\def\ZREF@NAME@bot{bot}
\def\ZREF@MARKS@DefineProp#1{%
  \zref@ifpropundefined{#1\ZREF@TempName}{%
    \ifnum\ZREF@TempNum=\ltx@zero
      \begingroup
        \edef\x{\endgroup
          \noexpand\zref@newprop*{#1\ZREF@TempName}[]{%
            \expandafter\noexpand\csname#1mark\endcsname
          }%
        }%
      \x
    \else
      \begingroup
        \edef\x{\endgroup
          \noexpand\zref@newprop*{#1\ZREF@TempName}[]{%
            \expandafter\noexpand\csname#1marks\endcsname
            \ZREF@TempNum
          }%
        }%
      \x
    \fi
  }{%
    \PackageWarning{zref-marks}{%
      \string\zref@marks@register\ltx@space does not generate the%
      \MessageBreak
      new property `#1\ZREF@TempName', because\MessageBreak
      it is already defined%
    }%
  }%
}
%    \end{macrocode}
%    \begin{macrocode}
%</marks>
%    \end{macrocode}
%
% \subsection{Module \xmodule{runs}}
%
%    This module does not use the label-reference-system.
%    The reference changes with each \hologo{LaTeX}~run and would
%    force a rerun warning always.
%    \begin{macrocode}
%<*runs>
\NeedsTeXFormat{LaTeX2e}
\ProvidesPackage{zref-runs}%
  [2020-07-03 v2.32 Module runs for zref (HO)]%
%    \end{macrocode}
%
%    \begin{macro}{\zruns}
%    \begin{macrocode}
\providecommand*{\zruns}{0}%
\AtBeginDocument{%
  \edef\zruns{\number\numexpr\zruns+1}%
  \begingroup
    \def\on@line{}%
    \PackageInfo{zref-runs}{LaTeX runs: \zruns}%
    \if@filesw
      \immediate\write\@mainaux{%
        \string\gdef\string\zruns{\zruns}%
      }%
    \fi
  \endgroup
}
%    \end{macrocode}
%    \end{macro}
%
%    \begin{macrocode}
%</runs>
%    \end{macrocode}
%
%
% \subsection{Module \xmodule{perpage}}
%
%    \begin{macrocode}
%<*perpage>
\NeedsTeXFormat{LaTeX2e}
\ProvidesPackage{zref-perpage}%
  [2020-07-03 v2.32 Module perpage for zref (HO)]%
\RequirePackage{zref-base}[2019/11/29]
\ifx\ZREF@base@ok Y%
\else
  \expandafter\endinput
\fi
%    \end{macrocode}
%
%    This module resets a counter at page boundaries.
%    Because of the asynchronous output routine
%    page counter properties cannot be asked directly,
%    references are necessary.
%
%    For detecting changed pages module \xmodule{abspage}
%    is loaded.
%    \begin{macrocode}
\RequirePackage{zref-abspage}[2019/11/29]
%    \end{macrocode}
%
%    We group the properties for the needed references
%    in the property list |perpage|. The property |pagevalue|
%    records the correct value of the page counter.
%    \begin{macrocode}
\ZREF@NewPropPageValue
\zref@newlist{perpage}
\zref@addprops{perpage}{abspage,page,pagevalue}
%    \end{macrocode}
%    The page value, known by the reference mechanism, will be
%    stored in counter |zpage|.
%    \begin{macrocode}
\newcounter{zpage}
%    \end{macrocode}
%    Counter |zref@unique| helps in generating unique reference
%    names.
%    \begin{macrocode}
\zref@require@unique
%    \end{macrocode}
%    In order to be able to reset the counter, we hook here
%    into \cs{stepcounter}. In fact two nested hooks are used
%    to allow other packages to use the first hook at the beginning
%    of \cs{stepcounter}.
%    \begin{macrocode}
\let\ZREF@org@stepcounter\stepcounter
\def\stepcounter#1{%
  \ifcsname @stepcounterhook@#1\endcsname
    \csname @stepcounterhook@#1\endcsname
  \fi
  \ZREF@org@stepcounter{#1}%
}
%    \end{macrocode}
%    \cs{@stpelt} must be adapted due to the change in latex
%    2015-01, see https://github.com/ho-tex/zref/issues/26
%    \begin{macrocode}
\let\ZREF@org@@stpelt\@stpelt
\def\@stpelt#1{%
  \ifcsname ZREF@perpage@#1\endcsname
    \begingroup
      \let\stepcounter\ZREF@org@stepcounter
      \ZREF@org@@stpelt{#1}%
    \endgroup
    \expandafter\ltx@gobbletwo
  \fi
  \ZREF@org@@stpelt{#1}%
}
%    \end{macrocode}
%
%    \begin{macro}{\zmakeperpage}
%    Makro \cs{zmakeperpage} resets a counter at each page
%    break. It uses the same syntax and semantics as
%    \cs{MakePerPage} from package \xpackage{perpage} \cite{perpage}.
%    The initial start value can be given by the optional
%    argument. Default is one that means after the first
%    \cs{stepcounter} on a new page the counter starts with
%    one.
%    \begin{macrocode}
\ZREF@IfDefinable\zmakeperpage\def{%
  {%
    \@ifnextchar[\ZREF@makeperpage@opt{\ZREF@@makeperpage[\ltx@zero]}%
  }%
}
%    \end{macrocode}
%    \end{macro}
%    We hook before the counter is incremented in \cs{stepcounter},
%    package \xpackage{perpage} afterwards. Thus a little
%    calculation is necessary.
%    \begin{macrocode}
\def\ZREF@makeperpage@opt[#1]{%
  \begingroup
    \edef\x{\endgroup
      \noexpand\ZREF@@makeperpage[\number\numexpr#1-1\relax]%
    }%
  \x
}
%    \end{macrocode}
%    \begin{macrocode}
\def\ZREF@@makeperpage[#1]#2{%
  \@ifundefined{@stepcounterhook@#2}{%
    \expandafter\gdef\csname @stepcounterhook@#2\endcsname{}%
  }{}%
  \expandafter\gdef\csname ZREF@perpage@#2\endcsname{%
    \ZREF@@perpage@step{#2}{#1}%
  }%
  \expandafter\g@addto@macro\csname @stepcounterhook@#2\endcsname{%
    \ifcsname ZREF@perpage@#2\endcsname
      \csname ZREF@perpage@#2\endcsname
    \fi
  }%
}
%    \end{macrocode}
%    \begin{macro}{\ZREF@@perpage@step}
%    The heart of this module follows.
%    \begin{macrocode}
\def\ZREF@@perpage@step#1#2{%
%    \end{macrocode}
%    First the reference is generated.
%    \begin{macrocode}
  \global\advance\c@zref@unique\ltx@one
  \begingroup
    \expandafter
    \zref@labelbylist\expandafter{\thezref@unique}{perpage}%
%    \end{macrocode}
%    The \cs{expandafter} commands are necessary, because \cs{ZREF@temp}
%    is also used inside of \cs{zref@labelbylist}.
%
%    The evaluation of the reference follows. If the reference
%    is not yet kwown, we use the page counter as approximation.
%    \begin{macrocode}
    \zref@ifrefundefined\thezref@unique{%
      \global\c@zpage=\c@page
      \global\let\thezpage\thepage
      \expandafter\xdef\csname ZREF@abspage@#1\endcsname{%
        \number\c@abspage
      }%
    }{%
%    \end{macrocode}
%    The reference is used to set \cs{thezpage} and counter |zpage|.
%    \begin{macrocode}
      \global\c@zpage=\zref@extract\thezref@unique{pagevalue}\relax
      \xdef\thezpage{\noexpand\zref@extract{\thezref@unique}{page}}%
      \expandafter\xdef\csname ZREF@abspage@#1\endcsname{%
        \zref@extractdefault\thezref@unique
            {abspage}{\number\c@abspage}%
      }%
    }%
%    \end{macrocode}
%    Page changes are detected by a changed absolute page number.
%    \begin{macrocode}
    \expandafter\ifx\csname ZREF@abspage@#1\expandafter\endcsname
                    \csname ZREF@currentabspage@#1\endcsname
    \else
      \global\csname c@#1\endcsname=#2\relax
      \global\expandafter\let
          \csname ZREF@currentabspage@#1\expandafter\endcsname
          \csname ZREF@abspage@#1\endcsname
    \fi
  \endgroup
}
%    \end{macrocode}
%    \end{macro}
%    \begin{macro}{\zunmakeperpage}
%    Macro \cs{zunmakeperpage} cancels the effect of \cs{zmakeperpage}.
%    \begin{macrocode}
\ZREF@IfDefinable\zunmakeperpage\def{%
  #1{%
    \global\expandafter
    \let\csname ZREF@perpage@#1\endcsname\@undefined
  }%
}
%    \end{macrocode}
%    \end{macro}
%
%    \begin{macrocode}
%</perpage>
%    \end{macrocode}
%
%
% \subsection{Module \xmodule{titleref}}
%
%    \begin{macrocode}
%<*titleref>
\NeedsTeXFormat{LaTeX2e}
\ProvidesPackage{zref-titleref}%
  [2020-07-03 v2.32 Module titleref for zref (HO)]%
\RequirePackage{zref-base}[2019/11/29]
\ifx\ZREF@base@ok Y%
\else
  \expandafter\endinput
\fi
\RequirePackage{gettitlestring}[2009/12/08]
%    \end{macrocode}
%
% \subsubsection{Implementation}
%    \begin{macrocode}
\RequirePackage{keyval}
%    \end{macrocode}
%
%    This module makes section and caption titles available
%    for the reference system. It uses some of the ideas
%    of package \xpackage{nameref} and \xpackage{titleref}.
%
%    Now we can add the property \verb|title| is added to
%    the main property list.
%    \begin{macrocode}
\ZREF@NewPropTitle
\zref@addprop\ZREF@mainlist{title}%
%    \end{macrocode}
%
%    The title strings go into the \xfile{.aux} file, thus
%    they need some kind of protection.
%    Package \xpackage{titleref} uses a protected expansion
%    method. The advantage is that this can be used to
%    cleanup the string and to remove \cs{label}, \cs{index}
%    and other macros unwanted for referencing.
%    But there is the risk that fragile stuff can break.
%
%    Therefore package \xpackage{nameref} does not expand
%    the string. Thus the entries can safely be written to
%    the \xfile{.aux} file. But potentially dangerous
%    macros such as \cs{label} remain in the string and
%    can cause problems when using the string in references.
%
%    \begin{macro}{\ifzref@titleref@expand}
%    The switch \cs{ifzref@titleref@expand} distinguishes
%    between the both methods. Package \xpackage{nameref}'s
%    behaviour is achieved by setting the switch to false,
%    otherwise \xpackage{titleref}'s expansion is used.
%    Default is false.
%    \begin{macrocode}
\newif\ifzref@titleref@expand
%    \end{macrocode}
%    \end{macro}
%
%    \begin{macro}{\ZREF@titleref@hook}
%    The hook \cs{ZREF@titleref@hook} allows to extend
%    the cleanup for the expansion method. Thus
%    unnecessary macros can be removed or dangerous commands
%    removed. The hook is executed before the expansion of
%    \cs{zref@titleref@current}.
%    \begin{macrocode}
\let\ZREF@titleref@hook\ltx@empty
%    \end{macrocode}
%    \end{macro}
%    \begin{macro}{\zref@titleref@cleanup}
%    The hook should not be used directly, instead we provide the
%    macro \cs{zref@titleref@cleanup} to add stuff to the hook
%    and prevents that a previous non-empty content is not discarded
%    accidently.
%    \begin{macrocode}
\ZREF@Robust\def\zref@titleref@cleanup#1{%
   \begingroup
    \toks@\expandafter{%
      \ZREF@titleref@hook
      #1%
    }%
  \expandafter\endgroup
  \expandafter\def\expandafter\ZREF@titleref@hook\expandafter{%
    \the\toks@
  }%
}%
%    \end{macrocode}
%    \end{macro}
%
%    \begin{macro}{\ifzref@titleref@stripperiod}
%    Sometimes a title contains a period at the end.
%    Package \xpackage{nameref} removes this. This behaviour
%    is controlled by the switch \cs{ifzref@titleref@stripperiod}
%    and works regardless of the setting of option \xoption{expand}.
%    Period stripping is the default.
%    \begin{macrocode}
\newif\ifzref@titleref@stripperiod
\zref@titleref@stripperiodtrue
%    \end{macrocode}
%    \end{macro}
%
%    \begin{macro}{\zref@titleref@setcurrent}
%    Macro \cs{zref@titleref@setcurrent} sets a new current title
%    stored in \cs{zref@titleref@current}. Some cleanup and
%    expansion is performed that can be controlled by the
%    previous switches.
%    \begin{macrocode}
\ZREF@Robust\def\zref@titleref@setcurrent#1{%
  \ifzref@titleref@expand
    \GetTitleStringExpand{#1}%
  \else
    \GetTitleStringNonExpand{#1}%
  \fi
  \edef\zref@titleref@current{%
    \detokenize\expandafter{\GetTitleStringResult}%
  }%
  \ifzref@titleref@stripperiod
    \edef\zref@titleref@current{%
      \expandafter\ZREF@stripperiod\zref@titleref@current
      \ltx@empty.\ltx@empty\@nil
    }%
  \fi
}%
\GetTitleStringDisableCommands{%
  \ZREF@titleref@hook
}
%    \end{macrocode}
%    \end{macro}
%    \begin{macro}{\ZREF@stripperiod}
%    If \cs{ZREF@stripperiod} is called, the argument consists of
%    space tokens and tokens with catcode 12 (other), because
%    of \hologo{eTeX}'s \cs{detokenize}.
%    \begin{macrocode}
\def\ZREF@stripperiod#1.\ltx@empty#2\@nil{#1}%
%    \end{macrocode}
%    \end{macro}
%
% \subsubsection{User interface}
%
%    \begin{macro}{\ztitlerefsetup}
%    The behaviour of module \xmodule{titleref} is controlled
%    by switches and a hook. They can be set by \cs{ztitlerefsetup}
%    with a key value interface, provided by package \xpackage{keyval}.
%    Also the current title can be given explicitly by the key
%    \verb|title|.
%    \begin{macrocode}
\define@key{ZREF@TR}{expand}[true]{%
  \csname zref@titleref@expand#1\endcsname
}%
\define@key{ZREF@TR}{stripperiod}[true]{%
  \csname zref@titleref@stripperiod#1\endcsname
}%
\define@key{ZREF@TR}{cleanup}{%
  \zref@titleref@cleanup{#1}%
}%
\define@key{ZREF@TR}{title}{%
  \def\zref@titleref@current{#1}%
}%
\ZREF@IfDefinable\ztitlerefsetup\def{%
  {\kvsetkeys{ZREF@TR}}%
}%
%    \end{macrocode}
%    \end{macro}
%
%    \begin{macro}{\ztitleref}
%    The user command \cs{ztitleref} references the title.
%    For safety \cs{label} is disabled to prevent multiply
%    defined references.
%    \begin{macrocode}
\ZREF@IfDefinable\ztitleref\def{%
  {\zref@wrapper@babel\ZREF@titleref}%
}%
\def\ZREF@titleref#1{%
  \begingroup
    \zref@refused{#1}%
    \let\label\ltx@gobble
    \zref@extract{#1}{title}%
  \endgroup
}%
%    \end{macrocode}
%    \end{macro}
%
% \subsubsection{Patches for section and caption commands}
%
%    The section and caption macros are patched to extract the
%    title data.
%
%    Captions of figures and tables.
%    \begin{macrocode}
\AtBeginDocument{%
  \ZREF@patch{@caption}{%
    \long\def\@caption#1[#2]{%
      \zref@titleref@setcurrent{#2}%
      \ZREF@org@@caption{#1}[{#2}]%
    }%
  }%
%    \end{macrocode}
%
%    Section commands without star. The title version for the
%    table of contents is used because it is usually shorter and
%    more robust.
%    \begin{macrocode}
  \ZREF@patch{@part}{%
    \def\@part[#1]{%
      \zref@titleref@setcurrent{#1}%
      \ZREF@org@@part[{#1}]%
    }%
  }%
  \ZREF@patch{@chapter}{%
    \def\@chapter[#1]{%
      \zref@titleref@setcurrent{#1}%
      \ZREF@org@@chapter[{#1}]%
    }%
  }%
  \ZREF@patch{@sect}{%
    \def\@sect#1#2#3#4#5#6[#7]{%
      \zref@titleref@setcurrent{#7}%
      \ZREF@org@@sect{#1}{#2}{#3}{#4}{#5}{#6}[{#7}]%
    }%
  }%
%    \end{macrocode}
%    The star versions of the section commands.
%    \begin{macrocode}
  \ZREF@patch{@spart}{%
    \def\@spart#1{%
      \zref@titleref@setcurrent{#1}%
      \ZREF@org@@spart{#1}%
    }%
  }%
  \ZREF@patch{@schapter}{%
    \def\@schapter#1{%
      \zref@titleref@setcurrent{#1}%
      \ZREF@org@@schapter{#1}%
    }%
  }%
  \ZREF@patch{@ssect}{%
    \def\@ssect#1#2#3#4#5{%
      \zref@titleref@setcurrent{#5}%
      \ZREF@org@@ssect{#1}{#2}{#3}{#4}{#5}%
    }%
  }%
%    \end{macrocode}
%
% \subsubsection{Environment \textsf{description}}
%
%    \begin{macrocode}
  \ZREF@patch{descriptionlabel}{%
    \def\descriptionlabel#1{%
      \zref@titleref@setcurrent{#1}%
      \ZREF@org@descriptionlabel{#1}%
    }%
  }%
%    \end{macrocode}
%
% \subsubsection{Class \xclass{memoir}}
%
%    \begin{macrocode}
  \@ifclassloaded{memoir}{%
    \ltx@IfUndefined{ifheadnameref}{}{%
      \def\@chapter[#1]#2{%
        \ltx@IfUndefined{ch@pt@c}{%
          \zref@titleref@setcurrent{#1}%
        }{%
          \ifx\ch@pt@c\ltx@empty
            \zref@titleref@setcurrent{#2}%
          \else
            \def\NR@temp{#1}%
            \ifx\NR@temp\ltx@empty
              \expandafter\zref@titleref@setcurrent
              \expandafter{\ch@pt@c}%
            \else
              \ifheadnameref
                \zref@titleref@setcurrent{#1}%
              \else
                \expandafter\zref@titleref@setcurrent
                \expandafter{\ch@pt@c}%
              \fi
            \fi
          \fi
        }%
        \ZREF@org@@chapter[{#1}]{#2}%
      }%
      \ZREF@patch{M@sect}{%
        \def\M@sect#1#2#3#4#5#6[#7][#8]{%
          \ifheadnameref
            \zref@titleref@setcurrent{#8}%
          \else
            \zref@titleref@setcurrent{#7}%
          \fi
          \ZREF@org@M@sect{#1}{#2}{#3}{#4}{#5}{#6}[{#7}][{#8}]%
        }%
      }%
    }%
  }{}%
%    \end{macrocode}
%
% \subsubsection{Class \xclass{beamer}}
%
%    \begin{macrocode}
  \@ifclassloaded{beamer}{%
    \ZREF@patch{beamer@section}{%
      \long\def\beamer@section[#1]{%
        \zref@titleref@setcurrent{#1}%
        \ZREF@org@beamer@section[{#1}]%
      }%
    }%
    \ZREF@patch{beamer@subsection}{%
      \long\def\beamer@subsection[#1]{%
        \zref@titleref@setcurrent{#1}%
        \ZREF@org@beamer@subsection[{#1}]%
      }%
    }%
    \ZREF@patch{beamer@subsubsection}{%
      \long\def\beamer@subsubsection[#1]{%
        \zref@titleref@setcurrent{#1}%
        \ZREF@org@beamer@subsubsection[{#1}]%
      }%
    }%
  }{}%
%    \end{macrocode}
%
% \subsubsection{Package \xpackage{titlesec}}
%
%    \begin{macrocode}
  \@ifpackageloaded{titlesec}{%
    \ZREF@patch{ttl@sect@i}{%
      \def\ttl@sect@i#1#2[#3]#4{%
        \zref@titleref@setcurrent{#4}%
        \ZREF@org@ttl@sect@i{#1}{#2}[{#3}]{#4}%
      }%
    }%
    \ZREF@patch{ttl@straight@i}{%
      \def\ttl@straight@i#1[#2]#3{%
        \def\ZREF@temp{#2}%
        \ifx\ZREF@temp\ltx@empty
          \zref@titleref@setcurrent{#3}%
        \else
          \zref@titleref@setcurrent{#2}%
        \fi
        \ZREF@org@ttl@straight@i{#1}[{#2}]{#3}%
      }%
    }%
  }{}%
%    \end{macrocode}
%
% \subsubsection{Package \xpackage{longtable}}
%
%    Package \xpackage{longtable}: some support for its \cs{caption}.
%    However \cs{label} inside the caption is not supported.
%    \begin{macrocode}
  \@ifpackageloaded{longtable}{%
    \ZREF@patch{LT@c@ption}{%
      \def\LT@c@ption#1[#2]#3{%
        \ZREF@org@LT@c@ption{#1}[{#2}]{#3}%
        \zref@titleref@setcurrent{#2}%
      }%
    }%
  }{}%
%    \end{macrocode}
%
% \subsubsection{Package \xpackage{listings}}
%
%    Package \xpackage{listings}: support for its caption.
%    \begin{macrocode}
  \@ifpackageloaded{listings}{%
    \ZREF@patch{lst@MakeCaption}{%
      \def\lst@MakeCaption{%
        \ifx\lst@label\ltx@empty
        \else
          \expandafter\zref@titleref@setcurrent\expandafter{%
            \lst@@caption
          }%
        \fi
        \ZREF@org@lst@MakeCaption
      }%
    }%
  }{}%
%    \end{macrocode}
%
% \subsubsection{Theorems}
%
%    \begin{macrocode}
  \ZREF@patch{@opargbegintheorem}{%
    \def\@opargbegintheorem#1#2#3{%
      \zref@titleref@setcurrent{#3}%
      \ZREF@org@@opargbegintheorem{#1}{#2}{#3}%
    }%
  }%
  \@ifpackageloaded{amsthm}{%
    \begingroup
      \edef\x{macro:\string#1\string#2[\string#3]}%
      \@onelevel@sanitize\x
      \def\y#1->#2\@nil{#1}%
      \edef\z{\expandafter\y\meaning\@begintheorem->\@nil}%
      \@onelevel@sanitize\z
    \expandafter\endgroup
    \ifx\x\z
      \ZREF@patch{@begintheorem}{%
        \def\@begintheorem#1#2[#3]{%
          \zref@titleref@setcurrent{#3}%
          \ZREF@org@@begintheorem{#1}{#2}[{#3}]%
        }%
      }%
    \fi
  }{}%
%    \end{macrocode}
%
%    \begin{macrocode}
}
%    \end{macrocode}
%
%    \begin{macrocode}
%</titleref>
%    \end{macrocode}
%
%
% \subsection{Module \xmodule{xr}}
%
%    \begin{macrocode}
%<*xr>
\NeedsTeXFormat{LaTeX2e}
\ProvidesPackage{zref-xr}%
  [2020-07-03 v2.32 Module xr for zref (HO)]%
\RequirePackage{zref-base}[2019/11/29]
\ifx\ZREF@base@ok Y%
\else
  \expandafter\endinput
\fi
%    \end{macrocode}
%    \begin{macrocode}
\RequirePackage{keyval}
\RequirePackage{kvoptions}[2010/02/22]
%    \end{macrocode}
%
%    We declare property |url|, because this is added,
%    if a reference is imported and has not already set this field.
%    Or if |hyperref| is used, then this property can be asked.
%    \begin{macrocode}
\zref@newprop{url}{}%
\zref@newprop{urluse}{}%
%    \end{macrocode}
%    \begin{macrocode}
\zref@newprop{externaldocument}{}%
%    \end{macrocode}
%
%    Most code, especially the handling of the \xfile{.aux}
%    files are taken from David Carlisle's \xpackage{xr} package.
%    Therefore I drop the documentation for these macros here.
%
%    \begin{macro}{\zref@xr@ext}
%    If the URL is not specied, then assume
%    processed file with a guessed extension.
%    Use the setting of \xpackage{hyperref} if available.
%    \begin{macrocode}
\providecommand*{\zref@xr@ext}{%
  \ltx@ifundefined{XR@ext}{pdf}{\XR@ext}%
}%
%    \end{macrocode}
%    \end{macro}
%    \begin{macro}{\ifZREF@xr@zreflabel}
%    The use of the star form of \cs{zexternaldocument}
%    is remembered in the switch \cs{ifZREF@xr@zreflabel}.
%    \begin{macrocode}
\newif\ifZREF@xr@zreflabel
%    \end{macrocode}
%    \end{macro}
%    \begin{macrocode}
\SetupKeyvalOptions{%
  family=ZREF@XR,%
  prefix=ZREF@xr@%
}
\DeclareBoolOption[true]{tozreflabel}
\DeclareBoolOption[false]{toltxlabel}
\DeclareBoolOption{verbose}
\define@key{ZREF@XR}{ext}{%
  \def\zref@xr@{#1}%
}
\DeclareBoolOption[false]{urluse}
%    \end{macrocode}
%    \begin{macro}{\zxrsetup}
%    \begin{macrocode}
\newcommand*{\zxrsetup}{%
  \kvsetkeys{ZREF@XR}%
}%
%    \end{macrocode}
%    \end{macro}
%
%    \begin{macro}{\ZREF@xr@URL}
%    \begin{macrocode}
\newcount\ZREF@xr@URL
\ZREF@xr@URL=\ltx@zero
%    \end{macrocode}
%    \end{macro}
%    \begin{macro}{\ZREF@xr@AddURL}
%    \begin{macrocode}
\def\ZREF@xr@AddURL#1{%
  \begingroup
    \def\ZREF@temp{#1}%
    \count@=\ltx@one
    \ZREF@xr@@AddUrl
  \endgroup
}
%    \end{macrocode}
%    \end{macro}
%    \begin{macro}{\ZREF@xr@@AddUrl}
%    \begin{macrocode}
\def\ZREF@xr@@AddUrl{%
  \ifnum\count@>\ZREF@xr@URL
    \global\advance\ZREF@xr@URL by\ltx@one
    \xdef\ZREF@xr@theURL{\romannumeral\ZREF@xr@URL}%
    \global\expandafter\let
        \csname Z@U@\ZREF@xr@theURL\endcsname\ZREF@temp
    \@PackageInfo{zref-xr}{%
      \ltx@backslashchar Z@U@\ZREF@xr@theURL:\MessageBreak
      \ZREF@temp\MessageBreak
    }%
  \else
    \expandafter
    \ifx\csname Z@U@\romannumeral\count@\endcsname\ZREF@temp
      \xdef\ZREF@xr@theURL{\romannumeral\count@}%
    \else
      \expandafter\expandafter\expandafter\ZREF@xr@@AddUrl
    \fi
  \fi
}
%    \end{macrocode}
%    \end{macro}
%
%    \begin{macro}{\zexternaldocument}
%    In its star form it looks for \cs{newlabel},
%    otherwise for \cs{zref@newlabel}.
%     Later we
%    will read \xfile{.aux} files that expects
%    |@| to have catcode 11 (letter).
%    \begin{macrocode}
\ZREF@IfDefinable\zexternaldocument\def{%
  {%
    \ZREF@NewPropAnchor
    \ZREF@NewPropTitle
    \begingroup
      \csname @safe@actives@true\endcsname
      \makeatletter
      \@ifstar{%
        \ZREF@xr@zreflabelfalse
        \@testopt\ZREF@xr@externaldocument{}%
      }{%
        \ZREF@xr@zreflabeltrue
        \@testopt\ZREF@xr@externaldocument{}%
      }%
  }%
}%
%    \end{macrocode}
%    \end{macro}
%
%    If the \cs{include} featuer was used, there can be
%    several \xfile{.aux} files. These files are read
%    one after another, especially they are not recursively
%    read in order to save read registers. Thus it can
%    happen that the read order of the newlabel commands
%    differs from \hologo{LaTeX}'s order using \cs{input}.
%
%    \begin{macro}{\ZREF@xr@externaldocument}
%    It reads the remaining arguments. \cs{newcommand}
%    comes in handy for the optional argument.
%    \begin{macrocode}
\def\ZREF@xr@externaldocument[#1]#2{%
    \def\ZREF@xr@prefix{#1}%
    \let\ZREF@xr@filelist\ltx@empty
    \edef\ZREF@xr@externalfile{#2}%
    \edef\ZREF@xr@file{\ZREF@xr@externalfile.aux}%
    \filename@parse{#2}%
    \@testopt\ZREF@xr@graburl{#2.\zref@xr@ext}%
}%
\def\ZREF@xr@graburl[#1]{%
    \edef\ZREF@xr@url{#1}%
    \ifZREF@xr@urluse
      \expandafter\ZREF@xr@AddURL\expandafter{\ZREF@xr@url}%
      \expandafter\def\expandafter\ZREF@xr@url
      \expandafter{\csname Z@U@\ZREF@xr@theURL\endcsname}%
    \fi
    \ZREF@xr@checkfile
  \endgroup
}%
%    \end{macrocode}
%    \end{macro}
%    \begin{macro}{\ZREF@xr@processfile}
%    We follow \xpackage{xr} here, \cs{IfFileExists} offers
%    a nicer test, but we have to open the file anyway.
%    \begin{macrocode}
\def\ZREF@xr@checkfile{%
  \openin\@inputcheck\ZREF@xr@file\relax
  \ifeof\@inputcheck
    \PackageWarning{zref-xr}{%
      File `\ZREF@xr@file' not found or empty,\MessageBreak
      labels not imported%
    }%
  \else
    \PackageInfo{zref-xr}{%
      Label \ifZREF@xr@zreflabel (zref) \fi
      import from `\ZREF@xr@file'%
    }%
    \def\ZREF@xr@found{0}%
    \def\ZREF@xr@ignored@empty{0}%
    \def\ZREF@xr@ignored@zref{0}%
    \def\ZREF@xr@ignored@ltx{0}%
    \ZREF@xr@processfile
    \closein\@inputcheck
    \begingroup
      \let\on@line\ltx@empty
      \PackageInfo{zref-xr}{%
        Statistics for `\ZREF@xr@file':\MessageBreak
        \ZREF@xr@found\space
        \ifZREF@xr@zreflabel zref\else LaTeX\fi\space
        label(s) found%
        \ifnum\ZREF@xr@ignored@empty>0 %
          ,\MessageBreak
          \ZREF@xr@ignored@empty\space empty label(s) ignored%
        \fi
        \ifnum\ZREF@xr@ignored@zref>0 %
          ,\MessageBreak
          \ZREF@xr@ignored@zref\space
          duplicated zref label(s) ignored%
        \fi
        \ifnum\ZREF@xr@ignored@ltx>0 %
          ,\MessageBreak
          \ZREF@xr@ignored@ltx\space
          duplicated latex label(s) ignored%
        \fi
      }%
    \endgroup
  \fi
  \ifx\ZREF@xr@filelist\ltx@empty
  \else
    \edef\ZREF@xr@file{%
      \expandafter\ltx@car\ZREF@xr@filelist\@nil
    }%
    \edef\ZREF@xr@filelist{%
      \expandafter\ltx@cdr\ZREF@xr@filelist\ltx@empty\@nil
    }%
    \expandafter\ZREF@xr@checkfile
  \fi
}%
%    \end{macrocode}
%    \end{macro}
%
%    \begin{macro}{\ZREF@xr@processfile}
%    \begin{macrocode}
\def\ZREF@xr@processfile{%
  \read\@inputcheck to\ZREF@xr@line
  \expandafter\ZREF@xr@processline\ZREF@xr@line..\ZREF@nil
  \ifeof\@inputcheck
  \else
    \expandafter\ZREF@xr@processfile
  \fi
}%
%    \end{macrocode}
%    \end{macro}
%    \begin{macro}{\ZREF@xr@processline}
%    The most work must be done for analyzing the
%    arguments of \cs{newlabel}.
%    \begin{macrocode}
\long\def\ZREF@xr@processline#1#2#3\ZREF@nil{%
  \def\x{#1}%
  \toks@{#2}%
  \ifZREF@xr@zreflabel
    \ifx\x\ZREF@xr@zref@newlabel
      \expandafter
      \ZREF@xr@process@zreflabel\ZREF@xr@line...\ZREF@nil
    \fi
  \else
    \ifx\x\ZREF@xr@newlabel
      \expandafter
      \ZREF@xr@process@label\ZREF@xr@line...[]\ZREF@nil
    \fi
  \fi
  \ifx\x\ZREF@xr@@input
    \edef\ZREF@xr@filelist{%
      \etex@unexpanded\expandafter{\ZREF@xr@filelist}%
      {\filename@area\the\toks@}%
    }%
  \fi
}%
\def\ZREF@xr@process@zreflabel\zref@newlabel#1#2#3\ZREF@nil{%
  \edef\ZREF@xr@refname{Z@R@\ZREF@xr@prefix#1}%
  \edef\ZREF@xr@found{\the\numexpr\ZREF@xr@found+1\relax}%
  \def\x{#2}%
  \edef\ZREF@xr@tempname{$temp$}%
  \edef\ZREF@xr@temprefname{Z@R@\ZREF@xr@tempname}%
  \let\ZREF@xr@list\x
  \ifx\ZREF@xr@list\ltx@empty
    \PackageWarningNoLine{zref-xr}{%
      Label `#1' without properties ignored\MessageBreak
      in file `\ZREF@xr@file'%
    }%
    \edef\ZREF@xr@ignored@empty{%
      \the\numexpr\ZREF@xr@ignored@empty+1\relax
    }%
  \else
    \expandafter\ZREF@xr@checklist\x\ZREF@nil
    \expandafter\let\csname\ZREF@xr@temprefname\endcsname\x
    \expandafter\ltx@LocalAppendToMacro
    \csname\ZREF@xr@temprefname\expandafter\endcsname
    \expandafter{%
      \expandafter\externaldocument\expandafter{%
        \ZREF@xr@externalfile
      }%
    }%
    \ZREF@xr@urlcheck\ZREF@xr@tempname
    \ifZREF@xr@tozreflabel
      \@ifundefined{\ZREF@xr@refname}{%
        \ifZREF@xr@verbose
          \PackageInfo{zref-xr}{%
            Import to zref label `\ZREF@xr@tempname#1'%
          }%
        \fi
        \global\expandafter
        \let\csname\ZREF@xr@refname\expandafter\endcsname
        \csname\ZREF@xr@temprefname\endcsname
      }{%
        \ZREF@xr@zref@ignorewarning{\ZREF@xr@prefix#1}%
      }%
    \fi
    \ifZREF@xr@toltxlabel
      \ZREF@xr@tolabel{\ZREF@xr@tempname}{\ZREF@xr@prefix#1}%
    \fi
  \fi
}%
\def\ZREF@xr@process@label\newlabel#1#2#3[#4]#5\ZREF@nil{%
  \def\ZREF@xr@refname{Z@R@\ZREF@xr@prefix#1}%
  \edef\ZREF@xr@found{\the\numexpr\ZREF@xr@found+1\relax}%
  \def\x{#2}%
  \edef\ZREF@xr@tempname{$temp$}%
  \edef\ZREF@xr@temprefname{Z@R@\ZREF@xr@tempname}%
  \expandafter\ZREF@xr@scanparams
      \csname\ZREF@xr@temprefname\expandafter\endcsname
      \x{}{}{}{}{}\ZREF@nil
  \ifx\\#4\\%
  \else
    % ntheorem knows an optional argument at the end of \newlabel
    \ZREF@NewPropTheotype
    \expandafter\ltx@LocalAppendToMacro
        \csname\ZREF@xr@temprefname\endcsname{\theotype{#4}}%
  \fi
  \expandafter\ltx@LocalAppendToMacro
  \csname\ZREF@xr@temprefname\expandafter\endcsname\expandafter{%
    \expandafter\externaldocument\expandafter{%
      \ZREF@xr@externalfile
    }%
  }%
  \ZREF@xr@urlcheck\ZREF@xr@tempname
  \ifZREF@xr@tozreflabel
    \@ifundefined{\ZREF@xr@refname}{%
      \ifZREF@xr@verbose
        \PackageInfo{zref-xr}{%
          Import to zref label `\ZREF@xr@prefix#1'%
        }%
      \fi
      \global\expandafter
      \let\csname\ZREF@xr@refname\expandafter\endcsname
      \csname\ZREF@xr@temprefname\endcsname
    }{%
      \ZREF@xr@zref@ignorewarning{\ZREF@xr@prefix#1}%
    }%
  \fi
  \ifZREF@xr@toltxlabel
    \ZREF@xr@tolabel{\ZREF@xr@tempname}{\ZREF@xr@prefix#1}%
  \fi
}
\def\ZREF@xr@zref@newlabel{\zref@newlabel}%
\def\ZREF@xr@newlabel{\newlabel}%
\def\ZREF@xr@@input{\@input}%
\def\ZREF@xr@relax{\relax}%
%    \end{macrocode}
%    \end{macro}
%    \begin{macro}{\ZREF@xr@tolabel}
%    \begin{macrocode}
\def\ZREF@xr@tolabel#1#2{%
  \ifZREF@xr@verbose
    \PackageInfo{zref-xr}{%
      Import to LaTeX label `#2'%
    }%
  \fi
  \zref@wrapper@unexpanded{%
    \expandafter\xdef\csname r@#2\endcsname{%
      {%
        \ltx@ifundefined{M@TitleReference}{%
          \ltx@ifundefined{TR@TitleReference}{%
            \zref@extractdefault{#1}{default}{}%
          }{%
            \noexpand\TR@TitleReference
            {\zref@extractdefault{#1}{default}{}}%
            {\zref@extractdefault{#1}{title}{}}%
          }%
        }{%
          \noexpand\M@TitleReference
          {\zref@extractdefault{#1}{default}{}}%
          {\zref@extractdefault{#1}{title}{}}%
        }%
      }%
      {\zref@extractdefault{#1}{page}{}}%
      \ltx@ifpackageloaded{nameref}{%
        {\zref@extractdefault{#1}{title}{}}%
        {\zref@extractdefault{#1}{anchor}{}}%
        \zref@ifrefcontainsprop{#1}{urluse}{%
          {\zref@extractdefault{#1}{urluse}{}}%
        }{%
          {\zref@extractdefault{#1}{url}{}}%
        }%
      }{}%
    }%
  }%
}
%    \end{macrocode}
%    \end{macro}
%
%    \begin{macro}{\ZREF@xr@zref@ignorewarning}
%    \begin{macrocode}
\def\ZREF@xr@zref@ignorewarning#1{%
  \PackageWarningNoLine{zref-xr}{%
    Zref label `#1' is already in use\MessageBreak
    in file `\ZREF@xr@file'%
  }%
  \edef\ZREF@xr@ignored@zref{%
    \the\numexpr\ZREF@xr@ignored@zref+1%
  }%
}%
%    \end{macrocode}
%    \end{macro}
%
%    \begin{macro}{\ZREF@xr@ltx@ignorewarning}
%    \begin{macrocode}
\def\ZREF@xr@ltx@ignorewarning#1{%
  \PackageWarningNoLine{zref-xr}{%
    LaTeX label `#1' is already in use\MessageBreak
    in file `\ZREF@xr@file'%
  }%
  \edef\ZREF@xr@ignored@ltx{%
    \the\numexpr\ZREF@xr@ignored@ltx+1%
  }%
}%
%    \end{macrocode}
%    \end{macro}
%
%    \begin{macro}{\ZREF@xr@checklist}
%    \begin{macrocode}
\def\ZREF@xr@checklist#1#2#3\ZREF@nil{%
  \ifx\@undefined#1\relax
    \expandafter\ZREF@xr@checkkey\string#1\@nil
  \fi
  \ifx\\#3\\%
  \else
    \ltx@ReturnAfterFi{%
      \ZREF@xr@checklist#3\ZREF@nil
    }%
  \fi
}%
\def\ZREF@xr@checkkey#1#2\@nil{%
  \zref@ifpropundefined{#2}{%
    \zref@newprop{#2}{}%
  }{}%
}%
%    \end{macrocode}
%    \end{macro}
%
%    \begin{macro}{\ZREF@xr@scanparams}
%    \begin{macrocode}
\def\ZREF@xr@scanparams#1#2#3#4#5#6#7\ZREF@nil{%
  \let#1\ltx@empty
  \ZREF@foundfalse
  \ZREF@xr@scantitleref#1#2\TR@TitleReference{}{}\ZREF@nil
  \ifZREF@found
  \else
    \ltx@LocalAppendToMacro#1{\default{#2}}%
  \fi
  % page
  \ltx@LocalAppendToMacro#1{\page{#3}}%
  % nameref title
  \ifZREF@found
  \else
    \ifx\\#4\\%
    \else
      \def\ZREF@xr@temp{#4}%
      \ifx\ZREF@xr@temp\ZREF@xr@relax
      \else
        \ltx@LocalAppendToMacro#1{\title{#4}}%
      \fi
    \fi
  \fi
  % anchor
  \ifx\\#5\\%
  \else
    \ltx@LocalAppendToMacro#1{\anchor{#5}}%
  \fi
  \ifx\\#6\\%
  \else
    \ifZREF@xr@urluse
      \ZREF@xr@AddURL{#6}%
      \expandafter\ltx@LocalAppendToMacro\expandafter#1%
      \expandafter{%
        \expandafter\urluse\expandafter{%
          \csname Z@U@\ZREF@xr@theURL\endcsname
        }%
      }%
    \else
      \ltx@LocalAppendToMacro#1{\url{#6}}%
    \fi
  \fi
}%
%    \end{macrocode}
%    \end{macro}
%
%    \begin{macro}{\ZREF@xr@scantitleref}
%    \begin{macrocode}
\def\ZREF@xr@scantitleref#1#2\TR@TitleReference#3#4#5\ZREF@nil{%
  \ifx\\#5\\%
  \else
    \ltx@LocalAppendToMacro#1{%
      \default{#3}%
      \title{#4}%
    }%
    \ZREF@foundtrue
  \fi
}%
%    \end{macrocode}
%    \end{macro}
%
%    \begin{macro}{\ZREF@xr@urlcheck}%
%    \begin{macrocode}
\def\ZREF@xr@urlcheck#1{%
  \zref@ifrefcontainsprop{#1}{anchor}{%
    \zref@ifrefcontainsprop{#1}{url}{%
    }{%
      \expandafter
      \ltx@LocalAppendToMacro\csname Z@R@#1\expandafter\endcsname
      \expandafter{%
        \csname url\ifZREF@xr@urluse use\fi
        \expandafter\endcsname\expandafter{\ZREF@xr@url}%
      }%
    }%
  }{%
  }%
}%
%    \end{macrocode}
%    \end{macro}
%
%    \begin{macrocode}
%</xr>
%    \end{macrocode}
%
%
% \subsection{Module \xmodule{hyperref}}
%
%    UNFINISHED :-(
%
%    \begin{macrocode}
%<*hyperref>
\NeedsTeXFormat{LaTeX2e}
\ProvidesPackage{zref-hyperref}%
  [2020-07-03 v2.32 Module hyperref for zref (HO)]%
\RequirePackage{zref-base}[2019/11/29]
\ifx\ZREF@base@ok Y%
\else
  \expandafter\endinput
\fi
%    \end{macrocode}
%
%    \begin{macrocode}
\ZREF@NewPropAnchor
\zref@addprop\ZREF@mainlist{anchor}%
%    \end{macrocode}
%
%    \begin{macrocode}
%</hyperref>
%    \end{macrocode}
%
%
% \subsection{Module \xmodule{savepos}}
%
%    Module \xmodule{savepos} provides an interface for
%    \pdfTeX's \cmd{\pdfsavepos}, see the manual for
%    \pdfTeX.
%
% \subsubsection{Identification}
%
%    \begin{macrocode}
%<*savepos>
\NeedsTeXFormat{LaTeX2e}
\ProvidesPackage{zref-savepos}%
  [2020-07-03 v2.32 Module savepos for zref (HO)]%
\RequirePackage{zref-base}[2019/11/29]
\ifx\ZREF@base@ok Y%
\else
  \expandafter\endinput
\fi
%    \end{macrocode}
%
% \subsubsection{Availability}
%
%    First we check, whether the feature is available.
%    \begin{macrocode}
\ifx\directlua\@undefined
\ltx@IfUndefined{pdfsavepos}{%
  \PackageError\ZREF@name{%
    \string\pdfsavepos\space is not supported.\MessageBreak
    It is provided by pdfTeX (1.40) or XeTeX%
  }\ZREF@UpdatePdfTeX
  \endinput
}{}%
\fi
%    \end{macrocode}
%    In PDF mode we are done. However support for DVI mode
%    was added later in version 1.40.0. In earlier versions
%    \cs{pdfsavepos} is defined, but its execution raises
%    an error. Note that \hologo{XeTeX} also provides \cs{pdfsavepos}.
%    \begin{macrocode}
\ifpdf
\else
  \ltx@IfUndefined{pdftexversion}{%
  }{%
    \ifnum\pdftexversion<140 %
      \PackageError\ZREF@name{%
        \string\pdfsavepos\space is not supported in DVI mode%
        \MessageBreak
        of this pdfTeX version%
      }\ZREF@UpdatePdfTeX
      \expandafter\expandafter\expandafter\endinput
    \fi
  }%
\fi
%    \end{macrocode}
%
% \subsubsection{Setup}
%
%    \begin{macrocode}
\zref@newlist{savepos}
\ifx\directlua\@undefined
  \zref@newprop*{posx}[0]{\the\pdflastxpos}
  \zref@newprop*{posy}[0]{\the\pdflastypos}
\else
  \zref@newprop*{posx}[0]{\the\lastxpos}
  \zref@newprop*{posy}[0]{\the\lastypos}
\fi
\zref@addprops{savepos}{posx,posy}
%    \end{macrocode}
%
% \subsubsection{User macros}
%
%    \begin{macro}{\zref@savepos}
%    \begin{macrocode}
\ifx\directlua\@undefined
  \def\zref@savepos{%
    \if@filesw
      \pdfsavepos
    \fi
  }
\else
  \def\zref@savepos{%
    \if@filesw
      \savepos
    \fi
  }
\fi
%    \end{macrocode}
%    \end{macro}
%
%    \begin{macro}{\ZREF@zsavepos}
%    \begin{macrocode}
\def\ZREF@zsavepos#1#2#3{%
  \@bsphack
  \if@filesw
    \zref@savepos
    #1{#3}{#2}%
    \ltx@IfUndefined{TeXXeTstate}{%
    }{%
      \ifnum\TeXXeTstate=\ltx@zero
      \else
        \zref@savepos
      \fi
    }%
  \fi
  \@esphack
}
%    \end{macrocode}
%    \end{macro}
%
%    \begin{macro}{\zsavepos}
%    The current location is stored in a reference with the
%    given name.
%    \begin{macrocode}
\ZREF@IfDefinable\zsavepos\def{%
  {%
    \ZREF@zsavepos\zref@labelbylist{savepos}%
  }%
}
%    \end{macrocode}
%    \end{macro}
%
%    \begin{macro}{\zsaveposx}
%    \begin{macrocode}
\ZREF@IfDefinable\zsaveposx\def{%
  {%
    \ZREF@zsavepos\zref@labelbyprops{posx}%
  }%
}
%    \end{macrocode}
%    \end{macro}
%    \begin{macro}{\zsaveposy}
%    \begin{macrocode}
\ZREF@IfDefinable\zsaveposy\def{%
  {%
    \ZREF@zsavepos\zref@labelbyprops{posy}%
  }%
}
%    \end{macrocode}
%    \end{macro}
%
%    \begin{macro}{\zposx}
%    \begin{macro}{\zposy}
%    The horizontal and vertical position are available by
%    \cs{zposx} and \cs{zposy}. Do not rely on absolute
%    positions. They differ in DVI and PDF mode of \pdfTeX.
%    Use differences instead. The unit of the position numbers
%    is sp.
%    \begin{macrocode}
\newcommand*{\zposx}[1]{%
  \zref@extract{#1}{posx}%
}%
\newcommand*{\zposy}[1]{%
  \zref@extract{#1}{posy}%
}%
%    \end{macrocode}
%    \end{macro}
%    \end{macro}
%
%    Typically horizontal and vertical positions are used
%    inside calculations. Therefore the extracting macros
%    should be expandable and babel's patch is not applyable.
%
%    Also it is in the responsibility of the user to marked
%    used positions by \cs{zrefused}
%    in order to notify \hologo{LaTeX} about undefined references.
%
%    \begin{macro}{\ZREF@savepos@ok}
%    \begin{macrocode}
\let\ZREF@savepos@ok=Y
%    \end{macrocode}
%    \end{macro}
%
%    \begin{macrocode}
%</savepos>
%    \end{macrocode}
%
%
% \subsection{Module \xmodule{abspos}}
%
% \subsubsection{Identification}
%
%    \begin{macrocode}
%<*abspos>
\NeedsTeXFormat{LaTeX2e}
\ProvidesPackage{zref-abspos}%
  [2020-07-03 v2.32 Module abspos for zref (HO)]%
\RequirePackage{zref-base}[2019/11/29]
\ifx\ZREF@base@ok Y%
\else
  \expandafter\endinput
\fi
%    \end{macrocode}
%
%    \begin{macrocode}
\RequirePackage{zref-savepos}[2019/11/29]
\ifx\ZREF@savepos@ok Y%
\else
  \expandafter\endinput
\fi
%    \end{macrocode}
%    \begin{macrocode}
\RequirePackage{zref-pagelayout}[2019/11/29]
\zref@addprop{savepos}{abspage}
\zref@addprop{savepos}{pagevalue}
%    \end{macrocode}
%
%    \begin{macro}{\zref@absposx}
%    \begin{macrocode}
\newcommand*{\zref@absposx}[3]{%
  \number
    \expandafter\zref@absposnumx\expandafter{%
      \number\zref@extractdefault{#1}{abspage}{0}%
    }{#2}{#3}%
  \ltx@space
}
%    \end{macrocode}
%    \end{macro}
%
%    \begin{macro}{\zref@absposy}
%    \begin{macrocode}
\newcommand*{\zref@absposy}[3]{%
  \number
  \expandafter\zref@absposnumy\expandafter{%
    \number\zref@extractdefault{#1}{abspage}{0}%
  }{#2}{#3}%
  \ltx@space
}
%    \end{macrocode}
%    \end{macro}
%
%    \begin{macro}{\zref@absposnumx}
%    \begin{macrocode}
\newcommand*{\zref@absposnumx}[3]{%
  \number
%  \ifnum#1>\ltx@zero
%    \zref@ifrefundefined{thepage#1}{%
%      0%
%    }{%
      \numexpr\ZREF@absposnum{thepage#1}{#2}{x}{#3}\relax
%    }%
%  \else
%    0%
%  \fi
}
%    \end{macrocode}
%    \end{macro}
%
%    \begin{macro}{\zref@absposnumy}
%    \begin{macrocode}
\newcommand*{\zref@absposnumy}[3]{%
  \number
%  \ifnum#1>\ltx@zero
%    \zref@ifrefundefined{thepage#1}{%
%      0%
%    }{%
      \numexpr\ZREF@absposnum{thepage#1}{#2}{y}{#3}\relax
%    }%
%  \else
%    0%
%  \fi
}
%    \end{macrocode}
%    \end{macro}
%
%    \begin{macro}{\ZREF@absposnum}
%    \begin{macrocode}
\def\ZREF@absposnum#1#2#3#4{%
  \ltx@ifundefined{ZREF@abspos@#2@#3@#4}{%
    0%
  }{%
    \csname ZREF@abspos@#2@#3@#4\endcsname{#1}%
  }%
}
%    \end{macrocode}
%    \end{macro}
%
%    \begin{macro}{\zref@def@absposx}
%    \begin{macrocode}
\ZREF@Robust\def\zref@def@absposx#1{%
  \zref@wrapper@babel{\ZREF@def@abspos{#1}\zref@absposx}%
}
%    \end{macrocode}
%    \end{macro}
%
%    \begin{macro}{\zref@def@absposy}
%    \begin{macrocode}
\ZREF@Robust\def\zref@def@absposy#1{%
  \zref@wrapper@babel{\ZREF@def@abspos{#1}\zref@absposy}%
}
%    \end{macrocode}
%    \end{macro}
%
%    \begin{macro}{\zref@def@absposnumx}
%    \begin{macrocode}
\ZREF@Robust\def\zref@def@absposnumx#1{%
  \zref@wrapper@babel{\ZREF@def@abspos{#1}\zref@absposnumx}%
}
%    \end{macrocode}
%    \end{macro}
%
%    \begin{macro}{\zref@def@absposnumy}
%    \begin{macrocode}
\ZREF@Robust\def\zref@def@absposnumy#1{%
  \zref@wrapper@babel{\ZREF@def@abspos{#1}\zref@absposnumy}%
}
%    \end{macrocode}
%    \end{macro}
%    \begin{macro}{\ZREF@def@abspos}
%    \begin{macrocode}
\def\ZREF@def@absposnumy#1#2#3#4#5{%
  \edef#1{#2{#3}{#4}{#5}}%
}
%    \end{macrocode}
%    \end{macro}
%
%    \begin{macro}{\zref@absposused}
%    \begin{macrocode}
\ZREF@Robust\def\zref@absposused{%
  \zref@wrapper@babel\ZREF@abspos@used
}
%    \end{macrocode}
%    \end{macro}
%    \begin{macro}{\ZREF@abspos@used}
%    \begin{macrocode}
\def\ZREF@abspos@used#1{%
  \zref@refused{#1}%
  \zref@ifrefundefined{#1}{%
  }{%
    \begingroup
      \edef\ZREF@temp{%
        \zref@extractdefault{#1}{abspage}{0}%
      }%
      \ifnum\ZREF@temp>\ltx@zero
        \zref@refused{thepage\ZREF@temp}%
      \else
        \@PackageError{zref-abspos}{%
          \string\zref@pos@label@used\ltx@space
          needs property `abspage'\MessageBreak
          in label `#1'%
        }\@ehc
      \fi
    \endgroup
  }%
}
%    \end{macrocode}
%    \end{macro}
%
%    \begin{macro}{\zref@absposnumused}
%    \begin{macrocode}
\newcommand*{\zref@absposnumused}[1]{%
  \ifnum#1>\ltx@zero
    \zref@refused{thepage\number#1}%
  \else
    \@PackageError{zref-abspos}{%
      Invalid absolute page number (#1)\MessageBreak
      for \string\zref@pos@num@used.\MessageBreak
      A positive integer is expected%
    }\@ehc
  \fi
}
%    \end{macrocode}
%    \end{macro}
%
%    \begin{macro}{\zref@ifabsposundefined}
%    \begin{macrocode}
\def\zref@ifabsposundefined#1{%
  \zref@ifrefundefined{#1}\ltx@firsttwo{%
    \expandafter\zref@ifabsposnumundefined\expandafter{%
      \number\zref@extractdefault{#1}{abspage}{0}%
    }%
  }%
}
%    \end{macrocode}
%    \end{macro}
%
%    \begin{macro}{\zref@ifabsposnumundefined}
%    \begin{macrocode}
\def\zref@ifabsposnumundefined#1{%
  \ifnum\ZREF@number{#1}>\ltx@zero
    \zref@ifrefundefined{thepage#1}%
        \ltx@firstoftwo\ltx@secondoftwo
  \else
    \expandafter\ltx@firstoftwo
  \fi
}
%    \end{macrocode}
%    \end{macro}
%
% \subsubsection{Media}
%
%    \begin{macro}{\ZREF@abspos@media@width}
%    \begin{macrocode}
\edef\ZREF@abspos@media@width{%
  \ltx@ifundefined{pdfpagewidth}{%
    \ltx@ifundefined{mediawidth}{%
      \ltx@ifundefined{stockwidth}{%
        paperwidth%
      }{%
        stockwidth%
      }%
    }{%
      mediawidth%
    }%
  }{%
    pdfpagewidth%
  }%
}
\ifluatex
\def\ZREF@abspos@media@width{pdfpagewidth}%
\fi
%    \end{macrocode}
%    \end{macro}
%    \begin{macro}{\ZREF@abspos@media@height}
%    \begin{macrocode}
\edef\ZREF@abspos@media@height{%
  \ltx@ifundefined{pdfpageheight}{%
    \ltx@ifundefined{mediaheight}{%
      \ltx@ifundefined{stockheight}{%
        paperheight%
      }{%
        stockheight%
      }%
    }{%
      mediaheight%
    }%
  }{%
    \noexpand\ifcase\pdfpageheight
      \ltx@ifundefined{stockheight}{%
        paperheight%
      }{%
        stockheight%
      }%
    \noexpand\else
      pdfpageheight%
    \noexpand\fi
  }%
}
\ifluatex
\edef\ZREF@abspos@media@height{%
   \noexpand\ifcase\pageheight
      \ltx@ifundefined{stockheight}{%
        paperheight%
      }{%
        stockheight%
      }%
    \noexpand\else
      pdfpageheight%
    \noexpand\fi}%
\fi
%    \end{macrocode}
%    \end{macro}
%
%    \begin{macro}{\ZREF@abspos@media@x@left}
%    \begin{macrocode}
\def\ZREF@abspos@media@x@left#1{%
  0%
}
%    \end{macrocode}
%    \end{macro}
%    \begin{macro}{\ZREF@abspos@media@x@right}
%    \begin{macrocode}
\def\ZREF@abspos@media@x@right#1{%
  \zref@extract{#1}\ZREF@abspos@media@width
}
%    \end{macrocode}
%    \end{macro}
%    \begin{macro}{\ZREF@abspos@media@x@center}
%    \begin{macrocode}
\def\ZREF@abspos@media@x@center#1{%
  \ZREF@abspos@media@x@left{#1}%
  +\zref@extract{#1}\ZREF@abspos@media@width/2%
}
%    \end{macrocode}
%    \end{macro}
%
%    \begin{macro}{\ZREF@abspos@media@y@top}
%    \begin{macrocode}
\def\ZREF@abspos@media@y@top#1{%
  \zref@extract{#1}\ZREF@abspos@media@height
}
%    \end{macrocode}
%    \end{macro}
%    \begin{macro}{\ZREF@abspos@media@y@bottom}
%    \begin{macrocode}
\def\ZREF@abspos@media@y@bottom#1{%
  0%
}
%    \end{macrocode}
%    \end{macro}
%    \begin{macro}{\ZREF@abspos@media@y@center}
%    \begin{macrocode}
\def\ZREF@abspos@media@y@center#1{%
  \zref@extract{#1}\ZREF@abspos@media@height/2%
}
%    \end{macrocode}
%    \end{macro}
%
% \subsubsection{Paper}
%
%    \begin{macro}{\ZREF@abspos@paper@x@left}
%    \begin{macrocode}
\def\ZREF@abspos@paper@x@left#1{%
  0%
}
%    \end{macrocode}
%    \end{macro}
%    \begin{macro}{\ZREF@abspos@paper@x@right}
%    \begin{macrocode}
\def\ZREF@abspos@paper@x@right#1{%
  \zref@extract{#1}{paperwidth}%
}
%    \end{macrocode}
%    \end{macro}
%    \begin{macro}{\ZREF@abspos@paper@x@center}
%    \begin{macrocode}
\def\ZREF@abspos@paper@x@center#1{%
  \zref@extract{#1}{paperwidth}/2%
}
%    \end{macrocode}
%    \end{macro}
%
%    \begin{macro}{\ZREF@abspos@paper@y@top}
%    \begin{macrocode}
\let\ZREF@abspos@paper@y@top\ZREF@abspos@media@y@top
%    \end{macrocode}
%    \end{macro}
%    \begin{macro}{\ZREF@abspos@paper@y@bottom}
%    \begin{macrocode}
\def\ZREF@abspos@paper@y@bottom#1{%
  \ZREF@abspos@paper@y@top{#1}%
  -\zref@extract{#1}{paperheight}%
}
%    \end{macrocode}
%    \end{macro}
%    \begin{macro}{\ZREF@abspos@paper@y@center}
%    \begin{macrocode}
\def\ZREF@abspos@paper@y@center#1{%
  \ZREF@abspos@paper@y@top{#1}%
  -\zref@extract{#1}{paperheight}/2%
}
%    \end{macrocode}
%    \end{macro}
%
% \subsubsection{Origin}
%
% There doesn't seem a good reason to make these tests depend on pdf mode
%  in current engines, so comment out the \verb|\ifpdf| tests.
%    \begin{macro}{\ZREF@abspos@origin@x}
%    \begin{macrocode}
\let\ZREF@temp\ltx@two
\ltx@IfUndefined{pdfhorigin}{}{%
%  \ifpdf
    \let\ZREF@temp\ltx@zero
%  \fi
}
\ifluatex
% \ifpdf
  \let\ZREF@temp\ltx@zero
% \fi
\fi

\ifx\ZREF@temp\ltx@two
  \ifnum\mag=1000 %
    \let\ZREF@temp\ltx@one
  \fi
\fi
\ifcase\ZREF@temp
  \def\ZREF@abspos@origin@x#1{%
    \zref@extract{#1}{pdfhorigin}%
  }%
\or
  \def\ZREF@abspos@origin@x#1{%
    4736286%
  }%
\or
  \def\ZREF@abspos@origin@x#1{%
    \numexpr\mag/1000*\dimexpr 1truein\relax\relax
  }%
\fi
%    \end{macrocode}
%    \end{macro}
%    \begin{macro}{\ZREF@abspos@origin@y}
%    \begin{macrocode}
\let\ZREF@temp\ltx@two
\ltx@IfUndefined{pdfvorigin}{}{%
%  \ifpdf
    \let\ZREF@temp\ltx@zero
%  \fi
}
\ifluatex
%  \ifpdf
    \let\ZREF@temp\ltx@zero
%  \fi
\fi
\ifx\ZREF@temp\ltx@two
  \ifnum\mag=1000 %
    \let\ZREF@temp\ltx@one
  \fi
\fi
\ifcase\ZREF@temp
  \def\ZREF@abspos@origin@y#1{%
    \zref@extract{#1}{pdfvorigin}%
  }%
\or
  \def\ZREF@abspos@origin@y#1{%
    4736286%
  }%
\or
  \def\ZREF@abspos@origin@y#1{%
    \numexpr\mag/1000*\dimexpr 1truein\relax\relax
   }%
\fi
%    \end{macrocode}
%    \end{macro}
%
% \subsubsection{Header}
%
%    \begin{macro}{\ZREF@abspos@head@x@left}
%    \begin{macrocode}
\def\ZREF@abspos@head@x@left#1{%
  \ZREF@abspos@paper@x@left{#1}%
  +\ZREF@abspos@origin@x{#1}%
  +\zref@extract{#1}{hoffset}%
  +\ifodd\zref@extractdefault{#1}{pagevalue}{\number\c@page} %
     \zref@extract{#1}{oddsidemargin}%
   \else
     \zref@extract{#1}{evensidemargin}%
   \fi
}
%    \end{macrocode}
%    \end{macro}
%    \begin{macro}{\ZREF@abspos@head@x@right}
%    \begin{macrocode}
\def\ZREF@abspos@head@x@right#1{%
  \ZREF@abspos@head@x@left{#1}%
  +\zref@extract{#1}{textwidth}%
}
%    \end{macrocode}
%    \end{macro}
%    \begin{macro}{\ZREF@abspos@head@x@center}
%    \begin{macrocode}
\def\ZREF@abspos@head@x@center#1{%
  \ZREF@abspos@head@x@left{#1}%
  +\zref@extract{#1}{textwidth}/2%
}
%    \end{macrocode}
%    \end{macro}
%
%    \begin{macro}{\ZREF@abspos@head@y@top}
%    \begin{macrocode}
\def\ZREF@abspos@head@y@top#1{%
  \ZREF@abspos@paper@y@top{#1}%
  -\ZREF@abspos@origin@y{#1}%
  -\zref@extract{#1}{voffset}%
  -\zref@extract{#1}{topmargin}%
}
%    \end{macrocode}
%    \end{macro}
%    \begin{macro}{\ZREF@abspos@head@y@bottom}
%    \begin{macrocode}
\def\ZREF@abspos@head@y@bottom#1{%
  \ZREF@abspos@head@y@top{#1}%
  -\zref@extract{#1}{headheight}%
}
%    \end{macrocode}
%    \end{macro}
%    \begin{macro}{\ZREF@abspos@head@y@center}
%    \begin{macrocode}
\def\ZREF@abspos@head@y@center#1{%
  \ZREF@abspos@head@y@top{#1}%
  -\zref@extract{#1}{headheight}/2%
}
%    \end{macrocode}
%    \end{macro}
%
% \subsubsection{Body}
%
%    \begin{macro}{\ZREF@abspos@body@x@left}
%    \begin{macrocode}
\let\ZREF@abspos@body@x@left\ZREF@abspos@head@x@left
%    \end{macrocode}
%    \end{macro}
%    \begin{macro}{\ZREF@abspos@body@x@right}
%    \begin{macrocode}
\let\ZREF@abspos@body@x@right\ZREF@abspos@head@x@right
%    \end{macrocode}
%    \end{macro}
%    \begin{macro}{\ZREF@abspos@body@x@center}
%    \begin{macrocode}
\let\ZREF@abspos@body@x@center\ZREF@abspos@head@x@center
%    \end{macrocode}
%    \end{macro}
%
%    \begin{macro}{\ZREF@abspos@body@y@top}
%    \begin{macrocode}
\def\ZREF@abspos@body@y@top#1{%
  \ZREF@abspos@head@y@bottom{#1}%
  -\zref@extract{#1}{headsep}%
}
%    \end{macrocode}
%    \end{macro}
%    \begin{macro}{\ZREF@abspos@body@y@bottom}
%    \begin{macrocode}
\def\ZREF@abspos@body@y@bottom#1{%
  \ZREF@abspos@body@y@top{#1}%
  -\zref@extract{#1}{textheight}%
}
%    \end{macrocode}
%    \end{macro}
%    \begin{macro}{\ZREF@abspos@body@y@center}
%    \begin{macrocode}
\def\ZREF@abspos@body@y@center#1{%
  \ZREF@abspos@body@y@top{#1}%
  -\zref@extract{#1}{textheight}/2%
}
%    \end{macrocode}
%    \end{macro}
%
% \subsubsection{Footer}
%
%    \begin{macro}{\ZREF@abspos@foot@x@left}
%    \begin{macrocode}
\let\ZREF@abspos@foot@x@left\ZREF@abspos@head@x@left
%    \end{macrocode}
%    \end{macro}
%    \begin{macro}{\ZREF@abspos@foot@x@right}
%    \begin{macrocode}
\let\ZREF@abspos@foot@x@right\ZREF@abspos@head@x@right
%    \end{macrocode}
%    \end{macro}
%    \begin{macro}{\ZREF@abspos@foot@x@center}
%    \begin{macrocode}
\let\ZREF@abspos@foot@x@center\ZREF@abspos@head@x@center
%    \end{macrocode}
%    \end{macro}
%
%    \begin{macro}{\ZREF@abspos@foot@y@bottom}
%    \begin{macrocode}
\def\ZREF@abspos@foot@y@bottom#1{%
  \ZREF@abspos@body@y@bottom{#1}%
  -\zref@extract{#1}{footskip}%
}
%    \end{macrocode}
%    \end{macro}
%
% \subsubsection{Marginal notes}
%
%    \begin{macro}{\ZREF@abspos@marginpar@x@left}
%    \begin{macrocode}
\def\ZREF@abspos@marginpar@x@left#1{%
  \ifodd\zref@extractdefault{#1}{pagevalue}{\number\c@page} %
    \ZREF@abspos@body@x@right{#1}%
    +\zref@extract{#1}{marginparsep}%
  \else
    \ZREF@abspos@body@x@left{#1}%
    -\zref@extract{#1}{marginparsep}%
    -\zref@extract{#1}{marginparwidth}%
  \fi
}
%    \end{macrocode}
%    \end{macro}
%    \begin{macro}{\ZREF@abspos@marginpar@x@right}
%    \begin{macrocode}
\def\ZREF@abspos@marginpar@x@right#1{%
  \ZREF@abspos@marginpar@x@left{#1}%
  +\zref@extract{#1}{marginparwidth}%
}
%    \end{macrocode}
%    \end{macro}
%    \begin{macro}{\ZREF@abspos@marginpar@x@center}
%    \begin{macrocode}
\def\ZREF@abspos@marginpar@x@center#1{%
  \ZREF@abspos@marginpar@x@left{#1}%
  +\zref@extract{#1}{marginparwidth}/2%
}
%    \end{macrocode}
%    \end{macro}
%
%    \begin{macro}{\ZREF@abspos@marginpar@y@top}
%    \begin{macrocode}
\let\ZREF@abspos@marginpar@y@top\ZREF@abspos@body@y@top
%    \end{macrocode}
%    \end{macro}
%    \begin{macro}{\ZREF@abspos@marginpar@y@bottom}
%    \begin{macrocode}
\let\ZREF@abspos@marginpar@y@bottom\ZREF@abspos@body@y@bottom
%    \end{macrocode}
%    \end{macro}
%    \begin{macro}{\ZREF@abspos@marginpar@y@center}
%    \begin{macrocode}
\let\ZREF@abspos@marginpar@y@center\ZREF@abspos@body@y@center
%    \end{macrocode}
%    \end{macro}
%
% \subsubsection{Stock paper}
%
%    \begin{macro}{\ZREF@abspos@stock@x@left}
%    \begin{macrocode}
\let\ZREF@abspos@stock@x@left\ZREF@abspos@paper@x@left
%    \end{macrocode}
%    \end{macro}
%    \begin{macro}{\ZREF@abspos@stock@x@right}
%    \begin{macrocode}
\let\ZREF@abspos@stock@x@right\ZREF@abspos@paper@x@right
%    \end{macrocode}
%    \end{macro}
%    \begin{macro}{\ZREF@abspos@stock@x@center}
%    \begin{macrocode}
\let\ZREF@abspos@stock@x@center\ZREF@abspos@paper@x@center
%    \end{macrocode}
%    \end{macro}
%
%    \begin{macro}{\ZREF@abspos@stock@y@top}
%    \begin{macrocode}
\let\ZREF@abspos@stock@y@top\ZREF@abspos@paper@y@top
%    \end{macrocode}
%    \end{macro}
%    \begin{macro}{\ZREF@abspos@stock@y@bottom}
%    \begin{macrocode}
\let\ZREF@abspos@stock@y@bottom\ZREF@abspos@paper@y@bottom
%    \end{macrocode}
%    \end{macro}
%    \begin{macro}{\ZREF@abspos@stock@y@center}
%    \begin{macrocode}
\let\ZREF@abspos@stock@y@center\ZREF@abspos@paper@y@center
%    \end{macrocode}
%    \end{macro}
%
%    \begin{macrocode}
%</abspos>
%    \end{macrocode}
%
%
% \subsection{Module \xmodule{dotfill}}
%
%    \begin{macrocode}
%<*dotfill>
\NeedsTeXFormat{LaTeX2e}
\ProvidesPackage{zref-dotfill}%
  [2020-07-03 v2.32 Module dotfill for zref (HO)]%
\RequirePackage{zref-base}[2019/11/29]
\ifx\ZREF@base@ok Y%
\else
  \expandafter\endinput
\fi
%    \end{macrocode}
%
%    For measuring the width of \cs{zdotfill} we use the features
%    provided by module \xmodule{savepos}.
%    \begin{macrocode}
\RequirePackage{zref-savepos}[2019/11/29]
%    \end{macrocode}
%    For automatically generated label names we use the unique
%    counter of module \xmodule{base}.
%    \begin{macrocode}
\zref@require@unique
%    \end{macrocode}
%    Configuration is done by the key value interface of package
%    \xpackage{keyval}.
%    \begin{macrocode}
\RequirePackage{keyval}
%    \end{macrocode}
%    The definitions of the keys follow.
%    \begin{macrocode}
\define@key{ZREF@DF}{unit}{%
  \def\ZREF@df@unit{#1}%
}
\define@key{ZREF@DF}{min}{%
  \def\ZREF@df@min{#1}%
}
\define@key{ZREF@DF}{dot}{%
  \def\ZREF@df@dot{#1}%
}
%    \end{macrocode}
%    Defaults are set, see user interface.
%    \begin{macrocode}
\providecommand\ZREF@df@min{2}
\providecommand\ZREF@df@unit{.44em}
\providecommand\ZREF@df@dot{.}
%    \end{macrocode}
%    \begin{macro}{\zdotfillsetup}
%    Configuration of \cs{zdotfill} is done by \cs{zdotfillsetup}.
%    \begin{macrocode}
\newcommand*{\zdotfillsetup}{\kvsetkeys{ZREF@DF}}
%    \end{macrocode}
%    \end{macro}
%
%    \begin{macro}{\zdotfill}
%    \cs{zdotfill} sets labels at the left and the right to get
%    the horizontal position. \cs{zsavepos} is not used, because
%    we do not need the vertical position.
%    \begin{macrocode}
\ZREF@IfDefinable\zdotfill\def{%
  {%
    \leavevmode
    \global\advance\c@zref@unique\ltx@one
    \begingroup
      \def\ZREF@temp{zref@\number\c@zref@unique}%
      \pdfsavepos
      \zref@labelbyprops{\thezref@unique L}{posx}%
      \setlength{\dimen@}{\ZREF@df@unit}%
      \zref@ifrefundefined{\thezref@unique R}{%
        \ZREF@dotfill
      }{%
        \ifnum\numexpr\zposx{\thezref@unique R}%
                      -\zposx{\thezref@unique L}\relax
            <\dimexpr\ZREF@df@min\dimen@\relax
          \hfill
        \else
          \ZREF@dotfill
        \fi
      }%
      \pdfsavepos
      \zref@labelbyprops{\thezref@unique R}{posx}%
    \endgroup
    \kern\z@
  }%
}
%    \end{macrocode}
%    \end{macro}
%    \begin{macro}{\ZREF@dotfill}
%    Help macro that actually sets the dots.
%    \begin{macrocode}
\def\ZREF@dotfill{%
  \cleaders\hb@xt@\dimen@{\hss\ZREF@df@dot\hss}\hfill
}
%    \end{macrocode}
%    \end{macro}
%
%    \begin{macrocode}
%</dotfill>
%    \end{macrocode}
%
% \subsection{Module \xmodule{env}}
%
%    \begin{macrocode}
%<*env>
\NeedsTeXFormat{LaTeX2e}
\ProvidesPackage{zref-env}%
  [2020-07-03 v2.32 Module env for zref (HO)]%
\RequirePackage{zref-base}[2019/11/29]
\ifx\ZREF@base@ok Y%
\else
  \expandafter\endinput
\fi
%    \end{macrocode}
%
%    \begin{macrocode}
\zref@newprop{envname}[]{\@currenvir}
\zref@newprop{envline}[]{\zref@env@line}
%    \end{macrocode}
%
%    \begin{macro}{\zref@env@line}
%    Macro \cs{zref@env@line} extracts the line number
%    from \cs{@currenvline}.
%    \begin{macrocode}
\def\zref@env@line{%
  \ifx\@currenvline\ltx@empty
  \else
    \expandafter
    \ZREF@ENV@line\@currenvline\ltx@empty line \ltx@empty\@nil
  \fi
}
%    \end{macrocode}
%    \end{macro}
%    \begin{macro}{\ZREF@ENV@line}
%    \begin{macrocode}
\def\ZREF@ENV@line#1line #2\ltx@empty#3\@nil{#2}%
%    \end{macrocode}
%    \end{macro}
%
%    \begin{macrocode}
%</env>
%    \end{macrocode}
%
%% \section{Installation}
%
% \subsection{Download}
%
% \paragraph{Package.} This package is available on
% CTAN\footnote{\CTANpkg{zref}}:
% \begin{description}
% \item[\CTAN{macros/latex/contrib/zref/zref.dtx}] The source file.
% \item[\CTAN{macros/latex/contrib/zref/zref.pdf}] Documentation.
% \end{description}
%
%
% \paragraph{Bundle.} All the packages of the bundle `zref'
% are also available in a TDS compliant ZIP archive. There
% the packages are already unpacked and the documentation files
% are generated. The files and directories obey the TDS standard.
% \begin{description}
% \item[\CTANinstall{install/macros/latex/contrib/zref.tds.zip}]
% \end{description}
% \emph{TDS} refers to the standard ``A Directory Structure
% for \TeX\ Files'' (\CTANpkg{tds}). Directories
% with \xfile{texmf} in their name are usually organized this way.
%
% \subsection{Bundle installation}
%
% \paragraph{Unpacking.} Unpack the \xfile{zref.tds.zip} in the
% TDS tree (also known as \xfile{texmf} tree) of your choice.
% Example (linux):
% \begin{quote}
%   |unzip zref.tds.zip -d ~/texmf|
% \end{quote}
%
%
% \subsection{Package installation}
%
% \paragraph{Unpacking.} The \xfile{.dtx} file is a self-extracting
% \docstrip\ archive. The files are extracted by running the
% \xfile{.dtx} through \plainTeX:
% \begin{quote}
%   \verb|tex zref.dtx|
% \end{quote}
%
% \paragraph{TDS.} Now the different files must be moved into
% the different directories in your installation TDS tree
% (also known as \xfile{texmf} tree):
% \begin{quote}
% \def\t{^^A
% \begin{tabular}{@{}>{\ttfamily}l@{ $\rightarrow$ }>{\ttfamily}l@{}}
%   zref.sty & tex/latex/zref/zref.sty\\
%   zref-base.sty & tex/latex/zref/zref-base.sty\\
%   zref-abspage.sty & tex/latex/zref/zref-abspage.sty\\
%   zref-abspos.sty & tex/latex/zref/zref-abspos.sty\\
%   zref-counter.sty & tex/latex/zref/zref-counter.sty\\
%   zref-dotfill.sty & tex/latex/zref/zref-dotfill.sty\\
%   zref-env.sty & tex/latex/zref/zref-env.sty\\
%   zref-hyperref.sty & tex/latex/zref/zref-hyperref.sty\\
%   zref-lastpage.sty & tex/latex/zref/zref-lastpage.sty\\
%   zref-marks.sty & tex/latex/zref/zref-marks.sty\\
%   zref-nextpage.sty & tex/latex/zref/zref-nextpage.sty\\
%   zref-pageattr.sty & tex/latex/zref/zref-pageattr.sty\\
%   zref-pagelayout.sty & tex/latex/zref/zref-pagelayout.sty\\
%   zref-perpage.sty & tex/latex/zref/zref-perpage.sty\\
%   zref-runs.sty & tex/latex/zref/zref-runs.sty\\
%   zref-savepos.sty & tex/latex/zref/zref-savepos.sty\\
%   zref-thepage.sty & tex/latex/zref/zref-thepage.sty\\
%   zref-titleref.sty & tex/latex/zref/zref-titleref.sty\\
%   zref-totpages.sty & tex/latex/zref/zref-totpages.sty\\
%   zref-user.sty & tex/latex/zref/zref-user.sty\\
%   zref-xr.sty & tex/latex/zref/zref-xr.sty\\
%   zref.pdf & doc/latex/zref/zref.pdf\\
%   zref-example.tex & doc/latex/zref/zref-example.tex\\
%   zref-example-lastpage.tex & doc/latex/zref/zref-example-lastpage.tex\\
%   zref-example-nextpage.tex & doc/latex/zref/zref-example-nextpage.tex\\
%   zref.dtx & source/latex/zref/zref.dtx\\
% \end{tabular}^^A
% }^^A
% \sbox0{\t}^^A
% \ifdim\wd0>\linewidth
%   \begingroup
%     \advance\linewidth by\leftmargin
%     \advance\linewidth by\rightmargin
%   \edef\x{\endgroup
%     \def\noexpand\lw{\the\linewidth}^^A
%   }\x
%   \def\lwbox{^^A
%     \leavevmode
%     \hbox to \linewidth{^^A
%       \kern-\leftmargin\relax
%       \hss
%       \usebox0
%       \hss
%       \kern-\rightmargin\relax
%     }^^A
%   }^^A
%   \ifdim\wd0>\lw
%     \sbox0{\small\t}^^A
%     \ifdim\wd0>\linewidth
%       \ifdim\wd0>\lw
%         \sbox0{\footnotesize\t}^^A
%         \ifdim\wd0>\linewidth
%           \ifdim\wd0>\lw
%             \sbox0{\scriptsize\t}^^A
%             \ifdim\wd0>\linewidth
%               \ifdim\wd0>\lw
%                 \sbox0{\tiny\t}^^A
%                 \ifdim\wd0>\linewidth
%                   \lwbox
%                 \else
%                   \usebox0
%                 \fi
%               \else
%                 \lwbox
%               \fi
%             \else
%               \usebox0
%             \fi
%           \else
%             \lwbox
%           \fi
%         \else
%           \usebox0
%         \fi
%       \else
%         \lwbox
%       \fi
%     \else
%       \usebox0
%     \fi
%   \else
%     \lwbox
%   \fi
% \else
%   \usebox0
% \fi
% \end{quote}
% If you have a \xfile{docstrip.cfg} that configures and enables \docstrip's
% TDS installing feature, then some files can already be in the right
% place, see the documentation of \docstrip.
%
% \subsection{Refresh file name databases}
%
% If your \TeX~distribution
% (\TeX\,Live, \mikTeX, \dots) relies on file name databases, you must refresh
% these. For example, \TeX\,Live\ users run \verb|texhash| or
% \verb|mktexlsr|.
%
% \subsection{Some details for the interested}
%
% \paragraph{Unpacking with \LaTeX.}
% The \xfile{.dtx} chooses its action depending on the format:
% \begin{description}
% \item[\plainTeX:] Run \docstrip\ and extract the files.
% \item[\LaTeX:] Generate the documentation.
% \end{description}
% If you insist on using \LaTeX\ for \docstrip\ (really,
% \docstrip\ does not need \LaTeX), then inform the autodetect routine
% about your intention:
% \begin{quote}
%   \verb|latex \let\install=y\input{zref.dtx}|
% \end{quote}
% Do not forget to quote the argument according to the demands
% of your shell.
%
% \paragraph{Generating the documentation.}
% You can use both the \xfile{.dtx} or the \xfile{.drv} to generate
% the documentation. The process can be configured by the
% configuration file \xfile{ltxdoc.cfg}. For instance, put this
% line into this file, if you want to have A4 as paper format:
% \begin{quote}
%   \verb|\PassOptionsToClass{a4paper}{article}|
% \end{quote}
% An example follows how to generate the
% documentation with pdf\LaTeX:
% \begin{quote}
%\begin{verbatim}
%pdflatex zref.dtx
%makeindex -s gind.ist zref.idx
%pdflatex zref.dtx
%makeindex -s gind.ist zref.idx
%pdflatex zref.dtx
%\end{verbatim}
% \end{quote}
%
% \begin{thebibliography}{99}
% \bibpackage{footmisc}{Robin Fairbairns}{2004/01/23 v5.3a}
%   {\CTANpkg{footmisc}}
% \bibpackage{hyperref}{Sebastian Rahtz, Heiko Oberdiek}{2006/08/16 v6.75c}
%   {\CTANpkg{hyperref}}
% \bibpackage{lastpage}{Jeff Goldberg}{1994/06/25 v0.1b}
%   {\CTANpkg{lastpage}}
% \bibpackage{nameref}{Sebastian Rahtz, Heiko Oberdiek}{2006/02/12 v2.24}
%   {\CTANpkg{nameref}}
% \bibpackage{perpage}{David Kastrup}{2002/12/20 v1.0}
%   {\CTANpkg{perpage}}
% \bibpackage{titleref}{Donald Arsenau}{2001/04/05 v3.1}
%   {\CTANpkg{titleref}}
% \bibpackage{totpages}{Wilhelm M\"uller}{1999/07/14 v1.00}
%   {\CTANpkg{totpages}}
% \bibpackage{xr}{David Carlisle}{1994/05/28 v5.02}
%   {\CTANpkg{xr}}
% \bibpackage{xr-hyper}{David Carlisle}{2000/03/22 v6.00beta4}
%   {\CTANpkg{xr-hyper}}
% \end{thebibliography}
%
% \begin{History}
%   \begin{Version}{2006/02/20 v1.0}
%   \item
%     First version.
%   \end{Version}
%   \begin{Version}{2006/05/03 v1.1}
%   \item
%     Module \xmodule{perpage} added.
%   \item
%     Module redesign as packages.
%   \end{Version}
%   \begin{Version}{2006/05/25 v1.2}
%   \item
%     Module \xmodule{dotfillmin} added.
%   \item
%     Module \xmodule{base}: macros \cs{zref@require@uniqe}
%     and \cs{thezref@unique} added
%     (used by modules \xmodule{titleref} and \xmodule{dotfillmin}).
%   \end{Version}
%   \begin{Version}{2006/09/08 v1.3}
%   \item
%     Typo fixes and English cleanup by Per Starback.
%   \end{Version}
%   \begin{Version}{2007/01/23 v1.4}
%   \item
%     Typo in macro name fixed in documentation.
%   \end{Version}
%   \begin{Version}{2007/02/18 v1.5}
%   \item
%     \cs{zref@getcurrent} added (suggestion of Igor Akkerman).
%   \item
%     Module \xmodule{savepos} also supports \hologo{XeTeX}.
%   \end{Version}
%   \begin{Version}{2007/04/06 v1.6}
%   \item
%     Fix in modules \xmodule{abspage} and \xmodule{base}:
%     Now counter |abspage| and |zref@unique|
%     are not remembered by \cs{include}.
%   \item
%     Beamer support for module \xmodule{titleref}.
%   \end{Version}
%   \begin{Version}{2007/04/17 v1.7}
%   \item
%     Package \xpackage{atbegshi} replaces \xpackage{everyshi}.
%   \end{Version}
%   \begin{Version}{2007/04/22 v1.8}
%   \item
%     \cs{zref@wrapper@babel} and \cs{zref@refused} are now expandable
%     if babel is not used or \cs{if@safe@actives} is already set to true.
%     (Feature request of Josselin Noirel)
%   \end{Version}
%   \begin{Version}{2007/05/02 v1.9}
%   \item
%     Module \xmodule{titleref}: Some support for \cs{caption} of package
%     \xpackage{longtable}, but only if \cs{label} is given after
%     \cs{caption}.
%   \end{Version}
%   \begin{Version}{2007/05/06 v2.0}
%   \item
%     Uses package \xpackage{etexcmds} for accessing \hologo{eTeX}'s \cs{unexpanded}.
%   \end{Version}
%   \begin{Version}{2007/05/28 v2.1}
%   \item
%     Module \xmodule{titleref} supports caption of package \xpackage{listings}.
%   \item
%     Fixes in module \xmodule{titleref} for support
%     of packages \xpackage{titlesec} and \xpackage{longtable}.
%   \end{Version}
%   \begin{Version}{2008/09/21 v2.2}
%   \item
%     Module \xmodule{base}: \cs{zref@iflistcontainsprop} is documented, but
%     a broken \cs{zref@listcontainsprop} implemented. Name and
%     implementation fixed (thanks Ohad Kammar).
%   \end{Version}
%   \begin{Version}{2008/10/01 v2.3}
%   \item
%     \cs{zref@localaddprop} added (feature request of Ohad Kammar).
%   \item
%     Module \xmodule{lastpage}: list `LastPage' added. Label `LastPage' will use
%     the properties of this list (default is empty) along with the
%     properties of the main list.
%   \end{Version}
%   \begin{Version}{2009/08/07 v2.4}
%   \item
%     Module \xmodule{runs} added.
%   \end{Version}
%   \begin{Version}{2009/12/06 v2.5}
%   \item
%     Module \xmodule{lastpage}: Uses package \xpackage{atveryend}.
%   \item
%     Module \xmodule{titleref}: Further commands are disabled
%     during string expansion, imported from package \xpackage{nameref}.
%   \end{Version}
%   \begin{Version}{2009/12/07 v2.6}
%   \item
%     Version date added for package \xpackage{atveryend}.
%   \end{Version}
%   \begin{Version}{2009/12/08 v2.7}
%   \item
%     Module \xmodule{titleref}: Use of package \xpackage{gettitlestring}.
%   \end{Version}
%   \begin{Version}{2010/03/26 v2.8}
%   \item
%     \cs{zifrefundefined} added.
%   \item
%     Module \xmodule{lastpage}: Macros \cs{zref@iflastpage}
%     and \cs{ziflastpage} added.
%   \item
%     Module \xmodule{thepage} added.
%   \item
%     Module \xmodule{nextpage} added.
%   \end{Version}
%   \begin{Version}{2010/03/29 v2.9}
%   \item
%     Module \xmodule{marks} added (without documentation).
%   \item
%     \cs{zref@addprop} now adds expanded property to list.
%   \item
%     Useless \cs{ZREF@ErrorNoLine} removed.
%   \end{Version}
%   \begin{Version}{2010/04/08 v2.10}
%   \item
%     Module \xmodule{xr} remembers the external document name
%     in property `externaldocument'.
%   \end{Version}
%   \begin{Version}{2010/04/15 v2.11}
%   \item
%     Module \xmodule{titleref}: Better support of class \xclass{memoir}.
%   \item
%     Module \xmodule{titleref}: Support of theorems.
%   \end{Version}
%   \begin{Version}{2010/04/17 v2.12}
%   \item
%     Module \xmodule{base}: \cs{zref@newprop} ensures global
%     empty default.
%   \item
%     Module \xmodule{xr}: Setup options \xoption{tozreflabel}
%     and \xoption{toltxlabel} added.
%   \end{Version}
%   \begin{Version}{2010/04/19 v2.13}
%   \item
%     \cs{zref@setcurrent} throws an error if the property
%     does not exist (Florent Chervet).
%   \item
%     \cs{zref@getcurrent} the documentation is fixed (Florent Chervet).
%     Also it returns the empty string in case of errors.
%   \item
%     \cs{zref@addprop} and \cs{zref@localaddprop} now take a list of
%     property names (feature request of Florent Chervet).
%   \item
%     Example for \cs{zref@wrapper@unexpanded} corrected (Florent Chervet).
%   \end{Version}
%   \begin{Version}{2010/04/22 v2.14}
%   \item
%     Bug fix for \cs{zref@getcurrent} second argument wasn't eaten in
%     case of unknown property.
%   \item
%     \cs{zref@getcurrent} supports \cs{zref@wrapper@unexpanded}.
%   \item
%     \cs{zref@wrapper@unexpanded} added for \cs{ZREF@xr@tolabel}.
%   \item
%     \cs{zref@extract}, \cs{zref@extractdefault}, \cs{zref@getcurrent}
%     are expandable in exact two steps
%     except inside \cs{zref@wrapper@unexpanded}.
%   \end{Version}
%   \begin{Version}{2010/04/23 v2.15}
%   \item
%     \cs{zexternaldocument} fixed for property `url' when
%     importing \cs{new@label} (bug found by Victor Ivrii).
%   \item
%     Two expansion steps also in \cs{zref@wrapper@unexpanded}.
%   \item
%     Nested calls of \cs{zref@wrapper@unexpanded} possible.
%   \end{Version}
%   \begin{Version}{2010/04/28 v2.16}
%   \item
%     More consequent use of package `ltxcmds' and `hologo'.
%   \item
%     Module \xmodule{pagelayout} added.
%   \item
%     Module \xmodule{pageattr} added.
%   \item
%     Robustness introduced for non-expandable interface macros.
%   \item
%     Internal change of the data format of property lists
%     (suggestion of Florent Chervet).
%   \item
%     Module \xmodule{titleref}: Support of environment \textsf{description}.
%   \end{Version}
%   \begin{Version}{2010/05/01 v2.17}
%   \item
%     \cs{zref@newprop} throws an error if the property already exists.
%   \item
%     Module \xmodule{xr}: Bug fix for the case of several \xext{aux} files
%     (bug found by Victor Ivrii).
%   \item
%     Module \xmodule{xr}: Property `urluse' and option \xoption{urluse} added.
%   \end{Version}
%   \begin{Version}{2010/05/13 v2.18}
%   \item
%     Module \xmodule{env} added.
%   \item
%     Module \xmodule{savepos}: \cs{zref@savepos} added.
%   \end{Version}
%   \begin{Version}{2010/10/22 v2.19}
%   \item
%     \cs{zref@addprop} and \cs{zref@localaddprop} are limited
%     to one property only
%     (incompatibility to versions v2.13 to v2.18).
%   \item
%     \cs{zref@addprops} and \cs{zref@localaddprops} added.
%   \item
%     \cs{zref@delprop} and \cs{zref@localdelprop} added.
%   \item
%     \cs{zref@labelbykv} and \cs{zkvlabel} (module \xmodule{user})
%     with keys \xoption{prop}, \xoption{list}, \xoption{delprop},
%     \xoption{immediate}, \xoption{values} added.
%   \end{Version}
%   \begin{Version}{2011/02/12 v2.20}
%   \item
%     Fix for warning in \xpackage{zref-xr}.
%   \end{Version}
%   \begin{Version}{2011/03/18 v2.21}
%   \item
%     Fix in module \xmodule{pagelayout} for \cs{zlistpagelayout}.
%   \item
%     Fix for \cs{zref@localaddprop} (probably since v2.19).
%   \end{Version}
%   \begin{Version}{2011/10/05 v2.22}
%   \item
%     Documentation fixed for \cs{zref@(local)addprop(s)}.
%   \item
%     Module \xmodule{base}: \cs{zref@def@extract}, \cs{zref@def@extractdefault}
%     added.
%   \item
%     Fix in module \xmodule{pagelayout}: Because of missing \cs{noexpand}
%     commands the values of the pagelayout properties on all pages were
%     the values at package loading.
%   \item
%     Module \xmodule{base}: \cs{zref@showprop} added.
%   \end{Version}
%   \begin{Version}{2011/12/05 v2.23}
%   \item
%     Module \xmodule{savepos}: \cs{zsaveposx} and \cs{zsaveposy} added.
%   \end{Version}
%   \begin{Version}{2012/04/04 v2.24}
%   \item
%     Module \xmodule{titleref}, package \xpackage{titlesec}:
%     some support for class `straight' (\cs{ttl@straight@i}) added.
%   \end{Version}
%   \begin{Version}{2016/05/16 v2.25}
%   \item
%     Documentation updates.
%   \end{Version}
%   \begin{Version}{2016/05/21 v2.26}
%   \item
%     update zref-savepos for new luatex
%   \end{Version}
%   \begin{Version}{2018/11/21 v2.27}
%   \item
%     adapted zref-perpage, see issue https://github.com/ho-tex/zref/issues/2
%   \end{Version}
%   \begin{Version}{2019/11/29 v2.28}
%   \item
%     Documentation updates.
%   \item
%     Use \xpackage{iftex} directly.
%   \end{Version}
%   \begin{Version}{2020-03-03 v2.29}
%   \item
%     adapted in module zref-pagelayout the properties pdfhorigin, pdfvorigin,
%     pdfpagewidth, pdfpageheight for luatex to the right primitives.
%   \item
%     Removed no longer needed code for older lualatex versions.
%   \item added some documentation of the abspos module.
%   \item adapted the abspos module to the new luatex primitives.
%   \item adapted pageattr module to the new luatex primitives.
%   \item added short documentation for pageattr module
%   \item use luatex command names directly in zref-savepos rather than defining pdftex compatibility names.
%   \item allow zref-abspos to use \verb|\pdf[vh]origin| in dvi mode.
%   \end{Version}
%   \begin{Version}{2020-03-04 v2.30}
%   \item
%     add pagevalue property to savepos in the abspos module (issue 1)
%   \end{Version}
%   \begin{Version}{2020-05-28 v2.31}
%   \item Adapted module zref-counter to use \cs{@currentcounter} in the next
%   \LaTeX version (issue 5)
%   \end{Version}
%   \begin{Version}{2020-07-03 v2.32}
%   \item Changed in zref-pagelayout the names of the shipout box dimensions
%    to adapt to the new hook management.
%   \end{Version}
% \end{History}
%
% \PrintIndex
%
% \Finale
\endinput

%        (quote the arguments according to the demands of your shell)
%
% Documentation:
%    (a) If zref.drv is present:
%           latex zref.drv
%    (b) Without zref.drv:
%           latex zref.dtx; ...
%    The class ltxdoc loads the configuration file ltxdoc.cfg
%    if available. Here you can specify further options, e.g.
%    use A4 as paper format:
%       \PassOptionsToClass{a4paper}{article}
%
%    Programm calls to get the documentation (example):
%       pdflatex zref.dtx
%       makeindex -s gind.ist zref.idx
%       pdflatex zref.dtx
%       makeindex -s gind.ist zref.idx
%       pdflatex zref.dtx
%
% Installation:
%    TDS:tex/latex/zref/zref.sty
%    TDS:tex/latex/zref/zref-base.sty
%    TDS:tex/latex/zref/zref-abspage.sty
%    TDS:tex/latex/zref/zref-abspos.sty
%    TDS:tex/latex/zref/zref-counter.sty
%    TDS:tex/latex/zref/zref-dotfill.sty
%    TDS:tex/latex/zref/zref-env.sty
%    TDS:tex/latex/zref/zref-hyperref.sty
%    TDS:tex/latex/zref/zref-lastpage.sty
%    TDS:tex/latex/zref/zref-marks.sty
%    TDS:tex/latex/zref/zref-nextpage.sty
%    TDS:tex/latex/zref/zref-pageattr.sty
%    TDS:tex/latex/zref/zref-pagelayout.sty
%    TDS:tex/latex/zref/zref-perpage.sty
%    TDS:tex/latex/zref/zref-runs.sty
%    TDS:tex/latex/zref/zref-savepos.sty
%    TDS:tex/latex/zref/zref-thepage.sty
%    TDS:tex/latex/zref/zref-titleref.sty
%    TDS:tex/latex/zref/zref-totpages.sty
%    TDS:tex/latex/zref/zref-user.sty
%    TDS:tex/latex/zref/zref-xr.sty
%    TDS:doc/latex/zref/zref.pdf
%    TDS:doc/latex/zref/zref-example.tex
%    TDS:doc/latex/zref/zref-example-lastpage.tex
%    TDS:doc/latex/zref/zref-example-nextpage.tex
%    TDS:source/latex/zref/zref.dtx
%
%<*ignore>
\begingroup
  \catcode123=1 %
  \catcode125=2 %
  \def\x{LaTeX2e}%
\expandafter\endgroup
\ifcase 0\ifx\install y1\fi\expandafter
         \ifx\csname processbatchFile\endcsname\relax\else1\fi
         \ifx\fmtname\x\else 1\fi\relax
\else\csname fi\endcsname
%</ignore>
%<*install>
\input docstrip.tex
\Msg{************************************************************************}
\Msg{* Installation}
\Msg{* Package: zref 2020-07-03 v2.32 A new reference scheme for LaTeX (HO)}
\Msg{************************************************************************}

\keepsilent
\askforoverwritefalse

\let\MetaPrefix\relax
\preamble

This is a generated file.

Project: zref
Version: 2020-07-03 v2.32

Copyright (C)
  2006-2012 Heiko Oberdiek
  2016-2020 Oberdiek Package Support Group

This work may be distributed and/or modified under the
conditions of the LaTeX Project Public License, either
version 1.3c of this license or (at your option) any later
version. This version of this license is in
   https://www.latex-project.org/lppl/lppl-1-3c.txt
and the latest version of this license is in
   https://www.latex-project.org/lppl.txt
and version 1.3 or later is part of all distributions of
LaTeX version 2005/12/01 or later.

This work has the LPPL maintenance status "maintained".

The Current Maintainers of this work are
Heiko Oberdiek and the Oberdiek Package Support Group
https://github.com/ho-tex/zref/issues


This work consists of the main source file zref.dtx
and the derived files
   zref.sty, zref.pdf, zref.ins, zref.drv, zref-base.sty,
   zref-abspage.sty, zref-abspos.sty, zref-counter.sty,
   zref-dotfill.sty, zref-env.sty, zref-hyperref.sty,
   zref-lastpage.sty, zref-marks.sty, zref-nextpage.sty,
   zref-pageattr.sty, zref-pagelayout.sty, zref-perpage.sty,
   zref-runs.sty, zref-savepos.sty, zref-thepage.sty,
   zref-titleref.sty, zref-totpages.sty, zref-user.sty,
   zref-xr.sty, zref-example.tex, zref-example-lastpage.tex,
   zref-example-nextpage.tex, zref-test1.tex,
   zref-test-base.tex, zref-test-runs.tex,
   zref-test-titleref-memoir.tex.

\endpreamble
\let\MetaPrefix\DoubleperCent

\generate{%
  \file{zref.ins}{\from{zref.dtx}{install}}%
  \file{zref.drv}{\from{zref.dtx}{driver}}%
  \usedir{tex/latex/zref}%
  \file{zref.sty}{\from{zref.dtx}{package}}%
  \file{zref-base.sty}{\from{zref.dtx}{base}}%
  \file{zref-abspage.sty}{\from{zref.dtx}{abspage}}%
  \file{zref-abspos.sty}{\from{zref.dtx}{abspos}}%
  \file{zref-counter.sty}{\from{zref.dtx}{counter}}%
  \file{zref-dotfill.sty}{\from{zref.dtx}{dotfill}}%
  \file{zref-env.sty}{\from{zref.dtx}{env}}%
  \file{zref-hyperref.sty}{\from{zref.dtx}{hyperref}}%
  \file{zref-lastpage.sty}{\from{zref.dtx}{lastpage}}%
  \file{zref-marks.sty}{\from{zref.dtx}{marks}}%
  \file{zref-nextpage.sty}{\from{zref.dtx}{nextpage}}%
  \file{zref-pageattr.sty}{\from{zref.dtx}{pageattr}}%
  \file{zref-pagelayout.sty}{\from{zref.dtx}{pagelayout}}%
  \file{zref-perpage.sty}{\from{zref.dtx}{perpage}}%
  \file{zref-runs.sty}{\from{zref.dtx}{runs}}%
  \file{zref-savepos.sty}{\from{zref.dtx}{savepos}}%
  \file{zref-thepage.sty}{\from{zref.dtx}{thepage}}%
  \file{zref-titleref.sty}{\from{zref.dtx}{titleref}}%
  \file{zref-totpages.sty}{\from{zref.dtx}{totpages}}%
  \file{zref-user.sty}{\from{zref.dtx}{user}}%
  \file{zref-xr.sty}{\from{zref.dtx}{xr}}%
  \usedir{doc/latex/zref}%
  \file{zref-example.tex}{\from{zref.dtx}{example}}%
  \file{zref-example-lastpage.tex}{\from{zref.dtx}{example-lastpage}}%
  \file{zref-example-nextpage.tex}{\from{zref.dtx}{example-nextpage}}%
%  \usedir{doc/latex/zref/test}%
%  \file{zref-test1.tex}{\from{zref.dtx}{test1}}%
%  \file{zref-test-base.tex}{\from{zref.dtx}{test-base}}%
%  \file{zref-test-runs.tex}{\from{zref.dtx}{test-runs}}%
%  \file{zref-test-titleref-memoir.tex}{\from{zref.dtx}{test-titleref-memoir}}%
}

\catcode32=13\relax% active space
\let =\space%
\Msg{************************************************************************}
\Msg{*}
\Msg{* To finish the installation you have to move the following}
\Msg{* files into a directory searched by TeX:}
\Msg{*}
\Msg{*     zref.sty, zref-base.sty, zref-abspage.sty, zref-abspos.sty,}
\Msg{*     zref-counter.sty, zref-dotfill.sty, zref-env.sty,}
\Msg{*     zref-hyperref.sty, zref-lastpage.sty, zref-marks.sty,}
\Msg{*     zref-nextpage.sty, zref-pageattr.sty, zref-pagelayout.sty,}
\Msg{*     zref-perpage.sty, zref-runs.sty, zref-savepos.sty,}
\Msg{*     zref-thepage.sty, zref-titleref.sty, zref-totpages.sty,}
\Msg{*     zref-user.sty, zref-xr.sty}
\Msg{*}
\Msg{* To produce the documentation run the file `zref.drv'}
\Msg{* through LaTeX.}
\Msg{*}
\Msg{* Happy TeXing!}
\Msg{*}
\Msg{************************************************************************}

\endbatchfile
%</install>
%<*ignore>
\fi
%</ignore>
%<*driver>
\NeedsTeXFormat{LaTeX2e}
\ProvidesFile{zref.drv}%
  [2020-07-03 v2.32 A new reference scheme for LaTeX (HO)]%
\documentclass{ltxdoc}
\usepackage{holtxdoc}[2011/11/22]
\setlength{\emergencystretch}{1em}
\begin{document}
  \DocInput{zref.dtx}%
\end{document}
%</driver>
% \fi
%
%
%
% \GetFileInfo{zref.drv}
%
% \title{The \xpackage{zref} package}
% \date{2020-07-03 v2.32}
% \author{Heiko Oberdiek\thanks
% {Please report any issues at \url{https://github.com/ho-tex/zref/issues}}}
%
% \maketitle
%
% \begin{abstract}
% Package \xpackage{zref} tries to get rid of the restriction
% in \hologo{LaTeX}'s reference system that only two properties are
% supported. The package implements an extensible referencing
% system, where properties are handled in a more flexible way.
% It offers an interface for macro programmers for the access
% to the system and some applications that uses the new
% reference scheme.
% \end{abstract}
%
% \tableofcontents
%
% \newcommand*{\metacs}[1]{\texttt{\itshape #1}}
%
% \newcommand{\markexp}{^^A
%   \textsuperscript{exp}^^A
%   \space
% }
% \newcommand{\markexptwo}{^^A
%   \textsuperscript{exp2}^^A
%   \space
% }
% \newcommand{\markrobust}{^^A
%   \textsuperscript{robust}^^A
%   \space
% }
% \newcommand{\marknowarn}{^^A
%   \textsuperscript{nowarn}^^A
%   \space
% }
% \newcommand{\markbabel}{^^A
%   \textsuperscript{babel}^^A
%   \space
% }
%
% \section{Introduction}
%
% Standard \hologo{LaTeX}'s reference system with \cs{label}, \cs{ref},
% and \cs{pageref} supports two properties, the apperance of
% the counter that is last incremented by \cs{refstepcounter} and
% the page with the \cs{label} command.
%
% Unhappily \hologo{LaTeX} does not provide an interface for adding
% another properties. Packages such as \xpackage{hyperref},
% \xpackage{nameref}, or \xpackage{titleref} are forced to
% use ugly hacks to extend the reference system.
% These ugly hacks are one of the causes for \xpackage{hyperref}'s
% difficulty regarding compatibility with other packages.
%
% \subsection{Standard \hologo{LaTeX} behaviour}
%
% References are created by the \cs{label} command:
% \begin{quote}
%   |\chapter{Second chapter}|\\
%   |\section{First section on page 7} % section 2.1|\\
%   |\label{myref}|
% \end{quote}
% Now \hologo{LaTeX} records the section number |2.1| and the page |7|
% in the reference. Internally the reference is a list with
% two entries:
% \begin{quote}
%   |\r@myref| $\rightarrow$ |{2.1}{7}|
% \end{quote}
% The length of the list if fixed in the \hologo{LaTeX} kernel,
% An interface for adding new properties is missing.
%
% There are several tries to add new properties:
% \begin{description}
% \item[\xpackage{hyperref}] uses a list of five
%   properties instead of the standard list with two entries.
%   This causes many compatibility problems with \hologo{LaTeX} and
%   other packages.
% \item[\xpackage{titleref}] stores its title data into the
%   first entry in the list. \hologo{LaTeX} is happy because it
%   does only see its list with two entries. The situation
%   becomes more difficult, if more properties are added
%   this way. Then the macros form a nested structure
%   inside the first reference argument for the label.
%   Expandable extractions will then become painful.
% \end{description}
%
% \subsection{Basic idea}
%
% Some time ago Morten H\o gholm sent me an experimental
% cross referencing mechanism as ``expl3'' code. His
% idea is:
% \begin{quote}
%   |\g_xref_mylabel_plist| $\rightarrow$\\
%   \mbox{}\qquad|\xref_dance_key{salsa}\xref_name_key{Morten}|\dots
% \end{quote}
% The entries have the following format:
% \begin{quote}
%   |\xref_|\meta{your key}|_key{|\meta{some text}|}|
% \end{quote}
% This approach is much more flexible:
% \begin{itemize}
%   \item New properties can easily be added, just use a new key.
%   \item The length of the list is not fixed. A reference
%         can use a subset of the keys.
%   \item The order of the entries does not matter.
% \end{itemize}
% Unhappily I am not familiar with the experimental code
% for \hologo{LaTeX3} that will need some time before its first release.
% Thus I have implemented it as \hologo{LaTeXe} package without disturbing
% the existing \hologo{LaTeX} reference system.
%
% \subsection{Interfaces}
%
% The package provides a generic \emph{interface for programmers}.
% Commands of this interface are prefixed by \cs{zref@}.
%
% Option \xoption{user} enabels the \emph{user interface}.
% Here the commands are prefixed by \cs{z} to avoid name
% clashes with existing macros.
%
% Then the packages provides some \emph{modules}. They
% are applications for the reference system and can also be
% considered as examples how to use the reference system.
%
% The modules can be loaded as packages. The package
% name is prefixed with |zref-|, for example:
%\begin{quote}
%\begin{verbatim}
%\RequirePackage{zref-abspage}
%\end{verbatim}
%\end{quote}
% This is the preferred way if the package is loaded from
% within other packages to avoid option clashes.
%
% As alternative package |zref| can be used and the modules
% are given as options:
%\begin{quote}
%\begin{verbatim}
%\usepackage[perpage,user]{zref}
%\end{verbatim}
%\end{quote}
%
% \section{Interface for programmers}
%
% The user interface is described in the next section \ref{sec:user}.
%
% \subsection{Entities}
%
% \paragraph{Reference.}
% Internally a reference is a list of key value pairs:
% \begin{quote}
%   |\Z@R@myref| $\rightarrow$
%   |\default{2.1}\page{7}|
% \end{quote}
% The generic format of a entry is:
% \begin{quote}
%   |\Z@R@|\meta{refname} $\rightarrow$
%   |\|\meta{propname}|{|\meta{value}|}|
% \end{quote}
% \meta{refname} is the name that denoted references
% (the name used in \cs{label} and \cs{ref}). \meta{propname}
% is the name of the property or key. The property key macro
% is never executed, it is used in parameter text matching only.
%
% \paragraph{Property.}
% Because the name of a property is used in a macro name
% that must survive the \xfile{.aux} file, the name is
% restricted to letters and `@'.
%
% \paragraph{Property list.}
% Often references are used for special purposes. Thus it
% saves memory if just the properties are used in this
% reference that are necessary for its purpose.
%
% Therefore this package uses the concept of \emph{property lists}.
% A property list is a set of properties. The set of properties
% that is used by the default \cs{label} command is the
% \emph{main property list}.
%
% \subsection{Property list}
%
% \markexp means that the implementation of the
% marked macro is expandable. \markexptwo goes a step further
% and marks the macro expandable in exact two expansion steps.
%
% \begin{declcs}{zref@newlist}
%   \M{listname}
% \end{declcs}
% Declares a new empty property list.
%
% \begin{declcs}{zref@addprop}
%   \M{listname} \M{propname}\\
%   \cs{zref@localaddprop} \M{listname} \M{propname}
% \end{declcs}
% Adds the property \meta{propname}
% to the property list \meta{listname}.
% The property and list must exist.
% The addition is global by \cs{zref@addprop} and limited to
% local scope by \cs{zref@localaddprop}.
% Between 2010/04/19 v2.13 and 2010/10/22 v2.19 a comma separated list
% of properties could be used as argument \meta{propname}.
% Since 2010/10/22 v2.19 the addition of several properties
% at once is supported by \cs{zref@addprops}.
%
% \begin{declcs}{zref@addprops}
%   \M{listname} \M{propname list}\\
%   \cs{zref@localaddprops} \M{listname} \M{propname list}
% \end{declcs}
% These macros add a comma separated list of properties
% \meta{propname list} to list \meta{listname}.
% \cs{zref@addprops} works globally and \cs{zref@localaddprops}
% locally. Since 2010/10/22 v2.19.
%
% \begin{declcs}{zref@listexists}
%   \M{listname} \M{then}
% \end{declcs}
% Executes \meta{then} if the property list \meta{listname} exists
% or raise an error otherwise.
%
% \begin{declcs}{zref@iflistundefined}\markexp \M{listname}
%   \M{then} \M{else}
% \end{declcs}
% Executes \meta{then} if the list exists or \meta{else}
% otherwise.
%
% \begin{declcs}{zref@iflistcontainsprop}
%   \M{listname} \M{propname} \M{then} \M{else}
% \end{declcs}
% Executes \meta{then} if the property \meta{propname} is part
% of property list \meta{listname} or otherwise it
% runs the \meta{else} part.
%
% \subsection{Property}
%
% \begin{declcs}{zref@newprop}
%   \* \M{propname} |[|\meta{default}|]| \M{value}
% \end{declcs}
% This command declares and configures a new property with
% name \meta{propname}.
%
% In case of unknown references
% or the property does not exist in the reference, the
% \meta{default} is used as value. If it is not specified
% here, a global default is used, see \cs{zref@setdefault}.
%
% The correct values of some properties are not known
% immediately but at page shipout time. Prominent example
% is the page number. These properties are declared with
% the star form of the command.
%
% \begin{declcs}{zref@setcurrent}
%   \M{propname} \M{value}
% \end{declcs}
% This sets the current value of the property \meta{propname}.
% It is a generalization of setting \hologo{LaTeX}'s \cs{currentlabel}.
%
% \begin{declcs}{zref@getcurrent}\markexptwo
%   \M{propname}
% \end{declcs}
% This returns the current value of the property \meta{propname}.
% The value may not be correct, especially if the property
% is bound to a page (start form of \cs{zref@newprop}) and the
% right value is only known at shipout time (e.g. property `page').
% In case of errors (e.g. unknown property) the empty string is
% returned.
%
% Since version 2010/04/22 v2.14 \cs{zref@getcurrent} supports
% \cs{zref@wrapper@unexpanded}.
%
% \begin{declcs}{zref@propexists}
%   \M{propname} \M{then}
% \end{declcs}
% Calls \meta{then} if the property \meta{propname} is available
% or generates an error message otherwise.
%
% \begin{declcs}{zref@ifpropundefined}\markexp
%   \M{propname} \M{then} \M{else}
% \end{declcs}
% Calls \meta{then} or \meta{else} depending on the existence
% of property \meta{propname}.
%
% \subsection{Reference generation}
%
% \begin{declcs}{zref@label}
%   \M{refname}
% \end{declcs}
% This works similar to \cs{label}. The reference \meta{refname} is
% created and put into the \xfile{.aux} file with the
% properties of the main property list.
%
% \begin{declcs}{zref@labelbylist}
%   \M{refname} \M{listname}
% \end{declcs}
% Same as \cs{zref@label} except that the properties are
% taken from the specified property list \meta{listname}.
%
% \begin{declcs}{zref@labelbyprops}
%   \M{refname} |{|\meta{propnameA}|,|\meta{propnameB}|,|\dots|}|
% \end{declcs}
% Same as \cs{zref@label} except that these properties
% are used that are given as comma separated list in the
% second argument.
%
% \begin{declcs}{zref@newlabel}
%   \M{refname} |{|\dots|}|
% \end{declcs}
% This is the macro that is used in the \xfile{.aux} file.
% It is basically the same as \cs{newlabel} apart from
% the format of the data in the second argument.
%
% \subsection{Data extraction}
%
% \begin{declcs}{zref@extractdefault}\markexptwo
%   \M{refname} \M{propname} \M{default}
% \end{declcs}
% This is the basic command that refernces the value of a
% property \meta{propname} for the reference \meta{refname}.
% In case of errors such as undefined reference the \meta{default}
% is used instead.
%
% \begin{declcs}{zref@extract}\markexptwo
%   \M{refname} \M{propname}
% \end{declcs}
% The command is an abbreviation for \cs{zref@extractdefault}.
% As default the default of the property is taken, otherwise
% the global default.
%
% Example for page references:
% \begin{quote}
% \begin{tabular}{@{}ll}
% \hologo{LaTeX}:& |\pageref{foobar}|\\
% \xpackage{zref}:& |\zref@extract{foobar}{page}|
% \end{tabular}
% \end{quote}
% Both \cs{zref@extract} and \cs{zref@extractdefault} are
% expandable. That means, these macros can directly
% be used in expandable calculations, see the example file.
% On the other side, babel's shorthands are not supported,
% there are no warnings in case of undefined references.
%
% If an user interface doesn't need expandable macros
% then it can use \cs{zref@refused} and \cs{zref@wrapper@babel}
% for its user macros.
%
% \begin{declcs}{zref@refused}
%   \M{refname}
% \end{declcs}
% This command is not expandable. It causes the warnings if
% the reference \meta{refname} is not defined. Use the
% \cs{zref@extract} commands inside expandable contexts and
% mark their use outside by \cs{zref@refused}, see the example
% file.
%
% \begin{declcs}{zref@def@extract}%
%     \M{cmd} \M{refname} \M{propname}\\
%   \cs{zref@def@extractdefault}
%     \M{cmd} \M{refname} \M{propname} \M{default}
% \end{declcs}
% Both macros extract the property \meta{propname} from the reference
% \meta{refname} the same way as macros \cs{zref@extract} and
% \cs{zref@extractdefault}. The result is stored in macro \meta{cmd}.
% Also \cs{zref@refused} is called to notify \hologo{LaTeX} that
% the reference \meta{refname} is used. Added in 2011/10/04 v2.22.
%
% \begin{declcs}{zref@ifrefundefined}\markexp
%   \M{refname} \M{then} \M{else}
% \end{declcs}
% Macro \cs{zref@ifrefundefined} calls arguments \meta{then} or \meta{else}
% dependent on the existence of the reference \meta{refname}.
%
% \begin{declcs}{zifrefundefined}
%   \M{refname} \M{then} \M{else}
% \end{declcs}
% Macro \cs{zifrefundefined} calls \cs{ref@refused} before
% executing \cs{zref@ifrefundefined}. Babel shorthands are
% supported in \meta{refname}.
%
% \begin{declcs}{zref@ifrefcontainsprop}\markexp
%   \M{refname} \M{propname} \M{then} \M{else}
% \end{declcs}
% Test whether a reference provides a property.
%
% \subsection{Setup}
%
% \begin{declcs}{zref@default}
% \end{declcs}
% Holds the global default for unknown values.
%
% \begin{declcs}{zref@setdefault}
%   |{|\meta{value}|}|
% \end{declcs}
% Sets the global default for unknown values.
% The global default is used, if a property does not specify
% an own default and the value for a property
% cannot be extracted. This can happen if the reference is
% unknown or the reference does not have the property.
%
% \begin{declcs}{zref@setmainlist}
%   |{|\meta{value}|}|
% \end{declcs}
% Sets the name of the main property list. The package
% sets and uses |main|.
%
% \subsection{Declared properties}
%
% \begin{quote}
% \renewcommand*{\arraystretch}{1.05}
% \setlength{\tabcolsep}{2\tabcolsep}
% \def\VV{\vphantom{\large \"Ag}}
% \def\EMPTY{{\itshape\textless empty\textgreater}}
% \def\pl#1{&#1&thepage&\cs{number}\cs{#1}}
% \begin{tabular}{@{}>{\sffamily}l|>{\ttfamily}l|>{\ttfamily}l|>{\ttfamily}l@{}}
%   \rmfamily Module&
%   \rmfamily Property&
%   \rmfamily Property list&
%   \rmfamily Default\\
%   \hline
%   (base)
%     & default & main & \EMPTY\\
%     & page & main & \EMPTY\\
%   \hline
%   abspage
%     & abspage & main & 0\\
%   \hline
%   counter
%     & counter & main & \EMPTY\\
%   \hline
%   hyperref
%     & anchor & main & \EMPTY\\
%     & url    & & \EMPTY\\
%   \hline
%   pageattr
%     & pdfpageattr & thepage & \dots\\
%     & pdfpagesattr & LastPage & \dots\\
%   \hline
%   pagelayout\footnotemark
%     \pl{mag}\\
%     \pl{paperwidth}\\
%     \pl{paperheight}\\
%     \pl{stockwidth}\\
%     \pl{stockheight}\\
%     \pl{pdfpageheight}\\
%     \pl{pdfpagewidth}\\
%     \pl{pdfhorigin}\\
%     \pl{pdfvorigin}\\
%     \pl{hoffset}\\
%     \pl{voffset}\\
%     \pl{topmargin}\\
%     \pl{oddsidemargin}\\
%     \pl{evensidemargin}\\
%     \pl{textwidth}\\
%     \pl{textheight}\\
%     \pl{headheight}\\
%     \pl{headsep}\\
%     \pl{footskip}\\
%     \pl{marginparwidth}\\
%     \pl{marginparsep}\\
%     \pl{columnwidth}\\
%     \pl{columnsep}\\
%   \hline
%   perpage
%     & pagevalue & perpage & 0\\
%     & page & perpage & \EMPTY\\
%     & abspage & perpage & 0\\
%   \hline
%   savepos
%     & posx & savepos & 0\\
%     & posy & savepos & 0\\
%   \hline
%   titleref
%     & title & main & \EMPTY\\
%   \hline
%   xr
%     & anchor   & & \EMPTY\\
%     & externaldocument & & \EMPTY\\
%     & theotype & & \EMPTY\\
%     & title    & & \EMPTY\\
%     & url      & & \EMPTY\\
% \end{tabular}
% \footnotetext{Module \xmodule{pagelayout} only defines properties
% if the parameter exists.}%
% \end{quote}
%
% \subsection{Wrapper for advanced situations}
%
% \begin{declcs}{zref@wrapper@babel}
%   |{|\dots|}| \M{name}
% \end{declcs}
% This macro helps to add shorthand support. The second
% argument is protected, then the code of the first argument
% is called with the protected name appended. Examples
% are in the sources.
%
% \begin{declcs}{zref@wrapper@immediate}
%   |{|\dots|}|
% \end{declcs}
% There are situations where a label must be written instantly
% to the \xfile{.aux} file, for example after the last page.
% If the \cs{zlabel} or \cs{label} command is put inside this wrapper,
% immediate writing is enabled. See the implementation for
% module \xmodule{lastpage} for an example of its use.
%
% \begin{declcs}{zref@wrapper@unexpanded}
%   |{|\dots|}|
% \end{declcs}
% Assuming someone wants to extract a value for property |bar|
% and store the result in a macro |\foo| without traces of
% the expanding macros and without expanding the value.
% This (theoretical?) problem can be solved by this wrapper:
% \begin{quote}
%\begin{verbatim}
%\zref@wrapper@unexpanded{%
%  \edef\foo{%
%    \zref@extract{someref}{bar}%
%  }%
%}
%\end{verbatim}
% \end{quote}
% The \cs{edef} forces the expansion of \cs{zref@extract},
% but the extraction of the value is prevented by
% the wrapper that uses \hologo{eTeX}'\ \cs{unexpanded} for this
% purpose. Supported macros are \cs{zref@extract}, \cs{zref@extractdefault}
% and since version 2010/04/22 v2.14 macro \cs{zref@getcurrent}.
%
% \subsection{Counter for unique names}
%
% Some modules (\xmodule{titleref} and \xmodule{dotfillmin})
% need unique names for automatically generated label names.
%
% \begin{declcs}{zref@require@unique}
% \end{declcs}
% This command creates the unique counter |zref@unique| if the
% counter does not already exist.
%
% \begin{declcs}{thezref@unique}
% \end{declcs}
% This command is used to generate unique label names.
%
% \section{User interface}\label{sec:user}
%
% \subsection{Module \xmodule{user}}
%
% The user interface for this package and its modules
% is enabled by \xpackage{zref}'s package option \xmodule{user}
% or package \xpackage{zref-user}.
% The names of user commands are prefixed by |z| in order
% to avoid name clashes with existing macros of the same
% functionality. Thus the package does not disturb the
% traditional reference scheme, both can be used together.
%
% The syntax descriptions contain the following markers
% that are intended as hints for programmers:
% \begin{quote}
% \begin{tabular}{@{}ll@{}}
%   \markbabel\unskip  & Babel shorthands are allowed.\\
%   \markrobust\unskip & Robust macro.\\
%   \markexp\unskip    & Expandable version:\\
%                      & \textbullet\ robust,
%                        unless the extracted values are fragile,\\
%                      & \textbullet\ no babel shorthand suport.\\
%   \markexptwo\unskip & Expandable like \markexp and:\\
%                      & \textbullet\ expandable in exact two steps.\\
% \end{tabular}
% \end{quote}
%
% The basic user interface of the package without modules
% are commands that mimic the standard \hologo{LaTeX} behaviour
% of \cs{label}, \cs{ref}, and \cs{pageref}:
%
% \begin{declcs}{zlabel}
%   \M{refname}\markbabel
% \end{declcs}
% Similar to \cs{label}. It generates a label with name
% \meta{refname} in the new reference scheme.
%
% \begin{declcs}{zref}
%   |[|\meta{propname}|]| \M{refname}\markbabel
% \end{declcs}
% Without optional argument similar to \cs{ref}, it returns
% the default reference property. This property is named
% |default|:
% \[
%   |\zref{|x|}| \equiv |\zref[default]{|x|}|
% \]
%
% \begin{declcs}{zpageref}
%   \M{refname}\markbabel
% \end{declcs}
% Convenience macro, similar to \cs{pageref}.
% \[
%   |\zpageref{|x|}| \equiv |\zref[page]{|x|}|
% \]
%
% \begin{declcs}{zrefused}
%   \M{refname}\markbabel
% \end{declcs}
% Some of the user commands in the modules are expandable.
% The use of such commands do not cause any undefined
% reference warnings, because inside of expandable
% contexts this is not possible. However, if there is
% a place outside of expandable contexts, \cs{refused}
% is strongly recommended. The reference \meta{refname}
% is marked as used, undefined ones will generate
% warnings.
%
% \subsection{Module \xmodule{abspage}}
%
% With the help of package \xpackage{atbegshi}
% a new counter |abspage| with absolute page numbers is provided.
% For technical and historical reasons the counter itself is zero based:
% if you use it directly on the first page, e.g
% with |\arabic{abspage}| you will get 0 as value. When using |\zref|
% the first page will be page 1 as expected.
% Also a new property |abspage| is defined and added to the
% main property list. Thus you can reference the absolute
% page number:
% \begin{quote}
%   |Section \zref{foo} is on page \zpageref{foo}.|\\
%   |This is page \zref[abspage]{foo}|\\
%   |of \zref[abspage]{LastPage}.|
% \end{quote}
% The example also makes use of module \xmodule{lastpage}.
%
% \subsection{Module \xmodule{lastpage}}
%
% Provides the functionality of package \xpackage{lastpage}
% \cite{lastpage} in the new reference scheme.
% The label |LastPage| is put at the end of the document.
% You can refer the last page number with:
% \begin{quote}
%   |\zref@extract{LastPage}{page}| (+ |\zref@refused{LastPage}|)\\
% \end{quote}
% or
% \begin{quote}
%   |\zpageref{LastPage}| (module \xmodule{user})
% \end{quote}
%
% Since version 2008/10/01 v2.3 the module defines the list |LastPage|.
% In addition to the properties of the main list label |LastPage| also
% stores the properties of this list |LastPage|. The default of this
% list is empty. The list can be used by the user to add additional
% properties for label |LastPage|.
%
% \subsubsection{Tests for last page}
%
% Since version 2010/03/26 v2.8 the macros \cs{zref@iflastpage}
% and \cs{ziflastpage} were added. They test the reference,
% whether it is a reference of the last page.
%
% \begin{declcs}{zref@iflastpage}\markexp \M{refname} \M{then} \M{else}
% \end{declcs}
% Macro \cs{zref@iflastpage} compares the references \meta{refname}
% with \meta{LastPage}. Basis of the comparison is the value of
% property |abspage|, because the values are different for different
% pages. This is not ensured by property |page|. Therefore module
% \xmodule{abspage} is loaded by module \xmodule{lastpage}.
% If both values of property |abspage| are present and match,
% then \meta{then} is executed, otherwise code \meta{else} is called.
% If one or both references are undefined or lack the property |abspage|,
% then \meta{else} is executed.
%
% Macro \cs{zref@iflastpage} is expandable, therefore \cs{zref@refused}
% should be called on \meta{refname} and \meta{LastPage}.
%
% \begin{declcs}{ziflastpage} \M{refname} \M{then} \M{else}
% \end{declcs}
% Macro \cs{ziflastpage} has the same function as \cs{zref@iflastpage},
% but adds support for babel shorthands in \meta{refname} and
% calls \cs{zref@refused}. However macro \cs{ziflastpage} is not
% expandable.
%
% \subsubsection{Example}
%
%    \begin{macrocode}
%<*example-lastpage>
%<<END_EXAMPLE
\NeedsTeXFormat{LaTeX2e}
\documentclass{report}

\newcounter{foo}
\renewcommand*{\thefoo}{\Alph{foo}}

\usepackage{zref-lastpage,zref-user}[2019/11/29]

\makeatletter
\zref@newprop{thefoo}{\thefoo}
\zref@newprop{valuefoo}{\the\value{foo}}
\zref@newprop{chapter}{\thechapter}
\zref@addprops{LastPage}{thefoo,valuefoo,chapter}
\makeatother

\newcommand*{\foo}{%
  \stepcounter{foo}%
  [Current foo: \thefoo]%
}

\begin{document}
  \chapter{First chapter}
  Last page is \zref{LastPage}.\\
  Last chapter is \zref[chapter]{LastPage}.\\
  Last foo is \zref[thefoo]{LastPage}.\\
  Last value of foo is \zref[valuefoo]{LastPage}.\\
  \foo
  \chapter{Second chapter}
  \foo\foo\foo
  \chapter{Last chapter}
  \foo
\end{document}
%END_EXAMPLE
%</example-lastpage>
%    \end{macrocode}
%
% \subsection{Module \xmodule{thepage}}
%
% This module \xmodule{thepage} loads module \xmodule{abspage},
% constructs a reference name using the absolute page number
% and remembers property |page|. Other properties can be added
% by adding them to the property list |thepage|.
%
% \begin{declcs}{zthepage} \M{absolute page number}\\
% \end{declcs}
% Macro \cs{zthepage} is basically a \cs{zpageref}. The reference
% name is yield by the \meta{absolute page number}. If the reference
% is not defined, then the default for property |page| is used.
%
% \begin{declcs}{zref@thepage@name}\markexp \M{absolute page number}
% \end{declcs}
% Macro \cs{zref@thepage@name} returns the internal reference
% name that is constructed using the \meta{absolute page number}.
% The internal reference name should not be used directly, because
% it might change in future versions.
%
% \begin{declcs}{zref@thepage}\markexp \M{absolute page number}\\
%   \cs{zref@thepage@refused} \M{absolute page number}
% \end{declcs}
% Macro \cs{zref@thepage} returns the page number (\cs{thepage})
% of \meta{absolute page number}. Because this macro is expandable,
% \cs{zref@thepage@refused} is used outside an expandable context
% to mark the reference as used.
%
%
% \subsection{Module \xmodule{nextpage}}
%
% \begin{declcs}{znextpage}
% \end{declcs}
% Macro \cs{znextpage} prints \cs{thepage} of the following page.
% It gets the current absolute page number by using a label.
% There are three cases for the next page:
% \begin{enumerate}
% \item The next page is not known yet because of undefined references.
%   Then \cs{zunknownnextpagename} is used instead. The default for
%   this macro is the default of property |page|.
% \item This page is the last page. Then \cs{znonextpagename}
%   is used. Its default is empty.
% \item The next page is known, then \cs{thepage} of the next
%   page is used (the value of property |page| of the next page).
% \end{enumerate}
%
% \subsubsection{Configuration}
%
% The behaviour can be configured by the following macros.
%
% \begin{declcs}{zunknownnextpagename}\\
%   \cs{znonextpagename}
% \end{declcs}
% If the next page is not known or available, then
% \cs{znextpage} uses these name macros as default.
% \cs{zunknownnextpagename} is used in case of undefined
% references. Default is the value of property |page| of the
% next page (\cs{thepage}). Module \xmodule{thepage} is used.
%
% Macro \cs{znonextpagename} is used, if the next page does
% not exists. That means that the current page is last page.
% The default is empty.
%
% \begin{declcs}{znextpagesetup} \M{unknown} \M{no next}
%   \M{next}
% \end{declcs}
% Acording to the case (see \cs{znextpage}) macro \cs{znextpage}
% calls an internal macro with an argument. The argument is
% either \cs{thepage} of the next page or one of \cs{zunknownnextpagename}
% or \cs{znonextpagename}. These internal macro can be changed by
% \cs{znextpagesetup}. It expects the definition texts for these
% three cases of a macro with one argument.
% The default is
% \begin{quote}
%   |\znextpagesetup{#1}{#1}{#1}|
% \end{quote}
%
% \subsubsection{Example}
%
%    \begin{macrocode}
%<*example-nextpage>
%<<END_EXAMPLE
\documentclass{book}

\usepackage{zref-nextpage}[2019/11/29]
\znextpagesetup
  {\thepage}% next page is unknown
  {\thepage\ (#1)}% this page is last page
  {\thepage\ $\rightarrow$ #1}% next page is known
\renewcommand*{\znonextpagename}{last page}

\usepackage{fancyhdr}
\pagestyle{fancy}
\fancyhf{}
\fancyhead[LE,RO]{\znextpage}
\fancypagestyle{plain}{%
  \fancyhf{}%
  \fancyhead[LE,RO]{\znextpage}%
}

\begin{document}
\frontmatter
  \tableofcontents
\mainmatter
  \chapter{Hello World}
  \clearpage
  \section{Last section}
\end{document}
%END_EXAMPLE
%</example-nextpage>
%    \end{macrocode}
%
%
% \subsection{Module \xmodule{totpages}}
%
% For the total number of pages of a document you need to
% know the absolute page number of the last page.
% Both modules \xmodule{abspage} and \xmodule{lastpage}
% are necessary and automatically enabled.
%
% \begin{declcs}{ztotpages}\markexp
% \end{declcs}
% Prints the total number of pages or |0| if this
% number is not yet known. It expands to an explicit number
% and can also used even in expandable calculations (\cs{numexpr})
% or counter assignments.
%
% \subsection{Module \xmodule{pagelayout}}
%
% The module defines additional properties for each parameter
% of the page layout that is effective during page shipout.
% The value of length parameters is given in sp without the
% unit as plain number.
%
% Some parameters are specific for a class (e.g.\@ \xoption{stockwidth}
% and \xoption{stockheight} for class \xclass{memoir}) or the
% \hologo{TeX} engine like \hologo{pdfTeX}. If the parameter is not
% available, then the property will not be defined. The default value
% of the property is the current setting of the parameter.
%
% The module \xmodule{thepage} is loaded that generates a label
% for each page. The properties of module \xmodule{pagelayout}
% are added to the property list |thepage| of module \xmodule{thepage}.
%
% List of properties:
% \begin{quote}
%   \def\x#1{\cs{#1}&#1}%
%   \begin{tabular}{l>{\sffamily}ll}
%     \textrm{parameter} & \textrm{property} & remarks\\
%     \hline
%     \x{mag}\\
%     \x{paperwidth}\\
%     \x{paperheight}\\
%     \x{stockwidth}& class \xclass{memoir}\\
%     \x{stockheight}& class \xclass{memoir}\\
%     \x{pdfpagewidth}& \hologo{pdfTeX}, \hologo{LuaTeX}\\
%     \x{pdfpageheight}& \hologo{pdfTeX}, \hologo{LuaTeX}\\
%     \x{pdfhorigin}& \hologo{pdfTeX}, \hologo{LuaTeX}\\
%     \x{pdfvorigin}& \hologo{pdfTeX}, \hologo{LuaTeX}\\
%     \x{hoffset}\\
%     \x{voffset}\\
%     \x{topmargin}\\
%     \x{oddsidemargin}\\
%     \x{evensidemargin}\\
%     \x{textwidth}\\
%     \x{textheight}\\
%     \x{headheight}\\
%     \x{headsep}\\
%     \x{footskip}\\
%     \x{marginparwidth}\\
%     \x{marginparsep}\\
%     \x{columnwidth}\\
%     \x{columnsep}\\
%   \end{tabular}
% \end{quote}
%
% \begin{declcs}{zlistpagelayout}
% \end{declcs}
% At the end of document the page layout parameter for each page are
% printed into the \xext{log} file if macro \cs{zlistpagelayout}
% is called before |\end{document}| (preamble is a good place).
%
% \subsection{Module \xmodule{marks}}
%
% ToDo.
%
% \subsection{Module \xmodule{runs}}
%
% Module \xmodule{runs} counts the \hologo{LaTeX} runs since last
% \xext{aux} file creation and prints the number in the
% \xext{log} file.
%
% \begin{declcs}{zruns}\markexp
% \end{declcs}
% Prints the the total number of \hologo{LaTeX} runs including
% the current one. It expands to an explicit number.
% Before |begin{document}| the value is zero meaning
% the \xext{aux} file is not read yet. If a previous
% \xext{aux} file exists, the value found there increased
% by one is the new number. Otherwise \cs{zruns} is set to one.
% \hologo{LaTeX}~runs where the \xext{aux} files are not rewritten
% are not counted (see \cs{nofiles}).
%
% \subsection{Module \xmodule{perpage}}
%
% With \cs{@addtoreset} or \cs{numberwithin} a counter can be
% reset if another counter is incremented. This do not work
% well if the other counter is the page counter. The page counter is
% incremented in the output routine that is often called asynchronous
% somewhere on the next page. A reference mechanism costs at least
% two \hologo{LaTeX} runs, but ensures correct page counter values.
%
% \begin{declcs}{zmakeperpage}
%   |[|\meta{reset}|]| \M{counter}
% \end{declcs}
% At the of a new page counter \meta{counter} starts
% counting with value \meta{reset} (default is \texttt{1}).
% The macro has the same
% syntax and semantics as \cs{MakePerPage} of package
% \xpackage{perpage} \cite{perpage}.
% Also \xoption{perpage} of package \xpackage{footmisc} \cite{footmisc}
% can easily be simulated by
% \begin{quote}
% |\zmakeperpage{footnote}| \itshape |% \usepackage[perpage]{footmisc}|
%\end{quote}
% If footnote symbols are used, some people dislike the
% first symbol \ensuremath{\dagger}. It can easily be skipped:
% \begin{quote}
% |\zmakeperpage[2]{footnote}|
% \end{quote}
%
% \begin{declcs}{thezpage}\\
%   counter |zpage|
% \end{declcs}
% If the formatted counter value of the counter that is reset at
% a new page contains the page value, then you can use \cs{thezpage},
% the page number of the current page. Or counter |zpage| can be
% used, if the page number should be formatted differently from
% the current page number. Example:
%\begin{quote}
%\begin{verbatim}
%\newcounter{foobar}
%\zmakeperpage{foobar}
%\renewcommand*{\thefoobar}{\thezpage-\arabic{foobar}}
% % or
%\renewcommand*{\thefoobar}{\roman{zpage}-\arabic{foobar}}
%\end{verbatim}
%\end{quote}
%
% \begin{declcs}{zunmakeperpage}
%   \M{counter}
% \end{declcs}
% The reset mechanism for this counter is deactivated.
%
%
% \subsection{Module \xmodule{counter}}
%
% This option just add the property |counter| to the main
% property list. The property stores the counter name,
% that was responsible for the reference. This is the
% property \xpackage{hyperref}'s \cs{autoref} feature
% uses. Thus this property |counter| may be useful
% for a reimplementation of the autoref feature,
% see the section \ref{sec:todo} with the todo list.
%
% \subsection{Module \xmodule{titleref}}
%
% This option makes section and caption titles available
% to the reference system similar to packages \xpackage{titleref}
% or \xpackage{nameref}.
%
% \begin{declcs}{ztitleref}
%   \M{refname}\markbabel
% \end{declcs}
% Print the section or caption title of reference \meta{refname},
% similar to \cs{nameref} or \cs{titleref}.
%
% \begin{declcs}{ztitlerefsetup}
%   |{|\textit{key$_1$=value$_1$, key$_2$=value$_2$, \dots}|}|
% \end{declcs}
% This command allows to configure the behaviour of module
% \xmodule{titleref}. The following keys are available:
% \begin{description}
% \let\bfseries\relax
% \item[\texttt{title=}\meta{value}]\mbox{}\\*
%   Sets the current title.
% \item[\texttt{stripperiod=true\string|false}]\mbox{}\\*
%   Follow package \xpackage{nameref} that
%   removes a last period. Default: |true|.
% \item[\texttt{expand=true\string|false}]\mbox{}\\*
%   Package \cs{titleref} expands the title first. This way
%   garbage and dangerous commands can be removed, e.g.
%   \cs{label}, \cs{index}\dots. See implementation section
%   for more details. Default is |false|.
% \item[\texttt{cleanup=\{}\textrm{\dots}\texttt{\}}]\mbox{}\\*
%   Hook to add own cleanup code, if method |expand| is used.
%   See implementation section for more details.
% \end{description}
%
% \subsection{Module \xmodule{savepos}}
%
% This option supports a feature that \pdfTeX\ provides
% (and \hologo{XeTeX}).
% \pdfTeX\ is able to tell the current position on the
% page. The page position is not instantly known. First the
% page must be constructed by \hologo{TeX}'s asynchronous output
% routine. Thus the time where the position is known is
% the page shipout time. Thus a reference system where the
% information is recorded in the first run and made available
% for use in the second run comes in handy.
%
% \begin{declcs}{zsavepos}
%   \M{refname}
% \end{declcs}
% It generates a reference with name \meta{refname}.
% The reference stores the location where \cs{zsavepos}
% is executed in properties |posx| and |posy|.
%
% \begin{declcs}{zsaveposx}
%   \M{refname}\\
%   \cs{zsaveposy}
%   \M{refname}
% \end{declcs}
% Same as \cs{zsavepos} except that only the |x| or |y| component
% of the position is stored. Since 2011/12/05 v2.23.
%
% \begin{declcs}{zposx}\markexp
%   \M{refname}\\
%   \SpecialUsageIndex{\zposy}\cs{zposy}\markexp \M{refname}
% \end{declcs}
% Get the position as number. Unit is sp. Horizontal
% positions by \cs{zposx} increase from left to right.
% Vertical positions by \cs{zposy} from bottom to top.
%
% Do not rely on absolute page numbers. Because of problems
% with the origin the numbers may differ in DVI or PDF mode
% of \pdfTeX. Therefore work with relative values by comparisons.
%
% Both \cs{zposx} and \cs{zposy} are expandable and can be
% used inside calculations (\cs{setcounter}, \cs{addtocounter},
% package \xpackage{calc}, \cs{numexpr}). However this property
% prevents from notifying \hologo{LaTeX} that the reference is actually
% used (the notifying is not expandable). Therefore you should
% mark the reference as used by \cs{zrefused}.
%
% This module uses \pdfTeX's \cs{pdfsavepos}, \cs{pdflastxpos},
% and \cs{pdflastypos}. They are available in PDF mode and
% since version 1.40.0 also in DVI mode.
%
% \begin{declcs}{zref@savepos}
% \end{declcs}
% Macro \cs{zref@savepos} performs the first part of \cs{zsavepos}
% by calling \cs{pdfsavepos} (if \xext{aux} files are writable).
%
% Thus \cs{zsavepos} is basically \cs{zref@savepos} followed
% by |\zref@labelbylist{|\texttt{\meta{refname}}|}{savepos}|.
% If \cs{TeXXeTstate} is detected and enabled, \cs{savepos}
% also adds \cs{zref@savepos} at the end to support \cs{beginR}
% where the whatits are processed in reverse order.
% The property list |savepos| contains the properties |posx|
% and |posy|.
%
% \subsection{Module \xmodule{abspos}}
% Module \xmodule{abspos} allows to get various values of the page layout. There is no user
% command, only a number of internal commands. For example:
% \begin{quote}
%   |\zref@absposx{|\meta{label}|}{|\meta{value}|}{|\meta{position}|}|\\
%   |\zref@absposy{|\meta{label}|}{|\meta{value}|}{|\meta{position}|}|
% \end{quote}
%
% The return value is like in the module \xmodule{savepos} a number representing a length in sp.
% The length are measured from the bottom left of the page.
%
% \meta{label} is a label set with \cs{zlabel} or \cs{zsavepos} that allows to retrieve the
% absolute page number.
%
% \meta{position} is for the x-command one of |left|, |right| or |center|. For the y-command
%  it is one of |top|, |bottom|, |center|.
%
% The possible content of \meta{value} can be seen in the following table. Be aware that the code
% makes some assumptions which are perhaps not always true -- for example that the left of
% the head is identical to the left of the body.
%
% \medskip
% \begin{tabular}{lll}
%  \textbf{value} & \textbf{axis} & \textbf{comments}\\\hline
%   media         & x             &   left=0, right=\cs{pdfpagewidth} \\
%   paper         & x             &   left=0, right=\cs{paperwidth}\\
%   stock         & x             &   derived from paper\\
%   media         & y             & bottom=0, top=\cs{pdfpageheigh}\\
%   paper         & y             & top=\cs{pdfpageheight}, bottom=top-\cs{paperheight}\\
%   stock         & y             & top  derived from paper\\
%   head          & x             & calculated with hoffset, horigin, etc\\
%   head          & y             & calculated\\
%   body          & x             & = head value\\%
%   body          & y             & = head bottom - \cs{headsep}\\
%   foot          & x             & = head\\
%   foot          & y             & calculated from body bottom and \cs{footskip}\\
%   marginpar     & x             &different on odd/even pages! \\
%   marginpar     & y             & = body %
% \end{tabular}
%
% \subsection{Module \xmodule{dotfill}}
%
% \begin{declcs}{zdotfill}
% \end{declcs}
% This package provides the command \cs{zdotfill} that works
% similar to \cs{dotfill}, but can be configured. Especially
% it suppresses the dots if a minimum number of dots cannot be set.
%
% \begin{declcs}{zdotfillsetup}
%   |{|\textit{key$_1$=value$_1$, key$_2$=value$_2$, \dots}|}|
% \end{declcs}
% This command allows to configure the behaviour of \cs{zdotfill}.
% The following keys are available:
% \begin{description}
% \let\bfseries\relax
% \item[\texttt{min=}\meta{count value}]\mbox{}\\*
%   If the actual number of dots are smaller than \meta{count value},
%   then the dots are suppressed. Default: |2|.
% \item[\texttt{unit=}\meta{dimen value}]\mbox{}\\*
%   The width of a dot unit is given by \meta{dimen value}.
%   Default: |0.44em| (same as the unit in \cs{dotfill}).
% \item[\texttt{dot=}\meta{value}]\mbox{}\\*
%   The dot itself is given by \meta{value}. Default: |.| (dot, same as
%   the dot in \cs{dotfill}).
% \end{description}
%
% \subsection{Module \xmodule{env}}
%
% This module defines two properties |envname| and |envline|. They
% remember the name of the environment and the line number at
% the start of the environment.
%
% \subsection{Module \xmodule{xr}}
%
% This package provides the functionality of package \xpackage{xr},
% see \cite{xr}. It also supports the syntax of \xpackage{xr-hyper}.
%
% \begin{declcs}{zexternaldocument}
%   \* |[|\meta{prefix}|]|\markbabel \M{external document}
%   |[|\meta{url}|]|
% \end{declcs}
% See \cs{externaldocument} for a description of this option.
% The found labels also get a property |externaldocument| that
% remembers \meta{external document}.
% The standard reference scheme and the scheme of this
% package use different name spaces for reference names.
% If the external document uses both systems. Then one
% import statement would put the names in one namespace
% and probably causing problems with multiple references
% of the same name. Thus the star form only looks for
% \cs{newlabel} in the \xfile{.aux} files, whereas without
% star only \cs{zref@newlabel}s are used.
%
% In the star form it tries to detect labels from
% \xpackage{hyperref}, \xpackage{titleref}, and \xpackage{ntheorem}.
% If such an extended property from the packages before cannot
% be found or are empty, they are not included in the
% imported reference.
%
% Warnings are given if a reference name is already in use
% and the item is ignored. Unknown properties will automatically
% be declared.
%
% If the external references contain |anchor| properties,
% then we need also a url to be able to address the external
% file. As default the filename is taken with a default
% extension.
%
% \begin{declcs}{zxrsetup}
%   |{|\textit{key$_1$=value$_1$, key$_2$=value$_2$, \dots}|}|
% \end{declcs}
% The following setup options are available:
% \begin{description}
% \item[\xoption{ext}:] It sets the default extension.
% \item[\xoption{tozreflabel}:] Boolean option. The found references
%   are imported as zref labels. This is enabled by default.
% \item[\xoption{toltxlabel}:] Boolean option. The found references
%   are imported as \hologo{LaTeX} labels. Packages \xpackage{nameref},
%   \xpackage{titleref} and class \xclass{memoir} are supported.
% \item[\xoption{urluse}:] Boolean option. If enabled, then a URL
%   is stored in a macro and the macro is put in property `urluse'.
%   The URL is not put in property `url'.
%   The purpose is to save \hologo{TeX} memory.
% \item[\xoption{verbose}:] Boolean option. List the imported labels in the
%   \xext{log} file. Default is |false|.
% \end{description}
%
% \begin{declcs}{zref@xr@ext}
% \end{declcs}
% If the \meta{url} is not specified in \cs{zref@externaldocument},
% then the url will be constructed with the file name and
% this macro as extension. \cs{XR@ext} is used if
% \xpackage{hyperref} is loaded, otherwise |pdf|.
%
% \subsection{Module \xmodule{pageattr}}
%
% This module allows to recover the content of the register
% \cs{pdfpageattr} and \cs{pdfpagesattr} in pdftex and the equivalent register
% in luatex. There is no user command. Programmer commands are
%
% \begin{quote}
% |\zref@pdfpageattr{|\meta{absolute page number}|}|
%
% |\zref@pdfpagesattr{|\meta{absolute page number}|}|
% \end{quote}
% \section{ToDo}\label{sec:todo}
%
% Among other things the following issues are left for future work:
% \begin{itemize}
% \item Other applications: autoref, hyperref, \dots
% \end{itemize}
%
% \section{Example}
%
%    \begin{macrocode}
%<*example>
%    \end{macrocode}
%
%    \begin{macrocode}
\documentclass{book}

\usepackage[ngerman]{babel}%

\usepackage[savepos,totpages,titleref,dotfill,counter,user]{zref}

%    \end{macrocode}
%
%    Chapters are wrapped inside \cs{ChapterStart} and
%    \cs{ChapterStop}. The first argument \verb|#1| of \cs{ChapterStart}
%    is used to form a label id \verb|chap:#1|. At the end of
%    the chapter another label is set by \cs{zref@wrapper@immediate}, because
%    otherwise at the end of document a deferred write would not
%    be written, because there is no page for shipout.
%
%    Also this example shows how chapter titles can be recorded.
%    A new property \verb|chaptitle| is declared and added to the
%    main property list. In \cs{ChapterStart} the current value
%    of the property is updated.
%    \begin{macrocode}
\makeatletter
\zref@newprop{chaptitle}{}
\zref@addprop{main}{chaptitle}

\newcommand*{\ChapterStart}[2]{%
  \cleardoublepage
  \def\current@chapid{#1}%
  \zref@setcurrent{chaptitle}{#2}%
  \chapter{#2}%
  \zlabel{chap:#1}%
}
\newcommand*{\ChapterStop}{%
  \cleardoublepage
  \zref@wrapper@immediate{%
    \zref@labelbyprops{chapend:\current@chapid}{abspage}%
  }%
}
%    \end{macrocode}
%    \cs{ChapterPages} calculates and returns the number of pages of the
%    referenced chapter.
%    \begin{macrocode}
\newcommand*{\ChapterPages}[1]{%
  \zrefused{chap:#1}%
  \zrefused{chapend:#1}%
  \number\numexpr
    \zref@extract{chapend:#1}{abspage}%
    -\zref@extract{chap:#1}{abspage}%
    +1\relax
}
\makeatother
%    \end{macrocode}
%    \begin{macrocode}
\begin{document}
%    \end{macrocode}
%    As exception we use \cs{makeatletter} here, because
%    this is just an example file that also should show some of
%    programmer's interface.
%    \begin{macrocode}
\makeatletter

\frontmatter
\zlabel{documentstart}

\begin{itemize}
\item
  The frontmatter part has
  \number\numexpr\zref@extract{chap:first}{abspage}-1\relax
  ~pages.
\item
  Chapter \zref{chap:first} has \ChapterPages{first} page(s).
\item
  Section \zref{hello} is on the
  \ifcase\numexpr
    \zref@extractdefault{hello}{page}{0}%
    -\zref@extractdefault{chap:first}{page}{0}%
    +1\relax
    ??\or first\or second\or third\or forth\fi
  ~page inside its chapter.
\item
  The document has
  \zref[abspage]{LastPage} pages.
  This number is \ifodd\ztotpages odd\else even\fi.
\item
  The last page is labeled with \zpageref{LastPage}.
\item
  The title of chapter \zref{chap:next} %
  is ``\zref[chaptitle]{chap:next}''.
\end{itemize}

\tableofcontents

\mainmatter
\ChapterStart{first}{First chapter}

%    \end{macrocode}
%    The user level commands should protect babel shorthands
%    where possible. On the other side, expandable extracting
%    macros are useful in calculations, see above the
%    examples with \cs{numexpr}.
%    \begin{macrocode}
\section{Test}
\zlabel{a"o}
Section \zref{a"o} on page
\zref@wrapper@babel\zref@extract{a"o}{page}.

Text.
\newpage

\section{Hello World}
\zlabel{hello}

\ChapterStop

\ChapterStart{next}{Next chapter with \emph{umlauts}: "a"o"u"s}

%    \end{macrocode}
%
%    Here an example follows that makes use of \pdfTeX's ``savepos''
%    feature. The position on the page is not known before the
%    page is constructed and shipped out. Therefore the position
%    ist stored in references and are available for calculations
%    in the next \hologo{LaTeX} compile run.
%    \begin{macrocode}
The width of the first column is
  \the\dimexpr \zposx{secondcol}sp - \zposx{firstcol}sp\relax,\\
the height difference of the two baselines is
  \the\dimexpr \zposy{firstcol}sp - \zposy{secondline}sp\relax:\\
\begin{tabular}{ll}
  \zsavepos{firstcol}Hello&\zsavepos{secondcol}World\\
  \zsavepos{secondline}Second line&foobar\\
\end{tabular}

%    \end{macrocode}
%    With \cs{zrefused} \hologo{LaTeX} is notified, if the references
%    are not yet available and \hologo{LaTeX} can generate the rerun
%    hint.
%    \begin{macrocode}
\zrefused{firstcol}
\zrefused{secondcol}
\zrefused{secondline}

\ChapterStop
%    \end{macrocode}
%    Test for module \cs{dotfill}.
%    \begin{macrocode}
\ChapterStart{dotfill}{Test for dotfill feature}
\newcommand*{\dftest}[1]{%
  #1&
  [\makebox[{#1}]{\dotfill}]&
  [\makebox[{#1}]{\zdotfill}]\\
}
\begin{tabular}{rll}
& [\verb|\dotfill|] & [\verb|\zdotfill|]\\
\dftest{0.43em}
\dftest{0.44em}
\dftest{0.45em}
\dftest{0.87em}
\dftest{0.88em}
\dftest{0.89em}
\dftest{1.31em}
\dftest{1.32em}
\dftest{1.33em}
\end{tabular}
\ChapterStop
\end{document}
%    \end{macrocode}
%
%    \begin{macrocode}
%</example>
%    \end{macrocode}
%
% \StopEventually{
% }
%
% \section{Implementation}
%
% \subsection{Package \xpackage{zref}}
%
% \subsubsection{Identification}
%
%    \begin{macrocode}
%<*package>
\NeedsTeXFormat{LaTeX2e}
\ProvidesPackage{zref}
  [2020-07-03 v2.32 A new reference scheme for LaTeX (HO)]%
%    \end{macrocode}
%
% \subsubsection{Load basic module}
%
%    \begin{macrocode}
\RequirePackage{zref-base}[2019/11/29]
%    \end{macrocode}
%    Abort package loading if \xpackage{zref-base} could not be
%    loaded successfully.
%    \begin{macrocode}
\@ifundefined{ZREF@base@ok}{\endinput}{}
%    \end{macrocode}
%
% \subsubsection{Process options}
%
%    Known modules are loaded and the release date is checked.
%    \begin{macrocode}
\def\ZREF@temp#1{%
  \DeclareOption{#1}{%
    \AtEndOfPackage{%
      \RequirePackage{zref-#1}[2019/11/29]%
    }%
  }%
}
\ZREF@temp{abspage}
\ZREF@temp{counter}
\ZREF@temp{dotfill}
\ZREF@temp{hyperref}
\ZREF@temp{lastpage}
\ZREF@temp{marks}
\ZREF@temp{nextpage}
\ZREF@temp{pageattr}
\ZREF@temp{pagelayout}
\ZREF@temp{perpage}
\ZREF@temp{runs}
\ZREF@temp{savepos}
\ZREF@temp{thepage}
\ZREF@temp{titleref}
\ZREF@temp{totpages}
\ZREF@temp{user}
\ZREF@temp{xr}
%    \end{macrocode}
%    \begin{macrocode}
\ProcessOptions\relax
%</package>
%    \end{macrocode}
%
%
% \subsection{Module \xmodule{base}}
%
% \subsubsection{Prefixes}
%
%    This package uses the following prefixes for macro names:
%    \begin{description}
%    \item[\cs{zref@}:] Macros of the programmer's interface.
%    \item[\cs{ZREF@}:] Internal macros.
%    \item[\cs{Z@L@}\metacs{listname}:]
%      The properties of the list \meta{listname}.
%    \item[\cs{Z@D@}\metacs{propname}:]
%      The default value for property \meta{propname}.
%    \item[\cs{Z@E@}\metacs{propname}:]
%      Extract function for property \meta{propname}.
%    \item[\cs{Z@X@}\metacs{propname}:]
%      Information whether a property value for property
%      \meta{propname} is expanded
%      immediately or at shipout time.
%    \item[\cs{Z@C@}\metacs{propname}:]
%      Current value of the property \meta{propname}.
%    \item[\cs{Z@R@}\metacs{labelname}:]
%      Data for reference \meta{labelname}.
%    \item[\cs{ZREF@org@}:] Original versions of patched commands.
%    \item[\cs{z}:] For macros in user land, defined if
%      module \xmodule{user} is set.
%    \end{description}
%    The following family names are used for keys defined
%    according to the \xpackage{keyval} package:
%    \begin{description}
%    \item[\texttt{ZREF@TR}:] Setup for module \xmodule{titleref}.
%    \end{description}
%
% \subsubsection{Identification}
%
%    \begin{macrocode}
%<*base>
\NeedsTeXFormat{LaTeX2e}
\ProvidesPackage{zref-base}%
  [2020-07-03 v2.32 Module base for zref (HO)]%
%    \end{macrocode}
%
% \subsubsection{Utilities}
%
%    \begin{macrocode}
\providecommand\IfFormatAtLeastTF{\@ifl@t@r\fmtversion}
\RequirePackage{ltxcmds}[2010/12/02]
\RequirePackage{infwarerr}[2010/04/08]
\RequirePackage{kvsetkeys}[2010/03/01]
\RequirePackage{kvdefinekeys}[2010/03/01]
\RequirePackage{pdftexcmds}[2010/04/01]
%    \end{macrocode}
%
%    \begin{macro}{\ZREF@name}
%    Several times the package name is used, thus we store it
%    in \cmd{\ZREF@name}.
%    \begin{macrocode}
\def\ZREF@name{zref}
%    \end{macrocode}
%    \end{macro}
%
%    \begin{macrocode}
\ltx@IfUndefined{protected}{%
  \RequirePackage{makerobust}[2006/03/18]%
%    \end{macrocode}
%    \begin{macro}{\ZREF@Robust}
%    \begin{macrocode}
  \def\ZREF@Robust#1#2{%
    \def\ZREF@temp{\MakeRobustcommand#2}%
    \afterassignment\ZREF@temp
    #1#2%
  }%
%    \end{macrocode}
%    \end{macro}
%    \begin{macrocode}
}{%
%    \end{macrocode}
%    \begin{macro}{\ZREF@Robust}
%    \begin{macrocode}
  \def\ZREF@Robust#1{%
    \protected#1%
  }%
%    \end{macrocode}
%    \end{macro}
%    \begin{macrocode}
}
%    \end{macrocode}
%    \begin{macro}{\ZREF@IfDefinable}
%    \begin{macrocode}
\def\ZREF@IfDefinable#1#2#3{%
  \@ifdefinable{#1}{%
    \ZREF@Robust{#2}#1#3%
  }%
}
%    \end{macrocode}
%    \end{macro}
%
%    \begin{macro}{\ZREF@UpdatePdfTeX}
%    \cs{ZREF@UpdatePdfTeX} is used as help message text in
%    error messages.
%    \begin{macrocode}
\def\ZREF@UpdatePdfTeX{Update pdfTeX.}
%    \end{macrocode}
%    \end{macro}
%
%    \begin{macro}{\ifZREF@found}
%    The following switch is usded in list processing.
%    \begin{macrocode}
\newif\ifZREF@found
%    \end{macrocode}
%    \end{macro}
%
%    \begin{macro}{\ZREF@patch}
%    Macro \cs{ZREF@patch} first checks the existence of
%    the command and safes it.
%    \begin{macrocode}
\def\ZREF@patch#1{%
  \ltx@IfUndefined{#1}{%
    \ltx@gobble
  }{%
    \expandafter\let\csname ZREF@org@#1\expandafter\endcsname
    \csname #1\endcsname
    \ltx@firstofone
  }%
}
%    \end{macrocode}
%    \end{macro}
%
% \subsubsection{Check for \hologo{eTeX}}
%
%    The use of \hologo{eTeX} should be standard nowadays for \hologo{LaTeX}.
%    We test for \hologo{eTeX} in order to use its features later.
%    \begin{macrocode}
\ltx@IfUndefined{eTeXversion}{%
  \PackageError\ZREF@name{%
    Missing support for eTeX; package is abandoned%
  }{%
    Use a TeX compiler that support eTeX and enable eTeX %
    in the format.%
  }%
  \endinput
}{}%
%    \end{macrocode}
%
%    \begin{macrocode}
\RequirePackage{etexcmds}[2007/09/09]
\ifetex@unexpanded
\else
  \PackageError\ZREF@name{%
    Missing e-TeX's \string\unexpanded.\MessageBreak
    Add \string\RequirePackage\string{etexcmds\string} before %
    \string\documentclass%
  }{%
    Probably you are using some package (e.g. ConTeXt) that %
    redefines \string\unexpanded%
  }%
  \expandafter\endinput
\fi
%    \end{macrocode}
%
% \subsubsection{Auxiliary file stuff}
%
%    We are using some commands in the \xfile{.aux} files.
%    However sometimes these auxiliary files are interpreted
%    by \hologo{LaTeX} processes that haven't loaded this package
%    (e.g. package \xpackage{xr}).
%    Therefore we provide dummy definitions.
%    \begin{macrocode}
\RequirePackage{auxhook}
\AddLineBeginAux{%
  \string\providecommand\string\zref@newlabel[2]{}%
}
%    \end{macrocode}
%    \begin{macro}{\ZREF@RefPrefix}
%    \begin{macrocode}
\def\ZREF@RefPrefix{Z@R}
%    \end{macrocode}
%    \end{macro}
%    \begin{macro}{\zref@newlabel}
%    For the implementation of \cs{zref@newlabel} we call
%    the same internal macro \cs{@newl@bel} that is used
%    in \cs{newlabel}. Thus we have for free:
%    \begin{itemize}
%    \item \cs{Z@R@}\metacs{labelname} is defined.
%    \item \hologo{LaTeX}'s check for multiple references.
%    \item \hologo{LaTeX}'s check for changed references.
%    \end{itemize}
%    \begin{macrocode}
\ZREF@Robust\edef\zref@newlabel{%
  \noexpand\@newl@bel{\ZREF@RefPrefix}%
}
%    \end{macrocode}
%    \end{macro}
%
% \subsubsection{Property lists}
%
%    \begin{macro}{\zref@newlist}
%    Property lists are stored as list of property names
%    enclosed in curly braces. \cs{zref@newlist}
%    creates a new list as empty list. Assignments to
%    property lists are global.
%    \begin{macrocode}
\ZREF@Robust\def\zref@newlist#1{%
  \zref@iflistundefined{#1}{%
    \@ifdefinable{Z@L@#1}{%
     \global\expandafter\let\csname Z@L@#1\endcsname\ltx@empty
     \PackageInfo\ZREF@name{New property list: #1}%
    }%
  }{%
    \PackageError\ZREF@name{%
      Property list `#1' already exists%
    }\@ehc
  }%
}
%    \end{macrocode}
%    \end{macro}
%
%    \begin{macro}{\zref@iflistundefined}
%    \cs{zref@iflistundefined} checks the existence of the
%    property list \verb|#1|. If the property list is present,
%    then \verb|#2| is executed and \verb|#3| otherwise.
%    \begin{macrocode}
\def\zref@iflistundefined#1{%
  \ltx@ifundefined{Z@L@#1}%
}
%    \end{macrocode}
%    \end{macro}
%    \begin{macro}{\zref@listexists}
%    \cs{zref@listexists} only executes \verb|#2| if the
%    property list \verb|#1| exists and raises an error
%    message otherwise.
%    \begin{macrocode}
\ZREF@Robust\def\zref@listexists#1{%
  \zref@iflistundefined{#1}{%
    \PackageError\ZREF@name{%
      Property list `#1' does not exist%
    }\@ehc
  }%
}
%    \end{macrocode}
%    \end{macro}
%
%    \begin{macro}{\zref@iflistcontainsprop}
%    \cs{zref@iflistcontainsprop} checks, whether a property \verb|#2| is
%    already present in a property list \verb|#1|.
%    \begin{macrocode}
\ZREF@Robust\def\zref@iflistcontainsprop#1#2{%
  \zref@iflistundefined{#1}{%
    \ltx@secondoftwo
  }{%
    \begingroup\expandafter\endgroup
    \expandafter\in@
    \csname#2\expandafter\expandafter\expandafter\endcsname
    \expandafter\expandafter\expandafter{\csname Z@L@#1\endcsname}%
    \csname ltx@\ifin@ first\else second\fi oftwo\endcsname
  }%
}
%    \end{macrocode}
%    \end{macro}
%
%    \begin{macro}{\zref@listforloop}
%    \begin{macrocode}
\def\zref@listforloop#1#2{%
  \zref@listexists{#1}{%
    \expandafter\expandafter\expandafter\@tfor
    \expandafter\expandafter\expandafter\zref@prop
    \expandafter\expandafter\expandafter:%
    \expandafter\expandafter\expandafter=%
    \csname Z@L@#1\endcsname
    \do{%
      \begingroup
        \escapechar=-1 %
        \edef\x{\endgroup
          \def\noexpand\zref@prop{%
            \expandafter\string\zref@prop
          }%
        }%
      \x
      #2\zref@prop
    }%
  }%
}
%    \end{macrocode}
%    \end{macro}
%
%    \begin{macro}{\zref@addprops}
%    \cs{zref@addprop} adds the properties \verb|#2| to the property
%    list \verb|#1|, if the property is not already in the list.
%    Otherwise a warning is given.
%    \begin{macrocode}
\ZREF@Robust\def\zref@addprops#1#2{%
  \zref@listexists{#1}{%
    \comma@parse{#2}{%
      \zref@propexists\comma@entry{%
        \zref@iflistcontainsprop{#1}\comma@entry{%
          \PackageWarning\ZREF@name{%
            Property `\comma@entry' is already in list `#1'%
          }%
        }{%
          \begingroup\expandafter\endgroup
          \expandafter\g@addto@macro
          \csname Z@L@#1\expandafter\endcsname
          \expandafter{\csname\comma@entry\endcsname}%
        }%
      }%
      \ltx@gobble
    }%
  }%
}
%    \end{macrocode}
%    \end{macro}
%    \begin{macro}{\zref@addprop}
%    \cs{zref@addprop} adds the property \verb|#2| to the property
%    list \verb|#1|, if the property is not already in the list.
%    Otherwise a warning is given.
%    \begin{macrocode}
\ZREF@Robust\def\zref@addprop#1#2{%
  \zref@listexists{#1}{%
    \zref@propexists{#2}{%
      \zref@iflistcontainsprop{#1}{#2}{%
        \PackageWarning\ZREF@name{%
          Property `#2' is already in list `#1'%
        }%
      }{%
        \begingroup\expandafter\endgroup
        \expandafter\g@addto@macro
        \csname Z@L@#1\expandafter\endcsname
        \expandafter{\csname#2\endcsname}%
      }%
    }%
  }%
}
%    \end{macrocode}
%    \end{macro}
%
%    \begin{macro}{\zref@localaddprops}
%    \begin{macrocode}
\ZREF@Robust\def\zref@localaddprops#1#2{%
  \zref@listexists{#1}{%
    \comma@parse{#2}{%
      \zref@propexists\comma@entry{%
        \zref@iflistcontainsprop{#1}\comma@entry{%
          \PackageWarning\ZREF@name{%
            Property `\comma@entry' is already in list `#1'%
          }%
        }{%
          \begingroup\expandafter\endgroup
          \expandafter\ltx@LocalAppendToMacro
          \csname Z@L@#1\expandafter\endcsname
          \expandafter{\csname\comma@entry\endcsname}%
        }%
      }%
      \ltx@gobble
    }%
  }%
}
%    \end{macrocode}
%    \end{macro}
%    \begin{macro}{\zref@localaddprop}
%    \begin{macrocode}
\ZREF@Robust\def\zref@localaddprop#1#2{%
  \zref@listexists{#1}{%
    \zref@propexists{#2}{%
      \zref@iflistcontainsprop{#1}{#2}{%
        \PackageWarning\ZREF@name{%
          Property `#2' is already in list `#1'%
        }%
      }{%
        \begingroup\expandafter\endgroup
        \expandafter\ltx@LocalAppendToMacro
        \csname Z@L@#1\expandafter\endcsname
        \expandafter{\csname#2\endcsname}%
      }%
    }%
  }%
}
%    \end{macrocode}
%    \end{macro}
%
%    \begin{macrocode}
\ltx@IfUndefined{pdf@strcmp}{%
%    \end{macrocode}
%    \begin{macro}{\zref@delprop}
%    \begin{macrocode}
  \ZREF@Robust\def\zref@delprop{%
    \ZREF@delprop\gdef
  }%
%    \end{macrocode}
%    \end{macro}
%    \begin{macro}{\zref@localdelprop}
%    \begin{macrocode}
  \ZREF@Robust\def\zref@localdelprop{%
    \ZREF@delprop\def
  }%
%    \end{macrocode}
%    \end{macro}
%    \begin{macro}{\ZREF@delprop}
%    \begin{macrocode}
  \def\ZREF@delprop#1#2#3{%
    \zref@listexists{#2}{%
      \begingroup
        \escapechar=-1 %
        \def\ZREF@param{#3}%
        \@onelevel@sanitize\ZREF@param
        \toks@{}%
        \expandafter\expandafter\expandafter\ZREF@@delprop
        \csname Z@L@#2\endcsname!%
      \expandafter\endgroup
      \expandafter#1\csname Z@L@#2\expandafter\endcsname
      \expandafter{%
        \the\toks@
      }%
    }%
  }%
%    \end{macrocode}
%    \end{macro}
%    \begin{macro}{\ZREF@@delprop}
%    \begin{macrocode}
  \def\ZREF@@delprop#1{%
    \expandafter\ZREF@@@delprop\expandafter{\string#1}#1%
  }%
%    \end{macrocode}
%    \end{macro}
%    \begin{macro}{\ZREF@@@delprop}
%    \begin{macrocode}
  \def\ZREF@@@delprop#1#2{%
    \ifx#2!%
    \else
      \def\ZREF@temp{#1}%
      \@onelevel@sanitize\ZREF@temp
      \ifx\ZREF@param\ZREF@temp
      \else
        \toks@\expandafter{%
          \the\expandafter\toks@\csname#1\endcsname
        }%
      \fi
      \expandafter\ZREF@@delprop
    \fi
  }%
%    \end{macrocode}
%    \end{macro}
%    \begin{macrocode}
}{%
%    \end{macrocode}
%    \begin{macro}{\zref@delprop}
%    \begin{macrocode}
  \ZREF@Robust\def\zref@delprop{%
    \ZREF@delprop\xdef
  }%
%    \end{macrocode}
%    \end{macro}
%    \begin{macro}{\zref@localdelprop}
%    \begin{macrocode}
  \ZREF@Robust\def\zref@localdelprop{%
    \ZREF@delprop\edef
  }%
%    \end{macrocode}
%    \end{macro}
%    \begin{macro}{\ZREF@delprop}
%    \begin{macrocode}
  \def\ZREF@delprop#1#2#3{%
    \zref@listexists{#2}{%
      \def\ZREF@param{#3}%
      \edef\ZREF@SavedEscapechar{\the\escapechar}%
      \escapechar=-1 %
      \expandafter#1\csname Z@L@#2%
      \expandafter\expandafter\expandafter\endcsname{%
        \expandafter\expandafter\expandafter\ZREF@@delprop
        \csname Z@L@#2\endcsname!%
      }%
      \escapechar=\ZREF@SavedEscapechar\relax
    }%
  }%
%    \end{macrocode}
%    \end{macro}
%    \begin{macro}{\ZREF@@delprop}
%    Caution: |#1| might be an \cs{if} or similar token.
%    \begin{macrocode}
  \def\ZREF@@delprop#1{%
    \expandafter\ZREF@@@delprop\expandafter{\string#1}#1%
  }%
%    \end{macrocode}
%    \end{macro}
%    \begin{macro}{\ZREF@@@delprop}
%    \begin{macrocode}
  \def\ZREF@@@delprop#1#2{%
    \ifx#2!%
    \else
      \ifnum\pdf@strcmp{#1}{\ZREF@param}=\ltx@zero
      \else
        \expandafter\noexpand\csname#1\endcsname
      \fi
      \expandafter\ZREF@@delprop
    \fi
  }%
%    \end{macrocode}
%    \end{macro}
%    \begin{macrocode}
}
%    \end{macrocode}
%
% \subsubsection{Properties}
%
%    \begin{macro}{\zref@ifpropundefined}
%    \cs{zref@ifpropundefined} checks the existence of the
%    property \verb|#1|. If the property is present,
%    then \verb|#2| is executed and \verb|#3| otherwise.
%    \begin{macrocode}
\def\zref@ifpropundefined#1{%
  \ltx@ifundefined{Z@E@#1}%
}
%    \end{macrocode}
%    \end{macro}
%    \begin{macro}{\zref@propexists}
%    Some macros rely on the existence of a property.
%    \cs{zref@propexists} only executes \verb|#2| if the
%    property \verb|#1| exists and raises an error
%    message otherwise.
%    \begin{macrocode}
\ZREF@Robust\def\zref@propexists#1{%
  \zref@ifpropundefined{#1}{%
    \PackageError\ZREF@name{%
      Property `#1' does not exist%
    }\@ehc
  }%
}
%    \end{macrocode}
%    \end{macro}
%
%    \begin{macro}{\zref@newprop}
%    A new property is declared by \cs{zref@newprop}, the
%    property name \meta{propname} is given in \verb|#1|. The
%    property is created and configured. If the star form
%    is given, then the expansion of the property value
%    is delayed to page shipout time, when the reference
%    is written to the \xfile{.aux} file.
%    \begin{description}
%    \item[\cs{Z@D@}\metacs{propname}:]
%      Stores the default value for this property.
%    \item[\cs{Z@E@}\metacs{propname}:]
%      Extract function.
%    \item[\cs{Z@X@}\metacs{propname}:]
%      Information whether the expansion of the property value
%      is delayed to shipout time.
%    \item[\cs{Z@C@}\metacs{propname}:]
%      Current value of the property.
%    \end{description}
%    \begin{macrocode}
\ZREF@Robust\def\zref@newprop{%
  \@ifstar{%
    \let\ZREF@X\noexpand
    \ZREF@newprop
  }{%
    \let\ZREF@X\ltx@empty
    \ZREF@newprop
  }%
}
%    \end{macrocode}
%    \end{macro}
%    \begin{macro}{\ZREF@newprop}
%    \begin{macrocode}
\def\ZREF@newprop#1{%
  \edef\ZREF@P{#1}%
  \@onelevel@sanitize\ZREF@P
  \begingroup
  \ifx\ZREF@P\ZREF@par
    \@PackageError\ZREF@name{%
      Invalid property name `\ZREF@P'%
    }{%
      The property name `par' is not allowed %
      because of internal reasons.%
      \MessageBreak
      \@ehc
    }%
    \def\ZREF@@newprop[##1]##2{\endgroup}%
  \else
    \zref@ifpropundefined\ZREF@P{%
      \endgroup
      \PackageInfo\ZREF@name{%
        New property: \ZREF@P
      }%
    }{%
      \@PackageError\ZREF@name{%
        Property `\ZREF@P' already exists%
      }\@ehc
      \def\ZREF@@newprop[##1]##2{\endgroup}%
    }%
  \fi
  \@ifnextchar[\ZREF@@newprop{\ZREF@@newprop[\zref@default]}%
}
%    \end{macrocode}
%    \end{macro}
%    \begin{macro}{\ZREF@par}
%    \begin{macrocode}
\def\ZREF@par{par}
\@onelevel@sanitize\ZREF@par
%    \end{macrocode}
%    \end{macro}
%    \begin{macro}{\ZREF@@newprop}
%    \begin{macrocode}
\def\ZREF@@newprop[#1]{%
  \global\@namedef{Z@D@\ZREF@P}{#1}%
  \global\expandafter\let\csname Z@X@\ZREF@P\endcsname\ZREF@X
  \begingroup\expandafter\endgroup
  \expandafter\ZREF@@@newprop\csname\ZREF@P\endcsname
  \expandafter\gdef\csname Z@C@\ZREF@P\endcsname{}%
  \zref@setcurrent\ZREF@P
}
\def\ZREF@@@newprop#1{%
  \expandafter
  \gdef\csname Z@E@\ZREF@P\endcsname##1#1##2##3\ZREF@nil{##2}%
}
%    \end{macrocode}
%    \end{macro}
%
%    \begin{macro}{\zref@showprop}
%    \begin{macrocode}
\ZREF@Robust\def\zref@showprop#1{%
  \zref@ifpropundefined{#1}{%
    \@PackageInfoNoLine{\ZREF@name}{%
      Show property `#1': <undefined>%
    }%
  }{%
    \begingroup
      \toks@\expandafter\expandafter\expandafter{%
        \csname Z@C@#1\endcsname
      }%
      \edef\ZREF@value{\the\toks@}%
      \ltx@onelevel@sanitize\ZREF@value
      \toks@\expandafter\expandafter\expandafter{%
        \csname Z@D@#1\endcsname
      }%
      \edef\ZREF@default{\the\toks@}%
      \ltx@onelevel@sanitize\ZREF@default
      \@PackageInfoNoLine{\ZREF@name}{%
        Show property `#1':\MessageBreak
        \expandafter\ifx\csname Z@X@#1\endcsname\ltx@empty
          Immediate %
        \else
          Delayed %
        \fi
        value: [\ZREF@value]\MessageBreak
        Default: [\ZREF@default]%
      }%
    \endgroup
  }%
}
%    \end{macrocode}
%    \end{macro}
%
%    \begin{macro}{\zref@setcurrent}
%    \cs{zref@setcurrent} sets the current value for a property.
%    \begin{macrocode}
\ZREF@Robust\def\zref@setcurrent#1#2{%
  \zref@propexists{#1}{%
    \expandafter\def\csname Z@C@#1\endcsname{#2}%
  }%
}
%    \end{macrocode}
%    \end{macro}
%
%    \begin{macro}{\ZREF@getcurrent}
%    \cs{zref@getcurrent} gets the current value for a property.
%    \begin{macrocode}
\def\ZREF@getcurrent#1{%
  \romannumeral0%
  \ltx@ifundefined{Z@C@#1}{%
    \ltx@space
  }{%
    \expandafter\expandafter\expandafter\ltx@space
    \csname Z@C@#1\endcsname
  }%
}
%    \end{macrocode}
%    \end{macro}
%    \begin{macro}{\ZREF@u@getcurrent}
%    \begin{macrocode}
\def\ZREF@wu@getcurrent#1{%
  \etex@unexpanded\expandafter\expandafter\expandafter{%
    \ZREF@getcurrent{#1}%
  }%
}
%    \end{macrocode}
%    \end{macro}
%    \begin{macro}{\zref@getcurrent}
%    \begin{macrocode}
\let\zref@getcurrent\ZREF@getcurrent
%    \end{macrocode}
%    \end{macro}
%
% \subsubsection{Reference generation}
%
%    \begin{macro}{\zref@label}
%    Label macro that uses the main property list.
%    \begin{macrocode}
\ZREF@Robust\def\zref@label#1{%
  \zref@labelbylist{#1}\ZREF@mainlist
}
%    \end{macrocode}
%    \end{macro}
%    \begin{macro}{\zref@labelbylist}
%    Label macro that stores the properties, specified in the
%    property list \verb|#2|.
%    \begin{macrocode}
\ZREF@Robust\def\zref@labelbylist#1#2{%
  \@bsphack
    \zref@listexists{#2}{%
      \expandafter\expandafter\expandafter\ZREF@label
      \expandafter\expandafter\expandafter{%
        \csname Z@L@#2\endcsname
      }{#1}%
    }%
  \@esphack
}
%    \end{macrocode}
%    \end{macro}
%    \begin{macro}{\zref@labelbyprops}
%    The properties are directly specified in a comma separated list.
%    \begin{macrocode}
\ZREF@Robust\def\zref@labelbyprops#1#2{%
  \@bsphack
    \begingroup
      \toks@{}%
      \comma@parse{#2}{%
        \zref@ifpropundefined\comma@entry{%
          \PackageWarning\ZREF@name{%
            Property `\comma@entry' is not known%
          }%
        }{%
          \toks@\expandafter{%
            \the\expandafter\toks@\csname\comma@entry\endcsname
          }%
        }%
        \ltx@gobble
      }%
    \expandafter\endgroup
    \expandafter\ZREF@label\expandafter{\the\toks@}{#1}%
  \@esphack
}
%    \end{macrocode}
%    \end{macro}
%    \begin{macro}{\zref@labelbykv}
%    \begin{macrocode}
\ZREF@Robust\def\zref@labelbykv#1#2{%
  \@bsphack
    \begingroup
      \let\Z@L@ZREF@temp\ltx@empty
      \kvsetkeys{ZREF@LABEL}{#1}%
      \ifZREF@immediate
        \expandafter\zref@wrapper@immediate\expandafter{%
          \expandafter\ZREF@label\expandafter{\Z@L@ZREF@temp}{#2}%
        }%
      \else
        \expandafter\ZREF@label\expandafter{\Z@L@ZREF@temp}{#2}%
      \fi
    \endgroup
  \@esphack
}
%    \end{macrocode}
%    \end{macro}
%    \begin{macrocode}
\kv@define@key{ZREF@LABEL}{prop}{%
  \edef\ZREF@param{#1}%
  \zref@propexists\ZREF@param{%
    \zref@iflistcontainsprop{ZREF@temp}\ZREF@param{}{%
      \begingroup\expandafter\endgroup
      \expandafter\ltx@LocalAppendToMacro
      \expandafter\Z@L@ZREF@temp
      \expandafter{\csname\ZREF@param\endcsname}%
    }%
  }%
}
\kv@define@key{ZREF@LABEL}{list}{%
  \zref@listforloop{#1}{%
    \zref@iflistcontainsprop{ZREF@temp}\zref@prop{}{%
      \begingroup\expandafter\endgroup
      \expandafter\ltx@LocalAppendToMacro
      \expandafter\Z@L@ZREF@temp
      \expandafter{\csname\zref@prop\endcsname}%
    }%
    \ltx@gobble
  }%
}
\kv@define@key{ZREF@LABEL}{delprop}{%
  \zref@propexists{#1}{%
    \zref@localdelprop{ZREF@temp}{#1}%
  }%
}
\kv@define@key{ZREF@LABEL}{immediate}[true]{%
  \edef\ZREF@param{#1}%
  \ifx\ZREF@param\ZREF@true
    \ZREF@immediatetrue
  \else
    \ifx\ZREF@param\ZREF@false
      \ZREF@immediatefalse
    \else
      \PackageWarning\ZREF@name{%
        Option `immediate' expects `true' or `false'.\MessageBreak
        Ignoring invalid value `\ZREF@param'%
      }%
    \fi
  \fi
}
%    \end{macrocode}
%    \begin{macro}{\ZREF@false}
%    \begin{macrocode}
\def\ZREF@false{false}
%    \end{macrocode}
%    \end{macro}
%    \begin{macro}{\ZREF@true}
%    \begin{macrocode}
\def\ZREF@true{true}
%    \end{macrocode}
%    \end{macro}
%
%    \begin{macrocode}
\kv@define@key{ZREF@LABEL}{values}[]{%
  \kv@parse{#1}{%
    \ifx\kv@value\relax
      \@PackageWarning\ZREF@name{%
        Missing value for property `\kv@key'%
      }%
      \expandafter\ltx@gobbletwo
    \else
      \expandafter\zref@setcurrent
    \fi
  }%
}
%    \end{macrocode}
%
%    \begin{macro}{\ifZREF@immediate}
%    The switch \cs{ifZREF@immediate} tells us, whether the
%    label should be written immediately or at page shipout time.
%    \cs{ZREF@label} need to be notified about this, because
%    it must disable the deferred execution of property values,
%    if the label is written immediately.
%    \begin{macrocode}
\newif\ifZREF@immediate
%    \end{macrocode}
%    \end{macro}
%    \begin{macro}{\zref@wrapper@immediate}
%    The argument of \cs{zref@wrapper@immediate} is executed inside
%    a group where \cs{write} is redefined by adding
%    \cs{immediate} before its execution. Also \cs{ZREF@label}
%    is notified via the switch \cs{ifZREF@immediate}.
%    \begin{macrocode}
\ZREF@Robust{\long\def}\zref@wrapper@immediate#1{%
  \begingroup
    \ZREF@immediatetrue
    \let\ZREF@org@write\write
    \def\write{\immediate\ZREF@org@write}%
    #1%
  \endgroup
}
%    \end{macrocode}
%    \end{macro}
%
%    \begin{macro}{\ZREF@label}
%    \cs{ZREF@label} writes the data in the \xfile{.aux} file.
%    \verb|#1| contains the list of valid properties,
%    \verb|#2| the name of the reference. In case of immediate
%    writing, the deferred execution of property values is
%    disabled. Also \thepage is made expandable in this case.
%    \begin{macrocode}
\def\ZREF@label#1#2{%
  \if@filesw
    \begingroup
      \ifZREF@immediate
        \let\ZREF@org@thepage\thepage
      \fi
      \protected@write\@auxout{%
        \ifZREF@immediate
          \let\thepage\ZREF@org@thepage
        \fi
        \let\ZREF@temp\ltx@empty
        \@tfor\ZREF@P:=#1\do{%
          \begingroup
            \escapechar=-1 %
            \edef\x{\endgroup
              \def\noexpand\ZREF@P{%
                \expandafter\string\ZREF@P
              }%
            }%
          \x
          \expandafter\ifx
              \csname
                \ifZREF@immediate
                  relax%
                \else
                  Z@X@\ZREF@P%
                \fi
              \endcsname
              \noexpand
            \expandafter\let\csname Z@C@\ZREF@P\endcsname\relax
          \fi
          \toks@\expandafter{\ZREF@temp}%
          \edef\ZREF@temp{%
            \the\toks@
            \ltx@backslashchar\ZREF@P{%
              \expandafter\noexpand\csname Z@C@\ZREF@P\endcsname
            }%
          }%
        }%
      }{%
        \string\zref@newlabel{#2}{\ZREF@temp}%
      }%
    \endgroup
  \fi
}
\def\ZREF@addtoks#1{%
  \toks@\expandafter\expandafter\expandafter{%
    \expandafter\the\expandafter\toks@#1%
  }%
}
%    \end{macrocode}
%    \end{macro}
%
% \subsubsection{Reference querying and extracting}
%
%    Design goal for the extracting macros is that the extraction
%    process is full expandable. Thus these macros can be used
%    in expandable contexts. But there are problems that cannot
%    be solved by full expandable macros:
%    \begin{itemize}
%    \item In standard \hologo{LaTeX} undefined references sets a flag
%      and generate a warning. Both actions are not expandable.
%    \item Babel's support for its shorthand uses commands that
%      use non-expandable assignments. However currently there
%      is hope, that primitives are added to \pdfTeX\ that
%      allows the detection of contexts. Then the shorthand
%      can detect, if they are executed inside \cs{csname}
%      and protect themselves automatically.
%    \end{itemize}
%    \begin{macro}{\zref@ifrefundefined}
%    If a reference \verb|#1| is undefined, then macro
%    \cs{zref@ifrefundefined} calls \verb|#2| and \verb|#3| otherwise.
%    \begin{macrocode}
\def\zref@ifrefundefined#1{%
  \ltx@ifundefined{Z@R@#1}%
}
%    \end{macrocode}
%    \end{macro}
%
%    \begin{macro}{\zifrefundefined}
%    If a reference \verb|#1| is undefined, then macro
%    \cs{zref@ifrefundefined} calls \verb|#2| and \verb|#3| otherwise.
%    Also the reference is marked used.
%    \begin{macrocode}
\ZREF@IfDefinable\zifrefundefined\def{%
  #1{%
    \zref@wrapper@babel\ZREF@ifrefundefined{#1}%
  }%
}
%    \end{macrocode}
%    \end{macro}
%    \begin{macro}{\ZREF@ifrefundefined}
%    \begin{macrocode}
\def\ZREF@ifrefundefined#1{%
  \zref@refused{#1}%
  \zref@ifrefundefined{#1}%
}
%    \end{macrocode}
%    \end{macro}
%
%    \begin{macro}{\zref@refused}
%    The problem with undefined references is addressed by
%    the macro \cs{zref@refused}. This can be used outside
%    the expandable context. In case of an undefined reference
%    the flag is set to notify \hologo{LaTeX} and a warning is given.
%    \begin{macrocode}
\ZREF@Robust\def\zref@refused#1{%
  \zref@wrapper@babel\ZREF@refused{#1}%
}
%    \end{macrocode}
%    \end{macro}
%    \begin{macro}{\ZREF@refused}
%    \begin{macrocode}
\def\ZREF@refused#1{%
  \zref@ifrefundefined{#1}{%
    \protect\G@refundefinedtrue
    \@latex@warning{%
      Reference `#1' on page \thepage \space undefined%
    }%
  }{}%
}
%    \end{macrocode}
%    \end{macro}
%
%    \begin{macro}{\zref@ifrefcontainsprop}
%    \cs{zref@ifrefcontainsprop} looks, if the reference \verb|#1|
%    has the property \verb|#2| and calls then \verb|#3| and
%    \verb|#4| otherwise.
%    \begin{macrocode}
\def\zref@ifrefcontainsprop#1#2{%
  \zref@ifrefundefined{#1}{%
    \ltx@secondoftwo
  }{%
    \expandafter\ZREF@ifrefcontainsprop
    \csname Z@E@#2\expandafter\endcsname
    \csname#2\expandafter\expandafter\expandafter\endcsname
    \expandafter\expandafter\expandafter{%
      \csname Z@R@#1\endcsname
    }%
  }%
}
\def\ZREF@ifrefcontainsprop#1#2#3{%
  \expandafter\ifx\expandafter\ZREF@novalue
  #1#3#2\ZREF@novalue\ZREF@nil\ltx@empty
    \expandafter\ltx@secondoftwo
  \else
    \expandafter\ltx@firstoftwo
  \fi
}
\def\ZREF@novalue{\ZREF@NOVALUE}
%    \end{macrocode}
%    \end{macro}
%
%    \begin{macro}{\zref@extract}
%    \cs{zref@extract} is an abbreviation for the case
%    that the default of the property is used as default
%    value.
%    \begin{macrocode}
\def\ZREF@extract#1#2{%
  \romannumeral0%
  \ltx@ifundefined{Z@D@#2}{%
    \expandafter\ltx@space\zref@default
  }{%
    \expandafter\expandafter\expandafter\ZREF@@extract
    \expandafter\expandafter\expandafter{%
      \csname Z@D@#2\endcsname
    }{#1}{#2}%
  }%
}
%    \end{macrocode}
%    \end{macro}
%    \begin{macro}{\ZREF@@extract}
%    \begin{macrocode}
\def\ZREF@@extract#1#2#3{%
  \expandafter\expandafter\expandafter\ltx@space
  \zref@extractdefault{#2}{#3}{#1}%
}
%    \end{macrocode}
%    \end{macro}
%    \begin{macro}{\ZREF@wu@extract}
%    \begin{macrocode}
\def\ZREF@wu@extract#1#2{%
  \etex@unexpanded\expandafter\expandafter\expandafter{%
    \ZREF@extract{#1}{#2}%
  }%
}
%    \end{macrocode}
%    \end{macro}
%    \begin{macro}{\zref@extract}
%    \begin{macrocode}
\let\zref@extract\ZREF@extract
%    \end{macrocode}
%    \end{macro}
%
%    \begin{macro}{\ZREF@extractdefault}
%    The basic extracting macro is \cs{zref@extractdefault} with
%    the reference name in \verb|#1|, the property in \verb|#2| and
%    the default value in \verb|#3| in case for problems.
%    \begin{macrocode}
\def\ZREF@extractdefault#1#2#3{%
  \romannumeral0%
  \zref@ifrefundefined{#1}\ltx@firstoftwo{%
    \zref@ifpropundefined{#2}\ltx@firstoftwo\ltx@secondoftwo
  }{%
    \ltx@space
    #3%
  }{%
    \expandafter\expandafter\expandafter\ltx@space
    \csname Z@E@#2\expandafter\expandafter\expandafter\endcsname
    \csname Z@R@#1\expandafter\endcsname
    \csname#2\endcsname{#3}\ZREF@nil
  }%
}
%    \end{macrocode}
%    \end{macro}
%    \begin{macro}{\ZREF@wu@extractdefault}
%    \begin{macrocode}
\def\ZREF@wu@extractdefault#1#2#3{%
  \etex@unexpanded\expandafter\expandafter\expandafter{%
    \ZREF@extractdefault{#1}{#2}{#3}%
  }%
}
%    \end{macrocode}
%    \end{macro}
%    \begin{macro}{\zref@extractdefault}
%    \begin{macrocode}
\let\zref@extractdefault\ZREF@extractdefault
%    \end{macrocode}
%    \end{macro}
%
%    \begin{macro}{\zref@def@extract}
%    \begin{macrocode}
\ZREF@Robust\def\zref@def@extract#1{%
  \zref@wrapper@babel{\ZREF@def@extract{#1}}%
}
%    \end{macrocode}
%    \end{macro}
%    \begin{macro}{\ZREF@def@extract}
%    \begin{macrocode}
\def\ZREF@def@extract#1#2#3{%
  \zref@refused{#2}%
  \expandafter\expandafter\expandafter\def
  \expandafter\expandafter\expandafter#1%
  \expandafter\expandafter\expandafter{%
    \zref@extract{#2}{#3}%
  }%
}
%    \end{macrocode}
%    \end{macro}
%    \begin{macro}{\zref@def@extractdefault}
%    \begin{macrocode}
\ZREF@Robust\def\zref@def@extractdefault#1{%
  \zref@wrapper@babel{\ZREF@def@extractdefault{#1}}%
}
%    \end{macrocode}
%    \end{macro}
%    \begin{macro}{\ZREF@def@extractdefault}
%    \begin{macrocode}
\def\ZREF@def@extractdefault#1#2#3#4{%
  \zref@refused{#2}%
  \expandafter\expandafter\expandafter\def
  \expandafter\expandafter\expandafter#1%
  \expandafter\expandafter\expandafter{%
    \zref@extractdefault{#2}{#3}{#4}%
  }%
}
%    \end{macrocode}
%    \end{macro}
%
%    \begin{macro}{\ZREF@wrapper@unexpanded}
%    \begin{macrocode}
\ZREF@Robust{\long\def}\ZREF@wrapper@unexpanded#1{%
  \let\zref@wrapper@unexpanded\ltx@firstofone
  \let\zref@getcurrent\ZREF@wu@getcurrent
  \let\zref@extractdefault\ZREF@wu@extractdefault
  \let\zref@extract\ZREF@wu@extract
  #1%
  \let\zref@wrapper@unexpanded\ZREF@wrapper@unexpanded
  \let\zref@getcurrent\ZREF@getcurrent
  \let\zref@extractdefault\ZREF@extractdefault
  \let\zref@extract\ZREF@extract
}
%    \end{macrocode}
%    \end{macro}
%    \begin{macro}{\zref@wrapper@unexpanded}
%    \begin{macrocode}
\ltx@IfUndefined{etex@unexpanded}{%
  \let\zref@wrapper@unexpanded\ltx@firstofone
}{%
  \let\zref@wrapper@unexpanded\ZREF@wrapper@unexpanded
}
%    \end{macrocode}
%    \end{macro}
%
% \subsubsection{Compatibility with \xpackage{babel}}
%
%    \begin{macro}{\zref@wrapper@babel}
%    \begin{macrocode}
\ZREF@Robust{\long\def}\zref@wrapper@babel#1#2{%
  \ifcsname if@safe@actives\endcsname
    \expandafter\ltx@firstofone
  \else
    \expandafter\ltx@secondoftwo
  \fi
  {%
    \if@safe@actives
      \expandafter\ltx@secondoftwo
    \else
      \expandafter\ltx@firstoftwo
    \fi
    {%
      \begingroup
        \csname @safe@activestrue\endcsname
        \edef\x{#2}%
      \expandafter\endgroup
      \expandafter\ZREF@wrapper@babel\expandafter{\x}{#1}%
    }%
  }{%
    #1{#2}%
  }%
}
\long\def\ZREF@wrapper@babel#1#2{%
  #2{#1}%
}
%    \end{macrocode}
%    \end{macro}
%
% \subsubsection{Unique counter support}
%
%    \begin{macro}{\zref@require@unique}
%    Generate the counter |zref@unique| if the counter does not
%    already exist.
%    \begin{macrocode}
\ZREF@Robust\def\zref@require@unique{%
  \@ifundefined{c@zref@unique}{%
    \begingroup
      \let\@addtoreset\ltx@gobbletwo
      \newcounter{zref@unique}%
    \endgroup
%    \end{macrocode}
%    \end{macro}
%
%    \begin{macro}{\thezref@unique}
%    \cs{thezref@unique} is used for automatically generated unique
%    labelnames.
%    \begin{macrocode}
    \renewcommand*{\thezref@unique}{%
      zref@\number\c@zref@unique
    }%
  }{}%
}
%    \end{macrocode}
%    \end{macro}
%
% \subsubsection{Utilities}
%
%    \begin{macro}{\ZREF@number}
%    \begin{macrocode}
\ltx@IfUndefined{numexpr}{%
  \def\ZREF@number#1{\number#1}%
}{%
  \def\ZREF@number#1{\the\numexpr(#1)\relax}%
}
%    \end{macrocode}
%    \end{macro}
%
% \subsubsection{Setup}
%
%    \begin{macro}{\zref@setdefault}
%    Standard \hologo{LaTeX} prints ``??'' in bold face if a reference
%    is not known. \cs{zref@default} holds the text that is
%    printed in case of unknown references and is used, if
%    the default was not specified during the definition
%    of the new property by \cs{ref@newprop}.
%    The global default value can be set by \cs{zref@setdefault}.
%    \begin{macrocode}
\ZREF@Robust\def\zref@setdefault#1{%
  \def\zref@default{#1}%
}
%    \end{macrocode}
%    \end{macro}
%    \begin{macro}{\zref@default}
%    Now we initialize \cs{zref@default} with the same value
%    that \hologo{LaTeX} uses for its undefined references.
%    \begin{macrocode}
\zref@setdefault{%
  \nfss@text{\reset@font\bfseries ??}%
}
%    \end{macrocode}
%    \end{macro}
%
% \paragraph{Main property list.}
%
%    \begin{macro}{\zref@setmainlist}
%    The name of the default property list is stored in
%    \cs{ZREF@mainlist} and can be set by \cs{zref@setmainlist}.
%    \begin{macrocode}
\ZREF@Robust\def\zref@setmainlist#1{%
  \def\ZREF@mainlist{#1}%
}
\zref@setmainlist{main}
%    \end{macrocode}
%    \end{macro}
%    Now we create the list.
%    \begin{macrocode}
\zref@newlist\ZREF@mainlist
%    \end{macrocode}
%
% \paragraph{Main properties.}
%
%    The two properties \texttt{default} and \texttt{page}
%    are created and added to the main property list.
%    They store the data that
%    standard \hologo{LaTeX} uses in its references created by
%    \cs{label}.
%    \begin{description}
%    \item[\texttt{default}] the apperance of the latest
%      counter that is incremented by \cs{refstepcounter}
%    \item[\texttt{page}] the apperance of the page counter
%    \end{description}
%    \begin{macrocode}
\zref@newprop{default}{\@currentlabel}
\zref@newprop*{page}{\thepage}
\zref@addprops\ZREF@mainlist{default,page}
%    \end{macrocode}
%
% \paragraph{Properties}
%
%    \begin{macro}{\ZREF@NewPropAnchor}
%    \begin{macrocode}
\def\ZREF@NewPropAnchor{%
  \zref@newprop{anchor}{%
    \ltx@ifundefined{@currentHref}{}{\@currentHref}%
  }%
  \global\let\ZREF@NewPropAnchor\relax
}
%    \end{macrocode}
%    \end{macro}
%    \begin{macro}{\zref@titleref@current}
%    Later we will redefine the section and caption macros
%    to catch the current title and remember the value in
%    \cs{zref@titleref@current}.
%    \begin{macro}{\ZREF@NewPropTitle}
%    \begin{macrocode}
\def\ZREF@NewPropTitle{%
  \gdef\zref@titleref@current{}%
  \zref@newprop{title}{\zref@titleref@current}%
  \global\let\ZREF@NewPropTitle\relax
}
%    \end{macrocode}
%    \end{macro}
%    \end{macro}
%    \begin{macro}{\ZREF@NewPropTheotype}
%    \begin{macrocode}
\def\ZREF@NewPropTheotype{%
  \zref@newprop{theotype}{}%
  \global\let\ZREF@NewPropTheotype\relax
}
%    \end{macrocode}
%    \end{macro}
%    \begin{macro}{\ZREF@NewPropPageValue}
%    \begin{macrocode}
\def\ZREF@NewPropPageValue{%
  \zref@newprop*{pagevalue}[0]{\number\c@page}%
  \global\let\ZREF@NewPropPageValue\relax
}
%    \end{macrocode}
%    \end{macro}
%
% \paragraph{Mark successful loading}
%
%    \begin{macrocode}
\let\ZREF@base@ok=Y
%</base>
%    \end{macrocode}
%
%
% \subsection{Module \xmodule{user}}
%
%    \begin{macrocode}
%<*user>
\NeedsTeXFormat{LaTeX2e}
\ProvidesPackage{zref-user}%
  [2020-07-03 v2.32 Module user for zref (HO)]%
\RequirePackage{zref-base}[2019/11/29]
\ifx\ZREF@base@ok Y%
\else
  \expandafter\endinput
\fi
%    \end{macrocode}
%
%    Module \xmodule{user} enables a small user interface.
%    All macros are prefixed by \cs{z}.
%
%    First we define the pendants to the standard \hologo{LaTeX} referencing
%    commands \cs{label}, \cs{ref}, and \cs{pageref}.
%
%    \begin{macro}{\zlabel}
%    Similar to \cs{label} the macro \cs{zlabel} writes a reference
%    entry in the \xfile{.aux} file. The main property list is used.
%    Also we add the babel patch. The \cs{label} command can also be
%    used inside section titles, but it must not go into the
%    table of contents. Therefore we have to check this situation.
%    \begin{macrocode}
\newcommand*\zlabel{%
  \ifx\label\ltx@gobble
    \expandafter\ltx@gobble
  \else
    \expandafter\zref@wrapper@babel\expandafter\zref@label
  \fi
}%
%    \end{macrocode}
%    \end{macro}
%
%    \begin{macro}{\zkvlabel}
%    \begin{macrocode}
\newcommand*{\zkvlabel}[1]{%
  \ifx\label\ltx@gobble
    \expandafter\ltx@gobblethree
  \fi
  \zref@wrapper@babel{\zref@labelbykv{#1}}%
}%
%    \end{macrocode}
%    \end{macro}
%
%    \begin{macro}{\zref}
%    Macro \cs{zref} is the corresponding macro for \cs{ref}.
%    Also it provides an optional argument in order to
%    select another property.
%    \begin{macrocode}
\newcommand*{\zref}[2][default]{% robust because of optional argument
  \zref@propexists{#1}{%
    \zref@wrapper@babel\ZREF@zref{#2}{#1}%
  }%
}%
\def\ZREF@zref#1{%
  \zref@refused{#1}%
  \zref@extract{#1}%
}%
%    \end{macrocode}
%    \end{macro}
%
%    \begin{macro}{\zpageref}
%    For macro \cs{zpageref} we just call \cs{zref}
%    with property \verb|page|.
%    \begin{macrocode}
\ZREF@IfDefinable\zpageref\def{%
  {\zref[page]}%
}
%    \end{macrocode}
%    \end{macro}
%
%    \begin{macro}{\zrefused}
%    For the following expandible user macros \cs{zrefused}
%    should be used to notify \hologo{LaTeX} in case of undefined
%    references.
%    \begin{macrocode}
\ZREF@IfDefinable\zrefused\def{%
  {\zref@refused}%
}
%    \end{macrocode}
%    \end{macro}
%
%    \begin{macrocode}
%</user>
%    \end{macrocode}
%
%
% \subsection{Module \xmodule{abspage}}
%
%    \begin{macrocode}
%<*abspage>
\NeedsTeXFormat{LaTeX2e}
\ProvidesPackage{zref-abspage}%
  [2020-07-03 v2.32 Module abspage for zref (HO)]%
\RequirePackage{zref-base}[2019/11/29]
\ifx\ZREF@base@ok Y%
\else
  \expandafter\endinput
\fi
%    \end{macrocode}
%
%    Module \xmodule{abspage} adds a new property \verb|abspage|
%    to the \verb|main| property list for absolute page numbers.
%    These are recorded by the help of package \xpackage{atbegshi}.
%    \begin{macrocode}
\RequirePackage{atbegshi}[2011/10/05]%
%    \end{macrocode}
%    The counter \verb|abspage| must not go in the clear list
%    of \verb|@ckpt| that is used to set counters in
%    \xfile{.aux} files of included \hologo{TeX} files.
%    \begin{macrocode}
\begingroup
  \let\@addtoreset\ltx@gobbletwo
  \newcounter{abspage}%
\endgroup
\setcounter{abspage}{0}%
\AtBeginShipout{%
  \stepcounter{abspage}%
}%
\zref@newprop*{abspage}[0]{\the\c@abspage}%
\zref@addprop\ZREF@mainlist{abspage}%
%    \end{macrocode}
%    Note that counter \verb|abspage| shows the
%    previous page during page processing. Before
%    shipout the counter is incremented.
%    Thus the property is correctly written with
%    deferred writing. If the counter is written
%    using \cs{zref@wrapper@immediate}, then the number
%    is too small by one.
%
%    \begin{macrocode}
%</abspage>
%    \end{macrocode}
%
%
% \subsection{Module \xmodule{counter}}
%
%    \begin{macrocode}
%<*counter>
\NeedsTeXFormat{LaTeX2e}
\ProvidesPackage{zref-counter}%
  [2020-07-03 v2.32 Module counter for zref (HO)]%
\RequirePackage{zref-base}[2019/11/29]
\ifx\ZREF@base@ok Y%
\else
  \expandafter\endinput
\fi
%    \end{macrocode}
%
%    For features such as \xpackage{hyperref}'s \cs{autoref}
%    we need the name of the counter. The property
%    \verb|counter| is defined and added to the main property
%    list. Starting with \LaTeX\ 2020-10-01 the proper can use
%    \@currentcounter. In older formats
%    \cs{refstepcounter} has to be patched but this can fail in some
%    cases, see issue \#5.
%    \begin{macrocode}
\@ifl@t@r\fmtversion{2020-10-01}
 {
  \zref@newprop{counter}{\@currentcounter}
  \zref@addprop\ZREF@mainlist{counter}
 }
 {
  \zref@newprop{counter}{}
  \zref@addprop\ZREF@mainlist{counter}
  \AtBeginDocument{%
    \ZREF@patch{refstepcounter}{%
      \def\refstepcounter#1{%
        \zref@setcurrent{counter}{#1}%
        \ZREF@org@refstepcounter{#1}%
      }%
    }%
  }
 }
%    \end{macrocode}
%    \begin{macrocode}
%</counter>
%    \end{macrocode}
%
%
% \subsection{Module \xmodule{lastpage}}
%
%    \begin{macrocode}
%<*lastpage>
\NeedsTeXFormat{LaTeX2e}
\ProvidesPackage{zref-lastpage}%
  [2020-07-03 v2.32 Module lastpage for zref (HO)]%
\RequirePackage{zref-base}[2019/11/29]
\RequirePackage{zref-abspage}[2019/11/29]
\RequirePackage{atveryend}[2009/12/07]
\ifx\ZREF@base@ok Y%
\else
  \expandafter\endinput
\fi
%    \end{macrocode}
%
%    The module \xmodule{lastpage} implements the service
%    of package \xpackage{lastpage} by setting a reference
%    \verb|LastPage| at the end of the document.
%    If module \xmodule{abspage} is given, also the absolute
%    page number is available, because the properties of
%    the main property list are used.
%    \begin{macrocode}
\zref@newlist{LastPage}
\AfterLastShipout{%
  \if@filesw
    \begingroup
      \advance\c@page\m@ne
      \toks@\expandafter\expandafter\expandafter{%
        \expandafter\Z@L@main
        \Z@L@LastPage
      }%
      \expandafter\zref@wrapper@immediate\expandafter{%
        \expandafter\ZREF@label\expandafter{\the\toks@}{LastPage}%
      }%
    \endgroup
  \fi
}
%    \end{macrocode}
%    \begin{macro}{\zref@iflastpage}
%    \begin{macrocode}
\def\zref@iflastpage#1{%
  \ifnum\zref@extractdefault{#1}{abspage}{-1}=%
        \zref@extractdefault{LastPage}{abspage}{-2} %
    \expandafter\ltx@firstoftwo
  \else
    \expandafter\ltx@secondoftwo
  \fi
}
%    \end{macrocode}
%    \end{macro}
%    \begin{macro}{\ziflastpage}
%    \begin{macrocode}
\ZREF@IfDefinable\ziflastpage\def{%
  {\zref@wrapper@babel\ZREF@iflastpage}%
}
%    \end{macrocode}
%    \end{macro}
%    \begin{macro}{ZREF@iflastpage}
%    \begin{macrocode}
\def\ZREF@iflastpage#1{%
  \zref@refused{LastPage}%
  \zref@refused{#1}%
  \zref@iflastpage{#1}%
}
%    \end{macrocode}
%    \end{macro}
%    \begin{macrocode}
%</lastpage>
%    \end{macrocode}
%
%
% \subsection{Module \xmodule{thepage}}
%
%    \begin{macrocode}
%<*thepage>
\NeedsTeXFormat{LaTeX2e}
\ProvidesPackage{zref-thepage}%
  [2020-07-03 v2.32 Module thepage for zref (HO)]%
\RequirePackage{zref-base}[2019/11/29]
\ifx\ZREF@base@ok Y%
\else
  \expandafter\endinput
\fi
%    \end{macrocode}
%
%    \begin{macrocode}
\RequirePackage{atbegshi}[2011/10/05]
\RequirePackage{zref-abspage}[2019/11/29]
%    \end{macrocode}
%    \begin{macrocode}
\zref@newlist{thepage}
\zref@addprop{thepage}{page}
\ZREF@NewPropPageValue
%    \end{macrocode}
%    \begin{macro}{\zref@thepage@atbegshi@hook}
%    \begin{macrocode}
\let\zref@thepage@atbegshi@hook\ltx@empty
%    \end{macrocode}
%    \end{macro}
%    \begin{macrocode}
\zref@addprop{thepage}{pagevalue}
\AtBeginShipout{%
  \AtBeginShipoutAddToBox{%
    \zref@thepage@atbegshi@hook
    \zref@labelbylist{thepage\the\value{abspage}}{thepage}%
  }%
}
%    \end{macrocode}
%    \begin{macro}{\zref@thepage@name}
%    \begin{macrocode}
\ltx@IfUndefined{numexpr}{%
  \def\zref@thepage@name#1{thepage\number#1}%
}{%
  \def\zref@thepage@name#1{thepage\the\numexpr#1}%
}
%    \end{macrocode}
%    \end{macro}
%    \begin{macro}{\zref@thepage}
%    \begin{macrocode}
\def\zref@thepage#1{%
  \zref@extract{\zref@thepage@name{#1}}{page}%
}%
%    \end{macrocode}
%    \end{macro}
%    \begin{macro}{\zref@thepage@refused}
%    \begin{macrocode}
\ZREF@Robust\def\zref@thepage@refused#1{%
  \zref@refused{\zref@thepage@name{#1}}%
}%
%    \end{macrocode}
%    \end{macro}
%    \begin{macro}{\zthepage}
%    \begin{macrocode}
\ZREF@IfDefinable\zthepage\def{%
  #1{%
    \zref@thepage@refused{#1}%
    \zref@thepage{#1}%
  }%
}
%    \end{macrocode}
%    \end{macro}
%    \begin{macrocode}
%</thepage>
%    \end{macrocode}
%
%
% \subsection{Module \xmodule{nextpage}}
%
%    \begin{macrocode}
%<*nextpage>
\NeedsTeXFormat{LaTeX2e}
\ProvidesPackage{zref-nextpage}%
  [2020-07-03 v2.32 Module nextpage for zref (HO)]%
\RequirePackage{zref-base}[2019/11/29]
\ifx\ZREF@base@ok Y%
\else
  \expandafter\endinput
\fi
%    \end{macrocode}
%
%    \begin{macrocode}
\RequirePackage{zref-abspage}[2019/11/29]
\RequirePackage{zref-thepage}[2019/11/29]
\RequirePackage{zref-lastpage}[2019/11/29]
\RequirePackage{uniquecounter}[2009/12/18]
%    \end{macrocode}
%    \begin{macrocode}
\UniqueCounterNew{znextpage}

\newcommand*{\znextpagesetup}{%
  \afterassignment\ZREF@np@setup@i
  \def\ZREF@np@call@unknown##1%
}
\def\ZREF@np@setup@i{%
  \afterassignment\ZREF@np@setup@ii
  \def\ZREF@np@call@nonext##1%
}
\def\ZREF@np@setup@ii{%
  \def\ZREF@np@call@next##1%
}
\def\ZREF@np@call@unknown#1{#1}
\def\ZREF@np@call@nonext#1{#1}
\def\ZREF@np@call@next#1{#1}
\ZREF@IfDefinable\znextpage\def{%
  {\UniqueCounterCall{znextpage}{\ZREF@nextpage}}%
}%
\newcommand*{\znonextpagename}{}
\newcommand*{\zunknownnextpagename}{\Z@D@page}
\def\ZREF@nextpage#1{%
  \begingroup
    \def\ZREF@refname@this{zref@np#1}%
    \zref@labelbyprops\ZREF@refname@this{abspage}%
    \chardef\ZREF@call=0 % unknown
    \ZREF@ifrefundefined\ZREF@refname@this{%
    }{%
      \edef\ZREF@pagenum@this{%
        \zref@extractdefault\ZREF@refname@this{abspage}{0}%
      }%
      \edef\ZREF@refname@next{%
        \zref@thepage@name{%
          \the\numexpr\ZREF@pagenum@this+1%
        }%
      }%
      \ifnum\ZREF@pagenum@this>0 %
        \ZREF@ifrefundefined{LastPage}{%
          \zref@ifrefundefined\ZREF@refname@next{%
          }{%
            \chardef\ZREF@call=2 % next page
          }%
        }{%
          \edef\ZREF@pagenum@last{%
            \zref@extractdefault{LastPage}{abspage}{0}%
          }%
          \ifnum\ZREF@pagenum@this<\ZREF@pagenum@last\ltx@space
            \ZREF@ifrefundefined\ZREF@refname@next{%
            }{%
              \chardef\ZREF@call=2 % next page
            }%
          \else
            \ifnum\ZREF@pagenum@this=\ZREF@pagenum@this\ltx@space
              \chardef\ZREF@call=1 % no next page
            \fi
          \fi
        }%
      \fi
    }%
    \edef\x{%
      \endgroup
      \ifcase\ZREF@call
        \noexpand\ZREF@np@call@unknown{%
          \noexpand\zunknownnextpagename
        }%
      \or
        \noexpand\ZREF@np@call@nonext{%
          \noexpand\znonextpagename
        }%
      \else
        \noexpand\ZREF@np@call@next{%
          \noexpand\zref@extract{\ZREF@refname@next}{page}%
        }%
      \fi
    }%
  \x
}
%    \end{macrocode}
%    \begin{macrocode}
%</nextpage>
%    \end{macrocode}
%
%
% \subsection{Module \xmodule{totpages}}
%
%    \begin{macrocode}
%<*totpages>
\NeedsTeXFormat{LaTeX2e}
\ProvidesPackage{zref-totpages}%
  [2020-07-03 v2.32 Module totpages for zref (HO)]%
\RequirePackage{zref-base}[2019/11/29]
\ifx\ZREF@base@ok Y%
\else
  \expandafter\endinput
\fi
%    \end{macrocode}
%
%    The absolute page number of the last page is the total
%    page number.
%    \begin{macrocode}
\RequirePackage{zref-abspage}[2019/11/29]
\RequirePackage{zref-lastpage}[2019/11/29]
%    \end{macrocode}
%
%    \begin{macro}{\ztotpages}
%    Macro \cs{ztotpages} contains the number of pages. It can be
%    used inside expandable calculations. It expands to zero if
%    the reference is not yet available.
%    \begin{macrocode}
\newcommand*{\ztotpages}{%
  \zref@extractdefault{LastPage}{abspage}{0}%
}
%    \end{macrocode}
%    \end{macro}
%    Also we mark the reference |LastPage| as used:
%    \begin{macrocode}
\AtBeginDocument{%
  \zref@refused{LastPage}%
}
%    \end{macrocode}
%    \begin{macrocode}
%</totpages>
%    \end{macrocode}
%
%
% \subsection{Module \xmodule{pagelayout}}
%
%    \begin{macrocode}
%<*pagelayout>
\NeedsTeXFormat{LaTeX2e}
\ProvidesPackage{zref-pagelayout}%
  [2020-07-03 v2.32 Module pagelayout for zref (HO)]%
\RequirePackage{zref-base}[2019/11/29]
\ifx\ZREF@base@ok Y%
\else
  \expandafter\endinput
\fi
%    \end{macrocode}
%
%    \begin{macrocode}
\RequirePackage{zref-thepage}[2019/11/29]
\RequirePackage{iftex}[2019/11/07]%
\RequirePackage{atveryend}[2010/03/24]
%    \end{macrocode}
%
%
% \subsubsection{Define layout properties}
%
%    \begin{macrocode}
\def\ZREF@temp#1{%
  \begingroup
    \escapechar=-1 %
  \ltx@ifundefined{\string#1}{\endgroup}{%
    \edef\x{%
      \endgroup
      \noexpand\zref@newprop*{\string#1}%
                             [\noexpand\number\noexpand#1]% hash-ok
                             {\noexpand\number\noexpand#1}%
      \noexpand\zref@addprop{thepage}{\string#1}%
    }%
    \x
  }%
}
\ZREF@temp\mag
\ZREF@temp\paperwidth
\ZREF@temp\paperheight
\ZREF@temp\stockwidth % memoir.cls, crop.sty
\ZREF@temp\stockheight % memoir.cls, crop.sty
\ZREF@temp\mediawidth % VTeX
\ZREF@temp\mediaheight % VTeX
\ifluatex
\zref@newprop*{pdfvorigin}%
                             [\number\pdfvariable vorigin]% hash-ok
                             {\number\pdfvariable vorigin}%
\zref@addprop{thepage}{pdfvorigin}
\zref@newprop*{pdfhorigin}%
                             [\number\pdfvariable horigin]% hash-ok
                             {\number\pdfvariable horigin}%
\zref@addprop{thepage}{pdfhorigin}
\zref@newprop*{pdfpageheight}%
                             [\number\pageheight]% hash-ok
                             {\number\pageheight}%
\zref@addprop{thepage}{pdfpageheight}
\zref@newprop*{pdfpagewidth}%
                             [\number\pagewidth]% hash-ok
                             {\number\pagewidth}%
\zref@addprop{thepage}{pdfpagewidth}
\else
\ZREF@temp\pdfpagewidth
\ZREF@temp\pdfpageheight
\ZREF@temp\pdfhorigin
\ZREF@temp\pdfvorigin
\fi
\ZREF@temp\hoffset
\ZREF@temp\voffset
\ZREF@temp\topmargin
\ZREF@temp\oddsidemargin
\ZREF@temp\evensidemargin
\ZREF@temp\textwidth
\ZREF@temp\textheight
\ZREF@temp\headheight
\ZREF@temp\headsep
\ZREF@temp\footskip
\ZREF@temp\marginparwidth
\ZREF@temp\marginparsep
\ZREF@temp\columnwidth
\ZREF@temp\columnsep
\ZREF@temp\trimedge % memoir.cls
\ZREF@temp\spinemargin % memoir.cls
\ZREF@temp\foremargin % memoir.cls
\ZREF@temp\trimtop % memoir.cls
\ZREF@temp\uppermargin % memoir.cls
\ZREF@temp\headmargin % memoir.cls
%    \end{macrocode}
%
%    \begin{macrocode}
\IfFormatAtLeastTF{2020/10/01}
 {
   \zref@newprop*{outputboxwd}[0pt]{\ShipoutBoxWidth}
   \zref@newprop*{outputboxht}[0pt]{\ShipoutBoxHeight}
   \zref@newprop*{outputboxdp}[0pt]{\ShipoutBoxDepth}
 }
 {
   \zref@newprop*{outputboxwd}[0pt]{\AtBeginShipoutBoxWidth}
   \zref@newprop*{outputboxht}[0pt]{\AtBeginShipoutBoxHeight}
   \zref@newprop*{outputboxdp}[0pt]{\AtBeginShipoutBoxDepth}
 }
%    \end{macrocode}
%    \begin{macrocode}
\zref@addprops{thepage}{outputboxwd,outputboxht,outputboxdp}
%    \end{macrocode}
%
%    \begin{macro}{\ifZREF@pl@list}
%    \begin{macrocode}
\ltx@newif\ifZREF@pl@list
%    \end{macrocode}
%    \end{macro}
%    \begin{macro}{\zref@listpagelayout}
%    \begin{macrocode}
\ZREF@IfDefinable\zlistpagelayout\def{%
  {\global\ZREF@pl@listtrue}%
}
%    \end{macrocode}
%    \end{macro}
%    \begin{macro}{\ZREF@pl@AfterLastShipout}
%    \begin{macrocode}
\def\ZREF@pl@AfterLastShipout{%
  \ifZREF@pl@list
    \edef\ZREF@page@max{\the\value{abspage}}%
    \ltx@ifundefined{ZREF@org@testdef}{%
      \let\ZREF@org@testdef\@testdef
      \def\@testdef##1##2##3{%
        \ZREF@org@testdef{##1}{##2}{##3}%
        \def\ZREF@temp{##1}%
        \ifx\ZREF@temp\ZREF@RefPrefix
          \expandafter\gdef\csname##1@##2\endcsname{##3}%
        \fi
      }%
    }{}%
    \AtVeryEndDocument{\ZREF@pl@AtVeryEnd}%
  \fi
}
%    \end{macrocode}
%    \end{macro}
%    \begin{macro}{\ZREF@pl@AtVeryEnd}
%    \begin{macrocode}
\def\ZREF@pl@AtVeryEnd{%
  \begingroup
    \toks@{Page layout parameters:\MessageBreak}%
    \count@=1 %
    \ZREF@pl@ListPage
    \edef\x{\endgroup
      \noexpand\@PackageInfoNoLine{zref-pagelayout}{\the\toks@}%
    }%
  \x
}
%    \end{macrocode}
%    \end{macro}
%    \begin{macro}{\ZREF@pl@ListPage}
%    \begin{macrocode}
\def\ZREF@pl@ListPage{%
  \edef\x{%
    \toks@={%
      \the\toks@
      Page \the\count@:\noexpand\MessageBreak
      \zref@ifrefundefined{thepage\the\count@}{}{%
        \ltx@space\ltx@space mag = %
        \zref@extract{thepage\the\count@}{mag}%
        \noexpand\MessageBreak
        \ZREF@pl@ListEntry{paperwidth}%
        \ZREF@pl@ListEntry{paperheight}%
        \ZREF@pl@ListEntry{stockwidth}%
        \ZREF@pl@ListEntry{stockheight}%
        \ZREF@pl@ListEntry{mediawidth}%
        \ZREF@pl@ListEntry{mediaheight}%
        \ZREF@pl@ListEntry{pdfpagewidth}%
        \ZREF@pl@ListEntry{pdfpageheight}%
        \ZREF@pl@ListEntry{pdfhorigin}%
        \ZREF@pl@ListEntry{pdfvorigin}%
        \ZREF@pl@ListEntry{hoffset}%
        \ZREF@pl@ListEntry{voffset}%
        \ZREF@pl@ListEntry{topmargin}%
        \ZREF@pl@ListEntry{oddsidemargin}%
        \ZREF@pl@ListEntry{evensidemargin}%
        \ZREF@pl@ListEntry{textwidth}%
        \ZREF@pl@ListEntry{textheight}%
        \ZREF@pl@ListEntry{headheight}%
        \ZREF@pl@ListEntry{headsep}%
        \ZREF@pl@ListEntry{footskip}%
        \ZREF@pl@ListEntry{marginparwidth}%
        \ZREF@pl@ListEntry{marginparsep}%
        \ZREF@pl@ListEntry{columnwidth}%
        \ZREF@pl@ListEntry{columnsep}%
        \ZREF@pl@ListEntry{trimedge}%
        \ZREF@pl@ListEntry{spinemargin}%
        \ZREF@pl@ListEntry{foremargin}%
        \ZREF@pl@ListEntry{trimtop}%
        \ZREF@pl@ListEntry{uppermargin}%
        \ZREF@pl@ListEntry{headmargin}%
      }%
    }%
  }\x
  \ifnum\ZREF@page@max>\count@
    \advance\count@ by\ltx@one
  \else
    \expandafter\ltx@gobble
  \fi
  \ZREF@pl@ListPage
}
%    \end{macrocode}
%    \end{macro}
%    \begin{macro}{\ZREF@pl@ListEntry}
%    \begin{macrocode}
\def\ZREF@pl@ListEntry#1{%
  \zref@ifpropundefined{#1}{%
  }{%
    \zref@ifrefcontainsprop{thepage\the\count@}{#1}{%
      \ltx@space\ltx@space#1 = %
      \zref@extract{thepage\the\count@}{#1}sp = %
      \the\dimexpr\zref@extract{thepage\the\count@}{#1}sp\relax
      \noexpand\MessageBreak
    }{}%
  }%
}
%    \end{macrocode}
%    \end{macro}
%    \begin{macrocode}
\AfterLastShipout{%
  \ZREF@pl@AfterLastShipout
}
%    \end{macrocode}
%
%    \begin{macrocode}
%</pagelayout>
%    \end{macrocode}
%
%
% \subsection{Module \xmodule{pageattr}}
%
%    \begin{macrocode}
%<*pageattr>
\NeedsTeXFormat{LaTeX2e}
\ProvidesPackage{zref-pageattr}%
  [2020-07-03 v2.32 Module pageattr for zref (HO)]%
\RequirePackage{zref-base}[2019/11/29]
\ifx\ZREF@base@ok Y%
\else
  \expandafter\endinput
\fi
%    \end{macrocode}
%
%    \begin{macrocode}
\RequirePackage{iftex}[2019/11/07]%
%    \end{macrocode}
%
%
%    \begin{macrocode}
\let\ZREF@temp=N%
\ifluatex
\expandafter\@firstoftwo
\else
\expandafter\@secondoftwo
\fi
{%luatex
 \RequirePackage{zref-thepage}[2019/11/29]
 \RequirePackage{zref-lastpage}[2019/11/29]%
 \zref@newprop*{pdfpageattr}[]{\zref@hex{\the\pdfvariable pageattr}}%
 \zref@addprop{thepage}{pdfpageattr}%
 \zref@newprop*{pdfpagesattr}[]{\zref@hex{\the\pdfvariable pagesattr}}%
 \zref@addprop{LastPage}{pdfpagesattr}%
 \let\ZREF@temp=Y%
}
{%other
 \ltx@IfUndefined{pdfpageattr}{%
  \@PackageInfoNoLine{zref-pageattr}{%
    \string\pdfpageattr\space is not available%
  }%
  \def\zref@pdfpageattr#1{}%
  \def\zref@pdfpageattr@used#1{}%
 }{%
  \RequirePackage{zref-thepage}[2019/11/29]%
  \zref@newprop*{pdfpageattr}[]{\zref@hex{\the\pdfpageattr}}%
  \zref@addprop{thepage}{pdfpageattr}%
  \let\ZREF@temp=Y%
 }
 \ltx@IfUndefined{pdfpagesattr}{%
  \@PackageInfoNoLine{zref-pageattr}{%
    \string\pdfpagesattr\space is not available%
  }%
  \def\zref@pdfpagesattr{}%
  \def\zref@pdfpagesattr@used{}%
 }{%
  \RequirePackage{zref-lastpage}[2019/11/29]%
  \zref@newprop*{pdfpagesattr}[]{\zref@hex{\the\pdfpagesattr}}%
  \zref@addprop{LastPage}{pdfpagesattr}%
  \let\ZREF@temp=Y%
 }%
}%
%    \end{macrocode}
%    \begin{macrocode}
\ifx\ZREF@temp N%
  \expandafter\endinput
\fi
%    \end{macrocode}
%    \begin{macrocode}
\RequirePackage{zref-abspage}[2019/11/29]
\RequirePackage{atveryend}[2010/03/24]
\RequirePackage{pdftexcmds}[2010/04/01]
\let\ZREF@temp=Y%
\ltx@IfUndefined{pdf@escapehex}{\let\ZREF@temp=N}{}
\ltx@IfUndefined{pdf@unescapehex}{\let\ZREF@temp=N}{}
\ifx\ZREF@temp N%
  \let\zref@hex\ltx@firstofone
  \let\zref@unhex\ltx@firstofone
\else
  \let\zref@hex\pdf@escapehex
  \let\zref@unhex\pdf@unescapehex
\fi
%    \end{macrocode}
%
%    \begin{macro}{\ifZREF@pa@list}
%    \begin{macrocode}
\ltx@newif\ifZREF@pa@list
%    \end{macrocode}
%    \end{macro}
%    \begin{macro}{\zref@listpageattr}
%    \begin{macrocode}
\ZREF@IfDefinable\zlistpageattr\def{%
  {\ZREF@pa@listtrue}%
}
%    \end{macrocode}
%    \end{macro}
%    \begin{macro}{\ZREF@pa@AfterLastShipout}
%    \begin{macrocode}
\def\ZREF@pa@AfterLastShipout{%
  \ifZREF@pa@list
    \edef\ZREF@page@max{\the\value{abspage}}%
    \ltx@ifundefined{ZREF@org@testdef}{%
      \let\ZREF@org@testdef\@testdef
      \def\@testdef##1##2##3{%
        \ZREF@org@testdef{##1}{##2}{##3}%
        \def\ZREF@temp{##1}%
        \ifx\ZREF@temp\ZREF@RefPrefix
          \expandafter\xdef\csname##1@##2\endcsname{##3}%
        \fi
      }%
    }{}%
    \AtVeryEndDocument{\ZREF@pa@AtVeryEnd}%
  \fi
}
%    \end{macrocode}
%    \end{macro}
%    \begin{macro}{\ZREF@pa@AtVeryEnd}
%    \begin{macrocode}
\let\ZREF@temp=Y%
\ltx@IfUndefined{pdfpageattr}{}{\let\ZREF@temp=N}
\ifluatex \let\ZREF@temp=N \fi
\ifx\ZREF@temp Y
 \expandafter\@firstoftwo
\else
 \expandafter\@secondoftwo
\fi
 {%
  \def\ZREF@pa@AtVeryEnd{}%
 }
 {%
  \def\ZREF@pa@AtVeryEnd{%
    \begingroup
      \toks@{List of \ltx@backslashchar
             \ifluatex pdfvariable\else pdf\fi
             pdfpageattr:\MessageBreak}%
      \count@=1 %
      \ZREF@pa@ListPage
      \edef\x{\endgroup
        \noexpand\@PackageInfoNoLine{zref-pageattr}{%
          \the\toks@
        }%
      }%
    \x
  }%
%    \end{macrocode}
%    \end{macro}
%    \begin{macro}{\zref@pageattr}
%    \begin{macrocode}
\def\zref@pdfpageattr#1{%
  \zref@unhex{%
    \zref@extract{thepage\ZREF@number{#1}}{pdfpageattr}%
  }%
}
% compability, \zref@pageattr was defined in older versions
\let\zref@pageattr\zref@pdfpageattr
%    \end{macrocode}
%    \end{macro}
%    \begin{macro}{\zref@pageattr@used}
%    \begin{macrocode}
\ZREF@Robust\def\zref@pageattr@used#1{%
  \zref@refused{thepage\ZREF@number{#1}}%
}
%    \end{macrocode}
%    \end{macro}
%    \begin{macro}{\ZREF@pa@ListPage}
%    \begin{macrocode}
  \def\ZREF@pa@ListPage{%
    \edef\x{%
      \toks@={%
        \the\toks@
        Page \the\count@:%
        \noexpand\MessageBreak
        \zref@ifrefundefined{thepage\the\count@}{}{%
          <<\zref@pdfpageattr\count@>>%
          \noexpand\MessageBreak
        }%
      }%
    }\x
    \ifnum\ZREF@page@max>\count@
      \advance\count@ by\ltx@one
    \else
      \expandafter\ltx@gobble
    \fi
    \ZREF@pa@ListPage
  }%
}
%    \end{macrocode}
%    \end{macro}
%    \begin{macrocode}
\let\ZREF@temp=Y%
\ltx@IfUndefined{pdfpagesattr}{}{\let\ZREF@temp=N}
\ifluatex \let\ZREF@temp=N \fi
\ifx\ZREF@temp N
 \expandafter\@firstofone
\fi
 {%
%    \end{macrocode}
%    \begin{macro}{\zref@pdfpagesattr}
%    \begin{macrocode}
  \def\zref@pdfpagesattr{%
    \zref@unhex{%
      \zref@extract{LastPage}{pdfpagesattr}%
    }%
  }%
%    \end{macrocode}
%    \end{macro}
%    \begin{macro}{\zref@pdfpagesattr@used}
%    \begin{macrocode}
  \ZREF@Robust\def\zref@pdfpagesattr@used{%
    \zref@refused{LastPage}%
  }%
%    \end{macrocode}
%    \end{macro}
%    \begin{macrocode}
  \ltx@LocalAppendToMacro\ZREF@pa@AtVeryEnd{%
    \@PackageInfoNoLine{zref-pageattr}{%
      \ltx@backslashchar
      \ifluatex pdfvariable\else pdf\fi
      pagesattr:\MessageBreak
      <<\zref@pdfpagesattr>>%
      \MessageBreak
    }%
  }%
}
\AfterLastShipout{%
  \ZREF@pa@AfterLastShipout
}
%    \end{macrocode}
%    \begin{macrocode}
%</pageattr>
%    \end{macrocode}
%
%
% \subsection{Module \xmodule{marks}}
%
%    \begin{macrocode}
%<*marks>
\NeedsTeXFormat{LaTeX2e}
\ProvidesPackage{zref-marks}%
  [2020-07-03 v2.32 Module marks for zref (HO)]%
\RequirePackage{zref-base}[2019/11/29]
\ifx\ZREF@base@ok Y%
\else
  \expandafter\endinput
\fi
%    \end{macrocode}
%
%    \begin{macrocode}
\newcommand*{\zref@marks@register}[3][]{%
  \edef\ZREF@TempName{#1}%
  \edef\ZREF@TempNum{\ZREF@number{#2}}%
  \ifnum\ZREF@TempNum<\ltx@zero %
    \PackageError\ZREF@name{%
      \string\zref@marks@register\ltx@space is called with invalid%
      \MessageBreak
      marks register number (\ZREF@TempNum)%
    }{%
      Use `0' or the command, defined by \string\newmarks.\MessageBreak
      \@ehc
    }%
  \else
    \ifx\ZREF@TempName\ltx@empty
      \edef\ZREF@TempName{mark\romannumeral\ZREF@TempNum}%
    \else
      \edef\ZREF@TempName{marks\ZREF@TempName}%
    \fi
    \ZREF@MARKS@DefineProp{top}%
    \ZREF@MARKS@DefineProp{first}%
    \ZREF@MARKS@DefineProp{bot}%
    \kv@parse{#3}{%
      \ifx\kv@value\relax
        \def\kv@value{top,first,bot}%
      \fi
      \edef\ZREF@temp{\expandafter\ltx@car\kv@key X\@nil}%
      \ifx\ZREF@temp\ZREF@STAR
        \edef\kv@key{\expandafter\ltx@cdr\kv@key\@nil}%
        \zref@newlist\kv@key
      \fi
      \expandafter\comma@parse\expandafter{\kv@value}{%
        \ifcase0\ifx\comma@entry\ZREF@NAME@top 1\else
                \ifx\comma@entry\ZREF@NAME@first 1\else
                \ifx\comma@entry\ZREF@NAME@bot 1\fi\fi\fi\ltx@space
          \PackageWarning{zref-marks}{%
            Use `top', `first' or `bot' for the list values%
            \MessageBreak
            in the third argument of \string\zref@marks@register.%
            \MessageBreak
            Ignoring unkown value `\comma@entry'%
          }%
        \else
          \zref@addprop{\kv@key}{\comma@entry\ZREF@TempName}%
        \fi
        \ltx@gobble
      }%
      \ltx@gobbletwo
    }%
  \fi
}
\def\ZREF@STAR{*}
\def\ZREF@NAME@top{top}
\def\ZREF@NAME@first{first}
\def\ZREF@NAME@bot{bot}
\def\ZREF@MARKS@DefineProp#1{%
  \zref@ifpropundefined{#1\ZREF@TempName}{%
    \ifnum\ZREF@TempNum=\ltx@zero
      \begingroup
        \edef\x{\endgroup
          \noexpand\zref@newprop*{#1\ZREF@TempName}[]{%
            \expandafter\noexpand\csname#1mark\endcsname
          }%
        }%
      \x
    \else
      \begingroup
        \edef\x{\endgroup
          \noexpand\zref@newprop*{#1\ZREF@TempName}[]{%
            \expandafter\noexpand\csname#1marks\endcsname
            \ZREF@TempNum
          }%
        }%
      \x
    \fi
  }{%
    \PackageWarning{zref-marks}{%
      \string\zref@marks@register\ltx@space does not generate the%
      \MessageBreak
      new property `#1\ZREF@TempName', because\MessageBreak
      it is already defined%
    }%
  }%
}
%    \end{macrocode}
%    \begin{macrocode}
%</marks>
%    \end{macrocode}
%
% \subsection{Module \xmodule{runs}}
%
%    This module does not use the label-reference-system.
%    The reference changes with each \hologo{LaTeX}~run and would
%    force a rerun warning always.
%    \begin{macrocode}
%<*runs>
\NeedsTeXFormat{LaTeX2e}
\ProvidesPackage{zref-runs}%
  [2020-07-03 v2.32 Module runs for zref (HO)]%
%    \end{macrocode}
%
%    \begin{macro}{\zruns}
%    \begin{macrocode}
\providecommand*{\zruns}{0}%
\AtBeginDocument{%
  \edef\zruns{\number\numexpr\zruns+1}%
  \begingroup
    \def\on@line{}%
    \PackageInfo{zref-runs}{LaTeX runs: \zruns}%
    \if@filesw
      \immediate\write\@mainaux{%
        \string\gdef\string\zruns{\zruns}%
      }%
    \fi
  \endgroup
}
%    \end{macrocode}
%    \end{macro}
%
%    \begin{macrocode}
%</runs>
%    \end{macrocode}
%
%
% \subsection{Module \xmodule{perpage}}
%
%    \begin{macrocode}
%<*perpage>
\NeedsTeXFormat{LaTeX2e}
\ProvidesPackage{zref-perpage}%
  [2020-07-03 v2.32 Module perpage for zref (HO)]%
\RequirePackage{zref-base}[2019/11/29]
\ifx\ZREF@base@ok Y%
\else
  \expandafter\endinput
\fi
%    \end{macrocode}
%
%    This module resets a counter at page boundaries.
%    Because of the asynchronous output routine
%    page counter properties cannot be asked directly,
%    references are necessary.
%
%    For detecting changed pages module \xmodule{abspage}
%    is loaded.
%    \begin{macrocode}
\RequirePackage{zref-abspage}[2019/11/29]
%    \end{macrocode}
%
%    We group the properties for the needed references
%    in the property list |perpage|. The property |pagevalue|
%    records the correct value of the page counter.
%    \begin{macrocode}
\ZREF@NewPropPageValue
\zref@newlist{perpage}
\zref@addprops{perpage}{abspage,page,pagevalue}
%    \end{macrocode}
%    The page value, known by the reference mechanism, will be
%    stored in counter |zpage|.
%    \begin{macrocode}
\newcounter{zpage}
%    \end{macrocode}
%    Counter |zref@unique| helps in generating unique reference
%    names.
%    \begin{macrocode}
\zref@require@unique
%    \end{macrocode}
%    In order to be able to reset the counter, we hook here
%    into \cs{stepcounter}. In fact two nested hooks are used
%    to allow other packages to use the first hook at the beginning
%    of \cs{stepcounter}.
%    \begin{macrocode}
\let\ZREF@org@stepcounter\stepcounter
\def\stepcounter#1{%
  \ifcsname @stepcounterhook@#1\endcsname
    \csname @stepcounterhook@#1\endcsname
  \fi
  \ZREF@org@stepcounter{#1}%
}
%    \end{macrocode}
%    \cs{@stpelt} must be adapted due to the change in latex
%    2015-01, see https://github.com/ho-tex/zref/issues/26
%    \begin{macrocode}
\let\ZREF@org@@stpelt\@stpelt
\def\@stpelt#1{%
  \ifcsname ZREF@perpage@#1\endcsname
    \begingroup
      \let\stepcounter\ZREF@org@stepcounter
      \ZREF@org@@stpelt{#1}%
    \endgroup
    \expandafter\ltx@gobbletwo
  \fi
  \ZREF@org@@stpelt{#1}%
}
%    \end{macrocode}
%
%    \begin{macro}{\zmakeperpage}
%    Makro \cs{zmakeperpage} resets a counter at each page
%    break. It uses the same syntax and semantics as
%    \cs{MakePerPage} from package \xpackage{perpage} \cite{perpage}.
%    The initial start value can be given by the optional
%    argument. Default is one that means after the first
%    \cs{stepcounter} on a new page the counter starts with
%    one.
%    \begin{macrocode}
\ZREF@IfDefinable\zmakeperpage\def{%
  {%
    \@ifnextchar[\ZREF@makeperpage@opt{\ZREF@@makeperpage[\ltx@zero]}%
  }%
}
%    \end{macrocode}
%    \end{macro}
%    We hook before the counter is incremented in \cs{stepcounter},
%    package \xpackage{perpage} afterwards. Thus a little
%    calculation is necessary.
%    \begin{macrocode}
\def\ZREF@makeperpage@opt[#1]{%
  \begingroup
    \edef\x{\endgroup
      \noexpand\ZREF@@makeperpage[\number\numexpr#1-1\relax]%
    }%
  \x
}
%    \end{macrocode}
%    \begin{macrocode}
\def\ZREF@@makeperpage[#1]#2{%
  \@ifundefined{@stepcounterhook@#2}{%
    \expandafter\gdef\csname @stepcounterhook@#2\endcsname{}%
  }{}%
  \expandafter\gdef\csname ZREF@perpage@#2\endcsname{%
    \ZREF@@perpage@step{#2}{#1}%
  }%
  \expandafter\g@addto@macro\csname @stepcounterhook@#2\endcsname{%
    \ifcsname ZREF@perpage@#2\endcsname
      \csname ZREF@perpage@#2\endcsname
    \fi
  }%
}
%    \end{macrocode}
%    \begin{macro}{\ZREF@@perpage@step}
%    The heart of this module follows.
%    \begin{macrocode}
\def\ZREF@@perpage@step#1#2{%
%    \end{macrocode}
%    First the reference is generated.
%    \begin{macrocode}
  \global\advance\c@zref@unique\ltx@one
  \begingroup
    \expandafter
    \zref@labelbylist\expandafter{\thezref@unique}{perpage}%
%    \end{macrocode}
%    The \cs{expandafter} commands are necessary, because \cs{ZREF@temp}
%    is also used inside of \cs{zref@labelbylist}.
%
%    The evaluation of the reference follows. If the reference
%    is not yet kwown, we use the page counter as approximation.
%    \begin{macrocode}
    \zref@ifrefundefined\thezref@unique{%
      \global\c@zpage=\c@page
      \global\let\thezpage\thepage
      \expandafter\xdef\csname ZREF@abspage@#1\endcsname{%
        \number\c@abspage
      }%
    }{%
%    \end{macrocode}
%    The reference is used to set \cs{thezpage} and counter |zpage|.
%    \begin{macrocode}
      \global\c@zpage=\zref@extract\thezref@unique{pagevalue}\relax
      \xdef\thezpage{\noexpand\zref@extract{\thezref@unique}{page}}%
      \expandafter\xdef\csname ZREF@abspage@#1\endcsname{%
        \zref@extractdefault\thezref@unique
            {abspage}{\number\c@abspage}%
      }%
    }%
%    \end{macrocode}
%    Page changes are detected by a changed absolute page number.
%    \begin{macrocode}
    \expandafter\ifx\csname ZREF@abspage@#1\expandafter\endcsname
                    \csname ZREF@currentabspage@#1\endcsname
    \else
      \global\csname c@#1\endcsname=#2\relax
      \global\expandafter\let
          \csname ZREF@currentabspage@#1\expandafter\endcsname
          \csname ZREF@abspage@#1\endcsname
    \fi
  \endgroup
}
%    \end{macrocode}
%    \end{macro}
%    \begin{macro}{\zunmakeperpage}
%    Macro \cs{zunmakeperpage} cancels the effect of \cs{zmakeperpage}.
%    \begin{macrocode}
\ZREF@IfDefinable\zunmakeperpage\def{%
  #1{%
    \global\expandafter
    \let\csname ZREF@perpage@#1\endcsname\@undefined
  }%
}
%    \end{macrocode}
%    \end{macro}
%
%    \begin{macrocode}
%</perpage>
%    \end{macrocode}
%
%
% \subsection{Module \xmodule{titleref}}
%
%    \begin{macrocode}
%<*titleref>
\NeedsTeXFormat{LaTeX2e}
\ProvidesPackage{zref-titleref}%
  [2020-07-03 v2.32 Module titleref for zref (HO)]%
\RequirePackage{zref-base}[2019/11/29]
\ifx\ZREF@base@ok Y%
\else
  \expandafter\endinput
\fi
\RequirePackage{gettitlestring}[2009/12/08]
%    \end{macrocode}
%
% \subsubsection{Implementation}
%    \begin{macrocode}
\RequirePackage{keyval}
%    \end{macrocode}
%
%    This module makes section and caption titles available
%    for the reference system. It uses some of the ideas
%    of package \xpackage{nameref} and \xpackage{titleref}.
%
%    Now we can add the property \verb|title| is added to
%    the main property list.
%    \begin{macrocode}
\ZREF@NewPropTitle
\zref@addprop\ZREF@mainlist{title}%
%    \end{macrocode}
%
%    The title strings go into the \xfile{.aux} file, thus
%    they need some kind of protection.
%    Package \xpackage{titleref} uses a protected expansion
%    method. The advantage is that this can be used to
%    cleanup the string and to remove \cs{label}, \cs{index}
%    and other macros unwanted for referencing.
%    But there is the risk that fragile stuff can break.
%
%    Therefore package \xpackage{nameref} does not expand
%    the string. Thus the entries can safely be written to
%    the \xfile{.aux} file. But potentially dangerous
%    macros such as \cs{label} remain in the string and
%    can cause problems when using the string in references.
%
%    \begin{macro}{\ifzref@titleref@expand}
%    The switch \cs{ifzref@titleref@expand} distinguishes
%    between the both methods. Package \xpackage{nameref}'s
%    behaviour is achieved by setting the switch to false,
%    otherwise \xpackage{titleref}'s expansion is used.
%    Default is false.
%    \begin{macrocode}
\newif\ifzref@titleref@expand
%    \end{macrocode}
%    \end{macro}
%
%    \begin{macro}{\ZREF@titleref@hook}
%    The hook \cs{ZREF@titleref@hook} allows to extend
%    the cleanup for the expansion method. Thus
%    unnecessary macros can be removed or dangerous commands
%    removed. The hook is executed before the expansion of
%    \cs{zref@titleref@current}.
%    \begin{macrocode}
\let\ZREF@titleref@hook\ltx@empty
%    \end{macrocode}
%    \end{macro}
%    \begin{macro}{\zref@titleref@cleanup}
%    The hook should not be used directly, instead we provide the
%    macro \cs{zref@titleref@cleanup} to add stuff to the hook
%    and prevents that a previous non-empty content is not discarded
%    accidently.
%    \begin{macrocode}
\ZREF@Robust\def\zref@titleref@cleanup#1{%
   \begingroup
    \toks@\expandafter{%
      \ZREF@titleref@hook
      #1%
    }%
  \expandafter\endgroup
  \expandafter\def\expandafter\ZREF@titleref@hook\expandafter{%
    \the\toks@
  }%
}%
%    \end{macrocode}
%    \end{macro}
%
%    \begin{macro}{\ifzref@titleref@stripperiod}
%    Sometimes a title contains a period at the end.
%    Package \xpackage{nameref} removes this. This behaviour
%    is controlled by the switch \cs{ifzref@titleref@stripperiod}
%    and works regardless of the setting of option \xoption{expand}.
%    Period stripping is the default.
%    \begin{macrocode}
\newif\ifzref@titleref@stripperiod
\zref@titleref@stripperiodtrue
%    \end{macrocode}
%    \end{macro}
%
%    \begin{macro}{\zref@titleref@setcurrent}
%    Macro \cs{zref@titleref@setcurrent} sets a new current title
%    stored in \cs{zref@titleref@current}. Some cleanup and
%    expansion is performed that can be controlled by the
%    previous switches.
%    \begin{macrocode}
\ZREF@Robust\def\zref@titleref@setcurrent#1{%
  \ifzref@titleref@expand
    \GetTitleStringExpand{#1}%
  \else
    \GetTitleStringNonExpand{#1}%
  \fi
  \edef\zref@titleref@current{%
    \detokenize\expandafter{\GetTitleStringResult}%
  }%
  \ifzref@titleref@stripperiod
    \edef\zref@titleref@current{%
      \expandafter\ZREF@stripperiod\zref@titleref@current
      \ltx@empty.\ltx@empty\@nil
    }%
  \fi
}%
\GetTitleStringDisableCommands{%
  \ZREF@titleref@hook
}
%    \end{macrocode}
%    \end{macro}
%    \begin{macro}{\ZREF@stripperiod}
%    If \cs{ZREF@stripperiod} is called, the argument consists of
%    space tokens and tokens with catcode 12 (other), because
%    of \hologo{eTeX}'s \cs{detokenize}.
%    \begin{macrocode}
\def\ZREF@stripperiod#1.\ltx@empty#2\@nil{#1}%
%    \end{macrocode}
%    \end{macro}
%
% \subsubsection{User interface}
%
%    \begin{macro}{\ztitlerefsetup}
%    The behaviour of module \xmodule{titleref} is controlled
%    by switches and a hook. They can be set by \cs{ztitlerefsetup}
%    with a key value interface, provided by package \xpackage{keyval}.
%    Also the current title can be given explicitly by the key
%    \verb|title|.
%    \begin{macrocode}
\define@key{ZREF@TR}{expand}[true]{%
  \csname zref@titleref@expand#1\endcsname
}%
\define@key{ZREF@TR}{stripperiod}[true]{%
  \csname zref@titleref@stripperiod#1\endcsname
}%
\define@key{ZREF@TR}{cleanup}{%
  \zref@titleref@cleanup{#1}%
}%
\define@key{ZREF@TR}{title}{%
  \def\zref@titleref@current{#1}%
}%
\ZREF@IfDefinable\ztitlerefsetup\def{%
  {\kvsetkeys{ZREF@TR}}%
}%
%    \end{macrocode}
%    \end{macro}
%
%    \begin{macro}{\ztitleref}
%    The user command \cs{ztitleref} references the title.
%    For safety \cs{label} is disabled to prevent multiply
%    defined references.
%    \begin{macrocode}
\ZREF@IfDefinable\ztitleref\def{%
  {\zref@wrapper@babel\ZREF@titleref}%
}%
\def\ZREF@titleref#1{%
  \begingroup
    \zref@refused{#1}%
    \let\label\ltx@gobble
    \zref@extract{#1}{title}%
  \endgroup
}%
%    \end{macrocode}
%    \end{macro}
%
% \subsubsection{Patches for section and caption commands}
%
%    The section and caption macros are patched to extract the
%    title data.
%
%    Captions of figures and tables.
%    \begin{macrocode}
\AtBeginDocument{%
  \ZREF@patch{@caption}{%
    \long\def\@caption#1[#2]{%
      \zref@titleref@setcurrent{#2}%
      \ZREF@org@@caption{#1}[{#2}]%
    }%
  }%
%    \end{macrocode}
%
%    Section commands without star. The title version for the
%    table of contents is used because it is usually shorter and
%    more robust.
%    \begin{macrocode}
  \ZREF@patch{@part}{%
    \def\@part[#1]{%
      \zref@titleref@setcurrent{#1}%
      \ZREF@org@@part[{#1}]%
    }%
  }%
  \ZREF@patch{@chapter}{%
    \def\@chapter[#1]{%
      \zref@titleref@setcurrent{#1}%
      \ZREF@org@@chapter[{#1}]%
    }%
  }%
  \ZREF@patch{@sect}{%
    \def\@sect#1#2#3#4#5#6[#7]{%
      \zref@titleref@setcurrent{#7}%
      \ZREF@org@@sect{#1}{#2}{#3}{#4}{#5}{#6}[{#7}]%
    }%
  }%
%    \end{macrocode}
%    The star versions of the section commands.
%    \begin{macrocode}
  \ZREF@patch{@spart}{%
    \def\@spart#1{%
      \zref@titleref@setcurrent{#1}%
      \ZREF@org@@spart{#1}%
    }%
  }%
  \ZREF@patch{@schapter}{%
    \def\@schapter#1{%
      \zref@titleref@setcurrent{#1}%
      \ZREF@org@@schapter{#1}%
    }%
  }%
  \ZREF@patch{@ssect}{%
    \def\@ssect#1#2#3#4#5{%
      \zref@titleref@setcurrent{#5}%
      \ZREF@org@@ssect{#1}{#2}{#3}{#4}{#5}%
    }%
  }%
%    \end{macrocode}
%
% \subsubsection{Environment \textsf{description}}
%
%    \begin{macrocode}
  \ZREF@patch{descriptionlabel}{%
    \def\descriptionlabel#1{%
      \zref@titleref@setcurrent{#1}%
      \ZREF@org@descriptionlabel{#1}%
    }%
  }%
%    \end{macrocode}
%
% \subsubsection{Class \xclass{memoir}}
%
%    \begin{macrocode}
  \@ifclassloaded{memoir}{%
    \ltx@IfUndefined{ifheadnameref}{}{%
      \def\@chapter[#1]#2{%
        \ltx@IfUndefined{ch@pt@c}{%
          \zref@titleref@setcurrent{#1}%
        }{%
          \ifx\ch@pt@c\ltx@empty
            \zref@titleref@setcurrent{#2}%
          \else
            \def\NR@temp{#1}%
            \ifx\NR@temp\ltx@empty
              \expandafter\zref@titleref@setcurrent
              \expandafter{\ch@pt@c}%
            \else
              \ifheadnameref
                \zref@titleref@setcurrent{#1}%
              \else
                \expandafter\zref@titleref@setcurrent
                \expandafter{\ch@pt@c}%
              \fi
            \fi
          \fi
        }%
        \ZREF@org@@chapter[{#1}]{#2}%
      }%
      \ZREF@patch{M@sect}{%
        \def\M@sect#1#2#3#4#5#6[#7][#8]{%
          \ifheadnameref
            \zref@titleref@setcurrent{#8}%
          \else
            \zref@titleref@setcurrent{#7}%
          \fi
          \ZREF@org@M@sect{#1}{#2}{#3}{#4}{#5}{#6}[{#7}][{#8}]%
        }%
      }%
    }%
  }{}%
%    \end{macrocode}
%
% \subsubsection{Class \xclass{beamer}}
%
%    \begin{macrocode}
  \@ifclassloaded{beamer}{%
    \ZREF@patch{beamer@section}{%
      \long\def\beamer@section[#1]{%
        \zref@titleref@setcurrent{#1}%
        \ZREF@org@beamer@section[{#1}]%
      }%
    }%
    \ZREF@patch{beamer@subsection}{%
      \long\def\beamer@subsection[#1]{%
        \zref@titleref@setcurrent{#1}%
        \ZREF@org@beamer@subsection[{#1}]%
      }%
    }%
    \ZREF@patch{beamer@subsubsection}{%
      \long\def\beamer@subsubsection[#1]{%
        \zref@titleref@setcurrent{#1}%
        \ZREF@org@beamer@subsubsection[{#1}]%
      }%
    }%
  }{}%
%    \end{macrocode}
%
% \subsubsection{Package \xpackage{titlesec}}
%
%    \begin{macrocode}
  \@ifpackageloaded{titlesec}{%
    \ZREF@patch{ttl@sect@i}{%
      \def\ttl@sect@i#1#2[#3]#4{%
        \zref@titleref@setcurrent{#4}%
        \ZREF@org@ttl@sect@i{#1}{#2}[{#3}]{#4}%
      }%
    }%
    \ZREF@patch{ttl@straight@i}{%
      \def\ttl@straight@i#1[#2]#3{%
        \def\ZREF@temp{#2}%
        \ifx\ZREF@temp\ltx@empty
          \zref@titleref@setcurrent{#3}%
        \else
          \zref@titleref@setcurrent{#2}%
        \fi
        \ZREF@org@ttl@straight@i{#1}[{#2}]{#3}%
      }%
    }%
  }{}%
%    \end{macrocode}
%
% \subsubsection{Package \xpackage{longtable}}
%
%    Package \xpackage{longtable}: some support for its \cs{caption}.
%    However \cs{label} inside the caption is not supported.
%    \begin{macrocode}
  \@ifpackageloaded{longtable}{%
    \ZREF@patch{LT@c@ption}{%
      \def\LT@c@ption#1[#2]#3{%
        \ZREF@org@LT@c@ption{#1}[{#2}]{#3}%
        \zref@titleref@setcurrent{#2}%
      }%
    }%
  }{}%
%    \end{macrocode}
%
% \subsubsection{Package \xpackage{listings}}
%
%    Package \xpackage{listings}: support for its caption.
%    \begin{macrocode}
  \@ifpackageloaded{listings}{%
    \ZREF@patch{lst@MakeCaption}{%
      \def\lst@MakeCaption{%
        \ifx\lst@label\ltx@empty
        \else
          \expandafter\zref@titleref@setcurrent\expandafter{%
            \lst@@caption
          }%
        \fi
        \ZREF@org@lst@MakeCaption
      }%
    }%
  }{}%
%    \end{macrocode}
%
% \subsubsection{Theorems}
%
%    \begin{macrocode}
  \ZREF@patch{@opargbegintheorem}{%
    \def\@opargbegintheorem#1#2#3{%
      \zref@titleref@setcurrent{#3}%
      \ZREF@org@@opargbegintheorem{#1}{#2}{#3}%
    }%
  }%
  \@ifpackageloaded{amsthm}{%
    \begingroup
      \edef\x{macro:\string#1\string#2[\string#3]}%
      \@onelevel@sanitize\x
      \def\y#1->#2\@nil{#1}%
      \edef\z{\expandafter\y\meaning\@begintheorem->\@nil}%
      \@onelevel@sanitize\z
    \expandafter\endgroup
    \ifx\x\z
      \ZREF@patch{@begintheorem}{%
        \def\@begintheorem#1#2[#3]{%
          \zref@titleref@setcurrent{#3}%
          \ZREF@org@@begintheorem{#1}{#2}[{#3}]%
        }%
      }%
    \fi
  }{}%
%    \end{macrocode}
%
%    \begin{macrocode}
}
%    \end{macrocode}
%
%    \begin{macrocode}
%</titleref>
%    \end{macrocode}
%
%
% \subsection{Module \xmodule{xr}}
%
%    \begin{macrocode}
%<*xr>
\NeedsTeXFormat{LaTeX2e}
\ProvidesPackage{zref-xr}%
  [2020-07-03 v2.32 Module xr for zref (HO)]%
\RequirePackage{zref-base}[2019/11/29]
\ifx\ZREF@base@ok Y%
\else
  \expandafter\endinput
\fi
%    \end{macrocode}
%    \begin{macrocode}
\RequirePackage{keyval}
\RequirePackage{kvoptions}[2010/02/22]
%    \end{macrocode}
%
%    We declare property |url|, because this is added,
%    if a reference is imported and has not already set this field.
%    Or if |hyperref| is used, then this property can be asked.
%    \begin{macrocode}
\zref@newprop{url}{}%
\zref@newprop{urluse}{}%
%    \end{macrocode}
%    \begin{macrocode}
\zref@newprop{externaldocument}{}%
%    \end{macrocode}
%
%    Most code, especially the handling of the \xfile{.aux}
%    files are taken from David Carlisle's \xpackage{xr} package.
%    Therefore I drop the documentation for these macros here.
%
%    \begin{macro}{\zref@xr@ext}
%    If the URL is not specied, then assume
%    processed file with a guessed extension.
%    Use the setting of \xpackage{hyperref} if available.
%    \begin{macrocode}
\providecommand*{\zref@xr@ext}{%
  \ltx@ifundefined{XR@ext}{pdf}{\XR@ext}%
}%
%    \end{macrocode}
%    \end{macro}
%    \begin{macro}{\ifZREF@xr@zreflabel}
%    The use of the star form of \cs{zexternaldocument}
%    is remembered in the switch \cs{ifZREF@xr@zreflabel}.
%    \begin{macrocode}
\newif\ifZREF@xr@zreflabel
%    \end{macrocode}
%    \end{macro}
%    \begin{macrocode}
\SetupKeyvalOptions{%
  family=ZREF@XR,%
  prefix=ZREF@xr@%
}
\DeclareBoolOption[true]{tozreflabel}
\DeclareBoolOption[false]{toltxlabel}
\DeclareBoolOption{verbose}
\define@key{ZREF@XR}{ext}{%
  \def\zref@xr@{#1}%
}
\DeclareBoolOption[false]{urluse}
%    \end{macrocode}
%    \begin{macro}{\zxrsetup}
%    \begin{macrocode}
\newcommand*{\zxrsetup}{%
  \kvsetkeys{ZREF@XR}%
}%
%    \end{macrocode}
%    \end{macro}
%
%    \begin{macro}{\ZREF@xr@URL}
%    \begin{macrocode}
\newcount\ZREF@xr@URL
\ZREF@xr@URL=\ltx@zero
%    \end{macrocode}
%    \end{macro}
%    \begin{macro}{\ZREF@xr@AddURL}
%    \begin{macrocode}
\def\ZREF@xr@AddURL#1{%
  \begingroup
    \def\ZREF@temp{#1}%
    \count@=\ltx@one
    \ZREF@xr@@AddUrl
  \endgroup
}
%    \end{macrocode}
%    \end{macro}
%    \begin{macro}{\ZREF@xr@@AddUrl}
%    \begin{macrocode}
\def\ZREF@xr@@AddUrl{%
  \ifnum\count@>\ZREF@xr@URL
    \global\advance\ZREF@xr@URL by\ltx@one
    \xdef\ZREF@xr@theURL{\romannumeral\ZREF@xr@URL}%
    \global\expandafter\let
        \csname Z@U@\ZREF@xr@theURL\endcsname\ZREF@temp
    \@PackageInfo{zref-xr}{%
      \ltx@backslashchar Z@U@\ZREF@xr@theURL:\MessageBreak
      \ZREF@temp\MessageBreak
    }%
  \else
    \expandafter
    \ifx\csname Z@U@\romannumeral\count@\endcsname\ZREF@temp
      \xdef\ZREF@xr@theURL{\romannumeral\count@}%
    \else
      \expandafter\expandafter\expandafter\ZREF@xr@@AddUrl
    \fi
  \fi
}
%    \end{macrocode}
%    \end{macro}
%
%    \begin{macro}{\zexternaldocument}
%    In its star form it looks for \cs{newlabel},
%    otherwise for \cs{zref@newlabel}.
%     Later we
%    will read \xfile{.aux} files that expects
%    |@| to have catcode 11 (letter).
%    \begin{macrocode}
\ZREF@IfDefinable\zexternaldocument\def{%
  {%
    \ZREF@NewPropAnchor
    \ZREF@NewPropTitle
    \begingroup
      \csname @safe@actives@true\endcsname
      \makeatletter
      \@ifstar{%
        \ZREF@xr@zreflabelfalse
        \@testopt\ZREF@xr@externaldocument{}%
      }{%
        \ZREF@xr@zreflabeltrue
        \@testopt\ZREF@xr@externaldocument{}%
      }%
  }%
}%
%    \end{macrocode}
%    \end{macro}
%
%    If the \cs{include} featuer was used, there can be
%    several \xfile{.aux} files. These files are read
%    one after another, especially they are not recursively
%    read in order to save read registers. Thus it can
%    happen that the read order of the newlabel commands
%    differs from \hologo{LaTeX}'s order using \cs{input}.
%
%    \begin{macro}{\ZREF@xr@externaldocument}
%    It reads the remaining arguments. \cs{newcommand}
%    comes in handy for the optional argument.
%    \begin{macrocode}
\def\ZREF@xr@externaldocument[#1]#2{%
    \def\ZREF@xr@prefix{#1}%
    \let\ZREF@xr@filelist\ltx@empty
    \edef\ZREF@xr@externalfile{#2}%
    \edef\ZREF@xr@file{\ZREF@xr@externalfile.aux}%
    \filename@parse{#2}%
    \@testopt\ZREF@xr@graburl{#2.\zref@xr@ext}%
}%
\def\ZREF@xr@graburl[#1]{%
    \edef\ZREF@xr@url{#1}%
    \ifZREF@xr@urluse
      \expandafter\ZREF@xr@AddURL\expandafter{\ZREF@xr@url}%
      \expandafter\def\expandafter\ZREF@xr@url
      \expandafter{\csname Z@U@\ZREF@xr@theURL\endcsname}%
    \fi
    \ZREF@xr@checkfile
  \endgroup
}%
%    \end{macrocode}
%    \end{macro}
%    \begin{macro}{\ZREF@xr@processfile}
%    We follow \xpackage{xr} here, \cs{IfFileExists} offers
%    a nicer test, but we have to open the file anyway.
%    \begin{macrocode}
\def\ZREF@xr@checkfile{%
  \openin\@inputcheck\ZREF@xr@file\relax
  \ifeof\@inputcheck
    \PackageWarning{zref-xr}{%
      File `\ZREF@xr@file' not found or empty,\MessageBreak
      labels not imported%
    }%
  \else
    \PackageInfo{zref-xr}{%
      Label \ifZREF@xr@zreflabel (zref) \fi
      import from `\ZREF@xr@file'%
    }%
    \def\ZREF@xr@found{0}%
    \def\ZREF@xr@ignored@empty{0}%
    \def\ZREF@xr@ignored@zref{0}%
    \def\ZREF@xr@ignored@ltx{0}%
    \ZREF@xr@processfile
    \closein\@inputcheck
    \begingroup
      \let\on@line\ltx@empty
      \PackageInfo{zref-xr}{%
        Statistics for `\ZREF@xr@file':\MessageBreak
        \ZREF@xr@found\space
        \ifZREF@xr@zreflabel zref\else LaTeX\fi\space
        label(s) found%
        \ifnum\ZREF@xr@ignored@empty>0 %
          ,\MessageBreak
          \ZREF@xr@ignored@empty\space empty label(s) ignored%
        \fi
        \ifnum\ZREF@xr@ignored@zref>0 %
          ,\MessageBreak
          \ZREF@xr@ignored@zref\space
          duplicated zref label(s) ignored%
        \fi
        \ifnum\ZREF@xr@ignored@ltx>0 %
          ,\MessageBreak
          \ZREF@xr@ignored@ltx\space
          duplicated latex label(s) ignored%
        \fi
      }%
    \endgroup
  \fi
  \ifx\ZREF@xr@filelist\ltx@empty
  \else
    \edef\ZREF@xr@file{%
      \expandafter\ltx@car\ZREF@xr@filelist\@nil
    }%
    \edef\ZREF@xr@filelist{%
      \expandafter\ltx@cdr\ZREF@xr@filelist\ltx@empty\@nil
    }%
    \expandafter\ZREF@xr@checkfile
  \fi
}%
%    \end{macrocode}
%    \end{macro}
%
%    \begin{macro}{\ZREF@xr@processfile}
%    \begin{macrocode}
\def\ZREF@xr@processfile{%
  \read\@inputcheck to\ZREF@xr@line
  \expandafter\ZREF@xr@processline\ZREF@xr@line..\ZREF@nil
  \ifeof\@inputcheck
  \else
    \expandafter\ZREF@xr@processfile
  \fi
}%
%    \end{macrocode}
%    \end{macro}
%    \begin{macro}{\ZREF@xr@processline}
%    The most work must be done for analyzing the
%    arguments of \cs{newlabel}.
%    \begin{macrocode}
\long\def\ZREF@xr@processline#1#2#3\ZREF@nil{%
  \def\x{#1}%
  \toks@{#2}%
  \ifZREF@xr@zreflabel
    \ifx\x\ZREF@xr@zref@newlabel
      \expandafter
      \ZREF@xr@process@zreflabel\ZREF@xr@line...\ZREF@nil
    \fi
  \else
    \ifx\x\ZREF@xr@newlabel
      \expandafter
      \ZREF@xr@process@label\ZREF@xr@line...[]\ZREF@nil
    \fi
  \fi
  \ifx\x\ZREF@xr@@input
    \edef\ZREF@xr@filelist{%
      \etex@unexpanded\expandafter{\ZREF@xr@filelist}%
      {\filename@area\the\toks@}%
    }%
  \fi
}%
\def\ZREF@xr@process@zreflabel\zref@newlabel#1#2#3\ZREF@nil{%
  \edef\ZREF@xr@refname{Z@R@\ZREF@xr@prefix#1}%
  \edef\ZREF@xr@found{\the\numexpr\ZREF@xr@found+1\relax}%
  \def\x{#2}%
  \edef\ZREF@xr@tempname{$temp$}%
  \edef\ZREF@xr@temprefname{Z@R@\ZREF@xr@tempname}%
  \let\ZREF@xr@list\x
  \ifx\ZREF@xr@list\ltx@empty
    \PackageWarningNoLine{zref-xr}{%
      Label `#1' without properties ignored\MessageBreak
      in file `\ZREF@xr@file'%
    }%
    \edef\ZREF@xr@ignored@empty{%
      \the\numexpr\ZREF@xr@ignored@empty+1\relax
    }%
  \else
    \expandafter\ZREF@xr@checklist\x\ZREF@nil
    \expandafter\let\csname\ZREF@xr@temprefname\endcsname\x
    \expandafter\ltx@LocalAppendToMacro
    \csname\ZREF@xr@temprefname\expandafter\endcsname
    \expandafter{%
      \expandafter\externaldocument\expandafter{%
        \ZREF@xr@externalfile
      }%
    }%
    \ZREF@xr@urlcheck\ZREF@xr@tempname
    \ifZREF@xr@tozreflabel
      \@ifundefined{\ZREF@xr@refname}{%
        \ifZREF@xr@verbose
          \PackageInfo{zref-xr}{%
            Import to zref label `\ZREF@xr@tempname#1'%
          }%
        \fi
        \global\expandafter
        \let\csname\ZREF@xr@refname\expandafter\endcsname
        \csname\ZREF@xr@temprefname\endcsname
      }{%
        \ZREF@xr@zref@ignorewarning{\ZREF@xr@prefix#1}%
      }%
    \fi
    \ifZREF@xr@toltxlabel
      \ZREF@xr@tolabel{\ZREF@xr@tempname}{\ZREF@xr@prefix#1}%
    \fi
  \fi
}%
\def\ZREF@xr@process@label\newlabel#1#2#3[#4]#5\ZREF@nil{%
  \def\ZREF@xr@refname{Z@R@\ZREF@xr@prefix#1}%
  \edef\ZREF@xr@found{\the\numexpr\ZREF@xr@found+1\relax}%
  \def\x{#2}%
  \edef\ZREF@xr@tempname{$temp$}%
  \edef\ZREF@xr@temprefname{Z@R@\ZREF@xr@tempname}%
  \expandafter\ZREF@xr@scanparams
      \csname\ZREF@xr@temprefname\expandafter\endcsname
      \x{}{}{}{}{}\ZREF@nil
  \ifx\\#4\\%
  \else
    % ntheorem knows an optional argument at the end of \newlabel
    \ZREF@NewPropTheotype
    \expandafter\ltx@LocalAppendToMacro
        \csname\ZREF@xr@temprefname\endcsname{\theotype{#4}}%
  \fi
  \expandafter\ltx@LocalAppendToMacro
  \csname\ZREF@xr@temprefname\expandafter\endcsname\expandafter{%
    \expandafter\externaldocument\expandafter{%
      \ZREF@xr@externalfile
    }%
  }%
  \ZREF@xr@urlcheck\ZREF@xr@tempname
  \ifZREF@xr@tozreflabel
    \@ifundefined{\ZREF@xr@refname}{%
      \ifZREF@xr@verbose
        \PackageInfo{zref-xr}{%
          Import to zref label `\ZREF@xr@prefix#1'%
        }%
      \fi
      \global\expandafter
      \let\csname\ZREF@xr@refname\expandafter\endcsname
      \csname\ZREF@xr@temprefname\endcsname
    }{%
      \ZREF@xr@zref@ignorewarning{\ZREF@xr@prefix#1}%
    }%
  \fi
  \ifZREF@xr@toltxlabel
    \ZREF@xr@tolabel{\ZREF@xr@tempname}{\ZREF@xr@prefix#1}%
  \fi
}
\def\ZREF@xr@zref@newlabel{\zref@newlabel}%
\def\ZREF@xr@newlabel{\newlabel}%
\def\ZREF@xr@@input{\@input}%
\def\ZREF@xr@relax{\relax}%
%    \end{macrocode}
%    \end{macro}
%    \begin{macro}{\ZREF@xr@tolabel}
%    \begin{macrocode}
\def\ZREF@xr@tolabel#1#2{%
  \ifZREF@xr@verbose
    \PackageInfo{zref-xr}{%
      Import to LaTeX label `#2'%
    }%
  \fi
  \zref@wrapper@unexpanded{%
    \expandafter\xdef\csname r@#2\endcsname{%
      {%
        \ltx@ifundefined{M@TitleReference}{%
          \ltx@ifundefined{TR@TitleReference}{%
            \zref@extractdefault{#1}{default}{}%
          }{%
            \noexpand\TR@TitleReference
            {\zref@extractdefault{#1}{default}{}}%
            {\zref@extractdefault{#1}{title}{}}%
          }%
        }{%
          \noexpand\M@TitleReference
          {\zref@extractdefault{#1}{default}{}}%
          {\zref@extractdefault{#1}{title}{}}%
        }%
      }%
      {\zref@extractdefault{#1}{page}{}}%
      \ltx@ifpackageloaded{nameref}{%
        {\zref@extractdefault{#1}{title}{}}%
        {\zref@extractdefault{#1}{anchor}{}}%
        \zref@ifrefcontainsprop{#1}{urluse}{%
          {\zref@extractdefault{#1}{urluse}{}}%
        }{%
          {\zref@extractdefault{#1}{url}{}}%
        }%
      }{}%
    }%
  }%
}
%    \end{macrocode}
%    \end{macro}
%
%    \begin{macro}{\ZREF@xr@zref@ignorewarning}
%    \begin{macrocode}
\def\ZREF@xr@zref@ignorewarning#1{%
  \PackageWarningNoLine{zref-xr}{%
    Zref label `#1' is already in use\MessageBreak
    in file `\ZREF@xr@file'%
  }%
  \edef\ZREF@xr@ignored@zref{%
    \the\numexpr\ZREF@xr@ignored@zref+1%
  }%
}%
%    \end{macrocode}
%    \end{macro}
%
%    \begin{macro}{\ZREF@xr@ltx@ignorewarning}
%    \begin{macrocode}
\def\ZREF@xr@ltx@ignorewarning#1{%
  \PackageWarningNoLine{zref-xr}{%
    LaTeX label `#1' is already in use\MessageBreak
    in file `\ZREF@xr@file'%
  }%
  \edef\ZREF@xr@ignored@ltx{%
    \the\numexpr\ZREF@xr@ignored@ltx+1%
  }%
}%
%    \end{macrocode}
%    \end{macro}
%
%    \begin{macro}{\ZREF@xr@checklist}
%    \begin{macrocode}
\def\ZREF@xr@checklist#1#2#3\ZREF@nil{%
  \ifx\@undefined#1\relax
    \expandafter\ZREF@xr@checkkey\string#1\@nil
  \fi
  \ifx\\#3\\%
  \else
    \ltx@ReturnAfterFi{%
      \ZREF@xr@checklist#3\ZREF@nil
    }%
  \fi
}%
\def\ZREF@xr@checkkey#1#2\@nil{%
  \zref@ifpropundefined{#2}{%
    \zref@newprop{#2}{}%
  }{}%
}%
%    \end{macrocode}
%    \end{macro}
%
%    \begin{macro}{\ZREF@xr@scanparams}
%    \begin{macrocode}
\def\ZREF@xr@scanparams#1#2#3#4#5#6#7\ZREF@nil{%
  \let#1\ltx@empty
  \ZREF@foundfalse
  \ZREF@xr@scantitleref#1#2\TR@TitleReference{}{}\ZREF@nil
  \ifZREF@found
  \else
    \ltx@LocalAppendToMacro#1{\default{#2}}%
  \fi
  % page
  \ltx@LocalAppendToMacro#1{\page{#3}}%
  % nameref title
  \ifZREF@found
  \else
    \ifx\\#4\\%
    \else
      \def\ZREF@xr@temp{#4}%
      \ifx\ZREF@xr@temp\ZREF@xr@relax
      \else
        \ltx@LocalAppendToMacro#1{\title{#4}}%
      \fi
    \fi
  \fi
  % anchor
  \ifx\\#5\\%
  \else
    \ltx@LocalAppendToMacro#1{\anchor{#5}}%
  \fi
  \ifx\\#6\\%
  \else
    \ifZREF@xr@urluse
      \ZREF@xr@AddURL{#6}%
      \expandafter\ltx@LocalAppendToMacro\expandafter#1%
      \expandafter{%
        \expandafter\urluse\expandafter{%
          \csname Z@U@\ZREF@xr@theURL\endcsname
        }%
      }%
    \else
      \ltx@LocalAppendToMacro#1{\url{#6}}%
    \fi
  \fi
}%
%    \end{macrocode}
%    \end{macro}
%
%    \begin{macro}{\ZREF@xr@scantitleref}
%    \begin{macrocode}
\def\ZREF@xr@scantitleref#1#2\TR@TitleReference#3#4#5\ZREF@nil{%
  \ifx\\#5\\%
  \else
    \ltx@LocalAppendToMacro#1{%
      \default{#3}%
      \title{#4}%
    }%
    \ZREF@foundtrue
  \fi
}%
%    \end{macrocode}
%    \end{macro}
%
%    \begin{macro}{\ZREF@xr@urlcheck}%
%    \begin{macrocode}
\def\ZREF@xr@urlcheck#1{%
  \zref@ifrefcontainsprop{#1}{anchor}{%
    \zref@ifrefcontainsprop{#1}{url}{%
    }{%
      \expandafter
      \ltx@LocalAppendToMacro\csname Z@R@#1\expandafter\endcsname
      \expandafter{%
        \csname url\ifZREF@xr@urluse use\fi
        \expandafter\endcsname\expandafter{\ZREF@xr@url}%
      }%
    }%
  }{%
  }%
}%
%    \end{macrocode}
%    \end{macro}
%
%    \begin{macrocode}
%</xr>
%    \end{macrocode}
%
%
% \subsection{Module \xmodule{hyperref}}
%
%    UNFINISHED :-(
%
%    \begin{macrocode}
%<*hyperref>
\NeedsTeXFormat{LaTeX2e}
\ProvidesPackage{zref-hyperref}%
  [2020-07-03 v2.32 Module hyperref for zref (HO)]%
\RequirePackage{zref-base}[2019/11/29]
\ifx\ZREF@base@ok Y%
\else
  \expandafter\endinput
\fi
%    \end{macrocode}
%
%    \begin{macrocode}
\ZREF@NewPropAnchor
\zref@addprop\ZREF@mainlist{anchor}%
%    \end{macrocode}
%
%    \begin{macrocode}
%</hyperref>
%    \end{macrocode}
%
%
% \subsection{Module \xmodule{savepos}}
%
%    Module \xmodule{savepos} provides an interface for
%    \pdfTeX's \cmd{\pdfsavepos}, see the manual for
%    \pdfTeX.
%
% \subsubsection{Identification}
%
%    \begin{macrocode}
%<*savepos>
\NeedsTeXFormat{LaTeX2e}
\ProvidesPackage{zref-savepos}%
  [2020-07-03 v2.32 Module savepos for zref (HO)]%
\RequirePackage{zref-base}[2019/11/29]
\ifx\ZREF@base@ok Y%
\else
  \expandafter\endinput
\fi
%    \end{macrocode}
%
% \subsubsection{Availability}
%
%    First we check, whether the feature is available.
%    \begin{macrocode}
\ifx\directlua\@undefined
\ltx@IfUndefined{pdfsavepos}{%
  \PackageError\ZREF@name{%
    \string\pdfsavepos\space is not supported.\MessageBreak
    It is provided by pdfTeX (1.40) or XeTeX%
  }\ZREF@UpdatePdfTeX
  \endinput
}{}%
\fi
%    \end{macrocode}
%    In PDF mode we are done. However support for DVI mode
%    was added later in version 1.40.0. In earlier versions
%    \cs{pdfsavepos} is defined, but its execution raises
%    an error. Note that \hologo{XeTeX} also provides \cs{pdfsavepos}.
%    \begin{macrocode}
\ifpdf
\else
  \ltx@IfUndefined{pdftexversion}{%
  }{%
    \ifnum\pdftexversion<140 %
      \PackageError\ZREF@name{%
        \string\pdfsavepos\space is not supported in DVI mode%
        \MessageBreak
        of this pdfTeX version%
      }\ZREF@UpdatePdfTeX
      \expandafter\expandafter\expandafter\endinput
    \fi
  }%
\fi
%    \end{macrocode}
%
% \subsubsection{Setup}
%
%    \begin{macrocode}
\zref@newlist{savepos}
\ifx\directlua\@undefined
  \zref@newprop*{posx}[0]{\the\pdflastxpos}
  \zref@newprop*{posy}[0]{\the\pdflastypos}
\else
  \zref@newprop*{posx}[0]{\the\lastxpos}
  \zref@newprop*{posy}[0]{\the\lastypos}
\fi
\zref@addprops{savepos}{posx,posy}
%    \end{macrocode}
%
% \subsubsection{User macros}
%
%    \begin{macro}{\zref@savepos}
%    \begin{macrocode}
\ifx\directlua\@undefined
  \def\zref@savepos{%
    \if@filesw
      \pdfsavepos
    \fi
  }
\else
  \def\zref@savepos{%
    \if@filesw
      \savepos
    \fi
  }
\fi
%    \end{macrocode}
%    \end{macro}
%
%    \begin{macro}{\ZREF@zsavepos}
%    \begin{macrocode}
\def\ZREF@zsavepos#1#2#3{%
  \@bsphack
  \if@filesw
    \zref@savepos
    #1{#3}{#2}%
    \ltx@IfUndefined{TeXXeTstate}{%
    }{%
      \ifnum\TeXXeTstate=\ltx@zero
      \else
        \zref@savepos
      \fi
    }%
  \fi
  \@esphack
}
%    \end{macrocode}
%    \end{macro}
%
%    \begin{macro}{\zsavepos}
%    The current location is stored in a reference with the
%    given name.
%    \begin{macrocode}
\ZREF@IfDefinable\zsavepos\def{%
  {%
    \ZREF@zsavepos\zref@labelbylist{savepos}%
  }%
}
%    \end{macrocode}
%    \end{macro}
%
%    \begin{macro}{\zsaveposx}
%    \begin{macrocode}
\ZREF@IfDefinable\zsaveposx\def{%
  {%
    \ZREF@zsavepos\zref@labelbyprops{posx}%
  }%
}
%    \end{macrocode}
%    \end{macro}
%    \begin{macro}{\zsaveposy}
%    \begin{macrocode}
\ZREF@IfDefinable\zsaveposy\def{%
  {%
    \ZREF@zsavepos\zref@labelbyprops{posy}%
  }%
}
%    \end{macrocode}
%    \end{macro}
%
%    \begin{macro}{\zposx}
%    \begin{macro}{\zposy}
%    The horizontal and vertical position are available by
%    \cs{zposx} and \cs{zposy}. Do not rely on absolute
%    positions. They differ in DVI and PDF mode of \pdfTeX.
%    Use differences instead. The unit of the position numbers
%    is sp.
%    \begin{macrocode}
\newcommand*{\zposx}[1]{%
  \zref@extract{#1}{posx}%
}%
\newcommand*{\zposy}[1]{%
  \zref@extract{#1}{posy}%
}%
%    \end{macrocode}
%    \end{macro}
%    \end{macro}
%
%    Typically horizontal and vertical positions are used
%    inside calculations. Therefore the extracting macros
%    should be expandable and babel's patch is not applyable.
%
%    Also it is in the responsibility of the user to marked
%    used positions by \cs{zrefused}
%    in order to notify \hologo{LaTeX} about undefined references.
%
%    \begin{macro}{\ZREF@savepos@ok}
%    \begin{macrocode}
\let\ZREF@savepos@ok=Y
%    \end{macrocode}
%    \end{macro}
%
%    \begin{macrocode}
%</savepos>
%    \end{macrocode}
%
%
% \subsection{Module \xmodule{abspos}}
%
% \subsubsection{Identification}
%
%    \begin{macrocode}
%<*abspos>
\NeedsTeXFormat{LaTeX2e}
\ProvidesPackage{zref-abspos}%
  [2020-07-03 v2.32 Module abspos for zref (HO)]%
\RequirePackage{zref-base}[2019/11/29]
\ifx\ZREF@base@ok Y%
\else
  \expandafter\endinput
\fi
%    \end{macrocode}
%
%    \begin{macrocode}
\RequirePackage{zref-savepos}[2019/11/29]
\ifx\ZREF@savepos@ok Y%
\else
  \expandafter\endinput
\fi
%    \end{macrocode}
%    \begin{macrocode}
\RequirePackage{zref-pagelayout}[2019/11/29]
\zref@addprop{savepos}{abspage}
\zref@addprop{savepos}{pagevalue}
%    \end{macrocode}
%
%    \begin{macro}{\zref@absposx}
%    \begin{macrocode}
\newcommand*{\zref@absposx}[3]{%
  \number
    \expandafter\zref@absposnumx\expandafter{%
      \number\zref@extractdefault{#1}{abspage}{0}%
    }{#2}{#3}%
  \ltx@space
}
%    \end{macrocode}
%    \end{macro}
%
%    \begin{macro}{\zref@absposy}
%    \begin{macrocode}
\newcommand*{\zref@absposy}[3]{%
  \number
  \expandafter\zref@absposnumy\expandafter{%
    \number\zref@extractdefault{#1}{abspage}{0}%
  }{#2}{#3}%
  \ltx@space
}
%    \end{macrocode}
%    \end{macro}
%
%    \begin{macro}{\zref@absposnumx}
%    \begin{macrocode}
\newcommand*{\zref@absposnumx}[3]{%
  \number
%  \ifnum#1>\ltx@zero
%    \zref@ifrefundefined{thepage#1}{%
%      0%
%    }{%
      \numexpr\ZREF@absposnum{thepage#1}{#2}{x}{#3}\relax
%    }%
%  \else
%    0%
%  \fi
}
%    \end{macrocode}
%    \end{macro}
%
%    \begin{macro}{\zref@absposnumy}
%    \begin{macrocode}
\newcommand*{\zref@absposnumy}[3]{%
  \number
%  \ifnum#1>\ltx@zero
%    \zref@ifrefundefined{thepage#1}{%
%      0%
%    }{%
      \numexpr\ZREF@absposnum{thepage#1}{#2}{y}{#3}\relax
%    }%
%  \else
%    0%
%  \fi
}
%    \end{macrocode}
%    \end{macro}
%
%    \begin{macro}{\ZREF@absposnum}
%    \begin{macrocode}
\def\ZREF@absposnum#1#2#3#4{%
  \ltx@ifundefined{ZREF@abspos@#2@#3@#4}{%
    0%
  }{%
    \csname ZREF@abspos@#2@#3@#4\endcsname{#1}%
  }%
}
%    \end{macrocode}
%    \end{macro}
%
%    \begin{macro}{\zref@def@absposx}
%    \begin{macrocode}
\ZREF@Robust\def\zref@def@absposx#1{%
  \zref@wrapper@babel{\ZREF@def@abspos{#1}\zref@absposx}%
}
%    \end{macrocode}
%    \end{macro}
%
%    \begin{macro}{\zref@def@absposy}
%    \begin{macrocode}
\ZREF@Robust\def\zref@def@absposy#1{%
  \zref@wrapper@babel{\ZREF@def@abspos{#1}\zref@absposy}%
}
%    \end{macrocode}
%    \end{macro}
%
%    \begin{macro}{\zref@def@absposnumx}
%    \begin{macrocode}
\ZREF@Robust\def\zref@def@absposnumx#1{%
  \zref@wrapper@babel{\ZREF@def@abspos{#1}\zref@absposnumx}%
}
%    \end{macrocode}
%    \end{macro}
%
%    \begin{macro}{\zref@def@absposnumy}
%    \begin{macrocode}
\ZREF@Robust\def\zref@def@absposnumy#1{%
  \zref@wrapper@babel{\ZREF@def@abspos{#1}\zref@absposnumy}%
}
%    \end{macrocode}
%    \end{macro}
%    \begin{macro}{\ZREF@def@abspos}
%    \begin{macrocode}
\def\ZREF@def@absposnumy#1#2#3#4#5{%
  \edef#1{#2{#3}{#4}{#5}}%
}
%    \end{macrocode}
%    \end{macro}
%
%    \begin{macro}{\zref@absposused}
%    \begin{macrocode}
\ZREF@Robust\def\zref@absposused{%
  \zref@wrapper@babel\ZREF@abspos@used
}
%    \end{macrocode}
%    \end{macro}
%    \begin{macro}{\ZREF@abspos@used}
%    \begin{macrocode}
\def\ZREF@abspos@used#1{%
  \zref@refused{#1}%
  \zref@ifrefundefined{#1}{%
  }{%
    \begingroup
      \edef\ZREF@temp{%
        \zref@extractdefault{#1}{abspage}{0}%
      }%
      \ifnum\ZREF@temp>\ltx@zero
        \zref@refused{thepage\ZREF@temp}%
      \else
        \@PackageError{zref-abspos}{%
          \string\zref@pos@label@used\ltx@space
          needs property `abspage'\MessageBreak
          in label `#1'%
        }\@ehc
      \fi
    \endgroup
  }%
}
%    \end{macrocode}
%    \end{macro}
%
%    \begin{macro}{\zref@absposnumused}
%    \begin{macrocode}
\newcommand*{\zref@absposnumused}[1]{%
  \ifnum#1>\ltx@zero
    \zref@refused{thepage\number#1}%
  \else
    \@PackageError{zref-abspos}{%
      Invalid absolute page number (#1)\MessageBreak
      for \string\zref@pos@num@used.\MessageBreak
      A positive integer is expected%
    }\@ehc
  \fi
}
%    \end{macrocode}
%    \end{macro}
%
%    \begin{macro}{\zref@ifabsposundefined}
%    \begin{macrocode}
\def\zref@ifabsposundefined#1{%
  \zref@ifrefundefined{#1}\ltx@firsttwo{%
    \expandafter\zref@ifabsposnumundefined\expandafter{%
      \number\zref@extractdefault{#1}{abspage}{0}%
    }%
  }%
}
%    \end{macrocode}
%    \end{macro}
%
%    \begin{macro}{\zref@ifabsposnumundefined}
%    \begin{macrocode}
\def\zref@ifabsposnumundefined#1{%
  \ifnum\ZREF@number{#1}>\ltx@zero
    \zref@ifrefundefined{thepage#1}%
        \ltx@firstoftwo\ltx@secondoftwo
  \else
    \expandafter\ltx@firstoftwo
  \fi
}
%    \end{macrocode}
%    \end{macro}
%
% \subsubsection{Media}
%
%    \begin{macro}{\ZREF@abspos@media@width}
%    \begin{macrocode}
\edef\ZREF@abspos@media@width{%
  \ltx@ifundefined{pdfpagewidth}{%
    \ltx@ifundefined{mediawidth}{%
      \ltx@ifundefined{stockwidth}{%
        paperwidth%
      }{%
        stockwidth%
      }%
    }{%
      mediawidth%
    }%
  }{%
    pdfpagewidth%
  }%
}
\ifluatex
\def\ZREF@abspos@media@width{pdfpagewidth}%
\fi
%    \end{macrocode}
%    \end{macro}
%    \begin{macro}{\ZREF@abspos@media@height}
%    \begin{macrocode}
\edef\ZREF@abspos@media@height{%
  \ltx@ifundefined{pdfpageheight}{%
    \ltx@ifundefined{mediaheight}{%
      \ltx@ifundefined{stockheight}{%
        paperheight%
      }{%
        stockheight%
      }%
    }{%
      mediaheight%
    }%
  }{%
    \noexpand\ifcase\pdfpageheight
      \ltx@ifundefined{stockheight}{%
        paperheight%
      }{%
        stockheight%
      }%
    \noexpand\else
      pdfpageheight%
    \noexpand\fi
  }%
}
\ifluatex
\edef\ZREF@abspos@media@height{%
   \noexpand\ifcase\pageheight
      \ltx@ifundefined{stockheight}{%
        paperheight%
      }{%
        stockheight%
      }%
    \noexpand\else
      pdfpageheight%
    \noexpand\fi}%
\fi
%    \end{macrocode}
%    \end{macro}
%
%    \begin{macro}{\ZREF@abspos@media@x@left}
%    \begin{macrocode}
\def\ZREF@abspos@media@x@left#1{%
  0%
}
%    \end{macrocode}
%    \end{macro}
%    \begin{macro}{\ZREF@abspos@media@x@right}
%    \begin{macrocode}
\def\ZREF@abspos@media@x@right#1{%
  \zref@extract{#1}\ZREF@abspos@media@width
}
%    \end{macrocode}
%    \end{macro}
%    \begin{macro}{\ZREF@abspos@media@x@center}
%    \begin{macrocode}
\def\ZREF@abspos@media@x@center#1{%
  \ZREF@abspos@media@x@left{#1}%
  +\zref@extract{#1}\ZREF@abspos@media@width/2%
}
%    \end{macrocode}
%    \end{macro}
%
%    \begin{macro}{\ZREF@abspos@media@y@top}
%    \begin{macrocode}
\def\ZREF@abspos@media@y@top#1{%
  \zref@extract{#1}\ZREF@abspos@media@height
}
%    \end{macrocode}
%    \end{macro}
%    \begin{macro}{\ZREF@abspos@media@y@bottom}
%    \begin{macrocode}
\def\ZREF@abspos@media@y@bottom#1{%
  0%
}
%    \end{macrocode}
%    \end{macro}
%    \begin{macro}{\ZREF@abspos@media@y@center}
%    \begin{macrocode}
\def\ZREF@abspos@media@y@center#1{%
  \zref@extract{#1}\ZREF@abspos@media@height/2%
}
%    \end{macrocode}
%    \end{macro}
%
% \subsubsection{Paper}
%
%    \begin{macro}{\ZREF@abspos@paper@x@left}
%    \begin{macrocode}
\def\ZREF@abspos@paper@x@left#1{%
  0%
}
%    \end{macrocode}
%    \end{macro}
%    \begin{macro}{\ZREF@abspos@paper@x@right}
%    \begin{macrocode}
\def\ZREF@abspos@paper@x@right#1{%
  \zref@extract{#1}{paperwidth}%
}
%    \end{macrocode}
%    \end{macro}
%    \begin{macro}{\ZREF@abspos@paper@x@center}
%    \begin{macrocode}
\def\ZREF@abspos@paper@x@center#1{%
  \zref@extract{#1}{paperwidth}/2%
}
%    \end{macrocode}
%    \end{macro}
%
%    \begin{macro}{\ZREF@abspos@paper@y@top}
%    \begin{macrocode}
\let\ZREF@abspos@paper@y@top\ZREF@abspos@media@y@top
%    \end{macrocode}
%    \end{macro}
%    \begin{macro}{\ZREF@abspos@paper@y@bottom}
%    \begin{macrocode}
\def\ZREF@abspos@paper@y@bottom#1{%
  \ZREF@abspos@paper@y@top{#1}%
  -\zref@extract{#1}{paperheight}%
}
%    \end{macrocode}
%    \end{macro}
%    \begin{macro}{\ZREF@abspos@paper@y@center}
%    \begin{macrocode}
\def\ZREF@abspos@paper@y@center#1{%
  \ZREF@abspos@paper@y@top{#1}%
  -\zref@extract{#1}{paperheight}/2%
}
%    \end{macrocode}
%    \end{macro}
%
% \subsubsection{Origin}
%
% There doesn't seem a good reason to make these tests depend on pdf mode
%  in current engines, so comment out the \verb|\ifpdf| tests.
%    \begin{macro}{\ZREF@abspos@origin@x}
%    \begin{macrocode}
\let\ZREF@temp\ltx@two
\ltx@IfUndefined{pdfhorigin}{}{%
%  \ifpdf
    \let\ZREF@temp\ltx@zero
%  \fi
}
\ifluatex
% \ifpdf
  \let\ZREF@temp\ltx@zero
% \fi
\fi

\ifx\ZREF@temp\ltx@two
  \ifnum\mag=1000 %
    \let\ZREF@temp\ltx@one
  \fi
\fi
\ifcase\ZREF@temp
  \def\ZREF@abspos@origin@x#1{%
    \zref@extract{#1}{pdfhorigin}%
  }%
\or
  \def\ZREF@abspos@origin@x#1{%
    4736286%
  }%
\or
  \def\ZREF@abspos@origin@x#1{%
    \numexpr\mag/1000*\dimexpr 1truein\relax\relax
  }%
\fi
%    \end{macrocode}
%    \end{macro}
%    \begin{macro}{\ZREF@abspos@origin@y}
%    \begin{macrocode}
\let\ZREF@temp\ltx@two
\ltx@IfUndefined{pdfvorigin}{}{%
%  \ifpdf
    \let\ZREF@temp\ltx@zero
%  \fi
}
\ifluatex
%  \ifpdf
    \let\ZREF@temp\ltx@zero
%  \fi
\fi
\ifx\ZREF@temp\ltx@two
  \ifnum\mag=1000 %
    \let\ZREF@temp\ltx@one
  \fi
\fi
\ifcase\ZREF@temp
  \def\ZREF@abspos@origin@y#1{%
    \zref@extract{#1}{pdfvorigin}%
  }%
\or
  \def\ZREF@abspos@origin@y#1{%
    4736286%
  }%
\or
  \def\ZREF@abspos@origin@y#1{%
    \numexpr\mag/1000*\dimexpr 1truein\relax\relax
   }%
\fi
%    \end{macrocode}
%    \end{macro}
%
% \subsubsection{Header}
%
%    \begin{macro}{\ZREF@abspos@head@x@left}
%    \begin{macrocode}
\def\ZREF@abspos@head@x@left#1{%
  \ZREF@abspos@paper@x@left{#1}%
  +\ZREF@abspos@origin@x{#1}%
  +\zref@extract{#1}{hoffset}%
  +\ifodd\zref@extractdefault{#1}{pagevalue}{\number\c@page} %
     \zref@extract{#1}{oddsidemargin}%
   \else
     \zref@extract{#1}{evensidemargin}%
   \fi
}
%    \end{macrocode}
%    \end{macro}
%    \begin{macro}{\ZREF@abspos@head@x@right}
%    \begin{macrocode}
\def\ZREF@abspos@head@x@right#1{%
  \ZREF@abspos@head@x@left{#1}%
  +\zref@extract{#1}{textwidth}%
}
%    \end{macrocode}
%    \end{macro}
%    \begin{macro}{\ZREF@abspos@head@x@center}
%    \begin{macrocode}
\def\ZREF@abspos@head@x@center#1{%
  \ZREF@abspos@head@x@left{#1}%
  +\zref@extract{#1}{textwidth}/2%
}
%    \end{macrocode}
%    \end{macro}
%
%    \begin{macro}{\ZREF@abspos@head@y@top}
%    \begin{macrocode}
\def\ZREF@abspos@head@y@top#1{%
  \ZREF@abspos@paper@y@top{#1}%
  -\ZREF@abspos@origin@y{#1}%
  -\zref@extract{#1}{voffset}%
  -\zref@extract{#1}{topmargin}%
}
%    \end{macrocode}
%    \end{macro}
%    \begin{macro}{\ZREF@abspos@head@y@bottom}
%    \begin{macrocode}
\def\ZREF@abspos@head@y@bottom#1{%
  \ZREF@abspos@head@y@top{#1}%
  -\zref@extract{#1}{headheight}%
}
%    \end{macrocode}
%    \end{macro}
%    \begin{macro}{\ZREF@abspos@head@y@center}
%    \begin{macrocode}
\def\ZREF@abspos@head@y@center#1{%
  \ZREF@abspos@head@y@top{#1}%
  -\zref@extract{#1}{headheight}/2%
}
%    \end{macrocode}
%    \end{macro}
%
% \subsubsection{Body}
%
%    \begin{macro}{\ZREF@abspos@body@x@left}
%    \begin{macrocode}
\let\ZREF@abspos@body@x@left\ZREF@abspos@head@x@left
%    \end{macrocode}
%    \end{macro}
%    \begin{macro}{\ZREF@abspos@body@x@right}
%    \begin{macrocode}
\let\ZREF@abspos@body@x@right\ZREF@abspos@head@x@right
%    \end{macrocode}
%    \end{macro}
%    \begin{macro}{\ZREF@abspos@body@x@center}
%    \begin{macrocode}
\let\ZREF@abspos@body@x@center\ZREF@abspos@head@x@center
%    \end{macrocode}
%    \end{macro}
%
%    \begin{macro}{\ZREF@abspos@body@y@top}
%    \begin{macrocode}
\def\ZREF@abspos@body@y@top#1{%
  \ZREF@abspos@head@y@bottom{#1}%
  -\zref@extract{#1}{headsep}%
}
%    \end{macrocode}
%    \end{macro}
%    \begin{macro}{\ZREF@abspos@body@y@bottom}
%    \begin{macrocode}
\def\ZREF@abspos@body@y@bottom#1{%
  \ZREF@abspos@body@y@top{#1}%
  -\zref@extract{#1}{textheight}%
}
%    \end{macrocode}
%    \end{macro}
%    \begin{macro}{\ZREF@abspos@body@y@center}
%    \begin{macrocode}
\def\ZREF@abspos@body@y@center#1{%
  \ZREF@abspos@body@y@top{#1}%
  -\zref@extract{#1}{textheight}/2%
}
%    \end{macrocode}
%    \end{macro}
%
% \subsubsection{Footer}
%
%    \begin{macro}{\ZREF@abspos@foot@x@left}
%    \begin{macrocode}
\let\ZREF@abspos@foot@x@left\ZREF@abspos@head@x@left
%    \end{macrocode}
%    \end{macro}
%    \begin{macro}{\ZREF@abspos@foot@x@right}
%    \begin{macrocode}
\let\ZREF@abspos@foot@x@right\ZREF@abspos@head@x@right
%    \end{macrocode}
%    \end{macro}
%    \begin{macro}{\ZREF@abspos@foot@x@center}
%    \begin{macrocode}
\let\ZREF@abspos@foot@x@center\ZREF@abspos@head@x@center
%    \end{macrocode}
%    \end{macro}
%
%    \begin{macro}{\ZREF@abspos@foot@y@bottom}
%    \begin{macrocode}
\def\ZREF@abspos@foot@y@bottom#1{%
  \ZREF@abspos@body@y@bottom{#1}%
  -\zref@extract{#1}{footskip}%
}
%    \end{macrocode}
%    \end{macro}
%
% \subsubsection{Marginal notes}
%
%    \begin{macro}{\ZREF@abspos@marginpar@x@left}
%    \begin{macrocode}
\def\ZREF@abspos@marginpar@x@left#1{%
  \ifodd\zref@extractdefault{#1}{pagevalue}{\number\c@page} %
    \ZREF@abspos@body@x@right{#1}%
    +\zref@extract{#1}{marginparsep}%
  \else
    \ZREF@abspos@body@x@left{#1}%
    -\zref@extract{#1}{marginparsep}%
    -\zref@extract{#1}{marginparwidth}%
  \fi
}
%    \end{macrocode}
%    \end{macro}
%    \begin{macro}{\ZREF@abspos@marginpar@x@right}
%    \begin{macrocode}
\def\ZREF@abspos@marginpar@x@right#1{%
  \ZREF@abspos@marginpar@x@left{#1}%
  +\zref@extract{#1}{marginparwidth}%
}
%    \end{macrocode}
%    \end{macro}
%    \begin{macro}{\ZREF@abspos@marginpar@x@center}
%    \begin{macrocode}
\def\ZREF@abspos@marginpar@x@center#1{%
  \ZREF@abspos@marginpar@x@left{#1}%
  +\zref@extract{#1}{marginparwidth}/2%
}
%    \end{macrocode}
%    \end{macro}
%
%    \begin{macro}{\ZREF@abspos@marginpar@y@top}
%    \begin{macrocode}
\let\ZREF@abspos@marginpar@y@top\ZREF@abspos@body@y@top
%    \end{macrocode}
%    \end{macro}
%    \begin{macro}{\ZREF@abspos@marginpar@y@bottom}
%    \begin{macrocode}
\let\ZREF@abspos@marginpar@y@bottom\ZREF@abspos@body@y@bottom
%    \end{macrocode}
%    \end{macro}
%    \begin{macro}{\ZREF@abspos@marginpar@y@center}
%    \begin{macrocode}
\let\ZREF@abspos@marginpar@y@center\ZREF@abspos@body@y@center
%    \end{macrocode}
%    \end{macro}
%
% \subsubsection{Stock paper}
%
%    \begin{macro}{\ZREF@abspos@stock@x@left}
%    \begin{macrocode}
\let\ZREF@abspos@stock@x@left\ZREF@abspos@paper@x@left
%    \end{macrocode}
%    \end{macro}
%    \begin{macro}{\ZREF@abspos@stock@x@right}
%    \begin{macrocode}
\let\ZREF@abspos@stock@x@right\ZREF@abspos@paper@x@right
%    \end{macrocode}
%    \end{macro}
%    \begin{macro}{\ZREF@abspos@stock@x@center}
%    \begin{macrocode}
\let\ZREF@abspos@stock@x@center\ZREF@abspos@paper@x@center
%    \end{macrocode}
%    \end{macro}
%
%    \begin{macro}{\ZREF@abspos@stock@y@top}
%    \begin{macrocode}
\let\ZREF@abspos@stock@y@top\ZREF@abspos@paper@y@top
%    \end{macrocode}
%    \end{macro}
%    \begin{macro}{\ZREF@abspos@stock@y@bottom}
%    \begin{macrocode}
\let\ZREF@abspos@stock@y@bottom\ZREF@abspos@paper@y@bottom
%    \end{macrocode}
%    \end{macro}
%    \begin{macro}{\ZREF@abspos@stock@y@center}
%    \begin{macrocode}
\let\ZREF@abspos@stock@y@center\ZREF@abspos@paper@y@center
%    \end{macrocode}
%    \end{macro}
%
%    \begin{macrocode}
%</abspos>
%    \end{macrocode}
%
%
% \subsection{Module \xmodule{dotfill}}
%
%    \begin{macrocode}
%<*dotfill>
\NeedsTeXFormat{LaTeX2e}
\ProvidesPackage{zref-dotfill}%
  [2020-07-03 v2.32 Module dotfill for zref (HO)]%
\RequirePackage{zref-base}[2019/11/29]
\ifx\ZREF@base@ok Y%
\else
  \expandafter\endinput
\fi
%    \end{macrocode}
%
%    For measuring the width of \cs{zdotfill} we use the features
%    provided by module \xmodule{savepos}.
%    \begin{macrocode}
\RequirePackage{zref-savepos}[2019/11/29]
%    \end{macrocode}
%    For automatically generated label names we use the unique
%    counter of module \xmodule{base}.
%    \begin{macrocode}
\zref@require@unique
%    \end{macrocode}
%    Configuration is done by the key value interface of package
%    \xpackage{keyval}.
%    \begin{macrocode}
\RequirePackage{keyval}
%    \end{macrocode}
%    The definitions of the keys follow.
%    \begin{macrocode}
\define@key{ZREF@DF}{unit}{%
  \def\ZREF@df@unit{#1}%
}
\define@key{ZREF@DF}{min}{%
  \def\ZREF@df@min{#1}%
}
\define@key{ZREF@DF}{dot}{%
  \def\ZREF@df@dot{#1}%
}
%    \end{macrocode}
%    Defaults are set, see user interface.
%    \begin{macrocode}
\providecommand\ZREF@df@min{2}
\providecommand\ZREF@df@unit{.44em}
\providecommand\ZREF@df@dot{.}
%    \end{macrocode}
%    \begin{macro}{\zdotfillsetup}
%    Configuration of \cs{zdotfill} is done by \cs{zdotfillsetup}.
%    \begin{macrocode}
\newcommand*{\zdotfillsetup}{\kvsetkeys{ZREF@DF}}
%    \end{macrocode}
%    \end{macro}
%
%    \begin{macro}{\zdotfill}
%    \cs{zdotfill} sets labels at the left and the right to get
%    the horizontal position. \cs{zsavepos} is not used, because
%    we do not need the vertical position.
%    \begin{macrocode}
\ZREF@IfDefinable\zdotfill\def{%
  {%
    \leavevmode
    \global\advance\c@zref@unique\ltx@one
    \begingroup
      \def\ZREF@temp{zref@\number\c@zref@unique}%
      \pdfsavepos
      \zref@labelbyprops{\thezref@unique L}{posx}%
      \setlength{\dimen@}{\ZREF@df@unit}%
      \zref@ifrefundefined{\thezref@unique R}{%
        \ZREF@dotfill
      }{%
        \ifnum\numexpr\zposx{\thezref@unique R}%
                      -\zposx{\thezref@unique L}\relax
            <\dimexpr\ZREF@df@min\dimen@\relax
          \hfill
        \else
          \ZREF@dotfill
        \fi
      }%
      \pdfsavepos
      \zref@labelbyprops{\thezref@unique R}{posx}%
    \endgroup
    \kern\z@
  }%
}
%    \end{macrocode}
%    \end{macro}
%    \begin{macro}{\ZREF@dotfill}
%    Help macro that actually sets the dots.
%    \begin{macrocode}
\def\ZREF@dotfill{%
  \cleaders\hb@xt@\dimen@{\hss\ZREF@df@dot\hss}\hfill
}
%    \end{macrocode}
%    \end{macro}
%
%    \begin{macrocode}
%</dotfill>
%    \end{macrocode}
%
% \subsection{Module \xmodule{env}}
%
%    \begin{macrocode}
%<*env>
\NeedsTeXFormat{LaTeX2e}
\ProvidesPackage{zref-env}%
  [2020-07-03 v2.32 Module env for zref (HO)]%
\RequirePackage{zref-base}[2019/11/29]
\ifx\ZREF@base@ok Y%
\else
  \expandafter\endinput
\fi
%    \end{macrocode}
%
%    \begin{macrocode}
\zref@newprop{envname}[]{\@currenvir}
\zref@newprop{envline}[]{\zref@env@line}
%    \end{macrocode}
%
%    \begin{macro}{\zref@env@line}
%    Macro \cs{zref@env@line} extracts the line number
%    from \cs{@currenvline}.
%    \begin{macrocode}
\def\zref@env@line{%
  \ifx\@currenvline\ltx@empty
  \else
    \expandafter
    \ZREF@ENV@line\@currenvline\ltx@empty line \ltx@empty\@nil
  \fi
}
%    \end{macrocode}
%    \end{macro}
%    \begin{macro}{\ZREF@ENV@line}
%    \begin{macrocode}
\def\ZREF@ENV@line#1line #2\ltx@empty#3\@nil{#2}%
%    \end{macrocode}
%    \end{macro}
%
%    \begin{macrocode}
%</env>
%    \end{macrocode}
%
%% \section{Installation}
%
% \subsection{Download}
%
% \paragraph{Package.} This package is available on
% CTAN\footnote{\CTANpkg{zref}}:
% \begin{description}
% \item[\CTAN{macros/latex/contrib/zref/zref.dtx}] The source file.
% \item[\CTAN{macros/latex/contrib/zref/zref.pdf}] Documentation.
% \end{description}
%
%
% \paragraph{Bundle.} All the packages of the bundle `zref'
% are also available in a TDS compliant ZIP archive. There
% the packages are already unpacked and the documentation files
% are generated. The files and directories obey the TDS standard.
% \begin{description}
% \item[\CTANinstall{install/macros/latex/contrib/zref.tds.zip}]
% \end{description}
% \emph{TDS} refers to the standard ``A Directory Structure
% for \TeX\ Files'' (\CTANpkg{tds}). Directories
% with \xfile{texmf} in their name are usually organized this way.
%
% \subsection{Bundle installation}
%
% \paragraph{Unpacking.} Unpack the \xfile{zref.tds.zip} in the
% TDS tree (also known as \xfile{texmf} tree) of your choice.
% Example (linux):
% \begin{quote}
%   |unzip zref.tds.zip -d ~/texmf|
% \end{quote}
%
%
% \subsection{Package installation}
%
% \paragraph{Unpacking.} The \xfile{.dtx} file is a self-extracting
% \docstrip\ archive. The files are extracted by running the
% \xfile{.dtx} through \plainTeX:
% \begin{quote}
%   \verb|tex zref.dtx|
% \end{quote}
%
% \paragraph{TDS.} Now the different files must be moved into
% the different directories in your installation TDS tree
% (also known as \xfile{texmf} tree):
% \begin{quote}
% \def\t{^^A
% \begin{tabular}{@{}>{\ttfamily}l@{ $\rightarrow$ }>{\ttfamily}l@{}}
%   zref.sty & tex/latex/zref/zref.sty\\
%   zref-base.sty & tex/latex/zref/zref-base.sty\\
%   zref-abspage.sty & tex/latex/zref/zref-abspage.sty\\
%   zref-abspos.sty & tex/latex/zref/zref-abspos.sty\\
%   zref-counter.sty & tex/latex/zref/zref-counter.sty\\
%   zref-dotfill.sty & tex/latex/zref/zref-dotfill.sty\\
%   zref-env.sty & tex/latex/zref/zref-env.sty\\
%   zref-hyperref.sty & tex/latex/zref/zref-hyperref.sty\\
%   zref-lastpage.sty & tex/latex/zref/zref-lastpage.sty\\
%   zref-marks.sty & tex/latex/zref/zref-marks.sty\\
%   zref-nextpage.sty & tex/latex/zref/zref-nextpage.sty\\
%   zref-pageattr.sty & tex/latex/zref/zref-pageattr.sty\\
%   zref-pagelayout.sty & tex/latex/zref/zref-pagelayout.sty\\
%   zref-perpage.sty & tex/latex/zref/zref-perpage.sty\\
%   zref-runs.sty & tex/latex/zref/zref-runs.sty\\
%   zref-savepos.sty & tex/latex/zref/zref-savepos.sty\\
%   zref-thepage.sty & tex/latex/zref/zref-thepage.sty\\
%   zref-titleref.sty & tex/latex/zref/zref-titleref.sty\\
%   zref-totpages.sty & tex/latex/zref/zref-totpages.sty\\
%   zref-user.sty & tex/latex/zref/zref-user.sty\\
%   zref-xr.sty & tex/latex/zref/zref-xr.sty\\
%   zref.pdf & doc/latex/zref/zref.pdf\\
%   zref-example.tex & doc/latex/zref/zref-example.tex\\
%   zref-example-lastpage.tex & doc/latex/zref/zref-example-lastpage.tex\\
%   zref-example-nextpage.tex & doc/latex/zref/zref-example-nextpage.tex\\
%   zref.dtx & source/latex/zref/zref.dtx\\
% \end{tabular}^^A
% }^^A
% \sbox0{\t}^^A
% \ifdim\wd0>\linewidth
%   \begingroup
%     \advance\linewidth by\leftmargin
%     \advance\linewidth by\rightmargin
%   \edef\x{\endgroup
%     \def\noexpand\lw{\the\linewidth}^^A
%   }\x
%   \def\lwbox{^^A
%     \leavevmode
%     \hbox to \linewidth{^^A
%       \kern-\leftmargin\relax
%       \hss
%       \usebox0
%       \hss
%       \kern-\rightmargin\relax
%     }^^A
%   }^^A
%   \ifdim\wd0>\lw
%     \sbox0{\small\t}^^A
%     \ifdim\wd0>\linewidth
%       \ifdim\wd0>\lw
%         \sbox0{\footnotesize\t}^^A
%         \ifdim\wd0>\linewidth
%           \ifdim\wd0>\lw
%             \sbox0{\scriptsize\t}^^A
%             \ifdim\wd0>\linewidth
%               \ifdim\wd0>\lw
%                 \sbox0{\tiny\t}^^A
%                 \ifdim\wd0>\linewidth
%                   \lwbox
%                 \else
%                   \usebox0
%                 \fi
%               \else
%                 \lwbox
%               \fi
%             \else
%               \usebox0
%             \fi
%           \else
%             \lwbox
%           \fi
%         \else
%           \usebox0
%         \fi
%       \else
%         \lwbox
%       \fi
%     \else
%       \usebox0
%     \fi
%   \else
%     \lwbox
%   \fi
% \else
%   \usebox0
% \fi
% \end{quote}
% If you have a \xfile{docstrip.cfg} that configures and enables \docstrip's
% TDS installing feature, then some files can already be in the right
% place, see the documentation of \docstrip.
%
% \subsection{Refresh file name databases}
%
% If your \TeX~distribution
% (\TeX\,Live, \mikTeX, \dots) relies on file name databases, you must refresh
% these. For example, \TeX\,Live\ users run \verb|texhash| or
% \verb|mktexlsr|.
%
% \subsection{Some details for the interested}
%
% \paragraph{Unpacking with \LaTeX.}
% The \xfile{.dtx} chooses its action depending on the format:
% \begin{description}
% \item[\plainTeX:] Run \docstrip\ and extract the files.
% \item[\LaTeX:] Generate the documentation.
% \end{description}
% If you insist on using \LaTeX\ for \docstrip\ (really,
% \docstrip\ does not need \LaTeX), then inform the autodetect routine
% about your intention:
% \begin{quote}
%   \verb|latex \let\install=y% \iffalse meta-comment
%
% File: zref.dtx
% Version: 2020-07-03 v2.32
% Info: A new reference scheme for LaTeX
%
% Copyright (C)
%    2006-2012  Heiko Oberdiek
%    2016-2020  Oberdiek Package Support Group
%    https://github.com/ho-tex/zref/issues
%
% This work may be distributed and/or modified under the
% conditions of the LaTeX Project Public License, either
% version 1.3c of this license or (at your option) any later
% version. This version of this license is in
%    https://www.latex-project.org/lppl/lppl-1-3c.txt
% and the latest version of this license is in
%    https://www.latex-project.org/lppl.txt
% and version 1.3 or later is part of all distributions of
% LaTeX version 2005/12/01 or later.
%
% This work has the LPPL maintenance status "maintained".
%
% The Current Maintainers of this work are
% Heiko Oberdiek and the Oberdiek Package Support Group
% https://github.com/ho-tex/zref/issues
%
% This work consists of the main source file zref.dtx
% and the derived files
%    zref.sty, zref.pdf, zref.ins, zref.drv, zref-base.sty,
%    zref-abspage.sty, zref-abspos.sty, zref-counter.sty,
%    zref-dotfill.sty, zref-env.sty, zref-hyperref.sty,
%    zref-lastpage.sty, zref-marks.sty, zref-nextpage.sty,
%    zref-pageattr.sty, zref-pagelayout.sty, zref-perpage.sty,
%    zref-runs.sty, zref-savepos.sty, zref-thepage.sty,
%    zref-titleref.sty, zref-totpages.sty, zref-user.sty,
%    zref-xr.sty, zref-example.tex, zref-example-lastpage.tex,
%    zref-example-nextpage.tex, zref-test1.tex,
%    zref-test-base.tex, zref-test-runs.tex,
%    zref-test-titleref-memoir.tex.
%
% Distribution:
%    CTAN:macros/latex/contrib/zref/zref.dtx
%    CTAN:macros/latex/contrib/zref/zref.pdf
%
% Unpacking:
%    (a) If zref.ins is present:
%           tex zref.ins
%    (b) Without zref.ins:
%           tex zref.dtx
%    (c) If you insist on using LaTeX
%           latex \let\install=y\input{zref.dtx}
%        (quote the arguments according to the demands of your shell)
%
% Documentation:
%    (a) If zref.drv is present:
%           latex zref.drv
%    (b) Without zref.drv:
%           latex zref.dtx; ...
%    The class ltxdoc loads the configuration file ltxdoc.cfg
%    if available. Here you can specify further options, e.g.
%    use A4 as paper format:
%       \PassOptionsToClass{a4paper}{article}
%
%    Programm calls to get the documentation (example):
%       pdflatex zref.dtx
%       makeindex -s gind.ist zref.idx
%       pdflatex zref.dtx
%       makeindex -s gind.ist zref.idx
%       pdflatex zref.dtx
%
% Installation:
%    TDS:tex/latex/zref/zref.sty
%    TDS:tex/latex/zref/zref-base.sty
%    TDS:tex/latex/zref/zref-abspage.sty
%    TDS:tex/latex/zref/zref-abspos.sty
%    TDS:tex/latex/zref/zref-counter.sty
%    TDS:tex/latex/zref/zref-dotfill.sty
%    TDS:tex/latex/zref/zref-env.sty
%    TDS:tex/latex/zref/zref-hyperref.sty
%    TDS:tex/latex/zref/zref-lastpage.sty
%    TDS:tex/latex/zref/zref-marks.sty
%    TDS:tex/latex/zref/zref-nextpage.sty
%    TDS:tex/latex/zref/zref-pageattr.sty
%    TDS:tex/latex/zref/zref-pagelayout.sty
%    TDS:tex/latex/zref/zref-perpage.sty
%    TDS:tex/latex/zref/zref-runs.sty
%    TDS:tex/latex/zref/zref-savepos.sty
%    TDS:tex/latex/zref/zref-thepage.sty
%    TDS:tex/latex/zref/zref-titleref.sty
%    TDS:tex/latex/zref/zref-totpages.sty
%    TDS:tex/latex/zref/zref-user.sty
%    TDS:tex/latex/zref/zref-xr.sty
%    TDS:doc/latex/zref/zref.pdf
%    TDS:doc/latex/zref/zref-example.tex
%    TDS:doc/latex/zref/zref-example-lastpage.tex
%    TDS:doc/latex/zref/zref-example-nextpage.tex
%    TDS:source/latex/zref/zref.dtx
%
%<*ignore>
\begingroup
  \catcode123=1 %
  \catcode125=2 %
  \def\x{LaTeX2e}%
\expandafter\endgroup
\ifcase 0\ifx\install y1\fi\expandafter
         \ifx\csname processbatchFile\endcsname\relax\else1\fi
         \ifx\fmtname\x\else 1\fi\relax
\else\csname fi\endcsname
%</ignore>
%<*install>
\input docstrip.tex
\Msg{************************************************************************}
\Msg{* Installation}
\Msg{* Package: zref 2020-07-03 v2.32 A new reference scheme for LaTeX (HO)}
\Msg{************************************************************************}

\keepsilent
\askforoverwritefalse

\let\MetaPrefix\relax
\preamble

This is a generated file.

Project: zref
Version: 2020-07-03 v2.32

Copyright (C)
  2006-2012 Heiko Oberdiek
  2016-2020 Oberdiek Package Support Group

This work may be distributed and/or modified under the
conditions of the LaTeX Project Public License, either
version 1.3c of this license or (at your option) any later
version. This version of this license is in
   https://www.latex-project.org/lppl/lppl-1-3c.txt
and the latest version of this license is in
   https://www.latex-project.org/lppl.txt
and version 1.3 or later is part of all distributions of
LaTeX version 2005/12/01 or later.

This work has the LPPL maintenance status "maintained".

The Current Maintainers of this work are
Heiko Oberdiek and the Oberdiek Package Support Group
https://github.com/ho-tex/zref/issues


This work consists of the main source file zref.dtx
and the derived files
   zref.sty, zref.pdf, zref.ins, zref.drv, zref-base.sty,
   zref-abspage.sty, zref-abspos.sty, zref-counter.sty,
   zref-dotfill.sty, zref-env.sty, zref-hyperref.sty,
   zref-lastpage.sty, zref-marks.sty, zref-nextpage.sty,
   zref-pageattr.sty, zref-pagelayout.sty, zref-perpage.sty,
   zref-runs.sty, zref-savepos.sty, zref-thepage.sty,
   zref-titleref.sty, zref-totpages.sty, zref-user.sty,
   zref-xr.sty, zref-example.tex, zref-example-lastpage.tex,
   zref-example-nextpage.tex, zref-test1.tex,
   zref-test-base.tex, zref-test-runs.tex,
   zref-test-titleref-memoir.tex.

\endpreamble
\let\MetaPrefix\DoubleperCent

\generate{%
  \file{zref.ins}{\from{zref.dtx}{install}}%
  \file{zref.drv}{\from{zref.dtx}{driver}}%
  \usedir{tex/latex/zref}%
  \file{zref.sty}{\from{zref.dtx}{package}}%
  \file{zref-base.sty}{\from{zref.dtx}{base}}%
  \file{zref-abspage.sty}{\from{zref.dtx}{abspage}}%
  \file{zref-abspos.sty}{\from{zref.dtx}{abspos}}%
  \file{zref-counter.sty}{\from{zref.dtx}{counter}}%
  \file{zref-dotfill.sty}{\from{zref.dtx}{dotfill}}%
  \file{zref-env.sty}{\from{zref.dtx}{env}}%
  \file{zref-hyperref.sty}{\from{zref.dtx}{hyperref}}%
  \file{zref-lastpage.sty}{\from{zref.dtx}{lastpage}}%
  \file{zref-marks.sty}{\from{zref.dtx}{marks}}%
  \file{zref-nextpage.sty}{\from{zref.dtx}{nextpage}}%
  \file{zref-pageattr.sty}{\from{zref.dtx}{pageattr}}%
  \file{zref-pagelayout.sty}{\from{zref.dtx}{pagelayout}}%
  \file{zref-perpage.sty}{\from{zref.dtx}{perpage}}%
  \file{zref-runs.sty}{\from{zref.dtx}{runs}}%
  \file{zref-savepos.sty}{\from{zref.dtx}{savepos}}%
  \file{zref-thepage.sty}{\from{zref.dtx}{thepage}}%
  \file{zref-titleref.sty}{\from{zref.dtx}{titleref}}%
  \file{zref-totpages.sty}{\from{zref.dtx}{totpages}}%
  \file{zref-user.sty}{\from{zref.dtx}{user}}%
  \file{zref-xr.sty}{\from{zref.dtx}{xr}}%
  \usedir{doc/latex/zref}%
  \file{zref-example.tex}{\from{zref.dtx}{example}}%
  \file{zref-example-lastpage.tex}{\from{zref.dtx}{example-lastpage}}%
  \file{zref-example-nextpage.tex}{\from{zref.dtx}{example-nextpage}}%
%  \usedir{doc/latex/zref/test}%
%  \file{zref-test1.tex}{\from{zref.dtx}{test1}}%
%  \file{zref-test-base.tex}{\from{zref.dtx}{test-base}}%
%  \file{zref-test-runs.tex}{\from{zref.dtx}{test-runs}}%
%  \file{zref-test-titleref-memoir.tex}{\from{zref.dtx}{test-titleref-memoir}}%
}

\catcode32=13\relax% active space
\let =\space%
\Msg{************************************************************************}
\Msg{*}
\Msg{* To finish the installation you have to move the following}
\Msg{* files into a directory searched by TeX:}
\Msg{*}
\Msg{*     zref.sty, zref-base.sty, zref-abspage.sty, zref-abspos.sty,}
\Msg{*     zref-counter.sty, zref-dotfill.sty, zref-env.sty,}
\Msg{*     zref-hyperref.sty, zref-lastpage.sty, zref-marks.sty,}
\Msg{*     zref-nextpage.sty, zref-pageattr.sty, zref-pagelayout.sty,}
\Msg{*     zref-perpage.sty, zref-runs.sty, zref-savepos.sty,}
\Msg{*     zref-thepage.sty, zref-titleref.sty, zref-totpages.sty,}
\Msg{*     zref-user.sty, zref-xr.sty}
\Msg{*}
\Msg{* To produce the documentation run the file `zref.drv'}
\Msg{* through LaTeX.}
\Msg{*}
\Msg{* Happy TeXing!}
\Msg{*}
\Msg{************************************************************************}

\endbatchfile
%</install>
%<*ignore>
\fi
%</ignore>
%<*driver>
\NeedsTeXFormat{LaTeX2e}
\ProvidesFile{zref.drv}%
  [2020-07-03 v2.32 A new reference scheme for LaTeX (HO)]%
\documentclass{ltxdoc}
\usepackage{holtxdoc}[2011/11/22]
\setlength{\emergencystretch}{1em}
\begin{document}
  \DocInput{zref.dtx}%
\end{document}
%</driver>
% \fi
%
%
%
% \GetFileInfo{zref.drv}
%
% \title{The \xpackage{zref} package}
% \date{2020-07-03 v2.32}
% \author{Heiko Oberdiek\thanks
% {Please report any issues at \url{https://github.com/ho-tex/zref/issues}}}
%
% \maketitle
%
% \begin{abstract}
% Package \xpackage{zref} tries to get rid of the restriction
% in \hologo{LaTeX}'s reference system that only two properties are
% supported. The package implements an extensible referencing
% system, where properties are handled in a more flexible way.
% It offers an interface for macro programmers for the access
% to the system and some applications that uses the new
% reference scheme.
% \end{abstract}
%
% \tableofcontents
%
% \newcommand*{\metacs}[1]{\texttt{\itshape #1}}
%
% \newcommand{\markexp}{^^A
%   \textsuperscript{exp}^^A
%   \space
% }
% \newcommand{\markexptwo}{^^A
%   \textsuperscript{exp2}^^A
%   \space
% }
% \newcommand{\markrobust}{^^A
%   \textsuperscript{robust}^^A
%   \space
% }
% \newcommand{\marknowarn}{^^A
%   \textsuperscript{nowarn}^^A
%   \space
% }
% \newcommand{\markbabel}{^^A
%   \textsuperscript{babel}^^A
%   \space
% }
%
% \section{Introduction}
%
% Standard \hologo{LaTeX}'s reference system with \cs{label}, \cs{ref},
% and \cs{pageref} supports two properties, the apperance of
% the counter that is last incremented by \cs{refstepcounter} and
% the page with the \cs{label} command.
%
% Unhappily \hologo{LaTeX} does not provide an interface for adding
% another properties. Packages such as \xpackage{hyperref},
% \xpackage{nameref}, or \xpackage{titleref} are forced to
% use ugly hacks to extend the reference system.
% These ugly hacks are one of the causes for \xpackage{hyperref}'s
% difficulty regarding compatibility with other packages.
%
% \subsection{Standard \hologo{LaTeX} behaviour}
%
% References are created by the \cs{label} command:
% \begin{quote}
%   |\chapter{Second chapter}|\\
%   |\section{First section on page 7} % section 2.1|\\
%   |\label{myref}|
% \end{quote}
% Now \hologo{LaTeX} records the section number |2.1| and the page |7|
% in the reference. Internally the reference is a list with
% two entries:
% \begin{quote}
%   |\r@myref| $\rightarrow$ |{2.1}{7}|
% \end{quote}
% The length of the list if fixed in the \hologo{LaTeX} kernel,
% An interface for adding new properties is missing.
%
% There are several tries to add new properties:
% \begin{description}
% \item[\xpackage{hyperref}] uses a list of five
%   properties instead of the standard list with two entries.
%   This causes many compatibility problems with \hologo{LaTeX} and
%   other packages.
% \item[\xpackage{titleref}] stores its title data into the
%   first entry in the list. \hologo{LaTeX} is happy because it
%   does only see its list with two entries. The situation
%   becomes more difficult, if more properties are added
%   this way. Then the macros form a nested structure
%   inside the first reference argument for the label.
%   Expandable extractions will then become painful.
% \end{description}
%
% \subsection{Basic idea}
%
% Some time ago Morten H\o gholm sent me an experimental
% cross referencing mechanism as ``expl3'' code. His
% idea is:
% \begin{quote}
%   |\g_xref_mylabel_plist| $\rightarrow$\\
%   \mbox{}\qquad|\xref_dance_key{salsa}\xref_name_key{Morten}|\dots
% \end{quote}
% The entries have the following format:
% \begin{quote}
%   |\xref_|\meta{your key}|_key{|\meta{some text}|}|
% \end{quote}
% This approach is much more flexible:
% \begin{itemize}
%   \item New properties can easily be added, just use a new key.
%   \item The length of the list is not fixed. A reference
%         can use a subset of the keys.
%   \item The order of the entries does not matter.
% \end{itemize}
% Unhappily I am not familiar with the experimental code
% for \hologo{LaTeX3} that will need some time before its first release.
% Thus I have implemented it as \hologo{LaTeXe} package without disturbing
% the existing \hologo{LaTeX} reference system.
%
% \subsection{Interfaces}
%
% The package provides a generic \emph{interface for programmers}.
% Commands of this interface are prefixed by \cs{zref@}.
%
% Option \xoption{user} enabels the \emph{user interface}.
% Here the commands are prefixed by \cs{z} to avoid name
% clashes with existing macros.
%
% Then the packages provides some \emph{modules}. They
% are applications for the reference system and can also be
% considered as examples how to use the reference system.
%
% The modules can be loaded as packages. The package
% name is prefixed with |zref-|, for example:
%\begin{quote}
%\begin{verbatim}
%\RequirePackage{zref-abspage}
%\end{verbatim}
%\end{quote}
% This is the preferred way if the package is loaded from
% within other packages to avoid option clashes.
%
% As alternative package |zref| can be used and the modules
% are given as options:
%\begin{quote}
%\begin{verbatim}
%\usepackage[perpage,user]{zref}
%\end{verbatim}
%\end{quote}
%
% \section{Interface for programmers}
%
% The user interface is described in the next section \ref{sec:user}.
%
% \subsection{Entities}
%
% \paragraph{Reference.}
% Internally a reference is a list of key value pairs:
% \begin{quote}
%   |\Z@R@myref| $\rightarrow$
%   |\default{2.1}\page{7}|
% \end{quote}
% The generic format of a entry is:
% \begin{quote}
%   |\Z@R@|\meta{refname} $\rightarrow$
%   |\|\meta{propname}|{|\meta{value}|}|
% \end{quote}
% \meta{refname} is the name that denoted references
% (the name used in \cs{label} and \cs{ref}). \meta{propname}
% is the name of the property or key. The property key macro
% is never executed, it is used in parameter text matching only.
%
% \paragraph{Property.}
% Because the name of a property is used in a macro name
% that must survive the \xfile{.aux} file, the name is
% restricted to letters and `@'.
%
% \paragraph{Property list.}
% Often references are used for special purposes. Thus it
% saves memory if just the properties are used in this
% reference that are necessary for its purpose.
%
% Therefore this package uses the concept of \emph{property lists}.
% A property list is a set of properties. The set of properties
% that is used by the default \cs{label} command is the
% \emph{main property list}.
%
% \subsection{Property list}
%
% \markexp means that the implementation of the
% marked macro is expandable. \markexptwo goes a step further
% and marks the macro expandable in exact two expansion steps.
%
% \begin{declcs}{zref@newlist}
%   \M{listname}
% \end{declcs}
% Declares a new empty property list.
%
% \begin{declcs}{zref@addprop}
%   \M{listname} \M{propname}\\
%   \cs{zref@localaddprop} \M{listname} \M{propname}
% \end{declcs}
% Adds the property \meta{propname}
% to the property list \meta{listname}.
% The property and list must exist.
% The addition is global by \cs{zref@addprop} and limited to
% local scope by \cs{zref@localaddprop}.
% Between 2010/04/19 v2.13 and 2010/10/22 v2.19 a comma separated list
% of properties could be used as argument \meta{propname}.
% Since 2010/10/22 v2.19 the addition of several properties
% at once is supported by \cs{zref@addprops}.
%
% \begin{declcs}{zref@addprops}
%   \M{listname} \M{propname list}\\
%   \cs{zref@localaddprops} \M{listname} \M{propname list}
% \end{declcs}
% These macros add a comma separated list of properties
% \meta{propname list} to list \meta{listname}.
% \cs{zref@addprops} works globally and \cs{zref@localaddprops}
% locally. Since 2010/10/22 v2.19.
%
% \begin{declcs}{zref@listexists}
%   \M{listname} \M{then}
% \end{declcs}
% Executes \meta{then} if the property list \meta{listname} exists
% or raise an error otherwise.
%
% \begin{declcs}{zref@iflistundefined}\markexp \M{listname}
%   \M{then} \M{else}
% \end{declcs}
% Executes \meta{then} if the list exists or \meta{else}
% otherwise.
%
% \begin{declcs}{zref@iflistcontainsprop}
%   \M{listname} \M{propname} \M{then} \M{else}
% \end{declcs}
% Executes \meta{then} if the property \meta{propname} is part
% of property list \meta{listname} or otherwise it
% runs the \meta{else} part.
%
% \subsection{Property}
%
% \begin{declcs}{zref@newprop}
%   \* \M{propname} |[|\meta{default}|]| \M{value}
% \end{declcs}
% This command declares and configures a new property with
% name \meta{propname}.
%
% In case of unknown references
% or the property does not exist in the reference, the
% \meta{default} is used as value. If it is not specified
% here, a global default is used, see \cs{zref@setdefault}.
%
% The correct values of some properties are not known
% immediately but at page shipout time. Prominent example
% is the page number. These properties are declared with
% the star form of the command.
%
% \begin{declcs}{zref@setcurrent}
%   \M{propname} \M{value}
% \end{declcs}
% This sets the current value of the property \meta{propname}.
% It is a generalization of setting \hologo{LaTeX}'s \cs{currentlabel}.
%
% \begin{declcs}{zref@getcurrent}\markexptwo
%   \M{propname}
% \end{declcs}
% This returns the current value of the property \meta{propname}.
% The value may not be correct, especially if the property
% is bound to a page (start form of \cs{zref@newprop}) and the
% right value is only known at shipout time (e.g. property `page').
% In case of errors (e.g. unknown property) the empty string is
% returned.
%
% Since version 2010/04/22 v2.14 \cs{zref@getcurrent} supports
% \cs{zref@wrapper@unexpanded}.
%
% \begin{declcs}{zref@propexists}
%   \M{propname} \M{then}
% \end{declcs}
% Calls \meta{then} if the property \meta{propname} is available
% or generates an error message otherwise.
%
% \begin{declcs}{zref@ifpropundefined}\markexp
%   \M{propname} \M{then} \M{else}
% \end{declcs}
% Calls \meta{then} or \meta{else} depending on the existence
% of property \meta{propname}.
%
% \subsection{Reference generation}
%
% \begin{declcs}{zref@label}
%   \M{refname}
% \end{declcs}
% This works similar to \cs{label}. The reference \meta{refname} is
% created and put into the \xfile{.aux} file with the
% properties of the main property list.
%
% \begin{declcs}{zref@labelbylist}
%   \M{refname} \M{listname}
% \end{declcs}
% Same as \cs{zref@label} except that the properties are
% taken from the specified property list \meta{listname}.
%
% \begin{declcs}{zref@labelbyprops}
%   \M{refname} |{|\meta{propnameA}|,|\meta{propnameB}|,|\dots|}|
% \end{declcs}
% Same as \cs{zref@label} except that these properties
% are used that are given as comma separated list in the
% second argument.
%
% \begin{declcs}{zref@newlabel}
%   \M{refname} |{|\dots|}|
% \end{declcs}
% This is the macro that is used in the \xfile{.aux} file.
% It is basically the same as \cs{newlabel} apart from
% the format of the data in the second argument.
%
% \subsection{Data extraction}
%
% \begin{declcs}{zref@extractdefault}\markexptwo
%   \M{refname} \M{propname} \M{default}
% \end{declcs}
% This is the basic command that refernces the value of a
% property \meta{propname} for the reference \meta{refname}.
% In case of errors such as undefined reference the \meta{default}
% is used instead.
%
% \begin{declcs}{zref@extract}\markexptwo
%   \M{refname} \M{propname}
% \end{declcs}
% The command is an abbreviation for \cs{zref@extractdefault}.
% As default the default of the property is taken, otherwise
% the global default.
%
% Example for page references:
% \begin{quote}
% \begin{tabular}{@{}ll}
% \hologo{LaTeX}:& |\pageref{foobar}|\\
% \xpackage{zref}:& |\zref@extract{foobar}{page}|
% \end{tabular}
% \end{quote}
% Both \cs{zref@extract} and \cs{zref@extractdefault} are
% expandable. That means, these macros can directly
% be used in expandable calculations, see the example file.
% On the other side, babel's shorthands are not supported,
% there are no warnings in case of undefined references.
%
% If an user interface doesn't need expandable macros
% then it can use \cs{zref@refused} and \cs{zref@wrapper@babel}
% for its user macros.
%
% \begin{declcs}{zref@refused}
%   \M{refname}
% \end{declcs}
% This command is not expandable. It causes the warnings if
% the reference \meta{refname} is not defined. Use the
% \cs{zref@extract} commands inside expandable contexts and
% mark their use outside by \cs{zref@refused}, see the example
% file.
%
% \begin{declcs}{zref@def@extract}%
%     \M{cmd} \M{refname} \M{propname}\\
%   \cs{zref@def@extractdefault}
%     \M{cmd} \M{refname} \M{propname} \M{default}
% \end{declcs}
% Both macros extract the property \meta{propname} from the reference
% \meta{refname} the same way as macros \cs{zref@extract} and
% \cs{zref@extractdefault}. The result is stored in macro \meta{cmd}.
% Also \cs{zref@refused} is called to notify \hologo{LaTeX} that
% the reference \meta{refname} is used. Added in 2011/10/04 v2.22.
%
% \begin{declcs}{zref@ifrefundefined}\markexp
%   \M{refname} \M{then} \M{else}
% \end{declcs}
% Macro \cs{zref@ifrefundefined} calls arguments \meta{then} or \meta{else}
% dependent on the existence of the reference \meta{refname}.
%
% \begin{declcs}{zifrefundefined}
%   \M{refname} \M{then} \M{else}
% \end{declcs}
% Macro \cs{zifrefundefined} calls \cs{ref@refused} before
% executing \cs{zref@ifrefundefined}. Babel shorthands are
% supported in \meta{refname}.
%
% \begin{declcs}{zref@ifrefcontainsprop}\markexp
%   \M{refname} \M{propname} \M{then} \M{else}
% \end{declcs}
% Test whether a reference provides a property.
%
% \subsection{Setup}
%
% \begin{declcs}{zref@default}
% \end{declcs}
% Holds the global default for unknown values.
%
% \begin{declcs}{zref@setdefault}
%   |{|\meta{value}|}|
% \end{declcs}
% Sets the global default for unknown values.
% The global default is used, if a property does not specify
% an own default and the value for a property
% cannot be extracted. This can happen if the reference is
% unknown or the reference does not have the property.
%
% \begin{declcs}{zref@setmainlist}
%   |{|\meta{value}|}|
% \end{declcs}
% Sets the name of the main property list. The package
% sets and uses |main|.
%
% \subsection{Declared properties}
%
% \begin{quote}
% \renewcommand*{\arraystretch}{1.05}
% \setlength{\tabcolsep}{2\tabcolsep}
% \def\VV{\vphantom{\large \"Ag}}
% \def\EMPTY{{\itshape\textless empty\textgreater}}
% \def\pl#1{&#1&thepage&\cs{number}\cs{#1}}
% \begin{tabular}{@{}>{\sffamily}l|>{\ttfamily}l|>{\ttfamily}l|>{\ttfamily}l@{}}
%   \rmfamily Module&
%   \rmfamily Property&
%   \rmfamily Property list&
%   \rmfamily Default\\
%   \hline
%   (base)
%     & default & main & \EMPTY\\
%     & page & main & \EMPTY\\
%   \hline
%   abspage
%     & abspage & main & 0\\
%   \hline
%   counter
%     & counter & main & \EMPTY\\
%   \hline
%   hyperref
%     & anchor & main & \EMPTY\\
%     & url    & & \EMPTY\\
%   \hline
%   pageattr
%     & pdfpageattr & thepage & \dots\\
%     & pdfpagesattr & LastPage & \dots\\
%   \hline
%   pagelayout\footnotemark
%     \pl{mag}\\
%     \pl{paperwidth}\\
%     \pl{paperheight}\\
%     \pl{stockwidth}\\
%     \pl{stockheight}\\
%     \pl{pdfpageheight}\\
%     \pl{pdfpagewidth}\\
%     \pl{pdfhorigin}\\
%     \pl{pdfvorigin}\\
%     \pl{hoffset}\\
%     \pl{voffset}\\
%     \pl{topmargin}\\
%     \pl{oddsidemargin}\\
%     \pl{evensidemargin}\\
%     \pl{textwidth}\\
%     \pl{textheight}\\
%     \pl{headheight}\\
%     \pl{headsep}\\
%     \pl{footskip}\\
%     \pl{marginparwidth}\\
%     \pl{marginparsep}\\
%     \pl{columnwidth}\\
%     \pl{columnsep}\\
%   \hline
%   perpage
%     & pagevalue & perpage & 0\\
%     & page & perpage & \EMPTY\\
%     & abspage & perpage & 0\\
%   \hline
%   savepos
%     & posx & savepos & 0\\
%     & posy & savepos & 0\\
%   \hline
%   titleref
%     & title & main & \EMPTY\\
%   \hline
%   xr
%     & anchor   & & \EMPTY\\
%     & externaldocument & & \EMPTY\\
%     & theotype & & \EMPTY\\
%     & title    & & \EMPTY\\
%     & url      & & \EMPTY\\
% \end{tabular}
% \footnotetext{Module \xmodule{pagelayout} only defines properties
% if the parameter exists.}%
% \end{quote}
%
% \subsection{Wrapper for advanced situations}
%
% \begin{declcs}{zref@wrapper@babel}
%   |{|\dots|}| \M{name}
% \end{declcs}
% This macro helps to add shorthand support. The second
% argument is protected, then the code of the first argument
% is called with the protected name appended. Examples
% are in the sources.
%
% \begin{declcs}{zref@wrapper@immediate}
%   |{|\dots|}|
% \end{declcs}
% There are situations where a label must be written instantly
% to the \xfile{.aux} file, for example after the last page.
% If the \cs{zlabel} or \cs{label} command is put inside this wrapper,
% immediate writing is enabled. See the implementation for
% module \xmodule{lastpage} for an example of its use.
%
% \begin{declcs}{zref@wrapper@unexpanded}
%   |{|\dots|}|
% \end{declcs}
% Assuming someone wants to extract a value for property |bar|
% and store the result in a macro |\foo| without traces of
% the expanding macros and without expanding the value.
% This (theoretical?) problem can be solved by this wrapper:
% \begin{quote}
%\begin{verbatim}
%\zref@wrapper@unexpanded{%
%  \edef\foo{%
%    \zref@extract{someref}{bar}%
%  }%
%}
%\end{verbatim}
% \end{quote}
% The \cs{edef} forces the expansion of \cs{zref@extract},
% but the extraction of the value is prevented by
% the wrapper that uses \hologo{eTeX}'\ \cs{unexpanded} for this
% purpose. Supported macros are \cs{zref@extract}, \cs{zref@extractdefault}
% and since version 2010/04/22 v2.14 macro \cs{zref@getcurrent}.
%
% \subsection{Counter for unique names}
%
% Some modules (\xmodule{titleref} and \xmodule{dotfillmin})
% need unique names for automatically generated label names.
%
% \begin{declcs}{zref@require@unique}
% \end{declcs}
% This command creates the unique counter |zref@unique| if the
% counter does not already exist.
%
% \begin{declcs}{thezref@unique}
% \end{declcs}
% This command is used to generate unique label names.
%
% \section{User interface}\label{sec:user}
%
% \subsection{Module \xmodule{user}}
%
% The user interface for this package and its modules
% is enabled by \xpackage{zref}'s package option \xmodule{user}
% or package \xpackage{zref-user}.
% The names of user commands are prefixed by |z| in order
% to avoid name clashes with existing macros of the same
% functionality. Thus the package does not disturb the
% traditional reference scheme, both can be used together.
%
% The syntax descriptions contain the following markers
% that are intended as hints for programmers:
% \begin{quote}
% \begin{tabular}{@{}ll@{}}
%   \markbabel\unskip  & Babel shorthands are allowed.\\
%   \markrobust\unskip & Robust macro.\\
%   \markexp\unskip    & Expandable version:\\
%                      & \textbullet\ robust,
%                        unless the extracted values are fragile,\\
%                      & \textbullet\ no babel shorthand suport.\\
%   \markexptwo\unskip & Expandable like \markexp and:\\
%                      & \textbullet\ expandable in exact two steps.\\
% \end{tabular}
% \end{quote}
%
% The basic user interface of the package without modules
% are commands that mimic the standard \hologo{LaTeX} behaviour
% of \cs{label}, \cs{ref}, and \cs{pageref}:
%
% \begin{declcs}{zlabel}
%   \M{refname}\markbabel
% \end{declcs}
% Similar to \cs{label}. It generates a label with name
% \meta{refname} in the new reference scheme.
%
% \begin{declcs}{zref}
%   |[|\meta{propname}|]| \M{refname}\markbabel
% \end{declcs}
% Without optional argument similar to \cs{ref}, it returns
% the default reference property. This property is named
% |default|:
% \[
%   |\zref{|x|}| \equiv |\zref[default]{|x|}|
% \]
%
% \begin{declcs}{zpageref}
%   \M{refname}\markbabel
% \end{declcs}
% Convenience macro, similar to \cs{pageref}.
% \[
%   |\zpageref{|x|}| \equiv |\zref[page]{|x|}|
% \]
%
% \begin{declcs}{zrefused}
%   \M{refname}\markbabel
% \end{declcs}
% Some of the user commands in the modules are expandable.
% The use of such commands do not cause any undefined
% reference warnings, because inside of expandable
% contexts this is not possible. However, if there is
% a place outside of expandable contexts, \cs{refused}
% is strongly recommended. The reference \meta{refname}
% is marked as used, undefined ones will generate
% warnings.
%
% \subsection{Module \xmodule{abspage}}
%
% With the help of package \xpackage{atbegshi}
% a new counter |abspage| with absolute page numbers is provided.
% For technical and historical reasons the counter itself is zero based:
% if you use it directly on the first page, e.g
% with |\arabic{abspage}| you will get 0 as value. When using |\zref|
% the first page will be page 1 as expected.
% Also a new property |abspage| is defined and added to the
% main property list. Thus you can reference the absolute
% page number:
% \begin{quote}
%   |Section \zref{foo} is on page \zpageref{foo}.|\\
%   |This is page \zref[abspage]{foo}|\\
%   |of \zref[abspage]{LastPage}.|
% \end{quote}
% The example also makes use of module \xmodule{lastpage}.
%
% \subsection{Module \xmodule{lastpage}}
%
% Provides the functionality of package \xpackage{lastpage}
% \cite{lastpage} in the new reference scheme.
% The label |LastPage| is put at the end of the document.
% You can refer the last page number with:
% \begin{quote}
%   |\zref@extract{LastPage}{page}| (+ |\zref@refused{LastPage}|)\\
% \end{quote}
% or
% \begin{quote}
%   |\zpageref{LastPage}| (module \xmodule{user})
% \end{quote}
%
% Since version 2008/10/01 v2.3 the module defines the list |LastPage|.
% In addition to the properties of the main list label |LastPage| also
% stores the properties of this list |LastPage|. The default of this
% list is empty. The list can be used by the user to add additional
% properties for label |LastPage|.
%
% \subsubsection{Tests for last page}
%
% Since version 2010/03/26 v2.8 the macros \cs{zref@iflastpage}
% and \cs{ziflastpage} were added. They test the reference,
% whether it is a reference of the last page.
%
% \begin{declcs}{zref@iflastpage}\markexp \M{refname} \M{then} \M{else}
% \end{declcs}
% Macro \cs{zref@iflastpage} compares the references \meta{refname}
% with \meta{LastPage}. Basis of the comparison is the value of
% property |abspage|, because the values are different for different
% pages. This is not ensured by property |page|. Therefore module
% \xmodule{abspage} is loaded by module \xmodule{lastpage}.
% If both values of property |abspage| are present and match,
% then \meta{then} is executed, otherwise code \meta{else} is called.
% If one or both references are undefined or lack the property |abspage|,
% then \meta{else} is executed.
%
% Macro \cs{zref@iflastpage} is expandable, therefore \cs{zref@refused}
% should be called on \meta{refname} and \meta{LastPage}.
%
% \begin{declcs}{ziflastpage} \M{refname} \M{then} \M{else}
% \end{declcs}
% Macro \cs{ziflastpage} has the same function as \cs{zref@iflastpage},
% but adds support for babel shorthands in \meta{refname} and
% calls \cs{zref@refused}. However macro \cs{ziflastpage} is not
% expandable.
%
% \subsubsection{Example}
%
%    \begin{macrocode}
%<*example-lastpage>
%<<END_EXAMPLE
\NeedsTeXFormat{LaTeX2e}
\documentclass{report}

\newcounter{foo}
\renewcommand*{\thefoo}{\Alph{foo}}

\usepackage{zref-lastpage,zref-user}[2019/11/29]

\makeatletter
\zref@newprop{thefoo}{\thefoo}
\zref@newprop{valuefoo}{\the\value{foo}}
\zref@newprop{chapter}{\thechapter}
\zref@addprops{LastPage}{thefoo,valuefoo,chapter}
\makeatother

\newcommand*{\foo}{%
  \stepcounter{foo}%
  [Current foo: \thefoo]%
}

\begin{document}
  \chapter{First chapter}
  Last page is \zref{LastPage}.\\
  Last chapter is \zref[chapter]{LastPage}.\\
  Last foo is \zref[thefoo]{LastPage}.\\
  Last value of foo is \zref[valuefoo]{LastPage}.\\
  \foo
  \chapter{Second chapter}
  \foo\foo\foo
  \chapter{Last chapter}
  \foo
\end{document}
%END_EXAMPLE
%</example-lastpage>
%    \end{macrocode}
%
% \subsection{Module \xmodule{thepage}}
%
% This module \xmodule{thepage} loads module \xmodule{abspage},
% constructs a reference name using the absolute page number
% and remembers property |page|. Other properties can be added
% by adding them to the property list |thepage|.
%
% \begin{declcs}{zthepage} \M{absolute page number}\\
% \end{declcs}
% Macro \cs{zthepage} is basically a \cs{zpageref}. The reference
% name is yield by the \meta{absolute page number}. If the reference
% is not defined, then the default for property |page| is used.
%
% \begin{declcs}{zref@thepage@name}\markexp \M{absolute page number}
% \end{declcs}
% Macro \cs{zref@thepage@name} returns the internal reference
% name that is constructed using the \meta{absolute page number}.
% The internal reference name should not be used directly, because
% it might change in future versions.
%
% \begin{declcs}{zref@thepage}\markexp \M{absolute page number}\\
%   \cs{zref@thepage@refused} \M{absolute page number}
% \end{declcs}
% Macro \cs{zref@thepage} returns the page number (\cs{thepage})
% of \meta{absolute page number}. Because this macro is expandable,
% \cs{zref@thepage@refused} is used outside an expandable context
% to mark the reference as used.
%
%
% \subsection{Module \xmodule{nextpage}}
%
% \begin{declcs}{znextpage}
% \end{declcs}
% Macro \cs{znextpage} prints \cs{thepage} of the following page.
% It gets the current absolute page number by using a label.
% There are three cases for the next page:
% \begin{enumerate}
% \item The next page is not known yet because of undefined references.
%   Then \cs{zunknownnextpagename} is used instead. The default for
%   this macro is the default of property |page|.
% \item This page is the last page. Then \cs{znonextpagename}
%   is used. Its default is empty.
% \item The next page is known, then \cs{thepage} of the next
%   page is used (the value of property |page| of the next page).
% \end{enumerate}
%
% \subsubsection{Configuration}
%
% The behaviour can be configured by the following macros.
%
% \begin{declcs}{zunknownnextpagename}\\
%   \cs{znonextpagename}
% \end{declcs}
% If the next page is not known or available, then
% \cs{znextpage} uses these name macros as default.
% \cs{zunknownnextpagename} is used in case of undefined
% references. Default is the value of property |page| of the
% next page (\cs{thepage}). Module \xmodule{thepage} is used.
%
% Macro \cs{znonextpagename} is used, if the next page does
% not exists. That means that the current page is last page.
% The default is empty.
%
% \begin{declcs}{znextpagesetup} \M{unknown} \M{no next}
%   \M{next}
% \end{declcs}
% Acording to the case (see \cs{znextpage}) macro \cs{znextpage}
% calls an internal macro with an argument. The argument is
% either \cs{thepage} of the next page or one of \cs{zunknownnextpagename}
% or \cs{znonextpagename}. These internal macro can be changed by
% \cs{znextpagesetup}. It expects the definition texts for these
% three cases of a macro with one argument.
% The default is
% \begin{quote}
%   |\znextpagesetup{#1}{#1}{#1}|
% \end{quote}
%
% \subsubsection{Example}
%
%    \begin{macrocode}
%<*example-nextpage>
%<<END_EXAMPLE
\documentclass{book}

\usepackage{zref-nextpage}[2019/11/29]
\znextpagesetup
  {\thepage}% next page is unknown
  {\thepage\ (#1)}% this page is last page
  {\thepage\ $\rightarrow$ #1}% next page is known
\renewcommand*{\znonextpagename}{last page}

\usepackage{fancyhdr}
\pagestyle{fancy}
\fancyhf{}
\fancyhead[LE,RO]{\znextpage}
\fancypagestyle{plain}{%
  \fancyhf{}%
  \fancyhead[LE,RO]{\znextpage}%
}

\begin{document}
\frontmatter
  \tableofcontents
\mainmatter
  \chapter{Hello World}
  \clearpage
  \section{Last section}
\end{document}
%END_EXAMPLE
%</example-nextpage>
%    \end{macrocode}
%
%
% \subsection{Module \xmodule{totpages}}
%
% For the total number of pages of a document you need to
% know the absolute page number of the last page.
% Both modules \xmodule{abspage} and \xmodule{lastpage}
% are necessary and automatically enabled.
%
% \begin{declcs}{ztotpages}\markexp
% \end{declcs}
% Prints the total number of pages or |0| if this
% number is not yet known. It expands to an explicit number
% and can also used even in expandable calculations (\cs{numexpr})
% or counter assignments.
%
% \subsection{Module \xmodule{pagelayout}}
%
% The module defines additional properties for each parameter
% of the page layout that is effective during page shipout.
% The value of length parameters is given in sp without the
% unit as plain number.
%
% Some parameters are specific for a class (e.g.\@ \xoption{stockwidth}
% and \xoption{stockheight} for class \xclass{memoir}) or the
% \hologo{TeX} engine like \hologo{pdfTeX}. If the parameter is not
% available, then the property will not be defined. The default value
% of the property is the current setting of the parameter.
%
% The module \xmodule{thepage} is loaded that generates a label
% for each page. The properties of module \xmodule{pagelayout}
% are added to the property list |thepage| of module \xmodule{thepage}.
%
% List of properties:
% \begin{quote}
%   \def\x#1{\cs{#1}&#1}%
%   \begin{tabular}{l>{\sffamily}ll}
%     \textrm{parameter} & \textrm{property} & remarks\\
%     \hline
%     \x{mag}\\
%     \x{paperwidth}\\
%     \x{paperheight}\\
%     \x{stockwidth}& class \xclass{memoir}\\
%     \x{stockheight}& class \xclass{memoir}\\
%     \x{pdfpagewidth}& \hologo{pdfTeX}, \hologo{LuaTeX}\\
%     \x{pdfpageheight}& \hologo{pdfTeX}, \hologo{LuaTeX}\\
%     \x{pdfhorigin}& \hologo{pdfTeX}, \hologo{LuaTeX}\\
%     \x{pdfvorigin}& \hologo{pdfTeX}, \hologo{LuaTeX}\\
%     \x{hoffset}\\
%     \x{voffset}\\
%     \x{topmargin}\\
%     \x{oddsidemargin}\\
%     \x{evensidemargin}\\
%     \x{textwidth}\\
%     \x{textheight}\\
%     \x{headheight}\\
%     \x{headsep}\\
%     \x{footskip}\\
%     \x{marginparwidth}\\
%     \x{marginparsep}\\
%     \x{columnwidth}\\
%     \x{columnsep}\\
%   \end{tabular}
% \end{quote}
%
% \begin{declcs}{zlistpagelayout}
% \end{declcs}
% At the end of document the page layout parameter for each page are
% printed into the \xext{log} file if macro \cs{zlistpagelayout}
% is called before |\end{document}| (preamble is a good place).
%
% \subsection{Module \xmodule{marks}}
%
% ToDo.
%
% \subsection{Module \xmodule{runs}}
%
% Module \xmodule{runs} counts the \hologo{LaTeX} runs since last
% \xext{aux} file creation and prints the number in the
% \xext{log} file.
%
% \begin{declcs}{zruns}\markexp
% \end{declcs}
% Prints the the total number of \hologo{LaTeX} runs including
% the current one. It expands to an explicit number.
% Before |begin{document}| the value is zero meaning
% the \xext{aux} file is not read yet. If a previous
% \xext{aux} file exists, the value found there increased
% by one is the new number. Otherwise \cs{zruns} is set to one.
% \hologo{LaTeX}~runs where the \xext{aux} files are not rewritten
% are not counted (see \cs{nofiles}).
%
% \subsection{Module \xmodule{perpage}}
%
% With \cs{@addtoreset} or \cs{numberwithin} a counter can be
% reset if another counter is incremented. This do not work
% well if the other counter is the page counter. The page counter is
% incremented in the output routine that is often called asynchronous
% somewhere on the next page. A reference mechanism costs at least
% two \hologo{LaTeX} runs, but ensures correct page counter values.
%
% \begin{declcs}{zmakeperpage}
%   |[|\meta{reset}|]| \M{counter}
% \end{declcs}
% At the of a new page counter \meta{counter} starts
% counting with value \meta{reset} (default is \texttt{1}).
% The macro has the same
% syntax and semantics as \cs{MakePerPage} of package
% \xpackage{perpage} \cite{perpage}.
% Also \xoption{perpage} of package \xpackage{footmisc} \cite{footmisc}
% can easily be simulated by
% \begin{quote}
% |\zmakeperpage{footnote}| \itshape |% \usepackage[perpage]{footmisc}|
%\end{quote}
% If footnote symbols are used, some people dislike the
% first symbol \ensuremath{\dagger}. It can easily be skipped:
% \begin{quote}
% |\zmakeperpage[2]{footnote}|
% \end{quote}
%
% \begin{declcs}{thezpage}\\
%   counter |zpage|
% \end{declcs}
% If the formatted counter value of the counter that is reset at
% a new page contains the page value, then you can use \cs{thezpage},
% the page number of the current page. Or counter |zpage| can be
% used, if the page number should be formatted differently from
% the current page number. Example:
%\begin{quote}
%\begin{verbatim}
%\newcounter{foobar}
%\zmakeperpage{foobar}
%\renewcommand*{\thefoobar}{\thezpage-\arabic{foobar}}
% % or
%\renewcommand*{\thefoobar}{\roman{zpage}-\arabic{foobar}}
%\end{verbatim}
%\end{quote}
%
% \begin{declcs}{zunmakeperpage}
%   \M{counter}
% \end{declcs}
% The reset mechanism for this counter is deactivated.
%
%
% \subsection{Module \xmodule{counter}}
%
% This option just add the property |counter| to the main
% property list. The property stores the counter name,
% that was responsible for the reference. This is the
% property \xpackage{hyperref}'s \cs{autoref} feature
% uses. Thus this property |counter| may be useful
% for a reimplementation of the autoref feature,
% see the section \ref{sec:todo} with the todo list.
%
% \subsection{Module \xmodule{titleref}}
%
% This option makes section and caption titles available
% to the reference system similar to packages \xpackage{titleref}
% or \xpackage{nameref}.
%
% \begin{declcs}{ztitleref}
%   \M{refname}\markbabel
% \end{declcs}
% Print the section or caption title of reference \meta{refname},
% similar to \cs{nameref} or \cs{titleref}.
%
% \begin{declcs}{ztitlerefsetup}
%   |{|\textit{key$_1$=value$_1$, key$_2$=value$_2$, \dots}|}|
% \end{declcs}
% This command allows to configure the behaviour of module
% \xmodule{titleref}. The following keys are available:
% \begin{description}
% \let\bfseries\relax
% \item[\texttt{title=}\meta{value}]\mbox{}\\*
%   Sets the current title.
% \item[\texttt{stripperiod=true\string|false}]\mbox{}\\*
%   Follow package \xpackage{nameref} that
%   removes a last period. Default: |true|.
% \item[\texttt{expand=true\string|false}]\mbox{}\\*
%   Package \cs{titleref} expands the title first. This way
%   garbage and dangerous commands can be removed, e.g.
%   \cs{label}, \cs{index}\dots. See implementation section
%   for more details. Default is |false|.
% \item[\texttt{cleanup=\{}\textrm{\dots}\texttt{\}}]\mbox{}\\*
%   Hook to add own cleanup code, if method |expand| is used.
%   See implementation section for more details.
% \end{description}
%
% \subsection{Module \xmodule{savepos}}
%
% This option supports a feature that \pdfTeX\ provides
% (and \hologo{XeTeX}).
% \pdfTeX\ is able to tell the current position on the
% page. The page position is not instantly known. First the
% page must be constructed by \hologo{TeX}'s asynchronous output
% routine. Thus the time where the position is known is
% the page shipout time. Thus a reference system where the
% information is recorded in the first run and made available
% for use in the second run comes in handy.
%
% \begin{declcs}{zsavepos}
%   \M{refname}
% \end{declcs}
% It generates a reference with name \meta{refname}.
% The reference stores the location where \cs{zsavepos}
% is executed in properties |posx| and |posy|.
%
% \begin{declcs}{zsaveposx}
%   \M{refname}\\
%   \cs{zsaveposy}
%   \M{refname}
% \end{declcs}
% Same as \cs{zsavepos} except that only the |x| or |y| component
% of the position is stored. Since 2011/12/05 v2.23.
%
% \begin{declcs}{zposx}\markexp
%   \M{refname}\\
%   \SpecialUsageIndex{\zposy}\cs{zposy}\markexp \M{refname}
% \end{declcs}
% Get the position as number. Unit is sp. Horizontal
% positions by \cs{zposx} increase from left to right.
% Vertical positions by \cs{zposy} from bottom to top.
%
% Do not rely on absolute page numbers. Because of problems
% with the origin the numbers may differ in DVI or PDF mode
% of \pdfTeX. Therefore work with relative values by comparisons.
%
% Both \cs{zposx} and \cs{zposy} are expandable and can be
% used inside calculations (\cs{setcounter}, \cs{addtocounter},
% package \xpackage{calc}, \cs{numexpr}). However this property
% prevents from notifying \hologo{LaTeX} that the reference is actually
% used (the notifying is not expandable). Therefore you should
% mark the reference as used by \cs{zrefused}.
%
% This module uses \pdfTeX's \cs{pdfsavepos}, \cs{pdflastxpos},
% and \cs{pdflastypos}. They are available in PDF mode and
% since version 1.40.0 also in DVI mode.
%
% \begin{declcs}{zref@savepos}
% \end{declcs}
% Macro \cs{zref@savepos} performs the first part of \cs{zsavepos}
% by calling \cs{pdfsavepos} (if \xext{aux} files are writable).
%
% Thus \cs{zsavepos} is basically \cs{zref@savepos} followed
% by |\zref@labelbylist{|\texttt{\meta{refname}}|}{savepos}|.
% If \cs{TeXXeTstate} is detected and enabled, \cs{savepos}
% also adds \cs{zref@savepos} at the end to support \cs{beginR}
% where the whatits are processed in reverse order.
% The property list |savepos| contains the properties |posx|
% and |posy|.
%
% \subsection{Module \xmodule{abspos}}
% Module \xmodule{abspos} allows to get various values of the page layout. There is no user
% command, only a number of internal commands. For example:
% \begin{quote}
%   |\zref@absposx{|\meta{label}|}{|\meta{value}|}{|\meta{position}|}|\\
%   |\zref@absposy{|\meta{label}|}{|\meta{value}|}{|\meta{position}|}|
% \end{quote}
%
% The return value is like in the module \xmodule{savepos} a number representing a length in sp.
% The length are measured from the bottom left of the page.
%
% \meta{label} is a label set with \cs{zlabel} or \cs{zsavepos} that allows to retrieve the
% absolute page number.
%
% \meta{position} is for the x-command one of |left|, |right| or |center|. For the y-command
%  it is one of |top|, |bottom|, |center|.
%
% The possible content of \meta{value} can be seen in the following table. Be aware that the code
% makes some assumptions which are perhaps not always true -- for example that the left of
% the head is identical to the left of the body.
%
% \medskip
% \begin{tabular}{lll}
%  \textbf{value} & \textbf{axis} & \textbf{comments}\\\hline
%   media         & x             &   left=0, right=\cs{pdfpagewidth} \\
%   paper         & x             &   left=0, right=\cs{paperwidth}\\
%   stock         & x             &   derived from paper\\
%   media         & y             & bottom=0, top=\cs{pdfpageheigh}\\
%   paper         & y             & top=\cs{pdfpageheight}, bottom=top-\cs{paperheight}\\
%   stock         & y             & top  derived from paper\\
%   head          & x             & calculated with hoffset, horigin, etc\\
%   head          & y             & calculated\\
%   body          & x             & = head value\\%
%   body          & y             & = head bottom - \cs{headsep}\\
%   foot          & x             & = head\\
%   foot          & y             & calculated from body bottom and \cs{footskip}\\
%   marginpar     & x             &different on odd/even pages! \\
%   marginpar     & y             & = body %
% \end{tabular}
%
% \subsection{Module \xmodule{dotfill}}
%
% \begin{declcs}{zdotfill}
% \end{declcs}
% This package provides the command \cs{zdotfill} that works
% similar to \cs{dotfill}, but can be configured. Especially
% it suppresses the dots if a minimum number of dots cannot be set.
%
% \begin{declcs}{zdotfillsetup}
%   |{|\textit{key$_1$=value$_1$, key$_2$=value$_2$, \dots}|}|
% \end{declcs}
% This command allows to configure the behaviour of \cs{zdotfill}.
% The following keys are available:
% \begin{description}
% \let\bfseries\relax
% \item[\texttt{min=}\meta{count value}]\mbox{}\\*
%   If the actual number of dots are smaller than \meta{count value},
%   then the dots are suppressed. Default: |2|.
% \item[\texttt{unit=}\meta{dimen value}]\mbox{}\\*
%   The width of a dot unit is given by \meta{dimen value}.
%   Default: |0.44em| (same as the unit in \cs{dotfill}).
% \item[\texttt{dot=}\meta{value}]\mbox{}\\*
%   The dot itself is given by \meta{value}. Default: |.| (dot, same as
%   the dot in \cs{dotfill}).
% \end{description}
%
% \subsection{Module \xmodule{env}}
%
% This module defines two properties |envname| and |envline|. They
% remember the name of the environment and the line number at
% the start of the environment.
%
% \subsection{Module \xmodule{xr}}
%
% This package provides the functionality of package \xpackage{xr},
% see \cite{xr}. It also supports the syntax of \xpackage{xr-hyper}.
%
% \begin{declcs}{zexternaldocument}
%   \* |[|\meta{prefix}|]|\markbabel \M{external document}
%   |[|\meta{url}|]|
% \end{declcs}
% See \cs{externaldocument} for a description of this option.
% The found labels also get a property |externaldocument| that
% remembers \meta{external document}.
% The standard reference scheme and the scheme of this
% package use different name spaces for reference names.
% If the external document uses both systems. Then one
% import statement would put the names in one namespace
% and probably causing problems with multiple references
% of the same name. Thus the star form only looks for
% \cs{newlabel} in the \xfile{.aux} files, whereas without
% star only \cs{zref@newlabel}s are used.
%
% In the star form it tries to detect labels from
% \xpackage{hyperref}, \xpackage{titleref}, and \xpackage{ntheorem}.
% If such an extended property from the packages before cannot
% be found or are empty, they are not included in the
% imported reference.
%
% Warnings are given if a reference name is already in use
% and the item is ignored. Unknown properties will automatically
% be declared.
%
% If the external references contain |anchor| properties,
% then we need also a url to be able to address the external
% file. As default the filename is taken with a default
% extension.
%
% \begin{declcs}{zxrsetup}
%   |{|\textit{key$_1$=value$_1$, key$_2$=value$_2$, \dots}|}|
% \end{declcs}
% The following setup options are available:
% \begin{description}
% \item[\xoption{ext}:] It sets the default extension.
% \item[\xoption{tozreflabel}:] Boolean option. The found references
%   are imported as zref labels. This is enabled by default.
% \item[\xoption{toltxlabel}:] Boolean option. The found references
%   are imported as \hologo{LaTeX} labels. Packages \xpackage{nameref},
%   \xpackage{titleref} and class \xclass{memoir} are supported.
% \item[\xoption{urluse}:] Boolean option. If enabled, then a URL
%   is stored in a macro and the macro is put in property `urluse'.
%   The URL is not put in property `url'.
%   The purpose is to save \hologo{TeX} memory.
% \item[\xoption{verbose}:] Boolean option. List the imported labels in the
%   \xext{log} file. Default is |false|.
% \end{description}
%
% \begin{declcs}{zref@xr@ext}
% \end{declcs}
% If the \meta{url} is not specified in \cs{zref@externaldocument},
% then the url will be constructed with the file name and
% this macro as extension. \cs{XR@ext} is used if
% \xpackage{hyperref} is loaded, otherwise |pdf|.
%
% \subsection{Module \xmodule{pageattr}}
%
% This module allows to recover the content of the register
% \cs{pdfpageattr} and \cs{pdfpagesattr} in pdftex and the equivalent register
% in luatex. There is no user command. Programmer commands are
%
% \begin{quote}
% |\zref@pdfpageattr{|\meta{absolute page number}|}|
%
% |\zref@pdfpagesattr{|\meta{absolute page number}|}|
% \end{quote}
% \section{ToDo}\label{sec:todo}
%
% Among other things the following issues are left for future work:
% \begin{itemize}
% \item Other applications: autoref, hyperref, \dots
% \end{itemize}
%
% \section{Example}
%
%    \begin{macrocode}
%<*example>
%    \end{macrocode}
%
%    \begin{macrocode}
\documentclass{book}

\usepackage[ngerman]{babel}%

\usepackage[savepos,totpages,titleref,dotfill,counter,user]{zref}

%    \end{macrocode}
%
%    Chapters are wrapped inside \cs{ChapterStart} and
%    \cs{ChapterStop}. The first argument \verb|#1| of \cs{ChapterStart}
%    is used to form a label id \verb|chap:#1|. At the end of
%    the chapter another label is set by \cs{zref@wrapper@immediate}, because
%    otherwise at the end of document a deferred write would not
%    be written, because there is no page for shipout.
%
%    Also this example shows how chapter titles can be recorded.
%    A new property \verb|chaptitle| is declared and added to the
%    main property list. In \cs{ChapterStart} the current value
%    of the property is updated.
%    \begin{macrocode}
\makeatletter
\zref@newprop{chaptitle}{}
\zref@addprop{main}{chaptitle}

\newcommand*{\ChapterStart}[2]{%
  \cleardoublepage
  \def\current@chapid{#1}%
  \zref@setcurrent{chaptitle}{#2}%
  \chapter{#2}%
  \zlabel{chap:#1}%
}
\newcommand*{\ChapterStop}{%
  \cleardoublepage
  \zref@wrapper@immediate{%
    \zref@labelbyprops{chapend:\current@chapid}{abspage}%
  }%
}
%    \end{macrocode}
%    \cs{ChapterPages} calculates and returns the number of pages of the
%    referenced chapter.
%    \begin{macrocode}
\newcommand*{\ChapterPages}[1]{%
  \zrefused{chap:#1}%
  \zrefused{chapend:#1}%
  \number\numexpr
    \zref@extract{chapend:#1}{abspage}%
    -\zref@extract{chap:#1}{abspage}%
    +1\relax
}
\makeatother
%    \end{macrocode}
%    \begin{macrocode}
\begin{document}
%    \end{macrocode}
%    As exception we use \cs{makeatletter} here, because
%    this is just an example file that also should show some of
%    programmer's interface.
%    \begin{macrocode}
\makeatletter

\frontmatter
\zlabel{documentstart}

\begin{itemize}
\item
  The frontmatter part has
  \number\numexpr\zref@extract{chap:first}{abspage}-1\relax
  ~pages.
\item
  Chapter \zref{chap:first} has \ChapterPages{first} page(s).
\item
  Section \zref{hello} is on the
  \ifcase\numexpr
    \zref@extractdefault{hello}{page}{0}%
    -\zref@extractdefault{chap:first}{page}{0}%
    +1\relax
    ??\or first\or second\or third\or forth\fi
  ~page inside its chapter.
\item
  The document has
  \zref[abspage]{LastPage} pages.
  This number is \ifodd\ztotpages odd\else even\fi.
\item
  The last page is labeled with \zpageref{LastPage}.
\item
  The title of chapter \zref{chap:next} %
  is ``\zref[chaptitle]{chap:next}''.
\end{itemize}

\tableofcontents

\mainmatter
\ChapterStart{first}{First chapter}

%    \end{macrocode}
%    The user level commands should protect babel shorthands
%    where possible. On the other side, expandable extracting
%    macros are useful in calculations, see above the
%    examples with \cs{numexpr}.
%    \begin{macrocode}
\section{Test}
\zlabel{a"o}
Section \zref{a"o} on page
\zref@wrapper@babel\zref@extract{a"o}{page}.

Text.
\newpage

\section{Hello World}
\zlabel{hello}

\ChapterStop

\ChapterStart{next}{Next chapter with \emph{umlauts}: "a"o"u"s}

%    \end{macrocode}
%
%    Here an example follows that makes use of \pdfTeX's ``savepos''
%    feature. The position on the page is not known before the
%    page is constructed and shipped out. Therefore the position
%    ist stored in references and are available for calculations
%    in the next \hologo{LaTeX} compile run.
%    \begin{macrocode}
The width of the first column is
  \the\dimexpr \zposx{secondcol}sp - \zposx{firstcol}sp\relax,\\
the height difference of the two baselines is
  \the\dimexpr \zposy{firstcol}sp - \zposy{secondline}sp\relax:\\
\begin{tabular}{ll}
  \zsavepos{firstcol}Hello&\zsavepos{secondcol}World\\
  \zsavepos{secondline}Second line&foobar\\
\end{tabular}

%    \end{macrocode}
%    With \cs{zrefused} \hologo{LaTeX} is notified, if the references
%    are not yet available and \hologo{LaTeX} can generate the rerun
%    hint.
%    \begin{macrocode}
\zrefused{firstcol}
\zrefused{secondcol}
\zrefused{secondline}

\ChapterStop
%    \end{macrocode}
%    Test for module \cs{dotfill}.
%    \begin{macrocode}
\ChapterStart{dotfill}{Test for dotfill feature}
\newcommand*{\dftest}[1]{%
  #1&
  [\makebox[{#1}]{\dotfill}]&
  [\makebox[{#1}]{\zdotfill}]\\
}
\begin{tabular}{rll}
& [\verb|\dotfill|] & [\verb|\zdotfill|]\\
\dftest{0.43em}
\dftest{0.44em}
\dftest{0.45em}
\dftest{0.87em}
\dftest{0.88em}
\dftest{0.89em}
\dftest{1.31em}
\dftest{1.32em}
\dftest{1.33em}
\end{tabular}
\ChapterStop
\end{document}
%    \end{macrocode}
%
%    \begin{macrocode}
%</example>
%    \end{macrocode}
%
% \StopEventually{
% }
%
% \section{Implementation}
%
% \subsection{Package \xpackage{zref}}
%
% \subsubsection{Identification}
%
%    \begin{macrocode}
%<*package>
\NeedsTeXFormat{LaTeX2e}
\ProvidesPackage{zref}
  [2020-07-03 v2.32 A new reference scheme for LaTeX (HO)]%
%    \end{macrocode}
%
% \subsubsection{Load basic module}
%
%    \begin{macrocode}
\RequirePackage{zref-base}[2019/11/29]
%    \end{macrocode}
%    Abort package loading if \xpackage{zref-base} could not be
%    loaded successfully.
%    \begin{macrocode}
\@ifundefined{ZREF@base@ok}{\endinput}{}
%    \end{macrocode}
%
% \subsubsection{Process options}
%
%    Known modules are loaded and the release date is checked.
%    \begin{macrocode}
\def\ZREF@temp#1{%
  \DeclareOption{#1}{%
    \AtEndOfPackage{%
      \RequirePackage{zref-#1}[2019/11/29]%
    }%
  }%
}
\ZREF@temp{abspage}
\ZREF@temp{counter}
\ZREF@temp{dotfill}
\ZREF@temp{hyperref}
\ZREF@temp{lastpage}
\ZREF@temp{marks}
\ZREF@temp{nextpage}
\ZREF@temp{pageattr}
\ZREF@temp{pagelayout}
\ZREF@temp{perpage}
\ZREF@temp{runs}
\ZREF@temp{savepos}
\ZREF@temp{thepage}
\ZREF@temp{titleref}
\ZREF@temp{totpages}
\ZREF@temp{user}
\ZREF@temp{xr}
%    \end{macrocode}
%    \begin{macrocode}
\ProcessOptions\relax
%</package>
%    \end{macrocode}
%
%
% \subsection{Module \xmodule{base}}
%
% \subsubsection{Prefixes}
%
%    This package uses the following prefixes for macro names:
%    \begin{description}
%    \item[\cs{zref@}:] Macros of the programmer's interface.
%    \item[\cs{ZREF@}:] Internal macros.
%    \item[\cs{Z@L@}\metacs{listname}:]
%      The properties of the list \meta{listname}.
%    \item[\cs{Z@D@}\metacs{propname}:]
%      The default value for property \meta{propname}.
%    \item[\cs{Z@E@}\metacs{propname}:]
%      Extract function for property \meta{propname}.
%    \item[\cs{Z@X@}\metacs{propname}:]
%      Information whether a property value for property
%      \meta{propname} is expanded
%      immediately or at shipout time.
%    \item[\cs{Z@C@}\metacs{propname}:]
%      Current value of the property \meta{propname}.
%    \item[\cs{Z@R@}\metacs{labelname}:]
%      Data for reference \meta{labelname}.
%    \item[\cs{ZREF@org@}:] Original versions of patched commands.
%    \item[\cs{z}:] For macros in user land, defined if
%      module \xmodule{user} is set.
%    \end{description}
%    The following family names are used for keys defined
%    according to the \xpackage{keyval} package:
%    \begin{description}
%    \item[\texttt{ZREF@TR}:] Setup for module \xmodule{titleref}.
%    \end{description}
%
% \subsubsection{Identification}
%
%    \begin{macrocode}
%<*base>
\NeedsTeXFormat{LaTeX2e}
\ProvidesPackage{zref-base}%
  [2020-07-03 v2.32 Module base for zref (HO)]%
%    \end{macrocode}
%
% \subsubsection{Utilities}
%
%    \begin{macrocode}
\providecommand\IfFormatAtLeastTF{\@ifl@t@r\fmtversion}
\RequirePackage{ltxcmds}[2010/12/02]
\RequirePackage{infwarerr}[2010/04/08]
\RequirePackage{kvsetkeys}[2010/03/01]
\RequirePackage{kvdefinekeys}[2010/03/01]
\RequirePackage{pdftexcmds}[2010/04/01]
%    \end{macrocode}
%
%    \begin{macro}{\ZREF@name}
%    Several times the package name is used, thus we store it
%    in \cmd{\ZREF@name}.
%    \begin{macrocode}
\def\ZREF@name{zref}
%    \end{macrocode}
%    \end{macro}
%
%    \begin{macrocode}
\ltx@IfUndefined{protected}{%
  \RequirePackage{makerobust}[2006/03/18]%
%    \end{macrocode}
%    \begin{macro}{\ZREF@Robust}
%    \begin{macrocode}
  \def\ZREF@Robust#1#2{%
    \def\ZREF@temp{\MakeRobustcommand#2}%
    \afterassignment\ZREF@temp
    #1#2%
  }%
%    \end{macrocode}
%    \end{macro}
%    \begin{macrocode}
}{%
%    \end{macrocode}
%    \begin{macro}{\ZREF@Robust}
%    \begin{macrocode}
  \def\ZREF@Robust#1{%
    \protected#1%
  }%
%    \end{macrocode}
%    \end{macro}
%    \begin{macrocode}
}
%    \end{macrocode}
%    \begin{macro}{\ZREF@IfDefinable}
%    \begin{macrocode}
\def\ZREF@IfDefinable#1#2#3{%
  \@ifdefinable{#1}{%
    \ZREF@Robust{#2}#1#3%
  }%
}
%    \end{macrocode}
%    \end{macro}
%
%    \begin{macro}{\ZREF@UpdatePdfTeX}
%    \cs{ZREF@UpdatePdfTeX} is used as help message text in
%    error messages.
%    \begin{macrocode}
\def\ZREF@UpdatePdfTeX{Update pdfTeX.}
%    \end{macrocode}
%    \end{macro}
%
%    \begin{macro}{\ifZREF@found}
%    The following switch is usded in list processing.
%    \begin{macrocode}
\newif\ifZREF@found
%    \end{macrocode}
%    \end{macro}
%
%    \begin{macro}{\ZREF@patch}
%    Macro \cs{ZREF@patch} first checks the existence of
%    the command and safes it.
%    \begin{macrocode}
\def\ZREF@patch#1{%
  \ltx@IfUndefined{#1}{%
    \ltx@gobble
  }{%
    \expandafter\let\csname ZREF@org@#1\expandafter\endcsname
    \csname #1\endcsname
    \ltx@firstofone
  }%
}
%    \end{macrocode}
%    \end{macro}
%
% \subsubsection{Check for \hologo{eTeX}}
%
%    The use of \hologo{eTeX} should be standard nowadays for \hologo{LaTeX}.
%    We test for \hologo{eTeX} in order to use its features later.
%    \begin{macrocode}
\ltx@IfUndefined{eTeXversion}{%
  \PackageError\ZREF@name{%
    Missing support for eTeX; package is abandoned%
  }{%
    Use a TeX compiler that support eTeX and enable eTeX %
    in the format.%
  }%
  \endinput
}{}%
%    \end{macrocode}
%
%    \begin{macrocode}
\RequirePackage{etexcmds}[2007/09/09]
\ifetex@unexpanded
\else
  \PackageError\ZREF@name{%
    Missing e-TeX's \string\unexpanded.\MessageBreak
    Add \string\RequirePackage\string{etexcmds\string} before %
    \string\documentclass%
  }{%
    Probably you are using some package (e.g. ConTeXt) that %
    redefines \string\unexpanded%
  }%
  \expandafter\endinput
\fi
%    \end{macrocode}
%
% \subsubsection{Auxiliary file stuff}
%
%    We are using some commands in the \xfile{.aux} files.
%    However sometimes these auxiliary files are interpreted
%    by \hologo{LaTeX} processes that haven't loaded this package
%    (e.g. package \xpackage{xr}).
%    Therefore we provide dummy definitions.
%    \begin{macrocode}
\RequirePackage{auxhook}
\AddLineBeginAux{%
  \string\providecommand\string\zref@newlabel[2]{}%
}
%    \end{macrocode}
%    \begin{macro}{\ZREF@RefPrefix}
%    \begin{macrocode}
\def\ZREF@RefPrefix{Z@R}
%    \end{macrocode}
%    \end{macro}
%    \begin{macro}{\zref@newlabel}
%    For the implementation of \cs{zref@newlabel} we call
%    the same internal macro \cs{@newl@bel} that is used
%    in \cs{newlabel}. Thus we have for free:
%    \begin{itemize}
%    \item \cs{Z@R@}\metacs{labelname} is defined.
%    \item \hologo{LaTeX}'s check for multiple references.
%    \item \hologo{LaTeX}'s check for changed references.
%    \end{itemize}
%    \begin{macrocode}
\ZREF@Robust\edef\zref@newlabel{%
  \noexpand\@newl@bel{\ZREF@RefPrefix}%
}
%    \end{macrocode}
%    \end{macro}
%
% \subsubsection{Property lists}
%
%    \begin{macro}{\zref@newlist}
%    Property lists are stored as list of property names
%    enclosed in curly braces. \cs{zref@newlist}
%    creates a new list as empty list. Assignments to
%    property lists are global.
%    \begin{macrocode}
\ZREF@Robust\def\zref@newlist#1{%
  \zref@iflistundefined{#1}{%
    \@ifdefinable{Z@L@#1}{%
     \global\expandafter\let\csname Z@L@#1\endcsname\ltx@empty
     \PackageInfo\ZREF@name{New property list: #1}%
    }%
  }{%
    \PackageError\ZREF@name{%
      Property list `#1' already exists%
    }\@ehc
  }%
}
%    \end{macrocode}
%    \end{macro}
%
%    \begin{macro}{\zref@iflistundefined}
%    \cs{zref@iflistundefined} checks the existence of the
%    property list \verb|#1|. If the property list is present,
%    then \verb|#2| is executed and \verb|#3| otherwise.
%    \begin{macrocode}
\def\zref@iflistundefined#1{%
  \ltx@ifundefined{Z@L@#1}%
}
%    \end{macrocode}
%    \end{macro}
%    \begin{macro}{\zref@listexists}
%    \cs{zref@listexists} only executes \verb|#2| if the
%    property list \verb|#1| exists and raises an error
%    message otherwise.
%    \begin{macrocode}
\ZREF@Robust\def\zref@listexists#1{%
  \zref@iflistundefined{#1}{%
    \PackageError\ZREF@name{%
      Property list `#1' does not exist%
    }\@ehc
  }%
}
%    \end{macrocode}
%    \end{macro}
%
%    \begin{macro}{\zref@iflistcontainsprop}
%    \cs{zref@iflistcontainsprop} checks, whether a property \verb|#2| is
%    already present in a property list \verb|#1|.
%    \begin{macrocode}
\ZREF@Robust\def\zref@iflistcontainsprop#1#2{%
  \zref@iflistundefined{#1}{%
    \ltx@secondoftwo
  }{%
    \begingroup\expandafter\endgroup
    \expandafter\in@
    \csname#2\expandafter\expandafter\expandafter\endcsname
    \expandafter\expandafter\expandafter{\csname Z@L@#1\endcsname}%
    \csname ltx@\ifin@ first\else second\fi oftwo\endcsname
  }%
}
%    \end{macrocode}
%    \end{macro}
%
%    \begin{macro}{\zref@listforloop}
%    \begin{macrocode}
\def\zref@listforloop#1#2{%
  \zref@listexists{#1}{%
    \expandafter\expandafter\expandafter\@tfor
    \expandafter\expandafter\expandafter\zref@prop
    \expandafter\expandafter\expandafter:%
    \expandafter\expandafter\expandafter=%
    \csname Z@L@#1\endcsname
    \do{%
      \begingroup
        \escapechar=-1 %
        \edef\x{\endgroup
          \def\noexpand\zref@prop{%
            \expandafter\string\zref@prop
          }%
        }%
      \x
      #2\zref@prop
    }%
  }%
}
%    \end{macrocode}
%    \end{macro}
%
%    \begin{macro}{\zref@addprops}
%    \cs{zref@addprop} adds the properties \verb|#2| to the property
%    list \verb|#1|, if the property is not already in the list.
%    Otherwise a warning is given.
%    \begin{macrocode}
\ZREF@Robust\def\zref@addprops#1#2{%
  \zref@listexists{#1}{%
    \comma@parse{#2}{%
      \zref@propexists\comma@entry{%
        \zref@iflistcontainsprop{#1}\comma@entry{%
          \PackageWarning\ZREF@name{%
            Property `\comma@entry' is already in list `#1'%
          }%
        }{%
          \begingroup\expandafter\endgroup
          \expandafter\g@addto@macro
          \csname Z@L@#1\expandafter\endcsname
          \expandafter{\csname\comma@entry\endcsname}%
        }%
      }%
      \ltx@gobble
    }%
  }%
}
%    \end{macrocode}
%    \end{macro}
%    \begin{macro}{\zref@addprop}
%    \cs{zref@addprop} adds the property \verb|#2| to the property
%    list \verb|#1|, if the property is not already in the list.
%    Otherwise a warning is given.
%    \begin{macrocode}
\ZREF@Robust\def\zref@addprop#1#2{%
  \zref@listexists{#1}{%
    \zref@propexists{#2}{%
      \zref@iflistcontainsprop{#1}{#2}{%
        \PackageWarning\ZREF@name{%
          Property `#2' is already in list `#1'%
        }%
      }{%
        \begingroup\expandafter\endgroup
        \expandafter\g@addto@macro
        \csname Z@L@#1\expandafter\endcsname
        \expandafter{\csname#2\endcsname}%
      }%
    }%
  }%
}
%    \end{macrocode}
%    \end{macro}
%
%    \begin{macro}{\zref@localaddprops}
%    \begin{macrocode}
\ZREF@Robust\def\zref@localaddprops#1#2{%
  \zref@listexists{#1}{%
    \comma@parse{#2}{%
      \zref@propexists\comma@entry{%
        \zref@iflistcontainsprop{#1}\comma@entry{%
          \PackageWarning\ZREF@name{%
            Property `\comma@entry' is already in list `#1'%
          }%
        }{%
          \begingroup\expandafter\endgroup
          \expandafter\ltx@LocalAppendToMacro
          \csname Z@L@#1\expandafter\endcsname
          \expandafter{\csname\comma@entry\endcsname}%
        }%
      }%
      \ltx@gobble
    }%
  }%
}
%    \end{macrocode}
%    \end{macro}
%    \begin{macro}{\zref@localaddprop}
%    \begin{macrocode}
\ZREF@Robust\def\zref@localaddprop#1#2{%
  \zref@listexists{#1}{%
    \zref@propexists{#2}{%
      \zref@iflistcontainsprop{#1}{#2}{%
        \PackageWarning\ZREF@name{%
          Property `#2' is already in list `#1'%
        }%
      }{%
        \begingroup\expandafter\endgroup
        \expandafter\ltx@LocalAppendToMacro
        \csname Z@L@#1\expandafter\endcsname
        \expandafter{\csname#2\endcsname}%
      }%
    }%
  }%
}
%    \end{macrocode}
%    \end{macro}
%
%    \begin{macrocode}
\ltx@IfUndefined{pdf@strcmp}{%
%    \end{macrocode}
%    \begin{macro}{\zref@delprop}
%    \begin{macrocode}
  \ZREF@Robust\def\zref@delprop{%
    \ZREF@delprop\gdef
  }%
%    \end{macrocode}
%    \end{macro}
%    \begin{macro}{\zref@localdelprop}
%    \begin{macrocode}
  \ZREF@Robust\def\zref@localdelprop{%
    \ZREF@delprop\def
  }%
%    \end{macrocode}
%    \end{macro}
%    \begin{macro}{\ZREF@delprop}
%    \begin{macrocode}
  \def\ZREF@delprop#1#2#3{%
    \zref@listexists{#2}{%
      \begingroup
        \escapechar=-1 %
        \def\ZREF@param{#3}%
        \@onelevel@sanitize\ZREF@param
        \toks@{}%
        \expandafter\expandafter\expandafter\ZREF@@delprop
        \csname Z@L@#2\endcsname!%
      \expandafter\endgroup
      \expandafter#1\csname Z@L@#2\expandafter\endcsname
      \expandafter{%
        \the\toks@
      }%
    }%
  }%
%    \end{macrocode}
%    \end{macro}
%    \begin{macro}{\ZREF@@delprop}
%    \begin{macrocode}
  \def\ZREF@@delprop#1{%
    \expandafter\ZREF@@@delprop\expandafter{\string#1}#1%
  }%
%    \end{macrocode}
%    \end{macro}
%    \begin{macro}{\ZREF@@@delprop}
%    \begin{macrocode}
  \def\ZREF@@@delprop#1#2{%
    \ifx#2!%
    \else
      \def\ZREF@temp{#1}%
      \@onelevel@sanitize\ZREF@temp
      \ifx\ZREF@param\ZREF@temp
      \else
        \toks@\expandafter{%
          \the\expandafter\toks@\csname#1\endcsname
        }%
      \fi
      \expandafter\ZREF@@delprop
    \fi
  }%
%    \end{macrocode}
%    \end{macro}
%    \begin{macrocode}
}{%
%    \end{macrocode}
%    \begin{macro}{\zref@delprop}
%    \begin{macrocode}
  \ZREF@Robust\def\zref@delprop{%
    \ZREF@delprop\xdef
  }%
%    \end{macrocode}
%    \end{macro}
%    \begin{macro}{\zref@localdelprop}
%    \begin{macrocode}
  \ZREF@Robust\def\zref@localdelprop{%
    \ZREF@delprop\edef
  }%
%    \end{macrocode}
%    \end{macro}
%    \begin{macro}{\ZREF@delprop}
%    \begin{macrocode}
  \def\ZREF@delprop#1#2#3{%
    \zref@listexists{#2}{%
      \def\ZREF@param{#3}%
      \edef\ZREF@SavedEscapechar{\the\escapechar}%
      \escapechar=-1 %
      \expandafter#1\csname Z@L@#2%
      \expandafter\expandafter\expandafter\endcsname{%
        \expandafter\expandafter\expandafter\ZREF@@delprop
        \csname Z@L@#2\endcsname!%
      }%
      \escapechar=\ZREF@SavedEscapechar\relax
    }%
  }%
%    \end{macrocode}
%    \end{macro}
%    \begin{macro}{\ZREF@@delprop}
%    Caution: |#1| might be an \cs{if} or similar token.
%    \begin{macrocode}
  \def\ZREF@@delprop#1{%
    \expandafter\ZREF@@@delprop\expandafter{\string#1}#1%
  }%
%    \end{macrocode}
%    \end{macro}
%    \begin{macro}{\ZREF@@@delprop}
%    \begin{macrocode}
  \def\ZREF@@@delprop#1#2{%
    \ifx#2!%
    \else
      \ifnum\pdf@strcmp{#1}{\ZREF@param}=\ltx@zero
      \else
        \expandafter\noexpand\csname#1\endcsname
      \fi
      \expandafter\ZREF@@delprop
    \fi
  }%
%    \end{macrocode}
%    \end{macro}
%    \begin{macrocode}
}
%    \end{macrocode}
%
% \subsubsection{Properties}
%
%    \begin{macro}{\zref@ifpropundefined}
%    \cs{zref@ifpropundefined} checks the existence of the
%    property \verb|#1|. If the property is present,
%    then \verb|#2| is executed and \verb|#3| otherwise.
%    \begin{macrocode}
\def\zref@ifpropundefined#1{%
  \ltx@ifundefined{Z@E@#1}%
}
%    \end{macrocode}
%    \end{macro}
%    \begin{macro}{\zref@propexists}
%    Some macros rely on the existence of a property.
%    \cs{zref@propexists} only executes \verb|#2| if the
%    property \verb|#1| exists and raises an error
%    message otherwise.
%    \begin{macrocode}
\ZREF@Robust\def\zref@propexists#1{%
  \zref@ifpropundefined{#1}{%
    \PackageError\ZREF@name{%
      Property `#1' does not exist%
    }\@ehc
  }%
}
%    \end{macrocode}
%    \end{macro}
%
%    \begin{macro}{\zref@newprop}
%    A new property is declared by \cs{zref@newprop}, the
%    property name \meta{propname} is given in \verb|#1|. The
%    property is created and configured. If the star form
%    is given, then the expansion of the property value
%    is delayed to page shipout time, when the reference
%    is written to the \xfile{.aux} file.
%    \begin{description}
%    \item[\cs{Z@D@}\metacs{propname}:]
%      Stores the default value for this property.
%    \item[\cs{Z@E@}\metacs{propname}:]
%      Extract function.
%    \item[\cs{Z@X@}\metacs{propname}:]
%      Information whether the expansion of the property value
%      is delayed to shipout time.
%    \item[\cs{Z@C@}\metacs{propname}:]
%      Current value of the property.
%    \end{description}
%    \begin{macrocode}
\ZREF@Robust\def\zref@newprop{%
  \@ifstar{%
    \let\ZREF@X\noexpand
    \ZREF@newprop
  }{%
    \let\ZREF@X\ltx@empty
    \ZREF@newprop
  }%
}
%    \end{macrocode}
%    \end{macro}
%    \begin{macro}{\ZREF@newprop}
%    \begin{macrocode}
\def\ZREF@newprop#1{%
  \edef\ZREF@P{#1}%
  \@onelevel@sanitize\ZREF@P
  \begingroup
  \ifx\ZREF@P\ZREF@par
    \@PackageError\ZREF@name{%
      Invalid property name `\ZREF@P'%
    }{%
      The property name `par' is not allowed %
      because of internal reasons.%
      \MessageBreak
      \@ehc
    }%
    \def\ZREF@@newprop[##1]##2{\endgroup}%
  \else
    \zref@ifpropundefined\ZREF@P{%
      \endgroup
      \PackageInfo\ZREF@name{%
        New property: \ZREF@P
      }%
    }{%
      \@PackageError\ZREF@name{%
        Property `\ZREF@P' already exists%
      }\@ehc
      \def\ZREF@@newprop[##1]##2{\endgroup}%
    }%
  \fi
  \@ifnextchar[\ZREF@@newprop{\ZREF@@newprop[\zref@default]}%
}
%    \end{macrocode}
%    \end{macro}
%    \begin{macro}{\ZREF@par}
%    \begin{macrocode}
\def\ZREF@par{par}
\@onelevel@sanitize\ZREF@par
%    \end{macrocode}
%    \end{macro}
%    \begin{macro}{\ZREF@@newprop}
%    \begin{macrocode}
\def\ZREF@@newprop[#1]{%
  \global\@namedef{Z@D@\ZREF@P}{#1}%
  \global\expandafter\let\csname Z@X@\ZREF@P\endcsname\ZREF@X
  \begingroup\expandafter\endgroup
  \expandafter\ZREF@@@newprop\csname\ZREF@P\endcsname
  \expandafter\gdef\csname Z@C@\ZREF@P\endcsname{}%
  \zref@setcurrent\ZREF@P
}
\def\ZREF@@@newprop#1{%
  \expandafter
  \gdef\csname Z@E@\ZREF@P\endcsname##1#1##2##3\ZREF@nil{##2}%
}
%    \end{macrocode}
%    \end{macro}
%
%    \begin{macro}{\zref@showprop}
%    \begin{macrocode}
\ZREF@Robust\def\zref@showprop#1{%
  \zref@ifpropundefined{#1}{%
    \@PackageInfoNoLine{\ZREF@name}{%
      Show property `#1': <undefined>%
    }%
  }{%
    \begingroup
      \toks@\expandafter\expandafter\expandafter{%
        \csname Z@C@#1\endcsname
      }%
      \edef\ZREF@value{\the\toks@}%
      \ltx@onelevel@sanitize\ZREF@value
      \toks@\expandafter\expandafter\expandafter{%
        \csname Z@D@#1\endcsname
      }%
      \edef\ZREF@default{\the\toks@}%
      \ltx@onelevel@sanitize\ZREF@default
      \@PackageInfoNoLine{\ZREF@name}{%
        Show property `#1':\MessageBreak
        \expandafter\ifx\csname Z@X@#1\endcsname\ltx@empty
          Immediate %
        \else
          Delayed %
        \fi
        value: [\ZREF@value]\MessageBreak
        Default: [\ZREF@default]%
      }%
    \endgroup
  }%
}
%    \end{macrocode}
%    \end{macro}
%
%    \begin{macro}{\zref@setcurrent}
%    \cs{zref@setcurrent} sets the current value for a property.
%    \begin{macrocode}
\ZREF@Robust\def\zref@setcurrent#1#2{%
  \zref@propexists{#1}{%
    \expandafter\def\csname Z@C@#1\endcsname{#2}%
  }%
}
%    \end{macrocode}
%    \end{macro}
%
%    \begin{macro}{\ZREF@getcurrent}
%    \cs{zref@getcurrent} gets the current value for a property.
%    \begin{macrocode}
\def\ZREF@getcurrent#1{%
  \romannumeral0%
  \ltx@ifundefined{Z@C@#1}{%
    \ltx@space
  }{%
    \expandafter\expandafter\expandafter\ltx@space
    \csname Z@C@#1\endcsname
  }%
}
%    \end{macrocode}
%    \end{macro}
%    \begin{macro}{\ZREF@u@getcurrent}
%    \begin{macrocode}
\def\ZREF@wu@getcurrent#1{%
  \etex@unexpanded\expandafter\expandafter\expandafter{%
    \ZREF@getcurrent{#1}%
  }%
}
%    \end{macrocode}
%    \end{macro}
%    \begin{macro}{\zref@getcurrent}
%    \begin{macrocode}
\let\zref@getcurrent\ZREF@getcurrent
%    \end{macrocode}
%    \end{macro}
%
% \subsubsection{Reference generation}
%
%    \begin{macro}{\zref@label}
%    Label macro that uses the main property list.
%    \begin{macrocode}
\ZREF@Robust\def\zref@label#1{%
  \zref@labelbylist{#1}\ZREF@mainlist
}
%    \end{macrocode}
%    \end{macro}
%    \begin{macro}{\zref@labelbylist}
%    Label macro that stores the properties, specified in the
%    property list \verb|#2|.
%    \begin{macrocode}
\ZREF@Robust\def\zref@labelbylist#1#2{%
  \@bsphack
    \zref@listexists{#2}{%
      \expandafter\expandafter\expandafter\ZREF@label
      \expandafter\expandafter\expandafter{%
        \csname Z@L@#2\endcsname
      }{#1}%
    }%
  \@esphack
}
%    \end{macrocode}
%    \end{macro}
%    \begin{macro}{\zref@labelbyprops}
%    The properties are directly specified in a comma separated list.
%    \begin{macrocode}
\ZREF@Robust\def\zref@labelbyprops#1#2{%
  \@bsphack
    \begingroup
      \toks@{}%
      \comma@parse{#2}{%
        \zref@ifpropundefined\comma@entry{%
          \PackageWarning\ZREF@name{%
            Property `\comma@entry' is not known%
          }%
        }{%
          \toks@\expandafter{%
            \the\expandafter\toks@\csname\comma@entry\endcsname
          }%
        }%
        \ltx@gobble
      }%
    \expandafter\endgroup
    \expandafter\ZREF@label\expandafter{\the\toks@}{#1}%
  \@esphack
}
%    \end{macrocode}
%    \end{macro}
%    \begin{macro}{\zref@labelbykv}
%    \begin{macrocode}
\ZREF@Robust\def\zref@labelbykv#1#2{%
  \@bsphack
    \begingroup
      \let\Z@L@ZREF@temp\ltx@empty
      \kvsetkeys{ZREF@LABEL}{#1}%
      \ifZREF@immediate
        \expandafter\zref@wrapper@immediate\expandafter{%
          \expandafter\ZREF@label\expandafter{\Z@L@ZREF@temp}{#2}%
        }%
      \else
        \expandafter\ZREF@label\expandafter{\Z@L@ZREF@temp}{#2}%
      \fi
    \endgroup
  \@esphack
}
%    \end{macrocode}
%    \end{macro}
%    \begin{macrocode}
\kv@define@key{ZREF@LABEL}{prop}{%
  \edef\ZREF@param{#1}%
  \zref@propexists\ZREF@param{%
    \zref@iflistcontainsprop{ZREF@temp}\ZREF@param{}{%
      \begingroup\expandafter\endgroup
      \expandafter\ltx@LocalAppendToMacro
      \expandafter\Z@L@ZREF@temp
      \expandafter{\csname\ZREF@param\endcsname}%
    }%
  }%
}
\kv@define@key{ZREF@LABEL}{list}{%
  \zref@listforloop{#1}{%
    \zref@iflistcontainsprop{ZREF@temp}\zref@prop{}{%
      \begingroup\expandafter\endgroup
      \expandafter\ltx@LocalAppendToMacro
      \expandafter\Z@L@ZREF@temp
      \expandafter{\csname\zref@prop\endcsname}%
    }%
    \ltx@gobble
  }%
}
\kv@define@key{ZREF@LABEL}{delprop}{%
  \zref@propexists{#1}{%
    \zref@localdelprop{ZREF@temp}{#1}%
  }%
}
\kv@define@key{ZREF@LABEL}{immediate}[true]{%
  \edef\ZREF@param{#1}%
  \ifx\ZREF@param\ZREF@true
    \ZREF@immediatetrue
  \else
    \ifx\ZREF@param\ZREF@false
      \ZREF@immediatefalse
    \else
      \PackageWarning\ZREF@name{%
        Option `immediate' expects `true' or `false'.\MessageBreak
        Ignoring invalid value `\ZREF@param'%
      }%
    \fi
  \fi
}
%    \end{macrocode}
%    \begin{macro}{\ZREF@false}
%    \begin{macrocode}
\def\ZREF@false{false}
%    \end{macrocode}
%    \end{macro}
%    \begin{macro}{\ZREF@true}
%    \begin{macrocode}
\def\ZREF@true{true}
%    \end{macrocode}
%    \end{macro}
%
%    \begin{macrocode}
\kv@define@key{ZREF@LABEL}{values}[]{%
  \kv@parse{#1}{%
    \ifx\kv@value\relax
      \@PackageWarning\ZREF@name{%
        Missing value for property `\kv@key'%
      }%
      \expandafter\ltx@gobbletwo
    \else
      \expandafter\zref@setcurrent
    \fi
  }%
}
%    \end{macrocode}
%
%    \begin{macro}{\ifZREF@immediate}
%    The switch \cs{ifZREF@immediate} tells us, whether the
%    label should be written immediately or at page shipout time.
%    \cs{ZREF@label} need to be notified about this, because
%    it must disable the deferred execution of property values,
%    if the label is written immediately.
%    \begin{macrocode}
\newif\ifZREF@immediate
%    \end{macrocode}
%    \end{macro}
%    \begin{macro}{\zref@wrapper@immediate}
%    The argument of \cs{zref@wrapper@immediate} is executed inside
%    a group where \cs{write} is redefined by adding
%    \cs{immediate} before its execution. Also \cs{ZREF@label}
%    is notified via the switch \cs{ifZREF@immediate}.
%    \begin{macrocode}
\ZREF@Robust{\long\def}\zref@wrapper@immediate#1{%
  \begingroup
    \ZREF@immediatetrue
    \let\ZREF@org@write\write
    \def\write{\immediate\ZREF@org@write}%
    #1%
  \endgroup
}
%    \end{macrocode}
%    \end{macro}
%
%    \begin{macro}{\ZREF@label}
%    \cs{ZREF@label} writes the data in the \xfile{.aux} file.
%    \verb|#1| contains the list of valid properties,
%    \verb|#2| the name of the reference. In case of immediate
%    writing, the deferred execution of property values is
%    disabled. Also \thepage is made expandable in this case.
%    \begin{macrocode}
\def\ZREF@label#1#2{%
  \if@filesw
    \begingroup
      \ifZREF@immediate
        \let\ZREF@org@thepage\thepage
      \fi
      \protected@write\@auxout{%
        \ifZREF@immediate
          \let\thepage\ZREF@org@thepage
        \fi
        \let\ZREF@temp\ltx@empty
        \@tfor\ZREF@P:=#1\do{%
          \begingroup
            \escapechar=-1 %
            \edef\x{\endgroup
              \def\noexpand\ZREF@P{%
                \expandafter\string\ZREF@P
              }%
            }%
          \x
          \expandafter\ifx
              \csname
                \ifZREF@immediate
                  relax%
                \else
                  Z@X@\ZREF@P%
                \fi
              \endcsname
              \noexpand
            \expandafter\let\csname Z@C@\ZREF@P\endcsname\relax
          \fi
          \toks@\expandafter{\ZREF@temp}%
          \edef\ZREF@temp{%
            \the\toks@
            \ltx@backslashchar\ZREF@P{%
              \expandafter\noexpand\csname Z@C@\ZREF@P\endcsname
            }%
          }%
        }%
      }{%
        \string\zref@newlabel{#2}{\ZREF@temp}%
      }%
    \endgroup
  \fi
}
\def\ZREF@addtoks#1{%
  \toks@\expandafter\expandafter\expandafter{%
    \expandafter\the\expandafter\toks@#1%
  }%
}
%    \end{macrocode}
%    \end{macro}
%
% \subsubsection{Reference querying and extracting}
%
%    Design goal for the extracting macros is that the extraction
%    process is full expandable. Thus these macros can be used
%    in expandable contexts. But there are problems that cannot
%    be solved by full expandable macros:
%    \begin{itemize}
%    \item In standard \hologo{LaTeX} undefined references sets a flag
%      and generate a warning. Both actions are not expandable.
%    \item Babel's support for its shorthand uses commands that
%      use non-expandable assignments. However currently there
%      is hope, that primitives are added to \pdfTeX\ that
%      allows the detection of contexts. Then the shorthand
%      can detect, if they are executed inside \cs{csname}
%      and protect themselves automatically.
%    \end{itemize}
%    \begin{macro}{\zref@ifrefundefined}
%    If a reference \verb|#1| is undefined, then macro
%    \cs{zref@ifrefundefined} calls \verb|#2| and \verb|#3| otherwise.
%    \begin{macrocode}
\def\zref@ifrefundefined#1{%
  \ltx@ifundefined{Z@R@#1}%
}
%    \end{macrocode}
%    \end{macro}
%
%    \begin{macro}{\zifrefundefined}
%    If a reference \verb|#1| is undefined, then macro
%    \cs{zref@ifrefundefined} calls \verb|#2| and \verb|#3| otherwise.
%    Also the reference is marked used.
%    \begin{macrocode}
\ZREF@IfDefinable\zifrefundefined\def{%
  #1{%
    \zref@wrapper@babel\ZREF@ifrefundefined{#1}%
  }%
}
%    \end{macrocode}
%    \end{macro}
%    \begin{macro}{\ZREF@ifrefundefined}
%    \begin{macrocode}
\def\ZREF@ifrefundefined#1{%
  \zref@refused{#1}%
  \zref@ifrefundefined{#1}%
}
%    \end{macrocode}
%    \end{macro}
%
%    \begin{macro}{\zref@refused}
%    The problem with undefined references is addressed by
%    the macro \cs{zref@refused}. This can be used outside
%    the expandable context. In case of an undefined reference
%    the flag is set to notify \hologo{LaTeX} and a warning is given.
%    \begin{macrocode}
\ZREF@Robust\def\zref@refused#1{%
  \zref@wrapper@babel\ZREF@refused{#1}%
}
%    \end{macrocode}
%    \end{macro}
%    \begin{macro}{\ZREF@refused}
%    \begin{macrocode}
\def\ZREF@refused#1{%
  \zref@ifrefundefined{#1}{%
    \protect\G@refundefinedtrue
    \@latex@warning{%
      Reference `#1' on page \thepage \space undefined%
    }%
  }{}%
}
%    \end{macrocode}
%    \end{macro}
%
%    \begin{macro}{\zref@ifrefcontainsprop}
%    \cs{zref@ifrefcontainsprop} looks, if the reference \verb|#1|
%    has the property \verb|#2| and calls then \verb|#3| and
%    \verb|#4| otherwise.
%    \begin{macrocode}
\def\zref@ifrefcontainsprop#1#2{%
  \zref@ifrefundefined{#1}{%
    \ltx@secondoftwo
  }{%
    \expandafter\ZREF@ifrefcontainsprop
    \csname Z@E@#2\expandafter\endcsname
    \csname#2\expandafter\expandafter\expandafter\endcsname
    \expandafter\expandafter\expandafter{%
      \csname Z@R@#1\endcsname
    }%
  }%
}
\def\ZREF@ifrefcontainsprop#1#2#3{%
  \expandafter\ifx\expandafter\ZREF@novalue
  #1#3#2\ZREF@novalue\ZREF@nil\ltx@empty
    \expandafter\ltx@secondoftwo
  \else
    \expandafter\ltx@firstoftwo
  \fi
}
\def\ZREF@novalue{\ZREF@NOVALUE}
%    \end{macrocode}
%    \end{macro}
%
%    \begin{macro}{\zref@extract}
%    \cs{zref@extract} is an abbreviation for the case
%    that the default of the property is used as default
%    value.
%    \begin{macrocode}
\def\ZREF@extract#1#2{%
  \romannumeral0%
  \ltx@ifundefined{Z@D@#2}{%
    \expandafter\ltx@space\zref@default
  }{%
    \expandafter\expandafter\expandafter\ZREF@@extract
    \expandafter\expandafter\expandafter{%
      \csname Z@D@#2\endcsname
    }{#1}{#2}%
  }%
}
%    \end{macrocode}
%    \end{macro}
%    \begin{macro}{\ZREF@@extract}
%    \begin{macrocode}
\def\ZREF@@extract#1#2#3{%
  \expandafter\expandafter\expandafter\ltx@space
  \zref@extractdefault{#2}{#3}{#1}%
}
%    \end{macrocode}
%    \end{macro}
%    \begin{macro}{\ZREF@wu@extract}
%    \begin{macrocode}
\def\ZREF@wu@extract#1#2{%
  \etex@unexpanded\expandafter\expandafter\expandafter{%
    \ZREF@extract{#1}{#2}%
  }%
}
%    \end{macrocode}
%    \end{macro}
%    \begin{macro}{\zref@extract}
%    \begin{macrocode}
\let\zref@extract\ZREF@extract
%    \end{macrocode}
%    \end{macro}
%
%    \begin{macro}{\ZREF@extractdefault}
%    The basic extracting macro is \cs{zref@extractdefault} with
%    the reference name in \verb|#1|, the property in \verb|#2| and
%    the default value in \verb|#3| in case for problems.
%    \begin{macrocode}
\def\ZREF@extractdefault#1#2#3{%
  \romannumeral0%
  \zref@ifrefundefined{#1}\ltx@firstoftwo{%
    \zref@ifpropundefined{#2}\ltx@firstoftwo\ltx@secondoftwo
  }{%
    \ltx@space
    #3%
  }{%
    \expandafter\expandafter\expandafter\ltx@space
    \csname Z@E@#2\expandafter\expandafter\expandafter\endcsname
    \csname Z@R@#1\expandafter\endcsname
    \csname#2\endcsname{#3}\ZREF@nil
  }%
}
%    \end{macrocode}
%    \end{macro}
%    \begin{macro}{\ZREF@wu@extractdefault}
%    \begin{macrocode}
\def\ZREF@wu@extractdefault#1#2#3{%
  \etex@unexpanded\expandafter\expandafter\expandafter{%
    \ZREF@extractdefault{#1}{#2}{#3}%
  }%
}
%    \end{macrocode}
%    \end{macro}
%    \begin{macro}{\zref@extractdefault}
%    \begin{macrocode}
\let\zref@extractdefault\ZREF@extractdefault
%    \end{macrocode}
%    \end{macro}
%
%    \begin{macro}{\zref@def@extract}
%    \begin{macrocode}
\ZREF@Robust\def\zref@def@extract#1{%
  \zref@wrapper@babel{\ZREF@def@extract{#1}}%
}
%    \end{macrocode}
%    \end{macro}
%    \begin{macro}{\ZREF@def@extract}
%    \begin{macrocode}
\def\ZREF@def@extract#1#2#3{%
  \zref@refused{#2}%
  \expandafter\expandafter\expandafter\def
  \expandafter\expandafter\expandafter#1%
  \expandafter\expandafter\expandafter{%
    \zref@extract{#2}{#3}%
  }%
}
%    \end{macrocode}
%    \end{macro}
%    \begin{macro}{\zref@def@extractdefault}
%    \begin{macrocode}
\ZREF@Robust\def\zref@def@extractdefault#1{%
  \zref@wrapper@babel{\ZREF@def@extractdefault{#1}}%
}
%    \end{macrocode}
%    \end{macro}
%    \begin{macro}{\ZREF@def@extractdefault}
%    \begin{macrocode}
\def\ZREF@def@extractdefault#1#2#3#4{%
  \zref@refused{#2}%
  \expandafter\expandafter\expandafter\def
  \expandafter\expandafter\expandafter#1%
  \expandafter\expandafter\expandafter{%
    \zref@extractdefault{#2}{#3}{#4}%
  }%
}
%    \end{macrocode}
%    \end{macro}
%
%    \begin{macro}{\ZREF@wrapper@unexpanded}
%    \begin{macrocode}
\ZREF@Robust{\long\def}\ZREF@wrapper@unexpanded#1{%
  \let\zref@wrapper@unexpanded\ltx@firstofone
  \let\zref@getcurrent\ZREF@wu@getcurrent
  \let\zref@extractdefault\ZREF@wu@extractdefault
  \let\zref@extract\ZREF@wu@extract
  #1%
  \let\zref@wrapper@unexpanded\ZREF@wrapper@unexpanded
  \let\zref@getcurrent\ZREF@getcurrent
  \let\zref@extractdefault\ZREF@extractdefault
  \let\zref@extract\ZREF@extract
}
%    \end{macrocode}
%    \end{macro}
%    \begin{macro}{\zref@wrapper@unexpanded}
%    \begin{macrocode}
\ltx@IfUndefined{etex@unexpanded}{%
  \let\zref@wrapper@unexpanded\ltx@firstofone
}{%
  \let\zref@wrapper@unexpanded\ZREF@wrapper@unexpanded
}
%    \end{macrocode}
%    \end{macro}
%
% \subsubsection{Compatibility with \xpackage{babel}}
%
%    \begin{macro}{\zref@wrapper@babel}
%    \begin{macrocode}
\ZREF@Robust{\long\def}\zref@wrapper@babel#1#2{%
  \ifcsname if@safe@actives\endcsname
    \expandafter\ltx@firstofone
  \else
    \expandafter\ltx@secondoftwo
  \fi
  {%
    \if@safe@actives
      \expandafter\ltx@secondoftwo
    \else
      \expandafter\ltx@firstoftwo
    \fi
    {%
      \begingroup
        \csname @safe@activestrue\endcsname
        \edef\x{#2}%
      \expandafter\endgroup
      \expandafter\ZREF@wrapper@babel\expandafter{\x}{#1}%
    }%
  }{%
    #1{#2}%
  }%
}
\long\def\ZREF@wrapper@babel#1#2{%
  #2{#1}%
}
%    \end{macrocode}
%    \end{macro}
%
% \subsubsection{Unique counter support}
%
%    \begin{macro}{\zref@require@unique}
%    Generate the counter |zref@unique| if the counter does not
%    already exist.
%    \begin{macrocode}
\ZREF@Robust\def\zref@require@unique{%
  \@ifundefined{c@zref@unique}{%
    \begingroup
      \let\@addtoreset\ltx@gobbletwo
      \newcounter{zref@unique}%
    \endgroup
%    \end{macrocode}
%    \end{macro}
%
%    \begin{macro}{\thezref@unique}
%    \cs{thezref@unique} is used for automatically generated unique
%    labelnames.
%    \begin{macrocode}
    \renewcommand*{\thezref@unique}{%
      zref@\number\c@zref@unique
    }%
  }{}%
}
%    \end{macrocode}
%    \end{macro}
%
% \subsubsection{Utilities}
%
%    \begin{macro}{\ZREF@number}
%    \begin{macrocode}
\ltx@IfUndefined{numexpr}{%
  \def\ZREF@number#1{\number#1}%
}{%
  \def\ZREF@number#1{\the\numexpr(#1)\relax}%
}
%    \end{macrocode}
%    \end{macro}
%
% \subsubsection{Setup}
%
%    \begin{macro}{\zref@setdefault}
%    Standard \hologo{LaTeX} prints ``??'' in bold face if a reference
%    is not known. \cs{zref@default} holds the text that is
%    printed in case of unknown references and is used, if
%    the default was not specified during the definition
%    of the new property by \cs{ref@newprop}.
%    The global default value can be set by \cs{zref@setdefault}.
%    \begin{macrocode}
\ZREF@Robust\def\zref@setdefault#1{%
  \def\zref@default{#1}%
}
%    \end{macrocode}
%    \end{macro}
%    \begin{macro}{\zref@default}
%    Now we initialize \cs{zref@default} with the same value
%    that \hologo{LaTeX} uses for its undefined references.
%    \begin{macrocode}
\zref@setdefault{%
  \nfss@text{\reset@font\bfseries ??}%
}
%    \end{macrocode}
%    \end{macro}
%
% \paragraph{Main property list.}
%
%    \begin{macro}{\zref@setmainlist}
%    The name of the default property list is stored in
%    \cs{ZREF@mainlist} and can be set by \cs{zref@setmainlist}.
%    \begin{macrocode}
\ZREF@Robust\def\zref@setmainlist#1{%
  \def\ZREF@mainlist{#1}%
}
\zref@setmainlist{main}
%    \end{macrocode}
%    \end{macro}
%    Now we create the list.
%    \begin{macrocode}
\zref@newlist\ZREF@mainlist
%    \end{macrocode}
%
% \paragraph{Main properties.}
%
%    The two properties \texttt{default} and \texttt{page}
%    are created and added to the main property list.
%    They store the data that
%    standard \hologo{LaTeX} uses in its references created by
%    \cs{label}.
%    \begin{description}
%    \item[\texttt{default}] the apperance of the latest
%      counter that is incremented by \cs{refstepcounter}
%    \item[\texttt{page}] the apperance of the page counter
%    \end{description}
%    \begin{macrocode}
\zref@newprop{default}{\@currentlabel}
\zref@newprop*{page}{\thepage}
\zref@addprops\ZREF@mainlist{default,page}
%    \end{macrocode}
%
% \paragraph{Properties}
%
%    \begin{macro}{\ZREF@NewPropAnchor}
%    \begin{macrocode}
\def\ZREF@NewPropAnchor{%
  \zref@newprop{anchor}{%
    \ltx@ifundefined{@currentHref}{}{\@currentHref}%
  }%
  \global\let\ZREF@NewPropAnchor\relax
}
%    \end{macrocode}
%    \end{macro}
%    \begin{macro}{\zref@titleref@current}
%    Later we will redefine the section and caption macros
%    to catch the current title and remember the value in
%    \cs{zref@titleref@current}.
%    \begin{macro}{\ZREF@NewPropTitle}
%    \begin{macrocode}
\def\ZREF@NewPropTitle{%
  \gdef\zref@titleref@current{}%
  \zref@newprop{title}{\zref@titleref@current}%
  \global\let\ZREF@NewPropTitle\relax
}
%    \end{macrocode}
%    \end{macro}
%    \end{macro}
%    \begin{macro}{\ZREF@NewPropTheotype}
%    \begin{macrocode}
\def\ZREF@NewPropTheotype{%
  \zref@newprop{theotype}{}%
  \global\let\ZREF@NewPropTheotype\relax
}
%    \end{macrocode}
%    \end{macro}
%    \begin{macro}{\ZREF@NewPropPageValue}
%    \begin{macrocode}
\def\ZREF@NewPropPageValue{%
  \zref@newprop*{pagevalue}[0]{\number\c@page}%
  \global\let\ZREF@NewPropPageValue\relax
}
%    \end{macrocode}
%    \end{macro}
%
% \paragraph{Mark successful loading}
%
%    \begin{macrocode}
\let\ZREF@base@ok=Y
%</base>
%    \end{macrocode}
%
%
% \subsection{Module \xmodule{user}}
%
%    \begin{macrocode}
%<*user>
\NeedsTeXFormat{LaTeX2e}
\ProvidesPackage{zref-user}%
  [2020-07-03 v2.32 Module user for zref (HO)]%
\RequirePackage{zref-base}[2019/11/29]
\ifx\ZREF@base@ok Y%
\else
  \expandafter\endinput
\fi
%    \end{macrocode}
%
%    Module \xmodule{user} enables a small user interface.
%    All macros are prefixed by \cs{z}.
%
%    First we define the pendants to the standard \hologo{LaTeX} referencing
%    commands \cs{label}, \cs{ref}, and \cs{pageref}.
%
%    \begin{macro}{\zlabel}
%    Similar to \cs{label} the macro \cs{zlabel} writes a reference
%    entry in the \xfile{.aux} file. The main property list is used.
%    Also we add the babel patch. The \cs{label} command can also be
%    used inside section titles, but it must not go into the
%    table of contents. Therefore we have to check this situation.
%    \begin{macrocode}
\newcommand*\zlabel{%
  \ifx\label\ltx@gobble
    \expandafter\ltx@gobble
  \else
    \expandafter\zref@wrapper@babel\expandafter\zref@label
  \fi
}%
%    \end{macrocode}
%    \end{macro}
%
%    \begin{macro}{\zkvlabel}
%    \begin{macrocode}
\newcommand*{\zkvlabel}[1]{%
  \ifx\label\ltx@gobble
    \expandafter\ltx@gobblethree
  \fi
  \zref@wrapper@babel{\zref@labelbykv{#1}}%
}%
%    \end{macrocode}
%    \end{macro}
%
%    \begin{macro}{\zref}
%    Macro \cs{zref} is the corresponding macro for \cs{ref}.
%    Also it provides an optional argument in order to
%    select another property.
%    \begin{macrocode}
\newcommand*{\zref}[2][default]{% robust because of optional argument
  \zref@propexists{#1}{%
    \zref@wrapper@babel\ZREF@zref{#2}{#1}%
  }%
}%
\def\ZREF@zref#1{%
  \zref@refused{#1}%
  \zref@extract{#1}%
}%
%    \end{macrocode}
%    \end{macro}
%
%    \begin{macro}{\zpageref}
%    For macro \cs{zpageref} we just call \cs{zref}
%    with property \verb|page|.
%    \begin{macrocode}
\ZREF@IfDefinable\zpageref\def{%
  {\zref[page]}%
}
%    \end{macrocode}
%    \end{macro}
%
%    \begin{macro}{\zrefused}
%    For the following expandible user macros \cs{zrefused}
%    should be used to notify \hologo{LaTeX} in case of undefined
%    references.
%    \begin{macrocode}
\ZREF@IfDefinable\zrefused\def{%
  {\zref@refused}%
}
%    \end{macrocode}
%    \end{macro}
%
%    \begin{macrocode}
%</user>
%    \end{macrocode}
%
%
% \subsection{Module \xmodule{abspage}}
%
%    \begin{macrocode}
%<*abspage>
\NeedsTeXFormat{LaTeX2e}
\ProvidesPackage{zref-abspage}%
  [2020-07-03 v2.32 Module abspage for zref (HO)]%
\RequirePackage{zref-base}[2019/11/29]
\ifx\ZREF@base@ok Y%
\else
  \expandafter\endinput
\fi
%    \end{macrocode}
%
%    Module \xmodule{abspage} adds a new property \verb|abspage|
%    to the \verb|main| property list for absolute page numbers.
%    These are recorded by the help of package \xpackage{atbegshi}.
%    \begin{macrocode}
\RequirePackage{atbegshi}[2011/10/05]%
%    \end{macrocode}
%    The counter \verb|abspage| must not go in the clear list
%    of \verb|@ckpt| that is used to set counters in
%    \xfile{.aux} files of included \hologo{TeX} files.
%    \begin{macrocode}
\begingroup
  \let\@addtoreset\ltx@gobbletwo
  \newcounter{abspage}%
\endgroup
\setcounter{abspage}{0}%
\AtBeginShipout{%
  \stepcounter{abspage}%
}%
\zref@newprop*{abspage}[0]{\the\c@abspage}%
\zref@addprop\ZREF@mainlist{abspage}%
%    \end{macrocode}
%    Note that counter \verb|abspage| shows the
%    previous page during page processing. Before
%    shipout the counter is incremented.
%    Thus the property is correctly written with
%    deferred writing. If the counter is written
%    using \cs{zref@wrapper@immediate}, then the number
%    is too small by one.
%
%    \begin{macrocode}
%</abspage>
%    \end{macrocode}
%
%
% \subsection{Module \xmodule{counter}}
%
%    \begin{macrocode}
%<*counter>
\NeedsTeXFormat{LaTeX2e}
\ProvidesPackage{zref-counter}%
  [2020-07-03 v2.32 Module counter for zref (HO)]%
\RequirePackage{zref-base}[2019/11/29]
\ifx\ZREF@base@ok Y%
\else
  \expandafter\endinput
\fi
%    \end{macrocode}
%
%    For features such as \xpackage{hyperref}'s \cs{autoref}
%    we need the name of the counter. The property
%    \verb|counter| is defined and added to the main property
%    list. Starting with \LaTeX\ 2020-10-01 the proper can use
%    \@currentcounter. In older formats
%    \cs{refstepcounter} has to be patched but this can fail in some
%    cases, see issue \#5.
%    \begin{macrocode}
\@ifl@t@r\fmtversion{2020-10-01}
 {
  \zref@newprop{counter}{\@currentcounter}
  \zref@addprop\ZREF@mainlist{counter}
 }
 {
  \zref@newprop{counter}{}
  \zref@addprop\ZREF@mainlist{counter}
  \AtBeginDocument{%
    \ZREF@patch{refstepcounter}{%
      \def\refstepcounter#1{%
        \zref@setcurrent{counter}{#1}%
        \ZREF@org@refstepcounter{#1}%
      }%
    }%
  }
 }
%    \end{macrocode}
%    \begin{macrocode}
%</counter>
%    \end{macrocode}
%
%
% \subsection{Module \xmodule{lastpage}}
%
%    \begin{macrocode}
%<*lastpage>
\NeedsTeXFormat{LaTeX2e}
\ProvidesPackage{zref-lastpage}%
  [2020-07-03 v2.32 Module lastpage for zref (HO)]%
\RequirePackage{zref-base}[2019/11/29]
\RequirePackage{zref-abspage}[2019/11/29]
\RequirePackage{atveryend}[2009/12/07]
\ifx\ZREF@base@ok Y%
\else
  \expandafter\endinput
\fi
%    \end{macrocode}
%
%    The module \xmodule{lastpage} implements the service
%    of package \xpackage{lastpage} by setting a reference
%    \verb|LastPage| at the end of the document.
%    If module \xmodule{abspage} is given, also the absolute
%    page number is available, because the properties of
%    the main property list are used.
%    \begin{macrocode}
\zref@newlist{LastPage}
\AfterLastShipout{%
  \if@filesw
    \begingroup
      \advance\c@page\m@ne
      \toks@\expandafter\expandafter\expandafter{%
        \expandafter\Z@L@main
        \Z@L@LastPage
      }%
      \expandafter\zref@wrapper@immediate\expandafter{%
        \expandafter\ZREF@label\expandafter{\the\toks@}{LastPage}%
      }%
    \endgroup
  \fi
}
%    \end{macrocode}
%    \begin{macro}{\zref@iflastpage}
%    \begin{macrocode}
\def\zref@iflastpage#1{%
  \ifnum\zref@extractdefault{#1}{abspage}{-1}=%
        \zref@extractdefault{LastPage}{abspage}{-2} %
    \expandafter\ltx@firstoftwo
  \else
    \expandafter\ltx@secondoftwo
  \fi
}
%    \end{macrocode}
%    \end{macro}
%    \begin{macro}{\ziflastpage}
%    \begin{macrocode}
\ZREF@IfDefinable\ziflastpage\def{%
  {\zref@wrapper@babel\ZREF@iflastpage}%
}
%    \end{macrocode}
%    \end{macro}
%    \begin{macro}{ZREF@iflastpage}
%    \begin{macrocode}
\def\ZREF@iflastpage#1{%
  \zref@refused{LastPage}%
  \zref@refused{#1}%
  \zref@iflastpage{#1}%
}
%    \end{macrocode}
%    \end{macro}
%    \begin{macrocode}
%</lastpage>
%    \end{macrocode}
%
%
% \subsection{Module \xmodule{thepage}}
%
%    \begin{macrocode}
%<*thepage>
\NeedsTeXFormat{LaTeX2e}
\ProvidesPackage{zref-thepage}%
  [2020-07-03 v2.32 Module thepage for zref (HO)]%
\RequirePackage{zref-base}[2019/11/29]
\ifx\ZREF@base@ok Y%
\else
  \expandafter\endinput
\fi
%    \end{macrocode}
%
%    \begin{macrocode}
\RequirePackage{atbegshi}[2011/10/05]
\RequirePackage{zref-abspage}[2019/11/29]
%    \end{macrocode}
%    \begin{macrocode}
\zref@newlist{thepage}
\zref@addprop{thepage}{page}
\ZREF@NewPropPageValue
%    \end{macrocode}
%    \begin{macro}{\zref@thepage@atbegshi@hook}
%    \begin{macrocode}
\let\zref@thepage@atbegshi@hook\ltx@empty
%    \end{macrocode}
%    \end{macro}
%    \begin{macrocode}
\zref@addprop{thepage}{pagevalue}
\AtBeginShipout{%
  \AtBeginShipoutAddToBox{%
    \zref@thepage@atbegshi@hook
    \zref@labelbylist{thepage\the\value{abspage}}{thepage}%
  }%
}
%    \end{macrocode}
%    \begin{macro}{\zref@thepage@name}
%    \begin{macrocode}
\ltx@IfUndefined{numexpr}{%
  \def\zref@thepage@name#1{thepage\number#1}%
}{%
  \def\zref@thepage@name#1{thepage\the\numexpr#1}%
}
%    \end{macrocode}
%    \end{macro}
%    \begin{macro}{\zref@thepage}
%    \begin{macrocode}
\def\zref@thepage#1{%
  \zref@extract{\zref@thepage@name{#1}}{page}%
}%
%    \end{macrocode}
%    \end{macro}
%    \begin{macro}{\zref@thepage@refused}
%    \begin{macrocode}
\ZREF@Robust\def\zref@thepage@refused#1{%
  \zref@refused{\zref@thepage@name{#1}}%
}%
%    \end{macrocode}
%    \end{macro}
%    \begin{macro}{\zthepage}
%    \begin{macrocode}
\ZREF@IfDefinable\zthepage\def{%
  #1{%
    \zref@thepage@refused{#1}%
    \zref@thepage{#1}%
  }%
}
%    \end{macrocode}
%    \end{macro}
%    \begin{macrocode}
%</thepage>
%    \end{macrocode}
%
%
% \subsection{Module \xmodule{nextpage}}
%
%    \begin{macrocode}
%<*nextpage>
\NeedsTeXFormat{LaTeX2e}
\ProvidesPackage{zref-nextpage}%
  [2020-07-03 v2.32 Module nextpage for zref (HO)]%
\RequirePackage{zref-base}[2019/11/29]
\ifx\ZREF@base@ok Y%
\else
  \expandafter\endinput
\fi
%    \end{macrocode}
%
%    \begin{macrocode}
\RequirePackage{zref-abspage}[2019/11/29]
\RequirePackage{zref-thepage}[2019/11/29]
\RequirePackage{zref-lastpage}[2019/11/29]
\RequirePackage{uniquecounter}[2009/12/18]
%    \end{macrocode}
%    \begin{macrocode}
\UniqueCounterNew{znextpage}

\newcommand*{\znextpagesetup}{%
  \afterassignment\ZREF@np@setup@i
  \def\ZREF@np@call@unknown##1%
}
\def\ZREF@np@setup@i{%
  \afterassignment\ZREF@np@setup@ii
  \def\ZREF@np@call@nonext##1%
}
\def\ZREF@np@setup@ii{%
  \def\ZREF@np@call@next##1%
}
\def\ZREF@np@call@unknown#1{#1}
\def\ZREF@np@call@nonext#1{#1}
\def\ZREF@np@call@next#1{#1}
\ZREF@IfDefinable\znextpage\def{%
  {\UniqueCounterCall{znextpage}{\ZREF@nextpage}}%
}%
\newcommand*{\znonextpagename}{}
\newcommand*{\zunknownnextpagename}{\Z@D@page}
\def\ZREF@nextpage#1{%
  \begingroup
    \def\ZREF@refname@this{zref@np#1}%
    \zref@labelbyprops\ZREF@refname@this{abspage}%
    \chardef\ZREF@call=0 % unknown
    \ZREF@ifrefundefined\ZREF@refname@this{%
    }{%
      \edef\ZREF@pagenum@this{%
        \zref@extractdefault\ZREF@refname@this{abspage}{0}%
      }%
      \edef\ZREF@refname@next{%
        \zref@thepage@name{%
          \the\numexpr\ZREF@pagenum@this+1%
        }%
      }%
      \ifnum\ZREF@pagenum@this>0 %
        \ZREF@ifrefundefined{LastPage}{%
          \zref@ifrefundefined\ZREF@refname@next{%
          }{%
            \chardef\ZREF@call=2 % next page
          }%
        }{%
          \edef\ZREF@pagenum@last{%
            \zref@extractdefault{LastPage}{abspage}{0}%
          }%
          \ifnum\ZREF@pagenum@this<\ZREF@pagenum@last\ltx@space
            \ZREF@ifrefundefined\ZREF@refname@next{%
            }{%
              \chardef\ZREF@call=2 % next page
            }%
          \else
            \ifnum\ZREF@pagenum@this=\ZREF@pagenum@this\ltx@space
              \chardef\ZREF@call=1 % no next page
            \fi
          \fi
        }%
      \fi
    }%
    \edef\x{%
      \endgroup
      \ifcase\ZREF@call
        \noexpand\ZREF@np@call@unknown{%
          \noexpand\zunknownnextpagename
        }%
      \or
        \noexpand\ZREF@np@call@nonext{%
          \noexpand\znonextpagename
        }%
      \else
        \noexpand\ZREF@np@call@next{%
          \noexpand\zref@extract{\ZREF@refname@next}{page}%
        }%
      \fi
    }%
  \x
}
%    \end{macrocode}
%    \begin{macrocode}
%</nextpage>
%    \end{macrocode}
%
%
% \subsection{Module \xmodule{totpages}}
%
%    \begin{macrocode}
%<*totpages>
\NeedsTeXFormat{LaTeX2e}
\ProvidesPackage{zref-totpages}%
  [2020-07-03 v2.32 Module totpages for zref (HO)]%
\RequirePackage{zref-base}[2019/11/29]
\ifx\ZREF@base@ok Y%
\else
  \expandafter\endinput
\fi
%    \end{macrocode}
%
%    The absolute page number of the last page is the total
%    page number.
%    \begin{macrocode}
\RequirePackage{zref-abspage}[2019/11/29]
\RequirePackage{zref-lastpage}[2019/11/29]
%    \end{macrocode}
%
%    \begin{macro}{\ztotpages}
%    Macro \cs{ztotpages} contains the number of pages. It can be
%    used inside expandable calculations. It expands to zero if
%    the reference is not yet available.
%    \begin{macrocode}
\newcommand*{\ztotpages}{%
  \zref@extractdefault{LastPage}{abspage}{0}%
}
%    \end{macrocode}
%    \end{macro}
%    Also we mark the reference |LastPage| as used:
%    \begin{macrocode}
\AtBeginDocument{%
  \zref@refused{LastPage}%
}
%    \end{macrocode}
%    \begin{macrocode}
%</totpages>
%    \end{macrocode}
%
%
% \subsection{Module \xmodule{pagelayout}}
%
%    \begin{macrocode}
%<*pagelayout>
\NeedsTeXFormat{LaTeX2e}
\ProvidesPackage{zref-pagelayout}%
  [2020-07-03 v2.32 Module pagelayout for zref (HO)]%
\RequirePackage{zref-base}[2019/11/29]
\ifx\ZREF@base@ok Y%
\else
  \expandafter\endinput
\fi
%    \end{macrocode}
%
%    \begin{macrocode}
\RequirePackage{zref-thepage}[2019/11/29]
\RequirePackage{iftex}[2019/11/07]%
\RequirePackage{atveryend}[2010/03/24]
%    \end{macrocode}
%
%
% \subsubsection{Define layout properties}
%
%    \begin{macrocode}
\def\ZREF@temp#1{%
  \begingroup
    \escapechar=-1 %
  \ltx@ifundefined{\string#1}{\endgroup}{%
    \edef\x{%
      \endgroup
      \noexpand\zref@newprop*{\string#1}%
                             [\noexpand\number\noexpand#1]% hash-ok
                             {\noexpand\number\noexpand#1}%
      \noexpand\zref@addprop{thepage}{\string#1}%
    }%
    \x
  }%
}
\ZREF@temp\mag
\ZREF@temp\paperwidth
\ZREF@temp\paperheight
\ZREF@temp\stockwidth % memoir.cls, crop.sty
\ZREF@temp\stockheight % memoir.cls, crop.sty
\ZREF@temp\mediawidth % VTeX
\ZREF@temp\mediaheight % VTeX
\ifluatex
\zref@newprop*{pdfvorigin}%
                             [\number\pdfvariable vorigin]% hash-ok
                             {\number\pdfvariable vorigin}%
\zref@addprop{thepage}{pdfvorigin}
\zref@newprop*{pdfhorigin}%
                             [\number\pdfvariable horigin]% hash-ok
                             {\number\pdfvariable horigin}%
\zref@addprop{thepage}{pdfhorigin}
\zref@newprop*{pdfpageheight}%
                             [\number\pageheight]% hash-ok
                             {\number\pageheight}%
\zref@addprop{thepage}{pdfpageheight}
\zref@newprop*{pdfpagewidth}%
                             [\number\pagewidth]% hash-ok
                             {\number\pagewidth}%
\zref@addprop{thepage}{pdfpagewidth}
\else
\ZREF@temp\pdfpagewidth
\ZREF@temp\pdfpageheight
\ZREF@temp\pdfhorigin
\ZREF@temp\pdfvorigin
\fi
\ZREF@temp\hoffset
\ZREF@temp\voffset
\ZREF@temp\topmargin
\ZREF@temp\oddsidemargin
\ZREF@temp\evensidemargin
\ZREF@temp\textwidth
\ZREF@temp\textheight
\ZREF@temp\headheight
\ZREF@temp\headsep
\ZREF@temp\footskip
\ZREF@temp\marginparwidth
\ZREF@temp\marginparsep
\ZREF@temp\columnwidth
\ZREF@temp\columnsep
\ZREF@temp\trimedge % memoir.cls
\ZREF@temp\spinemargin % memoir.cls
\ZREF@temp\foremargin % memoir.cls
\ZREF@temp\trimtop % memoir.cls
\ZREF@temp\uppermargin % memoir.cls
\ZREF@temp\headmargin % memoir.cls
%    \end{macrocode}
%
%    \begin{macrocode}
\IfFormatAtLeastTF{2020/10/01}
 {
   \zref@newprop*{outputboxwd}[0pt]{\ShipoutBoxWidth}
   \zref@newprop*{outputboxht}[0pt]{\ShipoutBoxHeight}
   \zref@newprop*{outputboxdp}[0pt]{\ShipoutBoxDepth}
 }
 {
   \zref@newprop*{outputboxwd}[0pt]{\AtBeginShipoutBoxWidth}
   \zref@newprop*{outputboxht}[0pt]{\AtBeginShipoutBoxHeight}
   \zref@newprop*{outputboxdp}[0pt]{\AtBeginShipoutBoxDepth}
 }
%    \end{macrocode}
%    \begin{macrocode}
\zref@addprops{thepage}{outputboxwd,outputboxht,outputboxdp}
%    \end{macrocode}
%
%    \begin{macro}{\ifZREF@pl@list}
%    \begin{macrocode}
\ltx@newif\ifZREF@pl@list
%    \end{macrocode}
%    \end{macro}
%    \begin{macro}{\zref@listpagelayout}
%    \begin{macrocode}
\ZREF@IfDefinable\zlistpagelayout\def{%
  {\global\ZREF@pl@listtrue}%
}
%    \end{macrocode}
%    \end{macro}
%    \begin{macro}{\ZREF@pl@AfterLastShipout}
%    \begin{macrocode}
\def\ZREF@pl@AfterLastShipout{%
  \ifZREF@pl@list
    \edef\ZREF@page@max{\the\value{abspage}}%
    \ltx@ifundefined{ZREF@org@testdef}{%
      \let\ZREF@org@testdef\@testdef
      \def\@testdef##1##2##3{%
        \ZREF@org@testdef{##1}{##2}{##3}%
        \def\ZREF@temp{##1}%
        \ifx\ZREF@temp\ZREF@RefPrefix
          \expandafter\gdef\csname##1@##2\endcsname{##3}%
        \fi
      }%
    }{}%
    \AtVeryEndDocument{\ZREF@pl@AtVeryEnd}%
  \fi
}
%    \end{macrocode}
%    \end{macro}
%    \begin{macro}{\ZREF@pl@AtVeryEnd}
%    \begin{macrocode}
\def\ZREF@pl@AtVeryEnd{%
  \begingroup
    \toks@{Page layout parameters:\MessageBreak}%
    \count@=1 %
    \ZREF@pl@ListPage
    \edef\x{\endgroup
      \noexpand\@PackageInfoNoLine{zref-pagelayout}{\the\toks@}%
    }%
  \x
}
%    \end{macrocode}
%    \end{macro}
%    \begin{macro}{\ZREF@pl@ListPage}
%    \begin{macrocode}
\def\ZREF@pl@ListPage{%
  \edef\x{%
    \toks@={%
      \the\toks@
      Page \the\count@:\noexpand\MessageBreak
      \zref@ifrefundefined{thepage\the\count@}{}{%
        \ltx@space\ltx@space mag = %
        \zref@extract{thepage\the\count@}{mag}%
        \noexpand\MessageBreak
        \ZREF@pl@ListEntry{paperwidth}%
        \ZREF@pl@ListEntry{paperheight}%
        \ZREF@pl@ListEntry{stockwidth}%
        \ZREF@pl@ListEntry{stockheight}%
        \ZREF@pl@ListEntry{mediawidth}%
        \ZREF@pl@ListEntry{mediaheight}%
        \ZREF@pl@ListEntry{pdfpagewidth}%
        \ZREF@pl@ListEntry{pdfpageheight}%
        \ZREF@pl@ListEntry{pdfhorigin}%
        \ZREF@pl@ListEntry{pdfvorigin}%
        \ZREF@pl@ListEntry{hoffset}%
        \ZREF@pl@ListEntry{voffset}%
        \ZREF@pl@ListEntry{topmargin}%
        \ZREF@pl@ListEntry{oddsidemargin}%
        \ZREF@pl@ListEntry{evensidemargin}%
        \ZREF@pl@ListEntry{textwidth}%
        \ZREF@pl@ListEntry{textheight}%
        \ZREF@pl@ListEntry{headheight}%
        \ZREF@pl@ListEntry{headsep}%
        \ZREF@pl@ListEntry{footskip}%
        \ZREF@pl@ListEntry{marginparwidth}%
        \ZREF@pl@ListEntry{marginparsep}%
        \ZREF@pl@ListEntry{columnwidth}%
        \ZREF@pl@ListEntry{columnsep}%
        \ZREF@pl@ListEntry{trimedge}%
        \ZREF@pl@ListEntry{spinemargin}%
        \ZREF@pl@ListEntry{foremargin}%
        \ZREF@pl@ListEntry{trimtop}%
        \ZREF@pl@ListEntry{uppermargin}%
        \ZREF@pl@ListEntry{headmargin}%
      }%
    }%
  }\x
  \ifnum\ZREF@page@max>\count@
    \advance\count@ by\ltx@one
  \else
    \expandafter\ltx@gobble
  \fi
  \ZREF@pl@ListPage
}
%    \end{macrocode}
%    \end{macro}
%    \begin{macro}{\ZREF@pl@ListEntry}
%    \begin{macrocode}
\def\ZREF@pl@ListEntry#1{%
  \zref@ifpropundefined{#1}{%
  }{%
    \zref@ifrefcontainsprop{thepage\the\count@}{#1}{%
      \ltx@space\ltx@space#1 = %
      \zref@extract{thepage\the\count@}{#1}sp = %
      \the\dimexpr\zref@extract{thepage\the\count@}{#1}sp\relax
      \noexpand\MessageBreak
    }{}%
  }%
}
%    \end{macrocode}
%    \end{macro}
%    \begin{macrocode}
\AfterLastShipout{%
  \ZREF@pl@AfterLastShipout
}
%    \end{macrocode}
%
%    \begin{macrocode}
%</pagelayout>
%    \end{macrocode}
%
%
% \subsection{Module \xmodule{pageattr}}
%
%    \begin{macrocode}
%<*pageattr>
\NeedsTeXFormat{LaTeX2e}
\ProvidesPackage{zref-pageattr}%
  [2020-07-03 v2.32 Module pageattr for zref (HO)]%
\RequirePackage{zref-base}[2019/11/29]
\ifx\ZREF@base@ok Y%
\else
  \expandafter\endinput
\fi
%    \end{macrocode}
%
%    \begin{macrocode}
\RequirePackage{iftex}[2019/11/07]%
%    \end{macrocode}
%
%
%    \begin{macrocode}
\let\ZREF@temp=N%
\ifluatex
\expandafter\@firstoftwo
\else
\expandafter\@secondoftwo
\fi
{%luatex
 \RequirePackage{zref-thepage}[2019/11/29]
 \RequirePackage{zref-lastpage}[2019/11/29]%
 \zref@newprop*{pdfpageattr}[]{\zref@hex{\the\pdfvariable pageattr}}%
 \zref@addprop{thepage}{pdfpageattr}%
 \zref@newprop*{pdfpagesattr}[]{\zref@hex{\the\pdfvariable pagesattr}}%
 \zref@addprop{LastPage}{pdfpagesattr}%
 \let\ZREF@temp=Y%
}
{%other
 \ltx@IfUndefined{pdfpageattr}{%
  \@PackageInfoNoLine{zref-pageattr}{%
    \string\pdfpageattr\space is not available%
  }%
  \def\zref@pdfpageattr#1{}%
  \def\zref@pdfpageattr@used#1{}%
 }{%
  \RequirePackage{zref-thepage}[2019/11/29]%
  \zref@newprop*{pdfpageattr}[]{\zref@hex{\the\pdfpageattr}}%
  \zref@addprop{thepage}{pdfpageattr}%
  \let\ZREF@temp=Y%
 }
 \ltx@IfUndefined{pdfpagesattr}{%
  \@PackageInfoNoLine{zref-pageattr}{%
    \string\pdfpagesattr\space is not available%
  }%
  \def\zref@pdfpagesattr{}%
  \def\zref@pdfpagesattr@used{}%
 }{%
  \RequirePackage{zref-lastpage}[2019/11/29]%
  \zref@newprop*{pdfpagesattr}[]{\zref@hex{\the\pdfpagesattr}}%
  \zref@addprop{LastPage}{pdfpagesattr}%
  \let\ZREF@temp=Y%
 }%
}%
%    \end{macrocode}
%    \begin{macrocode}
\ifx\ZREF@temp N%
  \expandafter\endinput
\fi
%    \end{macrocode}
%    \begin{macrocode}
\RequirePackage{zref-abspage}[2019/11/29]
\RequirePackage{atveryend}[2010/03/24]
\RequirePackage{pdftexcmds}[2010/04/01]
\let\ZREF@temp=Y%
\ltx@IfUndefined{pdf@escapehex}{\let\ZREF@temp=N}{}
\ltx@IfUndefined{pdf@unescapehex}{\let\ZREF@temp=N}{}
\ifx\ZREF@temp N%
  \let\zref@hex\ltx@firstofone
  \let\zref@unhex\ltx@firstofone
\else
  \let\zref@hex\pdf@escapehex
  \let\zref@unhex\pdf@unescapehex
\fi
%    \end{macrocode}
%
%    \begin{macro}{\ifZREF@pa@list}
%    \begin{macrocode}
\ltx@newif\ifZREF@pa@list
%    \end{macrocode}
%    \end{macro}
%    \begin{macro}{\zref@listpageattr}
%    \begin{macrocode}
\ZREF@IfDefinable\zlistpageattr\def{%
  {\ZREF@pa@listtrue}%
}
%    \end{macrocode}
%    \end{macro}
%    \begin{macro}{\ZREF@pa@AfterLastShipout}
%    \begin{macrocode}
\def\ZREF@pa@AfterLastShipout{%
  \ifZREF@pa@list
    \edef\ZREF@page@max{\the\value{abspage}}%
    \ltx@ifundefined{ZREF@org@testdef}{%
      \let\ZREF@org@testdef\@testdef
      \def\@testdef##1##2##3{%
        \ZREF@org@testdef{##1}{##2}{##3}%
        \def\ZREF@temp{##1}%
        \ifx\ZREF@temp\ZREF@RefPrefix
          \expandafter\xdef\csname##1@##2\endcsname{##3}%
        \fi
      }%
    }{}%
    \AtVeryEndDocument{\ZREF@pa@AtVeryEnd}%
  \fi
}
%    \end{macrocode}
%    \end{macro}
%    \begin{macro}{\ZREF@pa@AtVeryEnd}
%    \begin{macrocode}
\let\ZREF@temp=Y%
\ltx@IfUndefined{pdfpageattr}{}{\let\ZREF@temp=N}
\ifluatex \let\ZREF@temp=N \fi
\ifx\ZREF@temp Y
 \expandafter\@firstoftwo
\else
 \expandafter\@secondoftwo
\fi
 {%
  \def\ZREF@pa@AtVeryEnd{}%
 }
 {%
  \def\ZREF@pa@AtVeryEnd{%
    \begingroup
      \toks@{List of \ltx@backslashchar
             \ifluatex pdfvariable\else pdf\fi
             pdfpageattr:\MessageBreak}%
      \count@=1 %
      \ZREF@pa@ListPage
      \edef\x{\endgroup
        \noexpand\@PackageInfoNoLine{zref-pageattr}{%
          \the\toks@
        }%
      }%
    \x
  }%
%    \end{macrocode}
%    \end{macro}
%    \begin{macro}{\zref@pageattr}
%    \begin{macrocode}
\def\zref@pdfpageattr#1{%
  \zref@unhex{%
    \zref@extract{thepage\ZREF@number{#1}}{pdfpageattr}%
  }%
}
% compability, \zref@pageattr was defined in older versions
\let\zref@pageattr\zref@pdfpageattr
%    \end{macrocode}
%    \end{macro}
%    \begin{macro}{\zref@pageattr@used}
%    \begin{macrocode}
\ZREF@Robust\def\zref@pageattr@used#1{%
  \zref@refused{thepage\ZREF@number{#1}}%
}
%    \end{macrocode}
%    \end{macro}
%    \begin{macro}{\ZREF@pa@ListPage}
%    \begin{macrocode}
  \def\ZREF@pa@ListPage{%
    \edef\x{%
      \toks@={%
        \the\toks@
        Page \the\count@:%
        \noexpand\MessageBreak
        \zref@ifrefundefined{thepage\the\count@}{}{%
          <<\zref@pdfpageattr\count@>>%
          \noexpand\MessageBreak
        }%
      }%
    }\x
    \ifnum\ZREF@page@max>\count@
      \advance\count@ by\ltx@one
    \else
      \expandafter\ltx@gobble
    \fi
    \ZREF@pa@ListPage
  }%
}
%    \end{macrocode}
%    \end{macro}
%    \begin{macrocode}
\let\ZREF@temp=Y%
\ltx@IfUndefined{pdfpagesattr}{}{\let\ZREF@temp=N}
\ifluatex \let\ZREF@temp=N \fi
\ifx\ZREF@temp N
 \expandafter\@firstofone
\fi
 {%
%    \end{macrocode}
%    \begin{macro}{\zref@pdfpagesattr}
%    \begin{macrocode}
  \def\zref@pdfpagesattr{%
    \zref@unhex{%
      \zref@extract{LastPage}{pdfpagesattr}%
    }%
  }%
%    \end{macrocode}
%    \end{macro}
%    \begin{macro}{\zref@pdfpagesattr@used}
%    \begin{macrocode}
  \ZREF@Robust\def\zref@pdfpagesattr@used{%
    \zref@refused{LastPage}%
  }%
%    \end{macrocode}
%    \end{macro}
%    \begin{macrocode}
  \ltx@LocalAppendToMacro\ZREF@pa@AtVeryEnd{%
    \@PackageInfoNoLine{zref-pageattr}{%
      \ltx@backslashchar
      \ifluatex pdfvariable\else pdf\fi
      pagesattr:\MessageBreak
      <<\zref@pdfpagesattr>>%
      \MessageBreak
    }%
  }%
}
\AfterLastShipout{%
  \ZREF@pa@AfterLastShipout
}
%    \end{macrocode}
%    \begin{macrocode}
%</pageattr>
%    \end{macrocode}
%
%
% \subsection{Module \xmodule{marks}}
%
%    \begin{macrocode}
%<*marks>
\NeedsTeXFormat{LaTeX2e}
\ProvidesPackage{zref-marks}%
  [2020-07-03 v2.32 Module marks for zref (HO)]%
\RequirePackage{zref-base}[2019/11/29]
\ifx\ZREF@base@ok Y%
\else
  \expandafter\endinput
\fi
%    \end{macrocode}
%
%    \begin{macrocode}
\newcommand*{\zref@marks@register}[3][]{%
  \edef\ZREF@TempName{#1}%
  \edef\ZREF@TempNum{\ZREF@number{#2}}%
  \ifnum\ZREF@TempNum<\ltx@zero %
    \PackageError\ZREF@name{%
      \string\zref@marks@register\ltx@space is called with invalid%
      \MessageBreak
      marks register number (\ZREF@TempNum)%
    }{%
      Use `0' or the command, defined by \string\newmarks.\MessageBreak
      \@ehc
    }%
  \else
    \ifx\ZREF@TempName\ltx@empty
      \edef\ZREF@TempName{mark\romannumeral\ZREF@TempNum}%
    \else
      \edef\ZREF@TempName{marks\ZREF@TempName}%
    \fi
    \ZREF@MARKS@DefineProp{top}%
    \ZREF@MARKS@DefineProp{first}%
    \ZREF@MARKS@DefineProp{bot}%
    \kv@parse{#3}{%
      \ifx\kv@value\relax
        \def\kv@value{top,first,bot}%
      \fi
      \edef\ZREF@temp{\expandafter\ltx@car\kv@key X\@nil}%
      \ifx\ZREF@temp\ZREF@STAR
        \edef\kv@key{\expandafter\ltx@cdr\kv@key\@nil}%
        \zref@newlist\kv@key
      \fi
      \expandafter\comma@parse\expandafter{\kv@value}{%
        \ifcase0\ifx\comma@entry\ZREF@NAME@top 1\else
                \ifx\comma@entry\ZREF@NAME@first 1\else
                \ifx\comma@entry\ZREF@NAME@bot 1\fi\fi\fi\ltx@space
          \PackageWarning{zref-marks}{%
            Use `top', `first' or `bot' for the list values%
            \MessageBreak
            in the third argument of \string\zref@marks@register.%
            \MessageBreak
            Ignoring unkown value `\comma@entry'%
          }%
        \else
          \zref@addprop{\kv@key}{\comma@entry\ZREF@TempName}%
        \fi
        \ltx@gobble
      }%
      \ltx@gobbletwo
    }%
  \fi
}
\def\ZREF@STAR{*}
\def\ZREF@NAME@top{top}
\def\ZREF@NAME@first{first}
\def\ZREF@NAME@bot{bot}
\def\ZREF@MARKS@DefineProp#1{%
  \zref@ifpropundefined{#1\ZREF@TempName}{%
    \ifnum\ZREF@TempNum=\ltx@zero
      \begingroup
        \edef\x{\endgroup
          \noexpand\zref@newprop*{#1\ZREF@TempName}[]{%
            \expandafter\noexpand\csname#1mark\endcsname
          }%
        }%
      \x
    \else
      \begingroup
        \edef\x{\endgroup
          \noexpand\zref@newprop*{#1\ZREF@TempName}[]{%
            \expandafter\noexpand\csname#1marks\endcsname
            \ZREF@TempNum
          }%
        }%
      \x
    \fi
  }{%
    \PackageWarning{zref-marks}{%
      \string\zref@marks@register\ltx@space does not generate the%
      \MessageBreak
      new property `#1\ZREF@TempName', because\MessageBreak
      it is already defined%
    }%
  }%
}
%    \end{macrocode}
%    \begin{macrocode}
%</marks>
%    \end{macrocode}
%
% \subsection{Module \xmodule{runs}}
%
%    This module does not use the label-reference-system.
%    The reference changes with each \hologo{LaTeX}~run and would
%    force a rerun warning always.
%    \begin{macrocode}
%<*runs>
\NeedsTeXFormat{LaTeX2e}
\ProvidesPackage{zref-runs}%
  [2020-07-03 v2.32 Module runs for zref (HO)]%
%    \end{macrocode}
%
%    \begin{macro}{\zruns}
%    \begin{macrocode}
\providecommand*{\zruns}{0}%
\AtBeginDocument{%
  \edef\zruns{\number\numexpr\zruns+1}%
  \begingroup
    \def\on@line{}%
    \PackageInfo{zref-runs}{LaTeX runs: \zruns}%
    \if@filesw
      \immediate\write\@mainaux{%
        \string\gdef\string\zruns{\zruns}%
      }%
    \fi
  \endgroup
}
%    \end{macrocode}
%    \end{macro}
%
%    \begin{macrocode}
%</runs>
%    \end{macrocode}
%
%
% \subsection{Module \xmodule{perpage}}
%
%    \begin{macrocode}
%<*perpage>
\NeedsTeXFormat{LaTeX2e}
\ProvidesPackage{zref-perpage}%
  [2020-07-03 v2.32 Module perpage for zref (HO)]%
\RequirePackage{zref-base}[2019/11/29]
\ifx\ZREF@base@ok Y%
\else
  \expandafter\endinput
\fi
%    \end{macrocode}
%
%    This module resets a counter at page boundaries.
%    Because of the asynchronous output routine
%    page counter properties cannot be asked directly,
%    references are necessary.
%
%    For detecting changed pages module \xmodule{abspage}
%    is loaded.
%    \begin{macrocode}
\RequirePackage{zref-abspage}[2019/11/29]
%    \end{macrocode}
%
%    We group the properties for the needed references
%    in the property list |perpage|. The property |pagevalue|
%    records the correct value of the page counter.
%    \begin{macrocode}
\ZREF@NewPropPageValue
\zref@newlist{perpage}
\zref@addprops{perpage}{abspage,page,pagevalue}
%    \end{macrocode}
%    The page value, known by the reference mechanism, will be
%    stored in counter |zpage|.
%    \begin{macrocode}
\newcounter{zpage}
%    \end{macrocode}
%    Counter |zref@unique| helps in generating unique reference
%    names.
%    \begin{macrocode}
\zref@require@unique
%    \end{macrocode}
%    In order to be able to reset the counter, we hook here
%    into \cs{stepcounter}. In fact two nested hooks are used
%    to allow other packages to use the first hook at the beginning
%    of \cs{stepcounter}.
%    \begin{macrocode}
\let\ZREF@org@stepcounter\stepcounter
\def\stepcounter#1{%
  \ifcsname @stepcounterhook@#1\endcsname
    \csname @stepcounterhook@#1\endcsname
  \fi
  \ZREF@org@stepcounter{#1}%
}
%    \end{macrocode}
%    \cs{@stpelt} must be adapted due to the change in latex
%    2015-01, see https://github.com/ho-tex/zref/issues/26
%    \begin{macrocode}
\let\ZREF@org@@stpelt\@stpelt
\def\@stpelt#1{%
  \ifcsname ZREF@perpage@#1\endcsname
    \begingroup
      \let\stepcounter\ZREF@org@stepcounter
      \ZREF@org@@stpelt{#1}%
    \endgroup
    \expandafter\ltx@gobbletwo
  \fi
  \ZREF@org@@stpelt{#1}%
}
%    \end{macrocode}
%
%    \begin{macro}{\zmakeperpage}
%    Makro \cs{zmakeperpage} resets a counter at each page
%    break. It uses the same syntax and semantics as
%    \cs{MakePerPage} from package \xpackage{perpage} \cite{perpage}.
%    The initial start value can be given by the optional
%    argument. Default is one that means after the first
%    \cs{stepcounter} on a new page the counter starts with
%    one.
%    \begin{macrocode}
\ZREF@IfDefinable\zmakeperpage\def{%
  {%
    \@ifnextchar[\ZREF@makeperpage@opt{\ZREF@@makeperpage[\ltx@zero]}%
  }%
}
%    \end{macrocode}
%    \end{macro}
%    We hook before the counter is incremented in \cs{stepcounter},
%    package \xpackage{perpage} afterwards. Thus a little
%    calculation is necessary.
%    \begin{macrocode}
\def\ZREF@makeperpage@opt[#1]{%
  \begingroup
    \edef\x{\endgroup
      \noexpand\ZREF@@makeperpage[\number\numexpr#1-1\relax]%
    }%
  \x
}
%    \end{macrocode}
%    \begin{macrocode}
\def\ZREF@@makeperpage[#1]#2{%
  \@ifundefined{@stepcounterhook@#2}{%
    \expandafter\gdef\csname @stepcounterhook@#2\endcsname{}%
  }{}%
  \expandafter\gdef\csname ZREF@perpage@#2\endcsname{%
    \ZREF@@perpage@step{#2}{#1}%
  }%
  \expandafter\g@addto@macro\csname @stepcounterhook@#2\endcsname{%
    \ifcsname ZREF@perpage@#2\endcsname
      \csname ZREF@perpage@#2\endcsname
    \fi
  }%
}
%    \end{macrocode}
%    \begin{macro}{\ZREF@@perpage@step}
%    The heart of this module follows.
%    \begin{macrocode}
\def\ZREF@@perpage@step#1#2{%
%    \end{macrocode}
%    First the reference is generated.
%    \begin{macrocode}
  \global\advance\c@zref@unique\ltx@one
  \begingroup
    \expandafter
    \zref@labelbylist\expandafter{\thezref@unique}{perpage}%
%    \end{macrocode}
%    The \cs{expandafter} commands are necessary, because \cs{ZREF@temp}
%    is also used inside of \cs{zref@labelbylist}.
%
%    The evaluation of the reference follows. If the reference
%    is not yet kwown, we use the page counter as approximation.
%    \begin{macrocode}
    \zref@ifrefundefined\thezref@unique{%
      \global\c@zpage=\c@page
      \global\let\thezpage\thepage
      \expandafter\xdef\csname ZREF@abspage@#1\endcsname{%
        \number\c@abspage
      }%
    }{%
%    \end{macrocode}
%    The reference is used to set \cs{thezpage} and counter |zpage|.
%    \begin{macrocode}
      \global\c@zpage=\zref@extract\thezref@unique{pagevalue}\relax
      \xdef\thezpage{\noexpand\zref@extract{\thezref@unique}{page}}%
      \expandafter\xdef\csname ZREF@abspage@#1\endcsname{%
        \zref@extractdefault\thezref@unique
            {abspage}{\number\c@abspage}%
      }%
    }%
%    \end{macrocode}
%    Page changes are detected by a changed absolute page number.
%    \begin{macrocode}
    \expandafter\ifx\csname ZREF@abspage@#1\expandafter\endcsname
                    \csname ZREF@currentabspage@#1\endcsname
    \else
      \global\csname c@#1\endcsname=#2\relax
      \global\expandafter\let
          \csname ZREF@currentabspage@#1\expandafter\endcsname
          \csname ZREF@abspage@#1\endcsname
    \fi
  \endgroup
}
%    \end{macrocode}
%    \end{macro}
%    \begin{macro}{\zunmakeperpage}
%    Macro \cs{zunmakeperpage} cancels the effect of \cs{zmakeperpage}.
%    \begin{macrocode}
\ZREF@IfDefinable\zunmakeperpage\def{%
  #1{%
    \global\expandafter
    \let\csname ZREF@perpage@#1\endcsname\@undefined
  }%
}
%    \end{macrocode}
%    \end{macro}
%
%    \begin{macrocode}
%</perpage>
%    \end{macrocode}
%
%
% \subsection{Module \xmodule{titleref}}
%
%    \begin{macrocode}
%<*titleref>
\NeedsTeXFormat{LaTeX2e}
\ProvidesPackage{zref-titleref}%
  [2020-07-03 v2.32 Module titleref for zref (HO)]%
\RequirePackage{zref-base}[2019/11/29]
\ifx\ZREF@base@ok Y%
\else
  \expandafter\endinput
\fi
\RequirePackage{gettitlestring}[2009/12/08]
%    \end{macrocode}
%
% \subsubsection{Implementation}
%    \begin{macrocode}
\RequirePackage{keyval}
%    \end{macrocode}
%
%    This module makes section and caption titles available
%    for the reference system. It uses some of the ideas
%    of package \xpackage{nameref} and \xpackage{titleref}.
%
%    Now we can add the property \verb|title| is added to
%    the main property list.
%    \begin{macrocode}
\ZREF@NewPropTitle
\zref@addprop\ZREF@mainlist{title}%
%    \end{macrocode}
%
%    The title strings go into the \xfile{.aux} file, thus
%    they need some kind of protection.
%    Package \xpackage{titleref} uses a protected expansion
%    method. The advantage is that this can be used to
%    cleanup the string and to remove \cs{label}, \cs{index}
%    and other macros unwanted for referencing.
%    But there is the risk that fragile stuff can break.
%
%    Therefore package \xpackage{nameref} does not expand
%    the string. Thus the entries can safely be written to
%    the \xfile{.aux} file. But potentially dangerous
%    macros such as \cs{label} remain in the string and
%    can cause problems when using the string in references.
%
%    \begin{macro}{\ifzref@titleref@expand}
%    The switch \cs{ifzref@titleref@expand} distinguishes
%    between the both methods. Package \xpackage{nameref}'s
%    behaviour is achieved by setting the switch to false,
%    otherwise \xpackage{titleref}'s expansion is used.
%    Default is false.
%    \begin{macrocode}
\newif\ifzref@titleref@expand
%    \end{macrocode}
%    \end{macro}
%
%    \begin{macro}{\ZREF@titleref@hook}
%    The hook \cs{ZREF@titleref@hook} allows to extend
%    the cleanup for the expansion method. Thus
%    unnecessary macros can be removed or dangerous commands
%    removed. The hook is executed before the expansion of
%    \cs{zref@titleref@current}.
%    \begin{macrocode}
\let\ZREF@titleref@hook\ltx@empty
%    \end{macrocode}
%    \end{macro}
%    \begin{macro}{\zref@titleref@cleanup}
%    The hook should not be used directly, instead we provide the
%    macro \cs{zref@titleref@cleanup} to add stuff to the hook
%    and prevents that a previous non-empty content is not discarded
%    accidently.
%    \begin{macrocode}
\ZREF@Robust\def\zref@titleref@cleanup#1{%
   \begingroup
    \toks@\expandafter{%
      \ZREF@titleref@hook
      #1%
    }%
  \expandafter\endgroup
  \expandafter\def\expandafter\ZREF@titleref@hook\expandafter{%
    \the\toks@
  }%
}%
%    \end{macrocode}
%    \end{macro}
%
%    \begin{macro}{\ifzref@titleref@stripperiod}
%    Sometimes a title contains a period at the end.
%    Package \xpackage{nameref} removes this. This behaviour
%    is controlled by the switch \cs{ifzref@titleref@stripperiod}
%    and works regardless of the setting of option \xoption{expand}.
%    Period stripping is the default.
%    \begin{macrocode}
\newif\ifzref@titleref@stripperiod
\zref@titleref@stripperiodtrue
%    \end{macrocode}
%    \end{macro}
%
%    \begin{macro}{\zref@titleref@setcurrent}
%    Macro \cs{zref@titleref@setcurrent} sets a new current title
%    stored in \cs{zref@titleref@current}. Some cleanup and
%    expansion is performed that can be controlled by the
%    previous switches.
%    \begin{macrocode}
\ZREF@Robust\def\zref@titleref@setcurrent#1{%
  \ifzref@titleref@expand
    \GetTitleStringExpand{#1}%
  \else
    \GetTitleStringNonExpand{#1}%
  \fi
  \edef\zref@titleref@current{%
    \detokenize\expandafter{\GetTitleStringResult}%
  }%
  \ifzref@titleref@stripperiod
    \edef\zref@titleref@current{%
      \expandafter\ZREF@stripperiod\zref@titleref@current
      \ltx@empty.\ltx@empty\@nil
    }%
  \fi
}%
\GetTitleStringDisableCommands{%
  \ZREF@titleref@hook
}
%    \end{macrocode}
%    \end{macro}
%    \begin{macro}{\ZREF@stripperiod}
%    If \cs{ZREF@stripperiod} is called, the argument consists of
%    space tokens and tokens with catcode 12 (other), because
%    of \hologo{eTeX}'s \cs{detokenize}.
%    \begin{macrocode}
\def\ZREF@stripperiod#1.\ltx@empty#2\@nil{#1}%
%    \end{macrocode}
%    \end{macro}
%
% \subsubsection{User interface}
%
%    \begin{macro}{\ztitlerefsetup}
%    The behaviour of module \xmodule{titleref} is controlled
%    by switches and a hook. They can be set by \cs{ztitlerefsetup}
%    with a key value interface, provided by package \xpackage{keyval}.
%    Also the current title can be given explicitly by the key
%    \verb|title|.
%    \begin{macrocode}
\define@key{ZREF@TR}{expand}[true]{%
  \csname zref@titleref@expand#1\endcsname
}%
\define@key{ZREF@TR}{stripperiod}[true]{%
  \csname zref@titleref@stripperiod#1\endcsname
}%
\define@key{ZREF@TR}{cleanup}{%
  \zref@titleref@cleanup{#1}%
}%
\define@key{ZREF@TR}{title}{%
  \def\zref@titleref@current{#1}%
}%
\ZREF@IfDefinable\ztitlerefsetup\def{%
  {\kvsetkeys{ZREF@TR}}%
}%
%    \end{macrocode}
%    \end{macro}
%
%    \begin{macro}{\ztitleref}
%    The user command \cs{ztitleref} references the title.
%    For safety \cs{label} is disabled to prevent multiply
%    defined references.
%    \begin{macrocode}
\ZREF@IfDefinable\ztitleref\def{%
  {\zref@wrapper@babel\ZREF@titleref}%
}%
\def\ZREF@titleref#1{%
  \begingroup
    \zref@refused{#1}%
    \let\label\ltx@gobble
    \zref@extract{#1}{title}%
  \endgroup
}%
%    \end{macrocode}
%    \end{macro}
%
% \subsubsection{Patches for section and caption commands}
%
%    The section and caption macros are patched to extract the
%    title data.
%
%    Captions of figures and tables.
%    \begin{macrocode}
\AtBeginDocument{%
  \ZREF@patch{@caption}{%
    \long\def\@caption#1[#2]{%
      \zref@titleref@setcurrent{#2}%
      \ZREF@org@@caption{#1}[{#2}]%
    }%
  }%
%    \end{macrocode}
%
%    Section commands without star. The title version for the
%    table of contents is used because it is usually shorter and
%    more robust.
%    \begin{macrocode}
  \ZREF@patch{@part}{%
    \def\@part[#1]{%
      \zref@titleref@setcurrent{#1}%
      \ZREF@org@@part[{#1}]%
    }%
  }%
  \ZREF@patch{@chapter}{%
    \def\@chapter[#1]{%
      \zref@titleref@setcurrent{#1}%
      \ZREF@org@@chapter[{#1}]%
    }%
  }%
  \ZREF@patch{@sect}{%
    \def\@sect#1#2#3#4#5#6[#7]{%
      \zref@titleref@setcurrent{#7}%
      \ZREF@org@@sect{#1}{#2}{#3}{#4}{#5}{#6}[{#7}]%
    }%
  }%
%    \end{macrocode}
%    The star versions of the section commands.
%    \begin{macrocode}
  \ZREF@patch{@spart}{%
    \def\@spart#1{%
      \zref@titleref@setcurrent{#1}%
      \ZREF@org@@spart{#1}%
    }%
  }%
  \ZREF@patch{@schapter}{%
    \def\@schapter#1{%
      \zref@titleref@setcurrent{#1}%
      \ZREF@org@@schapter{#1}%
    }%
  }%
  \ZREF@patch{@ssect}{%
    \def\@ssect#1#2#3#4#5{%
      \zref@titleref@setcurrent{#5}%
      \ZREF@org@@ssect{#1}{#2}{#3}{#4}{#5}%
    }%
  }%
%    \end{macrocode}
%
% \subsubsection{Environment \textsf{description}}
%
%    \begin{macrocode}
  \ZREF@patch{descriptionlabel}{%
    \def\descriptionlabel#1{%
      \zref@titleref@setcurrent{#1}%
      \ZREF@org@descriptionlabel{#1}%
    }%
  }%
%    \end{macrocode}
%
% \subsubsection{Class \xclass{memoir}}
%
%    \begin{macrocode}
  \@ifclassloaded{memoir}{%
    \ltx@IfUndefined{ifheadnameref}{}{%
      \def\@chapter[#1]#2{%
        \ltx@IfUndefined{ch@pt@c}{%
          \zref@titleref@setcurrent{#1}%
        }{%
          \ifx\ch@pt@c\ltx@empty
            \zref@titleref@setcurrent{#2}%
          \else
            \def\NR@temp{#1}%
            \ifx\NR@temp\ltx@empty
              \expandafter\zref@titleref@setcurrent
              \expandafter{\ch@pt@c}%
            \else
              \ifheadnameref
                \zref@titleref@setcurrent{#1}%
              \else
                \expandafter\zref@titleref@setcurrent
                \expandafter{\ch@pt@c}%
              \fi
            \fi
          \fi
        }%
        \ZREF@org@@chapter[{#1}]{#2}%
      }%
      \ZREF@patch{M@sect}{%
        \def\M@sect#1#2#3#4#5#6[#7][#8]{%
          \ifheadnameref
            \zref@titleref@setcurrent{#8}%
          \else
            \zref@titleref@setcurrent{#7}%
          \fi
          \ZREF@org@M@sect{#1}{#2}{#3}{#4}{#5}{#6}[{#7}][{#8}]%
        }%
      }%
    }%
  }{}%
%    \end{macrocode}
%
% \subsubsection{Class \xclass{beamer}}
%
%    \begin{macrocode}
  \@ifclassloaded{beamer}{%
    \ZREF@patch{beamer@section}{%
      \long\def\beamer@section[#1]{%
        \zref@titleref@setcurrent{#1}%
        \ZREF@org@beamer@section[{#1}]%
      }%
    }%
    \ZREF@patch{beamer@subsection}{%
      \long\def\beamer@subsection[#1]{%
        \zref@titleref@setcurrent{#1}%
        \ZREF@org@beamer@subsection[{#1}]%
      }%
    }%
    \ZREF@patch{beamer@subsubsection}{%
      \long\def\beamer@subsubsection[#1]{%
        \zref@titleref@setcurrent{#1}%
        \ZREF@org@beamer@subsubsection[{#1}]%
      }%
    }%
  }{}%
%    \end{macrocode}
%
% \subsubsection{Package \xpackage{titlesec}}
%
%    \begin{macrocode}
  \@ifpackageloaded{titlesec}{%
    \ZREF@patch{ttl@sect@i}{%
      \def\ttl@sect@i#1#2[#3]#4{%
        \zref@titleref@setcurrent{#4}%
        \ZREF@org@ttl@sect@i{#1}{#2}[{#3}]{#4}%
      }%
    }%
    \ZREF@patch{ttl@straight@i}{%
      \def\ttl@straight@i#1[#2]#3{%
        \def\ZREF@temp{#2}%
        \ifx\ZREF@temp\ltx@empty
          \zref@titleref@setcurrent{#3}%
        \else
          \zref@titleref@setcurrent{#2}%
        \fi
        \ZREF@org@ttl@straight@i{#1}[{#2}]{#3}%
      }%
    }%
  }{}%
%    \end{macrocode}
%
% \subsubsection{Package \xpackage{longtable}}
%
%    Package \xpackage{longtable}: some support for its \cs{caption}.
%    However \cs{label} inside the caption is not supported.
%    \begin{macrocode}
  \@ifpackageloaded{longtable}{%
    \ZREF@patch{LT@c@ption}{%
      \def\LT@c@ption#1[#2]#3{%
        \ZREF@org@LT@c@ption{#1}[{#2}]{#3}%
        \zref@titleref@setcurrent{#2}%
      }%
    }%
  }{}%
%    \end{macrocode}
%
% \subsubsection{Package \xpackage{listings}}
%
%    Package \xpackage{listings}: support for its caption.
%    \begin{macrocode}
  \@ifpackageloaded{listings}{%
    \ZREF@patch{lst@MakeCaption}{%
      \def\lst@MakeCaption{%
        \ifx\lst@label\ltx@empty
        \else
          \expandafter\zref@titleref@setcurrent\expandafter{%
            \lst@@caption
          }%
        \fi
        \ZREF@org@lst@MakeCaption
      }%
    }%
  }{}%
%    \end{macrocode}
%
% \subsubsection{Theorems}
%
%    \begin{macrocode}
  \ZREF@patch{@opargbegintheorem}{%
    \def\@opargbegintheorem#1#2#3{%
      \zref@titleref@setcurrent{#3}%
      \ZREF@org@@opargbegintheorem{#1}{#2}{#3}%
    }%
  }%
  \@ifpackageloaded{amsthm}{%
    \begingroup
      \edef\x{macro:\string#1\string#2[\string#3]}%
      \@onelevel@sanitize\x
      \def\y#1->#2\@nil{#1}%
      \edef\z{\expandafter\y\meaning\@begintheorem->\@nil}%
      \@onelevel@sanitize\z
    \expandafter\endgroup
    \ifx\x\z
      \ZREF@patch{@begintheorem}{%
        \def\@begintheorem#1#2[#3]{%
          \zref@titleref@setcurrent{#3}%
          \ZREF@org@@begintheorem{#1}{#2}[{#3}]%
        }%
      }%
    \fi
  }{}%
%    \end{macrocode}
%
%    \begin{macrocode}
}
%    \end{macrocode}
%
%    \begin{macrocode}
%</titleref>
%    \end{macrocode}
%
%
% \subsection{Module \xmodule{xr}}
%
%    \begin{macrocode}
%<*xr>
\NeedsTeXFormat{LaTeX2e}
\ProvidesPackage{zref-xr}%
  [2020-07-03 v2.32 Module xr for zref (HO)]%
\RequirePackage{zref-base}[2019/11/29]
\ifx\ZREF@base@ok Y%
\else
  \expandafter\endinput
\fi
%    \end{macrocode}
%    \begin{macrocode}
\RequirePackage{keyval}
\RequirePackage{kvoptions}[2010/02/22]
%    \end{macrocode}
%
%    We declare property |url|, because this is added,
%    if a reference is imported and has not already set this field.
%    Or if |hyperref| is used, then this property can be asked.
%    \begin{macrocode}
\zref@newprop{url}{}%
\zref@newprop{urluse}{}%
%    \end{macrocode}
%    \begin{macrocode}
\zref@newprop{externaldocument}{}%
%    \end{macrocode}
%
%    Most code, especially the handling of the \xfile{.aux}
%    files are taken from David Carlisle's \xpackage{xr} package.
%    Therefore I drop the documentation for these macros here.
%
%    \begin{macro}{\zref@xr@ext}
%    If the URL is not specied, then assume
%    processed file with a guessed extension.
%    Use the setting of \xpackage{hyperref} if available.
%    \begin{macrocode}
\providecommand*{\zref@xr@ext}{%
  \ltx@ifundefined{XR@ext}{pdf}{\XR@ext}%
}%
%    \end{macrocode}
%    \end{macro}
%    \begin{macro}{\ifZREF@xr@zreflabel}
%    The use of the star form of \cs{zexternaldocument}
%    is remembered in the switch \cs{ifZREF@xr@zreflabel}.
%    \begin{macrocode}
\newif\ifZREF@xr@zreflabel
%    \end{macrocode}
%    \end{macro}
%    \begin{macrocode}
\SetupKeyvalOptions{%
  family=ZREF@XR,%
  prefix=ZREF@xr@%
}
\DeclareBoolOption[true]{tozreflabel}
\DeclareBoolOption[false]{toltxlabel}
\DeclareBoolOption{verbose}
\define@key{ZREF@XR}{ext}{%
  \def\zref@xr@{#1}%
}
\DeclareBoolOption[false]{urluse}
%    \end{macrocode}
%    \begin{macro}{\zxrsetup}
%    \begin{macrocode}
\newcommand*{\zxrsetup}{%
  \kvsetkeys{ZREF@XR}%
}%
%    \end{macrocode}
%    \end{macro}
%
%    \begin{macro}{\ZREF@xr@URL}
%    \begin{macrocode}
\newcount\ZREF@xr@URL
\ZREF@xr@URL=\ltx@zero
%    \end{macrocode}
%    \end{macro}
%    \begin{macro}{\ZREF@xr@AddURL}
%    \begin{macrocode}
\def\ZREF@xr@AddURL#1{%
  \begingroup
    \def\ZREF@temp{#1}%
    \count@=\ltx@one
    \ZREF@xr@@AddUrl
  \endgroup
}
%    \end{macrocode}
%    \end{macro}
%    \begin{macro}{\ZREF@xr@@AddUrl}
%    \begin{macrocode}
\def\ZREF@xr@@AddUrl{%
  \ifnum\count@>\ZREF@xr@URL
    \global\advance\ZREF@xr@URL by\ltx@one
    \xdef\ZREF@xr@theURL{\romannumeral\ZREF@xr@URL}%
    \global\expandafter\let
        \csname Z@U@\ZREF@xr@theURL\endcsname\ZREF@temp
    \@PackageInfo{zref-xr}{%
      \ltx@backslashchar Z@U@\ZREF@xr@theURL:\MessageBreak
      \ZREF@temp\MessageBreak
    }%
  \else
    \expandafter
    \ifx\csname Z@U@\romannumeral\count@\endcsname\ZREF@temp
      \xdef\ZREF@xr@theURL{\romannumeral\count@}%
    \else
      \expandafter\expandafter\expandafter\ZREF@xr@@AddUrl
    \fi
  \fi
}
%    \end{macrocode}
%    \end{macro}
%
%    \begin{macro}{\zexternaldocument}
%    In its star form it looks for \cs{newlabel},
%    otherwise for \cs{zref@newlabel}.
%     Later we
%    will read \xfile{.aux} files that expects
%    |@| to have catcode 11 (letter).
%    \begin{macrocode}
\ZREF@IfDefinable\zexternaldocument\def{%
  {%
    \ZREF@NewPropAnchor
    \ZREF@NewPropTitle
    \begingroup
      \csname @safe@actives@true\endcsname
      \makeatletter
      \@ifstar{%
        \ZREF@xr@zreflabelfalse
        \@testopt\ZREF@xr@externaldocument{}%
      }{%
        \ZREF@xr@zreflabeltrue
        \@testopt\ZREF@xr@externaldocument{}%
      }%
  }%
}%
%    \end{macrocode}
%    \end{macro}
%
%    If the \cs{include} featuer was used, there can be
%    several \xfile{.aux} files. These files are read
%    one after another, especially they are not recursively
%    read in order to save read registers. Thus it can
%    happen that the read order of the newlabel commands
%    differs from \hologo{LaTeX}'s order using \cs{input}.
%
%    \begin{macro}{\ZREF@xr@externaldocument}
%    It reads the remaining arguments. \cs{newcommand}
%    comes in handy for the optional argument.
%    \begin{macrocode}
\def\ZREF@xr@externaldocument[#1]#2{%
    \def\ZREF@xr@prefix{#1}%
    \let\ZREF@xr@filelist\ltx@empty
    \edef\ZREF@xr@externalfile{#2}%
    \edef\ZREF@xr@file{\ZREF@xr@externalfile.aux}%
    \filename@parse{#2}%
    \@testopt\ZREF@xr@graburl{#2.\zref@xr@ext}%
}%
\def\ZREF@xr@graburl[#1]{%
    \edef\ZREF@xr@url{#1}%
    \ifZREF@xr@urluse
      \expandafter\ZREF@xr@AddURL\expandafter{\ZREF@xr@url}%
      \expandafter\def\expandafter\ZREF@xr@url
      \expandafter{\csname Z@U@\ZREF@xr@theURL\endcsname}%
    \fi
    \ZREF@xr@checkfile
  \endgroup
}%
%    \end{macrocode}
%    \end{macro}
%    \begin{macro}{\ZREF@xr@processfile}
%    We follow \xpackage{xr} here, \cs{IfFileExists} offers
%    a nicer test, but we have to open the file anyway.
%    \begin{macrocode}
\def\ZREF@xr@checkfile{%
  \openin\@inputcheck\ZREF@xr@file\relax
  \ifeof\@inputcheck
    \PackageWarning{zref-xr}{%
      File `\ZREF@xr@file' not found or empty,\MessageBreak
      labels not imported%
    }%
  \else
    \PackageInfo{zref-xr}{%
      Label \ifZREF@xr@zreflabel (zref) \fi
      import from `\ZREF@xr@file'%
    }%
    \def\ZREF@xr@found{0}%
    \def\ZREF@xr@ignored@empty{0}%
    \def\ZREF@xr@ignored@zref{0}%
    \def\ZREF@xr@ignored@ltx{0}%
    \ZREF@xr@processfile
    \closein\@inputcheck
    \begingroup
      \let\on@line\ltx@empty
      \PackageInfo{zref-xr}{%
        Statistics for `\ZREF@xr@file':\MessageBreak
        \ZREF@xr@found\space
        \ifZREF@xr@zreflabel zref\else LaTeX\fi\space
        label(s) found%
        \ifnum\ZREF@xr@ignored@empty>0 %
          ,\MessageBreak
          \ZREF@xr@ignored@empty\space empty label(s) ignored%
        \fi
        \ifnum\ZREF@xr@ignored@zref>0 %
          ,\MessageBreak
          \ZREF@xr@ignored@zref\space
          duplicated zref label(s) ignored%
        \fi
        \ifnum\ZREF@xr@ignored@ltx>0 %
          ,\MessageBreak
          \ZREF@xr@ignored@ltx\space
          duplicated latex label(s) ignored%
        \fi
      }%
    \endgroup
  \fi
  \ifx\ZREF@xr@filelist\ltx@empty
  \else
    \edef\ZREF@xr@file{%
      \expandafter\ltx@car\ZREF@xr@filelist\@nil
    }%
    \edef\ZREF@xr@filelist{%
      \expandafter\ltx@cdr\ZREF@xr@filelist\ltx@empty\@nil
    }%
    \expandafter\ZREF@xr@checkfile
  \fi
}%
%    \end{macrocode}
%    \end{macro}
%
%    \begin{macro}{\ZREF@xr@processfile}
%    \begin{macrocode}
\def\ZREF@xr@processfile{%
  \read\@inputcheck to\ZREF@xr@line
  \expandafter\ZREF@xr@processline\ZREF@xr@line..\ZREF@nil
  \ifeof\@inputcheck
  \else
    \expandafter\ZREF@xr@processfile
  \fi
}%
%    \end{macrocode}
%    \end{macro}
%    \begin{macro}{\ZREF@xr@processline}
%    The most work must be done for analyzing the
%    arguments of \cs{newlabel}.
%    \begin{macrocode}
\long\def\ZREF@xr@processline#1#2#3\ZREF@nil{%
  \def\x{#1}%
  \toks@{#2}%
  \ifZREF@xr@zreflabel
    \ifx\x\ZREF@xr@zref@newlabel
      \expandafter
      \ZREF@xr@process@zreflabel\ZREF@xr@line...\ZREF@nil
    \fi
  \else
    \ifx\x\ZREF@xr@newlabel
      \expandafter
      \ZREF@xr@process@label\ZREF@xr@line...[]\ZREF@nil
    \fi
  \fi
  \ifx\x\ZREF@xr@@input
    \edef\ZREF@xr@filelist{%
      \etex@unexpanded\expandafter{\ZREF@xr@filelist}%
      {\filename@area\the\toks@}%
    }%
  \fi
}%
\def\ZREF@xr@process@zreflabel\zref@newlabel#1#2#3\ZREF@nil{%
  \edef\ZREF@xr@refname{Z@R@\ZREF@xr@prefix#1}%
  \edef\ZREF@xr@found{\the\numexpr\ZREF@xr@found+1\relax}%
  \def\x{#2}%
  \edef\ZREF@xr@tempname{$temp$}%
  \edef\ZREF@xr@temprefname{Z@R@\ZREF@xr@tempname}%
  \let\ZREF@xr@list\x
  \ifx\ZREF@xr@list\ltx@empty
    \PackageWarningNoLine{zref-xr}{%
      Label `#1' without properties ignored\MessageBreak
      in file `\ZREF@xr@file'%
    }%
    \edef\ZREF@xr@ignored@empty{%
      \the\numexpr\ZREF@xr@ignored@empty+1\relax
    }%
  \else
    \expandafter\ZREF@xr@checklist\x\ZREF@nil
    \expandafter\let\csname\ZREF@xr@temprefname\endcsname\x
    \expandafter\ltx@LocalAppendToMacro
    \csname\ZREF@xr@temprefname\expandafter\endcsname
    \expandafter{%
      \expandafter\externaldocument\expandafter{%
        \ZREF@xr@externalfile
      }%
    }%
    \ZREF@xr@urlcheck\ZREF@xr@tempname
    \ifZREF@xr@tozreflabel
      \@ifundefined{\ZREF@xr@refname}{%
        \ifZREF@xr@verbose
          \PackageInfo{zref-xr}{%
            Import to zref label `\ZREF@xr@tempname#1'%
          }%
        \fi
        \global\expandafter
        \let\csname\ZREF@xr@refname\expandafter\endcsname
        \csname\ZREF@xr@temprefname\endcsname
      }{%
        \ZREF@xr@zref@ignorewarning{\ZREF@xr@prefix#1}%
      }%
    \fi
    \ifZREF@xr@toltxlabel
      \ZREF@xr@tolabel{\ZREF@xr@tempname}{\ZREF@xr@prefix#1}%
    \fi
  \fi
}%
\def\ZREF@xr@process@label\newlabel#1#2#3[#4]#5\ZREF@nil{%
  \def\ZREF@xr@refname{Z@R@\ZREF@xr@prefix#1}%
  \edef\ZREF@xr@found{\the\numexpr\ZREF@xr@found+1\relax}%
  \def\x{#2}%
  \edef\ZREF@xr@tempname{$temp$}%
  \edef\ZREF@xr@temprefname{Z@R@\ZREF@xr@tempname}%
  \expandafter\ZREF@xr@scanparams
      \csname\ZREF@xr@temprefname\expandafter\endcsname
      \x{}{}{}{}{}\ZREF@nil
  \ifx\\#4\\%
  \else
    % ntheorem knows an optional argument at the end of \newlabel
    \ZREF@NewPropTheotype
    \expandafter\ltx@LocalAppendToMacro
        \csname\ZREF@xr@temprefname\endcsname{\theotype{#4}}%
  \fi
  \expandafter\ltx@LocalAppendToMacro
  \csname\ZREF@xr@temprefname\expandafter\endcsname\expandafter{%
    \expandafter\externaldocument\expandafter{%
      \ZREF@xr@externalfile
    }%
  }%
  \ZREF@xr@urlcheck\ZREF@xr@tempname
  \ifZREF@xr@tozreflabel
    \@ifundefined{\ZREF@xr@refname}{%
      \ifZREF@xr@verbose
        \PackageInfo{zref-xr}{%
          Import to zref label `\ZREF@xr@prefix#1'%
        }%
      \fi
      \global\expandafter
      \let\csname\ZREF@xr@refname\expandafter\endcsname
      \csname\ZREF@xr@temprefname\endcsname
    }{%
      \ZREF@xr@zref@ignorewarning{\ZREF@xr@prefix#1}%
    }%
  \fi
  \ifZREF@xr@toltxlabel
    \ZREF@xr@tolabel{\ZREF@xr@tempname}{\ZREF@xr@prefix#1}%
  \fi
}
\def\ZREF@xr@zref@newlabel{\zref@newlabel}%
\def\ZREF@xr@newlabel{\newlabel}%
\def\ZREF@xr@@input{\@input}%
\def\ZREF@xr@relax{\relax}%
%    \end{macrocode}
%    \end{macro}
%    \begin{macro}{\ZREF@xr@tolabel}
%    \begin{macrocode}
\def\ZREF@xr@tolabel#1#2{%
  \ifZREF@xr@verbose
    \PackageInfo{zref-xr}{%
      Import to LaTeX label `#2'%
    }%
  \fi
  \zref@wrapper@unexpanded{%
    \expandafter\xdef\csname r@#2\endcsname{%
      {%
        \ltx@ifundefined{M@TitleReference}{%
          \ltx@ifundefined{TR@TitleReference}{%
            \zref@extractdefault{#1}{default}{}%
          }{%
            \noexpand\TR@TitleReference
            {\zref@extractdefault{#1}{default}{}}%
            {\zref@extractdefault{#1}{title}{}}%
          }%
        }{%
          \noexpand\M@TitleReference
          {\zref@extractdefault{#1}{default}{}}%
          {\zref@extractdefault{#1}{title}{}}%
        }%
      }%
      {\zref@extractdefault{#1}{page}{}}%
      \ltx@ifpackageloaded{nameref}{%
        {\zref@extractdefault{#1}{title}{}}%
        {\zref@extractdefault{#1}{anchor}{}}%
        \zref@ifrefcontainsprop{#1}{urluse}{%
          {\zref@extractdefault{#1}{urluse}{}}%
        }{%
          {\zref@extractdefault{#1}{url}{}}%
        }%
      }{}%
    }%
  }%
}
%    \end{macrocode}
%    \end{macro}
%
%    \begin{macro}{\ZREF@xr@zref@ignorewarning}
%    \begin{macrocode}
\def\ZREF@xr@zref@ignorewarning#1{%
  \PackageWarningNoLine{zref-xr}{%
    Zref label `#1' is already in use\MessageBreak
    in file `\ZREF@xr@file'%
  }%
  \edef\ZREF@xr@ignored@zref{%
    \the\numexpr\ZREF@xr@ignored@zref+1%
  }%
}%
%    \end{macrocode}
%    \end{macro}
%
%    \begin{macro}{\ZREF@xr@ltx@ignorewarning}
%    \begin{macrocode}
\def\ZREF@xr@ltx@ignorewarning#1{%
  \PackageWarningNoLine{zref-xr}{%
    LaTeX label `#1' is already in use\MessageBreak
    in file `\ZREF@xr@file'%
  }%
  \edef\ZREF@xr@ignored@ltx{%
    \the\numexpr\ZREF@xr@ignored@ltx+1%
  }%
}%
%    \end{macrocode}
%    \end{macro}
%
%    \begin{macro}{\ZREF@xr@checklist}
%    \begin{macrocode}
\def\ZREF@xr@checklist#1#2#3\ZREF@nil{%
  \ifx\@undefined#1\relax
    \expandafter\ZREF@xr@checkkey\string#1\@nil
  \fi
  \ifx\\#3\\%
  \else
    \ltx@ReturnAfterFi{%
      \ZREF@xr@checklist#3\ZREF@nil
    }%
  \fi
}%
\def\ZREF@xr@checkkey#1#2\@nil{%
  \zref@ifpropundefined{#2}{%
    \zref@newprop{#2}{}%
  }{}%
}%
%    \end{macrocode}
%    \end{macro}
%
%    \begin{macro}{\ZREF@xr@scanparams}
%    \begin{macrocode}
\def\ZREF@xr@scanparams#1#2#3#4#5#6#7\ZREF@nil{%
  \let#1\ltx@empty
  \ZREF@foundfalse
  \ZREF@xr@scantitleref#1#2\TR@TitleReference{}{}\ZREF@nil
  \ifZREF@found
  \else
    \ltx@LocalAppendToMacro#1{\default{#2}}%
  \fi
  % page
  \ltx@LocalAppendToMacro#1{\page{#3}}%
  % nameref title
  \ifZREF@found
  \else
    \ifx\\#4\\%
    \else
      \def\ZREF@xr@temp{#4}%
      \ifx\ZREF@xr@temp\ZREF@xr@relax
      \else
        \ltx@LocalAppendToMacro#1{\title{#4}}%
      \fi
    \fi
  \fi
  % anchor
  \ifx\\#5\\%
  \else
    \ltx@LocalAppendToMacro#1{\anchor{#5}}%
  \fi
  \ifx\\#6\\%
  \else
    \ifZREF@xr@urluse
      \ZREF@xr@AddURL{#6}%
      \expandafter\ltx@LocalAppendToMacro\expandafter#1%
      \expandafter{%
        \expandafter\urluse\expandafter{%
          \csname Z@U@\ZREF@xr@theURL\endcsname
        }%
      }%
    \else
      \ltx@LocalAppendToMacro#1{\url{#6}}%
    \fi
  \fi
}%
%    \end{macrocode}
%    \end{macro}
%
%    \begin{macro}{\ZREF@xr@scantitleref}
%    \begin{macrocode}
\def\ZREF@xr@scantitleref#1#2\TR@TitleReference#3#4#5\ZREF@nil{%
  \ifx\\#5\\%
  \else
    \ltx@LocalAppendToMacro#1{%
      \default{#3}%
      \title{#4}%
    }%
    \ZREF@foundtrue
  \fi
}%
%    \end{macrocode}
%    \end{macro}
%
%    \begin{macro}{\ZREF@xr@urlcheck}%
%    \begin{macrocode}
\def\ZREF@xr@urlcheck#1{%
  \zref@ifrefcontainsprop{#1}{anchor}{%
    \zref@ifrefcontainsprop{#1}{url}{%
    }{%
      \expandafter
      \ltx@LocalAppendToMacro\csname Z@R@#1\expandafter\endcsname
      \expandafter{%
        \csname url\ifZREF@xr@urluse use\fi
        \expandafter\endcsname\expandafter{\ZREF@xr@url}%
      }%
    }%
  }{%
  }%
}%
%    \end{macrocode}
%    \end{macro}
%
%    \begin{macrocode}
%</xr>
%    \end{macrocode}
%
%
% \subsection{Module \xmodule{hyperref}}
%
%    UNFINISHED :-(
%
%    \begin{macrocode}
%<*hyperref>
\NeedsTeXFormat{LaTeX2e}
\ProvidesPackage{zref-hyperref}%
  [2020-07-03 v2.32 Module hyperref for zref (HO)]%
\RequirePackage{zref-base}[2019/11/29]
\ifx\ZREF@base@ok Y%
\else
  \expandafter\endinput
\fi
%    \end{macrocode}
%
%    \begin{macrocode}
\ZREF@NewPropAnchor
\zref@addprop\ZREF@mainlist{anchor}%
%    \end{macrocode}
%
%    \begin{macrocode}
%</hyperref>
%    \end{macrocode}
%
%
% \subsection{Module \xmodule{savepos}}
%
%    Module \xmodule{savepos} provides an interface for
%    \pdfTeX's \cmd{\pdfsavepos}, see the manual for
%    \pdfTeX.
%
% \subsubsection{Identification}
%
%    \begin{macrocode}
%<*savepos>
\NeedsTeXFormat{LaTeX2e}
\ProvidesPackage{zref-savepos}%
  [2020-07-03 v2.32 Module savepos for zref (HO)]%
\RequirePackage{zref-base}[2019/11/29]
\ifx\ZREF@base@ok Y%
\else
  \expandafter\endinput
\fi
%    \end{macrocode}
%
% \subsubsection{Availability}
%
%    First we check, whether the feature is available.
%    \begin{macrocode}
\ifx\directlua\@undefined
\ltx@IfUndefined{pdfsavepos}{%
  \PackageError\ZREF@name{%
    \string\pdfsavepos\space is not supported.\MessageBreak
    It is provided by pdfTeX (1.40) or XeTeX%
  }\ZREF@UpdatePdfTeX
  \endinput
}{}%
\fi
%    \end{macrocode}
%    In PDF mode we are done. However support for DVI mode
%    was added later in version 1.40.0. In earlier versions
%    \cs{pdfsavepos} is defined, but its execution raises
%    an error. Note that \hologo{XeTeX} also provides \cs{pdfsavepos}.
%    \begin{macrocode}
\ifpdf
\else
  \ltx@IfUndefined{pdftexversion}{%
  }{%
    \ifnum\pdftexversion<140 %
      \PackageError\ZREF@name{%
        \string\pdfsavepos\space is not supported in DVI mode%
        \MessageBreak
        of this pdfTeX version%
      }\ZREF@UpdatePdfTeX
      \expandafter\expandafter\expandafter\endinput
    \fi
  }%
\fi
%    \end{macrocode}
%
% \subsubsection{Setup}
%
%    \begin{macrocode}
\zref@newlist{savepos}
\ifx\directlua\@undefined
  \zref@newprop*{posx}[0]{\the\pdflastxpos}
  \zref@newprop*{posy}[0]{\the\pdflastypos}
\else
  \zref@newprop*{posx}[0]{\the\lastxpos}
  \zref@newprop*{posy}[0]{\the\lastypos}
\fi
\zref@addprops{savepos}{posx,posy}
%    \end{macrocode}
%
% \subsubsection{User macros}
%
%    \begin{macro}{\zref@savepos}
%    \begin{macrocode}
\ifx\directlua\@undefined
  \def\zref@savepos{%
    \if@filesw
      \pdfsavepos
    \fi
  }
\else
  \def\zref@savepos{%
    \if@filesw
      \savepos
    \fi
  }
\fi
%    \end{macrocode}
%    \end{macro}
%
%    \begin{macro}{\ZREF@zsavepos}
%    \begin{macrocode}
\def\ZREF@zsavepos#1#2#3{%
  \@bsphack
  \if@filesw
    \zref@savepos
    #1{#3}{#2}%
    \ltx@IfUndefined{TeXXeTstate}{%
    }{%
      \ifnum\TeXXeTstate=\ltx@zero
      \else
        \zref@savepos
      \fi
    }%
  \fi
  \@esphack
}
%    \end{macrocode}
%    \end{macro}
%
%    \begin{macro}{\zsavepos}
%    The current location is stored in a reference with the
%    given name.
%    \begin{macrocode}
\ZREF@IfDefinable\zsavepos\def{%
  {%
    \ZREF@zsavepos\zref@labelbylist{savepos}%
  }%
}
%    \end{macrocode}
%    \end{macro}
%
%    \begin{macro}{\zsaveposx}
%    \begin{macrocode}
\ZREF@IfDefinable\zsaveposx\def{%
  {%
    \ZREF@zsavepos\zref@labelbyprops{posx}%
  }%
}
%    \end{macrocode}
%    \end{macro}
%    \begin{macro}{\zsaveposy}
%    \begin{macrocode}
\ZREF@IfDefinable\zsaveposy\def{%
  {%
    \ZREF@zsavepos\zref@labelbyprops{posy}%
  }%
}
%    \end{macrocode}
%    \end{macro}
%
%    \begin{macro}{\zposx}
%    \begin{macro}{\zposy}
%    The horizontal and vertical position are available by
%    \cs{zposx} and \cs{zposy}. Do not rely on absolute
%    positions. They differ in DVI and PDF mode of \pdfTeX.
%    Use differences instead. The unit of the position numbers
%    is sp.
%    \begin{macrocode}
\newcommand*{\zposx}[1]{%
  \zref@extract{#1}{posx}%
}%
\newcommand*{\zposy}[1]{%
  \zref@extract{#1}{posy}%
}%
%    \end{macrocode}
%    \end{macro}
%    \end{macro}
%
%    Typically horizontal and vertical positions are used
%    inside calculations. Therefore the extracting macros
%    should be expandable and babel's patch is not applyable.
%
%    Also it is in the responsibility of the user to marked
%    used positions by \cs{zrefused}
%    in order to notify \hologo{LaTeX} about undefined references.
%
%    \begin{macro}{\ZREF@savepos@ok}
%    \begin{macrocode}
\let\ZREF@savepos@ok=Y
%    \end{macrocode}
%    \end{macro}
%
%    \begin{macrocode}
%</savepos>
%    \end{macrocode}
%
%
% \subsection{Module \xmodule{abspos}}
%
% \subsubsection{Identification}
%
%    \begin{macrocode}
%<*abspos>
\NeedsTeXFormat{LaTeX2e}
\ProvidesPackage{zref-abspos}%
  [2020-07-03 v2.32 Module abspos for zref (HO)]%
\RequirePackage{zref-base}[2019/11/29]
\ifx\ZREF@base@ok Y%
\else
  \expandafter\endinput
\fi
%    \end{macrocode}
%
%    \begin{macrocode}
\RequirePackage{zref-savepos}[2019/11/29]
\ifx\ZREF@savepos@ok Y%
\else
  \expandafter\endinput
\fi
%    \end{macrocode}
%    \begin{macrocode}
\RequirePackage{zref-pagelayout}[2019/11/29]
\zref@addprop{savepos}{abspage}
\zref@addprop{savepos}{pagevalue}
%    \end{macrocode}
%
%    \begin{macro}{\zref@absposx}
%    \begin{macrocode}
\newcommand*{\zref@absposx}[3]{%
  \number
    \expandafter\zref@absposnumx\expandafter{%
      \number\zref@extractdefault{#1}{abspage}{0}%
    }{#2}{#3}%
  \ltx@space
}
%    \end{macrocode}
%    \end{macro}
%
%    \begin{macro}{\zref@absposy}
%    \begin{macrocode}
\newcommand*{\zref@absposy}[3]{%
  \number
  \expandafter\zref@absposnumy\expandafter{%
    \number\zref@extractdefault{#1}{abspage}{0}%
  }{#2}{#3}%
  \ltx@space
}
%    \end{macrocode}
%    \end{macro}
%
%    \begin{macro}{\zref@absposnumx}
%    \begin{macrocode}
\newcommand*{\zref@absposnumx}[3]{%
  \number
%  \ifnum#1>\ltx@zero
%    \zref@ifrefundefined{thepage#1}{%
%      0%
%    }{%
      \numexpr\ZREF@absposnum{thepage#1}{#2}{x}{#3}\relax
%    }%
%  \else
%    0%
%  \fi
}
%    \end{macrocode}
%    \end{macro}
%
%    \begin{macro}{\zref@absposnumy}
%    \begin{macrocode}
\newcommand*{\zref@absposnumy}[3]{%
  \number
%  \ifnum#1>\ltx@zero
%    \zref@ifrefundefined{thepage#1}{%
%      0%
%    }{%
      \numexpr\ZREF@absposnum{thepage#1}{#2}{y}{#3}\relax
%    }%
%  \else
%    0%
%  \fi
}
%    \end{macrocode}
%    \end{macro}
%
%    \begin{macro}{\ZREF@absposnum}
%    \begin{macrocode}
\def\ZREF@absposnum#1#2#3#4{%
  \ltx@ifundefined{ZREF@abspos@#2@#3@#4}{%
    0%
  }{%
    \csname ZREF@abspos@#2@#3@#4\endcsname{#1}%
  }%
}
%    \end{macrocode}
%    \end{macro}
%
%    \begin{macro}{\zref@def@absposx}
%    \begin{macrocode}
\ZREF@Robust\def\zref@def@absposx#1{%
  \zref@wrapper@babel{\ZREF@def@abspos{#1}\zref@absposx}%
}
%    \end{macrocode}
%    \end{macro}
%
%    \begin{macro}{\zref@def@absposy}
%    \begin{macrocode}
\ZREF@Robust\def\zref@def@absposy#1{%
  \zref@wrapper@babel{\ZREF@def@abspos{#1}\zref@absposy}%
}
%    \end{macrocode}
%    \end{macro}
%
%    \begin{macro}{\zref@def@absposnumx}
%    \begin{macrocode}
\ZREF@Robust\def\zref@def@absposnumx#1{%
  \zref@wrapper@babel{\ZREF@def@abspos{#1}\zref@absposnumx}%
}
%    \end{macrocode}
%    \end{macro}
%
%    \begin{macro}{\zref@def@absposnumy}
%    \begin{macrocode}
\ZREF@Robust\def\zref@def@absposnumy#1{%
  \zref@wrapper@babel{\ZREF@def@abspos{#1}\zref@absposnumy}%
}
%    \end{macrocode}
%    \end{macro}
%    \begin{macro}{\ZREF@def@abspos}
%    \begin{macrocode}
\def\ZREF@def@absposnumy#1#2#3#4#5{%
  \edef#1{#2{#3}{#4}{#5}}%
}
%    \end{macrocode}
%    \end{macro}
%
%    \begin{macro}{\zref@absposused}
%    \begin{macrocode}
\ZREF@Robust\def\zref@absposused{%
  \zref@wrapper@babel\ZREF@abspos@used
}
%    \end{macrocode}
%    \end{macro}
%    \begin{macro}{\ZREF@abspos@used}
%    \begin{macrocode}
\def\ZREF@abspos@used#1{%
  \zref@refused{#1}%
  \zref@ifrefundefined{#1}{%
  }{%
    \begingroup
      \edef\ZREF@temp{%
        \zref@extractdefault{#1}{abspage}{0}%
      }%
      \ifnum\ZREF@temp>\ltx@zero
        \zref@refused{thepage\ZREF@temp}%
      \else
        \@PackageError{zref-abspos}{%
          \string\zref@pos@label@used\ltx@space
          needs property `abspage'\MessageBreak
          in label `#1'%
        }\@ehc
      \fi
    \endgroup
  }%
}
%    \end{macrocode}
%    \end{macro}
%
%    \begin{macro}{\zref@absposnumused}
%    \begin{macrocode}
\newcommand*{\zref@absposnumused}[1]{%
  \ifnum#1>\ltx@zero
    \zref@refused{thepage\number#1}%
  \else
    \@PackageError{zref-abspos}{%
      Invalid absolute page number (#1)\MessageBreak
      for \string\zref@pos@num@used.\MessageBreak
      A positive integer is expected%
    }\@ehc
  \fi
}
%    \end{macrocode}
%    \end{macro}
%
%    \begin{macro}{\zref@ifabsposundefined}
%    \begin{macrocode}
\def\zref@ifabsposundefined#1{%
  \zref@ifrefundefined{#1}\ltx@firsttwo{%
    \expandafter\zref@ifabsposnumundefined\expandafter{%
      \number\zref@extractdefault{#1}{abspage}{0}%
    }%
  }%
}
%    \end{macrocode}
%    \end{macro}
%
%    \begin{macro}{\zref@ifabsposnumundefined}
%    \begin{macrocode}
\def\zref@ifabsposnumundefined#1{%
  \ifnum\ZREF@number{#1}>\ltx@zero
    \zref@ifrefundefined{thepage#1}%
        \ltx@firstoftwo\ltx@secondoftwo
  \else
    \expandafter\ltx@firstoftwo
  \fi
}
%    \end{macrocode}
%    \end{macro}
%
% \subsubsection{Media}
%
%    \begin{macro}{\ZREF@abspos@media@width}
%    \begin{macrocode}
\edef\ZREF@abspos@media@width{%
  \ltx@ifundefined{pdfpagewidth}{%
    \ltx@ifundefined{mediawidth}{%
      \ltx@ifundefined{stockwidth}{%
        paperwidth%
      }{%
        stockwidth%
      }%
    }{%
      mediawidth%
    }%
  }{%
    pdfpagewidth%
  }%
}
\ifluatex
\def\ZREF@abspos@media@width{pdfpagewidth}%
\fi
%    \end{macrocode}
%    \end{macro}
%    \begin{macro}{\ZREF@abspos@media@height}
%    \begin{macrocode}
\edef\ZREF@abspos@media@height{%
  \ltx@ifundefined{pdfpageheight}{%
    \ltx@ifundefined{mediaheight}{%
      \ltx@ifundefined{stockheight}{%
        paperheight%
      }{%
        stockheight%
      }%
    }{%
      mediaheight%
    }%
  }{%
    \noexpand\ifcase\pdfpageheight
      \ltx@ifundefined{stockheight}{%
        paperheight%
      }{%
        stockheight%
      }%
    \noexpand\else
      pdfpageheight%
    \noexpand\fi
  }%
}
\ifluatex
\edef\ZREF@abspos@media@height{%
   \noexpand\ifcase\pageheight
      \ltx@ifundefined{stockheight}{%
        paperheight%
      }{%
        stockheight%
      }%
    \noexpand\else
      pdfpageheight%
    \noexpand\fi}%
\fi
%    \end{macrocode}
%    \end{macro}
%
%    \begin{macro}{\ZREF@abspos@media@x@left}
%    \begin{macrocode}
\def\ZREF@abspos@media@x@left#1{%
  0%
}
%    \end{macrocode}
%    \end{macro}
%    \begin{macro}{\ZREF@abspos@media@x@right}
%    \begin{macrocode}
\def\ZREF@abspos@media@x@right#1{%
  \zref@extract{#1}\ZREF@abspos@media@width
}
%    \end{macrocode}
%    \end{macro}
%    \begin{macro}{\ZREF@abspos@media@x@center}
%    \begin{macrocode}
\def\ZREF@abspos@media@x@center#1{%
  \ZREF@abspos@media@x@left{#1}%
  +\zref@extract{#1}\ZREF@abspos@media@width/2%
}
%    \end{macrocode}
%    \end{macro}
%
%    \begin{macro}{\ZREF@abspos@media@y@top}
%    \begin{macrocode}
\def\ZREF@abspos@media@y@top#1{%
  \zref@extract{#1}\ZREF@abspos@media@height
}
%    \end{macrocode}
%    \end{macro}
%    \begin{macro}{\ZREF@abspos@media@y@bottom}
%    \begin{macrocode}
\def\ZREF@abspos@media@y@bottom#1{%
  0%
}
%    \end{macrocode}
%    \end{macro}
%    \begin{macro}{\ZREF@abspos@media@y@center}
%    \begin{macrocode}
\def\ZREF@abspos@media@y@center#1{%
  \zref@extract{#1}\ZREF@abspos@media@height/2%
}
%    \end{macrocode}
%    \end{macro}
%
% \subsubsection{Paper}
%
%    \begin{macro}{\ZREF@abspos@paper@x@left}
%    \begin{macrocode}
\def\ZREF@abspos@paper@x@left#1{%
  0%
}
%    \end{macrocode}
%    \end{macro}
%    \begin{macro}{\ZREF@abspos@paper@x@right}
%    \begin{macrocode}
\def\ZREF@abspos@paper@x@right#1{%
  \zref@extract{#1}{paperwidth}%
}
%    \end{macrocode}
%    \end{macro}
%    \begin{macro}{\ZREF@abspos@paper@x@center}
%    \begin{macrocode}
\def\ZREF@abspos@paper@x@center#1{%
  \zref@extract{#1}{paperwidth}/2%
}
%    \end{macrocode}
%    \end{macro}
%
%    \begin{macro}{\ZREF@abspos@paper@y@top}
%    \begin{macrocode}
\let\ZREF@abspos@paper@y@top\ZREF@abspos@media@y@top
%    \end{macrocode}
%    \end{macro}
%    \begin{macro}{\ZREF@abspos@paper@y@bottom}
%    \begin{macrocode}
\def\ZREF@abspos@paper@y@bottom#1{%
  \ZREF@abspos@paper@y@top{#1}%
  -\zref@extract{#1}{paperheight}%
}
%    \end{macrocode}
%    \end{macro}
%    \begin{macro}{\ZREF@abspos@paper@y@center}
%    \begin{macrocode}
\def\ZREF@abspos@paper@y@center#1{%
  \ZREF@abspos@paper@y@top{#1}%
  -\zref@extract{#1}{paperheight}/2%
}
%    \end{macrocode}
%    \end{macro}
%
% \subsubsection{Origin}
%
% There doesn't seem a good reason to make these tests depend on pdf mode
%  in current engines, so comment out the \verb|\ifpdf| tests.
%    \begin{macro}{\ZREF@abspos@origin@x}
%    \begin{macrocode}
\let\ZREF@temp\ltx@two
\ltx@IfUndefined{pdfhorigin}{}{%
%  \ifpdf
    \let\ZREF@temp\ltx@zero
%  \fi
}
\ifluatex
% \ifpdf
  \let\ZREF@temp\ltx@zero
% \fi
\fi

\ifx\ZREF@temp\ltx@two
  \ifnum\mag=1000 %
    \let\ZREF@temp\ltx@one
  \fi
\fi
\ifcase\ZREF@temp
  \def\ZREF@abspos@origin@x#1{%
    \zref@extract{#1}{pdfhorigin}%
  }%
\or
  \def\ZREF@abspos@origin@x#1{%
    4736286%
  }%
\or
  \def\ZREF@abspos@origin@x#1{%
    \numexpr\mag/1000*\dimexpr 1truein\relax\relax
  }%
\fi
%    \end{macrocode}
%    \end{macro}
%    \begin{macro}{\ZREF@abspos@origin@y}
%    \begin{macrocode}
\let\ZREF@temp\ltx@two
\ltx@IfUndefined{pdfvorigin}{}{%
%  \ifpdf
    \let\ZREF@temp\ltx@zero
%  \fi
}
\ifluatex
%  \ifpdf
    \let\ZREF@temp\ltx@zero
%  \fi
\fi
\ifx\ZREF@temp\ltx@two
  \ifnum\mag=1000 %
    \let\ZREF@temp\ltx@one
  \fi
\fi
\ifcase\ZREF@temp
  \def\ZREF@abspos@origin@y#1{%
    \zref@extract{#1}{pdfvorigin}%
  }%
\or
  \def\ZREF@abspos@origin@y#1{%
    4736286%
  }%
\or
  \def\ZREF@abspos@origin@y#1{%
    \numexpr\mag/1000*\dimexpr 1truein\relax\relax
   }%
\fi
%    \end{macrocode}
%    \end{macro}
%
% \subsubsection{Header}
%
%    \begin{macro}{\ZREF@abspos@head@x@left}
%    \begin{macrocode}
\def\ZREF@abspos@head@x@left#1{%
  \ZREF@abspos@paper@x@left{#1}%
  +\ZREF@abspos@origin@x{#1}%
  +\zref@extract{#1}{hoffset}%
  +\ifodd\zref@extractdefault{#1}{pagevalue}{\number\c@page} %
     \zref@extract{#1}{oddsidemargin}%
   \else
     \zref@extract{#1}{evensidemargin}%
   \fi
}
%    \end{macrocode}
%    \end{macro}
%    \begin{macro}{\ZREF@abspos@head@x@right}
%    \begin{macrocode}
\def\ZREF@abspos@head@x@right#1{%
  \ZREF@abspos@head@x@left{#1}%
  +\zref@extract{#1}{textwidth}%
}
%    \end{macrocode}
%    \end{macro}
%    \begin{macro}{\ZREF@abspos@head@x@center}
%    \begin{macrocode}
\def\ZREF@abspos@head@x@center#1{%
  \ZREF@abspos@head@x@left{#1}%
  +\zref@extract{#1}{textwidth}/2%
}
%    \end{macrocode}
%    \end{macro}
%
%    \begin{macro}{\ZREF@abspos@head@y@top}
%    \begin{macrocode}
\def\ZREF@abspos@head@y@top#1{%
  \ZREF@abspos@paper@y@top{#1}%
  -\ZREF@abspos@origin@y{#1}%
  -\zref@extract{#1}{voffset}%
  -\zref@extract{#1}{topmargin}%
}
%    \end{macrocode}
%    \end{macro}
%    \begin{macro}{\ZREF@abspos@head@y@bottom}
%    \begin{macrocode}
\def\ZREF@abspos@head@y@bottom#1{%
  \ZREF@abspos@head@y@top{#1}%
  -\zref@extract{#1}{headheight}%
}
%    \end{macrocode}
%    \end{macro}
%    \begin{macro}{\ZREF@abspos@head@y@center}
%    \begin{macrocode}
\def\ZREF@abspos@head@y@center#1{%
  \ZREF@abspos@head@y@top{#1}%
  -\zref@extract{#1}{headheight}/2%
}
%    \end{macrocode}
%    \end{macro}
%
% \subsubsection{Body}
%
%    \begin{macro}{\ZREF@abspos@body@x@left}
%    \begin{macrocode}
\let\ZREF@abspos@body@x@left\ZREF@abspos@head@x@left
%    \end{macrocode}
%    \end{macro}
%    \begin{macro}{\ZREF@abspos@body@x@right}
%    \begin{macrocode}
\let\ZREF@abspos@body@x@right\ZREF@abspos@head@x@right
%    \end{macrocode}
%    \end{macro}
%    \begin{macro}{\ZREF@abspos@body@x@center}
%    \begin{macrocode}
\let\ZREF@abspos@body@x@center\ZREF@abspos@head@x@center
%    \end{macrocode}
%    \end{macro}
%
%    \begin{macro}{\ZREF@abspos@body@y@top}
%    \begin{macrocode}
\def\ZREF@abspos@body@y@top#1{%
  \ZREF@abspos@head@y@bottom{#1}%
  -\zref@extract{#1}{headsep}%
}
%    \end{macrocode}
%    \end{macro}
%    \begin{macro}{\ZREF@abspos@body@y@bottom}
%    \begin{macrocode}
\def\ZREF@abspos@body@y@bottom#1{%
  \ZREF@abspos@body@y@top{#1}%
  -\zref@extract{#1}{textheight}%
}
%    \end{macrocode}
%    \end{macro}
%    \begin{macro}{\ZREF@abspos@body@y@center}
%    \begin{macrocode}
\def\ZREF@abspos@body@y@center#1{%
  \ZREF@abspos@body@y@top{#1}%
  -\zref@extract{#1}{textheight}/2%
}
%    \end{macrocode}
%    \end{macro}
%
% \subsubsection{Footer}
%
%    \begin{macro}{\ZREF@abspos@foot@x@left}
%    \begin{macrocode}
\let\ZREF@abspos@foot@x@left\ZREF@abspos@head@x@left
%    \end{macrocode}
%    \end{macro}
%    \begin{macro}{\ZREF@abspos@foot@x@right}
%    \begin{macrocode}
\let\ZREF@abspos@foot@x@right\ZREF@abspos@head@x@right
%    \end{macrocode}
%    \end{macro}
%    \begin{macro}{\ZREF@abspos@foot@x@center}
%    \begin{macrocode}
\let\ZREF@abspos@foot@x@center\ZREF@abspos@head@x@center
%    \end{macrocode}
%    \end{macro}
%
%    \begin{macro}{\ZREF@abspos@foot@y@bottom}
%    \begin{macrocode}
\def\ZREF@abspos@foot@y@bottom#1{%
  \ZREF@abspos@body@y@bottom{#1}%
  -\zref@extract{#1}{footskip}%
}
%    \end{macrocode}
%    \end{macro}
%
% \subsubsection{Marginal notes}
%
%    \begin{macro}{\ZREF@abspos@marginpar@x@left}
%    \begin{macrocode}
\def\ZREF@abspos@marginpar@x@left#1{%
  \ifodd\zref@extractdefault{#1}{pagevalue}{\number\c@page} %
    \ZREF@abspos@body@x@right{#1}%
    +\zref@extract{#1}{marginparsep}%
  \else
    \ZREF@abspos@body@x@left{#1}%
    -\zref@extract{#1}{marginparsep}%
    -\zref@extract{#1}{marginparwidth}%
  \fi
}
%    \end{macrocode}
%    \end{macro}
%    \begin{macro}{\ZREF@abspos@marginpar@x@right}
%    \begin{macrocode}
\def\ZREF@abspos@marginpar@x@right#1{%
  \ZREF@abspos@marginpar@x@left{#1}%
  +\zref@extract{#1}{marginparwidth}%
}
%    \end{macrocode}
%    \end{macro}
%    \begin{macro}{\ZREF@abspos@marginpar@x@center}
%    \begin{macrocode}
\def\ZREF@abspos@marginpar@x@center#1{%
  \ZREF@abspos@marginpar@x@left{#1}%
  +\zref@extract{#1}{marginparwidth}/2%
}
%    \end{macrocode}
%    \end{macro}
%
%    \begin{macro}{\ZREF@abspos@marginpar@y@top}
%    \begin{macrocode}
\let\ZREF@abspos@marginpar@y@top\ZREF@abspos@body@y@top
%    \end{macrocode}
%    \end{macro}
%    \begin{macro}{\ZREF@abspos@marginpar@y@bottom}
%    \begin{macrocode}
\let\ZREF@abspos@marginpar@y@bottom\ZREF@abspos@body@y@bottom
%    \end{macrocode}
%    \end{macro}
%    \begin{macro}{\ZREF@abspos@marginpar@y@center}
%    \begin{macrocode}
\let\ZREF@abspos@marginpar@y@center\ZREF@abspos@body@y@center
%    \end{macrocode}
%    \end{macro}
%
% \subsubsection{Stock paper}
%
%    \begin{macro}{\ZREF@abspos@stock@x@left}
%    \begin{macrocode}
\let\ZREF@abspos@stock@x@left\ZREF@abspos@paper@x@left
%    \end{macrocode}
%    \end{macro}
%    \begin{macro}{\ZREF@abspos@stock@x@right}
%    \begin{macrocode}
\let\ZREF@abspos@stock@x@right\ZREF@abspos@paper@x@right
%    \end{macrocode}
%    \end{macro}
%    \begin{macro}{\ZREF@abspos@stock@x@center}
%    \begin{macrocode}
\let\ZREF@abspos@stock@x@center\ZREF@abspos@paper@x@center
%    \end{macrocode}
%    \end{macro}
%
%    \begin{macro}{\ZREF@abspos@stock@y@top}
%    \begin{macrocode}
\let\ZREF@abspos@stock@y@top\ZREF@abspos@paper@y@top
%    \end{macrocode}
%    \end{macro}
%    \begin{macro}{\ZREF@abspos@stock@y@bottom}
%    \begin{macrocode}
\let\ZREF@abspos@stock@y@bottom\ZREF@abspos@paper@y@bottom
%    \end{macrocode}
%    \end{macro}
%    \begin{macro}{\ZREF@abspos@stock@y@center}
%    \begin{macrocode}
\let\ZREF@abspos@stock@y@center\ZREF@abspos@paper@y@center
%    \end{macrocode}
%    \end{macro}
%
%    \begin{macrocode}
%</abspos>
%    \end{macrocode}
%
%
% \subsection{Module \xmodule{dotfill}}
%
%    \begin{macrocode}
%<*dotfill>
\NeedsTeXFormat{LaTeX2e}
\ProvidesPackage{zref-dotfill}%
  [2020-07-03 v2.32 Module dotfill for zref (HO)]%
\RequirePackage{zref-base}[2019/11/29]
\ifx\ZREF@base@ok Y%
\else
  \expandafter\endinput
\fi
%    \end{macrocode}
%
%    For measuring the width of \cs{zdotfill} we use the features
%    provided by module \xmodule{savepos}.
%    \begin{macrocode}
\RequirePackage{zref-savepos}[2019/11/29]
%    \end{macrocode}
%    For automatically generated label names we use the unique
%    counter of module \xmodule{base}.
%    \begin{macrocode}
\zref@require@unique
%    \end{macrocode}
%    Configuration is done by the key value interface of package
%    \xpackage{keyval}.
%    \begin{macrocode}
\RequirePackage{keyval}
%    \end{macrocode}
%    The definitions of the keys follow.
%    \begin{macrocode}
\define@key{ZREF@DF}{unit}{%
  \def\ZREF@df@unit{#1}%
}
\define@key{ZREF@DF}{min}{%
  \def\ZREF@df@min{#1}%
}
\define@key{ZREF@DF}{dot}{%
  \def\ZREF@df@dot{#1}%
}
%    \end{macrocode}
%    Defaults are set, see user interface.
%    \begin{macrocode}
\providecommand\ZREF@df@min{2}
\providecommand\ZREF@df@unit{.44em}
\providecommand\ZREF@df@dot{.}
%    \end{macrocode}
%    \begin{macro}{\zdotfillsetup}
%    Configuration of \cs{zdotfill} is done by \cs{zdotfillsetup}.
%    \begin{macrocode}
\newcommand*{\zdotfillsetup}{\kvsetkeys{ZREF@DF}}
%    \end{macrocode}
%    \end{macro}
%
%    \begin{macro}{\zdotfill}
%    \cs{zdotfill} sets labels at the left and the right to get
%    the horizontal position. \cs{zsavepos} is not used, because
%    we do not need the vertical position.
%    \begin{macrocode}
\ZREF@IfDefinable\zdotfill\def{%
  {%
    \leavevmode
    \global\advance\c@zref@unique\ltx@one
    \begingroup
      \def\ZREF@temp{zref@\number\c@zref@unique}%
      \pdfsavepos
      \zref@labelbyprops{\thezref@unique L}{posx}%
      \setlength{\dimen@}{\ZREF@df@unit}%
      \zref@ifrefundefined{\thezref@unique R}{%
        \ZREF@dotfill
      }{%
        \ifnum\numexpr\zposx{\thezref@unique R}%
                      -\zposx{\thezref@unique L}\relax
            <\dimexpr\ZREF@df@min\dimen@\relax
          \hfill
        \else
          \ZREF@dotfill
        \fi
      }%
      \pdfsavepos
      \zref@labelbyprops{\thezref@unique R}{posx}%
    \endgroup
    \kern\z@
  }%
}
%    \end{macrocode}
%    \end{macro}
%    \begin{macro}{\ZREF@dotfill}
%    Help macro that actually sets the dots.
%    \begin{macrocode}
\def\ZREF@dotfill{%
  \cleaders\hb@xt@\dimen@{\hss\ZREF@df@dot\hss}\hfill
}
%    \end{macrocode}
%    \end{macro}
%
%    \begin{macrocode}
%</dotfill>
%    \end{macrocode}
%
% \subsection{Module \xmodule{env}}
%
%    \begin{macrocode}
%<*env>
\NeedsTeXFormat{LaTeX2e}
\ProvidesPackage{zref-env}%
  [2020-07-03 v2.32 Module env for zref (HO)]%
\RequirePackage{zref-base}[2019/11/29]
\ifx\ZREF@base@ok Y%
\else
  \expandafter\endinput
\fi
%    \end{macrocode}
%
%    \begin{macrocode}
\zref@newprop{envname}[]{\@currenvir}
\zref@newprop{envline}[]{\zref@env@line}
%    \end{macrocode}
%
%    \begin{macro}{\zref@env@line}
%    Macro \cs{zref@env@line} extracts the line number
%    from \cs{@currenvline}.
%    \begin{macrocode}
\def\zref@env@line{%
  \ifx\@currenvline\ltx@empty
  \else
    \expandafter
    \ZREF@ENV@line\@currenvline\ltx@empty line \ltx@empty\@nil
  \fi
}
%    \end{macrocode}
%    \end{macro}
%    \begin{macro}{\ZREF@ENV@line}
%    \begin{macrocode}
\def\ZREF@ENV@line#1line #2\ltx@empty#3\@nil{#2}%
%    \end{macrocode}
%    \end{macro}
%
%    \begin{macrocode}
%</env>
%    \end{macrocode}
%
%% \section{Installation}
%
% \subsection{Download}
%
% \paragraph{Package.} This package is available on
% CTAN\footnote{\CTANpkg{zref}}:
% \begin{description}
% \item[\CTAN{macros/latex/contrib/zref/zref.dtx}] The source file.
% \item[\CTAN{macros/latex/contrib/zref/zref.pdf}] Documentation.
% \end{description}
%
%
% \paragraph{Bundle.} All the packages of the bundle `zref'
% are also available in a TDS compliant ZIP archive. There
% the packages are already unpacked and the documentation files
% are generated. The files and directories obey the TDS standard.
% \begin{description}
% \item[\CTANinstall{install/macros/latex/contrib/zref.tds.zip}]
% \end{description}
% \emph{TDS} refers to the standard ``A Directory Structure
% for \TeX\ Files'' (\CTANpkg{tds}). Directories
% with \xfile{texmf} in their name are usually organized this way.
%
% \subsection{Bundle installation}
%
% \paragraph{Unpacking.} Unpack the \xfile{zref.tds.zip} in the
% TDS tree (also known as \xfile{texmf} tree) of your choice.
% Example (linux):
% \begin{quote}
%   |unzip zref.tds.zip -d ~/texmf|
% \end{quote}
%
%
% \subsection{Package installation}
%
% \paragraph{Unpacking.} The \xfile{.dtx} file is a self-extracting
% \docstrip\ archive. The files are extracted by running the
% \xfile{.dtx} through \plainTeX:
% \begin{quote}
%   \verb|tex zref.dtx|
% \end{quote}
%
% \paragraph{TDS.} Now the different files must be moved into
% the different directories in your installation TDS tree
% (also known as \xfile{texmf} tree):
% \begin{quote}
% \def\t{^^A
% \begin{tabular}{@{}>{\ttfamily}l@{ $\rightarrow$ }>{\ttfamily}l@{}}
%   zref.sty & tex/latex/zref/zref.sty\\
%   zref-base.sty & tex/latex/zref/zref-base.sty\\
%   zref-abspage.sty & tex/latex/zref/zref-abspage.sty\\
%   zref-abspos.sty & tex/latex/zref/zref-abspos.sty\\
%   zref-counter.sty & tex/latex/zref/zref-counter.sty\\
%   zref-dotfill.sty & tex/latex/zref/zref-dotfill.sty\\
%   zref-env.sty & tex/latex/zref/zref-env.sty\\
%   zref-hyperref.sty & tex/latex/zref/zref-hyperref.sty\\
%   zref-lastpage.sty & tex/latex/zref/zref-lastpage.sty\\
%   zref-marks.sty & tex/latex/zref/zref-marks.sty\\
%   zref-nextpage.sty & tex/latex/zref/zref-nextpage.sty\\
%   zref-pageattr.sty & tex/latex/zref/zref-pageattr.sty\\
%   zref-pagelayout.sty & tex/latex/zref/zref-pagelayout.sty\\
%   zref-perpage.sty & tex/latex/zref/zref-perpage.sty\\
%   zref-runs.sty & tex/latex/zref/zref-runs.sty\\
%   zref-savepos.sty & tex/latex/zref/zref-savepos.sty\\
%   zref-thepage.sty & tex/latex/zref/zref-thepage.sty\\
%   zref-titleref.sty & tex/latex/zref/zref-titleref.sty\\
%   zref-totpages.sty & tex/latex/zref/zref-totpages.sty\\
%   zref-user.sty & tex/latex/zref/zref-user.sty\\
%   zref-xr.sty & tex/latex/zref/zref-xr.sty\\
%   zref.pdf & doc/latex/zref/zref.pdf\\
%   zref-example.tex & doc/latex/zref/zref-example.tex\\
%   zref-example-lastpage.tex & doc/latex/zref/zref-example-lastpage.tex\\
%   zref-example-nextpage.tex & doc/latex/zref/zref-example-nextpage.tex\\
%   zref.dtx & source/latex/zref/zref.dtx\\
% \end{tabular}^^A
% }^^A
% \sbox0{\t}^^A
% \ifdim\wd0>\linewidth
%   \begingroup
%     \advance\linewidth by\leftmargin
%     \advance\linewidth by\rightmargin
%   \edef\x{\endgroup
%     \def\noexpand\lw{\the\linewidth}^^A
%   }\x
%   \def\lwbox{^^A
%     \leavevmode
%     \hbox to \linewidth{^^A
%       \kern-\leftmargin\relax
%       \hss
%       \usebox0
%       \hss
%       \kern-\rightmargin\relax
%     }^^A
%   }^^A
%   \ifdim\wd0>\lw
%     \sbox0{\small\t}^^A
%     \ifdim\wd0>\linewidth
%       \ifdim\wd0>\lw
%         \sbox0{\footnotesize\t}^^A
%         \ifdim\wd0>\linewidth
%           \ifdim\wd0>\lw
%             \sbox0{\scriptsize\t}^^A
%             \ifdim\wd0>\linewidth
%               \ifdim\wd0>\lw
%                 \sbox0{\tiny\t}^^A
%                 \ifdim\wd0>\linewidth
%                   \lwbox
%                 \else
%                   \usebox0
%                 \fi
%               \else
%                 \lwbox
%               \fi
%             \else
%               \usebox0
%             \fi
%           \else
%             \lwbox
%           \fi
%         \else
%           \usebox0
%         \fi
%       \else
%         \lwbox
%       \fi
%     \else
%       \usebox0
%     \fi
%   \else
%     \lwbox
%   \fi
% \else
%   \usebox0
% \fi
% \end{quote}
% If you have a \xfile{docstrip.cfg} that configures and enables \docstrip's
% TDS installing feature, then some files can already be in the right
% place, see the documentation of \docstrip.
%
% \subsection{Refresh file name databases}
%
% If your \TeX~distribution
% (\TeX\,Live, \mikTeX, \dots) relies on file name databases, you must refresh
% these. For example, \TeX\,Live\ users run \verb|texhash| or
% \verb|mktexlsr|.
%
% \subsection{Some details for the interested}
%
% \paragraph{Unpacking with \LaTeX.}
% The \xfile{.dtx} chooses its action depending on the format:
% \begin{description}
% \item[\plainTeX:] Run \docstrip\ and extract the files.
% \item[\LaTeX:] Generate the documentation.
% \end{description}
% If you insist on using \LaTeX\ for \docstrip\ (really,
% \docstrip\ does not need \LaTeX), then inform the autodetect routine
% about your intention:
% \begin{quote}
%   \verb|latex \let\install=y\input{zref.dtx}|
% \end{quote}
% Do not forget to quote the argument according to the demands
% of your shell.
%
% \paragraph{Generating the documentation.}
% You can use both the \xfile{.dtx} or the \xfile{.drv} to generate
% the documentation. The process can be configured by the
% configuration file \xfile{ltxdoc.cfg}. For instance, put this
% line into this file, if you want to have A4 as paper format:
% \begin{quote}
%   \verb|\PassOptionsToClass{a4paper}{article}|
% \end{quote}
% An example follows how to generate the
% documentation with pdf\LaTeX:
% \begin{quote}
%\begin{verbatim}
%pdflatex zref.dtx
%makeindex -s gind.ist zref.idx
%pdflatex zref.dtx
%makeindex -s gind.ist zref.idx
%pdflatex zref.dtx
%\end{verbatim}
% \end{quote}
%
% \begin{thebibliography}{99}
% \bibpackage{footmisc}{Robin Fairbairns}{2004/01/23 v5.3a}
%   {\CTANpkg{footmisc}}
% \bibpackage{hyperref}{Sebastian Rahtz, Heiko Oberdiek}{2006/08/16 v6.75c}
%   {\CTANpkg{hyperref}}
% \bibpackage{lastpage}{Jeff Goldberg}{1994/06/25 v0.1b}
%   {\CTANpkg{lastpage}}
% \bibpackage{nameref}{Sebastian Rahtz, Heiko Oberdiek}{2006/02/12 v2.24}
%   {\CTANpkg{nameref}}
% \bibpackage{perpage}{David Kastrup}{2002/12/20 v1.0}
%   {\CTANpkg{perpage}}
% \bibpackage{titleref}{Donald Arsenau}{2001/04/05 v3.1}
%   {\CTANpkg{titleref}}
% \bibpackage{totpages}{Wilhelm M\"uller}{1999/07/14 v1.00}
%   {\CTANpkg{totpages}}
% \bibpackage{xr}{David Carlisle}{1994/05/28 v5.02}
%   {\CTANpkg{xr}}
% \bibpackage{xr-hyper}{David Carlisle}{2000/03/22 v6.00beta4}
%   {\CTANpkg{xr-hyper}}
% \end{thebibliography}
%
% \begin{History}
%   \begin{Version}{2006/02/20 v1.0}
%   \item
%     First version.
%   \end{Version}
%   \begin{Version}{2006/05/03 v1.1}
%   \item
%     Module \xmodule{perpage} added.
%   \item
%     Module redesign as packages.
%   \end{Version}
%   \begin{Version}{2006/05/25 v1.2}
%   \item
%     Module \xmodule{dotfillmin} added.
%   \item
%     Module \xmodule{base}: macros \cs{zref@require@uniqe}
%     and \cs{thezref@unique} added
%     (used by modules \xmodule{titleref} and \xmodule{dotfillmin}).
%   \end{Version}
%   \begin{Version}{2006/09/08 v1.3}
%   \item
%     Typo fixes and English cleanup by Per Starback.
%   \end{Version}
%   \begin{Version}{2007/01/23 v1.4}
%   \item
%     Typo in macro name fixed in documentation.
%   \end{Version}
%   \begin{Version}{2007/02/18 v1.5}
%   \item
%     \cs{zref@getcurrent} added (suggestion of Igor Akkerman).
%   \item
%     Module \xmodule{savepos} also supports \hologo{XeTeX}.
%   \end{Version}
%   \begin{Version}{2007/04/06 v1.6}
%   \item
%     Fix in modules \xmodule{abspage} and \xmodule{base}:
%     Now counter |abspage| and |zref@unique|
%     are not remembered by \cs{include}.
%   \item
%     Beamer support for module \xmodule{titleref}.
%   \end{Version}
%   \begin{Version}{2007/04/17 v1.7}
%   \item
%     Package \xpackage{atbegshi} replaces \xpackage{everyshi}.
%   \end{Version}
%   \begin{Version}{2007/04/22 v1.8}
%   \item
%     \cs{zref@wrapper@babel} and \cs{zref@refused} are now expandable
%     if babel is not used or \cs{if@safe@actives} is already set to true.
%     (Feature request of Josselin Noirel)
%   \end{Version}
%   \begin{Version}{2007/05/02 v1.9}
%   \item
%     Module \xmodule{titleref}: Some support for \cs{caption} of package
%     \xpackage{longtable}, but only if \cs{label} is given after
%     \cs{caption}.
%   \end{Version}
%   \begin{Version}{2007/05/06 v2.0}
%   \item
%     Uses package \xpackage{etexcmds} for accessing \hologo{eTeX}'s \cs{unexpanded}.
%   \end{Version}
%   \begin{Version}{2007/05/28 v2.1}
%   \item
%     Module \xmodule{titleref} supports caption of package \xpackage{listings}.
%   \item
%     Fixes in module \xmodule{titleref} for support
%     of packages \xpackage{titlesec} and \xpackage{longtable}.
%   \end{Version}
%   \begin{Version}{2008/09/21 v2.2}
%   \item
%     Module \xmodule{base}: \cs{zref@iflistcontainsprop} is documented, but
%     a broken \cs{zref@listcontainsprop} implemented. Name and
%     implementation fixed (thanks Ohad Kammar).
%   \end{Version}
%   \begin{Version}{2008/10/01 v2.3}
%   \item
%     \cs{zref@localaddprop} added (feature request of Ohad Kammar).
%   \item
%     Module \xmodule{lastpage}: list `LastPage' added. Label `LastPage' will use
%     the properties of this list (default is empty) along with the
%     properties of the main list.
%   \end{Version}
%   \begin{Version}{2009/08/07 v2.4}
%   \item
%     Module \xmodule{runs} added.
%   \end{Version}
%   \begin{Version}{2009/12/06 v2.5}
%   \item
%     Module \xmodule{lastpage}: Uses package \xpackage{atveryend}.
%   \item
%     Module \xmodule{titleref}: Further commands are disabled
%     during string expansion, imported from package \xpackage{nameref}.
%   \end{Version}
%   \begin{Version}{2009/12/07 v2.6}
%   \item
%     Version date added for package \xpackage{atveryend}.
%   \end{Version}
%   \begin{Version}{2009/12/08 v2.7}
%   \item
%     Module \xmodule{titleref}: Use of package \xpackage{gettitlestring}.
%   \end{Version}
%   \begin{Version}{2010/03/26 v2.8}
%   \item
%     \cs{zifrefundefined} added.
%   \item
%     Module \xmodule{lastpage}: Macros \cs{zref@iflastpage}
%     and \cs{ziflastpage} added.
%   \item
%     Module \xmodule{thepage} added.
%   \item
%     Module \xmodule{nextpage} added.
%   \end{Version}
%   \begin{Version}{2010/03/29 v2.9}
%   \item
%     Module \xmodule{marks} added (without documentation).
%   \item
%     \cs{zref@addprop} now adds expanded property to list.
%   \item
%     Useless \cs{ZREF@ErrorNoLine} removed.
%   \end{Version}
%   \begin{Version}{2010/04/08 v2.10}
%   \item
%     Module \xmodule{xr} remembers the external document name
%     in property `externaldocument'.
%   \end{Version}
%   \begin{Version}{2010/04/15 v2.11}
%   \item
%     Module \xmodule{titleref}: Better support of class \xclass{memoir}.
%   \item
%     Module \xmodule{titleref}: Support of theorems.
%   \end{Version}
%   \begin{Version}{2010/04/17 v2.12}
%   \item
%     Module \xmodule{base}: \cs{zref@newprop} ensures global
%     empty default.
%   \item
%     Module \xmodule{xr}: Setup options \xoption{tozreflabel}
%     and \xoption{toltxlabel} added.
%   \end{Version}
%   \begin{Version}{2010/04/19 v2.13}
%   \item
%     \cs{zref@setcurrent} throws an error if the property
%     does not exist (Florent Chervet).
%   \item
%     \cs{zref@getcurrent} the documentation is fixed (Florent Chervet).
%     Also it returns the empty string in case of errors.
%   \item
%     \cs{zref@addprop} and \cs{zref@localaddprop} now take a list of
%     property names (feature request of Florent Chervet).
%   \item
%     Example for \cs{zref@wrapper@unexpanded} corrected (Florent Chervet).
%   \end{Version}
%   \begin{Version}{2010/04/22 v2.14}
%   \item
%     Bug fix for \cs{zref@getcurrent} second argument wasn't eaten in
%     case of unknown property.
%   \item
%     \cs{zref@getcurrent} supports \cs{zref@wrapper@unexpanded}.
%   \item
%     \cs{zref@wrapper@unexpanded} added for \cs{ZREF@xr@tolabel}.
%   \item
%     \cs{zref@extract}, \cs{zref@extractdefault}, \cs{zref@getcurrent}
%     are expandable in exact two steps
%     except inside \cs{zref@wrapper@unexpanded}.
%   \end{Version}
%   \begin{Version}{2010/04/23 v2.15}
%   \item
%     \cs{zexternaldocument} fixed for property `url' when
%     importing \cs{new@label} (bug found by Victor Ivrii).
%   \item
%     Two expansion steps also in \cs{zref@wrapper@unexpanded}.
%   \item
%     Nested calls of \cs{zref@wrapper@unexpanded} possible.
%   \end{Version}
%   \begin{Version}{2010/04/28 v2.16}
%   \item
%     More consequent use of package `ltxcmds' and `hologo'.
%   \item
%     Module \xmodule{pagelayout} added.
%   \item
%     Module \xmodule{pageattr} added.
%   \item
%     Robustness introduced for non-expandable interface macros.
%   \item
%     Internal change of the data format of property lists
%     (suggestion of Florent Chervet).
%   \item
%     Module \xmodule{titleref}: Support of environment \textsf{description}.
%   \end{Version}
%   \begin{Version}{2010/05/01 v2.17}
%   \item
%     \cs{zref@newprop} throws an error if the property already exists.
%   \item
%     Module \xmodule{xr}: Bug fix for the case of several \xext{aux} files
%     (bug found by Victor Ivrii).
%   \item
%     Module \xmodule{xr}: Property `urluse' and option \xoption{urluse} added.
%   \end{Version}
%   \begin{Version}{2010/05/13 v2.18}
%   \item
%     Module \xmodule{env} added.
%   \item
%     Module \xmodule{savepos}: \cs{zref@savepos} added.
%   \end{Version}
%   \begin{Version}{2010/10/22 v2.19}
%   \item
%     \cs{zref@addprop} and \cs{zref@localaddprop} are limited
%     to one property only
%     (incompatibility to versions v2.13 to v2.18).
%   \item
%     \cs{zref@addprops} and \cs{zref@localaddprops} added.
%   \item
%     \cs{zref@delprop} and \cs{zref@localdelprop} added.
%   \item
%     \cs{zref@labelbykv} and \cs{zkvlabel} (module \xmodule{user})
%     with keys \xoption{prop}, \xoption{list}, \xoption{delprop},
%     \xoption{immediate}, \xoption{values} added.
%   \end{Version}
%   \begin{Version}{2011/02/12 v2.20}
%   \item
%     Fix for warning in \xpackage{zref-xr}.
%   \end{Version}
%   \begin{Version}{2011/03/18 v2.21}
%   \item
%     Fix in module \xmodule{pagelayout} for \cs{zlistpagelayout}.
%   \item
%     Fix for \cs{zref@localaddprop} (probably since v2.19).
%   \end{Version}
%   \begin{Version}{2011/10/05 v2.22}
%   \item
%     Documentation fixed for \cs{zref@(local)addprop(s)}.
%   \item
%     Module \xmodule{base}: \cs{zref@def@extract}, \cs{zref@def@extractdefault}
%     added.
%   \item
%     Fix in module \xmodule{pagelayout}: Because of missing \cs{noexpand}
%     commands the values of the pagelayout properties on all pages were
%     the values at package loading.
%   \item
%     Module \xmodule{base}: \cs{zref@showprop} added.
%   \end{Version}
%   \begin{Version}{2011/12/05 v2.23}
%   \item
%     Module \xmodule{savepos}: \cs{zsaveposx} and \cs{zsaveposy} added.
%   \end{Version}
%   \begin{Version}{2012/04/04 v2.24}
%   \item
%     Module \xmodule{titleref}, package \xpackage{titlesec}:
%     some support for class `straight' (\cs{ttl@straight@i}) added.
%   \end{Version}
%   \begin{Version}{2016/05/16 v2.25}
%   \item
%     Documentation updates.
%   \end{Version}
%   \begin{Version}{2016/05/21 v2.26}
%   \item
%     update zref-savepos for new luatex
%   \end{Version}
%   \begin{Version}{2018/11/21 v2.27}
%   \item
%     adapted zref-perpage, see issue https://github.com/ho-tex/zref/issues/2
%   \end{Version}
%   \begin{Version}{2019/11/29 v2.28}
%   \item
%     Documentation updates.
%   \item
%     Use \xpackage{iftex} directly.
%   \end{Version}
%   \begin{Version}{2020-03-03 v2.29}
%   \item
%     adapted in module zref-pagelayout the properties pdfhorigin, pdfvorigin,
%     pdfpagewidth, pdfpageheight for luatex to the right primitives.
%   \item
%     Removed no longer needed code for older lualatex versions.
%   \item added some documentation of the abspos module.
%   \item adapted the abspos module to the new luatex primitives.
%   \item adapted pageattr module to the new luatex primitives.
%   \item added short documentation for pageattr module
%   \item use luatex command names directly in zref-savepos rather than defining pdftex compatibility names.
%   \item allow zref-abspos to use \verb|\pdf[vh]origin| in dvi mode.
%   \end{Version}
%   \begin{Version}{2020-03-04 v2.30}
%   \item
%     add pagevalue property to savepos in the abspos module (issue 1)
%   \end{Version}
%   \begin{Version}{2020-05-28 v2.31}
%   \item Adapted module zref-counter to use \cs{@currentcounter} in the next
%   \LaTeX version (issue 5)
%   \end{Version}
%   \begin{Version}{2020-07-03 v2.32}
%   \item Changed in zref-pagelayout the names of the shipout box dimensions
%    to adapt to the new hook management.
%   \end{Version}
% \end{History}
%
% \PrintIndex
%
% \Finale
\endinput
|
% \end{quote}
% Do not forget to quote the argument according to the demands
% of your shell.
%
% \paragraph{Generating the documentation.}
% You can use both the \xfile{.dtx} or the \xfile{.drv} to generate
% the documentation. The process can be configured by the
% configuration file \xfile{ltxdoc.cfg}. For instance, put this
% line into this file, if you want to have A4 as paper format:
% \begin{quote}
%   \verb|\PassOptionsToClass{a4paper}{article}|
% \end{quote}
% An example follows how to generate the
% documentation with pdf\LaTeX:
% \begin{quote}
%\begin{verbatim}
%pdflatex zref.dtx
%makeindex -s gind.ist zref.idx
%pdflatex zref.dtx
%makeindex -s gind.ist zref.idx
%pdflatex zref.dtx
%\end{verbatim}
% \end{quote}
%
% \begin{thebibliography}{99}
% \bibpackage{footmisc}{Robin Fairbairns}{2004/01/23 v5.3a}
%   {\CTANpkg{footmisc}}
% \bibpackage{hyperref}{Sebastian Rahtz, Heiko Oberdiek}{2006/08/16 v6.75c}
%   {\CTANpkg{hyperref}}
% \bibpackage{lastpage}{Jeff Goldberg}{1994/06/25 v0.1b}
%   {\CTANpkg{lastpage}}
% \bibpackage{nameref}{Sebastian Rahtz, Heiko Oberdiek}{2006/02/12 v2.24}
%   {\CTANpkg{nameref}}
% \bibpackage{perpage}{David Kastrup}{2002/12/20 v1.0}
%   {\CTANpkg{perpage}}
% \bibpackage{titleref}{Donald Arsenau}{2001/04/05 v3.1}
%   {\CTANpkg{titleref}}
% \bibpackage{totpages}{Wilhelm M\"uller}{1999/07/14 v1.00}
%   {\CTANpkg{totpages}}
% \bibpackage{xr}{David Carlisle}{1994/05/28 v5.02}
%   {\CTANpkg{xr}}
% \bibpackage{xr-hyper}{David Carlisle}{2000/03/22 v6.00beta4}
%   {\CTANpkg{xr-hyper}}
% \end{thebibliography}
%
% \begin{History}
%   \begin{Version}{2006/02/20 v1.0}
%   \item
%     First version.
%   \end{Version}
%   \begin{Version}{2006/05/03 v1.1}
%   \item
%     Module \xmodule{perpage} added.
%   \item
%     Module redesign as packages.
%   \end{Version}
%   \begin{Version}{2006/05/25 v1.2}
%   \item
%     Module \xmodule{dotfillmin} added.
%   \item
%     Module \xmodule{base}: macros \cs{zref@require@uniqe}
%     and \cs{thezref@unique} added
%     (used by modules \xmodule{titleref} and \xmodule{dotfillmin}).
%   \end{Version}
%   \begin{Version}{2006/09/08 v1.3}
%   \item
%     Typo fixes and English cleanup by Per Starback.
%   \end{Version}
%   \begin{Version}{2007/01/23 v1.4}
%   \item
%     Typo in macro name fixed in documentation.
%   \end{Version}
%   \begin{Version}{2007/02/18 v1.5}
%   \item
%     \cs{zref@getcurrent} added (suggestion of Igor Akkerman).
%   \item
%     Module \xmodule{savepos} also supports \hologo{XeTeX}.
%   \end{Version}
%   \begin{Version}{2007/04/06 v1.6}
%   \item
%     Fix in modules \xmodule{abspage} and \xmodule{base}:
%     Now counter |abspage| and |zref@unique|
%     are not remembered by \cs{include}.
%   \item
%     Beamer support for module \xmodule{titleref}.
%   \end{Version}
%   \begin{Version}{2007/04/17 v1.7}
%   \item
%     Package \xpackage{atbegshi} replaces \xpackage{everyshi}.
%   \end{Version}
%   \begin{Version}{2007/04/22 v1.8}
%   \item
%     \cs{zref@wrapper@babel} and \cs{zref@refused} are now expandable
%     if babel is not used or \cs{if@safe@actives} is already set to true.
%     (Feature request of Josselin Noirel)
%   \end{Version}
%   \begin{Version}{2007/05/02 v1.9}
%   \item
%     Module \xmodule{titleref}: Some support for \cs{caption} of package
%     \xpackage{longtable}, but only if \cs{label} is given after
%     \cs{caption}.
%   \end{Version}
%   \begin{Version}{2007/05/06 v2.0}
%   \item
%     Uses package \xpackage{etexcmds} for accessing \hologo{eTeX}'s \cs{unexpanded}.
%   \end{Version}
%   \begin{Version}{2007/05/28 v2.1}
%   \item
%     Module \xmodule{titleref} supports caption of package \xpackage{listings}.
%   \item
%     Fixes in module \xmodule{titleref} for support
%     of packages \xpackage{titlesec} and \xpackage{longtable}.
%   \end{Version}
%   \begin{Version}{2008/09/21 v2.2}
%   \item
%     Module \xmodule{base}: \cs{zref@iflistcontainsprop} is documented, but
%     a broken \cs{zref@listcontainsprop} implemented. Name and
%     implementation fixed (thanks Ohad Kammar).
%   \end{Version}
%   \begin{Version}{2008/10/01 v2.3}
%   \item
%     \cs{zref@localaddprop} added (feature request of Ohad Kammar).
%   \item
%     Module \xmodule{lastpage}: list `LastPage' added. Label `LastPage' will use
%     the properties of this list (default is empty) along with the
%     properties of the main list.
%   \end{Version}
%   \begin{Version}{2009/08/07 v2.4}
%   \item
%     Module \xmodule{runs} added.
%   \end{Version}
%   \begin{Version}{2009/12/06 v2.5}
%   \item
%     Module \xmodule{lastpage}: Uses package \xpackage{atveryend}.
%   \item
%     Module \xmodule{titleref}: Further commands are disabled
%     during string expansion, imported from package \xpackage{nameref}.
%   \end{Version}
%   \begin{Version}{2009/12/07 v2.6}
%   \item
%     Version date added for package \xpackage{atveryend}.
%   \end{Version}
%   \begin{Version}{2009/12/08 v2.7}
%   \item
%     Module \xmodule{titleref}: Use of package \xpackage{gettitlestring}.
%   \end{Version}
%   \begin{Version}{2010/03/26 v2.8}
%   \item
%     \cs{zifrefundefined} added.
%   \item
%     Module \xmodule{lastpage}: Macros \cs{zref@iflastpage}
%     and \cs{ziflastpage} added.
%   \item
%     Module \xmodule{thepage} added.
%   \item
%     Module \xmodule{nextpage} added.
%   \end{Version}
%   \begin{Version}{2010/03/29 v2.9}
%   \item
%     Module \xmodule{marks} added (without documentation).
%   \item
%     \cs{zref@addprop} now adds expanded property to list.
%   \item
%     Useless \cs{ZREF@ErrorNoLine} removed.
%   \end{Version}
%   \begin{Version}{2010/04/08 v2.10}
%   \item
%     Module \xmodule{xr} remembers the external document name
%     in property `externaldocument'.
%   \end{Version}
%   \begin{Version}{2010/04/15 v2.11}
%   \item
%     Module \xmodule{titleref}: Better support of class \xclass{memoir}.
%   \item
%     Module \xmodule{titleref}: Support of theorems.
%   \end{Version}
%   \begin{Version}{2010/04/17 v2.12}
%   \item
%     Module \xmodule{base}: \cs{zref@newprop} ensures global
%     empty default.
%   \item
%     Module \xmodule{xr}: Setup options \xoption{tozreflabel}
%     and \xoption{toltxlabel} added.
%   \end{Version}
%   \begin{Version}{2010/04/19 v2.13}
%   \item
%     \cs{zref@setcurrent} throws an error if the property
%     does not exist (Florent Chervet).
%   \item
%     \cs{zref@getcurrent} the documentation is fixed (Florent Chervet).
%     Also it returns the empty string in case of errors.
%   \item
%     \cs{zref@addprop} and \cs{zref@localaddprop} now take a list of
%     property names (feature request of Florent Chervet).
%   \item
%     Example for \cs{zref@wrapper@unexpanded} corrected (Florent Chervet).
%   \end{Version}
%   \begin{Version}{2010/04/22 v2.14}
%   \item
%     Bug fix for \cs{zref@getcurrent} second argument wasn't eaten in
%     case of unknown property.
%   \item
%     \cs{zref@getcurrent} supports \cs{zref@wrapper@unexpanded}.
%   \item
%     \cs{zref@wrapper@unexpanded} added for \cs{ZREF@xr@tolabel}.
%   \item
%     \cs{zref@extract}, \cs{zref@extractdefault}, \cs{zref@getcurrent}
%     are expandable in exact two steps
%     except inside \cs{zref@wrapper@unexpanded}.
%   \end{Version}
%   \begin{Version}{2010/04/23 v2.15}
%   \item
%     \cs{zexternaldocument} fixed for property `url' when
%     importing \cs{new@label} (bug found by Victor Ivrii).
%   \item
%     Two expansion steps also in \cs{zref@wrapper@unexpanded}.
%   \item
%     Nested calls of \cs{zref@wrapper@unexpanded} possible.
%   \end{Version}
%   \begin{Version}{2010/04/28 v2.16}
%   \item
%     More consequent use of package `ltxcmds' and `hologo'.
%   \item
%     Module \xmodule{pagelayout} added.
%   \item
%     Module \xmodule{pageattr} added.
%   \item
%     Robustness introduced for non-expandable interface macros.
%   \item
%     Internal change of the data format of property lists
%     (suggestion of Florent Chervet).
%   \item
%     Module \xmodule{titleref}: Support of environment \textsf{description}.
%   \end{Version}
%   \begin{Version}{2010/05/01 v2.17}
%   \item
%     \cs{zref@newprop} throws an error if the property already exists.
%   \item
%     Module \xmodule{xr}: Bug fix for the case of several \xext{aux} files
%     (bug found by Victor Ivrii).
%   \item
%     Module \xmodule{xr}: Property `urluse' and option \xoption{urluse} added.
%   \end{Version}
%   \begin{Version}{2010/05/13 v2.18}
%   \item
%     Module \xmodule{env} added.
%   \item
%     Module \xmodule{savepos}: \cs{zref@savepos} added.
%   \end{Version}
%   \begin{Version}{2010/10/22 v2.19}
%   \item
%     \cs{zref@addprop} and \cs{zref@localaddprop} are limited
%     to one property only
%     (incompatibility to versions v2.13 to v2.18).
%   \item
%     \cs{zref@addprops} and \cs{zref@localaddprops} added.
%   \item
%     \cs{zref@delprop} and \cs{zref@localdelprop} added.
%   \item
%     \cs{zref@labelbykv} and \cs{zkvlabel} (module \xmodule{user})
%     with keys \xoption{prop}, \xoption{list}, \xoption{delprop},
%     \xoption{immediate}, \xoption{values} added.
%   \end{Version}
%   \begin{Version}{2011/02/12 v2.20}
%   \item
%     Fix for warning in \xpackage{zref-xr}.
%   \end{Version}
%   \begin{Version}{2011/03/18 v2.21}
%   \item
%     Fix in module \xmodule{pagelayout} for \cs{zlistpagelayout}.
%   \item
%     Fix for \cs{zref@localaddprop} (probably since v2.19).
%   \end{Version}
%   \begin{Version}{2011/10/05 v2.22}
%   \item
%     Documentation fixed for \cs{zref@(local)addprop(s)}.
%   \item
%     Module \xmodule{base}: \cs{zref@def@extract}, \cs{zref@def@extractdefault}
%     added.
%   \item
%     Fix in module \xmodule{pagelayout}: Because of missing \cs{noexpand}
%     commands the values of the pagelayout properties on all pages were
%     the values at package loading.
%   \item
%     Module \xmodule{base}: \cs{zref@showprop} added.
%   \end{Version}
%   \begin{Version}{2011/12/05 v2.23}
%   \item
%     Module \xmodule{savepos}: \cs{zsaveposx} and \cs{zsaveposy} added.
%   \end{Version}
%   \begin{Version}{2012/04/04 v2.24}
%   \item
%     Module \xmodule{titleref}, package \xpackage{titlesec}:
%     some support for class `straight' (\cs{ttl@straight@i}) added.
%   \end{Version}
%   \begin{Version}{2016/05/16 v2.25}
%   \item
%     Documentation updates.
%   \end{Version}
%   \begin{Version}{2016/05/21 v2.26}
%   \item
%     update zref-savepos for new luatex
%   \end{Version}
%   \begin{Version}{2018/11/21 v2.27}
%   \item
%     adapted zref-perpage, see issue https://github.com/ho-tex/zref/issues/2
%   \end{Version}
%   \begin{Version}{2019/11/29 v2.28}
%   \item
%     Documentation updates.
%   \item
%     Use \xpackage{iftex} directly.
%   \end{Version}
%   \begin{Version}{2020-03-03 v2.29}
%   \item
%     adapted in module zref-pagelayout the properties pdfhorigin, pdfvorigin,
%     pdfpagewidth, pdfpageheight for luatex to the right primitives.
%   \item
%     Removed no longer needed code for older lualatex versions.
%   \item added some documentation of the abspos module.
%   \item adapted the abspos module to the new luatex primitives.
%   \item adapted pageattr module to the new luatex primitives.
%   \item added short documentation for pageattr module
%   \item use luatex command names directly in zref-savepos rather than defining pdftex compatibility names.
%   \item allow zref-abspos to use \verb|\pdf[vh]origin| in dvi mode.
%   \end{Version}
%   \begin{Version}{2020-03-04 v2.30}
%   \item
%     add pagevalue property to savepos in the abspos module (issue 1)
%   \end{Version}
%   \begin{Version}{2020-05-28 v2.31}
%   \item Adapted module zref-counter to use \cs{@currentcounter} in the next
%   \LaTeX version (issue 5)
%   \end{Version}
%   \begin{Version}{2020-07-03 v2.32}
%   \item Changed in zref-pagelayout the names of the shipout box dimensions
%    to adapt to the new hook management.
%   \end{Version}
% \end{History}
%
% \PrintIndex
%
% \Finale
\endinput

%        (quote the arguments according to the demands of your shell)
%
% Documentation:
%    (a) If zref.drv is present:
%           latex zref.drv
%    (b) Without zref.drv:
%           latex zref.dtx; ...
%    The class ltxdoc loads the configuration file ltxdoc.cfg
%    if available. Here you can specify further options, e.g.
%    use A4 as paper format:
%       \PassOptionsToClass{a4paper}{article}
%
%    Programm calls to get the documentation (example):
%       pdflatex zref.dtx
%       makeindex -s gind.ist zref.idx
%       pdflatex zref.dtx
%       makeindex -s gind.ist zref.idx
%       pdflatex zref.dtx
%
% Installation:
%    TDS:tex/latex/zref/zref.sty
%    TDS:tex/latex/zref/zref-base.sty
%    TDS:tex/latex/zref/zref-abspage.sty
%    TDS:tex/latex/zref/zref-abspos.sty
%    TDS:tex/latex/zref/zref-counter.sty
%    TDS:tex/latex/zref/zref-dotfill.sty
%    TDS:tex/latex/zref/zref-env.sty
%    TDS:tex/latex/zref/zref-hyperref.sty
%    TDS:tex/latex/zref/zref-lastpage.sty
%    TDS:tex/latex/zref/zref-marks.sty
%    TDS:tex/latex/zref/zref-nextpage.sty
%    TDS:tex/latex/zref/zref-pageattr.sty
%    TDS:tex/latex/zref/zref-pagelayout.sty
%    TDS:tex/latex/zref/zref-perpage.sty
%    TDS:tex/latex/zref/zref-runs.sty
%    TDS:tex/latex/zref/zref-savepos.sty
%    TDS:tex/latex/zref/zref-thepage.sty
%    TDS:tex/latex/zref/zref-titleref.sty
%    TDS:tex/latex/zref/zref-totpages.sty
%    TDS:tex/latex/zref/zref-user.sty
%    TDS:tex/latex/zref/zref-xr.sty
%    TDS:doc/latex/zref/zref.pdf
%    TDS:doc/latex/zref/zref-example.tex
%    TDS:doc/latex/zref/zref-example-lastpage.tex
%    TDS:doc/latex/zref/zref-example-nextpage.tex
%    TDS:source/latex/zref/zref.dtx
%
%<*ignore>
\begingroup
  \catcode123=1 %
  \catcode125=2 %
  \def\x{LaTeX2e}%
\expandafter\endgroup
\ifcase 0\ifx\install y1\fi\expandafter
         \ifx\csname processbatchFile\endcsname\relax\else1\fi
         \ifx\fmtname\x\else 1\fi\relax
\else\csname fi\endcsname
%</ignore>
%<*install>
\input docstrip.tex
\Msg{************************************************************************}
\Msg{* Installation}
\Msg{* Package: zref 2020-07-03 v2.32 A new reference scheme for LaTeX (HO)}
\Msg{************************************************************************}

\keepsilent
\askforoverwritefalse

\let\MetaPrefix\relax
\preamble

This is a generated file.

Project: zref
Version: 2020-07-03 v2.32

Copyright (C)
  2006-2012 Heiko Oberdiek
  2016-2020 Oberdiek Package Support Group

This work may be distributed and/or modified under the
conditions of the LaTeX Project Public License, either
version 1.3c of this license or (at your option) any later
version. This version of this license is in
   https://www.latex-project.org/lppl/lppl-1-3c.txt
and the latest version of this license is in
   https://www.latex-project.org/lppl.txt
and version 1.3 or later is part of all distributions of
LaTeX version 2005/12/01 or later.

This work has the LPPL maintenance status "maintained".

The Current Maintainers of this work are
Heiko Oberdiek and the Oberdiek Package Support Group
https://github.com/ho-tex/zref/issues


This work consists of the main source file zref.dtx
and the derived files
   zref.sty, zref.pdf, zref.ins, zref.drv, zref-base.sty,
   zref-abspage.sty, zref-abspos.sty, zref-counter.sty,
   zref-dotfill.sty, zref-env.sty, zref-hyperref.sty,
   zref-lastpage.sty, zref-marks.sty, zref-nextpage.sty,
   zref-pageattr.sty, zref-pagelayout.sty, zref-perpage.sty,
   zref-runs.sty, zref-savepos.sty, zref-thepage.sty,
   zref-titleref.sty, zref-totpages.sty, zref-user.sty,
   zref-xr.sty, zref-example.tex, zref-example-lastpage.tex,
   zref-example-nextpage.tex, zref-test1.tex,
   zref-test-base.tex, zref-test-runs.tex,
   zref-test-titleref-memoir.tex.

\endpreamble
\let\MetaPrefix\DoubleperCent

\generate{%
  \file{zref.ins}{\from{zref.dtx}{install}}%
  \file{zref.drv}{\from{zref.dtx}{driver}}%
  \usedir{tex/latex/zref}%
  \file{zref.sty}{\from{zref.dtx}{package}}%
  \file{zref-base.sty}{\from{zref.dtx}{base}}%
  \file{zref-abspage.sty}{\from{zref.dtx}{abspage}}%
  \file{zref-abspos.sty}{\from{zref.dtx}{abspos}}%
  \file{zref-counter.sty}{\from{zref.dtx}{counter}}%
  \file{zref-dotfill.sty}{\from{zref.dtx}{dotfill}}%
  \file{zref-env.sty}{\from{zref.dtx}{env}}%
  \file{zref-hyperref.sty}{\from{zref.dtx}{hyperref}}%
  \file{zref-lastpage.sty}{\from{zref.dtx}{lastpage}}%
  \file{zref-marks.sty}{\from{zref.dtx}{marks}}%
  \file{zref-nextpage.sty}{\from{zref.dtx}{nextpage}}%
  \file{zref-pageattr.sty}{\from{zref.dtx}{pageattr}}%
  \file{zref-pagelayout.sty}{\from{zref.dtx}{pagelayout}}%
  \file{zref-perpage.sty}{\from{zref.dtx}{perpage}}%
  \file{zref-runs.sty}{\from{zref.dtx}{runs}}%
  \file{zref-savepos.sty}{\from{zref.dtx}{savepos}}%
  \file{zref-thepage.sty}{\from{zref.dtx}{thepage}}%
  \file{zref-titleref.sty}{\from{zref.dtx}{titleref}}%
  \file{zref-totpages.sty}{\from{zref.dtx}{totpages}}%
  \file{zref-user.sty}{\from{zref.dtx}{user}}%
  \file{zref-xr.sty}{\from{zref.dtx}{xr}}%
  \usedir{doc/latex/zref}%
  \file{zref-example.tex}{\from{zref.dtx}{example}}%
  \file{zref-example-lastpage.tex}{\from{zref.dtx}{example-lastpage}}%
  \file{zref-example-nextpage.tex}{\from{zref.dtx}{example-nextpage}}%
%  \usedir{doc/latex/zref/test}%
%  \file{zref-test1.tex}{\from{zref.dtx}{test1}}%
%  \file{zref-test-base.tex}{\from{zref.dtx}{test-base}}%
%  \file{zref-test-runs.tex}{\from{zref.dtx}{test-runs}}%
%  \file{zref-test-titleref-memoir.tex}{\from{zref.dtx}{test-titleref-memoir}}%
}

\catcode32=13\relax% active space
\let =\space%
\Msg{************************************************************************}
\Msg{*}
\Msg{* To finish the installation you have to move the following}
\Msg{* files into a directory searched by TeX:}
\Msg{*}
\Msg{*     zref.sty, zref-base.sty, zref-abspage.sty, zref-abspos.sty,}
\Msg{*     zref-counter.sty, zref-dotfill.sty, zref-env.sty,}
\Msg{*     zref-hyperref.sty, zref-lastpage.sty, zref-marks.sty,}
\Msg{*     zref-nextpage.sty, zref-pageattr.sty, zref-pagelayout.sty,}
\Msg{*     zref-perpage.sty, zref-runs.sty, zref-savepos.sty,}
\Msg{*     zref-thepage.sty, zref-titleref.sty, zref-totpages.sty,}
\Msg{*     zref-user.sty, zref-xr.sty}
\Msg{*}
\Msg{* To produce the documentation run the file `zref.drv'}
\Msg{* through LaTeX.}
\Msg{*}
\Msg{* Happy TeXing!}
\Msg{*}
\Msg{************************************************************************}

\endbatchfile
%</install>
%<*ignore>
\fi
%</ignore>
%<*driver>
\NeedsTeXFormat{LaTeX2e}
\ProvidesFile{zref.drv}%
  [2020-07-03 v2.32 A new reference scheme for LaTeX (HO)]%
\documentclass{ltxdoc}
\usepackage{holtxdoc}[2011/11/22]
\setlength{\emergencystretch}{1em}
\begin{document}
  \DocInput{zref.dtx}%
\end{document}
%</driver>
% \fi
%
%
%
% \GetFileInfo{zref.drv}
%
% \title{The \xpackage{zref} package}
% \date{2020-07-03 v2.32}
% \author{Heiko Oberdiek\thanks
% {Please report any issues at \url{https://github.com/ho-tex/zref/issues}}}
%
% \maketitle
%
% \begin{abstract}
% Package \xpackage{zref} tries to get rid of the restriction
% in \hologo{LaTeX}'s reference system that only two properties are
% supported. The package implements an extensible referencing
% system, where properties are handled in a more flexible way.
% It offers an interface for macro programmers for the access
% to the system and some applications that uses the new
% reference scheme.
% \end{abstract}
%
% \tableofcontents
%
% \newcommand*{\metacs}[1]{\texttt{\itshape #1}}
%
% \newcommand{\markexp}{^^A
%   \textsuperscript{exp}^^A
%   \space
% }
% \newcommand{\markexptwo}{^^A
%   \textsuperscript{exp2}^^A
%   \space
% }
% \newcommand{\markrobust}{^^A
%   \textsuperscript{robust}^^A
%   \space
% }
% \newcommand{\marknowarn}{^^A
%   \textsuperscript{nowarn}^^A
%   \space
% }
% \newcommand{\markbabel}{^^A
%   \textsuperscript{babel}^^A
%   \space
% }
%
% \section{Introduction}
%
% Standard \hologo{LaTeX}'s reference system with \cs{label}, \cs{ref},
% and \cs{pageref} supports two properties, the apperance of
% the counter that is last incremented by \cs{refstepcounter} and
% the page with the \cs{label} command.
%
% Unhappily \hologo{LaTeX} does not provide an interface for adding
% another properties. Packages such as \xpackage{hyperref},
% \xpackage{nameref}, or \xpackage{titleref} are forced to
% use ugly hacks to extend the reference system.
% These ugly hacks are one of the causes for \xpackage{hyperref}'s
% difficulty regarding compatibility with other packages.
%
% \subsection{Standard \hologo{LaTeX} behaviour}
%
% References are created by the \cs{label} command:
% \begin{quote}
%   |\chapter{Second chapter}|\\
%   |\section{First section on page 7} % section 2.1|\\
%   |\label{myref}|
% \end{quote}
% Now \hologo{LaTeX} records the section number |2.1| and the page |7|
% in the reference. Internally the reference is a list with
% two entries:
% \begin{quote}
%   |\r@myref| $\rightarrow$ |{2.1}{7}|
% \end{quote}
% The length of the list if fixed in the \hologo{LaTeX} kernel,
% An interface for adding new properties is missing.
%
% There are several tries to add new properties:
% \begin{description}
% \item[\xpackage{hyperref}] uses a list of five
%   properties instead of the standard list with two entries.
%   This causes many compatibility problems with \hologo{LaTeX} and
%   other packages.
% \item[\xpackage{titleref}] stores its title data into the
%   first entry in the list. \hologo{LaTeX} is happy because it
%   does only see its list with two entries. The situation
%   becomes more difficult, if more properties are added
%   this way. Then the macros form a nested structure
%   inside the first reference argument for the label.
%   Expandable extractions will then become painful.
% \end{description}
%
% \subsection{Basic idea}
%
% Some time ago Morten H\o gholm sent me an experimental
% cross referencing mechanism as ``expl3'' code. His
% idea is:
% \begin{quote}
%   |\g_xref_mylabel_plist| $\rightarrow$\\
%   \mbox{}\qquad|\xref_dance_key{salsa}\xref_name_key{Morten}|\dots
% \end{quote}
% The entries have the following format:
% \begin{quote}
%   |\xref_|\meta{your key}|_key{|\meta{some text}|}|
% \end{quote}
% This approach is much more flexible:
% \begin{itemize}
%   \item New properties can easily be added, just use a new key.
%   \item The length of the list is not fixed. A reference
%         can use a subset of the keys.
%   \item The order of the entries does not matter.
% \end{itemize}
% Unhappily I am not familiar with the experimental code
% for \hologo{LaTeX3} that will need some time before its first release.
% Thus I have implemented it as \hologo{LaTeXe} package without disturbing
% the existing \hologo{LaTeX} reference system.
%
% \subsection{Interfaces}
%
% The package provides a generic \emph{interface for programmers}.
% Commands of this interface are prefixed by \cs{zref@}.
%
% Option \xoption{user} enabels the \emph{user interface}.
% Here the commands are prefixed by \cs{z} to avoid name
% clashes with existing macros.
%
% Then the packages provides some \emph{modules}. They
% are applications for the reference system and can also be
% considered as examples how to use the reference system.
%
% The modules can be loaded as packages. The package
% name is prefixed with |zref-|, for example:
%\begin{quote}
%\begin{verbatim}
%\RequirePackage{zref-abspage}
%\end{verbatim}
%\end{quote}
% This is the preferred way if the package is loaded from
% within other packages to avoid option clashes.
%
% As alternative package |zref| can be used and the modules
% are given as options:
%\begin{quote}
%\begin{verbatim}
%\usepackage[perpage,user]{zref}
%\end{verbatim}
%\end{quote}
%
% \section{Interface for programmers}
%
% The user interface is described in the next section \ref{sec:user}.
%
% \subsection{Entities}
%
% \paragraph{Reference.}
% Internally a reference is a list of key value pairs:
% \begin{quote}
%   |\Z@R@myref| $\rightarrow$
%   |\default{2.1}\page{7}|
% \end{quote}
% The generic format of a entry is:
% \begin{quote}
%   |\Z@R@|\meta{refname} $\rightarrow$
%   |\|\meta{propname}|{|\meta{value}|}|
% \end{quote}
% \meta{refname} is the name that denoted references
% (the name used in \cs{label} and \cs{ref}). \meta{propname}
% is the name of the property or key. The property key macro
% is never executed, it is used in parameter text matching only.
%
% \paragraph{Property.}
% Because the name of a property is used in a macro name
% that must survive the \xfile{.aux} file, the name is
% restricted to letters and `@'.
%
% \paragraph{Property list.}
% Often references are used for special purposes. Thus it
% saves memory if just the properties are used in this
% reference that are necessary for its purpose.
%
% Therefore this package uses the concept of \emph{property lists}.
% A property list is a set of properties. The set of properties
% that is used by the default \cs{label} command is the
% \emph{main property list}.
%
% \subsection{Property list}
%
% \markexp means that the implementation of the
% marked macro is expandable. \markexptwo goes a step further
% and marks the macro expandable in exact two expansion steps.
%
% \begin{declcs}{zref@newlist}
%   \M{listname}
% \end{declcs}
% Declares a new empty property list.
%
% \begin{declcs}{zref@addprop}
%   \M{listname} \M{propname}\\
%   \cs{zref@localaddprop} \M{listname} \M{propname}
% \end{declcs}
% Adds the property \meta{propname}
% to the property list \meta{listname}.
% The property and list must exist.
% The addition is global by \cs{zref@addprop} and limited to
% local scope by \cs{zref@localaddprop}.
% Between 2010/04/19 v2.13 and 2010/10/22 v2.19 a comma separated list
% of properties could be used as argument \meta{propname}.
% Since 2010/10/22 v2.19 the addition of several properties
% at once is supported by \cs{zref@addprops}.
%
% \begin{declcs}{zref@addprops}
%   \M{listname} \M{propname list}\\
%   \cs{zref@localaddprops} \M{listname} \M{propname list}
% \end{declcs}
% These macros add a comma separated list of properties
% \meta{propname list} to list \meta{listname}.
% \cs{zref@addprops} works globally and \cs{zref@localaddprops}
% locally. Since 2010/10/22 v2.19.
%
% \begin{declcs}{zref@listexists}
%   \M{listname} \M{then}
% \end{declcs}
% Executes \meta{then} if the property list \meta{listname} exists
% or raise an error otherwise.
%
% \begin{declcs}{zref@iflistundefined}\markexp \M{listname}
%   \M{then} \M{else}
% \end{declcs}
% Executes \meta{then} if the list exists or \meta{else}
% otherwise.
%
% \begin{declcs}{zref@iflistcontainsprop}
%   \M{listname} \M{propname} \M{then} \M{else}
% \end{declcs}
% Executes \meta{then} if the property \meta{propname} is part
% of property list \meta{listname} or otherwise it
% runs the \meta{else} part.
%
% \subsection{Property}
%
% \begin{declcs}{zref@newprop}
%   \* \M{propname} |[|\meta{default}|]| \M{value}
% \end{declcs}
% This command declares and configures a new property with
% name \meta{propname}.
%
% In case of unknown references
% or the property does not exist in the reference, the
% \meta{default} is used as value. If it is not specified
% here, a global default is used, see \cs{zref@setdefault}.
%
% The correct values of some properties are not known
% immediately but at page shipout time. Prominent example
% is the page number. These properties are declared with
% the star form of the command.
%
% \begin{declcs}{zref@setcurrent}
%   \M{propname} \M{value}
% \end{declcs}
% This sets the current value of the property \meta{propname}.
% It is a generalization of setting \hologo{LaTeX}'s \cs{currentlabel}.
%
% \begin{declcs}{zref@getcurrent}\markexptwo
%   \M{propname}
% \end{declcs}
% This returns the current value of the property \meta{propname}.
% The value may not be correct, especially if the property
% is bound to a page (start form of \cs{zref@newprop}) and the
% right value is only known at shipout time (e.g. property `page').
% In case of errors (e.g. unknown property) the empty string is
% returned.
%
% Since version 2010/04/22 v2.14 \cs{zref@getcurrent} supports
% \cs{zref@wrapper@unexpanded}.
%
% \begin{declcs}{zref@propexists}
%   \M{propname} \M{then}
% \end{declcs}
% Calls \meta{then} if the property \meta{propname} is available
% or generates an error message otherwise.
%
% \begin{declcs}{zref@ifpropundefined}\markexp
%   \M{propname} \M{then} \M{else}
% \end{declcs}
% Calls \meta{then} or \meta{else} depending on the existence
% of property \meta{propname}.
%
% \subsection{Reference generation}
%
% \begin{declcs}{zref@label}
%   \M{refname}
% \end{declcs}
% This works similar to \cs{label}. The reference \meta{refname} is
% created and put into the \xfile{.aux} file with the
% properties of the main property list.
%
% \begin{declcs}{zref@labelbylist}
%   \M{refname} \M{listname}
% \end{declcs}
% Same as \cs{zref@label} except that the properties are
% taken from the specified property list \meta{listname}.
%
% \begin{declcs}{zref@labelbyprops}
%   \M{refname} |{|\meta{propnameA}|,|\meta{propnameB}|,|\dots|}|
% \end{declcs}
% Same as \cs{zref@label} except that these properties
% are used that are given as comma separated list in the
% second argument.
%
% \begin{declcs}{zref@newlabel}
%   \M{refname} |{|\dots|}|
% \end{declcs}
% This is the macro that is used in the \xfile{.aux} file.
% It is basically the same as \cs{newlabel} apart from
% the format of the data in the second argument.
%
% \subsection{Data extraction}
%
% \begin{declcs}{zref@extractdefault}\markexptwo
%   \M{refname} \M{propname} \M{default}
% \end{declcs}
% This is the basic command that refernces the value of a
% property \meta{propname} for the reference \meta{refname}.
% In case of errors such as undefined reference the \meta{default}
% is used instead.
%
% \begin{declcs}{zref@extract}\markexptwo
%   \M{refname} \M{propname}
% \end{declcs}
% The command is an abbreviation for \cs{zref@extractdefault}.
% As default the default of the property is taken, otherwise
% the global default.
%
% Example for page references:
% \begin{quote}
% \begin{tabular}{@{}ll}
% \hologo{LaTeX}:& |\pageref{foobar}|\\
% \xpackage{zref}:& |\zref@extract{foobar}{page}|
% \end{tabular}
% \end{quote}
% Both \cs{zref@extract} and \cs{zref@extractdefault} are
% expandable. That means, these macros can directly
% be used in expandable calculations, see the example file.
% On the other side, babel's shorthands are not supported,
% there are no warnings in case of undefined references.
%
% If an user interface doesn't need expandable macros
% then it can use \cs{zref@refused} and \cs{zref@wrapper@babel}
% for its user macros.
%
% \begin{declcs}{zref@refused}
%   \M{refname}
% \end{declcs}
% This command is not expandable. It causes the warnings if
% the reference \meta{refname} is not defined. Use the
% \cs{zref@extract} commands inside expandable contexts and
% mark their use outside by \cs{zref@refused}, see the example
% file.
%
% \begin{declcs}{zref@def@extract}%
%     \M{cmd} \M{refname} \M{propname}\\
%   \cs{zref@def@extractdefault}
%     \M{cmd} \M{refname} \M{propname} \M{default}
% \end{declcs}
% Both macros extract the property \meta{propname} from the reference
% \meta{refname} the same way as macros \cs{zref@extract} and
% \cs{zref@extractdefault}. The result is stored in macro \meta{cmd}.
% Also \cs{zref@refused} is called to notify \hologo{LaTeX} that
% the reference \meta{refname} is used. Added in 2011/10/04 v2.22.
%
% \begin{declcs}{zref@ifrefundefined}\markexp
%   \M{refname} \M{then} \M{else}
% \end{declcs}
% Macro \cs{zref@ifrefundefined} calls arguments \meta{then} or \meta{else}
% dependent on the existence of the reference \meta{refname}.
%
% \begin{declcs}{zifrefundefined}
%   \M{refname} \M{then} \M{else}
% \end{declcs}
% Macro \cs{zifrefundefined} calls \cs{ref@refused} before
% executing \cs{zref@ifrefundefined}. Babel shorthands are
% supported in \meta{refname}.
%
% \begin{declcs}{zref@ifrefcontainsprop}\markexp
%   \M{refname} \M{propname} \M{then} \M{else}
% \end{declcs}
% Test whether a reference provides a property.
%
% \subsection{Setup}
%
% \begin{declcs}{zref@default}
% \end{declcs}
% Holds the global default for unknown values.
%
% \begin{declcs}{zref@setdefault}
%   |{|\meta{value}|}|
% \end{declcs}
% Sets the global default for unknown values.
% The global default is used, if a property does not specify
% an own default and the value for a property
% cannot be extracted. This can happen if the reference is
% unknown or the reference does not have the property.
%
% \begin{declcs}{zref@setmainlist}
%   |{|\meta{value}|}|
% \end{declcs}
% Sets the name of the main property list. The package
% sets and uses |main|.
%
% \subsection{Declared properties}
%
% \begin{quote}
% \renewcommand*{\arraystretch}{1.05}
% \setlength{\tabcolsep}{2\tabcolsep}
% \def\VV{\vphantom{\large \"Ag}}
% \def\EMPTY{{\itshape\textless empty\textgreater}}
% \def\pl#1{&#1&thepage&\cs{number}\cs{#1}}
% \begin{tabular}{@{}>{\sffamily}l|>{\ttfamily}l|>{\ttfamily}l|>{\ttfamily}l@{}}
%   \rmfamily Module&
%   \rmfamily Property&
%   \rmfamily Property list&
%   \rmfamily Default\\
%   \hline
%   (base)
%     & default & main & \EMPTY\\
%     & page & main & \EMPTY\\
%   \hline
%   abspage
%     & abspage & main & 0\\
%   \hline
%   counter
%     & counter & main & \EMPTY\\
%   \hline
%   hyperref
%     & anchor & main & \EMPTY\\
%     & url    & & \EMPTY\\
%   \hline
%   pageattr
%     & pdfpageattr & thepage & \dots\\
%     & pdfpagesattr & LastPage & \dots\\
%   \hline
%   pagelayout\footnotemark
%     \pl{mag}\\
%     \pl{paperwidth}\\
%     \pl{paperheight}\\
%     \pl{stockwidth}\\
%     \pl{stockheight}\\
%     \pl{pdfpageheight}\\
%     \pl{pdfpagewidth}\\
%     \pl{pdfhorigin}\\
%     \pl{pdfvorigin}\\
%     \pl{hoffset}\\
%     \pl{voffset}\\
%     \pl{topmargin}\\
%     \pl{oddsidemargin}\\
%     \pl{evensidemargin}\\
%     \pl{textwidth}\\
%     \pl{textheight}\\
%     \pl{headheight}\\
%     \pl{headsep}\\
%     \pl{footskip}\\
%     \pl{marginparwidth}\\
%     \pl{marginparsep}\\
%     \pl{columnwidth}\\
%     \pl{columnsep}\\
%   \hline
%   perpage
%     & pagevalue & perpage & 0\\
%     & page & perpage & \EMPTY\\
%     & abspage & perpage & 0\\
%   \hline
%   savepos
%     & posx & savepos & 0\\
%     & posy & savepos & 0\\
%   \hline
%   titleref
%     & title & main & \EMPTY\\
%   \hline
%   xr
%     & anchor   & & \EMPTY\\
%     & externaldocument & & \EMPTY\\
%     & theotype & & \EMPTY\\
%     & title    & & \EMPTY\\
%     & url      & & \EMPTY\\
% \end{tabular}
% \footnotetext{Module \xmodule{pagelayout} only defines properties
% if the parameter exists.}%
% \end{quote}
%
% \subsection{Wrapper for advanced situations}
%
% \begin{declcs}{zref@wrapper@babel}
%   |{|\dots|}| \M{name}
% \end{declcs}
% This macro helps to add shorthand support. The second
% argument is protected, then the code of the first argument
% is called with the protected name appended. Examples
% are in the sources.
%
% \begin{declcs}{zref@wrapper@immediate}
%   |{|\dots|}|
% \end{declcs}
% There are situations where a label must be written instantly
% to the \xfile{.aux} file, for example after the last page.
% If the \cs{zlabel} or \cs{label} command is put inside this wrapper,
% immediate writing is enabled. See the implementation for
% module \xmodule{lastpage} for an example of its use.
%
% \begin{declcs}{zref@wrapper@unexpanded}
%   |{|\dots|}|
% \end{declcs}
% Assuming someone wants to extract a value for property |bar|
% and store the result in a macro |\foo| without traces of
% the expanding macros and without expanding the value.
% This (theoretical?) problem can be solved by this wrapper:
% \begin{quote}
%\begin{verbatim}
%\zref@wrapper@unexpanded{%
%  \edef\foo{%
%    \zref@extract{someref}{bar}%
%  }%
%}
%\end{verbatim}
% \end{quote}
% The \cs{edef} forces the expansion of \cs{zref@extract},
% but the extraction of the value is prevented by
% the wrapper that uses \hologo{eTeX}'\ \cs{unexpanded} for this
% purpose. Supported macros are \cs{zref@extract}, \cs{zref@extractdefault}
% and since version 2010/04/22 v2.14 macro \cs{zref@getcurrent}.
%
% \subsection{Counter for unique names}
%
% Some modules (\xmodule{titleref} and \xmodule{dotfillmin})
% need unique names for automatically generated label names.
%
% \begin{declcs}{zref@require@unique}
% \end{declcs}
% This command creates the unique counter |zref@unique| if the
% counter does not already exist.
%
% \begin{declcs}{thezref@unique}
% \end{declcs}
% This command is used to generate unique label names.
%
% \section{User interface}\label{sec:user}
%
% \subsection{Module \xmodule{user}}
%
% The user interface for this package and its modules
% is enabled by \xpackage{zref}'s package option \xmodule{user}
% or package \xpackage{zref-user}.
% The names of user commands are prefixed by |z| in order
% to avoid name clashes with existing macros of the same
% functionality. Thus the package does not disturb the
% traditional reference scheme, both can be used together.
%
% The syntax descriptions contain the following markers
% that are intended as hints for programmers:
% \begin{quote}
% \begin{tabular}{@{}ll@{}}
%   \markbabel\unskip  & Babel shorthands are allowed.\\
%   \markrobust\unskip & Robust macro.\\
%   \markexp\unskip    & Expandable version:\\
%                      & \textbullet\ robust,
%                        unless the extracted values are fragile,\\
%                      & \textbullet\ no babel shorthand suport.\\
%   \markexptwo\unskip & Expandable like \markexp and:\\
%                      & \textbullet\ expandable in exact two steps.\\
% \end{tabular}
% \end{quote}
%
% The basic user interface of the package without modules
% are commands that mimic the standard \hologo{LaTeX} behaviour
% of \cs{label}, \cs{ref}, and \cs{pageref}:
%
% \begin{declcs}{zlabel}
%   \M{refname}\markbabel
% \end{declcs}
% Similar to \cs{label}. It generates a label with name
% \meta{refname} in the new reference scheme.
%
% \begin{declcs}{zref}
%   |[|\meta{propname}|]| \M{refname}\markbabel
% \end{declcs}
% Without optional argument similar to \cs{ref}, it returns
% the default reference property. This property is named
% |default|:
% \[
%   |\zref{|x|}| \equiv |\zref[default]{|x|}|
% \]
%
% \begin{declcs}{zpageref}
%   \M{refname}\markbabel
% \end{declcs}
% Convenience macro, similar to \cs{pageref}.
% \[
%   |\zpageref{|x|}| \equiv |\zref[page]{|x|}|
% \]
%
% \begin{declcs}{zrefused}
%   \M{refname}\markbabel
% \end{declcs}
% Some of the user commands in the modules are expandable.
% The use of such commands do not cause any undefined
% reference warnings, because inside of expandable
% contexts this is not possible. However, if there is
% a place outside of expandable contexts, \cs{refused}
% is strongly recommended. The reference \meta{refname}
% is marked as used, undefined ones will generate
% warnings.
%
% \subsection{Module \xmodule{abspage}}
%
% With the help of package \xpackage{atbegshi}
% a new counter |abspage| with absolute page numbers is provided.
% For technical and historical reasons the counter itself is zero based:
% if you use it directly on the first page, e.g
% with |\arabic{abspage}| you will get 0 as value. When using |\zref|
% the first page will be page 1 as expected.
% Also a new property |abspage| is defined and added to the
% main property list. Thus you can reference the absolute
% page number:
% \begin{quote}
%   |Section \zref{foo} is on page \zpageref{foo}.|\\
%   |This is page \zref[abspage]{foo}|\\
%   |of \zref[abspage]{LastPage}.|
% \end{quote}
% The example also makes use of module \xmodule{lastpage}.
%
% \subsection{Module \xmodule{lastpage}}
%
% Provides the functionality of package \xpackage{lastpage}
% \cite{lastpage} in the new reference scheme.
% The label |LastPage| is put at the end of the document.
% You can refer the last page number with:
% \begin{quote}
%   |\zref@extract{LastPage}{page}| (+ |\zref@refused{LastPage}|)\\
% \end{quote}
% or
% \begin{quote}
%   |\zpageref{LastPage}| (module \xmodule{user})
% \end{quote}
%
% Since version 2008/10/01 v2.3 the module defines the list |LastPage|.
% In addition to the properties of the main list label |LastPage| also
% stores the properties of this list |LastPage|. The default of this
% list is empty. The list can be used by the user to add additional
% properties for label |LastPage|.
%
% \subsubsection{Tests for last page}
%
% Since version 2010/03/26 v2.8 the macros \cs{zref@iflastpage}
% and \cs{ziflastpage} were added. They test the reference,
% whether it is a reference of the last page.
%
% \begin{declcs}{zref@iflastpage}\markexp \M{refname} \M{then} \M{else}
% \end{declcs}
% Macro \cs{zref@iflastpage} compares the references \meta{refname}
% with \meta{LastPage}. Basis of the comparison is the value of
% property |abspage|, because the values are different for different
% pages. This is not ensured by property |page|. Therefore module
% \xmodule{abspage} is loaded by module \xmodule{lastpage}.
% If both values of property |abspage| are present and match,
% then \meta{then} is executed, otherwise code \meta{else} is called.
% If one or both references are undefined or lack the property |abspage|,
% then \meta{else} is executed.
%
% Macro \cs{zref@iflastpage} is expandable, therefore \cs{zref@refused}
% should be called on \meta{refname} and \meta{LastPage}.
%
% \begin{declcs}{ziflastpage} \M{refname} \M{then} \M{else}
% \end{declcs}
% Macro \cs{ziflastpage} has the same function as \cs{zref@iflastpage},
% but adds support for babel shorthands in \meta{refname} and
% calls \cs{zref@refused}. However macro \cs{ziflastpage} is not
% expandable.
%
% \subsubsection{Example}
%
%    \begin{macrocode}
%<*example-lastpage>
%<<END_EXAMPLE
\NeedsTeXFormat{LaTeX2e}
\documentclass{report}

\newcounter{foo}
\renewcommand*{\thefoo}{\Alph{foo}}

\usepackage{zref-lastpage,zref-user}[2019/11/29]

\makeatletter
\zref@newprop{thefoo}{\thefoo}
\zref@newprop{valuefoo}{\the\value{foo}}
\zref@newprop{chapter}{\thechapter}
\zref@addprops{LastPage}{thefoo,valuefoo,chapter}
\makeatother

\newcommand*{\foo}{%
  \stepcounter{foo}%
  [Current foo: \thefoo]%
}

\begin{document}
  \chapter{First chapter}
  Last page is \zref{LastPage}.\\
  Last chapter is \zref[chapter]{LastPage}.\\
  Last foo is \zref[thefoo]{LastPage}.\\
  Last value of foo is \zref[valuefoo]{LastPage}.\\
  \foo
  \chapter{Second chapter}
  \foo\foo\foo
  \chapter{Last chapter}
  \foo
\end{document}
%END_EXAMPLE
%</example-lastpage>
%    \end{macrocode}
%
% \subsection{Module \xmodule{thepage}}
%
% This module \xmodule{thepage} loads module \xmodule{abspage},
% constructs a reference name using the absolute page number
% and remembers property |page|. Other properties can be added
% by adding them to the property list |thepage|.
%
% \begin{declcs}{zthepage} \M{absolute page number}\\
% \end{declcs}
% Macro \cs{zthepage} is basically a \cs{zpageref}. The reference
% name is yield by the \meta{absolute page number}. If the reference
% is not defined, then the default for property |page| is used.
%
% \begin{declcs}{zref@thepage@name}\markexp \M{absolute page number}
% \end{declcs}
% Macro \cs{zref@thepage@name} returns the internal reference
% name that is constructed using the \meta{absolute page number}.
% The internal reference name should not be used directly, because
% it might change in future versions.
%
% \begin{declcs}{zref@thepage}\markexp \M{absolute page number}\\
%   \cs{zref@thepage@refused} \M{absolute page number}
% \end{declcs}
% Macro \cs{zref@thepage} returns the page number (\cs{thepage})
% of \meta{absolute page number}. Because this macro is expandable,
% \cs{zref@thepage@refused} is used outside an expandable context
% to mark the reference as used.
%
%
% \subsection{Module \xmodule{nextpage}}
%
% \begin{declcs}{znextpage}
% \end{declcs}
% Macro \cs{znextpage} prints \cs{thepage} of the following page.
% It gets the current absolute page number by using a label.
% There are three cases for the next page:
% \begin{enumerate}
% \item The next page is not known yet because of undefined references.
%   Then \cs{zunknownnextpagename} is used instead. The default for
%   this macro is the default of property |page|.
% \item This page is the last page. Then \cs{znonextpagename}
%   is used. Its default is empty.
% \item The next page is known, then \cs{thepage} of the next
%   page is used (the value of property |page| of the next page).
% \end{enumerate}
%
% \subsubsection{Configuration}
%
% The behaviour can be configured by the following macros.
%
% \begin{declcs}{zunknownnextpagename}\\
%   \cs{znonextpagename}
% \end{declcs}
% If the next page is not known or available, then
% \cs{znextpage} uses these name macros as default.
% \cs{zunknownnextpagename} is used in case of undefined
% references. Default is the value of property |page| of the
% next page (\cs{thepage}). Module \xmodule{thepage} is used.
%
% Macro \cs{znonextpagename} is used, if the next page does
% not exists. That means that the current page is last page.
% The default is empty.
%
% \begin{declcs}{znextpagesetup} \M{unknown} \M{no next}
%   \M{next}
% \end{declcs}
% Acording to the case (see \cs{znextpage}) macro \cs{znextpage}
% calls an internal macro with an argument. The argument is
% either \cs{thepage} of the next page or one of \cs{zunknownnextpagename}
% or \cs{znonextpagename}. These internal macro can be changed by
% \cs{znextpagesetup}. It expects the definition texts for these
% three cases of a macro with one argument.
% The default is
% \begin{quote}
%   |\znextpagesetup{#1}{#1}{#1}|
% \end{quote}
%
% \subsubsection{Example}
%
%    \begin{macrocode}
%<*example-nextpage>
%<<END_EXAMPLE
\documentclass{book}

\usepackage{zref-nextpage}[2019/11/29]
\znextpagesetup
  {\thepage}% next page is unknown
  {\thepage\ (#1)}% this page is last page
  {\thepage\ $\rightarrow$ #1}% next page is known
\renewcommand*{\znonextpagename}{last page}

\usepackage{fancyhdr}
\pagestyle{fancy}
\fancyhf{}
\fancyhead[LE,RO]{\znextpage}
\fancypagestyle{plain}{%
  \fancyhf{}%
  \fancyhead[LE,RO]{\znextpage}%
}

\begin{document}
\frontmatter
  \tableofcontents
\mainmatter
  \chapter{Hello World}
  \clearpage
  \section{Last section}
\end{document}
%END_EXAMPLE
%</example-nextpage>
%    \end{macrocode}
%
%
% \subsection{Module \xmodule{totpages}}
%
% For the total number of pages of a document you need to
% know the absolute page number of the last page.
% Both modules \xmodule{abspage} and \xmodule{lastpage}
% are necessary and automatically enabled.
%
% \begin{declcs}{ztotpages}\markexp
% \end{declcs}
% Prints the total number of pages or |0| if this
% number is not yet known. It expands to an explicit number
% and can also used even in expandable calculations (\cs{numexpr})
% or counter assignments.
%
% \subsection{Module \xmodule{pagelayout}}
%
% The module defines additional properties for each parameter
% of the page layout that is effective during page shipout.
% The value of length parameters is given in sp without the
% unit as plain number.
%
% Some parameters are specific for a class (e.g.\@ \xoption{stockwidth}
% and \xoption{stockheight} for class \xclass{memoir}) or the
% \hologo{TeX} engine like \hologo{pdfTeX}. If the parameter is not
% available, then the property will not be defined. The default value
% of the property is the current setting of the parameter.
%
% The module \xmodule{thepage} is loaded that generates a label
% for each page. The properties of module \xmodule{pagelayout}
% are added to the property list |thepage| of module \xmodule{thepage}.
%
% List of properties:
% \begin{quote}
%   \def\x#1{\cs{#1}&#1}%
%   \begin{tabular}{l>{\sffamily}ll}
%     \textrm{parameter} & \textrm{property} & remarks\\
%     \hline
%     \x{mag}\\
%     \x{paperwidth}\\
%     \x{paperheight}\\
%     \x{stockwidth}& class \xclass{memoir}\\
%     \x{stockheight}& class \xclass{memoir}\\
%     \x{pdfpagewidth}& \hologo{pdfTeX}, \hologo{LuaTeX}\\
%     \x{pdfpageheight}& \hologo{pdfTeX}, \hologo{LuaTeX}\\
%     \x{pdfhorigin}& \hologo{pdfTeX}, \hologo{LuaTeX}\\
%     \x{pdfvorigin}& \hologo{pdfTeX}, \hologo{LuaTeX}\\
%     \x{hoffset}\\
%     \x{voffset}\\
%     \x{topmargin}\\
%     \x{oddsidemargin}\\
%     \x{evensidemargin}\\
%     \x{textwidth}\\
%     \x{textheight}\\
%     \x{headheight}\\
%     \x{headsep}\\
%     \x{footskip}\\
%     \x{marginparwidth}\\
%     \x{marginparsep}\\
%     \x{columnwidth}\\
%     \x{columnsep}\\
%   \end{tabular}
% \end{quote}
%
% \begin{declcs}{zlistpagelayout}
% \end{declcs}
% At the end of document the page layout parameter for each page are
% printed into the \xext{log} file if macro \cs{zlistpagelayout}
% is called before |\end{document}| (preamble is a good place).
%
% \subsection{Module \xmodule{marks}}
%
% ToDo.
%
% \subsection{Module \xmodule{runs}}
%
% Module \xmodule{runs} counts the \hologo{LaTeX} runs since last
% \xext{aux} file creation and prints the number in the
% \xext{log} file.
%
% \begin{declcs}{zruns}\markexp
% \end{declcs}
% Prints the the total number of \hologo{LaTeX} runs including
% the current one. It expands to an explicit number.
% Before |begin{document}| the value is zero meaning
% the \xext{aux} file is not read yet. If a previous
% \xext{aux} file exists, the value found there increased
% by one is the new number. Otherwise \cs{zruns} is set to one.
% \hologo{LaTeX}~runs where the \xext{aux} files are not rewritten
% are not counted (see \cs{nofiles}).
%
% \subsection{Module \xmodule{perpage}}
%
% With \cs{@addtoreset} or \cs{numberwithin} a counter can be
% reset if another counter is incremented. This do not work
% well if the other counter is the page counter. The page counter is
% incremented in the output routine that is often called asynchronous
% somewhere on the next page. A reference mechanism costs at least
% two \hologo{LaTeX} runs, but ensures correct page counter values.
%
% \begin{declcs}{zmakeperpage}
%   |[|\meta{reset}|]| \M{counter}
% \end{declcs}
% At the of a new page counter \meta{counter} starts
% counting with value \meta{reset} (default is \texttt{1}).
% The macro has the same
% syntax and semantics as \cs{MakePerPage} of package
% \xpackage{perpage} \cite{perpage}.
% Also \xoption{perpage} of package \xpackage{footmisc} \cite{footmisc}
% can easily be simulated by
% \begin{quote}
% |\zmakeperpage{footnote}| \itshape |% \usepackage[perpage]{footmisc}|
%\end{quote}
% If footnote symbols are used, some people dislike the
% first symbol \ensuremath{\dagger}. It can easily be skipped:
% \begin{quote}
% |\zmakeperpage[2]{footnote}|
% \end{quote}
%
% \begin{declcs}{thezpage}\\
%   counter |zpage|
% \end{declcs}
% If the formatted counter value of the counter that is reset at
% a new page contains the page value, then you can use \cs{thezpage},
% the page number of the current page. Or counter |zpage| can be
% used, if the page number should be formatted differently from
% the current page number. Example:
%\begin{quote}
%\begin{verbatim}
%\newcounter{foobar}
%\zmakeperpage{foobar}
%\renewcommand*{\thefoobar}{\thezpage-\arabic{foobar}}
% % or
%\renewcommand*{\thefoobar}{\roman{zpage}-\arabic{foobar}}
%\end{verbatim}
%\end{quote}
%
% \begin{declcs}{zunmakeperpage}
%   \M{counter}
% \end{declcs}
% The reset mechanism for this counter is deactivated.
%
%
% \subsection{Module \xmodule{counter}}
%
% This option just add the property |counter| to the main
% property list. The property stores the counter name,
% that was responsible for the reference. This is the
% property \xpackage{hyperref}'s \cs{autoref} feature
% uses. Thus this property |counter| may be useful
% for a reimplementation of the autoref feature,
% see the section \ref{sec:todo} with the todo list.
%
% \subsection{Module \xmodule{titleref}}
%
% This option makes section and caption titles available
% to the reference system similar to packages \xpackage{titleref}
% or \xpackage{nameref}.
%
% \begin{declcs}{ztitleref}
%   \M{refname}\markbabel
% \end{declcs}
% Print the section or caption title of reference \meta{refname},
% similar to \cs{nameref} or \cs{titleref}.
%
% \begin{declcs}{ztitlerefsetup}
%   |{|\textit{key$_1$=value$_1$, key$_2$=value$_2$, \dots}|}|
% \end{declcs}
% This command allows to configure the behaviour of module
% \xmodule{titleref}. The following keys are available:
% \begin{description}
% \let\bfseries\relax
% \item[\texttt{title=}\meta{value}]\mbox{}\\*
%   Sets the current title.
% \item[\texttt{stripperiod=true\string|false}]\mbox{}\\*
%   Follow package \xpackage{nameref} that
%   removes a last period. Default: |true|.
% \item[\texttt{expand=true\string|false}]\mbox{}\\*
%   Package \cs{titleref} expands the title first. This way
%   garbage and dangerous commands can be removed, e.g.
%   \cs{label}, \cs{index}\dots. See implementation section
%   for more details. Default is |false|.
% \item[\texttt{cleanup=\{}\textrm{\dots}\texttt{\}}]\mbox{}\\*
%   Hook to add own cleanup code, if method |expand| is used.
%   See implementation section for more details.
% \end{description}
%
% \subsection{Module \xmodule{savepos}}
%
% This option supports a feature that \pdfTeX\ provides
% (and \hologo{XeTeX}).
% \pdfTeX\ is able to tell the current position on the
% page. The page position is not instantly known. First the
% page must be constructed by \hologo{TeX}'s asynchronous output
% routine. Thus the time where the position is known is
% the page shipout time. Thus a reference system where the
% information is recorded in the first run and made available
% for use in the second run comes in handy.
%
% \begin{declcs}{zsavepos}
%   \M{refname}
% \end{declcs}
% It generates a reference with name \meta{refname}.
% The reference stores the location where \cs{zsavepos}
% is executed in properties |posx| and |posy|.
%
% \begin{declcs}{zsaveposx}
%   \M{refname}\\
%   \cs{zsaveposy}
%   \M{refname}
% \end{declcs}
% Same as \cs{zsavepos} except that only the |x| or |y| component
% of the position is stored. Since 2011/12/05 v2.23.
%
% \begin{declcs}{zposx}\markexp
%   \M{refname}\\
%   \SpecialUsageIndex{\zposy}\cs{zposy}\markexp \M{refname}
% \end{declcs}
% Get the position as number. Unit is sp. Horizontal
% positions by \cs{zposx} increase from left to right.
% Vertical positions by \cs{zposy} from bottom to top.
%
% Do not rely on absolute page numbers. Because of problems
% with the origin the numbers may differ in DVI or PDF mode
% of \pdfTeX. Therefore work with relative values by comparisons.
%
% Both \cs{zposx} and \cs{zposy} are expandable and can be
% used inside calculations (\cs{setcounter}, \cs{addtocounter},
% package \xpackage{calc}, \cs{numexpr}). However this property
% prevents from notifying \hologo{LaTeX} that the reference is actually
% used (the notifying is not expandable). Therefore you should
% mark the reference as used by \cs{zrefused}.
%
% This module uses \pdfTeX's \cs{pdfsavepos}, \cs{pdflastxpos},
% and \cs{pdflastypos}. They are available in PDF mode and
% since version 1.40.0 also in DVI mode.
%
% \begin{declcs}{zref@savepos}
% \end{declcs}
% Macro \cs{zref@savepos} performs the first part of \cs{zsavepos}
% by calling \cs{pdfsavepos} (if \xext{aux} files are writable).
%
% Thus \cs{zsavepos} is basically \cs{zref@savepos} followed
% by |\zref@labelbylist{|\texttt{\meta{refname}}|}{savepos}|.
% If \cs{TeXXeTstate} is detected and enabled, \cs{savepos}
% also adds \cs{zref@savepos} at the end to support \cs{beginR}
% where the whatits are processed in reverse order.
% The property list |savepos| contains the properties |posx|
% and |posy|.
%
% \subsection{Module \xmodule{abspos}}
% Module \xmodule{abspos} allows to get various values of the page layout. There is no user
% command, only a number of internal commands. For example:
% \begin{quote}
%   |\zref@absposx{|\meta{label}|}{|\meta{value}|}{|\meta{position}|}|\\
%   |\zref@absposy{|\meta{label}|}{|\meta{value}|}{|\meta{position}|}|
% \end{quote}
%
% The return value is like in the module \xmodule{savepos} a number representing a length in sp.
% The length are measured from the bottom left of the page.
%
% \meta{label} is a label set with \cs{zlabel} or \cs{zsavepos} that allows to retrieve the
% absolute page number.
%
% \meta{position} is for the x-command one of |left|, |right| or |center|. For the y-command
%  it is one of |top|, |bottom|, |center|.
%
% The possible content of \meta{value} can be seen in the following table. Be aware that the code
% makes some assumptions which are perhaps not always true -- for example that the left of
% the head is identical to the left of the body.
%
% \medskip
% \begin{tabular}{lll}
%  \textbf{value} & \textbf{axis} & \textbf{comments}\\\hline
%   media         & x             &   left=0, right=\cs{pdfpagewidth} \\
%   paper         & x             &   left=0, right=\cs{paperwidth}\\
%   stock         & x             &   derived from paper\\
%   media         & y             & bottom=0, top=\cs{pdfpageheigh}\\
%   paper         & y             & top=\cs{pdfpageheight}, bottom=top-\cs{paperheight}\\
%   stock         & y             & top  derived from paper\\
%   head          & x             & calculated with hoffset, horigin, etc\\
%   head          & y             & calculated\\
%   body          & x             & = head value\\%
%   body          & y             & = head bottom - \cs{headsep}\\
%   foot          & x             & = head\\
%   foot          & y             & calculated from body bottom and \cs{footskip}\\
%   marginpar     & x             &different on odd/even pages! \\
%   marginpar     & y             & = body %
% \end{tabular}
%
% \subsection{Module \xmodule{dotfill}}
%
% \begin{declcs}{zdotfill}
% \end{declcs}
% This package provides the command \cs{zdotfill} that works
% similar to \cs{dotfill}, but can be configured. Especially
% it suppresses the dots if a minimum number of dots cannot be set.
%
% \begin{declcs}{zdotfillsetup}
%   |{|\textit{key$_1$=value$_1$, key$_2$=value$_2$, \dots}|}|
% \end{declcs}
% This command allows to configure the behaviour of \cs{zdotfill}.
% The following keys are available:
% \begin{description}
% \let\bfseries\relax
% \item[\texttt{min=}\meta{count value}]\mbox{}\\*
%   If the actual number of dots are smaller than \meta{count value},
%   then the dots are suppressed. Default: |2|.
% \item[\texttt{unit=}\meta{dimen value}]\mbox{}\\*
%   The width of a dot unit is given by \meta{dimen value}.
%   Default: |0.44em| (same as the unit in \cs{dotfill}).
% \item[\texttt{dot=}\meta{value}]\mbox{}\\*
%   The dot itself is given by \meta{value}. Default: |.| (dot, same as
%   the dot in \cs{dotfill}).
% \end{description}
%
% \subsection{Module \xmodule{env}}
%
% This module defines two properties |envname| and |envline|. They
% remember the name of the environment and the line number at
% the start of the environment.
%
% \subsection{Module \xmodule{xr}}
%
% This package provides the functionality of package \xpackage{xr},
% see \cite{xr}. It also supports the syntax of \xpackage{xr-hyper}.
%
% \begin{declcs}{zexternaldocument}
%   \* |[|\meta{prefix}|]|\markbabel \M{external document}
%   |[|\meta{url}|]|
% \end{declcs}
% See \cs{externaldocument} for a description of this option.
% The found labels also get a property |externaldocument| that
% remembers \meta{external document}.
% The standard reference scheme and the scheme of this
% package use different name spaces for reference names.
% If the external document uses both systems. Then one
% import statement would put the names in one namespace
% and probably causing problems with multiple references
% of the same name. Thus the star form only looks for
% \cs{newlabel} in the \xfile{.aux} files, whereas without
% star only \cs{zref@newlabel}s are used.
%
% In the star form it tries to detect labels from
% \xpackage{hyperref}, \xpackage{titleref}, and \xpackage{ntheorem}.
% If such an extended property from the packages before cannot
% be found or are empty, they are not included in the
% imported reference.
%
% Warnings are given if a reference name is already in use
% and the item is ignored. Unknown properties will automatically
% be declared.
%
% If the external references contain |anchor| properties,
% then we need also a url to be able to address the external
% file. As default the filename is taken with a default
% extension.
%
% \begin{declcs}{zxrsetup}
%   |{|\textit{key$_1$=value$_1$, key$_2$=value$_2$, \dots}|}|
% \end{declcs}
% The following setup options are available:
% \begin{description}
% \item[\xoption{ext}:] It sets the default extension.
% \item[\xoption{tozreflabel}:] Boolean option. The found references
%   are imported as zref labels. This is enabled by default.
% \item[\xoption{toltxlabel}:] Boolean option. The found references
%   are imported as \hologo{LaTeX} labels. Packages \xpackage{nameref},
%   \xpackage{titleref} and class \xclass{memoir} are supported.
% \item[\xoption{urluse}:] Boolean option. If enabled, then a URL
%   is stored in a macro and the macro is put in property `urluse'.
%   The URL is not put in property `url'.
%   The purpose is to save \hologo{TeX} memory.
% \item[\xoption{verbose}:] Boolean option. List the imported labels in the
%   \xext{log} file. Default is |false|.
% \end{description}
%
% \begin{declcs}{zref@xr@ext}
% \end{declcs}
% If the \meta{url} is not specified in \cs{zref@externaldocument},
% then the url will be constructed with the file name and
% this macro as extension. \cs{XR@ext} is used if
% \xpackage{hyperref} is loaded, otherwise |pdf|.
%
% \subsection{Module \xmodule{pageattr}}
%
% This module allows to recover the content of the register
% \cs{pdfpageattr} and \cs{pdfpagesattr} in pdftex and the equivalent register
% in luatex. There is no user command. Programmer commands are
%
% \begin{quote}
% |\zref@pdfpageattr{|\meta{absolute page number}|}|
%
% |\zref@pdfpagesattr{|\meta{absolute page number}|}|
% \end{quote}
% \section{ToDo}\label{sec:todo}
%
% Among other things the following issues are left for future work:
% \begin{itemize}
% \item Other applications: autoref, hyperref, \dots
% \end{itemize}
%
% \section{Example}
%
%    \begin{macrocode}
%<*example>
%    \end{macrocode}
%
%    \begin{macrocode}
\documentclass{book}

\usepackage[ngerman]{babel}%

\usepackage[savepos,totpages,titleref,dotfill,counter,user]{zref}

%    \end{macrocode}
%
%    Chapters are wrapped inside \cs{ChapterStart} and
%    \cs{ChapterStop}. The first argument \verb|#1| of \cs{ChapterStart}
%    is used to form a label id \verb|chap:#1|. At the end of
%    the chapter another label is set by \cs{zref@wrapper@immediate}, because
%    otherwise at the end of document a deferred write would not
%    be written, because there is no page for shipout.
%
%    Also this example shows how chapter titles can be recorded.
%    A new property \verb|chaptitle| is declared and added to the
%    main property list. In \cs{ChapterStart} the current value
%    of the property is updated.
%    \begin{macrocode}
\makeatletter
\zref@newprop{chaptitle}{}
\zref@addprop{main}{chaptitle}

\newcommand*{\ChapterStart}[2]{%
  \cleardoublepage
  \def\current@chapid{#1}%
  \zref@setcurrent{chaptitle}{#2}%
  \chapter{#2}%
  \zlabel{chap:#1}%
}
\newcommand*{\ChapterStop}{%
  \cleardoublepage
  \zref@wrapper@immediate{%
    \zref@labelbyprops{chapend:\current@chapid}{abspage}%
  }%
}
%    \end{macrocode}
%    \cs{ChapterPages} calculates and returns the number of pages of the
%    referenced chapter.
%    \begin{macrocode}
\newcommand*{\ChapterPages}[1]{%
  \zrefused{chap:#1}%
  \zrefused{chapend:#1}%
  \number\numexpr
    \zref@extract{chapend:#1}{abspage}%
    -\zref@extract{chap:#1}{abspage}%
    +1\relax
}
\makeatother
%    \end{macrocode}
%    \begin{macrocode}
\begin{document}
%    \end{macrocode}
%    As exception we use \cs{makeatletter} here, because
%    this is just an example file that also should show some of
%    programmer's interface.
%    \begin{macrocode}
\makeatletter

\frontmatter
\zlabel{documentstart}

\begin{itemize}
\item
  The frontmatter part has
  \number\numexpr\zref@extract{chap:first}{abspage}-1\relax
  ~pages.
\item
  Chapter \zref{chap:first} has \ChapterPages{first} page(s).
\item
  Section \zref{hello} is on the
  \ifcase\numexpr
    \zref@extractdefault{hello}{page}{0}%
    -\zref@extractdefault{chap:first}{page}{0}%
    +1\relax
    ??\or first\or second\or third\or forth\fi
  ~page inside its chapter.
\item
  The document has
  \zref[abspage]{LastPage} pages.
  This number is \ifodd\ztotpages odd\else even\fi.
\item
  The last page is labeled with \zpageref{LastPage}.
\item
  The title of chapter \zref{chap:next} %
  is ``\zref[chaptitle]{chap:next}''.
\end{itemize}

\tableofcontents

\mainmatter
\ChapterStart{first}{First chapter}

%    \end{macrocode}
%    The user level commands should protect babel shorthands
%    where possible. On the other side, expandable extracting
%    macros are useful in calculations, see above the
%    examples with \cs{numexpr}.
%    \begin{macrocode}
\section{Test}
\zlabel{a"o}
Section \zref{a"o} on page
\zref@wrapper@babel\zref@extract{a"o}{page}.

Text.
\newpage

\section{Hello World}
\zlabel{hello}

\ChapterStop

\ChapterStart{next}{Next chapter with \emph{umlauts}: "a"o"u"s}

%    \end{macrocode}
%
%    Here an example follows that makes use of \pdfTeX's ``savepos''
%    feature. The position on the page is not known before the
%    page is constructed and shipped out. Therefore the position
%    ist stored in references and are available for calculations
%    in the next \hologo{LaTeX} compile run.
%    \begin{macrocode}
The width of the first column is
  \the\dimexpr \zposx{secondcol}sp - \zposx{firstcol}sp\relax,\\
the height difference of the two baselines is
  \the\dimexpr \zposy{firstcol}sp - \zposy{secondline}sp\relax:\\
\begin{tabular}{ll}
  \zsavepos{firstcol}Hello&\zsavepos{secondcol}World\\
  \zsavepos{secondline}Second line&foobar\\
\end{tabular}

%    \end{macrocode}
%    With \cs{zrefused} \hologo{LaTeX} is notified, if the references
%    are not yet available and \hologo{LaTeX} can generate the rerun
%    hint.
%    \begin{macrocode}
\zrefused{firstcol}
\zrefused{secondcol}
\zrefused{secondline}

\ChapterStop
%    \end{macrocode}
%    Test for module \cs{dotfill}.
%    \begin{macrocode}
\ChapterStart{dotfill}{Test for dotfill feature}
\newcommand*{\dftest}[1]{%
  #1&
  [\makebox[{#1}]{\dotfill}]&
  [\makebox[{#1}]{\zdotfill}]\\
}
\begin{tabular}{rll}
& [\verb|\dotfill|] & [\verb|\zdotfill|]\\
\dftest{0.43em}
\dftest{0.44em}
\dftest{0.45em}
\dftest{0.87em}
\dftest{0.88em}
\dftest{0.89em}
\dftest{1.31em}
\dftest{1.32em}
\dftest{1.33em}
\end{tabular}
\ChapterStop
\end{document}
%    \end{macrocode}
%
%    \begin{macrocode}
%</example>
%    \end{macrocode}
%
% \StopEventually{
% }
%
% \section{Implementation}
%
% \subsection{Package \xpackage{zref}}
%
% \subsubsection{Identification}
%
%    \begin{macrocode}
%<*package>
\NeedsTeXFormat{LaTeX2e}
\ProvidesPackage{zref}
  [2020-07-03 v2.32 A new reference scheme for LaTeX (HO)]%
%    \end{macrocode}
%
% \subsubsection{Load basic module}
%
%    \begin{macrocode}
\RequirePackage{zref-base}[2019/11/29]
%    \end{macrocode}
%    Abort package loading if \xpackage{zref-base} could not be
%    loaded successfully.
%    \begin{macrocode}
\@ifundefined{ZREF@base@ok}{\endinput}{}
%    \end{macrocode}
%
% \subsubsection{Process options}
%
%    Known modules are loaded and the release date is checked.
%    \begin{macrocode}
\def\ZREF@temp#1{%
  \DeclareOption{#1}{%
    \AtEndOfPackage{%
      \RequirePackage{zref-#1}[2019/11/29]%
    }%
  }%
}
\ZREF@temp{abspage}
\ZREF@temp{counter}
\ZREF@temp{dotfill}
\ZREF@temp{hyperref}
\ZREF@temp{lastpage}
\ZREF@temp{marks}
\ZREF@temp{nextpage}
\ZREF@temp{pageattr}
\ZREF@temp{pagelayout}
\ZREF@temp{perpage}
\ZREF@temp{runs}
\ZREF@temp{savepos}
\ZREF@temp{thepage}
\ZREF@temp{titleref}
\ZREF@temp{totpages}
\ZREF@temp{user}
\ZREF@temp{xr}
%    \end{macrocode}
%    \begin{macrocode}
\ProcessOptions\relax
%</package>
%    \end{macrocode}
%
%
% \subsection{Module \xmodule{base}}
%
% \subsubsection{Prefixes}
%
%    This package uses the following prefixes for macro names:
%    \begin{description}
%    \item[\cs{zref@}:] Macros of the programmer's interface.
%    \item[\cs{ZREF@}:] Internal macros.
%    \item[\cs{Z@L@}\metacs{listname}:]
%      The properties of the list \meta{listname}.
%    \item[\cs{Z@D@}\metacs{propname}:]
%      The default value for property \meta{propname}.
%    \item[\cs{Z@E@}\metacs{propname}:]
%      Extract function for property \meta{propname}.
%    \item[\cs{Z@X@}\metacs{propname}:]
%      Information whether a property value for property
%      \meta{propname} is expanded
%      immediately or at shipout time.
%    \item[\cs{Z@C@}\metacs{propname}:]
%      Current value of the property \meta{propname}.
%    \item[\cs{Z@R@}\metacs{labelname}:]
%      Data for reference \meta{labelname}.
%    \item[\cs{ZREF@org@}:] Original versions of patched commands.
%    \item[\cs{z}:] For macros in user land, defined if
%      module \xmodule{user} is set.
%    \end{description}
%    The following family names are used for keys defined
%    according to the \xpackage{keyval} package:
%    \begin{description}
%    \item[\texttt{ZREF@TR}:] Setup for module \xmodule{titleref}.
%    \end{description}
%
% \subsubsection{Identification}
%
%    \begin{macrocode}
%<*base>
\NeedsTeXFormat{LaTeX2e}
\ProvidesPackage{zref-base}%
  [2020-07-03 v2.32 Module base for zref (HO)]%
%    \end{macrocode}
%
% \subsubsection{Utilities}
%
%    \begin{macrocode}
\providecommand\IfFormatAtLeastTF{\@ifl@t@r\fmtversion}
\RequirePackage{ltxcmds}[2010/12/02]
\RequirePackage{infwarerr}[2010/04/08]
\RequirePackage{kvsetkeys}[2010/03/01]
\RequirePackage{kvdefinekeys}[2010/03/01]
\RequirePackage{pdftexcmds}[2010/04/01]
%    \end{macrocode}
%
%    \begin{macro}{\ZREF@name}
%    Several times the package name is used, thus we store it
%    in \cmd{\ZREF@name}.
%    \begin{macrocode}
\def\ZREF@name{zref}
%    \end{macrocode}
%    \end{macro}
%
%    \begin{macrocode}
\ltx@IfUndefined{protected}{%
  \RequirePackage{makerobust}[2006/03/18]%
%    \end{macrocode}
%    \begin{macro}{\ZREF@Robust}
%    \begin{macrocode}
  \def\ZREF@Robust#1#2{%
    \def\ZREF@temp{\MakeRobustcommand#2}%
    \afterassignment\ZREF@temp
    #1#2%
  }%
%    \end{macrocode}
%    \end{macro}
%    \begin{macrocode}
}{%
%    \end{macrocode}
%    \begin{macro}{\ZREF@Robust}
%    \begin{macrocode}
  \def\ZREF@Robust#1{%
    \protected#1%
  }%
%    \end{macrocode}
%    \end{macro}
%    \begin{macrocode}
}
%    \end{macrocode}
%    \begin{macro}{\ZREF@IfDefinable}
%    \begin{macrocode}
\def\ZREF@IfDefinable#1#2#3{%
  \@ifdefinable{#1}{%
    \ZREF@Robust{#2}#1#3%
  }%
}
%    \end{macrocode}
%    \end{macro}
%
%    \begin{macro}{\ZREF@UpdatePdfTeX}
%    \cs{ZREF@UpdatePdfTeX} is used as help message text in
%    error messages.
%    \begin{macrocode}
\def\ZREF@UpdatePdfTeX{Update pdfTeX.}
%    \end{macrocode}
%    \end{macro}
%
%    \begin{macro}{\ifZREF@found}
%    The following switch is usded in list processing.
%    \begin{macrocode}
\newif\ifZREF@found
%    \end{macrocode}
%    \end{macro}
%
%    \begin{macro}{\ZREF@patch}
%    Macro \cs{ZREF@patch} first checks the existence of
%    the command and safes it.
%    \begin{macrocode}
\def\ZREF@patch#1{%
  \ltx@IfUndefined{#1}{%
    \ltx@gobble
  }{%
    \expandafter\let\csname ZREF@org@#1\expandafter\endcsname
    \csname #1\endcsname
    \ltx@firstofone
  }%
}
%    \end{macrocode}
%    \end{macro}
%
% \subsubsection{Check for \hologo{eTeX}}
%
%    The use of \hologo{eTeX} should be standard nowadays for \hologo{LaTeX}.
%    We test for \hologo{eTeX} in order to use its features later.
%    \begin{macrocode}
\ltx@IfUndefined{eTeXversion}{%
  \PackageError\ZREF@name{%
    Missing support for eTeX; package is abandoned%
  }{%
    Use a TeX compiler that support eTeX and enable eTeX %
    in the format.%
  }%
  \endinput
}{}%
%    \end{macrocode}
%
%    \begin{macrocode}
\RequirePackage{etexcmds}[2007/09/09]
\ifetex@unexpanded
\else
  \PackageError\ZREF@name{%
    Missing e-TeX's \string\unexpanded.\MessageBreak
    Add \string\RequirePackage\string{etexcmds\string} before %
    \string\documentclass%
  }{%
    Probably you are using some package (e.g. ConTeXt) that %
    redefines \string\unexpanded%
  }%
  \expandafter\endinput
\fi
%    \end{macrocode}
%
% \subsubsection{Auxiliary file stuff}
%
%    We are using some commands in the \xfile{.aux} files.
%    However sometimes these auxiliary files are interpreted
%    by \hologo{LaTeX} processes that haven't loaded this package
%    (e.g. package \xpackage{xr}).
%    Therefore we provide dummy definitions.
%    \begin{macrocode}
\RequirePackage{auxhook}
\AddLineBeginAux{%
  \string\providecommand\string\zref@newlabel[2]{}%
}
%    \end{macrocode}
%    \begin{macro}{\ZREF@RefPrefix}
%    \begin{macrocode}
\def\ZREF@RefPrefix{Z@R}
%    \end{macrocode}
%    \end{macro}
%    \begin{macro}{\zref@newlabel}
%    For the implementation of \cs{zref@newlabel} we call
%    the same internal macro \cs{@newl@bel} that is used
%    in \cs{newlabel}. Thus we have for free:
%    \begin{itemize}
%    \item \cs{Z@R@}\metacs{labelname} is defined.
%    \item \hologo{LaTeX}'s check for multiple references.
%    \item \hologo{LaTeX}'s check for changed references.
%    \end{itemize}
%    \begin{macrocode}
\ZREF@Robust\edef\zref@newlabel{%
  \noexpand\@newl@bel{\ZREF@RefPrefix}%
}
%    \end{macrocode}
%    \end{macro}
%
% \subsubsection{Property lists}
%
%    \begin{macro}{\zref@newlist}
%    Property lists are stored as list of property names
%    enclosed in curly braces. \cs{zref@newlist}
%    creates a new list as empty list. Assignments to
%    property lists are global.
%    \begin{macrocode}
\ZREF@Robust\def\zref@newlist#1{%
  \zref@iflistundefined{#1}{%
    \@ifdefinable{Z@L@#1}{%
     \global\expandafter\let\csname Z@L@#1\endcsname\ltx@empty
     \PackageInfo\ZREF@name{New property list: #1}%
    }%
  }{%
    \PackageError\ZREF@name{%
      Property list `#1' already exists%
    }\@ehc
  }%
}
%    \end{macrocode}
%    \end{macro}
%
%    \begin{macro}{\zref@iflistundefined}
%    \cs{zref@iflistundefined} checks the existence of the
%    property list \verb|#1|. If the property list is present,
%    then \verb|#2| is executed and \verb|#3| otherwise.
%    \begin{macrocode}
\def\zref@iflistundefined#1{%
  \ltx@ifundefined{Z@L@#1}%
}
%    \end{macrocode}
%    \end{macro}
%    \begin{macro}{\zref@listexists}
%    \cs{zref@listexists} only executes \verb|#2| if the
%    property list \verb|#1| exists and raises an error
%    message otherwise.
%    \begin{macrocode}
\ZREF@Robust\def\zref@listexists#1{%
  \zref@iflistundefined{#1}{%
    \PackageError\ZREF@name{%
      Property list `#1' does not exist%
    }\@ehc
  }%
}
%    \end{macrocode}
%    \end{macro}
%
%    \begin{macro}{\zref@iflistcontainsprop}
%    \cs{zref@iflistcontainsprop} checks, whether a property \verb|#2| is
%    already present in a property list \verb|#1|.
%    \begin{macrocode}
\ZREF@Robust\def\zref@iflistcontainsprop#1#2{%
  \zref@iflistundefined{#1}{%
    \ltx@secondoftwo
  }{%
    \begingroup\expandafter\endgroup
    \expandafter\in@
    \csname#2\expandafter\expandafter\expandafter\endcsname
    \expandafter\expandafter\expandafter{\csname Z@L@#1\endcsname}%
    \csname ltx@\ifin@ first\else second\fi oftwo\endcsname
  }%
}
%    \end{macrocode}
%    \end{macro}
%
%    \begin{macro}{\zref@listforloop}
%    \begin{macrocode}
\def\zref@listforloop#1#2{%
  \zref@listexists{#1}{%
    \expandafter\expandafter\expandafter\@tfor
    \expandafter\expandafter\expandafter\zref@prop
    \expandafter\expandafter\expandafter:%
    \expandafter\expandafter\expandafter=%
    \csname Z@L@#1\endcsname
    \do{%
      \begingroup
        \escapechar=-1 %
        \edef\x{\endgroup
          \def\noexpand\zref@prop{%
            \expandafter\string\zref@prop
          }%
        }%
      \x
      #2\zref@prop
    }%
  }%
}
%    \end{macrocode}
%    \end{macro}
%
%    \begin{macro}{\zref@addprops}
%    \cs{zref@addprop} adds the properties \verb|#2| to the property
%    list \verb|#1|, if the property is not already in the list.
%    Otherwise a warning is given.
%    \begin{macrocode}
\ZREF@Robust\def\zref@addprops#1#2{%
  \zref@listexists{#1}{%
    \comma@parse{#2}{%
      \zref@propexists\comma@entry{%
        \zref@iflistcontainsprop{#1}\comma@entry{%
          \PackageWarning\ZREF@name{%
            Property `\comma@entry' is already in list `#1'%
          }%
        }{%
          \begingroup\expandafter\endgroup
          \expandafter\g@addto@macro
          \csname Z@L@#1\expandafter\endcsname
          \expandafter{\csname\comma@entry\endcsname}%
        }%
      }%
      \ltx@gobble
    }%
  }%
}
%    \end{macrocode}
%    \end{macro}
%    \begin{macro}{\zref@addprop}
%    \cs{zref@addprop} adds the property \verb|#2| to the property
%    list \verb|#1|, if the property is not already in the list.
%    Otherwise a warning is given.
%    \begin{macrocode}
\ZREF@Robust\def\zref@addprop#1#2{%
  \zref@listexists{#1}{%
    \zref@propexists{#2}{%
      \zref@iflistcontainsprop{#1}{#2}{%
        \PackageWarning\ZREF@name{%
          Property `#2' is already in list `#1'%
        }%
      }{%
        \begingroup\expandafter\endgroup
        \expandafter\g@addto@macro
        \csname Z@L@#1\expandafter\endcsname
        \expandafter{\csname#2\endcsname}%
      }%
    }%
  }%
}
%    \end{macrocode}
%    \end{macro}
%
%    \begin{macro}{\zref@localaddprops}
%    \begin{macrocode}
\ZREF@Robust\def\zref@localaddprops#1#2{%
  \zref@listexists{#1}{%
    \comma@parse{#2}{%
      \zref@propexists\comma@entry{%
        \zref@iflistcontainsprop{#1}\comma@entry{%
          \PackageWarning\ZREF@name{%
            Property `\comma@entry' is already in list `#1'%
          }%
        }{%
          \begingroup\expandafter\endgroup
          \expandafter\ltx@LocalAppendToMacro
          \csname Z@L@#1\expandafter\endcsname
          \expandafter{\csname\comma@entry\endcsname}%
        }%
      }%
      \ltx@gobble
    }%
  }%
}
%    \end{macrocode}
%    \end{macro}
%    \begin{macro}{\zref@localaddprop}
%    \begin{macrocode}
\ZREF@Robust\def\zref@localaddprop#1#2{%
  \zref@listexists{#1}{%
    \zref@propexists{#2}{%
      \zref@iflistcontainsprop{#1}{#2}{%
        \PackageWarning\ZREF@name{%
          Property `#2' is already in list `#1'%
        }%
      }{%
        \begingroup\expandafter\endgroup
        \expandafter\ltx@LocalAppendToMacro
        \csname Z@L@#1\expandafter\endcsname
        \expandafter{\csname#2\endcsname}%
      }%
    }%
  }%
}
%    \end{macrocode}
%    \end{macro}
%
%    \begin{macrocode}
\ltx@IfUndefined{pdf@strcmp}{%
%    \end{macrocode}
%    \begin{macro}{\zref@delprop}
%    \begin{macrocode}
  \ZREF@Robust\def\zref@delprop{%
    \ZREF@delprop\gdef
  }%
%    \end{macrocode}
%    \end{macro}
%    \begin{macro}{\zref@localdelprop}
%    \begin{macrocode}
  \ZREF@Robust\def\zref@localdelprop{%
    \ZREF@delprop\def
  }%
%    \end{macrocode}
%    \end{macro}
%    \begin{macro}{\ZREF@delprop}
%    \begin{macrocode}
  \def\ZREF@delprop#1#2#3{%
    \zref@listexists{#2}{%
      \begingroup
        \escapechar=-1 %
        \def\ZREF@param{#3}%
        \@onelevel@sanitize\ZREF@param
        \toks@{}%
        \expandafter\expandafter\expandafter\ZREF@@delprop
        \csname Z@L@#2\endcsname!%
      \expandafter\endgroup
      \expandafter#1\csname Z@L@#2\expandafter\endcsname
      \expandafter{%
        \the\toks@
      }%
    }%
  }%
%    \end{macrocode}
%    \end{macro}
%    \begin{macro}{\ZREF@@delprop}
%    \begin{macrocode}
  \def\ZREF@@delprop#1{%
    \expandafter\ZREF@@@delprop\expandafter{\string#1}#1%
  }%
%    \end{macrocode}
%    \end{macro}
%    \begin{macro}{\ZREF@@@delprop}
%    \begin{macrocode}
  \def\ZREF@@@delprop#1#2{%
    \ifx#2!%
    \else
      \def\ZREF@temp{#1}%
      \@onelevel@sanitize\ZREF@temp
      \ifx\ZREF@param\ZREF@temp
      \else
        \toks@\expandafter{%
          \the\expandafter\toks@\csname#1\endcsname
        }%
      \fi
      \expandafter\ZREF@@delprop
    \fi
  }%
%    \end{macrocode}
%    \end{macro}
%    \begin{macrocode}
}{%
%    \end{macrocode}
%    \begin{macro}{\zref@delprop}
%    \begin{macrocode}
  \ZREF@Robust\def\zref@delprop{%
    \ZREF@delprop\xdef
  }%
%    \end{macrocode}
%    \end{macro}
%    \begin{macro}{\zref@localdelprop}
%    \begin{macrocode}
  \ZREF@Robust\def\zref@localdelprop{%
    \ZREF@delprop\edef
  }%
%    \end{macrocode}
%    \end{macro}
%    \begin{macro}{\ZREF@delprop}
%    \begin{macrocode}
  \def\ZREF@delprop#1#2#3{%
    \zref@listexists{#2}{%
      \def\ZREF@param{#3}%
      \edef\ZREF@SavedEscapechar{\the\escapechar}%
      \escapechar=-1 %
      \expandafter#1\csname Z@L@#2%
      \expandafter\expandafter\expandafter\endcsname{%
        \expandafter\expandafter\expandafter\ZREF@@delprop
        \csname Z@L@#2\endcsname!%
      }%
      \escapechar=\ZREF@SavedEscapechar\relax
    }%
  }%
%    \end{macrocode}
%    \end{macro}
%    \begin{macro}{\ZREF@@delprop}
%    Caution: |#1| might be an \cs{if} or similar token.
%    \begin{macrocode}
  \def\ZREF@@delprop#1{%
    \expandafter\ZREF@@@delprop\expandafter{\string#1}#1%
  }%
%    \end{macrocode}
%    \end{macro}
%    \begin{macro}{\ZREF@@@delprop}
%    \begin{macrocode}
  \def\ZREF@@@delprop#1#2{%
    \ifx#2!%
    \else
      \ifnum\pdf@strcmp{#1}{\ZREF@param}=\ltx@zero
      \else
        \expandafter\noexpand\csname#1\endcsname
      \fi
      \expandafter\ZREF@@delprop
    \fi
  }%
%    \end{macrocode}
%    \end{macro}
%    \begin{macrocode}
}
%    \end{macrocode}
%
% \subsubsection{Properties}
%
%    \begin{macro}{\zref@ifpropundefined}
%    \cs{zref@ifpropundefined} checks the existence of the
%    property \verb|#1|. If the property is present,
%    then \verb|#2| is executed and \verb|#3| otherwise.
%    \begin{macrocode}
\def\zref@ifpropundefined#1{%
  \ltx@ifundefined{Z@E@#1}%
}
%    \end{macrocode}
%    \end{macro}
%    \begin{macro}{\zref@propexists}
%    Some macros rely on the existence of a property.
%    \cs{zref@propexists} only executes \verb|#2| if the
%    property \verb|#1| exists and raises an error
%    message otherwise.
%    \begin{macrocode}
\ZREF@Robust\def\zref@propexists#1{%
  \zref@ifpropundefined{#1}{%
    \PackageError\ZREF@name{%
      Property `#1' does not exist%
    }\@ehc
  }%
}
%    \end{macrocode}
%    \end{macro}
%
%    \begin{macro}{\zref@newprop}
%    A new property is declared by \cs{zref@newprop}, the
%    property name \meta{propname} is given in \verb|#1|. The
%    property is created and configured. If the star form
%    is given, then the expansion of the property value
%    is delayed to page shipout time, when the reference
%    is written to the \xfile{.aux} file.
%    \begin{description}
%    \item[\cs{Z@D@}\metacs{propname}:]
%      Stores the default value for this property.
%    \item[\cs{Z@E@}\metacs{propname}:]
%      Extract function.
%    \item[\cs{Z@X@}\metacs{propname}:]
%      Information whether the expansion of the property value
%      is delayed to shipout time.
%    \item[\cs{Z@C@}\metacs{propname}:]
%      Current value of the property.
%    \end{description}
%    \begin{macrocode}
\ZREF@Robust\def\zref@newprop{%
  \@ifstar{%
    \let\ZREF@X\noexpand
    \ZREF@newprop
  }{%
    \let\ZREF@X\ltx@empty
    \ZREF@newprop
  }%
}
%    \end{macrocode}
%    \end{macro}
%    \begin{macro}{\ZREF@newprop}
%    \begin{macrocode}
\def\ZREF@newprop#1{%
  \edef\ZREF@P{#1}%
  \@onelevel@sanitize\ZREF@P
  \begingroup
  \ifx\ZREF@P\ZREF@par
    \@PackageError\ZREF@name{%
      Invalid property name `\ZREF@P'%
    }{%
      The property name `par' is not allowed %
      because of internal reasons.%
      \MessageBreak
      \@ehc
    }%
    \def\ZREF@@newprop[##1]##2{\endgroup}%
  \else
    \zref@ifpropundefined\ZREF@P{%
      \endgroup
      \PackageInfo\ZREF@name{%
        New property: \ZREF@P
      }%
    }{%
      \@PackageError\ZREF@name{%
        Property `\ZREF@P' already exists%
      }\@ehc
      \def\ZREF@@newprop[##1]##2{\endgroup}%
    }%
  \fi
  \@ifnextchar[\ZREF@@newprop{\ZREF@@newprop[\zref@default]}%
}
%    \end{macrocode}
%    \end{macro}
%    \begin{macro}{\ZREF@par}
%    \begin{macrocode}
\def\ZREF@par{par}
\@onelevel@sanitize\ZREF@par
%    \end{macrocode}
%    \end{macro}
%    \begin{macro}{\ZREF@@newprop}
%    \begin{macrocode}
\def\ZREF@@newprop[#1]{%
  \global\@namedef{Z@D@\ZREF@P}{#1}%
  \global\expandafter\let\csname Z@X@\ZREF@P\endcsname\ZREF@X
  \begingroup\expandafter\endgroup
  \expandafter\ZREF@@@newprop\csname\ZREF@P\endcsname
  \expandafter\gdef\csname Z@C@\ZREF@P\endcsname{}%
  \zref@setcurrent\ZREF@P
}
\def\ZREF@@@newprop#1{%
  \expandafter
  \gdef\csname Z@E@\ZREF@P\endcsname##1#1##2##3\ZREF@nil{##2}%
}
%    \end{macrocode}
%    \end{macro}
%
%    \begin{macro}{\zref@showprop}
%    \begin{macrocode}
\ZREF@Robust\def\zref@showprop#1{%
  \zref@ifpropundefined{#1}{%
    \@PackageInfoNoLine{\ZREF@name}{%
      Show property `#1': <undefined>%
    }%
  }{%
    \begingroup
      \toks@\expandafter\expandafter\expandafter{%
        \csname Z@C@#1\endcsname
      }%
      \edef\ZREF@value{\the\toks@}%
      \ltx@onelevel@sanitize\ZREF@value
      \toks@\expandafter\expandafter\expandafter{%
        \csname Z@D@#1\endcsname
      }%
      \edef\ZREF@default{\the\toks@}%
      \ltx@onelevel@sanitize\ZREF@default
      \@PackageInfoNoLine{\ZREF@name}{%
        Show property `#1':\MessageBreak
        \expandafter\ifx\csname Z@X@#1\endcsname\ltx@empty
          Immediate %
        \else
          Delayed %
        \fi
        value: [\ZREF@value]\MessageBreak
        Default: [\ZREF@default]%
      }%
    \endgroup
  }%
}
%    \end{macrocode}
%    \end{macro}
%
%    \begin{macro}{\zref@setcurrent}
%    \cs{zref@setcurrent} sets the current value for a property.
%    \begin{macrocode}
\ZREF@Robust\def\zref@setcurrent#1#2{%
  \zref@propexists{#1}{%
    \expandafter\def\csname Z@C@#1\endcsname{#2}%
  }%
}
%    \end{macrocode}
%    \end{macro}
%
%    \begin{macro}{\ZREF@getcurrent}
%    \cs{zref@getcurrent} gets the current value for a property.
%    \begin{macrocode}
\def\ZREF@getcurrent#1{%
  \romannumeral0%
  \ltx@ifundefined{Z@C@#1}{%
    \ltx@space
  }{%
    \expandafter\expandafter\expandafter\ltx@space
    \csname Z@C@#1\endcsname
  }%
}
%    \end{macrocode}
%    \end{macro}
%    \begin{macro}{\ZREF@u@getcurrent}
%    \begin{macrocode}
\def\ZREF@wu@getcurrent#1{%
  \etex@unexpanded\expandafter\expandafter\expandafter{%
    \ZREF@getcurrent{#1}%
  }%
}
%    \end{macrocode}
%    \end{macro}
%    \begin{macro}{\zref@getcurrent}
%    \begin{macrocode}
\let\zref@getcurrent\ZREF@getcurrent
%    \end{macrocode}
%    \end{macro}
%
% \subsubsection{Reference generation}
%
%    \begin{macro}{\zref@label}
%    Label macro that uses the main property list.
%    \begin{macrocode}
\ZREF@Robust\def\zref@label#1{%
  \zref@labelbylist{#1}\ZREF@mainlist
}
%    \end{macrocode}
%    \end{macro}
%    \begin{macro}{\zref@labelbylist}
%    Label macro that stores the properties, specified in the
%    property list \verb|#2|.
%    \begin{macrocode}
\ZREF@Robust\def\zref@labelbylist#1#2{%
  \@bsphack
    \zref@listexists{#2}{%
      \expandafter\expandafter\expandafter\ZREF@label
      \expandafter\expandafter\expandafter{%
        \csname Z@L@#2\endcsname
      }{#1}%
    }%
  \@esphack
}
%    \end{macrocode}
%    \end{macro}
%    \begin{macro}{\zref@labelbyprops}
%    The properties are directly specified in a comma separated list.
%    \begin{macrocode}
\ZREF@Robust\def\zref@labelbyprops#1#2{%
  \@bsphack
    \begingroup
      \toks@{}%
      \comma@parse{#2}{%
        \zref@ifpropundefined\comma@entry{%
          \PackageWarning\ZREF@name{%
            Property `\comma@entry' is not known%
          }%
        }{%
          \toks@\expandafter{%
            \the\expandafter\toks@\csname\comma@entry\endcsname
          }%
        }%
        \ltx@gobble
      }%
    \expandafter\endgroup
    \expandafter\ZREF@label\expandafter{\the\toks@}{#1}%
  \@esphack
}
%    \end{macrocode}
%    \end{macro}
%    \begin{macro}{\zref@labelbykv}
%    \begin{macrocode}
\ZREF@Robust\def\zref@labelbykv#1#2{%
  \@bsphack
    \begingroup
      \let\Z@L@ZREF@temp\ltx@empty
      \kvsetkeys{ZREF@LABEL}{#1}%
      \ifZREF@immediate
        \expandafter\zref@wrapper@immediate\expandafter{%
          \expandafter\ZREF@label\expandafter{\Z@L@ZREF@temp}{#2}%
        }%
      \else
        \expandafter\ZREF@label\expandafter{\Z@L@ZREF@temp}{#2}%
      \fi
    \endgroup
  \@esphack
}
%    \end{macrocode}
%    \end{macro}
%    \begin{macrocode}
\kv@define@key{ZREF@LABEL}{prop}{%
  \edef\ZREF@param{#1}%
  \zref@propexists\ZREF@param{%
    \zref@iflistcontainsprop{ZREF@temp}\ZREF@param{}{%
      \begingroup\expandafter\endgroup
      \expandafter\ltx@LocalAppendToMacro
      \expandafter\Z@L@ZREF@temp
      \expandafter{\csname\ZREF@param\endcsname}%
    }%
  }%
}
\kv@define@key{ZREF@LABEL}{list}{%
  \zref@listforloop{#1}{%
    \zref@iflistcontainsprop{ZREF@temp}\zref@prop{}{%
      \begingroup\expandafter\endgroup
      \expandafter\ltx@LocalAppendToMacro
      \expandafter\Z@L@ZREF@temp
      \expandafter{\csname\zref@prop\endcsname}%
    }%
    \ltx@gobble
  }%
}
\kv@define@key{ZREF@LABEL}{delprop}{%
  \zref@propexists{#1}{%
    \zref@localdelprop{ZREF@temp}{#1}%
  }%
}
\kv@define@key{ZREF@LABEL}{immediate}[true]{%
  \edef\ZREF@param{#1}%
  \ifx\ZREF@param\ZREF@true
    \ZREF@immediatetrue
  \else
    \ifx\ZREF@param\ZREF@false
      \ZREF@immediatefalse
    \else
      \PackageWarning\ZREF@name{%
        Option `immediate' expects `true' or `false'.\MessageBreak
        Ignoring invalid value `\ZREF@param'%
      }%
    \fi
  \fi
}
%    \end{macrocode}
%    \begin{macro}{\ZREF@false}
%    \begin{macrocode}
\def\ZREF@false{false}
%    \end{macrocode}
%    \end{macro}
%    \begin{macro}{\ZREF@true}
%    \begin{macrocode}
\def\ZREF@true{true}
%    \end{macrocode}
%    \end{macro}
%
%    \begin{macrocode}
\kv@define@key{ZREF@LABEL}{values}[]{%
  \kv@parse{#1}{%
    \ifx\kv@value\relax
      \@PackageWarning\ZREF@name{%
        Missing value for property `\kv@key'%
      }%
      \expandafter\ltx@gobbletwo
    \else
      \expandafter\zref@setcurrent
    \fi
  }%
}
%    \end{macrocode}
%
%    \begin{macro}{\ifZREF@immediate}
%    The switch \cs{ifZREF@immediate} tells us, whether the
%    label should be written immediately or at page shipout time.
%    \cs{ZREF@label} need to be notified about this, because
%    it must disable the deferred execution of property values,
%    if the label is written immediately.
%    \begin{macrocode}
\newif\ifZREF@immediate
%    \end{macrocode}
%    \end{macro}
%    \begin{macro}{\zref@wrapper@immediate}
%    The argument of \cs{zref@wrapper@immediate} is executed inside
%    a group where \cs{write} is redefined by adding
%    \cs{immediate} before its execution. Also \cs{ZREF@label}
%    is notified via the switch \cs{ifZREF@immediate}.
%    \begin{macrocode}
\ZREF@Robust{\long\def}\zref@wrapper@immediate#1{%
  \begingroup
    \ZREF@immediatetrue
    \let\ZREF@org@write\write
    \def\write{\immediate\ZREF@org@write}%
    #1%
  \endgroup
}
%    \end{macrocode}
%    \end{macro}
%
%    \begin{macro}{\ZREF@label}
%    \cs{ZREF@label} writes the data in the \xfile{.aux} file.
%    \verb|#1| contains the list of valid properties,
%    \verb|#2| the name of the reference. In case of immediate
%    writing, the deferred execution of property values is
%    disabled. Also \thepage is made expandable in this case.
%    \begin{macrocode}
\def\ZREF@label#1#2{%
  \if@filesw
    \begingroup
      \ifZREF@immediate
        \let\ZREF@org@thepage\thepage
      \fi
      \protected@write\@auxout{%
        \ifZREF@immediate
          \let\thepage\ZREF@org@thepage
        \fi
        \let\ZREF@temp\ltx@empty
        \@tfor\ZREF@P:=#1\do{%
          \begingroup
            \escapechar=-1 %
            \edef\x{\endgroup
              \def\noexpand\ZREF@P{%
                \expandafter\string\ZREF@P
              }%
            }%
          \x
          \expandafter\ifx
              \csname
                \ifZREF@immediate
                  relax%
                \else
                  Z@X@\ZREF@P%
                \fi
              \endcsname
              \noexpand
            \expandafter\let\csname Z@C@\ZREF@P\endcsname\relax
          \fi
          \toks@\expandafter{\ZREF@temp}%
          \edef\ZREF@temp{%
            \the\toks@
            \ltx@backslashchar\ZREF@P{%
              \expandafter\noexpand\csname Z@C@\ZREF@P\endcsname
            }%
          }%
        }%
      }{%
        \string\zref@newlabel{#2}{\ZREF@temp}%
      }%
    \endgroup
  \fi
}
\def\ZREF@addtoks#1{%
  \toks@\expandafter\expandafter\expandafter{%
    \expandafter\the\expandafter\toks@#1%
  }%
}
%    \end{macrocode}
%    \end{macro}
%
% \subsubsection{Reference querying and extracting}
%
%    Design goal for the extracting macros is that the extraction
%    process is full expandable. Thus these macros can be used
%    in expandable contexts. But there are problems that cannot
%    be solved by full expandable macros:
%    \begin{itemize}
%    \item In standard \hologo{LaTeX} undefined references sets a flag
%      and generate a warning. Both actions are not expandable.
%    \item Babel's support for its shorthand uses commands that
%      use non-expandable assignments. However currently there
%      is hope, that primitives are added to \pdfTeX\ that
%      allows the detection of contexts. Then the shorthand
%      can detect, if they are executed inside \cs{csname}
%      and protect themselves automatically.
%    \end{itemize}
%    \begin{macro}{\zref@ifrefundefined}
%    If a reference \verb|#1| is undefined, then macro
%    \cs{zref@ifrefundefined} calls \verb|#2| and \verb|#3| otherwise.
%    \begin{macrocode}
\def\zref@ifrefundefined#1{%
  \ltx@ifundefined{Z@R@#1}%
}
%    \end{macrocode}
%    \end{macro}
%
%    \begin{macro}{\zifrefundefined}
%    If a reference \verb|#1| is undefined, then macro
%    \cs{zref@ifrefundefined} calls \verb|#2| and \verb|#3| otherwise.
%    Also the reference is marked used.
%    \begin{macrocode}
\ZREF@IfDefinable\zifrefundefined\def{%
  #1{%
    \zref@wrapper@babel\ZREF@ifrefundefined{#1}%
  }%
}
%    \end{macrocode}
%    \end{macro}
%    \begin{macro}{\ZREF@ifrefundefined}
%    \begin{macrocode}
\def\ZREF@ifrefundefined#1{%
  \zref@refused{#1}%
  \zref@ifrefundefined{#1}%
}
%    \end{macrocode}
%    \end{macro}
%
%    \begin{macro}{\zref@refused}
%    The problem with undefined references is addressed by
%    the macro \cs{zref@refused}. This can be used outside
%    the expandable context. In case of an undefined reference
%    the flag is set to notify \hologo{LaTeX} and a warning is given.
%    \begin{macrocode}
\ZREF@Robust\def\zref@refused#1{%
  \zref@wrapper@babel\ZREF@refused{#1}%
}
%    \end{macrocode}
%    \end{macro}
%    \begin{macro}{\ZREF@refused}
%    \begin{macrocode}
\def\ZREF@refused#1{%
  \zref@ifrefundefined{#1}{%
    \protect\G@refundefinedtrue
    \@latex@warning{%
      Reference `#1' on page \thepage \space undefined%
    }%
  }{}%
}
%    \end{macrocode}
%    \end{macro}
%
%    \begin{macro}{\zref@ifrefcontainsprop}
%    \cs{zref@ifrefcontainsprop} looks, if the reference \verb|#1|
%    has the property \verb|#2| and calls then \verb|#3| and
%    \verb|#4| otherwise.
%    \begin{macrocode}
\def\zref@ifrefcontainsprop#1#2{%
  \zref@ifrefundefined{#1}{%
    \ltx@secondoftwo
  }{%
    \expandafter\ZREF@ifrefcontainsprop
    \csname Z@E@#2\expandafter\endcsname
    \csname#2\expandafter\expandafter\expandafter\endcsname
    \expandafter\expandafter\expandafter{%
      \csname Z@R@#1\endcsname
    }%
  }%
}
\def\ZREF@ifrefcontainsprop#1#2#3{%
  \expandafter\ifx\expandafter\ZREF@novalue
  #1#3#2\ZREF@novalue\ZREF@nil\ltx@empty
    \expandafter\ltx@secondoftwo
  \else
    \expandafter\ltx@firstoftwo
  \fi
}
\def\ZREF@novalue{\ZREF@NOVALUE}
%    \end{macrocode}
%    \end{macro}
%
%    \begin{macro}{\zref@extract}
%    \cs{zref@extract} is an abbreviation for the case
%    that the default of the property is used as default
%    value.
%    \begin{macrocode}
\def\ZREF@extract#1#2{%
  \romannumeral0%
  \ltx@ifundefined{Z@D@#2}{%
    \expandafter\ltx@space\zref@default
  }{%
    \expandafter\expandafter\expandafter\ZREF@@extract
    \expandafter\expandafter\expandafter{%
      \csname Z@D@#2\endcsname
    }{#1}{#2}%
  }%
}
%    \end{macrocode}
%    \end{macro}
%    \begin{macro}{\ZREF@@extract}
%    \begin{macrocode}
\def\ZREF@@extract#1#2#3{%
  \expandafter\expandafter\expandafter\ltx@space
  \zref@extractdefault{#2}{#3}{#1}%
}
%    \end{macrocode}
%    \end{macro}
%    \begin{macro}{\ZREF@wu@extract}
%    \begin{macrocode}
\def\ZREF@wu@extract#1#2{%
  \etex@unexpanded\expandafter\expandafter\expandafter{%
    \ZREF@extract{#1}{#2}%
  }%
}
%    \end{macrocode}
%    \end{macro}
%    \begin{macro}{\zref@extract}
%    \begin{macrocode}
\let\zref@extract\ZREF@extract
%    \end{macrocode}
%    \end{macro}
%
%    \begin{macro}{\ZREF@extractdefault}
%    The basic extracting macro is \cs{zref@extractdefault} with
%    the reference name in \verb|#1|, the property in \verb|#2| and
%    the default value in \verb|#3| in case for problems.
%    \begin{macrocode}
\def\ZREF@extractdefault#1#2#3{%
  \romannumeral0%
  \zref@ifrefundefined{#1}\ltx@firstoftwo{%
    \zref@ifpropundefined{#2}\ltx@firstoftwo\ltx@secondoftwo
  }{%
    \ltx@space
    #3%
  }{%
    \expandafter\expandafter\expandafter\ltx@space
    \csname Z@E@#2\expandafter\expandafter\expandafter\endcsname
    \csname Z@R@#1\expandafter\endcsname
    \csname#2\endcsname{#3}\ZREF@nil
  }%
}
%    \end{macrocode}
%    \end{macro}
%    \begin{macro}{\ZREF@wu@extractdefault}
%    \begin{macrocode}
\def\ZREF@wu@extractdefault#1#2#3{%
  \etex@unexpanded\expandafter\expandafter\expandafter{%
    \ZREF@extractdefault{#1}{#2}{#3}%
  }%
}
%    \end{macrocode}
%    \end{macro}
%    \begin{macro}{\zref@extractdefault}
%    \begin{macrocode}
\let\zref@extractdefault\ZREF@extractdefault
%    \end{macrocode}
%    \end{macro}
%
%    \begin{macro}{\zref@def@extract}
%    \begin{macrocode}
\ZREF@Robust\def\zref@def@extract#1{%
  \zref@wrapper@babel{\ZREF@def@extract{#1}}%
}
%    \end{macrocode}
%    \end{macro}
%    \begin{macro}{\ZREF@def@extract}
%    \begin{macrocode}
\def\ZREF@def@extract#1#2#3{%
  \zref@refused{#2}%
  \expandafter\expandafter\expandafter\def
  \expandafter\expandafter\expandafter#1%
  \expandafter\expandafter\expandafter{%
    \zref@extract{#2}{#3}%
  }%
}
%    \end{macrocode}
%    \end{macro}
%    \begin{macro}{\zref@def@extractdefault}
%    \begin{macrocode}
\ZREF@Robust\def\zref@def@extractdefault#1{%
  \zref@wrapper@babel{\ZREF@def@extractdefault{#1}}%
}
%    \end{macrocode}
%    \end{macro}
%    \begin{macro}{\ZREF@def@extractdefault}
%    \begin{macrocode}
\def\ZREF@def@extractdefault#1#2#3#4{%
  \zref@refused{#2}%
  \expandafter\expandafter\expandafter\def
  \expandafter\expandafter\expandafter#1%
  \expandafter\expandafter\expandafter{%
    \zref@extractdefault{#2}{#3}{#4}%
  }%
}
%    \end{macrocode}
%    \end{macro}
%
%    \begin{macro}{\ZREF@wrapper@unexpanded}
%    \begin{macrocode}
\ZREF@Robust{\long\def}\ZREF@wrapper@unexpanded#1{%
  \let\zref@wrapper@unexpanded\ltx@firstofone
  \let\zref@getcurrent\ZREF@wu@getcurrent
  \let\zref@extractdefault\ZREF@wu@extractdefault
  \let\zref@extract\ZREF@wu@extract
  #1%
  \let\zref@wrapper@unexpanded\ZREF@wrapper@unexpanded
  \let\zref@getcurrent\ZREF@getcurrent
  \let\zref@extractdefault\ZREF@extractdefault
  \let\zref@extract\ZREF@extract
}
%    \end{macrocode}
%    \end{macro}
%    \begin{macro}{\zref@wrapper@unexpanded}
%    \begin{macrocode}
\ltx@IfUndefined{etex@unexpanded}{%
  \let\zref@wrapper@unexpanded\ltx@firstofone
}{%
  \let\zref@wrapper@unexpanded\ZREF@wrapper@unexpanded
}
%    \end{macrocode}
%    \end{macro}
%
% \subsubsection{Compatibility with \xpackage{babel}}
%
%    \begin{macro}{\zref@wrapper@babel}
%    \begin{macrocode}
\ZREF@Robust{\long\def}\zref@wrapper@babel#1#2{%
  \ifcsname if@safe@actives\endcsname
    \expandafter\ltx@firstofone
  \else
    \expandafter\ltx@secondoftwo
  \fi
  {%
    \if@safe@actives
      \expandafter\ltx@secondoftwo
    \else
      \expandafter\ltx@firstoftwo
    \fi
    {%
      \begingroup
        \csname @safe@activestrue\endcsname
        \edef\x{#2}%
      \expandafter\endgroup
      \expandafter\ZREF@wrapper@babel\expandafter{\x}{#1}%
    }%
  }{%
    #1{#2}%
  }%
}
\long\def\ZREF@wrapper@babel#1#2{%
  #2{#1}%
}
%    \end{macrocode}
%    \end{macro}
%
% \subsubsection{Unique counter support}
%
%    \begin{macro}{\zref@require@unique}
%    Generate the counter |zref@unique| if the counter does not
%    already exist.
%    \begin{macrocode}
\ZREF@Robust\def\zref@require@unique{%
  \@ifundefined{c@zref@unique}{%
    \begingroup
      \let\@addtoreset\ltx@gobbletwo
      \newcounter{zref@unique}%
    \endgroup
%    \end{macrocode}
%    \end{macro}
%
%    \begin{macro}{\thezref@unique}
%    \cs{thezref@unique} is used for automatically generated unique
%    labelnames.
%    \begin{macrocode}
    \renewcommand*{\thezref@unique}{%
      zref@\number\c@zref@unique
    }%
  }{}%
}
%    \end{macrocode}
%    \end{macro}
%
% \subsubsection{Utilities}
%
%    \begin{macro}{\ZREF@number}
%    \begin{macrocode}
\ltx@IfUndefined{numexpr}{%
  \def\ZREF@number#1{\number#1}%
}{%
  \def\ZREF@number#1{\the\numexpr(#1)\relax}%
}
%    \end{macrocode}
%    \end{macro}
%
% \subsubsection{Setup}
%
%    \begin{macro}{\zref@setdefault}
%    Standard \hologo{LaTeX} prints ``??'' in bold face if a reference
%    is not known. \cs{zref@default} holds the text that is
%    printed in case of unknown references and is used, if
%    the default was not specified during the definition
%    of the new property by \cs{ref@newprop}.
%    The global default value can be set by \cs{zref@setdefault}.
%    \begin{macrocode}
\ZREF@Robust\def\zref@setdefault#1{%
  \def\zref@default{#1}%
}
%    \end{macrocode}
%    \end{macro}
%    \begin{macro}{\zref@default}
%    Now we initialize \cs{zref@default} with the same value
%    that \hologo{LaTeX} uses for its undefined references.
%    \begin{macrocode}
\zref@setdefault{%
  \nfss@text{\reset@font\bfseries ??}%
}
%    \end{macrocode}
%    \end{macro}
%
% \paragraph{Main property list.}
%
%    \begin{macro}{\zref@setmainlist}
%    The name of the default property list is stored in
%    \cs{ZREF@mainlist} and can be set by \cs{zref@setmainlist}.
%    \begin{macrocode}
\ZREF@Robust\def\zref@setmainlist#1{%
  \def\ZREF@mainlist{#1}%
}
\zref@setmainlist{main}
%    \end{macrocode}
%    \end{macro}
%    Now we create the list.
%    \begin{macrocode}
\zref@newlist\ZREF@mainlist
%    \end{macrocode}
%
% \paragraph{Main properties.}
%
%    The two properties \texttt{default} and \texttt{page}
%    are created and added to the main property list.
%    They store the data that
%    standard \hologo{LaTeX} uses in its references created by
%    \cs{label}.
%    \begin{description}
%    \item[\texttt{default}] the apperance of the latest
%      counter that is incremented by \cs{refstepcounter}
%    \item[\texttt{page}] the apperance of the page counter
%    \end{description}
%    \begin{macrocode}
\zref@newprop{default}{\@currentlabel}
\zref@newprop*{page}{\thepage}
\zref@addprops\ZREF@mainlist{default,page}
%    \end{macrocode}
%
% \paragraph{Properties}
%
%    \begin{macro}{\ZREF@NewPropAnchor}
%    \begin{macrocode}
\def\ZREF@NewPropAnchor{%
  \zref@newprop{anchor}{%
    \ltx@ifundefined{@currentHref}{}{\@currentHref}%
  }%
  \global\let\ZREF@NewPropAnchor\relax
}
%    \end{macrocode}
%    \end{macro}
%    \begin{macro}{\zref@titleref@current}
%    Later we will redefine the section and caption macros
%    to catch the current title and remember the value in
%    \cs{zref@titleref@current}.
%    \begin{macro}{\ZREF@NewPropTitle}
%    \begin{macrocode}
\def\ZREF@NewPropTitle{%
  \gdef\zref@titleref@current{}%
  \zref@newprop{title}{\zref@titleref@current}%
  \global\let\ZREF@NewPropTitle\relax
}
%    \end{macrocode}
%    \end{macro}
%    \end{macro}
%    \begin{macro}{\ZREF@NewPropTheotype}
%    \begin{macrocode}
\def\ZREF@NewPropTheotype{%
  \zref@newprop{theotype}{}%
  \global\let\ZREF@NewPropTheotype\relax
}
%    \end{macrocode}
%    \end{macro}
%    \begin{macro}{\ZREF@NewPropPageValue}
%    \begin{macrocode}
\def\ZREF@NewPropPageValue{%
  \zref@newprop*{pagevalue}[0]{\number\c@page}%
  \global\let\ZREF@NewPropPageValue\relax
}
%    \end{macrocode}
%    \end{macro}
%
% \paragraph{Mark successful loading}
%
%    \begin{macrocode}
\let\ZREF@base@ok=Y
%</base>
%    \end{macrocode}
%
%
% \subsection{Module \xmodule{user}}
%
%    \begin{macrocode}
%<*user>
\NeedsTeXFormat{LaTeX2e}
\ProvidesPackage{zref-user}%
  [2020-07-03 v2.32 Module user for zref (HO)]%
\RequirePackage{zref-base}[2019/11/29]
\ifx\ZREF@base@ok Y%
\else
  \expandafter\endinput
\fi
%    \end{macrocode}
%
%    Module \xmodule{user} enables a small user interface.
%    All macros are prefixed by \cs{z}.
%
%    First we define the pendants to the standard \hologo{LaTeX} referencing
%    commands \cs{label}, \cs{ref}, and \cs{pageref}.
%
%    \begin{macro}{\zlabel}
%    Similar to \cs{label} the macro \cs{zlabel} writes a reference
%    entry in the \xfile{.aux} file. The main property list is used.
%    Also we add the babel patch. The \cs{label} command can also be
%    used inside section titles, but it must not go into the
%    table of contents. Therefore we have to check this situation.
%    \begin{macrocode}
\newcommand*\zlabel{%
  \ifx\label\ltx@gobble
    \expandafter\ltx@gobble
  \else
    \expandafter\zref@wrapper@babel\expandafter\zref@label
  \fi
}%
%    \end{macrocode}
%    \end{macro}
%
%    \begin{macro}{\zkvlabel}
%    \begin{macrocode}
\newcommand*{\zkvlabel}[1]{%
  \ifx\label\ltx@gobble
    \expandafter\ltx@gobblethree
  \fi
  \zref@wrapper@babel{\zref@labelbykv{#1}}%
}%
%    \end{macrocode}
%    \end{macro}
%
%    \begin{macro}{\zref}
%    Macro \cs{zref} is the corresponding macro for \cs{ref}.
%    Also it provides an optional argument in order to
%    select another property.
%    \begin{macrocode}
\newcommand*{\zref}[2][default]{% robust because of optional argument
  \zref@propexists{#1}{%
    \zref@wrapper@babel\ZREF@zref{#2}{#1}%
  }%
}%
\def\ZREF@zref#1{%
  \zref@refused{#1}%
  \zref@extract{#1}%
}%
%    \end{macrocode}
%    \end{macro}
%
%    \begin{macro}{\zpageref}
%    For macro \cs{zpageref} we just call \cs{zref}
%    with property \verb|page|.
%    \begin{macrocode}
\ZREF@IfDefinable\zpageref\def{%
  {\zref[page]}%
}
%    \end{macrocode}
%    \end{macro}
%
%    \begin{macro}{\zrefused}
%    For the following expandible user macros \cs{zrefused}
%    should be used to notify \hologo{LaTeX} in case of undefined
%    references.
%    \begin{macrocode}
\ZREF@IfDefinable\zrefused\def{%
  {\zref@refused}%
}
%    \end{macrocode}
%    \end{macro}
%
%    \begin{macrocode}
%</user>
%    \end{macrocode}
%
%
% \subsection{Module \xmodule{abspage}}
%
%    \begin{macrocode}
%<*abspage>
\NeedsTeXFormat{LaTeX2e}
\ProvidesPackage{zref-abspage}%
  [2020-07-03 v2.32 Module abspage for zref (HO)]%
\RequirePackage{zref-base}[2019/11/29]
\ifx\ZREF@base@ok Y%
\else
  \expandafter\endinput
\fi
%    \end{macrocode}
%
%    Module \xmodule{abspage} adds a new property \verb|abspage|
%    to the \verb|main| property list for absolute page numbers.
%    These are recorded by the help of package \xpackage{atbegshi}.
%    \begin{macrocode}
\RequirePackage{atbegshi}[2011/10/05]%
%    \end{macrocode}
%    The counter \verb|abspage| must not go in the clear list
%    of \verb|@ckpt| that is used to set counters in
%    \xfile{.aux} files of included \hologo{TeX} files.
%    \begin{macrocode}
\begingroup
  \let\@addtoreset\ltx@gobbletwo
  \newcounter{abspage}%
\endgroup
\setcounter{abspage}{0}%
\AtBeginShipout{%
  \stepcounter{abspage}%
}%
\zref@newprop*{abspage}[0]{\the\c@abspage}%
\zref@addprop\ZREF@mainlist{abspage}%
%    \end{macrocode}
%    Note that counter \verb|abspage| shows the
%    previous page during page processing. Before
%    shipout the counter is incremented.
%    Thus the property is correctly written with
%    deferred writing. If the counter is written
%    using \cs{zref@wrapper@immediate}, then the number
%    is too small by one.
%
%    \begin{macrocode}
%</abspage>
%    \end{macrocode}
%
%
% \subsection{Module \xmodule{counter}}
%
%    \begin{macrocode}
%<*counter>
\NeedsTeXFormat{LaTeX2e}
\ProvidesPackage{zref-counter}%
  [2020-07-03 v2.32 Module counter for zref (HO)]%
\RequirePackage{zref-base}[2019/11/29]
\ifx\ZREF@base@ok Y%
\else
  \expandafter\endinput
\fi
%    \end{macrocode}
%
%    For features such as \xpackage{hyperref}'s \cs{autoref}
%    we need the name of the counter. The property
%    \verb|counter| is defined and added to the main property
%    list. Starting with \LaTeX\ 2020-10-01 the proper can use
%    \@currentcounter. In older formats
%    \cs{refstepcounter} has to be patched but this can fail in some
%    cases, see issue \#5.
%    \begin{macrocode}
\@ifl@t@r\fmtversion{2020-10-01}
 {
  \zref@newprop{counter}{\@currentcounter}
  \zref@addprop\ZREF@mainlist{counter}
 }
 {
  \zref@newprop{counter}{}
  \zref@addprop\ZREF@mainlist{counter}
  \AtBeginDocument{%
    \ZREF@patch{refstepcounter}{%
      \def\refstepcounter#1{%
        \zref@setcurrent{counter}{#1}%
        \ZREF@org@refstepcounter{#1}%
      }%
    }%
  }
 }
%    \end{macrocode}
%    \begin{macrocode}
%</counter>
%    \end{macrocode}
%
%
% \subsection{Module \xmodule{lastpage}}
%
%    \begin{macrocode}
%<*lastpage>
\NeedsTeXFormat{LaTeX2e}
\ProvidesPackage{zref-lastpage}%
  [2020-07-03 v2.32 Module lastpage for zref (HO)]%
\RequirePackage{zref-base}[2019/11/29]
\RequirePackage{zref-abspage}[2019/11/29]
\RequirePackage{atveryend}[2009/12/07]
\ifx\ZREF@base@ok Y%
\else
  \expandafter\endinput
\fi
%    \end{macrocode}
%
%    The module \xmodule{lastpage} implements the service
%    of package \xpackage{lastpage} by setting a reference
%    \verb|LastPage| at the end of the document.
%    If module \xmodule{abspage} is given, also the absolute
%    page number is available, because the properties of
%    the main property list are used.
%    \begin{macrocode}
\zref@newlist{LastPage}
\AfterLastShipout{%
  \if@filesw
    \begingroup
      \advance\c@page\m@ne
      \toks@\expandafter\expandafter\expandafter{%
        \expandafter\Z@L@main
        \Z@L@LastPage
      }%
      \expandafter\zref@wrapper@immediate\expandafter{%
        \expandafter\ZREF@label\expandafter{\the\toks@}{LastPage}%
      }%
    \endgroup
  \fi
}
%    \end{macrocode}
%    \begin{macro}{\zref@iflastpage}
%    \begin{macrocode}
\def\zref@iflastpage#1{%
  \ifnum\zref@extractdefault{#1}{abspage}{-1}=%
        \zref@extractdefault{LastPage}{abspage}{-2} %
    \expandafter\ltx@firstoftwo
  \else
    \expandafter\ltx@secondoftwo
  \fi
}
%    \end{macrocode}
%    \end{macro}
%    \begin{macro}{\ziflastpage}
%    \begin{macrocode}
\ZREF@IfDefinable\ziflastpage\def{%
  {\zref@wrapper@babel\ZREF@iflastpage}%
}
%    \end{macrocode}
%    \end{macro}
%    \begin{macro}{ZREF@iflastpage}
%    \begin{macrocode}
\def\ZREF@iflastpage#1{%
  \zref@refused{LastPage}%
  \zref@refused{#1}%
  \zref@iflastpage{#1}%
}
%    \end{macrocode}
%    \end{macro}
%    \begin{macrocode}
%</lastpage>
%    \end{macrocode}
%
%
% \subsection{Module \xmodule{thepage}}
%
%    \begin{macrocode}
%<*thepage>
\NeedsTeXFormat{LaTeX2e}
\ProvidesPackage{zref-thepage}%
  [2020-07-03 v2.32 Module thepage for zref (HO)]%
\RequirePackage{zref-base}[2019/11/29]
\ifx\ZREF@base@ok Y%
\else
  \expandafter\endinput
\fi
%    \end{macrocode}
%
%    \begin{macrocode}
\RequirePackage{atbegshi}[2011/10/05]
\RequirePackage{zref-abspage}[2019/11/29]
%    \end{macrocode}
%    \begin{macrocode}
\zref@newlist{thepage}
\zref@addprop{thepage}{page}
\ZREF@NewPropPageValue
%    \end{macrocode}
%    \begin{macro}{\zref@thepage@atbegshi@hook}
%    \begin{macrocode}
\let\zref@thepage@atbegshi@hook\ltx@empty
%    \end{macrocode}
%    \end{macro}
%    \begin{macrocode}
\zref@addprop{thepage}{pagevalue}
\AtBeginShipout{%
  \AtBeginShipoutAddToBox{%
    \zref@thepage@atbegshi@hook
    \zref@labelbylist{thepage\the\value{abspage}}{thepage}%
  }%
}
%    \end{macrocode}
%    \begin{macro}{\zref@thepage@name}
%    \begin{macrocode}
\ltx@IfUndefined{numexpr}{%
  \def\zref@thepage@name#1{thepage\number#1}%
}{%
  \def\zref@thepage@name#1{thepage\the\numexpr#1}%
}
%    \end{macrocode}
%    \end{macro}
%    \begin{macro}{\zref@thepage}
%    \begin{macrocode}
\def\zref@thepage#1{%
  \zref@extract{\zref@thepage@name{#1}}{page}%
}%
%    \end{macrocode}
%    \end{macro}
%    \begin{macro}{\zref@thepage@refused}
%    \begin{macrocode}
\ZREF@Robust\def\zref@thepage@refused#1{%
  \zref@refused{\zref@thepage@name{#1}}%
}%
%    \end{macrocode}
%    \end{macro}
%    \begin{macro}{\zthepage}
%    \begin{macrocode}
\ZREF@IfDefinable\zthepage\def{%
  #1{%
    \zref@thepage@refused{#1}%
    \zref@thepage{#1}%
  }%
}
%    \end{macrocode}
%    \end{macro}
%    \begin{macrocode}
%</thepage>
%    \end{macrocode}
%
%
% \subsection{Module \xmodule{nextpage}}
%
%    \begin{macrocode}
%<*nextpage>
\NeedsTeXFormat{LaTeX2e}
\ProvidesPackage{zref-nextpage}%
  [2020-07-03 v2.32 Module nextpage for zref (HO)]%
\RequirePackage{zref-base}[2019/11/29]
\ifx\ZREF@base@ok Y%
\else
  \expandafter\endinput
\fi
%    \end{macrocode}
%
%    \begin{macrocode}
\RequirePackage{zref-abspage}[2019/11/29]
\RequirePackage{zref-thepage}[2019/11/29]
\RequirePackage{zref-lastpage}[2019/11/29]
\RequirePackage{uniquecounter}[2009/12/18]
%    \end{macrocode}
%    \begin{macrocode}
\UniqueCounterNew{znextpage}

\newcommand*{\znextpagesetup}{%
  \afterassignment\ZREF@np@setup@i
  \def\ZREF@np@call@unknown##1%
}
\def\ZREF@np@setup@i{%
  \afterassignment\ZREF@np@setup@ii
  \def\ZREF@np@call@nonext##1%
}
\def\ZREF@np@setup@ii{%
  \def\ZREF@np@call@next##1%
}
\def\ZREF@np@call@unknown#1{#1}
\def\ZREF@np@call@nonext#1{#1}
\def\ZREF@np@call@next#1{#1}
\ZREF@IfDefinable\znextpage\def{%
  {\UniqueCounterCall{znextpage}{\ZREF@nextpage}}%
}%
\newcommand*{\znonextpagename}{}
\newcommand*{\zunknownnextpagename}{\Z@D@page}
\def\ZREF@nextpage#1{%
  \begingroup
    \def\ZREF@refname@this{zref@np#1}%
    \zref@labelbyprops\ZREF@refname@this{abspage}%
    \chardef\ZREF@call=0 % unknown
    \ZREF@ifrefundefined\ZREF@refname@this{%
    }{%
      \edef\ZREF@pagenum@this{%
        \zref@extractdefault\ZREF@refname@this{abspage}{0}%
      }%
      \edef\ZREF@refname@next{%
        \zref@thepage@name{%
          \the\numexpr\ZREF@pagenum@this+1%
        }%
      }%
      \ifnum\ZREF@pagenum@this>0 %
        \ZREF@ifrefundefined{LastPage}{%
          \zref@ifrefundefined\ZREF@refname@next{%
          }{%
            \chardef\ZREF@call=2 % next page
          }%
        }{%
          \edef\ZREF@pagenum@last{%
            \zref@extractdefault{LastPage}{abspage}{0}%
          }%
          \ifnum\ZREF@pagenum@this<\ZREF@pagenum@last\ltx@space
            \ZREF@ifrefundefined\ZREF@refname@next{%
            }{%
              \chardef\ZREF@call=2 % next page
            }%
          \else
            \ifnum\ZREF@pagenum@this=\ZREF@pagenum@this\ltx@space
              \chardef\ZREF@call=1 % no next page
            \fi
          \fi
        }%
      \fi
    }%
    \edef\x{%
      \endgroup
      \ifcase\ZREF@call
        \noexpand\ZREF@np@call@unknown{%
          \noexpand\zunknownnextpagename
        }%
      \or
        \noexpand\ZREF@np@call@nonext{%
          \noexpand\znonextpagename
        }%
      \else
        \noexpand\ZREF@np@call@next{%
          \noexpand\zref@extract{\ZREF@refname@next}{page}%
        }%
      \fi
    }%
  \x
}
%    \end{macrocode}
%    \begin{macrocode}
%</nextpage>
%    \end{macrocode}
%
%
% \subsection{Module \xmodule{totpages}}
%
%    \begin{macrocode}
%<*totpages>
\NeedsTeXFormat{LaTeX2e}
\ProvidesPackage{zref-totpages}%
  [2020-07-03 v2.32 Module totpages for zref (HO)]%
\RequirePackage{zref-base}[2019/11/29]
\ifx\ZREF@base@ok Y%
\else
  \expandafter\endinput
\fi
%    \end{macrocode}
%
%    The absolute page number of the last page is the total
%    page number.
%    \begin{macrocode}
\RequirePackage{zref-abspage}[2019/11/29]
\RequirePackage{zref-lastpage}[2019/11/29]
%    \end{macrocode}
%
%    \begin{macro}{\ztotpages}
%    Macro \cs{ztotpages} contains the number of pages. It can be
%    used inside expandable calculations. It expands to zero if
%    the reference is not yet available.
%    \begin{macrocode}
\newcommand*{\ztotpages}{%
  \zref@extractdefault{LastPage}{abspage}{0}%
}
%    \end{macrocode}
%    \end{macro}
%    Also we mark the reference |LastPage| as used:
%    \begin{macrocode}
\AtBeginDocument{%
  \zref@refused{LastPage}%
}
%    \end{macrocode}
%    \begin{macrocode}
%</totpages>
%    \end{macrocode}
%
%
% \subsection{Module \xmodule{pagelayout}}
%
%    \begin{macrocode}
%<*pagelayout>
\NeedsTeXFormat{LaTeX2e}
\ProvidesPackage{zref-pagelayout}%
  [2020-07-03 v2.32 Module pagelayout for zref (HO)]%
\RequirePackage{zref-base}[2019/11/29]
\ifx\ZREF@base@ok Y%
\else
  \expandafter\endinput
\fi
%    \end{macrocode}
%
%    \begin{macrocode}
\RequirePackage{zref-thepage}[2019/11/29]
\RequirePackage{iftex}[2019/11/07]%
\RequirePackage{atveryend}[2010/03/24]
%    \end{macrocode}
%
%
% \subsubsection{Define layout properties}
%
%    \begin{macrocode}
\def\ZREF@temp#1{%
  \begingroup
    \escapechar=-1 %
  \ltx@ifundefined{\string#1}{\endgroup}{%
    \edef\x{%
      \endgroup
      \noexpand\zref@newprop*{\string#1}%
                             [\noexpand\number\noexpand#1]% hash-ok
                             {\noexpand\number\noexpand#1}%
      \noexpand\zref@addprop{thepage}{\string#1}%
    }%
    \x
  }%
}
\ZREF@temp\mag
\ZREF@temp\paperwidth
\ZREF@temp\paperheight
\ZREF@temp\stockwidth % memoir.cls, crop.sty
\ZREF@temp\stockheight % memoir.cls, crop.sty
\ZREF@temp\mediawidth % VTeX
\ZREF@temp\mediaheight % VTeX
\ifluatex
\zref@newprop*{pdfvorigin}%
                             [\number\pdfvariable vorigin]% hash-ok
                             {\number\pdfvariable vorigin}%
\zref@addprop{thepage}{pdfvorigin}
\zref@newprop*{pdfhorigin}%
                             [\number\pdfvariable horigin]% hash-ok
                             {\number\pdfvariable horigin}%
\zref@addprop{thepage}{pdfhorigin}
\zref@newprop*{pdfpageheight}%
                             [\number\pageheight]% hash-ok
                             {\number\pageheight}%
\zref@addprop{thepage}{pdfpageheight}
\zref@newprop*{pdfpagewidth}%
                             [\number\pagewidth]% hash-ok
                             {\number\pagewidth}%
\zref@addprop{thepage}{pdfpagewidth}
\else
\ZREF@temp\pdfpagewidth
\ZREF@temp\pdfpageheight
\ZREF@temp\pdfhorigin
\ZREF@temp\pdfvorigin
\fi
\ZREF@temp\hoffset
\ZREF@temp\voffset
\ZREF@temp\topmargin
\ZREF@temp\oddsidemargin
\ZREF@temp\evensidemargin
\ZREF@temp\textwidth
\ZREF@temp\textheight
\ZREF@temp\headheight
\ZREF@temp\headsep
\ZREF@temp\footskip
\ZREF@temp\marginparwidth
\ZREF@temp\marginparsep
\ZREF@temp\columnwidth
\ZREF@temp\columnsep
\ZREF@temp\trimedge % memoir.cls
\ZREF@temp\spinemargin % memoir.cls
\ZREF@temp\foremargin % memoir.cls
\ZREF@temp\trimtop % memoir.cls
\ZREF@temp\uppermargin % memoir.cls
\ZREF@temp\headmargin % memoir.cls
%    \end{macrocode}
%
%    \begin{macrocode}
\IfFormatAtLeastTF{2020/10/01}
 {
   \zref@newprop*{outputboxwd}[0pt]{\ShipoutBoxWidth}
   \zref@newprop*{outputboxht}[0pt]{\ShipoutBoxHeight}
   \zref@newprop*{outputboxdp}[0pt]{\ShipoutBoxDepth}
 }
 {
   \zref@newprop*{outputboxwd}[0pt]{\AtBeginShipoutBoxWidth}
   \zref@newprop*{outputboxht}[0pt]{\AtBeginShipoutBoxHeight}
   \zref@newprop*{outputboxdp}[0pt]{\AtBeginShipoutBoxDepth}
 }
%    \end{macrocode}
%    \begin{macrocode}
\zref@addprops{thepage}{outputboxwd,outputboxht,outputboxdp}
%    \end{macrocode}
%
%    \begin{macro}{\ifZREF@pl@list}
%    \begin{macrocode}
\ltx@newif\ifZREF@pl@list
%    \end{macrocode}
%    \end{macro}
%    \begin{macro}{\zref@listpagelayout}
%    \begin{macrocode}
\ZREF@IfDefinable\zlistpagelayout\def{%
  {\global\ZREF@pl@listtrue}%
}
%    \end{macrocode}
%    \end{macro}
%    \begin{macro}{\ZREF@pl@AfterLastShipout}
%    \begin{macrocode}
\def\ZREF@pl@AfterLastShipout{%
  \ifZREF@pl@list
    \edef\ZREF@page@max{\the\value{abspage}}%
    \ltx@ifundefined{ZREF@org@testdef}{%
      \let\ZREF@org@testdef\@testdef
      \def\@testdef##1##2##3{%
        \ZREF@org@testdef{##1}{##2}{##3}%
        \def\ZREF@temp{##1}%
        \ifx\ZREF@temp\ZREF@RefPrefix
          \expandafter\gdef\csname##1@##2\endcsname{##3}%
        \fi
      }%
    }{}%
    \AtVeryEndDocument{\ZREF@pl@AtVeryEnd}%
  \fi
}
%    \end{macrocode}
%    \end{macro}
%    \begin{macro}{\ZREF@pl@AtVeryEnd}
%    \begin{macrocode}
\def\ZREF@pl@AtVeryEnd{%
  \begingroup
    \toks@{Page layout parameters:\MessageBreak}%
    \count@=1 %
    \ZREF@pl@ListPage
    \edef\x{\endgroup
      \noexpand\@PackageInfoNoLine{zref-pagelayout}{\the\toks@}%
    }%
  \x
}
%    \end{macrocode}
%    \end{macro}
%    \begin{macro}{\ZREF@pl@ListPage}
%    \begin{macrocode}
\def\ZREF@pl@ListPage{%
  \edef\x{%
    \toks@={%
      \the\toks@
      Page \the\count@:\noexpand\MessageBreak
      \zref@ifrefundefined{thepage\the\count@}{}{%
        \ltx@space\ltx@space mag = %
        \zref@extract{thepage\the\count@}{mag}%
        \noexpand\MessageBreak
        \ZREF@pl@ListEntry{paperwidth}%
        \ZREF@pl@ListEntry{paperheight}%
        \ZREF@pl@ListEntry{stockwidth}%
        \ZREF@pl@ListEntry{stockheight}%
        \ZREF@pl@ListEntry{mediawidth}%
        \ZREF@pl@ListEntry{mediaheight}%
        \ZREF@pl@ListEntry{pdfpagewidth}%
        \ZREF@pl@ListEntry{pdfpageheight}%
        \ZREF@pl@ListEntry{pdfhorigin}%
        \ZREF@pl@ListEntry{pdfvorigin}%
        \ZREF@pl@ListEntry{hoffset}%
        \ZREF@pl@ListEntry{voffset}%
        \ZREF@pl@ListEntry{topmargin}%
        \ZREF@pl@ListEntry{oddsidemargin}%
        \ZREF@pl@ListEntry{evensidemargin}%
        \ZREF@pl@ListEntry{textwidth}%
        \ZREF@pl@ListEntry{textheight}%
        \ZREF@pl@ListEntry{headheight}%
        \ZREF@pl@ListEntry{headsep}%
        \ZREF@pl@ListEntry{footskip}%
        \ZREF@pl@ListEntry{marginparwidth}%
        \ZREF@pl@ListEntry{marginparsep}%
        \ZREF@pl@ListEntry{columnwidth}%
        \ZREF@pl@ListEntry{columnsep}%
        \ZREF@pl@ListEntry{trimedge}%
        \ZREF@pl@ListEntry{spinemargin}%
        \ZREF@pl@ListEntry{foremargin}%
        \ZREF@pl@ListEntry{trimtop}%
        \ZREF@pl@ListEntry{uppermargin}%
        \ZREF@pl@ListEntry{headmargin}%
      }%
    }%
  }\x
  \ifnum\ZREF@page@max>\count@
    \advance\count@ by\ltx@one
  \else
    \expandafter\ltx@gobble
  \fi
  \ZREF@pl@ListPage
}
%    \end{macrocode}
%    \end{macro}
%    \begin{macro}{\ZREF@pl@ListEntry}
%    \begin{macrocode}
\def\ZREF@pl@ListEntry#1{%
  \zref@ifpropundefined{#1}{%
  }{%
    \zref@ifrefcontainsprop{thepage\the\count@}{#1}{%
      \ltx@space\ltx@space#1 = %
      \zref@extract{thepage\the\count@}{#1}sp = %
      \the\dimexpr\zref@extract{thepage\the\count@}{#1}sp\relax
      \noexpand\MessageBreak
    }{}%
  }%
}
%    \end{macrocode}
%    \end{macro}
%    \begin{macrocode}
\AfterLastShipout{%
  \ZREF@pl@AfterLastShipout
}
%    \end{macrocode}
%
%    \begin{macrocode}
%</pagelayout>
%    \end{macrocode}
%
%
% \subsection{Module \xmodule{pageattr}}
%
%    \begin{macrocode}
%<*pageattr>
\NeedsTeXFormat{LaTeX2e}
\ProvidesPackage{zref-pageattr}%
  [2020-07-03 v2.32 Module pageattr for zref (HO)]%
\RequirePackage{zref-base}[2019/11/29]
\ifx\ZREF@base@ok Y%
\else
  \expandafter\endinput
\fi
%    \end{macrocode}
%
%    \begin{macrocode}
\RequirePackage{iftex}[2019/11/07]%
%    \end{macrocode}
%
%
%    \begin{macrocode}
\let\ZREF@temp=N%
\ifluatex
\expandafter\@firstoftwo
\else
\expandafter\@secondoftwo
\fi
{%luatex
 \RequirePackage{zref-thepage}[2019/11/29]
 \RequirePackage{zref-lastpage}[2019/11/29]%
 \zref@newprop*{pdfpageattr}[]{\zref@hex{\the\pdfvariable pageattr}}%
 \zref@addprop{thepage}{pdfpageattr}%
 \zref@newprop*{pdfpagesattr}[]{\zref@hex{\the\pdfvariable pagesattr}}%
 \zref@addprop{LastPage}{pdfpagesattr}%
 \let\ZREF@temp=Y%
}
{%other
 \ltx@IfUndefined{pdfpageattr}{%
  \@PackageInfoNoLine{zref-pageattr}{%
    \string\pdfpageattr\space is not available%
  }%
  \def\zref@pdfpageattr#1{}%
  \def\zref@pdfpageattr@used#1{}%
 }{%
  \RequirePackage{zref-thepage}[2019/11/29]%
  \zref@newprop*{pdfpageattr}[]{\zref@hex{\the\pdfpageattr}}%
  \zref@addprop{thepage}{pdfpageattr}%
  \let\ZREF@temp=Y%
 }
 \ltx@IfUndefined{pdfpagesattr}{%
  \@PackageInfoNoLine{zref-pageattr}{%
    \string\pdfpagesattr\space is not available%
  }%
  \def\zref@pdfpagesattr{}%
  \def\zref@pdfpagesattr@used{}%
 }{%
  \RequirePackage{zref-lastpage}[2019/11/29]%
  \zref@newprop*{pdfpagesattr}[]{\zref@hex{\the\pdfpagesattr}}%
  \zref@addprop{LastPage}{pdfpagesattr}%
  \let\ZREF@temp=Y%
 }%
}%
%    \end{macrocode}
%    \begin{macrocode}
\ifx\ZREF@temp N%
  \expandafter\endinput
\fi
%    \end{macrocode}
%    \begin{macrocode}
\RequirePackage{zref-abspage}[2019/11/29]
\RequirePackage{atveryend}[2010/03/24]
\RequirePackage{pdftexcmds}[2010/04/01]
\let\ZREF@temp=Y%
\ltx@IfUndefined{pdf@escapehex}{\let\ZREF@temp=N}{}
\ltx@IfUndefined{pdf@unescapehex}{\let\ZREF@temp=N}{}
\ifx\ZREF@temp N%
  \let\zref@hex\ltx@firstofone
  \let\zref@unhex\ltx@firstofone
\else
  \let\zref@hex\pdf@escapehex
  \let\zref@unhex\pdf@unescapehex
\fi
%    \end{macrocode}
%
%    \begin{macro}{\ifZREF@pa@list}
%    \begin{macrocode}
\ltx@newif\ifZREF@pa@list
%    \end{macrocode}
%    \end{macro}
%    \begin{macro}{\zref@listpageattr}
%    \begin{macrocode}
\ZREF@IfDefinable\zlistpageattr\def{%
  {\ZREF@pa@listtrue}%
}
%    \end{macrocode}
%    \end{macro}
%    \begin{macro}{\ZREF@pa@AfterLastShipout}
%    \begin{macrocode}
\def\ZREF@pa@AfterLastShipout{%
  \ifZREF@pa@list
    \edef\ZREF@page@max{\the\value{abspage}}%
    \ltx@ifundefined{ZREF@org@testdef}{%
      \let\ZREF@org@testdef\@testdef
      \def\@testdef##1##2##3{%
        \ZREF@org@testdef{##1}{##2}{##3}%
        \def\ZREF@temp{##1}%
        \ifx\ZREF@temp\ZREF@RefPrefix
          \expandafter\xdef\csname##1@##2\endcsname{##3}%
        \fi
      }%
    }{}%
    \AtVeryEndDocument{\ZREF@pa@AtVeryEnd}%
  \fi
}
%    \end{macrocode}
%    \end{macro}
%    \begin{macro}{\ZREF@pa@AtVeryEnd}
%    \begin{macrocode}
\let\ZREF@temp=Y%
\ltx@IfUndefined{pdfpageattr}{}{\let\ZREF@temp=N}
\ifluatex \let\ZREF@temp=N \fi
\ifx\ZREF@temp Y
 \expandafter\@firstoftwo
\else
 \expandafter\@secondoftwo
\fi
 {%
  \def\ZREF@pa@AtVeryEnd{}%
 }
 {%
  \def\ZREF@pa@AtVeryEnd{%
    \begingroup
      \toks@{List of \ltx@backslashchar
             \ifluatex pdfvariable\else pdf\fi
             pdfpageattr:\MessageBreak}%
      \count@=1 %
      \ZREF@pa@ListPage
      \edef\x{\endgroup
        \noexpand\@PackageInfoNoLine{zref-pageattr}{%
          \the\toks@
        }%
      }%
    \x
  }%
%    \end{macrocode}
%    \end{macro}
%    \begin{macro}{\zref@pageattr}
%    \begin{macrocode}
\def\zref@pdfpageattr#1{%
  \zref@unhex{%
    \zref@extract{thepage\ZREF@number{#1}}{pdfpageattr}%
  }%
}
% compability, \zref@pageattr was defined in older versions
\let\zref@pageattr\zref@pdfpageattr
%    \end{macrocode}
%    \end{macro}
%    \begin{macro}{\zref@pageattr@used}
%    \begin{macrocode}
\ZREF@Robust\def\zref@pageattr@used#1{%
  \zref@refused{thepage\ZREF@number{#1}}%
}
%    \end{macrocode}
%    \end{macro}
%    \begin{macro}{\ZREF@pa@ListPage}
%    \begin{macrocode}
  \def\ZREF@pa@ListPage{%
    \edef\x{%
      \toks@={%
        \the\toks@
        Page \the\count@:%
        \noexpand\MessageBreak
        \zref@ifrefundefined{thepage\the\count@}{}{%
          <<\zref@pdfpageattr\count@>>%
          \noexpand\MessageBreak
        }%
      }%
    }\x
    \ifnum\ZREF@page@max>\count@
      \advance\count@ by\ltx@one
    \else
      \expandafter\ltx@gobble
    \fi
    \ZREF@pa@ListPage
  }%
}
%    \end{macrocode}
%    \end{macro}
%    \begin{macrocode}
\let\ZREF@temp=Y%
\ltx@IfUndefined{pdfpagesattr}{}{\let\ZREF@temp=N}
\ifluatex \let\ZREF@temp=N \fi
\ifx\ZREF@temp N
 \expandafter\@firstofone
\fi
 {%
%    \end{macrocode}
%    \begin{macro}{\zref@pdfpagesattr}
%    \begin{macrocode}
  \def\zref@pdfpagesattr{%
    \zref@unhex{%
      \zref@extract{LastPage}{pdfpagesattr}%
    }%
  }%
%    \end{macrocode}
%    \end{macro}
%    \begin{macro}{\zref@pdfpagesattr@used}
%    \begin{macrocode}
  \ZREF@Robust\def\zref@pdfpagesattr@used{%
    \zref@refused{LastPage}%
  }%
%    \end{macrocode}
%    \end{macro}
%    \begin{macrocode}
  \ltx@LocalAppendToMacro\ZREF@pa@AtVeryEnd{%
    \@PackageInfoNoLine{zref-pageattr}{%
      \ltx@backslashchar
      \ifluatex pdfvariable\else pdf\fi
      pagesattr:\MessageBreak
      <<\zref@pdfpagesattr>>%
      \MessageBreak
    }%
  }%
}
\AfterLastShipout{%
  \ZREF@pa@AfterLastShipout
}
%    \end{macrocode}
%    \begin{macrocode}
%</pageattr>
%    \end{macrocode}
%
%
% \subsection{Module \xmodule{marks}}
%
%    \begin{macrocode}
%<*marks>
\NeedsTeXFormat{LaTeX2e}
\ProvidesPackage{zref-marks}%
  [2020-07-03 v2.32 Module marks for zref (HO)]%
\RequirePackage{zref-base}[2019/11/29]
\ifx\ZREF@base@ok Y%
\else
  \expandafter\endinput
\fi
%    \end{macrocode}
%
%    \begin{macrocode}
\newcommand*{\zref@marks@register}[3][]{%
  \edef\ZREF@TempName{#1}%
  \edef\ZREF@TempNum{\ZREF@number{#2}}%
  \ifnum\ZREF@TempNum<\ltx@zero %
    \PackageError\ZREF@name{%
      \string\zref@marks@register\ltx@space is called with invalid%
      \MessageBreak
      marks register number (\ZREF@TempNum)%
    }{%
      Use `0' or the command, defined by \string\newmarks.\MessageBreak
      \@ehc
    }%
  \else
    \ifx\ZREF@TempName\ltx@empty
      \edef\ZREF@TempName{mark\romannumeral\ZREF@TempNum}%
    \else
      \edef\ZREF@TempName{marks\ZREF@TempName}%
    \fi
    \ZREF@MARKS@DefineProp{top}%
    \ZREF@MARKS@DefineProp{first}%
    \ZREF@MARKS@DefineProp{bot}%
    \kv@parse{#3}{%
      \ifx\kv@value\relax
        \def\kv@value{top,first,bot}%
      \fi
      \edef\ZREF@temp{\expandafter\ltx@car\kv@key X\@nil}%
      \ifx\ZREF@temp\ZREF@STAR
        \edef\kv@key{\expandafter\ltx@cdr\kv@key\@nil}%
        \zref@newlist\kv@key
      \fi
      \expandafter\comma@parse\expandafter{\kv@value}{%
        \ifcase0\ifx\comma@entry\ZREF@NAME@top 1\else
                \ifx\comma@entry\ZREF@NAME@first 1\else
                \ifx\comma@entry\ZREF@NAME@bot 1\fi\fi\fi\ltx@space
          \PackageWarning{zref-marks}{%
            Use `top', `first' or `bot' for the list values%
            \MessageBreak
            in the third argument of \string\zref@marks@register.%
            \MessageBreak
            Ignoring unkown value `\comma@entry'%
          }%
        \else
          \zref@addprop{\kv@key}{\comma@entry\ZREF@TempName}%
        \fi
        \ltx@gobble
      }%
      \ltx@gobbletwo
    }%
  \fi
}
\def\ZREF@STAR{*}
\def\ZREF@NAME@top{top}
\def\ZREF@NAME@first{first}
\def\ZREF@NAME@bot{bot}
\def\ZREF@MARKS@DefineProp#1{%
  \zref@ifpropundefined{#1\ZREF@TempName}{%
    \ifnum\ZREF@TempNum=\ltx@zero
      \begingroup
        \edef\x{\endgroup
          \noexpand\zref@newprop*{#1\ZREF@TempName}[]{%
            \expandafter\noexpand\csname#1mark\endcsname
          }%
        }%
      \x
    \else
      \begingroup
        \edef\x{\endgroup
          \noexpand\zref@newprop*{#1\ZREF@TempName}[]{%
            \expandafter\noexpand\csname#1marks\endcsname
            \ZREF@TempNum
          }%
        }%
      \x
    \fi
  }{%
    \PackageWarning{zref-marks}{%
      \string\zref@marks@register\ltx@space does not generate the%
      \MessageBreak
      new property `#1\ZREF@TempName', because\MessageBreak
      it is already defined%
    }%
  }%
}
%    \end{macrocode}
%    \begin{macrocode}
%</marks>
%    \end{macrocode}
%
% \subsection{Module \xmodule{runs}}
%
%    This module does not use the label-reference-system.
%    The reference changes with each \hologo{LaTeX}~run and would
%    force a rerun warning always.
%    \begin{macrocode}
%<*runs>
\NeedsTeXFormat{LaTeX2e}
\ProvidesPackage{zref-runs}%
  [2020-07-03 v2.32 Module runs for zref (HO)]%
%    \end{macrocode}
%
%    \begin{macro}{\zruns}
%    \begin{macrocode}
\providecommand*{\zruns}{0}%
\AtBeginDocument{%
  \edef\zruns{\number\numexpr\zruns+1}%
  \begingroup
    \def\on@line{}%
    \PackageInfo{zref-runs}{LaTeX runs: \zruns}%
    \if@filesw
      \immediate\write\@mainaux{%
        \string\gdef\string\zruns{\zruns}%
      }%
    \fi
  \endgroup
}
%    \end{macrocode}
%    \end{macro}
%
%    \begin{macrocode}
%</runs>
%    \end{macrocode}
%
%
% \subsection{Module \xmodule{perpage}}
%
%    \begin{macrocode}
%<*perpage>
\NeedsTeXFormat{LaTeX2e}
\ProvidesPackage{zref-perpage}%
  [2020-07-03 v2.32 Module perpage for zref (HO)]%
\RequirePackage{zref-base}[2019/11/29]
\ifx\ZREF@base@ok Y%
\else
  \expandafter\endinput
\fi
%    \end{macrocode}
%
%    This module resets a counter at page boundaries.
%    Because of the asynchronous output routine
%    page counter properties cannot be asked directly,
%    references are necessary.
%
%    For detecting changed pages module \xmodule{abspage}
%    is loaded.
%    \begin{macrocode}
\RequirePackage{zref-abspage}[2019/11/29]
%    \end{macrocode}
%
%    We group the properties for the needed references
%    in the property list |perpage|. The property |pagevalue|
%    records the correct value of the page counter.
%    \begin{macrocode}
\ZREF@NewPropPageValue
\zref@newlist{perpage}
\zref@addprops{perpage}{abspage,page,pagevalue}
%    \end{macrocode}
%    The page value, known by the reference mechanism, will be
%    stored in counter |zpage|.
%    \begin{macrocode}
\newcounter{zpage}
%    \end{macrocode}
%    Counter |zref@unique| helps in generating unique reference
%    names.
%    \begin{macrocode}
\zref@require@unique
%    \end{macrocode}
%    In order to be able to reset the counter, we hook here
%    into \cs{stepcounter}. In fact two nested hooks are used
%    to allow other packages to use the first hook at the beginning
%    of \cs{stepcounter}.
%    \begin{macrocode}
\let\ZREF@org@stepcounter\stepcounter
\def\stepcounter#1{%
  \ifcsname @stepcounterhook@#1\endcsname
    \csname @stepcounterhook@#1\endcsname
  \fi
  \ZREF@org@stepcounter{#1}%
}
%    \end{macrocode}
%    \cs{@stpelt} must be adapted due to the change in latex
%    2015-01, see https://github.com/ho-tex/zref/issues/26
%    \begin{macrocode}
\let\ZREF@org@@stpelt\@stpelt
\def\@stpelt#1{%
  \ifcsname ZREF@perpage@#1\endcsname
    \begingroup
      \let\stepcounter\ZREF@org@stepcounter
      \ZREF@org@@stpelt{#1}%
    \endgroup
    \expandafter\ltx@gobbletwo
  \fi
  \ZREF@org@@stpelt{#1}%
}
%    \end{macrocode}
%
%    \begin{macro}{\zmakeperpage}
%    Makro \cs{zmakeperpage} resets a counter at each page
%    break. It uses the same syntax and semantics as
%    \cs{MakePerPage} from package \xpackage{perpage} \cite{perpage}.
%    The initial start value can be given by the optional
%    argument. Default is one that means after the first
%    \cs{stepcounter} on a new page the counter starts with
%    one.
%    \begin{macrocode}
\ZREF@IfDefinable\zmakeperpage\def{%
  {%
    \@ifnextchar[\ZREF@makeperpage@opt{\ZREF@@makeperpage[\ltx@zero]}%
  }%
}
%    \end{macrocode}
%    \end{macro}
%    We hook before the counter is incremented in \cs{stepcounter},
%    package \xpackage{perpage} afterwards. Thus a little
%    calculation is necessary.
%    \begin{macrocode}
\def\ZREF@makeperpage@opt[#1]{%
  \begingroup
    \edef\x{\endgroup
      \noexpand\ZREF@@makeperpage[\number\numexpr#1-1\relax]%
    }%
  \x
}
%    \end{macrocode}
%    \begin{macrocode}
\def\ZREF@@makeperpage[#1]#2{%
  \@ifundefined{@stepcounterhook@#2}{%
    \expandafter\gdef\csname @stepcounterhook@#2\endcsname{}%
  }{}%
  \expandafter\gdef\csname ZREF@perpage@#2\endcsname{%
    \ZREF@@perpage@step{#2}{#1}%
  }%
  \expandafter\g@addto@macro\csname @stepcounterhook@#2\endcsname{%
    \ifcsname ZREF@perpage@#2\endcsname
      \csname ZREF@perpage@#2\endcsname
    \fi
  }%
}
%    \end{macrocode}
%    \begin{macro}{\ZREF@@perpage@step}
%    The heart of this module follows.
%    \begin{macrocode}
\def\ZREF@@perpage@step#1#2{%
%    \end{macrocode}
%    First the reference is generated.
%    \begin{macrocode}
  \global\advance\c@zref@unique\ltx@one
  \begingroup
    \expandafter
    \zref@labelbylist\expandafter{\thezref@unique}{perpage}%
%    \end{macrocode}
%    The \cs{expandafter} commands are necessary, because \cs{ZREF@temp}
%    is also used inside of \cs{zref@labelbylist}.
%
%    The evaluation of the reference follows. If the reference
%    is not yet kwown, we use the page counter as approximation.
%    \begin{macrocode}
    \zref@ifrefundefined\thezref@unique{%
      \global\c@zpage=\c@page
      \global\let\thezpage\thepage
      \expandafter\xdef\csname ZREF@abspage@#1\endcsname{%
        \number\c@abspage
      }%
    }{%
%    \end{macrocode}
%    The reference is used to set \cs{thezpage} and counter |zpage|.
%    \begin{macrocode}
      \global\c@zpage=\zref@extract\thezref@unique{pagevalue}\relax
      \xdef\thezpage{\noexpand\zref@extract{\thezref@unique}{page}}%
      \expandafter\xdef\csname ZREF@abspage@#1\endcsname{%
        \zref@extractdefault\thezref@unique
            {abspage}{\number\c@abspage}%
      }%
    }%
%    \end{macrocode}
%    Page changes are detected by a changed absolute page number.
%    \begin{macrocode}
    \expandafter\ifx\csname ZREF@abspage@#1\expandafter\endcsname
                    \csname ZREF@currentabspage@#1\endcsname
    \else
      \global\csname c@#1\endcsname=#2\relax
      \global\expandafter\let
          \csname ZREF@currentabspage@#1\expandafter\endcsname
          \csname ZREF@abspage@#1\endcsname
    \fi
  \endgroup
}
%    \end{macrocode}
%    \end{macro}
%    \begin{macro}{\zunmakeperpage}
%    Macro \cs{zunmakeperpage} cancels the effect of \cs{zmakeperpage}.
%    \begin{macrocode}
\ZREF@IfDefinable\zunmakeperpage\def{%
  #1{%
    \global\expandafter
    \let\csname ZREF@perpage@#1\endcsname\@undefined
  }%
}
%    \end{macrocode}
%    \end{macro}
%
%    \begin{macrocode}
%</perpage>
%    \end{macrocode}
%
%
% \subsection{Module \xmodule{titleref}}
%
%    \begin{macrocode}
%<*titleref>
\NeedsTeXFormat{LaTeX2e}
\ProvidesPackage{zref-titleref}%
  [2020-07-03 v2.32 Module titleref for zref (HO)]%
\RequirePackage{zref-base}[2019/11/29]
\ifx\ZREF@base@ok Y%
\else
  \expandafter\endinput
\fi
\RequirePackage{gettitlestring}[2009/12/08]
%    \end{macrocode}
%
% \subsubsection{Implementation}
%    \begin{macrocode}
\RequirePackage{keyval}
%    \end{macrocode}
%
%    This module makes section and caption titles available
%    for the reference system. It uses some of the ideas
%    of package \xpackage{nameref} and \xpackage{titleref}.
%
%    Now we can add the property \verb|title| is added to
%    the main property list.
%    \begin{macrocode}
\ZREF@NewPropTitle
\zref@addprop\ZREF@mainlist{title}%
%    \end{macrocode}
%
%    The title strings go into the \xfile{.aux} file, thus
%    they need some kind of protection.
%    Package \xpackage{titleref} uses a protected expansion
%    method. The advantage is that this can be used to
%    cleanup the string and to remove \cs{label}, \cs{index}
%    and other macros unwanted for referencing.
%    But there is the risk that fragile stuff can break.
%
%    Therefore package \xpackage{nameref} does not expand
%    the string. Thus the entries can safely be written to
%    the \xfile{.aux} file. But potentially dangerous
%    macros such as \cs{label} remain in the string and
%    can cause problems when using the string in references.
%
%    \begin{macro}{\ifzref@titleref@expand}
%    The switch \cs{ifzref@titleref@expand} distinguishes
%    between the both methods. Package \xpackage{nameref}'s
%    behaviour is achieved by setting the switch to false,
%    otherwise \xpackage{titleref}'s expansion is used.
%    Default is false.
%    \begin{macrocode}
\newif\ifzref@titleref@expand
%    \end{macrocode}
%    \end{macro}
%
%    \begin{macro}{\ZREF@titleref@hook}
%    The hook \cs{ZREF@titleref@hook} allows to extend
%    the cleanup for the expansion method. Thus
%    unnecessary macros can be removed or dangerous commands
%    removed. The hook is executed before the expansion of
%    \cs{zref@titleref@current}.
%    \begin{macrocode}
\let\ZREF@titleref@hook\ltx@empty
%    \end{macrocode}
%    \end{macro}
%    \begin{macro}{\zref@titleref@cleanup}
%    The hook should not be used directly, instead we provide the
%    macro \cs{zref@titleref@cleanup} to add stuff to the hook
%    and prevents that a previous non-empty content is not discarded
%    accidently.
%    \begin{macrocode}
\ZREF@Robust\def\zref@titleref@cleanup#1{%
   \begingroup
    \toks@\expandafter{%
      \ZREF@titleref@hook
      #1%
    }%
  \expandafter\endgroup
  \expandafter\def\expandafter\ZREF@titleref@hook\expandafter{%
    \the\toks@
  }%
}%
%    \end{macrocode}
%    \end{macro}
%
%    \begin{macro}{\ifzref@titleref@stripperiod}
%    Sometimes a title contains a period at the end.
%    Package \xpackage{nameref} removes this. This behaviour
%    is controlled by the switch \cs{ifzref@titleref@stripperiod}
%    and works regardless of the setting of option \xoption{expand}.
%    Period stripping is the default.
%    \begin{macrocode}
\newif\ifzref@titleref@stripperiod
\zref@titleref@stripperiodtrue
%    \end{macrocode}
%    \end{macro}
%
%    \begin{macro}{\zref@titleref@setcurrent}
%    Macro \cs{zref@titleref@setcurrent} sets a new current title
%    stored in \cs{zref@titleref@current}. Some cleanup and
%    expansion is performed that can be controlled by the
%    previous switches.
%    \begin{macrocode}
\ZREF@Robust\def\zref@titleref@setcurrent#1{%
  \ifzref@titleref@expand
    \GetTitleStringExpand{#1}%
  \else
    \GetTitleStringNonExpand{#1}%
  \fi
  \edef\zref@titleref@current{%
    \detokenize\expandafter{\GetTitleStringResult}%
  }%
  \ifzref@titleref@stripperiod
    \edef\zref@titleref@current{%
      \expandafter\ZREF@stripperiod\zref@titleref@current
      \ltx@empty.\ltx@empty\@nil
    }%
  \fi
}%
\GetTitleStringDisableCommands{%
  \ZREF@titleref@hook
}
%    \end{macrocode}
%    \end{macro}
%    \begin{macro}{\ZREF@stripperiod}
%    If \cs{ZREF@stripperiod} is called, the argument consists of
%    space tokens and tokens with catcode 12 (other), because
%    of \hologo{eTeX}'s \cs{detokenize}.
%    \begin{macrocode}
\def\ZREF@stripperiod#1.\ltx@empty#2\@nil{#1}%
%    \end{macrocode}
%    \end{macro}
%
% \subsubsection{User interface}
%
%    \begin{macro}{\ztitlerefsetup}
%    The behaviour of module \xmodule{titleref} is controlled
%    by switches and a hook. They can be set by \cs{ztitlerefsetup}
%    with a key value interface, provided by package \xpackage{keyval}.
%    Also the current title can be given explicitly by the key
%    \verb|title|.
%    \begin{macrocode}
\define@key{ZREF@TR}{expand}[true]{%
  \csname zref@titleref@expand#1\endcsname
}%
\define@key{ZREF@TR}{stripperiod}[true]{%
  \csname zref@titleref@stripperiod#1\endcsname
}%
\define@key{ZREF@TR}{cleanup}{%
  \zref@titleref@cleanup{#1}%
}%
\define@key{ZREF@TR}{title}{%
  \def\zref@titleref@current{#1}%
}%
\ZREF@IfDefinable\ztitlerefsetup\def{%
  {\kvsetkeys{ZREF@TR}}%
}%
%    \end{macrocode}
%    \end{macro}
%
%    \begin{macro}{\ztitleref}
%    The user command \cs{ztitleref} references the title.
%    For safety \cs{label} is disabled to prevent multiply
%    defined references.
%    \begin{macrocode}
\ZREF@IfDefinable\ztitleref\def{%
  {\zref@wrapper@babel\ZREF@titleref}%
}%
\def\ZREF@titleref#1{%
  \begingroup
    \zref@refused{#1}%
    \let\label\ltx@gobble
    \zref@extract{#1}{title}%
  \endgroup
}%
%    \end{macrocode}
%    \end{macro}
%
% \subsubsection{Patches for section and caption commands}
%
%    The section and caption macros are patched to extract the
%    title data.
%
%    Captions of figures and tables.
%    \begin{macrocode}
\AtBeginDocument{%
  \ZREF@patch{@caption}{%
    \long\def\@caption#1[#2]{%
      \zref@titleref@setcurrent{#2}%
      \ZREF@org@@caption{#1}[{#2}]%
    }%
  }%
%    \end{macrocode}
%
%    Section commands without star. The title version for the
%    table of contents is used because it is usually shorter and
%    more robust.
%    \begin{macrocode}
  \ZREF@patch{@part}{%
    \def\@part[#1]{%
      \zref@titleref@setcurrent{#1}%
      \ZREF@org@@part[{#1}]%
    }%
  }%
  \ZREF@patch{@chapter}{%
    \def\@chapter[#1]{%
      \zref@titleref@setcurrent{#1}%
      \ZREF@org@@chapter[{#1}]%
    }%
  }%
  \ZREF@patch{@sect}{%
    \def\@sect#1#2#3#4#5#6[#7]{%
      \zref@titleref@setcurrent{#7}%
      \ZREF@org@@sect{#1}{#2}{#3}{#4}{#5}{#6}[{#7}]%
    }%
  }%
%    \end{macrocode}
%    The star versions of the section commands.
%    \begin{macrocode}
  \ZREF@patch{@spart}{%
    \def\@spart#1{%
      \zref@titleref@setcurrent{#1}%
      \ZREF@org@@spart{#1}%
    }%
  }%
  \ZREF@patch{@schapter}{%
    \def\@schapter#1{%
      \zref@titleref@setcurrent{#1}%
      \ZREF@org@@schapter{#1}%
    }%
  }%
  \ZREF@patch{@ssect}{%
    \def\@ssect#1#2#3#4#5{%
      \zref@titleref@setcurrent{#5}%
      \ZREF@org@@ssect{#1}{#2}{#3}{#4}{#5}%
    }%
  }%
%    \end{macrocode}
%
% \subsubsection{Environment \textsf{description}}
%
%    \begin{macrocode}
  \ZREF@patch{descriptionlabel}{%
    \def\descriptionlabel#1{%
      \zref@titleref@setcurrent{#1}%
      \ZREF@org@descriptionlabel{#1}%
    }%
  }%
%    \end{macrocode}
%
% \subsubsection{Class \xclass{memoir}}
%
%    \begin{macrocode}
  \@ifclassloaded{memoir}{%
    \ltx@IfUndefined{ifheadnameref}{}{%
      \def\@chapter[#1]#2{%
        \ltx@IfUndefined{ch@pt@c}{%
          \zref@titleref@setcurrent{#1}%
        }{%
          \ifx\ch@pt@c\ltx@empty
            \zref@titleref@setcurrent{#2}%
          \else
            \def\NR@temp{#1}%
            \ifx\NR@temp\ltx@empty
              \expandafter\zref@titleref@setcurrent
              \expandafter{\ch@pt@c}%
            \else
              \ifheadnameref
                \zref@titleref@setcurrent{#1}%
              \else
                \expandafter\zref@titleref@setcurrent
                \expandafter{\ch@pt@c}%
              \fi
            \fi
          \fi
        }%
        \ZREF@org@@chapter[{#1}]{#2}%
      }%
      \ZREF@patch{M@sect}{%
        \def\M@sect#1#2#3#4#5#6[#7][#8]{%
          \ifheadnameref
            \zref@titleref@setcurrent{#8}%
          \else
            \zref@titleref@setcurrent{#7}%
          \fi
          \ZREF@org@M@sect{#1}{#2}{#3}{#4}{#5}{#6}[{#7}][{#8}]%
        }%
      }%
    }%
  }{}%
%    \end{macrocode}
%
% \subsubsection{Class \xclass{beamer}}
%
%    \begin{macrocode}
  \@ifclassloaded{beamer}{%
    \ZREF@patch{beamer@section}{%
      \long\def\beamer@section[#1]{%
        \zref@titleref@setcurrent{#1}%
        \ZREF@org@beamer@section[{#1}]%
      }%
    }%
    \ZREF@patch{beamer@subsection}{%
      \long\def\beamer@subsection[#1]{%
        \zref@titleref@setcurrent{#1}%
        \ZREF@org@beamer@subsection[{#1}]%
      }%
    }%
    \ZREF@patch{beamer@subsubsection}{%
      \long\def\beamer@subsubsection[#1]{%
        \zref@titleref@setcurrent{#1}%
        \ZREF@org@beamer@subsubsection[{#1}]%
      }%
    }%
  }{}%
%    \end{macrocode}
%
% \subsubsection{Package \xpackage{titlesec}}
%
%    \begin{macrocode}
  \@ifpackageloaded{titlesec}{%
    \ZREF@patch{ttl@sect@i}{%
      \def\ttl@sect@i#1#2[#3]#4{%
        \zref@titleref@setcurrent{#4}%
        \ZREF@org@ttl@sect@i{#1}{#2}[{#3}]{#4}%
      }%
    }%
    \ZREF@patch{ttl@straight@i}{%
      \def\ttl@straight@i#1[#2]#3{%
        \def\ZREF@temp{#2}%
        \ifx\ZREF@temp\ltx@empty
          \zref@titleref@setcurrent{#3}%
        \else
          \zref@titleref@setcurrent{#2}%
        \fi
        \ZREF@org@ttl@straight@i{#1}[{#2}]{#3}%
      }%
    }%
  }{}%
%    \end{macrocode}
%
% \subsubsection{Package \xpackage{longtable}}
%
%    Package \xpackage{longtable}: some support for its \cs{caption}.
%    However \cs{label} inside the caption is not supported.
%    \begin{macrocode}
  \@ifpackageloaded{longtable}{%
    \ZREF@patch{LT@c@ption}{%
      \def\LT@c@ption#1[#2]#3{%
        \ZREF@org@LT@c@ption{#1}[{#2}]{#3}%
        \zref@titleref@setcurrent{#2}%
      }%
    }%
  }{}%
%    \end{macrocode}
%
% \subsubsection{Package \xpackage{listings}}
%
%    Package \xpackage{listings}: support for its caption.
%    \begin{macrocode}
  \@ifpackageloaded{listings}{%
    \ZREF@patch{lst@MakeCaption}{%
      \def\lst@MakeCaption{%
        \ifx\lst@label\ltx@empty
        \else
          \expandafter\zref@titleref@setcurrent\expandafter{%
            \lst@@caption
          }%
        \fi
        \ZREF@org@lst@MakeCaption
      }%
    }%
  }{}%
%    \end{macrocode}
%
% \subsubsection{Theorems}
%
%    \begin{macrocode}
  \ZREF@patch{@opargbegintheorem}{%
    \def\@opargbegintheorem#1#2#3{%
      \zref@titleref@setcurrent{#3}%
      \ZREF@org@@opargbegintheorem{#1}{#2}{#3}%
    }%
  }%
  \@ifpackageloaded{amsthm}{%
    \begingroup
      \edef\x{macro:\string#1\string#2[\string#3]}%
      \@onelevel@sanitize\x
      \def\y#1->#2\@nil{#1}%
      \edef\z{\expandafter\y\meaning\@begintheorem->\@nil}%
      \@onelevel@sanitize\z
    \expandafter\endgroup
    \ifx\x\z
      \ZREF@patch{@begintheorem}{%
        \def\@begintheorem#1#2[#3]{%
          \zref@titleref@setcurrent{#3}%
          \ZREF@org@@begintheorem{#1}{#2}[{#3}]%
        }%
      }%
    \fi
  }{}%
%    \end{macrocode}
%
%    \begin{macrocode}
}
%    \end{macrocode}
%
%    \begin{macrocode}
%</titleref>
%    \end{macrocode}
%
%
% \subsection{Module \xmodule{xr}}
%
%    \begin{macrocode}
%<*xr>
\NeedsTeXFormat{LaTeX2e}
\ProvidesPackage{zref-xr}%
  [2020-07-03 v2.32 Module xr for zref (HO)]%
\RequirePackage{zref-base}[2019/11/29]
\ifx\ZREF@base@ok Y%
\else
  \expandafter\endinput
\fi
%    \end{macrocode}
%    \begin{macrocode}
\RequirePackage{keyval}
\RequirePackage{kvoptions}[2010/02/22]
%    \end{macrocode}
%
%    We declare property |url|, because this is added,
%    if a reference is imported and has not already set this field.
%    Or if |hyperref| is used, then this property can be asked.
%    \begin{macrocode}
\zref@newprop{url}{}%
\zref@newprop{urluse}{}%
%    \end{macrocode}
%    \begin{macrocode}
\zref@newprop{externaldocument}{}%
%    \end{macrocode}
%
%    Most code, especially the handling of the \xfile{.aux}
%    files are taken from David Carlisle's \xpackage{xr} package.
%    Therefore I drop the documentation for these macros here.
%
%    \begin{macro}{\zref@xr@ext}
%    If the URL is not specied, then assume
%    processed file with a guessed extension.
%    Use the setting of \xpackage{hyperref} if available.
%    \begin{macrocode}
\providecommand*{\zref@xr@ext}{%
  \ltx@ifundefined{XR@ext}{pdf}{\XR@ext}%
}%
%    \end{macrocode}
%    \end{macro}
%    \begin{macro}{\ifZREF@xr@zreflabel}
%    The use of the star form of \cs{zexternaldocument}
%    is remembered in the switch \cs{ifZREF@xr@zreflabel}.
%    \begin{macrocode}
\newif\ifZREF@xr@zreflabel
%    \end{macrocode}
%    \end{macro}
%    \begin{macrocode}
\SetupKeyvalOptions{%
  family=ZREF@XR,%
  prefix=ZREF@xr@%
}
\DeclareBoolOption[true]{tozreflabel}
\DeclareBoolOption[false]{toltxlabel}
\DeclareBoolOption{verbose}
\define@key{ZREF@XR}{ext}{%
  \def\zref@xr@{#1}%
}
\DeclareBoolOption[false]{urluse}
%    \end{macrocode}
%    \begin{macro}{\zxrsetup}
%    \begin{macrocode}
\newcommand*{\zxrsetup}{%
  \kvsetkeys{ZREF@XR}%
}%
%    \end{macrocode}
%    \end{macro}
%
%    \begin{macro}{\ZREF@xr@URL}
%    \begin{macrocode}
\newcount\ZREF@xr@URL
\ZREF@xr@URL=\ltx@zero
%    \end{macrocode}
%    \end{macro}
%    \begin{macro}{\ZREF@xr@AddURL}
%    \begin{macrocode}
\def\ZREF@xr@AddURL#1{%
  \begingroup
    \def\ZREF@temp{#1}%
    \count@=\ltx@one
    \ZREF@xr@@AddUrl
  \endgroup
}
%    \end{macrocode}
%    \end{macro}
%    \begin{macro}{\ZREF@xr@@AddUrl}
%    \begin{macrocode}
\def\ZREF@xr@@AddUrl{%
  \ifnum\count@>\ZREF@xr@URL
    \global\advance\ZREF@xr@URL by\ltx@one
    \xdef\ZREF@xr@theURL{\romannumeral\ZREF@xr@URL}%
    \global\expandafter\let
        \csname Z@U@\ZREF@xr@theURL\endcsname\ZREF@temp
    \@PackageInfo{zref-xr}{%
      \ltx@backslashchar Z@U@\ZREF@xr@theURL:\MessageBreak
      \ZREF@temp\MessageBreak
    }%
  \else
    \expandafter
    \ifx\csname Z@U@\romannumeral\count@\endcsname\ZREF@temp
      \xdef\ZREF@xr@theURL{\romannumeral\count@}%
    \else
      \expandafter\expandafter\expandafter\ZREF@xr@@AddUrl
    \fi
  \fi
}
%    \end{macrocode}
%    \end{macro}
%
%    \begin{macro}{\zexternaldocument}
%    In its star form it looks for \cs{newlabel},
%    otherwise for \cs{zref@newlabel}.
%     Later we
%    will read \xfile{.aux} files that expects
%    |@| to have catcode 11 (letter).
%    \begin{macrocode}
\ZREF@IfDefinable\zexternaldocument\def{%
  {%
    \ZREF@NewPropAnchor
    \ZREF@NewPropTitle
    \begingroup
      \csname @safe@actives@true\endcsname
      \makeatletter
      \@ifstar{%
        \ZREF@xr@zreflabelfalse
        \@testopt\ZREF@xr@externaldocument{}%
      }{%
        \ZREF@xr@zreflabeltrue
        \@testopt\ZREF@xr@externaldocument{}%
      }%
  }%
}%
%    \end{macrocode}
%    \end{macro}
%
%    If the \cs{include} featuer was used, there can be
%    several \xfile{.aux} files. These files are read
%    one after another, especially they are not recursively
%    read in order to save read registers. Thus it can
%    happen that the read order of the newlabel commands
%    differs from \hologo{LaTeX}'s order using \cs{input}.
%
%    \begin{macro}{\ZREF@xr@externaldocument}
%    It reads the remaining arguments. \cs{newcommand}
%    comes in handy for the optional argument.
%    \begin{macrocode}
\def\ZREF@xr@externaldocument[#1]#2{%
    \def\ZREF@xr@prefix{#1}%
    \let\ZREF@xr@filelist\ltx@empty
    \edef\ZREF@xr@externalfile{#2}%
    \edef\ZREF@xr@file{\ZREF@xr@externalfile.aux}%
    \filename@parse{#2}%
    \@testopt\ZREF@xr@graburl{#2.\zref@xr@ext}%
}%
\def\ZREF@xr@graburl[#1]{%
    \edef\ZREF@xr@url{#1}%
    \ifZREF@xr@urluse
      \expandafter\ZREF@xr@AddURL\expandafter{\ZREF@xr@url}%
      \expandafter\def\expandafter\ZREF@xr@url
      \expandafter{\csname Z@U@\ZREF@xr@theURL\endcsname}%
    \fi
    \ZREF@xr@checkfile
  \endgroup
}%
%    \end{macrocode}
%    \end{macro}
%    \begin{macro}{\ZREF@xr@processfile}
%    We follow \xpackage{xr} here, \cs{IfFileExists} offers
%    a nicer test, but we have to open the file anyway.
%    \begin{macrocode}
\def\ZREF@xr@checkfile{%
  \openin\@inputcheck\ZREF@xr@file\relax
  \ifeof\@inputcheck
    \PackageWarning{zref-xr}{%
      File `\ZREF@xr@file' not found or empty,\MessageBreak
      labels not imported%
    }%
  \else
    \PackageInfo{zref-xr}{%
      Label \ifZREF@xr@zreflabel (zref) \fi
      import from `\ZREF@xr@file'%
    }%
    \def\ZREF@xr@found{0}%
    \def\ZREF@xr@ignored@empty{0}%
    \def\ZREF@xr@ignored@zref{0}%
    \def\ZREF@xr@ignored@ltx{0}%
    \ZREF@xr@processfile
    \closein\@inputcheck
    \begingroup
      \let\on@line\ltx@empty
      \PackageInfo{zref-xr}{%
        Statistics for `\ZREF@xr@file':\MessageBreak
        \ZREF@xr@found\space
        \ifZREF@xr@zreflabel zref\else LaTeX\fi\space
        label(s) found%
        \ifnum\ZREF@xr@ignored@empty>0 %
          ,\MessageBreak
          \ZREF@xr@ignored@empty\space empty label(s) ignored%
        \fi
        \ifnum\ZREF@xr@ignored@zref>0 %
          ,\MessageBreak
          \ZREF@xr@ignored@zref\space
          duplicated zref label(s) ignored%
        \fi
        \ifnum\ZREF@xr@ignored@ltx>0 %
          ,\MessageBreak
          \ZREF@xr@ignored@ltx\space
          duplicated latex label(s) ignored%
        \fi
      }%
    \endgroup
  \fi
  \ifx\ZREF@xr@filelist\ltx@empty
  \else
    \edef\ZREF@xr@file{%
      \expandafter\ltx@car\ZREF@xr@filelist\@nil
    }%
    \edef\ZREF@xr@filelist{%
      \expandafter\ltx@cdr\ZREF@xr@filelist\ltx@empty\@nil
    }%
    \expandafter\ZREF@xr@checkfile
  \fi
}%
%    \end{macrocode}
%    \end{macro}
%
%    \begin{macro}{\ZREF@xr@processfile}
%    \begin{macrocode}
\def\ZREF@xr@processfile{%
  \read\@inputcheck to\ZREF@xr@line
  \expandafter\ZREF@xr@processline\ZREF@xr@line..\ZREF@nil
  \ifeof\@inputcheck
  \else
    \expandafter\ZREF@xr@processfile
  \fi
}%
%    \end{macrocode}
%    \end{macro}
%    \begin{macro}{\ZREF@xr@processline}
%    The most work must be done for analyzing the
%    arguments of \cs{newlabel}.
%    \begin{macrocode}
\long\def\ZREF@xr@processline#1#2#3\ZREF@nil{%
  \def\x{#1}%
  \toks@{#2}%
  \ifZREF@xr@zreflabel
    \ifx\x\ZREF@xr@zref@newlabel
      \expandafter
      \ZREF@xr@process@zreflabel\ZREF@xr@line...\ZREF@nil
    \fi
  \else
    \ifx\x\ZREF@xr@newlabel
      \expandafter
      \ZREF@xr@process@label\ZREF@xr@line...[]\ZREF@nil
    \fi
  \fi
  \ifx\x\ZREF@xr@@input
    \edef\ZREF@xr@filelist{%
      \etex@unexpanded\expandafter{\ZREF@xr@filelist}%
      {\filename@area\the\toks@}%
    }%
  \fi
}%
\def\ZREF@xr@process@zreflabel\zref@newlabel#1#2#3\ZREF@nil{%
  \edef\ZREF@xr@refname{Z@R@\ZREF@xr@prefix#1}%
  \edef\ZREF@xr@found{\the\numexpr\ZREF@xr@found+1\relax}%
  \def\x{#2}%
  \edef\ZREF@xr@tempname{$temp$}%
  \edef\ZREF@xr@temprefname{Z@R@\ZREF@xr@tempname}%
  \let\ZREF@xr@list\x
  \ifx\ZREF@xr@list\ltx@empty
    \PackageWarningNoLine{zref-xr}{%
      Label `#1' without properties ignored\MessageBreak
      in file `\ZREF@xr@file'%
    }%
    \edef\ZREF@xr@ignored@empty{%
      \the\numexpr\ZREF@xr@ignored@empty+1\relax
    }%
  \else
    \expandafter\ZREF@xr@checklist\x\ZREF@nil
    \expandafter\let\csname\ZREF@xr@temprefname\endcsname\x
    \expandafter\ltx@LocalAppendToMacro
    \csname\ZREF@xr@temprefname\expandafter\endcsname
    \expandafter{%
      \expandafter\externaldocument\expandafter{%
        \ZREF@xr@externalfile
      }%
    }%
    \ZREF@xr@urlcheck\ZREF@xr@tempname
    \ifZREF@xr@tozreflabel
      \@ifundefined{\ZREF@xr@refname}{%
        \ifZREF@xr@verbose
          \PackageInfo{zref-xr}{%
            Import to zref label `\ZREF@xr@tempname#1'%
          }%
        \fi
        \global\expandafter
        \let\csname\ZREF@xr@refname\expandafter\endcsname
        \csname\ZREF@xr@temprefname\endcsname
      }{%
        \ZREF@xr@zref@ignorewarning{\ZREF@xr@prefix#1}%
      }%
    \fi
    \ifZREF@xr@toltxlabel
      \ZREF@xr@tolabel{\ZREF@xr@tempname}{\ZREF@xr@prefix#1}%
    \fi
  \fi
}%
\def\ZREF@xr@process@label\newlabel#1#2#3[#4]#5\ZREF@nil{%
  \def\ZREF@xr@refname{Z@R@\ZREF@xr@prefix#1}%
  \edef\ZREF@xr@found{\the\numexpr\ZREF@xr@found+1\relax}%
  \def\x{#2}%
  \edef\ZREF@xr@tempname{$temp$}%
  \edef\ZREF@xr@temprefname{Z@R@\ZREF@xr@tempname}%
  \expandafter\ZREF@xr@scanparams
      \csname\ZREF@xr@temprefname\expandafter\endcsname
      \x{}{}{}{}{}\ZREF@nil
  \ifx\\#4\\%
  \else
    % ntheorem knows an optional argument at the end of \newlabel
    \ZREF@NewPropTheotype
    \expandafter\ltx@LocalAppendToMacro
        \csname\ZREF@xr@temprefname\endcsname{\theotype{#4}}%
  \fi
  \expandafter\ltx@LocalAppendToMacro
  \csname\ZREF@xr@temprefname\expandafter\endcsname\expandafter{%
    \expandafter\externaldocument\expandafter{%
      \ZREF@xr@externalfile
    }%
  }%
  \ZREF@xr@urlcheck\ZREF@xr@tempname
  \ifZREF@xr@tozreflabel
    \@ifundefined{\ZREF@xr@refname}{%
      \ifZREF@xr@verbose
        \PackageInfo{zref-xr}{%
          Import to zref label `\ZREF@xr@prefix#1'%
        }%
      \fi
      \global\expandafter
      \let\csname\ZREF@xr@refname\expandafter\endcsname
      \csname\ZREF@xr@temprefname\endcsname
    }{%
      \ZREF@xr@zref@ignorewarning{\ZREF@xr@prefix#1}%
    }%
  \fi
  \ifZREF@xr@toltxlabel
    \ZREF@xr@tolabel{\ZREF@xr@tempname}{\ZREF@xr@prefix#1}%
  \fi
}
\def\ZREF@xr@zref@newlabel{\zref@newlabel}%
\def\ZREF@xr@newlabel{\newlabel}%
\def\ZREF@xr@@input{\@input}%
\def\ZREF@xr@relax{\relax}%
%    \end{macrocode}
%    \end{macro}
%    \begin{macro}{\ZREF@xr@tolabel}
%    \begin{macrocode}
\def\ZREF@xr@tolabel#1#2{%
  \ifZREF@xr@verbose
    \PackageInfo{zref-xr}{%
      Import to LaTeX label `#2'%
    }%
  \fi
  \zref@wrapper@unexpanded{%
    \expandafter\xdef\csname r@#2\endcsname{%
      {%
        \ltx@ifundefined{M@TitleReference}{%
          \ltx@ifundefined{TR@TitleReference}{%
            \zref@extractdefault{#1}{default}{}%
          }{%
            \noexpand\TR@TitleReference
            {\zref@extractdefault{#1}{default}{}}%
            {\zref@extractdefault{#1}{title}{}}%
          }%
        }{%
          \noexpand\M@TitleReference
          {\zref@extractdefault{#1}{default}{}}%
          {\zref@extractdefault{#1}{title}{}}%
        }%
      }%
      {\zref@extractdefault{#1}{page}{}}%
      \ltx@ifpackageloaded{nameref}{%
        {\zref@extractdefault{#1}{title}{}}%
        {\zref@extractdefault{#1}{anchor}{}}%
        \zref@ifrefcontainsprop{#1}{urluse}{%
          {\zref@extractdefault{#1}{urluse}{}}%
        }{%
          {\zref@extractdefault{#1}{url}{}}%
        }%
      }{}%
    }%
  }%
}
%    \end{macrocode}
%    \end{macro}
%
%    \begin{macro}{\ZREF@xr@zref@ignorewarning}
%    \begin{macrocode}
\def\ZREF@xr@zref@ignorewarning#1{%
  \PackageWarningNoLine{zref-xr}{%
    Zref label `#1' is already in use\MessageBreak
    in file `\ZREF@xr@file'%
  }%
  \edef\ZREF@xr@ignored@zref{%
    \the\numexpr\ZREF@xr@ignored@zref+1%
  }%
}%
%    \end{macrocode}
%    \end{macro}
%
%    \begin{macro}{\ZREF@xr@ltx@ignorewarning}
%    \begin{macrocode}
\def\ZREF@xr@ltx@ignorewarning#1{%
  \PackageWarningNoLine{zref-xr}{%
    LaTeX label `#1' is already in use\MessageBreak
    in file `\ZREF@xr@file'%
  }%
  \edef\ZREF@xr@ignored@ltx{%
    \the\numexpr\ZREF@xr@ignored@ltx+1%
  }%
}%
%    \end{macrocode}
%    \end{macro}
%
%    \begin{macro}{\ZREF@xr@checklist}
%    \begin{macrocode}
\def\ZREF@xr@checklist#1#2#3\ZREF@nil{%
  \ifx\@undefined#1\relax
    \expandafter\ZREF@xr@checkkey\string#1\@nil
  \fi
  \ifx\\#3\\%
  \else
    \ltx@ReturnAfterFi{%
      \ZREF@xr@checklist#3\ZREF@nil
    }%
  \fi
}%
\def\ZREF@xr@checkkey#1#2\@nil{%
  \zref@ifpropundefined{#2}{%
    \zref@newprop{#2}{}%
  }{}%
}%
%    \end{macrocode}
%    \end{macro}
%
%    \begin{macro}{\ZREF@xr@scanparams}
%    \begin{macrocode}
\def\ZREF@xr@scanparams#1#2#3#4#5#6#7\ZREF@nil{%
  \let#1\ltx@empty
  \ZREF@foundfalse
  \ZREF@xr@scantitleref#1#2\TR@TitleReference{}{}\ZREF@nil
  \ifZREF@found
  \else
    \ltx@LocalAppendToMacro#1{\default{#2}}%
  \fi
  % page
  \ltx@LocalAppendToMacro#1{\page{#3}}%
  % nameref title
  \ifZREF@found
  \else
    \ifx\\#4\\%
    \else
      \def\ZREF@xr@temp{#4}%
      \ifx\ZREF@xr@temp\ZREF@xr@relax
      \else
        \ltx@LocalAppendToMacro#1{\title{#4}}%
      \fi
    \fi
  \fi
  % anchor
  \ifx\\#5\\%
  \else
    \ltx@LocalAppendToMacro#1{\anchor{#5}}%
  \fi
  \ifx\\#6\\%
  \else
    \ifZREF@xr@urluse
      \ZREF@xr@AddURL{#6}%
      \expandafter\ltx@LocalAppendToMacro\expandafter#1%
      \expandafter{%
        \expandafter\urluse\expandafter{%
          \csname Z@U@\ZREF@xr@theURL\endcsname
        }%
      }%
    \else
      \ltx@LocalAppendToMacro#1{\url{#6}}%
    \fi
  \fi
}%
%    \end{macrocode}
%    \end{macro}
%
%    \begin{macro}{\ZREF@xr@scantitleref}
%    \begin{macrocode}
\def\ZREF@xr@scantitleref#1#2\TR@TitleReference#3#4#5\ZREF@nil{%
  \ifx\\#5\\%
  \else
    \ltx@LocalAppendToMacro#1{%
      \default{#3}%
      \title{#4}%
    }%
    \ZREF@foundtrue
  \fi
}%
%    \end{macrocode}
%    \end{macro}
%
%    \begin{macro}{\ZREF@xr@urlcheck}%
%    \begin{macrocode}
\def\ZREF@xr@urlcheck#1{%
  \zref@ifrefcontainsprop{#1}{anchor}{%
    \zref@ifrefcontainsprop{#1}{url}{%
    }{%
      \expandafter
      \ltx@LocalAppendToMacro\csname Z@R@#1\expandafter\endcsname
      \expandafter{%
        \csname url\ifZREF@xr@urluse use\fi
        \expandafter\endcsname\expandafter{\ZREF@xr@url}%
      }%
    }%
  }{%
  }%
}%
%    \end{macrocode}
%    \end{macro}
%
%    \begin{macrocode}
%</xr>
%    \end{macrocode}
%
%
% \subsection{Module \xmodule{hyperref}}
%
%    UNFINISHED :-(
%
%    \begin{macrocode}
%<*hyperref>
\NeedsTeXFormat{LaTeX2e}
\ProvidesPackage{zref-hyperref}%
  [2020-07-03 v2.32 Module hyperref for zref (HO)]%
\RequirePackage{zref-base}[2019/11/29]
\ifx\ZREF@base@ok Y%
\else
  \expandafter\endinput
\fi
%    \end{macrocode}
%
%    \begin{macrocode}
\ZREF@NewPropAnchor
\zref@addprop\ZREF@mainlist{anchor}%
%    \end{macrocode}
%
%    \begin{macrocode}
%</hyperref>
%    \end{macrocode}
%
%
% \subsection{Module \xmodule{savepos}}
%
%    Module \xmodule{savepos} provides an interface for
%    \pdfTeX's \cmd{\pdfsavepos}, see the manual for
%    \pdfTeX.
%
% \subsubsection{Identification}
%
%    \begin{macrocode}
%<*savepos>
\NeedsTeXFormat{LaTeX2e}
\ProvidesPackage{zref-savepos}%
  [2020-07-03 v2.32 Module savepos for zref (HO)]%
\RequirePackage{zref-base}[2019/11/29]
\ifx\ZREF@base@ok Y%
\else
  \expandafter\endinput
\fi
%    \end{macrocode}
%
% \subsubsection{Availability}
%
%    First we check, whether the feature is available.
%    \begin{macrocode}
\ifx\directlua\@undefined
\ltx@IfUndefined{pdfsavepos}{%
  \PackageError\ZREF@name{%
    \string\pdfsavepos\space is not supported.\MessageBreak
    It is provided by pdfTeX (1.40) or XeTeX%
  }\ZREF@UpdatePdfTeX
  \endinput
}{}%
\fi
%    \end{macrocode}
%    In PDF mode we are done. However support for DVI mode
%    was added later in version 1.40.0. In earlier versions
%    \cs{pdfsavepos} is defined, but its execution raises
%    an error. Note that \hologo{XeTeX} also provides \cs{pdfsavepos}.
%    \begin{macrocode}
\ifpdf
\else
  \ltx@IfUndefined{pdftexversion}{%
  }{%
    \ifnum\pdftexversion<140 %
      \PackageError\ZREF@name{%
        \string\pdfsavepos\space is not supported in DVI mode%
        \MessageBreak
        of this pdfTeX version%
      }\ZREF@UpdatePdfTeX
      \expandafter\expandafter\expandafter\endinput
    \fi
  }%
\fi
%    \end{macrocode}
%
% \subsubsection{Setup}
%
%    \begin{macrocode}
\zref@newlist{savepos}
\ifx\directlua\@undefined
  \zref@newprop*{posx}[0]{\the\pdflastxpos}
  \zref@newprop*{posy}[0]{\the\pdflastypos}
\else
  \zref@newprop*{posx}[0]{\the\lastxpos}
  \zref@newprop*{posy}[0]{\the\lastypos}
\fi
\zref@addprops{savepos}{posx,posy}
%    \end{macrocode}
%
% \subsubsection{User macros}
%
%    \begin{macro}{\zref@savepos}
%    \begin{macrocode}
\ifx\directlua\@undefined
  \def\zref@savepos{%
    \if@filesw
      \pdfsavepos
    \fi
  }
\else
  \def\zref@savepos{%
    \if@filesw
      \savepos
    \fi
  }
\fi
%    \end{macrocode}
%    \end{macro}
%
%    \begin{macro}{\ZREF@zsavepos}
%    \begin{macrocode}
\def\ZREF@zsavepos#1#2#3{%
  \@bsphack
  \if@filesw
    \zref@savepos
    #1{#3}{#2}%
    \ltx@IfUndefined{TeXXeTstate}{%
    }{%
      \ifnum\TeXXeTstate=\ltx@zero
      \else
        \zref@savepos
      \fi
    }%
  \fi
  \@esphack
}
%    \end{macrocode}
%    \end{macro}
%
%    \begin{macro}{\zsavepos}
%    The current location is stored in a reference with the
%    given name.
%    \begin{macrocode}
\ZREF@IfDefinable\zsavepos\def{%
  {%
    \ZREF@zsavepos\zref@labelbylist{savepos}%
  }%
}
%    \end{macrocode}
%    \end{macro}
%
%    \begin{macro}{\zsaveposx}
%    \begin{macrocode}
\ZREF@IfDefinable\zsaveposx\def{%
  {%
    \ZREF@zsavepos\zref@labelbyprops{posx}%
  }%
}
%    \end{macrocode}
%    \end{macro}
%    \begin{macro}{\zsaveposy}
%    \begin{macrocode}
\ZREF@IfDefinable\zsaveposy\def{%
  {%
    \ZREF@zsavepos\zref@labelbyprops{posy}%
  }%
}
%    \end{macrocode}
%    \end{macro}
%
%    \begin{macro}{\zposx}
%    \begin{macro}{\zposy}
%    The horizontal and vertical position are available by
%    \cs{zposx} and \cs{zposy}. Do not rely on absolute
%    positions. They differ in DVI and PDF mode of \pdfTeX.
%    Use differences instead. The unit of the position numbers
%    is sp.
%    \begin{macrocode}
\newcommand*{\zposx}[1]{%
  \zref@extract{#1}{posx}%
}%
\newcommand*{\zposy}[1]{%
  \zref@extract{#1}{posy}%
}%
%    \end{macrocode}
%    \end{macro}
%    \end{macro}
%
%    Typically horizontal and vertical positions are used
%    inside calculations. Therefore the extracting macros
%    should be expandable and babel's patch is not applyable.
%
%    Also it is in the responsibility of the user to marked
%    used positions by \cs{zrefused}
%    in order to notify \hologo{LaTeX} about undefined references.
%
%    \begin{macro}{\ZREF@savepos@ok}
%    \begin{macrocode}
\let\ZREF@savepos@ok=Y
%    \end{macrocode}
%    \end{macro}
%
%    \begin{macrocode}
%</savepos>
%    \end{macrocode}
%
%
% \subsection{Module \xmodule{abspos}}
%
% \subsubsection{Identification}
%
%    \begin{macrocode}
%<*abspos>
\NeedsTeXFormat{LaTeX2e}
\ProvidesPackage{zref-abspos}%
  [2020-07-03 v2.32 Module abspos for zref (HO)]%
\RequirePackage{zref-base}[2019/11/29]
\ifx\ZREF@base@ok Y%
\else
  \expandafter\endinput
\fi
%    \end{macrocode}
%
%    \begin{macrocode}
\RequirePackage{zref-savepos}[2019/11/29]
\ifx\ZREF@savepos@ok Y%
\else
  \expandafter\endinput
\fi
%    \end{macrocode}
%    \begin{macrocode}
\RequirePackage{zref-pagelayout}[2019/11/29]
\zref@addprop{savepos}{abspage}
\zref@addprop{savepos}{pagevalue}
%    \end{macrocode}
%
%    \begin{macro}{\zref@absposx}
%    \begin{macrocode}
\newcommand*{\zref@absposx}[3]{%
  \number
    \expandafter\zref@absposnumx\expandafter{%
      \number\zref@extractdefault{#1}{abspage}{0}%
    }{#2}{#3}%
  \ltx@space
}
%    \end{macrocode}
%    \end{macro}
%
%    \begin{macro}{\zref@absposy}
%    \begin{macrocode}
\newcommand*{\zref@absposy}[3]{%
  \number
  \expandafter\zref@absposnumy\expandafter{%
    \number\zref@extractdefault{#1}{abspage}{0}%
  }{#2}{#3}%
  \ltx@space
}
%    \end{macrocode}
%    \end{macro}
%
%    \begin{macro}{\zref@absposnumx}
%    \begin{macrocode}
\newcommand*{\zref@absposnumx}[3]{%
  \number
%  \ifnum#1>\ltx@zero
%    \zref@ifrefundefined{thepage#1}{%
%      0%
%    }{%
      \numexpr\ZREF@absposnum{thepage#1}{#2}{x}{#3}\relax
%    }%
%  \else
%    0%
%  \fi
}
%    \end{macrocode}
%    \end{macro}
%
%    \begin{macro}{\zref@absposnumy}
%    \begin{macrocode}
\newcommand*{\zref@absposnumy}[3]{%
  \number
%  \ifnum#1>\ltx@zero
%    \zref@ifrefundefined{thepage#1}{%
%      0%
%    }{%
      \numexpr\ZREF@absposnum{thepage#1}{#2}{y}{#3}\relax
%    }%
%  \else
%    0%
%  \fi
}
%    \end{macrocode}
%    \end{macro}
%
%    \begin{macro}{\ZREF@absposnum}
%    \begin{macrocode}
\def\ZREF@absposnum#1#2#3#4{%
  \ltx@ifundefined{ZREF@abspos@#2@#3@#4}{%
    0%
  }{%
    \csname ZREF@abspos@#2@#3@#4\endcsname{#1}%
  }%
}
%    \end{macrocode}
%    \end{macro}
%
%    \begin{macro}{\zref@def@absposx}
%    \begin{macrocode}
\ZREF@Robust\def\zref@def@absposx#1{%
  \zref@wrapper@babel{\ZREF@def@abspos{#1}\zref@absposx}%
}
%    \end{macrocode}
%    \end{macro}
%
%    \begin{macro}{\zref@def@absposy}
%    \begin{macrocode}
\ZREF@Robust\def\zref@def@absposy#1{%
  \zref@wrapper@babel{\ZREF@def@abspos{#1}\zref@absposy}%
}
%    \end{macrocode}
%    \end{macro}
%
%    \begin{macro}{\zref@def@absposnumx}
%    \begin{macrocode}
\ZREF@Robust\def\zref@def@absposnumx#1{%
  \zref@wrapper@babel{\ZREF@def@abspos{#1}\zref@absposnumx}%
}
%    \end{macrocode}
%    \end{macro}
%
%    \begin{macro}{\zref@def@absposnumy}
%    \begin{macrocode}
\ZREF@Robust\def\zref@def@absposnumy#1{%
  \zref@wrapper@babel{\ZREF@def@abspos{#1}\zref@absposnumy}%
}
%    \end{macrocode}
%    \end{macro}
%    \begin{macro}{\ZREF@def@abspos}
%    \begin{macrocode}
\def\ZREF@def@absposnumy#1#2#3#4#5{%
  \edef#1{#2{#3}{#4}{#5}}%
}
%    \end{macrocode}
%    \end{macro}
%
%    \begin{macro}{\zref@absposused}
%    \begin{macrocode}
\ZREF@Robust\def\zref@absposused{%
  \zref@wrapper@babel\ZREF@abspos@used
}
%    \end{macrocode}
%    \end{macro}
%    \begin{macro}{\ZREF@abspos@used}
%    \begin{macrocode}
\def\ZREF@abspos@used#1{%
  \zref@refused{#1}%
  \zref@ifrefundefined{#1}{%
  }{%
    \begingroup
      \edef\ZREF@temp{%
        \zref@extractdefault{#1}{abspage}{0}%
      }%
      \ifnum\ZREF@temp>\ltx@zero
        \zref@refused{thepage\ZREF@temp}%
      \else
        \@PackageError{zref-abspos}{%
          \string\zref@pos@label@used\ltx@space
          needs property `abspage'\MessageBreak
          in label `#1'%
        }\@ehc
      \fi
    \endgroup
  }%
}
%    \end{macrocode}
%    \end{macro}
%
%    \begin{macro}{\zref@absposnumused}
%    \begin{macrocode}
\newcommand*{\zref@absposnumused}[1]{%
  \ifnum#1>\ltx@zero
    \zref@refused{thepage\number#1}%
  \else
    \@PackageError{zref-abspos}{%
      Invalid absolute page number (#1)\MessageBreak
      for \string\zref@pos@num@used.\MessageBreak
      A positive integer is expected%
    }\@ehc
  \fi
}
%    \end{macrocode}
%    \end{macro}
%
%    \begin{macro}{\zref@ifabsposundefined}
%    \begin{macrocode}
\def\zref@ifabsposundefined#1{%
  \zref@ifrefundefined{#1}\ltx@firsttwo{%
    \expandafter\zref@ifabsposnumundefined\expandafter{%
      \number\zref@extractdefault{#1}{abspage}{0}%
    }%
  }%
}
%    \end{macrocode}
%    \end{macro}
%
%    \begin{macro}{\zref@ifabsposnumundefined}
%    \begin{macrocode}
\def\zref@ifabsposnumundefined#1{%
  \ifnum\ZREF@number{#1}>\ltx@zero
    \zref@ifrefundefined{thepage#1}%
        \ltx@firstoftwo\ltx@secondoftwo
  \else
    \expandafter\ltx@firstoftwo
  \fi
}
%    \end{macrocode}
%    \end{macro}
%
% \subsubsection{Media}
%
%    \begin{macro}{\ZREF@abspos@media@width}
%    \begin{macrocode}
\edef\ZREF@abspos@media@width{%
  \ltx@ifundefined{pdfpagewidth}{%
    \ltx@ifundefined{mediawidth}{%
      \ltx@ifundefined{stockwidth}{%
        paperwidth%
      }{%
        stockwidth%
      }%
    }{%
      mediawidth%
    }%
  }{%
    pdfpagewidth%
  }%
}
\ifluatex
\def\ZREF@abspos@media@width{pdfpagewidth}%
\fi
%    \end{macrocode}
%    \end{macro}
%    \begin{macro}{\ZREF@abspos@media@height}
%    \begin{macrocode}
\edef\ZREF@abspos@media@height{%
  \ltx@ifundefined{pdfpageheight}{%
    \ltx@ifundefined{mediaheight}{%
      \ltx@ifundefined{stockheight}{%
        paperheight%
      }{%
        stockheight%
      }%
    }{%
      mediaheight%
    }%
  }{%
    \noexpand\ifcase\pdfpageheight
      \ltx@ifundefined{stockheight}{%
        paperheight%
      }{%
        stockheight%
      }%
    \noexpand\else
      pdfpageheight%
    \noexpand\fi
  }%
}
\ifluatex
\edef\ZREF@abspos@media@height{%
   \noexpand\ifcase\pageheight
      \ltx@ifundefined{stockheight}{%
        paperheight%
      }{%
        stockheight%
      }%
    \noexpand\else
      pdfpageheight%
    \noexpand\fi}%
\fi
%    \end{macrocode}
%    \end{macro}
%
%    \begin{macro}{\ZREF@abspos@media@x@left}
%    \begin{macrocode}
\def\ZREF@abspos@media@x@left#1{%
  0%
}
%    \end{macrocode}
%    \end{macro}
%    \begin{macro}{\ZREF@abspos@media@x@right}
%    \begin{macrocode}
\def\ZREF@abspos@media@x@right#1{%
  \zref@extract{#1}\ZREF@abspos@media@width
}
%    \end{macrocode}
%    \end{macro}
%    \begin{macro}{\ZREF@abspos@media@x@center}
%    \begin{macrocode}
\def\ZREF@abspos@media@x@center#1{%
  \ZREF@abspos@media@x@left{#1}%
  +\zref@extract{#1}\ZREF@abspos@media@width/2%
}
%    \end{macrocode}
%    \end{macro}
%
%    \begin{macro}{\ZREF@abspos@media@y@top}
%    \begin{macrocode}
\def\ZREF@abspos@media@y@top#1{%
  \zref@extract{#1}\ZREF@abspos@media@height
}
%    \end{macrocode}
%    \end{macro}
%    \begin{macro}{\ZREF@abspos@media@y@bottom}
%    \begin{macrocode}
\def\ZREF@abspos@media@y@bottom#1{%
  0%
}
%    \end{macrocode}
%    \end{macro}
%    \begin{macro}{\ZREF@abspos@media@y@center}
%    \begin{macrocode}
\def\ZREF@abspos@media@y@center#1{%
  \zref@extract{#1}\ZREF@abspos@media@height/2%
}
%    \end{macrocode}
%    \end{macro}
%
% \subsubsection{Paper}
%
%    \begin{macro}{\ZREF@abspos@paper@x@left}
%    \begin{macrocode}
\def\ZREF@abspos@paper@x@left#1{%
  0%
}
%    \end{macrocode}
%    \end{macro}
%    \begin{macro}{\ZREF@abspos@paper@x@right}
%    \begin{macrocode}
\def\ZREF@abspos@paper@x@right#1{%
  \zref@extract{#1}{paperwidth}%
}
%    \end{macrocode}
%    \end{macro}
%    \begin{macro}{\ZREF@abspos@paper@x@center}
%    \begin{macrocode}
\def\ZREF@abspos@paper@x@center#1{%
  \zref@extract{#1}{paperwidth}/2%
}
%    \end{macrocode}
%    \end{macro}
%
%    \begin{macro}{\ZREF@abspos@paper@y@top}
%    \begin{macrocode}
\let\ZREF@abspos@paper@y@top\ZREF@abspos@media@y@top
%    \end{macrocode}
%    \end{macro}
%    \begin{macro}{\ZREF@abspos@paper@y@bottom}
%    \begin{macrocode}
\def\ZREF@abspos@paper@y@bottom#1{%
  \ZREF@abspos@paper@y@top{#1}%
  -\zref@extract{#1}{paperheight}%
}
%    \end{macrocode}
%    \end{macro}
%    \begin{macro}{\ZREF@abspos@paper@y@center}
%    \begin{macrocode}
\def\ZREF@abspos@paper@y@center#1{%
  \ZREF@abspos@paper@y@top{#1}%
  -\zref@extract{#1}{paperheight}/2%
}
%    \end{macrocode}
%    \end{macro}
%
% \subsubsection{Origin}
%
% There doesn't seem a good reason to make these tests depend on pdf mode
%  in current engines, so comment out the \verb|\ifpdf| tests.
%    \begin{macro}{\ZREF@abspos@origin@x}
%    \begin{macrocode}
\let\ZREF@temp\ltx@two
\ltx@IfUndefined{pdfhorigin}{}{%
%  \ifpdf
    \let\ZREF@temp\ltx@zero
%  \fi
}
\ifluatex
% \ifpdf
  \let\ZREF@temp\ltx@zero
% \fi
\fi

\ifx\ZREF@temp\ltx@two
  \ifnum\mag=1000 %
    \let\ZREF@temp\ltx@one
  \fi
\fi
\ifcase\ZREF@temp
  \def\ZREF@abspos@origin@x#1{%
    \zref@extract{#1}{pdfhorigin}%
  }%
\or
  \def\ZREF@abspos@origin@x#1{%
    4736286%
  }%
\or
  \def\ZREF@abspos@origin@x#1{%
    \numexpr\mag/1000*\dimexpr 1truein\relax\relax
  }%
\fi
%    \end{macrocode}
%    \end{macro}
%    \begin{macro}{\ZREF@abspos@origin@y}
%    \begin{macrocode}
\let\ZREF@temp\ltx@two
\ltx@IfUndefined{pdfvorigin}{}{%
%  \ifpdf
    \let\ZREF@temp\ltx@zero
%  \fi
}
\ifluatex
%  \ifpdf
    \let\ZREF@temp\ltx@zero
%  \fi
\fi
\ifx\ZREF@temp\ltx@two
  \ifnum\mag=1000 %
    \let\ZREF@temp\ltx@one
  \fi
\fi
\ifcase\ZREF@temp
  \def\ZREF@abspos@origin@y#1{%
    \zref@extract{#1}{pdfvorigin}%
  }%
\or
  \def\ZREF@abspos@origin@y#1{%
    4736286%
  }%
\or
  \def\ZREF@abspos@origin@y#1{%
    \numexpr\mag/1000*\dimexpr 1truein\relax\relax
   }%
\fi
%    \end{macrocode}
%    \end{macro}
%
% \subsubsection{Header}
%
%    \begin{macro}{\ZREF@abspos@head@x@left}
%    \begin{macrocode}
\def\ZREF@abspos@head@x@left#1{%
  \ZREF@abspos@paper@x@left{#1}%
  +\ZREF@abspos@origin@x{#1}%
  +\zref@extract{#1}{hoffset}%
  +\ifodd\zref@extractdefault{#1}{pagevalue}{\number\c@page} %
     \zref@extract{#1}{oddsidemargin}%
   \else
     \zref@extract{#1}{evensidemargin}%
   \fi
}
%    \end{macrocode}
%    \end{macro}
%    \begin{macro}{\ZREF@abspos@head@x@right}
%    \begin{macrocode}
\def\ZREF@abspos@head@x@right#1{%
  \ZREF@abspos@head@x@left{#1}%
  +\zref@extract{#1}{textwidth}%
}
%    \end{macrocode}
%    \end{macro}
%    \begin{macro}{\ZREF@abspos@head@x@center}
%    \begin{macrocode}
\def\ZREF@abspos@head@x@center#1{%
  \ZREF@abspos@head@x@left{#1}%
  +\zref@extract{#1}{textwidth}/2%
}
%    \end{macrocode}
%    \end{macro}
%
%    \begin{macro}{\ZREF@abspos@head@y@top}
%    \begin{macrocode}
\def\ZREF@abspos@head@y@top#1{%
  \ZREF@abspos@paper@y@top{#1}%
  -\ZREF@abspos@origin@y{#1}%
  -\zref@extract{#1}{voffset}%
  -\zref@extract{#1}{topmargin}%
}
%    \end{macrocode}
%    \end{macro}
%    \begin{macro}{\ZREF@abspos@head@y@bottom}
%    \begin{macrocode}
\def\ZREF@abspos@head@y@bottom#1{%
  \ZREF@abspos@head@y@top{#1}%
  -\zref@extract{#1}{headheight}%
}
%    \end{macrocode}
%    \end{macro}
%    \begin{macro}{\ZREF@abspos@head@y@center}
%    \begin{macrocode}
\def\ZREF@abspos@head@y@center#1{%
  \ZREF@abspos@head@y@top{#1}%
  -\zref@extract{#1}{headheight}/2%
}
%    \end{macrocode}
%    \end{macro}
%
% \subsubsection{Body}
%
%    \begin{macro}{\ZREF@abspos@body@x@left}
%    \begin{macrocode}
\let\ZREF@abspos@body@x@left\ZREF@abspos@head@x@left
%    \end{macrocode}
%    \end{macro}
%    \begin{macro}{\ZREF@abspos@body@x@right}
%    \begin{macrocode}
\let\ZREF@abspos@body@x@right\ZREF@abspos@head@x@right
%    \end{macrocode}
%    \end{macro}
%    \begin{macro}{\ZREF@abspos@body@x@center}
%    \begin{macrocode}
\let\ZREF@abspos@body@x@center\ZREF@abspos@head@x@center
%    \end{macrocode}
%    \end{macro}
%
%    \begin{macro}{\ZREF@abspos@body@y@top}
%    \begin{macrocode}
\def\ZREF@abspos@body@y@top#1{%
  \ZREF@abspos@head@y@bottom{#1}%
  -\zref@extract{#1}{headsep}%
}
%    \end{macrocode}
%    \end{macro}
%    \begin{macro}{\ZREF@abspos@body@y@bottom}
%    \begin{macrocode}
\def\ZREF@abspos@body@y@bottom#1{%
  \ZREF@abspos@body@y@top{#1}%
  -\zref@extract{#1}{textheight}%
}
%    \end{macrocode}
%    \end{macro}
%    \begin{macro}{\ZREF@abspos@body@y@center}
%    \begin{macrocode}
\def\ZREF@abspos@body@y@center#1{%
  \ZREF@abspos@body@y@top{#1}%
  -\zref@extract{#1}{textheight}/2%
}
%    \end{macrocode}
%    \end{macro}
%
% \subsubsection{Footer}
%
%    \begin{macro}{\ZREF@abspos@foot@x@left}
%    \begin{macrocode}
\let\ZREF@abspos@foot@x@left\ZREF@abspos@head@x@left
%    \end{macrocode}
%    \end{macro}
%    \begin{macro}{\ZREF@abspos@foot@x@right}
%    \begin{macrocode}
\let\ZREF@abspos@foot@x@right\ZREF@abspos@head@x@right
%    \end{macrocode}
%    \end{macro}
%    \begin{macro}{\ZREF@abspos@foot@x@center}
%    \begin{macrocode}
\let\ZREF@abspos@foot@x@center\ZREF@abspos@head@x@center
%    \end{macrocode}
%    \end{macro}
%
%    \begin{macro}{\ZREF@abspos@foot@y@bottom}
%    \begin{macrocode}
\def\ZREF@abspos@foot@y@bottom#1{%
  \ZREF@abspos@body@y@bottom{#1}%
  -\zref@extract{#1}{footskip}%
}
%    \end{macrocode}
%    \end{macro}
%
% \subsubsection{Marginal notes}
%
%    \begin{macro}{\ZREF@abspos@marginpar@x@left}
%    \begin{macrocode}
\def\ZREF@abspos@marginpar@x@left#1{%
  \ifodd\zref@extractdefault{#1}{pagevalue}{\number\c@page} %
    \ZREF@abspos@body@x@right{#1}%
    +\zref@extract{#1}{marginparsep}%
  \else
    \ZREF@abspos@body@x@left{#1}%
    -\zref@extract{#1}{marginparsep}%
    -\zref@extract{#1}{marginparwidth}%
  \fi
}
%    \end{macrocode}
%    \end{macro}
%    \begin{macro}{\ZREF@abspos@marginpar@x@right}
%    \begin{macrocode}
\def\ZREF@abspos@marginpar@x@right#1{%
  \ZREF@abspos@marginpar@x@left{#1}%
  +\zref@extract{#1}{marginparwidth}%
}
%    \end{macrocode}
%    \end{macro}
%    \begin{macro}{\ZREF@abspos@marginpar@x@center}
%    \begin{macrocode}
\def\ZREF@abspos@marginpar@x@center#1{%
  \ZREF@abspos@marginpar@x@left{#1}%
  +\zref@extract{#1}{marginparwidth}/2%
}
%    \end{macrocode}
%    \end{macro}
%
%    \begin{macro}{\ZREF@abspos@marginpar@y@top}
%    \begin{macrocode}
\let\ZREF@abspos@marginpar@y@top\ZREF@abspos@body@y@top
%    \end{macrocode}
%    \end{macro}
%    \begin{macro}{\ZREF@abspos@marginpar@y@bottom}
%    \begin{macrocode}
\let\ZREF@abspos@marginpar@y@bottom\ZREF@abspos@body@y@bottom
%    \end{macrocode}
%    \end{macro}
%    \begin{macro}{\ZREF@abspos@marginpar@y@center}
%    \begin{macrocode}
\let\ZREF@abspos@marginpar@y@center\ZREF@abspos@body@y@center
%    \end{macrocode}
%    \end{macro}
%
% \subsubsection{Stock paper}
%
%    \begin{macro}{\ZREF@abspos@stock@x@left}
%    \begin{macrocode}
\let\ZREF@abspos@stock@x@left\ZREF@abspos@paper@x@left
%    \end{macrocode}
%    \end{macro}
%    \begin{macro}{\ZREF@abspos@stock@x@right}
%    \begin{macrocode}
\let\ZREF@abspos@stock@x@right\ZREF@abspos@paper@x@right
%    \end{macrocode}
%    \end{macro}
%    \begin{macro}{\ZREF@abspos@stock@x@center}
%    \begin{macrocode}
\let\ZREF@abspos@stock@x@center\ZREF@abspos@paper@x@center
%    \end{macrocode}
%    \end{macro}
%
%    \begin{macro}{\ZREF@abspos@stock@y@top}
%    \begin{macrocode}
\let\ZREF@abspos@stock@y@top\ZREF@abspos@paper@y@top
%    \end{macrocode}
%    \end{macro}
%    \begin{macro}{\ZREF@abspos@stock@y@bottom}
%    \begin{macrocode}
\let\ZREF@abspos@stock@y@bottom\ZREF@abspos@paper@y@bottom
%    \end{macrocode}
%    \end{macro}
%    \begin{macro}{\ZREF@abspos@stock@y@center}
%    \begin{macrocode}
\let\ZREF@abspos@stock@y@center\ZREF@abspos@paper@y@center
%    \end{macrocode}
%    \end{macro}
%
%    \begin{macrocode}
%</abspos>
%    \end{macrocode}
%
%
% \subsection{Module \xmodule{dotfill}}
%
%    \begin{macrocode}
%<*dotfill>
\NeedsTeXFormat{LaTeX2e}
\ProvidesPackage{zref-dotfill}%
  [2020-07-03 v2.32 Module dotfill for zref (HO)]%
\RequirePackage{zref-base}[2019/11/29]
\ifx\ZREF@base@ok Y%
\else
  \expandafter\endinput
\fi
%    \end{macrocode}
%
%    For measuring the width of \cs{zdotfill} we use the features
%    provided by module \xmodule{savepos}.
%    \begin{macrocode}
\RequirePackage{zref-savepos}[2019/11/29]
%    \end{macrocode}
%    For automatically generated label names we use the unique
%    counter of module \xmodule{base}.
%    \begin{macrocode}
\zref@require@unique
%    \end{macrocode}
%    Configuration is done by the key value interface of package
%    \xpackage{keyval}.
%    \begin{macrocode}
\RequirePackage{keyval}
%    \end{macrocode}
%    The definitions of the keys follow.
%    \begin{macrocode}
\define@key{ZREF@DF}{unit}{%
  \def\ZREF@df@unit{#1}%
}
\define@key{ZREF@DF}{min}{%
  \def\ZREF@df@min{#1}%
}
\define@key{ZREF@DF}{dot}{%
  \def\ZREF@df@dot{#1}%
}
%    \end{macrocode}
%    Defaults are set, see user interface.
%    \begin{macrocode}
\providecommand\ZREF@df@min{2}
\providecommand\ZREF@df@unit{.44em}
\providecommand\ZREF@df@dot{.}
%    \end{macrocode}
%    \begin{macro}{\zdotfillsetup}
%    Configuration of \cs{zdotfill} is done by \cs{zdotfillsetup}.
%    \begin{macrocode}
\newcommand*{\zdotfillsetup}{\kvsetkeys{ZREF@DF}}
%    \end{macrocode}
%    \end{macro}
%
%    \begin{macro}{\zdotfill}
%    \cs{zdotfill} sets labels at the left and the right to get
%    the horizontal position. \cs{zsavepos} is not used, because
%    we do not need the vertical position.
%    \begin{macrocode}
\ZREF@IfDefinable\zdotfill\def{%
  {%
    \leavevmode
    \global\advance\c@zref@unique\ltx@one
    \begingroup
      \def\ZREF@temp{zref@\number\c@zref@unique}%
      \pdfsavepos
      \zref@labelbyprops{\thezref@unique L}{posx}%
      \setlength{\dimen@}{\ZREF@df@unit}%
      \zref@ifrefundefined{\thezref@unique R}{%
        \ZREF@dotfill
      }{%
        \ifnum\numexpr\zposx{\thezref@unique R}%
                      -\zposx{\thezref@unique L}\relax
            <\dimexpr\ZREF@df@min\dimen@\relax
          \hfill
        \else
          \ZREF@dotfill
        \fi
      }%
      \pdfsavepos
      \zref@labelbyprops{\thezref@unique R}{posx}%
    \endgroup
    \kern\z@
  }%
}
%    \end{macrocode}
%    \end{macro}
%    \begin{macro}{\ZREF@dotfill}
%    Help macro that actually sets the dots.
%    \begin{macrocode}
\def\ZREF@dotfill{%
  \cleaders\hb@xt@\dimen@{\hss\ZREF@df@dot\hss}\hfill
}
%    \end{macrocode}
%    \end{macro}
%
%    \begin{macrocode}
%</dotfill>
%    \end{macrocode}
%
% \subsection{Module \xmodule{env}}
%
%    \begin{macrocode}
%<*env>
\NeedsTeXFormat{LaTeX2e}
\ProvidesPackage{zref-env}%
  [2020-07-03 v2.32 Module env for zref (HO)]%
\RequirePackage{zref-base}[2019/11/29]
\ifx\ZREF@base@ok Y%
\else
  \expandafter\endinput
\fi
%    \end{macrocode}
%
%    \begin{macrocode}
\zref@newprop{envname}[]{\@currenvir}
\zref@newprop{envline}[]{\zref@env@line}
%    \end{macrocode}
%
%    \begin{macro}{\zref@env@line}
%    Macro \cs{zref@env@line} extracts the line number
%    from \cs{@currenvline}.
%    \begin{macrocode}
\def\zref@env@line{%
  \ifx\@currenvline\ltx@empty
  \else
    \expandafter
    \ZREF@ENV@line\@currenvline\ltx@empty line \ltx@empty\@nil
  \fi
}
%    \end{macrocode}
%    \end{macro}
%    \begin{macro}{\ZREF@ENV@line}
%    \begin{macrocode}
\def\ZREF@ENV@line#1line #2\ltx@empty#3\@nil{#2}%
%    \end{macrocode}
%    \end{macro}
%
%    \begin{macrocode}
%</env>
%    \end{macrocode}
%
%% \section{Installation}
%
% \subsection{Download}
%
% \paragraph{Package.} This package is available on
% CTAN\footnote{\CTANpkg{zref}}:
% \begin{description}
% \item[\CTAN{macros/latex/contrib/zref/zref.dtx}] The source file.
% \item[\CTAN{macros/latex/contrib/zref/zref.pdf}] Documentation.
% \end{description}
%
%
% \paragraph{Bundle.} All the packages of the bundle `zref'
% are also available in a TDS compliant ZIP archive. There
% the packages are already unpacked and the documentation files
% are generated. The files and directories obey the TDS standard.
% \begin{description}
% \item[\CTANinstall{install/macros/latex/contrib/zref.tds.zip}]
% \end{description}
% \emph{TDS} refers to the standard ``A Directory Structure
% for \TeX\ Files'' (\CTANpkg{tds}). Directories
% with \xfile{texmf} in their name are usually organized this way.
%
% \subsection{Bundle installation}
%
% \paragraph{Unpacking.} Unpack the \xfile{zref.tds.zip} in the
% TDS tree (also known as \xfile{texmf} tree) of your choice.
% Example (linux):
% \begin{quote}
%   |unzip zref.tds.zip -d ~/texmf|
% \end{quote}
%
%
% \subsection{Package installation}
%
% \paragraph{Unpacking.} The \xfile{.dtx} file is a self-extracting
% \docstrip\ archive. The files are extracted by running the
% \xfile{.dtx} through \plainTeX:
% \begin{quote}
%   \verb|tex zref.dtx|
% \end{quote}
%
% \paragraph{TDS.} Now the different files must be moved into
% the different directories in your installation TDS tree
% (also known as \xfile{texmf} tree):
% \begin{quote}
% \def\t{^^A
% \begin{tabular}{@{}>{\ttfamily}l@{ $\rightarrow$ }>{\ttfamily}l@{}}
%   zref.sty & tex/latex/zref/zref.sty\\
%   zref-base.sty & tex/latex/zref/zref-base.sty\\
%   zref-abspage.sty & tex/latex/zref/zref-abspage.sty\\
%   zref-abspos.sty & tex/latex/zref/zref-abspos.sty\\
%   zref-counter.sty & tex/latex/zref/zref-counter.sty\\
%   zref-dotfill.sty & tex/latex/zref/zref-dotfill.sty\\
%   zref-env.sty & tex/latex/zref/zref-env.sty\\
%   zref-hyperref.sty & tex/latex/zref/zref-hyperref.sty\\
%   zref-lastpage.sty & tex/latex/zref/zref-lastpage.sty\\
%   zref-marks.sty & tex/latex/zref/zref-marks.sty\\
%   zref-nextpage.sty & tex/latex/zref/zref-nextpage.sty\\
%   zref-pageattr.sty & tex/latex/zref/zref-pageattr.sty\\
%   zref-pagelayout.sty & tex/latex/zref/zref-pagelayout.sty\\
%   zref-perpage.sty & tex/latex/zref/zref-perpage.sty\\
%   zref-runs.sty & tex/latex/zref/zref-runs.sty\\
%   zref-savepos.sty & tex/latex/zref/zref-savepos.sty\\
%   zref-thepage.sty & tex/latex/zref/zref-thepage.sty\\
%   zref-titleref.sty & tex/latex/zref/zref-titleref.sty\\
%   zref-totpages.sty & tex/latex/zref/zref-totpages.sty\\
%   zref-user.sty & tex/latex/zref/zref-user.sty\\
%   zref-xr.sty & tex/latex/zref/zref-xr.sty\\
%   zref.pdf & doc/latex/zref/zref.pdf\\
%   zref-example.tex & doc/latex/zref/zref-example.tex\\
%   zref-example-lastpage.tex & doc/latex/zref/zref-example-lastpage.tex\\
%   zref-example-nextpage.tex & doc/latex/zref/zref-example-nextpage.tex\\
%   zref.dtx & source/latex/zref/zref.dtx\\
% \end{tabular}^^A
% }^^A
% \sbox0{\t}^^A
% \ifdim\wd0>\linewidth
%   \begingroup
%     \advance\linewidth by\leftmargin
%     \advance\linewidth by\rightmargin
%   \edef\x{\endgroup
%     \def\noexpand\lw{\the\linewidth}^^A
%   }\x
%   \def\lwbox{^^A
%     \leavevmode
%     \hbox to \linewidth{^^A
%       \kern-\leftmargin\relax
%       \hss
%       \usebox0
%       \hss
%       \kern-\rightmargin\relax
%     }^^A
%   }^^A
%   \ifdim\wd0>\lw
%     \sbox0{\small\t}^^A
%     \ifdim\wd0>\linewidth
%       \ifdim\wd0>\lw
%         \sbox0{\footnotesize\t}^^A
%         \ifdim\wd0>\linewidth
%           \ifdim\wd0>\lw
%             \sbox0{\scriptsize\t}^^A
%             \ifdim\wd0>\linewidth
%               \ifdim\wd0>\lw
%                 \sbox0{\tiny\t}^^A
%                 \ifdim\wd0>\linewidth
%                   \lwbox
%                 \else
%                   \usebox0
%                 \fi
%               \else
%                 \lwbox
%               \fi
%             \else
%               \usebox0
%             \fi
%           \else
%             \lwbox
%           \fi
%         \else
%           \usebox0
%         \fi
%       \else
%         \lwbox
%       \fi
%     \else
%       \usebox0
%     \fi
%   \else
%     \lwbox
%   \fi
% \else
%   \usebox0
% \fi
% \end{quote}
% If you have a \xfile{docstrip.cfg} that configures and enables \docstrip's
% TDS installing feature, then some files can already be in the right
% place, see the documentation of \docstrip.
%
% \subsection{Refresh file name databases}
%
% If your \TeX~distribution
% (\TeX\,Live, \mikTeX, \dots) relies on file name databases, you must refresh
% these. For example, \TeX\,Live\ users run \verb|texhash| or
% \verb|mktexlsr|.
%
% \subsection{Some details for the interested}
%
% \paragraph{Unpacking with \LaTeX.}
% The \xfile{.dtx} chooses its action depending on the format:
% \begin{description}
% \item[\plainTeX:] Run \docstrip\ and extract the files.
% \item[\LaTeX:] Generate the documentation.
% \end{description}
% If you insist on using \LaTeX\ for \docstrip\ (really,
% \docstrip\ does not need \LaTeX), then inform the autodetect routine
% about your intention:
% \begin{quote}
%   \verb|latex \let\install=y% \iffalse meta-comment
%
% File: zref.dtx
% Version: 2020-07-03 v2.32
% Info: A new reference scheme for LaTeX
%
% Copyright (C)
%    2006-2012  Heiko Oberdiek
%    2016-2020  Oberdiek Package Support Group
%    https://github.com/ho-tex/zref/issues
%
% This work may be distributed and/or modified under the
% conditions of the LaTeX Project Public License, either
% version 1.3c of this license or (at your option) any later
% version. This version of this license is in
%    https://www.latex-project.org/lppl/lppl-1-3c.txt
% and the latest version of this license is in
%    https://www.latex-project.org/lppl.txt
% and version 1.3 or later is part of all distributions of
% LaTeX version 2005/12/01 or later.
%
% This work has the LPPL maintenance status "maintained".
%
% The Current Maintainers of this work are
% Heiko Oberdiek and the Oberdiek Package Support Group
% https://github.com/ho-tex/zref/issues
%
% This work consists of the main source file zref.dtx
% and the derived files
%    zref.sty, zref.pdf, zref.ins, zref.drv, zref-base.sty,
%    zref-abspage.sty, zref-abspos.sty, zref-counter.sty,
%    zref-dotfill.sty, zref-env.sty, zref-hyperref.sty,
%    zref-lastpage.sty, zref-marks.sty, zref-nextpage.sty,
%    zref-pageattr.sty, zref-pagelayout.sty, zref-perpage.sty,
%    zref-runs.sty, zref-savepos.sty, zref-thepage.sty,
%    zref-titleref.sty, zref-totpages.sty, zref-user.sty,
%    zref-xr.sty, zref-example.tex, zref-example-lastpage.tex,
%    zref-example-nextpage.tex, zref-test1.tex,
%    zref-test-base.tex, zref-test-runs.tex,
%    zref-test-titleref-memoir.tex.
%
% Distribution:
%    CTAN:macros/latex/contrib/zref/zref.dtx
%    CTAN:macros/latex/contrib/zref/zref.pdf
%
% Unpacking:
%    (a) If zref.ins is present:
%           tex zref.ins
%    (b) Without zref.ins:
%           tex zref.dtx
%    (c) If you insist on using LaTeX
%           latex \let\install=y% \iffalse meta-comment
%
% File: zref.dtx
% Version: 2020-07-03 v2.32
% Info: A new reference scheme for LaTeX
%
% Copyright (C)
%    2006-2012  Heiko Oberdiek
%    2016-2020  Oberdiek Package Support Group
%    https://github.com/ho-tex/zref/issues
%
% This work may be distributed and/or modified under the
% conditions of the LaTeX Project Public License, either
% version 1.3c of this license or (at your option) any later
% version. This version of this license is in
%    https://www.latex-project.org/lppl/lppl-1-3c.txt
% and the latest version of this license is in
%    https://www.latex-project.org/lppl.txt
% and version 1.3 or later is part of all distributions of
% LaTeX version 2005/12/01 or later.
%
% This work has the LPPL maintenance status "maintained".
%
% The Current Maintainers of this work are
% Heiko Oberdiek and the Oberdiek Package Support Group
% https://github.com/ho-tex/zref/issues
%
% This work consists of the main source file zref.dtx
% and the derived files
%    zref.sty, zref.pdf, zref.ins, zref.drv, zref-base.sty,
%    zref-abspage.sty, zref-abspos.sty, zref-counter.sty,
%    zref-dotfill.sty, zref-env.sty, zref-hyperref.sty,
%    zref-lastpage.sty, zref-marks.sty, zref-nextpage.sty,
%    zref-pageattr.sty, zref-pagelayout.sty, zref-perpage.sty,
%    zref-runs.sty, zref-savepos.sty, zref-thepage.sty,
%    zref-titleref.sty, zref-totpages.sty, zref-user.sty,
%    zref-xr.sty, zref-example.tex, zref-example-lastpage.tex,
%    zref-example-nextpage.tex, zref-test1.tex,
%    zref-test-base.tex, zref-test-runs.tex,
%    zref-test-titleref-memoir.tex.
%
% Distribution:
%    CTAN:macros/latex/contrib/zref/zref.dtx
%    CTAN:macros/latex/contrib/zref/zref.pdf
%
% Unpacking:
%    (a) If zref.ins is present:
%           tex zref.ins
%    (b) Without zref.ins:
%           tex zref.dtx
%    (c) If you insist on using LaTeX
%           latex \let\install=y\input{zref.dtx}
%        (quote the arguments according to the demands of your shell)
%
% Documentation:
%    (a) If zref.drv is present:
%           latex zref.drv
%    (b) Without zref.drv:
%           latex zref.dtx; ...
%    The class ltxdoc loads the configuration file ltxdoc.cfg
%    if available. Here you can specify further options, e.g.
%    use A4 as paper format:
%       \PassOptionsToClass{a4paper}{article}
%
%    Programm calls to get the documentation (example):
%       pdflatex zref.dtx
%       makeindex -s gind.ist zref.idx
%       pdflatex zref.dtx
%       makeindex -s gind.ist zref.idx
%       pdflatex zref.dtx
%
% Installation:
%    TDS:tex/latex/zref/zref.sty
%    TDS:tex/latex/zref/zref-base.sty
%    TDS:tex/latex/zref/zref-abspage.sty
%    TDS:tex/latex/zref/zref-abspos.sty
%    TDS:tex/latex/zref/zref-counter.sty
%    TDS:tex/latex/zref/zref-dotfill.sty
%    TDS:tex/latex/zref/zref-env.sty
%    TDS:tex/latex/zref/zref-hyperref.sty
%    TDS:tex/latex/zref/zref-lastpage.sty
%    TDS:tex/latex/zref/zref-marks.sty
%    TDS:tex/latex/zref/zref-nextpage.sty
%    TDS:tex/latex/zref/zref-pageattr.sty
%    TDS:tex/latex/zref/zref-pagelayout.sty
%    TDS:tex/latex/zref/zref-perpage.sty
%    TDS:tex/latex/zref/zref-runs.sty
%    TDS:tex/latex/zref/zref-savepos.sty
%    TDS:tex/latex/zref/zref-thepage.sty
%    TDS:tex/latex/zref/zref-titleref.sty
%    TDS:tex/latex/zref/zref-totpages.sty
%    TDS:tex/latex/zref/zref-user.sty
%    TDS:tex/latex/zref/zref-xr.sty
%    TDS:doc/latex/zref/zref.pdf
%    TDS:doc/latex/zref/zref-example.tex
%    TDS:doc/latex/zref/zref-example-lastpage.tex
%    TDS:doc/latex/zref/zref-example-nextpage.tex
%    TDS:source/latex/zref/zref.dtx
%
%<*ignore>
\begingroup
  \catcode123=1 %
  \catcode125=2 %
  \def\x{LaTeX2e}%
\expandafter\endgroup
\ifcase 0\ifx\install y1\fi\expandafter
         \ifx\csname processbatchFile\endcsname\relax\else1\fi
         \ifx\fmtname\x\else 1\fi\relax
\else\csname fi\endcsname
%</ignore>
%<*install>
\input docstrip.tex
\Msg{************************************************************************}
\Msg{* Installation}
\Msg{* Package: zref 2020-07-03 v2.32 A new reference scheme for LaTeX (HO)}
\Msg{************************************************************************}

\keepsilent
\askforoverwritefalse

\let\MetaPrefix\relax
\preamble

This is a generated file.

Project: zref
Version: 2020-07-03 v2.32

Copyright (C)
  2006-2012 Heiko Oberdiek
  2016-2020 Oberdiek Package Support Group

This work may be distributed and/or modified under the
conditions of the LaTeX Project Public License, either
version 1.3c of this license or (at your option) any later
version. This version of this license is in
   https://www.latex-project.org/lppl/lppl-1-3c.txt
and the latest version of this license is in
   https://www.latex-project.org/lppl.txt
and version 1.3 or later is part of all distributions of
LaTeX version 2005/12/01 or later.

This work has the LPPL maintenance status "maintained".

The Current Maintainers of this work are
Heiko Oberdiek and the Oberdiek Package Support Group
https://github.com/ho-tex/zref/issues


This work consists of the main source file zref.dtx
and the derived files
   zref.sty, zref.pdf, zref.ins, zref.drv, zref-base.sty,
   zref-abspage.sty, zref-abspos.sty, zref-counter.sty,
   zref-dotfill.sty, zref-env.sty, zref-hyperref.sty,
   zref-lastpage.sty, zref-marks.sty, zref-nextpage.sty,
   zref-pageattr.sty, zref-pagelayout.sty, zref-perpage.sty,
   zref-runs.sty, zref-savepos.sty, zref-thepage.sty,
   zref-titleref.sty, zref-totpages.sty, zref-user.sty,
   zref-xr.sty, zref-example.tex, zref-example-lastpage.tex,
   zref-example-nextpage.tex, zref-test1.tex,
   zref-test-base.tex, zref-test-runs.tex,
   zref-test-titleref-memoir.tex.

\endpreamble
\let\MetaPrefix\DoubleperCent

\generate{%
  \file{zref.ins}{\from{zref.dtx}{install}}%
  \file{zref.drv}{\from{zref.dtx}{driver}}%
  \usedir{tex/latex/zref}%
  \file{zref.sty}{\from{zref.dtx}{package}}%
  \file{zref-base.sty}{\from{zref.dtx}{base}}%
  \file{zref-abspage.sty}{\from{zref.dtx}{abspage}}%
  \file{zref-abspos.sty}{\from{zref.dtx}{abspos}}%
  \file{zref-counter.sty}{\from{zref.dtx}{counter}}%
  \file{zref-dotfill.sty}{\from{zref.dtx}{dotfill}}%
  \file{zref-env.sty}{\from{zref.dtx}{env}}%
  \file{zref-hyperref.sty}{\from{zref.dtx}{hyperref}}%
  \file{zref-lastpage.sty}{\from{zref.dtx}{lastpage}}%
  \file{zref-marks.sty}{\from{zref.dtx}{marks}}%
  \file{zref-nextpage.sty}{\from{zref.dtx}{nextpage}}%
  \file{zref-pageattr.sty}{\from{zref.dtx}{pageattr}}%
  \file{zref-pagelayout.sty}{\from{zref.dtx}{pagelayout}}%
  \file{zref-perpage.sty}{\from{zref.dtx}{perpage}}%
  \file{zref-runs.sty}{\from{zref.dtx}{runs}}%
  \file{zref-savepos.sty}{\from{zref.dtx}{savepos}}%
  \file{zref-thepage.sty}{\from{zref.dtx}{thepage}}%
  \file{zref-titleref.sty}{\from{zref.dtx}{titleref}}%
  \file{zref-totpages.sty}{\from{zref.dtx}{totpages}}%
  \file{zref-user.sty}{\from{zref.dtx}{user}}%
  \file{zref-xr.sty}{\from{zref.dtx}{xr}}%
  \usedir{doc/latex/zref}%
  \file{zref-example.tex}{\from{zref.dtx}{example}}%
  \file{zref-example-lastpage.tex}{\from{zref.dtx}{example-lastpage}}%
  \file{zref-example-nextpage.tex}{\from{zref.dtx}{example-nextpage}}%
%  \usedir{doc/latex/zref/test}%
%  \file{zref-test1.tex}{\from{zref.dtx}{test1}}%
%  \file{zref-test-base.tex}{\from{zref.dtx}{test-base}}%
%  \file{zref-test-runs.tex}{\from{zref.dtx}{test-runs}}%
%  \file{zref-test-titleref-memoir.tex}{\from{zref.dtx}{test-titleref-memoir}}%
}

\catcode32=13\relax% active space
\let =\space%
\Msg{************************************************************************}
\Msg{*}
\Msg{* To finish the installation you have to move the following}
\Msg{* files into a directory searched by TeX:}
\Msg{*}
\Msg{*     zref.sty, zref-base.sty, zref-abspage.sty, zref-abspos.sty,}
\Msg{*     zref-counter.sty, zref-dotfill.sty, zref-env.sty,}
\Msg{*     zref-hyperref.sty, zref-lastpage.sty, zref-marks.sty,}
\Msg{*     zref-nextpage.sty, zref-pageattr.sty, zref-pagelayout.sty,}
\Msg{*     zref-perpage.sty, zref-runs.sty, zref-savepos.sty,}
\Msg{*     zref-thepage.sty, zref-titleref.sty, zref-totpages.sty,}
\Msg{*     zref-user.sty, zref-xr.sty}
\Msg{*}
\Msg{* To produce the documentation run the file `zref.drv'}
\Msg{* through LaTeX.}
\Msg{*}
\Msg{* Happy TeXing!}
\Msg{*}
\Msg{************************************************************************}

\endbatchfile
%</install>
%<*ignore>
\fi
%</ignore>
%<*driver>
\NeedsTeXFormat{LaTeX2e}
\ProvidesFile{zref.drv}%
  [2020-07-03 v2.32 A new reference scheme for LaTeX (HO)]%
\documentclass{ltxdoc}
\usepackage{holtxdoc}[2011/11/22]
\setlength{\emergencystretch}{1em}
\begin{document}
  \DocInput{zref.dtx}%
\end{document}
%</driver>
% \fi
%
%
%
% \GetFileInfo{zref.drv}
%
% \title{The \xpackage{zref} package}
% \date{2020-07-03 v2.32}
% \author{Heiko Oberdiek\thanks
% {Please report any issues at \url{https://github.com/ho-tex/zref/issues}}}
%
% \maketitle
%
% \begin{abstract}
% Package \xpackage{zref} tries to get rid of the restriction
% in \hologo{LaTeX}'s reference system that only two properties are
% supported. The package implements an extensible referencing
% system, where properties are handled in a more flexible way.
% It offers an interface for macro programmers for the access
% to the system and some applications that uses the new
% reference scheme.
% \end{abstract}
%
% \tableofcontents
%
% \newcommand*{\metacs}[1]{\texttt{\itshape #1}}
%
% \newcommand{\markexp}{^^A
%   \textsuperscript{exp}^^A
%   \space
% }
% \newcommand{\markexptwo}{^^A
%   \textsuperscript{exp2}^^A
%   \space
% }
% \newcommand{\markrobust}{^^A
%   \textsuperscript{robust}^^A
%   \space
% }
% \newcommand{\marknowarn}{^^A
%   \textsuperscript{nowarn}^^A
%   \space
% }
% \newcommand{\markbabel}{^^A
%   \textsuperscript{babel}^^A
%   \space
% }
%
% \section{Introduction}
%
% Standard \hologo{LaTeX}'s reference system with \cs{label}, \cs{ref},
% and \cs{pageref} supports two properties, the apperance of
% the counter that is last incremented by \cs{refstepcounter} and
% the page with the \cs{label} command.
%
% Unhappily \hologo{LaTeX} does not provide an interface for adding
% another properties. Packages such as \xpackage{hyperref},
% \xpackage{nameref}, or \xpackage{titleref} are forced to
% use ugly hacks to extend the reference system.
% These ugly hacks are one of the causes for \xpackage{hyperref}'s
% difficulty regarding compatibility with other packages.
%
% \subsection{Standard \hologo{LaTeX} behaviour}
%
% References are created by the \cs{label} command:
% \begin{quote}
%   |\chapter{Second chapter}|\\
%   |\section{First section on page 7} % section 2.1|\\
%   |\label{myref}|
% \end{quote}
% Now \hologo{LaTeX} records the section number |2.1| and the page |7|
% in the reference. Internally the reference is a list with
% two entries:
% \begin{quote}
%   |\r@myref| $\rightarrow$ |{2.1}{7}|
% \end{quote}
% The length of the list if fixed in the \hologo{LaTeX} kernel,
% An interface for adding new properties is missing.
%
% There are several tries to add new properties:
% \begin{description}
% \item[\xpackage{hyperref}] uses a list of five
%   properties instead of the standard list with two entries.
%   This causes many compatibility problems with \hologo{LaTeX} and
%   other packages.
% \item[\xpackage{titleref}] stores its title data into the
%   first entry in the list. \hologo{LaTeX} is happy because it
%   does only see its list with two entries. The situation
%   becomes more difficult, if more properties are added
%   this way. Then the macros form a nested structure
%   inside the first reference argument for the label.
%   Expandable extractions will then become painful.
% \end{description}
%
% \subsection{Basic idea}
%
% Some time ago Morten H\o gholm sent me an experimental
% cross referencing mechanism as ``expl3'' code. His
% idea is:
% \begin{quote}
%   |\g_xref_mylabel_plist| $\rightarrow$\\
%   \mbox{}\qquad|\xref_dance_key{salsa}\xref_name_key{Morten}|\dots
% \end{quote}
% The entries have the following format:
% \begin{quote}
%   |\xref_|\meta{your key}|_key{|\meta{some text}|}|
% \end{quote}
% This approach is much more flexible:
% \begin{itemize}
%   \item New properties can easily be added, just use a new key.
%   \item The length of the list is not fixed. A reference
%         can use a subset of the keys.
%   \item The order of the entries does not matter.
% \end{itemize}
% Unhappily I am not familiar with the experimental code
% for \hologo{LaTeX3} that will need some time before its first release.
% Thus I have implemented it as \hologo{LaTeXe} package without disturbing
% the existing \hologo{LaTeX} reference system.
%
% \subsection{Interfaces}
%
% The package provides a generic \emph{interface for programmers}.
% Commands of this interface are prefixed by \cs{zref@}.
%
% Option \xoption{user} enabels the \emph{user interface}.
% Here the commands are prefixed by \cs{z} to avoid name
% clashes with existing macros.
%
% Then the packages provides some \emph{modules}. They
% are applications for the reference system and can also be
% considered as examples how to use the reference system.
%
% The modules can be loaded as packages. The package
% name is prefixed with |zref-|, for example:
%\begin{quote}
%\begin{verbatim}
%\RequirePackage{zref-abspage}
%\end{verbatim}
%\end{quote}
% This is the preferred way if the package is loaded from
% within other packages to avoid option clashes.
%
% As alternative package |zref| can be used and the modules
% are given as options:
%\begin{quote}
%\begin{verbatim}
%\usepackage[perpage,user]{zref}
%\end{verbatim}
%\end{quote}
%
% \section{Interface for programmers}
%
% The user interface is described in the next section \ref{sec:user}.
%
% \subsection{Entities}
%
% \paragraph{Reference.}
% Internally a reference is a list of key value pairs:
% \begin{quote}
%   |\Z@R@myref| $\rightarrow$
%   |\default{2.1}\page{7}|
% \end{quote}
% The generic format of a entry is:
% \begin{quote}
%   |\Z@R@|\meta{refname} $\rightarrow$
%   |\|\meta{propname}|{|\meta{value}|}|
% \end{quote}
% \meta{refname} is the name that denoted references
% (the name used in \cs{label} and \cs{ref}). \meta{propname}
% is the name of the property or key. The property key macro
% is never executed, it is used in parameter text matching only.
%
% \paragraph{Property.}
% Because the name of a property is used in a macro name
% that must survive the \xfile{.aux} file, the name is
% restricted to letters and `@'.
%
% \paragraph{Property list.}
% Often references are used for special purposes. Thus it
% saves memory if just the properties are used in this
% reference that are necessary for its purpose.
%
% Therefore this package uses the concept of \emph{property lists}.
% A property list is a set of properties. The set of properties
% that is used by the default \cs{label} command is the
% \emph{main property list}.
%
% \subsection{Property list}
%
% \markexp means that the implementation of the
% marked macro is expandable. \markexptwo goes a step further
% and marks the macro expandable in exact two expansion steps.
%
% \begin{declcs}{zref@newlist}
%   \M{listname}
% \end{declcs}
% Declares a new empty property list.
%
% \begin{declcs}{zref@addprop}
%   \M{listname} \M{propname}\\
%   \cs{zref@localaddprop} \M{listname} \M{propname}
% \end{declcs}
% Adds the property \meta{propname}
% to the property list \meta{listname}.
% The property and list must exist.
% The addition is global by \cs{zref@addprop} and limited to
% local scope by \cs{zref@localaddprop}.
% Between 2010/04/19 v2.13 and 2010/10/22 v2.19 a comma separated list
% of properties could be used as argument \meta{propname}.
% Since 2010/10/22 v2.19 the addition of several properties
% at once is supported by \cs{zref@addprops}.
%
% \begin{declcs}{zref@addprops}
%   \M{listname} \M{propname list}\\
%   \cs{zref@localaddprops} \M{listname} \M{propname list}
% \end{declcs}
% These macros add a comma separated list of properties
% \meta{propname list} to list \meta{listname}.
% \cs{zref@addprops} works globally and \cs{zref@localaddprops}
% locally. Since 2010/10/22 v2.19.
%
% \begin{declcs}{zref@listexists}
%   \M{listname} \M{then}
% \end{declcs}
% Executes \meta{then} if the property list \meta{listname} exists
% or raise an error otherwise.
%
% \begin{declcs}{zref@iflistundefined}\markexp \M{listname}
%   \M{then} \M{else}
% \end{declcs}
% Executes \meta{then} if the list exists or \meta{else}
% otherwise.
%
% \begin{declcs}{zref@iflistcontainsprop}
%   \M{listname} \M{propname} \M{then} \M{else}
% \end{declcs}
% Executes \meta{then} if the property \meta{propname} is part
% of property list \meta{listname} or otherwise it
% runs the \meta{else} part.
%
% \subsection{Property}
%
% \begin{declcs}{zref@newprop}
%   \* \M{propname} |[|\meta{default}|]| \M{value}
% \end{declcs}
% This command declares and configures a new property with
% name \meta{propname}.
%
% In case of unknown references
% or the property does not exist in the reference, the
% \meta{default} is used as value. If it is not specified
% here, a global default is used, see \cs{zref@setdefault}.
%
% The correct values of some properties are not known
% immediately but at page shipout time. Prominent example
% is the page number. These properties are declared with
% the star form of the command.
%
% \begin{declcs}{zref@setcurrent}
%   \M{propname} \M{value}
% \end{declcs}
% This sets the current value of the property \meta{propname}.
% It is a generalization of setting \hologo{LaTeX}'s \cs{currentlabel}.
%
% \begin{declcs}{zref@getcurrent}\markexptwo
%   \M{propname}
% \end{declcs}
% This returns the current value of the property \meta{propname}.
% The value may not be correct, especially if the property
% is bound to a page (start form of \cs{zref@newprop}) and the
% right value is only known at shipout time (e.g. property `page').
% In case of errors (e.g. unknown property) the empty string is
% returned.
%
% Since version 2010/04/22 v2.14 \cs{zref@getcurrent} supports
% \cs{zref@wrapper@unexpanded}.
%
% \begin{declcs}{zref@propexists}
%   \M{propname} \M{then}
% \end{declcs}
% Calls \meta{then} if the property \meta{propname} is available
% or generates an error message otherwise.
%
% \begin{declcs}{zref@ifpropundefined}\markexp
%   \M{propname} \M{then} \M{else}
% \end{declcs}
% Calls \meta{then} or \meta{else} depending on the existence
% of property \meta{propname}.
%
% \subsection{Reference generation}
%
% \begin{declcs}{zref@label}
%   \M{refname}
% \end{declcs}
% This works similar to \cs{label}. The reference \meta{refname} is
% created and put into the \xfile{.aux} file with the
% properties of the main property list.
%
% \begin{declcs}{zref@labelbylist}
%   \M{refname} \M{listname}
% \end{declcs}
% Same as \cs{zref@label} except that the properties are
% taken from the specified property list \meta{listname}.
%
% \begin{declcs}{zref@labelbyprops}
%   \M{refname} |{|\meta{propnameA}|,|\meta{propnameB}|,|\dots|}|
% \end{declcs}
% Same as \cs{zref@label} except that these properties
% are used that are given as comma separated list in the
% second argument.
%
% \begin{declcs}{zref@newlabel}
%   \M{refname} |{|\dots|}|
% \end{declcs}
% This is the macro that is used in the \xfile{.aux} file.
% It is basically the same as \cs{newlabel} apart from
% the format of the data in the second argument.
%
% \subsection{Data extraction}
%
% \begin{declcs}{zref@extractdefault}\markexptwo
%   \M{refname} \M{propname} \M{default}
% \end{declcs}
% This is the basic command that refernces the value of a
% property \meta{propname} for the reference \meta{refname}.
% In case of errors such as undefined reference the \meta{default}
% is used instead.
%
% \begin{declcs}{zref@extract}\markexptwo
%   \M{refname} \M{propname}
% \end{declcs}
% The command is an abbreviation for \cs{zref@extractdefault}.
% As default the default of the property is taken, otherwise
% the global default.
%
% Example for page references:
% \begin{quote}
% \begin{tabular}{@{}ll}
% \hologo{LaTeX}:& |\pageref{foobar}|\\
% \xpackage{zref}:& |\zref@extract{foobar}{page}|
% \end{tabular}
% \end{quote}
% Both \cs{zref@extract} and \cs{zref@extractdefault} are
% expandable. That means, these macros can directly
% be used in expandable calculations, see the example file.
% On the other side, babel's shorthands are not supported,
% there are no warnings in case of undefined references.
%
% If an user interface doesn't need expandable macros
% then it can use \cs{zref@refused} and \cs{zref@wrapper@babel}
% for its user macros.
%
% \begin{declcs}{zref@refused}
%   \M{refname}
% \end{declcs}
% This command is not expandable. It causes the warnings if
% the reference \meta{refname} is not defined. Use the
% \cs{zref@extract} commands inside expandable contexts and
% mark their use outside by \cs{zref@refused}, see the example
% file.
%
% \begin{declcs}{zref@def@extract}%
%     \M{cmd} \M{refname} \M{propname}\\
%   \cs{zref@def@extractdefault}
%     \M{cmd} \M{refname} \M{propname} \M{default}
% \end{declcs}
% Both macros extract the property \meta{propname} from the reference
% \meta{refname} the same way as macros \cs{zref@extract} and
% \cs{zref@extractdefault}. The result is stored in macro \meta{cmd}.
% Also \cs{zref@refused} is called to notify \hologo{LaTeX} that
% the reference \meta{refname} is used. Added in 2011/10/04 v2.22.
%
% \begin{declcs}{zref@ifrefundefined}\markexp
%   \M{refname} \M{then} \M{else}
% \end{declcs}
% Macro \cs{zref@ifrefundefined} calls arguments \meta{then} or \meta{else}
% dependent on the existence of the reference \meta{refname}.
%
% \begin{declcs}{zifrefundefined}
%   \M{refname} \M{then} \M{else}
% \end{declcs}
% Macro \cs{zifrefundefined} calls \cs{ref@refused} before
% executing \cs{zref@ifrefundefined}. Babel shorthands are
% supported in \meta{refname}.
%
% \begin{declcs}{zref@ifrefcontainsprop}\markexp
%   \M{refname} \M{propname} \M{then} \M{else}
% \end{declcs}
% Test whether a reference provides a property.
%
% \subsection{Setup}
%
% \begin{declcs}{zref@default}
% \end{declcs}
% Holds the global default for unknown values.
%
% \begin{declcs}{zref@setdefault}
%   |{|\meta{value}|}|
% \end{declcs}
% Sets the global default for unknown values.
% The global default is used, if a property does not specify
% an own default and the value for a property
% cannot be extracted. This can happen if the reference is
% unknown or the reference does not have the property.
%
% \begin{declcs}{zref@setmainlist}
%   |{|\meta{value}|}|
% \end{declcs}
% Sets the name of the main property list. The package
% sets and uses |main|.
%
% \subsection{Declared properties}
%
% \begin{quote}
% \renewcommand*{\arraystretch}{1.05}
% \setlength{\tabcolsep}{2\tabcolsep}
% \def\VV{\vphantom{\large \"Ag}}
% \def\EMPTY{{\itshape\textless empty\textgreater}}
% \def\pl#1{&#1&thepage&\cs{number}\cs{#1}}
% \begin{tabular}{@{}>{\sffamily}l|>{\ttfamily}l|>{\ttfamily}l|>{\ttfamily}l@{}}
%   \rmfamily Module&
%   \rmfamily Property&
%   \rmfamily Property list&
%   \rmfamily Default\\
%   \hline
%   (base)
%     & default & main & \EMPTY\\
%     & page & main & \EMPTY\\
%   \hline
%   abspage
%     & abspage & main & 0\\
%   \hline
%   counter
%     & counter & main & \EMPTY\\
%   \hline
%   hyperref
%     & anchor & main & \EMPTY\\
%     & url    & & \EMPTY\\
%   \hline
%   pageattr
%     & pdfpageattr & thepage & \dots\\
%     & pdfpagesattr & LastPage & \dots\\
%   \hline
%   pagelayout\footnotemark
%     \pl{mag}\\
%     \pl{paperwidth}\\
%     \pl{paperheight}\\
%     \pl{stockwidth}\\
%     \pl{stockheight}\\
%     \pl{pdfpageheight}\\
%     \pl{pdfpagewidth}\\
%     \pl{pdfhorigin}\\
%     \pl{pdfvorigin}\\
%     \pl{hoffset}\\
%     \pl{voffset}\\
%     \pl{topmargin}\\
%     \pl{oddsidemargin}\\
%     \pl{evensidemargin}\\
%     \pl{textwidth}\\
%     \pl{textheight}\\
%     \pl{headheight}\\
%     \pl{headsep}\\
%     \pl{footskip}\\
%     \pl{marginparwidth}\\
%     \pl{marginparsep}\\
%     \pl{columnwidth}\\
%     \pl{columnsep}\\
%   \hline
%   perpage
%     & pagevalue & perpage & 0\\
%     & page & perpage & \EMPTY\\
%     & abspage & perpage & 0\\
%   \hline
%   savepos
%     & posx & savepos & 0\\
%     & posy & savepos & 0\\
%   \hline
%   titleref
%     & title & main & \EMPTY\\
%   \hline
%   xr
%     & anchor   & & \EMPTY\\
%     & externaldocument & & \EMPTY\\
%     & theotype & & \EMPTY\\
%     & title    & & \EMPTY\\
%     & url      & & \EMPTY\\
% \end{tabular}
% \footnotetext{Module \xmodule{pagelayout} only defines properties
% if the parameter exists.}%
% \end{quote}
%
% \subsection{Wrapper for advanced situations}
%
% \begin{declcs}{zref@wrapper@babel}
%   |{|\dots|}| \M{name}
% \end{declcs}
% This macro helps to add shorthand support. The second
% argument is protected, then the code of the first argument
% is called with the protected name appended. Examples
% are in the sources.
%
% \begin{declcs}{zref@wrapper@immediate}
%   |{|\dots|}|
% \end{declcs}
% There are situations where a label must be written instantly
% to the \xfile{.aux} file, for example after the last page.
% If the \cs{zlabel} or \cs{label} command is put inside this wrapper,
% immediate writing is enabled. See the implementation for
% module \xmodule{lastpage} for an example of its use.
%
% \begin{declcs}{zref@wrapper@unexpanded}
%   |{|\dots|}|
% \end{declcs}
% Assuming someone wants to extract a value for property |bar|
% and store the result in a macro |\foo| without traces of
% the expanding macros and without expanding the value.
% This (theoretical?) problem can be solved by this wrapper:
% \begin{quote}
%\begin{verbatim}
%\zref@wrapper@unexpanded{%
%  \edef\foo{%
%    \zref@extract{someref}{bar}%
%  }%
%}
%\end{verbatim}
% \end{quote}
% The \cs{edef} forces the expansion of \cs{zref@extract},
% but the extraction of the value is prevented by
% the wrapper that uses \hologo{eTeX}'\ \cs{unexpanded} for this
% purpose. Supported macros are \cs{zref@extract}, \cs{zref@extractdefault}
% and since version 2010/04/22 v2.14 macro \cs{zref@getcurrent}.
%
% \subsection{Counter for unique names}
%
% Some modules (\xmodule{titleref} and \xmodule{dotfillmin})
% need unique names for automatically generated label names.
%
% \begin{declcs}{zref@require@unique}
% \end{declcs}
% This command creates the unique counter |zref@unique| if the
% counter does not already exist.
%
% \begin{declcs}{thezref@unique}
% \end{declcs}
% This command is used to generate unique label names.
%
% \section{User interface}\label{sec:user}
%
% \subsection{Module \xmodule{user}}
%
% The user interface for this package and its modules
% is enabled by \xpackage{zref}'s package option \xmodule{user}
% or package \xpackage{zref-user}.
% The names of user commands are prefixed by |z| in order
% to avoid name clashes with existing macros of the same
% functionality. Thus the package does not disturb the
% traditional reference scheme, both can be used together.
%
% The syntax descriptions contain the following markers
% that are intended as hints for programmers:
% \begin{quote}
% \begin{tabular}{@{}ll@{}}
%   \markbabel\unskip  & Babel shorthands are allowed.\\
%   \markrobust\unskip & Robust macro.\\
%   \markexp\unskip    & Expandable version:\\
%                      & \textbullet\ robust,
%                        unless the extracted values are fragile,\\
%                      & \textbullet\ no babel shorthand suport.\\
%   \markexptwo\unskip & Expandable like \markexp and:\\
%                      & \textbullet\ expandable in exact two steps.\\
% \end{tabular}
% \end{quote}
%
% The basic user interface of the package without modules
% are commands that mimic the standard \hologo{LaTeX} behaviour
% of \cs{label}, \cs{ref}, and \cs{pageref}:
%
% \begin{declcs}{zlabel}
%   \M{refname}\markbabel
% \end{declcs}
% Similar to \cs{label}. It generates a label with name
% \meta{refname} in the new reference scheme.
%
% \begin{declcs}{zref}
%   |[|\meta{propname}|]| \M{refname}\markbabel
% \end{declcs}
% Without optional argument similar to \cs{ref}, it returns
% the default reference property. This property is named
% |default|:
% \[
%   |\zref{|x|}| \equiv |\zref[default]{|x|}|
% \]
%
% \begin{declcs}{zpageref}
%   \M{refname}\markbabel
% \end{declcs}
% Convenience macro, similar to \cs{pageref}.
% \[
%   |\zpageref{|x|}| \equiv |\zref[page]{|x|}|
% \]
%
% \begin{declcs}{zrefused}
%   \M{refname}\markbabel
% \end{declcs}
% Some of the user commands in the modules are expandable.
% The use of such commands do not cause any undefined
% reference warnings, because inside of expandable
% contexts this is not possible. However, if there is
% a place outside of expandable contexts, \cs{refused}
% is strongly recommended. The reference \meta{refname}
% is marked as used, undefined ones will generate
% warnings.
%
% \subsection{Module \xmodule{abspage}}
%
% With the help of package \xpackage{atbegshi}
% a new counter |abspage| with absolute page numbers is provided.
% For technical and historical reasons the counter itself is zero based:
% if you use it directly on the first page, e.g
% with |\arabic{abspage}| you will get 0 as value. When using |\zref|
% the first page will be page 1 as expected.
% Also a new property |abspage| is defined and added to the
% main property list. Thus you can reference the absolute
% page number:
% \begin{quote}
%   |Section \zref{foo} is on page \zpageref{foo}.|\\
%   |This is page \zref[abspage]{foo}|\\
%   |of \zref[abspage]{LastPage}.|
% \end{quote}
% The example also makes use of module \xmodule{lastpage}.
%
% \subsection{Module \xmodule{lastpage}}
%
% Provides the functionality of package \xpackage{lastpage}
% \cite{lastpage} in the new reference scheme.
% The label |LastPage| is put at the end of the document.
% You can refer the last page number with:
% \begin{quote}
%   |\zref@extract{LastPage}{page}| (+ |\zref@refused{LastPage}|)\\
% \end{quote}
% or
% \begin{quote}
%   |\zpageref{LastPage}| (module \xmodule{user})
% \end{quote}
%
% Since version 2008/10/01 v2.3 the module defines the list |LastPage|.
% In addition to the properties of the main list label |LastPage| also
% stores the properties of this list |LastPage|. The default of this
% list is empty. The list can be used by the user to add additional
% properties for label |LastPage|.
%
% \subsubsection{Tests for last page}
%
% Since version 2010/03/26 v2.8 the macros \cs{zref@iflastpage}
% and \cs{ziflastpage} were added. They test the reference,
% whether it is a reference of the last page.
%
% \begin{declcs}{zref@iflastpage}\markexp \M{refname} \M{then} \M{else}
% \end{declcs}
% Macro \cs{zref@iflastpage} compares the references \meta{refname}
% with \meta{LastPage}. Basis of the comparison is the value of
% property |abspage|, because the values are different for different
% pages. This is not ensured by property |page|. Therefore module
% \xmodule{abspage} is loaded by module \xmodule{lastpage}.
% If both values of property |abspage| are present and match,
% then \meta{then} is executed, otherwise code \meta{else} is called.
% If one or both references are undefined or lack the property |abspage|,
% then \meta{else} is executed.
%
% Macro \cs{zref@iflastpage} is expandable, therefore \cs{zref@refused}
% should be called on \meta{refname} and \meta{LastPage}.
%
% \begin{declcs}{ziflastpage} \M{refname} \M{then} \M{else}
% \end{declcs}
% Macro \cs{ziflastpage} has the same function as \cs{zref@iflastpage},
% but adds support for babel shorthands in \meta{refname} and
% calls \cs{zref@refused}. However macro \cs{ziflastpage} is not
% expandable.
%
% \subsubsection{Example}
%
%    \begin{macrocode}
%<*example-lastpage>
%<<END_EXAMPLE
\NeedsTeXFormat{LaTeX2e}
\documentclass{report}

\newcounter{foo}
\renewcommand*{\thefoo}{\Alph{foo}}

\usepackage{zref-lastpage,zref-user}[2019/11/29]

\makeatletter
\zref@newprop{thefoo}{\thefoo}
\zref@newprop{valuefoo}{\the\value{foo}}
\zref@newprop{chapter}{\thechapter}
\zref@addprops{LastPage}{thefoo,valuefoo,chapter}
\makeatother

\newcommand*{\foo}{%
  \stepcounter{foo}%
  [Current foo: \thefoo]%
}

\begin{document}
  \chapter{First chapter}
  Last page is \zref{LastPage}.\\
  Last chapter is \zref[chapter]{LastPage}.\\
  Last foo is \zref[thefoo]{LastPage}.\\
  Last value of foo is \zref[valuefoo]{LastPage}.\\
  \foo
  \chapter{Second chapter}
  \foo\foo\foo
  \chapter{Last chapter}
  \foo
\end{document}
%END_EXAMPLE
%</example-lastpage>
%    \end{macrocode}
%
% \subsection{Module \xmodule{thepage}}
%
% This module \xmodule{thepage} loads module \xmodule{abspage},
% constructs a reference name using the absolute page number
% and remembers property |page|. Other properties can be added
% by adding them to the property list |thepage|.
%
% \begin{declcs}{zthepage} \M{absolute page number}\\
% \end{declcs}
% Macro \cs{zthepage} is basically a \cs{zpageref}. The reference
% name is yield by the \meta{absolute page number}. If the reference
% is not defined, then the default for property |page| is used.
%
% \begin{declcs}{zref@thepage@name}\markexp \M{absolute page number}
% \end{declcs}
% Macro \cs{zref@thepage@name} returns the internal reference
% name that is constructed using the \meta{absolute page number}.
% The internal reference name should not be used directly, because
% it might change in future versions.
%
% \begin{declcs}{zref@thepage}\markexp \M{absolute page number}\\
%   \cs{zref@thepage@refused} \M{absolute page number}
% \end{declcs}
% Macro \cs{zref@thepage} returns the page number (\cs{thepage})
% of \meta{absolute page number}. Because this macro is expandable,
% \cs{zref@thepage@refused} is used outside an expandable context
% to mark the reference as used.
%
%
% \subsection{Module \xmodule{nextpage}}
%
% \begin{declcs}{znextpage}
% \end{declcs}
% Macro \cs{znextpage} prints \cs{thepage} of the following page.
% It gets the current absolute page number by using a label.
% There are three cases for the next page:
% \begin{enumerate}
% \item The next page is not known yet because of undefined references.
%   Then \cs{zunknownnextpagename} is used instead. The default for
%   this macro is the default of property |page|.
% \item This page is the last page. Then \cs{znonextpagename}
%   is used. Its default is empty.
% \item The next page is known, then \cs{thepage} of the next
%   page is used (the value of property |page| of the next page).
% \end{enumerate}
%
% \subsubsection{Configuration}
%
% The behaviour can be configured by the following macros.
%
% \begin{declcs}{zunknownnextpagename}\\
%   \cs{znonextpagename}
% \end{declcs}
% If the next page is not known or available, then
% \cs{znextpage} uses these name macros as default.
% \cs{zunknownnextpagename} is used in case of undefined
% references. Default is the value of property |page| of the
% next page (\cs{thepage}). Module \xmodule{thepage} is used.
%
% Macro \cs{znonextpagename} is used, if the next page does
% not exists. That means that the current page is last page.
% The default is empty.
%
% \begin{declcs}{znextpagesetup} \M{unknown} \M{no next}
%   \M{next}
% \end{declcs}
% Acording to the case (see \cs{znextpage}) macro \cs{znextpage}
% calls an internal macro with an argument. The argument is
% either \cs{thepage} of the next page or one of \cs{zunknownnextpagename}
% or \cs{znonextpagename}. These internal macro can be changed by
% \cs{znextpagesetup}. It expects the definition texts for these
% three cases of a macro with one argument.
% The default is
% \begin{quote}
%   |\znextpagesetup{#1}{#1}{#1}|
% \end{quote}
%
% \subsubsection{Example}
%
%    \begin{macrocode}
%<*example-nextpage>
%<<END_EXAMPLE
\documentclass{book}

\usepackage{zref-nextpage}[2019/11/29]
\znextpagesetup
  {\thepage}% next page is unknown
  {\thepage\ (#1)}% this page is last page
  {\thepage\ $\rightarrow$ #1}% next page is known
\renewcommand*{\znonextpagename}{last page}

\usepackage{fancyhdr}
\pagestyle{fancy}
\fancyhf{}
\fancyhead[LE,RO]{\znextpage}
\fancypagestyle{plain}{%
  \fancyhf{}%
  \fancyhead[LE,RO]{\znextpage}%
}

\begin{document}
\frontmatter
  \tableofcontents
\mainmatter
  \chapter{Hello World}
  \clearpage
  \section{Last section}
\end{document}
%END_EXAMPLE
%</example-nextpage>
%    \end{macrocode}
%
%
% \subsection{Module \xmodule{totpages}}
%
% For the total number of pages of a document you need to
% know the absolute page number of the last page.
% Both modules \xmodule{abspage} and \xmodule{lastpage}
% are necessary and automatically enabled.
%
% \begin{declcs}{ztotpages}\markexp
% \end{declcs}
% Prints the total number of pages or |0| if this
% number is not yet known. It expands to an explicit number
% and can also used even in expandable calculations (\cs{numexpr})
% or counter assignments.
%
% \subsection{Module \xmodule{pagelayout}}
%
% The module defines additional properties for each parameter
% of the page layout that is effective during page shipout.
% The value of length parameters is given in sp without the
% unit as plain number.
%
% Some parameters are specific for a class (e.g.\@ \xoption{stockwidth}
% and \xoption{stockheight} for class \xclass{memoir}) or the
% \hologo{TeX} engine like \hologo{pdfTeX}. If the parameter is not
% available, then the property will not be defined. The default value
% of the property is the current setting of the parameter.
%
% The module \xmodule{thepage} is loaded that generates a label
% for each page. The properties of module \xmodule{pagelayout}
% are added to the property list |thepage| of module \xmodule{thepage}.
%
% List of properties:
% \begin{quote}
%   \def\x#1{\cs{#1}&#1}%
%   \begin{tabular}{l>{\sffamily}ll}
%     \textrm{parameter} & \textrm{property} & remarks\\
%     \hline
%     \x{mag}\\
%     \x{paperwidth}\\
%     \x{paperheight}\\
%     \x{stockwidth}& class \xclass{memoir}\\
%     \x{stockheight}& class \xclass{memoir}\\
%     \x{pdfpagewidth}& \hologo{pdfTeX}, \hologo{LuaTeX}\\
%     \x{pdfpageheight}& \hologo{pdfTeX}, \hologo{LuaTeX}\\
%     \x{pdfhorigin}& \hologo{pdfTeX}, \hologo{LuaTeX}\\
%     \x{pdfvorigin}& \hologo{pdfTeX}, \hologo{LuaTeX}\\
%     \x{hoffset}\\
%     \x{voffset}\\
%     \x{topmargin}\\
%     \x{oddsidemargin}\\
%     \x{evensidemargin}\\
%     \x{textwidth}\\
%     \x{textheight}\\
%     \x{headheight}\\
%     \x{headsep}\\
%     \x{footskip}\\
%     \x{marginparwidth}\\
%     \x{marginparsep}\\
%     \x{columnwidth}\\
%     \x{columnsep}\\
%   \end{tabular}
% \end{quote}
%
% \begin{declcs}{zlistpagelayout}
% \end{declcs}
% At the end of document the page layout parameter for each page are
% printed into the \xext{log} file if macro \cs{zlistpagelayout}
% is called before |\end{document}| (preamble is a good place).
%
% \subsection{Module \xmodule{marks}}
%
% ToDo.
%
% \subsection{Module \xmodule{runs}}
%
% Module \xmodule{runs} counts the \hologo{LaTeX} runs since last
% \xext{aux} file creation and prints the number in the
% \xext{log} file.
%
% \begin{declcs}{zruns}\markexp
% \end{declcs}
% Prints the the total number of \hologo{LaTeX} runs including
% the current one. It expands to an explicit number.
% Before |begin{document}| the value is zero meaning
% the \xext{aux} file is not read yet. If a previous
% \xext{aux} file exists, the value found there increased
% by one is the new number. Otherwise \cs{zruns} is set to one.
% \hologo{LaTeX}~runs where the \xext{aux} files are not rewritten
% are not counted (see \cs{nofiles}).
%
% \subsection{Module \xmodule{perpage}}
%
% With \cs{@addtoreset} or \cs{numberwithin} a counter can be
% reset if another counter is incremented. This do not work
% well if the other counter is the page counter. The page counter is
% incremented in the output routine that is often called asynchronous
% somewhere on the next page. A reference mechanism costs at least
% two \hologo{LaTeX} runs, but ensures correct page counter values.
%
% \begin{declcs}{zmakeperpage}
%   |[|\meta{reset}|]| \M{counter}
% \end{declcs}
% At the of a new page counter \meta{counter} starts
% counting with value \meta{reset} (default is \texttt{1}).
% The macro has the same
% syntax and semantics as \cs{MakePerPage} of package
% \xpackage{perpage} \cite{perpage}.
% Also \xoption{perpage} of package \xpackage{footmisc} \cite{footmisc}
% can easily be simulated by
% \begin{quote}
% |\zmakeperpage{footnote}| \itshape |% \usepackage[perpage]{footmisc}|
%\end{quote}
% If footnote symbols are used, some people dislike the
% first symbol \ensuremath{\dagger}. It can easily be skipped:
% \begin{quote}
% |\zmakeperpage[2]{footnote}|
% \end{quote}
%
% \begin{declcs}{thezpage}\\
%   counter |zpage|
% \end{declcs}
% If the formatted counter value of the counter that is reset at
% a new page contains the page value, then you can use \cs{thezpage},
% the page number of the current page. Or counter |zpage| can be
% used, if the page number should be formatted differently from
% the current page number. Example:
%\begin{quote}
%\begin{verbatim}
%\newcounter{foobar}
%\zmakeperpage{foobar}
%\renewcommand*{\thefoobar}{\thezpage-\arabic{foobar}}
% % or
%\renewcommand*{\thefoobar}{\roman{zpage}-\arabic{foobar}}
%\end{verbatim}
%\end{quote}
%
% \begin{declcs}{zunmakeperpage}
%   \M{counter}
% \end{declcs}
% The reset mechanism for this counter is deactivated.
%
%
% \subsection{Module \xmodule{counter}}
%
% This option just add the property |counter| to the main
% property list. The property stores the counter name,
% that was responsible for the reference. This is the
% property \xpackage{hyperref}'s \cs{autoref} feature
% uses. Thus this property |counter| may be useful
% for a reimplementation of the autoref feature,
% see the section \ref{sec:todo} with the todo list.
%
% \subsection{Module \xmodule{titleref}}
%
% This option makes section and caption titles available
% to the reference system similar to packages \xpackage{titleref}
% or \xpackage{nameref}.
%
% \begin{declcs}{ztitleref}
%   \M{refname}\markbabel
% \end{declcs}
% Print the section or caption title of reference \meta{refname},
% similar to \cs{nameref} or \cs{titleref}.
%
% \begin{declcs}{ztitlerefsetup}
%   |{|\textit{key$_1$=value$_1$, key$_2$=value$_2$, \dots}|}|
% \end{declcs}
% This command allows to configure the behaviour of module
% \xmodule{titleref}. The following keys are available:
% \begin{description}
% \let\bfseries\relax
% \item[\texttt{title=}\meta{value}]\mbox{}\\*
%   Sets the current title.
% \item[\texttt{stripperiod=true\string|false}]\mbox{}\\*
%   Follow package \xpackage{nameref} that
%   removes a last period. Default: |true|.
% \item[\texttt{expand=true\string|false}]\mbox{}\\*
%   Package \cs{titleref} expands the title first. This way
%   garbage and dangerous commands can be removed, e.g.
%   \cs{label}, \cs{index}\dots. See implementation section
%   for more details. Default is |false|.
% \item[\texttt{cleanup=\{}\textrm{\dots}\texttt{\}}]\mbox{}\\*
%   Hook to add own cleanup code, if method |expand| is used.
%   See implementation section for more details.
% \end{description}
%
% \subsection{Module \xmodule{savepos}}
%
% This option supports a feature that \pdfTeX\ provides
% (and \hologo{XeTeX}).
% \pdfTeX\ is able to tell the current position on the
% page. The page position is not instantly known. First the
% page must be constructed by \hologo{TeX}'s asynchronous output
% routine. Thus the time where the position is known is
% the page shipout time. Thus a reference system where the
% information is recorded in the first run and made available
% for use in the second run comes in handy.
%
% \begin{declcs}{zsavepos}
%   \M{refname}
% \end{declcs}
% It generates a reference with name \meta{refname}.
% The reference stores the location where \cs{zsavepos}
% is executed in properties |posx| and |posy|.
%
% \begin{declcs}{zsaveposx}
%   \M{refname}\\
%   \cs{zsaveposy}
%   \M{refname}
% \end{declcs}
% Same as \cs{zsavepos} except that only the |x| or |y| component
% of the position is stored. Since 2011/12/05 v2.23.
%
% \begin{declcs}{zposx}\markexp
%   \M{refname}\\
%   \SpecialUsageIndex{\zposy}\cs{zposy}\markexp \M{refname}
% \end{declcs}
% Get the position as number. Unit is sp. Horizontal
% positions by \cs{zposx} increase from left to right.
% Vertical positions by \cs{zposy} from bottom to top.
%
% Do not rely on absolute page numbers. Because of problems
% with the origin the numbers may differ in DVI or PDF mode
% of \pdfTeX. Therefore work with relative values by comparisons.
%
% Both \cs{zposx} and \cs{zposy} are expandable and can be
% used inside calculations (\cs{setcounter}, \cs{addtocounter},
% package \xpackage{calc}, \cs{numexpr}). However this property
% prevents from notifying \hologo{LaTeX} that the reference is actually
% used (the notifying is not expandable). Therefore you should
% mark the reference as used by \cs{zrefused}.
%
% This module uses \pdfTeX's \cs{pdfsavepos}, \cs{pdflastxpos},
% and \cs{pdflastypos}. They are available in PDF mode and
% since version 1.40.0 also in DVI mode.
%
% \begin{declcs}{zref@savepos}
% \end{declcs}
% Macro \cs{zref@savepos} performs the first part of \cs{zsavepos}
% by calling \cs{pdfsavepos} (if \xext{aux} files are writable).
%
% Thus \cs{zsavepos} is basically \cs{zref@savepos} followed
% by |\zref@labelbylist{|\texttt{\meta{refname}}|}{savepos}|.
% If \cs{TeXXeTstate} is detected and enabled, \cs{savepos}
% also adds \cs{zref@savepos} at the end to support \cs{beginR}
% where the whatits are processed in reverse order.
% The property list |savepos| contains the properties |posx|
% and |posy|.
%
% \subsection{Module \xmodule{abspos}}
% Module \xmodule{abspos} allows to get various values of the page layout. There is no user
% command, only a number of internal commands. For example:
% \begin{quote}
%   |\zref@absposx{|\meta{label}|}{|\meta{value}|}{|\meta{position}|}|\\
%   |\zref@absposy{|\meta{label}|}{|\meta{value}|}{|\meta{position}|}|
% \end{quote}
%
% The return value is like in the module \xmodule{savepos} a number representing a length in sp.
% The length are measured from the bottom left of the page.
%
% \meta{label} is a label set with \cs{zlabel} or \cs{zsavepos} that allows to retrieve the
% absolute page number.
%
% \meta{position} is for the x-command one of |left|, |right| or |center|. For the y-command
%  it is one of |top|, |bottom|, |center|.
%
% The possible content of \meta{value} can be seen in the following table. Be aware that the code
% makes some assumptions which are perhaps not always true -- for example that the left of
% the head is identical to the left of the body.
%
% \medskip
% \begin{tabular}{lll}
%  \textbf{value} & \textbf{axis} & \textbf{comments}\\\hline
%   media         & x             &   left=0, right=\cs{pdfpagewidth} \\
%   paper         & x             &   left=0, right=\cs{paperwidth}\\
%   stock         & x             &   derived from paper\\
%   media         & y             & bottom=0, top=\cs{pdfpageheigh}\\
%   paper         & y             & top=\cs{pdfpageheight}, bottom=top-\cs{paperheight}\\
%   stock         & y             & top  derived from paper\\
%   head          & x             & calculated with hoffset, horigin, etc\\
%   head          & y             & calculated\\
%   body          & x             & = head value\\%
%   body          & y             & = head bottom - \cs{headsep}\\
%   foot          & x             & = head\\
%   foot          & y             & calculated from body bottom and \cs{footskip}\\
%   marginpar     & x             &different on odd/even pages! \\
%   marginpar     & y             & = body %
% \end{tabular}
%
% \subsection{Module \xmodule{dotfill}}
%
% \begin{declcs}{zdotfill}
% \end{declcs}
% This package provides the command \cs{zdotfill} that works
% similar to \cs{dotfill}, but can be configured. Especially
% it suppresses the dots if a minimum number of dots cannot be set.
%
% \begin{declcs}{zdotfillsetup}
%   |{|\textit{key$_1$=value$_1$, key$_2$=value$_2$, \dots}|}|
% \end{declcs}
% This command allows to configure the behaviour of \cs{zdotfill}.
% The following keys are available:
% \begin{description}
% \let\bfseries\relax
% \item[\texttt{min=}\meta{count value}]\mbox{}\\*
%   If the actual number of dots are smaller than \meta{count value},
%   then the dots are suppressed. Default: |2|.
% \item[\texttt{unit=}\meta{dimen value}]\mbox{}\\*
%   The width of a dot unit is given by \meta{dimen value}.
%   Default: |0.44em| (same as the unit in \cs{dotfill}).
% \item[\texttt{dot=}\meta{value}]\mbox{}\\*
%   The dot itself is given by \meta{value}. Default: |.| (dot, same as
%   the dot in \cs{dotfill}).
% \end{description}
%
% \subsection{Module \xmodule{env}}
%
% This module defines two properties |envname| and |envline|. They
% remember the name of the environment and the line number at
% the start of the environment.
%
% \subsection{Module \xmodule{xr}}
%
% This package provides the functionality of package \xpackage{xr},
% see \cite{xr}. It also supports the syntax of \xpackage{xr-hyper}.
%
% \begin{declcs}{zexternaldocument}
%   \* |[|\meta{prefix}|]|\markbabel \M{external document}
%   |[|\meta{url}|]|
% \end{declcs}
% See \cs{externaldocument} for a description of this option.
% The found labels also get a property |externaldocument| that
% remembers \meta{external document}.
% The standard reference scheme and the scheme of this
% package use different name spaces for reference names.
% If the external document uses both systems. Then one
% import statement would put the names in one namespace
% and probably causing problems with multiple references
% of the same name. Thus the star form only looks for
% \cs{newlabel} in the \xfile{.aux} files, whereas without
% star only \cs{zref@newlabel}s are used.
%
% In the star form it tries to detect labels from
% \xpackage{hyperref}, \xpackage{titleref}, and \xpackage{ntheorem}.
% If such an extended property from the packages before cannot
% be found or are empty, they are not included in the
% imported reference.
%
% Warnings are given if a reference name is already in use
% and the item is ignored. Unknown properties will automatically
% be declared.
%
% If the external references contain |anchor| properties,
% then we need also a url to be able to address the external
% file. As default the filename is taken with a default
% extension.
%
% \begin{declcs}{zxrsetup}
%   |{|\textit{key$_1$=value$_1$, key$_2$=value$_2$, \dots}|}|
% \end{declcs}
% The following setup options are available:
% \begin{description}
% \item[\xoption{ext}:] It sets the default extension.
% \item[\xoption{tozreflabel}:] Boolean option. The found references
%   are imported as zref labels. This is enabled by default.
% \item[\xoption{toltxlabel}:] Boolean option. The found references
%   are imported as \hologo{LaTeX} labels. Packages \xpackage{nameref},
%   \xpackage{titleref} and class \xclass{memoir} are supported.
% \item[\xoption{urluse}:] Boolean option. If enabled, then a URL
%   is stored in a macro and the macro is put in property `urluse'.
%   The URL is not put in property `url'.
%   The purpose is to save \hologo{TeX} memory.
% \item[\xoption{verbose}:] Boolean option. List the imported labels in the
%   \xext{log} file. Default is |false|.
% \end{description}
%
% \begin{declcs}{zref@xr@ext}
% \end{declcs}
% If the \meta{url} is not specified in \cs{zref@externaldocument},
% then the url will be constructed with the file name and
% this macro as extension. \cs{XR@ext} is used if
% \xpackage{hyperref} is loaded, otherwise |pdf|.
%
% \subsection{Module \xmodule{pageattr}}
%
% This module allows to recover the content of the register
% \cs{pdfpageattr} and \cs{pdfpagesattr} in pdftex and the equivalent register
% in luatex. There is no user command. Programmer commands are
%
% \begin{quote}
% |\zref@pdfpageattr{|\meta{absolute page number}|}|
%
% |\zref@pdfpagesattr{|\meta{absolute page number}|}|
% \end{quote}
% \section{ToDo}\label{sec:todo}
%
% Among other things the following issues are left for future work:
% \begin{itemize}
% \item Other applications: autoref, hyperref, \dots
% \end{itemize}
%
% \section{Example}
%
%    \begin{macrocode}
%<*example>
%    \end{macrocode}
%
%    \begin{macrocode}
\documentclass{book}

\usepackage[ngerman]{babel}%

\usepackage[savepos,totpages,titleref,dotfill,counter,user]{zref}

%    \end{macrocode}
%
%    Chapters are wrapped inside \cs{ChapterStart} and
%    \cs{ChapterStop}. The first argument \verb|#1| of \cs{ChapterStart}
%    is used to form a label id \verb|chap:#1|. At the end of
%    the chapter another label is set by \cs{zref@wrapper@immediate}, because
%    otherwise at the end of document a deferred write would not
%    be written, because there is no page for shipout.
%
%    Also this example shows how chapter titles can be recorded.
%    A new property \verb|chaptitle| is declared and added to the
%    main property list. In \cs{ChapterStart} the current value
%    of the property is updated.
%    \begin{macrocode}
\makeatletter
\zref@newprop{chaptitle}{}
\zref@addprop{main}{chaptitle}

\newcommand*{\ChapterStart}[2]{%
  \cleardoublepage
  \def\current@chapid{#1}%
  \zref@setcurrent{chaptitle}{#2}%
  \chapter{#2}%
  \zlabel{chap:#1}%
}
\newcommand*{\ChapterStop}{%
  \cleardoublepage
  \zref@wrapper@immediate{%
    \zref@labelbyprops{chapend:\current@chapid}{abspage}%
  }%
}
%    \end{macrocode}
%    \cs{ChapterPages} calculates and returns the number of pages of the
%    referenced chapter.
%    \begin{macrocode}
\newcommand*{\ChapterPages}[1]{%
  \zrefused{chap:#1}%
  \zrefused{chapend:#1}%
  \number\numexpr
    \zref@extract{chapend:#1}{abspage}%
    -\zref@extract{chap:#1}{abspage}%
    +1\relax
}
\makeatother
%    \end{macrocode}
%    \begin{macrocode}
\begin{document}
%    \end{macrocode}
%    As exception we use \cs{makeatletter} here, because
%    this is just an example file that also should show some of
%    programmer's interface.
%    \begin{macrocode}
\makeatletter

\frontmatter
\zlabel{documentstart}

\begin{itemize}
\item
  The frontmatter part has
  \number\numexpr\zref@extract{chap:first}{abspage}-1\relax
  ~pages.
\item
  Chapter \zref{chap:first} has \ChapterPages{first} page(s).
\item
  Section \zref{hello} is on the
  \ifcase\numexpr
    \zref@extractdefault{hello}{page}{0}%
    -\zref@extractdefault{chap:first}{page}{0}%
    +1\relax
    ??\or first\or second\or third\or forth\fi
  ~page inside its chapter.
\item
  The document has
  \zref[abspage]{LastPage} pages.
  This number is \ifodd\ztotpages odd\else even\fi.
\item
  The last page is labeled with \zpageref{LastPage}.
\item
  The title of chapter \zref{chap:next} %
  is ``\zref[chaptitle]{chap:next}''.
\end{itemize}

\tableofcontents

\mainmatter
\ChapterStart{first}{First chapter}

%    \end{macrocode}
%    The user level commands should protect babel shorthands
%    where possible. On the other side, expandable extracting
%    macros are useful in calculations, see above the
%    examples with \cs{numexpr}.
%    \begin{macrocode}
\section{Test}
\zlabel{a"o}
Section \zref{a"o} on page
\zref@wrapper@babel\zref@extract{a"o}{page}.

Text.
\newpage

\section{Hello World}
\zlabel{hello}

\ChapterStop

\ChapterStart{next}{Next chapter with \emph{umlauts}: "a"o"u"s}

%    \end{macrocode}
%
%    Here an example follows that makes use of \pdfTeX's ``savepos''
%    feature. The position on the page is not known before the
%    page is constructed and shipped out. Therefore the position
%    ist stored in references and are available for calculations
%    in the next \hologo{LaTeX} compile run.
%    \begin{macrocode}
The width of the first column is
  \the\dimexpr \zposx{secondcol}sp - \zposx{firstcol}sp\relax,\\
the height difference of the two baselines is
  \the\dimexpr \zposy{firstcol}sp - \zposy{secondline}sp\relax:\\
\begin{tabular}{ll}
  \zsavepos{firstcol}Hello&\zsavepos{secondcol}World\\
  \zsavepos{secondline}Second line&foobar\\
\end{tabular}

%    \end{macrocode}
%    With \cs{zrefused} \hologo{LaTeX} is notified, if the references
%    are not yet available and \hologo{LaTeX} can generate the rerun
%    hint.
%    \begin{macrocode}
\zrefused{firstcol}
\zrefused{secondcol}
\zrefused{secondline}

\ChapterStop
%    \end{macrocode}
%    Test for module \cs{dotfill}.
%    \begin{macrocode}
\ChapterStart{dotfill}{Test for dotfill feature}
\newcommand*{\dftest}[1]{%
  #1&
  [\makebox[{#1}]{\dotfill}]&
  [\makebox[{#1}]{\zdotfill}]\\
}
\begin{tabular}{rll}
& [\verb|\dotfill|] & [\verb|\zdotfill|]\\
\dftest{0.43em}
\dftest{0.44em}
\dftest{0.45em}
\dftest{0.87em}
\dftest{0.88em}
\dftest{0.89em}
\dftest{1.31em}
\dftest{1.32em}
\dftest{1.33em}
\end{tabular}
\ChapterStop
\end{document}
%    \end{macrocode}
%
%    \begin{macrocode}
%</example>
%    \end{macrocode}
%
% \StopEventually{
% }
%
% \section{Implementation}
%
% \subsection{Package \xpackage{zref}}
%
% \subsubsection{Identification}
%
%    \begin{macrocode}
%<*package>
\NeedsTeXFormat{LaTeX2e}
\ProvidesPackage{zref}
  [2020-07-03 v2.32 A new reference scheme for LaTeX (HO)]%
%    \end{macrocode}
%
% \subsubsection{Load basic module}
%
%    \begin{macrocode}
\RequirePackage{zref-base}[2019/11/29]
%    \end{macrocode}
%    Abort package loading if \xpackage{zref-base} could not be
%    loaded successfully.
%    \begin{macrocode}
\@ifundefined{ZREF@base@ok}{\endinput}{}
%    \end{macrocode}
%
% \subsubsection{Process options}
%
%    Known modules are loaded and the release date is checked.
%    \begin{macrocode}
\def\ZREF@temp#1{%
  \DeclareOption{#1}{%
    \AtEndOfPackage{%
      \RequirePackage{zref-#1}[2019/11/29]%
    }%
  }%
}
\ZREF@temp{abspage}
\ZREF@temp{counter}
\ZREF@temp{dotfill}
\ZREF@temp{hyperref}
\ZREF@temp{lastpage}
\ZREF@temp{marks}
\ZREF@temp{nextpage}
\ZREF@temp{pageattr}
\ZREF@temp{pagelayout}
\ZREF@temp{perpage}
\ZREF@temp{runs}
\ZREF@temp{savepos}
\ZREF@temp{thepage}
\ZREF@temp{titleref}
\ZREF@temp{totpages}
\ZREF@temp{user}
\ZREF@temp{xr}
%    \end{macrocode}
%    \begin{macrocode}
\ProcessOptions\relax
%</package>
%    \end{macrocode}
%
%
% \subsection{Module \xmodule{base}}
%
% \subsubsection{Prefixes}
%
%    This package uses the following prefixes for macro names:
%    \begin{description}
%    \item[\cs{zref@}:] Macros of the programmer's interface.
%    \item[\cs{ZREF@}:] Internal macros.
%    \item[\cs{Z@L@}\metacs{listname}:]
%      The properties of the list \meta{listname}.
%    \item[\cs{Z@D@}\metacs{propname}:]
%      The default value for property \meta{propname}.
%    \item[\cs{Z@E@}\metacs{propname}:]
%      Extract function for property \meta{propname}.
%    \item[\cs{Z@X@}\metacs{propname}:]
%      Information whether a property value for property
%      \meta{propname} is expanded
%      immediately or at shipout time.
%    \item[\cs{Z@C@}\metacs{propname}:]
%      Current value of the property \meta{propname}.
%    \item[\cs{Z@R@}\metacs{labelname}:]
%      Data for reference \meta{labelname}.
%    \item[\cs{ZREF@org@}:] Original versions of patched commands.
%    \item[\cs{z}:] For macros in user land, defined if
%      module \xmodule{user} is set.
%    \end{description}
%    The following family names are used for keys defined
%    according to the \xpackage{keyval} package:
%    \begin{description}
%    \item[\texttt{ZREF@TR}:] Setup for module \xmodule{titleref}.
%    \end{description}
%
% \subsubsection{Identification}
%
%    \begin{macrocode}
%<*base>
\NeedsTeXFormat{LaTeX2e}
\ProvidesPackage{zref-base}%
  [2020-07-03 v2.32 Module base for zref (HO)]%
%    \end{macrocode}
%
% \subsubsection{Utilities}
%
%    \begin{macrocode}
\providecommand\IfFormatAtLeastTF{\@ifl@t@r\fmtversion}
\RequirePackage{ltxcmds}[2010/12/02]
\RequirePackage{infwarerr}[2010/04/08]
\RequirePackage{kvsetkeys}[2010/03/01]
\RequirePackage{kvdefinekeys}[2010/03/01]
\RequirePackage{pdftexcmds}[2010/04/01]
%    \end{macrocode}
%
%    \begin{macro}{\ZREF@name}
%    Several times the package name is used, thus we store it
%    in \cmd{\ZREF@name}.
%    \begin{macrocode}
\def\ZREF@name{zref}
%    \end{macrocode}
%    \end{macro}
%
%    \begin{macrocode}
\ltx@IfUndefined{protected}{%
  \RequirePackage{makerobust}[2006/03/18]%
%    \end{macrocode}
%    \begin{macro}{\ZREF@Robust}
%    \begin{macrocode}
  \def\ZREF@Robust#1#2{%
    \def\ZREF@temp{\MakeRobustcommand#2}%
    \afterassignment\ZREF@temp
    #1#2%
  }%
%    \end{macrocode}
%    \end{macro}
%    \begin{macrocode}
}{%
%    \end{macrocode}
%    \begin{macro}{\ZREF@Robust}
%    \begin{macrocode}
  \def\ZREF@Robust#1{%
    \protected#1%
  }%
%    \end{macrocode}
%    \end{macro}
%    \begin{macrocode}
}
%    \end{macrocode}
%    \begin{macro}{\ZREF@IfDefinable}
%    \begin{macrocode}
\def\ZREF@IfDefinable#1#2#3{%
  \@ifdefinable{#1}{%
    \ZREF@Robust{#2}#1#3%
  }%
}
%    \end{macrocode}
%    \end{macro}
%
%    \begin{macro}{\ZREF@UpdatePdfTeX}
%    \cs{ZREF@UpdatePdfTeX} is used as help message text in
%    error messages.
%    \begin{macrocode}
\def\ZREF@UpdatePdfTeX{Update pdfTeX.}
%    \end{macrocode}
%    \end{macro}
%
%    \begin{macro}{\ifZREF@found}
%    The following switch is usded in list processing.
%    \begin{macrocode}
\newif\ifZREF@found
%    \end{macrocode}
%    \end{macro}
%
%    \begin{macro}{\ZREF@patch}
%    Macro \cs{ZREF@patch} first checks the existence of
%    the command and safes it.
%    \begin{macrocode}
\def\ZREF@patch#1{%
  \ltx@IfUndefined{#1}{%
    \ltx@gobble
  }{%
    \expandafter\let\csname ZREF@org@#1\expandafter\endcsname
    \csname #1\endcsname
    \ltx@firstofone
  }%
}
%    \end{macrocode}
%    \end{macro}
%
% \subsubsection{Check for \hologo{eTeX}}
%
%    The use of \hologo{eTeX} should be standard nowadays for \hologo{LaTeX}.
%    We test for \hologo{eTeX} in order to use its features later.
%    \begin{macrocode}
\ltx@IfUndefined{eTeXversion}{%
  \PackageError\ZREF@name{%
    Missing support for eTeX; package is abandoned%
  }{%
    Use a TeX compiler that support eTeX and enable eTeX %
    in the format.%
  }%
  \endinput
}{}%
%    \end{macrocode}
%
%    \begin{macrocode}
\RequirePackage{etexcmds}[2007/09/09]
\ifetex@unexpanded
\else
  \PackageError\ZREF@name{%
    Missing e-TeX's \string\unexpanded.\MessageBreak
    Add \string\RequirePackage\string{etexcmds\string} before %
    \string\documentclass%
  }{%
    Probably you are using some package (e.g. ConTeXt) that %
    redefines \string\unexpanded%
  }%
  \expandafter\endinput
\fi
%    \end{macrocode}
%
% \subsubsection{Auxiliary file stuff}
%
%    We are using some commands in the \xfile{.aux} files.
%    However sometimes these auxiliary files are interpreted
%    by \hologo{LaTeX} processes that haven't loaded this package
%    (e.g. package \xpackage{xr}).
%    Therefore we provide dummy definitions.
%    \begin{macrocode}
\RequirePackage{auxhook}
\AddLineBeginAux{%
  \string\providecommand\string\zref@newlabel[2]{}%
}
%    \end{macrocode}
%    \begin{macro}{\ZREF@RefPrefix}
%    \begin{macrocode}
\def\ZREF@RefPrefix{Z@R}
%    \end{macrocode}
%    \end{macro}
%    \begin{macro}{\zref@newlabel}
%    For the implementation of \cs{zref@newlabel} we call
%    the same internal macro \cs{@newl@bel} that is used
%    in \cs{newlabel}. Thus we have for free:
%    \begin{itemize}
%    \item \cs{Z@R@}\metacs{labelname} is defined.
%    \item \hologo{LaTeX}'s check for multiple references.
%    \item \hologo{LaTeX}'s check for changed references.
%    \end{itemize}
%    \begin{macrocode}
\ZREF@Robust\edef\zref@newlabel{%
  \noexpand\@newl@bel{\ZREF@RefPrefix}%
}
%    \end{macrocode}
%    \end{macro}
%
% \subsubsection{Property lists}
%
%    \begin{macro}{\zref@newlist}
%    Property lists are stored as list of property names
%    enclosed in curly braces. \cs{zref@newlist}
%    creates a new list as empty list. Assignments to
%    property lists are global.
%    \begin{macrocode}
\ZREF@Robust\def\zref@newlist#1{%
  \zref@iflistundefined{#1}{%
    \@ifdefinable{Z@L@#1}{%
     \global\expandafter\let\csname Z@L@#1\endcsname\ltx@empty
     \PackageInfo\ZREF@name{New property list: #1}%
    }%
  }{%
    \PackageError\ZREF@name{%
      Property list `#1' already exists%
    }\@ehc
  }%
}
%    \end{macrocode}
%    \end{macro}
%
%    \begin{macro}{\zref@iflistundefined}
%    \cs{zref@iflistundefined} checks the existence of the
%    property list \verb|#1|. If the property list is present,
%    then \verb|#2| is executed and \verb|#3| otherwise.
%    \begin{macrocode}
\def\zref@iflistundefined#1{%
  \ltx@ifundefined{Z@L@#1}%
}
%    \end{macrocode}
%    \end{macro}
%    \begin{macro}{\zref@listexists}
%    \cs{zref@listexists} only executes \verb|#2| if the
%    property list \verb|#1| exists and raises an error
%    message otherwise.
%    \begin{macrocode}
\ZREF@Robust\def\zref@listexists#1{%
  \zref@iflistundefined{#1}{%
    \PackageError\ZREF@name{%
      Property list `#1' does not exist%
    }\@ehc
  }%
}
%    \end{macrocode}
%    \end{macro}
%
%    \begin{macro}{\zref@iflistcontainsprop}
%    \cs{zref@iflistcontainsprop} checks, whether a property \verb|#2| is
%    already present in a property list \verb|#1|.
%    \begin{macrocode}
\ZREF@Robust\def\zref@iflistcontainsprop#1#2{%
  \zref@iflistundefined{#1}{%
    \ltx@secondoftwo
  }{%
    \begingroup\expandafter\endgroup
    \expandafter\in@
    \csname#2\expandafter\expandafter\expandafter\endcsname
    \expandafter\expandafter\expandafter{\csname Z@L@#1\endcsname}%
    \csname ltx@\ifin@ first\else second\fi oftwo\endcsname
  }%
}
%    \end{macrocode}
%    \end{macro}
%
%    \begin{macro}{\zref@listforloop}
%    \begin{macrocode}
\def\zref@listforloop#1#2{%
  \zref@listexists{#1}{%
    \expandafter\expandafter\expandafter\@tfor
    \expandafter\expandafter\expandafter\zref@prop
    \expandafter\expandafter\expandafter:%
    \expandafter\expandafter\expandafter=%
    \csname Z@L@#1\endcsname
    \do{%
      \begingroup
        \escapechar=-1 %
        \edef\x{\endgroup
          \def\noexpand\zref@prop{%
            \expandafter\string\zref@prop
          }%
        }%
      \x
      #2\zref@prop
    }%
  }%
}
%    \end{macrocode}
%    \end{macro}
%
%    \begin{macro}{\zref@addprops}
%    \cs{zref@addprop} adds the properties \verb|#2| to the property
%    list \verb|#1|, if the property is not already in the list.
%    Otherwise a warning is given.
%    \begin{macrocode}
\ZREF@Robust\def\zref@addprops#1#2{%
  \zref@listexists{#1}{%
    \comma@parse{#2}{%
      \zref@propexists\comma@entry{%
        \zref@iflistcontainsprop{#1}\comma@entry{%
          \PackageWarning\ZREF@name{%
            Property `\comma@entry' is already in list `#1'%
          }%
        }{%
          \begingroup\expandafter\endgroup
          \expandafter\g@addto@macro
          \csname Z@L@#1\expandafter\endcsname
          \expandafter{\csname\comma@entry\endcsname}%
        }%
      }%
      \ltx@gobble
    }%
  }%
}
%    \end{macrocode}
%    \end{macro}
%    \begin{macro}{\zref@addprop}
%    \cs{zref@addprop} adds the property \verb|#2| to the property
%    list \verb|#1|, if the property is not already in the list.
%    Otherwise a warning is given.
%    \begin{macrocode}
\ZREF@Robust\def\zref@addprop#1#2{%
  \zref@listexists{#1}{%
    \zref@propexists{#2}{%
      \zref@iflistcontainsprop{#1}{#2}{%
        \PackageWarning\ZREF@name{%
          Property `#2' is already in list `#1'%
        }%
      }{%
        \begingroup\expandafter\endgroup
        \expandafter\g@addto@macro
        \csname Z@L@#1\expandafter\endcsname
        \expandafter{\csname#2\endcsname}%
      }%
    }%
  }%
}
%    \end{macrocode}
%    \end{macro}
%
%    \begin{macro}{\zref@localaddprops}
%    \begin{macrocode}
\ZREF@Robust\def\zref@localaddprops#1#2{%
  \zref@listexists{#1}{%
    \comma@parse{#2}{%
      \zref@propexists\comma@entry{%
        \zref@iflistcontainsprop{#1}\comma@entry{%
          \PackageWarning\ZREF@name{%
            Property `\comma@entry' is already in list `#1'%
          }%
        }{%
          \begingroup\expandafter\endgroup
          \expandafter\ltx@LocalAppendToMacro
          \csname Z@L@#1\expandafter\endcsname
          \expandafter{\csname\comma@entry\endcsname}%
        }%
      }%
      \ltx@gobble
    }%
  }%
}
%    \end{macrocode}
%    \end{macro}
%    \begin{macro}{\zref@localaddprop}
%    \begin{macrocode}
\ZREF@Robust\def\zref@localaddprop#1#2{%
  \zref@listexists{#1}{%
    \zref@propexists{#2}{%
      \zref@iflistcontainsprop{#1}{#2}{%
        \PackageWarning\ZREF@name{%
          Property `#2' is already in list `#1'%
        }%
      }{%
        \begingroup\expandafter\endgroup
        \expandafter\ltx@LocalAppendToMacro
        \csname Z@L@#1\expandafter\endcsname
        \expandafter{\csname#2\endcsname}%
      }%
    }%
  }%
}
%    \end{macrocode}
%    \end{macro}
%
%    \begin{macrocode}
\ltx@IfUndefined{pdf@strcmp}{%
%    \end{macrocode}
%    \begin{macro}{\zref@delprop}
%    \begin{macrocode}
  \ZREF@Robust\def\zref@delprop{%
    \ZREF@delprop\gdef
  }%
%    \end{macrocode}
%    \end{macro}
%    \begin{macro}{\zref@localdelprop}
%    \begin{macrocode}
  \ZREF@Robust\def\zref@localdelprop{%
    \ZREF@delprop\def
  }%
%    \end{macrocode}
%    \end{macro}
%    \begin{macro}{\ZREF@delprop}
%    \begin{macrocode}
  \def\ZREF@delprop#1#2#3{%
    \zref@listexists{#2}{%
      \begingroup
        \escapechar=-1 %
        \def\ZREF@param{#3}%
        \@onelevel@sanitize\ZREF@param
        \toks@{}%
        \expandafter\expandafter\expandafter\ZREF@@delprop
        \csname Z@L@#2\endcsname!%
      \expandafter\endgroup
      \expandafter#1\csname Z@L@#2\expandafter\endcsname
      \expandafter{%
        \the\toks@
      }%
    }%
  }%
%    \end{macrocode}
%    \end{macro}
%    \begin{macro}{\ZREF@@delprop}
%    \begin{macrocode}
  \def\ZREF@@delprop#1{%
    \expandafter\ZREF@@@delprop\expandafter{\string#1}#1%
  }%
%    \end{macrocode}
%    \end{macro}
%    \begin{macro}{\ZREF@@@delprop}
%    \begin{macrocode}
  \def\ZREF@@@delprop#1#2{%
    \ifx#2!%
    \else
      \def\ZREF@temp{#1}%
      \@onelevel@sanitize\ZREF@temp
      \ifx\ZREF@param\ZREF@temp
      \else
        \toks@\expandafter{%
          \the\expandafter\toks@\csname#1\endcsname
        }%
      \fi
      \expandafter\ZREF@@delprop
    \fi
  }%
%    \end{macrocode}
%    \end{macro}
%    \begin{macrocode}
}{%
%    \end{macrocode}
%    \begin{macro}{\zref@delprop}
%    \begin{macrocode}
  \ZREF@Robust\def\zref@delprop{%
    \ZREF@delprop\xdef
  }%
%    \end{macrocode}
%    \end{macro}
%    \begin{macro}{\zref@localdelprop}
%    \begin{macrocode}
  \ZREF@Robust\def\zref@localdelprop{%
    \ZREF@delprop\edef
  }%
%    \end{macrocode}
%    \end{macro}
%    \begin{macro}{\ZREF@delprop}
%    \begin{macrocode}
  \def\ZREF@delprop#1#2#3{%
    \zref@listexists{#2}{%
      \def\ZREF@param{#3}%
      \edef\ZREF@SavedEscapechar{\the\escapechar}%
      \escapechar=-1 %
      \expandafter#1\csname Z@L@#2%
      \expandafter\expandafter\expandafter\endcsname{%
        \expandafter\expandafter\expandafter\ZREF@@delprop
        \csname Z@L@#2\endcsname!%
      }%
      \escapechar=\ZREF@SavedEscapechar\relax
    }%
  }%
%    \end{macrocode}
%    \end{macro}
%    \begin{macro}{\ZREF@@delprop}
%    Caution: |#1| might be an \cs{if} or similar token.
%    \begin{macrocode}
  \def\ZREF@@delprop#1{%
    \expandafter\ZREF@@@delprop\expandafter{\string#1}#1%
  }%
%    \end{macrocode}
%    \end{macro}
%    \begin{macro}{\ZREF@@@delprop}
%    \begin{macrocode}
  \def\ZREF@@@delprop#1#2{%
    \ifx#2!%
    \else
      \ifnum\pdf@strcmp{#1}{\ZREF@param}=\ltx@zero
      \else
        \expandafter\noexpand\csname#1\endcsname
      \fi
      \expandafter\ZREF@@delprop
    \fi
  }%
%    \end{macrocode}
%    \end{macro}
%    \begin{macrocode}
}
%    \end{macrocode}
%
% \subsubsection{Properties}
%
%    \begin{macro}{\zref@ifpropundefined}
%    \cs{zref@ifpropundefined} checks the existence of the
%    property \verb|#1|. If the property is present,
%    then \verb|#2| is executed and \verb|#3| otherwise.
%    \begin{macrocode}
\def\zref@ifpropundefined#1{%
  \ltx@ifundefined{Z@E@#1}%
}
%    \end{macrocode}
%    \end{macro}
%    \begin{macro}{\zref@propexists}
%    Some macros rely on the existence of a property.
%    \cs{zref@propexists} only executes \verb|#2| if the
%    property \verb|#1| exists and raises an error
%    message otherwise.
%    \begin{macrocode}
\ZREF@Robust\def\zref@propexists#1{%
  \zref@ifpropundefined{#1}{%
    \PackageError\ZREF@name{%
      Property `#1' does not exist%
    }\@ehc
  }%
}
%    \end{macrocode}
%    \end{macro}
%
%    \begin{macro}{\zref@newprop}
%    A new property is declared by \cs{zref@newprop}, the
%    property name \meta{propname} is given in \verb|#1|. The
%    property is created and configured. If the star form
%    is given, then the expansion of the property value
%    is delayed to page shipout time, when the reference
%    is written to the \xfile{.aux} file.
%    \begin{description}
%    \item[\cs{Z@D@}\metacs{propname}:]
%      Stores the default value for this property.
%    \item[\cs{Z@E@}\metacs{propname}:]
%      Extract function.
%    \item[\cs{Z@X@}\metacs{propname}:]
%      Information whether the expansion of the property value
%      is delayed to shipout time.
%    \item[\cs{Z@C@}\metacs{propname}:]
%      Current value of the property.
%    \end{description}
%    \begin{macrocode}
\ZREF@Robust\def\zref@newprop{%
  \@ifstar{%
    \let\ZREF@X\noexpand
    \ZREF@newprop
  }{%
    \let\ZREF@X\ltx@empty
    \ZREF@newprop
  }%
}
%    \end{macrocode}
%    \end{macro}
%    \begin{macro}{\ZREF@newprop}
%    \begin{macrocode}
\def\ZREF@newprop#1{%
  \edef\ZREF@P{#1}%
  \@onelevel@sanitize\ZREF@P
  \begingroup
  \ifx\ZREF@P\ZREF@par
    \@PackageError\ZREF@name{%
      Invalid property name `\ZREF@P'%
    }{%
      The property name `par' is not allowed %
      because of internal reasons.%
      \MessageBreak
      \@ehc
    }%
    \def\ZREF@@newprop[##1]##2{\endgroup}%
  \else
    \zref@ifpropundefined\ZREF@P{%
      \endgroup
      \PackageInfo\ZREF@name{%
        New property: \ZREF@P
      }%
    }{%
      \@PackageError\ZREF@name{%
        Property `\ZREF@P' already exists%
      }\@ehc
      \def\ZREF@@newprop[##1]##2{\endgroup}%
    }%
  \fi
  \@ifnextchar[\ZREF@@newprop{\ZREF@@newprop[\zref@default]}%
}
%    \end{macrocode}
%    \end{macro}
%    \begin{macro}{\ZREF@par}
%    \begin{macrocode}
\def\ZREF@par{par}
\@onelevel@sanitize\ZREF@par
%    \end{macrocode}
%    \end{macro}
%    \begin{macro}{\ZREF@@newprop}
%    \begin{macrocode}
\def\ZREF@@newprop[#1]{%
  \global\@namedef{Z@D@\ZREF@P}{#1}%
  \global\expandafter\let\csname Z@X@\ZREF@P\endcsname\ZREF@X
  \begingroup\expandafter\endgroup
  \expandafter\ZREF@@@newprop\csname\ZREF@P\endcsname
  \expandafter\gdef\csname Z@C@\ZREF@P\endcsname{}%
  \zref@setcurrent\ZREF@P
}
\def\ZREF@@@newprop#1{%
  \expandafter
  \gdef\csname Z@E@\ZREF@P\endcsname##1#1##2##3\ZREF@nil{##2}%
}
%    \end{macrocode}
%    \end{macro}
%
%    \begin{macro}{\zref@showprop}
%    \begin{macrocode}
\ZREF@Robust\def\zref@showprop#1{%
  \zref@ifpropundefined{#1}{%
    \@PackageInfoNoLine{\ZREF@name}{%
      Show property `#1': <undefined>%
    }%
  }{%
    \begingroup
      \toks@\expandafter\expandafter\expandafter{%
        \csname Z@C@#1\endcsname
      }%
      \edef\ZREF@value{\the\toks@}%
      \ltx@onelevel@sanitize\ZREF@value
      \toks@\expandafter\expandafter\expandafter{%
        \csname Z@D@#1\endcsname
      }%
      \edef\ZREF@default{\the\toks@}%
      \ltx@onelevel@sanitize\ZREF@default
      \@PackageInfoNoLine{\ZREF@name}{%
        Show property `#1':\MessageBreak
        \expandafter\ifx\csname Z@X@#1\endcsname\ltx@empty
          Immediate %
        \else
          Delayed %
        \fi
        value: [\ZREF@value]\MessageBreak
        Default: [\ZREF@default]%
      }%
    \endgroup
  }%
}
%    \end{macrocode}
%    \end{macro}
%
%    \begin{macro}{\zref@setcurrent}
%    \cs{zref@setcurrent} sets the current value for a property.
%    \begin{macrocode}
\ZREF@Robust\def\zref@setcurrent#1#2{%
  \zref@propexists{#1}{%
    \expandafter\def\csname Z@C@#1\endcsname{#2}%
  }%
}
%    \end{macrocode}
%    \end{macro}
%
%    \begin{macro}{\ZREF@getcurrent}
%    \cs{zref@getcurrent} gets the current value for a property.
%    \begin{macrocode}
\def\ZREF@getcurrent#1{%
  \romannumeral0%
  \ltx@ifundefined{Z@C@#1}{%
    \ltx@space
  }{%
    \expandafter\expandafter\expandafter\ltx@space
    \csname Z@C@#1\endcsname
  }%
}
%    \end{macrocode}
%    \end{macro}
%    \begin{macro}{\ZREF@u@getcurrent}
%    \begin{macrocode}
\def\ZREF@wu@getcurrent#1{%
  \etex@unexpanded\expandafter\expandafter\expandafter{%
    \ZREF@getcurrent{#1}%
  }%
}
%    \end{macrocode}
%    \end{macro}
%    \begin{macro}{\zref@getcurrent}
%    \begin{macrocode}
\let\zref@getcurrent\ZREF@getcurrent
%    \end{macrocode}
%    \end{macro}
%
% \subsubsection{Reference generation}
%
%    \begin{macro}{\zref@label}
%    Label macro that uses the main property list.
%    \begin{macrocode}
\ZREF@Robust\def\zref@label#1{%
  \zref@labelbylist{#1}\ZREF@mainlist
}
%    \end{macrocode}
%    \end{macro}
%    \begin{macro}{\zref@labelbylist}
%    Label macro that stores the properties, specified in the
%    property list \verb|#2|.
%    \begin{macrocode}
\ZREF@Robust\def\zref@labelbylist#1#2{%
  \@bsphack
    \zref@listexists{#2}{%
      \expandafter\expandafter\expandafter\ZREF@label
      \expandafter\expandafter\expandafter{%
        \csname Z@L@#2\endcsname
      }{#1}%
    }%
  \@esphack
}
%    \end{macrocode}
%    \end{macro}
%    \begin{macro}{\zref@labelbyprops}
%    The properties are directly specified in a comma separated list.
%    \begin{macrocode}
\ZREF@Robust\def\zref@labelbyprops#1#2{%
  \@bsphack
    \begingroup
      \toks@{}%
      \comma@parse{#2}{%
        \zref@ifpropundefined\comma@entry{%
          \PackageWarning\ZREF@name{%
            Property `\comma@entry' is not known%
          }%
        }{%
          \toks@\expandafter{%
            \the\expandafter\toks@\csname\comma@entry\endcsname
          }%
        }%
        \ltx@gobble
      }%
    \expandafter\endgroup
    \expandafter\ZREF@label\expandafter{\the\toks@}{#1}%
  \@esphack
}
%    \end{macrocode}
%    \end{macro}
%    \begin{macro}{\zref@labelbykv}
%    \begin{macrocode}
\ZREF@Robust\def\zref@labelbykv#1#2{%
  \@bsphack
    \begingroup
      \let\Z@L@ZREF@temp\ltx@empty
      \kvsetkeys{ZREF@LABEL}{#1}%
      \ifZREF@immediate
        \expandafter\zref@wrapper@immediate\expandafter{%
          \expandafter\ZREF@label\expandafter{\Z@L@ZREF@temp}{#2}%
        }%
      \else
        \expandafter\ZREF@label\expandafter{\Z@L@ZREF@temp}{#2}%
      \fi
    \endgroup
  \@esphack
}
%    \end{macrocode}
%    \end{macro}
%    \begin{macrocode}
\kv@define@key{ZREF@LABEL}{prop}{%
  \edef\ZREF@param{#1}%
  \zref@propexists\ZREF@param{%
    \zref@iflistcontainsprop{ZREF@temp}\ZREF@param{}{%
      \begingroup\expandafter\endgroup
      \expandafter\ltx@LocalAppendToMacro
      \expandafter\Z@L@ZREF@temp
      \expandafter{\csname\ZREF@param\endcsname}%
    }%
  }%
}
\kv@define@key{ZREF@LABEL}{list}{%
  \zref@listforloop{#1}{%
    \zref@iflistcontainsprop{ZREF@temp}\zref@prop{}{%
      \begingroup\expandafter\endgroup
      \expandafter\ltx@LocalAppendToMacro
      \expandafter\Z@L@ZREF@temp
      \expandafter{\csname\zref@prop\endcsname}%
    }%
    \ltx@gobble
  }%
}
\kv@define@key{ZREF@LABEL}{delprop}{%
  \zref@propexists{#1}{%
    \zref@localdelprop{ZREF@temp}{#1}%
  }%
}
\kv@define@key{ZREF@LABEL}{immediate}[true]{%
  \edef\ZREF@param{#1}%
  \ifx\ZREF@param\ZREF@true
    \ZREF@immediatetrue
  \else
    \ifx\ZREF@param\ZREF@false
      \ZREF@immediatefalse
    \else
      \PackageWarning\ZREF@name{%
        Option `immediate' expects `true' or `false'.\MessageBreak
        Ignoring invalid value `\ZREF@param'%
      }%
    \fi
  \fi
}
%    \end{macrocode}
%    \begin{macro}{\ZREF@false}
%    \begin{macrocode}
\def\ZREF@false{false}
%    \end{macrocode}
%    \end{macro}
%    \begin{macro}{\ZREF@true}
%    \begin{macrocode}
\def\ZREF@true{true}
%    \end{macrocode}
%    \end{macro}
%
%    \begin{macrocode}
\kv@define@key{ZREF@LABEL}{values}[]{%
  \kv@parse{#1}{%
    \ifx\kv@value\relax
      \@PackageWarning\ZREF@name{%
        Missing value for property `\kv@key'%
      }%
      \expandafter\ltx@gobbletwo
    \else
      \expandafter\zref@setcurrent
    \fi
  }%
}
%    \end{macrocode}
%
%    \begin{macro}{\ifZREF@immediate}
%    The switch \cs{ifZREF@immediate} tells us, whether the
%    label should be written immediately or at page shipout time.
%    \cs{ZREF@label} need to be notified about this, because
%    it must disable the deferred execution of property values,
%    if the label is written immediately.
%    \begin{macrocode}
\newif\ifZREF@immediate
%    \end{macrocode}
%    \end{macro}
%    \begin{macro}{\zref@wrapper@immediate}
%    The argument of \cs{zref@wrapper@immediate} is executed inside
%    a group where \cs{write} is redefined by adding
%    \cs{immediate} before its execution. Also \cs{ZREF@label}
%    is notified via the switch \cs{ifZREF@immediate}.
%    \begin{macrocode}
\ZREF@Robust{\long\def}\zref@wrapper@immediate#1{%
  \begingroup
    \ZREF@immediatetrue
    \let\ZREF@org@write\write
    \def\write{\immediate\ZREF@org@write}%
    #1%
  \endgroup
}
%    \end{macrocode}
%    \end{macro}
%
%    \begin{macro}{\ZREF@label}
%    \cs{ZREF@label} writes the data in the \xfile{.aux} file.
%    \verb|#1| contains the list of valid properties,
%    \verb|#2| the name of the reference. In case of immediate
%    writing, the deferred execution of property values is
%    disabled. Also \thepage is made expandable in this case.
%    \begin{macrocode}
\def\ZREF@label#1#2{%
  \if@filesw
    \begingroup
      \ifZREF@immediate
        \let\ZREF@org@thepage\thepage
      \fi
      \protected@write\@auxout{%
        \ifZREF@immediate
          \let\thepage\ZREF@org@thepage
        \fi
        \let\ZREF@temp\ltx@empty
        \@tfor\ZREF@P:=#1\do{%
          \begingroup
            \escapechar=-1 %
            \edef\x{\endgroup
              \def\noexpand\ZREF@P{%
                \expandafter\string\ZREF@P
              }%
            }%
          \x
          \expandafter\ifx
              \csname
                \ifZREF@immediate
                  relax%
                \else
                  Z@X@\ZREF@P%
                \fi
              \endcsname
              \noexpand
            \expandafter\let\csname Z@C@\ZREF@P\endcsname\relax
          \fi
          \toks@\expandafter{\ZREF@temp}%
          \edef\ZREF@temp{%
            \the\toks@
            \ltx@backslashchar\ZREF@P{%
              \expandafter\noexpand\csname Z@C@\ZREF@P\endcsname
            }%
          }%
        }%
      }{%
        \string\zref@newlabel{#2}{\ZREF@temp}%
      }%
    \endgroup
  \fi
}
\def\ZREF@addtoks#1{%
  \toks@\expandafter\expandafter\expandafter{%
    \expandafter\the\expandafter\toks@#1%
  }%
}
%    \end{macrocode}
%    \end{macro}
%
% \subsubsection{Reference querying and extracting}
%
%    Design goal for the extracting macros is that the extraction
%    process is full expandable. Thus these macros can be used
%    in expandable contexts. But there are problems that cannot
%    be solved by full expandable macros:
%    \begin{itemize}
%    \item In standard \hologo{LaTeX} undefined references sets a flag
%      and generate a warning. Both actions are not expandable.
%    \item Babel's support for its shorthand uses commands that
%      use non-expandable assignments. However currently there
%      is hope, that primitives are added to \pdfTeX\ that
%      allows the detection of contexts. Then the shorthand
%      can detect, if they are executed inside \cs{csname}
%      and protect themselves automatically.
%    \end{itemize}
%    \begin{macro}{\zref@ifrefundefined}
%    If a reference \verb|#1| is undefined, then macro
%    \cs{zref@ifrefundefined} calls \verb|#2| and \verb|#3| otherwise.
%    \begin{macrocode}
\def\zref@ifrefundefined#1{%
  \ltx@ifundefined{Z@R@#1}%
}
%    \end{macrocode}
%    \end{macro}
%
%    \begin{macro}{\zifrefundefined}
%    If a reference \verb|#1| is undefined, then macro
%    \cs{zref@ifrefundefined} calls \verb|#2| and \verb|#3| otherwise.
%    Also the reference is marked used.
%    \begin{macrocode}
\ZREF@IfDefinable\zifrefundefined\def{%
  #1{%
    \zref@wrapper@babel\ZREF@ifrefundefined{#1}%
  }%
}
%    \end{macrocode}
%    \end{macro}
%    \begin{macro}{\ZREF@ifrefundefined}
%    \begin{macrocode}
\def\ZREF@ifrefundefined#1{%
  \zref@refused{#1}%
  \zref@ifrefundefined{#1}%
}
%    \end{macrocode}
%    \end{macro}
%
%    \begin{macro}{\zref@refused}
%    The problem with undefined references is addressed by
%    the macro \cs{zref@refused}. This can be used outside
%    the expandable context. In case of an undefined reference
%    the flag is set to notify \hologo{LaTeX} and a warning is given.
%    \begin{macrocode}
\ZREF@Robust\def\zref@refused#1{%
  \zref@wrapper@babel\ZREF@refused{#1}%
}
%    \end{macrocode}
%    \end{macro}
%    \begin{macro}{\ZREF@refused}
%    \begin{macrocode}
\def\ZREF@refused#1{%
  \zref@ifrefundefined{#1}{%
    \protect\G@refundefinedtrue
    \@latex@warning{%
      Reference `#1' on page \thepage \space undefined%
    }%
  }{}%
}
%    \end{macrocode}
%    \end{macro}
%
%    \begin{macro}{\zref@ifrefcontainsprop}
%    \cs{zref@ifrefcontainsprop} looks, if the reference \verb|#1|
%    has the property \verb|#2| and calls then \verb|#3| and
%    \verb|#4| otherwise.
%    \begin{macrocode}
\def\zref@ifrefcontainsprop#1#2{%
  \zref@ifrefundefined{#1}{%
    \ltx@secondoftwo
  }{%
    \expandafter\ZREF@ifrefcontainsprop
    \csname Z@E@#2\expandafter\endcsname
    \csname#2\expandafter\expandafter\expandafter\endcsname
    \expandafter\expandafter\expandafter{%
      \csname Z@R@#1\endcsname
    }%
  }%
}
\def\ZREF@ifrefcontainsprop#1#2#3{%
  \expandafter\ifx\expandafter\ZREF@novalue
  #1#3#2\ZREF@novalue\ZREF@nil\ltx@empty
    \expandafter\ltx@secondoftwo
  \else
    \expandafter\ltx@firstoftwo
  \fi
}
\def\ZREF@novalue{\ZREF@NOVALUE}
%    \end{macrocode}
%    \end{macro}
%
%    \begin{macro}{\zref@extract}
%    \cs{zref@extract} is an abbreviation for the case
%    that the default of the property is used as default
%    value.
%    \begin{macrocode}
\def\ZREF@extract#1#2{%
  \romannumeral0%
  \ltx@ifundefined{Z@D@#2}{%
    \expandafter\ltx@space\zref@default
  }{%
    \expandafter\expandafter\expandafter\ZREF@@extract
    \expandafter\expandafter\expandafter{%
      \csname Z@D@#2\endcsname
    }{#1}{#2}%
  }%
}
%    \end{macrocode}
%    \end{macro}
%    \begin{macro}{\ZREF@@extract}
%    \begin{macrocode}
\def\ZREF@@extract#1#2#3{%
  \expandafter\expandafter\expandafter\ltx@space
  \zref@extractdefault{#2}{#3}{#1}%
}
%    \end{macrocode}
%    \end{macro}
%    \begin{macro}{\ZREF@wu@extract}
%    \begin{macrocode}
\def\ZREF@wu@extract#1#2{%
  \etex@unexpanded\expandafter\expandafter\expandafter{%
    \ZREF@extract{#1}{#2}%
  }%
}
%    \end{macrocode}
%    \end{macro}
%    \begin{macro}{\zref@extract}
%    \begin{macrocode}
\let\zref@extract\ZREF@extract
%    \end{macrocode}
%    \end{macro}
%
%    \begin{macro}{\ZREF@extractdefault}
%    The basic extracting macro is \cs{zref@extractdefault} with
%    the reference name in \verb|#1|, the property in \verb|#2| and
%    the default value in \verb|#3| in case for problems.
%    \begin{macrocode}
\def\ZREF@extractdefault#1#2#3{%
  \romannumeral0%
  \zref@ifrefundefined{#1}\ltx@firstoftwo{%
    \zref@ifpropundefined{#2}\ltx@firstoftwo\ltx@secondoftwo
  }{%
    \ltx@space
    #3%
  }{%
    \expandafter\expandafter\expandafter\ltx@space
    \csname Z@E@#2\expandafter\expandafter\expandafter\endcsname
    \csname Z@R@#1\expandafter\endcsname
    \csname#2\endcsname{#3}\ZREF@nil
  }%
}
%    \end{macrocode}
%    \end{macro}
%    \begin{macro}{\ZREF@wu@extractdefault}
%    \begin{macrocode}
\def\ZREF@wu@extractdefault#1#2#3{%
  \etex@unexpanded\expandafter\expandafter\expandafter{%
    \ZREF@extractdefault{#1}{#2}{#3}%
  }%
}
%    \end{macrocode}
%    \end{macro}
%    \begin{macro}{\zref@extractdefault}
%    \begin{macrocode}
\let\zref@extractdefault\ZREF@extractdefault
%    \end{macrocode}
%    \end{macro}
%
%    \begin{macro}{\zref@def@extract}
%    \begin{macrocode}
\ZREF@Robust\def\zref@def@extract#1{%
  \zref@wrapper@babel{\ZREF@def@extract{#1}}%
}
%    \end{macrocode}
%    \end{macro}
%    \begin{macro}{\ZREF@def@extract}
%    \begin{macrocode}
\def\ZREF@def@extract#1#2#3{%
  \zref@refused{#2}%
  \expandafter\expandafter\expandafter\def
  \expandafter\expandafter\expandafter#1%
  \expandafter\expandafter\expandafter{%
    \zref@extract{#2}{#3}%
  }%
}
%    \end{macrocode}
%    \end{macro}
%    \begin{macro}{\zref@def@extractdefault}
%    \begin{macrocode}
\ZREF@Robust\def\zref@def@extractdefault#1{%
  \zref@wrapper@babel{\ZREF@def@extractdefault{#1}}%
}
%    \end{macrocode}
%    \end{macro}
%    \begin{macro}{\ZREF@def@extractdefault}
%    \begin{macrocode}
\def\ZREF@def@extractdefault#1#2#3#4{%
  \zref@refused{#2}%
  \expandafter\expandafter\expandafter\def
  \expandafter\expandafter\expandafter#1%
  \expandafter\expandafter\expandafter{%
    \zref@extractdefault{#2}{#3}{#4}%
  }%
}
%    \end{macrocode}
%    \end{macro}
%
%    \begin{macro}{\ZREF@wrapper@unexpanded}
%    \begin{macrocode}
\ZREF@Robust{\long\def}\ZREF@wrapper@unexpanded#1{%
  \let\zref@wrapper@unexpanded\ltx@firstofone
  \let\zref@getcurrent\ZREF@wu@getcurrent
  \let\zref@extractdefault\ZREF@wu@extractdefault
  \let\zref@extract\ZREF@wu@extract
  #1%
  \let\zref@wrapper@unexpanded\ZREF@wrapper@unexpanded
  \let\zref@getcurrent\ZREF@getcurrent
  \let\zref@extractdefault\ZREF@extractdefault
  \let\zref@extract\ZREF@extract
}
%    \end{macrocode}
%    \end{macro}
%    \begin{macro}{\zref@wrapper@unexpanded}
%    \begin{macrocode}
\ltx@IfUndefined{etex@unexpanded}{%
  \let\zref@wrapper@unexpanded\ltx@firstofone
}{%
  \let\zref@wrapper@unexpanded\ZREF@wrapper@unexpanded
}
%    \end{macrocode}
%    \end{macro}
%
% \subsubsection{Compatibility with \xpackage{babel}}
%
%    \begin{macro}{\zref@wrapper@babel}
%    \begin{macrocode}
\ZREF@Robust{\long\def}\zref@wrapper@babel#1#2{%
  \ifcsname if@safe@actives\endcsname
    \expandafter\ltx@firstofone
  \else
    \expandafter\ltx@secondoftwo
  \fi
  {%
    \if@safe@actives
      \expandafter\ltx@secondoftwo
    \else
      \expandafter\ltx@firstoftwo
    \fi
    {%
      \begingroup
        \csname @safe@activestrue\endcsname
        \edef\x{#2}%
      \expandafter\endgroup
      \expandafter\ZREF@wrapper@babel\expandafter{\x}{#1}%
    }%
  }{%
    #1{#2}%
  }%
}
\long\def\ZREF@wrapper@babel#1#2{%
  #2{#1}%
}
%    \end{macrocode}
%    \end{macro}
%
% \subsubsection{Unique counter support}
%
%    \begin{macro}{\zref@require@unique}
%    Generate the counter |zref@unique| if the counter does not
%    already exist.
%    \begin{macrocode}
\ZREF@Robust\def\zref@require@unique{%
  \@ifundefined{c@zref@unique}{%
    \begingroup
      \let\@addtoreset\ltx@gobbletwo
      \newcounter{zref@unique}%
    \endgroup
%    \end{macrocode}
%    \end{macro}
%
%    \begin{macro}{\thezref@unique}
%    \cs{thezref@unique} is used for automatically generated unique
%    labelnames.
%    \begin{macrocode}
    \renewcommand*{\thezref@unique}{%
      zref@\number\c@zref@unique
    }%
  }{}%
}
%    \end{macrocode}
%    \end{macro}
%
% \subsubsection{Utilities}
%
%    \begin{macro}{\ZREF@number}
%    \begin{macrocode}
\ltx@IfUndefined{numexpr}{%
  \def\ZREF@number#1{\number#1}%
}{%
  \def\ZREF@number#1{\the\numexpr(#1)\relax}%
}
%    \end{macrocode}
%    \end{macro}
%
% \subsubsection{Setup}
%
%    \begin{macro}{\zref@setdefault}
%    Standard \hologo{LaTeX} prints ``??'' in bold face if a reference
%    is not known. \cs{zref@default} holds the text that is
%    printed in case of unknown references and is used, if
%    the default was not specified during the definition
%    of the new property by \cs{ref@newprop}.
%    The global default value can be set by \cs{zref@setdefault}.
%    \begin{macrocode}
\ZREF@Robust\def\zref@setdefault#1{%
  \def\zref@default{#1}%
}
%    \end{macrocode}
%    \end{macro}
%    \begin{macro}{\zref@default}
%    Now we initialize \cs{zref@default} with the same value
%    that \hologo{LaTeX} uses for its undefined references.
%    \begin{macrocode}
\zref@setdefault{%
  \nfss@text{\reset@font\bfseries ??}%
}
%    \end{macrocode}
%    \end{macro}
%
% \paragraph{Main property list.}
%
%    \begin{macro}{\zref@setmainlist}
%    The name of the default property list is stored in
%    \cs{ZREF@mainlist} and can be set by \cs{zref@setmainlist}.
%    \begin{macrocode}
\ZREF@Robust\def\zref@setmainlist#1{%
  \def\ZREF@mainlist{#1}%
}
\zref@setmainlist{main}
%    \end{macrocode}
%    \end{macro}
%    Now we create the list.
%    \begin{macrocode}
\zref@newlist\ZREF@mainlist
%    \end{macrocode}
%
% \paragraph{Main properties.}
%
%    The two properties \texttt{default} and \texttt{page}
%    are created and added to the main property list.
%    They store the data that
%    standard \hologo{LaTeX} uses in its references created by
%    \cs{label}.
%    \begin{description}
%    \item[\texttt{default}] the apperance of the latest
%      counter that is incremented by \cs{refstepcounter}
%    \item[\texttt{page}] the apperance of the page counter
%    \end{description}
%    \begin{macrocode}
\zref@newprop{default}{\@currentlabel}
\zref@newprop*{page}{\thepage}
\zref@addprops\ZREF@mainlist{default,page}
%    \end{macrocode}
%
% \paragraph{Properties}
%
%    \begin{macro}{\ZREF@NewPropAnchor}
%    \begin{macrocode}
\def\ZREF@NewPropAnchor{%
  \zref@newprop{anchor}{%
    \ltx@ifundefined{@currentHref}{}{\@currentHref}%
  }%
  \global\let\ZREF@NewPropAnchor\relax
}
%    \end{macrocode}
%    \end{macro}
%    \begin{macro}{\zref@titleref@current}
%    Later we will redefine the section and caption macros
%    to catch the current title and remember the value in
%    \cs{zref@titleref@current}.
%    \begin{macro}{\ZREF@NewPropTitle}
%    \begin{macrocode}
\def\ZREF@NewPropTitle{%
  \gdef\zref@titleref@current{}%
  \zref@newprop{title}{\zref@titleref@current}%
  \global\let\ZREF@NewPropTitle\relax
}
%    \end{macrocode}
%    \end{macro}
%    \end{macro}
%    \begin{macro}{\ZREF@NewPropTheotype}
%    \begin{macrocode}
\def\ZREF@NewPropTheotype{%
  \zref@newprop{theotype}{}%
  \global\let\ZREF@NewPropTheotype\relax
}
%    \end{macrocode}
%    \end{macro}
%    \begin{macro}{\ZREF@NewPropPageValue}
%    \begin{macrocode}
\def\ZREF@NewPropPageValue{%
  \zref@newprop*{pagevalue}[0]{\number\c@page}%
  \global\let\ZREF@NewPropPageValue\relax
}
%    \end{macrocode}
%    \end{macro}
%
% \paragraph{Mark successful loading}
%
%    \begin{macrocode}
\let\ZREF@base@ok=Y
%</base>
%    \end{macrocode}
%
%
% \subsection{Module \xmodule{user}}
%
%    \begin{macrocode}
%<*user>
\NeedsTeXFormat{LaTeX2e}
\ProvidesPackage{zref-user}%
  [2020-07-03 v2.32 Module user for zref (HO)]%
\RequirePackage{zref-base}[2019/11/29]
\ifx\ZREF@base@ok Y%
\else
  \expandafter\endinput
\fi
%    \end{macrocode}
%
%    Module \xmodule{user} enables a small user interface.
%    All macros are prefixed by \cs{z}.
%
%    First we define the pendants to the standard \hologo{LaTeX} referencing
%    commands \cs{label}, \cs{ref}, and \cs{pageref}.
%
%    \begin{macro}{\zlabel}
%    Similar to \cs{label} the macro \cs{zlabel} writes a reference
%    entry in the \xfile{.aux} file. The main property list is used.
%    Also we add the babel patch. The \cs{label} command can also be
%    used inside section titles, but it must not go into the
%    table of contents. Therefore we have to check this situation.
%    \begin{macrocode}
\newcommand*\zlabel{%
  \ifx\label\ltx@gobble
    \expandafter\ltx@gobble
  \else
    \expandafter\zref@wrapper@babel\expandafter\zref@label
  \fi
}%
%    \end{macrocode}
%    \end{macro}
%
%    \begin{macro}{\zkvlabel}
%    \begin{macrocode}
\newcommand*{\zkvlabel}[1]{%
  \ifx\label\ltx@gobble
    \expandafter\ltx@gobblethree
  \fi
  \zref@wrapper@babel{\zref@labelbykv{#1}}%
}%
%    \end{macrocode}
%    \end{macro}
%
%    \begin{macro}{\zref}
%    Macro \cs{zref} is the corresponding macro for \cs{ref}.
%    Also it provides an optional argument in order to
%    select another property.
%    \begin{macrocode}
\newcommand*{\zref}[2][default]{% robust because of optional argument
  \zref@propexists{#1}{%
    \zref@wrapper@babel\ZREF@zref{#2}{#1}%
  }%
}%
\def\ZREF@zref#1{%
  \zref@refused{#1}%
  \zref@extract{#1}%
}%
%    \end{macrocode}
%    \end{macro}
%
%    \begin{macro}{\zpageref}
%    For macro \cs{zpageref} we just call \cs{zref}
%    with property \verb|page|.
%    \begin{macrocode}
\ZREF@IfDefinable\zpageref\def{%
  {\zref[page]}%
}
%    \end{macrocode}
%    \end{macro}
%
%    \begin{macro}{\zrefused}
%    For the following expandible user macros \cs{zrefused}
%    should be used to notify \hologo{LaTeX} in case of undefined
%    references.
%    \begin{macrocode}
\ZREF@IfDefinable\zrefused\def{%
  {\zref@refused}%
}
%    \end{macrocode}
%    \end{macro}
%
%    \begin{macrocode}
%</user>
%    \end{macrocode}
%
%
% \subsection{Module \xmodule{abspage}}
%
%    \begin{macrocode}
%<*abspage>
\NeedsTeXFormat{LaTeX2e}
\ProvidesPackage{zref-abspage}%
  [2020-07-03 v2.32 Module abspage for zref (HO)]%
\RequirePackage{zref-base}[2019/11/29]
\ifx\ZREF@base@ok Y%
\else
  \expandafter\endinput
\fi
%    \end{macrocode}
%
%    Module \xmodule{abspage} adds a new property \verb|abspage|
%    to the \verb|main| property list for absolute page numbers.
%    These are recorded by the help of package \xpackage{atbegshi}.
%    \begin{macrocode}
\RequirePackage{atbegshi}[2011/10/05]%
%    \end{macrocode}
%    The counter \verb|abspage| must not go in the clear list
%    of \verb|@ckpt| that is used to set counters in
%    \xfile{.aux} files of included \hologo{TeX} files.
%    \begin{macrocode}
\begingroup
  \let\@addtoreset\ltx@gobbletwo
  \newcounter{abspage}%
\endgroup
\setcounter{abspage}{0}%
\AtBeginShipout{%
  \stepcounter{abspage}%
}%
\zref@newprop*{abspage}[0]{\the\c@abspage}%
\zref@addprop\ZREF@mainlist{abspage}%
%    \end{macrocode}
%    Note that counter \verb|abspage| shows the
%    previous page during page processing. Before
%    shipout the counter is incremented.
%    Thus the property is correctly written with
%    deferred writing. If the counter is written
%    using \cs{zref@wrapper@immediate}, then the number
%    is too small by one.
%
%    \begin{macrocode}
%</abspage>
%    \end{macrocode}
%
%
% \subsection{Module \xmodule{counter}}
%
%    \begin{macrocode}
%<*counter>
\NeedsTeXFormat{LaTeX2e}
\ProvidesPackage{zref-counter}%
  [2020-07-03 v2.32 Module counter for zref (HO)]%
\RequirePackage{zref-base}[2019/11/29]
\ifx\ZREF@base@ok Y%
\else
  \expandafter\endinput
\fi
%    \end{macrocode}
%
%    For features such as \xpackage{hyperref}'s \cs{autoref}
%    we need the name of the counter. The property
%    \verb|counter| is defined and added to the main property
%    list. Starting with \LaTeX\ 2020-10-01 the proper can use
%    \@currentcounter. In older formats
%    \cs{refstepcounter} has to be patched but this can fail in some
%    cases, see issue \#5.
%    \begin{macrocode}
\@ifl@t@r\fmtversion{2020-10-01}
 {
  \zref@newprop{counter}{\@currentcounter}
  \zref@addprop\ZREF@mainlist{counter}
 }
 {
  \zref@newprop{counter}{}
  \zref@addprop\ZREF@mainlist{counter}
  \AtBeginDocument{%
    \ZREF@patch{refstepcounter}{%
      \def\refstepcounter#1{%
        \zref@setcurrent{counter}{#1}%
        \ZREF@org@refstepcounter{#1}%
      }%
    }%
  }
 }
%    \end{macrocode}
%    \begin{macrocode}
%</counter>
%    \end{macrocode}
%
%
% \subsection{Module \xmodule{lastpage}}
%
%    \begin{macrocode}
%<*lastpage>
\NeedsTeXFormat{LaTeX2e}
\ProvidesPackage{zref-lastpage}%
  [2020-07-03 v2.32 Module lastpage for zref (HO)]%
\RequirePackage{zref-base}[2019/11/29]
\RequirePackage{zref-abspage}[2019/11/29]
\RequirePackage{atveryend}[2009/12/07]
\ifx\ZREF@base@ok Y%
\else
  \expandafter\endinput
\fi
%    \end{macrocode}
%
%    The module \xmodule{lastpage} implements the service
%    of package \xpackage{lastpage} by setting a reference
%    \verb|LastPage| at the end of the document.
%    If module \xmodule{abspage} is given, also the absolute
%    page number is available, because the properties of
%    the main property list are used.
%    \begin{macrocode}
\zref@newlist{LastPage}
\AfterLastShipout{%
  \if@filesw
    \begingroup
      \advance\c@page\m@ne
      \toks@\expandafter\expandafter\expandafter{%
        \expandafter\Z@L@main
        \Z@L@LastPage
      }%
      \expandafter\zref@wrapper@immediate\expandafter{%
        \expandafter\ZREF@label\expandafter{\the\toks@}{LastPage}%
      }%
    \endgroup
  \fi
}
%    \end{macrocode}
%    \begin{macro}{\zref@iflastpage}
%    \begin{macrocode}
\def\zref@iflastpage#1{%
  \ifnum\zref@extractdefault{#1}{abspage}{-1}=%
        \zref@extractdefault{LastPage}{abspage}{-2} %
    \expandafter\ltx@firstoftwo
  \else
    \expandafter\ltx@secondoftwo
  \fi
}
%    \end{macrocode}
%    \end{macro}
%    \begin{macro}{\ziflastpage}
%    \begin{macrocode}
\ZREF@IfDefinable\ziflastpage\def{%
  {\zref@wrapper@babel\ZREF@iflastpage}%
}
%    \end{macrocode}
%    \end{macro}
%    \begin{macro}{ZREF@iflastpage}
%    \begin{macrocode}
\def\ZREF@iflastpage#1{%
  \zref@refused{LastPage}%
  \zref@refused{#1}%
  \zref@iflastpage{#1}%
}
%    \end{macrocode}
%    \end{macro}
%    \begin{macrocode}
%</lastpage>
%    \end{macrocode}
%
%
% \subsection{Module \xmodule{thepage}}
%
%    \begin{macrocode}
%<*thepage>
\NeedsTeXFormat{LaTeX2e}
\ProvidesPackage{zref-thepage}%
  [2020-07-03 v2.32 Module thepage for zref (HO)]%
\RequirePackage{zref-base}[2019/11/29]
\ifx\ZREF@base@ok Y%
\else
  \expandafter\endinput
\fi
%    \end{macrocode}
%
%    \begin{macrocode}
\RequirePackage{atbegshi}[2011/10/05]
\RequirePackage{zref-abspage}[2019/11/29]
%    \end{macrocode}
%    \begin{macrocode}
\zref@newlist{thepage}
\zref@addprop{thepage}{page}
\ZREF@NewPropPageValue
%    \end{macrocode}
%    \begin{macro}{\zref@thepage@atbegshi@hook}
%    \begin{macrocode}
\let\zref@thepage@atbegshi@hook\ltx@empty
%    \end{macrocode}
%    \end{macro}
%    \begin{macrocode}
\zref@addprop{thepage}{pagevalue}
\AtBeginShipout{%
  \AtBeginShipoutAddToBox{%
    \zref@thepage@atbegshi@hook
    \zref@labelbylist{thepage\the\value{abspage}}{thepage}%
  }%
}
%    \end{macrocode}
%    \begin{macro}{\zref@thepage@name}
%    \begin{macrocode}
\ltx@IfUndefined{numexpr}{%
  \def\zref@thepage@name#1{thepage\number#1}%
}{%
  \def\zref@thepage@name#1{thepage\the\numexpr#1}%
}
%    \end{macrocode}
%    \end{macro}
%    \begin{macro}{\zref@thepage}
%    \begin{macrocode}
\def\zref@thepage#1{%
  \zref@extract{\zref@thepage@name{#1}}{page}%
}%
%    \end{macrocode}
%    \end{macro}
%    \begin{macro}{\zref@thepage@refused}
%    \begin{macrocode}
\ZREF@Robust\def\zref@thepage@refused#1{%
  \zref@refused{\zref@thepage@name{#1}}%
}%
%    \end{macrocode}
%    \end{macro}
%    \begin{macro}{\zthepage}
%    \begin{macrocode}
\ZREF@IfDefinable\zthepage\def{%
  #1{%
    \zref@thepage@refused{#1}%
    \zref@thepage{#1}%
  }%
}
%    \end{macrocode}
%    \end{macro}
%    \begin{macrocode}
%</thepage>
%    \end{macrocode}
%
%
% \subsection{Module \xmodule{nextpage}}
%
%    \begin{macrocode}
%<*nextpage>
\NeedsTeXFormat{LaTeX2e}
\ProvidesPackage{zref-nextpage}%
  [2020-07-03 v2.32 Module nextpage for zref (HO)]%
\RequirePackage{zref-base}[2019/11/29]
\ifx\ZREF@base@ok Y%
\else
  \expandafter\endinput
\fi
%    \end{macrocode}
%
%    \begin{macrocode}
\RequirePackage{zref-abspage}[2019/11/29]
\RequirePackage{zref-thepage}[2019/11/29]
\RequirePackage{zref-lastpage}[2019/11/29]
\RequirePackage{uniquecounter}[2009/12/18]
%    \end{macrocode}
%    \begin{macrocode}
\UniqueCounterNew{znextpage}

\newcommand*{\znextpagesetup}{%
  \afterassignment\ZREF@np@setup@i
  \def\ZREF@np@call@unknown##1%
}
\def\ZREF@np@setup@i{%
  \afterassignment\ZREF@np@setup@ii
  \def\ZREF@np@call@nonext##1%
}
\def\ZREF@np@setup@ii{%
  \def\ZREF@np@call@next##1%
}
\def\ZREF@np@call@unknown#1{#1}
\def\ZREF@np@call@nonext#1{#1}
\def\ZREF@np@call@next#1{#1}
\ZREF@IfDefinable\znextpage\def{%
  {\UniqueCounterCall{znextpage}{\ZREF@nextpage}}%
}%
\newcommand*{\znonextpagename}{}
\newcommand*{\zunknownnextpagename}{\Z@D@page}
\def\ZREF@nextpage#1{%
  \begingroup
    \def\ZREF@refname@this{zref@np#1}%
    \zref@labelbyprops\ZREF@refname@this{abspage}%
    \chardef\ZREF@call=0 % unknown
    \ZREF@ifrefundefined\ZREF@refname@this{%
    }{%
      \edef\ZREF@pagenum@this{%
        \zref@extractdefault\ZREF@refname@this{abspage}{0}%
      }%
      \edef\ZREF@refname@next{%
        \zref@thepage@name{%
          \the\numexpr\ZREF@pagenum@this+1%
        }%
      }%
      \ifnum\ZREF@pagenum@this>0 %
        \ZREF@ifrefundefined{LastPage}{%
          \zref@ifrefundefined\ZREF@refname@next{%
          }{%
            \chardef\ZREF@call=2 % next page
          }%
        }{%
          \edef\ZREF@pagenum@last{%
            \zref@extractdefault{LastPage}{abspage}{0}%
          }%
          \ifnum\ZREF@pagenum@this<\ZREF@pagenum@last\ltx@space
            \ZREF@ifrefundefined\ZREF@refname@next{%
            }{%
              \chardef\ZREF@call=2 % next page
            }%
          \else
            \ifnum\ZREF@pagenum@this=\ZREF@pagenum@this\ltx@space
              \chardef\ZREF@call=1 % no next page
            \fi
          \fi
        }%
      \fi
    }%
    \edef\x{%
      \endgroup
      \ifcase\ZREF@call
        \noexpand\ZREF@np@call@unknown{%
          \noexpand\zunknownnextpagename
        }%
      \or
        \noexpand\ZREF@np@call@nonext{%
          \noexpand\znonextpagename
        }%
      \else
        \noexpand\ZREF@np@call@next{%
          \noexpand\zref@extract{\ZREF@refname@next}{page}%
        }%
      \fi
    }%
  \x
}
%    \end{macrocode}
%    \begin{macrocode}
%</nextpage>
%    \end{macrocode}
%
%
% \subsection{Module \xmodule{totpages}}
%
%    \begin{macrocode}
%<*totpages>
\NeedsTeXFormat{LaTeX2e}
\ProvidesPackage{zref-totpages}%
  [2020-07-03 v2.32 Module totpages for zref (HO)]%
\RequirePackage{zref-base}[2019/11/29]
\ifx\ZREF@base@ok Y%
\else
  \expandafter\endinput
\fi
%    \end{macrocode}
%
%    The absolute page number of the last page is the total
%    page number.
%    \begin{macrocode}
\RequirePackage{zref-abspage}[2019/11/29]
\RequirePackage{zref-lastpage}[2019/11/29]
%    \end{macrocode}
%
%    \begin{macro}{\ztotpages}
%    Macro \cs{ztotpages} contains the number of pages. It can be
%    used inside expandable calculations. It expands to zero if
%    the reference is not yet available.
%    \begin{macrocode}
\newcommand*{\ztotpages}{%
  \zref@extractdefault{LastPage}{abspage}{0}%
}
%    \end{macrocode}
%    \end{macro}
%    Also we mark the reference |LastPage| as used:
%    \begin{macrocode}
\AtBeginDocument{%
  \zref@refused{LastPage}%
}
%    \end{macrocode}
%    \begin{macrocode}
%</totpages>
%    \end{macrocode}
%
%
% \subsection{Module \xmodule{pagelayout}}
%
%    \begin{macrocode}
%<*pagelayout>
\NeedsTeXFormat{LaTeX2e}
\ProvidesPackage{zref-pagelayout}%
  [2020-07-03 v2.32 Module pagelayout for zref (HO)]%
\RequirePackage{zref-base}[2019/11/29]
\ifx\ZREF@base@ok Y%
\else
  \expandafter\endinput
\fi
%    \end{macrocode}
%
%    \begin{macrocode}
\RequirePackage{zref-thepage}[2019/11/29]
\RequirePackage{iftex}[2019/11/07]%
\RequirePackage{atveryend}[2010/03/24]
%    \end{macrocode}
%
%
% \subsubsection{Define layout properties}
%
%    \begin{macrocode}
\def\ZREF@temp#1{%
  \begingroup
    \escapechar=-1 %
  \ltx@ifundefined{\string#1}{\endgroup}{%
    \edef\x{%
      \endgroup
      \noexpand\zref@newprop*{\string#1}%
                             [\noexpand\number\noexpand#1]% hash-ok
                             {\noexpand\number\noexpand#1}%
      \noexpand\zref@addprop{thepage}{\string#1}%
    }%
    \x
  }%
}
\ZREF@temp\mag
\ZREF@temp\paperwidth
\ZREF@temp\paperheight
\ZREF@temp\stockwidth % memoir.cls, crop.sty
\ZREF@temp\stockheight % memoir.cls, crop.sty
\ZREF@temp\mediawidth % VTeX
\ZREF@temp\mediaheight % VTeX
\ifluatex
\zref@newprop*{pdfvorigin}%
                             [\number\pdfvariable vorigin]% hash-ok
                             {\number\pdfvariable vorigin}%
\zref@addprop{thepage}{pdfvorigin}
\zref@newprop*{pdfhorigin}%
                             [\number\pdfvariable horigin]% hash-ok
                             {\number\pdfvariable horigin}%
\zref@addprop{thepage}{pdfhorigin}
\zref@newprop*{pdfpageheight}%
                             [\number\pageheight]% hash-ok
                             {\number\pageheight}%
\zref@addprop{thepage}{pdfpageheight}
\zref@newprop*{pdfpagewidth}%
                             [\number\pagewidth]% hash-ok
                             {\number\pagewidth}%
\zref@addprop{thepage}{pdfpagewidth}
\else
\ZREF@temp\pdfpagewidth
\ZREF@temp\pdfpageheight
\ZREF@temp\pdfhorigin
\ZREF@temp\pdfvorigin
\fi
\ZREF@temp\hoffset
\ZREF@temp\voffset
\ZREF@temp\topmargin
\ZREF@temp\oddsidemargin
\ZREF@temp\evensidemargin
\ZREF@temp\textwidth
\ZREF@temp\textheight
\ZREF@temp\headheight
\ZREF@temp\headsep
\ZREF@temp\footskip
\ZREF@temp\marginparwidth
\ZREF@temp\marginparsep
\ZREF@temp\columnwidth
\ZREF@temp\columnsep
\ZREF@temp\trimedge % memoir.cls
\ZREF@temp\spinemargin % memoir.cls
\ZREF@temp\foremargin % memoir.cls
\ZREF@temp\trimtop % memoir.cls
\ZREF@temp\uppermargin % memoir.cls
\ZREF@temp\headmargin % memoir.cls
%    \end{macrocode}
%
%    \begin{macrocode}
\IfFormatAtLeastTF{2020/10/01}
 {
   \zref@newprop*{outputboxwd}[0pt]{\ShipoutBoxWidth}
   \zref@newprop*{outputboxht}[0pt]{\ShipoutBoxHeight}
   \zref@newprop*{outputboxdp}[0pt]{\ShipoutBoxDepth}
 }
 {
   \zref@newprop*{outputboxwd}[0pt]{\AtBeginShipoutBoxWidth}
   \zref@newprop*{outputboxht}[0pt]{\AtBeginShipoutBoxHeight}
   \zref@newprop*{outputboxdp}[0pt]{\AtBeginShipoutBoxDepth}
 }
%    \end{macrocode}
%    \begin{macrocode}
\zref@addprops{thepage}{outputboxwd,outputboxht,outputboxdp}
%    \end{macrocode}
%
%    \begin{macro}{\ifZREF@pl@list}
%    \begin{macrocode}
\ltx@newif\ifZREF@pl@list
%    \end{macrocode}
%    \end{macro}
%    \begin{macro}{\zref@listpagelayout}
%    \begin{macrocode}
\ZREF@IfDefinable\zlistpagelayout\def{%
  {\global\ZREF@pl@listtrue}%
}
%    \end{macrocode}
%    \end{macro}
%    \begin{macro}{\ZREF@pl@AfterLastShipout}
%    \begin{macrocode}
\def\ZREF@pl@AfterLastShipout{%
  \ifZREF@pl@list
    \edef\ZREF@page@max{\the\value{abspage}}%
    \ltx@ifundefined{ZREF@org@testdef}{%
      \let\ZREF@org@testdef\@testdef
      \def\@testdef##1##2##3{%
        \ZREF@org@testdef{##1}{##2}{##3}%
        \def\ZREF@temp{##1}%
        \ifx\ZREF@temp\ZREF@RefPrefix
          \expandafter\gdef\csname##1@##2\endcsname{##3}%
        \fi
      }%
    }{}%
    \AtVeryEndDocument{\ZREF@pl@AtVeryEnd}%
  \fi
}
%    \end{macrocode}
%    \end{macro}
%    \begin{macro}{\ZREF@pl@AtVeryEnd}
%    \begin{macrocode}
\def\ZREF@pl@AtVeryEnd{%
  \begingroup
    \toks@{Page layout parameters:\MessageBreak}%
    \count@=1 %
    \ZREF@pl@ListPage
    \edef\x{\endgroup
      \noexpand\@PackageInfoNoLine{zref-pagelayout}{\the\toks@}%
    }%
  \x
}
%    \end{macrocode}
%    \end{macro}
%    \begin{macro}{\ZREF@pl@ListPage}
%    \begin{macrocode}
\def\ZREF@pl@ListPage{%
  \edef\x{%
    \toks@={%
      \the\toks@
      Page \the\count@:\noexpand\MessageBreak
      \zref@ifrefundefined{thepage\the\count@}{}{%
        \ltx@space\ltx@space mag = %
        \zref@extract{thepage\the\count@}{mag}%
        \noexpand\MessageBreak
        \ZREF@pl@ListEntry{paperwidth}%
        \ZREF@pl@ListEntry{paperheight}%
        \ZREF@pl@ListEntry{stockwidth}%
        \ZREF@pl@ListEntry{stockheight}%
        \ZREF@pl@ListEntry{mediawidth}%
        \ZREF@pl@ListEntry{mediaheight}%
        \ZREF@pl@ListEntry{pdfpagewidth}%
        \ZREF@pl@ListEntry{pdfpageheight}%
        \ZREF@pl@ListEntry{pdfhorigin}%
        \ZREF@pl@ListEntry{pdfvorigin}%
        \ZREF@pl@ListEntry{hoffset}%
        \ZREF@pl@ListEntry{voffset}%
        \ZREF@pl@ListEntry{topmargin}%
        \ZREF@pl@ListEntry{oddsidemargin}%
        \ZREF@pl@ListEntry{evensidemargin}%
        \ZREF@pl@ListEntry{textwidth}%
        \ZREF@pl@ListEntry{textheight}%
        \ZREF@pl@ListEntry{headheight}%
        \ZREF@pl@ListEntry{headsep}%
        \ZREF@pl@ListEntry{footskip}%
        \ZREF@pl@ListEntry{marginparwidth}%
        \ZREF@pl@ListEntry{marginparsep}%
        \ZREF@pl@ListEntry{columnwidth}%
        \ZREF@pl@ListEntry{columnsep}%
        \ZREF@pl@ListEntry{trimedge}%
        \ZREF@pl@ListEntry{spinemargin}%
        \ZREF@pl@ListEntry{foremargin}%
        \ZREF@pl@ListEntry{trimtop}%
        \ZREF@pl@ListEntry{uppermargin}%
        \ZREF@pl@ListEntry{headmargin}%
      }%
    }%
  }\x
  \ifnum\ZREF@page@max>\count@
    \advance\count@ by\ltx@one
  \else
    \expandafter\ltx@gobble
  \fi
  \ZREF@pl@ListPage
}
%    \end{macrocode}
%    \end{macro}
%    \begin{macro}{\ZREF@pl@ListEntry}
%    \begin{macrocode}
\def\ZREF@pl@ListEntry#1{%
  \zref@ifpropundefined{#1}{%
  }{%
    \zref@ifrefcontainsprop{thepage\the\count@}{#1}{%
      \ltx@space\ltx@space#1 = %
      \zref@extract{thepage\the\count@}{#1}sp = %
      \the\dimexpr\zref@extract{thepage\the\count@}{#1}sp\relax
      \noexpand\MessageBreak
    }{}%
  }%
}
%    \end{macrocode}
%    \end{macro}
%    \begin{macrocode}
\AfterLastShipout{%
  \ZREF@pl@AfterLastShipout
}
%    \end{macrocode}
%
%    \begin{macrocode}
%</pagelayout>
%    \end{macrocode}
%
%
% \subsection{Module \xmodule{pageattr}}
%
%    \begin{macrocode}
%<*pageattr>
\NeedsTeXFormat{LaTeX2e}
\ProvidesPackage{zref-pageattr}%
  [2020-07-03 v2.32 Module pageattr for zref (HO)]%
\RequirePackage{zref-base}[2019/11/29]
\ifx\ZREF@base@ok Y%
\else
  \expandafter\endinput
\fi
%    \end{macrocode}
%
%    \begin{macrocode}
\RequirePackage{iftex}[2019/11/07]%
%    \end{macrocode}
%
%
%    \begin{macrocode}
\let\ZREF@temp=N%
\ifluatex
\expandafter\@firstoftwo
\else
\expandafter\@secondoftwo
\fi
{%luatex
 \RequirePackage{zref-thepage}[2019/11/29]
 \RequirePackage{zref-lastpage}[2019/11/29]%
 \zref@newprop*{pdfpageattr}[]{\zref@hex{\the\pdfvariable pageattr}}%
 \zref@addprop{thepage}{pdfpageattr}%
 \zref@newprop*{pdfpagesattr}[]{\zref@hex{\the\pdfvariable pagesattr}}%
 \zref@addprop{LastPage}{pdfpagesattr}%
 \let\ZREF@temp=Y%
}
{%other
 \ltx@IfUndefined{pdfpageattr}{%
  \@PackageInfoNoLine{zref-pageattr}{%
    \string\pdfpageattr\space is not available%
  }%
  \def\zref@pdfpageattr#1{}%
  \def\zref@pdfpageattr@used#1{}%
 }{%
  \RequirePackage{zref-thepage}[2019/11/29]%
  \zref@newprop*{pdfpageattr}[]{\zref@hex{\the\pdfpageattr}}%
  \zref@addprop{thepage}{pdfpageattr}%
  \let\ZREF@temp=Y%
 }
 \ltx@IfUndefined{pdfpagesattr}{%
  \@PackageInfoNoLine{zref-pageattr}{%
    \string\pdfpagesattr\space is not available%
  }%
  \def\zref@pdfpagesattr{}%
  \def\zref@pdfpagesattr@used{}%
 }{%
  \RequirePackage{zref-lastpage}[2019/11/29]%
  \zref@newprop*{pdfpagesattr}[]{\zref@hex{\the\pdfpagesattr}}%
  \zref@addprop{LastPage}{pdfpagesattr}%
  \let\ZREF@temp=Y%
 }%
}%
%    \end{macrocode}
%    \begin{macrocode}
\ifx\ZREF@temp N%
  \expandafter\endinput
\fi
%    \end{macrocode}
%    \begin{macrocode}
\RequirePackage{zref-abspage}[2019/11/29]
\RequirePackage{atveryend}[2010/03/24]
\RequirePackage{pdftexcmds}[2010/04/01]
\let\ZREF@temp=Y%
\ltx@IfUndefined{pdf@escapehex}{\let\ZREF@temp=N}{}
\ltx@IfUndefined{pdf@unescapehex}{\let\ZREF@temp=N}{}
\ifx\ZREF@temp N%
  \let\zref@hex\ltx@firstofone
  \let\zref@unhex\ltx@firstofone
\else
  \let\zref@hex\pdf@escapehex
  \let\zref@unhex\pdf@unescapehex
\fi
%    \end{macrocode}
%
%    \begin{macro}{\ifZREF@pa@list}
%    \begin{macrocode}
\ltx@newif\ifZREF@pa@list
%    \end{macrocode}
%    \end{macro}
%    \begin{macro}{\zref@listpageattr}
%    \begin{macrocode}
\ZREF@IfDefinable\zlistpageattr\def{%
  {\ZREF@pa@listtrue}%
}
%    \end{macrocode}
%    \end{macro}
%    \begin{macro}{\ZREF@pa@AfterLastShipout}
%    \begin{macrocode}
\def\ZREF@pa@AfterLastShipout{%
  \ifZREF@pa@list
    \edef\ZREF@page@max{\the\value{abspage}}%
    \ltx@ifundefined{ZREF@org@testdef}{%
      \let\ZREF@org@testdef\@testdef
      \def\@testdef##1##2##3{%
        \ZREF@org@testdef{##1}{##2}{##3}%
        \def\ZREF@temp{##1}%
        \ifx\ZREF@temp\ZREF@RefPrefix
          \expandafter\xdef\csname##1@##2\endcsname{##3}%
        \fi
      }%
    }{}%
    \AtVeryEndDocument{\ZREF@pa@AtVeryEnd}%
  \fi
}
%    \end{macrocode}
%    \end{macro}
%    \begin{macro}{\ZREF@pa@AtVeryEnd}
%    \begin{macrocode}
\let\ZREF@temp=Y%
\ltx@IfUndefined{pdfpageattr}{}{\let\ZREF@temp=N}
\ifluatex \let\ZREF@temp=N \fi
\ifx\ZREF@temp Y
 \expandafter\@firstoftwo
\else
 \expandafter\@secondoftwo
\fi
 {%
  \def\ZREF@pa@AtVeryEnd{}%
 }
 {%
  \def\ZREF@pa@AtVeryEnd{%
    \begingroup
      \toks@{List of \ltx@backslashchar
             \ifluatex pdfvariable\else pdf\fi
             pdfpageattr:\MessageBreak}%
      \count@=1 %
      \ZREF@pa@ListPage
      \edef\x{\endgroup
        \noexpand\@PackageInfoNoLine{zref-pageattr}{%
          \the\toks@
        }%
      }%
    \x
  }%
%    \end{macrocode}
%    \end{macro}
%    \begin{macro}{\zref@pageattr}
%    \begin{macrocode}
\def\zref@pdfpageattr#1{%
  \zref@unhex{%
    \zref@extract{thepage\ZREF@number{#1}}{pdfpageattr}%
  }%
}
% compability, \zref@pageattr was defined in older versions
\let\zref@pageattr\zref@pdfpageattr
%    \end{macrocode}
%    \end{macro}
%    \begin{macro}{\zref@pageattr@used}
%    \begin{macrocode}
\ZREF@Robust\def\zref@pageattr@used#1{%
  \zref@refused{thepage\ZREF@number{#1}}%
}
%    \end{macrocode}
%    \end{macro}
%    \begin{macro}{\ZREF@pa@ListPage}
%    \begin{macrocode}
  \def\ZREF@pa@ListPage{%
    \edef\x{%
      \toks@={%
        \the\toks@
        Page \the\count@:%
        \noexpand\MessageBreak
        \zref@ifrefundefined{thepage\the\count@}{}{%
          <<\zref@pdfpageattr\count@>>%
          \noexpand\MessageBreak
        }%
      }%
    }\x
    \ifnum\ZREF@page@max>\count@
      \advance\count@ by\ltx@one
    \else
      \expandafter\ltx@gobble
    \fi
    \ZREF@pa@ListPage
  }%
}
%    \end{macrocode}
%    \end{macro}
%    \begin{macrocode}
\let\ZREF@temp=Y%
\ltx@IfUndefined{pdfpagesattr}{}{\let\ZREF@temp=N}
\ifluatex \let\ZREF@temp=N \fi
\ifx\ZREF@temp N
 \expandafter\@firstofone
\fi
 {%
%    \end{macrocode}
%    \begin{macro}{\zref@pdfpagesattr}
%    \begin{macrocode}
  \def\zref@pdfpagesattr{%
    \zref@unhex{%
      \zref@extract{LastPage}{pdfpagesattr}%
    }%
  }%
%    \end{macrocode}
%    \end{macro}
%    \begin{macro}{\zref@pdfpagesattr@used}
%    \begin{macrocode}
  \ZREF@Robust\def\zref@pdfpagesattr@used{%
    \zref@refused{LastPage}%
  }%
%    \end{macrocode}
%    \end{macro}
%    \begin{macrocode}
  \ltx@LocalAppendToMacro\ZREF@pa@AtVeryEnd{%
    \@PackageInfoNoLine{zref-pageattr}{%
      \ltx@backslashchar
      \ifluatex pdfvariable\else pdf\fi
      pagesattr:\MessageBreak
      <<\zref@pdfpagesattr>>%
      \MessageBreak
    }%
  }%
}
\AfterLastShipout{%
  \ZREF@pa@AfterLastShipout
}
%    \end{macrocode}
%    \begin{macrocode}
%</pageattr>
%    \end{macrocode}
%
%
% \subsection{Module \xmodule{marks}}
%
%    \begin{macrocode}
%<*marks>
\NeedsTeXFormat{LaTeX2e}
\ProvidesPackage{zref-marks}%
  [2020-07-03 v2.32 Module marks for zref (HO)]%
\RequirePackage{zref-base}[2019/11/29]
\ifx\ZREF@base@ok Y%
\else
  \expandafter\endinput
\fi
%    \end{macrocode}
%
%    \begin{macrocode}
\newcommand*{\zref@marks@register}[3][]{%
  \edef\ZREF@TempName{#1}%
  \edef\ZREF@TempNum{\ZREF@number{#2}}%
  \ifnum\ZREF@TempNum<\ltx@zero %
    \PackageError\ZREF@name{%
      \string\zref@marks@register\ltx@space is called with invalid%
      \MessageBreak
      marks register number (\ZREF@TempNum)%
    }{%
      Use `0' or the command, defined by \string\newmarks.\MessageBreak
      \@ehc
    }%
  \else
    \ifx\ZREF@TempName\ltx@empty
      \edef\ZREF@TempName{mark\romannumeral\ZREF@TempNum}%
    \else
      \edef\ZREF@TempName{marks\ZREF@TempName}%
    \fi
    \ZREF@MARKS@DefineProp{top}%
    \ZREF@MARKS@DefineProp{first}%
    \ZREF@MARKS@DefineProp{bot}%
    \kv@parse{#3}{%
      \ifx\kv@value\relax
        \def\kv@value{top,first,bot}%
      \fi
      \edef\ZREF@temp{\expandafter\ltx@car\kv@key X\@nil}%
      \ifx\ZREF@temp\ZREF@STAR
        \edef\kv@key{\expandafter\ltx@cdr\kv@key\@nil}%
        \zref@newlist\kv@key
      \fi
      \expandafter\comma@parse\expandafter{\kv@value}{%
        \ifcase0\ifx\comma@entry\ZREF@NAME@top 1\else
                \ifx\comma@entry\ZREF@NAME@first 1\else
                \ifx\comma@entry\ZREF@NAME@bot 1\fi\fi\fi\ltx@space
          \PackageWarning{zref-marks}{%
            Use `top', `first' or `bot' for the list values%
            \MessageBreak
            in the third argument of \string\zref@marks@register.%
            \MessageBreak
            Ignoring unkown value `\comma@entry'%
          }%
        \else
          \zref@addprop{\kv@key}{\comma@entry\ZREF@TempName}%
        \fi
        \ltx@gobble
      }%
      \ltx@gobbletwo
    }%
  \fi
}
\def\ZREF@STAR{*}
\def\ZREF@NAME@top{top}
\def\ZREF@NAME@first{first}
\def\ZREF@NAME@bot{bot}
\def\ZREF@MARKS@DefineProp#1{%
  \zref@ifpropundefined{#1\ZREF@TempName}{%
    \ifnum\ZREF@TempNum=\ltx@zero
      \begingroup
        \edef\x{\endgroup
          \noexpand\zref@newprop*{#1\ZREF@TempName}[]{%
            \expandafter\noexpand\csname#1mark\endcsname
          }%
        }%
      \x
    \else
      \begingroup
        \edef\x{\endgroup
          \noexpand\zref@newprop*{#1\ZREF@TempName}[]{%
            \expandafter\noexpand\csname#1marks\endcsname
            \ZREF@TempNum
          }%
        }%
      \x
    \fi
  }{%
    \PackageWarning{zref-marks}{%
      \string\zref@marks@register\ltx@space does not generate the%
      \MessageBreak
      new property `#1\ZREF@TempName', because\MessageBreak
      it is already defined%
    }%
  }%
}
%    \end{macrocode}
%    \begin{macrocode}
%</marks>
%    \end{macrocode}
%
% \subsection{Module \xmodule{runs}}
%
%    This module does not use the label-reference-system.
%    The reference changes with each \hologo{LaTeX}~run and would
%    force a rerun warning always.
%    \begin{macrocode}
%<*runs>
\NeedsTeXFormat{LaTeX2e}
\ProvidesPackage{zref-runs}%
  [2020-07-03 v2.32 Module runs for zref (HO)]%
%    \end{macrocode}
%
%    \begin{macro}{\zruns}
%    \begin{macrocode}
\providecommand*{\zruns}{0}%
\AtBeginDocument{%
  \edef\zruns{\number\numexpr\zruns+1}%
  \begingroup
    \def\on@line{}%
    \PackageInfo{zref-runs}{LaTeX runs: \zruns}%
    \if@filesw
      \immediate\write\@mainaux{%
        \string\gdef\string\zruns{\zruns}%
      }%
    \fi
  \endgroup
}
%    \end{macrocode}
%    \end{macro}
%
%    \begin{macrocode}
%</runs>
%    \end{macrocode}
%
%
% \subsection{Module \xmodule{perpage}}
%
%    \begin{macrocode}
%<*perpage>
\NeedsTeXFormat{LaTeX2e}
\ProvidesPackage{zref-perpage}%
  [2020-07-03 v2.32 Module perpage for zref (HO)]%
\RequirePackage{zref-base}[2019/11/29]
\ifx\ZREF@base@ok Y%
\else
  \expandafter\endinput
\fi
%    \end{macrocode}
%
%    This module resets a counter at page boundaries.
%    Because of the asynchronous output routine
%    page counter properties cannot be asked directly,
%    references are necessary.
%
%    For detecting changed pages module \xmodule{abspage}
%    is loaded.
%    \begin{macrocode}
\RequirePackage{zref-abspage}[2019/11/29]
%    \end{macrocode}
%
%    We group the properties for the needed references
%    in the property list |perpage|. The property |pagevalue|
%    records the correct value of the page counter.
%    \begin{macrocode}
\ZREF@NewPropPageValue
\zref@newlist{perpage}
\zref@addprops{perpage}{abspage,page,pagevalue}
%    \end{macrocode}
%    The page value, known by the reference mechanism, will be
%    stored in counter |zpage|.
%    \begin{macrocode}
\newcounter{zpage}
%    \end{macrocode}
%    Counter |zref@unique| helps in generating unique reference
%    names.
%    \begin{macrocode}
\zref@require@unique
%    \end{macrocode}
%    In order to be able to reset the counter, we hook here
%    into \cs{stepcounter}. In fact two nested hooks are used
%    to allow other packages to use the first hook at the beginning
%    of \cs{stepcounter}.
%    \begin{macrocode}
\let\ZREF@org@stepcounter\stepcounter
\def\stepcounter#1{%
  \ifcsname @stepcounterhook@#1\endcsname
    \csname @stepcounterhook@#1\endcsname
  \fi
  \ZREF@org@stepcounter{#1}%
}
%    \end{macrocode}
%    \cs{@stpelt} must be adapted due to the change in latex
%    2015-01, see https://github.com/ho-tex/zref/issues/26
%    \begin{macrocode}
\let\ZREF@org@@stpelt\@stpelt
\def\@stpelt#1{%
  \ifcsname ZREF@perpage@#1\endcsname
    \begingroup
      \let\stepcounter\ZREF@org@stepcounter
      \ZREF@org@@stpelt{#1}%
    \endgroup
    \expandafter\ltx@gobbletwo
  \fi
  \ZREF@org@@stpelt{#1}%
}
%    \end{macrocode}
%
%    \begin{macro}{\zmakeperpage}
%    Makro \cs{zmakeperpage} resets a counter at each page
%    break. It uses the same syntax and semantics as
%    \cs{MakePerPage} from package \xpackage{perpage} \cite{perpage}.
%    The initial start value can be given by the optional
%    argument. Default is one that means after the first
%    \cs{stepcounter} on a new page the counter starts with
%    one.
%    \begin{macrocode}
\ZREF@IfDefinable\zmakeperpage\def{%
  {%
    \@ifnextchar[\ZREF@makeperpage@opt{\ZREF@@makeperpage[\ltx@zero]}%
  }%
}
%    \end{macrocode}
%    \end{macro}
%    We hook before the counter is incremented in \cs{stepcounter},
%    package \xpackage{perpage} afterwards. Thus a little
%    calculation is necessary.
%    \begin{macrocode}
\def\ZREF@makeperpage@opt[#1]{%
  \begingroup
    \edef\x{\endgroup
      \noexpand\ZREF@@makeperpage[\number\numexpr#1-1\relax]%
    }%
  \x
}
%    \end{macrocode}
%    \begin{macrocode}
\def\ZREF@@makeperpage[#1]#2{%
  \@ifundefined{@stepcounterhook@#2}{%
    \expandafter\gdef\csname @stepcounterhook@#2\endcsname{}%
  }{}%
  \expandafter\gdef\csname ZREF@perpage@#2\endcsname{%
    \ZREF@@perpage@step{#2}{#1}%
  }%
  \expandafter\g@addto@macro\csname @stepcounterhook@#2\endcsname{%
    \ifcsname ZREF@perpage@#2\endcsname
      \csname ZREF@perpage@#2\endcsname
    \fi
  }%
}
%    \end{macrocode}
%    \begin{macro}{\ZREF@@perpage@step}
%    The heart of this module follows.
%    \begin{macrocode}
\def\ZREF@@perpage@step#1#2{%
%    \end{macrocode}
%    First the reference is generated.
%    \begin{macrocode}
  \global\advance\c@zref@unique\ltx@one
  \begingroup
    \expandafter
    \zref@labelbylist\expandafter{\thezref@unique}{perpage}%
%    \end{macrocode}
%    The \cs{expandafter} commands are necessary, because \cs{ZREF@temp}
%    is also used inside of \cs{zref@labelbylist}.
%
%    The evaluation of the reference follows. If the reference
%    is not yet kwown, we use the page counter as approximation.
%    \begin{macrocode}
    \zref@ifrefundefined\thezref@unique{%
      \global\c@zpage=\c@page
      \global\let\thezpage\thepage
      \expandafter\xdef\csname ZREF@abspage@#1\endcsname{%
        \number\c@abspage
      }%
    }{%
%    \end{macrocode}
%    The reference is used to set \cs{thezpage} and counter |zpage|.
%    \begin{macrocode}
      \global\c@zpage=\zref@extract\thezref@unique{pagevalue}\relax
      \xdef\thezpage{\noexpand\zref@extract{\thezref@unique}{page}}%
      \expandafter\xdef\csname ZREF@abspage@#1\endcsname{%
        \zref@extractdefault\thezref@unique
            {abspage}{\number\c@abspage}%
      }%
    }%
%    \end{macrocode}
%    Page changes are detected by a changed absolute page number.
%    \begin{macrocode}
    \expandafter\ifx\csname ZREF@abspage@#1\expandafter\endcsname
                    \csname ZREF@currentabspage@#1\endcsname
    \else
      \global\csname c@#1\endcsname=#2\relax
      \global\expandafter\let
          \csname ZREF@currentabspage@#1\expandafter\endcsname
          \csname ZREF@abspage@#1\endcsname
    \fi
  \endgroup
}
%    \end{macrocode}
%    \end{macro}
%    \begin{macro}{\zunmakeperpage}
%    Macro \cs{zunmakeperpage} cancels the effect of \cs{zmakeperpage}.
%    \begin{macrocode}
\ZREF@IfDefinable\zunmakeperpage\def{%
  #1{%
    \global\expandafter
    \let\csname ZREF@perpage@#1\endcsname\@undefined
  }%
}
%    \end{macrocode}
%    \end{macro}
%
%    \begin{macrocode}
%</perpage>
%    \end{macrocode}
%
%
% \subsection{Module \xmodule{titleref}}
%
%    \begin{macrocode}
%<*titleref>
\NeedsTeXFormat{LaTeX2e}
\ProvidesPackage{zref-titleref}%
  [2020-07-03 v2.32 Module titleref for zref (HO)]%
\RequirePackage{zref-base}[2019/11/29]
\ifx\ZREF@base@ok Y%
\else
  \expandafter\endinput
\fi
\RequirePackage{gettitlestring}[2009/12/08]
%    \end{macrocode}
%
% \subsubsection{Implementation}
%    \begin{macrocode}
\RequirePackage{keyval}
%    \end{macrocode}
%
%    This module makes section and caption titles available
%    for the reference system. It uses some of the ideas
%    of package \xpackage{nameref} and \xpackage{titleref}.
%
%    Now we can add the property \verb|title| is added to
%    the main property list.
%    \begin{macrocode}
\ZREF@NewPropTitle
\zref@addprop\ZREF@mainlist{title}%
%    \end{macrocode}
%
%    The title strings go into the \xfile{.aux} file, thus
%    they need some kind of protection.
%    Package \xpackage{titleref} uses a protected expansion
%    method. The advantage is that this can be used to
%    cleanup the string and to remove \cs{label}, \cs{index}
%    and other macros unwanted for referencing.
%    But there is the risk that fragile stuff can break.
%
%    Therefore package \xpackage{nameref} does not expand
%    the string. Thus the entries can safely be written to
%    the \xfile{.aux} file. But potentially dangerous
%    macros such as \cs{label} remain in the string and
%    can cause problems when using the string in references.
%
%    \begin{macro}{\ifzref@titleref@expand}
%    The switch \cs{ifzref@titleref@expand} distinguishes
%    between the both methods. Package \xpackage{nameref}'s
%    behaviour is achieved by setting the switch to false,
%    otherwise \xpackage{titleref}'s expansion is used.
%    Default is false.
%    \begin{macrocode}
\newif\ifzref@titleref@expand
%    \end{macrocode}
%    \end{macro}
%
%    \begin{macro}{\ZREF@titleref@hook}
%    The hook \cs{ZREF@titleref@hook} allows to extend
%    the cleanup for the expansion method. Thus
%    unnecessary macros can be removed or dangerous commands
%    removed. The hook is executed before the expansion of
%    \cs{zref@titleref@current}.
%    \begin{macrocode}
\let\ZREF@titleref@hook\ltx@empty
%    \end{macrocode}
%    \end{macro}
%    \begin{macro}{\zref@titleref@cleanup}
%    The hook should not be used directly, instead we provide the
%    macro \cs{zref@titleref@cleanup} to add stuff to the hook
%    and prevents that a previous non-empty content is not discarded
%    accidently.
%    \begin{macrocode}
\ZREF@Robust\def\zref@titleref@cleanup#1{%
   \begingroup
    \toks@\expandafter{%
      \ZREF@titleref@hook
      #1%
    }%
  \expandafter\endgroup
  \expandafter\def\expandafter\ZREF@titleref@hook\expandafter{%
    \the\toks@
  }%
}%
%    \end{macrocode}
%    \end{macro}
%
%    \begin{macro}{\ifzref@titleref@stripperiod}
%    Sometimes a title contains a period at the end.
%    Package \xpackage{nameref} removes this. This behaviour
%    is controlled by the switch \cs{ifzref@titleref@stripperiod}
%    and works regardless of the setting of option \xoption{expand}.
%    Period stripping is the default.
%    \begin{macrocode}
\newif\ifzref@titleref@stripperiod
\zref@titleref@stripperiodtrue
%    \end{macrocode}
%    \end{macro}
%
%    \begin{macro}{\zref@titleref@setcurrent}
%    Macro \cs{zref@titleref@setcurrent} sets a new current title
%    stored in \cs{zref@titleref@current}. Some cleanup and
%    expansion is performed that can be controlled by the
%    previous switches.
%    \begin{macrocode}
\ZREF@Robust\def\zref@titleref@setcurrent#1{%
  \ifzref@titleref@expand
    \GetTitleStringExpand{#1}%
  \else
    \GetTitleStringNonExpand{#1}%
  \fi
  \edef\zref@titleref@current{%
    \detokenize\expandafter{\GetTitleStringResult}%
  }%
  \ifzref@titleref@stripperiod
    \edef\zref@titleref@current{%
      \expandafter\ZREF@stripperiod\zref@titleref@current
      \ltx@empty.\ltx@empty\@nil
    }%
  \fi
}%
\GetTitleStringDisableCommands{%
  \ZREF@titleref@hook
}
%    \end{macrocode}
%    \end{macro}
%    \begin{macro}{\ZREF@stripperiod}
%    If \cs{ZREF@stripperiod} is called, the argument consists of
%    space tokens and tokens with catcode 12 (other), because
%    of \hologo{eTeX}'s \cs{detokenize}.
%    \begin{macrocode}
\def\ZREF@stripperiod#1.\ltx@empty#2\@nil{#1}%
%    \end{macrocode}
%    \end{macro}
%
% \subsubsection{User interface}
%
%    \begin{macro}{\ztitlerefsetup}
%    The behaviour of module \xmodule{titleref} is controlled
%    by switches and a hook. They can be set by \cs{ztitlerefsetup}
%    with a key value interface, provided by package \xpackage{keyval}.
%    Also the current title can be given explicitly by the key
%    \verb|title|.
%    \begin{macrocode}
\define@key{ZREF@TR}{expand}[true]{%
  \csname zref@titleref@expand#1\endcsname
}%
\define@key{ZREF@TR}{stripperiod}[true]{%
  \csname zref@titleref@stripperiod#1\endcsname
}%
\define@key{ZREF@TR}{cleanup}{%
  \zref@titleref@cleanup{#1}%
}%
\define@key{ZREF@TR}{title}{%
  \def\zref@titleref@current{#1}%
}%
\ZREF@IfDefinable\ztitlerefsetup\def{%
  {\kvsetkeys{ZREF@TR}}%
}%
%    \end{macrocode}
%    \end{macro}
%
%    \begin{macro}{\ztitleref}
%    The user command \cs{ztitleref} references the title.
%    For safety \cs{label} is disabled to prevent multiply
%    defined references.
%    \begin{macrocode}
\ZREF@IfDefinable\ztitleref\def{%
  {\zref@wrapper@babel\ZREF@titleref}%
}%
\def\ZREF@titleref#1{%
  \begingroup
    \zref@refused{#1}%
    \let\label\ltx@gobble
    \zref@extract{#1}{title}%
  \endgroup
}%
%    \end{macrocode}
%    \end{macro}
%
% \subsubsection{Patches for section and caption commands}
%
%    The section and caption macros are patched to extract the
%    title data.
%
%    Captions of figures and tables.
%    \begin{macrocode}
\AtBeginDocument{%
  \ZREF@patch{@caption}{%
    \long\def\@caption#1[#2]{%
      \zref@titleref@setcurrent{#2}%
      \ZREF@org@@caption{#1}[{#2}]%
    }%
  }%
%    \end{macrocode}
%
%    Section commands without star. The title version for the
%    table of contents is used because it is usually shorter and
%    more robust.
%    \begin{macrocode}
  \ZREF@patch{@part}{%
    \def\@part[#1]{%
      \zref@titleref@setcurrent{#1}%
      \ZREF@org@@part[{#1}]%
    }%
  }%
  \ZREF@patch{@chapter}{%
    \def\@chapter[#1]{%
      \zref@titleref@setcurrent{#1}%
      \ZREF@org@@chapter[{#1}]%
    }%
  }%
  \ZREF@patch{@sect}{%
    \def\@sect#1#2#3#4#5#6[#7]{%
      \zref@titleref@setcurrent{#7}%
      \ZREF@org@@sect{#1}{#2}{#3}{#4}{#5}{#6}[{#7}]%
    }%
  }%
%    \end{macrocode}
%    The star versions of the section commands.
%    \begin{macrocode}
  \ZREF@patch{@spart}{%
    \def\@spart#1{%
      \zref@titleref@setcurrent{#1}%
      \ZREF@org@@spart{#1}%
    }%
  }%
  \ZREF@patch{@schapter}{%
    \def\@schapter#1{%
      \zref@titleref@setcurrent{#1}%
      \ZREF@org@@schapter{#1}%
    }%
  }%
  \ZREF@patch{@ssect}{%
    \def\@ssect#1#2#3#4#5{%
      \zref@titleref@setcurrent{#5}%
      \ZREF@org@@ssect{#1}{#2}{#3}{#4}{#5}%
    }%
  }%
%    \end{macrocode}
%
% \subsubsection{Environment \textsf{description}}
%
%    \begin{macrocode}
  \ZREF@patch{descriptionlabel}{%
    \def\descriptionlabel#1{%
      \zref@titleref@setcurrent{#1}%
      \ZREF@org@descriptionlabel{#1}%
    }%
  }%
%    \end{macrocode}
%
% \subsubsection{Class \xclass{memoir}}
%
%    \begin{macrocode}
  \@ifclassloaded{memoir}{%
    \ltx@IfUndefined{ifheadnameref}{}{%
      \def\@chapter[#1]#2{%
        \ltx@IfUndefined{ch@pt@c}{%
          \zref@titleref@setcurrent{#1}%
        }{%
          \ifx\ch@pt@c\ltx@empty
            \zref@titleref@setcurrent{#2}%
          \else
            \def\NR@temp{#1}%
            \ifx\NR@temp\ltx@empty
              \expandafter\zref@titleref@setcurrent
              \expandafter{\ch@pt@c}%
            \else
              \ifheadnameref
                \zref@titleref@setcurrent{#1}%
              \else
                \expandafter\zref@titleref@setcurrent
                \expandafter{\ch@pt@c}%
              \fi
            \fi
          \fi
        }%
        \ZREF@org@@chapter[{#1}]{#2}%
      }%
      \ZREF@patch{M@sect}{%
        \def\M@sect#1#2#3#4#5#6[#7][#8]{%
          \ifheadnameref
            \zref@titleref@setcurrent{#8}%
          \else
            \zref@titleref@setcurrent{#7}%
          \fi
          \ZREF@org@M@sect{#1}{#2}{#3}{#4}{#5}{#6}[{#7}][{#8}]%
        }%
      }%
    }%
  }{}%
%    \end{macrocode}
%
% \subsubsection{Class \xclass{beamer}}
%
%    \begin{macrocode}
  \@ifclassloaded{beamer}{%
    \ZREF@patch{beamer@section}{%
      \long\def\beamer@section[#1]{%
        \zref@titleref@setcurrent{#1}%
        \ZREF@org@beamer@section[{#1}]%
      }%
    }%
    \ZREF@patch{beamer@subsection}{%
      \long\def\beamer@subsection[#1]{%
        \zref@titleref@setcurrent{#1}%
        \ZREF@org@beamer@subsection[{#1}]%
      }%
    }%
    \ZREF@patch{beamer@subsubsection}{%
      \long\def\beamer@subsubsection[#1]{%
        \zref@titleref@setcurrent{#1}%
        \ZREF@org@beamer@subsubsection[{#1}]%
      }%
    }%
  }{}%
%    \end{macrocode}
%
% \subsubsection{Package \xpackage{titlesec}}
%
%    \begin{macrocode}
  \@ifpackageloaded{titlesec}{%
    \ZREF@patch{ttl@sect@i}{%
      \def\ttl@sect@i#1#2[#3]#4{%
        \zref@titleref@setcurrent{#4}%
        \ZREF@org@ttl@sect@i{#1}{#2}[{#3}]{#4}%
      }%
    }%
    \ZREF@patch{ttl@straight@i}{%
      \def\ttl@straight@i#1[#2]#3{%
        \def\ZREF@temp{#2}%
        \ifx\ZREF@temp\ltx@empty
          \zref@titleref@setcurrent{#3}%
        \else
          \zref@titleref@setcurrent{#2}%
        \fi
        \ZREF@org@ttl@straight@i{#1}[{#2}]{#3}%
      }%
    }%
  }{}%
%    \end{macrocode}
%
% \subsubsection{Package \xpackage{longtable}}
%
%    Package \xpackage{longtable}: some support for its \cs{caption}.
%    However \cs{label} inside the caption is not supported.
%    \begin{macrocode}
  \@ifpackageloaded{longtable}{%
    \ZREF@patch{LT@c@ption}{%
      \def\LT@c@ption#1[#2]#3{%
        \ZREF@org@LT@c@ption{#1}[{#2}]{#3}%
        \zref@titleref@setcurrent{#2}%
      }%
    }%
  }{}%
%    \end{macrocode}
%
% \subsubsection{Package \xpackage{listings}}
%
%    Package \xpackage{listings}: support for its caption.
%    \begin{macrocode}
  \@ifpackageloaded{listings}{%
    \ZREF@patch{lst@MakeCaption}{%
      \def\lst@MakeCaption{%
        \ifx\lst@label\ltx@empty
        \else
          \expandafter\zref@titleref@setcurrent\expandafter{%
            \lst@@caption
          }%
        \fi
        \ZREF@org@lst@MakeCaption
      }%
    }%
  }{}%
%    \end{macrocode}
%
% \subsubsection{Theorems}
%
%    \begin{macrocode}
  \ZREF@patch{@opargbegintheorem}{%
    \def\@opargbegintheorem#1#2#3{%
      \zref@titleref@setcurrent{#3}%
      \ZREF@org@@opargbegintheorem{#1}{#2}{#3}%
    }%
  }%
  \@ifpackageloaded{amsthm}{%
    \begingroup
      \edef\x{macro:\string#1\string#2[\string#3]}%
      \@onelevel@sanitize\x
      \def\y#1->#2\@nil{#1}%
      \edef\z{\expandafter\y\meaning\@begintheorem->\@nil}%
      \@onelevel@sanitize\z
    \expandafter\endgroup
    \ifx\x\z
      \ZREF@patch{@begintheorem}{%
        \def\@begintheorem#1#2[#3]{%
          \zref@titleref@setcurrent{#3}%
          \ZREF@org@@begintheorem{#1}{#2}[{#3}]%
        }%
      }%
    \fi
  }{}%
%    \end{macrocode}
%
%    \begin{macrocode}
}
%    \end{macrocode}
%
%    \begin{macrocode}
%</titleref>
%    \end{macrocode}
%
%
% \subsection{Module \xmodule{xr}}
%
%    \begin{macrocode}
%<*xr>
\NeedsTeXFormat{LaTeX2e}
\ProvidesPackage{zref-xr}%
  [2020-07-03 v2.32 Module xr for zref (HO)]%
\RequirePackage{zref-base}[2019/11/29]
\ifx\ZREF@base@ok Y%
\else
  \expandafter\endinput
\fi
%    \end{macrocode}
%    \begin{macrocode}
\RequirePackage{keyval}
\RequirePackage{kvoptions}[2010/02/22]
%    \end{macrocode}
%
%    We declare property |url|, because this is added,
%    if a reference is imported and has not already set this field.
%    Or if |hyperref| is used, then this property can be asked.
%    \begin{macrocode}
\zref@newprop{url}{}%
\zref@newprop{urluse}{}%
%    \end{macrocode}
%    \begin{macrocode}
\zref@newprop{externaldocument}{}%
%    \end{macrocode}
%
%    Most code, especially the handling of the \xfile{.aux}
%    files are taken from David Carlisle's \xpackage{xr} package.
%    Therefore I drop the documentation for these macros here.
%
%    \begin{macro}{\zref@xr@ext}
%    If the URL is not specied, then assume
%    processed file with a guessed extension.
%    Use the setting of \xpackage{hyperref} if available.
%    \begin{macrocode}
\providecommand*{\zref@xr@ext}{%
  \ltx@ifundefined{XR@ext}{pdf}{\XR@ext}%
}%
%    \end{macrocode}
%    \end{macro}
%    \begin{macro}{\ifZREF@xr@zreflabel}
%    The use of the star form of \cs{zexternaldocument}
%    is remembered in the switch \cs{ifZREF@xr@zreflabel}.
%    \begin{macrocode}
\newif\ifZREF@xr@zreflabel
%    \end{macrocode}
%    \end{macro}
%    \begin{macrocode}
\SetupKeyvalOptions{%
  family=ZREF@XR,%
  prefix=ZREF@xr@%
}
\DeclareBoolOption[true]{tozreflabel}
\DeclareBoolOption[false]{toltxlabel}
\DeclareBoolOption{verbose}
\define@key{ZREF@XR}{ext}{%
  \def\zref@xr@{#1}%
}
\DeclareBoolOption[false]{urluse}
%    \end{macrocode}
%    \begin{macro}{\zxrsetup}
%    \begin{macrocode}
\newcommand*{\zxrsetup}{%
  \kvsetkeys{ZREF@XR}%
}%
%    \end{macrocode}
%    \end{macro}
%
%    \begin{macro}{\ZREF@xr@URL}
%    \begin{macrocode}
\newcount\ZREF@xr@URL
\ZREF@xr@URL=\ltx@zero
%    \end{macrocode}
%    \end{macro}
%    \begin{macro}{\ZREF@xr@AddURL}
%    \begin{macrocode}
\def\ZREF@xr@AddURL#1{%
  \begingroup
    \def\ZREF@temp{#1}%
    \count@=\ltx@one
    \ZREF@xr@@AddUrl
  \endgroup
}
%    \end{macrocode}
%    \end{macro}
%    \begin{macro}{\ZREF@xr@@AddUrl}
%    \begin{macrocode}
\def\ZREF@xr@@AddUrl{%
  \ifnum\count@>\ZREF@xr@URL
    \global\advance\ZREF@xr@URL by\ltx@one
    \xdef\ZREF@xr@theURL{\romannumeral\ZREF@xr@URL}%
    \global\expandafter\let
        \csname Z@U@\ZREF@xr@theURL\endcsname\ZREF@temp
    \@PackageInfo{zref-xr}{%
      \ltx@backslashchar Z@U@\ZREF@xr@theURL:\MessageBreak
      \ZREF@temp\MessageBreak
    }%
  \else
    \expandafter
    \ifx\csname Z@U@\romannumeral\count@\endcsname\ZREF@temp
      \xdef\ZREF@xr@theURL{\romannumeral\count@}%
    \else
      \expandafter\expandafter\expandafter\ZREF@xr@@AddUrl
    \fi
  \fi
}
%    \end{macrocode}
%    \end{macro}
%
%    \begin{macro}{\zexternaldocument}
%    In its star form it looks for \cs{newlabel},
%    otherwise for \cs{zref@newlabel}.
%     Later we
%    will read \xfile{.aux} files that expects
%    |@| to have catcode 11 (letter).
%    \begin{macrocode}
\ZREF@IfDefinable\zexternaldocument\def{%
  {%
    \ZREF@NewPropAnchor
    \ZREF@NewPropTitle
    \begingroup
      \csname @safe@actives@true\endcsname
      \makeatletter
      \@ifstar{%
        \ZREF@xr@zreflabelfalse
        \@testopt\ZREF@xr@externaldocument{}%
      }{%
        \ZREF@xr@zreflabeltrue
        \@testopt\ZREF@xr@externaldocument{}%
      }%
  }%
}%
%    \end{macrocode}
%    \end{macro}
%
%    If the \cs{include} featuer was used, there can be
%    several \xfile{.aux} files. These files are read
%    one after another, especially they are not recursively
%    read in order to save read registers. Thus it can
%    happen that the read order of the newlabel commands
%    differs from \hologo{LaTeX}'s order using \cs{input}.
%
%    \begin{macro}{\ZREF@xr@externaldocument}
%    It reads the remaining arguments. \cs{newcommand}
%    comes in handy for the optional argument.
%    \begin{macrocode}
\def\ZREF@xr@externaldocument[#1]#2{%
    \def\ZREF@xr@prefix{#1}%
    \let\ZREF@xr@filelist\ltx@empty
    \edef\ZREF@xr@externalfile{#2}%
    \edef\ZREF@xr@file{\ZREF@xr@externalfile.aux}%
    \filename@parse{#2}%
    \@testopt\ZREF@xr@graburl{#2.\zref@xr@ext}%
}%
\def\ZREF@xr@graburl[#1]{%
    \edef\ZREF@xr@url{#1}%
    \ifZREF@xr@urluse
      \expandafter\ZREF@xr@AddURL\expandafter{\ZREF@xr@url}%
      \expandafter\def\expandafter\ZREF@xr@url
      \expandafter{\csname Z@U@\ZREF@xr@theURL\endcsname}%
    \fi
    \ZREF@xr@checkfile
  \endgroup
}%
%    \end{macrocode}
%    \end{macro}
%    \begin{macro}{\ZREF@xr@processfile}
%    We follow \xpackage{xr} here, \cs{IfFileExists} offers
%    a nicer test, but we have to open the file anyway.
%    \begin{macrocode}
\def\ZREF@xr@checkfile{%
  \openin\@inputcheck\ZREF@xr@file\relax
  \ifeof\@inputcheck
    \PackageWarning{zref-xr}{%
      File `\ZREF@xr@file' not found or empty,\MessageBreak
      labels not imported%
    }%
  \else
    \PackageInfo{zref-xr}{%
      Label \ifZREF@xr@zreflabel (zref) \fi
      import from `\ZREF@xr@file'%
    }%
    \def\ZREF@xr@found{0}%
    \def\ZREF@xr@ignored@empty{0}%
    \def\ZREF@xr@ignored@zref{0}%
    \def\ZREF@xr@ignored@ltx{0}%
    \ZREF@xr@processfile
    \closein\@inputcheck
    \begingroup
      \let\on@line\ltx@empty
      \PackageInfo{zref-xr}{%
        Statistics for `\ZREF@xr@file':\MessageBreak
        \ZREF@xr@found\space
        \ifZREF@xr@zreflabel zref\else LaTeX\fi\space
        label(s) found%
        \ifnum\ZREF@xr@ignored@empty>0 %
          ,\MessageBreak
          \ZREF@xr@ignored@empty\space empty label(s) ignored%
        \fi
        \ifnum\ZREF@xr@ignored@zref>0 %
          ,\MessageBreak
          \ZREF@xr@ignored@zref\space
          duplicated zref label(s) ignored%
        \fi
        \ifnum\ZREF@xr@ignored@ltx>0 %
          ,\MessageBreak
          \ZREF@xr@ignored@ltx\space
          duplicated latex label(s) ignored%
        \fi
      }%
    \endgroup
  \fi
  \ifx\ZREF@xr@filelist\ltx@empty
  \else
    \edef\ZREF@xr@file{%
      \expandafter\ltx@car\ZREF@xr@filelist\@nil
    }%
    \edef\ZREF@xr@filelist{%
      \expandafter\ltx@cdr\ZREF@xr@filelist\ltx@empty\@nil
    }%
    \expandafter\ZREF@xr@checkfile
  \fi
}%
%    \end{macrocode}
%    \end{macro}
%
%    \begin{macro}{\ZREF@xr@processfile}
%    \begin{macrocode}
\def\ZREF@xr@processfile{%
  \read\@inputcheck to\ZREF@xr@line
  \expandafter\ZREF@xr@processline\ZREF@xr@line..\ZREF@nil
  \ifeof\@inputcheck
  \else
    \expandafter\ZREF@xr@processfile
  \fi
}%
%    \end{macrocode}
%    \end{macro}
%    \begin{macro}{\ZREF@xr@processline}
%    The most work must be done for analyzing the
%    arguments of \cs{newlabel}.
%    \begin{macrocode}
\long\def\ZREF@xr@processline#1#2#3\ZREF@nil{%
  \def\x{#1}%
  \toks@{#2}%
  \ifZREF@xr@zreflabel
    \ifx\x\ZREF@xr@zref@newlabel
      \expandafter
      \ZREF@xr@process@zreflabel\ZREF@xr@line...\ZREF@nil
    \fi
  \else
    \ifx\x\ZREF@xr@newlabel
      \expandafter
      \ZREF@xr@process@label\ZREF@xr@line...[]\ZREF@nil
    \fi
  \fi
  \ifx\x\ZREF@xr@@input
    \edef\ZREF@xr@filelist{%
      \etex@unexpanded\expandafter{\ZREF@xr@filelist}%
      {\filename@area\the\toks@}%
    }%
  \fi
}%
\def\ZREF@xr@process@zreflabel\zref@newlabel#1#2#3\ZREF@nil{%
  \edef\ZREF@xr@refname{Z@R@\ZREF@xr@prefix#1}%
  \edef\ZREF@xr@found{\the\numexpr\ZREF@xr@found+1\relax}%
  \def\x{#2}%
  \edef\ZREF@xr@tempname{$temp$}%
  \edef\ZREF@xr@temprefname{Z@R@\ZREF@xr@tempname}%
  \let\ZREF@xr@list\x
  \ifx\ZREF@xr@list\ltx@empty
    \PackageWarningNoLine{zref-xr}{%
      Label `#1' without properties ignored\MessageBreak
      in file `\ZREF@xr@file'%
    }%
    \edef\ZREF@xr@ignored@empty{%
      \the\numexpr\ZREF@xr@ignored@empty+1\relax
    }%
  \else
    \expandafter\ZREF@xr@checklist\x\ZREF@nil
    \expandafter\let\csname\ZREF@xr@temprefname\endcsname\x
    \expandafter\ltx@LocalAppendToMacro
    \csname\ZREF@xr@temprefname\expandafter\endcsname
    \expandafter{%
      \expandafter\externaldocument\expandafter{%
        \ZREF@xr@externalfile
      }%
    }%
    \ZREF@xr@urlcheck\ZREF@xr@tempname
    \ifZREF@xr@tozreflabel
      \@ifundefined{\ZREF@xr@refname}{%
        \ifZREF@xr@verbose
          \PackageInfo{zref-xr}{%
            Import to zref label `\ZREF@xr@tempname#1'%
          }%
        \fi
        \global\expandafter
        \let\csname\ZREF@xr@refname\expandafter\endcsname
        \csname\ZREF@xr@temprefname\endcsname
      }{%
        \ZREF@xr@zref@ignorewarning{\ZREF@xr@prefix#1}%
      }%
    \fi
    \ifZREF@xr@toltxlabel
      \ZREF@xr@tolabel{\ZREF@xr@tempname}{\ZREF@xr@prefix#1}%
    \fi
  \fi
}%
\def\ZREF@xr@process@label\newlabel#1#2#3[#4]#5\ZREF@nil{%
  \def\ZREF@xr@refname{Z@R@\ZREF@xr@prefix#1}%
  \edef\ZREF@xr@found{\the\numexpr\ZREF@xr@found+1\relax}%
  \def\x{#2}%
  \edef\ZREF@xr@tempname{$temp$}%
  \edef\ZREF@xr@temprefname{Z@R@\ZREF@xr@tempname}%
  \expandafter\ZREF@xr@scanparams
      \csname\ZREF@xr@temprefname\expandafter\endcsname
      \x{}{}{}{}{}\ZREF@nil
  \ifx\\#4\\%
  \else
    % ntheorem knows an optional argument at the end of \newlabel
    \ZREF@NewPropTheotype
    \expandafter\ltx@LocalAppendToMacro
        \csname\ZREF@xr@temprefname\endcsname{\theotype{#4}}%
  \fi
  \expandafter\ltx@LocalAppendToMacro
  \csname\ZREF@xr@temprefname\expandafter\endcsname\expandafter{%
    \expandafter\externaldocument\expandafter{%
      \ZREF@xr@externalfile
    }%
  }%
  \ZREF@xr@urlcheck\ZREF@xr@tempname
  \ifZREF@xr@tozreflabel
    \@ifundefined{\ZREF@xr@refname}{%
      \ifZREF@xr@verbose
        \PackageInfo{zref-xr}{%
          Import to zref label `\ZREF@xr@prefix#1'%
        }%
      \fi
      \global\expandafter
      \let\csname\ZREF@xr@refname\expandafter\endcsname
      \csname\ZREF@xr@temprefname\endcsname
    }{%
      \ZREF@xr@zref@ignorewarning{\ZREF@xr@prefix#1}%
    }%
  \fi
  \ifZREF@xr@toltxlabel
    \ZREF@xr@tolabel{\ZREF@xr@tempname}{\ZREF@xr@prefix#1}%
  \fi
}
\def\ZREF@xr@zref@newlabel{\zref@newlabel}%
\def\ZREF@xr@newlabel{\newlabel}%
\def\ZREF@xr@@input{\@input}%
\def\ZREF@xr@relax{\relax}%
%    \end{macrocode}
%    \end{macro}
%    \begin{macro}{\ZREF@xr@tolabel}
%    \begin{macrocode}
\def\ZREF@xr@tolabel#1#2{%
  \ifZREF@xr@verbose
    \PackageInfo{zref-xr}{%
      Import to LaTeX label `#2'%
    }%
  \fi
  \zref@wrapper@unexpanded{%
    \expandafter\xdef\csname r@#2\endcsname{%
      {%
        \ltx@ifundefined{M@TitleReference}{%
          \ltx@ifundefined{TR@TitleReference}{%
            \zref@extractdefault{#1}{default}{}%
          }{%
            \noexpand\TR@TitleReference
            {\zref@extractdefault{#1}{default}{}}%
            {\zref@extractdefault{#1}{title}{}}%
          }%
        }{%
          \noexpand\M@TitleReference
          {\zref@extractdefault{#1}{default}{}}%
          {\zref@extractdefault{#1}{title}{}}%
        }%
      }%
      {\zref@extractdefault{#1}{page}{}}%
      \ltx@ifpackageloaded{nameref}{%
        {\zref@extractdefault{#1}{title}{}}%
        {\zref@extractdefault{#1}{anchor}{}}%
        \zref@ifrefcontainsprop{#1}{urluse}{%
          {\zref@extractdefault{#1}{urluse}{}}%
        }{%
          {\zref@extractdefault{#1}{url}{}}%
        }%
      }{}%
    }%
  }%
}
%    \end{macrocode}
%    \end{macro}
%
%    \begin{macro}{\ZREF@xr@zref@ignorewarning}
%    \begin{macrocode}
\def\ZREF@xr@zref@ignorewarning#1{%
  \PackageWarningNoLine{zref-xr}{%
    Zref label `#1' is already in use\MessageBreak
    in file `\ZREF@xr@file'%
  }%
  \edef\ZREF@xr@ignored@zref{%
    \the\numexpr\ZREF@xr@ignored@zref+1%
  }%
}%
%    \end{macrocode}
%    \end{macro}
%
%    \begin{macro}{\ZREF@xr@ltx@ignorewarning}
%    \begin{macrocode}
\def\ZREF@xr@ltx@ignorewarning#1{%
  \PackageWarningNoLine{zref-xr}{%
    LaTeX label `#1' is already in use\MessageBreak
    in file `\ZREF@xr@file'%
  }%
  \edef\ZREF@xr@ignored@ltx{%
    \the\numexpr\ZREF@xr@ignored@ltx+1%
  }%
}%
%    \end{macrocode}
%    \end{macro}
%
%    \begin{macro}{\ZREF@xr@checklist}
%    \begin{macrocode}
\def\ZREF@xr@checklist#1#2#3\ZREF@nil{%
  \ifx\@undefined#1\relax
    \expandafter\ZREF@xr@checkkey\string#1\@nil
  \fi
  \ifx\\#3\\%
  \else
    \ltx@ReturnAfterFi{%
      \ZREF@xr@checklist#3\ZREF@nil
    }%
  \fi
}%
\def\ZREF@xr@checkkey#1#2\@nil{%
  \zref@ifpropundefined{#2}{%
    \zref@newprop{#2}{}%
  }{}%
}%
%    \end{macrocode}
%    \end{macro}
%
%    \begin{macro}{\ZREF@xr@scanparams}
%    \begin{macrocode}
\def\ZREF@xr@scanparams#1#2#3#4#5#6#7\ZREF@nil{%
  \let#1\ltx@empty
  \ZREF@foundfalse
  \ZREF@xr@scantitleref#1#2\TR@TitleReference{}{}\ZREF@nil
  \ifZREF@found
  \else
    \ltx@LocalAppendToMacro#1{\default{#2}}%
  \fi
  % page
  \ltx@LocalAppendToMacro#1{\page{#3}}%
  % nameref title
  \ifZREF@found
  \else
    \ifx\\#4\\%
    \else
      \def\ZREF@xr@temp{#4}%
      \ifx\ZREF@xr@temp\ZREF@xr@relax
      \else
        \ltx@LocalAppendToMacro#1{\title{#4}}%
      \fi
    \fi
  \fi
  % anchor
  \ifx\\#5\\%
  \else
    \ltx@LocalAppendToMacro#1{\anchor{#5}}%
  \fi
  \ifx\\#6\\%
  \else
    \ifZREF@xr@urluse
      \ZREF@xr@AddURL{#6}%
      \expandafter\ltx@LocalAppendToMacro\expandafter#1%
      \expandafter{%
        \expandafter\urluse\expandafter{%
          \csname Z@U@\ZREF@xr@theURL\endcsname
        }%
      }%
    \else
      \ltx@LocalAppendToMacro#1{\url{#6}}%
    \fi
  \fi
}%
%    \end{macrocode}
%    \end{macro}
%
%    \begin{macro}{\ZREF@xr@scantitleref}
%    \begin{macrocode}
\def\ZREF@xr@scantitleref#1#2\TR@TitleReference#3#4#5\ZREF@nil{%
  \ifx\\#5\\%
  \else
    \ltx@LocalAppendToMacro#1{%
      \default{#3}%
      \title{#4}%
    }%
    \ZREF@foundtrue
  \fi
}%
%    \end{macrocode}
%    \end{macro}
%
%    \begin{macro}{\ZREF@xr@urlcheck}%
%    \begin{macrocode}
\def\ZREF@xr@urlcheck#1{%
  \zref@ifrefcontainsprop{#1}{anchor}{%
    \zref@ifrefcontainsprop{#1}{url}{%
    }{%
      \expandafter
      \ltx@LocalAppendToMacro\csname Z@R@#1\expandafter\endcsname
      \expandafter{%
        \csname url\ifZREF@xr@urluse use\fi
        \expandafter\endcsname\expandafter{\ZREF@xr@url}%
      }%
    }%
  }{%
  }%
}%
%    \end{macrocode}
%    \end{macro}
%
%    \begin{macrocode}
%</xr>
%    \end{macrocode}
%
%
% \subsection{Module \xmodule{hyperref}}
%
%    UNFINISHED :-(
%
%    \begin{macrocode}
%<*hyperref>
\NeedsTeXFormat{LaTeX2e}
\ProvidesPackage{zref-hyperref}%
  [2020-07-03 v2.32 Module hyperref for zref (HO)]%
\RequirePackage{zref-base}[2019/11/29]
\ifx\ZREF@base@ok Y%
\else
  \expandafter\endinput
\fi
%    \end{macrocode}
%
%    \begin{macrocode}
\ZREF@NewPropAnchor
\zref@addprop\ZREF@mainlist{anchor}%
%    \end{macrocode}
%
%    \begin{macrocode}
%</hyperref>
%    \end{macrocode}
%
%
% \subsection{Module \xmodule{savepos}}
%
%    Module \xmodule{savepos} provides an interface for
%    \pdfTeX's \cmd{\pdfsavepos}, see the manual for
%    \pdfTeX.
%
% \subsubsection{Identification}
%
%    \begin{macrocode}
%<*savepos>
\NeedsTeXFormat{LaTeX2e}
\ProvidesPackage{zref-savepos}%
  [2020-07-03 v2.32 Module savepos for zref (HO)]%
\RequirePackage{zref-base}[2019/11/29]
\ifx\ZREF@base@ok Y%
\else
  \expandafter\endinput
\fi
%    \end{macrocode}
%
% \subsubsection{Availability}
%
%    First we check, whether the feature is available.
%    \begin{macrocode}
\ifx\directlua\@undefined
\ltx@IfUndefined{pdfsavepos}{%
  \PackageError\ZREF@name{%
    \string\pdfsavepos\space is not supported.\MessageBreak
    It is provided by pdfTeX (1.40) or XeTeX%
  }\ZREF@UpdatePdfTeX
  \endinput
}{}%
\fi
%    \end{macrocode}
%    In PDF mode we are done. However support for DVI mode
%    was added later in version 1.40.0. In earlier versions
%    \cs{pdfsavepos} is defined, but its execution raises
%    an error. Note that \hologo{XeTeX} also provides \cs{pdfsavepos}.
%    \begin{macrocode}
\ifpdf
\else
  \ltx@IfUndefined{pdftexversion}{%
  }{%
    \ifnum\pdftexversion<140 %
      \PackageError\ZREF@name{%
        \string\pdfsavepos\space is not supported in DVI mode%
        \MessageBreak
        of this pdfTeX version%
      }\ZREF@UpdatePdfTeX
      \expandafter\expandafter\expandafter\endinput
    \fi
  }%
\fi
%    \end{macrocode}
%
% \subsubsection{Setup}
%
%    \begin{macrocode}
\zref@newlist{savepos}
\ifx\directlua\@undefined
  \zref@newprop*{posx}[0]{\the\pdflastxpos}
  \zref@newprop*{posy}[0]{\the\pdflastypos}
\else
  \zref@newprop*{posx}[0]{\the\lastxpos}
  \zref@newprop*{posy}[0]{\the\lastypos}
\fi
\zref@addprops{savepos}{posx,posy}
%    \end{macrocode}
%
% \subsubsection{User macros}
%
%    \begin{macro}{\zref@savepos}
%    \begin{macrocode}
\ifx\directlua\@undefined
  \def\zref@savepos{%
    \if@filesw
      \pdfsavepos
    \fi
  }
\else
  \def\zref@savepos{%
    \if@filesw
      \savepos
    \fi
  }
\fi
%    \end{macrocode}
%    \end{macro}
%
%    \begin{macro}{\ZREF@zsavepos}
%    \begin{macrocode}
\def\ZREF@zsavepos#1#2#3{%
  \@bsphack
  \if@filesw
    \zref@savepos
    #1{#3}{#2}%
    \ltx@IfUndefined{TeXXeTstate}{%
    }{%
      \ifnum\TeXXeTstate=\ltx@zero
      \else
        \zref@savepos
      \fi
    }%
  \fi
  \@esphack
}
%    \end{macrocode}
%    \end{macro}
%
%    \begin{macro}{\zsavepos}
%    The current location is stored in a reference with the
%    given name.
%    \begin{macrocode}
\ZREF@IfDefinable\zsavepos\def{%
  {%
    \ZREF@zsavepos\zref@labelbylist{savepos}%
  }%
}
%    \end{macrocode}
%    \end{macro}
%
%    \begin{macro}{\zsaveposx}
%    \begin{macrocode}
\ZREF@IfDefinable\zsaveposx\def{%
  {%
    \ZREF@zsavepos\zref@labelbyprops{posx}%
  }%
}
%    \end{macrocode}
%    \end{macro}
%    \begin{macro}{\zsaveposy}
%    \begin{macrocode}
\ZREF@IfDefinable\zsaveposy\def{%
  {%
    \ZREF@zsavepos\zref@labelbyprops{posy}%
  }%
}
%    \end{macrocode}
%    \end{macro}
%
%    \begin{macro}{\zposx}
%    \begin{macro}{\zposy}
%    The horizontal and vertical position are available by
%    \cs{zposx} and \cs{zposy}. Do not rely on absolute
%    positions. They differ in DVI and PDF mode of \pdfTeX.
%    Use differences instead. The unit of the position numbers
%    is sp.
%    \begin{macrocode}
\newcommand*{\zposx}[1]{%
  \zref@extract{#1}{posx}%
}%
\newcommand*{\zposy}[1]{%
  \zref@extract{#1}{posy}%
}%
%    \end{macrocode}
%    \end{macro}
%    \end{macro}
%
%    Typically horizontal and vertical positions are used
%    inside calculations. Therefore the extracting macros
%    should be expandable and babel's patch is not applyable.
%
%    Also it is in the responsibility of the user to marked
%    used positions by \cs{zrefused}
%    in order to notify \hologo{LaTeX} about undefined references.
%
%    \begin{macro}{\ZREF@savepos@ok}
%    \begin{macrocode}
\let\ZREF@savepos@ok=Y
%    \end{macrocode}
%    \end{macro}
%
%    \begin{macrocode}
%</savepos>
%    \end{macrocode}
%
%
% \subsection{Module \xmodule{abspos}}
%
% \subsubsection{Identification}
%
%    \begin{macrocode}
%<*abspos>
\NeedsTeXFormat{LaTeX2e}
\ProvidesPackage{zref-abspos}%
  [2020-07-03 v2.32 Module abspos for zref (HO)]%
\RequirePackage{zref-base}[2019/11/29]
\ifx\ZREF@base@ok Y%
\else
  \expandafter\endinput
\fi
%    \end{macrocode}
%
%    \begin{macrocode}
\RequirePackage{zref-savepos}[2019/11/29]
\ifx\ZREF@savepos@ok Y%
\else
  \expandafter\endinput
\fi
%    \end{macrocode}
%    \begin{macrocode}
\RequirePackage{zref-pagelayout}[2019/11/29]
\zref@addprop{savepos}{abspage}
\zref@addprop{savepos}{pagevalue}
%    \end{macrocode}
%
%    \begin{macro}{\zref@absposx}
%    \begin{macrocode}
\newcommand*{\zref@absposx}[3]{%
  \number
    \expandafter\zref@absposnumx\expandafter{%
      \number\zref@extractdefault{#1}{abspage}{0}%
    }{#2}{#3}%
  \ltx@space
}
%    \end{macrocode}
%    \end{macro}
%
%    \begin{macro}{\zref@absposy}
%    \begin{macrocode}
\newcommand*{\zref@absposy}[3]{%
  \number
  \expandafter\zref@absposnumy\expandafter{%
    \number\zref@extractdefault{#1}{abspage}{0}%
  }{#2}{#3}%
  \ltx@space
}
%    \end{macrocode}
%    \end{macro}
%
%    \begin{macro}{\zref@absposnumx}
%    \begin{macrocode}
\newcommand*{\zref@absposnumx}[3]{%
  \number
%  \ifnum#1>\ltx@zero
%    \zref@ifrefundefined{thepage#1}{%
%      0%
%    }{%
      \numexpr\ZREF@absposnum{thepage#1}{#2}{x}{#3}\relax
%    }%
%  \else
%    0%
%  \fi
}
%    \end{macrocode}
%    \end{macro}
%
%    \begin{macro}{\zref@absposnumy}
%    \begin{macrocode}
\newcommand*{\zref@absposnumy}[3]{%
  \number
%  \ifnum#1>\ltx@zero
%    \zref@ifrefundefined{thepage#1}{%
%      0%
%    }{%
      \numexpr\ZREF@absposnum{thepage#1}{#2}{y}{#3}\relax
%    }%
%  \else
%    0%
%  \fi
}
%    \end{macrocode}
%    \end{macro}
%
%    \begin{macro}{\ZREF@absposnum}
%    \begin{macrocode}
\def\ZREF@absposnum#1#2#3#4{%
  \ltx@ifundefined{ZREF@abspos@#2@#3@#4}{%
    0%
  }{%
    \csname ZREF@abspos@#2@#3@#4\endcsname{#1}%
  }%
}
%    \end{macrocode}
%    \end{macro}
%
%    \begin{macro}{\zref@def@absposx}
%    \begin{macrocode}
\ZREF@Robust\def\zref@def@absposx#1{%
  \zref@wrapper@babel{\ZREF@def@abspos{#1}\zref@absposx}%
}
%    \end{macrocode}
%    \end{macro}
%
%    \begin{macro}{\zref@def@absposy}
%    \begin{macrocode}
\ZREF@Robust\def\zref@def@absposy#1{%
  \zref@wrapper@babel{\ZREF@def@abspos{#1}\zref@absposy}%
}
%    \end{macrocode}
%    \end{macro}
%
%    \begin{macro}{\zref@def@absposnumx}
%    \begin{macrocode}
\ZREF@Robust\def\zref@def@absposnumx#1{%
  \zref@wrapper@babel{\ZREF@def@abspos{#1}\zref@absposnumx}%
}
%    \end{macrocode}
%    \end{macro}
%
%    \begin{macro}{\zref@def@absposnumy}
%    \begin{macrocode}
\ZREF@Robust\def\zref@def@absposnumy#1{%
  \zref@wrapper@babel{\ZREF@def@abspos{#1}\zref@absposnumy}%
}
%    \end{macrocode}
%    \end{macro}
%    \begin{macro}{\ZREF@def@abspos}
%    \begin{macrocode}
\def\ZREF@def@absposnumy#1#2#3#4#5{%
  \edef#1{#2{#3}{#4}{#5}}%
}
%    \end{macrocode}
%    \end{macro}
%
%    \begin{macro}{\zref@absposused}
%    \begin{macrocode}
\ZREF@Robust\def\zref@absposused{%
  \zref@wrapper@babel\ZREF@abspos@used
}
%    \end{macrocode}
%    \end{macro}
%    \begin{macro}{\ZREF@abspos@used}
%    \begin{macrocode}
\def\ZREF@abspos@used#1{%
  \zref@refused{#1}%
  \zref@ifrefundefined{#1}{%
  }{%
    \begingroup
      \edef\ZREF@temp{%
        \zref@extractdefault{#1}{abspage}{0}%
      }%
      \ifnum\ZREF@temp>\ltx@zero
        \zref@refused{thepage\ZREF@temp}%
      \else
        \@PackageError{zref-abspos}{%
          \string\zref@pos@label@used\ltx@space
          needs property `abspage'\MessageBreak
          in label `#1'%
        }\@ehc
      \fi
    \endgroup
  }%
}
%    \end{macrocode}
%    \end{macro}
%
%    \begin{macro}{\zref@absposnumused}
%    \begin{macrocode}
\newcommand*{\zref@absposnumused}[1]{%
  \ifnum#1>\ltx@zero
    \zref@refused{thepage\number#1}%
  \else
    \@PackageError{zref-abspos}{%
      Invalid absolute page number (#1)\MessageBreak
      for \string\zref@pos@num@used.\MessageBreak
      A positive integer is expected%
    }\@ehc
  \fi
}
%    \end{macrocode}
%    \end{macro}
%
%    \begin{macro}{\zref@ifabsposundefined}
%    \begin{macrocode}
\def\zref@ifabsposundefined#1{%
  \zref@ifrefundefined{#1}\ltx@firsttwo{%
    \expandafter\zref@ifabsposnumundefined\expandafter{%
      \number\zref@extractdefault{#1}{abspage}{0}%
    }%
  }%
}
%    \end{macrocode}
%    \end{macro}
%
%    \begin{macro}{\zref@ifabsposnumundefined}
%    \begin{macrocode}
\def\zref@ifabsposnumundefined#1{%
  \ifnum\ZREF@number{#1}>\ltx@zero
    \zref@ifrefundefined{thepage#1}%
        \ltx@firstoftwo\ltx@secondoftwo
  \else
    \expandafter\ltx@firstoftwo
  \fi
}
%    \end{macrocode}
%    \end{macro}
%
% \subsubsection{Media}
%
%    \begin{macro}{\ZREF@abspos@media@width}
%    \begin{macrocode}
\edef\ZREF@abspos@media@width{%
  \ltx@ifundefined{pdfpagewidth}{%
    \ltx@ifundefined{mediawidth}{%
      \ltx@ifundefined{stockwidth}{%
        paperwidth%
      }{%
        stockwidth%
      }%
    }{%
      mediawidth%
    }%
  }{%
    pdfpagewidth%
  }%
}
\ifluatex
\def\ZREF@abspos@media@width{pdfpagewidth}%
\fi
%    \end{macrocode}
%    \end{macro}
%    \begin{macro}{\ZREF@abspos@media@height}
%    \begin{macrocode}
\edef\ZREF@abspos@media@height{%
  \ltx@ifundefined{pdfpageheight}{%
    \ltx@ifundefined{mediaheight}{%
      \ltx@ifundefined{stockheight}{%
        paperheight%
      }{%
        stockheight%
      }%
    }{%
      mediaheight%
    }%
  }{%
    \noexpand\ifcase\pdfpageheight
      \ltx@ifundefined{stockheight}{%
        paperheight%
      }{%
        stockheight%
      }%
    \noexpand\else
      pdfpageheight%
    \noexpand\fi
  }%
}
\ifluatex
\edef\ZREF@abspos@media@height{%
   \noexpand\ifcase\pageheight
      \ltx@ifundefined{stockheight}{%
        paperheight%
      }{%
        stockheight%
      }%
    \noexpand\else
      pdfpageheight%
    \noexpand\fi}%
\fi
%    \end{macrocode}
%    \end{macro}
%
%    \begin{macro}{\ZREF@abspos@media@x@left}
%    \begin{macrocode}
\def\ZREF@abspos@media@x@left#1{%
  0%
}
%    \end{macrocode}
%    \end{macro}
%    \begin{macro}{\ZREF@abspos@media@x@right}
%    \begin{macrocode}
\def\ZREF@abspos@media@x@right#1{%
  \zref@extract{#1}\ZREF@abspos@media@width
}
%    \end{macrocode}
%    \end{macro}
%    \begin{macro}{\ZREF@abspos@media@x@center}
%    \begin{macrocode}
\def\ZREF@abspos@media@x@center#1{%
  \ZREF@abspos@media@x@left{#1}%
  +\zref@extract{#1}\ZREF@abspos@media@width/2%
}
%    \end{macrocode}
%    \end{macro}
%
%    \begin{macro}{\ZREF@abspos@media@y@top}
%    \begin{macrocode}
\def\ZREF@abspos@media@y@top#1{%
  \zref@extract{#1}\ZREF@abspos@media@height
}
%    \end{macrocode}
%    \end{macro}
%    \begin{macro}{\ZREF@abspos@media@y@bottom}
%    \begin{macrocode}
\def\ZREF@abspos@media@y@bottom#1{%
  0%
}
%    \end{macrocode}
%    \end{macro}
%    \begin{macro}{\ZREF@abspos@media@y@center}
%    \begin{macrocode}
\def\ZREF@abspos@media@y@center#1{%
  \zref@extract{#1}\ZREF@abspos@media@height/2%
}
%    \end{macrocode}
%    \end{macro}
%
% \subsubsection{Paper}
%
%    \begin{macro}{\ZREF@abspos@paper@x@left}
%    \begin{macrocode}
\def\ZREF@abspos@paper@x@left#1{%
  0%
}
%    \end{macrocode}
%    \end{macro}
%    \begin{macro}{\ZREF@abspos@paper@x@right}
%    \begin{macrocode}
\def\ZREF@abspos@paper@x@right#1{%
  \zref@extract{#1}{paperwidth}%
}
%    \end{macrocode}
%    \end{macro}
%    \begin{macro}{\ZREF@abspos@paper@x@center}
%    \begin{macrocode}
\def\ZREF@abspos@paper@x@center#1{%
  \zref@extract{#1}{paperwidth}/2%
}
%    \end{macrocode}
%    \end{macro}
%
%    \begin{macro}{\ZREF@abspos@paper@y@top}
%    \begin{macrocode}
\let\ZREF@abspos@paper@y@top\ZREF@abspos@media@y@top
%    \end{macrocode}
%    \end{macro}
%    \begin{macro}{\ZREF@abspos@paper@y@bottom}
%    \begin{macrocode}
\def\ZREF@abspos@paper@y@bottom#1{%
  \ZREF@abspos@paper@y@top{#1}%
  -\zref@extract{#1}{paperheight}%
}
%    \end{macrocode}
%    \end{macro}
%    \begin{macro}{\ZREF@abspos@paper@y@center}
%    \begin{macrocode}
\def\ZREF@abspos@paper@y@center#1{%
  \ZREF@abspos@paper@y@top{#1}%
  -\zref@extract{#1}{paperheight}/2%
}
%    \end{macrocode}
%    \end{macro}
%
% \subsubsection{Origin}
%
% There doesn't seem a good reason to make these tests depend on pdf mode
%  in current engines, so comment out the \verb|\ifpdf| tests.
%    \begin{macro}{\ZREF@abspos@origin@x}
%    \begin{macrocode}
\let\ZREF@temp\ltx@two
\ltx@IfUndefined{pdfhorigin}{}{%
%  \ifpdf
    \let\ZREF@temp\ltx@zero
%  \fi
}
\ifluatex
% \ifpdf
  \let\ZREF@temp\ltx@zero
% \fi
\fi

\ifx\ZREF@temp\ltx@two
  \ifnum\mag=1000 %
    \let\ZREF@temp\ltx@one
  \fi
\fi
\ifcase\ZREF@temp
  \def\ZREF@abspos@origin@x#1{%
    \zref@extract{#1}{pdfhorigin}%
  }%
\or
  \def\ZREF@abspos@origin@x#1{%
    4736286%
  }%
\or
  \def\ZREF@abspos@origin@x#1{%
    \numexpr\mag/1000*\dimexpr 1truein\relax\relax
  }%
\fi
%    \end{macrocode}
%    \end{macro}
%    \begin{macro}{\ZREF@abspos@origin@y}
%    \begin{macrocode}
\let\ZREF@temp\ltx@two
\ltx@IfUndefined{pdfvorigin}{}{%
%  \ifpdf
    \let\ZREF@temp\ltx@zero
%  \fi
}
\ifluatex
%  \ifpdf
    \let\ZREF@temp\ltx@zero
%  \fi
\fi
\ifx\ZREF@temp\ltx@two
  \ifnum\mag=1000 %
    \let\ZREF@temp\ltx@one
  \fi
\fi
\ifcase\ZREF@temp
  \def\ZREF@abspos@origin@y#1{%
    \zref@extract{#1}{pdfvorigin}%
  }%
\or
  \def\ZREF@abspos@origin@y#1{%
    4736286%
  }%
\or
  \def\ZREF@abspos@origin@y#1{%
    \numexpr\mag/1000*\dimexpr 1truein\relax\relax
   }%
\fi
%    \end{macrocode}
%    \end{macro}
%
% \subsubsection{Header}
%
%    \begin{macro}{\ZREF@abspos@head@x@left}
%    \begin{macrocode}
\def\ZREF@abspos@head@x@left#1{%
  \ZREF@abspos@paper@x@left{#1}%
  +\ZREF@abspos@origin@x{#1}%
  +\zref@extract{#1}{hoffset}%
  +\ifodd\zref@extractdefault{#1}{pagevalue}{\number\c@page} %
     \zref@extract{#1}{oddsidemargin}%
   \else
     \zref@extract{#1}{evensidemargin}%
   \fi
}
%    \end{macrocode}
%    \end{macro}
%    \begin{macro}{\ZREF@abspos@head@x@right}
%    \begin{macrocode}
\def\ZREF@abspos@head@x@right#1{%
  \ZREF@abspos@head@x@left{#1}%
  +\zref@extract{#1}{textwidth}%
}
%    \end{macrocode}
%    \end{macro}
%    \begin{macro}{\ZREF@abspos@head@x@center}
%    \begin{macrocode}
\def\ZREF@abspos@head@x@center#1{%
  \ZREF@abspos@head@x@left{#1}%
  +\zref@extract{#1}{textwidth}/2%
}
%    \end{macrocode}
%    \end{macro}
%
%    \begin{macro}{\ZREF@abspos@head@y@top}
%    \begin{macrocode}
\def\ZREF@abspos@head@y@top#1{%
  \ZREF@abspos@paper@y@top{#1}%
  -\ZREF@abspos@origin@y{#1}%
  -\zref@extract{#1}{voffset}%
  -\zref@extract{#1}{topmargin}%
}
%    \end{macrocode}
%    \end{macro}
%    \begin{macro}{\ZREF@abspos@head@y@bottom}
%    \begin{macrocode}
\def\ZREF@abspos@head@y@bottom#1{%
  \ZREF@abspos@head@y@top{#1}%
  -\zref@extract{#1}{headheight}%
}
%    \end{macrocode}
%    \end{macro}
%    \begin{macro}{\ZREF@abspos@head@y@center}
%    \begin{macrocode}
\def\ZREF@abspos@head@y@center#1{%
  \ZREF@abspos@head@y@top{#1}%
  -\zref@extract{#1}{headheight}/2%
}
%    \end{macrocode}
%    \end{macro}
%
% \subsubsection{Body}
%
%    \begin{macro}{\ZREF@abspos@body@x@left}
%    \begin{macrocode}
\let\ZREF@abspos@body@x@left\ZREF@abspos@head@x@left
%    \end{macrocode}
%    \end{macro}
%    \begin{macro}{\ZREF@abspos@body@x@right}
%    \begin{macrocode}
\let\ZREF@abspos@body@x@right\ZREF@abspos@head@x@right
%    \end{macrocode}
%    \end{macro}
%    \begin{macro}{\ZREF@abspos@body@x@center}
%    \begin{macrocode}
\let\ZREF@abspos@body@x@center\ZREF@abspos@head@x@center
%    \end{macrocode}
%    \end{macro}
%
%    \begin{macro}{\ZREF@abspos@body@y@top}
%    \begin{macrocode}
\def\ZREF@abspos@body@y@top#1{%
  \ZREF@abspos@head@y@bottom{#1}%
  -\zref@extract{#1}{headsep}%
}
%    \end{macrocode}
%    \end{macro}
%    \begin{macro}{\ZREF@abspos@body@y@bottom}
%    \begin{macrocode}
\def\ZREF@abspos@body@y@bottom#1{%
  \ZREF@abspos@body@y@top{#1}%
  -\zref@extract{#1}{textheight}%
}
%    \end{macrocode}
%    \end{macro}
%    \begin{macro}{\ZREF@abspos@body@y@center}
%    \begin{macrocode}
\def\ZREF@abspos@body@y@center#1{%
  \ZREF@abspos@body@y@top{#1}%
  -\zref@extract{#1}{textheight}/2%
}
%    \end{macrocode}
%    \end{macro}
%
% \subsubsection{Footer}
%
%    \begin{macro}{\ZREF@abspos@foot@x@left}
%    \begin{macrocode}
\let\ZREF@abspos@foot@x@left\ZREF@abspos@head@x@left
%    \end{macrocode}
%    \end{macro}
%    \begin{macro}{\ZREF@abspos@foot@x@right}
%    \begin{macrocode}
\let\ZREF@abspos@foot@x@right\ZREF@abspos@head@x@right
%    \end{macrocode}
%    \end{macro}
%    \begin{macro}{\ZREF@abspos@foot@x@center}
%    \begin{macrocode}
\let\ZREF@abspos@foot@x@center\ZREF@abspos@head@x@center
%    \end{macrocode}
%    \end{macro}
%
%    \begin{macro}{\ZREF@abspos@foot@y@bottom}
%    \begin{macrocode}
\def\ZREF@abspos@foot@y@bottom#1{%
  \ZREF@abspos@body@y@bottom{#1}%
  -\zref@extract{#1}{footskip}%
}
%    \end{macrocode}
%    \end{macro}
%
% \subsubsection{Marginal notes}
%
%    \begin{macro}{\ZREF@abspos@marginpar@x@left}
%    \begin{macrocode}
\def\ZREF@abspos@marginpar@x@left#1{%
  \ifodd\zref@extractdefault{#1}{pagevalue}{\number\c@page} %
    \ZREF@abspos@body@x@right{#1}%
    +\zref@extract{#1}{marginparsep}%
  \else
    \ZREF@abspos@body@x@left{#1}%
    -\zref@extract{#1}{marginparsep}%
    -\zref@extract{#1}{marginparwidth}%
  \fi
}
%    \end{macrocode}
%    \end{macro}
%    \begin{macro}{\ZREF@abspos@marginpar@x@right}
%    \begin{macrocode}
\def\ZREF@abspos@marginpar@x@right#1{%
  \ZREF@abspos@marginpar@x@left{#1}%
  +\zref@extract{#1}{marginparwidth}%
}
%    \end{macrocode}
%    \end{macro}
%    \begin{macro}{\ZREF@abspos@marginpar@x@center}
%    \begin{macrocode}
\def\ZREF@abspos@marginpar@x@center#1{%
  \ZREF@abspos@marginpar@x@left{#1}%
  +\zref@extract{#1}{marginparwidth}/2%
}
%    \end{macrocode}
%    \end{macro}
%
%    \begin{macro}{\ZREF@abspos@marginpar@y@top}
%    \begin{macrocode}
\let\ZREF@abspos@marginpar@y@top\ZREF@abspos@body@y@top
%    \end{macrocode}
%    \end{macro}
%    \begin{macro}{\ZREF@abspos@marginpar@y@bottom}
%    \begin{macrocode}
\let\ZREF@abspos@marginpar@y@bottom\ZREF@abspos@body@y@bottom
%    \end{macrocode}
%    \end{macro}
%    \begin{macro}{\ZREF@abspos@marginpar@y@center}
%    \begin{macrocode}
\let\ZREF@abspos@marginpar@y@center\ZREF@abspos@body@y@center
%    \end{macrocode}
%    \end{macro}
%
% \subsubsection{Stock paper}
%
%    \begin{macro}{\ZREF@abspos@stock@x@left}
%    \begin{macrocode}
\let\ZREF@abspos@stock@x@left\ZREF@abspos@paper@x@left
%    \end{macrocode}
%    \end{macro}
%    \begin{macro}{\ZREF@abspos@stock@x@right}
%    \begin{macrocode}
\let\ZREF@abspos@stock@x@right\ZREF@abspos@paper@x@right
%    \end{macrocode}
%    \end{macro}
%    \begin{macro}{\ZREF@abspos@stock@x@center}
%    \begin{macrocode}
\let\ZREF@abspos@stock@x@center\ZREF@abspos@paper@x@center
%    \end{macrocode}
%    \end{macro}
%
%    \begin{macro}{\ZREF@abspos@stock@y@top}
%    \begin{macrocode}
\let\ZREF@abspos@stock@y@top\ZREF@abspos@paper@y@top
%    \end{macrocode}
%    \end{macro}
%    \begin{macro}{\ZREF@abspos@stock@y@bottom}
%    \begin{macrocode}
\let\ZREF@abspos@stock@y@bottom\ZREF@abspos@paper@y@bottom
%    \end{macrocode}
%    \end{macro}
%    \begin{macro}{\ZREF@abspos@stock@y@center}
%    \begin{macrocode}
\let\ZREF@abspos@stock@y@center\ZREF@abspos@paper@y@center
%    \end{macrocode}
%    \end{macro}
%
%    \begin{macrocode}
%</abspos>
%    \end{macrocode}
%
%
% \subsection{Module \xmodule{dotfill}}
%
%    \begin{macrocode}
%<*dotfill>
\NeedsTeXFormat{LaTeX2e}
\ProvidesPackage{zref-dotfill}%
  [2020-07-03 v2.32 Module dotfill for zref (HO)]%
\RequirePackage{zref-base}[2019/11/29]
\ifx\ZREF@base@ok Y%
\else
  \expandafter\endinput
\fi
%    \end{macrocode}
%
%    For measuring the width of \cs{zdotfill} we use the features
%    provided by module \xmodule{savepos}.
%    \begin{macrocode}
\RequirePackage{zref-savepos}[2019/11/29]
%    \end{macrocode}
%    For automatically generated label names we use the unique
%    counter of module \xmodule{base}.
%    \begin{macrocode}
\zref@require@unique
%    \end{macrocode}
%    Configuration is done by the key value interface of package
%    \xpackage{keyval}.
%    \begin{macrocode}
\RequirePackage{keyval}
%    \end{macrocode}
%    The definitions of the keys follow.
%    \begin{macrocode}
\define@key{ZREF@DF}{unit}{%
  \def\ZREF@df@unit{#1}%
}
\define@key{ZREF@DF}{min}{%
  \def\ZREF@df@min{#1}%
}
\define@key{ZREF@DF}{dot}{%
  \def\ZREF@df@dot{#1}%
}
%    \end{macrocode}
%    Defaults are set, see user interface.
%    \begin{macrocode}
\providecommand\ZREF@df@min{2}
\providecommand\ZREF@df@unit{.44em}
\providecommand\ZREF@df@dot{.}
%    \end{macrocode}
%    \begin{macro}{\zdotfillsetup}
%    Configuration of \cs{zdotfill} is done by \cs{zdotfillsetup}.
%    \begin{macrocode}
\newcommand*{\zdotfillsetup}{\kvsetkeys{ZREF@DF}}
%    \end{macrocode}
%    \end{macro}
%
%    \begin{macro}{\zdotfill}
%    \cs{zdotfill} sets labels at the left and the right to get
%    the horizontal position. \cs{zsavepos} is not used, because
%    we do not need the vertical position.
%    \begin{macrocode}
\ZREF@IfDefinable\zdotfill\def{%
  {%
    \leavevmode
    \global\advance\c@zref@unique\ltx@one
    \begingroup
      \def\ZREF@temp{zref@\number\c@zref@unique}%
      \pdfsavepos
      \zref@labelbyprops{\thezref@unique L}{posx}%
      \setlength{\dimen@}{\ZREF@df@unit}%
      \zref@ifrefundefined{\thezref@unique R}{%
        \ZREF@dotfill
      }{%
        \ifnum\numexpr\zposx{\thezref@unique R}%
                      -\zposx{\thezref@unique L}\relax
            <\dimexpr\ZREF@df@min\dimen@\relax
          \hfill
        \else
          \ZREF@dotfill
        \fi
      }%
      \pdfsavepos
      \zref@labelbyprops{\thezref@unique R}{posx}%
    \endgroup
    \kern\z@
  }%
}
%    \end{macrocode}
%    \end{macro}
%    \begin{macro}{\ZREF@dotfill}
%    Help macro that actually sets the dots.
%    \begin{macrocode}
\def\ZREF@dotfill{%
  \cleaders\hb@xt@\dimen@{\hss\ZREF@df@dot\hss}\hfill
}
%    \end{macrocode}
%    \end{macro}
%
%    \begin{macrocode}
%</dotfill>
%    \end{macrocode}
%
% \subsection{Module \xmodule{env}}
%
%    \begin{macrocode}
%<*env>
\NeedsTeXFormat{LaTeX2e}
\ProvidesPackage{zref-env}%
  [2020-07-03 v2.32 Module env for zref (HO)]%
\RequirePackage{zref-base}[2019/11/29]
\ifx\ZREF@base@ok Y%
\else
  \expandafter\endinput
\fi
%    \end{macrocode}
%
%    \begin{macrocode}
\zref@newprop{envname}[]{\@currenvir}
\zref@newprop{envline}[]{\zref@env@line}
%    \end{macrocode}
%
%    \begin{macro}{\zref@env@line}
%    Macro \cs{zref@env@line} extracts the line number
%    from \cs{@currenvline}.
%    \begin{macrocode}
\def\zref@env@line{%
  \ifx\@currenvline\ltx@empty
  \else
    \expandafter
    \ZREF@ENV@line\@currenvline\ltx@empty line \ltx@empty\@nil
  \fi
}
%    \end{macrocode}
%    \end{macro}
%    \begin{macro}{\ZREF@ENV@line}
%    \begin{macrocode}
\def\ZREF@ENV@line#1line #2\ltx@empty#3\@nil{#2}%
%    \end{macrocode}
%    \end{macro}
%
%    \begin{macrocode}
%</env>
%    \end{macrocode}
%
%% \section{Installation}
%
% \subsection{Download}
%
% \paragraph{Package.} This package is available on
% CTAN\footnote{\CTANpkg{zref}}:
% \begin{description}
% \item[\CTAN{macros/latex/contrib/zref/zref.dtx}] The source file.
% \item[\CTAN{macros/latex/contrib/zref/zref.pdf}] Documentation.
% \end{description}
%
%
% \paragraph{Bundle.} All the packages of the bundle `zref'
% are also available in a TDS compliant ZIP archive. There
% the packages are already unpacked and the documentation files
% are generated. The files and directories obey the TDS standard.
% \begin{description}
% \item[\CTANinstall{install/macros/latex/contrib/zref.tds.zip}]
% \end{description}
% \emph{TDS} refers to the standard ``A Directory Structure
% for \TeX\ Files'' (\CTANpkg{tds}). Directories
% with \xfile{texmf} in their name are usually organized this way.
%
% \subsection{Bundle installation}
%
% \paragraph{Unpacking.} Unpack the \xfile{zref.tds.zip} in the
% TDS tree (also known as \xfile{texmf} tree) of your choice.
% Example (linux):
% \begin{quote}
%   |unzip zref.tds.zip -d ~/texmf|
% \end{quote}
%
%
% \subsection{Package installation}
%
% \paragraph{Unpacking.} The \xfile{.dtx} file is a self-extracting
% \docstrip\ archive. The files are extracted by running the
% \xfile{.dtx} through \plainTeX:
% \begin{quote}
%   \verb|tex zref.dtx|
% \end{quote}
%
% \paragraph{TDS.} Now the different files must be moved into
% the different directories in your installation TDS tree
% (also known as \xfile{texmf} tree):
% \begin{quote}
% \def\t{^^A
% \begin{tabular}{@{}>{\ttfamily}l@{ $\rightarrow$ }>{\ttfamily}l@{}}
%   zref.sty & tex/latex/zref/zref.sty\\
%   zref-base.sty & tex/latex/zref/zref-base.sty\\
%   zref-abspage.sty & tex/latex/zref/zref-abspage.sty\\
%   zref-abspos.sty & tex/latex/zref/zref-abspos.sty\\
%   zref-counter.sty & tex/latex/zref/zref-counter.sty\\
%   zref-dotfill.sty & tex/latex/zref/zref-dotfill.sty\\
%   zref-env.sty & tex/latex/zref/zref-env.sty\\
%   zref-hyperref.sty & tex/latex/zref/zref-hyperref.sty\\
%   zref-lastpage.sty & tex/latex/zref/zref-lastpage.sty\\
%   zref-marks.sty & tex/latex/zref/zref-marks.sty\\
%   zref-nextpage.sty & tex/latex/zref/zref-nextpage.sty\\
%   zref-pageattr.sty & tex/latex/zref/zref-pageattr.sty\\
%   zref-pagelayout.sty & tex/latex/zref/zref-pagelayout.sty\\
%   zref-perpage.sty & tex/latex/zref/zref-perpage.sty\\
%   zref-runs.sty & tex/latex/zref/zref-runs.sty\\
%   zref-savepos.sty & tex/latex/zref/zref-savepos.sty\\
%   zref-thepage.sty & tex/latex/zref/zref-thepage.sty\\
%   zref-titleref.sty & tex/latex/zref/zref-titleref.sty\\
%   zref-totpages.sty & tex/latex/zref/zref-totpages.sty\\
%   zref-user.sty & tex/latex/zref/zref-user.sty\\
%   zref-xr.sty & tex/latex/zref/zref-xr.sty\\
%   zref.pdf & doc/latex/zref/zref.pdf\\
%   zref-example.tex & doc/latex/zref/zref-example.tex\\
%   zref-example-lastpage.tex & doc/latex/zref/zref-example-lastpage.tex\\
%   zref-example-nextpage.tex & doc/latex/zref/zref-example-nextpage.tex\\
%   zref.dtx & source/latex/zref/zref.dtx\\
% \end{tabular}^^A
% }^^A
% \sbox0{\t}^^A
% \ifdim\wd0>\linewidth
%   \begingroup
%     \advance\linewidth by\leftmargin
%     \advance\linewidth by\rightmargin
%   \edef\x{\endgroup
%     \def\noexpand\lw{\the\linewidth}^^A
%   }\x
%   \def\lwbox{^^A
%     \leavevmode
%     \hbox to \linewidth{^^A
%       \kern-\leftmargin\relax
%       \hss
%       \usebox0
%       \hss
%       \kern-\rightmargin\relax
%     }^^A
%   }^^A
%   \ifdim\wd0>\lw
%     \sbox0{\small\t}^^A
%     \ifdim\wd0>\linewidth
%       \ifdim\wd0>\lw
%         \sbox0{\footnotesize\t}^^A
%         \ifdim\wd0>\linewidth
%           \ifdim\wd0>\lw
%             \sbox0{\scriptsize\t}^^A
%             \ifdim\wd0>\linewidth
%               \ifdim\wd0>\lw
%                 \sbox0{\tiny\t}^^A
%                 \ifdim\wd0>\linewidth
%                   \lwbox
%                 \else
%                   \usebox0
%                 \fi
%               \else
%                 \lwbox
%               \fi
%             \else
%               \usebox0
%             \fi
%           \else
%             \lwbox
%           \fi
%         \else
%           \usebox0
%         \fi
%       \else
%         \lwbox
%       \fi
%     \else
%       \usebox0
%     \fi
%   \else
%     \lwbox
%   \fi
% \else
%   \usebox0
% \fi
% \end{quote}
% If you have a \xfile{docstrip.cfg} that configures and enables \docstrip's
% TDS installing feature, then some files can already be in the right
% place, see the documentation of \docstrip.
%
% \subsection{Refresh file name databases}
%
% If your \TeX~distribution
% (\TeX\,Live, \mikTeX, \dots) relies on file name databases, you must refresh
% these. For example, \TeX\,Live\ users run \verb|texhash| or
% \verb|mktexlsr|.
%
% \subsection{Some details for the interested}
%
% \paragraph{Unpacking with \LaTeX.}
% The \xfile{.dtx} chooses its action depending on the format:
% \begin{description}
% \item[\plainTeX:] Run \docstrip\ and extract the files.
% \item[\LaTeX:] Generate the documentation.
% \end{description}
% If you insist on using \LaTeX\ for \docstrip\ (really,
% \docstrip\ does not need \LaTeX), then inform the autodetect routine
% about your intention:
% \begin{quote}
%   \verb|latex \let\install=y\input{zref.dtx}|
% \end{quote}
% Do not forget to quote the argument according to the demands
% of your shell.
%
% \paragraph{Generating the documentation.}
% You can use both the \xfile{.dtx} or the \xfile{.drv} to generate
% the documentation. The process can be configured by the
% configuration file \xfile{ltxdoc.cfg}. For instance, put this
% line into this file, if you want to have A4 as paper format:
% \begin{quote}
%   \verb|\PassOptionsToClass{a4paper}{article}|
% \end{quote}
% An example follows how to generate the
% documentation with pdf\LaTeX:
% \begin{quote}
%\begin{verbatim}
%pdflatex zref.dtx
%makeindex -s gind.ist zref.idx
%pdflatex zref.dtx
%makeindex -s gind.ist zref.idx
%pdflatex zref.dtx
%\end{verbatim}
% \end{quote}
%
% \begin{thebibliography}{99}
% \bibpackage{footmisc}{Robin Fairbairns}{2004/01/23 v5.3a}
%   {\CTANpkg{footmisc}}
% \bibpackage{hyperref}{Sebastian Rahtz, Heiko Oberdiek}{2006/08/16 v6.75c}
%   {\CTANpkg{hyperref}}
% \bibpackage{lastpage}{Jeff Goldberg}{1994/06/25 v0.1b}
%   {\CTANpkg{lastpage}}
% \bibpackage{nameref}{Sebastian Rahtz, Heiko Oberdiek}{2006/02/12 v2.24}
%   {\CTANpkg{nameref}}
% \bibpackage{perpage}{David Kastrup}{2002/12/20 v1.0}
%   {\CTANpkg{perpage}}
% \bibpackage{titleref}{Donald Arsenau}{2001/04/05 v3.1}
%   {\CTANpkg{titleref}}
% \bibpackage{totpages}{Wilhelm M\"uller}{1999/07/14 v1.00}
%   {\CTANpkg{totpages}}
% \bibpackage{xr}{David Carlisle}{1994/05/28 v5.02}
%   {\CTANpkg{xr}}
% \bibpackage{xr-hyper}{David Carlisle}{2000/03/22 v6.00beta4}
%   {\CTANpkg{xr-hyper}}
% \end{thebibliography}
%
% \begin{History}
%   \begin{Version}{2006/02/20 v1.0}
%   \item
%     First version.
%   \end{Version}
%   \begin{Version}{2006/05/03 v1.1}
%   \item
%     Module \xmodule{perpage} added.
%   \item
%     Module redesign as packages.
%   \end{Version}
%   \begin{Version}{2006/05/25 v1.2}
%   \item
%     Module \xmodule{dotfillmin} added.
%   \item
%     Module \xmodule{base}: macros \cs{zref@require@uniqe}
%     and \cs{thezref@unique} added
%     (used by modules \xmodule{titleref} and \xmodule{dotfillmin}).
%   \end{Version}
%   \begin{Version}{2006/09/08 v1.3}
%   \item
%     Typo fixes and English cleanup by Per Starback.
%   \end{Version}
%   \begin{Version}{2007/01/23 v1.4}
%   \item
%     Typo in macro name fixed in documentation.
%   \end{Version}
%   \begin{Version}{2007/02/18 v1.5}
%   \item
%     \cs{zref@getcurrent} added (suggestion of Igor Akkerman).
%   \item
%     Module \xmodule{savepos} also supports \hologo{XeTeX}.
%   \end{Version}
%   \begin{Version}{2007/04/06 v1.6}
%   \item
%     Fix in modules \xmodule{abspage} and \xmodule{base}:
%     Now counter |abspage| and |zref@unique|
%     are not remembered by \cs{include}.
%   \item
%     Beamer support for module \xmodule{titleref}.
%   \end{Version}
%   \begin{Version}{2007/04/17 v1.7}
%   \item
%     Package \xpackage{atbegshi} replaces \xpackage{everyshi}.
%   \end{Version}
%   \begin{Version}{2007/04/22 v1.8}
%   \item
%     \cs{zref@wrapper@babel} and \cs{zref@refused} are now expandable
%     if babel is not used or \cs{if@safe@actives} is already set to true.
%     (Feature request of Josselin Noirel)
%   \end{Version}
%   \begin{Version}{2007/05/02 v1.9}
%   \item
%     Module \xmodule{titleref}: Some support for \cs{caption} of package
%     \xpackage{longtable}, but only if \cs{label} is given after
%     \cs{caption}.
%   \end{Version}
%   \begin{Version}{2007/05/06 v2.0}
%   \item
%     Uses package \xpackage{etexcmds} for accessing \hologo{eTeX}'s \cs{unexpanded}.
%   \end{Version}
%   \begin{Version}{2007/05/28 v2.1}
%   \item
%     Module \xmodule{titleref} supports caption of package \xpackage{listings}.
%   \item
%     Fixes in module \xmodule{titleref} for support
%     of packages \xpackage{titlesec} and \xpackage{longtable}.
%   \end{Version}
%   \begin{Version}{2008/09/21 v2.2}
%   \item
%     Module \xmodule{base}: \cs{zref@iflistcontainsprop} is documented, but
%     a broken \cs{zref@listcontainsprop} implemented. Name and
%     implementation fixed (thanks Ohad Kammar).
%   \end{Version}
%   \begin{Version}{2008/10/01 v2.3}
%   \item
%     \cs{zref@localaddprop} added (feature request of Ohad Kammar).
%   \item
%     Module \xmodule{lastpage}: list `LastPage' added. Label `LastPage' will use
%     the properties of this list (default is empty) along with the
%     properties of the main list.
%   \end{Version}
%   \begin{Version}{2009/08/07 v2.4}
%   \item
%     Module \xmodule{runs} added.
%   \end{Version}
%   \begin{Version}{2009/12/06 v2.5}
%   \item
%     Module \xmodule{lastpage}: Uses package \xpackage{atveryend}.
%   \item
%     Module \xmodule{titleref}: Further commands are disabled
%     during string expansion, imported from package \xpackage{nameref}.
%   \end{Version}
%   \begin{Version}{2009/12/07 v2.6}
%   \item
%     Version date added for package \xpackage{atveryend}.
%   \end{Version}
%   \begin{Version}{2009/12/08 v2.7}
%   \item
%     Module \xmodule{titleref}: Use of package \xpackage{gettitlestring}.
%   \end{Version}
%   \begin{Version}{2010/03/26 v2.8}
%   \item
%     \cs{zifrefundefined} added.
%   \item
%     Module \xmodule{lastpage}: Macros \cs{zref@iflastpage}
%     and \cs{ziflastpage} added.
%   \item
%     Module \xmodule{thepage} added.
%   \item
%     Module \xmodule{nextpage} added.
%   \end{Version}
%   \begin{Version}{2010/03/29 v2.9}
%   \item
%     Module \xmodule{marks} added (without documentation).
%   \item
%     \cs{zref@addprop} now adds expanded property to list.
%   \item
%     Useless \cs{ZREF@ErrorNoLine} removed.
%   \end{Version}
%   \begin{Version}{2010/04/08 v2.10}
%   \item
%     Module \xmodule{xr} remembers the external document name
%     in property `externaldocument'.
%   \end{Version}
%   \begin{Version}{2010/04/15 v2.11}
%   \item
%     Module \xmodule{titleref}: Better support of class \xclass{memoir}.
%   \item
%     Module \xmodule{titleref}: Support of theorems.
%   \end{Version}
%   \begin{Version}{2010/04/17 v2.12}
%   \item
%     Module \xmodule{base}: \cs{zref@newprop} ensures global
%     empty default.
%   \item
%     Module \xmodule{xr}: Setup options \xoption{tozreflabel}
%     and \xoption{toltxlabel} added.
%   \end{Version}
%   \begin{Version}{2010/04/19 v2.13}
%   \item
%     \cs{zref@setcurrent} throws an error if the property
%     does not exist (Florent Chervet).
%   \item
%     \cs{zref@getcurrent} the documentation is fixed (Florent Chervet).
%     Also it returns the empty string in case of errors.
%   \item
%     \cs{zref@addprop} and \cs{zref@localaddprop} now take a list of
%     property names (feature request of Florent Chervet).
%   \item
%     Example for \cs{zref@wrapper@unexpanded} corrected (Florent Chervet).
%   \end{Version}
%   \begin{Version}{2010/04/22 v2.14}
%   \item
%     Bug fix for \cs{zref@getcurrent} second argument wasn't eaten in
%     case of unknown property.
%   \item
%     \cs{zref@getcurrent} supports \cs{zref@wrapper@unexpanded}.
%   \item
%     \cs{zref@wrapper@unexpanded} added for \cs{ZREF@xr@tolabel}.
%   \item
%     \cs{zref@extract}, \cs{zref@extractdefault}, \cs{zref@getcurrent}
%     are expandable in exact two steps
%     except inside \cs{zref@wrapper@unexpanded}.
%   \end{Version}
%   \begin{Version}{2010/04/23 v2.15}
%   \item
%     \cs{zexternaldocument} fixed for property `url' when
%     importing \cs{new@label} (bug found by Victor Ivrii).
%   \item
%     Two expansion steps also in \cs{zref@wrapper@unexpanded}.
%   \item
%     Nested calls of \cs{zref@wrapper@unexpanded} possible.
%   \end{Version}
%   \begin{Version}{2010/04/28 v2.16}
%   \item
%     More consequent use of package `ltxcmds' and `hologo'.
%   \item
%     Module \xmodule{pagelayout} added.
%   \item
%     Module \xmodule{pageattr} added.
%   \item
%     Robustness introduced for non-expandable interface macros.
%   \item
%     Internal change of the data format of property lists
%     (suggestion of Florent Chervet).
%   \item
%     Module \xmodule{titleref}: Support of environment \textsf{description}.
%   \end{Version}
%   \begin{Version}{2010/05/01 v2.17}
%   \item
%     \cs{zref@newprop} throws an error if the property already exists.
%   \item
%     Module \xmodule{xr}: Bug fix for the case of several \xext{aux} files
%     (bug found by Victor Ivrii).
%   \item
%     Module \xmodule{xr}: Property `urluse' and option \xoption{urluse} added.
%   \end{Version}
%   \begin{Version}{2010/05/13 v2.18}
%   \item
%     Module \xmodule{env} added.
%   \item
%     Module \xmodule{savepos}: \cs{zref@savepos} added.
%   \end{Version}
%   \begin{Version}{2010/10/22 v2.19}
%   \item
%     \cs{zref@addprop} and \cs{zref@localaddprop} are limited
%     to one property only
%     (incompatibility to versions v2.13 to v2.18).
%   \item
%     \cs{zref@addprops} and \cs{zref@localaddprops} added.
%   \item
%     \cs{zref@delprop} and \cs{zref@localdelprop} added.
%   \item
%     \cs{zref@labelbykv} and \cs{zkvlabel} (module \xmodule{user})
%     with keys \xoption{prop}, \xoption{list}, \xoption{delprop},
%     \xoption{immediate}, \xoption{values} added.
%   \end{Version}
%   \begin{Version}{2011/02/12 v2.20}
%   \item
%     Fix for warning in \xpackage{zref-xr}.
%   \end{Version}
%   \begin{Version}{2011/03/18 v2.21}
%   \item
%     Fix in module \xmodule{pagelayout} for \cs{zlistpagelayout}.
%   \item
%     Fix for \cs{zref@localaddprop} (probably since v2.19).
%   \end{Version}
%   \begin{Version}{2011/10/05 v2.22}
%   \item
%     Documentation fixed for \cs{zref@(local)addprop(s)}.
%   \item
%     Module \xmodule{base}: \cs{zref@def@extract}, \cs{zref@def@extractdefault}
%     added.
%   \item
%     Fix in module \xmodule{pagelayout}: Because of missing \cs{noexpand}
%     commands the values of the pagelayout properties on all pages were
%     the values at package loading.
%   \item
%     Module \xmodule{base}: \cs{zref@showprop} added.
%   \end{Version}
%   \begin{Version}{2011/12/05 v2.23}
%   \item
%     Module \xmodule{savepos}: \cs{zsaveposx} and \cs{zsaveposy} added.
%   \end{Version}
%   \begin{Version}{2012/04/04 v2.24}
%   \item
%     Module \xmodule{titleref}, package \xpackage{titlesec}:
%     some support for class `straight' (\cs{ttl@straight@i}) added.
%   \end{Version}
%   \begin{Version}{2016/05/16 v2.25}
%   \item
%     Documentation updates.
%   \end{Version}
%   \begin{Version}{2016/05/21 v2.26}
%   \item
%     update zref-savepos for new luatex
%   \end{Version}
%   \begin{Version}{2018/11/21 v2.27}
%   \item
%     adapted zref-perpage, see issue https://github.com/ho-tex/zref/issues/2
%   \end{Version}
%   \begin{Version}{2019/11/29 v2.28}
%   \item
%     Documentation updates.
%   \item
%     Use \xpackage{iftex} directly.
%   \end{Version}
%   \begin{Version}{2020-03-03 v2.29}
%   \item
%     adapted in module zref-pagelayout the properties pdfhorigin, pdfvorigin,
%     pdfpagewidth, pdfpageheight for luatex to the right primitives.
%   \item
%     Removed no longer needed code for older lualatex versions.
%   \item added some documentation of the abspos module.
%   \item adapted the abspos module to the new luatex primitives.
%   \item adapted pageattr module to the new luatex primitives.
%   \item added short documentation for pageattr module
%   \item use luatex command names directly in zref-savepos rather than defining pdftex compatibility names.
%   \item allow zref-abspos to use \verb|\pdf[vh]origin| in dvi mode.
%   \end{Version}
%   \begin{Version}{2020-03-04 v2.30}
%   \item
%     add pagevalue property to savepos in the abspos module (issue 1)
%   \end{Version}
%   \begin{Version}{2020-05-28 v2.31}
%   \item Adapted module zref-counter to use \cs{@currentcounter} in the next
%   \LaTeX version (issue 5)
%   \end{Version}
%   \begin{Version}{2020-07-03 v2.32}
%   \item Changed in zref-pagelayout the names of the shipout box dimensions
%    to adapt to the new hook management.
%   \end{Version}
% \end{History}
%
% \PrintIndex
%
% \Finale
\endinput

%        (quote the arguments according to the demands of your shell)
%
% Documentation:
%    (a) If zref.drv is present:
%           latex zref.drv
%    (b) Without zref.drv:
%           latex zref.dtx; ...
%    The class ltxdoc loads the configuration file ltxdoc.cfg
%    if available. Here you can specify further options, e.g.
%    use A4 as paper format:
%       \PassOptionsToClass{a4paper}{article}
%
%    Programm calls to get the documentation (example):
%       pdflatex zref.dtx
%       makeindex -s gind.ist zref.idx
%       pdflatex zref.dtx
%       makeindex -s gind.ist zref.idx
%       pdflatex zref.dtx
%
% Installation:
%    TDS:tex/latex/zref/zref.sty
%    TDS:tex/latex/zref/zref-base.sty
%    TDS:tex/latex/zref/zref-abspage.sty
%    TDS:tex/latex/zref/zref-abspos.sty
%    TDS:tex/latex/zref/zref-counter.sty
%    TDS:tex/latex/zref/zref-dotfill.sty
%    TDS:tex/latex/zref/zref-env.sty
%    TDS:tex/latex/zref/zref-hyperref.sty
%    TDS:tex/latex/zref/zref-lastpage.sty
%    TDS:tex/latex/zref/zref-marks.sty
%    TDS:tex/latex/zref/zref-nextpage.sty
%    TDS:tex/latex/zref/zref-pageattr.sty
%    TDS:tex/latex/zref/zref-pagelayout.sty
%    TDS:tex/latex/zref/zref-perpage.sty
%    TDS:tex/latex/zref/zref-runs.sty
%    TDS:tex/latex/zref/zref-savepos.sty
%    TDS:tex/latex/zref/zref-thepage.sty
%    TDS:tex/latex/zref/zref-titleref.sty
%    TDS:tex/latex/zref/zref-totpages.sty
%    TDS:tex/latex/zref/zref-user.sty
%    TDS:tex/latex/zref/zref-xr.sty
%    TDS:doc/latex/zref/zref.pdf
%    TDS:doc/latex/zref/zref-example.tex
%    TDS:doc/latex/zref/zref-example-lastpage.tex
%    TDS:doc/latex/zref/zref-example-nextpage.tex
%    TDS:source/latex/zref/zref.dtx
%
%<*ignore>
\begingroup
  \catcode123=1 %
  \catcode125=2 %
  \def\x{LaTeX2e}%
\expandafter\endgroup
\ifcase 0\ifx\install y1\fi\expandafter
         \ifx\csname processbatchFile\endcsname\relax\else1\fi
         \ifx\fmtname\x\else 1\fi\relax
\else\csname fi\endcsname
%</ignore>
%<*install>
\input docstrip.tex
\Msg{************************************************************************}
\Msg{* Installation}
\Msg{* Package: zref 2020-07-03 v2.32 A new reference scheme for LaTeX (HO)}
\Msg{************************************************************************}

\keepsilent
\askforoverwritefalse

\let\MetaPrefix\relax
\preamble

This is a generated file.

Project: zref
Version: 2020-07-03 v2.32

Copyright (C)
  2006-2012 Heiko Oberdiek
  2016-2020 Oberdiek Package Support Group

This work may be distributed and/or modified under the
conditions of the LaTeX Project Public License, either
version 1.3c of this license or (at your option) any later
version. This version of this license is in
   https://www.latex-project.org/lppl/lppl-1-3c.txt
and the latest version of this license is in
   https://www.latex-project.org/lppl.txt
and version 1.3 or later is part of all distributions of
LaTeX version 2005/12/01 or later.

This work has the LPPL maintenance status "maintained".

The Current Maintainers of this work are
Heiko Oberdiek and the Oberdiek Package Support Group
https://github.com/ho-tex/zref/issues


This work consists of the main source file zref.dtx
and the derived files
   zref.sty, zref.pdf, zref.ins, zref.drv, zref-base.sty,
   zref-abspage.sty, zref-abspos.sty, zref-counter.sty,
   zref-dotfill.sty, zref-env.sty, zref-hyperref.sty,
   zref-lastpage.sty, zref-marks.sty, zref-nextpage.sty,
   zref-pageattr.sty, zref-pagelayout.sty, zref-perpage.sty,
   zref-runs.sty, zref-savepos.sty, zref-thepage.sty,
   zref-titleref.sty, zref-totpages.sty, zref-user.sty,
   zref-xr.sty, zref-example.tex, zref-example-lastpage.tex,
   zref-example-nextpage.tex, zref-test1.tex,
   zref-test-base.tex, zref-test-runs.tex,
   zref-test-titleref-memoir.tex.

\endpreamble
\let\MetaPrefix\DoubleperCent

\generate{%
  \file{zref.ins}{\from{zref.dtx}{install}}%
  \file{zref.drv}{\from{zref.dtx}{driver}}%
  \usedir{tex/latex/zref}%
  \file{zref.sty}{\from{zref.dtx}{package}}%
  \file{zref-base.sty}{\from{zref.dtx}{base}}%
  \file{zref-abspage.sty}{\from{zref.dtx}{abspage}}%
  \file{zref-abspos.sty}{\from{zref.dtx}{abspos}}%
  \file{zref-counter.sty}{\from{zref.dtx}{counter}}%
  \file{zref-dotfill.sty}{\from{zref.dtx}{dotfill}}%
  \file{zref-env.sty}{\from{zref.dtx}{env}}%
  \file{zref-hyperref.sty}{\from{zref.dtx}{hyperref}}%
  \file{zref-lastpage.sty}{\from{zref.dtx}{lastpage}}%
  \file{zref-marks.sty}{\from{zref.dtx}{marks}}%
  \file{zref-nextpage.sty}{\from{zref.dtx}{nextpage}}%
  \file{zref-pageattr.sty}{\from{zref.dtx}{pageattr}}%
  \file{zref-pagelayout.sty}{\from{zref.dtx}{pagelayout}}%
  \file{zref-perpage.sty}{\from{zref.dtx}{perpage}}%
  \file{zref-runs.sty}{\from{zref.dtx}{runs}}%
  \file{zref-savepos.sty}{\from{zref.dtx}{savepos}}%
  \file{zref-thepage.sty}{\from{zref.dtx}{thepage}}%
  \file{zref-titleref.sty}{\from{zref.dtx}{titleref}}%
  \file{zref-totpages.sty}{\from{zref.dtx}{totpages}}%
  \file{zref-user.sty}{\from{zref.dtx}{user}}%
  \file{zref-xr.sty}{\from{zref.dtx}{xr}}%
  \usedir{doc/latex/zref}%
  \file{zref-example.tex}{\from{zref.dtx}{example}}%
  \file{zref-example-lastpage.tex}{\from{zref.dtx}{example-lastpage}}%
  \file{zref-example-nextpage.tex}{\from{zref.dtx}{example-nextpage}}%
%  \usedir{doc/latex/zref/test}%
%  \file{zref-test1.tex}{\from{zref.dtx}{test1}}%
%  \file{zref-test-base.tex}{\from{zref.dtx}{test-base}}%
%  \file{zref-test-runs.tex}{\from{zref.dtx}{test-runs}}%
%  \file{zref-test-titleref-memoir.tex}{\from{zref.dtx}{test-titleref-memoir}}%
}

\catcode32=13\relax% active space
\let =\space%
\Msg{************************************************************************}
\Msg{*}
\Msg{* To finish the installation you have to move the following}
\Msg{* files into a directory searched by TeX:}
\Msg{*}
\Msg{*     zref.sty, zref-base.sty, zref-abspage.sty, zref-abspos.sty,}
\Msg{*     zref-counter.sty, zref-dotfill.sty, zref-env.sty,}
\Msg{*     zref-hyperref.sty, zref-lastpage.sty, zref-marks.sty,}
\Msg{*     zref-nextpage.sty, zref-pageattr.sty, zref-pagelayout.sty,}
\Msg{*     zref-perpage.sty, zref-runs.sty, zref-savepos.sty,}
\Msg{*     zref-thepage.sty, zref-titleref.sty, zref-totpages.sty,}
\Msg{*     zref-user.sty, zref-xr.sty}
\Msg{*}
\Msg{* To produce the documentation run the file `zref.drv'}
\Msg{* through LaTeX.}
\Msg{*}
\Msg{* Happy TeXing!}
\Msg{*}
\Msg{************************************************************************}

\endbatchfile
%</install>
%<*ignore>
\fi
%</ignore>
%<*driver>
\NeedsTeXFormat{LaTeX2e}
\ProvidesFile{zref.drv}%
  [2020-07-03 v2.32 A new reference scheme for LaTeX (HO)]%
\documentclass{ltxdoc}
\usepackage{holtxdoc}[2011/11/22]
\setlength{\emergencystretch}{1em}
\begin{document}
  \DocInput{zref.dtx}%
\end{document}
%</driver>
% \fi
%
%
%
% \GetFileInfo{zref.drv}
%
% \title{The \xpackage{zref} package}
% \date{2020-07-03 v2.32}
% \author{Heiko Oberdiek\thanks
% {Please report any issues at \url{https://github.com/ho-tex/zref/issues}}}
%
% \maketitle
%
% \begin{abstract}
% Package \xpackage{zref} tries to get rid of the restriction
% in \hologo{LaTeX}'s reference system that only two properties are
% supported. The package implements an extensible referencing
% system, where properties are handled in a more flexible way.
% It offers an interface for macro programmers for the access
% to the system and some applications that uses the new
% reference scheme.
% \end{abstract}
%
% \tableofcontents
%
% \newcommand*{\metacs}[1]{\texttt{\itshape #1}}
%
% \newcommand{\markexp}{^^A
%   \textsuperscript{exp}^^A
%   \space
% }
% \newcommand{\markexptwo}{^^A
%   \textsuperscript{exp2}^^A
%   \space
% }
% \newcommand{\markrobust}{^^A
%   \textsuperscript{robust}^^A
%   \space
% }
% \newcommand{\marknowarn}{^^A
%   \textsuperscript{nowarn}^^A
%   \space
% }
% \newcommand{\markbabel}{^^A
%   \textsuperscript{babel}^^A
%   \space
% }
%
% \section{Introduction}
%
% Standard \hologo{LaTeX}'s reference system with \cs{label}, \cs{ref},
% and \cs{pageref} supports two properties, the apperance of
% the counter that is last incremented by \cs{refstepcounter} and
% the page with the \cs{label} command.
%
% Unhappily \hologo{LaTeX} does not provide an interface for adding
% another properties. Packages such as \xpackage{hyperref},
% \xpackage{nameref}, or \xpackage{titleref} are forced to
% use ugly hacks to extend the reference system.
% These ugly hacks are one of the causes for \xpackage{hyperref}'s
% difficulty regarding compatibility with other packages.
%
% \subsection{Standard \hologo{LaTeX} behaviour}
%
% References are created by the \cs{label} command:
% \begin{quote}
%   |\chapter{Second chapter}|\\
%   |\section{First section on page 7} % section 2.1|\\
%   |\label{myref}|
% \end{quote}
% Now \hologo{LaTeX} records the section number |2.1| and the page |7|
% in the reference. Internally the reference is a list with
% two entries:
% \begin{quote}
%   |\r@myref| $\rightarrow$ |{2.1}{7}|
% \end{quote}
% The length of the list if fixed in the \hologo{LaTeX} kernel,
% An interface for adding new properties is missing.
%
% There are several tries to add new properties:
% \begin{description}
% \item[\xpackage{hyperref}] uses a list of five
%   properties instead of the standard list with two entries.
%   This causes many compatibility problems with \hologo{LaTeX} and
%   other packages.
% \item[\xpackage{titleref}] stores its title data into the
%   first entry in the list. \hologo{LaTeX} is happy because it
%   does only see its list with two entries. The situation
%   becomes more difficult, if more properties are added
%   this way. Then the macros form a nested structure
%   inside the first reference argument for the label.
%   Expandable extractions will then become painful.
% \end{description}
%
% \subsection{Basic idea}
%
% Some time ago Morten H\o gholm sent me an experimental
% cross referencing mechanism as ``expl3'' code. His
% idea is:
% \begin{quote}
%   |\g_xref_mylabel_plist| $\rightarrow$\\
%   \mbox{}\qquad|\xref_dance_key{salsa}\xref_name_key{Morten}|\dots
% \end{quote}
% The entries have the following format:
% \begin{quote}
%   |\xref_|\meta{your key}|_key{|\meta{some text}|}|
% \end{quote}
% This approach is much more flexible:
% \begin{itemize}
%   \item New properties can easily be added, just use a new key.
%   \item The length of the list is not fixed. A reference
%         can use a subset of the keys.
%   \item The order of the entries does not matter.
% \end{itemize}
% Unhappily I am not familiar with the experimental code
% for \hologo{LaTeX3} that will need some time before its first release.
% Thus I have implemented it as \hologo{LaTeXe} package without disturbing
% the existing \hologo{LaTeX} reference system.
%
% \subsection{Interfaces}
%
% The package provides a generic \emph{interface for programmers}.
% Commands of this interface are prefixed by \cs{zref@}.
%
% Option \xoption{user} enabels the \emph{user interface}.
% Here the commands are prefixed by \cs{z} to avoid name
% clashes with existing macros.
%
% Then the packages provides some \emph{modules}. They
% are applications for the reference system and can also be
% considered as examples how to use the reference system.
%
% The modules can be loaded as packages. The package
% name is prefixed with |zref-|, for example:
%\begin{quote}
%\begin{verbatim}
%\RequirePackage{zref-abspage}
%\end{verbatim}
%\end{quote}
% This is the preferred way if the package is loaded from
% within other packages to avoid option clashes.
%
% As alternative package |zref| can be used and the modules
% are given as options:
%\begin{quote}
%\begin{verbatim}
%\usepackage[perpage,user]{zref}
%\end{verbatim}
%\end{quote}
%
% \section{Interface for programmers}
%
% The user interface is described in the next section \ref{sec:user}.
%
% \subsection{Entities}
%
% \paragraph{Reference.}
% Internally a reference is a list of key value pairs:
% \begin{quote}
%   |\Z@R@myref| $\rightarrow$
%   |\default{2.1}\page{7}|
% \end{quote}
% The generic format of a entry is:
% \begin{quote}
%   |\Z@R@|\meta{refname} $\rightarrow$
%   |\|\meta{propname}|{|\meta{value}|}|
% \end{quote}
% \meta{refname} is the name that denoted references
% (the name used in \cs{label} and \cs{ref}). \meta{propname}
% is the name of the property or key. The property key macro
% is never executed, it is used in parameter text matching only.
%
% \paragraph{Property.}
% Because the name of a property is used in a macro name
% that must survive the \xfile{.aux} file, the name is
% restricted to letters and `@'.
%
% \paragraph{Property list.}
% Often references are used for special purposes. Thus it
% saves memory if just the properties are used in this
% reference that are necessary for its purpose.
%
% Therefore this package uses the concept of \emph{property lists}.
% A property list is a set of properties. The set of properties
% that is used by the default \cs{label} command is the
% \emph{main property list}.
%
% \subsection{Property list}
%
% \markexp means that the implementation of the
% marked macro is expandable. \markexptwo goes a step further
% and marks the macro expandable in exact two expansion steps.
%
% \begin{declcs}{zref@newlist}
%   \M{listname}
% \end{declcs}
% Declares a new empty property list.
%
% \begin{declcs}{zref@addprop}
%   \M{listname} \M{propname}\\
%   \cs{zref@localaddprop} \M{listname} \M{propname}
% \end{declcs}
% Adds the property \meta{propname}
% to the property list \meta{listname}.
% The property and list must exist.
% The addition is global by \cs{zref@addprop} and limited to
% local scope by \cs{zref@localaddprop}.
% Between 2010/04/19 v2.13 and 2010/10/22 v2.19 a comma separated list
% of properties could be used as argument \meta{propname}.
% Since 2010/10/22 v2.19 the addition of several properties
% at once is supported by \cs{zref@addprops}.
%
% \begin{declcs}{zref@addprops}
%   \M{listname} \M{propname list}\\
%   \cs{zref@localaddprops} \M{listname} \M{propname list}
% \end{declcs}
% These macros add a comma separated list of properties
% \meta{propname list} to list \meta{listname}.
% \cs{zref@addprops} works globally and \cs{zref@localaddprops}
% locally. Since 2010/10/22 v2.19.
%
% \begin{declcs}{zref@listexists}
%   \M{listname} \M{then}
% \end{declcs}
% Executes \meta{then} if the property list \meta{listname} exists
% or raise an error otherwise.
%
% \begin{declcs}{zref@iflistundefined}\markexp \M{listname}
%   \M{then} \M{else}
% \end{declcs}
% Executes \meta{then} if the list exists or \meta{else}
% otherwise.
%
% \begin{declcs}{zref@iflistcontainsprop}
%   \M{listname} \M{propname} \M{then} \M{else}
% \end{declcs}
% Executes \meta{then} if the property \meta{propname} is part
% of property list \meta{listname} or otherwise it
% runs the \meta{else} part.
%
% \subsection{Property}
%
% \begin{declcs}{zref@newprop}
%   \* \M{propname} |[|\meta{default}|]| \M{value}
% \end{declcs}
% This command declares and configures a new property with
% name \meta{propname}.
%
% In case of unknown references
% or the property does not exist in the reference, the
% \meta{default} is used as value. If it is not specified
% here, a global default is used, see \cs{zref@setdefault}.
%
% The correct values of some properties are not known
% immediately but at page shipout time. Prominent example
% is the page number. These properties are declared with
% the star form of the command.
%
% \begin{declcs}{zref@setcurrent}
%   \M{propname} \M{value}
% \end{declcs}
% This sets the current value of the property \meta{propname}.
% It is a generalization of setting \hologo{LaTeX}'s \cs{currentlabel}.
%
% \begin{declcs}{zref@getcurrent}\markexptwo
%   \M{propname}
% \end{declcs}
% This returns the current value of the property \meta{propname}.
% The value may not be correct, especially if the property
% is bound to a page (start form of \cs{zref@newprop}) and the
% right value is only known at shipout time (e.g. property `page').
% In case of errors (e.g. unknown property) the empty string is
% returned.
%
% Since version 2010/04/22 v2.14 \cs{zref@getcurrent} supports
% \cs{zref@wrapper@unexpanded}.
%
% \begin{declcs}{zref@propexists}
%   \M{propname} \M{then}
% \end{declcs}
% Calls \meta{then} if the property \meta{propname} is available
% or generates an error message otherwise.
%
% \begin{declcs}{zref@ifpropundefined}\markexp
%   \M{propname} \M{then} \M{else}
% \end{declcs}
% Calls \meta{then} or \meta{else} depending on the existence
% of property \meta{propname}.
%
% \subsection{Reference generation}
%
% \begin{declcs}{zref@label}
%   \M{refname}
% \end{declcs}
% This works similar to \cs{label}. The reference \meta{refname} is
% created and put into the \xfile{.aux} file with the
% properties of the main property list.
%
% \begin{declcs}{zref@labelbylist}
%   \M{refname} \M{listname}
% \end{declcs}
% Same as \cs{zref@label} except that the properties are
% taken from the specified property list \meta{listname}.
%
% \begin{declcs}{zref@labelbyprops}
%   \M{refname} |{|\meta{propnameA}|,|\meta{propnameB}|,|\dots|}|
% \end{declcs}
% Same as \cs{zref@label} except that these properties
% are used that are given as comma separated list in the
% second argument.
%
% \begin{declcs}{zref@newlabel}
%   \M{refname} |{|\dots|}|
% \end{declcs}
% This is the macro that is used in the \xfile{.aux} file.
% It is basically the same as \cs{newlabel} apart from
% the format of the data in the second argument.
%
% \subsection{Data extraction}
%
% \begin{declcs}{zref@extractdefault}\markexptwo
%   \M{refname} \M{propname} \M{default}
% \end{declcs}
% This is the basic command that refernces the value of a
% property \meta{propname} for the reference \meta{refname}.
% In case of errors such as undefined reference the \meta{default}
% is used instead.
%
% \begin{declcs}{zref@extract}\markexptwo
%   \M{refname} \M{propname}
% \end{declcs}
% The command is an abbreviation for \cs{zref@extractdefault}.
% As default the default of the property is taken, otherwise
% the global default.
%
% Example for page references:
% \begin{quote}
% \begin{tabular}{@{}ll}
% \hologo{LaTeX}:& |\pageref{foobar}|\\
% \xpackage{zref}:& |\zref@extract{foobar}{page}|
% \end{tabular}
% \end{quote}
% Both \cs{zref@extract} and \cs{zref@extractdefault} are
% expandable. That means, these macros can directly
% be used in expandable calculations, see the example file.
% On the other side, babel's shorthands are not supported,
% there are no warnings in case of undefined references.
%
% If an user interface doesn't need expandable macros
% then it can use \cs{zref@refused} and \cs{zref@wrapper@babel}
% for its user macros.
%
% \begin{declcs}{zref@refused}
%   \M{refname}
% \end{declcs}
% This command is not expandable. It causes the warnings if
% the reference \meta{refname} is not defined. Use the
% \cs{zref@extract} commands inside expandable contexts and
% mark their use outside by \cs{zref@refused}, see the example
% file.
%
% \begin{declcs}{zref@def@extract}%
%     \M{cmd} \M{refname} \M{propname}\\
%   \cs{zref@def@extractdefault}
%     \M{cmd} \M{refname} \M{propname} \M{default}
% \end{declcs}
% Both macros extract the property \meta{propname} from the reference
% \meta{refname} the same way as macros \cs{zref@extract} and
% \cs{zref@extractdefault}. The result is stored in macro \meta{cmd}.
% Also \cs{zref@refused} is called to notify \hologo{LaTeX} that
% the reference \meta{refname} is used. Added in 2011/10/04 v2.22.
%
% \begin{declcs}{zref@ifrefundefined}\markexp
%   \M{refname} \M{then} \M{else}
% \end{declcs}
% Macro \cs{zref@ifrefundefined} calls arguments \meta{then} or \meta{else}
% dependent on the existence of the reference \meta{refname}.
%
% \begin{declcs}{zifrefundefined}
%   \M{refname} \M{then} \M{else}
% \end{declcs}
% Macro \cs{zifrefundefined} calls \cs{ref@refused} before
% executing \cs{zref@ifrefundefined}. Babel shorthands are
% supported in \meta{refname}.
%
% \begin{declcs}{zref@ifrefcontainsprop}\markexp
%   \M{refname} \M{propname} \M{then} \M{else}
% \end{declcs}
% Test whether a reference provides a property.
%
% \subsection{Setup}
%
% \begin{declcs}{zref@default}
% \end{declcs}
% Holds the global default for unknown values.
%
% \begin{declcs}{zref@setdefault}
%   |{|\meta{value}|}|
% \end{declcs}
% Sets the global default for unknown values.
% The global default is used, if a property does not specify
% an own default and the value for a property
% cannot be extracted. This can happen if the reference is
% unknown or the reference does not have the property.
%
% \begin{declcs}{zref@setmainlist}
%   |{|\meta{value}|}|
% \end{declcs}
% Sets the name of the main property list. The package
% sets and uses |main|.
%
% \subsection{Declared properties}
%
% \begin{quote}
% \renewcommand*{\arraystretch}{1.05}
% \setlength{\tabcolsep}{2\tabcolsep}
% \def\VV{\vphantom{\large \"Ag}}
% \def\EMPTY{{\itshape\textless empty\textgreater}}
% \def\pl#1{&#1&thepage&\cs{number}\cs{#1}}
% \begin{tabular}{@{}>{\sffamily}l|>{\ttfamily}l|>{\ttfamily}l|>{\ttfamily}l@{}}
%   \rmfamily Module&
%   \rmfamily Property&
%   \rmfamily Property list&
%   \rmfamily Default\\
%   \hline
%   (base)
%     & default & main & \EMPTY\\
%     & page & main & \EMPTY\\
%   \hline
%   abspage
%     & abspage & main & 0\\
%   \hline
%   counter
%     & counter & main & \EMPTY\\
%   \hline
%   hyperref
%     & anchor & main & \EMPTY\\
%     & url    & & \EMPTY\\
%   \hline
%   pageattr
%     & pdfpageattr & thepage & \dots\\
%     & pdfpagesattr & LastPage & \dots\\
%   \hline
%   pagelayout\footnotemark
%     \pl{mag}\\
%     \pl{paperwidth}\\
%     \pl{paperheight}\\
%     \pl{stockwidth}\\
%     \pl{stockheight}\\
%     \pl{pdfpageheight}\\
%     \pl{pdfpagewidth}\\
%     \pl{pdfhorigin}\\
%     \pl{pdfvorigin}\\
%     \pl{hoffset}\\
%     \pl{voffset}\\
%     \pl{topmargin}\\
%     \pl{oddsidemargin}\\
%     \pl{evensidemargin}\\
%     \pl{textwidth}\\
%     \pl{textheight}\\
%     \pl{headheight}\\
%     \pl{headsep}\\
%     \pl{footskip}\\
%     \pl{marginparwidth}\\
%     \pl{marginparsep}\\
%     \pl{columnwidth}\\
%     \pl{columnsep}\\
%   \hline
%   perpage
%     & pagevalue & perpage & 0\\
%     & page & perpage & \EMPTY\\
%     & abspage & perpage & 0\\
%   \hline
%   savepos
%     & posx & savepos & 0\\
%     & posy & savepos & 0\\
%   \hline
%   titleref
%     & title & main & \EMPTY\\
%   \hline
%   xr
%     & anchor   & & \EMPTY\\
%     & externaldocument & & \EMPTY\\
%     & theotype & & \EMPTY\\
%     & title    & & \EMPTY\\
%     & url      & & \EMPTY\\
% \end{tabular}
% \footnotetext{Module \xmodule{pagelayout} only defines properties
% if the parameter exists.}%
% \end{quote}
%
% \subsection{Wrapper for advanced situations}
%
% \begin{declcs}{zref@wrapper@babel}
%   |{|\dots|}| \M{name}
% \end{declcs}
% This macro helps to add shorthand support. The second
% argument is protected, then the code of the first argument
% is called with the protected name appended. Examples
% are in the sources.
%
% \begin{declcs}{zref@wrapper@immediate}
%   |{|\dots|}|
% \end{declcs}
% There are situations where a label must be written instantly
% to the \xfile{.aux} file, for example after the last page.
% If the \cs{zlabel} or \cs{label} command is put inside this wrapper,
% immediate writing is enabled. See the implementation for
% module \xmodule{lastpage} for an example of its use.
%
% \begin{declcs}{zref@wrapper@unexpanded}
%   |{|\dots|}|
% \end{declcs}
% Assuming someone wants to extract a value for property |bar|
% and store the result in a macro |\foo| without traces of
% the expanding macros and without expanding the value.
% This (theoretical?) problem can be solved by this wrapper:
% \begin{quote}
%\begin{verbatim}
%\zref@wrapper@unexpanded{%
%  \edef\foo{%
%    \zref@extract{someref}{bar}%
%  }%
%}
%\end{verbatim}
% \end{quote}
% The \cs{edef} forces the expansion of \cs{zref@extract},
% but the extraction of the value is prevented by
% the wrapper that uses \hologo{eTeX}'\ \cs{unexpanded} for this
% purpose. Supported macros are \cs{zref@extract}, \cs{zref@extractdefault}
% and since version 2010/04/22 v2.14 macro \cs{zref@getcurrent}.
%
% \subsection{Counter for unique names}
%
% Some modules (\xmodule{titleref} and \xmodule{dotfillmin})
% need unique names for automatically generated label names.
%
% \begin{declcs}{zref@require@unique}
% \end{declcs}
% This command creates the unique counter |zref@unique| if the
% counter does not already exist.
%
% \begin{declcs}{thezref@unique}
% \end{declcs}
% This command is used to generate unique label names.
%
% \section{User interface}\label{sec:user}
%
% \subsection{Module \xmodule{user}}
%
% The user interface for this package and its modules
% is enabled by \xpackage{zref}'s package option \xmodule{user}
% or package \xpackage{zref-user}.
% The names of user commands are prefixed by |z| in order
% to avoid name clashes with existing macros of the same
% functionality. Thus the package does not disturb the
% traditional reference scheme, both can be used together.
%
% The syntax descriptions contain the following markers
% that are intended as hints for programmers:
% \begin{quote}
% \begin{tabular}{@{}ll@{}}
%   \markbabel\unskip  & Babel shorthands are allowed.\\
%   \markrobust\unskip & Robust macro.\\
%   \markexp\unskip    & Expandable version:\\
%                      & \textbullet\ robust,
%                        unless the extracted values are fragile,\\
%                      & \textbullet\ no babel shorthand suport.\\
%   \markexptwo\unskip & Expandable like \markexp and:\\
%                      & \textbullet\ expandable in exact two steps.\\
% \end{tabular}
% \end{quote}
%
% The basic user interface of the package without modules
% are commands that mimic the standard \hologo{LaTeX} behaviour
% of \cs{label}, \cs{ref}, and \cs{pageref}:
%
% \begin{declcs}{zlabel}
%   \M{refname}\markbabel
% \end{declcs}
% Similar to \cs{label}. It generates a label with name
% \meta{refname} in the new reference scheme.
%
% \begin{declcs}{zref}
%   |[|\meta{propname}|]| \M{refname}\markbabel
% \end{declcs}
% Without optional argument similar to \cs{ref}, it returns
% the default reference property. This property is named
% |default|:
% \[
%   |\zref{|x|}| \equiv |\zref[default]{|x|}|
% \]
%
% \begin{declcs}{zpageref}
%   \M{refname}\markbabel
% \end{declcs}
% Convenience macro, similar to \cs{pageref}.
% \[
%   |\zpageref{|x|}| \equiv |\zref[page]{|x|}|
% \]
%
% \begin{declcs}{zrefused}
%   \M{refname}\markbabel
% \end{declcs}
% Some of the user commands in the modules are expandable.
% The use of such commands do not cause any undefined
% reference warnings, because inside of expandable
% contexts this is not possible. However, if there is
% a place outside of expandable contexts, \cs{refused}
% is strongly recommended. The reference \meta{refname}
% is marked as used, undefined ones will generate
% warnings.
%
% \subsection{Module \xmodule{abspage}}
%
% With the help of package \xpackage{atbegshi}
% a new counter |abspage| with absolute page numbers is provided.
% For technical and historical reasons the counter itself is zero based:
% if you use it directly on the first page, e.g
% with |\arabic{abspage}| you will get 0 as value. When using |\zref|
% the first page will be page 1 as expected.
% Also a new property |abspage| is defined and added to the
% main property list. Thus you can reference the absolute
% page number:
% \begin{quote}
%   |Section \zref{foo} is on page \zpageref{foo}.|\\
%   |This is page \zref[abspage]{foo}|\\
%   |of \zref[abspage]{LastPage}.|
% \end{quote}
% The example also makes use of module \xmodule{lastpage}.
%
% \subsection{Module \xmodule{lastpage}}
%
% Provides the functionality of package \xpackage{lastpage}
% \cite{lastpage} in the new reference scheme.
% The label |LastPage| is put at the end of the document.
% You can refer the last page number with:
% \begin{quote}
%   |\zref@extract{LastPage}{page}| (+ |\zref@refused{LastPage}|)\\
% \end{quote}
% or
% \begin{quote}
%   |\zpageref{LastPage}| (module \xmodule{user})
% \end{quote}
%
% Since version 2008/10/01 v2.3 the module defines the list |LastPage|.
% In addition to the properties of the main list label |LastPage| also
% stores the properties of this list |LastPage|. The default of this
% list is empty. The list can be used by the user to add additional
% properties for label |LastPage|.
%
% \subsubsection{Tests for last page}
%
% Since version 2010/03/26 v2.8 the macros \cs{zref@iflastpage}
% and \cs{ziflastpage} were added. They test the reference,
% whether it is a reference of the last page.
%
% \begin{declcs}{zref@iflastpage}\markexp \M{refname} \M{then} \M{else}
% \end{declcs}
% Macro \cs{zref@iflastpage} compares the references \meta{refname}
% with \meta{LastPage}. Basis of the comparison is the value of
% property |abspage|, because the values are different for different
% pages. This is not ensured by property |page|. Therefore module
% \xmodule{abspage} is loaded by module \xmodule{lastpage}.
% If both values of property |abspage| are present and match,
% then \meta{then} is executed, otherwise code \meta{else} is called.
% If one or both references are undefined or lack the property |abspage|,
% then \meta{else} is executed.
%
% Macro \cs{zref@iflastpage} is expandable, therefore \cs{zref@refused}
% should be called on \meta{refname} and \meta{LastPage}.
%
% \begin{declcs}{ziflastpage} \M{refname} \M{then} \M{else}
% \end{declcs}
% Macro \cs{ziflastpage} has the same function as \cs{zref@iflastpage},
% but adds support for babel shorthands in \meta{refname} and
% calls \cs{zref@refused}. However macro \cs{ziflastpage} is not
% expandable.
%
% \subsubsection{Example}
%
%    \begin{macrocode}
%<*example-lastpage>
%<<END_EXAMPLE
\NeedsTeXFormat{LaTeX2e}
\documentclass{report}

\newcounter{foo}
\renewcommand*{\thefoo}{\Alph{foo}}

\usepackage{zref-lastpage,zref-user}[2019/11/29]

\makeatletter
\zref@newprop{thefoo}{\thefoo}
\zref@newprop{valuefoo}{\the\value{foo}}
\zref@newprop{chapter}{\thechapter}
\zref@addprops{LastPage}{thefoo,valuefoo,chapter}
\makeatother

\newcommand*{\foo}{%
  \stepcounter{foo}%
  [Current foo: \thefoo]%
}

\begin{document}
  \chapter{First chapter}
  Last page is \zref{LastPage}.\\
  Last chapter is \zref[chapter]{LastPage}.\\
  Last foo is \zref[thefoo]{LastPage}.\\
  Last value of foo is \zref[valuefoo]{LastPage}.\\
  \foo
  \chapter{Second chapter}
  \foo\foo\foo
  \chapter{Last chapter}
  \foo
\end{document}
%END_EXAMPLE
%</example-lastpage>
%    \end{macrocode}
%
% \subsection{Module \xmodule{thepage}}
%
% This module \xmodule{thepage} loads module \xmodule{abspage},
% constructs a reference name using the absolute page number
% and remembers property |page|. Other properties can be added
% by adding them to the property list |thepage|.
%
% \begin{declcs}{zthepage} \M{absolute page number}\\
% \end{declcs}
% Macro \cs{zthepage} is basically a \cs{zpageref}. The reference
% name is yield by the \meta{absolute page number}. If the reference
% is not defined, then the default for property |page| is used.
%
% \begin{declcs}{zref@thepage@name}\markexp \M{absolute page number}
% \end{declcs}
% Macro \cs{zref@thepage@name} returns the internal reference
% name that is constructed using the \meta{absolute page number}.
% The internal reference name should not be used directly, because
% it might change in future versions.
%
% \begin{declcs}{zref@thepage}\markexp \M{absolute page number}\\
%   \cs{zref@thepage@refused} \M{absolute page number}
% \end{declcs}
% Macro \cs{zref@thepage} returns the page number (\cs{thepage})
% of \meta{absolute page number}. Because this macro is expandable,
% \cs{zref@thepage@refused} is used outside an expandable context
% to mark the reference as used.
%
%
% \subsection{Module \xmodule{nextpage}}
%
% \begin{declcs}{znextpage}
% \end{declcs}
% Macro \cs{znextpage} prints \cs{thepage} of the following page.
% It gets the current absolute page number by using a label.
% There are three cases for the next page:
% \begin{enumerate}
% \item The next page is not known yet because of undefined references.
%   Then \cs{zunknownnextpagename} is used instead. The default for
%   this macro is the default of property |page|.
% \item This page is the last page. Then \cs{znonextpagename}
%   is used. Its default is empty.
% \item The next page is known, then \cs{thepage} of the next
%   page is used (the value of property |page| of the next page).
% \end{enumerate}
%
% \subsubsection{Configuration}
%
% The behaviour can be configured by the following macros.
%
% \begin{declcs}{zunknownnextpagename}\\
%   \cs{znonextpagename}
% \end{declcs}
% If the next page is not known or available, then
% \cs{znextpage} uses these name macros as default.
% \cs{zunknownnextpagename} is used in case of undefined
% references. Default is the value of property |page| of the
% next page (\cs{thepage}). Module \xmodule{thepage} is used.
%
% Macro \cs{znonextpagename} is used, if the next page does
% not exists. That means that the current page is last page.
% The default is empty.
%
% \begin{declcs}{znextpagesetup} \M{unknown} \M{no next}
%   \M{next}
% \end{declcs}
% Acording to the case (see \cs{znextpage}) macro \cs{znextpage}
% calls an internal macro with an argument. The argument is
% either \cs{thepage} of the next page or one of \cs{zunknownnextpagename}
% or \cs{znonextpagename}. These internal macro can be changed by
% \cs{znextpagesetup}. It expects the definition texts for these
% three cases of a macro with one argument.
% The default is
% \begin{quote}
%   |\znextpagesetup{#1}{#1}{#1}|
% \end{quote}
%
% \subsubsection{Example}
%
%    \begin{macrocode}
%<*example-nextpage>
%<<END_EXAMPLE
\documentclass{book}

\usepackage{zref-nextpage}[2019/11/29]
\znextpagesetup
  {\thepage}% next page is unknown
  {\thepage\ (#1)}% this page is last page
  {\thepage\ $\rightarrow$ #1}% next page is known
\renewcommand*{\znonextpagename}{last page}

\usepackage{fancyhdr}
\pagestyle{fancy}
\fancyhf{}
\fancyhead[LE,RO]{\znextpage}
\fancypagestyle{plain}{%
  \fancyhf{}%
  \fancyhead[LE,RO]{\znextpage}%
}

\begin{document}
\frontmatter
  \tableofcontents
\mainmatter
  \chapter{Hello World}
  \clearpage
  \section{Last section}
\end{document}
%END_EXAMPLE
%</example-nextpage>
%    \end{macrocode}
%
%
% \subsection{Module \xmodule{totpages}}
%
% For the total number of pages of a document you need to
% know the absolute page number of the last page.
% Both modules \xmodule{abspage} and \xmodule{lastpage}
% are necessary and automatically enabled.
%
% \begin{declcs}{ztotpages}\markexp
% \end{declcs}
% Prints the total number of pages or |0| if this
% number is not yet known. It expands to an explicit number
% and can also used even in expandable calculations (\cs{numexpr})
% or counter assignments.
%
% \subsection{Module \xmodule{pagelayout}}
%
% The module defines additional properties for each parameter
% of the page layout that is effective during page shipout.
% The value of length parameters is given in sp without the
% unit as plain number.
%
% Some parameters are specific for a class (e.g.\@ \xoption{stockwidth}
% and \xoption{stockheight} for class \xclass{memoir}) or the
% \hologo{TeX} engine like \hologo{pdfTeX}. If the parameter is not
% available, then the property will not be defined. The default value
% of the property is the current setting of the parameter.
%
% The module \xmodule{thepage} is loaded that generates a label
% for each page. The properties of module \xmodule{pagelayout}
% are added to the property list |thepage| of module \xmodule{thepage}.
%
% List of properties:
% \begin{quote}
%   \def\x#1{\cs{#1}&#1}%
%   \begin{tabular}{l>{\sffamily}ll}
%     \textrm{parameter} & \textrm{property} & remarks\\
%     \hline
%     \x{mag}\\
%     \x{paperwidth}\\
%     \x{paperheight}\\
%     \x{stockwidth}& class \xclass{memoir}\\
%     \x{stockheight}& class \xclass{memoir}\\
%     \x{pdfpagewidth}& \hologo{pdfTeX}, \hologo{LuaTeX}\\
%     \x{pdfpageheight}& \hologo{pdfTeX}, \hologo{LuaTeX}\\
%     \x{pdfhorigin}& \hologo{pdfTeX}, \hologo{LuaTeX}\\
%     \x{pdfvorigin}& \hologo{pdfTeX}, \hologo{LuaTeX}\\
%     \x{hoffset}\\
%     \x{voffset}\\
%     \x{topmargin}\\
%     \x{oddsidemargin}\\
%     \x{evensidemargin}\\
%     \x{textwidth}\\
%     \x{textheight}\\
%     \x{headheight}\\
%     \x{headsep}\\
%     \x{footskip}\\
%     \x{marginparwidth}\\
%     \x{marginparsep}\\
%     \x{columnwidth}\\
%     \x{columnsep}\\
%   \end{tabular}
% \end{quote}
%
% \begin{declcs}{zlistpagelayout}
% \end{declcs}
% At the end of document the page layout parameter for each page are
% printed into the \xext{log} file if macro \cs{zlistpagelayout}
% is called before |\end{document}| (preamble is a good place).
%
% \subsection{Module \xmodule{marks}}
%
% ToDo.
%
% \subsection{Module \xmodule{runs}}
%
% Module \xmodule{runs} counts the \hologo{LaTeX} runs since last
% \xext{aux} file creation and prints the number in the
% \xext{log} file.
%
% \begin{declcs}{zruns}\markexp
% \end{declcs}
% Prints the the total number of \hologo{LaTeX} runs including
% the current one. It expands to an explicit number.
% Before |begin{document}| the value is zero meaning
% the \xext{aux} file is not read yet. If a previous
% \xext{aux} file exists, the value found there increased
% by one is the new number. Otherwise \cs{zruns} is set to one.
% \hologo{LaTeX}~runs where the \xext{aux} files are not rewritten
% are not counted (see \cs{nofiles}).
%
% \subsection{Module \xmodule{perpage}}
%
% With \cs{@addtoreset} or \cs{numberwithin} a counter can be
% reset if another counter is incremented. This do not work
% well if the other counter is the page counter. The page counter is
% incremented in the output routine that is often called asynchronous
% somewhere on the next page. A reference mechanism costs at least
% two \hologo{LaTeX} runs, but ensures correct page counter values.
%
% \begin{declcs}{zmakeperpage}
%   |[|\meta{reset}|]| \M{counter}
% \end{declcs}
% At the of a new page counter \meta{counter} starts
% counting with value \meta{reset} (default is \texttt{1}).
% The macro has the same
% syntax and semantics as \cs{MakePerPage} of package
% \xpackage{perpage} \cite{perpage}.
% Also \xoption{perpage} of package \xpackage{footmisc} \cite{footmisc}
% can easily be simulated by
% \begin{quote}
% |\zmakeperpage{footnote}| \itshape |% \usepackage[perpage]{footmisc}|
%\end{quote}
% If footnote symbols are used, some people dislike the
% first symbol \ensuremath{\dagger}. It can easily be skipped:
% \begin{quote}
% |\zmakeperpage[2]{footnote}|
% \end{quote}
%
% \begin{declcs}{thezpage}\\
%   counter |zpage|
% \end{declcs}
% If the formatted counter value of the counter that is reset at
% a new page contains the page value, then you can use \cs{thezpage},
% the page number of the current page. Or counter |zpage| can be
% used, if the page number should be formatted differently from
% the current page number. Example:
%\begin{quote}
%\begin{verbatim}
%\newcounter{foobar}
%\zmakeperpage{foobar}
%\renewcommand*{\thefoobar}{\thezpage-\arabic{foobar}}
% % or
%\renewcommand*{\thefoobar}{\roman{zpage}-\arabic{foobar}}
%\end{verbatim}
%\end{quote}
%
% \begin{declcs}{zunmakeperpage}
%   \M{counter}
% \end{declcs}
% The reset mechanism for this counter is deactivated.
%
%
% \subsection{Module \xmodule{counter}}
%
% This option just add the property |counter| to the main
% property list. The property stores the counter name,
% that was responsible for the reference. This is the
% property \xpackage{hyperref}'s \cs{autoref} feature
% uses. Thus this property |counter| may be useful
% for a reimplementation of the autoref feature,
% see the section \ref{sec:todo} with the todo list.
%
% \subsection{Module \xmodule{titleref}}
%
% This option makes section and caption titles available
% to the reference system similar to packages \xpackage{titleref}
% or \xpackage{nameref}.
%
% \begin{declcs}{ztitleref}
%   \M{refname}\markbabel
% \end{declcs}
% Print the section or caption title of reference \meta{refname},
% similar to \cs{nameref} or \cs{titleref}.
%
% \begin{declcs}{ztitlerefsetup}
%   |{|\textit{key$_1$=value$_1$, key$_2$=value$_2$, \dots}|}|
% \end{declcs}
% This command allows to configure the behaviour of module
% \xmodule{titleref}. The following keys are available:
% \begin{description}
% \let\bfseries\relax
% \item[\texttt{title=}\meta{value}]\mbox{}\\*
%   Sets the current title.
% \item[\texttt{stripperiod=true\string|false}]\mbox{}\\*
%   Follow package \xpackage{nameref} that
%   removes a last period. Default: |true|.
% \item[\texttt{expand=true\string|false}]\mbox{}\\*
%   Package \cs{titleref} expands the title first. This way
%   garbage and dangerous commands can be removed, e.g.
%   \cs{label}, \cs{index}\dots. See implementation section
%   for more details. Default is |false|.
% \item[\texttt{cleanup=\{}\textrm{\dots}\texttt{\}}]\mbox{}\\*
%   Hook to add own cleanup code, if method |expand| is used.
%   See implementation section for more details.
% \end{description}
%
% \subsection{Module \xmodule{savepos}}
%
% This option supports a feature that \pdfTeX\ provides
% (and \hologo{XeTeX}).
% \pdfTeX\ is able to tell the current position on the
% page. The page position is not instantly known. First the
% page must be constructed by \hologo{TeX}'s asynchronous output
% routine. Thus the time where the position is known is
% the page shipout time. Thus a reference system where the
% information is recorded in the first run and made available
% for use in the second run comes in handy.
%
% \begin{declcs}{zsavepos}
%   \M{refname}
% \end{declcs}
% It generates a reference with name \meta{refname}.
% The reference stores the location where \cs{zsavepos}
% is executed in properties |posx| and |posy|.
%
% \begin{declcs}{zsaveposx}
%   \M{refname}\\
%   \cs{zsaveposy}
%   \M{refname}
% \end{declcs}
% Same as \cs{zsavepos} except that only the |x| or |y| component
% of the position is stored. Since 2011/12/05 v2.23.
%
% \begin{declcs}{zposx}\markexp
%   \M{refname}\\
%   \SpecialUsageIndex{\zposy}\cs{zposy}\markexp \M{refname}
% \end{declcs}
% Get the position as number. Unit is sp. Horizontal
% positions by \cs{zposx} increase from left to right.
% Vertical positions by \cs{zposy} from bottom to top.
%
% Do not rely on absolute page numbers. Because of problems
% with the origin the numbers may differ in DVI or PDF mode
% of \pdfTeX. Therefore work with relative values by comparisons.
%
% Both \cs{zposx} and \cs{zposy} are expandable and can be
% used inside calculations (\cs{setcounter}, \cs{addtocounter},
% package \xpackage{calc}, \cs{numexpr}). However this property
% prevents from notifying \hologo{LaTeX} that the reference is actually
% used (the notifying is not expandable). Therefore you should
% mark the reference as used by \cs{zrefused}.
%
% This module uses \pdfTeX's \cs{pdfsavepos}, \cs{pdflastxpos},
% and \cs{pdflastypos}. They are available in PDF mode and
% since version 1.40.0 also in DVI mode.
%
% \begin{declcs}{zref@savepos}
% \end{declcs}
% Macro \cs{zref@savepos} performs the first part of \cs{zsavepos}
% by calling \cs{pdfsavepos} (if \xext{aux} files are writable).
%
% Thus \cs{zsavepos} is basically \cs{zref@savepos} followed
% by |\zref@labelbylist{|\texttt{\meta{refname}}|}{savepos}|.
% If \cs{TeXXeTstate} is detected and enabled, \cs{savepos}
% also adds \cs{zref@savepos} at the end to support \cs{beginR}
% where the whatits are processed in reverse order.
% The property list |savepos| contains the properties |posx|
% and |posy|.
%
% \subsection{Module \xmodule{abspos}}
% Module \xmodule{abspos} allows to get various values of the page layout. There is no user
% command, only a number of internal commands. For example:
% \begin{quote}
%   |\zref@absposx{|\meta{label}|}{|\meta{value}|}{|\meta{position}|}|\\
%   |\zref@absposy{|\meta{label}|}{|\meta{value}|}{|\meta{position}|}|
% \end{quote}
%
% The return value is like in the module \xmodule{savepos} a number representing a length in sp.
% The length are measured from the bottom left of the page.
%
% \meta{label} is a label set with \cs{zlabel} or \cs{zsavepos} that allows to retrieve the
% absolute page number.
%
% \meta{position} is for the x-command one of |left|, |right| or |center|. For the y-command
%  it is one of |top|, |bottom|, |center|.
%
% The possible content of \meta{value} can be seen in the following table. Be aware that the code
% makes some assumptions which are perhaps not always true -- for example that the left of
% the head is identical to the left of the body.
%
% \medskip
% \begin{tabular}{lll}
%  \textbf{value} & \textbf{axis} & \textbf{comments}\\\hline
%   media         & x             &   left=0, right=\cs{pdfpagewidth} \\
%   paper         & x             &   left=0, right=\cs{paperwidth}\\
%   stock         & x             &   derived from paper\\
%   media         & y             & bottom=0, top=\cs{pdfpageheigh}\\
%   paper         & y             & top=\cs{pdfpageheight}, bottom=top-\cs{paperheight}\\
%   stock         & y             & top  derived from paper\\
%   head          & x             & calculated with hoffset, horigin, etc\\
%   head          & y             & calculated\\
%   body          & x             & = head value\\%
%   body          & y             & = head bottom - \cs{headsep}\\
%   foot          & x             & = head\\
%   foot          & y             & calculated from body bottom and \cs{footskip}\\
%   marginpar     & x             &different on odd/even pages! \\
%   marginpar     & y             & = body %
% \end{tabular}
%
% \subsection{Module \xmodule{dotfill}}
%
% \begin{declcs}{zdotfill}
% \end{declcs}
% This package provides the command \cs{zdotfill} that works
% similar to \cs{dotfill}, but can be configured. Especially
% it suppresses the dots if a minimum number of dots cannot be set.
%
% \begin{declcs}{zdotfillsetup}
%   |{|\textit{key$_1$=value$_1$, key$_2$=value$_2$, \dots}|}|
% \end{declcs}
% This command allows to configure the behaviour of \cs{zdotfill}.
% The following keys are available:
% \begin{description}
% \let\bfseries\relax
% \item[\texttt{min=}\meta{count value}]\mbox{}\\*
%   If the actual number of dots are smaller than \meta{count value},
%   then the dots are suppressed. Default: |2|.
% \item[\texttt{unit=}\meta{dimen value}]\mbox{}\\*
%   The width of a dot unit is given by \meta{dimen value}.
%   Default: |0.44em| (same as the unit in \cs{dotfill}).
% \item[\texttt{dot=}\meta{value}]\mbox{}\\*
%   The dot itself is given by \meta{value}. Default: |.| (dot, same as
%   the dot in \cs{dotfill}).
% \end{description}
%
% \subsection{Module \xmodule{env}}
%
% This module defines two properties |envname| and |envline|. They
% remember the name of the environment and the line number at
% the start of the environment.
%
% \subsection{Module \xmodule{xr}}
%
% This package provides the functionality of package \xpackage{xr},
% see \cite{xr}. It also supports the syntax of \xpackage{xr-hyper}.
%
% \begin{declcs}{zexternaldocument}
%   \* |[|\meta{prefix}|]|\markbabel \M{external document}
%   |[|\meta{url}|]|
% \end{declcs}
% See \cs{externaldocument} for a description of this option.
% The found labels also get a property |externaldocument| that
% remembers \meta{external document}.
% The standard reference scheme and the scheme of this
% package use different name spaces for reference names.
% If the external document uses both systems. Then one
% import statement would put the names in one namespace
% and probably causing problems with multiple references
% of the same name. Thus the star form only looks for
% \cs{newlabel} in the \xfile{.aux} files, whereas without
% star only \cs{zref@newlabel}s are used.
%
% In the star form it tries to detect labels from
% \xpackage{hyperref}, \xpackage{titleref}, and \xpackage{ntheorem}.
% If such an extended property from the packages before cannot
% be found or are empty, they are not included in the
% imported reference.
%
% Warnings are given if a reference name is already in use
% and the item is ignored. Unknown properties will automatically
% be declared.
%
% If the external references contain |anchor| properties,
% then we need also a url to be able to address the external
% file. As default the filename is taken with a default
% extension.
%
% \begin{declcs}{zxrsetup}
%   |{|\textit{key$_1$=value$_1$, key$_2$=value$_2$, \dots}|}|
% \end{declcs}
% The following setup options are available:
% \begin{description}
% \item[\xoption{ext}:] It sets the default extension.
% \item[\xoption{tozreflabel}:] Boolean option. The found references
%   are imported as zref labels. This is enabled by default.
% \item[\xoption{toltxlabel}:] Boolean option. The found references
%   are imported as \hologo{LaTeX} labels. Packages \xpackage{nameref},
%   \xpackage{titleref} and class \xclass{memoir} are supported.
% \item[\xoption{urluse}:] Boolean option. If enabled, then a URL
%   is stored in a macro and the macro is put in property `urluse'.
%   The URL is not put in property `url'.
%   The purpose is to save \hologo{TeX} memory.
% \item[\xoption{verbose}:] Boolean option. List the imported labels in the
%   \xext{log} file. Default is |false|.
% \end{description}
%
% \begin{declcs}{zref@xr@ext}
% \end{declcs}
% If the \meta{url} is not specified in \cs{zref@externaldocument},
% then the url will be constructed with the file name and
% this macro as extension. \cs{XR@ext} is used if
% \xpackage{hyperref} is loaded, otherwise |pdf|.
%
% \subsection{Module \xmodule{pageattr}}
%
% This module allows to recover the content of the register
% \cs{pdfpageattr} and \cs{pdfpagesattr} in pdftex and the equivalent register
% in luatex. There is no user command. Programmer commands are
%
% \begin{quote}
% |\zref@pdfpageattr{|\meta{absolute page number}|}|
%
% |\zref@pdfpagesattr{|\meta{absolute page number}|}|
% \end{quote}
% \section{ToDo}\label{sec:todo}
%
% Among other things the following issues are left for future work:
% \begin{itemize}
% \item Other applications: autoref, hyperref, \dots
% \end{itemize}
%
% \section{Example}
%
%    \begin{macrocode}
%<*example>
%    \end{macrocode}
%
%    \begin{macrocode}
\documentclass{book}

\usepackage[ngerman]{babel}%

\usepackage[savepos,totpages,titleref,dotfill,counter,user]{zref}

%    \end{macrocode}
%
%    Chapters are wrapped inside \cs{ChapterStart} and
%    \cs{ChapterStop}. The first argument \verb|#1| of \cs{ChapterStart}
%    is used to form a label id \verb|chap:#1|. At the end of
%    the chapter another label is set by \cs{zref@wrapper@immediate}, because
%    otherwise at the end of document a deferred write would not
%    be written, because there is no page for shipout.
%
%    Also this example shows how chapter titles can be recorded.
%    A new property \verb|chaptitle| is declared and added to the
%    main property list. In \cs{ChapterStart} the current value
%    of the property is updated.
%    \begin{macrocode}
\makeatletter
\zref@newprop{chaptitle}{}
\zref@addprop{main}{chaptitle}

\newcommand*{\ChapterStart}[2]{%
  \cleardoublepage
  \def\current@chapid{#1}%
  \zref@setcurrent{chaptitle}{#2}%
  \chapter{#2}%
  \zlabel{chap:#1}%
}
\newcommand*{\ChapterStop}{%
  \cleardoublepage
  \zref@wrapper@immediate{%
    \zref@labelbyprops{chapend:\current@chapid}{abspage}%
  }%
}
%    \end{macrocode}
%    \cs{ChapterPages} calculates and returns the number of pages of the
%    referenced chapter.
%    \begin{macrocode}
\newcommand*{\ChapterPages}[1]{%
  \zrefused{chap:#1}%
  \zrefused{chapend:#1}%
  \number\numexpr
    \zref@extract{chapend:#1}{abspage}%
    -\zref@extract{chap:#1}{abspage}%
    +1\relax
}
\makeatother
%    \end{macrocode}
%    \begin{macrocode}
\begin{document}
%    \end{macrocode}
%    As exception we use \cs{makeatletter} here, because
%    this is just an example file that also should show some of
%    programmer's interface.
%    \begin{macrocode}
\makeatletter

\frontmatter
\zlabel{documentstart}

\begin{itemize}
\item
  The frontmatter part has
  \number\numexpr\zref@extract{chap:first}{abspage}-1\relax
  ~pages.
\item
  Chapter \zref{chap:first} has \ChapterPages{first} page(s).
\item
  Section \zref{hello} is on the
  \ifcase\numexpr
    \zref@extractdefault{hello}{page}{0}%
    -\zref@extractdefault{chap:first}{page}{0}%
    +1\relax
    ??\or first\or second\or third\or forth\fi
  ~page inside its chapter.
\item
  The document has
  \zref[abspage]{LastPage} pages.
  This number is \ifodd\ztotpages odd\else even\fi.
\item
  The last page is labeled with \zpageref{LastPage}.
\item
  The title of chapter \zref{chap:next} %
  is ``\zref[chaptitle]{chap:next}''.
\end{itemize}

\tableofcontents

\mainmatter
\ChapterStart{first}{First chapter}

%    \end{macrocode}
%    The user level commands should protect babel shorthands
%    where possible. On the other side, expandable extracting
%    macros are useful in calculations, see above the
%    examples with \cs{numexpr}.
%    \begin{macrocode}
\section{Test}
\zlabel{a"o}
Section \zref{a"o} on page
\zref@wrapper@babel\zref@extract{a"o}{page}.

Text.
\newpage

\section{Hello World}
\zlabel{hello}

\ChapterStop

\ChapterStart{next}{Next chapter with \emph{umlauts}: "a"o"u"s}

%    \end{macrocode}
%
%    Here an example follows that makes use of \pdfTeX's ``savepos''
%    feature. The position on the page is not known before the
%    page is constructed and shipped out. Therefore the position
%    ist stored in references and are available for calculations
%    in the next \hologo{LaTeX} compile run.
%    \begin{macrocode}
The width of the first column is
  \the\dimexpr \zposx{secondcol}sp - \zposx{firstcol}sp\relax,\\
the height difference of the two baselines is
  \the\dimexpr \zposy{firstcol}sp - \zposy{secondline}sp\relax:\\
\begin{tabular}{ll}
  \zsavepos{firstcol}Hello&\zsavepos{secondcol}World\\
  \zsavepos{secondline}Second line&foobar\\
\end{tabular}

%    \end{macrocode}
%    With \cs{zrefused} \hologo{LaTeX} is notified, if the references
%    are not yet available and \hologo{LaTeX} can generate the rerun
%    hint.
%    \begin{macrocode}
\zrefused{firstcol}
\zrefused{secondcol}
\zrefused{secondline}

\ChapterStop
%    \end{macrocode}
%    Test for module \cs{dotfill}.
%    \begin{macrocode}
\ChapterStart{dotfill}{Test for dotfill feature}
\newcommand*{\dftest}[1]{%
  #1&
  [\makebox[{#1}]{\dotfill}]&
  [\makebox[{#1}]{\zdotfill}]\\
}
\begin{tabular}{rll}
& [\verb|\dotfill|] & [\verb|\zdotfill|]\\
\dftest{0.43em}
\dftest{0.44em}
\dftest{0.45em}
\dftest{0.87em}
\dftest{0.88em}
\dftest{0.89em}
\dftest{1.31em}
\dftest{1.32em}
\dftest{1.33em}
\end{tabular}
\ChapterStop
\end{document}
%    \end{macrocode}
%
%    \begin{macrocode}
%</example>
%    \end{macrocode}
%
% \StopEventually{
% }
%
% \section{Implementation}
%
% \subsection{Package \xpackage{zref}}
%
% \subsubsection{Identification}
%
%    \begin{macrocode}
%<*package>
\NeedsTeXFormat{LaTeX2e}
\ProvidesPackage{zref}
  [2020-07-03 v2.32 A new reference scheme for LaTeX (HO)]%
%    \end{macrocode}
%
% \subsubsection{Load basic module}
%
%    \begin{macrocode}
\RequirePackage{zref-base}[2019/11/29]
%    \end{macrocode}
%    Abort package loading if \xpackage{zref-base} could not be
%    loaded successfully.
%    \begin{macrocode}
\@ifundefined{ZREF@base@ok}{\endinput}{}
%    \end{macrocode}
%
% \subsubsection{Process options}
%
%    Known modules are loaded and the release date is checked.
%    \begin{macrocode}
\def\ZREF@temp#1{%
  \DeclareOption{#1}{%
    \AtEndOfPackage{%
      \RequirePackage{zref-#1}[2019/11/29]%
    }%
  }%
}
\ZREF@temp{abspage}
\ZREF@temp{counter}
\ZREF@temp{dotfill}
\ZREF@temp{hyperref}
\ZREF@temp{lastpage}
\ZREF@temp{marks}
\ZREF@temp{nextpage}
\ZREF@temp{pageattr}
\ZREF@temp{pagelayout}
\ZREF@temp{perpage}
\ZREF@temp{runs}
\ZREF@temp{savepos}
\ZREF@temp{thepage}
\ZREF@temp{titleref}
\ZREF@temp{totpages}
\ZREF@temp{user}
\ZREF@temp{xr}
%    \end{macrocode}
%    \begin{macrocode}
\ProcessOptions\relax
%</package>
%    \end{macrocode}
%
%
% \subsection{Module \xmodule{base}}
%
% \subsubsection{Prefixes}
%
%    This package uses the following prefixes for macro names:
%    \begin{description}
%    \item[\cs{zref@}:] Macros of the programmer's interface.
%    \item[\cs{ZREF@}:] Internal macros.
%    \item[\cs{Z@L@}\metacs{listname}:]
%      The properties of the list \meta{listname}.
%    \item[\cs{Z@D@}\metacs{propname}:]
%      The default value for property \meta{propname}.
%    \item[\cs{Z@E@}\metacs{propname}:]
%      Extract function for property \meta{propname}.
%    \item[\cs{Z@X@}\metacs{propname}:]
%      Information whether a property value for property
%      \meta{propname} is expanded
%      immediately or at shipout time.
%    \item[\cs{Z@C@}\metacs{propname}:]
%      Current value of the property \meta{propname}.
%    \item[\cs{Z@R@}\metacs{labelname}:]
%      Data for reference \meta{labelname}.
%    \item[\cs{ZREF@org@}:] Original versions of patched commands.
%    \item[\cs{z}:] For macros in user land, defined if
%      module \xmodule{user} is set.
%    \end{description}
%    The following family names are used for keys defined
%    according to the \xpackage{keyval} package:
%    \begin{description}
%    \item[\texttt{ZREF@TR}:] Setup for module \xmodule{titleref}.
%    \end{description}
%
% \subsubsection{Identification}
%
%    \begin{macrocode}
%<*base>
\NeedsTeXFormat{LaTeX2e}
\ProvidesPackage{zref-base}%
  [2020-07-03 v2.32 Module base for zref (HO)]%
%    \end{macrocode}
%
% \subsubsection{Utilities}
%
%    \begin{macrocode}
\providecommand\IfFormatAtLeastTF{\@ifl@t@r\fmtversion}
\RequirePackage{ltxcmds}[2010/12/02]
\RequirePackage{infwarerr}[2010/04/08]
\RequirePackage{kvsetkeys}[2010/03/01]
\RequirePackage{kvdefinekeys}[2010/03/01]
\RequirePackage{pdftexcmds}[2010/04/01]
%    \end{macrocode}
%
%    \begin{macro}{\ZREF@name}
%    Several times the package name is used, thus we store it
%    in \cmd{\ZREF@name}.
%    \begin{macrocode}
\def\ZREF@name{zref}
%    \end{macrocode}
%    \end{macro}
%
%    \begin{macrocode}
\ltx@IfUndefined{protected}{%
  \RequirePackage{makerobust}[2006/03/18]%
%    \end{macrocode}
%    \begin{macro}{\ZREF@Robust}
%    \begin{macrocode}
  \def\ZREF@Robust#1#2{%
    \def\ZREF@temp{\MakeRobustcommand#2}%
    \afterassignment\ZREF@temp
    #1#2%
  }%
%    \end{macrocode}
%    \end{macro}
%    \begin{macrocode}
}{%
%    \end{macrocode}
%    \begin{macro}{\ZREF@Robust}
%    \begin{macrocode}
  \def\ZREF@Robust#1{%
    \protected#1%
  }%
%    \end{macrocode}
%    \end{macro}
%    \begin{macrocode}
}
%    \end{macrocode}
%    \begin{macro}{\ZREF@IfDefinable}
%    \begin{macrocode}
\def\ZREF@IfDefinable#1#2#3{%
  \@ifdefinable{#1}{%
    \ZREF@Robust{#2}#1#3%
  }%
}
%    \end{macrocode}
%    \end{macro}
%
%    \begin{macro}{\ZREF@UpdatePdfTeX}
%    \cs{ZREF@UpdatePdfTeX} is used as help message text in
%    error messages.
%    \begin{macrocode}
\def\ZREF@UpdatePdfTeX{Update pdfTeX.}
%    \end{macrocode}
%    \end{macro}
%
%    \begin{macro}{\ifZREF@found}
%    The following switch is usded in list processing.
%    \begin{macrocode}
\newif\ifZREF@found
%    \end{macrocode}
%    \end{macro}
%
%    \begin{macro}{\ZREF@patch}
%    Macro \cs{ZREF@patch} first checks the existence of
%    the command and safes it.
%    \begin{macrocode}
\def\ZREF@patch#1{%
  \ltx@IfUndefined{#1}{%
    \ltx@gobble
  }{%
    \expandafter\let\csname ZREF@org@#1\expandafter\endcsname
    \csname #1\endcsname
    \ltx@firstofone
  }%
}
%    \end{macrocode}
%    \end{macro}
%
% \subsubsection{Check for \hologo{eTeX}}
%
%    The use of \hologo{eTeX} should be standard nowadays for \hologo{LaTeX}.
%    We test for \hologo{eTeX} in order to use its features later.
%    \begin{macrocode}
\ltx@IfUndefined{eTeXversion}{%
  \PackageError\ZREF@name{%
    Missing support for eTeX; package is abandoned%
  }{%
    Use a TeX compiler that support eTeX and enable eTeX %
    in the format.%
  }%
  \endinput
}{}%
%    \end{macrocode}
%
%    \begin{macrocode}
\RequirePackage{etexcmds}[2007/09/09]
\ifetex@unexpanded
\else
  \PackageError\ZREF@name{%
    Missing e-TeX's \string\unexpanded.\MessageBreak
    Add \string\RequirePackage\string{etexcmds\string} before %
    \string\documentclass%
  }{%
    Probably you are using some package (e.g. ConTeXt) that %
    redefines \string\unexpanded%
  }%
  \expandafter\endinput
\fi
%    \end{macrocode}
%
% \subsubsection{Auxiliary file stuff}
%
%    We are using some commands in the \xfile{.aux} files.
%    However sometimes these auxiliary files are interpreted
%    by \hologo{LaTeX} processes that haven't loaded this package
%    (e.g. package \xpackage{xr}).
%    Therefore we provide dummy definitions.
%    \begin{macrocode}
\RequirePackage{auxhook}
\AddLineBeginAux{%
  \string\providecommand\string\zref@newlabel[2]{}%
}
%    \end{macrocode}
%    \begin{macro}{\ZREF@RefPrefix}
%    \begin{macrocode}
\def\ZREF@RefPrefix{Z@R}
%    \end{macrocode}
%    \end{macro}
%    \begin{macro}{\zref@newlabel}
%    For the implementation of \cs{zref@newlabel} we call
%    the same internal macro \cs{@newl@bel} that is used
%    in \cs{newlabel}. Thus we have for free:
%    \begin{itemize}
%    \item \cs{Z@R@}\metacs{labelname} is defined.
%    \item \hologo{LaTeX}'s check for multiple references.
%    \item \hologo{LaTeX}'s check for changed references.
%    \end{itemize}
%    \begin{macrocode}
\ZREF@Robust\edef\zref@newlabel{%
  \noexpand\@newl@bel{\ZREF@RefPrefix}%
}
%    \end{macrocode}
%    \end{macro}
%
% \subsubsection{Property lists}
%
%    \begin{macro}{\zref@newlist}
%    Property lists are stored as list of property names
%    enclosed in curly braces. \cs{zref@newlist}
%    creates a new list as empty list. Assignments to
%    property lists are global.
%    \begin{macrocode}
\ZREF@Robust\def\zref@newlist#1{%
  \zref@iflistundefined{#1}{%
    \@ifdefinable{Z@L@#1}{%
     \global\expandafter\let\csname Z@L@#1\endcsname\ltx@empty
     \PackageInfo\ZREF@name{New property list: #1}%
    }%
  }{%
    \PackageError\ZREF@name{%
      Property list `#1' already exists%
    }\@ehc
  }%
}
%    \end{macrocode}
%    \end{macro}
%
%    \begin{macro}{\zref@iflistundefined}
%    \cs{zref@iflistundefined} checks the existence of the
%    property list \verb|#1|. If the property list is present,
%    then \verb|#2| is executed and \verb|#3| otherwise.
%    \begin{macrocode}
\def\zref@iflistundefined#1{%
  \ltx@ifundefined{Z@L@#1}%
}
%    \end{macrocode}
%    \end{macro}
%    \begin{macro}{\zref@listexists}
%    \cs{zref@listexists} only executes \verb|#2| if the
%    property list \verb|#1| exists and raises an error
%    message otherwise.
%    \begin{macrocode}
\ZREF@Robust\def\zref@listexists#1{%
  \zref@iflistundefined{#1}{%
    \PackageError\ZREF@name{%
      Property list `#1' does not exist%
    }\@ehc
  }%
}
%    \end{macrocode}
%    \end{macro}
%
%    \begin{macro}{\zref@iflistcontainsprop}
%    \cs{zref@iflistcontainsprop} checks, whether a property \verb|#2| is
%    already present in a property list \verb|#1|.
%    \begin{macrocode}
\ZREF@Robust\def\zref@iflistcontainsprop#1#2{%
  \zref@iflistundefined{#1}{%
    \ltx@secondoftwo
  }{%
    \begingroup\expandafter\endgroup
    \expandafter\in@
    \csname#2\expandafter\expandafter\expandafter\endcsname
    \expandafter\expandafter\expandafter{\csname Z@L@#1\endcsname}%
    \csname ltx@\ifin@ first\else second\fi oftwo\endcsname
  }%
}
%    \end{macrocode}
%    \end{macro}
%
%    \begin{macro}{\zref@listforloop}
%    \begin{macrocode}
\def\zref@listforloop#1#2{%
  \zref@listexists{#1}{%
    \expandafter\expandafter\expandafter\@tfor
    \expandafter\expandafter\expandafter\zref@prop
    \expandafter\expandafter\expandafter:%
    \expandafter\expandafter\expandafter=%
    \csname Z@L@#1\endcsname
    \do{%
      \begingroup
        \escapechar=-1 %
        \edef\x{\endgroup
          \def\noexpand\zref@prop{%
            \expandafter\string\zref@prop
          }%
        }%
      \x
      #2\zref@prop
    }%
  }%
}
%    \end{macrocode}
%    \end{macro}
%
%    \begin{macro}{\zref@addprops}
%    \cs{zref@addprop} adds the properties \verb|#2| to the property
%    list \verb|#1|, if the property is not already in the list.
%    Otherwise a warning is given.
%    \begin{macrocode}
\ZREF@Robust\def\zref@addprops#1#2{%
  \zref@listexists{#1}{%
    \comma@parse{#2}{%
      \zref@propexists\comma@entry{%
        \zref@iflistcontainsprop{#1}\comma@entry{%
          \PackageWarning\ZREF@name{%
            Property `\comma@entry' is already in list `#1'%
          }%
        }{%
          \begingroup\expandafter\endgroup
          \expandafter\g@addto@macro
          \csname Z@L@#1\expandafter\endcsname
          \expandafter{\csname\comma@entry\endcsname}%
        }%
      }%
      \ltx@gobble
    }%
  }%
}
%    \end{macrocode}
%    \end{macro}
%    \begin{macro}{\zref@addprop}
%    \cs{zref@addprop} adds the property \verb|#2| to the property
%    list \verb|#1|, if the property is not already in the list.
%    Otherwise a warning is given.
%    \begin{macrocode}
\ZREF@Robust\def\zref@addprop#1#2{%
  \zref@listexists{#1}{%
    \zref@propexists{#2}{%
      \zref@iflistcontainsprop{#1}{#2}{%
        \PackageWarning\ZREF@name{%
          Property `#2' is already in list `#1'%
        }%
      }{%
        \begingroup\expandafter\endgroup
        \expandafter\g@addto@macro
        \csname Z@L@#1\expandafter\endcsname
        \expandafter{\csname#2\endcsname}%
      }%
    }%
  }%
}
%    \end{macrocode}
%    \end{macro}
%
%    \begin{macro}{\zref@localaddprops}
%    \begin{macrocode}
\ZREF@Robust\def\zref@localaddprops#1#2{%
  \zref@listexists{#1}{%
    \comma@parse{#2}{%
      \zref@propexists\comma@entry{%
        \zref@iflistcontainsprop{#1}\comma@entry{%
          \PackageWarning\ZREF@name{%
            Property `\comma@entry' is already in list `#1'%
          }%
        }{%
          \begingroup\expandafter\endgroup
          \expandafter\ltx@LocalAppendToMacro
          \csname Z@L@#1\expandafter\endcsname
          \expandafter{\csname\comma@entry\endcsname}%
        }%
      }%
      \ltx@gobble
    }%
  }%
}
%    \end{macrocode}
%    \end{macro}
%    \begin{macro}{\zref@localaddprop}
%    \begin{macrocode}
\ZREF@Robust\def\zref@localaddprop#1#2{%
  \zref@listexists{#1}{%
    \zref@propexists{#2}{%
      \zref@iflistcontainsprop{#1}{#2}{%
        \PackageWarning\ZREF@name{%
          Property `#2' is already in list `#1'%
        }%
      }{%
        \begingroup\expandafter\endgroup
        \expandafter\ltx@LocalAppendToMacro
        \csname Z@L@#1\expandafter\endcsname
        \expandafter{\csname#2\endcsname}%
      }%
    }%
  }%
}
%    \end{macrocode}
%    \end{macro}
%
%    \begin{macrocode}
\ltx@IfUndefined{pdf@strcmp}{%
%    \end{macrocode}
%    \begin{macro}{\zref@delprop}
%    \begin{macrocode}
  \ZREF@Robust\def\zref@delprop{%
    \ZREF@delprop\gdef
  }%
%    \end{macrocode}
%    \end{macro}
%    \begin{macro}{\zref@localdelprop}
%    \begin{macrocode}
  \ZREF@Robust\def\zref@localdelprop{%
    \ZREF@delprop\def
  }%
%    \end{macrocode}
%    \end{macro}
%    \begin{macro}{\ZREF@delprop}
%    \begin{macrocode}
  \def\ZREF@delprop#1#2#3{%
    \zref@listexists{#2}{%
      \begingroup
        \escapechar=-1 %
        \def\ZREF@param{#3}%
        \@onelevel@sanitize\ZREF@param
        \toks@{}%
        \expandafter\expandafter\expandafter\ZREF@@delprop
        \csname Z@L@#2\endcsname!%
      \expandafter\endgroup
      \expandafter#1\csname Z@L@#2\expandafter\endcsname
      \expandafter{%
        \the\toks@
      }%
    }%
  }%
%    \end{macrocode}
%    \end{macro}
%    \begin{macro}{\ZREF@@delprop}
%    \begin{macrocode}
  \def\ZREF@@delprop#1{%
    \expandafter\ZREF@@@delprop\expandafter{\string#1}#1%
  }%
%    \end{macrocode}
%    \end{macro}
%    \begin{macro}{\ZREF@@@delprop}
%    \begin{macrocode}
  \def\ZREF@@@delprop#1#2{%
    \ifx#2!%
    \else
      \def\ZREF@temp{#1}%
      \@onelevel@sanitize\ZREF@temp
      \ifx\ZREF@param\ZREF@temp
      \else
        \toks@\expandafter{%
          \the\expandafter\toks@\csname#1\endcsname
        }%
      \fi
      \expandafter\ZREF@@delprop
    \fi
  }%
%    \end{macrocode}
%    \end{macro}
%    \begin{macrocode}
}{%
%    \end{macrocode}
%    \begin{macro}{\zref@delprop}
%    \begin{macrocode}
  \ZREF@Robust\def\zref@delprop{%
    \ZREF@delprop\xdef
  }%
%    \end{macrocode}
%    \end{macro}
%    \begin{macro}{\zref@localdelprop}
%    \begin{macrocode}
  \ZREF@Robust\def\zref@localdelprop{%
    \ZREF@delprop\edef
  }%
%    \end{macrocode}
%    \end{macro}
%    \begin{macro}{\ZREF@delprop}
%    \begin{macrocode}
  \def\ZREF@delprop#1#2#3{%
    \zref@listexists{#2}{%
      \def\ZREF@param{#3}%
      \edef\ZREF@SavedEscapechar{\the\escapechar}%
      \escapechar=-1 %
      \expandafter#1\csname Z@L@#2%
      \expandafter\expandafter\expandafter\endcsname{%
        \expandafter\expandafter\expandafter\ZREF@@delprop
        \csname Z@L@#2\endcsname!%
      }%
      \escapechar=\ZREF@SavedEscapechar\relax
    }%
  }%
%    \end{macrocode}
%    \end{macro}
%    \begin{macro}{\ZREF@@delprop}
%    Caution: |#1| might be an \cs{if} or similar token.
%    \begin{macrocode}
  \def\ZREF@@delprop#1{%
    \expandafter\ZREF@@@delprop\expandafter{\string#1}#1%
  }%
%    \end{macrocode}
%    \end{macro}
%    \begin{macro}{\ZREF@@@delprop}
%    \begin{macrocode}
  \def\ZREF@@@delprop#1#2{%
    \ifx#2!%
    \else
      \ifnum\pdf@strcmp{#1}{\ZREF@param}=\ltx@zero
      \else
        \expandafter\noexpand\csname#1\endcsname
      \fi
      \expandafter\ZREF@@delprop
    \fi
  }%
%    \end{macrocode}
%    \end{macro}
%    \begin{macrocode}
}
%    \end{macrocode}
%
% \subsubsection{Properties}
%
%    \begin{macro}{\zref@ifpropundefined}
%    \cs{zref@ifpropundefined} checks the existence of the
%    property \verb|#1|. If the property is present,
%    then \verb|#2| is executed and \verb|#3| otherwise.
%    \begin{macrocode}
\def\zref@ifpropundefined#1{%
  \ltx@ifundefined{Z@E@#1}%
}
%    \end{macrocode}
%    \end{macro}
%    \begin{macro}{\zref@propexists}
%    Some macros rely on the existence of a property.
%    \cs{zref@propexists} only executes \verb|#2| if the
%    property \verb|#1| exists and raises an error
%    message otherwise.
%    \begin{macrocode}
\ZREF@Robust\def\zref@propexists#1{%
  \zref@ifpropundefined{#1}{%
    \PackageError\ZREF@name{%
      Property `#1' does not exist%
    }\@ehc
  }%
}
%    \end{macrocode}
%    \end{macro}
%
%    \begin{macro}{\zref@newprop}
%    A new property is declared by \cs{zref@newprop}, the
%    property name \meta{propname} is given in \verb|#1|. The
%    property is created and configured. If the star form
%    is given, then the expansion of the property value
%    is delayed to page shipout time, when the reference
%    is written to the \xfile{.aux} file.
%    \begin{description}
%    \item[\cs{Z@D@}\metacs{propname}:]
%      Stores the default value for this property.
%    \item[\cs{Z@E@}\metacs{propname}:]
%      Extract function.
%    \item[\cs{Z@X@}\metacs{propname}:]
%      Information whether the expansion of the property value
%      is delayed to shipout time.
%    \item[\cs{Z@C@}\metacs{propname}:]
%      Current value of the property.
%    \end{description}
%    \begin{macrocode}
\ZREF@Robust\def\zref@newprop{%
  \@ifstar{%
    \let\ZREF@X\noexpand
    \ZREF@newprop
  }{%
    \let\ZREF@X\ltx@empty
    \ZREF@newprop
  }%
}
%    \end{macrocode}
%    \end{macro}
%    \begin{macro}{\ZREF@newprop}
%    \begin{macrocode}
\def\ZREF@newprop#1{%
  \edef\ZREF@P{#1}%
  \@onelevel@sanitize\ZREF@P
  \begingroup
  \ifx\ZREF@P\ZREF@par
    \@PackageError\ZREF@name{%
      Invalid property name `\ZREF@P'%
    }{%
      The property name `par' is not allowed %
      because of internal reasons.%
      \MessageBreak
      \@ehc
    }%
    \def\ZREF@@newprop[##1]##2{\endgroup}%
  \else
    \zref@ifpropundefined\ZREF@P{%
      \endgroup
      \PackageInfo\ZREF@name{%
        New property: \ZREF@P
      }%
    }{%
      \@PackageError\ZREF@name{%
        Property `\ZREF@P' already exists%
      }\@ehc
      \def\ZREF@@newprop[##1]##2{\endgroup}%
    }%
  \fi
  \@ifnextchar[\ZREF@@newprop{\ZREF@@newprop[\zref@default]}%
}
%    \end{macrocode}
%    \end{macro}
%    \begin{macro}{\ZREF@par}
%    \begin{macrocode}
\def\ZREF@par{par}
\@onelevel@sanitize\ZREF@par
%    \end{macrocode}
%    \end{macro}
%    \begin{macro}{\ZREF@@newprop}
%    \begin{macrocode}
\def\ZREF@@newprop[#1]{%
  \global\@namedef{Z@D@\ZREF@P}{#1}%
  \global\expandafter\let\csname Z@X@\ZREF@P\endcsname\ZREF@X
  \begingroup\expandafter\endgroup
  \expandafter\ZREF@@@newprop\csname\ZREF@P\endcsname
  \expandafter\gdef\csname Z@C@\ZREF@P\endcsname{}%
  \zref@setcurrent\ZREF@P
}
\def\ZREF@@@newprop#1{%
  \expandafter
  \gdef\csname Z@E@\ZREF@P\endcsname##1#1##2##3\ZREF@nil{##2}%
}
%    \end{macrocode}
%    \end{macro}
%
%    \begin{macro}{\zref@showprop}
%    \begin{macrocode}
\ZREF@Robust\def\zref@showprop#1{%
  \zref@ifpropundefined{#1}{%
    \@PackageInfoNoLine{\ZREF@name}{%
      Show property `#1': <undefined>%
    }%
  }{%
    \begingroup
      \toks@\expandafter\expandafter\expandafter{%
        \csname Z@C@#1\endcsname
      }%
      \edef\ZREF@value{\the\toks@}%
      \ltx@onelevel@sanitize\ZREF@value
      \toks@\expandafter\expandafter\expandafter{%
        \csname Z@D@#1\endcsname
      }%
      \edef\ZREF@default{\the\toks@}%
      \ltx@onelevel@sanitize\ZREF@default
      \@PackageInfoNoLine{\ZREF@name}{%
        Show property `#1':\MessageBreak
        \expandafter\ifx\csname Z@X@#1\endcsname\ltx@empty
          Immediate %
        \else
          Delayed %
        \fi
        value: [\ZREF@value]\MessageBreak
        Default: [\ZREF@default]%
      }%
    \endgroup
  }%
}
%    \end{macrocode}
%    \end{macro}
%
%    \begin{macro}{\zref@setcurrent}
%    \cs{zref@setcurrent} sets the current value for a property.
%    \begin{macrocode}
\ZREF@Robust\def\zref@setcurrent#1#2{%
  \zref@propexists{#1}{%
    \expandafter\def\csname Z@C@#1\endcsname{#2}%
  }%
}
%    \end{macrocode}
%    \end{macro}
%
%    \begin{macro}{\ZREF@getcurrent}
%    \cs{zref@getcurrent} gets the current value for a property.
%    \begin{macrocode}
\def\ZREF@getcurrent#1{%
  \romannumeral0%
  \ltx@ifundefined{Z@C@#1}{%
    \ltx@space
  }{%
    \expandafter\expandafter\expandafter\ltx@space
    \csname Z@C@#1\endcsname
  }%
}
%    \end{macrocode}
%    \end{macro}
%    \begin{macro}{\ZREF@u@getcurrent}
%    \begin{macrocode}
\def\ZREF@wu@getcurrent#1{%
  \etex@unexpanded\expandafter\expandafter\expandafter{%
    \ZREF@getcurrent{#1}%
  }%
}
%    \end{macrocode}
%    \end{macro}
%    \begin{macro}{\zref@getcurrent}
%    \begin{macrocode}
\let\zref@getcurrent\ZREF@getcurrent
%    \end{macrocode}
%    \end{macro}
%
% \subsubsection{Reference generation}
%
%    \begin{macro}{\zref@label}
%    Label macro that uses the main property list.
%    \begin{macrocode}
\ZREF@Robust\def\zref@label#1{%
  \zref@labelbylist{#1}\ZREF@mainlist
}
%    \end{macrocode}
%    \end{macro}
%    \begin{macro}{\zref@labelbylist}
%    Label macro that stores the properties, specified in the
%    property list \verb|#2|.
%    \begin{macrocode}
\ZREF@Robust\def\zref@labelbylist#1#2{%
  \@bsphack
    \zref@listexists{#2}{%
      \expandafter\expandafter\expandafter\ZREF@label
      \expandafter\expandafter\expandafter{%
        \csname Z@L@#2\endcsname
      }{#1}%
    }%
  \@esphack
}
%    \end{macrocode}
%    \end{macro}
%    \begin{macro}{\zref@labelbyprops}
%    The properties are directly specified in a comma separated list.
%    \begin{macrocode}
\ZREF@Robust\def\zref@labelbyprops#1#2{%
  \@bsphack
    \begingroup
      \toks@{}%
      \comma@parse{#2}{%
        \zref@ifpropundefined\comma@entry{%
          \PackageWarning\ZREF@name{%
            Property `\comma@entry' is not known%
          }%
        }{%
          \toks@\expandafter{%
            \the\expandafter\toks@\csname\comma@entry\endcsname
          }%
        }%
        \ltx@gobble
      }%
    \expandafter\endgroup
    \expandafter\ZREF@label\expandafter{\the\toks@}{#1}%
  \@esphack
}
%    \end{macrocode}
%    \end{macro}
%    \begin{macro}{\zref@labelbykv}
%    \begin{macrocode}
\ZREF@Robust\def\zref@labelbykv#1#2{%
  \@bsphack
    \begingroup
      \let\Z@L@ZREF@temp\ltx@empty
      \kvsetkeys{ZREF@LABEL}{#1}%
      \ifZREF@immediate
        \expandafter\zref@wrapper@immediate\expandafter{%
          \expandafter\ZREF@label\expandafter{\Z@L@ZREF@temp}{#2}%
        }%
      \else
        \expandafter\ZREF@label\expandafter{\Z@L@ZREF@temp}{#2}%
      \fi
    \endgroup
  \@esphack
}
%    \end{macrocode}
%    \end{macro}
%    \begin{macrocode}
\kv@define@key{ZREF@LABEL}{prop}{%
  \edef\ZREF@param{#1}%
  \zref@propexists\ZREF@param{%
    \zref@iflistcontainsprop{ZREF@temp}\ZREF@param{}{%
      \begingroup\expandafter\endgroup
      \expandafter\ltx@LocalAppendToMacro
      \expandafter\Z@L@ZREF@temp
      \expandafter{\csname\ZREF@param\endcsname}%
    }%
  }%
}
\kv@define@key{ZREF@LABEL}{list}{%
  \zref@listforloop{#1}{%
    \zref@iflistcontainsprop{ZREF@temp}\zref@prop{}{%
      \begingroup\expandafter\endgroup
      \expandafter\ltx@LocalAppendToMacro
      \expandafter\Z@L@ZREF@temp
      \expandafter{\csname\zref@prop\endcsname}%
    }%
    \ltx@gobble
  }%
}
\kv@define@key{ZREF@LABEL}{delprop}{%
  \zref@propexists{#1}{%
    \zref@localdelprop{ZREF@temp}{#1}%
  }%
}
\kv@define@key{ZREF@LABEL}{immediate}[true]{%
  \edef\ZREF@param{#1}%
  \ifx\ZREF@param\ZREF@true
    \ZREF@immediatetrue
  \else
    \ifx\ZREF@param\ZREF@false
      \ZREF@immediatefalse
    \else
      \PackageWarning\ZREF@name{%
        Option `immediate' expects `true' or `false'.\MessageBreak
        Ignoring invalid value `\ZREF@param'%
      }%
    \fi
  \fi
}
%    \end{macrocode}
%    \begin{macro}{\ZREF@false}
%    \begin{macrocode}
\def\ZREF@false{false}
%    \end{macrocode}
%    \end{macro}
%    \begin{macro}{\ZREF@true}
%    \begin{macrocode}
\def\ZREF@true{true}
%    \end{macrocode}
%    \end{macro}
%
%    \begin{macrocode}
\kv@define@key{ZREF@LABEL}{values}[]{%
  \kv@parse{#1}{%
    \ifx\kv@value\relax
      \@PackageWarning\ZREF@name{%
        Missing value for property `\kv@key'%
      }%
      \expandafter\ltx@gobbletwo
    \else
      \expandafter\zref@setcurrent
    \fi
  }%
}
%    \end{macrocode}
%
%    \begin{macro}{\ifZREF@immediate}
%    The switch \cs{ifZREF@immediate} tells us, whether the
%    label should be written immediately or at page shipout time.
%    \cs{ZREF@label} need to be notified about this, because
%    it must disable the deferred execution of property values,
%    if the label is written immediately.
%    \begin{macrocode}
\newif\ifZREF@immediate
%    \end{macrocode}
%    \end{macro}
%    \begin{macro}{\zref@wrapper@immediate}
%    The argument of \cs{zref@wrapper@immediate} is executed inside
%    a group where \cs{write} is redefined by adding
%    \cs{immediate} before its execution. Also \cs{ZREF@label}
%    is notified via the switch \cs{ifZREF@immediate}.
%    \begin{macrocode}
\ZREF@Robust{\long\def}\zref@wrapper@immediate#1{%
  \begingroup
    \ZREF@immediatetrue
    \let\ZREF@org@write\write
    \def\write{\immediate\ZREF@org@write}%
    #1%
  \endgroup
}
%    \end{macrocode}
%    \end{macro}
%
%    \begin{macro}{\ZREF@label}
%    \cs{ZREF@label} writes the data in the \xfile{.aux} file.
%    \verb|#1| contains the list of valid properties,
%    \verb|#2| the name of the reference. In case of immediate
%    writing, the deferred execution of property values is
%    disabled. Also \thepage is made expandable in this case.
%    \begin{macrocode}
\def\ZREF@label#1#2{%
  \if@filesw
    \begingroup
      \ifZREF@immediate
        \let\ZREF@org@thepage\thepage
      \fi
      \protected@write\@auxout{%
        \ifZREF@immediate
          \let\thepage\ZREF@org@thepage
        \fi
        \let\ZREF@temp\ltx@empty
        \@tfor\ZREF@P:=#1\do{%
          \begingroup
            \escapechar=-1 %
            \edef\x{\endgroup
              \def\noexpand\ZREF@P{%
                \expandafter\string\ZREF@P
              }%
            }%
          \x
          \expandafter\ifx
              \csname
                \ifZREF@immediate
                  relax%
                \else
                  Z@X@\ZREF@P%
                \fi
              \endcsname
              \noexpand
            \expandafter\let\csname Z@C@\ZREF@P\endcsname\relax
          \fi
          \toks@\expandafter{\ZREF@temp}%
          \edef\ZREF@temp{%
            \the\toks@
            \ltx@backslashchar\ZREF@P{%
              \expandafter\noexpand\csname Z@C@\ZREF@P\endcsname
            }%
          }%
        }%
      }{%
        \string\zref@newlabel{#2}{\ZREF@temp}%
      }%
    \endgroup
  \fi
}
\def\ZREF@addtoks#1{%
  \toks@\expandafter\expandafter\expandafter{%
    \expandafter\the\expandafter\toks@#1%
  }%
}
%    \end{macrocode}
%    \end{macro}
%
% \subsubsection{Reference querying and extracting}
%
%    Design goal for the extracting macros is that the extraction
%    process is full expandable. Thus these macros can be used
%    in expandable contexts. But there are problems that cannot
%    be solved by full expandable macros:
%    \begin{itemize}
%    \item In standard \hologo{LaTeX} undefined references sets a flag
%      and generate a warning. Both actions are not expandable.
%    \item Babel's support for its shorthand uses commands that
%      use non-expandable assignments. However currently there
%      is hope, that primitives are added to \pdfTeX\ that
%      allows the detection of contexts. Then the shorthand
%      can detect, if they are executed inside \cs{csname}
%      and protect themselves automatically.
%    \end{itemize}
%    \begin{macro}{\zref@ifrefundefined}
%    If a reference \verb|#1| is undefined, then macro
%    \cs{zref@ifrefundefined} calls \verb|#2| and \verb|#3| otherwise.
%    \begin{macrocode}
\def\zref@ifrefundefined#1{%
  \ltx@ifundefined{Z@R@#1}%
}
%    \end{macrocode}
%    \end{macro}
%
%    \begin{macro}{\zifrefundefined}
%    If a reference \verb|#1| is undefined, then macro
%    \cs{zref@ifrefundefined} calls \verb|#2| and \verb|#3| otherwise.
%    Also the reference is marked used.
%    \begin{macrocode}
\ZREF@IfDefinable\zifrefundefined\def{%
  #1{%
    \zref@wrapper@babel\ZREF@ifrefundefined{#1}%
  }%
}
%    \end{macrocode}
%    \end{macro}
%    \begin{macro}{\ZREF@ifrefundefined}
%    \begin{macrocode}
\def\ZREF@ifrefundefined#1{%
  \zref@refused{#1}%
  \zref@ifrefundefined{#1}%
}
%    \end{macrocode}
%    \end{macro}
%
%    \begin{macro}{\zref@refused}
%    The problem with undefined references is addressed by
%    the macro \cs{zref@refused}. This can be used outside
%    the expandable context. In case of an undefined reference
%    the flag is set to notify \hologo{LaTeX} and a warning is given.
%    \begin{macrocode}
\ZREF@Robust\def\zref@refused#1{%
  \zref@wrapper@babel\ZREF@refused{#1}%
}
%    \end{macrocode}
%    \end{macro}
%    \begin{macro}{\ZREF@refused}
%    \begin{macrocode}
\def\ZREF@refused#1{%
  \zref@ifrefundefined{#1}{%
    \protect\G@refundefinedtrue
    \@latex@warning{%
      Reference `#1' on page \thepage \space undefined%
    }%
  }{}%
}
%    \end{macrocode}
%    \end{macro}
%
%    \begin{macro}{\zref@ifrefcontainsprop}
%    \cs{zref@ifrefcontainsprop} looks, if the reference \verb|#1|
%    has the property \verb|#2| and calls then \verb|#3| and
%    \verb|#4| otherwise.
%    \begin{macrocode}
\def\zref@ifrefcontainsprop#1#2{%
  \zref@ifrefundefined{#1}{%
    \ltx@secondoftwo
  }{%
    \expandafter\ZREF@ifrefcontainsprop
    \csname Z@E@#2\expandafter\endcsname
    \csname#2\expandafter\expandafter\expandafter\endcsname
    \expandafter\expandafter\expandafter{%
      \csname Z@R@#1\endcsname
    }%
  }%
}
\def\ZREF@ifrefcontainsprop#1#2#3{%
  \expandafter\ifx\expandafter\ZREF@novalue
  #1#3#2\ZREF@novalue\ZREF@nil\ltx@empty
    \expandafter\ltx@secondoftwo
  \else
    \expandafter\ltx@firstoftwo
  \fi
}
\def\ZREF@novalue{\ZREF@NOVALUE}
%    \end{macrocode}
%    \end{macro}
%
%    \begin{macro}{\zref@extract}
%    \cs{zref@extract} is an abbreviation for the case
%    that the default of the property is used as default
%    value.
%    \begin{macrocode}
\def\ZREF@extract#1#2{%
  \romannumeral0%
  \ltx@ifundefined{Z@D@#2}{%
    \expandafter\ltx@space\zref@default
  }{%
    \expandafter\expandafter\expandafter\ZREF@@extract
    \expandafter\expandafter\expandafter{%
      \csname Z@D@#2\endcsname
    }{#1}{#2}%
  }%
}
%    \end{macrocode}
%    \end{macro}
%    \begin{macro}{\ZREF@@extract}
%    \begin{macrocode}
\def\ZREF@@extract#1#2#3{%
  \expandafter\expandafter\expandafter\ltx@space
  \zref@extractdefault{#2}{#3}{#1}%
}
%    \end{macrocode}
%    \end{macro}
%    \begin{macro}{\ZREF@wu@extract}
%    \begin{macrocode}
\def\ZREF@wu@extract#1#2{%
  \etex@unexpanded\expandafter\expandafter\expandafter{%
    \ZREF@extract{#1}{#2}%
  }%
}
%    \end{macrocode}
%    \end{macro}
%    \begin{macro}{\zref@extract}
%    \begin{macrocode}
\let\zref@extract\ZREF@extract
%    \end{macrocode}
%    \end{macro}
%
%    \begin{macro}{\ZREF@extractdefault}
%    The basic extracting macro is \cs{zref@extractdefault} with
%    the reference name in \verb|#1|, the property in \verb|#2| and
%    the default value in \verb|#3| in case for problems.
%    \begin{macrocode}
\def\ZREF@extractdefault#1#2#3{%
  \romannumeral0%
  \zref@ifrefundefined{#1}\ltx@firstoftwo{%
    \zref@ifpropundefined{#2}\ltx@firstoftwo\ltx@secondoftwo
  }{%
    \ltx@space
    #3%
  }{%
    \expandafter\expandafter\expandafter\ltx@space
    \csname Z@E@#2\expandafter\expandafter\expandafter\endcsname
    \csname Z@R@#1\expandafter\endcsname
    \csname#2\endcsname{#3}\ZREF@nil
  }%
}
%    \end{macrocode}
%    \end{macro}
%    \begin{macro}{\ZREF@wu@extractdefault}
%    \begin{macrocode}
\def\ZREF@wu@extractdefault#1#2#3{%
  \etex@unexpanded\expandafter\expandafter\expandafter{%
    \ZREF@extractdefault{#1}{#2}{#3}%
  }%
}
%    \end{macrocode}
%    \end{macro}
%    \begin{macro}{\zref@extractdefault}
%    \begin{macrocode}
\let\zref@extractdefault\ZREF@extractdefault
%    \end{macrocode}
%    \end{macro}
%
%    \begin{macro}{\zref@def@extract}
%    \begin{macrocode}
\ZREF@Robust\def\zref@def@extract#1{%
  \zref@wrapper@babel{\ZREF@def@extract{#1}}%
}
%    \end{macrocode}
%    \end{macro}
%    \begin{macro}{\ZREF@def@extract}
%    \begin{macrocode}
\def\ZREF@def@extract#1#2#3{%
  \zref@refused{#2}%
  \expandafter\expandafter\expandafter\def
  \expandafter\expandafter\expandafter#1%
  \expandafter\expandafter\expandafter{%
    \zref@extract{#2}{#3}%
  }%
}
%    \end{macrocode}
%    \end{macro}
%    \begin{macro}{\zref@def@extractdefault}
%    \begin{macrocode}
\ZREF@Robust\def\zref@def@extractdefault#1{%
  \zref@wrapper@babel{\ZREF@def@extractdefault{#1}}%
}
%    \end{macrocode}
%    \end{macro}
%    \begin{macro}{\ZREF@def@extractdefault}
%    \begin{macrocode}
\def\ZREF@def@extractdefault#1#2#3#4{%
  \zref@refused{#2}%
  \expandafter\expandafter\expandafter\def
  \expandafter\expandafter\expandafter#1%
  \expandafter\expandafter\expandafter{%
    \zref@extractdefault{#2}{#3}{#4}%
  }%
}
%    \end{macrocode}
%    \end{macro}
%
%    \begin{macro}{\ZREF@wrapper@unexpanded}
%    \begin{macrocode}
\ZREF@Robust{\long\def}\ZREF@wrapper@unexpanded#1{%
  \let\zref@wrapper@unexpanded\ltx@firstofone
  \let\zref@getcurrent\ZREF@wu@getcurrent
  \let\zref@extractdefault\ZREF@wu@extractdefault
  \let\zref@extract\ZREF@wu@extract
  #1%
  \let\zref@wrapper@unexpanded\ZREF@wrapper@unexpanded
  \let\zref@getcurrent\ZREF@getcurrent
  \let\zref@extractdefault\ZREF@extractdefault
  \let\zref@extract\ZREF@extract
}
%    \end{macrocode}
%    \end{macro}
%    \begin{macro}{\zref@wrapper@unexpanded}
%    \begin{macrocode}
\ltx@IfUndefined{etex@unexpanded}{%
  \let\zref@wrapper@unexpanded\ltx@firstofone
}{%
  \let\zref@wrapper@unexpanded\ZREF@wrapper@unexpanded
}
%    \end{macrocode}
%    \end{macro}
%
% \subsubsection{Compatibility with \xpackage{babel}}
%
%    \begin{macro}{\zref@wrapper@babel}
%    \begin{macrocode}
\ZREF@Robust{\long\def}\zref@wrapper@babel#1#2{%
  \ifcsname if@safe@actives\endcsname
    \expandafter\ltx@firstofone
  \else
    \expandafter\ltx@secondoftwo
  \fi
  {%
    \if@safe@actives
      \expandafter\ltx@secondoftwo
    \else
      \expandafter\ltx@firstoftwo
    \fi
    {%
      \begingroup
        \csname @safe@activestrue\endcsname
        \edef\x{#2}%
      \expandafter\endgroup
      \expandafter\ZREF@wrapper@babel\expandafter{\x}{#1}%
    }%
  }{%
    #1{#2}%
  }%
}
\long\def\ZREF@wrapper@babel#1#2{%
  #2{#1}%
}
%    \end{macrocode}
%    \end{macro}
%
% \subsubsection{Unique counter support}
%
%    \begin{macro}{\zref@require@unique}
%    Generate the counter |zref@unique| if the counter does not
%    already exist.
%    \begin{macrocode}
\ZREF@Robust\def\zref@require@unique{%
  \@ifundefined{c@zref@unique}{%
    \begingroup
      \let\@addtoreset\ltx@gobbletwo
      \newcounter{zref@unique}%
    \endgroup
%    \end{macrocode}
%    \end{macro}
%
%    \begin{macro}{\thezref@unique}
%    \cs{thezref@unique} is used for automatically generated unique
%    labelnames.
%    \begin{macrocode}
    \renewcommand*{\thezref@unique}{%
      zref@\number\c@zref@unique
    }%
  }{}%
}
%    \end{macrocode}
%    \end{macro}
%
% \subsubsection{Utilities}
%
%    \begin{macro}{\ZREF@number}
%    \begin{macrocode}
\ltx@IfUndefined{numexpr}{%
  \def\ZREF@number#1{\number#1}%
}{%
  \def\ZREF@number#1{\the\numexpr(#1)\relax}%
}
%    \end{macrocode}
%    \end{macro}
%
% \subsubsection{Setup}
%
%    \begin{macro}{\zref@setdefault}
%    Standard \hologo{LaTeX} prints ``??'' in bold face if a reference
%    is not known. \cs{zref@default} holds the text that is
%    printed in case of unknown references and is used, if
%    the default was not specified during the definition
%    of the new property by \cs{ref@newprop}.
%    The global default value can be set by \cs{zref@setdefault}.
%    \begin{macrocode}
\ZREF@Robust\def\zref@setdefault#1{%
  \def\zref@default{#1}%
}
%    \end{macrocode}
%    \end{macro}
%    \begin{macro}{\zref@default}
%    Now we initialize \cs{zref@default} with the same value
%    that \hologo{LaTeX} uses for its undefined references.
%    \begin{macrocode}
\zref@setdefault{%
  \nfss@text{\reset@font\bfseries ??}%
}
%    \end{macrocode}
%    \end{macro}
%
% \paragraph{Main property list.}
%
%    \begin{macro}{\zref@setmainlist}
%    The name of the default property list is stored in
%    \cs{ZREF@mainlist} and can be set by \cs{zref@setmainlist}.
%    \begin{macrocode}
\ZREF@Robust\def\zref@setmainlist#1{%
  \def\ZREF@mainlist{#1}%
}
\zref@setmainlist{main}
%    \end{macrocode}
%    \end{macro}
%    Now we create the list.
%    \begin{macrocode}
\zref@newlist\ZREF@mainlist
%    \end{macrocode}
%
% \paragraph{Main properties.}
%
%    The two properties \texttt{default} and \texttt{page}
%    are created and added to the main property list.
%    They store the data that
%    standard \hologo{LaTeX} uses in its references created by
%    \cs{label}.
%    \begin{description}
%    \item[\texttt{default}] the apperance of the latest
%      counter that is incremented by \cs{refstepcounter}
%    \item[\texttt{page}] the apperance of the page counter
%    \end{description}
%    \begin{macrocode}
\zref@newprop{default}{\@currentlabel}
\zref@newprop*{page}{\thepage}
\zref@addprops\ZREF@mainlist{default,page}
%    \end{macrocode}
%
% \paragraph{Properties}
%
%    \begin{macro}{\ZREF@NewPropAnchor}
%    \begin{macrocode}
\def\ZREF@NewPropAnchor{%
  \zref@newprop{anchor}{%
    \ltx@ifundefined{@currentHref}{}{\@currentHref}%
  }%
  \global\let\ZREF@NewPropAnchor\relax
}
%    \end{macrocode}
%    \end{macro}
%    \begin{macro}{\zref@titleref@current}
%    Later we will redefine the section and caption macros
%    to catch the current title and remember the value in
%    \cs{zref@titleref@current}.
%    \begin{macro}{\ZREF@NewPropTitle}
%    \begin{macrocode}
\def\ZREF@NewPropTitle{%
  \gdef\zref@titleref@current{}%
  \zref@newprop{title}{\zref@titleref@current}%
  \global\let\ZREF@NewPropTitle\relax
}
%    \end{macrocode}
%    \end{macro}
%    \end{macro}
%    \begin{macro}{\ZREF@NewPropTheotype}
%    \begin{macrocode}
\def\ZREF@NewPropTheotype{%
  \zref@newprop{theotype}{}%
  \global\let\ZREF@NewPropTheotype\relax
}
%    \end{macrocode}
%    \end{macro}
%    \begin{macro}{\ZREF@NewPropPageValue}
%    \begin{macrocode}
\def\ZREF@NewPropPageValue{%
  \zref@newprop*{pagevalue}[0]{\number\c@page}%
  \global\let\ZREF@NewPropPageValue\relax
}
%    \end{macrocode}
%    \end{macro}
%
% \paragraph{Mark successful loading}
%
%    \begin{macrocode}
\let\ZREF@base@ok=Y
%</base>
%    \end{macrocode}
%
%
% \subsection{Module \xmodule{user}}
%
%    \begin{macrocode}
%<*user>
\NeedsTeXFormat{LaTeX2e}
\ProvidesPackage{zref-user}%
  [2020-07-03 v2.32 Module user for zref (HO)]%
\RequirePackage{zref-base}[2019/11/29]
\ifx\ZREF@base@ok Y%
\else
  \expandafter\endinput
\fi
%    \end{macrocode}
%
%    Module \xmodule{user} enables a small user interface.
%    All macros are prefixed by \cs{z}.
%
%    First we define the pendants to the standard \hologo{LaTeX} referencing
%    commands \cs{label}, \cs{ref}, and \cs{pageref}.
%
%    \begin{macro}{\zlabel}
%    Similar to \cs{label} the macro \cs{zlabel} writes a reference
%    entry in the \xfile{.aux} file. The main property list is used.
%    Also we add the babel patch. The \cs{label} command can also be
%    used inside section titles, but it must not go into the
%    table of contents. Therefore we have to check this situation.
%    \begin{macrocode}
\newcommand*\zlabel{%
  \ifx\label\ltx@gobble
    \expandafter\ltx@gobble
  \else
    \expandafter\zref@wrapper@babel\expandafter\zref@label
  \fi
}%
%    \end{macrocode}
%    \end{macro}
%
%    \begin{macro}{\zkvlabel}
%    \begin{macrocode}
\newcommand*{\zkvlabel}[1]{%
  \ifx\label\ltx@gobble
    \expandafter\ltx@gobblethree
  \fi
  \zref@wrapper@babel{\zref@labelbykv{#1}}%
}%
%    \end{macrocode}
%    \end{macro}
%
%    \begin{macro}{\zref}
%    Macro \cs{zref} is the corresponding macro for \cs{ref}.
%    Also it provides an optional argument in order to
%    select another property.
%    \begin{macrocode}
\newcommand*{\zref}[2][default]{% robust because of optional argument
  \zref@propexists{#1}{%
    \zref@wrapper@babel\ZREF@zref{#2}{#1}%
  }%
}%
\def\ZREF@zref#1{%
  \zref@refused{#1}%
  \zref@extract{#1}%
}%
%    \end{macrocode}
%    \end{macro}
%
%    \begin{macro}{\zpageref}
%    For macro \cs{zpageref} we just call \cs{zref}
%    with property \verb|page|.
%    \begin{macrocode}
\ZREF@IfDefinable\zpageref\def{%
  {\zref[page]}%
}
%    \end{macrocode}
%    \end{macro}
%
%    \begin{macro}{\zrefused}
%    For the following expandible user macros \cs{zrefused}
%    should be used to notify \hologo{LaTeX} in case of undefined
%    references.
%    \begin{macrocode}
\ZREF@IfDefinable\zrefused\def{%
  {\zref@refused}%
}
%    \end{macrocode}
%    \end{macro}
%
%    \begin{macrocode}
%</user>
%    \end{macrocode}
%
%
% \subsection{Module \xmodule{abspage}}
%
%    \begin{macrocode}
%<*abspage>
\NeedsTeXFormat{LaTeX2e}
\ProvidesPackage{zref-abspage}%
  [2020-07-03 v2.32 Module abspage for zref (HO)]%
\RequirePackage{zref-base}[2019/11/29]
\ifx\ZREF@base@ok Y%
\else
  \expandafter\endinput
\fi
%    \end{macrocode}
%
%    Module \xmodule{abspage} adds a new property \verb|abspage|
%    to the \verb|main| property list for absolute page numbers.
%    These are recorded by the help of package \xpackage{atbegshi}.
%    \begin{macrocode}
\RequirePackage{atbegshi}[2011/10/05]%
%    \end{macrocode}
%    The counter \verb|abspage| must not go in the clear list
%    of \verb|@ckpt| that is used to set counters in
%    \xfile{.aux} files of included \hologo{TeX} files.
%    \begin{macrocode}
\begingroup
  \let\@addtoreset\ltx@gobbletwo
  \newcounter{abspage}%
\endgroup
\setcounter{abspage}{0}%
\AtBeginShipout{%
  \stepcounter{abspage}%
}%
\zref@newprop*{abspage}[0]{\the\c@abspage}%
\zref@addprop\ZREF@mainlist{abspage}%
%    \end{macrocode}
%    Note that counter \verb|abspage| shows the
%    previous page during page processing. Before
%    shipout the counter is incremented.
%    Thus the property is correctly written with
%    deferred writing. If the counter is written
%    using \cs{zref@wrapper@immediate}, then the number
%    is too small by one.
%
%    \begin{macrocode}
%</abspage>
%    \end{macrocode}
%
%
% \subsection{Module \xmodule{counter}}
%
%    \begin{macrocode}
%<*counter>
\NeedsTeXFormat{LaTeX2e}
\ProvidesPackage{zref-counter}%
  [2020-07-03 v2.32 Module counter for zref (HO)]%
\RequirePackage{zref-base}[2019/11/29]
\ifx\ZREF@base@ok Y%
\else
  \expandafter\endinput
\fi
%    \end{macrocode}
%
%    For features such as \xpackage{hyperref}'s \cs{autoref}
%    we need the name of the counter. The property
%    \verb|counter| is defined and added to the main property
%    list. Starting with \LaTeX\ 2020-10-01 the proper can use
%    \@currentcounter. In older formats
%    \cs{refstepcounter} has to be patched but this can fail in some
%    cases, see issue \#5.
%    \begin{macrocode}
\@ifl@t@r\fmtversion{2020-10-01}
 {
  \zref@newprop{counter}{\@currentcounter}
  \zref@addprop\ZREF@mainlist{counter}
 }
 {
  \zref@newprop{counter}{}
  \zref@addprop\ZREF@mainlist{counter}
  \AtBeginDocument{%
    \ZREF@patch{refstepcounter}{%
      \def\refstepcounter#1{%
        \zref@setcurrent{counter}{#1}%
        \ZREF@org@refstepcounter{#1}%
      }%
    }%
  }
 }
%    \end{macrocode}
%    \begin{macrocode}
%</counter>
%    \end{macrocode}
%
%
% \subsection{Module \xmodule{lastpage}}
%
%    \begin{macrocode}
%<*lastpage>
\NeedsTeXFormat{LaTeX2e}
\ProvidesPackage{zref-lastpage}%
  [2020-07-03 v2.32 Module lastpage for zref (HO)]%
\RequirePackage{zref-base}[2019/11/29]
\RequirePackage{zref-abspage}[2019/11/29]
\RequirePackage{atveryend}[2009/12/07]
\ifx\ZREF@base@ok Y%
\else
  \expandafter\endinput
\fi
%    \end{macrocode}
%
%    The module \xmodule{lastpage} implements the service
%    of package \xpackage{lastpage} by setting a reference
%    \verb|LastPage| at the end of the document.
%    If module \xmodule{abspage} is given, also the absolute
%    page number is available, because the properties of
%    the main property list are used.
%    \begin{macrocode}
\zref@newlist{LastPage}
\AfterLastShipout{%
  \if@filesw
    \begingroup
      \advance\c@page\m@ne
      \toks@\expandafter\expandafter\expandafter{%
        \expandafter\Z@L@main
        \Z@L@LastPage
      }%
      \expandafter\zref@wrapper@immediate\expandafter{%
        \expandafter\ZREF@label\expandafter{\the\toks@}{LastPage}%
      }%
    \endgroup
  \fi
}
%    \end{macrocode}
%    \begin{macro}{\zref@iflastpage}
%    \begin{macrocode}
\def\zref@iflastpage#1{%
  \ifnum\zref@extractdefault{#1}{abspage}{-1}=%
        \zref@extractdefault{LastPage}{abspage}{-2} %
    \expandafter\ltx@firstoftwo
  \else
    \expandafter\ltx@secondoftwo
  \fi
}
%    \end{macrocode}
%    \end{macro}
%    \begin{macro}{\ziflastpage}
%    \begin{macrocode}
\ZREF@IfDefinable\ziflastpage\def{%
  {\zref@wrapper@babel\ZREF@iflastpage}%
}
%    \end{macrocode}
%    \end{macro}
%    \begin{macro}{ZREF@iflastpage}
%    \begin{macrocode}
\def\ZREF@iflastpage#1{%
  \zref@refused{LastPage}%
  \zref@refused{#1}%
  \zref@iflastpage{#1}%
}
%    \end{macrocode}
%    \end{macro}
%    \begin{macrocode}
%</lastpage>
%    \end{macrocode}
%
%
% \subsection{Module \xmodule{thepage}}
%
%    \begin{macrocode}
%<*thepage>
\NeedsTeXFormat{LaTeX2e}
\ProvidesPackage{zref-thepage}%
  [2020-07-03 v2.32 Module thepage for zref (HO)]%
\RequirePackage{zref-base}[2019/11/29]
\ifx\ZREF@base@ok Y%
\else
  \expandafter\endinput
\fi
%    \end{macrocode}
%
%    \begin{macrocode}
\RequirePackage{atbegshi}[2011/10/05]
\RequirePackage{zref-abspage}[2019/11/29]
%    \end{macrocode}
%    \begin{macrocode}
\zref@newlist{thepage}
\zref@addprop{thepage}{page}
\ZREF@NewPropPageValue
%    \end{macrocode}
%    \begin{macro}{\zref@thepage@atbegshi@hook}
%    \begin{macrocode}
\let\zref@thepage@atbegshi@hook\ltx@empty
%    \end{macrocode}
%    \end{macro}
%    \begin{macrocode}
\zref@addprop{thepage}{pagevalue}
\AtBeginShipout{%
  \AtBeginShipoutAddToBox{%
    \zref@thepage@atbegshi@hook
    \zref@labelbylist{thepage\the\value{abspage}}{thepage}%
  }%
}
%    \end{macrocode}
%    \begin{macro}{\zref@thepage@name}
%    \begin{macrocode}
\ltx@IfUndefined{numexpr}{%
  \def\zref@thepage@name#1{thepage\number#1}%
}{%
  \def\zref@thepage@name#1{thepage\the\numexpr#1}%
}
%    \end{macrocode}
%    \end{macro}
%    \begin{macro}{\zref@thepage}
%    \begin{macrocode}
\def\zref@thepage#1{%
  \zref@extract{\zref@thepage@name{#1}}{page}%
}%
%    \end{macrocode}
%    \end{macro}
%    \begin{macro}{\zref@thepage@refused}
%    \begin{macrocode}
\ZREF@Robust\def\zref@thepage@refused#1{%
  \zref@refused{\zref@thepage@name{#1}}%
}%
%    \end{macrocode}
%    \end{macro}
%    \begin{macro}{\zthepage}
%    \begin{macrocode}
\ZREF@IfDefinable\zthepage\def{%
  #1{%
    \zref@thepage@refused{#1}%
    \zref@thepage{#1}%
  }%
}
%    \end{macrocode}
%    \end{macro}
%    \begin{macrocode}
%</thepage>
%    \end{macrocode}
%
%
% \subsection{Module \xmodule{nextpage}}
%
%    \begin{macrocode}
%<*nextpage>
\NeedsTeXFormat{LaTeX2e}
\ProvidesPackage{zref-nextpage}%
  [2020-07-03 v2.32 Module nextpage for zref (HO)]%
\RequirePackage{zref-base}[2019/11/29]
\ifx\ZREF@base@ok Y%
\else
  \expandafter\endinput
\fi
%    \end{macrocode}
%
%    \begin{macrocode}
\RequirePackage{zref-abspage}[2019/11/29]
\RequirePackage{zref-thepage}[2019/11/29]
\RequirePackage{zref-lastpage}[2019/11/29]
\RequirePackage{uniquecounter}[2009/12/18]
%    \end{macrocode}
%    \begin{macrocode}
\UniqueCounterNew{znextpage}

\newcommand*{\znextpagesetup}{%
  \afterassignment\ZREF@np@setup@i
  \def\ZREF@np@call@unknown##1%
}
\def\ZREF@np@setup@i{%
  \afterassignment\ZREF@np@setup@ii
  \def\ZREF@np@call@nonext##1%
}
\def\ZREF@np@setup@ii{%
  \def\ZREF@np@call@next##1%
}
\def\ZREF@np@call@unknown#1{#1}
\def\ZREF@np@call@nonext#1{#1}
\def\ZREF@np@call@next#1{#1}
\ZREF@IfDefinable\znextpage\def{%
  {\UniqueCounterCall{znextpage}{\ZREF@nextpage}}%
}%
\newcommand*{\znonextpagename}{}
\newcommand*{\zunknownnextpagename}{\Z@D@page}
\def\ZREF@nextpage#1{%
  \begingroup
    \def\ZREF@refname@this{zref@np#1}%
    \zref@labelbyprops\ZREF@refname@this{abspage}%
    \chardef\ZREF@call=0 % unknown
    \ZREF@ifrefundefined\ZREF@refname@this{%
    }{%
      \edef\ZREF@pagenum@this{%
        \zref@extractdefault\ZREF@refname@this{abspage}{0}%
      }%
      \edef\ZREF@refname@next{%
        \zref@thepage@name{%
          \the\numexpr\ZREF@pagenum@this+1%
        }%
      }%
      \ifnum\ZREF@pagenum@this>0 %
        \ZREF@ifrefundefined{LastPage}{%
          \zref@ifrefundefined\ZREF@refname@next{%
          }{%
            \chardef\ZREF@call=2 % next page
          }%
        }{%
          \edef\ZREF@pagenum@last{%
            \zref@extractdefault{LastPage}{abspage}{0}%
          }%
          \ifnum\ZREF@pagenum@this<\ZREF@pagenum@last\ltx@space
            \ZREF@ifrefundefined\ZREF@refname@next{%
            }{%
              \chardef\ZREF@call=2 % next page
            }%
          \else
            \ifnum\ZREF@pagenum@this=\ZREF@pagenum@this\ltx@space
              \chardef\ZREF@call=1 % no next page
            \fi
          \fi
        }%
      \fi
    }%
    \edef\x{%
      \endgroup
      \ifcase\ZREF@call
        \noexpand\ZREF@np@call@unknown{%
          \noexpand\zunknownnextpagename
        }%
      \or
        \noexpand\ZREF@np@call@nonext{%
          \noexpand\znonextpagename
        }%
      \else
        \noexpand\ZREF@np@call@next{%
          \noexpand\zref@extract{\ZREF@refname@next}{page}%
        }%
      \fi
    }%
  \x
}
%    \end{macrocode}
%    \begin{macrocode}
%</nextpage>
%    \end{macrocode}
%
%
% \subsection{Module \xmodule{totpages}}
%
%    \begin{macrocode}
%<*totpages>
\NeedsTeXFormat{LaTeX2e}
\ProvidesPackage{zref-totpages}%
  [2020-07-03 v2.32 Module totpages for zref (HO)]%
\RequirePackage{zref-base}[2019/11/29]
\ifx\ZREF@base@ok Y%
\else
  \expandafter\endinput
\fi
%    \end{macrocode}
%
%    The absolute page number of the last page is the total
%    page number.
%    \begin{macrocode}
\RequirePackage{zref-abspage}[2019/11/29]
\RequirePackage{zref-lastpage}[2019/11/29]
%    \end{macrocode}
%
%    \begin{macro}{\ztotpages}
%    Macro \cs{ztotpages} contains the number of pages. It can be
%    used inside expandable calculations. It expands to zero if
%    the reference is not yet available.
%    \begin{macrocode}
\newcommand*{\ztotpages}{%
  \zref@extractdefault{LastPage}{abspage}{0}%
}
%    \end{macrocode}
%    \end{macro}
%    Also we mark the reference |LastPage| as used:
%    \begin{macrocode}
\AtBeginDocument{%
  \zref@refused{LastPage}%
}
%    \end{macrocode}
%    \begin{macrocode}
%</totpages>
%    \end{macrocode}
%
%
% \subsection{Module \xmodule{pagelayout}}
%
%    \begin{macrocode}
%<*pagelayout>
\NeedsTeXFormat{LaTeX2e}
\ProvidesPackage{zref-pagelayout}%
  [2020-07-03 v2.32 Module pagelayout for zref (HO)]%
\RequirePackage{zref-base}[2019/11/29]
\ifx\ZREF@base@ok Y%
\else
  \expandafter\endinput
\fi
%    \end{macrocode}
%
%    \begin{macrocode}
\RequirePackage{zref-thepage}[2019/11/29]
\RequirePackage{iftex}[2019/11/07]%
\RequirePackage{atveryend}[2010/03/24]
%    \end{macrocode}
%
%
% \subsubsection{Define layout properties}
%
%    \begin{macrocode}
\def\ZREF@temp#1{%
  \begingroup
    \escapechar=-1 %
  \ltx@ifundefined{\string#1}{\endgroup}{%
    \edef\x{%
      \endgroup
      \noexpand\zref@newprop*{\string#1}%
                             [\noexpand\number\noexpand#1]% hash-ok
                             {\noexpand\number\noexpand#1}%
      \noexpand\zref@addprop{thepage}{\string#1}%
    }%
    \x
  }%
}
\ZREF@temp\mag
\ZREF@temp\paperwidth
\ZREF@temp\paperheight
\ZREF@temp\stockwidth % memoir.cls, crop.sty
\ZREF@temp\stockheight % memoir.cls, crop.sty
\ZREF@temp\mediawidth % VTeX
\ZREF@temp\mediaheight % VTeX
\ifluatex
\zref@newprop*{pdfvorigin}%
                             [\number\pdfvariable vorigin]% hash-ok
                             {\number\pdfvariable vorigin}%
\zref@addprop{thepage}{pdfvorigin}
\zref@newprop*{pdfhorigin}%
                             [\number\pdfvariable horigin]% hash-ok
                             {\number\pdfvariable horigin}%
\zref@addprop{thepage}{pdfhorigin}
\zref@newprop*{pdfpageheight}%
                             [\number\pageheight]% hash-ok
                             {\number\pageheight}%
\zref@addprop{thepage}{pdfpageheight}
\zref@newprop*{pdfpagewidth}%
                             [\number\pagewidth]% hash-ok
                             {\number\pagewidth}%
\zref@addprop{thepage}{pdfpagewidth}
\else
\ZREF@temp\pdfpagewidth
\ZREF@temp\pdfpageheight
\ZREF@temp\pdfhorigin
\ZREF@temp\pdfvorigin
\fi
\ZREF@temp\hoffset
\ZREF@temp\voffset
\ZREF@temp\topmargin
\ZREF@temp\oddsidemargin
\ZREF@temp\evensidemargin
\ZREF@temp\textwidth
\ZREF@temp\textheight
\ZREF@temp\headheight
\ZREF@temp\headsep
\ZREF@temp\footskip
\ZREF@temp\marginparwidth
\ZREF@temp\marginparsep
\ZREF@temp\columnwidth
\ZREF@temp\columnsep
\ZREF@temp\trimedge % memoir.cls
\ZREF@temp\spinemargin % memoir.cls
\ZREF@temp\foremargin % memoir.cls
\ZREF@temp\trimtop % memoir.cls
\ZREF@temp\uppermargin % memoir.cls
\ZREF@temp\headmargin % memoir.cls
%    \end{macrocode}
%
%    \begin{macrocode}
\IfFormatAtLeastTF{2020/10/01}
 {
   \zref@newprop*{outputboxwd}[0pt]{\ShipoutBoxWidth}
   \zref@newprop*{outputboxht}[0pt]{\ShipoutBoxHeight}
   \zref@newprop*{outputboxdp}[0pt]{\ShipoutBoxDepth}
 }
 {
   \zref@newprop*{outputboxwd}[0pt]{\AtBeginShipoutBoxWidth}
   \zref@newprop*{outputboxht}[0pt]{\AtBeginShipoutBoxHeight}
   \zref@newprop*{outputboxdp}[0pt]{\AtBeginShipoutBoxDepth}
 }
%    \end{macrocode}
%    \begin{macrocode}
\zref@addprops{thepage}{outputboxwd,outputboxht,outputboxdp}
%    \end{macrocode}
%
%    \begin{macro}{\ifZREF@pl@list}
%    \begin{macrocode}
\ltx@newif\ifZREF@pl@list
%    \end{macrocode}
%    \end{macro}
%    \begin{macro}{\zref@listpagelayout}
%    \begin{macrocode}
\ZREF@IfDefinable\zlistpagelayout\def{%
  {\global\ZREF@pl@listtrue}%
}
%    \end{macrocode}
%    \end{macro}
%    \begin{macro}{\ZREF@pl@AfterLastShipout}
%    \begin{macrocode}
\def\ZREF@pl@AfterLastShipout{%
  \ifZREF@pl@list
    \edef\ZREF@page@max{\the\value{abspage}}%
    \ltx@ifundefined{ZREF@org@testdef}{%
      \let\ZREF@org@testdef\@testdef
      \def\@testdef##1##2##3{%
        \ZREF@org@testdef{##1}{##2}{##3}%
        \def\ZREF@temp{##1}%
        \ifx\ZREF@temp\ZREF@RefPrefix
          \expandafter\gdef\csname##1@##2\endcsname{##3}%
        \fi
      }%
    }{}%
    \AtVeryEndDocument{\ZREF@pl@AtVeryEnd}%
  \fi
}
%    \end{macrocode}
%    \end{macro}
%    \begin{macro}{\ZREF@pl@AtVeryEnd}
%    \begin{macrocode}
\def\ZREF@pl@AtVeryEnd{%
  \begingroup
    \toks@{Page layout parameters:\MessageBreak}%
    \count@=1 %
    \ZREF@pl@ListPage
    \edef\x{\endgroup
      \noexpand\@PackageInfoNoLine{zref-pagelayout}{\the\toks@}%
    }%
  \x
}
%    \end{macrocode}
%    \end{macro}
%    \begin{macro}{\ZREF@pl@ListPage}
%    \begin{macrocode}
\def\ZREF@pl@ListPage{%
  \edef\x{%
    \toks@={%
      \the\toks@
      Page \the\count@:\noexpand\MessageBreak
      \zref@ifrefundefined{thepage\the\count@}{}{%
        \ltx@space\ltx@space mag = %
        \zref@extract{thepage\the\count@}{mag}%
        \noexpand\MessageBreak
        \ZREF@pl@ListEntry{paperwidth}%
        \ZREF@pl@ListEntry{paperheight}%
        \ZREF@pl@ListEntry{stockwidth}%
        \ZREF@pl@ListEntry{stockheight}%
        \ZREF@pl@ListEntry{mediawidth}%
        \ZREF@pl@ListEntry{mediaheight}%
        \ZREF@pl@ListEntry{pdfpagewidth}%
        \ZREF@pl@ListEntry{pdfpageheight}%
        \ZREF@pl@ListEntry{pdfhorigin}%
        \ZREF@pl@ListEntry{pdfvorigin}%
        \ZREF@pl@ListEntry{hoffset}%
        \ZREF@pl@ListEntry{voffset}%
        \ZREF@pl@ListEntry{topmargin}%
        \ZREF@pl@ListEntry{oddsidemargin}%
        \ZREF@pl@ListEntry{evensidemargin}%
        \ZREF@pl@ListEntry{textwidth}%
        \ZREF@pl@ListEntry{textheight}%
        \ZREF@pl@ListEntry{headheight}%
        \ZREF@pl@ListEntry{headsep}%
        \ZREF@pl@ListEntry{footskip}%
        \ZREF@pl@ListEntry{marginparwidth}%
        \ZREF@pl@ListEntry{marginparsep}%
        \ZREF@pl@ListEntry{columnwidth}%
        \ZREF@pl@ListEntry{columnsep}%
        \ZREF@pl@ListEntry{trimedge}%
        \ZREF@pl@ListEntry{spinemargin}%
        \ZREF@pl@ListEntry{foremargin}%
        \ZREF@pl@ListEntry{trimtop}%
        \ZREF@pl@ListEntry{uppermargin}%
        \ZREF@pl@ListEntry{headmargin}%
      }%
    }%
  }\x
  \ifnum\ZREF@page@max>\count@
    \advance\count@ by\ltx@one
  \else
    \expandafter\ltx@gobble
  \fi
  \ZREF@pl@ListPage
}
%    \end{macrocode}
%    \end{macro}
%    \begin{macro}{\ZREF@pl@ListEntry}
%    \begin{macrocode}
\def\ZREF@pl@ListEntry#1{%
  \zref@ifpropundefined{#1}{%
  }{%
    \zref@ifrefcontainsprop{thepage\the\count@}{#1}{%
      \ltx@space\ltx@space#1 = %
      \zref@extract{thepage\the\count@}{#1}sp = %
      \the\dimexpr\zref@extract{thepage\the\count@}{#1}sp\relax
      \noexpand\MessageBreak
    }{}%
  }%
}
%    \end{macrocode}
%    \end{macro}
%    \begin{macrocode}
\AfterLastShipout{%
  \ZREF@pl@AfterLastShipout
}
%    \end{macrocode}
%
%    \begin{macrocode}
%</pagelayout>
%    \end{macrocode}
%
%
% \subsection{Module \xmodule{pageattr}}
%
%    \begin{macrocode}
%<*pageattr>
\NeedsTeXFormat{LaTeX2e}
\ProvidesPackage{zref-pageattr}%
  [2020-07-03 v2.32 Module pageattr for zref (HO)]%
\RequirePackage{zref-base}[2019/11/29]
\ifx\ZREF@base@ok Y%
\else
  \expandafter\endinput
\fi
%    \end{macrocode}
%
%    \begin{macrocode}
\RequirePackage{iftex}[2019/11/07]%
%    \end{macrocode}
%
%
%    \begin{macrocode}
\let\ZREF@temp=N%
\ifluatex
\expandafter\@firstoftwo
\else
\expandafter\@secondoftwo
\fi
{%luatex
 \RequirePackage{zref-thepage}[2019/11/29]
 \RequirePackage{zref-lastpage}[2019/11/29]%
 \zref@newprop*{pdfpageattr}[]{\zref@hex{\the\pdfvariable pageattr}}%
 \zref@addprop{thepage}{pdfpageattr}%
 \zref@newprop*{pdfpagesattr}[]{\zref@hex{\the\pdfvariable pagesattr}}%
 \zref@addprop{LastPage}{pdfpagesattr}%
 \let\ZREF@temp=Y%
}
{%other
 \ltx@IfUndefined{pdfpageattr}{%
  \@PackageInfoNoLine{zref-pageattr}{%
    \string\pdfpageattr\space is not available%
  }%
  \def\zref@pdfpageattr#1{}%
  \def\zref@pdfpageattr@used#1{}%
 }{%
  \RequirePackage{zref-thepage}[2019/11/29]%
  \zref@newprop*{pdfpageattr}[]{\zref@hex{\the\pdfpageattr}}%
  \zref@addprop{thepage}{pdfpageattr}%
  \let\ZREF@temp=Y%
 }
 \ltx@IfUndefined{pdfpagesattr}{%
  \@PackageInfoNoLine{zref-pageattr}{%
    \string\pdfpagesattr\space is not available%
  }%
  \def\zref@pdfpagesattr{}%
  \def\zref@pdfpagesattr@used{}%
 }{%
  \RequirePackage{zref-lastpage}[2019/11/29]%
  \zref@newprop*{pdfpagesattr}[]{\zref@hex{\the\pdfpagesattr}}%
  \zref@addprop{LastPage}{pdfpagesattr}%
  \let\ZREF@temp=Y%
 }%
}%
%    \end{macrocode}
%    \begin{macrocode}
\ifx\ZREF@temp N%
  \expandafter\endinput
\fi
%    \end{macrocode}
%    \begin{macrocode}
\RequirePackage{zref-abspage}[2019/11/29]
\RequirePackage{atveryend}[2010/03/24]
\RequirePackage{pdftexcmds}[2010/04/01]
\let\ZREF@temp=Y%
\ltx@IfUndefined{pdf@escapehex}{\let\ZREF@temp=N}{}
\ltx@IfUndefined{pdf@unescapehex}{\let\ZREF@temp=N}{}
\ifx\ZREF@temp N%
  \let\zref@hex\ltx@firstofone
  \let\zref@unhex\ltx@firstofone
\else
  \let\zref@hex\pdf@escapehex
  \let\zref@unhex\pdf@unescapehex
\fi
%    \end{macrocode}
%
%    \begin{macro}{\ifZREF@pa@list}
%    \begin{macrocode}
\ltx@newif\ifZREF@pa@list
%    \end{macrocode}
%    \end{macro}
%    \begin{macro}{\zref@listpageattr}
%    \begin{macrocode}
\ZREF@IfDefinable\zlistpageattr\def{%
  {\ZREF@pa@listtrue}%
}
%    \end{macrocode}
%    \end{macro}
%    \begin{macro}{\ZREF@pa@AfterLastShipout}
%    \begin{macrocode}
\def\ZREF@pa@AfterLastShipout{%
  \ifZREF@pa@list
    \edef\ZREF@page@max{\the\value{abspage}}%
    \ltx@ifundefined{ZREF@org@testdef}{%
      \let\ZREF@org@testdef\@testdef
      \def\@testdef##1##2##3{%
        \ZREF@org@testdef{##1}{##2}{##3}%
        \def\ZREF@temp{##1}%
        \ifx\ZREF@temp\ZREF@RefPrefix
          \expandafter\xdef\csname##1@##2\endcsname{##3}%
        \fi
      }%
    }{}%
    \AtVeryEndDocument{\ZREF@pa@AtVeryEnd}%
  \fi
}
%    \end{macrocode}
%    \end{macro}
%    \begin{macro}{\ZREF@pa@AtVeryEnd}
%    \begin{macrocode}
\let\ZREF@temp=Y%
\ltx@IfUndefined{pdfpageattr}{}{\let\ZREF@temp=N}
\ifluatex \let\ZREF@temp=N \fi
\ifx\ZREF@temp Y
 \expandafter\@firstoftwo
\else
 \expandafter\@secondoftwo
\fi
 {%
  \def\ZREF@pa@AtVeryEnd{}%
 }
 {%
  \def\ZREF@pa@AtVeryEnd{%
    \begingroup
      \toks@{List of \ltx@backslashchar
             \ifluatex pdfvariable\else pdf\fi
             pdfpageattr:\MessageBreak}%
      \count@=1 %
      \ZREF@pa@ListPage
      \edef\x{\endgroup
        \noexpand\@PackageInfoNoLine{zref-pageattr}{%
          \the\toks@
        }%
      }%
    \x
  }%
%    \end{macrocode}
%    \end{macro}
%    \begin{macro}{\zref@pageattr}
%    \begin{macrocode}
\def\zref@pdfpageattr#1{%
  \zref@unhex{%
    \zref@extract{thepage\ZREF@number{#1}}{pdfpageattr}%
  }%
}
% compability, \zref@pageattr was defined in older versions
\let\zref@pageattr\zref@pdfpageattr
%    \end{macrocode}
%    \end{macro}
%    \begin{macro}{\zref@pageattr@used}
%    \begin{macrocode}
\ZREF@Robust\def\zref@pageattr@used#1{%
  \zref@refused{thepage\ZREF@number{#1}}%
}
%    \end{macrocode}
%    \end{macro}
%    \begin{macro}{\ZREF@pa@ListPage}
%    \begin{macrocode}
  \def\ZREF@pa@ListPage{%
    \edef\x{%
      \toks@={%
        \the\toks@
        Page \the\count@:%
        \noexpand\MessageBreak
        \zref@ifrefundefined{thepage\the\count@}{}{%
          <<\zref@pdfpageattr\count@>>%
          \noexpand\MessageBreak
        }%
      }%
    }\x
    \ifnum\ZREF@page@max>\count@
      \advance\count@ by\ltx@one
    \else
      \expandafter\ltx@gobble
    \fi
    \ZREF@pa@ListPage
  }%
}
%    \end{macrocode}
%    \end{macro}
%    \begin{macrocode}
\let\ZREF@temp=Y%
\ltx@IfUndefined{pdfpagesattr}{}{\let\ZREF@temp=N}
\ifluatex \let\ZREF@temp=N \fi
\ifx\ZREF@temp N
 \expandafter\@firstofone
\fi
 {%
%    \end{macrocode}
%    \begin{macro}{\zref@pdfpagesattr}
%    \begin{macrocode}
  \def\zref@pdfpagesattr{%
    \zref@unhex{%
      \zref@extract{LastPage}{pdfpagesattr}%
    }%
  }%
%    \end{macrocode}
%    \end{macro}
%    \begin{macro}{\zref@pdfpagesattr@used}
%    \begin{macrocode}
  \ZREF@Robust\def\zref@pdfpagesattr@used{%
    \zref@refused{LastPage}%
  }%
%    \end{macrocode}
%    \end{macro}
%    \begin{macrocode}
  \ltx@LocalAppendToMacro\ZREF@pa@AtVeryEnd{%
    \@PackageInfoNoLine{zref-pageattr}{%
      \ltx@backslashchar
      \ifluatex pdfvariable\else pdf\fi
      pagesattr:\MessageBreak
      <<\zref@pdfpagesattr>>%
      \MessageBreak
    }%
  }%
}
\AfterLastShipout{%
  \ZREF@pa@AfterLastShipout
}
%    \end{macrocode}
%    \begin{macrocode}
%</pageattr>
%    \end{macrocode}
%
%
% \subsection{Module \xmodule{marks}}
%
%    \begin{macrocode}
%<*marks>
\NeedsTeXFormat{LaTeX2e}
\ProvidesPackage{zref-marks}%
  [2020-07-03 v2.32 Module marks for zref (HO)]%
\RequirePackage{zref-base}[2019/11/29]
\ifx\ZREF@base@ok Y%
\else
  \expandafter\endinput
\fi
%    \end{macrocode}
%
%    \begin{macrocode}
\newcommand*{\zref@marks@register}[3][]{%
  \edef\ZREF@TempName{#1}%
  \edef\ZREF@TempNum{\ZREF@number{#2}}%
  \ifnum\ZREF@TempNum<\ltx@zero %
    \PackageError\ZREF@name{%
      \string\zref@marks@register\ltx@space is called with invalid%
      \MessageBreak
      marks register number (\ZREF@TempNum)%
    }{%
      Use `0' or the command, defined by \string\newmarks.\MessageBreak
      \@ehc
    }%
  \else
    \ifx\ZREF@TempName\ltx@empty
      \edef\ZREF@TempName{mark\romannumeral\ZREF@TempNum}%
    \else
      \edef\ZREF@TempName{marks\ZREF@TempName}%
    \fi
    \ZREF@MARKS@DefineProp{top}%
    \ZREF@MARKS@DefineProp{first}%
    \ZREF@MARKS@DefineProp{bot}%
    \kv@parse{#3}{%
      \ifx\kv@value\relax
        \def\kv@value{top,first,bot}%
      \fi
      \edef\ZREF@temp{\expandafter\ltx@car\kv@key X\@nil}%
      \ifx\ZREF@temp\ZREF@STAR
        \edef\kv@key{\expandafter\ltx@cdr\kv@key\@nil}%
        \zref@newlist\kv@key
      \fi
      \expandafter\comma@parse\expandafter{\kv@value}{%
        \ifcase0\ifx\comma@entry\ZREF@NAME@top 1\else
                \ifx\comma@entry\ZREF@NAME@first 1\else
                \ifx\comma@entry\ZREF@NAME@bot 1\fi\fi\fi\ltx@space
          \PackageWarning{zref-marks}{%
            Use `top', `first' or `bot' for the list values%
            \MessageBreak
            in the third argument of \string\zref@marks@register.%
            \MessageBreak
            Ignoring unkown value `\comma@entry'%
          }%
        \else
          \zref@addprop{\kv@key}{\comma@entry\ZREF@TempName}%
        \fi
        \ltx@gobble
      }%
      \ltx@gobbletwo
    }%
  \fi
}
\def\ZREF@STAR{*}
\def\ZREF@NAME@top{top}
\def\ZREF@NAME@first{first}
\def\ZREF@NAME@bot{bot}
\def\ZREF@MARKS@DefineProp#1{%
  \zref@ifpropundefined{#1\ZREF@TempName}{%
    \ifnum\ZREF@TempNum=\ltx@zero
      \begingroup
        \edef\x{\endgroup
          \noexpand\zref@newprop*{#1\ZREF@TempName}[]{%
            \expandafter\noexpand\csname#1mark\endcsname
          }%
        }%
      \x
    \else
      \begingroup
        \edef\x{\endgroup
          \noexpand\zref@newprop*{#1\ZREF@TempName}[]{%
            \expandafter\noexpand\csname#1marks\endcsname
            \ZREF@TempNum
          }%
        }%
      \x
    \fi
  }{%
    \PackageWarning{zref-marks}{%
      \string\zref@marks@register\ltx@space does not generate the%
      \MessageBreak
      new property `#1\ZREF@TempName', because\MessageBreak
      it is already defined%
    }%
  }%
}
%    \end{macrocode}
%    \begin{macrocode}
%</marks>
%    \end{macrocode}
%
% \subsection{Module \xmodule{runs}}
%
%    This module does not use the label-reference-system.
%    The reference changes with each \hologo{LaTeX}~run and would
%    force a rerun warning always.
%    \begin{macrocode}
%<*runs>
\NeedsTeXFormat{LaTeX2e}
\ProvidesPackage{zref-runs}%
  [2020-07-03 v2.32 Module runs for zref (HO)]%
%    \end{macrocode}
%
%    \begin{macro}{\zruns}
%    \begin{macrocode}
\providecommand*{\zruns}{0}%
\AtBeginDocument{%
  \edef\zruns{\number\numexpr\zruns+1}%
  \begingroup
    \def\on@line{}%
    \PackageInfo{zref-runs}{LaTeX runs: \zruns}%
    \if@filesw
      \immediate\write\@mainaux{%
        \string\gdef\string\zruns{\zruns}%
      }%
    \fi
  \endgroup
}
%    \end{macrocode}
%    \end{macro}
%
%    \begin{macrocode}
%</runs>
%    \end{macrocode}
%
%
% \subsection{Module \xmodule{perpage}}
%
%    \begin{macrocode}
%<*perpage>
\NeedsTeXFormat{LaTeX2e}
\ProvidesPackage{zref-perpage}%
  [2020-07-03 v2.32 Module perpage for zref (HO)]%
\RequirePackage{zref-base}[2019/11/29]
\ifx\ZREF@base@ok Y%
\else
  \expandafter\endinput
\fi
%    \end{macrocode}
%
%    This module resets a counter at page boundaries.
%    Because of the asynchronous output routine
%    page counter properties cannot be asked directly,
%    references are necessary.
%
%    For detecting changed pages module \xmodule{abspage}
%    is loaded.
%    \begin{macrocode}
\RequirePackage{zref-abspage}[2019/11/29]
%    \end{macrocode}
%
%    We group the properties for the needed references
%    in the property list |perpage|. The property |pagevalue|
%    records the correct value of the page counter.
%    \begin{macrocode}
\ZREF@NewPropPageValue
\zref@newlist{perpage}
\zref@addprops{perpage}{abspage,page,pagevalue}
%    \end{macrocode}
%    The page value, known by the reference mechanism, will be
%    stored in counter |zpage|.
%    \begin{macrocode}
\newcounter{zpage}
%    \end{macrocode}
%    Counter |zref@unique| helps in generating unique reference
%    names.
%    \begin{macrocode}
\zref@require@unique
%    \end{macrocode}
%    In order to be able to reset the counter, we hook here
%    into \cs{stepcounter}. In fact two nested hooks are used
%    to allow other packages to use the first hook at the beginning
%    of \cs{stepcounter}.
%    \begin{macrocode}
\let\ZREF@org@stepcounter\stepcounter
\def\stepcounter#1{%
  \ifcsname @stepcounterhook@#1\endcsname
    \csname @stepcounterhook@#1\endcsname
  \fi
  \ZREF@org@stepcounter{#1}%
}
%    \end{macrocode}
%    \cs{@stpelt} must be adapted due to the change in latex
%    2015-01, see https://github.com/ho-tex/zref/issues/26
%    \begin{macrocode}
\let\ZREF@org@@stpelt\@stpelt
\def\@stpelt#1{%
  \ifcsname ZREF@perpage@#1\endcsname
    \begingroup
      \let\stepcounter\ZREF@org@stepcounter
      \ZREF@org@@stpelt{#1}%
    \endgroup
    \expandafter\ltx@gobbletwo
  \fi
  \ZREF@org@@stpelt{#1}%
}
%    \end{macrocode}
%
%    \begin{macro}{\zmakeperpage}
%    Makro \cs{zmakeperpage} resets a counter at each page
%    break. It uses the same syntax and semantics as
%    \cs{MakePerPage} from package \xpackage{perpage} \cite{perpage}.
%    The initial start value can be given by the optional
%    argument. Default is one that means after the first
%    \cs{stepcounter} on a new page the counter starts with
%    one.
%    \begin{macrocode}
\ZREF@IfDefinable\zmakeperpage\def{%
  {%
    \@ifnextchar[\ZREF@makeperpage@opt{\ZREF@@makeperpage[\ltx@zero]}%
  }%
}
%    \end{macrocode}
%    \end{macro}
%    We hook before the counter is incremented in \cs{stepcounter},
%    package \xpackage{perpage} afterwards. Thus a little
%    calculation is necessary.
%    \begin{macrocode}
\def\ZREF@makeperpage@opt[#1]{%
  \begingroup
    \edef\x{\endgroup
      \noexpand\ZREF@@makeperpage[\number\numexpr#1-1\relax]%
    }%
  \x
}
%    \end{macrocode}
%    \begin{macrocode}
\def\ZREF@@makeperpage[#1]#2{%
  \@ifundefined{@stepcounterhook@#2}{%
    \expandafter\gdef\csname @stepcounterhook@#2\endcsname{}%
  }{}%
  \expandafter\gdef\csname ZREF@perpage@#2\endcsname{%
    \ZREF@@perpage@step{#2}{#1}%
  }%
  \expandafter\g@addto@macro\csname @stepcounterhook@#2\endcsname{%
    \ifcsname ZREF@perpage@#2\endcsname
      \csname ZREF@perpage@#2\endcsname
    \fi
  }%
}
%    \end{macrocode}
%    \begin{macro}{\ZREF@@perpage@step}
%    The heart of this module follows.
%    \begin{macrocode}
\def\ZREF@@perpage@step#1#2{%
%    \end{macrocode}
%    First the reference is generated.
%    \begin{macrocode}
  \global\advance\c@zref@unique\ltx@one
  \begingroup
    \expandafter
    \zref@labelbylist\expandafter{\thezref@unique}{perpage}%
%    \end{macrocode}
%    The \cs{expandafter} commands are necessary, because \cs{ZREF@temp}
%    is also used inside of \cs{zref@labelbylist}.
%
%    The evaluation of the reference follows. If the reference
%    is not yet kwown, we use the page counter as approximation.
%    \begin{macrocode}
    \zref@ifrefundefined\thezref@unique{%
      \global\c@zpage=\c@page
      \global\let\thezpage\thepage
      \expandafter\xdef\csname ZREF@abspage@#1\endcsname{%
        \number\c@abspage
      }%
    }{%
%    \end{macrocode}
%    The reference is used to set \cs{thezpage} and counter |zpage|.
%    \begin{macrocode}
      \global\c@zpage=\zref@extract\thezref@unique{pagevalue}\relax
      \xdef\thezpage{\noexpand\zref@extract{\thezref@unique}{page}}%
      \expandafter\xdef\csname ZREF@abspage@#1\endcsname{%
        \zref@extractdefault\thezref@unique
            {abspage}{\number\c@abspage}%
      }%
    }%
%    \end{macrocode}
%    Page changes are detected by a changed absolute page number.
%    \begin{macrocode}
    \expandafter\ifx\csname ZREF@abspage@#1\expandafter\endcsname
                    \csname ZREF@currentabspage@#1\endcsname
    \else
      \global\csname c@#1\endcsname=#2\relax
      \global\expandafter\let
          \csname ZREF@currentabspage@#1\expandafter\endcsname
          \csname ZREF@abspage@#1\endcsname
    \fi
  \endgroup
}
%    \end{macrocode}
%    \end{macro}
%    \begin{macro}{\zunmakeperpage}
%    Macro \cs{zunmakeperpage} cancels the effect of \cs{zmakeperpage}.
%    \begin{macrocode}
\ZREF@IfDefinable\zunmakeperpage\def{%
  #1{%
    \global\expandafter
    \let\csname ZREF@perpage@#1\endcsname\@undefined
  }%
}
%    \end{macrocode}
%    \end{macro}
%
%    \begin{macrocode}
%</perpage>
%    \end{macrocode}
%
%
% \subsection{Module \xmodule{titleref}}
%
%    \begin{macrocode}
%<*titleref>
\NeedsTeXFormat{LaTeX2e}
\ProvidesPackage{zref-titleref}%
  [2020-07-03 v2.32 Module titleref for zref (HO)]%
\RequirePackage{zref-base}[2019/11/29]
\ifx\ZREF@base@ok Y%
\else
  \expandafter\endinput
\fi
\RequirePackage{gettitlestring}[2009/12/08]
%    \end{macrocode}
%
% \subsubsection{Implementation}
%    \begin{macrocode}
\RequirePackage{keyval}
%    \end{macrocode}
%
%    This module makes section and caption titles available
%    for the reference system. It uses some of the ideas
%    of package \xpackage{nameref} and \xpackage{titleref}.
%
%    Now we can add the property \verb|title| is added to
%    the main property list.
%    \begin{macrocode}
\ZREF@NewPropTitle
\zref@addprop\ZREF@mainlist{title}%
%    \end{macrocode}
%
%    The title strings go into the \xfile{.aux} file, thus
%    they need some kind of protection.
%    Package \xpackage{titleref} uses a protected expansion
%    method. The advantage is that this can be used to
%    cleanup the string and to remove \cs{label}, \cs{index}
%    and other macros unwanted for referencing.
%    But there is the risk that fragile stuff can break.
%
%    Therefore package \xpackage{nameref} does not expand
%    the string. Thus the entries can safely be written to
%    the \xfile{.aux} file. But potentially dangerous
%    macros such as \cs{label} remain in the string and
%    can cause problems when using the string in references.
%
%    \begin{macro}{\ifzref@titleref@expand}
%    The switch \cs{ifzref@titleref@expand} distinguishes
%    between the both methods. Package \xpackage{nameref}'s
%    behaviour is achieved by setting the switch to false,
%    otherwise \xpackage{titleref}'s expansion is used.
%    Default is false.
%    \begin{macrocode}
\newif\ifzref@titleref@expand
%    \end{macrocode}
%    \end{macro}
%
%    \begin{macro}{\ZREF@titleref@hook}
%    The hook \cs{ZREF@titleref@hook} allows to extend
%    the cleanup for the expansion method. Thus
%    unnecessary macros can be removed or dangerous commands
%    removed. The hook is executed before the expansion of
%    \cs{zref@titleref@current}.
%    \begin{macrocode}
\let\ZREF@titleref@hook\ltx@empty
%    \end{macrocode}
%    \end{macro}
%    \begin{macro}{\zref@titleref@cleanup}
%    The hook should not be used directly, instead we provide the
%    macro \cs{zref@titleref@cleanup} to add stuff to the hook
%    and prevents that a previous non-empty content is not discarded
%    accidently.
%    \begin{macrocode}
\ZREF@Robust\def\zref@titleref@cleanup#1{%
   \begingroup
    \toks@\expandafter{%
      \ZREF@titleref@hook
      #1%
    }%
  \expandafter\endgroup
  \expandafter\def\expandafter\ZREF@titleref@hook\expandafter{%
    \the\toks@
  }%
}%
%    \end{macrocode}
%    \end{macro}
%
%    \begin{macro}{\ifzref@titleref@stripperiod}
%    Sometimes a title contains a period at the end.
%    Package \xpackage{nameref} removes this. This behaviour
%    is controlled by the switch \cs{ifzref@titleref@stripperiod}
%    and works regardless of the setting of option \xoption{expand}.
%    Period stripping is the default.
%    \begin{macrocode}
\newif\ifzref@titleref@stripperiod
\zref@titleref@stripperiodtrue
%    \end{macrocode}
%    \end{macro}
%
%    \begin{macro}{\zref@titleref@setcurrent}
%    Macro \cs{zref@titleref@setcurrent} sets a new current title
%    stored in \cs{zref@titleref@current}. Some cleanup and
%    expansion is performed that can be controlled by the
%    previous switches.
%    \begin{macrocode}
\ZREF@Robust\def\zref@titleref@setcurrent#1{%
  \ifzref@titleref@expand
    \GetTitleStringExpand{#1}%
  \else
    \GetTitleStringNonExpand{#1}%
  \fi
  \edef\zref@titleref@current{%
    \detokenize\expandafter{\GetTitleStringResult}%
  }%
  \ifzref@titleref@stripperiod
    \edef\zref@titleref@current{%
      \expandafter\ZREF@stripperiod\zref@titleref@current
      \ltx@empty.\ltx@empty\@nil
    }%
  \fi
}%
\GetTitleStringDisableCommands{%
  \ZREF@titleref@hook
}
%    \end{macrocode}
%    \end{macro}
%    \begin{macro}{\ZREF@stripperiod}
%    If \cs{ZREF@stripperiod} is called, the argument consists of
%    space tokens and tokens with catcode 12 (other), because
%    of \hologo{eTeX}'s \cs{detokenize}.
%    \begin{macrocode}
\def\ZREF@stripperiod#1.\ltx@empty#2\@nil{#1}%
%    \end{macrocode}
%    \end{macro}
%
% \subsubsection{User interface}
%
%    \begin{macro}{\ztitlerefsetup}
%    The behaviour of module \xmodule{titleref} is controlled
%    by switches and a hook. They can be set by \cs{ztitlerefsetup}
%    with a key value interface, provided by package \xpackage{keyval}.
%    Also the current title can be given explicitly by the key
%    \verb|title|.
%    \begin{macrocode}
\define@key{ZREF@TR}{expand}[true]{%
  \csname zref@titleref@expand#1\endcsname
}%
\define@key{ZREF@TR}{stripperiod}[true]{%
  \csname zref@titleref@stripperiod#1\endcsname
}%
\define@key{ZREF@TR}{cleanup}{%
  \zref@titleref@cleanup{#1}%
}%
\define@key{ZREF@TR}{title}{%
  \def\zref@titleref@current{#1}%
}%
\ZREF@IfDefinable\ztitlerefsetup\def{%
  {\kvsetkeys{ZREF@TR}}%
}%
%    \end{macrocode}
%    \end{macro}
%
%    \begin{macro}{\ztitleref}
%    The user command \cs{ztitleref} references the title.
%    For safety \cs{label} is disabled to prevent multiply
%    defined references.
%    \begin{macrocode}
\ZREF@IfDefinable\ztitleref\def{%
  {\zref@wrapper@babel\ZREF@titleref}%
}%
\def\ZREF@titleref#1{%
  \begingroup
    \zref@refused{#1}%
    \let\label\ltx@gobble
    \zref@extract{#1}{title}%
  \endgroup
}%
%    \end{macrocode}
%    \end{macro}
%
% \subsubsection{Patches for section and caption commands}
%
%    The section and caption macros are patched to extract the
%    title data.
%
%    Captions of figures and tables.
%    \begin{macrocode}
\AtBeginDocument{%
  \ZREF@patch{@caption}{%
    \long\def\@caption#1[#2]{%
      \zref@titleref@setcurrent{#2}%
      \ZREF@org@@caption{#1}[{#2}]%
    }%
  }%
%    \end{macrocode}
%
%    Section commands without star. The title version for the
%    table of contents is used because it is usually shorter and
%    more robust.
%    \begin{macrocode}
  \ZREF@patch{@part}{%
    \def\@part[#1]{%
      \zref@titleref@setcurrent{#1}%
      \ZREF@org@@part[{#1}]%
    }%
  }%
  \ZREF@patch{@chapter}{%
    \def\@chapter[#1]{%
      \zref@titleref@setcurrent{#1}%
      \ZREF@org@@chapter[{#1}]%
    }%
  }%
  \ZREF@patch{@sect}{%
    \def\@sect#1#2#3#4#5#6[#7]{%
      \zref@titleref@setcurrent{#7}%
      \ZREF@org@@sect{#1}{#2}{#3}{#4}{#5}{#6}[{#7}]%
    }%
  }%
%    \end{macrocode}
%    The star versions of the section commands.
%    \begin{macrocode}
  \ZREF@patch{@spart}{%
    \def\@spart#1{%
      \zref@titleref@setcurrent{#1}%
      \ZREF@org@@spart{#1}%
    }%
  }%
  \ZREF@patch{@schapter}{%
    \def\@schapter#1{%
      \zref@titleref@setcurrent{#1}%
      \ZREF@org@@schapter{#1}%
    }%
  }%
  \ZREF@patch{@ssect}{%
    \def\@ssect#1#2#3#4#5{%
      \zref@titleref@setcurrent{#5}%
      \ZREF@org@@ssect{#1}{#2}{#3}{#4}{#5}%
    }%
  }%
%    \end{macrocode}
%
% \subsubsection{Environment \textsf{description}}
%
%    \begin{macrocode}
  \ZREF@patch{descriptionlabel}{%
    \def\descriptionlabel#1{%
      \zref@titleref@setcurrent{#1}%
      \ZREF@org@descriptionlabel{#1}%
    }%
  }%
%    \end{macrocode}
%
% \subsubsection{Class \xclass{memoir}}
%
%    \begin{macrocode}
  \@ifclassloaded{memoir}{%
    \ltx@IfUndefined{ifheadnameref}{}{%
      \def\@chapter[#1]#2{%
        \ltx@IfUndefined{ch@pt@c}{%
          \zref@titleref@setcurrent{#1}%
        }{%
          \ifx\ch@pt@c\ltx@empty
            \zref@titleref@setcurrent{#2}%
          \else
            \def\NR@temp{#1}%
            \ifx\NR@temp\ltx@empty
              \expandafter\zref@titleref@setcurrent
              \expandafter{\ch@pt@c}%
            \else
              \ifheadnameref
                \zref@titleref@setcurrent{#1}%
              \else
                \expandafter\zref@titleref@setcurrent
                \expandafter{\ch@pt@c}%
              \fi
            \fi
          \fi
        }%
        \ZREF@org@@chapter[{#1}]{#2}%
      }%
      \ZREF@patch{M@sect}{%
        \def\M@sect#1#2#3#4#5#6[#7][#8]{%
          \ifheadnameref
            \zref@titleref@setcurrent{#8}%
          \else
            \zref@titleref@setcurrent{#7}%
          \fi
          \ZREF@org@M@sect{#1}{#2}{#3}{#4}{#5}{#6}[{#7}][{#8}]%
        }%
      }%
    }%
  }{}%
%    \end{macrocode}
%
% \subsubsection{Class \xclass{beamer}}
%
%    \begin{macrocode}
  \@ifclassloaded{beamer}{%
    \ZREF@patch{beamer@section}{%
      \long\def\beamer@section[#1]{%
        \zref@titleref@setcurrent{#1}%
        \ZREF@org@beamer@section[{#1}]%
      }%
    }%
    \ZREF@patch{beamer@subsection}{%
      \long\def\beamer@subsection[#1]{%
        \zref@titleref@setcurrent{#1}%
        \ZREF@org@beamer@subsection[{#1}]%
      }%
    }%
    \ZREF@patch{beamer@subsubsection}{%
      \long\def\beamer@subsubsection[#1]{%
        \zref@titleref@setcurrent{#1}%
        \ZREF@org@beamer@subsubsection[{#1}]%
      }%
    }%
  }{}%
%    \end{macrocode}
%
% \subsubsection{Package \xpackage{titlesec}}
%
%    \begin{macrocode}
  \@ifpackageloaded{titlesec}{%
    \ZREF@patch{ttl@sect@i}{%
      \def\ttl@sect@i#1#2[#3]#4{%
        \zref@titleref@setcurrent{#4}%
        \ZREF@org@ttl@sect@i{#1}{#2}[{#3}]{#4}%
      }%
    }%
    \ZREF@patch{ttl@straight@i}{%
      \def\ttl@straight@i#1[#2]#3{%
        \def\ZREF@temp{#2}%
        \ifx\ZREF@temp\ltx@empty
          \zref@titleref@setcurrent{#3}%
        \else
          \zref@titleref@setcurrent{#2}%
        \fi
        \ZREF@org@ttl@straight@i{#1}[{#2}]{#3}%
      }%
    }%
  }{}%
%    \end{macrocode}
%
% \subsubsection{Package \xpackage{longtable}}
%
%    Package \xpackage{longtable}: some support for its \cs{caption}.
%    However \cs{label} inside the caption is not supported.
%    \begin{macrocode}
  \@ifpackageloaded{longtable}{%
    \ZREF@patch{LT@c@ption}{%
      \def\LT@c@ption#1[#2]#3{%
        \ZREF@org@LT@c@ption{#1}[{#2}]{#3}%
        \zref@titleref@setcurrent{#2}%
      }%
    }%
  }{}%
%    \end{macrocode}
%
% \subsubsection{Package \xpackage{listings}}
%
%    Package \xpackage{listings}: support for its caption.
%    \begin{macrocode}
  \@ifpackageloaded{listings}{%
    \ZREF@patch{lst@MakeCaption}{%
      \def\lst@MakeCaption{%
        \ifx\lst@label\ltx@empty
        \else
          \expandafter\zref@titleref@setcurrent\expandafter{%
            \lst@@caption
          }%
        \fi
        \ZREF@org@lst@MakeCaption
      }%
    }%
  }{}%
%    \end{macrocode}
%
% \subsubsection{Theorems}
%
%    \begin{macrocode}
  \ZREF@patch{@opargbegintheorem}{%
    \def\@opargbegintheorem#1#2#3{%
      \zref@titleref@setcurrent{#3}%
      \ZREF@org@@opargbegintheorem{#1}{#2}{#3}%
    }%
  }%
  \@ifpackageloaded{amsthm}{%
    \begingroup
      \edef\x{macro:\string#1\string#2[\string#3]}%
      \@onelevel@sanitize\x
      \def\y#1->#2\@nil{#1}%
      \edef\z{\expandafter\y\meaning\@begintheorem->\@nil}%
      \@onelevel@sanitize\z
    \expandafter\endgroup
    \ifx\x\z
      \ZREF@patch{@begintheorem}{%
        \def\@begintheorem#1#2[#3]{%
          \zref@titleref@setcurrent{#3}%
          \ZREF@org@@begintheorem{#1}{#2}[{#3}]%
        }%
      }%
    \fi
  }{}%
%    \end{macrocode}
%
%    \begin{macrocode}
}
%    \end{macrocode}
%
%    \begin{macrocode}
%</titleref>
%    \end{macrocode}
%
%
% \subsection{Module \xmodule{xr}}
%
%    \begin{macrocode}
%<*xr>
\NeedsTeXFormat{LaTeX2e}
\ProvidesPackage{zref-xr}%
  [2020-07-03 v2.32 Module xr for zref (HO)]%
\RequirePackage{zref-base}[2019/11/29]
\ifx\ZREF@base@ok Y%
\else
  \expandafter\endinput
\fi
%    \end{macrocode}
%    \begin{macrocode}
\RequirePackage{keyval}
\RequirePackage{kvoptions}[2010/02/22]
%    \end{macrocode}
%
%    We declare property |url|, because this is added,
%    if a reference is imported and has not already set this field.
%    Or if |hyperref| is used, then this property can be asked.
%    \begin{macrocode}
\zref@newprop{url}{}%
\zref@newprop{urluse}{}%
%    \end{macrocode}
%    \begin{macrocode}
\zref@newprop{externaldocument}{}%
%    \end{macrocode}
%
%    Most code, especially the handling of the \xfile{.aux}
%    files are taken from David Carlisle's \xpackage{xr} package.
%    Therefore I drop the documentation for these macros here.
%
%    \begin{macro}{\zref@xr@ext}
%    If the URL is not specied, then assume
%    processed file with a guessed extension.
%    Use the setting of \xpackage{hyperref} if available.
%    \begin{macrocode}
\providecommand*{\zref@xr@ext}{%
  \ltx@ifundefined{XR@ext}{pdf}{\XR@ext}%
}%
%    \end{macrocode}
%    \end{macro}
%    \begin{macro}{\ifZREF@xr@zreflabel}
%    The use of the star form of \cs{zexternaldocument}
%    is remembered in the switch \cs{ifZREF@xr@zreflabel}.
%    \begin{macrocode}
\newif\ifZREF@xr@zreflabel
%    \end{macrocode}
%    \end{macro}
%    \begin{macrocode}
\SetupKeyvalOptions{%
  family=ZREF@XR,%
  prefix=ZREF@xr@%
}
\DeclareBoolOption[true]{tozreflabel}
\DeclareBoolOption[false]{toltxlabel}
\DeclareBoolOption{verbose}
\define@key{ZREF@XR}{ext}{%
  \def\zref@xr@{#1}%
}
\DeclareBoolOption[false]{urluse}
%    \end{macrocode}
%    \begin{macro}{\zxrsetup}
%    \begin{macrocode}
\newcommand*{\zxrsetup}{%
  \kvsetkeys{ZREF@XR}%
}%
%    \end{macrocode}
%    \end{macro}
%
%    \begin{macro}{\ZREF@xr@URL}
%    \begin{macrocode}
\newcount\ZREF@xr@URL
\ZREF@xr@URL=\ltx@zero
%    \end{macrocode}
%    \end{macro}
%    \begin{macro}{\ZREF@xr@AddURL}
%    \begin{macrocode}
\def\ZREF@xr@AddURL#1{%
  \begingroup
    \def\ZREF@temp{#1}%
    \count@=\ltx@one
    \ZREF@xr@@AddUrl
  \endgroup
}
%    \end{macrocode}
%    \end{macro}
%    \begin{macro}{\ZREF@xr@@AddUrl}
%    \begin{macrocode}
\def\ZREF@xr@@AddUrl{%
  \ifnum\count@>\ZREF@xr@URL
    \global\advance\ZREF@xr@URL by\ltx@one
    \xdef\ZREF@xr@theURL{\romannumeral\ZREF@xr@URL}%
    \global\expandafter\let
        \csname Z@U@\ZREF@xr@theURL\endcsname\ZREF@temp
    \@PackageInfo{zref-xr}{%
      \ltx@backslashchar Z@U@\ZREF@xr@theURL:\MessageBreak
      \ZREF@temp\MessageBreak
    }%
  \else
    \expandafter
    \ifx\csname Z@U@\romannumeral\count@\endcsname\ZREF@temp
      \xdef\ZREF@xr@theURL{\romannumeral\count@}%
    \else
      \expandafter\expandafter\expandafter\ZREF@xr@@AddUrl
    \fi
  \fi
}
%    \end{macrocode}
%    \end{macro}
%
%    \begin{macro}{\zexternaldocument}
%    In its star form it looks for \cs{newlabel},
%    otherwise for \cs{zref@newlabel}.
%     Later we
%    will read \xfile{.aux} files that expects
%    |@| to have catcode 11 (letter).
%    \begin{macrocode}
\ZREF@IfDefinable\zexternaldocument\def{%
  {%
    \ZREF@NewPropAnchor
    \ZREF@NewPropTitle
    \begingroup
      \csname @safe@actives@true\endcsname
      \makeatletter
      \@ifstar{%
        \ZREF@xr@zreflabelfalse
        \@testopt\ZREF@xr@externaldocument{}%
      }{%
        \ZREF@xr@zreflabeltrue
        \@testopt\ZREF@xr@externaldocument{}%
      }%
  }%
}%
%    \end{macrocode}
%    \end{macro}
%
%    If the \cs{include} featuer was used, there can be
%    several \xfile{.aux} files. These files are read
%    one after another, especially they are not recursively
%    read in order to save read registers. Thus it can
%    happen that the read order of the newlabel commands
%    differs from \hologo{LaTeX}'s order using \cs{input}.
%
%    \begin{macro}{\ZREF@xr@externaldocument}
%    It reads the remaining arguments. \cs{newcommand}
%    comes in handy for the optional argument.
%    \begin{macrocode}
\def\ZREF@xr@externaldocument[#1]#2{%
    \def\ZREF@xr@prefix{#1}%
    \let\ZREF@xr@filelist\ltx@empty
    \edef\ZREF@xr@externalfile{#2}%
    \edef\ZREF@xr@file{\ZREF@xr@externalfile.aux}%
    \filename@parse{#2}%
    \@testopt\ZREF@xr@graburl{#2.\zref@xr@ext}%
}%
\def\ZREF@xr@graburl[#1]{%
    \edef\ZREF@xr@url{#1}%
    \ifZREF@xr@urluse
      \expandafter\ZREF@xr@AddURL\expandafter{\ZREF@xr@url}%
      \expandafter\def\expandafter\ZREF@xr@url
      \expandafter{\csname Z@U@\ZREF@xr@theURL\endcsname}%
    \fi
    \ZREF@xr@checkfile
  \endgroup
}%
%    \end{macrocode}
%    \end{macro}
%    \begin{macro}{\ZREF@xr@processfile}
%    We follow \xpackage{xr} here, \cs{IfFileExists} offers
%    a nicer test, but we have to open the file anyway.
%    \begin{macrocode}
\def\ZREF@xr@checkfile{%
  \openin\@inputcheck\ZREF@xr@file\relax
  \ifeof\@inputcheck
    \PackageWarning{zref-xr}{%
      File `\ZREF@xr@file' not found or empty,\MessageBreak
      labels not imported%
    }%
  \else
    \PackageInfo{zref-xr}{%
      Label \ifZREF@xr@zreflabel (zref) \fi
      import from `\ZREF@xr@file'%
    }%
    \def\ZREF@xr@found{0}%
    \def\ZREF@xr@ignored@empty{0}%
    \def\ZREF@xr@ignored@zref{0}%
    \def\ZREF@xr@ignored@ltx{0}%
    \ZREF@xr@processfile
    \closein\@inputcheck
    \begingroup
      \let\on@line\ltx@empty
      \PackageInfo{zref-xr}{%
        Statistics for `\ZREF@xr@file':\MessageBreak
        \ZREF@xr@found\space
        \ifZREF@xr@zreflabel zref\else LaTeX\fi\space
        label(s) found%
        \ifnum\ZREF@xr@ignored@empty>0 %
          ,\MessageBreak
          \ZREF@xr@ignored@empty\space empty label(s) ignored%
        \fi
        \ifnum\ZREF@xr@ignored@zref>0 %
          ,\MessageBreak
          \ZREF@xr@ignored@zref\space
          duplicated zref label(s) ignored%
        \fi
        \ifnum\ZREF@xr@ignored@ltx>0 %
          ,\MessageBreak
          \ZREF@xr@ignored@ltx\space
          duplicated latex label(s) ignored%
        \fi
      }%
    \endgroup
  \fi
  \ifx\ZREF@xr@filelist\ltx@empty
  \else
    \edef\ZREF@xr@file{%
      \expandafter\ltx@car\ZREF@xr@filelist\@nil
    }%
    \edef\ZREF@xr@filelist{%
      \expandafter\ltx@cdr\ZREF@xr@filelist\ltx@empty\@nil
    }%
    \expandafter\ZREF@xr@checkfile
  \fi
}%
%    \end{macrocode}
%    \end{macro}
%
%    \begin{macro}{\ZREF@xr@processfile}
%    \begin{macrocode}
\def\ZREF@xr@processfile{%
  \read\@inputcheck to\ZREF@xr@line
  \expandafter\ZREF@xr@processline\ZREF@xr@line..\ZREF@nil
  \ifeof\@inputcheck
  \else
    \expandafter\ZREF@xr@processfile
  \fi
}%
%    \end{macrocode}
%    \end{macro}
%    \begin{macro}{\ZREF@xr@processline}
%    The most work must be done for analyzing the
%    arguments of \cs{newlabel}.
%    \begin{macrocode}
\long\def\ZREF@xr@processline#1#2#3\ZREF@nil{%
  \def\x{#1}%
  \toks@{#2}%
  \ifZREF@xr@zreflabel
    \ifx\x\ZREF@xr@zref@newlabel
      \expandafter
      \ZREF@xr@process@zreflabel\ZREF@xr@line...\ZREF@nil
    \fi
  \else
    \ifx\x\ZREF@xr@newlabel
      \expandafter
      \ZREF@xr@process@label\ZREF@xr@line...[]\ZREF@nil
    \fi
  \fi
  \ifx\x\ZREF@xr@@input
    \edef\ZREF@xr@filelist{%
      \etex@unexpanded\expandafter{\ZREF@xr@filelist}%
      {\filename@area\the\toks@}%
    }%
  \fi
}%
\def\ZREF@xr@process@zreflabel\zref@newlabel#1#2#3\ZREF@nil{%
  \edef\ZREF@xr@refname{Z@R@\ZREF@xr@prefix#1}%
  \edef\ZREF@xr@found{\the\numexpr\ZREF@xr@found+1\relax}%
  \def\x{#2}%
  \edef\ZREF@xr@tempname{$temp$}%
  \edef\ZREF@xr@temprefname{Z@R@\ZREF@xr@tempname}%
  \let\ZREF@xr@list\x
  \ifx\ZREF@xr@list\ltx@empty
    \PackageWarningNoLine{zref-xr}{%
      Label `#1' without properties ignored\MessageBreak
      in file `\ZREF@xr@file'%
    }%
    \edef\ZREF@xr@ignored@empty{%
      \the\numexpr\ZREF@xr@ignored@empty+1\relax
    }%
  \else
    \expandafter\ZREF@xr@checklist\x\ZREF@nil
    \expandafter\let\csname\ZREF@xr@temprefname\endcsname\x
    \expandafter\ltx@LocalAppendToMacro
    \csname\ZREF@xr@temprefname\expandafter\endcsname
    \expandafter{%
      \expandafter\externaldocument\expandafter{%
        \ZREF@xr@externalfile
      }%
    }%
    \ZREF@xr@urlcheck\ZREF@xr@tempname
    \ifZREF@xr@tozreflabel
      \@ifundefined{\ZREF@xr@refname}{%
        \ifZREF@xr@verbose
          \PackageInfo{zref-xr}{%
            Import to zref label `\ZREF@xr@tempname#1'%
          }%
        \fi
        \global\expandafter
        \let\csname\ZREF@xr@refname\expandafter\endcsname
        \csname\ZREF@xr@temprefname\endcsname
      }{%
        \ZREF@xr@zref@ignorewarning{\ZREF@xr@prefix#1}%
      }%
    \fi
    \ifZREF@xr@toltxlabel
      \ZREF@xr@tolabel{\ZREF@xr@tempname}{\ZREF@xr@prefix#1}%
    \fi
  \fi
}%
\def\ZREF@xr@process@label\newlabel#1#2#3[#4]#5\ZREF@nil{%
  \def\ZREF@xr@refname{Z@R@\ZREF@xr@prefix#1}%
  \edef\ZREF@xr@found{\the\numexpr\ZREF@xr@found+1\relax}%
  \def\x{#2}%
  \edef\ZREF@xr@tempname{$temp$}%
  \edef\ZREF@xr@temprefname{Z@R@\ZREF@xr@tempname}%
  \expandafter\ZREF@xr@scanparams
      \csname\ZREF@xr@temprefname\expandafter\endcsname
      \x{}{}{}{}{}\ZREF@nil
  \ifx\\#4\\%
  \else
    % ntheorem knows an optional argument at the end of \newlabel
    \ZREF@NewPropTheotype
    \expandafter\ltx@LocalAppendToMacro
        \csname\ZREF@xr@temprefname\endcsname{\theotype{#4}}%
  \fi
  \expandafter\ltx@LocalAppendToMacro
  \csname\ZREF@xr@temprefname\expandafter\endcsname\expandafter{%
    \expandafter\externaldocument\expandafter{%
      \ZREF@xr@externalfile
    }%
  }%
  \ZREF@xr@urlcheck\ZREF@xr@tempname
  \ifZREF@xr@tozreflabel
    \@ifundefined{\ZREF@xr@refname}{%
      \ifZREF@xr@verbose
        \PackageInfo{zref-xr}{%
          Import to zref label `\ZREF@xr@prefix#1'%
        }%
      \fi
      \global\expandafter
      \let\csname\ZREF@xr@refname\expandafter\endcsname
      \csname\ZREF@xr@temprefname\endcsname
    }{%
      \ZREF@xr@zref@ignorewarning{\ZREF@xr@prefix#1}%
    }%
  \fi
  \ifZREF@xr@toltxlabel
    \ZREF@xr@tolabel{\ZREF@xr@tempname}{\ZREF@xr@prefix#1}%
  \fi
}
\def\ZREF@xr@zref@newlabel{\zref@newlabel}%
\def\ZREF@xr@newlabel{\newlabel}%
\def\ZREF@xr@@input{\@input}%
\def\ZREF@xr@relax{\relax}%
%    \end{macrocode}
%    \end{macro}
%    \begin{macro}{\ZREF@xr@tolabel}
%    \begin{macrocode}
\def\ZREF@xr@tolabel#1#2{%
  \ifZREF@xr@verbose
    \PackageInfo{zref-xr}{%
      Import to LaTeX label `#2'%
    }%
  \fi
  \zref@wrapper@unexpanded{%
    \expandafter\xdef\csname r@#2\endcsname{%
      {%
        \ltx@ifundefined{M@TitleReference}{%
          \ltx@ifundefined{TR@TitleReference}{%
            \zref@extractdefault{#1}{default}{}%
          }{%
            \noexpand\TR@TitleReference
            {\zref@extractdefault{#1}{default}{}}%
            {\zref@extractdefault{#1}{title}{}}%
          }%
        }{%
          \noexpand\M@TitleReference
          {\zref@extractdefault{#1}{default}{}}%
          {\zref@extractdefault{#1}{title}{}}%
        }%
      }%
      {\zref@extractdefault{#1}{page}{}}%
      \ltx@ifpackageloaded{nameref}{%
        {\zref@extractdefault{#1}{title}{}}%
        {\zref@extractdefault{#1}{anchor}{}}%
        \zref@ifrefcontainsprop{#1}{urluse}{%
          {\zref@extractdefault{#1}{urluse}{}}%
        }{%
          {\zref@extractdefault{#1}{url}{}}%
        }%
      }{}%
    }%
  }%
}
%    \end{macrocode}
%    \end{macro}
%
%    \begin{macro}{\ZREF@xr@zref@ignorewarning}
%    \begin{macrocode}
\def\ZREF@xr@zref@ignorewarning#1{%
  \PackageWarningNoLine{zref-xr}{%
    Zref label `#1' is already in use\MessageBreak
    in file `\ZREF@xr@file'%
  }%
  \edef\ZREF@xr@ignored@zref{%
    \the\numexpr\ZREF@xr@ignored@zref+1%
  }%
}%
%    \end{macrocode}
%    \end{macro}
%
%    \begin{macro}{\ZREF@xr@ltx@ignorewarning}
%    \begin{macrocode}
\def\ZREF@xr@ltx@ignorewarning#1{%
  \PackageWarningNoLine{zref-xr}{%
    LaTeX label `#1' is already in use\MessageBreak
    in file `\ZREF@xr@file'%
  }%
  \edef\ZREF@xr@ignored@ltx{%
    \the\numexpr\ZREF@xr@ignored@ltx+1%
  }%
}%
%    \end{macrocode}
%    \end{macro}
%
%    \begin{macro}{\ZREF@xr@checklist}
%    \begin{macrocode}
\def\ZREF@xr@checklist#1#2#3\ZREF@nil{%
  \ifx\@undefined#1\relax
    \expandafter\ZREF@xr@checkkey\string#1\@nil
  \fi
  \ifx\\#3\\%
  \else
    \ltx@ReturnAfterFi{%
      \ZREF@xr@checklist#3\ZREF@nil
    }%
  \fi
}%
\def\ZREF@xr@checkkey#1#2\@nil{%
  \zref@ifpropundefined{#2}{%
    \zref@newprop{#2}{}%
  }{}%
}%
%    \end{macrocode}
%    \end{macro}
%
%    \begin{macro}{\ZREF@xr@scanparams}
%    \begin{macrocode}
\def\ZREF@xr@scanparams#1#2#3#4#5#6#7\ZREF@nil{%
  \let#1\ltx@empty
  \ZREF@foundfalse
  \ZREF@xr@scantitleref#1#2\TR@TitleReference{}{}\ZREF@nil
  \ifZREF@found
  \else
    \ltx@LocalAppendToMacro#1{\default{#2}}%
  \fi
  % page
  \ltx@LocalAppendToMacro#1{\page{#3}}%
  % nameref title
  \ifZREF@found
  \else
    \ifx\\#4\\%
    \else
      \def\ZREF@xr@temp{#4}%
      \ifx\ZREF@xr@temp\ZREF@xr@relax
      \else
        \ltx@LocalAppendToMacro#1{\title{#4}}%
      \fi
    \fi
  \fi
  % anchor
  \ifx\\#5\\%
  \else
    \ltx@LocalAppendToMacro#1{\anchor{#5}}%
  \fi
  \ifx\\#6\\%
  \else
    \ifZREF@xr@urluse
      \ZREF@xr@AddURL{#6}%
      \expandafter\ltx@LocalAppendToMacro\expandafter#1%
      \expandafter{%
        \expandafter\urluse\expandafter{%
          \csname Z@U@\ZREF@xr@theURL\endcsname
        }%
      }%
    \else
      \ltx@LocalAppendToMacro#1{\url{#6}}%
    \fi
  \fi
}%
%    \end{macrocode}
%    \end{macro}
%
%    \begin{macro}{\ZREF@xr@scantitleref}
%    \begin{macrocode}
\def\ZREF@xr@scantitleref#1#2\TR@TitleReference#3#4#5\ZREF@nil{%
  \ifx\\#5\\%
  \else
    \ltx@LocalAppendToMacro#1{%
      \default{#3}%
      \title{#4}%
    }%
    \ZREF@foundtrue
  \fi
}%
%    \end{macrocode}
%    \end{macro}
%
%    \begin{macro}{\ZREF@xr@urlcheck}%
%    \begin{macrocode}
\def\ZREF@xr@urlcheck#1{%
  \zref@ifrefcontainsprop{#1}{anchor}{%
    \zref@ifrefcontainsprop{#1}{url}{%
    }{%
      \expandafter
      \ltx@LocalAppendToMacro\csname Z@R@#1\expandafter\endcsname
      \expandafter{%
        \csname url\ifZREF@xr@urluse use\fi
        \expandafter\endcsname\expandafter{\ZREF@xr@url}%
      }%
    }%
  }{%
  }%
}%
%    \end{macrocode}
%    \end{macro}
%
%    \begin{macrocode}
%</xr>
%    \end{macrocode}
%
%
% \subsection{Module \xmodule{hyperref}}
%
%    UNFINISHED :-(
%
%    \begin{macrocode}
%<*hyperref>
\NeedsTeXFormat{LaTeX2e}
\ProvidesPackage{zref-hyperref}%
  [2020-07-03 v2.32 Module hyperref for zref (HO)]%
\RequirePackage{zref-base}[2019/11/29]
\ifx\ZREF@base@ok Y%
\else
  \expandafter\endinput
\fi
%    \end{macrocode}
%
%    \begin{macrocode}
\ZREF@NewPropAnchor
\zref@addprop\ZREF@mainlist{anchor}%
%    \end{macrocode}
%
%    \begin{macrocode}
%</hyperref>
%    \end{macrocode}
%
%
% \subsection{Module \xmodule{savepos}}
%
%    Module \xmodule{savepos} provides an interface for
%    \pdfTeX's \cmd{\pdfsavepos}, see the manual for
%    \pdfTeX.
%
% \subsubsection{Identification}
%
%    \begin{macrocode}
%<*savepos>
\NeedsTeXFormat{LaTeX2e}
\ProvidesPackage{zref-savepos}%
  [2020-07-03 v2.32 Module savepos for zref (HO)]%
\RequirePackage{zref-base}[2019/11/29]
\ifx\ZREF@base@ok Y%
\else
  \expandafter\endinput
\fi
%    \end{macrocode}
%
% \subsubsection{Availability}
%
%    First we check, whether the feature is available.
%    \begin{macrocode}
\ifx\directlua\@undefined
\ltx@IfUndefined{pdfsavepos}{%
  \PackageError\ZREF@name{%
    \string\pdfsavepos\space is not supported.\MessageBreak
    It is provided by pdfTeX (1.40) or XeTeX%
  }\ZREF@UpdatePdfTeX
  \endinput
}{}%
\fi
%    \end{macrocode}
%    In PDF mode we are done. However support for DVI mode
%    was added later in version 1.40.0. In earlier versions
%    \cs{pdfsavepos} is defined, but its execution raises
%    an error. Note that \hologo{XeTeX} also provides \cs{pdfsavepos}.
%    \begin{macrocode}
\ifpdf
\else
  \ltx@IfUndefined{pdftexversion}{%
  }{%
    \ifnum\pdftexversion<140 %
      \PackageError\ZREF@name{%
        \string\pdfsavepos\space is not supported in DVI mode%
        \MessageBreak
        of this pdfTeX version%
      }\ZREF@UpdatePdfTeX
      \expandafter\expandafter\expandafter\endinput
    \fi
  }%
\fi
%    \end{macrocode}
%
% \subsubsection{Setup}
%
%    \begin{macrocode}
\zref@newlist{savepos}
\ifx\directlua\@undefined
  \zref@newprop*{posx}[0]{\the\pdflastxpos}
  \zref@newprop*{posy}[0]{\the\pdflastypos}
\else
  \zref@newprop*{posx}[0]{\the\lastxpos}
  \zref@newprop*{posy}[0]{\the\lastypos}
\fi
\zref@addprops{savepos}{posx,posy}
%    \end{macrocode}
%
% \subsubsection{User macros}
%
%    \begin{macro}{\zref@savepos}
%    \begin{macrocode}
\ifx\directlua\@undefined
  \def\zref@savepos{%
    \if@filesw
      \pdfsavepos
    \fi
  }
\else
  \def\zref@savepos{%
    \if@filesw
      \savepos
    \fi
  }
\fi
%    \end{macrocode}
%    \end{macro}
%
%    \begin{macro}{\ZREF@zsavepos}
%    \begin{macrocode}
\def\ZREF@zsavepos#1#2#3{%
  \@bsphack
  \if@filesw
    \zref@savepos
    #1{#3}{#2}%
    \ltx@IfUndefined{TeXXeTstate}{%
    }{%
      \ifnum\TeXXeTstate=\ltx@zero
      \else
        \zref@savepos
      \fi
    }%
  \fi
  \@esphack
}
%    \end{macrocode}
%    \end{macro}
%
%    \begin{macro}{\zsavepos}
%    The current location is stored in a reference with the
%    given name.
%    \begin{macrocode}
\ZREF@IfDefinable\zsavepos\def{%
  {%
    \ZREF@zsavepos\zref@labelbylist{savepos}%
  }%
}
%    \end{macrocode}
%    \end{macro}
%
%    \begin{macro}{\zsaveposx}
%    \begin{macrocode}
\ZREF@IfDefinable\zsaveposx\def{%
  {%
    \ZREF@zsavepos\zref@labelbyprops{posx}%
  }%
}
%    \end{macrocode}
%    \end{macro}
%    \begin{macro}{\zsaveposy}
%    \begin{macrocode}
\ZREF@IfDefinable\zsaveposy\def{%
  {%
    \ZREF@zsavepos\zref@labelbyprops{posy}%
  }%
}
%    \end{macrocode}
%    \end{macro}
%
%    \begin{macro}{\zposx}
%    \begin{macro}{\zposy}
%    The horizontal and vertical position are available by
%    \cs{zposx} and \cs{zposy}. Do not rely on absolute
%    positions. They differ in DVI and PDF mode of \pdfTeX.
%    Use differences instead. The unit of the position numbers
%    is sp.
%    \begin{macrocode}
\newcommand*{\zposx}[1]{%
  \zref@extract{#1}{posx}%
}%
\newcommand*{\zposy}[1]{%
  \zref@extract{#1}{posy}%
}%
%    \end{macrocode}
%    \end{macro}
%    \end{macro}
%
%    Typically horizontal and vertical positions are used
%    inside calculations. Therefore the extracting macros
%    should be expandable and babel's patch is not applyable.
%
%    Also it is in the responsibility of the user to marked
%    used positions by \cs{zrefused}
%    in order to notify \hologo{LaTeX} about undefined references.
%
%    \begin{macro}{\ZREF@savepos@ok}
%    \begin{macrocode}
\let\ZREF@savepos@ok=Y
%    \end{macrocode}
%    \end{macro}
%
%    \begin{macrocode}
%</savepos>
%    \end{macrocode}
%
%
% \subsection{Module \xmodule{abspos}}
%
% \subsubsection{Identification}
%
%    \begin{macrocode}
%<*abspos>
\NeedsTeXFormat{LaTeX2e}
\ProvidesPackage{zref-abspos}%
  [2020-07-03 v2.32 Module abspos for zref (HO)]%
\RequirePackage{zref-base}[2019/11/29]
\ifx\ZREF@base@ok Y%
\else
  \expandafter\endinput
\fi
%    \end{macrocode}
%
%    \begin{macrocode}
\RequirePackage{zref-savepos}[2019/11/29]
\ifx\ZREF@savepos@ok Y%
\else
  \expandafter\endinput
\fi
%    \end{macrocode}
%    \begin{macrocode}
\RequirePackage{zref-pagelayout}[2019/11/29]
\zref@addprop{savepos}{abspage}
\zref@addprop{savepos}{pagevalue}
%    \end{macrocode}
%
%    \begin{macro}{\zref@absposx}
%    \begin{macrocode}
\newcommand*{\zref@absposx}[3]{%
  \number
    \expandafter\zref@absposnumx\expandafter{%
      \number\zref@extractdefault{#1}{abspage}{0}%
    }{#2}{#3}%
  \ltx@space
}
%    \end{macrocode}
%    \end{macro}
%
%    \begin{macro}{\zref@absposy}
%    \begin{macrocode}
\newcommand*{\zref@absposy}[3]{%
  \number
  \expandafter\zref@absposnumy\expandafter{%
    \number\zref@extractdefault{#1}{abspage}{0}%
  }{#2}{#3}%
  \ltx@space
}
%    \end{macrocode}
%    \end{macro}
%
%    \begin{macro}{\zref@absposnumx}
%    \begin{macrocode}
\newcommand*{\zref@absposnumx}[3]{%
  \number
%  \ifnum#1>\ltx@zero
%    \zref@ifrefundefined{thepage#1}{%
%      0%
%    }{%
      \numexpr\ZREF@absposnum{thepage#1}{#2}{x}{#3}\relax
%    }%
%  \else
%    0%
%  \fi
}
%    \end{macrocode}
%    \end{macro}
%
%    \begin{macro}{\zref@absposnumy}
%    \begin{macrocode}
\newcommand*{\zref@absposnumy}[3]{%
  \number
%  \ifnum#1>\ltx@zero
%    \zref@ifrefundefined{thepage#1}{%
%      0%
%    }{%
      \numexpr\ZREF@absposnum{thepage#1}{#2}{y}{#3}\relax
%    }%
%  \else
%    0%
%  \fi
}
%    \end{macrocode}
%    \end{macro}
%
%    \begin{macro}{\ZREF@absposnum}
%    \begin{macrocode}
\def\ZREF@absposnum#1#2#3#4{%
  \ltx@ifundefined{ZREF@abspos@#2@#3@#4}{%
    0%
  }{%
    \csname ZREF@abspos@#2@#3@#4\endcsname{#1}%
  }%
}
%    \end{macrocode}
%    \end{macro}
%
%    \begin{macro}{\zref@def@absposx}
%    \begin{macrocode}
\ZREF@Robust\def\zref@def@absposx#1{%
  \zref@wrapper@babel{\ZREF@def@abspos{#1}\zref@absposx}%
}
%    \end{macrocode}
%    \end{macro}
%
%    \begin{macro}{\zref@def@absposy}
%    \begin{macrocode}
\ZREF@Robust\def\zref@def@absposy#1{%
  \zref@wrapper@babel{\ZREF@def@abspos{#1}\zref@absposy}%
}
%    \end{macrocode}
%    \end{macro}
%
%    \begin{macro}{\zref@def@absposnumx}
%    \begin{macrocode}
\ZREF@Robust\def\zref@def@absposnumx#1{%
  \zref@wrapper@babel{\ZREF@def@abspos{#1}\zref@absposnumx}%
}
%    \end{macrocode}
%    \end{macro}
%
%    \begin{macro}{\zref@def@absposnumy}
%    \begin{macrocode}
\ZREF@Robust\def\zref@def@absposnumy#1{%
  \zref@wrapper@babel{\ZREF@def@abspos{#1}\zref@absposnumy}%
}
%    \end{macrocode}
%    \end{macro}
%    \begin{macro}{\ZREF@def@abspos}
%    \begin{macrocode}
\def\ZREF@def@absposnumy#1#2#3#4#5{%
  \edef#1{#2{#3}{#4}{#5}}%
}
%    \end{macrocode}
%    \end{macro}
%
%    \begin{macro}{\zref@absposused}
%    \begin{macrocode}
\ZREF@Robust\def\zref@absposused{%
  \zref@wrapper@babel\ZREF@abspos@used
}
%    \end{macrocode}
%    \end{macro}
%    \begin{macro}{\ZREF@abspos@used}
%    \begin{macrocode}
\def\ZREF@abspos@used#1{%
  \zref@refused{#1}%
  \zref@ifrefundefined{#1}{%
  }{%
    \begingroup
      \edef\ZREF@temp{%
        \zref@extractdefault{#1}{abspage}{0}%
      }%
      \ifnum\ZREF@temp>\ltx@zero
        \zref@refused{thepage\ZREF@temp}%
      \else
        \@PackageError{zref-abspos}{%
          \string\zref@pos@label@used\ltx@space
          needs property `abspage'\MessageBreak
          in label `#1'%
        }\@ehc
      \fi
    \endgroup
  }%
}
%    \end{macrocode}
%    \end{macro}
%
%    \begin{macro}{\zref@absposnumused}
%    \begin{macrocode}
\newcommand*{\zref@absposnumused}[1]{%
  \ifnum#1>\ltx@zero
    \zref@refused{thepage\number#1}%
  \else
    \@PackageError{zref-abspos}{%
      Invalid absolute page number (#1)\MessageBreak
      for \string\zref@pos@num@used.\MessageBreak
      A positive integer is expected%
    }\@ehc
  \fi
}
%    \end{macrocode}
%    \end{macro}
%
%    \begin{macro}{\zref@ifabsposundefined}
%    \begin{macrocode}
\def\zref@ifabsposundefined#1{%
  \zref@ifrefundefined{#1}\ltx@firsttwo{%
    \expandafter\zref@ifabsposnumundefined\expandafter{%
      \number\zref@extractdefault{#1}{abspage}{0}%
    }%
  }%
}
%    \end{macrocode}
%    \end{macro}
%
%    \begin{macro}{\zref@ifabsposnumundefined}
%    \begin{macrocode}
\def\zref@ifabsposnumundefined#1{%
  \ifnum\ZREF@number{#1}>\ltx@zero
    \zref@ifrefundefined{thepage#1}%
        \ltx@firstoftwo\ltx@secondoftwo
  \else
    \expandafter\ltx@firstoftwo
  \fi
}
%    \end{macrocode}
%    \end{macro}
%
% \subsubsection{Media}
%
%    \begin{macro}{\ZREF@abspos@media@width}
%    \begin{macrocode}
\edef\ZREF@abspos@media@width{%
  \ltx@ifundefined{pdfpagewidth}{%
    \ltx@ifundefined{mediawidth}{%
      \ltx@ifundefined{stockwidth}{%
        paperwidth%
      }{%
        stockwidth%
      }%
    }{%
      mediawidth%
    }%
  }{%
    pdfpagewidth%
  }%
}
\ifluatex
\def\ZREF@abspos@media@width{pdfpagewidth}%
\fi
%    \end{macrocode}
%    \end{macro}
%    \begin{macro}{\ZREF@abspos@media@height}
%    \begin{macrocode}
\edef\ZREF@abspos@media@height{%
  \ltx@ifundefined{pdfpageheight}{%
    \ltx@ifundefined{mediaheight}{%
      \ltx@ifundefined{stockheight}{%
        paperheight%
      }{%
        stockheight%
      }%
    }{%
      mediaheight%
    }%
  }{%
    \noexpand\ifcase\pdfpageheight
      \ltx@ifundefined{stockheight}{%
        paperheight%
      }{%
        stockheight%
      }%
    \noexpand\else
      pdfpageheight%
    \noexpand\fi
  }%
}
\ifluatex
\edef\ZREF@abspos@media@height{%
   \noexpand\ifcase\pageheight
      \ltx@ifundefined{stockheight}{%
        paperheight%
      }{%
        stockheight%
      }%
    \noexpand\else
      pdfpageheight%
    \noexpand\fi}%
\fi
%    \end{macrocode}
%    \end{macro}
%
%    \begin{macro}{\ZREF@abspos@media@x@left}
%    \begin{macrocode}
\def\ZREF@abspos@media@x@left#1{%
  0%
}
%    \end{macrocode}
%    \end{macro}
%    \begin{macro}{\ZREF@abspos@media@x@right}
%    \begin{macrocode}
\def\ZREF@abspos@media@x@right#1{%
  \zref@extract{#1}\ZREF@abspos@media@width
}
%    \end{macrocode}
%    \end{macro}
%    \begin{macro}{\ZREF@abspos@media@x@center}
%    \begin{macrocode}
\def\ZREF@abspos@media@x@center#1{%
  \ZREF@abspos@media@x@left{#1}%
  +\zref@extract{#1}\ZREF@abspos@media@width/2%
}
%    \end{macrocode}
%    \end{macro}
%
%    \begin{macro}{\ZREF@abspos@media@y@top}
%    \begin{macrocode}
\def\ZREF@abspos@media@y@top#1{%
  \zref@extract{#1}\ZREF@abspos@media@height
}
%    \end{macrocode}
%    \end{macro}
%    \begin{macro}{\ZREF@abspos@media@y@bottom}
%    \begin{macrocode}
\def\ZREF@abspos@media@y@bottom#1{%
  0%
}
%    \end{macrocode}
%    \end{macro}
%    \begin{macro}{\ZREF@abspos@media@y@center}
%    \begin{macrocode}
\def\ZREF@abspos@media@y@center#1{%
  \zref@extract{#1}\ZREF@abspos@media@height/2%
}
%    \end{macrocode}
%    \end{macro}
%
% \subsubsection{Paper}
%
%    \begin{macro}{\ZREF@abspos@paper@x@left}
%    \begin{macrocode}
\def\ZREF@abspos@paper@x@left#1{%
  0%
}
%    \end{macrocode}
%    \end{macro}
%    \begin{macro}{\ZREF@abspos@paper@x@right}
%    \begin{macrocode}
\def\ZREF@abspos@paper@x@right#1{%
  \zref@extract{#1}{paperwidth}%
}
%    \end{macrocode}
%    \end{macro}
%    \begin{macro}{\ZREF@abspos@paper@x@center}
%    \begin{macrocode}
\def\ZREF@abspos@paper@x@center#1{%
  \zref@extract{#1}{paperwidth}/2%
}
%    \end{macrocode}
%    \end{macro}
%
%    \begin{macro}{\ZREF@abspos@paper@y@top}
%    \begin{macrocode}
\let\ZREF@abspos@paper@y@top\ZREF@abspos@media@y@top
%    \end{macrocode}
%    \end{macro}
%    \begin{macro}{\ZREF@abspos@paper@y@bottom}
%    \begin{macrocode}
\def\ZREF@abspos@paper@y@bottom#1{%
  \ZREF@abspos@paper@y@top{#1}%
  -\zref@extract{#1}{paperheight}%
}
%    \end{macrocode}
%    \end{macro}
%    \begin{macro}{\ZREF@abspos@paper@y@center}
%    \begin{macrocode}
\def\ZREF@abspos@paper@y@center#1{%
  \ZREF@abspos@paper@y@top{#1}%
  -\zref@extract{#1}{paperheight}/2%
}
%    \end{macrocode}
%    \end{macro}
%
% \subsubsection{Origin}
%
% There doesn't seem a good reason to make these tests depend on pdf mode
%  in current engines, so comment out the \verb|\ifpdf| tests.
%    \begin{macro}{\ZREF@abspos@origin@x}
%    \begin{macrocode}
\let\ZREF@temp\ltx@two
\ltx@IfUndefined{pdfhorigin}{}{%
%  \ifpdf
    \let\ZREF@temp\ltx@zero
%  \fi
}
\ifluatex
% \ifpdf
  \let\ZREF@temp\ltx@zero
% \fi
\fi

\ifx\ZREF@temp\ltx@two
  \ifnum\mag=1000 %
    \let\ZREF@temp\ltx@one
  \fi
\fi
\ifcase\ZREF@temp
  \def\ZREF@abspos@origin@x#1{%
    \zref@extract{#1}{pdfhorigin}%
  }%
\or
  \def\ZREF@abspos@origin@x#1{%
    4736286%
  }%
\or
  \def\ZREF@abspos@origin@x#1{%
    \numexpr\mag/1000*\dimexpr 1truein\relax\relax
  }%
\fi
%    \end{macrocode}
%    \end{macro}
%    \begin{macro}{\ZREF@abspos@origin@y}
%    \begin{macrocode}
\let\ZREF@temp\ltx@two
\ltx@IfUndefined{pdfvorigin}{}{%
%  \ifpdf
    \let\ZREF@temp\ltx@zero
%  \fi
}
\ifluatex
%  \ifpdf
    \let\ZREF@temp\ltx@zero
%  \fi
\fi
\ifx\ZREF@temp\ltx@two
  \ifnum\mag=1000 %
    \let\ZREF@temp\ltx@one
  \fi
\fi
\ifcase\ZREF@temp
  \def\ZREF@abspos@origin@y#1{%
    \zref@extract{#1}{pdfvorigin}%
  }%
\or
  \def\ZREF@abspos@origin@y#1{%
    4736286%
  }%
\or
  \def\ZREF@abspos@origin@y#1{%
    \numexpr\mag/1000*\dimexpr 1truein\relax\relax
   }%
\fi
%    \end{macrocode}
%    \end{macro}
%
% \subsubsection{Header}
%
%    \begin{macro}{\ZREF@abspos@head@x@left}
%    \begin{macrocode}
\def\ZREF@abspos@head@x@left#1{%
  \ZREF@abspos@paper@x@left{#1}%
  +\ZREF@abspos@origin@x{#1}%
  +\zref@extract{#1}{hoffset}%
  +\ifodd\zref@extractdefault{#1}{pagevalue}{\number\c@page} %
     \zref@extract{#1}{oddsidemargin}%
   \else
     \zref@extract{#1}{evensidemargin}%
   \fi
}
%    \end{macrocode}
%    \end{macro}
%    \begin{macro}{\ZREF@abspos@head@x@right}
%    \begin{macrocode}
\def\ZREF@abspos@head@x@right#1{%
  \ZREF@abspos@head@x@left{#1}%
  +\zref@extract{#1}{textwidth}%
}
%    \end{macrocode}
%    \end{macro}
%    \begin{macro}{\ZREF@abspos@head@x@center}
%    \begin{macrocode}
\def\ZREF@abspos@head@x@center#1{%
  \ZREF@abspos@head@x@left{#1}%
  +\zref@extract{#1}{textwidth}/2%
}
%    \end{macrocode}
%    \end{macro}
%
%    \begin{macro}{\ZREF@abspos@head@y@top}
%    \begin{macrocode}
\def\ZREF@abspos@head@y@top#1{%
  \ZREF@abspos@paper@y@top{#1}%
  -\ZREF@abspos@origin@y{#1}%
  -\zref@extract{#1}{voffset}%
  -\zref@extract{#1}{topmargin}%
}
%    \end{macrocode}
%    \end{macro}
%    \begin{macro}{\ZREF@abspos@head@y@bottom}
%    \begin{macrocode}
\def\ZREF@abspos@head@y@bottom#1{%
  \ZREF@abspos@head@y@top{#1}%
  -\zref@extract{#1}{headheight}%
}
%    \end{macrocode}
%    \end{macro}
%    \begin{macro}{\ZREF@abspos@head@y@center}
%    \begin{macrocode}
\def\ZREF@abspos@head@y@center#1{%
  \ZREF@abspos@head@y@top{#1}%
  -\zref@extract{#1}{headheight}/2%
}
%    \end{macrocode}
%    \end{macro}
%
% \subsubsection{Body}
%
%    \begin{macro}{\ZREF@abspos@body@x@left}
%    \begin{macrocode}
\let\ZREF@abspos@body@x@left\ZREF@abspos@head@x@left
%    \end{macrocode}
%    \end{macro}
%    \begin{macro}{\ZREF@abspos@body@x@right}
%    \begin{macrocode}
\let\ZREF@abspos@body@x@right\ZREF@abspos@head@x@right
%    \end{macrocode}
%    \end{macro}
%    \begin{macro}{\ZREF@abspos@body@x@center}
%    \begin{macrocode}
\let\ZREF@abspos@body@x@center\ZREF@abspos@head@x@center
%    \end{macrocode}
%    \end{macro}
%
%    \begin{macro}{\ZREF@abspos@body@y@top}
%    \begin{macrocode}
\def\ZREF@abspos@body@y@top#1{%
  \ZREF@abspos@head@y@bottom{#1}%
  -\zref@extract{#1}{headsep}%
}
%    \end{macrocode}
%    \end{macro}
%    \begin{macro}{\ZREF@abspos@body@y@bottom}
%    \begin{macrocode}
\def\ZREF@abspos@body@y@bottom#1{%
  \ZREF@abspos@body@y@top{#1}%
  -\zref@extract{#1}{textheight}%
}
%    \end{macrocode}
%    \end{macro}
%    \begin{macro}{\ZREF@abspos@body@y@center}
%    \begin{macrocode}
\def\ZREF@abspos@body@y@center#1{%
  \ZREF@abspos@body@y@top{#1}%
  -\zref@extract{#1}{textheight}/2%
}
%    \end{macrocode}
%    \end{macro}
%
% \subsubsection{Footer}
%
%    \begin{macro}{\ZREF@abspos@foot@x@left}
%    \begin{macrocode}
\let\ZREF@abspos@foot@x@left\ZREF@abspos@head@x@left
%    \end{macrocode}
%    \end{macro}
%    \begin{macro}{\ZREF@abspos@foot@x@right}
%    \begin{macrocode}
\let\ZREF@abspos@foot@x@right\ZREF@abspos@head@x@right
%    \end{macrocode}
%    \end{macro}
%    \begin{macro}{\ZREF@abspos@foot@x@center}
%    \begin{macrocode}
\let\ZREF@abspos@foot@x@center\ZREF@abspos@head@x@center
%    \end{macrocode}
%    \end{macro}
%
%    \begin{macro}{\ZREF@abspos@foot@y@bottom}
%    \begin{macrocode}
\def\ZREF@abspos@foot@y@bottom#1{%
  \ZREF@abspos@body@y@bottom{#1}%
  -\zref@extract{#1}{footskip}%
}
%    \end{macrocode}
%    \end{macro}
%
% \subsubsection{Marginal notes}
%
%    \begin{macro}{\ZREF@abspos@marginpar@x@left}
%    \begin{macrocode}
\def\ZREF@abspos@marginpar@x@left#1{%
  \ifodd\zref@extractdefault{#1}{pagevalue}{\number\c@page} %
    \ZREF@abspos@body@x@right{#1}%
    +\zref@extract{#1}{marginparsep}%
  \else
    \ZREF@abspos@body@x@left{#1}%
    -\zref@extract{#1}{marginparsep}%
    -\zref@extract{#1}{marginparwidth}%
  \fi
}
%    \end{macrocode}
%    \end{macro}
%    \begin{macro}{\ZREF@abspos@marginpar@x@right}
%    \begin{macrocode}
\def\ZREF@abspos@marginpar@x@right#1{%
  \ZREF@abspos@marginpar@x@left{#1}%
  +\zref@extract{#1}{marginparwidth}%
}
%    \end{macrocode}
%    \end{macro}
%    \begin{macro}{\ZREF@abspos@marginpar@x@center}
%    \begin{macrocode}
\def\ZREF@abspos@marginpar@x@center#1{%
  \ZREF@abspos@marginpar@x@left{#1}%
  +\zref@extract{#1}{marginparwidth}/2%
}
%    \end{macrocode}
%    \end{macro}
%
%    \begin{macro}{\ZREF@abspos@marginpar@y@top}
%    \begin{macrocode}
\let\ZREF@abspos@marginpar@y@top\ZREF@abspos@body@y@top
%    \end{macrocode}
%    \end{macro}
%    \begin{macro}{\ZREF@abspos@marginpar@y@bottom}
%    \begin{macrocode}
\let\ZREF@abspos@marginpar@y@bottom\ZREF@abspos@body@y@bottom
%    \end{macrocode}
%    \end{macro}
%    \begin{macro}{\ZREF@abspos@marginpar@y@center}
%    \begin{macrocode}
\let\ZREF@abspos@marginpar@y@center\ZREF@abspos@body@y@center
%    \end{macrocode}
%    \end{macro}
%
% \subsubsection{Stock paper}
%
%    \begin{macro}{\ZREF@abspos@stock@x@left}
%    \begin{macrocode}
\let\ZREF@abspos@stock@x@left\ZREF@abspos@paper@x@left
%    \end{macrocode}
%    \end{macro}
%    \begin{macro}{\ZREF@abspos@stock@x@right}
%    \begin{macrocode}
\let\ZREF@abspos@stock@x@right\ZREF@abspos@paper@x@right
%    \end{macrocode}
%    \end{macro}
%    \begin{macro}{\ZREF@abspos@stock@x@center}
%    \begin{macrocode}
\let\ZREF@abspos@stock@x@center\ZREF@abspos@paper@x@center
%    \end{macrocode}
%    \end{macro}
%
%    \begin{macro}{\ZREF@abspos@stock@y@top}
%    \begin{macrocode}
\let\ZREF@abspos@stock@y@top\ZREF@abspos@paper@y@top
%    \end{macrocode}
%    \end{macro}
%    \begin{macro}{\ZREF@abspos@stock@y@bottom}
%    \begin{macrocode}
\let\ZREF@abspos@stock@y@bottom\ZREF@abspos@paper@y@bottom
%    \end{macrocode}
%    \end{macro}
%    \begin{macro}{\ZREF@abspos@stock@y@center}
%    \begin{macrocode}
\let\ZREF@abspos@stock@y@center\ZREF@abspos@paper@y@center
%    \end{macrocode}
%    \end{macro}
%
%    \begin{macrocode}
%</abspos>
%    \end{macrocode}
%
%
% \subsection{Module \xmodule{dotfill}}
%
%    \begin{macrocode}
%<*dotfill>
\NeedsTeXFormat{LaTeX2e}
\ProvidesPackage{zref-dotfill}%
  [2020-07-03 v2.32 Module dotfill for zref (HO)]%
\RequirePackage{zref-base}[2019/11/29]
\ifx\ZREF@base@ok Y%
\else
  \expandafter\endinput
\fi
%    \end{macrocode}
%
%    For measuring the width of \cs{zdotfill} we use the features
%    provided by module \xmodule{savepos}.
%    \begin{macrocode}
\RequirePackage{zref-savepos}[2019/11/29]
%    \end{macrocode}
%    For automatically generated label names we use the unique
%    counter of module \xmodule{base}.
%    \begin{macrocode}
\zref@require@unique
%    \end{macrocode}
%    Configuration is done by the key value interface of package
%    \xpackage{keyval}.
%    \begin{macrocode}
\RequirePackage{keyval}
%    \end{macrocode}
%    The definitions of the keys follow.
%    \begin{macrocode}
\define@key{ZREF@DF}{unit}{%
  \def\ZREF@df@unit{#1}%
}
\define@key{ZREF@DF}{min}{%
  \def\ZREF@df@min{#1}%
}
\define@key{ZREF@DF}{dot}{%
  \def\ZREF@df@dot{#1}%
}
%    \end{macrocode}
%    Defaults are set, see user interface.
%    \begin{macrocode}
\providecommand\ZREF@df@min{2}
\providecommand\ZREF@df@unit{.44em}
\providecommand\ZREF@df@dot{.}
%    \end{macrocode}
%    \begin{macro}{\zdotfillsetup}
%    Configuration of \cs{zdotfill} is done by \cs{zdotfillsetup}.
%    \begin{macrocode}
\newcommand*{\zdotfillsetup}{\kvsetkeys{ZREF@DF}}
%    \end{macrocode}
%    \end{macro}
%
%    \begin{macro}{\zdotfill}
%    \cs{zdotfill} sets labels at the left and the right to get
%    the horizontal position. \cs{zsavepos} is not used, because
%    we do not need the vertical position.
%    \begin{macrocode}
\ZREF@IfDefinable\zdotfill\def{%
  {%
    \leavevmode
    \global\advance\c@zref@unique\ltx@one
    \begingroup
      \def\ZREF@temp{zref@\number\c@zref@unique}%
      \pdfsavepos
      \zref@labelbyprops{\thezref@unique L}{posx}%
      \setlength{\dimen@}{\ZREF@df@unit}%
      \zref@ifrefundefined{\thezref@unique R}{%
        \ZREF@dotfill
      }{%
        \ifnum\numexpr\zposx{\thezref@unique R}%
                      -\zposx{\thezref@unique L}\relax
            <\dimexpr\ZREF@df@min\dimen@\relax
          \hfill
        \else
          \ZREF@dotfill
        \fi
      }%
      \pdfsavepos
      \zref@labelbyprops{\thezref@unique R}{posx}%
    \endgroup
    \kern\z@
  }%
}
%    \end{macrocode}
%    \end{macro}
%    \begin{macro}{\ZREF@dotfill}
%    Help macro that actually sets the dots.
%    \begin{macrocode}
\def\ZREF@dotfill{%
  \cleaders\hb@xt@\dimen@{\hss\ZREF@df@dot\hss}\hfill
}
%    \end{macrocode}
%    \end{macro}
%
%    \begin{macrocode}
%</dotfill>
%    \end{macrocode}
%
% \subsection{Module \xmodule{env}}
%
%    \begin{macrocode}
%<*env>
\NeedsTeXFormat{LaTeX2e}
\ProvidesPackage{zref-env}%
  [2020-07-03 v2.32 Module env for zref (HO)]%
\RequirePackage{zref-base}[2019/11/29]
\ifx\ZREF@base@ok Y%
\else
  \expandafter\endinput
\fi
%    \end{macrocode}
%
%    \begin{macrocode}
\zref@newprop{envname}[]{\@currenvir}
\zref@newprop{envline}[]{\zref@env@line}
%    \end{macrocode}
%
%    \begin{macro}{\zref@env@line}
%    Macro \cs{zref@env@line} extracts the line number
%    from \cs{@currenvline}.
%    \begin{macrocode}
\def\zref@env@line{%
  \ifx\@currenvline\ltx@empty
  \else
    \expandafter
    \ZREF@ENV@line\@currenvline\ltx@empty line \ltx@empty\@nil
  \fi
}
%    \end{macrocode}
%    \end{macro}
%    \begin{macro}{\ZREF@ENV@line}
%    \begin{macrocode}
\def\ZREF@ENV@line#1line #2\ltx@empty#3\@nil{#2}%
%    \end{macrocode}
%    \end{macro}
%
%    \begin{macrocode}
%</env>
%    \end{macrocode}
%
%% \section{Installation}
%
% \subsection{Download}
%
% \paragraph{Package.} This package is available on
% CTAN\footnote{\CTANpkg{zref}}:
% \begin{description}
% \item[\CTAN{macros/latex/contrib/zref/zref.dtx}] The source file.
% \item[\CTAN{macros/latex/contrib/zref/zref.pdf}] Documentation.
% \end{description}
%
%
% \paragraph{Bundle.} All the packages of the bundle `zref'
% are also available in a TDS compliant ZIP archive. There
% the packages are already unpacked and the documentation files
% are generated. The files and directories obey the TDS standard.
% \begin{description}
% \item[\CTANinstall{install/macros/latex/contrib/zref.tds.zip}]
% \end{description}
% \emph{TDS} refers to the standard ``A Directory Structure
% for \TeX\ Files'' (\CTANpkg{tds}). Directories
% with \xfile{texmf} in their name are usually organized this way.
%
% \subsection{Bundle installation}
%
% \paragraph{Unpacking.} Unpack the \xfile{zref.tds.zip} in the
% TDS tree (also known as \xfile{texmf} tree) of your choice.
% Example (linux):
% \begin{quote}
%   |unzip zref.tds.zip -d ~/texmf|
% \end{quote}
%
%
% \subsection{Package installation}
%
% \paragraph{Unpacking.} The \xfile{.dtx} file is a self-extracting
% \docstrip\ archive. The files are extracted by running the
% \xfile{.dtx} through \plainTeX:
% \begin{quote}
%   \verb|tex zref.dtx|
% \end{quote}
%
% \paragraph{TDS.} Now the different files must be moved into
% the different directories in your installation TDS tree
% (also known as \xfile{texmf} tree):
% \begin{quote}
% \def\t{^^A
% \begin{tabular}{@{}>{\ttfamily}l@{ $\rightarrow$ }>{\ttfamily}l@{}}
%   zref.sty & tex/latex/zref/zref.sty\\
%   zref-base.sty & tex/latex/zref/zref-base.sty\\
%   zref-abspage.sty & tex/latex/zref/zref-abspage.sty\\
%   zref-abspos.sty & tex/latex/zref/zref-abspos.sty\\
%   zref-counter.sty & tex/latex/zref/zref-counter.sty\\
%   zref-dotfill.sty & tex/latex/zref/zref-dotfill.sty\\
%   zref-env.sty & tex/latex/zref/zref-env.sty\\
%   zref-hyperref.sty & tex/latex/zref/zref-hyperref.sty\\
%   zref-lastpage.sty & tex/latex/zref/zref-lastpage.sty\\
%   zref-marks.sty & tex/latex/zref/zref-marks.sty\\
%   zref-nextpage.sty & tex/latex/zref/zref-nextpage.sty\\
%   zref-pageattr.sty & tex/latex/zref/zref-pageattr.sty\\
%   zref-pagelayout.sty & tex/latex/zref/zref-pagelayout.sty\\
%   zref-perpage.sty & tex/latex/zref/zref-perpage.sty\\
%   zref-runs.sty & tex/latex/zref/zref-runs.sty\\
%   zref-savepos.sty & tex/latex/zref/zref-savepos.sty\\
%   zref-thepage.sty & tex/latex/zref/zref-thepage.sty\\
%   zref-titleref.sty & tex/latex/zref/zref-titleref.sty\\
%   zref-totpages.sty & tex/latex/zref/zref-totpages.sty\\
%   zref-user.sty & tex/latex/zref/zref-user.sty\\
%   zref-xr.sty & tex/latex/zref/zref-xr.sty\\
%   zref.pdf & doc/latex/zref/zref.pdf\\
%   zref-example.tex & doc/latex/zref/zref-example.tex\\
%   zref-example-lastpage.tex & doc/latex/zref/zref-example-lastpage.tex\\
%   zref-example-nextpage.tex & doc/latex/zref/zref-example-nextpage.tex\\
%   zref.dtx & source/latex/zref/zref.dtx\\
% \end{tabular}^^A
% }^^A
% \sbox0{\t}^^A
% \ifdim\wd0>\linewidth
%   \begingroup
%     \advance\linewidth by\leftmargin
%     \advance\linewidth by\rightmargin
%   \edef\x{\endgroup
%     \def\noexpand\lw{\the\linewidth}^^A
%   }\x
%   \def\lwbox{^^A
%     \leavevmode
%     \hbox to \linewidth{^^A
%       \kern-\leftmargin\relax
%       \hss
%       \usebox0
%       \hss
%       \kern-\rightmargin\relax
%     }^^A
%   }^^A
%   \ifdim\wd0>\lw
%     \sbox0{\small\t}^^A
%     \ifdim\wd0>\linewidth
%       \ifdim\wd0>\lw
%         \sbox0{\footnotesize\t}^^A
%         \ifdim\wd0>\linewidth
%           \ifdim\wd0>\lw
%             \sbox0{\scriptsize\t}^^A
%             \ifdim\wd0>\linewidth
%               \ifdim\wd0>\lw
%                 \sbox0{\tiny\t}^^A
%                 \ifdim\wd0>\linewidth
%                   \lwbox
%                 \else
%                   \usebox0
%                 \fi
%               \else
%                 \lwbox
%               \fi
%             \else
%               \usebox0
%             \fi
%           \else
%             \lwbox
%           \fi
%         \else
%           \usebox0
%         \fi
%       \else
%         \lwbox
%       \fi
%     \else
%       \usebox0
%     \fi
%   \else
%     \lwbox
%   \fi
% \else
%   \usebox0
% \fi
% \end{quote}
% If you have a \xfile{docstrip.cfg} that configures and enables \docstrip's
% TDS installing feature, then some files can already be in the right
% place, see the documentation of \docstrip.
%
% \subsection{Refresh file name databases}
%
% If your \TeX~distribution
% (\TeX\,Live, \mikTeX, \dots) relies on file name databases, you must refresh
% these. For example, \TeX\,Live\ users run \verb|texhash| or
% \verb|mktexlsr|.
%
% \subsection{Some details for the interested}
%
% \paragraph{Unpacking with \LaTeX.}
% The \xfile{.dtx} chooses its action depending on the format:
% \begin{description}
% \item[\plainTeX:] Run \docstrip\ and extract the files.
% \item[\LaTeX:] Generate the documentation.
% \end{description}
% If you insist on using \LaTeX\ for \docstrip\ (really,
% \docstrip\ does not need \LaTeX), then inform the autodetect routine
% about your intention:
% \begin{quote}
%   \verb|latex \let\install=y% \iffalse meta-comment
%
% File: zref.dtx
% Version: 2020-07-03 v2.32
% Info: A new reference scheme for LaTeX
%
% Copyright (C)
%    2006-2012  Heiko Oberdiek
%    2016-2020  Oberdiek Package Support Group
%    https://github.com/ho-tex/zref/issues
%
% This work may be distributed and/or modified under the
% conditions of the LaTeX Project Public License, either
% version 1.3c of this license or (at your option) any later
% version. This version of this license is in
%    https://www.latex-project.org/lppl/lppl-1-3c.txt
% and the latest version of this license is in
%    https://www.latex-project.org/lppl.txt
% and version 1.3 or later is part of all distributions of
% LaTeX version 2005/12/01 or later.
%
% This work has the LPPL maintenance status "maintained".
%
% The Current Maintainers of this work are
% Heiko Oberdiek and the Oberdiek Package Support Group
% https://github.com/ho-tex/zref/issues
%
% This work consists of the main source file zref.dtx
% and the derived files
%    zref.sty, zref.pdf, zref.ins, zref.drv, zref-base.sty,
%    zref-abspage.sty, zref-abspos.sty, zref-counter.sty,
%    zref-dotfill.sty, zref-env.sty, zref-hyperref.sty,
%    zref-lastpage.sty, zref-marks.sty, zref-nextpage.sty,
%    zref-pageattr.sty, zref-pagelayout.sty, zref-perpage.sty,
%    zref-runs.sty, zref-savepos.sty, zref-thepage.sty,
%    zref-titleref.sty, zref-totpages.sty, zref-user.sty,
%    zref-xr.sty, zref-example.tex, zref-example-lastpage.tex,
%    zref-example-nextpage.tex, zref-test1.tex,
%    zref-test-base.tex, zref-test-runs.tex,
%    zref-test-titleref-memoir.tex.
%
% Distribution:
%    CTAN:macros/latex/contrib/zref/zref.dtx
%    CTAN:macros/latex/contrib/zref/zref.pdf
%
% Unpacking:
%    (a) If zref.ins is present:
%           tex zref.ins
%    (b) Without zref.ins:
%           tex zref.dtx
%    (c) If you insist on using LaTeX
%           latex \let\install=y\input{zref.dtx}
%        (quote the arguments according to the demands of your shell)
%
% Documentation:
%    (a) If zref.drv is present:
%           latex zref.drv
%    (b) Without zref.drv:
%           latex zref.dtx; ...
%    The class ltxdoc loads the configuration file ltxdoc.cfg
%    if available. Here you can specify further options, e.g.
%    use A4 as paper format:
%       \PassOptionsToClass{a4paper}{article}
%
%    Programm calls to get the documentation (example):
%       pdflatex zref.dtx
%       makeindex -s gind.ist zref.idx
%       pdflatex zref.dtx
%       makeindex -s gind.ist zref.idx
%       pdflatex zref.dtx
%
% Installation:
%    TDS:tex/latex/zref/zref.sty
%    TDS:tex/latex/zref/zref-base.sty
%    TDS:tex/latex/zref/zref-abspage.sty
%    TDS:tex/latex/zref/zref-abspos.sty
%    TDS:tex/latex/zref/zref-counter.sty
%    TDS:tex/latex/zref/zref-dotfill.sty
%    TDS:tex/latex/zref/zref-env.sty
%    TDS:tex/latex/zref/zref-hyperref.sty
%    TDS:tex/latex/zref/zref-lastpage.sty
%    TDS:tex/latex/zref/zref-marks.sty
%    TDS:tex/latex/zref/zref-nextpage.sty
%    TDS:tex/latex/zref/zref-pageattr.sty
%    TDS:tex/latex/zref/zref-pagelayout.sty
%    TDS:tex/latex/zref/zref-perpage.sty
%    TDS:tex/latex/zref/zref-runs.sty
%    TDS:tex/latex/zref/zref-savepos.sty
%    TDS:tex/latex/zref/zref-thepage.sty
%    TDS:tex/latex/zref/zref-titleref.sty
%    TDS:tex/latex/zref/zref-totpages.sty
%    TDS:tex/latex/zref/zref-user.sty
%    TDS:tex/latex/zref/zref-xr.sty
%    TDS:doc/latex/zref/zref.pdf
%    TDS:doc/latex/zref/zref-example.tex
%    TDS:doc/latex/zref/zref-example-lastpage.tex
%    TDS:doc/latex/zref/zref-example-nextpage.tex
%    TDS:source/latex/zref/zref.dtx
%
%<*ignore>
\begingroup
  \catcode123=1 %
  \catcode125=2 %
  \def\x{LaTeX2e}%
\expandafter\endgroup
\ifcase 0\ifx\install y1\fi\expandafter
         \ifx\csname processbatchFile\endcsname\relax\else1\fi
         \ifx\fmtname\x\else 1\fi\relax
\else\csname fi\endcsname
%</ignore>
%<*install>
\input docstrip.tex
\Msg{************************************************************************}
\Msg{* Installation}
\Msg{* Package: zref 2020-07-03 v2.32 A new reference scheme for LaTeX (HO)}
\Msg{************************************************************************}

\keepsilent
\askforoverwritefalse

\let\MetaPrefix\relax
\preamble

This is a generated file.

Project: zref
Version: 2020-07-03 v2.32

Copyright (C)
  2006-2012 Heiko Oberdiek
  2016-2020 Oberdiek Package Support Group

This work may be distributed and/or modified under the
conditions of the LaTeX Project Public License, either
version 1.3c of this license or (at your option) any later
version. This version of this license is in
   https://www.latex-project.org/lppl/lppl-1-3c.txt
and the latest version of this license is in
   https://www.latex-project.org/lppl.txt
and version 1.3 or later is part of all distributions of
LaTeX version 2005/12/01 or later.

This work has the LPPL maintenance status "maintained".

The Current Maintainers of this work are
Heiko Oberdiek and the Oberdiek Package Support Group
https://github.com/ho-tex/zref/issues


This work consists of the main source file zref.dtx
and the derived files
   zref.sty, zref.pdf, zref.ins, zref.drv, zref-base.sty,
   zref-abspage.sty, zref-abspos.sty, zref-counter.sty,
   zref-dotfill.sty, zref-env.sty, zref-hyperref.sty,
   zref-lastpage.sty, zref-marks.sty, zref-nextpage.sty,
   zref-pageattr.sty, zref-pagelayout.sty, zref-perpage.sty,
   zref-runs.sty, zref-savepos.sty, zref-thepage.sty,
   zref-titleref.sty, zref-totpages.sty, zref-user.sty,
   zref-xr.sty, zref-example.tex, zref-example-lastpage.tex,
   zref-example-nextpage.tex, zref-test1.tex,
   zref-test-base.tex, zref-test-runs.tex,
   zref-test-titleref-memoir.tex.

\endpreamble
\let\MetaPrefix\DoubleperCent

\generate{%
  \file{zref.ins}{\from{zref.dtx}{install}}%
  \file{zref.drv}{\from{zref.dtx}{driver}}%
  \usedir{tex/latex/zref}%
  \file{zref.sty}{\from{zref.dtx}{package}}%
  \file{zref-base.sty}{\from{zref.dtx}{base}}%
  \file{zref-abspage.sty}{\from{zref.dtx}{abspage}}%
  \file{zref-abspos.sty}{\from{zref.dtx}{abspos}}%
  \file{zref-counter.sty}{\from{zref.dtx}{counter}}%
  \file{zref-dotfill.sty}{\from{zref.dtx}{dotfill}}%
  \file{zref-env.sty}{\from{zref.dtx}{env}}%
  \file{zref-hyperref.sty}{\from{zref.dtx}{hyperref}}%
  \file{zref-lastpage.sty}{\from{zref.dtx}{lastpage}}%
  \file{zref-marks.sty}{\from{zref.dtx}{marks}}%
  \file{zref-nextpage.sty}{\from{zref.dtx}{nextpage}}%
  \file{zref-pageattr.sty}{\from{zref.dtx}{pageattr}}%
  \file{zref-pagelayout.sty}{\from{zref.dtx}{pagelayout}}%
  \file{zref-perpage.sty}{\from{zref.dtx}{perpage}}%
  \file{zref-runs.sty}{\from{zref.dtx}{runs}}%
  \file{zref-savepos.sty}{\from{zref.dtx}{savepos}}%
  \file{zref-thepage.sty}{\from{zref.dtx}{thepage}}%
  \file{zref-titleref.sty}{\from{zref.dtx}{titleref}}%
  \file{zref-totpages.sty}{\from{zref.dtx}{totpages}}%
  \file{zref-user.sty}{\from{zref.dtx}{user}}%
  \file{zref-xr.sty}{\from{zref.dtx}{xr}}%
  \usedir{doc/latex/zref}%
  \file{zref-example.tex}{\from{zref.dtx}{example}}%
  \file{zref-example-lastpage.tex}{\from{zref.dtx}{example-lastpage}}%
  \file{zref-example-nextpage.tex}{\from{zref.dtx}{example-nextpage}}%
%  \usedir{doc/latex/zref/test}%
%  \file{zref-test1.tex}{\from{zref.dtx}{test1}}%
%  \file{zref-test-base.tex}{\from{zref.dtx}{test-base}}%
%  \file{zref-test-runs.tex}{\from{zref.dtx}{test-runs}}%
%  \file{zref-test-titleref-memoir.tex}{\from{zref.dtx}{test-titleref-memoir}}%
}

\catcode32=13\relax% active space
\let =\space%
\Msg{************************************************************************}
\Msg{*}
\Msg{* To finish the installation you have to move the following}
\Msg{* files into a directory searched by TeX:}
\Msg{*}
\Msg{*     zref.sty, zref-base.sty, zref-abspage.sty, zref-abspos.sty,}
\Msg{*     zref-counter.sty, zref-dotfill.sty, zref-env.sty,}
\Msg{*     zref-hyperref.sty, zref-lastpage.sty, zref-marks.sty,}
\Msg{*     zref-nextpage.sty, zref-pageattr.sty, zref-pagelayout.sty,}
\Msg{*     zref-perpage.sty, zref-runs.sty, zref-savepos.sty,}
\Msg{*     zref-thepage.sty, zref-titleref.sty, zref-totpages.sty,}
\Msg{*     zref-user.sty, zref-xr.sty}
\Msg{*}
\Msg{* To produce the documentation run the file `zref.drv'}
\Msg{* through LaTeX.}
\Msg{*}
\Msg{* Happy TeXing!}
\Msg{*}
\Msg{************************************************************************}

\endbatchfile
%</install>
%<*ignore>
\fi
%</ignore>
%<*driver>
\NeedsTeXFormat{LaTeX2e}
\ProvidesFile{zref.drv}%
  [2020-07-03 v2.32 A new reference scheme for LaTeX (HO)]%
\documentclass{ltxdoc}
\usepackage{holtxdoc}[2011/11/22]
\setlength{\emergencystretch}{1em}
\begin{document}
  \DocInput{zref.dtx}%
\end{document}
%</driver>
% \fi
%
%
%
% \GetFileInfo{zref.drv}
%
% \title{The \xpackage{zref} package}
% \date{2020-07-03 v2.32}
% \author{Heiko Oberdiek\thanks
% {Please report any issues at \url{https://github.com/ho-tex/zref/issues}}}
%
% \maketitle
%
% \begin{abstract}
% Package \xpackage{zref} tries to get rid of the restriction
% in \hologo{LaTeX}'s reference system that only two properties are
% supported. The package implements an extensible referencing
% system, where properties are handled in a more flexible way.
% It offers an interface for macro programmers for the access
% to the system and some applications that uses the new
% reference scheme.
% \end{abstract}
%
% \tableofcontents
%
% \newcommand*{\metacs}[1]{\texttt{\itshape #1}}
%
% \newcommand{\markexp}{^^A
%   \textsuperscript{exp}^^A
%   \space
% }
% \newcommand{\markexptwo}{^^A
%   \textsuperscript{exp2}^^A
%   \space
% }
% \newcommand{\markrobust}{^^A
%   \textsuperscript{robust}^^A
%   \space
% }
% \newcommand{\marknowarn}{^^A
%   \textsuperscript{nowarn}^^A
%   \space
% }
% \newcommand{\markbabel}{^^A
%   \textsuperscript{babel}^^A
%   \space
% }
%
% \section{Introduction}
%
% Standard \hologo{LaTeX}'s reference system with \cs{label}, \cs{ref},
% and \cs{pageref} supports two properties, the apperance of
% the counter that is last incremented by \cs{refstepcounter} and
% the page with the \cs{label} command.
%
% Unhappily \hologo{LaTeX} does not provide an interface for adding
% another properties. Packages such as \xpackage{hyperref},
% \xpackage{nameref}, or \xpackage{titleref} are forced to
% use ugly hacks to extend the reference system.
% These ugly hacks are one of the causes for \xpackage{hyperref}'s
% difficulty regarding compatibility with other packages.
%
% \subsection{Standard \hologo{LaTeX} behaviour}
%
% References are created by the \cs{label} command:
% \begin{quote}
%   |\chapter{Second chapter}|\\
%   |\section{First section on page 7} % section 2.1|\\
%   |\label{myref}|
% \end{quote}
% Now \hologo{LaTeX} records the section number |2.1| and the page |7|
% in the reference. Internally the reference is a list with
% two entries:
% \begin{quote}
%   |\r@myref| $\rightarrow$ |{2.1}{7}|
% \end{quote}
% The length of the list if fixed in the \hologo{LaTeX} kernel,
% An interface for adding new properties is missing.
%
% There are several tries to add new properties:
% \begin{description}
% \item[\xpackage{hyperref}] uses a list of five
%   properties instead of the standard list with two entries.
%   This causes many compatibility problems with \hologo{LaTeX} and
%   other packages.
% \item[\xpackage{titleref}] stores its title data into the
%   first entry in the list. \hologo{LaTeX} is happy because it
%   does only see its list with two entries. The situation
%   becomes more difficult, if more properties are added
%   this way. Then the macros form a nested structure
%   inside the first reference argument for the label.
%   Expandable extractions will then become painful.
% \end{description}
%
% \subsection{Basic idea}
%
% Some time ago Morten H\o gholm sent me an experimental
% cross referencing mechanism as ``expl3'' code. His
% idea is:
% \begin{quote}
%   |\g_xref_mylabel_plist| $\rightarrow$\\
%   \mbox{}\qquad|\xref_dance_key{salsa}\xref_name_key{Morten}|\dots
% \end{quote}
% The entries have the following format:
% \begin{quote}
%   |\xref_|\meta{your key}|_key{|\meta{some text}|}|
% \end{quote}
% This approach is much more flexible:
% \begin{itemize}
%   \item New properties can easily be added, just use a new key.
%   \item The length of the list is not fixed. A reference
%         can use a subset of the keys.
%   \item The order of the entries does not matter.
% \end{itemize}
% Unhappily I am not familiar with the experimental code
% for \hologo{LaTeX3} that will need some time before its first release.
% Thus I have implemented it as \hologo{LaTeXe} package without disturbing
% the existing \hologo{LaTeX} reference system.
%
% \subsection{Interfaces}
%
% The package provides a generic \emph{interface for programmers}.
% Commands of this interface are prefixed by \cs{zref@}.
%
% Option \xoption{user} enabels the \emph{user interface}.
% Here the commands are prefixed by \cs{z} to avoid name
% clashes with existing macros.
%
% Then the packages provides some \emph{modules}. They
% are applications for the reference system and can also be
% considered as examples how to use the reference system.
%
% The modules can be loaded as packages. The package
% name is prefixed with |zref-|, for example:
%\begin{quote}
%\begin{verbatim}
%\RequirePackage{zref-abspage}
%\end{verbatim}
%\end{quote}
% This is the preferred way if the package is loaded from
% within other packages to avoid option clashes.
%
% As alternative package |zref| can be used and the modules
% are given as options:
%\begin{quote}
%\begin{verbatim}
%\usepackage[perpage,user]{zref}
%\end{verbatim}
%\end{quote}
%
% \section{Interface for programmers}
%
% The user interface is described in the next section \ref{sec:user}.
%
% \subsection{Entities}
%
% \paragraph{Reference.}
% Internally a reference is a list of key value pairs:
% \begin{quote}
%   |\Z@R@myref| $\rightarrow$
%   |\default{2.1}\page{7}|
% \end{quote}
% The generic format of a entry is:
% \begin{quote}
%   |\Z@R@|\meta{refname} $\rightarrow$
%   |\|\meta{propname}|{|\meta{value}|}|
% \end{quote}
% \meta{refname} is the name that denoted references
% (the name used in \cs{label} and \cs{ref}). \meta{propname}
% is the name of the property or key. The property key macro
% is never executed, it is used in parameter text matching only.
%
% \paragraph{Property.}
% Because the name of a property is used in a macro name
% that must survive the \xfile{.aux} file, the name is
% restricted to letters and `@'.
%
% \paragraph{Property list.}
% Often references are used for special purposes. Thus it
% saves memory if just the properties are used in this
% reference that are necessary for its purpose.
%
% Therefore this package uses the concept of \emph{property lists}.
% A property list is a set of properties. The set of properties
% that is used by the default \cs{label} command is the
% \emph{main property list}.
%
% \subsection{Property list}
%
% \markexp means that the implementation of the
% marked macro is expandable. \markexptwo goes a step further
% and marks the macro expandable in exact two expansion steps.
%
% \begin{declcs}{zref@newlist}
%   \M{listname}
% \end{declcs}
% Declares a new empty property list.
%
% \begin{declcs}{zref@addprop}
%   \M{listname} \M{propname}\\
%   \cs{zref@localaddprop} \M{listname} \M{propname}
% \end{declcs}
% Adds the property \meta{propname}
% to the property list \meta{listname}.
% The property and list must exist.
% The addition is global by \cs{zref@addprop} and limited to
% local scope by \cs{zref@localaddprop}.
% Between 2010/04/19 v2.13 and 2010/10/22 v2.19 a comma separated list
% of properties could be used as argument \meta{propname}.
% Since 2010/10/22 v2.19 the addition of several properties
% at once is supported by \cs{zref@addprops}.
%
% \begin{declcs}{zref@addprops}
%   \M{listname} \M{propname list}\\
%   \cs{zref@localaddprops} \M{listname} \M{propname list}
% \end{declcs}
% These macros add a comma separated list of properties
% \meta{propname list} to list \meta{listname}.
% \cs{zref@addprops} works globally and \cs{zref@localaddprops}
% locally. Since 2010/10/22 v2.19.
%
% \begin{declcs}{zref@listexists}
%   \M{listname} \M{then}
% \end{declcs}
% Executes \meta{then} if the property list \meta{listname} exists
% or raise an error otherwise.
%
% \begin{declcs}{zref@iflistundefined}\markexp \M{listname}
%   \M{then} \M{else}
% \end{declcs}
% Executes \meta{then} if the list exists or \meta{else}
% otherwise.
%
% \begin{declcs}{zref@iflistcontainsprop}
%   \M{listname} \M{propname} \M{then} \M{else}
% \end{declcs}
% Executes \meta{then} if the property \meta{propname} is part
% of property list \meta{listname} or otherwise it
% runs the \meta{else} part.
%
% \subsection{Property}
%
% \begin{declcs}{zref@newprop}
%   \* \M{propname} |[|\meta{default}|]| \M{value}
% \end{declcs}
% This command declares and configures a new property with
% name \meta{propname}.
%
% In case of unknown references
% or the property does not exist in the reference, the
% \meta{default} is used as value. If it is not specified
% here, a global default is used, see \cs{zref@setdefault}.
%
% The correct values of some properties are not known
% immediately but at page shipout time. Prominent example
% is the page number. These properties are declared with
% the star form of the command.
%
% \begin{declcs}{zref@setcurrent}
%   \M{propname} \M{value}
% \end{declcs}
% This sets the current value of the property \meta{propname}.
% It is a generalization of setting \hologo{LaTeX}'s \cs{currentlabel}.
%
% \begin{declcs}{zref@getcurrent}\markexptwo
%   \M{propname}
% \end{declcs}
% This returns the current value of the property \meta{propname}.
% The value may not be correct, especially if the property
% is bound to a page (start form of \cs{zref@newprop}) and the
% right value is only known at shipout time (e.g. property `page').
% In case of errors (e.g. unknown property) the empty string is
% returned.
%
% Since version 2010/04/22 v2.14 \cs{zref@getcurrent} supports
% \cs{zref@wrapper@unexpanded}.
%
% \begin{declcs}{zref@propexists}
%   \M{propname} \M{then}
% \end{declcs}
% Calls \meta{then} if the property \meta{propname} is available
% or generates an error message otherwise.
%
% \begin{declcs}{zref@ifpropundefined}\markexp
%   \M{propname} \M{then} \M{else}
% \end{declcs}
% Calls \meta{then} or \meta{else} depending on the existence
% of property \meta{propname}.
%
% \subsection{Reference generation}
%
% \begin{declcs}{zref@label}
%   \M{refname}
% \end{declcs}
% This works similar to \cs{label}. The reference \meta{refname} is
% created and put into the \xfile{.aux} file with the
% properties of the main property list.
%
% \begin{declcs}{zref@labelbylist}
%   \M{refname} \M{listname}
% \end{declcs}
% Same as \cs{zref@label} except that the properties are
% taken from the specified property list \meta{listname}.
%
% \begin{declcs}{zref@labelbyprops}
%   \M{refname} |{|\meta{propnameA}|,|\meta{propnameB}|,|\dots|}|
% \end{declcs}
% Same as \cs{zref@label} except that these properties
% are used that are given as comma separated list in the
% second argument.
%
% \begin{declcs}{zref@newlabel}
%   \M{refname} |{|\dots|}|
% \end{declcs}
% This is the macro that is used in the \xfile{.aux} file.
% It is basically the same as \cs{newlabel} apart from
% the format of the data in the second argument.
%
% \subsection{Data extraction}
%
% \begin{declcs}{zref@extractdefault}\markexptwo
%   \M{refname} \M{propname} \M{default}
% \end{declcs}
% This is the basic command that refernces the value of a
% property \meta{propname} for the reference \meta{refname}.
% In case of errors such as undefined reference the \meta{default}
% is used instead.
%
% \begin{declcs}{zref@extract}\markexptwo
%   \M{refname} \M{propname}
% \end{declcs}
% The command is an abbreviation for \cs{zref@extractdefault}.
% As default the default of the property is taken, otherwise
% the global default.
%
% Example for page references:
% \begin{quote}
% \begin{tabular}{@{}ll}
% \hologo{LaTeX}:& |\pageref{foobar}|\\
% \xpackage{zref}:& |\zref@extract{foobar}{page}|
% \end{tabular}
% \end{quote}
% Both \cs{zref@extract} and \cs{zref@extractdefault} are
% expandable. That means, these macros can directly
% be used in expandable calculations, see the example file.
% On the other side, babel's shorthands are not supported,
% there are no warnings in case of undefined references.
%
% If an user interface doesn't need expandable macros
% then it can use \cs{zref@refused} and \cs{zref@wrapper@babel}
% for its user macros.
%
% \begin{declcs}{zref@refused}
%   \M{refname}
% \end{declcs}
% This command is not expandable. It causes the warnings if
% the reference \meta{refname} is not defined. Use the
% \cs{zref@extract} commands inside expandable contexts and
% mark their use outside by \cs{zref@refused}, see the example
% file.
%
% \begin{declcs}{zref@def@extract}%
%     \M{cmd} \M{refname} \M{propname}\\
%   \cs{zref@def@extractdefault}
%     \M{cmd} \M{refname} \M{propname} \M{default}
% \end{declcs}
% Both macros extract the property \meta{propname} from the reference
% \meta{refname} the same way as macros \cs{zref@extract} and
% \cs{zref@extractdefault}. The result is stored in macro \meta{cmd}.
% Also \cs{zref@refused} is called to notify \hologo{LaTeX} that
% the reference \meta{refname} is used. Added in 2011/10/04 v2.22.
%
% \begin{declcs}{zref@ifrefundefined}\markexp
%   \M{refname} \M{then} \M{else}
% \end{declcs}
% Macro \cs{zref@ifrefundefined} calls arguments \meta{then} or \meta{else}
% dependent on the existence of the reference \meta{refname}.
%
% \begin{declcs}{zifrefundefined}
%   \M{refname} \M{then} \M{else}
% \end{declcs}
% Macro \cs{zifrefundefined} calls \cs{ref@refused} before
% executing \cs{zref@ifrefundefined}. Babel shorthands are
% supported in \meta{refname}.
%
% \begin{declcs}{zref@ifrefcontainsprop}\markexp
%   \M{refname} \M{propname} \M{then} \M{else}
% \end{declcs}
% Test whether a reference provides a property.
%
% \subsection{Setup}
%
% \begin{declcs}{zref@default}
% \end{declcs}
% Holds the global default for unknown values.
%
% \begin{declcs}{zref@setdefault}
%   |{|\meta{value}|}|
% \end{declcs}
% Sets the global default for unknown values.
% The global default is used, if a property does not specify
% an own default and the value for a property
% cannot be extracted. This can happen if the reference is
% unknown or the reference does not have the property.
%
% \begin{declcs}{zref@setmainlist}
%   |{|\meta{value}|}|
% \end{declcs}
% Sets the name of the main property list. The package
% sets and uses |main|.
%
% \subsection{Declared properties}
%
% \begin{quote}
% \renewcommand*{\arraystretch}{1.05}
% \setlength{\tabcolsep}{2\tabcolsep}
% \def\VV{\vphantom{\large \"Ag}}
% \def\EMPTY{{\itshape\textless empty\textgreater}}
% \def\pl#1{&#1&thepage&\cs{number}\cs{#1}}
% \begin{tabular}{@{}>{\sffamily}l|>{\ttfamily}l|>{\ttfamily}l|>{\ttfamily}l@{}}
%   \rmfamily Module&
%   \rmfamily Property&
%   \rmfamily Property list&
%   \rmfamily Default\\
%   \hline
%   (base)
%     & default & main & \EMPTY\\
%     & page & main & \EMPTY\\
%   \hline
%   abspage
%     & abspage & main & 0\\
%   \hline
%   counter
%     & counter & main & \EMPTY\\
%   \hline
%   hyperref
%     & anchor & main & \EMPTY\\
%     & url    & & \EMPTY\\
%   \hline
%   pageattr
%     & pdfpageattr & thepage & \dots\\
%     & pdfpagesattr & LastPage & \dots\\
%   \hline
%   pagelayout\footnotemark
%     \pl{mag}\\
%     \pl{paperwidth}\\
%     \pl{paperheight}\\
%     \pl{stockwidth}\\
%     \pl{stockheight}\\
%     \pl{pdfpageheight}\\
%     \pl{pdfpagewidth}\\
%     \pl{pdfhorigin}\\
%     \pl{pdfvorigin}\\
%     \pl{hoffset}\\
%     \pl{voffset}\\
%     \pl{topmargin}\\
%     \pl{oddsidemargin}\\
%     \pl{evensidemargin}\\
%     \pl{textwidth}\\
%     \pl{textheight}\\
%     \pl{headheight}\\
%     \pl{headsep}\\
%     \pl{footskip}\\
%     \pl{marginparwidth}\\
%     \pl{marginparsep}\\
%     \pl{columnwidth}\\
%     \pl{columnsep}\\
%   \hline
%   perpage
%     & pagevalue & perpage & 0\\
%     & page & perpage & \EMPTY\\
%     & abspage & perpage & 0\\
%   \hline
%   savepos
%     & posx & savepos & 0\\
%     & posy & savepos & 0\\
%   \hline
%   titleref
%     & title & main & \EMPTY\\
%   \hline
%   xr
%     & anchor   & & \EMPTY\\
%     & externaldocument & & \EMPTY\\
%     & theotype & & \EMPTY\\
%     & title    & & \EMPTY\\
%     & url      & & \EMPTY\\
% \end{tabular}
% \footnotetext{Module \xmodule{pagelayout} only defines properties
% if the parameter exists.}%
% \end{quote}
%
% \subsection{Wrapper for advanced situations}
%
% \begin{declcs}{zref@wrapper@babel}
%   |{|\dots|}| \M{name}
% \end{declcs}
% This macro helps to add shorthand support. The second
% argument is protected, then the code of the first argument
% is called with the protected name appended. Examples
% are in the sources.
%
% \begin{declcs}{zref@wrapper@immediate}
%   |{|\dots|}|
% \end{declcs}
% There are situations where a label must be written instantly
% to the \xfile{.aux} file, for example after the last page.
% If the \cs{zlabel} or \cs{label} command is put inside this wrapper,
% immediate writing is enabled. See the implementation for
% module \xmodule{lastpage} for an example of its use.
%
% \begin{declcs}{zref@wrapper@unexpanded}
%   |{|\dots|}|
% \end{declcs}
% Assuming someone wants to extract a value for property |bar|
% and store the result in a macro |\foo| without traces of
% the expanding macros and without expanding the value.
% This (theoretical?) problem can be solved by this wrapper:
% \begin{quote}
%\begin{verbatim}
%\zref@wrapper@unexpanded{%
%  \edef\foo{%
%    \zref@extract{someref}{bar}%
%  }%
%}
%\end{verbatim}
% \end{quote}
% The \cs{edef} forces the expansion of \cs{zref@extract},
% but the extraction of the value is prevented by
% the wrapper that uses \hologo{eTeX}'\ \cs{unexpanded} for this
% purpose. Supported macros are \cs{zref@extract}, \cs{zref@extractdefault}
% and since version 2010/04/22 v2.14 macro \cs{zref@getcurrent}.
%
% \subsection{Counter for unique names}
%
% Some modules (\xmodule{titleref} and \xmodule{dotfillmin})
% need unique names for automatically generated label names.
%
% \begin{declcs}{zref@require@unique}
% \end{declcs}
% This command creates the unique counter |zref@unique| if the
% counter does not already exist.
%
% \begin{declcs}{thezref@unique}
% \end{declcs}
% This command is used to generate unique label names.
%
% \section{User interface}\label{sec:user}
%
% \subsection{Module \xmodule{user}}
%
% The user interface for this package and its modules
% is enabled by \xpackage{zref}'s package option \xmodule{user}
% or package \xpackage{zref-user}.
% The names of user commands are prefixed by |z| in order
% to avoid name clashes with existing macros of the same
% functionality. Thus the package does not disturb the
% traditional reference scheme, both can be used together.
%
% The syntax descriptions contain the following markers
% that are intended as hints for programmers:
% \begin{quote}
% \begin{tabular}{@{}ll@{}}
%   \markbabel\unskip  & Babel shorthands are allowed.\\
%   \markrobust\unskip & Robust macro.\\
%   \markexp\unskip    & Expandable version:\\
%                      & \textbullet\ robust,
%                        unless the extracted values are fragile,\\
%                      & \textbullet\ no babel shorthand suport.\\
%   \markexptwo\unskip & Expandable like \markexp and:\\
%                      & \textbullet\ expandable in exact two steps.\\
% \end{tabular}
% \end{quote}
%
% The basic user interface of the package without modules
% are commands that mimic the standard \hologo{LaTeX} behaviour
% of \cs{label}, \cs{ref}, and \cs{pageref}:
%
% \begin{declcs}{zlabel}
%   \M{refname}\markbabel
% \end{declcs}
% Similar to \cs{label}. It generates a label with name
% \meta{refname} in the new reference scheme.
%
% \begin{declcs}{zref}
%   |[|\meta{propname}|]| \M{refname}\markbabel
% \end{declcs}
% Without optional argument similar to \cs{ref}, it returns
% the default reference property. This property is named
% |default|:
% \[
%   |\zref{|x|}| \equiv |\zref[default]{|x|}|
% \]
%
% \begin{declcs}{zpageref}
%   \M{refname}\markbabel
% \end{declcs}
% Convenience macro, similar to \cs{pageref}.
% \[
%   |\zpageref{|x|}| \equiv |\zref[page]{|x|}|
% \]
%
% \begin{declcs}{zrefused}
%   \M{refname}\markbabel
% \end{declcs}
% Some of the user commands in the modules are expandable.
% The use of such commands do not cause any undefined
% reference warnings, because inside of expandable
% contexts this is not possible. However, if there is
% a place outside of expandable contexts, \cs{refused}
% is strongly recommended. The reference \meta{refname}
% is marked as used, undefined ones will generate
% warnings.
%
% \subsection{Module \xmodule{abspage}}
%
% With the help of package \xpackage{atbegshi}
% a new counter |abspage| with absolute page numbers is provided.
% For technical and historical reasons the counter itself is zero based:
% if you use it directly on the first page, e.g
% with |\arabic{abspage}| you will get 0 as value. When using |\zref|
% the first page will be page 1 as expected.
% Also a new property |abspage| is defined and added to the
% main property list. Thus you can reference the absolute
% page number:
% \begin{quote}
%   |Section \zref{foo} is on page \zpageref{foo}.|\\
%   |This is page \zref[abspage]{foo}|\\
%   |of \zref[abspage]{LastPage}.|
% \end{quote}
% The example also makes use of module \xmodule{lastpage}.
%
% \subsection{Module \xmodule{lastpage}}
%
% Provides the functionality of package \xpackage{lastpage}
% \cite{lastpage} in the new reference scheme.
% The label |LastPage| is put at the end of the document.
% You can refer the last page number with:
% \begin{quote}
%   |\zref@extract{LastPage}{page}| (+ |\zref@refused{LastPage}|)\\
% \end{quote}
% or
% \begin{quote}
%   |\zpageref{LastPage}| (module \xmodule{user})
% \end{quote}
%
% Since version 2008/10/01 v2.3 the module defines the list |LastPage|.
% In addition to the properties of the main list label |LastPage| also
% stores the properties of this list |LastPage|. The default of this
% list is empty. The list can be used by the user to add additional
% properties for label |LastPage|.
%
% \subsubsection{Tests for last page}
%
% Since version 2010/03/26 v2.8 the macros \cs{zref@iflastpage}
% and \cs{ziflastpage} were added. They test the reference,
% whether it is a reference of the last page.
%
% \begin{declcs}{zref@iflastpage}\markexp \M{refname} \M{then} \M{else}
% \end{declcs}
% Macro \cs{zref@iflastpage} compares the references \meta{refname}
% with \meta{LastPage}. Basis of the comparison is the value of
% property |abspage|, because the values are different for different
% pages. This is not ensured by property |page|. Therefore module
% \xmodule{abspage} is loaded by module \xmodule{lastpage}.
% If both values of property |abspage| are present and match,
% then \meta{then} is executed, otherwise code \meta{else} is called.
% If one or both references are undefined or lack the property |abspage|,
% then \meta{else} is executed.
%
% Macro \cs{zref@iflastpage} is expandable, therefore \cs{zref@refused}
% should be called on \meta{refname} and \meta{LastPage}.
%
% \begin{declcs}{ziflastpage} \M{refname} \M{then} \M{else}
% \end{declcs}
% Macro \cs{ziflastpage} has the same function as \cs{zref@iflastpage},
% but adds support for babel shorthands in \meta{refname} and
% calls \cs{zref@refused}. However macro \cs{ziflastpage} is not
% expandable.
%
% \subsubsection{Example}
%
%    \begin{macrocode}
%<*example-lastpage>
%<<END_EXAMPLE
\NeedsTeXFormat{LaTeX2e}
\documentclass{report}

\newcounter{foo}
\renewcommand*{\thefoo}{\Alph{foo}}

\usepackage{zref-lastpage,zref-user}[2019/11/29]

\makeatletter
\zref@newprop{thefoo}{\thefoo}
\zref@newprop{valuefoo}{\the\value{foo}}
\zref@newprop{chapter}{\thechapter}
\zref@addprops{LastPage}{thefoo,valuefoo,chapter}
\makeatother

\newcommand*{\foo}{%
  \stepcounter{foo}%
  [Current foo: \thefoo]%
}

\begin{document}
  \chapter{First chapter}
  Last page is \zref{LastPage}.\\
  Last chapter is \zref[chapter]{LastPage}.\\
  Last foo is \zref[thefoo]{LastPage}.\\
  Last value of foo is \zref[valuefoo]{LastPage}.\\
  \foo
  \chapter{Second chapter}
  \foo\foo\foo
  \chapter{Last chapter}
  \foo
\end{document}
%END_EXAMPLE
%</example-lastpage>
%    \end{macrocode}
%
% \subsection{Module \xmodule{thepage}}
%
% This module \xmodule{thepage} loads module \xmodule{abspage},
% constructs a reference name using the absolute page number
% and remembers property |page|. Other properties can be added
% by adding them to the property list |thepage|.
%
% \begin{declcs}{zthepage} \M{absolute page number}\\
% \end{declcs}
% Macro \cs{zthepage} is basically a \cs{zpageref}. The reference
% name is yield by the \meta{absolute page number}. If the reference
% is not defined, then the default for property |page| is used.
%
% \begin{declcs}{zref@thepage@name}\markexp \M{absolute page number}
% \end{declcs}
% Macro \cs{zref@thepage@name} returns the internal reference
% name that is constructed using the \meta{absolute page number}.
% The internal reference name should not be used directly, because
% it might change in future versions.
%
% \begin{declcs}{zref@thepage}\markexp \M{absolute page number}\\
%   \cs{zref@thepage@refused} \M{absolute page number}
% \end{declcs}
% Macro \cs{zref@thepage} returns the page number (\cs{thepage})
% of \meta{absolute page number}. Because this macro is expandable,
% \cs{zref@thepage@refused} is used outside an expandable context
% to mark the reference as used.
%
%
% \subsection{Module \xmodule{nextpage}}
%
% \begin{declcs}{znextpage}
% \end{declcs}
% Macro \cs{znextpage} prints \cs{thepage} of the following page.
% It gets the current absolute page number by using a label.
% There are three cases for the next page:
% \begin{enumerate}
% \item The next page is not known yet because of undefined references.
%   Then \cs{zunknownnextpagename} is used instead. The default for
%   this macro is the default of property |page|.
% \item This page is the last page. Then \cs{znonextpagename}
%   is used. Its default is empty.
% \item The next page is known, then \cs{thepage} of the next
%   page is used (the value of property |page| of the next page).
% \end{enumerate}
%
% \subsubsection{Configuration}
%
% The behaviour can be configured by the following macros.
%
% \begin{declcs}{zunknownnextpagename}\\
%   \cs{znonextpagename}
% \end{declcs}
% If the next page is not known or available, then
% \cs{znextpage} uses these name macros as default.
% \cs{zunknownnextpagename} is used in case of undefined
% references. Default is the value of property |page| of the
% next page (\cs{thepage}). Module \xmodule{thepage} is used.
%
% Macro \cs{znonextpagename} is used, if the next page does
% not exists. That means that the current page is last page.
% The default is empty.
%
% \begin{declcs}{znextpagesetup} \M{unknown} \M{no next}
%   \M{next}
% \end{declcs}
% Acording to the case (see \cs{znextpage}) macro \cs{znextpage}
% calls an internal macro with an argument. The argument is
% either \cs{thepage} of the next page or one of \cs{zunknownnextpagename}
% or \cs{znonextpagename}. These internal macro can be changed by
% \cs{znextpagesetup}. It expects the definition texts for these
% three cases of a macro with one argument.
% The default is
% \begin{quote}
%   |\znextpagesetup{#1}{#1}{#1}|
% \end{quote}
%
% \subsubsection{Example}
%
%    \begin{macrocode}
%<*example-nextpage>
%<<END_EXAMPLE
\documentclass{book}

\usepackage{zref-nextpage}[2019/11/29]
\znextpagesetup
  {\thepage}% next page is unknown
  {\thepage\ (#1)}% this page is last page
  {\thepage\ $\rightarrow$ #1}% next page is known
\renewcommand*{\znonextpagename}{last page}

\usepackage{fancyhdr}
\pagestyle{fancy}
\fancyhf{}
\fancyhead[LE,RO]{\znextpage}
\fancypagestyle{plain}{%
  \fancyhf{}%
  \fancyhead[LE,RO]{\znextpage}%
}

\begin{document}
\frontmatter
  \tableofcontents
\mainmatter
  \chapter{Hello World}
  \clearpage
  \section{Last section}
\end{document}
%END_EXAMPLE
%</example-nextpage>
%    \end{macrocode}
%
%
% \subsection{Module \xmodule{totpages}}
%
% For the total number of pages of a document you need to
% know the absolute page number of the last page.
% Both modules \xmodule{abspage} and \xmodule{lastpage}
% are necessary and automatically enabled.
%
% \begin{declcs}{ztotpages}\markexp
% \end{declcs}
% Prints the total number of pages or |0| if this
% number is not yet known. It expands to an explicit number
% and can also used even in expandable calculations (\cs{numexpr})
% or counter assignments.
%
% \subsection{Module \xmodule{pagelayout}}
%
% The module defines additional properties for each parameter
% of the page layout that is effective during page shipout.
% The value of length parameters is given in sp without the
% unit as plain number.
%
% Some parameters are specific for a class (e.g.\@ \xoption{stockwidth}
% and \xoption{stockheight} for class \xclass{memoir}) or the
% \hologo{TeX} engine like \hologo{pdfTeX}. If the parameter is not
% available, then the property will not be defined. The default value
% of the property is the current setting of the parameter.
%
% The module \xmodule{thepage} is loaded that generates a label
% for each page. The properties of module \xmodule{pagelayout}
% are added to the property list |thepage| of module \xmodule{thepage}.
%
% List of properties:
% \begin{quote}
%   \def\x#1{\cs{#1}&#1}%
%   \begin{tabular}{l>{\sffamily}ll}
%     \textrm{parameter} & \textrm{property} & remarks\\
%     \hline
%     \x{mag}\\
%     \x{paperwidth}\\
%     \x{paperheight}\\
%     \x{stockwidth}& class \xclass{memoir}\\
%     \x{stockheight}& class \xclass{memoir}\\
%     \x{pdfpagewidth}& \hologo{pdfTeX}, \hologo{LuaTeX}\\
%     \x{pdfpageheight}& \hologo{pdfTeX}, \hologo{LuaTeX}\\
%     \x{pdfhorigin}& \hologo{pdfTeX}, \hologo{LuaTeX}\\
%     \x{pdfvorigin}& \hologo{pdfTeX}, \hologo{LuaTeX}\\
%     \x{hoffset}\\
%     \x{voffset}\\
%     \x{topmargin}\\
%     \x{oddsidemargin}\\
%     \x{evensidemargin}\\
%     \x{textwidth}\\
%     \x{textheight}\\
%     \x{headheight}\\
%     \x{headsep}\\
%     \x{footskip}\\
%     \x{marginparwidth}\\
%     \x{marginparsep}\\
%     \x{columnwidth}\\
%     \x{columnsep}\\
%   \end{tabular}
% \end{quote}
%
% \begin{declcs}{zlistpagelayout}
% \end{declcs}
% At the end of document the page layout parameter for each page are
% printed into the \xext{log} file if macro \cs{zlistpagelayout}
% is called before |\end{document}| (preamble is a good place).
%
% \subsection{Module \xmodule{marks}}
%
% ToDo.
%
% \subsection{Module \xmodule{runs}}
%
% Module \xmodule{runs} counts the \hologo{LaTeX} runs since last
% \xext{aux} file creation and prints the number in the
% \xext{log} file.
%
% \begin{declcs}{zruns}\markexp
% \end{declcs}
% Prints the the total number of \hologo{LaTeX} runs including
% the current one. It expands to an explicit number.
% Before |begin{document}| the value is zero meaning
% the \xext{aux} file is not read yet. If a previous
% \xext{aux} file exists, the value found there increased
% by one is the new number. Otherwise \cs{zruns} is set to one.
% \hologo{LaTeX}~runs where the \xext{aux} files are not rewritten
% are not counted (see \cs{nofiles}).
%
% \subsection{Module \xmodule{perpage}}
%
% With \cs{@addtoreset} or \cs{numberwithin} a counter can be
% reset if another counter is incremented. This do not work
% well if the other counter is the page counter. The page counter is
% incremented in the output routine that is often called asynchronous
% somewhere on the next page. A reference mechanism costs at least
% two \hologo{LaTeX} runs, but ensures correct page counter values.
%
% \begin{declcs}{zmakeperpage}
%   |[|\meta{reset}|]| \M{counter}
% \end{declcs}
% At the of a new page counter \meta{counter} starts
% counting with value \meta{reset} (default is \texttt{1}).
% The macro has the same
% syntax and semantics as \cs{MakePerPage} of package
% \xpackage{perpage} \cite{perpage}.
% Also \xoption{perpage} of package \xpackage{footmisc} \cite{footmisc}
% can easily be simulated by
% \begin{quote}
% |\zmakeperpage{footnote}| \itshape |% \usepackage[perpage]{footmisc}|
%\end{quote}
% If footnote symbols are used, some people dislike the
% first symbol \ensuremath{\dagger}. It can easily be skipped:
% \begin{quote}
% |\zmakeperpage[2]{footnote}|
% \end{quote}
%
% \begin{declcs}{thezpage}\\
%   counter |zpage|
% \end{declcs}
% If the formatted counter value of the counter that is reset at
% a new page contains the page value, then you can use \cs{thezpage},
% the page number of the current page. Or counter |zpage| can be
% used, if the page number should be formatted differently from
% the current page number. Example:
%\begin{quote}
%\begin{verbatim}
%\newcounter{foobar}
%\zmakeperpage{foobar}
%\renewcommand*{\thefoobar}{\thezpage-\arabic{foobar}}
% % or
%\renewcommand*{\thefoobar}{\roman{zpage}-\arabic{foobar}}
%\end{verbatim}
%\end{quote}
%
% \begin{declcs}{zunmakeperpage}
%   \M{counter}
% \end{declcs}
% The reset mechanism for this counter is deactivated.
%
%
% \subsection{Module \xmodule{counter}}
%
% This option just add the property |counter| to the main
% property list. The property stores the counter name,
% that was responsible for the reference. This is the
% property \xpackage{hyperref}'s \cs{autoref} feature
% uses. Thus this property |counter| may be useful
% for a reimplementation of the autoref feature,
% see the section \ref{sec:todo} with the todo list.
%
% \subsection{Module \xmodule{titleref}}
%
% This option makes section and caption titles available
% to the reference system similar to packages \xpackage{titleref}
% or \xpackage{nameref}.
%
% \begin{declcs}{ztitleref}
%   \M{refname}\markbabel
% \end{declcs}
% Print the section or caption title of reference \meta{refname},
% similar to \cs{nameref} or \cs{titleref}.
%
% \begin{declcs}{ztitlerefsetup}
%   |{|\textit{key$_1$=value$_1$, key$_2$=value$_2$, \dots}|}|
% \end{declcs}
% This command allows to configure the behaviour of module
% \xmodule{titleref}. The following keys are available:
% \begin{description}
% \let\bfseries\relax
% \item[\texttt{title=}\meta{value}]\mbox{}\\*
%   Sets the current title.
% \item[\texttt{stripperiod=true\string|false}]\mbox{}\\*
%   Follow package \xpackage{nameref} that
%   removes a last period. Default: |true|.
% \item[\texttt{expand=true\string|false}]\mbox{}\\*
%   Package \cs{titleref} expands the title first. This way
%   garbage and dangerous commands can be removed, e.g.
%   \cs{label}, \cs{index}\dots. See implementation section
%   for more details. Default is |false|.
% \item[\texttt{cleanup=\{}\textrm{\dots}\texttt{\}}]\mbox{}\\*
%   Hook to add own cleanup code, if method |expand| is used.
%   See implementation section for more details.
% \end{description}
%
% \subsection{Module \xmodule{savepos}}
%
% This option supports a feature that \pdfTeX\ provides
% (and \hologo{XeTeX}).
% \pdfTeX\ is able to tell the current position on the
% page. The page position is not instantly known. First the
% page must be constructed by \hologo{TeX}'s asynchronous output
% routine. Thus the time where the position is known is
% the page shipout time. Thus a reference system where the
% information is recorded in the first run and made available
% for use in the second run comes in handy.
%
% \begin{declcs}{zsavepos}
%   \M{refname}
% \end{declcs}
% It generates a reference with name \meta{refname}.
% The reference stores the location where \cs{zsavepos}
% is executed in properties |posx| and |posy|.
%
% \begin{declcs}{zsaveposx}
%   \M{refname}\\
%   \cs{zsaveposy}
%   \M{refname}
% \end{declcs}
% Same as \cs{zsavepos} except that only the |x| or |y| component
% of the position is stored. Since 2011/12/05 v2.23.
%
% \begin{declcs}{zposx}\markexp
%   \M{refname}\\
%   \SpecialUsageIndex{\zposy}\cs{zposy}\markexp \M{refname}
% \end{declcs}
% Get the position as number. Unit is sp. Horizontal
% positions by \cs{zposx} increase from left to right.
% Vertical positions by \cs{zposy} from bottom to top.
%
% Do not rely on absolute page numbers. Because of problems
% with the origin the numbers may differ in DVI or PDF mode
% of \pdfTeX. Therefore work with relative values by comparisons.
%
% Both \cs{zposx} and \cs{zposy} are expandable and can be
% used inside calculations (\cs{setcounter}, \cs{addtocounter},
% package \xpackage{calc}, \cs{numexpr}). However this property
% prevents from notifying \hologo{LaTeX} that the reference is actually
% used (the notifying is not expandable). Therefore you should
% mark the reference as used by \cs{zrefused}.
%
% This module uses \pdfTeX's \cs{pdfsavepos}, \cs{pdflastxpos},
% and \cs{pdflastypos}. They are available in PDF mode and
% since version 1.40.0 also in DVI mode.
%
% \begin{declcs}{zref@savepos}
% \end{declcs}
% Macro \cs{zref@savepos} performs the first part of \cs{zsavepos}
% by calling \cs{pdfsavepos} (if \xext{aux} files are writable).
%
% Thus \cs{zsavepos} is basically \cs{zref@savepos} followed
% by |\zref@labelbylist{|\texttt{\meta{refname}}|}{savepos}|.
% If \cs{TeXXeTstate} is detected and enabled, \cs{savepos}
% also adds \cs{zref@savepos} at the end to support \cs{beginR}
% where the whatits are processed in reverse order.
% The property list |savepos| contains the properties |posx|
% and |posy|.
%
% \subsection{Module \xmodule{abspos}}
% Module \xmodule{abspos} allows to get various values of the page layout. There is no user
% command, only a number of internal commands. For example:
% \begin{quote}
%   |\zref@absposx{|\meta{label}|}{|\meta{value}|}{|\meta{position}|}|\\
%   |\zref@absposy{|\meta{label}|}{|\meta{value}|}{|\meta{position}|}|
% \end{quote}
%
% The return value is like in the module \xmodule{savepos} a number representing a length in sp.
% The length are measured from the bottom left of the page.
%
% \meta{label} is a label set with \cs{zlabel} or \cs{zsavepos} that allows to retrieve the
% absolute page number.
%
% \meta{position} is for the x-command one of |left|, |right| or |center|. For the y-command
%  it is one of |top|, |bottom|, |center|.
%
% The possible content of \meta{value} can be seen in the following table. Be aware that the code
% makes some assumptions which are perhaps not always true -- for example that the left of
% the head is identical to the left of the body.
%
% \medskip
% \begin{tabular}{lll}
%  \textbf{value} & \textbf{axis} & \textbf{comments}\\\hline
%   media         & x             &   left=0, right=\cs{pdfpagewidth} \\
%   paper         & x             &   left=0, right=\cs{paperwidth}\\
%   stock         & x             &   derived from paper\\
%   media         & y             & bottom=0, top=\cs{pdfpageheigh}\\
%   paper         & y             & top=\cs{pdfpageheight}, bottom=top-\cs{paperheight}\\
%   stock         & y             & top  derived from paper\\
%   head          & x             & calculated with hoffset, horigin, etc\\
%   head          & y             & calculated\\
%   body          & x             & = head value\\%
%   body          & y             & = head bottom - \cs{headsep}\\
%   foot          & x             & = head\\
%   foot          & y             & calculated from body bottom and \cs{footskip}\\
%   marginpar     & x             &different on odd/even pages! \\
%   marginpar     & y             & = body %
% \end{tabular}
%
% \subsection{Module \xmodule{dotfill}}
%
% \begin{declcs}{zdotfill}
% \end{declcs}
% This package provides the command \cs{zdotfill} that works
% similar to \cs{dotfill}, but can be configured. Especially
% it suppresses the dots if a minimum number of dots cannot be set.
%
% \begin{declcs}{zdotfillsetup}
%   |{|\textit{key$_1$=value$_1$, key$_2$=value$_2$, \dots}|}|
% \end{declcs}
% This command allows to configure the behaviour of \cs{zdotfill}.
% The following keys are available:
% \begin{description}
% \let\bfseries\relax
% \item[\texttt{min=}\meta{count value}]\mbox{}\\*
%   If the actual number of dots are smaller than \meta{count value},
%   then the dots are suppressed. Default: |2|.
% \item[\texttt{unit=}\meta{dimen value}]\mbox{}\\*
%   The width of a dot unit is given by \meta{dimen value}.
%   Default: |0.44em| (same as the unit in \cs{dotfill}).
% \item[\texttt{dot=}\meta{value}]\mbox{}\\*
%   The dot itself is given by \meta{value}. Default: |.| (dot, same as
%   the dot in \cs{dotfill}).
% \end{description}
%
% \subsection{Module \xmodule{env}}
%
% This module defines two properties |envname| and |envline|. They
% remember the name of the environment and the line number at
% the start of the environment.
%
% \subsection{Module \xmodule{xr}}
%
% This package provides the functionality of package \xpackage{xr},
% see \cite{xr}. It also supports the syntax of \xpackage{xr-hyper}.
%
% \begin{declcs}{zexternaldocument}
%   \* |[|\meta{prefix}|]|\markbabel \M{external document}
%   |[|\meta{url}|]|
% \end{declcs}
% See \cs{externaldocument} for a description of this option.
% The found labels also get a property |externaldocument| that
% remembers \meta{external document}.
% The standard reference scheme and the scheme of this
% package use different name spaces for reference names.
% If the external document uses both systems. Then one
% import statement would put the names in one namespace
% and probably causing problems with multiple references
% of the same name. Thus the star form only looks for
% \cs{newlabel} in the \xfile{.aux} files, whereas without
% star only \cs{zref@newlabel}s are used.
%
% In the star form it tries to detect labels from
% \xpackage{hyperref}, \xpackage{titleref}, and \xpackage{ntheorem}.
% If such an extended property from the packages before cannot
% be found or are empty, they are not included in the
% imported reference.
%
% Warnings are given if a reference name is already in use
% and the item is ignored. Unknown properties will automatically
% be declared.
%
% If the external references contain |anchor| properties,
% then we need also a url to be able to address the external
% file. As default the filename is taken with a default
% extension.
%
% \begin{declcs}{zxrsetup}
%   |{|\textit{key$_1$=value$_1$, key$_2$=value$_2$, \dots}|}|
% \end{declcs}
% The following setup options are available:
% \begin{description}
% \item[\xoption{ext}:] It sets the default extension.
% \item[\xoption{tozreflabel}:] Boolean option. The found references
%   are imported as zref labels. This is enabled by default.
% \item[\xoption{toltxlabel}:] Boolean option. The found references
%   are imported as \hologo{LaTeX} labels. Packages \xpackage{nameref},
%   \xpackage{titleref} and class \xclass{memoir} are supported.
% \item[\xoption{urluse}:] Boolean option. If enabled, then a URL
%   is stored in a macro and the macro is put in property `urluse'.
%   The URL is not put in property `url'.
%   The purpose is to save \hologo{TeX} memory.
% \item[\xoption{verbose}:] Boolean option. List the imported labels in the
%   \xext{log} file. Default is |false|.
% \end{description}
%
% \begin{declcs}{zref@xr@ext}
% \end{declcs}
% If the \meta{url} is not specified in \cs{zref@externaldocument},
% then the url will be constructed with the file name and
% this macro as extension. \cs{XR@ext} is used if
% \xpackage{hyperref} is loaded, otherwise |pdf|.
%
% \subsection{Module \xmodule{pageattr}}
%
% This module allows to recover the content of the register
% \cs{pdfpageattr} and \cs{pdfpagesattr} in pdftex and the equivalent register
% in luatex. There is no user command. Programmer commands are
%
% \begin{quote}
% |\zref@pdfpageattr{|\meta{absolute page number}|}|
%
% |\zref@pdfpagesattr{|\meta{absolute page number}|}|
% \end{quote}
% \section{ToDo}\label{sec:todo}
%
% Among other things the following issues are left for future work:
% \begin{itemize}
% \item Other applications: autoref, hyperref, \dots
% \end{itemize}
%
% \section{Example}
%
%    \begin{macrocode}
%<*example>
%    \end{macrocode}
%
%    \begin{macrocode}
\documentclass{book}

\usepackage[ngerman]{babel}%

\usepackage[savepos,totpages,titleref,dotfill,counter,user]{zref}

%    \end{macrocode}
%
%    Chapters are wrapped inside \cs{ChapterStart} and
%    \cs{ChapterStop}. The first argument \verb|#1| of \cs{ChapterStart}
%    is used to form a label id \verb|chap:#1|. At the end of
%    the chapter another label is set by \cs{zref@wrapper@immediate}, because
%    otherwise at the end of document a deferred write would not
%    be written, because there is no page for shipout.
%
%    Also this example shows how chapter titles can be recorded.
%    A new property \verb|chaptitle| is declared and added to the
%    main property list. In \cs{ChapterStart} the current value
%    of the property is updated.
%    \begin{macrocode}
\makeatletter
\zref@newprop{chaptitle}{}
\zref@addprop{main}{chaptitle}

\newcommand*{\ChapterStart}[2]{%
  \cleardoublepage
  \def\current@chapid{#1}%
  \zref@setcurrent{chaptitle}{#2}%
  \chapter{#2}%
  \zlabel{chap:#1}%
}
\newcommand*{\ChapterStop}{%
  \cleardoublepage
  \zref@wrapper@immediate{%
    \zref@labelbyprops{chapend:\current@chapid}{abspage}%
  }%
}
%    \end{macrocode}
%    \cs{ChapterPages} calculates and returns the number of pages of the
%    referenced chapter.
%    \begin{macrocode}
\newcommand*{\ChapterPages}[1]{%
  \zrefused{chap:#1}%
  \zrefused{chapend:#1}%
  \number\numexpr
    \zref@extract{chapend:#1}{abspage}%
    -\zref@extract{chap:#1}{abspage}%
    +1\relax
}
\makeatother
%    \end{macrocode}
%    \begin{macrocode}
\begin{document}
%    \end{macrocode}
%    As exception we use \cs{makeatletter} here, because
%    this is just an example file that also should show some of
%    programmer's interface.
%    \begin{macrocode}
\makeatletter

\frontmatter
\zlabel{documentstart}

\begin{itemize}
\item
  The frontmatter part has
  \number\numexpr\zref@extract{chap:first}{abspage}-1\relax
  ~pages.
\item
  Chapter \zref{chap:first} has \ChapterPages{first} page(s).
\item
  Section \zref{hello} is on the
  \ifcase\numexpr
    \zref@extractdefault{hello}{page}{0}%
    -\zref@extractdefault{chap:first}{page}{0}%
    +1\relax
    ??\or first\or second\or third\or forth\fi
  ~page inside its chapter.
\item
  The document has
  \zref[abspage]{LastPage} pages.
  This number is \ifodd\ztotpages odd\else even\fi.
\item
  The last page is labeled with \zpageref{LastPage}.
\item
  The title of chapter \zref{chap:next} %
  is ``\zref[chaptitle]{chap:next}''.
\end{itemize}

\tableofcontents

\mainmatter
\ChapterStart{first}{First chapter}

%    \end{macrocode}
%    The user level commands should protect babel shorthands
%    where possible. On the other side, expandable extracting
%    macros are useful in calculations, see above the
%    examples with \cs{numexpr}.
%    \begin{macrocode}
\section{Test}
\zlabel{a"o}
Section \zref{a"o} on page
\zref@wrapper@babel\zref@extract{a"o}{page}.

Text.
\newpage

\section{Hello World}
\zlabel{hello}

\ChapterStop

\ChapterStart{next}{Next chapter with \emph{umlauts}: "a"o"u"s}

%    \end{macrocode}
%
%    Here an example follows that makes use of \pdfTeX's ``savepos''
%    feature. The position on the page is not known before the
%    page is constructed and shipped out. Therefore the position
%    ist stored in references and are available for calculations
%    in the next \hologo{LaTeX} compile run.
%    \begin{macrocode}
The width of the first column is
  \the\dimexpr \zposx{secondcol}sp - \zposx{firstcol}sp\relax,\\
the height difference of the two baselines is
  \the\dimexpr \zposy{firstcol}sp - \zposy{secondline}sp\relax:\\
\begin{tabular}{ll}
  \zsavepos{firstcol}Hello&\zsavepos{secondcol}World\\
  \zsavepos{secondline}Second line&foobar\\
\end{tabular}

%    \end{macrocode}
%    With \cs{zrefused} \hologo{LaTeX} is notified, if the references
%    are not yet available and \hologo{LaTeX} can generate the rerun
%    hint.
%    \begin{macrocode}
\zrefused{firstcol}
\zrefused{secondcol}
\zrefused{secondline}

\ChapterStop
%    \end{macrocode}
%    Test for module \cs{dotfill}.
%    \begin{macrocode}
\ChapterStart{dotfill}{Test for dotfill feature}
\newcommand*{\dftest}[1]{%
  #1&
  [\makebox[{#1}]{\dotfill}]&
  [\makebox[{#1}]{\zdotfill}]\\
}
\begin{tabular}{rll}
& [\verb|\dotfill|] & [\verb|\zdotfill|]\\
\dftest{0.43em}
\dftest{0.44em}
\dftest{0.45em}
\dftest{0.87em}
\dftest{0.88em}
\dftest{0.89em}
\dftest{1.31em}
\dftest{1.32em}
\dftest{1.33em}
\end{tabular}
\ChapterStop
\end{document}
%    \end{macrocode}
%
%    \begin{macrocode}
%</example>
%    \end{macrocode}
%
% \StopEventually{
% }
%
% \section{Implementation}
%
% \subsection{Package \xpackage{zref}}
%
% \subsubsection{Identification}
%
%    \begin{macrocode}
%<*package>
\NeedsTeXFormat{LaTeX2e}
\ProvidesPackage{zref}
  [2020-07-03 v2.32 A new reference scheme for LaTeX (HO)]%
%    \end{macrocode}
%
% \subsubsection{Load basic module}
%
%    \begin{macrocode}
\RequirePackage{zref-base}[2019/11/29]
%    \end{macrocode}
%    Abort package loading if \xpackage{zref-base} could not be
%    loaded successfully.
%    \begin{macrocode}
\@ifundefined{ZREF@base@ok}{\endinput}{}
%    \end{macrocode}
%
% \subsubsection{Process options}
%
%    Known modules are loaded and the release date is checked.
%    \begin{macrocode}
\def\ZREF@temp#1{%
  \DeclareOption{#1}{%
    \AtEndOfPackage{%
      \RequirePackage{zref-#1}[2019/11/29]%
    }%
  }%
}
\ZREF@temp{abspage}
\ZREF@temp{counter}
\ZREF@temp{dotfill}
\ZREF@temp{hyperref}
\ZREF@temp{lastpage}
\ZREF@temp{marks}
\ZREF@temp{nextpage}
\ZREF@temp{pageattr}
\ZREF@temp{pagelayout}
\ZREF@temp{perpage}
\ZREF@temp{runs}
\ZREF@temp{savepos}
\ZREF@temp{thepage}
\ZREF@temp{titleref}
\ZREF@temp{totpages}
\ZREF@temp{user}
\ZREF@temp{xr}
%    \end{macrocode}
%    \begin{macrocode}
\ProcessOptions\relax
%</package>
%    \end{macrocode}
%
%
% \subsection{Module \xmodule{base}}
%
% \subsubsection{Prefixes}
%
%    This package uses the following prefixes for macro names:
%    \begin{description}
%    \item[\cs{zref@}:] Macros of the programmer's interface.
%    \item[\cs{ZREF@}:] Internal macros.
%    \item[\cs{Z@L@}\metacs{listname}:]
%      The properties of the list \meta{listname}.
%    \item[\cs{Z@D@}\metacs{propname}:]
%      The default value for property \meta{propname}.
%    \item[\cs{Z@E@}\metacs{propname}:]
%      Extract function for property \meta{propname}.
%    \item[\cs{Z@X@}\metacs{propname}:]
%      Information whether a property value for property
%      \meta{propname} is expanded
%      immediately or at shipout time.
%    \item[\cs{Z@C@}\metacs{propname}:]
%      Current value of the property \meta{propname}.
%    \item[\cs{Z@R@}\metacs{labelname}:]
%      Data for reference \meta{labelname}.
%    \item[\cs{ZREF@org@}:] Original versions of patched commands.
%    \item[\cs{z}:] For macros in user land, defined if
%      module \xmodule{user} is set.
%    \end{description}
%    The following family names are used for keys defined
%    according to the \xpackage{keyval} package:
%    \begin{description}
%    \item[\texttt{ZREF@TR}:] Setup for module \xmodule{titleref}.
%    \end{description}
%
% \subsubsection{Identification}
%
%    \begin{macrocode}
%<*base>
\NeedsTeXFormat{LaTeX2e}
\ProvidesPackage{zref-base}%
  [2020-07-03 v2.32 Module base for zref (HO)]%
%    \end{macrocode}
%
% \subsubsection{Utilities}
%
%    \begin{macrocode}
\providecommand\IfFormatAtLeastTF{\@ifl@t@r\fmtversion}
\RequirePackage{ltxcmds}[2010/12/02]
\RequirePackage{infwarerr}[2010/04/08]
\RequirePackage{kvsetkeys}[2010/03/01]
\RequirePackage{kvdefinekeys}[2010/03/01]
\RequirePackage{pdftexcmds}[2010/04/01]
%    \end{macrocode}
%
%    \begin{macro}{\ZREF@name}
%    Several times the package name is used, thus we store it
%    in \cmd{\ZREF@name}.
%    \begin{macrocode}
\def\ZREF@name{zref}
%    \end{macrocode}
%    \end{macro}
%
%    \begin{macrocode}
\ltx@IfUndefined{protected}{%
  \RequirePackage{makerobust}[2006/03/18]%
%    \end{macrocode}
%    \begin{macro}{\ZREF@Robust}
%    \begin{macrocode}
  \def\ZREF@Robust#1#2{%
    \def\ZREF@temp{\MakeRobustcommand#2}%
    \afterassignment\ZREF@temp
    #1#2%
  }%
%    \end{macrocode}
%    \end{macro}
%    \begin{macrocode}
}{%
%    \end{macrocode}
%    \begin{macro}{\ZREF@Robust}
%    \begin{macrocode}
  \def\ZREF@Robust#1{%
    \protected#1%
  }%
%    \end{macrocode}
%    \end{macro}
%    \begin{macrocode}
}
%    \end{macrocode}
%    \begin{macro}{\ZREF@IfDefinable}
%    \begin{macrocode}
\def\ZREF@IfDefinable#1#2#3{%
  \@ifdefinable{#1}{%
    \ZREF@Robust{#2}#1#3%
  }%
}
%    \end{macrocode}
%    \end{macro}
%
%    \begin{macro}{\ZREF@UpdatePdfTeX}
%    \cs{ZREF@UpdatePdfTeX} is used as help message text in
%    error messages.
%    \begin{macrocode}
\def\ZREF@UpdatePdfTeX{Update pdfTeX.}
%    \end{macrocode}
%    \end{macro}
%
%    \begin{macro}{\ifZREF@found}
%    The following switch is usded in list processing.
%    \begin{macrocode}
\newif\ifZREF@found
%    \end{macrocode}
%    \end{macro}
%
%    \begin{macro}{\ZREF@patch}
%    Macro \cs{ZREF@patch} first checks the existence of
%    the command and safes it.
%    \begin{macrocode}
\def\ZREF@patch#1{%
  \ltx@IfUndefined{#1}{%
    \ltx@gobble
  }{%
    \expandafter\let\csname ZREF@org@#1\expandafter\endcsname
    \csname #1\endcsname
    \ltx@firstofone
  }%
}
%    \end{macrocode}
%    \end{macro}
%
% \subsubsection{Check for \hologo{eTeX}}
%
%    The use of \hologo{eTeX} should be standard nowadays for \hologo{LaTeX}.
%    We test for \hologo{eTeX} in order to use its features later.
%    \begin{macrocode}
\ltx@IfUndefined{eTeXversion}{%
  \PackageError\ZREF@name{%
    Missing support for eTeX; package is abandoned%
  }{%
    Use a TeX compiler that support eTeX and enable eTeX %
    in the format.%
  }%
  \endinput
}{}%
%    \end{macrocode}
%
%    \begin{macrocode}
\RequirePackage{etexcmds}[2007/09/09]
\ifetex@unexpanded
\else
  \PackageError\ZREF@name{%
    Missing e-TeX's \string\unexpanded.\MessageBreak
    Add \string\RequirePackage\string{etexcmds\string} before %
    \string\documentclass%
  }{%
    Probably you are using some package (e.g. ConTeXt) that %
    redefines \string\unexpanded%
  }%
  \expandafter\endinput
\fi
%    \end{macrocode}
%
% \subsubsection{Auxiliary file stuff}
%
%    We are using some commands in the \xfile{.aux} files.
%    However sometimes these auxiliary files are interpreted
%    by \hologo{LaTeX} processes that haven't loaded this package
%    (e.g. package \xpackage{xr}).
%    Therefore we provide dummy definitions.
%    \begin{macrocode}
\RequirePackage{auxhook}
\AddLineBeginAux{%
  \string\providecommand\string\zref@newlabel[2]{}%
}
%    \end{macrocode}
%    \begin{macro}{\ZREF@RefPrefix}
%    \begin{macrocode}
\def\ZREF@RefPrefix{Z@R}
%    \end{macrocode}
%    \end{macro}
%    \begin{macro}{\zref@newlabel}
%    For the implementation of \cs{zref@newlabel} we call
%    the same internal macro \cs{@newl@bel} that is used
%    in \cs{newlabel}. Thus we have for free:
%    \begin{itemize}
%    \item \cs{Z@R@}\metacs{labelname} is defined.
%    \item \hologo{LaTeX}'s check for multiple references.
%    \item \hologo{LaTeX}'s check for changed references.
%    \end{itemize}
%    \begin{macrocode}
\ZREF@Robust\edef\zref@newlabel{%
  \noexpand\@newl@bel{\ZREF@RefPrefix}%
}
%    \end{macrocode}
%    \end{macro}
%
% \subsubsection{Property lists}
%
%    \begin{macro}{\zref@newlist}
%    Property lists are stored as list of property names
%    enclosed in curly braces. \cs{zref@newlist}
%    creates a new list as empty list. Assignments to
%    property lists are global.
%    \begin{macrocode}
\ZREF@Robust\def\zref@newlist#1{%
  \zref@iflistundefined{#1}{%
    \@ifdefinable{Z@L@#1}{%
     \global\expandafter\let\csname Z@L@#1\endcsname\ltx@empty
     \PackageInfo\ZREF@name{New property list: #1}%
    }%
  }{%
    \PackageError\ZREF@name{%
      Property list `#1' already exists%
    }\@ehc
  }%
}
%    \end{macrocode}
%    \end{macro}
%
%    \begin{macro}{\zref@iflistundefined}
%    \cs{zref@iflistundefined} checks the existence of the
%    property list \verb|#1|. If the property list is present,
%    then \verb|#2| is executed and \verb|#3| otherwise.
%    \begin{macrocode}
\def\zref@iflistundefined#1{%
  \ltx@ifundefined{Z@L@#1}%
}
%    \end{macrocode}
%    \end{macro}
%    \begin{macro}{\zref@listexists}
%    \cs{zref@listexists} only executes \verb|#2| if the
%    property list \verb|#1| exists and raises an error
%    message otherwise.
%    \begin{macrocode}
\ZREF@Robust\def\zref@listexists#1{%
  \zref@iflistundefined{#1}{%
    \PackageError\ZREF@name{%
      Property list `#1' does not exist%
    }\@ehc
  }%
}
%    \end{macrocode}
%    \end{macro}
%
%    \begin{macro}{\zref@iflistcontainsprop}
%    \cs{zref@iflistcontainsprop} checks, whether a property \verb|#2| is
%    already present in a property list \verb|#1|.
%    \begin{macrocode}
\ZREF@Robust\def\zref@iflistcontainsprop#1#2{%
  \zref@iflistundefined{#1}{%
    \ltx@secondoftwo
  }{%
    \begingroup\expandafter\endgroup
    \expandafter\in@
    \csname#2\expandafter\expandafter\expandafter\endcsname
    \expandafter\expandafter\expandafter{\csname Z@L@#1\endcsname}%
    \csname ltx@\ifin@ first\else second\fi oftwo\endcsname
  }%
}
%    \end{macrocode}
%    \end{macro}
%
%    \begin{macro}{\zref@listforloop}
%    \begin{macrocode}
\def\zref@listforloop#1#2{%
  \zref@listexists{#1}{%
    \expandafter\expandafter\expandafter\@tfor
    \expandafter\expandafter\expandafter\zref@prop
    \expandafter\expandafter\expandafter:%
    \expandafter\expandafter\expandafter=%
    \csname Z@L@#1\endcsname
    \do{%
      \begingroup
        \escapechar=-1 %
        \edef\x{\endgroup
          \def\noexpand\zref@prop{%
            \expandafter\string\zref@prop
          }%
        }%
      \x
      #2\zref@prop
    }%
  }%
}
%    \end{macrocode}
%    \end{macro}
%
%    \begin{macro}{\zref@addprops}
%    \cs{zref@addprop} adds the properties \verb|#2| to the property
%    list \verb|#1|, if the property is not already in the list.
%    Otherwise a warning is given.
%    \begin{macrocode}
\ZREF@Robust\def\zref@addprops#1#2{%
  \zref@listexists{#1}{%
    \comma@parse{#2}{%
      \zref@propexists\comma@entry{%
        \zref@iflistcontainsprop{#1}\comma@entry{%
          \PackageWarning\ZREF@name{%
            Property `\comma@entry' is already in list `#1'%
          }%
        }{%
          \begingroup\expandafter\endgroup
          \expandafter\g@addto@macro
          \csname Z@L@#1\expandafter\endcsname
          \expandafter{\csname\comma@entry\endcsname}%
        }%
      }%
      \ltx@gobble
    }%
  }%
}
%    \end{macrocode}
%    \end{macro}
%    \begin{macro}{\zref@addprop}
%    \cs{zref@addprop} adds the property \verb|#2| to the property
%    list \verb|#1|, if the property is not already in the list.
%    Otherwise a warning is given.
%    \begin{macrocode}
\ZREF@Robust\def\zref@addprop#1#2{%
  \zref@listexists{#1}{%
    \zref@propexists{#2}{%
      \zref@iflistcontainsprop{#1}{#2}{%
        \PackageWarning\ZREF@name{%
          Property `#2' is already in list `#1'%
        }%
      }{%
        \begingroup\expandafter\endgroup
        \expandafter\g@addto@macro
        \csname Z@L@#1\expandafter\endcsname
        \expandafter{\csname#2\endcsname}%
      }%
    }%
  }%
}
%    \end{macrocode}
%    \end{macro}
%
%    \begin{macro}{\zref@localaddprops}
%    \begin{macrocode}
\ZREF@Robust\def\zref@localaddprops#1#2{%
  \zref@listexists{#1}{%
    \comma@parse{#2}{%
      \zref@propexists\comma@entry{%
        \zref@iflistcontainsprop{#1}\comma@entry{%
          \PackageWarning\ZREF@name{%
            Property `\comma@entry' is already in list `#1'%
          }%
        }{%
          \begingroup\expandafter\endgroup
          \expandafter\ltx@LocalAppendToMacro
          \csname Z@L@#1\expandafter\endcsname
          \expandafter{\csname\comma@entry\endcsname}%
        }%
      }%
      \ltx@gobble
    }%
  }%
}
%    \end{macrocode}
%    \end{macro}
%    \begin{macro}{\zref@localaddprop}
%    \begin{macrocode}
\ZREF@Robust\def\zref@localaddprop#1#2{%
  \zref@listexists{#1}{%
    \zref@propexists{#2}{%
      \zref@iflistcontainsprop{#1}{#2}{%
        \PackageWarning\ZREF@name{%
          Property `#2' is already in list `#1'%
        }%
      }{%
        \begingroup\expandafter\endgroup
        \expandafter\ltx@LocalAppendToMacro
        \csname Z@L@#1\expandafter\endcsname
        \expandafter{\csname#2\endcsname}%
      }%
    }%
  }%
}
%    \end{macrocode}
%    \end{macro}
%
%    \begin{macrocode}
\ltx@IfUndefined{pdf@strcmp}{%
%    \end{macrocode}
%    \begin{macro}{\zref@delprop}
%    \begin{macrocode}
  \ZREF@Robust\def\zref@delprop{%
    \ZREF@delprop\gdef
  }%
%    \end{macrocode}
%    \end{macro}
%    \begin{macro}{\zref@localdelprop}
%    \begin{macrocode}
  \ZREF@Robust\def\zref@localdelprop{%
    \ZREF@delprop\def
  }%
%    \end{macrocode}
%    \end{macro}
%    \begin{macro}{\ZREF@delprop}
%    \begin{macrocode}
  \def\ZREF@delprop#1#2#3{%
    \zref@listexists{#2}{%
      \begingroup
        \escapechar=-1 %
        \def\ZREF@param{#3}%
        \@onelevel@sanitize\ZREF@param
        \toks@{}%
        \expandafter\expandafter\expandafter\ZREF@@delprop
        \csname Z@L@#2\endcsname!%
      \expandafter\endgroup
      \expandafter#1\csname Z@L@#2\expandafter\endcsname
      \expandafter{%
        \the\toks@
      }%
    }%
  }%
%    \end{macrocode}
%    \end{macro}
%    \begin{macro}{\ZREF@@delprop}
%    \begin{macrocode}
  \def\ZREF@@delprop#1{%
    \expandafter\ZREF@@@delprop\expandafter{\string#1}#1%
  }%
%    \end{macrocode}
%    \end{macro}
%    \begin{macro}{\ZREF@@@delprop}
%    \begin{macrocode}
  \def\ZREF@@@delprop#1#2{%
    \ifx#2!%
    \else
      \def\ZREF@temp{#1}%
      \@onelevel@sanitize\ZREF@temp
      \ifx\ZREF@param\ZREF@temp
      \else
        \toks@\expandafter{%
          \the\expandafter\toks@\csname#1\endcsname
        }%
      \fi
      \expandafter\ZREF@@delprop
    \fi
  }%
%    \end{macrocode}
%    \end{macro}
%    \begin{macrocode}
}{%
%    \end{macrocode}
%    \begin{macro}{\zref@delprop}
%    \begin{macrocode}
  \ZREF@Robust\def\zref@delprop{%
    \ZREF@delprop\xdef
  }%
%    \end{macrocode}
%    \end{macro}
%    \begin{macro}{\zref@localdelprop}
%    \begin{macrocode}
  \ZREF@Robust\def\zref@localdelprop{%
    \ZREF@delprop\edef
  }%
%    \end{macrocode}
%    \end{macro}
%    \begin{macro}{\ZREF@delprop}
%    \begin{macrocode}
  \def\ZREF@delprop#1#2#3{%
    \zref@listexists{#2}{%
      \def\ZREF@param{#3}%
      \edef\ZREF@SavedEscapechar{\the\escapechar}%
      \escapechar=-1 %
      \expandafter#1\csname Z@L@#2%
      \expandafter\expandafter\expandafter\endcsname{%
        \expandafter\expandafter\expandafter\ZREF@@delprop
        \csname Z@L@#2\endcsname!%
      }%
      \escapechar=\ZREF@SavedEscapechar\relax
    }%
  }%
%    \end{macrocode}
%    \end{macro}
%    \begin{macro}{\ZREF@@delprop}
%    Caution: |#1| might be an \cs{if} or similar token.
%    \begin{macrocode}
  \def\ZREF@@delprop#1{%
    \expandafter\ZREF@@@delprop\expandafter{\string#1}#1%
  }%
%    \end{macrocode}
%    \end{macro}
%    \begin{macro}{\ZREF@@@delprop}
%    \begin{macrocode}
  \def\ZREF@@@delprop#1#2{%
    \ifx#2!%
    \else
      \ifnum\pdf@strcmp{#1}{\ZREF@param}=\ltx@zero
      \else
        \expandafter\noexpand\csname#1\endcsname
      \fi
      \expandafter\ZREF@@delprop
    \fi
  }%
%    \end{macrocode}
%    \end{macro}
%    \begin{macrocode}
}
%    \end{macrocode}
%
% \subsubsection{Properties}
%
%    \begin{macro}{\zref@ifpropundefined}
%    \cs{zref@ifpropundefined} checks the existence of the
%    property \verb|#1|. If the property is present,
%    then \verb|#2| is executed and \verb|#3| otherwise.
%    \begin{macrocode}
\def\zref@ifpropundefined#1{%
  \ltx@ifundefined{Z@E@#1}%
}
%    \end{macrocode}
%    \end{macro}
%    \begin{macro}{\zref@propexists}
%    Some macros rely on the existence of a property.
%    \cs{zref@propexists} only executes \verb|#2| if the
%    property \verb|#1| exists and raises an error
%    message otherwise.
%    \begin{macrocode}
\ZREF@Robust\def\zref@propexists#1{%
  \zref@ifpropundefined{#1}{%
    \PackageError\ZREF@name{%
      Property `#1' does not exist%
    }\@ehc
  }%
}
%    \end{macrocode}
%    \end{macro}
%
%    \begin{macro}{\zref@newprop}
%    A new property is declared by \cs{zref@newprop}, the
%    property name \meta{propname} is given in \verb|#1|. The
%    property is created and configured. If the star form
%    is given, then the expansion of the property value
%    is delayed to page shipout time, when the reference
%    is written to the \xfile{.aux} file.
%    \begin{description}
%    \item[\cs{Z@D@}\metacs{propname}:]
%      Stores the default value for this property.
%    \item[\cs{Z@E@}\metacs{propname}:]
%      Extract function.
%    \item[\cs{Z@X@}\metacs{propname}:]
%      Information whether the expansion of the property value
%      is delayed to shipout time.
%    \item[\cs{Z@C@}\metacs{propname}:]
%      Current value of the property.
%    \end{description}
%    \begin{macrocode}
\ZREF@Robust\def\zref@newprop{%
  \@ifstar{%
    \let\ZREF@X\noexpand
    \ZREF@newprop
  }{%
    \let\ZREF@X\ltx@empty
    \ZREF@newprop
  }%
}
%    \end{macrocode}
%    \end{macro}
%    \begin{macro}{\ZREF@newprop}
%    \begin{macrocode}
\def\ZREF@newprop#1{%
  \edef\ZREF@P{#1}%
  \@onelevel@sanitize\ZREF@P
  \begingroup
  \ifx\ZREF@P\ZREF@par
    \@PackageError\ZREF@name{%
      Invalid property name `\ZREF@P'%
    }{%
      The property name `par' is not allowed %
      because of internal reasons.%
      \MessageBreak
      \@ehc
    }%
    \def\ZREF@@newprop[##1]##2{\endgroup}%
  \else
    \zref@ifpropundefined\ZREF@P{%
      \endgroup
      \PackageInfo\ZREF@name{%
        New property: \ZREF@P
      }%
    }{%
      \@PackageError\ZREF@name{%
        Property `\ZREF@P' already exists%
      }\@ehc
      \def\ZREF@@newprop[##1]##2{\endgroup}%
    }%
  \fi
  \@ifnextchar[\ZREF@@newprop{\ZREF@@newprop[\zref@default]}%
}
%    \end{macrocode}
%    \end{macro}
%    \begin{macro}{\ZREF@par}
%    \begin{macrocode}
\def\ZREF@par{par}
\@onelevel@sanitize\ZREF@par
%    \end{macrocode}
%    \end{macro}
%    \begin{macro}{\ZREF@@newprop}
%    \begin{macrocode}
\def\ZREF@@newprop[#1]{%
  \global\@namedef{Z@D@\ZREF@P}{#1}%
  \global\expandafter\let\csname Z@X@\ZREF@P\endcsname\ZREF@X
  \begingroup\expandafter\endgroup
  \expandafter\ZREF@@@newprop\csname\ZREF@P\endcsname
  \expandafter\gdef\csname Z@C@\ZREF@P\endcsname{}%
  \zref@setcurrent\ZREF@P
}
\def\ZREF@@@newprop#1{%
  \expandafter
  \gdef\csname Z@E@\ZREF@P\endcsname##1#1##2##3\ZREF@nil{##2}%
}
%    \end{macrocode}
%    \end{macro}
%
%    \begin{macro}{\zref@showprop}
%    \begin{macrocode}
\ZREF@Robust\def\zref@showprop#1{%
  \zref@ifpropundefined{#1}{%
    \@PackageInfoNoLine{\ZREF@name}{%
      Show property `#1': <undefined>%
    }%
  }{%
    \begingroup
      \toks@\expandafter\expandafter\expandafter{%
        \csname Z@C@#1\endcsname
      }%
      \edef\ZREF@value{\the\toks@}%
      \ltx@onelevel@sanitize\ZREF@value
      \toks@\expandafter\expandafter\expandafter{%
        \csname Z@D@#1\endcsname
      }%
      \edef\ZREF@default{\the\toks@}%
      \ltx@onelevel@sanitize\ZREF@default
      \@PackageInfoNoLine{\ZREF@name}{%
        Show property `#1':\MessageBreak
        \expandafter\ifx\csname Z@X@#1\endcsname\ltx@empty
          Immediate %
        \else
          Delayed %
        \fi
        value: [\ZREF@value]\MessageBreak
        Default: [\ZREF@default]%
      }%
    \endgroup
  }%
}
%    \end{macrocode}
%    \end{macro}
%
%    \begin{macro}{\zref@setcurrent}
%    \cs{zref@setcurrent} sets the current value for a property.
%    \begin{macrocode}
\ZREF@Robust\def\zref@setcurrent#1#2{%
  \zref@propexists{#1}{%
    \expandafter\def\csname Z@C@#1\endcsname{#2}%
  }%
}
%    \end{macrocode}
%    \end{macro}
%
%    \begin{macro}{\ZREF@getcurrent}
%    \cs{zref@getcurrent} gets the current value for a property.
%    \begin{macrocode}
\def\ZREF@getcurrent#1{%
  \romannumeral0%
  \ltx@ifundefined{Z@C@#1}{%
    \ltx@space
  }{%
    \expandafter\expandafter\expandafter\ltx@space
    \csname Z@C@#1\endcsname
  }%
}
%    \end{macrocode}
%    \end{macro}
%    \begin{macro}{\ZREF@u@getcurrent}
%    \begin{macrocode}
\def\ZREF@wu@getcurrent#1{%
  \etex@unexpanded\expandafter\expandafter\expandafter{%
    \ZREF@getcurrent{#1}%
  }%
}
%    \end{macrocode}
%    \end{macro}
%    \begin{macro}{\zref@getcurrent}
%    \begin{macrocode}
\let\zref@getcurrent\ZREF@getcurrent
%    \end{macrocode}
%    \end{macro}
%
% \subsubsection{Reference generation}
%
%    \begin{macro}{\zref@label}
%    Label macro that uses the main property list.
%    \begin{macrocode}
\ZREF@Robust\def\zref@label#1{%
  \zref@labelbylist{#1}\ZREF@mainlist
}
%    \end{macrocode}
%    \end{macro}
%    \begin{macro}{\zref@labelbylist}
%    Label macro that stores the properties, specified in the
%    property list \verb|#2|.
%    \begin{macrocode}
\ZREF@Robust\def\zref@labelbylist#1#2{%
  \@bsphack
    \zref@listexists{#2}{%
      \expandafter\expandafter\expandafter\ZREF@label
      \expandafter\expandafter\expandafter{%
        \csname Z@L@#2\endcsname
      }{#1}%
    }%
  \@esphack
}
%    \end{macrocode}
%    \end{macro}
%    \begin{macro}{\zref@labelbyprops}
%    The properties are directly specified in a comma separated list.
%    \begin{macrocode}
\ZREF@Robust\def\zref@labelbyprops#1#2{%
  \@bsphack
    \begingroup
      \toks@{}%
      \comma@parse{#2}{%
        \zref@ifpropundefined\comma@entry{%
          \PackageWarning\ZREF@name{%
            Property `\comma@entry' is not known%
          }%
        }{%
          \toks@\expandafter{%
            \the\expandafter\toks@\csname\comma@entry\endcsname
          }%
        }%
        \ltx@gobble
      }%
    \expandafter\endgroup
    \expandafter\ZREF@label\expandafter{\the\toks@}{#1}%
  \@esphack
}
%    \end{macrocode}
%    \end{macro}
%    \begin{macro}{\zref@labelbykv}
%    \begin{macrocode}
\ZREF@Robust\def\zref@labelbykv#1#2{%
  \@bsphack
    \begingroup
      \let\Z@L@ZREF@temp\ltx@empty
      \kvsetkeys{ZREF@LABEL}{#1}%
      \ifZREF@immediate
        \expandafter\zref@wrapper@immediate\expandafter{%
          \expandafter\ZREF@label\expandafter{\Z@L@ZREF@temp}{#2}%
        }%
      \else
        \expandafter\ZREF@label\expandafter{\Z@L@ZREF@temp}{#2}%
      \fi
    \endgroup
  \@esphack
}
%    \end{macrocode}
%    \end{macro}
%    \begin{macrocode}
\kv@define@key{ZREF@LABEL}{prop}{%
  \edef\ZREF@param{#1}%
  \zref@propexists\ZREF@param{%
    \zref@iflistcontainsprop{ZREF@temp}\ZREF@param{}{%
      \begingroup\expandafter\endgroup
      \expandafter\ltx@LocalAppendToMacro
      \expandafter\Z@L@ZREF@temp
      \expandafter{\csname\ZREF@param\endcsname}%
    }%
  }%
}
\kv@define@key{ZREF@LABEL}{list}{%
  \zref@listforloop{#1}{%
    \zref@iflistcontainsprop{ZREF@temp}\zref@prop{}{%
      \begingroup\expandafter\endgroup
      \expandafter\ltx@LocalAppendToMacro
      \expandafter\Z@L@ZREF@temp
      \expandafter{\csname\zref@prop\endcsname}%
    }%
    \ltx@gobble
  }%
}
\kv@define@key{ZREF@LABEL}{delprop}{%
  \zref@propexists{#1}{%
    \zref@localdelprop{ZREF@temp}{#1}%
  }%
}
\kv@define@key{ZREF@LABEL}{immediate}[true]{%
  \edef\ZREF@param{#1}%
  \ifx\ZREF@param\ZREF@true
    \ZREF@immediatetrue
  \else
    \ifx\ZREF@param\ZREF@false
      \ZREF@immediatefalse
    \else
      \PackageWarning\ZREF@name{%
        Option `immediate' expects `true' or `false'.\MessageBreak
        Ignoring invalid value `\ZREF@param'%
      }%
    \fi
  \fi
}
%    \end{macrocode}
%    \begin{macro}{\ZREF@false}
%    \begin{macrocode}
\def\ZREF@false{false}
%    \end{macrocode}
%    \end{macro}
%    \begin{macro}{\ZREF@true}
%    \begin{macrocode}
\def\ZREF@true{true}
%    \end{macrocode}
%    \end{macro}
%
%    \begin{macrocode}
\kv@define@key{ZREF@LABEL}{values}[]{%
  \kv@parse{#1}{%
    \ifx\kv@value\relax
      \@PackageWarning\ZREF@name{%
        Missing value for property `\kv@key'%
      }%
      \expandafter\ltx@gobbletwo
    \else
      \expandafter\zref@setcurrent
    \fi
  }%
}
%    \end{macrocode}
%
%    \begin{macro}{\ifZREF@immediate}
%    The switch \cs{ifZREF@immediate} tells us, whether the
%    label should be written immediately or at page shipout time.
%    \cs{ZREF@label} need to be notified about this, because
%    it must disable the deferred execution of property values,
%    if the label is written immediately.
%    \begin{macrocode}
\newif\ifZREF@immediate
%    \end{macrocode}
%    \end{macro}
%    \begin{macro}{\zref@wrapper@immediate}
%    The argument of \cs{zref@wrapper@immediate} is executed inside
%    a group where \cs{write} is redefined by adding
%    \cs{immediate} before its execution. Also \cs{ZREF@label}
%    is notified via the switch \cs{ifZREF@immediate}.
%    \begin{macrocode}
\ZREF@Robust{\long\def}\zref@wrapper@immediate#1{%
  \begingroup
    \ZREF@immediatetrue
    \let\ZREF@org@write\write
    \def\write{\immediate\ZREF@org@write}%
    #1%
  \endgroup
}
%    \end{macrocode}
%    \end{macro}
%
%    \begin{macro}{\ZREF@label}
%    \cs{ZREF@label} writes the data in the \xfile{.aux} file.
%    \verb|#1| contains the list of valid properties,
%    \verb|#2| the name of the reference. In case of immediate
%    writing, the deferred execution of property values is
%    disabled. Also \thepage is made expandable in this case.
%    \begin{macrocode}
\def\ZREF@label#1#2{%
  \if@filesw
    \begingroup
      \ifZREF@immediate
        \let\ZREF@org@thepage\thepage
      \fi
      \protected@write\@auxout{%
        \ifZREF@immediate
          \let\thepage\ZREF@org@thepage
        \fi
        \let\ZREF@temp\ltx@empty
        \@tfor\ZREF@P:=#1\do{%
          \begingroup
            \escapechar=-1 %
            \edef\x{\endgroup
              \def\noexpand\ZREF@P{%
                \expandafter\string\ZREF@P
              }%
            }%
          \x
          \expandafter\ifx
              \csname
                \ifZREF@immediate
                  relax%
                \else
                  Z@X@\ZREF@P%
                \fi
              \endcsname
              \noexpand
            \expandafter\let\csname Z@C@\ZREF@P\endcsname\relax
          \fi
          \toks@\expandafter{\ZREF@temp}%
          \edef\ZREF@temp{%
            \the\toks@
            \ltx@backslashchar\ZREF@P{%
              \expandafter\noexpand\csname Z@C@\ZREF@P\endcsname
            }%
          }%
        }%
      }{%
        \string\zref@newlabel{#2}{\ZREF@temp}%
      }%
    \endgroup
  \fi
}
\def\ZREF@addtoks#1{%
  \toks@\expandafter\expandafter\expandafter{%
    \expandafter\the\expandafter\toks@#1%
  }%
}
%    \end{macrocode}
%    \end{macro}
%
% \subsubsection{Reference querying and extracting}
%
%    Design goal for the extracting macros is that the extraction
%    process is full expandable. Thus these macros can be used
%    in expandable contexts. But there are problems that cannot
%    be solved by full expandable macros:
%    \begin{itemize}
%    \item In standard \hologo{LaTeX} undefined references sets a flag
%      and generate a warning. Both actions are not expandable.
%    \item Babel's support for its shorthand uses commands that
%      use non-expandable assignments. However currently there
%      is hope, that primitives are added to \pdfTeX\ that
%      allows the detection of contexts. Then the shorthand
%      can detect, if they are executed inside \cs{csname}
%      and protect themselves automatically.
%    \end{itemize}
%    \begin{macro}{\zref@ifrefundefined}
%    If a reference \verb|#1| is undefined, then macro
%    \cs{zref@ifrefundefined} calls \verb|#2| and \verb|#3| otherwise.
%    \begin{macrocode}
\def\zref@ifrefundefined#1{%
  \ltx@ifundefined{Z@R@#1}%
}
%    \end{macrocode}
%    \end{macro}
%
%    \begin{macro}{\zifrefundefined}
%    If a reference \verb|#1| is undefined, then macro
%    \cs{zref@ifrefundefined} calls \verb|#2| and \verb|#3| otherwise.
%    Also the reference is marked used.
%    \begin{macrocode}
\ZREF@IfDefinable\zifrefundefined\def{%
  #1{%
    \zref@wrapper@babel\ZREF@ifrefundefined{#1}%
  }%
}
%    \end{macrocode}
%    \end{macro}
%    \begin{macro}{\ZREF@ifrefundefined}
%    \begin{macrocode}
\def\ZREF@ifrefundefined#1{%
  \zref@refused{#1}%
  \zref@ifrefundefined{#1}%
}
%    \end{macrocode}
%    \end{macro}
%
%    \begin{macro}{\zref@refused}
%    The problem with undefined references is addressed by
%    the macro \cs{zref@refused}. This can be used outside
%    the expandable context. In case of an undefined reference
%    the flag is set to notify \hologo{LaTeX} and a warning is given.
%    \begin{macrocode}
\ZREF@Robust\def\zref@refused#1{%
  \zref@wrapper@babel\ZREF@refused{#1}%
}
%    \end{macrocode}
%    \end{macro}
%    \begin{macro}{\ZREF@refused}
%    \begin{macrocode}
\def\ZREF@refused#1{%
  \zref@ifrefundefined{#1}{%
    \protect\G@refundefinedtrue
    \@latex@warning{%
      Reference `#1' on page \thepage \space undefined%
    }%
  }{}%
}
%    \end{macrocode}
%    \end{macro}
%
%    \begin{macro}{\zref@ifrefcontainsprop}
%    \cs{zref@ifrefcontainsprop} looks, if the reference \verb|#1|
%    has the property \verb|#2| and calls then \verb|#3| and
%    \verb|#4| otherwise.
%    \begin{macrocode}
\def\zref@ifrefcontainsprop#1#2{%
  \zref@ifrefundefined{#1}{%
    \ltx@secondoftwo
  }{%
    \expandafter\ZREF@ifrefcontainsprop
    \csname Z@E@#2\expandafter\endcsname
    \csname#2\expandafter\expandafter\expandafter\endcsname
    \expandafter\expandafter\expandafter{%
      \csname Z@R@#1\endcsname
    }%
  }%
}
\def\ZREF@ifrefcontainsprop#1#2#3{%
  \expandafter\ifx\expandafter\ZREF@novalue
  #1#3#2\ZREF@novalue\ZREF@nil\ltx@empty
    \expandafter\ltx@secondoftwo
  \else
    \expandafter\ltx@firstoftwo
  \fi
}
\def\ZREF@novalue{\ZREF@NOVALUE}
%    \end{macrocode}
%    \end{macro}
%
%    \begin{macro}{\zref@extract}
%    \cs{zref@extract} is an abbreviation for the case
%    that the default of the property is used as default
%    value.
%    \begin{macrocode}
\def\ZREF@extract#1#2{%
  \romannumeral0%
  \ltx@ifundefined{Z@D@#2}{%
    \expandafter\ltx@space\zref@default
  }{%
    \expandafter\expandafter\expandafter\ZREF@@extract
    \expandafter\expandafter\expandafter{%
      \csname Z@D@#2\endcsname
    }{#1}{#2}%
  }%
}
%    \end{macrocode}
%    \end{macro}
%    \begin{macro}{\ZREF@@extract}
%    \begin{macrocode}
\def\ZREF@@extract#1#2#3{%
  \expandafter\expandafter\expandafter\ltx@space
  \zref@extractdefault{#2}{#3}{#1}%
}
%    \end{macrocode}
%    \end{macro}
%    \begin{macro}{\ZREF@wu@extract}
%    \begin{macrocode}
\def\ZREF@wu@extract#1#2{%
  \etex@unexpanded\expandafter\expandafter\expandafter{%
    \ZREF@extract{#1}{#2}%
  }%
}
%    \end{macrocode}
%    \end{macro}
%    \begin{macro}{\zref@extract}
%    \begin{macrocode}
\let\zref@extract\ZREF@extract
%    \end{macrocode}
%    \end{macro}
%
%    \begin{macro}{\ZREF@extractdefault}
%    The basic extracting macro is \cs{zref@extractdefault} with
%    the reference name in \verb|#1|, the property in \verb|#2| and
%    the default value in \verb|#3| in case for problems.
%    \begin{macrocode}
\def\ZREF@extractdefault#1#2#3{%
  \romannumeral0%
  \zref@ifrefundefined{#1}\ltx@firstoftwo{%
    \zref@ifpropundefined{#2}\ltx@firstoftwo\ltx@secondoftwo
  }{%
    \ltx@space
    #3%
  }{%
    \expandafter\expandafter\expandafter\ltx@space
    \csname Z@E@#2\expandafter\expandafter\expandafter\endcsname
    \csname Z@R@#1\expandafter\endcsname
    \csname#2\endcsname{#3}\ZREF@nil
  }%
}
%    \end{macrocode}
%    \end{macro}
%    \begin{macro}{\ZREF@wu@extractdefault}
%    \begin{macrocode}
\def\ZREF@wu@extractdefault#1#2#3{%
  \etex@unexpanded\expandafter\expandafter\expandafter{%
    \ZREF@extractdefault{#1}{#2}{#3}%
  }%
}
%    \end{macrocode}
%    \end{macro}
%    \begin{macro}{\zref@extractdefault}
%    \begin{macrocode}
\let\zref@extractdefault\ZREF@extractdefault
%    \end{macrocode}
%    \end{macro}
%
%    \begin{macro}{\zref@def@extract}
%    \begin{macrocode}
\ZREF@Robust\def\zref@def@extract#1{%
  \zref@wrapper@babel{\ZREF@def@extract{#1}}%
}
%    \end{macrocode}
%    \end{macro}
%    \begin{macro}{\ZREF@def@extract}
%    \begin{macrocode}
\def\ZREF@def@extract#1#2#3{%
  \zref@refused{#2}%
  \expandafter\expandafter\expandafter\def
  \expandafter\expandafter\expandafter#1%
  \expandafter\expandafter\expandafter{%
    \zref@extract{#2}{#3}%
  }%
}
%    \end{macrocode}
%    \end{macro}
%    \begin{macro}{\zref@def@extractdefault}
%    \begin{macrocode}
\ZREF@Robust\def\zref@def@extractdefault#1{%
  \zref@wrapper@babel{\ZREF@def@extractdefault{#1}}%
}
%    \end{macrocode}
%    \end{macro}
%    \begin{macro}{\ZREF@def@extractdefault}
%    \begin{macrocode}
\def\ZREF@def@extractdefault#1#2#3#4{%
  \zref@refused{#2}%
  \expandafter\expandafter\expandafter\def
  \expandafter\expandafter\expandafter#1%
  \expandafter\expandafter\expandafter{%
    \zref@extractdefault{#2}{#3}{#4}%
  }%
}
%    \end{macrocode}
%    \end{macro}
%
%    \begin{macro}{\ZREF@wrapper@unexpanded}
%    \begin{macrocode}
\ZREF@Robust{\long\def}\ZREF@wrapper@unexpanded#1{%
  \let\zref@wrapper@unexpanded\ltx@firstofone
  \let\zref@getcurrent\ZREF@wu@getcurrent
  \let\zref@extractdefault\ZREF@wu@extractdefault
  \let\zref@extract\ZREF@wu@extract
  #1%
  \let\zref@wrapper@unexpanded\ZREF@wrapper@unexpanded
  \let\zref@getcurrent\ZREF@getcurrent
  \let\zref@extractdefault\ZREF@extractdefault
  \let\zref@extract\ZREF@extract
}
%    \end{macrocode}
%    \end{macro}
%    \begin{macro}{\zref@wrapper@unexpanded}
%    \begin{macrocode}
\ltx@IfUndefined{etex@unexpanded}{%
  \let\zref@wrapper@unexpanded\ltx@firstofone
}{%
  \let\zref@wrapper@unexpanded\ZREF@wrapper@unexpanded
}
%    \end{macrocode}
%    \end{macro}
%
% \subsubsection{Compatibility with \xpackage{babel}}
%
%    \begin{macro}{\zref@wrapper@babel}
%    \begin{macrocode}
\ZREF@Robust{\long\def}\zref@wrapper@babel#1#2{%
  \ifcsname if@safe@actives\endcsname
    \expandafter\ltx@firstofone
  \else
    \expandafter\ltx@secondoftwo
  \fi
  {%
    \if@safe@actives
      \expandafter\ltx@secondoftwo
    \else
      \expandafter\ltx@firstoftwo
    \fi
    {%
      \begingroup
        \csname @safe@activestrue\endcsname
        \edef\x{#2}%
      \expandafter\endgroup
      \expandafter\ZREF@wrapper@babel\expandafter{\x}{#1}%
    }%
  }{%
    #1{#2}%
  }%
}
\long\def\ZREF@wrapper@babel#1#2{%
  #2{#1}%
}
%    \end{macrocode}
%    \end{macro}
%
% \subsubsection{Unique counter support}
%
%    \begin{macro}{\zref@require@unique}
%    Generate the counter |zref@unique| if the counter does not
%    already exist.
%    \begin{macrocode}
\ZREF@Robust\def\zref@require@unique{%
  \@ifundefined{c@zref@unique}{%
    \begingroup
      \let\@addtoreset\ltx@gobbletwo
      \newcounter{zref@unique}%
    \endgroup
%    \end{macrocode}
%    \end{macro}
%
%    \begin{macro}{\thezref@unique}
%    \cs{thezref@unique} is used for automatically generated unique
%    labelnames.
%    \begin{macrocode}
    \renewcommand*{\thezref@unique}{%
      zref@\number\c@zref@unique
    }%
  }{}%
}
%    \end{macrocode}
%    \end{macro}
%
% \subsubsection{Utilities}
%
%    \begin{macro}{\ZREF@number}
%    \begin{macrocode}
\ltx@IfUndefined{numexpr}{%
  \def\ZREF@number#1{\number#1}%
}{%
  \def\ZREF@number#1{\the\numexpr(#1)\relax}%
}
%    \end{macrocode}
%    \end{macro}
%
% \subsubsection{Setup}
%
%    \begin{macro}{\zref@setdefault}
%    Standard \hologo{LaTeX} prints ``??'' in bold face if a reference
%    is not known. \cs{zref@default} holds the text that is
%    printed in case of unknown references and is used, if
%    the default was not specified during the definition
%    of the new property by \cs{ref@newprop}.
%    The global default value can be set by \cs{zref@setdefault}.
%    \begin{macrocode}
\ZREF@Robust\def\zref@setdefault#1{%
  \def\zref@default{#1}%
}
%    \end{macrocode}
%    \end{macro}
%    \begin{macro}{\zref@default}
%    Now we initialize \cs{zref@default} with the same value
%    that \hologo{LaTeX} uses for its undefined references.
%    \begin{macrocode}
\zref@setdefault{%
  \nfss@text{\reset@font\bfseries ??}%
}
%    \end{macrocode}
%    \end{macro}
%
% \paragraph{Main property list.}
%
%    \begin{macro}{\zref@setmainlist}
%    The name of the default property list is stored in
%    \cs{ZREF@mainlist} and can be set by \cs{zref@setmainlist}.
%    \begin{macrocode}
\ZREF@Robust\def\zref@setmainlist#1{%
  \def\ZREF@mainlist{#1}%
}
\zref@setmainlist{main}
%    \end{macrocode}
%    \end{macro}
%    Now we create the list.
%    \begin{macrocode}
\zref@newlist\ZREF@mainlist
%    \end{macrocode}
%
% \paragraph{Main properties.}
%
%    The two properties \texttt{default} and \texttt{page}
%    are created and added to the main property list.
%    They store the data that
%    standard \hologo{LaTeX} uses in its references created by
%    \cs{label}.
%    \begin{description}
%    \item[\texttt{default}] the apperance of the latest
%      counter that is incremented by \cs{refstepcounter}
%    \item[\texttt{page}] the apperance of the page counter
%    \end{description}
%    \begin{macrocode}
\zref@newprop{default}{\@currentlabel}
\zref@newprop*{page}{\thepage}
\zref@addprops\ZREF@mainlist{default,page}
%    \end{macrocode}
%
% \paragraph{Properties}
%
%    \begin{macro}{\ZREF@NewPropAnchor}
%    \begin{macrocode}
\def\ZREF@NewPropAnchor{%
  \zref@newprop{anchor}{%
    \ltx@ifundefined{@currentHref}{}{\@currentHref}%
  }%
  \global\let\ZREF@NewPropAnchor\relax
}
%    \end{macrocode}
%    \end{macro}
%    \begin{macro}{\zref@titleref@current}
%    Later we will redefine the section and caption macros
%    to catch the current title and remember the value in
%    \cs{zref@titleref@current}.
%    \begin{macro}{\ZREF@NewPropTitle}
%    \begin{macrocode}
\def\ZREF@NewPropTitle{%
  \gdef\zref@titleref@current{}%
  \zref@newprop{title}{\zref@titleref@current}%
  \global\let\ZREF@NewPropTitle\relax
}
%    \end{macrocode}
%    \end{macro}
%    \end{macro}
%    \begin{macro}{\ZREF@NewPropTheotype}
%    \begin{macrocode}
\def\ZREF@NewPropTheotype{%
  \zref@newprop{theotype}{}%
  \global\let\ZREF@NewPropTheotype\relax
}
%    \end{macrocode}
%    \end{macro}
%    \begin{macro}{\ZREF@NewPropPageValue}
%    \begin{macrocode}
\def\ZREF@NewPropPageValue{%
  \zref@newprop*{pagevalue}[0]{\number\c@page}%
  \global\let\ZREF@NewPropPageValue\relax
}
%    \end{macrocode}
%    \end{macro}
%
% \paragraph{Mark successful loading}
%
%    \begin{macrocode}
\let\ZREF@base@ok=Y
%</base>
%    \end{macrocode}
%
%
% \subsection{Module \xmodule{user}}
%
%    \begin{macrocode}
%<*user>
\NeedsTeXFormat{LaTeX2e}
\ProvidesPackage{zref-user}%
  [2020-07-03 v2.32 Module user for zref (HO)]%
\RequirePackage{zref-base}[2019/11/29]
\ifx\ZREF@base@ok Y%
\else
  \expandafter\endinput
\fi
%    \end{macrocode}
%
%    Module \xmodule{user} enables a small user interface.
%    All macros are prefixed by \cs{z}.
%
%    First we define the pendants to the standard \hologo{LaTeX} referencing
%    commands \cs{label}, \cs{ref}, and \cs{pageref}.
%
%    \begin{macro}{\zlabel}
%    Similar to \cs{label} the macro \cs{zlabel} writes a reference
%    entry in the \xfile{.aux} file. The main property list is used.
%    Also we add the babel patch. The \cs{label} command can also be
%    used inside section titles, but it must not go into the
%    table of contents. Therefore we have to check this situation.
%    \begin{macrocode}
\newcommand*\zlabel{%
  \ifx\label\ltx@gobble
    \expandafter\ltx@gobble
  \else
    \expandafter\zref@wrapper@babel\expandafter\zref@label
  \fi
}%
%    \end{macrocode}
%    \end{macro}
%
%    \begin{macro}{\zkvlabel}
%    \begin{macrocode}
\newcommand*{\zkvlabel}[1]{%
  \ifx\label\ltx@gobble
    \expandafter\ltx@gobblethree
  \fi
  \zref@wrapper@babel{\zref@labelbykv{#1}}%
}%
%    \end{macrocode}
%    \end{macro}
%
%    \begin{macro}{\zref}
%    Macro \cs{zref} is the corresponding macro for \cs{ref}.
%    Also it provides an optional argument in order to
%    select another property.
%    \begin{macrocode}
\newcommand*{\zref}[2][default]{% robust because of optional argument
  \zref@propexists{#1}{%
    \zref@wrapper@babel\ZREF@zref{#2}{#1}%
  }%
}%
\def\ZREF@zref#1{%
  \zref@refused{#1}%
  \zref@extract{#1}%
}%
%    \end{macrocode}
%    \end{macro}
%
%    \begin{macro}{\zpageref}
%    For macro \cs{zpageref} we just call \cs{zref}
%    with property \verb|page|.
%    \begin{macrocode}
\ZREF@IfDefinable\zpageref\def{%
  {\zref[page]}%
}
%    \end{macrocode}
%    \end{macro}
%
%    \begin{macro}{\zrefused}
%    For the following expandible user macros \cs{zrefused}
%    should be used to notify \hologo{LaTeX} in case of undefined
%    references.
%    \begin{macrocode}
\ZREF@IfDefinable\zrefused\def{%
  {\zref@refused}%
}
%    \end{macrocode}
%    \end{macro}
%
%    \begin{macrocode}
%</user>
%    \end{macrocode}
%
%
% \subsection{Module \xmodule{abspage}}
%
%    \begin{macrocode}
%<*abspage>
\NeedsTeXFormat{LaTeX2e}
\ProvidesPackage{zref-abspage}%
  [2020-07-03 v2.32 Module abspage for zref (HO)]%
\RequirePackage{zref-base}[2019/11/29]
\ifx\ZREF@base@ok Y%
\else
  \expandafter\endinput
\fi
%    \end{macrocode}
%
%    Module \xmodule{abspage} adds a new property \verb|abspage|
%    to the \verb|main| property list for absolute page numbers.
%    These are recorded by the help of package \xpackage{atbegshi}.
%    \begin{macrocode}
\RequirePackage{atbegshi}[2011/10/05]%
%    \end{macrocode}
%    The counter \verb|abspage| must not go in the clear list
%    of \verb|@ckpt| that is used to set counters in
%    \xfile{.aux} files of included \hologo{TeX} files.
%    \begin{macrocode}
\begingroup
  \let\@addtoreset\ltx@gobbletwo
  \newcounter{abspage}%
\endgroup
\setcounter{abspage}{0}%
\AtBeginShipout{%
  \stepcounter{abspage}%
}%
\zref@newprop*{abspage}[0]{\the\c@abspage}%
\zref@addprop\ZREF@mainlist{abspage}%
%    \end{macrocode}
%    Note that counter \verb|abspage| shows the
%    previous page during page processing. Before
%    shipout the counter is incremented.
%    Thus the property is correctly written with
%    deferred writing. If the counter is written
%    using \cs{zref@wrapper@immediate}, then the number
%    is too small by one.
%
%    \begin{macrocode}
%</abspage>
%    \end{macrocode}
%
%
% \subsection{Module \xmodule{counter}}
%
%    \begin{macrocode}
%<*counter>
\NeedsTeXFormat{LaTeX2e}
\ProvidesPackage{zref-counter}%
  [2020-07-03 v2.32 Module counter for zref (HO)]%
\RequirePackage{zref-base}[2019/11/29]
\ifx\ZREF@base@ok Y%
\else
  \expandafter\endinput
\fi
%    \end{macrocode}
%
%    For features such as \xpackage{hyperref}'s \cs{autoref}
%    we need the name of the counter. The property
%    \verb|counter| is defined and added to the main property
%    list. Starting with \LaTeX\ 2020-10-01 the proper can use
%    \@currentcounter. In older formats
%    \cs{refstepcounter} has to be patched but this can fail in some
%    cases, see issue \#5.
%    \begin{macrocode}
\@ifl@t@r\fmtversion{2020-10-01}
 {
  \zref@newprop{counter}{\@currentcounter}
  \zref@addprop\ZREF@mainlist{counter}
 }
 {
  \zref@newprop{counter}{}
  \zref@addprop\ZREF@mainlist{counter}
  \AtBeginDocument{%
    \ZREF@patch{refstepcounter}{%
      \def\refstepcounter#1{%
        \zref@setcurrent{counter}{#1}%
        \ZREF@org@refstepcounter{#1}%
      }%
    }%
  }
 }
%    \end{macrocode}
%    \begin{macrocode}
%</counter>
%    \end{macrocode}
%
%
% \subsection{Module \xmodule{lastpage}}
%
%    \begin{macrocode}
%<*lastpage>
\NeedsTeXFormat{LaTeX2e}
\ProvidesPackage{zref-lastpage}%
  [2020-07-03 v2.32 Module lastpage for zref (HO)]%
\RequirePackage{zref-base}[2019/11/29]
\RequirePackage{zref-abspage}[2019/11/29]
\RequirePackage{atveryend}[2009/12/07]
\ifx\ZREF@base@ok Y%
\else
  \expandafter\endinput
\fi
%    \end{macrocode}
%
%    The module \xmodule{lastpage} implements the service
%    of package \xpackage{lastpage} by setting a reference
%    \verb|LastPage| at the end of the document.
%    If module \xmodule{abspage} is given, also the absolute
%    page number is available, because the properties of
%    the main property list are used.
%    \begin{macrocode}
\zref@newlist{LastPage}
\AfterLastShipout{%
  \if@filesw
    \begingroup
      \advance\c@page\m@ne
      \toks@\expandafter\expandafter\expandafter{%
        \expandafter\Z@L@main
        \Z@L@LastPage
      }%
      \expandafter\zref@wrapper@immediate\expandafter{%
        \expandafter\ZREF@label\expandafter{\the\toks@}{LastPage}%
      }%
    \endgroup
  \fi
}
%    \end{macrocode}
%    \begin{macro}{\zref@iflastpage}
%    \begin{macrocode}
\def\zref@iflastpage#1{%
  \ifnum\zref@extractdefault{#1}{abspage}{-1}=%
        \zref@extractdefault{LastPage}{abspage}{-2} %
    \expandafter\ltx@firstoftwo
  \else
    \expandafter\ltx@secondoftwo
  \fi
}
%    \end{macrocode}
%    \end{macro}
%    \begin{macro}{\ziflastpage}
%    \begin{macrocode}
\ZREF@IfDefinable\ziflastpage\def{%
  {\zref@wrapper@babel\ZREF@iflastpage}%
}
%    \end{macrocode}
%    \end{macro}
%    \begin{macro}{ZREF@iflastpage}
%    \begin{macrocode}
\def\ZREF@iflastpage#1{%
  \zref@refused{LastPage}%
  \zref@refused{#1}%
  \zref@iflastpage{#1}%
}
%    \end{macrocode}
%    \end{macro}
%    \begin{macrocode}
%</lastpage>
%    \end{macrocode}
%
%
% \subsection{Module \xmodule{thepage}}
%
%    \begin{macrocode}
%<*thepage>
\NeedsTeXFormat{LaTeX2e}
\ProvidesPackage{zref-thepage}%
  [2020-07-03 v2.32 Module thepage for zref (HO)]%
\RequirePackage{zref-base}[2019/11/29]
\ifx\ZREF@base@ok Y%
\else
  \expandafter\endinput
\fi
%    \end{macrocode}
%
%    \begin{macrocode}
\RequirePackage{atbegshi}[2011/10/05]
\RequirePackage{zref-abspage}[2019/11/29]
%    \end{macrocode}
%    \begin{macrocode}
\zref@newlist{thepage}
\zref@addprop{thepage}{page}
\ZREF@NewPropPageValue
%    \end{macrocode}
%    \begin{macro}{\zref@thepage@atbegshi@hook}
%    \begin{macrocode}
\let\zref@thepage@atbegshi@hook\ltx@empty
%    \end{macrocode}
%    \end{macro}
%    \begin{macrocode}
\zref@addprop{thepage}{pagevalue}
\AtBeginShipout{%
  \AtBeginShipoutAddToBox{%
    \zref@thepage@atbegshi@hook
    \zref@labelbylist{thepage\the\value{abspage}}{thepage}%
  }%
}
%    \end{macrocode}
%    \begin{macro}{\zref@thepage@name}
%    \begin{macrocode}
\ltx@IfUndefined{numexpr}{%
  \def\zref@thepage@name#1{thepage\number#1}%
}{%
  \def\zref@thepage@name#1{thepage\the\numexpr#1}%
}
%    \end{macrocode}
%    \end{macro}
%    \begin{macro}{\zref@thepage}
%    \begin{macrocode}
\def\zref@thepage#1{%
  \zref@extract{\zref@thepage@name{#1}}{page}%
}%
%    \end{macrocode}
%    \end{macro}
%    \begin{macro}{\zref@thepage@refused}
%    \begin{macrocode}
\ZREF@Robust\def\zref@thepage@refused#1{%
  \zref@refused{\zref@thepage@name{#1}}%
}%
%    \end{macrocode}
%    \end{macro}
%    \begin{macro}{\zthepage}
%    \begin{macrocode}
\ZREF@IfDefinable\zthepage\def{%
  #1{%
    \zref@thepage@refused{#1}%
    \zref@thepage{#1}%
  }%
}
%    \end{macrocode}
%    \end{macro}
%    \begin{macrocode}
%</thepage>
%    \end{macrocode}
%
%
% \subsection{Module \xmodule{nextpage}}
%
%    \begin{macrocode}
%<*nextpage>
\NeedsTeXFormat{LaTeX2e}
\ProvidesPackage{zref-nextpage}%
  [2020-07-03 v2.32 Module nextpage for zref (HO)]%
\RequirePackage{zref-base}[2019/11/29]
\ifx\ZREF@base@ok Y%
\else
  \expandafter\endinput
\fi
%    \end{macrocode}
%
%    \begin{macrocode}
\RequirePackage{zref-abspage}[2019/11/29]
\RequirePackage{zref-thepage}[2019/11/29]
\RequirePackage{zref-lastpage}[2019/11/29]
\RequirePackage{uniquecounter}[2009/12/18]
%    \end{macrocode}
%    \begin{macrocode}
\UniqueCounterNew{znextpage}

\newcommand*{\znextpagesetup}{%
  \afterassignment\ZREF@np@setup@i
  \def\ZREF@np@call@unknown##1%
}
\def\ZREF@np@setup@i{%
  \afterassignment\ZREF@np@setup@ii
  \def\ZREF@np@call@nonext##1%
}
\def\ZREF@np@setup@ii{%
  \def\ZREF@np@call@next##1%
}
\def\ZREF@np@call@unknown#1{#1}
\def\ZREF@np@call@nonext#1{#1}
\def\ZREF@np@call@next#1{#1}
\ZREF@IfDefinable\znextpage\def{%
  {\UniqueCounterCall{znextpage}{\ZREF@nextpage}}%
}%
\newcommand*{\znonextpagename}{}
\newcommand*{\zunknownnextpagename}{\Z@D@page}
\def\ZREF@nextpage#1{%
  \begingroup
    \def\ZREF@refname@this{zref@np#1}%
    \zref@labelbyprops\ZREF@refname@this{abspage}%
    \chardef\ZREF@call=0 % unknown
    \ZREF@ifrefundefined\ZREF@refname@this{%
    }{%
      \edef\ZREF@pagenum@this{%
        \zref@extractdefault\ZREF@refname@this{abspage}{0}%
      }%
      \edef\ZREF@refname@next{%
        \zref@thepage@name{%
          \the\numexpr\ZREF@pagenum@this+1%
        }%
      }%
      \ifnum\ZREF@pagenum@this>0 %
        \ZREF@ifrefundefined{LastPage}{%
          \zref@ifrefundefined\ZREF@refname@next{%
          }{%
            \chardef\ZREF@call=2 % next page
          }%
        }{%
          \edef\ZREF@pagenum@last{%
            \zref@extractdefault{LastPage}{abspage}{0}%
          }%
          \ifnum\ZREF@pagenum@this<\ZREF@pagenum@last\ltx@space
            \ZREF@ifrefundefined\ZREF@refname@next{%
            }{%
              \chardef\ZREF@call=2 % next page
            }%
          \else
            \ifnum\ZREF@pagenum@this=\ZREF@pagenum@this\ltx@space
              \chardef\ZREF@call=1 % no next page
            \fi
          \fi
        }%
      \fi
    }%
    \edef\x{%
      \endgroup
      \ifcase\ZREF@call
        \noexpand\ZREF@np@call@unknown{%
          \noexpand\zunknownnextpagename
        }%
      \or
        \noexpand\ZREF@np@call@nonext{%
          \noexpand\znonextpagename
        }%
      \else
        \noexpand\ZREF@np@call@next{%
          \noexpand\zref@extract{\ZREF@refname@next}{page}%
        }%
      \fi
    }%
  \x
}
%    \end{macrocode}
%    \begin{macrocode}
%</nextpage>
%    \end{macrocode}
%
%
% \subsection{Module \xmodule{totpages}}
%
%    \begin{macrocode}
%<*totpages>
\NeedsTeXFormat{LaTeX2e}
\ProvidesPackage{zref-totpages}%
  [2020-07-03 v2.32 Module totpages for zref (HO)]%
\RequirePackage{zref-base}[2019/11/29]
\ifx\ZREF@base@ok Y%
\else
  \expandafter\endinput
\fi
%    \end{macrocode}
%
%    The absolute page number of the last page is the total
%    page number.
%    \begin{macrocode}
\RequirePackage{zref-abspage}[2019/11/29]
\RequirePackage{zref-lastpage}[2019/11/29]
%    \end{macrocode}
%
%    \begin{macro}{\ztotpages}
%    Macro \cs{ztotpages} contains the number of pages. It can be
%    used inside expandable calculations. It expands to zero if
%    the reference is not yet available.
%    \begin{macrocode}
\newcommand*{\ztotpages}{%
  \zref@extractdefault{LastPage}{abspage}{0}%
}
%    \end{macrocode}
%    \end{macro}
%    Also we mark the reference |LastPage| as used:
%    \begin{macrocode}
\AtBeginDocument{%
  \zref@refused{LastPage}%
}
%    \end{macrocode}
%    \begin{macrocode}
%</totpages>
%    \end{macrocode}
%
%
% \subsection{Module \xmodule{pagelayout}}
%
%    \begin{macrocode}
%<*pagelayout>
\NeedsTeXFormat{LaTeX2e}
\ProvidesPackage{zref-pagelayout}%
  [2020-07-03 v2.32 Module pagelayout for zref (HO)]%
\RequirePackage{zref-base}[2019/11/29]
\ifx\ZREF@base@ok Y%
\else
  \expandafter\endinput
\fi
%    \end{macrocode}
%
%    \begin{macrocode}
\RequirePackage{zref-thepage}[2019/11/29]
\RequirePackage{iftex}[2019/11/07]%
\RequirePackage{atveryend}[2010/03/24]
%    \end{macrocode}
%
%
% \subsubsection{Define layout properties}
%
%    \begin{macrocode}
\def\ZREF@temp#1{%
  \begingroup
    \escapechar=-1 %
  \ltx@ifundefined{\string#1}{\endgroup}{%
    \edef\x{%
      \endgroup
      \noexpand\zref@newprop*{\string#1}%
                             [\noexpand\number\noexpand#1]% hash-ok
                             {\noexpand\number\noexpand#1}%
      \noexpand\zref@addprop{thepage}{\string#1}%
    }%
    \x
  }%
}
\ZREF@temp\mag
\ZREF@temp\paperwidth
\ZREF@temp\paperheight
\ZREF@temp\stockwidth % memoir.cls, crop.sty
\ZREF@temp\stockheight % memoir.cls, crop.sty
\ZREF@temp\mediawidth % VTeX
\ZREF@temp\mediaheight % VTeX
\ifluatex
\zref@newprop*{pdfvorigin}%
                             [\number\pdfvariable vorigin]% hash-ok
                             {\number\pdfvariable vorigin}%
\zref@addprop{thepage}{pdfvorigin}
\zref@newprop*{pdfhorigin}%
                             [\number\pdfvariable horigin]% hash-ok
                             {\number\pdfvariable horigin}%
\zref@addprop{thepage}{pdfhorigin}
\zref@newprop*{pdfpageheight}%
                             [\number\pageheight]% hash-ok
                             {\number\pageheight}%
\zref@addprop{thepage}{pdfpageheight}
\zref@newprop*{pdfpagewidth}%
                             [\number\pagewidth]% hash-ok
                             {\number\pagewidth}%
\zref@addprop{thepage}{pdfpagewidth}
\else
\ZREF@temp\pdfpagewidth
\ZREF@temp\pdfpageheight
\ZREF@temp\pdfhorigin
\ZREF@temp\pdfvorigin
\fi
\ZREF@temp\hoffset
\ZREF@temp\voffset
\ZREF@temp\topmargin
\ZREF@temp\oddsidemargin
\ZREF@temp\evensidemargin
\ZREF@temp\textwidth
\ZREF@temp\textheight
\ZREF@temp\headheight
\ZREF@temp\headsep
\ZREF@temp\footskip
\ZREF@temp\marginparwidth
\ZREF@temp\marginparsep
\ZREF@temp\columnwidth
\ZREF@temp\columnsep
\ZREF@temp\trimedge % memoir.cls
\ZREF@temp\spinemargin % memoir.cls
\ZREF@temp\foremargin % memoir.cls
\ZREF@temp\trimtop % memoir.cls
\ZREF@temp\uppermargin % memoir.cls
\ZREF@temp\headmargin % memoir.cls
%    \end{macrocode}
%
%    \begin{macrocode}
\IfFormatAtLeastTF{2020/10/01}
 {
   \zref@newprop*{outputboxwd}[0pt]{\ShipoutBoxWidth}
   \zref@newprop*{outputboxht}[0pt]{\ShipoutBoxHeight}
   \zref@newprop*{outputboxdp}[0pt]{\ShipoutBoxDepth}
 }
 {
   \zref@newprop*{outputboxwd}[0pt]{\AtBeginShipoutBoxWidth}
   \zref@newprop*{outputboxht}[0pt]{\AtBeginShipoutBoxHeight}
   \zref@newprop*{outputboxdp}[0pt]{\AtBeginShipoutBoxDepth}
 }
%    \end{macrocode}
%    \begin{macrocode}
\zref@addprops{thepage}{outputboxwd,outputboxht,outputboxdp}
%    \end{macrocode}
%
%    \begin{macro}{\ifZREF@pl@list}
%    \begin{macrocode}
\ltx@newif\ifZREF@pl@list
%    \end{macrocode}
%    \end{macro}
%    \begin{macro}{\zref@listpagelayout}
%    \begin{macrocode}
\ZREF@IfDefinable\zlistpagelayout\def{%
  {\global\ZREF@pl@listtrue}%
}
%    \end{macrocode}
%    \end{macro}
%    \begin{macro}{\ZREF@pl@AfterLastShipout}
%    \begin{macrocode}
\def\ZREF@pl@AfterLastShipout{%
  \ifZREF@pl@list
    \edef\ZREF@page@max{\the\value{abspage}}%
    \ltx@ifundefined{ZREF@org@testdef}{%
      \let\ZREF@org@testdef\@testdef
      \def\@testdef##1##2##3{%
        \ZREF@org@testdef{##1}{##2}{##3}%
        \def\ZREF@temp{##1}%
        \ifx\ZREF@temp\ZREF@RefPrefix
          \expandafter\gdef\csname##1@##2\endcsname{##3}%
        \fi
      }%
    }{}%
    \AtVeryEndDocument{\ZREF@pl@AtVeryEnd}%
  \fi
}
%    \end{macrocode}
%    \end{macro}
%    \begin{macro}{\ZREF@pl@AtVeryEnd}
%    \begin{macrocode}
\def\ZREF@pl@AtVeryEnd{%
  \begingroup
    \toks@{Page layout parameters:\MessageBreak}%
    \count@=1 %
    \ZREF@pl@ListPage
    \edef\x{\endgroup
      \noexpand\@PackageInfoNoLine{zref-pagelayout}{\the\toks@}%
    }%
  \x
}
%    \end{macrocode}
%    \end{macro}
%    \begin{macro}{\ZREF@pl@ListPage}
%    \begin{macrocode}
\def\ZREF@pl@ListPage{%
  \edef\x{%
    \toks@={%
      \the\toks@
      Page \the\count@:\noexpand\MessageBreak
      \zref@ifrefundefined{thepage\the\count@}{}{%
        \ltx@space\ltx@space mag = %
        \zref@extract{thepage\the\count@}{mag}%
        \noexpand\MessageBreak
        \ZREF@pl@ListEntry{paperwidth}%
        \ZREF@pl@ListEntry{paperheight}%
        \ZREF@pl@ListEntry{stockwidth}%
        \ZREF@pl@ListEntry{stockheight}%
        \ZREF@pl@ListEntry{mediawidth}%
        \ZREF@pl@ListEntry{mediaheight}%
        \ZREF@pl@ListEntry{pdfpagewidth}%
        \ZREF@pl@ListEntry{pdfpageheight}%
        \ZREF@pl@ListEntry{pdfhorigin}%
        \ZREF@pl@ListEntry{pdfvorigin}%
        \ZREF@pl@ListEntry{hoffset}%
        \ZREF@pl@ListEntry{voffset}%
        \ZREF@pl@ListEntry{topmargin}%
        \ZREF@pl@ListEntry{oddsidemargin}%
        \ZREF@pl@ListEntry{evensidemargin}%
        \ZREF@pl@ListEntry{textwidth}%
        \ZREF@pl@ListEntry{textheight}%
        \ZREF@pl@ListEntry{headheight}%
        \ZREF@pl@ListEntry{headsep}%
        \ZREF@pl@ListEntry{footskip}%
        \ZREF@pl@ListEntry{marginparwidth}%
        \ZREF@pl@ListEntry{marginparsep}%
        \ZREF@pl@ListEntry{columnwidth}%
        \ZREF@pl@ListEntry{columnsep}%
        \ZREF@pl@ListEntry{trimedge}%
        \ZREF@pl@ListEntry{spinemargin}%
        \ZREF@pl@ListEntry{foremargin}%
        \ZREF@pl@ListEntry{trimtop}%
        \ZREF@pl@ListEntry{uppermargin}%
        \ZREF@pl@ListEntry{headmargin}%
      }%
    }%
  }\x
  \ifnum\ZREF@page@max>\count@
    \advance\count@ by\ltx@one
  \else
    \expandafter\ltx@gobble
  \fi
  \ZREF@pl@ListPage
}
%    \end{macrocode}
%    \end{macro}
%    \begin{macro}{\ZREF@pl@ListEntry}
%    \begin{macrocode}
\def\ZREF@pl@ListEntry#1{%
  \zref@ifpropundefined{#1}{%
  }{%
    \zref@ifrefcontainsprop{thepage\the\count@}{#1}{%
      \ltx@space\ltx@space#1 = %
      \zref@extract{thepage\the\count@}{#1}sp = %
      \the\dimexpr\zref@extract{thepage\the\count@}{#1}sp\relax
      \noexpand\MessageBreak
    }{}%
  }%
}
%    \end{macrocode}
%    \end{macro}
%    \begin{macrocode}
\AfterLastShipout{%
  \ZREF@pl@AfterLastShipout
}
%    \end{macrocode}
%
%    \begin{macrocode}
%</pagelayout>
%    \end{macrocode}
%
%
% \subsection{Module \xmodule{pageattr}}
%
%    \begin{macrocode}
%<*pageattr>
\NeedsTeXFormat{LaTeX2e}
\ProvidesPackage{zref-pageattr}%
  [2020-07-03 v2.32 Module pageattr for zref (HO)]%
\RequirePackage{zref-base}[2019/11/29]
\ifx\ZREF@base@ok Y%
\else
  \expandafter\endinput
\fi
%    \end{macrocode}
%
%    \begin{macrocode}
\RequirePackage{iftex}[2019/11/07]%
%    \end{macrocode}
%
%
%    \begin{macrocode}
\let\ZREF@temp=N%
\ifluatex
\expandafter\@firstoftwo
\else
\expandafter\@secondoftwo
\fi
{%luatex
 \RequirePackage{zref-thepage}[2019/11/29]
 \RequirePackage{zref-lastpage}[2019/11/29]%
 \zref@newprop*{pdfpageattr}[]{\zref@hex{\the\pdfvariable pageattr}}%
 \zref@addprop{thepage}{pdfpageattr}%
 \zref@newprop*{pdfpagesattr}[]{\zref@hex{\the\pdfvariable pagesattr}}%
 \zref@addprop{LastPage}{pdfpagesattr}%
 \let\ZREF@temp=Y%
}
{%other
 \ltx@IfUndefined{pdfpageattr}{%
  \@PackageInfoNoLine{zref-pageattr}{%
    \string\pdfpageattr\space is not available%
  }%
  \def\zref@pdfpageattr#1{}%
  \def\zref@pdfpageattr@used#1{}%
 }{%
  \RequirePackage{zref-thepage}[2019/11/29]%
  \zref@newprop*{pdfpageattr}[]{\zref@hex{\the\pdfpageattr}}%
  \zref@addprop{thepage}{pdfpageattr}%
  \let\ZREF@temp=Y%
 }
 \ltx@IfUndefined{pdfpagesattr}{%
  \@PackageInfoNoLine{zref-pageattr}{%
    \string\pdfpagesattr\space is not available%
  }%
  \def\zref@pdfpagesattr{}%
  \def\zref@pdfpagesattr@used{}%
 }{%
  \RequirePackage{zref-lastpage}[2019/11/29]%
  \zref@newprop*{pdfpagesattr}[]{\zref@hex{\the\pdfpagesattr}}%
  \zref@addprop{LastPage}{pdfpagesattr}%
  \let\ZREF@temp=Y%
 }%
}%
%    \end{macrocode}
%    \begin{macrocode}
\ifx\ZREF@temp N%
  \expandafter\endinput
\fi
%    \end{macrocode}
%    \begin{macrocode}
\RequirePackage{zref-abspage}[2019/11/29]
\RequirePackage{atveryend}[2010/03/24]
\RequirePackage{pdftexcmds}[2010/04/01]
\let\ZREF@temp=Y%
\ltx@IfUndefined{pdf@escapehex}{\let\ZREF@temp=N}{}
\ltx@IfUndefined{pdf@unescapehex}{\let\ZREF@temp=N}{}
\ifx\ZREF@temp N%
  \let\zref@hex\ltx@firstofone
  \let\zref@unhex\ltx@firstofone
\else
  \let\zref@hex\pdf@escapehex
  \let\zref@unhex\pdf@unescapehex
\fi
%    \end{macrocode}
%
%    \begin{macro}{\ifZREF@pa@list}
%    \begin{macrocode}
\ltx@newif\ifZREF@pa@list
%    \end{macrocode}
%    \end{macro}
%    \begin{macro}{\zref@listpageattr}
%    \begin{macrocode}
\ZREF@IfDefinable\zlistpageattr\def{%
  {\ZREF@pa@listtrue}%
}
%    \end{macrocode}
%    \end{macro}
%    \begin{macro}{\ZREF@pa@AfterLastShipout}
%    \begin{macrocode}
\def\ZREF@pa@AfterLastShipout{%
  \ifZREF@pa@list
    \edef\ZREF@page@max{\the\value{abspage}}%
    \ltx@ifundefined{ZREF@org@testdef}{%
      \let\ZREF@org@testdef\@testdef
      \def\@testdef##1##2##3{%
        \ZREF@org@testdef{##1}{##2}{##3}%
        \def\ZREF@temp{##1}%
        \ifx\ZREF@temp\ZREF@RefPrefix
          \expandafter\xdef\csname##1@##2\endcsname{##3}%
        \fi
      }%
    }{}%
    \AtVeryEndDocument{\ZREF@pa@AtVeryEnd}%
  \fi
}
%    \end{macrocode}
%    \end{macro}
%    \begin{macro}{\ZREF@pa@AtVeryEnd}
%    \begin{macrocode}
\let\ZREF@temp=Y%
\ltx@IfUndefined{pdfpageattr}{}{\let\ZREF@temp=N}
\ifluatex \let\ZREF@temp=N \fi
\ifx\ZREF@temp Y
 \expandafter\@firstoftwo
\else
 \expandafter\@secondoftwo
\fi
 {%
  \def\ZREF@pa@AtVeryEnd{}%
 }
 {%
  \def\ZREF@pa@AtVeryEnd{%
    \begingroup
      \toks@{List of \ltx@backslashchar
             \ifluatex pdfvariable\else pdf\fi
             pdfpageattr:\MessageBreak}%
      \count@=1 %
      \ZREF@pa@ListPage
      \edef\x{\endgroup
        \noexpand\@PackageInfoNoLine{zref-pageattr}{%
          \the\toks@
        }%
      }%
    \x
  }%
%    \end{macrocode}
%    \end{macro}
%    \begin{macro}{\zref@pageattr}
%    \begin{macrocode}
\def\zref@pdfpageattr#1{%
  \zref@unhex{%
    \zref@extract{thepage\ZREF@number{#1}}{pdfpageattr}%
  }%
}
% compability, \zref@pageattr was defined in older versions
\let\zref@pageattr\zref@pdfpageattr
%    \end{macrocode}
%    \end{macro}
%    \begin{macro}{\zref@pageattr@used}
%    \begin{macrocode}
\ZREF@Robust\def\zref@pageattr@used#1{%
  \zref@refused{thepage\ZREF@number{#1}}%
}
%    \end{macrocode}
%    \end{macro}
%    \begin{macro}{\ZREF@pa@ListPage}
%    \begin{macrocode}
  \def\ZREF@pa@ListPage{%
    \edef\x{%
      \toks@={%
        \the\toks@
        Page \the\count@:%
        \noexpand\MessageBreak
        \zref@ifrefundefined{thepage\the\count@}{}{%
          <<\zref@pdfpageattr\count@>>%
          \noexpand\MessageBreak
        }%
      }%
    }\x
    \ifnum\ZREF@page@max>\count@
      \advance\count@ by\ltx@one
    \else
      \expandafter\ltx@gobble
    \fi
    \ZREF@pa@ListPage
  }%
}
%    \end{macrocode}
%    \end{macro}
%    \begin{macrocode}
\let\ZREF@temp=Y%
\ltx@IfUndefined{pdfpagesattr}{}{\let\ZREF@temp=N}
\ifluatex \let\ZREF@temp=N \fi
\ifx\ZREF@temp N
 \expandafter\@firstofone
\fi
 {%
%    \end{macrocode}
%    \begin{macro}{\zref@pdfpagesattr}
%    \begin{macrocode}
  \def\zref@pdfpagesattr{%
    \zref@unhex{%
      \zref@extract{LastPage}{pdfpagesattr}%
    }%
  }%
%    \end{macrocode}
%    \end{macro}
%    \begin{macro}{\zref@pdfpagesattr@used}
%    \begin{macrocode}
  \ZREF@Robust\def\zref@pdfpagesattr@used{%
    \zref@refused{LastPage}%
  }%
%    \end{macrocode}
%    \end{macro}
%    \begin{macrocode}
  \ltx@LocalAppendToMacro\ZREF@pa@AtVeryEnd{%
    \@PackageInfoNoLine{zref-pageattr}{%
      \ltx@backslashchar
      \ifluatex pdfvariable\else pdf\fi
      pagesattr:\MessageBreak
      <<\zref@pdfpagesattr>>%
      \MessageBreak
    }%
  }%
}
\AfterLastShipout{%
  \ZREF@pa@AfterLastShipout
}
%    \end{macrocode}
%    \begin{macrocode}
%</pageattr>
%    \end{macrocode}
%
%
% \subsection{Module \xmodule{marks}}
%
%    \begin{macrocode}
%<*marks>
\NeedsTeXFormat{LaTeX2e}
\ProvidesPackage{zref-marks}%
  [2020-07-03 v2.32 Module marks for zref (HO)]%
\RequirePackage{zref-base}[2019/11/29]
\ifx\ZREF@base@ok Y%
\else
  \expandafter\endinput
\fi
%    \end{macrocode}
%
%    \begin{macrocode}
\newcommand*{\zref@marks@register}[3][]{%
  \edef\ZREF@TempName{#1}%
  \edef\ZREF@TempNum{\ZREF@number{#2}}%
  \ifnum\ZREF@TempNum<\ltx@zero %
    \PackageError\ZREF@name{%
      \string\zref@marks@register\ltx@space is called with invalid%
      \MessageBreak
      marks register number (\ZREF@TempNum)%
    }{%
      Use `0' or the command, defined by \string\newmarks.\MessageBreak
      \@ehc
    }%
  \else
    \ifx\ZREF@TempName\ltx@empty
      \edef\ZREF@TempName{mark\romannumeral\ZREF@TempNum}%
    \else
      \edef\ZREF@TempName{marks\ZREF@TempName}%
    \fi
    \ZREF@MARKS@DefineProp{top}%
    \ZREF@MARKS@DefineProp{first}%
    \ZREF@MARKS@DefineProp{bot}%
    \kv@parse{#3}{%
      \ifx\kv@value\relax
        \def\kv@value{top,first,bot}%
      \fi
      \edef\ZREF@temp{\expandafter\ltx@car\kv@key X\@nil}%
      \ifx\ZREF@temp\ZREF@STAR
        \edef\kv@key{\expandafter\ltx@cdr\kv@key\@nil}%
        \zref@newlist\kv@key
      \fi
      \expandafter\comma@parse\expandafter{\kv@value}{%
        \ifcase0\ifx\comma@entry\ZREF@NAME@top 1\else
                \ifx\comma@entry\ZREF@NAME@first 1\else
                \ifx\comma@entry\ZREF@NAME@bot 1\fi\fi\fi\ltx@space
          \PackageWarning{zref-marks}{%
            Use `top', `first' or `bot' for the list values%
            \MessageBreak
            in the third argument of \string\zref@marks@register.%
            \MessageBreak
            Ignoring unkown value `\comma@entry'%
          }%
        \else
          \zref@addprop{\kv@key}{\comma@entry\ZREF@TempName}%
        \fi
        \ltx@gobble
      }%
      \ltx@gobbletwo
    }%
  \fi
}
\def\ZREF@STAR{*}
\def\ZREF@NAME@top{top}
\def\ZREF@NAME@first{first}
\def\ZREF@NAME@bot{bot}
\def\ZREF@MARKS@DefineProp#1{%
  \zref@ifpropundefined{#1\ZREF@TempName}{%
    \ifnum\ZREF@TempNum=\ltx@zero
      \begingroup
        \edef\x{\endgroup
          \noexpand\zref@newprop*{#1\ZREF@TempName}[]{%
            \expandafter\noexpand\csname#1mark\endcsname
          }%
        }%
      \x
    \else
      \begingroup
        \edef\x{\endgroup
          \noexpand\zref@newprop*{#1\ZREF@TempName}[]{%
            \expandafter\noexpand\csname#1marks\endcsname
            \ZREF@TempNum
          }%
        }%
      \x
    \fi
  }{%
    \PackageWarning{zref-marks}{%
      \string\zref@marks@register\ltx@space does not generate the%
      \MessageBreak
      new property `#1\ZREF@TempName', because\MessageBreak
      it is already defined%
    }%
  }%
}
%    \end{macrocode}
%    \begin{macrocode}
%</marks>
%    \end{macrocode}
%
% \subsection{Module \xmodule{runs}}
%
%    This module does not use the label-reference-system.
%    The reference changes with each \hologo{LaTeX}~run and would
%    force a rerun warning always.
%    \begin{macrocode}
%<*runs>
\NeedsTeXFormat{LaTeX2e}
\ProvidesPackage{zref-runs}%
  [2020-07-03 v2.32 Module runs for zref (HO)]%
%    \end{macrocode}
%
%    \begin{macro}{\zruns}
%    \begin{macrocode}
\providecommand*{\zruns}{0}%
\AtBeginDocument{%
  \edef\zruns{\number\numexpr\zruns+1}%
  \begingroup
    \def\on@line{}%
    \PackageInfo{zref-runs}{LaTeX runs: \zruns}%
    \if@filesw
      \immediate\write\@mainaux{%
        \string\gdef\string\zruns{\zruns}%
      }%
    \fi
  \endgroup
}
%    \end{macrocode}
%    \end{macro}
%
%    \begin{macrocode}
%</runs>
%    \end{macrocode}
%
%
% \subsection{Module \xmodule{perpage}}
%
%    \begin{macrocode}
%<*perpage>
\NeedsTeXFormat{LaTeX2e}
\ProvidesPackage{zref-perpage}%
  [2020-07-03 v2.32 Module perpage for zref (HO)]%
\RequirePackage{zref-base}[2019/11/29]
\ifx\ZREF@base@ok Y%
\else
  \expandafter\endinput
\fi
%    \end{macrocode}
%
%    This module resets a counter at page boundaries.
%    Because of the asynchronous output routine
%    page counter properties cannot be asked directly,
%    references are necessary.
%
%    For detecting changed pages module \xmodule{abspage}
%    is loaded.
%    \begin{macrocode}
\RequirePackage{zref-abspage}[2019/11/29]
%    \end{macrocode}
%
%    We group the properties for the needed references
%    in the property list |perpage|. The property |pagevalue|
%    records the correct value of the page counter.
%    \begin{macrocode}
\ZREF@NewPropPageValue
\zref@newlist{perpage}
\zref@addprops{perpage}{abspage,page,pagevalue}
%    \end{macrocode}
%    The page value, known by the reference mechanism, will be
%    stored in counter |zpage|.
%    \begin{macrocode}
\newcounter{zpage}
%    \end{macrocode}
%    Counter |zref@unique| helps in generating unique reference
%    names.
%    \begin{macrocode}
\zref@require@unique
%    \end{macrocode}
%    In order to be able to reset the counter, we hook here
%    into \cs{stepcounter}. In fact two nested hooks are used
%    to allow other packages to use the first hook at the beginning
%    of \cs{stepcounter}.
%    \begin{macrocode}
\let\ZREF@org@stepcounter\stepcounter
\def\stepcounter#1{%
  \ifcsname @stepcounterhook@#1\endcsname
    \csname @stepcounterhook@#1\endcsname
  \fi
  \ZREF@org@stepcounter{#1}%
}
%    \end{macrocode}
%    \cs{@stpelt} must be adapted due to the change in latex
%    2015-01, see https://github.com/ho-tex/zref/issues/26
%    \begin{macrocode}
\let\ZREF@org@@stpelt\@stpelt
\def\@stpelt#1{%
  \ifcsname ZREF@perpage@#1\endcsname
    \begingroup
      \let\stepcounter\ZREF@org@stepcounter
      \ZREF@org@@stpelt{#1}%
    \endgroup
    \expandafter\ltx@gobbletwo
  \fi
  \ZREF@org@@stpelt{#1}%
}
%    \end{macrocode}
%
%    \begin{macro}{\zmakeperpage}
%    Makro \cs{zmakeperpage} resets a counter at each page
%    break. It uses the same syntax and semantics as
%    \cs{MakePerPage} from package \xpackage{perpage} \cite{perpage}.
%    The initial start value can be given by the optional
%    argument. Default is one that means after the first
%    \cs{stepcounter} on a new page the counter starts with
%    one.
%    \begin{macrocode}
\ZREF@IfDefinable\zmakeperpage\def{%
  {%
    \@ifnextchar[\ZREF@makeperpage@opt{\ZREF@@makeperpage[\ltx@zero]}%
  }%
}
%    \end{macrocode}
%    \end{macro}
%    We hook before the counter is incremented in \cs{stepcounter},
%    package \xpackage{perpage} afterwards. Thus a little
%    calculation is necessary.
%    \begin{macrocode}
\def\ZREF@makeperpage@opt[#1]{%
  \begingroup
    \edef\x{\endgroup
      \noexpand\ZREF@@makeperpage[\number\numexpr#1-1\relax]%
    }%
  \x
}
%    \end{macrocode}
%    \begin{macrocode}
\def\ZREF@@makeperpage[#1]#2{%
  \@ifundefined{@stepcounterhook@#2}{%
    \expandafter\gdef\csname @stepcounterhook@#2\endcsname{}%
  }{}%
  \expandafter\gdef\csname ZREF@perpage@#2\endcsname{%
    \ZREF@@perpage@step{#2}{#1}%
  }%
  \expandafter\g@addto@macro\csname @stepcounterhook@#2\endcsname{%
    \ifcsname ZREF@perpage@#2\endcsname
      \csname ZREF@perpage@#2\endcsname
    \fi
  }%
}
%    \end{macrocode}
%    \begin{macro}{\ZREF@@perpage@step}
%    The heart of this module follows.
%    \begin{macrocode}
\def\ZREF@@perpage@step#1#2{%
%    \end{macrocode}
%    First the reference is generated.
%    \begin{macrocode}
  \global\advance\c@zref@unique\ltx@one
  \begingroup
    \expandafter
    \zref@labelbylist\expandafter{\thezref@unique}{perpage}%
%    \end{macrocode}
%    The \cs{expandafter} commands are necessary, because \cs{ZREF@temp}
%    is also used inside of \cs{zref@labelbylist}.
%
%    The evaluation of the reference follows. If the reference
%    is not yet kwown, we use the page counter as approximation.
%    \begin{macrocode}
    \zref@ifrefundefined\thezref@unique{%
      \global\c@zpage=\c@page
      \global\let\thezpage\thepage
      \expandafter\xdef\csname ZREF@abspage@#1\endcsname{%
        \number\c@abspage
      }%
    }{%
%    \end{macrocode}
%    The reference is used to set \cs{thezpage} and counter |zpage|.
%    \begin{macrocode}
      \global\c@zpage=\zref@extract\thezref@unique{pagevalue}\relax
      \xdef\thezpage{\noexpand\zref@extract{\thezref@unique}{page}}%
      \expandafter\xdef\csname ZREF@abspage@#1\endcsname{%
        \zref@extractdefault\thezref@unique
            {abspage}{\number\c@abspage}%
      }%
    }%
%    \end{macrocode}
%    Page changes are detected by a changed absolute page number.
%    \begin{macrocode}
    \expandafter\ifx\csname ZREF@abspage@#1\expandafter\endcsname
                    \csname ZREF@currentabspage@#1\endcsname
    \else
      \global\csname c@#1\endcsname=#2\relax
      \global\expandafter\let
          \csname ZREF@currentabspage@#1\expandafter\endcsname
          \csname ZREF@abspage@#1\endcsname
    \fi
  \endgroup
}
%    \end{macrocode}
%    \end{macro}
%    \begin{macro}{\zunmakeperpage}
%    Macro \cs{zunmakeperpage} cancels the effect of \cs{zmakeperpage}.
%    \begin{macrocode}
\ZREF@IfDefinable\zunmakeperpage\def{%
  #1{%
    \global\expandafter
    \let\csname ZREF@perpage@#1\endcsname\@undefined
  }%
}
%    \end{macrocode}
%    \end{macro}
%
%    \begin{macrocode}
%</perpage>
%    \end{macrocode}
%
%
% \subsection{Module \xmodule{titleref}}
%
%    \begin{macrocode}
%<*titleref>
\NeedsTeXFormat{LaTeX2e}
\ProvidesPackage{zref-titleref}%
  [2020-07-03 v2.32 Module titleref for zref (HO)]%
\RequirePackage{zref-base}[2019/11/29]
\ifx\ZREF@base@ok Y%
\else
  \expandafter\endinput
\fi
\RequirePackage{gettitlestring}[2009/12/08]
%    \end{macrocode}
%
% \subsubsection{Implementation}
%    \begin{macrocode}
\RequirePackage{keyval}
%    \end{macrocode}
%
%    This module makes section and caption titles available
%    for the reference system. It uses some of the ideas
%    of package \xpackage{nameref} and \xpackage{titleref}.
%
%    Now we can add the property \verb|title| is added to
%    the main property list.
%    \begin{macrocode}
\ZREF@NewPropTitle
\zref@addprop\ZREF@mainlist{title}%
%    \end{macrocode}
%
%    The title strings go into the \xfile{.aux} file, thus
%    they need some kind of protection.
%    Package \xpackage{titleref} uses a protected expansion
%    method. The advantage is that this can be used to
%    cleanup the string and to remove \cs{label}, \cs{index}
%    and other macros unwanted for referencing.
%    But there is the risk that fragile stuff can break.
%
%    Therefore package \xpackage{nameref} does not expand
%    the string. Thus the entries can safely be written to
%    the \xfile{.aux} file. But potentially dangerous
%    macros such as \cs{label} remain in the string and
%    can cause problems when using the string in references.
%
%    \begin{macro}{\ifzref@titleref@expand}
%    The switch \cs{ifzref@titleref@expand} distinguishes
%    between the both methods. Package \xpackage{nameref}'s
%    behaviour is achieved by setting the switch to false,
%    otherwise \xpackage{titleref}'s expansion is used.
%    Default is false.
%    \begin{macrocode}
\newif\ifzref@titleref@expand
%    \end{macrocode}
%    \end{macro}
%
%    \begin{macro}{\ZREF@titleref@hook}
%    The hook \cs{ZREF@titleref@hook} allows to extend
%    the cleanup for the expansion method. Thus
%    unnecessary macros can be removed or dangerous commands
%    removed. The hook is executed before the expansion of
%    \cs{zref@titleref@current}.
%    \begin{macrocode}
\let\ZREF@titleref@hook\ltx@empty
%    \end{macrocode}
%    \end{macro}
%    \begin{macro}{\zref@titleref@cleanup}
%    The hook should not be used directly, instead we provide the
%    macro \cs{zref@titleref@cleanup} to add stuff to the hook
%    and prevents that a previous non-empty content is not discarded
%    accidently.
%    \begin{macrocode}
\ZREF@Robust\def\zref@titleref@cleanup#1{%
   \begingroup
    \toks@\expandafter{%
      \ZREF@titleref@hook
      #1%
    }%
  \expandafter\endgroup
  \expandafter\def\expandafter\ZREF@titleref@hook\expandafter{%
    \the\toks@
  }%
}%
%    \end{macrocode}
%    \end{macro}
%
%    \begin{macro}{\ifzref@titleref@stripperiod}
%    Sometimes a title contains a period at the end.
%    Package \xpackage{nameref} removes this. This behaviour
%    is controlled by the switch \cs{ifzref@titleref@stripperiod}
%    and works regardless of the setting of option \xoption{expand}.
%    Period stripping is the default.
%    \begin{macrocode}
\newif\ifzref@titleref@stripperiod
\zref@titleref@stripperiodtrue
%    \end{macrocode}
%    \end{macro}
%
%    \begin{macro}{\zref@titleref@setcurrent}
%    Macro \cs{zref@titleref@setcurrent} sets a new current title
%    stored in \cs{zref@titleref@current}. Some cleanup and
%    expansion is performed that can be controlled by the
%    previous switches.
%    \begin{macrocode}
\ZREF@Robust\def\zref@titleref@setcurrent#1{%
  \ifzref@titleref@expand
    \GetTitleStringExpand{#1}%
  \else
    \GetTitleStringNonExpand{#1}%
  \fi
  \edef\zref@titleref@current{%
    \detokenize\expandafter{\GetTitleStringResult}%
  }%
  \ifzref@titleref@stripperiod
    \edef\zref@titleref@current{%
      \expandafter\ZREF@stripperiod\zref@titleref@current
      \ltx@empty.\ltx@empty\@nil
    }%
  \fi
}%
\GetTitleStringDisableCommands{%
  \ZREF@titleref@hook
}
%    \end{macrocode}
%    \end{macro}
%    \begin{macro}{\ZREF@stripperiod}
%    If \cs{ZREF@stripperiod} is called, the argument consists of
%    space tokens and tokens with catcode 12 (other), because
%    of \hologo{eTeX}'s \cs{detokenize}.
%    \begin{macrocode}
\def\ZREF@stripperiod#1.\ltx@empty#2\@nil{#1}%
%    \end{macrocode}
%    \end{macro}
%
% \subsubsection{User interface}
%
%    \begin{macro}{\ztitlerefsetup}
%    The behaviour of module \xmodule{titleref} is controlled
%    by switches and a hook. They can be set by \cs{ztitlerefsetup}
%    with a key value interface, provided by package \xpackage{keyval}.
%    Also the current title can be given explicitly by the key
%    \verb|title|.
%    \begin{macrocode}
\define@key{ZREF@TR}{expand}[true]{%
  \csname zref@titleref@expand#1\endcsname
}%
\define@key{ZREF@TR}{stripperiod}[true]{%
  \csname zref@titleref@stripperiod#1\endcsname
}%
\define@key{ZREF@TR}{cleanup}{%
  \zref@titleref@cleanup{#1}%
}%
\define@key{ZREF@TR}{title}{%
  \def\zref@titleref@current{#1}%
}%
\ZREF@IfDefinable\ztitlerefsetup\def{%
  {\kvsetkeys{ZREF@TR}}%
}%
%    \end{macrocode}
%    \end{macro}
%
%    \begin{macro}{\ztitleref}
%    The user command \cs{ztitleref} references the title.
%    For safety \cs{label} is disabled to prevent multiply
%    defined references.
%    \begin{macrocode}
\ZREF@IfDefinable\ztitleref\def{%
  {\zref@wrapper@babel\ZREF@titleref}%
}%
\def\ZREF@titleref#1{%
  \begingroup
    \zref@refused{#1}%
    \let\label\ltx@gobble
    \zref@extract{#1}{title}%
  \endgroup
}%
%    \end{macrocode}
%    \end{macro}
%
% \subsubsection{Patches for section and caption commands}
%
%    The section and caption macros are patched to extract the
%    title data.
%
%    Captions of figures and tables.
%    \begin{macrocode}
\AtBeginDocument{%
  \ZREF@patch{@caption}{%
    \long\def\@caption#1[#2]{%
      \zref@titleref@setcurrent{#2}%
      \ZREF@org@@caption{#1}[{#2}]%
    }%
  }%
%    \end{macrocode}
%
%    Section commands without star. The title version for the
%    table of contents is used because it is usually shorter and
%    more robust.
%    \begin{macrocode}
  \ZREF@patch{@part}{%
    \def\@part[#1]{%
      \zref@titleref@setcurrent{#1}%
      \ZREF@org@@part[{#1}]%
    }%
  }%
  \ZREF@patch{@chapter}{%
    \def\@chapter[#1]{%
      \zref@titleref@setcurrent{#1}%
      \ZREF@org@@chapter[{#1}]%
    }%
  }%
  \ZREF@patch{@sect}{%
    \def\@sect#1#2#3#4#5#6[#7]{%
      \zref@titleref@setcurrent{#7}%
      \ZREF@org@@sect{#1}{#2}{#3}{#4}{#5}{#6}[{#7}]%
    }%
  }%
%    \end{macrocode}
%    The star versions of the section commands.
%    \begin{macrocode}
  \ZREF@patch{@spart}{%
    \def\@spart#1{%
      \zref@titleref@setcurrent{#1}%
      \ZREF@org@@spart{#1}%
    }%
  }%
  \ZREF@patch{@schapter}{%
    \def\@schapter#1{%
      \zref@titleref@setcurrent{#1}%
      \ZREF@org@@schapter{#1}%
    }%
  }%
  \ZREF@patch{@ssect}{%
    \def\@ssect#1#2#3#4#5{%
      \zref@titleref@setcurrent{#5}%
      \ZREF@org@@ssect{#1}{#2}{#3}{#4}{#5}%
    }%
  }%
%    \end{macrocode}
%
% \subsubsection{Environment \textsf{description}}
%
%    \begin{macrocode}
  \ZREF@patch{descriptionlabel}{%
    \def\descriptionlabel#1{%
      \zref@titleref@setcurrent{#1}%
      \ZREF@org@descriptionlabel{#1}%
    }%
  }%
%    \end{macrocode}
%
% \subsubsection{Class \xclass{memoir}}
%
%    \begin{macrocode}
  \@ifclassloaded{memoir}{%
    \ltx@IfUndefined{ifheadnameref}{}{%
      \def\@chapter[#1]#2{%
        \ltx@IfUndefined{ch@pt@c}{%
          \zref@titleref@setcurrent{#1}%
        }{%
          \ifx\ch@pt@c\ltx@empty
            \zref@titleref@setcurrent{#2}%
          \else
            \def\NR@temp{#1}%
            \ifx\NR@temp\ltx@empty
              \expandafter\zref@titleref@setcurrent
              \expandafter{\ch@pt@c}%
            \else
              \ifheadnameref
                \zref@titleref@setcurrent{#1}%
              \else
                \expandafter\zref@titleref@setcurrent
                \expandafter{\ch@pt@c}%
              \fi
            \fi
          \fi
        }%
        \ZREF@org@@chapter[{#1}]{#2}%
      }%
      \ZREF@patch{M@sect}{%
        \def\M@sect#1#2#3#4#5#6[#7][#8]{%
          \ifheadnameref
            \zref@titleref@setcurrent{#8}%
          \else
            \zref@titleref@setcurrent{#7}%
          \fi
          \ZREF@org@M@sect{#1}{#2}{#3}{#4}{#5}{#6}[{#7}][{#8}]%
        }%
      }%
    }%
  }{}%
%    \end{macrocode}
%
% \subsubsection{Class \xclass{beamer}}
%
%    \begin{macrocode}
  \@ifclassloaded{beamer}{%
    \ZREF@patch{beamer@section}{%
      \long\def\beamer@section[#1]{%
        \zref@titleref@setcurrent{#1}%
        \ZREF@org@beamer@section[{#1}]%
      }%
    }%
    \ZREF@patch{beamer@subsection}{%
      \long\def\beamer@subsection[#1]{%
        \zref@titleref@setcurrent{#1}%
        \ZREF@org@beamer@subsection[{#1}]%
      }%
    }%
    \ZREF@patch{beamer@subsubsection}{%
      \long\def\beamer@subsubsection[#1]{%
        \zref@titleref@setcurrent{#1}%
        \ZREF@org@beamer@subsubsection[{#1}]%
      }%
    }%
  }{}%
%    \end{macrocode}
%
% \subsubsection{Package \xpackage{titlesec}}
%
%    \begin{macrocode}
  \@ifpackageloaded{titlesec}{%
    \ZREF@patch{ttl@sect@i}{%
      \def\ttl@sect@i#1#2[#3]#4{%
        \zref@titleref@setcurrent{#4}%
        \ZREF@org@ttl@sect@i{#1}{#2}[{#3}]{#4}%
      }%
    }%
    \ZREF@patch{ttl@straight@i}{%
      \def\ttl@straight@i#1[#2]#3{%
        \def\ZREF@temp{#2}%
        \ifx\ZREF@temp\ltx@empty
          \zref@titleref@setcurrent{#3}%
        \else
          \zref@titleref@setcurrent{#2}%
        \fi
        \ZREF@org@ttl@straight@i{#1}[{#2}]{#3}%
      }%
    }%
  }{}%
%    \end{macrocode}
%
% \subsubsection{Package \xpackage{longtable}}
%
%    Package \xpackage{longtable}: some support for its \cs{caption}.
%    However \cs{label} inside the caption is not supported.
%    \begin{macrocode}
  \@ifpackageloaded{longtable}{%
    \ZREF@patch{LT@c@ption}{%
      \def\LT@c@ption#1[#2]#3{%
        \ZREF@org@LT@c@ption{#1}[{#2}]{#3}%
        \zref@titleref@setcurrent{#2}%
      }%
    }%
  }{}%
%    \end{macrocode}
%
% \subsubsection{Package \xpackage{listings}}
%
%    Package \xpackage{listings}: support for its caption.
%    \begin{macrocode}
  \@ifpackageloaded{listings}{%
    \ZREF@patch{lst@MakeCaption}{%
      \def\lst@MakeCaption{%
        \ifx\lst@label\ltx@empty
        \else
          \expandafter\zref@titleref@setcurrent\expandafter{%
            \lst@@caption
          }%
        \fi
        \ZREF@org@lst@MakeCaption
      }%
    }%
  }{}%
%    \end{macrocode}
%
% \subsubsection{Theorems}
%
%    \begin{macrocode}
  \ZREF@patch{@opargbegintheorem}{%
    \def\@opargbegintheorem#1#2#3{%
      \zref@titleref@setcurrent{#3}%
      \ZREF@org@@opargbegintheorem{#1}{#2}{#3}%
    }%
  }%
  \@ifpackageloaded{amsthm}{%
    \begingroup
      \edef\x{macro:\string#1\string#2[\string#3]}%
      \@onelevel@sanitize\x
      \def\y#1->#2\@nil{#1}%
      \edef\z{\expandafter\y\meaning\@begintheorem->\@nil}%
      \@onelevel@sanitize\z
    \expandafter\endgroup
    \ifx\x\z
      \ZREF@patch{@begintheorem}{%
        \def\@begintheorem#1#2[#3]{%
          \zref@titleref@setcurrent{#3}%
          \ZREF@org@@begintheorem{#1}{#2}[{#3}]%
        }%
      }%
    \fi
  }{}%
%    \end{macrocode}
%
%    \begin{macrocode}
}
%    \end{macrocode}
%
%    \begin{macrocode}
%</titleref>
%    \end{macrocode}
%
%
% \subsection{Module \xmodule{xr}}
%
%    \begin{macrocode}
%<*xr>
\NeedsTeXFormat{LaTeX2e}
\ProvidesPackage{zref-xr}%
  [2020-07-03 v2.32 Module xr for zref (HO)]%
\RequirePackage{zref-base}[2019/11/29]
\ifx\ZREF@base@ok Y%
\else
  \expandafter\endinput
\fi
%    \end{macrocode}
%    \begin{macrocode}
\RequirePackage{keyval}
\RequirePackage{kvoptions}[2010/02/22]
%    \end{macrocode}
%
%    We declare property |url|, because this is added,
%    if a reference is imported and has not already set this field.
%    Or if |hyperref| is used, then this property can be asked.
%    \begin{macrocode}
\zref@newprop{url}{}%
\zref@newprop{urluse}{}%
%    \end{macrocode}
%    \begin{macrocode}
\zref@newprop{externaldocument}{}%
%    \end{macrocode}
%
%    Most code, especially the handling of the \xfile{.aux}
%    files are taken from David Carlisle's \xpackage{xr} package.
%    Therefore I drop the documentation for these macros here.
%
%    \begin{macro}{\zref@xr@ext}
%    If the URL is not specied, then assume
%    processed file with a guessed extension.
%    Use the setting of \xpackage{hyperref} if available.
%    \begin{macrocode}
\providecommand*{\zref@xr@ext}{%
  \ltx@ifundefined{XR@ext}{pdf}{\XR@ext}%
}%
%    \end{macrocode}
%    \end{macro}
%    \begin{macro}{\ifZREF@xr@zreflabel}
%    The use of the star form of \cs{zexternaldocument}
%    is remembered in the switch \cs{ifZREF@xr@zreflabel}.
%    \begin{macrocode}
\newif\ifZREF@xr@zreflabel
%    \end{macrocode}
%    \end{macro}
%    \begin{macrocode}
\SetupKeyvalOptions{%
  family=ZREF@XR,%
  prefix=ZREF@xr@%
}
\DeclareBoolOption[true]{tozreflabel}
\DeclareBoolOption[false]{toltxlabel}
\DeclareBoolOption{verbose}
\define@key{ZREF@XR}{ext}{%
  \def\zref@xr@{#1}%
}
\DeclareBoolOption[false]{urluse}
%    \end{macrocode}
%    \begin{macro}{\zxrsetup}
%    \begin{macrocode}
\newcommand*{\zxrsetup}{%
  \kvsetkeys{ZREF@XR}%
}%
%    \end{macrocode}
%    \end{macro}
%
%    \begin{macro}{\ZREF@xr@URL}
%    \begin{macrocode}
\newcount\ZREF@xr@URL
\ZREF@xr@URL=\ltx@zero
%    \end{macrocode}
%    \end{macro}
%    \begin{macro}{\ZREF@xr@AddURL}
%    \begin{macrocode}
\def\ZREF@xr@AddURL#1{%
  \begingroup
    \def\ZREF@temp{#1}%
    \count@=\ltx@one
    \ZREF@xr@@AddUrl
  \endgroup
}
%    \end{macrocode}
%    \end{macro}
%    \begin{macro}{\ZREF@xr@@AddUrl}
%    \begin{macrocode}
\def\ZREF@xr@@AddUrl{%
  \ifnum\count@>\ZREF@xr@URL
    \global\advance\ZREF@xr@URL by\ltx@one
    \xdef\ZREF@xr@theURL{\romannumeral\ZREF@xr@URL}%
    \global\expandafter\let
        \csname Z@U@\ZREF@xr@theURL\endcsname\ZREF@temp
    \@PackageInfo{zref-xr}{%
      \ltx@backslashchar Z@U@\ZREF@xr@theURL:\MessageBreak
      \ZREF@temp\MessageBreak
    }%
  \else
    \expandafter
    \ifx\csname Z@U@\romannumeral\count@\endcsname\ZREF@temp
      \xdef\ZREF@xr@theURL{\romannumeral\count@}%
    \else
      \expandafter\expandafter\expandafter\ZREF@xr@@AddUrl
    \fi
  \fi
}
%    \end{macrocode}
%    \end{macro}
%
%    \begin{macro}{\zexternaldocument}
%    In its star form it looks for \cs{newlabel},
%    otherwise for \cs{zref@newlabel}.
%     Later we
%    will read \xfile{.aux} files that expects
%    |@| to have catcode 11 (letter).
%    \begin{macrocode}
\ZREF@IfDefinable\zexternaldocument\def{%
  {%
    \ZREF@NewPropAnchor
    \ZREF@NewPropTitle
    \begingroup
      \csname @safe@actives@true\endcsname
      \makeatletter
      \@ifstar{%
        \ZREF@xr@zreflabelfalse
        \@testopt\ZREF@xr@externaldocument{}%
      }{%
        \ZREF@xr@zreflabeltrue
        \@testopt\ZREF@xr@externaldocument{}%
      }%
  }%
}%
%    \end{macrocode}
%    \end{macro}
%
%    If the \cs{include} featuer was used, there can be
%    several \xfile{.aux} files. These files are read
%    one after another, especially they are not recursively
%    read in order to save read registers. Thus it can
%    happen that the read order of the newlabel commands
%    differs from \hologo{LaTeX}'s order using \cs{input}.
%
%    \begin{macro}{\ZREF@xr@externaldocument}
%    It reads the remaining arguments. \cs{newcommand}
%    comes in handy for the optional argument.
%    \begin{macrocode}
\def\ZREF@xr@externaldocument[#1]#2{%
    \def\ZREF@xr@prefix{#1}%
    \let\ZREF@xr@filelist\ltx@empty
    \edef\ZREF@xr@externalfile{#2}%
    \edef\ZREF@xr@file{\ZREF@xr@externalfile.aux}%
    \filename@parse{#2}%
    \@testopt\ZREF@xr@graburl{#2.\zref@xr@ext}%
}%
\def\ZREF@xr@graburl[#1]{%
    \edef\ZREF@xr@url{#1}%
    \ifZREF@xr@urluse
      \expandafter\ZREF@xr@AddURL\expandafter{\ZREF@xr@url}%
      \expandafter\def\expandafter\ZREF@xr@url
      \expandafter{\csname Z@U@\ZREF@xr@theURL\endcsname}%
    \fi
    \ZREF@xr@checkfile
  \endgroup
}%
%    \end{macrocode}
%    \end{macro}
%    \begin{macro}{\ZREF@xr@processfile}
%    We follow \xpackage{xr} here, \cs{IfFileExists} offers
%    a nicer test, but we have to open the file anyway.
%    \begin{macrocode}
\def\ZREF@xr@checkfile{%
  \openin\@inputcheck\ZREF@xr@file\relax
  \ifeof\@inputcheck
    \PackageWarning{zref-xr}{%
      File `\ZREF@xr@file' not found or empty,\MessageBreak
      labels not imported%
    }%
  \else
    \PackageInfo{zref-xr}{%
      Label \ifZREF@xr@zreflabel (zref) \fi
      import from `\ZREF@xr@file'%
    }%
    \def\ZREF@xr@found{0}%
    \def\ZREF@xr@ignored@empty{0}%
    \def\ZREF@xr@ignored@zref{0}%
    \def\ZREF@xr@ignored@ltx{0}%
    \ZREF@xr@processfile
    \closein\@inputcheck
    \begingroup
      \let\on@line\ltx@empty
      \PackageInfo{zref-xr}{%
        Statistics for `\ZREF@xr@file':\MessageBreak
        \ZREF@xr@found\space
        \ifZREF@xr@zreflabel zref\else LaTeX\fi\space
        label(s) found%
        \ifnum\ZREF@xr@ignored@empty>0 %
          ,\MessageBreak
          \ZREF@xr@ignored@empty\space empty label(s) ignored%
        \fi
        \ifnum\ZREF@xr@ignored@zref>0 %
          ,\MessageBreak
          \ZREF@xr@ignored@zref\space
          duplicated zref label(s) ignored%
        \fi
        \ifnum\ZREF@xr@ignored@ltx>0 %
          ,\MessageBreak
          \ZREF@xr@ignored@ltx\space
          duplicated latex label(s) ignored%
        \fi
      }%
    \endgroup
  \fi
  \ifx\ZREF@xr@filelist\ltx@empty
  \else
    \edef\ZREF@xr@file{%
      \expandafter\ltx@car\ZREF@xr@filelist\@nil
    }%
    \edef\ZREF@xr@filelist{%
      \expandafter\ltx@cdr\ZREF@xr@filelist\ltx@empty\@nil
    }%
    \expandafter\ZREF@xr@checkfile
  \fi
}%
%    \end{macrocode}
%    \end{macro}
%
%    \begin{macro}{\ZREF@xr@processfile}
%    \begin{macrocode}
\def\ZREF@xr@processfile{%
  \read\@inputcheck to\ZREF@xr@line
  \expandafter\ZREF@xr@processline\ZREF@xr@line..\ZREF@nil
  \ifeof\@inputcheck
  \else
    \expandafter\ZREF@xr@processfile
  \fi
}%
%    \end{macrocode}
%    \end{macro}
%    \begin{macro}{\ZREF@xr@processline}
%    The most work must be done for analyzing the
%    arguments of \cs{newlabel}.
%    \begin{macrocode}
\long\def\ZREF@xr@processline#1#2#3\ZREF@nil{%
  \def\x{#1}%
  \toks@{#2}%
  \ifZREF@xr@zreflabel
    \ifx\x\ZREF@xr@zref@newlabel
      \expandafter
      \ZREF@xr@process@zreflabel\ZREF@xr@line...\ZREF@nil
    \fi
  \else
    \ifx\x\ZREF@xr@newlabel
      \expandafter
      \ZREF@xr@process@label\ZREF@xr@line...[]\ZREF@nil
    \fi
  \fi
  \ifx\x\ZREF@xr@@input
    \edef\ZREF@xr@filelist{%
      \etex@unexpanded\expandafter{\ZREF@xr@filelist}%
      {\filename@area\the\toks@}%
    }%
  \fi
}%
\def\ZREF@xr@process@zreflabel\zref@newlabel#1#2#3\ZREF@nil{%
  \edef\ZREF@xr@refname{Z@R@\ZREF@xr@prefix#1}%
  \edef\ZREF@xr@found{\the\numexpr\ZREF@xr@found+1\relax}%
  \def\x{#2}%
  \edef\ZREF@xr@tempname{$temp$}%
  \edef\ZREF@xr@temprefname{Z@R@\ZREF@xr@tempname}%
  \let\ZREF@xr@list\x
  \ifx\ZREF@xr@list\ltx@empty
    \PackageWarningNoLine{zref-xr}{%
      Label `#1' without properties ignored\MessageBreak
      in file `\ZREF@xr@file'%
    }%
    \edef\ZREF@xr@ignored@empty{%
      \the\numexpr\ZREF@xr@ignored@empty+1\relax
    }%
  \else
    \expandafter\ZREF@xr@checklist\x\ZREF@nil
    \expandafter\let\csname\ZREF@xr@temprefname\endcsname\x
    \expandafter\ltx@LocalAppendToMacro
    \csname\ZREF@xr@temprefname\expandafter\endcsname
    \expandafter{%
      \expandafter\externaldocument\expandafter{%
        \ZREF@xr@externalfile
      }%
    }%
    \ZREF@xr@urlcheck\ZREF@xr@tempname
    \ifZREF@xr@tozreflabel
      \@ifundefined{\ZREF@xr@refname}{%
        \ifZREF@xr@verbose
          \PackageInfo{zref-xr}{%
            Import to zref label `\ZREF@xr@tempname#1'%
          }%
        \fi
        \global\expandafter
        \let\csname\ZREF@xr@refname\expandafter\endcsname
        \csname\ZREF@xr@temprefname\endcsname
      }{%
        \ZREF@xr@zref@ignorewarning{\ZREF@xr@prefix#1}%
      }%
    \fi
    \ifZREF@xr@toltxlabel
      \ZREF@xr@tolabel{\ZREF@xr@tempname}{\ZREF@xr@prefix#1}%
    \fi
  \fi
}%
\def\ZREF@xr@process@label\newlabel#1#2#3[#4]#5\ZREF@nil{%
  \def\ZREF@xr@refname{Z@R@\ZREF@xr@prefix#1}%
  \edef\ZREF@xr@found{\the\numexpr\ZREF@xr@found+1\relax}%
  \def\x{#2}%
  \edef\ZREF@xr@tempname{$temp$}%
  \edef\ZREF@xr@temprefname{Z@R@\ZREF@xr@tempname}%
  \expandafter\ZREF@xr@scanparams
      \csname\ZREF@xr@temprefname\expandafter\endcsname
      \x{}{}{}{}{}\ZREF@nil
  \ifx\\#4\\%
  \else
    % ntheorem knows an optional argument at the end of \newlabel
    \ZREF@NewPropTheotype
    \expandafter\ltx@LocalAppendToMacro
        \csname\ZREF@xr@temprefname\endcsname{\theotype{#4}}%
  \fi
  \expandafter\ltx@LocalAppendToMacro
  \csname\ZREF@xr@temprefname\expandafter\endcsname\expandafter{%
    \expandafter\externaldocument\expandafter{%
      \ZREF@xr@externalfile
    }%
  }%
  \ZREF@xr@urlcheck\ZREF@xr@tempname
  \ifZREF@xr@tozreflabel
    \@ifundefined{\ZREF@xr@refname}{%
      \ifZREF@xr@verbose
        \PackageInfo{zref-xr}{%
          Import to zref label `\ZREF@xr@prefix#1'%
        }%
      \fi
      \global\expandafter
      \let\csname\ZREF@xr@refname\expandafter\endcsname
      \csname\ZREF@xr@temprefname\endcsname
    }{%
      \ZREF@xr@zref@ignorewarning{\ZREF@xr@prefix#1}%
    }%
  \fi
  \ifZREF@xr@toltxlabel
    \ZREF@xr@tolabel{\ZREF@xr@tempname}{\ZREF@xr@prefix#1}%
  \fi
}
\def\ZREF@xr@zref@newlabel{\zref@newlabel}%
\def\ZREF@xr@newlabel{\newlabel}%
\def\ZREF@xr@@input{\@input}%
\def\ZREF@xr@relax{\relax}%
%    \end{macrocode}
%    \end{macro}
%    \begin{macro}{\ZREF@xr@tolabel}
%    \begin{macrocode}
\def\ZREF@xr@tolabel#1#2{%
  \ifZREF@xr@verbose
    \PackageInfo{zref-xr}{%
      Import to LaTeX label `#2'%
    }%
  \fi
  \zref@wrapper@unexpanded{%
    \expandafter\xdef\csname r@#2\endcsname{%
      {%
        \ltx@ifundefined{M@TitleReference}{%
          \ltx@ifundefined{TR@TitleReference}{%
            \zref@extractdefault{#1}{default}{}%
          }{%
            \noexpand\TR@TitleReference
            {\zref@extractdefault{#1}{default}{}}%
            {\zref@extractdefault{#1}{title}{}}%
          }%
        }{%
          \noexpand\M@TitleReference
          {\zref@extractdefault{#1}{default}{}}%
          {\zref@extractdefault{#1}{title}{}}%
        }%
      }%
      {\zref@extractdefault{#1}{page}{}}%
      \ltx@ifpackageloaded{nameref}{%
        {\zref@extractdefault{#1}{title}{}}%
        {\zref@extractdefault{#1}{anchor}{}}%
        \zref@ifrefcontainsprop{#1}{urluse}{%
          {\zref@extractdefault{#1}{urluse}{}}%
        }{%
          {\zref@extractdefault{#1}{url}{}}%
        }%
      }{}%
    }%
  }%
}
%    \end{macrocode}
%    \end{macro}
%
%    \begin{macro}{\ZREF@xr@zref@ignorewarning}
%    \begin{macrocode}
\def\ZREF@xr@zref@ignorewarning#1{%
  \PackageWarningNoLine{zref-xr}{%
    Zref label `#1' is already in use\MessageBreak
    in file `\ZREF@xr@file'%
  }%
  \edef\ZREF@xr@ignored@zref{%
    \the\numexpr\ZREF@xr@ignored@zref+1%
  }%
}%
%    \end{macrocode}
%    \end{macro}
%
%    \begin{macro}{\ZREF@xr@ltx@ignorewarning}
%    \begin{macrocode}
\def\ZREF@xr@ltx@ignorewarning#1{%
  \PackageWarningNoLine{zref-xr}{%
    LaTeX label `#1' is already in use\MessageBreak
    in file `\ZREF@xr@file'%
  }%
  \edef\ZREF@xr@ignored@ltx{%
    \the\numexpr\ZREF@xr@ignored@ltx+1%
  }%
}%
%    \end{macrocode}
%    \end{macro}
%
%    \begin{macro}{\ZREF@xr@checklist}
%    \begin{macrocode}
\def\ZREF@xr@checklist#1#2#3\ZREF@nil{%
  \ifx\@undefined#1\relax
    \expandafter\ZREF@xr@checkkey\string#1\@nil
  \fi
  \ifx\\#3\\%
  \else
    \ltx@ReturnAfterFi{%
      \ZREF@xr@checklist#3\ZREF@nil
    }%
  \fi
}%
\def\ZREF@xr@checkkey#1#2\@nil{%
  \zref@ifpropundefined{#2}{%
    \zref@newprop{#2}{}%
  }{}%
}%
%    \end{macrocode}
%    \end{macro}
%
%    \begin{macro}{\ZREF@xr@scanparams}
%    \begin{macrocode}
\def\ZREF@xr@scanparams#1#2#3#4#5#6#7\ZREF@nil{%
  \let#1\ltx@empty
  \ZREF@foundfalse
  \ZREF@xr@scantitleref#1#2\TR@TitleReference{}{}\ZREF@nil
  \ifZREF@found
  \else
    \ltx@LocalAppendToMacro#1{\default{#2}}%
  \fi
  % page
  \ltx@LocalAppendToMacro#1{\page{#3}}%
  % nameref title
  \ifZREF@found
  \else
    \ifx\\#4\\%
    \else
      \def\ZREF@xr@temp{#4}%
      \ifx\ZREF@xr@temp\ZREF@xr@relax
      \else
        \ltx@LocalAppendToMacro#1{\title{#4}}%
      \fi
    \fi
  \fi
  % anchor
  \ifx\\#5\\%
  \else
    \ltx@LocalAppendToMacro#1{\anchor{#5}}%
  \fi
  \ifx\\#6\\%
  \else
    \ifZREF@xr@urluse
      \ZREF@xr@AddURL{#6}%
      \expandafter\ltx@LocalAppendToMacro\expandafter#1%
      \expandafter{%
        \expandafter\urluse\expandafter{%
          \csname Z@U@\ZREF@xr@theURL\endcsname
        }%
      }%
    \else
      \ltx@LocalAppendToMacro#1{\url{#6}}%
    \fi
  \fi
}%
%    \end{macrocode}
%    \end{macro}
%
%    \begin{macro}{\ZREF@xr@scantitleref}
%    \begin{macrocode}
\def\ZREF@xr@scantitleref#1#2\TR@TitleReference#3#4#5\ZREF@nil{%
  \ifx\\#5\\%
  \else
    \ltx@LocalAppendToMacro#1{%
      \default{#3}%
      \title{#4}%
    }%
    \ZREF@foundtrue
  \fi
}%
%    \end{macrocode}
%    \end{macro}
%
%    \begin{macro}{\ZREF@xr@urlcheck}%
%    \begin{macrocode}
\def\ZREF@xr@urlcheck#1{%
  \zref@ifrefcontainsprop{#1}{anchor}{%
    \zref@ifrefcontainsprop{#1}{url}{%
    }{%
      \expandafter
      \ltx@LocalAppendToMacro\csname Z@R@#1\expandafter\endcsname
      \expandafter{%
        \csname url\ifZREF@xr@urluse use\fi
        \expandafter\endcsname\expandafter{\ZREF@xr@url}%
      }%
    }%
  }{%
  }%
}%
%    \end{macrocode}
%    \end{macro}
%
%    \begin{macrocode}
%</xr>
%    \end{macrocode}
%
%
% \subsection{Module \xmodule{hyperref}}
%
%    UNFINISHED :-(
%
%    \begin{macrocode}
%<*hyperref>
\NeedsTeXFormat{LaTeX2e}
\ProvidesPackage{zref-hyperref}%
  [2020-07-03 v2.32 Module hyperref for zref (HO)]%
\RequirePackage{zref-base}[2019/11/29]
\ifx\ZREF@base@ok Y%
\else
  \expandafter\endinput
\fi
%    \end{macrocode}
%
%    \begin{macrocode}
\ZREF@NewPropAnchor
\zref@addprop\ZREF@mainlist{anchor}%
%    \end{macrocode}
%
%    \begin{macrocode}
%</hyperref>
%    \end{macrocode}
%
%
% \subsection{Module \xmodule{savepos}}
%
%    Module \xmodule{savepos} provides an interface for
%    \pdfTeX's \cmd{\pdfsavepos}, see the manual for
%    \pdfTeX.
%
% \subsubsection{Identification}
%
%    \begin{macrocode}
%<*savepos>
\NeedsTeXFormat{LaTeX2e}
\ProvidesPackage{zref-savepos}%
  [2020-07-03 v2.32 Module savepos for zref (HO)]%
\RequirePackage{zref-base}[2019/11/29]
\ifx\ZREF@base@ok Y%
\else
  \expandafter\endinput
\fi
%    \end{macrocode}
%
% \subsubsection{Availability}
%
%    First we check, whether the feature is available.
%    \begin{macrocode}
\ifx\directlua\@undefined
\ltx@IfUndefined{pdfsavepos}{%
  \PackageError\ZREF@name{%
    \string\pdfsavepos\space is not supported.\MessageBreak
    It is provided by pdfTeX (1.40) or XeTeX%
  }\ZREF@UpdatePdfTeX
  \endinput
}{}%
\fi
%    \end{macrocode}
%    In PDF mode we are done. However support for DVI mode
%    was added later in version 1.40.0. In earlier versions
%    \cs{pdfsavepos} is defined, but its execution raises
%    an error. Note that \hologo{XeTeX} also provides \cs{pdfsavepos}.
%    \begin{macrocode}
\ifpdf
\else
  \ltx@IfUndefined{pdftexversion}{%
  }{%
    \ifnum\pdftexversion<140 %
      \PackageError\ZREF@name{%
        \string\pdfsavepos\space is not supported in DVI mode%
        \MessageBreak
        of this pdfTeX version%
      }\ZREF@UpdatePdfTeX
      \expandafter\expandafter\expandafter\endinput
    \fi
  }%
\fi
%    \end{macrocode}
%
% \subsubsection{Setup}
%
%    \begin{macrocode}
\zref@newlist{savepos}
\ifx\directlua\@undefined
  \zref@newprop*{posx}[0]{\the\pdflastxpos}
  \zref@newprop*{posy}[0]{\the\pdflastypos}
\else
  \zref@newprop*{posx}[0]{\the\lastxpos}
  \zref@newprop*{posy}[0]{\the\lastypos}
\fi
\zref@addprops{savepos}{posx,posy}
%    \end{macrocode}
%
% \subsubsection{User macros}
%
%    \begin{macro}{\zref@savepos}
%    \begin{macrocode}
\ifx\directlua\@undefined
  \def\zref@savepos{%
    \if@filesw
      \pdfsavepos
    \fi
  }
\else
  \def\zref@savepos{%
    \if@filesw
      \savepos
    \fi
  }
\fi
%    \end{macrocode}
%    \end{macro}
%
%    \begin{macro}{\ZREF@zsavepos}
%    \begin{macrocode}
\def\ZREF@zsavepos#1#2#3{%
  \@bsphack
  \if@filesw
    \zref@savepos
    #1{#3}{#2}%
    \ltx@IfUndefined{TeXXeTstate}{%
    }{%
      \ifnum\TeXXeTstate=\ltx@zero
      \else
        \zref@savepos
      \fi
    }%
  \fi
  \@esphack
}
%    \end{macrocode}
%    \end{macro}
%
%    \begin{macro}{\zsavepos}
%    The current location is stored in a reference with the
%    given name.
%    \begin{macrocode}
\ZREF@IfDefinable\zsavepos\def{%
  {%
    \ZREF@zsavepos\zref@labelbylist{savepos}%
  }%
}
%    \end{macrocode}
%    \end{macro}
%
%    \begin{macro}{\zsaveposx}
%    \begin{macrocode}
\ZREF@IfDefinable\zsaveposx\def{%
  {%
    \ZREF@zsavepos\zref@labelbyprops{posx}%
  }%
}
%    \end{macrocode}
%    \end{macro}
%    \begin{macro}{\zsaveposy}
%    \begin{macrocode}
\ZREF@IfDefinable\zsaveposy\def{%
  {%
    \ZREF@zsavepos\zref@labelbyprops{posy}%
  }%
}
%    \end{macrocode}
%    \end{macro}
%
%    \begin{macro}{\zposx}
%    \begin{macro}{\zposy}
%    The horizontal and vertical position are available by
%    \cs{zposx} and \cs{zposy}. Do not rely on absolute
%    positions. They differ in DVI and PDF mode of \pdfTeX.
%    Use differences instead. The unit of the position numbers
%    is sp.
%    \begin{macrocode}
\newcommand*{\zposx}[1]{%
  \zref@extract{#1}{posx}%
}%
\newcommand*{\zposy}[1]{%
  \zref@extract{#1}{posy}%
}%
%    \end{macrocode}
%    \end{macro}
%    \end{macro}
%
%    Typically horizontal and vertical positions are used
%    inside calculations. Therefore the extracting macros
%    should be expandable and babel's patch is not applyable.
%
%    Also it is in the responsibility of the user to marked
%    used positions by \cs{zrefused}
%    in order to notify \hologo{LaTeX} about undefined references.
%
%    \begin{macro}{\ZREF@savepos@ok}
%    \begin{macrocode}
\let\ZREF@savepos@ok=Y
%    \end{macrocode}
%    \end{macro}
%
%    \begin{macrocode}
%</savepos>
%    \end{macrocode}
%
%
% \subsection{Module \xmodule{abspos}}
%
% \subsubsection{Identification}
%
%    \begin{macrocode}
%<*abspos>
\NeedsTeXFormat{LaTeX2e}
\ProvidesPackage{zref-abspos}%
  [2020-07-03 v2.32 Module abspos for zref (HO)]%
\RequirePackage{zref-base}[2019/11/29]
\ifx\ZREF@base@ok Y%
\else
  \expandafter\endinput
\fi
%    \end{macrocode}
%
%    \begin{macrocode}
\RequirePackage{zref-savepos}[2019/11/29]
\ifx\ZREF@savepos@ok Y%
\else
  \expandafter\endinput
\fi
%    \end{macrocode}
%    \begin{macrocode}
\RequirePackage{zref-pagelayout}[2019/11/29]
\zref@addprop{savepos}{abspage}
\zref@addprop{savepos}{pagevalue}
%    \end{macrocode}
%
%    \begin{macro}{\zref@absposx}
%    \begin{macrocode}
\newcommand*{\zref@absposx}[3]{%
  \number
    \expandafter\zref@absposnumx\expandafter{%
      \number\zref@extractdefault{#1}{abspage}{0}%
    }{#2}{#3}%
  \ltx@space
}
%    \end{macrocode}
%    \end{macro}
%
%    \begin{macro}{\zref@absposy}
%    \begin{macrocode}
\newcommand*{\zref@absposy}[3]{%
  \number
  \expandafter\zref@absposnumy\expandafter{%
    \number\zref@extractdefault{#1}{abspage}{0}%
  }{#2}{#3}%
  \ltx@space
}
%    \end{macrocode}
%    \end{macro}
%
%    \begin{macro}{\zref@absposnumx}
%    \begin{macrocode}
\newcommand*{\zref@absposnumx}[3]{%
  \number
%  \ifnum#1>\ltx@zero
%    \zref@ifrefundefined{thepage#1}{%
%      0%
%    }{%
      \numexpr\ZREF@absposnum{thepage#1}{#2}{x}{#3}\relax
%    }%
%  \else
%    0%
%  \fi
}
%    \end{macrocode}
%    \end{macro}
%
%    \begin{macro}{\zref@absposnumy}
%    \begin{macrocode}
\newcommand*{\zref@absposnumy}[3]{%
  \number
%  \ifnum#1>\ltx@zero
%    \zref@ifrefundefined{thepage#1}{%
%      0%
%    }{%
      \numexpr\ZREF@absposnum{thepage#1}{#2}{y}{#3}\relax
%    }%
%  \else
%    0%
%  \fi
}
%    \end{macrocode}
%    \end{macro}
%
%    \begin{macro}{\ZREF@absposnum}
%    \begin{macrocode}
\def\ZREF@absposnum#1#2#3#4{%
  \ltx@ifundefined{ZREF@abspos@#2@#3@#4}{%
    0%
  }{%
    \csname ZREF@abspos@#2@#3@#4\endcsname{#1}%
  }%
}
%    \end{macrocode}
%    \end{macro}
%
%    \begin{macro}{\zref@def@absposx}
%    \begin{macrocode}
\ZREF@Robust\def\zref@def@absposx#1{%
  \zref@wrapper@babel{\ZREF@def@abspos{#1}\zref@absposx}%
}
%    \end{macrocode}
%    \end{macro}
%
%    \begin{macro}{\zref@def@absposy}
%    \begin{macrocode}
\ZREF@Robust\def\zref@def@absposy#1{%
  \zref@wrapper@babel{\ZREF@def@abspos{#1}\zref@absposy}%
}
%    \end{macrocode}
%    \end{macro}
%
%    \begin{macro}{\zref@def@absposnumx}
%    \begin{macrocode}
\ZREF@Robust\def\zref@def@absposnumx#1{%
  \zref@wrapper@babel{\ZREF@def@abspos{#1}\zref@absposnumx}%
}
%    \end{macrocode}
%    \end{macro}
%
%    \begin{macro}{\zref@def@absposnumy}
%    \begin{macrocode}
\ZREF@Robust\def\zref@def@absposnumy#1{%
  \zref@wrapper@babel{\ZREF@def@abspos{#1}\zref@absposnumy}%
}
%    \end{macrocode}
%    \end{macro}
%    \begin{macro}{\ZREF@def@abspos}
%    \begin{macrocode}
\def\ZREF@def@absposnumy#1#2#3#4#5{%
  \edef#1{#2{#3}{#4}{#5}}%
}
%    \end{macrocode}
%    \end{macro}
%
%    \begin{macro}{\zref@absposused}
%    \begin{macrocode}
\ZREF@Robust\def\zref@absposused{%
  \zref@wrapper@babel\ZREF@abspos@used
}
%    \end{macrocode}
%    \end{macro}
%    \begin{macro}{\ZREF@abspos@used}
%    \begin{macrocode}
\def\ZREF@abspos@used#1{%
  \zref@refused{#1}%
  \zref@ifrefundefined{#1}{%
  }{%
    \begingroup
      \edef\ZREF@temp{%
        \zref@extractdefault{#1}{abspage}{0}%
      }%
      \ifnum\ZREF@temp>\ltx@zero
        \zref@refused{thepage\ZREF@temp}%
      \else
        \@PackageError{zref-abspos}{%
          \string\zref@pos@label@used\ltx@space
          needs property `abspage'\MessageBreak
          in label `#1'%
        }\@ehc
      \fi
    \endgroup
  }%
}
%    \end{macrocode}
%    \end{macro}
%
%    \begin{macro}{\zref@absposnumused}
%    \begin{macrocode}
\newcommand*{\zref@absposnumused}[1]{%
  \ifnum#1>\ltx@zero
    \zref@refused{thepage\number#1}%
  \else
    \@PackageError{zref-abspos}{%
      Invalid absolute page number (#1)\MessageBreak
      for \string\zref@pos@num@used.\MessageBreak
      A positive integer is expected%
    }\@ehc
  \fi
}
%    \end{macrocode}
%    \end{macro}
%
%    \begin{macro}{\zref@ifabsposundefined}
%    \begin{macrocode}
\def\zref@ifabsposundefined#1{%
  \zref@ifrefundefined{#1}\ltx@firsttwo{%
    \expandafter\zref@ifabsposnumundefined\expandafter{%
      \number\zref@extractdefault{#1}{abspage}{0}%
    }%
  }%
}
%    \end{macrocode}
%    \end{macro}
%
%    \begin{macro}{\zref@ifabsposnumundefined}
%    \begin{macrocode}
\def\zref@ifabsposnumundefined#1{%
  \ifnum\ZREF@number{#1}>\ltx@zero
    \zref@ifrefundefined{thepage#1}%
        \ltx@firstoftwo\ltx@secondoftwo
  \else
    \expandafter\ltx@firstoftwo
  \fi
}
%    \end{macrocode}
%    \end{macro}
%
% \subsubsection{Media}
%
%    \begin{macro}{\ZREF@abspos@media@width}
%    \begin{macrocode}
\edef\ZREF@abspos@media@width{%
  \ltx@ifundefined{pdfpagewidth}{%
    \ltx@ifundefined{mediawidth}{%
      \ltx@ifundefined{stockwidth}{%
        paperwidth%
      }{%
        stockwidth%
      }%
    }{%
      mediawidth%
    }%
  }{%
    pdfpagewidth%
  }%
}
\ifluatex
\def\ZREF@abspos@media@width{pdfpagewidth}%
\fi
%    \end{macrocode}
%    \end{macro}
%    \begin{macro}{\ZREF@abspos@media@height}
%    \begin{macrocode}
\edef\ZREF@abspos@media@height{%
  \ltx@ifundefined{pdfpageheight}{%
    \ltx@ifundefined{mediaheight}{%
      \ltx@ifundefined{stockheight}{%
        paperheight%
      }{%
        stockheight%
      }%
    }{%
      mediaheight%
    }%
  }{%
    \noexpand\ifcase\pdfpageheight
      \ltx@ifundefined{stockheight}{%
        paperheight%
      }{%
        stockheight%
      }%
    \noexpand\else
      pdfpageheight%
    \noexpand\fi
  }%
}
\ifluatex
\edef\ZREF@abspos@media@height{%
   \noexpand\ifcase\pageheight
      \ltx@ifundefined{stockheight}{%
        paperheight%
      }{%
        stockheight%
      }%
    \noexpand\else
      pdfpageheight%
    \noexpand\fi}%
\fi
%    \end{macrocode}
%    \end{macro}
%
%    \begin{macro}{\ZREF@abspos@media@x@left}
%    \begin{macrocode}
\def\ZREF@abspos@media@x@left#1{%
  0%
}
%    \end{macrocode}
%    \end{macro}
%    \begin{macro}{\ZREF@abspos@media@x@right}
%    \begin{macrocode}
\def\ZREF@abspos@media@x@right#1{%
  \zref@extract{#1}\ZREF@abspos@media@width
}
%    \end{macrocode}
%    \end{macro}
%    \begin{macro}{\ZREF@abspos@media@x@center}
%    \begin{macrocode}
\def\ZREF@abspos@media@x@center#1{%
  \ZREF@abspos@media@x@left{#1}%
  +\zref@extract{#1}\ZREF@abspos@media@width/2%
}
%    \end{macrocode}
%    \end{macro}
%
%    \begin{macro}{\ZREF@abspos@media@y@top}
%    \begin{macrocode}
\def\ZREF@abspos@media@y@top#1{%
  \zref@extract{#1}\ZREF@abspos@media@height
}
%    \end{macrocode}
%    \end{macro}
%    \begin{macro}{\ZREF@abspos@media@y@bottom}
%    \begin{macrocode}
\def\ZREF@abspos@media@y@bottom#1{%
  0%
}
%    \end{macrocode}
%    \end{macro}
%    \begin{macro}{\ZREF@abspos@media@y@center}
%    \begin{macrocode}
\def\ZREF@abspos@media@y@center#1{%
  \zref@extract{#1}\ZREF@abspos@media@height/2%
}
%    \end{macrocode}
%    \end{macro}
%
% \subsubsection{Paper}
%
%    \begin{macro}{\ZREF@abspos@paper@x@left}
%    \begin{macrocode}
\def\ZREF@abspos@paper@x@left#1{%
  0%
}
%    \end{macrocode}
%    \end{macro}
%    \begin{macro}{\ZREF@abspos@paper@x@right}
%    \begin{macrocode}
\def\ZREF@abspos@paper@x@right#1{%
  \zref@extract{#1}{paperwidth}%
}
%    \end{macrocode}
%    \end{macro}
%    \begin{macro}{\ZREF@abspos@paper@x@center}
%    \begin{macrocode}
\def\ZREF@abspos@paper@x@center#1{%
  \zref@extract{#1}{paperwidth}/2%
}
%    \end{macrocode}
%    \end{macro}
%
%    \begin{macro}{\ZREF@abspos@paper@y@top}
%    \begin{macrocode}
\let\ZREF@abspos@paper@y@top\ZREF@abspos@media@y@top
%    \end{macrocode}
%    \end{macro}
%    \begin{macro}{\ZREF@abspos@paper@y@bottom}
%    \begin{macrocode}
\def\ZREF@abspos@paper@y@bottom#1{%
  \ZREF@abspos@paper@y@top{#1}%
  -\zref@extract{#1}{paperheight}%
}
%    \end{macrocode}
%    \end{macro}
%    \begin{macro}{\ZREF@abspos@paper@y@center}
%    \begin{macrocode}
\def\ZREF@abspos@paper@y@center#1{%
  \ZREF@abspos@paper@y@top{#1}%
  -\zref@extract{#1}{paperheight}/2%
}
%    \end{macrocode}
%    \end{macro}
%
% \subsubsection{Origin}
%
% There doesn't seem a good reason to make these tests depend on pdf mode
%  in current engines, so comment out the \verb|\ifpdf| tests.
%    \begin{macro}{\ZREF@abspos@origin@x}
%    \begin{macrocode}
\let\ZREF@temp\ltx@two
\ltx@IfUndefined{pdfhorigin}{}{%
%  \ifpdf
    \let\ZREF@temp\ltx@zero
%  \fi
}
\ifluatex
% \ifpdf
  \let\ZREF@temp\ltx@zero
% \fi
\fi

\ifx\ZREF@temp\ltx@two
  \ifnum\mag=1000 %
    \let\ZREF@temp\ltx@one
  \fi
\fi
\ifcase\ZREF@temp
  \def\ZREF@abspos@origin@x#1{%
    \zref@extract{#1}{pdfhorigin}%
  }%
\or
  \def\ZREF@abspos@origin@x#1{%
    4736286%
  }%
\or
  \def\ZREF@abspos@origin@x#1{%
    \numexpr\mag/1000*\dimexpr 1truein\relax\relax
  }%
\fi
%    \end{macrocode}
%    \end{macro}
%    \begin{macro}{\ZREF@abspos@origin@y}
%    \begin{macrocode}
\let\ZREF@temp\ltx@two
\ltx@IfUndefined{pdfvorigin}{}{%
%  \ifpdf
    \let\ZREF@temp\ltx@zero
%  \fi
}
\ifluatex
%  \ifpdf
    \let\ZREF@temp\ltx@zero
%  \fi
\fi
\ifx\ZREF@temp\ltx@two
  \ifnum\mag=1000 %
    \let\ZREF@temp\ltx@one
  \fi
\fi
\ifcase\ZREF@temp
  \def\ZREF@abspos@origin@y#1{%
    \zref@extract{#1}{pdfvorigin}%
  }%
\or
  \def\ZREF@abspos@origin@y#1{%
    4736286%
  }%
\or
  \def\ZREF@abspos@origin@y#1{%
    \numexpr\mag/1000*\dimexpr 1truein\relax\relax
   }%
\fi
%    \end{macrocode}
%    \end{macro}
%
% \subsubsection{Header}
%
%    \begin{macro}{\ZREF@abspos@head@x@left}
%    \begin{macrocode}
\def\ZREF@abspos@head@x@left#1{%
  \ZREF@abspos@paper@x@left{#1}%
  +\ZREF@abspos@origin@x{#1}%
  +\zref@extract{#1}{hoffset}%
  +\ifodd\zref@extractdefault{#1}{pagevalue}{\number\c@page} %
     \zref@extract{#1}{oddsidemargin}%
   \else
     \zref@extract{#1}{evensidemargin}%
   \fi
}
%    \end{macrocode}
%    \end{macro}
%    \begin{macro}{\ZREF@abspos@head@x@right}
%    \begin{macrocode}
\def\ZREF@abspos@head@x@right#1{%
  \ZREF@abspos@head@x@left{#1}%
  +\zref@extract{#1}{textwidth}%
}
%    \end{macrocode}
%    \end{macro}
%    \begin{macro}{\ZREF@abspos@head@x@center}
%    \begin{macrocode}
\def\ZREF@abspos@head@x@center#1{%
  \ZREF@abspos@head@x@left{#1}%
  +\zref@extract{#1}{textwidth}/2%
}
%    \end{macrocode}
%    \end{macro}
%
%    \begin{macro}{\ZREF@abspos@head@y@top}
%    \begin{macrocode}
\def\ZREF@abspos@head@y@top#1{%
  \ZREF@abspos@paper@y@top{#1}%
  -\ZREF@abspos@origin@y{#1}%
  -\zref@extract{#1}{voffset}%
  -\zref@extract{#1}{topmargin}%
}
%    \end{macrocode}
%    \end{macro}
%    \begin{macro}{\ZREF@abspos@head@y@bottom}
%    \begin{macrocode}
\def\ZREF@abspos@head@y@bottom#1{%
  \ZREF@abspos@head@y@top{#1}%
  -\zref@extract{#1}{headheight}%
}
%    \end{macrocode}
%    \end{macro}
%    \begin{macro}{\ZREF@abspos@head@y@center}
%    \begin{macrocode}
\def\ZREF@abspos@head@y@center#1{%
  \ZREF@abspos@head@y@top{#1}%
  -\zref@extract{#1}{headheight}/2%
}
%    \end{macrocode}
%    \end{macro}
%
% \subsubsection{Body}
%
%    \begin{macro}{\ZREF@abspos@body@x@left}
%    \begin{macrocode}
\let\ZREF@abspos@body@x@left\ZREF@abspos@head@x@left
%    \end{macrocode}
%    \end{macro}
%    \begin{macro}{\ZREF@abspos@body@x@right}
%    \begin{macrocode}
\let\ZREF@abspos@body@x@right\ZREF@abspos@head@x@right
%    \end{macrocode}
%    \end{macro}
%    \begin{macro}{\ZREF@abspos@body@x@center}
%    \begin{macrocode}
\let\ZREF@abspos@body@x@center\ZREF@abspos@head@x@center
%    \end{macrocode}
%    \end{macro}
%
%    \begin{macro}{\ZREF@abspos@body@y@top}
%    \begin{macrocode}
\def\ZREF@abspos@body@y@top#1{%
  \ZREF@abspos@head@y@bottom{#1}%
  -\zref@extract{#1}{headsep}%
}
%    \end{macrocode}
%    \end{macro}
%    \begin{macro}{\ZREF@abspos@body@y@bottom}
%    \begin{macrocode}
\def\ZREF@abspos@body@y@bottom#1{%
  \ZREF@abspos@body@y@top{#1}%
  -\zref@extract{#1}{textheight}%
}
%    \end{macrocode}
%    \end{macro}
%    \begin{macro}{\ZREF@abspos@body@y@center}
%    \begin{macrocode}
\def\ZREF@abspos@body@y@center#1{%
  \ZREF@abspos@body@y@top{#1}%
  -\zref@extract{#1}{textheight}/2%
}
%    \end{macrocode}
%    \end{macro}
%
% \subsubsection{Footer}
%
%    \begin{macro}{\ZREF@abspos@foot@x@left}
%    \begin{macrocode}
\let\ZREF@abspos@foot@x@left\ZREF@abspos@head@x@left
%    \end{macrocode}
%    \end{macro}
%    \begin{macro}{\ZREF@abspos@foot@x@right}
%    \begin{macrocode}
\let\ZREF@abspos@foot@x@right\ZREF@abspos@head@x@right
%    \end{macrocode}
%    \end{macro}
%    \begin{macro}{\ZREF@abspos@foot@x@center}
%    \begin{macrocode}
\let\ZREF@abspos@foot@x@center\ZREF@abspos@head@x@center
%    \end{macrocode}
%    \end{macro}
%
%    \begin{macro}{\ZREF@abspos@foot@y@bottom}
%    \begin{macrocode}
\def\ZREF@abspos@foot@y@bottom#1{%
  \ZREF@abspos@body@y@bottom{#1}%
  -\zref@extract{#1}{footskip}%
}
%    \end{macrocode}
%    \end{macro}
%
% \subsubsection{Marginal notes}
%
%    \begin{macro}{\ZREF@abspos@marginpar@x@left}
%    \begin{macrocode}
\def\ZREF@abspos@marginpar@x@left#1{%
  \ifodd\zref@extractdefault{#1}{pagevalue}{\number\c@page} %
    \ZREF@abspos@body@x@right{#1}%
    +\zref@extract{#1}{marginparsep}%
  \else
    \ZREF@abspos@body@x@left{#1}%
    -\zref@extract{#1}{marginparsep}%
    -\zref@extract{#1}{marginparwidth}%
  \fi
}
%    \end{macrocode}
%    \end{macro}
%    \begin{macro}{\ZREF@abspos@marginpar@x@right}
%    \begin{macrocode}
\def\ZREF@abspos@marginpar@x@right#1{%
  \ZREF@abspos@marginpar@x@left{#1}%
  +\zref@extract{#1}{marginparwidth}%
}
%    \end{macrocode}
%    \end{macro}
%    \begin{macro}{\ZREF@abspos@marginpar@x@center}
%    \begin{macrocode}
\def\ZREF@abspos@marginpar@x@center#1{%
  \ZREF@abspos@marginpar@x@left{#1}%
  +\zref@extract{#1}{marginparwidth}/2%
}
%    \end{macrocode}
%    \end{macro}
%
%    \begin{macro}{\ZREF@abspos@marginpar@y@top}
%    \begin{macrocode}
\let\ZREF@abspos@marginpar@y@top\ZREF@abspos@body@y@top
%    \end{macrocode}
%    \end{macro}
%    \begin{macro}{\ZREF@abspos@marginpar@y@bottom}
%    \begin{macrocode}
\let\ZREF@abspos@marginpar@y@bottom\ZREF@abspos@body@y@bottom
%    \end{macrocode}
%    \end{macro}
%    \begin{macro}{\ZREF@abspos@marginpar@y@center}
%    \begin{macrocode}
\let\ZREF@abspos@marginpar@y@center\ZREF@abspos@body@y@center
%    \end{macrocode}
%    \end{macro}
%
% \subsubsection{Stock paper}
%
%    \begin{macro}{\ZREF@abspos@stock@x@left}
%    \begin{macrocode}
\let\ZREF@abspos@stock@x@left\ZREF@abspos@paper@x@left
%    \end{macrocode}
%    \end{macro}
%    \begin{macro}{\ZREF@abspos@stock@x@right}
%    \begin{macrocode}
\let\ZREF@abspos@stock@x@right\ZREF@abspos@paper@x@right
%    \end{macrocode}
%    \end{macro}
%    \begin{macro}{\ZREF@abspos@stock@x@center}
%    \begin{macrocode}
\let\ZREF@abspos@stock@x@center\ZREF@abspos@paper@x@center
%    \end{macrocode}
%    \end{macro}
%
%    \begin{macro}{\ZREF@abspos@stock@y@top}
%    \begin{macrocode}
\let\ZREF@abspos@stock@y@top\ZREF@abspos@paper@y@top
%    \end{macrocode}
%    \end{macro}
%    \begin{macro}{\ZREF@abspos@stock@y@bottom}
%    \begin{macrocode}
\let\ZREF@abspos@stock@y@bottom\ZREF@abspos@paper@y@bottom
%    \end{macrocode}
%    \end{macro}
%    \begin{macro}{\ZREF@abspos@stock@y@center}
%    \begin{macrocode}
\let\ZREF@abspos@stock@y@center\ZREF@abspos@paper@y@center
%    \end{macrocode}
%    \end{macro}
%
%    \begin{macrocode}
%</abspos>
%    \end{macrocode}
%
%
% \subsection{Module \xmodule{dotfill}}
%
%    \begin{macrocode}
%<*dotfill>
\NeedsTeXFormat{LaTeX2e}
\ProvidesPackage{zref-dotfill}%
  [2020-07-03 v2.32 Module dotfill for zref (HO)]%
\RequirePackage{zref-base}[2019/11/29]
\ifx\ZREF@base@ok Y%
\else
  \expandafter\endinput
\fi
%    \end{macrocode}
%
%    For measuring the width of \cs{zdotfill} we use the features
%    provided by module \xmodule{savepos}.
%    \begin{macrocode}
\RequirePackage{zref-savepos}[2019/11/29]
%    \end{macrocode}
%    For automatically generated label names we use the unique
%    counter of module \xmodule{base}.
%    \begin{macrocode}
\zref@require@unique
%    \end{macrocode}
%    Configuration is done by the key value interface of package
%    \xpackage{keyval}.
%    \begin{macrocode}
\RequirePackage{keyval}
%    \end{macrocode}
%    The definitions of the keys follow.
%    \begin{macrocode}
\define@key{ZREF@DF}{unit}{%
  \def\ZREF@df@unit{#1}%
}
\define@key{ZREF@DF}{min}{%
  \def\ZREF@df@min{#1}%
}
\define@key{ZREF@DF}{dot}{%
  \def\ZREF@df@dot{#1}%
}
%    \end{macrocode}
%    Defaults are set, see user interface.
%    \begin{macrocode}
\providecommand\ZREF@df@min{2}
\providecommand\ZREF@df@unit{.44em}
\providecommand\ZREF@df@dot{.}
%    \end{macrocode}
%    \begin{macro}{\zdotfillsetup}
%    Configuration of \cs{zdotfill} is done by \cs{zdotfillsetup}.
%    \begin{macrocode}
\newcommand*{\zdotfillsetup}{\kvsetkeys{ZREF@DF}}
%    \end{macrocode}
%    \end{macro}
%
%    \begin{macro}{\zdotfill}
%    \cs{zdotfill} sets labels at the left and the right to get
%    the horizontal position. \cs{zsavepos} is not used, because
%    we do not need the vertical position.
%    \begin{macrocode}
\ZREF@IfDefinable\zdotfill\def{%
  {%
    \leavevmode
    \global\advance\c@zref@unique\ltx@one
    \begingroup
      \def\ZREF@temp{zref@\number\c@zref@unique}%
      \pdfsavepos
      \zref@labelbyprops{\thezref@unique L}{posx}%
      \setlength{\dimen@}{\ZREF@df@unit}%
      \zref@ifrefundefined{\thezref@unique R}{%
        \ZREF@dotfill
      }{%
        \ifnum\numexpr\zposx{\thezref@unique R}%
                      -\zposx{\thezref@unique L}\relax
            <\dimexpr\ZREF@df@min\dimen@\relax
          \hfill
        \else
          \ZREF@dotfill
        \fi
      }%
      \pdfsavepos
      \zref@labelbyprops{\thezref@unique R}{posx}%
    \endgroup
    \kern\z@
  }%
}
%    \end{macrocode}
%    \end{macro}
%    \begin{macro}{\ZREF@dotfill}
%    Help macro that actually sets the dots.
%    \begin{macrocode}
\def\ZREF@dotfill{%
  \cleaders\hb@xt@\dimen@{\hss\ZREF@df@dot\hss}\hfill
}
%    \end{macrocode}
%    \end{macro}
%
%    \begin{macrocode}
%</dotfill>
%    \end{macrocode}
%
% \subsection{Module \xmodule{env}}
%
%    \begin{macrocode}
%<*env>
\NeedsTeXFormat{LaTeX2e}
\ProvidesPackage{zref-env}%
  [2020-07-03 v2.32 Module env for zref (HO)]%
\RequirePackage{zref-base}[2019/11/29]
\ifx\ZREF@base@ok Y%
\else
  \expandafter\endinput
\fi
%    \end{macrocode}
%
%    \begin{macrocode}
\zref@newprop{envname}[]{\@currenvir}
\zref@newprop{envline}[]{\zref@env@line}
%    \end{macrocode}
%
%    \begin{macro}{\zref@env@line}
%    Macro \cs{zref@env@line} extracts the line number
%    from \cs{@currenvline}.
%    \begin{macrocode}
\def\zref@env@line{%
  \ifx\@currenvline\ltx@empty
  \else
    \expandafter
    \ZREF@ENV@line\@currenvline\ltx@empty line \ltx@empty\@nil
  \fi
}
%    \end{macrocode}
%    \end{macro}
%    \begin{macro}{\ZREF@ENV@line}
%    \begin{macrocode}
\def\ZREF@ENV@line#1line #2\ltx@empty#3\@nil{#2}%
%    \end{macrocode}
%    \end{macro}
%
%    \begin{macrocode}
%</env>
%    \end{macrocode}
%
%% \section{Installation}
%
% \subsection{Download}
%
% \paragraph{Package.} This package is available on
% CTAN\footnote{\CTANpkg{zref}}:
% \begin{description}
% \item[\CTAN{macros/latex/contrib/zref/zref.dtx}] The source file.
% \item[\CTAN{macros/latex/contrib/zref/zref.pdf}] Documentation.
% \end{description}
%
%
% \paragraph{Bundle.} All the packages of the bundle `zref'
% are also available in a TDS compliant ZIP archive. There
% the packages are already unpacked and the documentation files
% are generated. The files and directories obey the TDS standard.
% \begin{description}
% \item[\CTANinstall{install/macros/latex/contrib/zref.tds.zip}]
% \end{description}
% \emph{TDS} refers to the standard ``A Directory Structure
% for \TeX\ Files'' (\CTANpkg{tds}). Directories
% with \xfile{texmf} in their name are usually organized this way.
%
% \subsection{Bundle installation}
%
% \paragraph{Unpacking.} Unpack the \xfile{zref.tds.zip} in the
% TDS tree (also known as \xfile{texmf} tree) of your choice.
% Example (linux):
% \begin{quote}
%   |unzip zref.tds.zip -d ~/texmf|
% \end{quote}
%
%
% \subsection{Package installation}
%
% \paragraph{Unpacking.} The \xfile{.dtx} file is a self-extracting
% \docstrip\ archive. The files are extracted by running the
% \xfile{.dtx} through \plainTeX:
% \begin{quote}
%   \verb|tex zref.dtx|
% \end{quote}
%
% \paragraph{TDS.} Now the different files must be moved into
% the different directories in your installation TDS tree
% (also known as \xfile{texmf} tree):
% \begin{quote}
% \def\t{^^A
% \begin{tabular}{@{}>{\ttfamily}l@{ $\rightarrow$ }>{\ttfamily}l@{}}
%   zref.sty & tex/latex/zref/zref.sty\\
%   zref-base.sty & tex/latex/zref/zref-base.sty\\
%   zref-abspage.sty & tex/latex/zref/zref-abspage.sty\\
%   zref-abspos.sty & tex/latex/zref/zref-abspos.sty\\
%   zref-counter.sty & tex/latex/zref/zref-counter.sty\\
%   zref-dotfill.sty & tex/latex/zref/zref-dotfill.sty\\
%   zref-env.sty & tex/latex/zref/zref-env.sty\\
%   zref-hyperref.sty & tex/latex/zref/zref-hyperref.sty\\
%   zref-lastpage.sty & tex/latex/zref/zref-lastpage.sty\\
%   zref-marks.sty & tex/latex/zref/zref-marks.sty\\
%   zref-nextpage.sty & tex/latex/zref/zref-nextpage.sty\\
%   zref-pageattr.sty & tex/latex/zref/zref-pageattr.sty\\
%   zref-pagelayout.sty & tex/latex/zref/zref-pagelayout.sty\\
%   zref-perpage.sty & tex/latex/zref/zref-perpage.sty\\
%   zref-runs.sty & tex/latex/zref/zref-runs.sty\\
%   zref-savepos.sty & tex/latex/zref/zref-savepos.sty\\
%   zref-thepage.sty & tex/latex/zref/zref-thepage.sty\\
%   zref-titleref.sty & tex/latex/zref/zref-titleref.sty\\
%   zref-totpages.sty & tex/latex/zref/zref-totpages.sty\\
%   zref-user.sty & tex/latex/zref/zref-user.sty\\
%   zref-xr.sty & tex/latex/zref/zref-xr.sty\\
%   zref.pdf & doc/latex/zref/zref.pdf\\
%   zref-example.tex & doc/latex/zref/zref-example.tex\\
%   zref-example-lastpage.tex & doc/latex/zref/zref-example-lastpage.tex\\
%   zref-example-nextpage.tex & doc/latex/zref/zref-example-nextpage.tex\\
%   zref.dtx & source/latex/zref/zref.dtx\\
% \end{tabular}^^A
% }^^A
% \sbox0{\t}^^A
% \ifdim\wd0>\linewidth
%   \begingroup
%     \advance\linewidth by\leftmargin
%     \advance\linewidth by\rightmargin
%   \edef\x{\endgroup
%     \def\noexpand\lw{\the\linewidth}^^A
%   }\x
%   \def\lwbox{^^A
%     \leavevmode
%     \hbox to \linewidth{^^A
%       \kern-\leftmargin\relax
%       \hss
%       \usebox0
%       \hss
%       \kern-\rightmargin\relax
%     }^^A
%   }^^A
%   \ifdim\wd0>\lw
%     \sbox0{\small\t}^^A
%     \ifdim\wd0>\linewidth
%       \ifdim\wd0>\lw
%         \sbox0{\footnotesize\t}^^A
%         \ifdim\wd0>\linewidth
%           \ifdim\wd0>\lw
%             \sbox0{\scriptsize\t}^^A
%             \ifdim\wd0>\linewidth
%               \ifdim\wd0>\lw
%                 \sbox0{\tiny\t}^^A
%                 \ifdim\wd0>\linewidth
%                   \lwbox
%                 \else
%                   \usebox0
%                 \fi
%               \else
%                 \lwbox
%               \fi
%             \else
%               \usebox0
%             \fi
%           \else
%             \lwbox
%           \fi
%         \else
%           \usebox0
%         \fi
%       \else
%         \lwbox
%       \fi
%     \else
%       \usebox0
%     \fi
%   \else
%     \lwbox
%   \fi
% \else
%   \usebox0
% \fi
% \end{quote}
% If you have a \xfile{docstrip.cfg} that configures and enables \docstrip's
% TDS installing feature, then some files can already be in the right
% place, see the documentation of \docstrip.
%
% \subsection{Refresh file name databases}
%
% If your \TeX~distribution
% (\TeX\,Live, \mikTeX, \dots) relies on file name databases, you must refresh
% these. For example, \TeX\,Live\ users run \verb|texhash| or
% \verb|mktexlsr|.
%
% \subsection{Some details for the interested}
%
% \paragraph{Unpacking with \LaTeX.}
% The \xfile{.dtx} chooses its action depending on the format:
% \begin{description}
% \item[\plainTeX:] Run \docstrip\ and extract the files.
% \item[\LaTeX:] Generate the documentation.
% \end{description}
% If you insist on using \LaTeX\ for \docstrip\ (really,
% \docstrip\ does not need \LaTeX), then inform the autodetect routine
% about your intention:
% \begin{quote}
%   \verb|latex \let\install=y\input{zref.dtx}|
% \end{quote}
% Do not forget to quote the argument according to the demands
% of your shell.
%
% \paragraph{Generating the documentation.}
% You can use both the \xfile{.dtx} or the \xfile{.drv} to generate
% the documentation. The process can be configured by the
% configuration file \xfile{ltxdoc.cfg}. For instance, put this
% line into this file, if you want to have A4 as paper format:
% \begin{quote}
%   \verb|\PassOptionsToClass{a4paper}{article}|
% \end{quote}
% An example follows how to generate the
% documentation with pdf\LaTeX:
% \begin{quote}
%\begin{verbatim}
%pdflatex zref.dtx
%makeindex -s gind.ist zref.idx
%pdflatex zref.dtx
%makeindex -s gind.ist zref.idx
%pdflatex zref.dtx
%\end{verbatim}
% \end{quote}
%
% \begin{thebibliography}{99}
% \bibpackage{footmisc}{Robin Fairbairns}{2004/01/23 v5.3a}
%   {\CTANpkg{footmisc}}
% \bibpackage{hyperref}{Sebastian Rahtz, Heiko Oberdiek}{2006/08/16 v6.75c}
%   {\CTANpkg{hyperref}}
% \bibpackage{lastpage}{Jeff Goldberg}{1994/06/25 v0.1b}
%   {\CTANpkg{lastpage}}
% \bibpackage{nameref}{Sebastian Rahtz, Heiko Oberdiek}{2006/02/12 v2.24}
%   {\CTANpkg{nameref}}
% \bibpackage{perpage}{David Kastrup}{2002/12/20 v1.0}
%   {\CTANpkg{perpage}}
% \bibpackage{titleref}{Donald Arsenau}{2001/04/05 v3.1}
%   {\CTANpkg{titleref}}
% \bibpackage{totpages}{Wilhelm M\"uller}{1999/07/14 v1.00}
%   {\CTANpkg{totpages}}
% \bibpackage{xr}{David Carlisle}{1994/05/28 v5.02}
%   {\CTANpkg{xr}}
% \bibpackage{xr-hyper}{David Carlisle}{2000/03/22 v6.00beta4}
%   {\CTANpkg{xr-hyper}}
% \end{thebibliography}
%
% \begin{History}
%   \begin{Version}{2006/02/20 v1.0}
%   \item
%     First version.
%   \end{Version}
%   \begin{Version}{2006/05/03 v1.1}
%   \item
%     Module \xmodule{perpage} added.
%   \item
%     Module redesign as packages.
%   \end{Version}
%   \begin{Version}{2006/05/25 v1.2}
%   \item
%     Module \xmodule{dotfillmin} added.
%   \item
%     Module \xmodule{base}: macros \cs{zref@require@uniqe}
%     and \cs{thezref@unique} added
%     (used by modules \xmodule{titleref} and \xmodule{dotfillmin}).
%   \end{Version}
%   \begin{Version}{2006/09/08 v1.3}
%   \item
%     Typo fixes and English cleanup by Per Starback.
%   \end{Version}
%   \begin{Version}{2007/01/23 v1.4}
%   \item
%     Typo in macro name fixed in documentation.
%   \end{Version}
%   \begin{Version}{2007/02/18 v1.5}
%   \item
%     \cs{zref@getcurrent} added (suggestion of Igor Akkerman).
%   \item
%     Module \xmodule{savepos} also supports \hologo{XeTeX}.
%   \end{Version}
%   \begin{Version}{2007/04/06 v1.6}
%   \item
%     Fix in modules \xmodule{abspage} and \xmodule{base}:
%     Now counter |abspage| and |zref@unique|
%     are not remembered by \cs{include}.
%   \item
%     Beamer support for module \xmodule{titleref}.
%   \end{Version}
%   \begin{Version}{2007/04/17 v1.7}
%   \item
%     Package \xpackage{atbegshi} replaces \xpackage{everyshi}.
%   \end{Version}
%   \begin{Version}{2007/04/22 v1.8}
%   \item
%     \cs{zref@wrapper@babel} and \cs{zref@refused} are now expandable
%     if babel is not used or \cs{if@safe@actives} is already set to true.
%     (Feature request of Josselin Noirel)
%   \end{Version}
%   \begin{Version}{2007/05/02 v1.9}
%   \item
%     Module \xmodule{titleref}: Some support for \cs{caption} of package
%     \xpackage{longtable}, but only if \cs{label} is given after
%     \cs{caption}.
%   \end{Version}
%   \begin{Version}{2007/05/06 v2.0}
%   \item
%     Uses package \xpackage{etexcmds} for accessing \hologo{eTeX}'s \cs{unexpanded}.
%   \end{Version}
%   \begin{Version}{2007/05/28 v2.1}
%   \item
%     Module \xmodule{titleref} supports caption of package \xpackage{listings}.
%   \item
%     Fixes in module \xmodule{titleref} for support
%     of packages \xpackage{titlesec} and \xpackage{longtable}.
%   \end{Version}
%   \begin{Version}{2008/09/21 v2.2}
%   \item
%     Module \xmodule{base}: \cs{zref@iflistcontainsprop} is documented, but
%     a broken \cs{zref@listcontainsprop} implemented. Name and
%     implementation fixed (thanks Ohad Kammar).
%   \end{Version}
%   \begin{Version}{2008/10/01 v2.3}
%   \item
%     \cs{zref@localaddprop} added (feature request of Ohad Kammar).
%   \item
%     Module \xmodule{lastpage}: list `LastPage' added. Label `LastPage' will use
%     the properties of this list (default is empty) along with the
%     properties of the main list.
%   \end{Version}
%   \begin{Version}{2009/08/07 v2.4}
%   \item
%     Module \xmodule{runs} added.
%   \end{Version}
%   \begin{Version}{2009/12/06 v2.5}
%   \item
%     Module \xmodule{lastpage}: Uses package \xpackage{atveryend}.
%   \item
%     Module \xmodule{titleref}: Further commands are disabled
%     during string expansion, imported from package \xpackage{nameref}.
%   \end{Version}
%   \begin{Version}{2009/12/07 v2.6}
%   \item
%     Version date added for package \xpackage{atveryend}.
%   \end{Version}
%   \begin{Version}{2009/12/08 v2.7}
%   \item
%     Module \xmodule{titleref}: Use of package \xpackage{gettitlestring}.
%   \end{Version}
%   \begin{Version}{2010/03/26 v2.8}
%   \item
%     \cs{zifrefundefined} added.
%   \item
%     Module \xmodule{lastpage}: Macros \cs{zref@iflastpage}
%     and \cs{ziflastpage} added.
%   \item
%     Module \xmodule{thepage} added.
%   \item
%     Module \xmodule{nextpage} added.
%   \end{Version}
%   \begin{Version}{2010/03/29 v2.9}
%   \item
%     Module \xmodule{marks} added (without documentation).
%   \item
%     \cs{zref@addprop} now adds expanded property to list.
%   \item
%     Useless \cs{ZREF@ErrorNoLine} removed.
%   \end{Version}
%   \begin{Version}{2010/04/08 v2.10}
%   \item
%     Module \xmodule{xr} remembers the external document name
%     in property `externaldocument'.
%   \end{Version}
%   \begin{Version}{2010/04/15 v2.11}
%   \item
%     Module \xmodule{titleref}: Better support of class \xclass{memoir}.
%   \item
%     Module \xmodule{titleref}: Support of theorems.
%   \end{Version}
%   \begin{Version}{2010/04/17 v2.12}
%   \item
%     Module \xmodule{base}: \cs{zref@newprop} ensures global
%     empty default.
%   \item
%     Module \xmodule{xr}: Setup options \xoption{tozreflabel}
%     and \xoption{toltxlabel} added.
%   \end{Version}
%   \begin{Version}{2010/04/19 v2.13}
%   \item
%     \cs{zref@setcurrent} throws an error if the property
%     does not exist (Florent Chervet).
%   \item
%     \cs{zref@getcurrent} the documentation is fixed (Florent Chervet).
%     Also it returns the empty string in case of errors.
%   \item
%     \cs{zref@addprop} and \cs{zref@localaddprop} now take a list of
%     property names (feature request of Florent Chervet).
%   \item
%     Example for \cs{zref@wrapper@unexpanded} corrected (Florent Chervet).
%   \end{Version}
%   \begin{Version}{2010/04/22 v2.14}
%   \item
%     Bug fix for \cs{zref@getcurrent} second argument wasn't eaten in
%     case of unknown property.
%   \item
%     \cs{zref@getcurrent} supports \cs{zref@wrapper@unexpanded}.
%   \item
%     \cs{zref@wrapper@unexpanded} added for \cs{ZREF@xr@tolabel}.
%   \item
%     \cs{zref@extract}, \cs{zref@extractdefault}, \cs{zref@getcurrent}
%     are expandable in exact two steps
%     except inside \cs{zref@wrapper@unexpanded}.
%   \end{Version}
%   \begin{Version}{2010/04/23 v2.15}
%   \item
%     \cs{zexternaldocument} fixed for property `url' when
%     importing \cs{new@label} (bug found by Victor Ivrii).
%   \item
%     Two expansion steps also in \cs{zref@wrapper@unexpanded}.
%   \item
%     Nested calls of \cs{zref@wrapper@unexpanded} possible.
%   \end{Version}
%   \begin{Version}{2010/04/28 v2.16}
%   \item
%     More consequent use of package `ltxcmds' and `hologo'.
%   \item
%     Module \xmodule{pagelayout} added.
%   \item
%     Module \xmodule{pageattr} added.
%   \item
%     Robustness introduced for non-expandable interface macros.
%   \item
%     Internal change of the data format of property lists
%     (suggestion of Florent Chervet).
%   \item
%     Module \xmodule{titleref}: Support of environment \textsf{description}.
%   \end{Version}
%   \begin{Version}{2010/05/01 v2.17}
%   \item
%     \cs{zref@newprop} throws an error if the property already exists.
%   \item
%     Module \xmodule{xr}: Bug fix for the case of several \xext{aux} files
%     (bug found by Victor Ivrii).
%   \item
%     Module \xmodule{xr}: Property `urluse' and option \xoption{urluse} added.
%   \end{Version}
%   \begin{Version}{2010/05/13 v2.18}
%   \item
%     Module \xmodule{env} added.
%   \item
%     Module \xmodule{savepos}: \cs{zref@savepos} added.
%   \end{Version}
%   \begin{Version}{2010/10/22 v2.19}
%   \item
%     \cs{zref@addprop} and \cs{zref@localaddprop} are limited
%     to one property only
%     (incompatibility to versions v2.13 to v2.18).
%   \item
%     \cs{zref@addprops} and \cs{zref@localaddprops} added.
%   \item
%     \cs{zref@delprop} and \cs{zref@localdelprop} added.
%   \item
%     \cs{zref@labelbykv} and \cs{zkvlabel} (module \xmodule{user})
%     with keys \xoption{prop}, \xoption{list}, \xoption{delprop},
%     \xoption{immediate}, \xoption{values} added.
%   \end{Version}
%   \begin{Version}{2011/02/12 v2.20}
%   \item
%     Fix for warning in \xpackage{zref-xr}.
%   \end{Version}
%   \begin{Version}{2011/03/18 v2.21}
%   \item
%     Fix in module \xmodule{pagelayout} for \cs{zlistpagelayout}.
%   \item
%     Fix for \cs{zref@localaddprop} (probably since v2.19).
%   \end{Version}
%   \begin{Version}{2011/10/05 v2.22}
%   \item
%     Documentation fixed for \cs{zref@(local)addprop(s)}.
%   \item
%     Module \xmodule{base}: \cs{zref@def@extract}, \cs{zref@def@extractdefault}
%     added.
%   \item
%     Fix in module \xmodule{pagelayout}: Because of missing \cs{noexpand}
%     commands the values of the pagelayout properties on all pages were
%     the values at package loading.
%   \item
%     Module \xmodule{base}: \cs{zref@showprop} added.
%   \end{Version}
%   \begin{Version}{2011/12/05 v2.23}
%   \item
%     Module \xmodule{savepos}: \cs{zsaveposx} and \cs{zsaveposy} added.
%   \end{Version}
%   \begin{Version}{2012/04/04 v2.24}
%   \item
%     Module \xmodule{titleref}, package \xpackage{titlesec}:
%     some support for class `straight' (\cs{ttl@straight@i}) added.
%   \end{Version}
%   \begin{Version}{2016/05/16 v2.25}
%   \item
%     Documentation updates.
%   \end{Version}
%   \begin{Version}{2016/05/21 v2.26}
%   \item
%     update zref-savepos for new luatex
%   \end{Version}
%   \begin{Version}{2018/11/21 v2.27}
%   \item
%     adapted zref-perpage, see issue https://github.com/ho-tex/zref/issues/2
%   \end{Version}
%   \begin{Version}{2019/11/29 v2.28}
%   \item
%     Documentation updates.
%   \item
%     Use \xpackage{iftex} directly.
%   \end{Version}
%   \begin{Version}{2020-03-03 v2.29}
%   \item
%     adapted in module zref-pagelayout the properties pdfhorigin, pdfvorigin,
%     pdfpagewidth, pdfpageheight for luatex to the right primitives.
%   \item
%     Removed no longer needed code for older lualatex versions.
%   \item added some documentation of the abspos module.
%   \item adapted the abspos module to the new luatex primitives.
%   \item adapted pageattr module to the new luatex primitives.
%   \item added short documentation for pageattr module
%   \item use luatex command names directly in zref-savepos rather than defining pdftex compatibility names.
%   \item allow zref-abspos to use \verb|\pdf[vh]origin| in dvi mode.
%   \end{Version}
%   \begin{Version}{2020-03-04 v2.30}
%   \item
%     add pagevalue property to savepos in the abspos module (issue 1)
%   \end{Version}
%   \begin{Version}{2020-05-28 v2.31}
%   \item Adapted module zref-counter to use \cs{@currentcounter} in the next
%   \LaTeX version (issue 5)
%   \end{Version}
%   \begin{Version}{2020-07-03 v2.32}
%   \item Changed in zref-pagelayout the names of the shipout box dimensions
%    to adapt to the new hook management.
%   \end{Version}
% \end{History}
%
% \PrintIndex
%
% \Finale
\endinput
|
% \end{quote}
% Do not forget to quote the argument according to the demands
% of your shell.
%
% \paragraph{Generating the documentation.}
% You can use both the \xfile{.dtx} or the \xfile{.drv} to generate
% the documentation. The process can be configured by the
% configuration file \xfile{ltxdoc.cfg}. For instance, put this
% line into this file, if you want to have A4 as paper format:
% \begin{quote}
%   \verb|\PassOptionsToClass{a4paper}{article}|
% \end{quote}
% An example follows how to generate the
% documentation with pdf\LaTeX:
% \begin{quote}
%\begin{verbatim}
%pdflatex zref.dtx
%makeindex -s gind.ist zref.idx
%pdflatex zref.dtx
%makeindex -s gind.ist zref.idx
%pdflatex zref.dtx
%\end{verbatim}
% \end{quote}
%
% \begin{thebibliography}{99}
% \bibpackage{footmisc}{Robin Fairbairns}{2004/01/23 v5.3a}
%   {\CTANpkg{footmisc}}
% \bibpackage{hyperref}{Sebastian Rahtz, Heiko Oberdiek}{2006/08/16 v6.75c}
%   {\CTANpkg{hyperref}}
% \bibpackage{lastpage}{Jeff Goldberg}{1994/06/25 v0.1b}
%   {\CTANpkg{lastpage}}
% \bibpackage{nameref}{Sebastian Rahtz, Heiko Oberdiek}{2006/02/12 v2.24}
%   {\CTANpkg{nameref}}
% \bibpackage{perpage}{David Kastrup}{2002/12/20 v1.0}
%   {\CTANpkg{perpage}}
% \bibpackage{titleref}{Donald Arsenau}{2001/04/05 v3.1}
%   {\CTANpkg{titleref}}
% \bibpackage{totpages}{Wilhelm M\"uller}{1999/07/14 v1.00}
%   {\CTANpkg{totpages}}
% \bibpackage{xr}{David Carlisle}{1994/05/28 v5.02}
%   {\CTANpkg{xr}}
% \bibpackage{xr-hyper}{David Carlisle}{2000/03/22 v6.00beta4}
%   {\CTANpkg{xr-hyper}}
% \end{thebibliography}
%
% \begin{History}
%   \begin{Version}{2006/02/20 v1.0}
%   \item
%     First version.
%   \end{Version}
%   \begin{Version}{2006/05/03 v1.1}
%   \item
%     Module \xmodule{perpage} added.
%   \item
%     Module redesign as packages.
%   \end{Version}
%   \begin{Version}{2006/05/25 v1.2}
%   \item
%     Module \xmodule{dotfillmin} added.
%   \item
%     Module \xmodule{base}: macros \cs{zref@require@uniqe}
%     and \cs{thezref@unique} added
%     (used by modules \xmodule{titleref} and \xmodule{dotfillmin}).
%   \end{Version}
%   \begin{Version}{2006/09/08 v1.3}
%   \item
%     Typo fixes and English cleanup by Per Starback.
%   \end{Version}
%   \begin{Version}{2007/01/23 v1.4}
%   \item
%     Typo in macro name fixed in documentation.
%   \end{Version}
%   \begin{Version}{2007/02/18 v1.5}
%   \item
%     \cs{zref@getcurrent} added (suggestion of Igor Akkerman).
%   \item
%     Module \xmodule{savepos} also supports \hologo{XeTeX}.
%   \end{Version}
%   \begin{Version}{2007/04/06 v1.6}
%   \item
%     Fix in modules \xmodule{abspage} and \xmodule{base}:
%     Now counter |abspage| and |zref@unique|
%     are not remembered by \cs{include}.
%   \item
%     Beamer support for module \xmodule{titleref}.
%   \end{Version}
%   \begin{Version}{2007/04/17 v1.7}
%   \item
%     Package \xpackage{atbegshi} replaces \xpackage{everyshi}.
%   \end{Version}
%   \begin{Version}{2007/04/22 v1.8}
%   \item
%     \cs{zref@wrapper@babel} and \cs{zref@refused} are now expandable
%     if babel is not used or \cs{if@safe@actives} is already set to true.
%     (Feature request of Josselin Noirel)
%   \end{Version}
%   \begin{Version}{2007/05/02 v1.9}
%   \item
%     Module \xmodule{titleref}: Some support for \cs{caption} of package
%     \xpackage{longtable}, but only if \cs{label} is given after
%     \cs{caption}.
%   \end{Version}
%   \begin{Version}{2007/05/06 v2.0}
%   \item
%     Uses package \xpackage{etexcmds} for accessing \hologo{eTeX}'s \cs{unexpanded}.
%   \end{Version}
%   \begin{Version}{2007/05/28 v2.1}
%   \item
%     Module \xmodule{titleref} supports caption of package \xpackage{listings}.
%   \item
%     Fixes in module \xmodule{titleref} for support
%     of packages \xpackage{titlesec} and \xpackage{longtable}.
%   \end{Version}
%   \begin{Version}{2008/09/21 v2.2}
%   \item
%     Module \xmodule{base}: \cs{zref@iflistcontainsprop} is documented, but
%     a broken \cs{zref@listcontainsprop} implemented. Name and
%     implementation fixed (thanks Ohad Kammar).
%   \end{Version}
%   \begin{Version}{2008/10/01 v2.3}
%   \item
%     \cs{zref@localaddprop} added (feature request of Ohad Kammar).
%   \item
%     Module \xmodule{lastpage}: list `LastPage' added. Label `LastPage' will use
%     the properties of this list (default is empty) along with the
%     properties of the main list.
%   \end{Version}
%   \begin{Version}{2009/08/07 v2.4}
%   \item
%     Module \xmodule{runs} added.
%   \end{Version}
%   \begin{Version}{2009/12/06 v2.5}
%   \item
%     Module \xmodule{lastpage}: Uses package \xpackage{atveryend}.
%   \item
%     Module \xmodule{titleref}: Further commands are disabled
%     during string expansion, imported from package \xpackage{nameref}.
%   \end{Version}
%   \begin{Version}{2009/12/07 v2.6}
%   \item
%     Version date added for package \xpackage{atveryend}.
%   \end{Version}
%   \begin{Version}{2009/12/08 v2.7}
%   \item
%     Module \xmodule{titleref}: Use of package \xpackage{gettitlestring}.
%   \end{Version}
%   \begin{Version}{2010/03/26 v2.8}
%   \item
%     \cs{zifrefundefined} added.
%   \item
%     Module \xmodule{lastpage}: Macros \cs{zref@iflastpage}
%     and \cs{ziflastpage} added.
%   \item
%     Module \xmodule{thepage} added.
%   \item
%     Module \xmodule{nextpage} added.
%   \end{Version}
%   \begin{Version}{2010/03/29 v2.9}
%   \item
%     Module \xmodule{marks} added (without documentation).
%   \item
%     \cs{zref@addprop} now adds expanded property to list.
%   \item
%     Useless \cs{ZREF@ErrorNoLine} removed.
%   \end{Version}
%   \begin{Version}{2010/04/08 v2.10}
%   \item
%     Module \xmodule{xr} remembers the external document name
%     in property `externaldocument'.
%   \end{Version}
%   \begin{Version}{2010/04/15 v2.11}
%   \item
%     Module \xmodule{titleref}: Better support of class \xclass{memoir}.
%   \item
%     Module \xmodule{titleref}: Support of theorems.
%   \end{Version}
%   \begin{Version}{2010/04/17 v2.12}
%   \item
%     Module \xmodule{base}: \cs{zref@newprop} ensures global
%     empty default.
%   \item
%     Module \xmodule{xr}: Setup options \xoption{tozreflabel}
%     and \xoption{toltxlabel} added.
%   \end{Version}
%   \begin{Version}{2010/04/19 v2.13}
%   \item
%     \cs{zref@setcurrent} throws an error if the property
%     does not exist (Florent Chervet).
%   \item
%     \cs{zref@getcurrent} the documentation is fixed (Florent Chervet).
%     Also it returns the empty string in case of errors.
%   \item
%     \cs{zref@addprop} and \cs{zref@localaddprop} now take a list of
%     property names (feature request of Florent Chervet).
%   \item
%     Example for \cs{zref@wrapper@unexpanded} corrected (Florent Chervet).
%   \end{Version}
%   \begin{Version}{2010/04/22 v2.14}
%   \item
%     Bug fix for \cs{zref@getcurrent} second argument wasn't eaten in
%     case of unknown property.
%   \item
%     \cs{zref@getcurrent} supports \cs{zref@wrapper@unexpanded}.
%   \item
%     \cs{zref@wrapper@unexpanded} added for \cs{ZREF@xr@tolabel}.
%   \item
%     \cs{zref@extract}, \cs{zref@extractdefault}, \cs{zref@getcurrent}
%     are expandable in exact two steps
%     except inside \cs{zref@wrapper@unexpanded}.
%   \end{Version}
%   \begin{Version}{2010/04/23 v2.15}
%   \item
%     \cs{zexternaldocument} fixed for property `url' when
%     importing \cs{new@label} (bug found by Victor Ivrii).
%   \item
%     Two expansion steps also in \cs{zref@wrapper@unexpanded}.
%   \item
%     Nested calls of \cs{zref@wrapper@unexpanded} possible.
%   \end{Version}
%   \begin{Version}{2010/04/28 v2.16}
%   \item
%     More consequent use of package `ltxcmds' and `hologo'.
%   \item
%     Module \xmodule{pagelayout} added.
%   \item
%     Module \xmodule{pageattr} added.
%   \item
%     Robustness introduced for non-expandable interface macros.
%   \item
%     Internal change of the data format of property lists
%     (suggestion of Florent Chervet).
%   \item
%     Module \xmodule{titleref}: Support of environment \textsf{description}.
%   \end{Version}
%   \begin{Version}{2010/05/01 v2.17}
%   \item
%     \cs{zref@newprop} throws an error if the property already exists.
%   \item
%     Module \xmodule{xr}: Bug fix for the case of several \xext{aux} files
%     (bug found by Victor Ivrii).
%   \item
%     Module \xmodule{xr}: Property `urluse' and option \xoption{urluse} added.
%   \end{Version}
%   \begin{Version}{2010/05/13 v2.18}
%   \item
%     Module \xmodule{env} added.
%   \item
%     Module \xmodule{savepos}: \cs{zref@savepos} added.
%   \end{Version}
%   \begin{Version}{2010/10/22 v2.19}
%   \item
%     \cs{zref@addprop} and \cs{zref@localaddprop} are limited
%     to one property only
%     (incompatibility to versions v2.13 to v2.18).
%   \item
%     \cs{zref@addprops} and \cs{zref@localaddprops} added.
%   \item
%     \cs{zref@delprop} and \cs{zref@localdelprop} added.
%   \item
%     \cs{zref@labelbykv} and \cs{zkvlabel} (module \xmodule{user})
%     with keys \xoption{prop}, \xoption{list}, \xoption{delprop},
%     \xoption{immediate}, \xoption{values} added.
%   \end{Version}
%   \begin{Version}{2011/02/12 v2.20}
%   \item
%     Fix for warning in \xpackage{zref-xr}.
%   \end{Version}
%   \begin{Version}{2011/03/18 v2.21}
%   \item
%     Fix in module \xmodule{pagelayout} for \cs{zlistpagelayout}.
%   \item
%     Fix for \cs{zref@localaddprop} (probably since v2.19).
%   \end{Version}
%   \begin{Version}{2011/10/05 v2.22}
%   \item
%     Documentation fixed for \cs{zref@(local)addprop(s)}.
%   \item
%     Module \xmodule{base}: \cs{zref@def@extract}, \cs{zref@def@extractdefault}
%     added.
%   \item
%     Fix in module \xmodule{pagelayout}: Because of missing \cs{noexpand}
%     commands the values of the pagelayout properties on all pages were
%     the values at package loading.
%   \item
%     Module \xmodule{base}: \cs{zref@showprop} added.
%   \end{Version}
%   \begin{Version}{2011/12/05 v2.23}
%   \item
%     Module \xmodule{savepos}: \cs{zsaveposx} and \cs{zsaveposy} added.
%   \end{Version}
%   \begin{Version}{2012/04/04 v2.24}
%   \item
%     Module \xmodule{titleref}, package \xpackage{titlesec}:
%     some support for class `straight' (\cs{ttl@straight@i}) added.
%   \end{Version}
%   \begin{Version}{2016/05/16 v2.25}
%   \item
%     Documentation updates.
%   \end{Version}
%   \begin{Version}{2016/05/21 v2.26}
%   \item
%     update zref-savepos for new luatex
%   \end{Version}
%   \begin{Version}{2018/11/21 v2.27}
%   \item
%     adapted zref-perpage, see issue https://github.com/ho-tex/zref/issues/2
%   \end{Version}
%   \begin{Version}{2019/11/29 v2.28}
%   \item
%     Documentation updates.
%   \item
%     Use \xpackage{iftex} directly.
%   \end{Version}
%   \begin{Version}{2020-03-03 v2.29}
%   \item
%     adapted in module zref-pagelayout the properties pdfhorigin, pdfvorigin,
%     pdfpagewidth, pdfpageheight for luatex to the right primitives.
%   \item
%     Removed no longer needed code for older lualatex versions.
%   \item added some documentation of the abspos module.
%   \item adapted the abspos module to the new luatex primitives.
%   \item adapted pageattr module to the new luatex primitives.
%   \item added short documentation for pageattr module
%   \item use luatex command names directly in zref-savepos rather than defining pdftex compatibility names.
%   \item allow zref-abspos to use \verb|\pdf[vh]origin| in dvi mode.
%   \end{Version}
%   \begin{Version}{2020-03-04 v2.30}
%   \item
%     add pagevalue property to savepos in the abspos module (issue 1)
%   \end{Version}
%   \begin{Version}{2020-05-28 v2.31}
%   \item Adapted module zref-counter to use \cs{@currentcounter} in the next
%   \LaTeX version (issue 5)
%   \end{Version}
%   \begin{Version}{2020-07-03 v2.32}
%   \item Changed in zref-pagelayout the names of the shipout box dimensions
%    to adapt to the new hook management.
%   \end{Version}
% \end{History}
%
% \PrintIndex
%
% \Finale
\endinput
|
% \end{quote}
% Do not forget to quote the argument according to the demands
% of your shell.
%
% \paragraph{Generating the documentation.}
% You can use both the \xfile{.dtx} or the \xfile{.drv} to generate
% the documentation. The process can be configured by the
% configuration file \xfile{ltxdoc.cfg}. For instance, put this
% line into this file, if you want to have A4 as paper format:
% \begin{quote}
%   \verb|\PassOptionsToClass{a4paper}{article}|
% \end{quote}
% An example follows how to generate the
% documentation with pdf\LaTeX:
% \begin{quote}
%\begin{verbatim}
%pdflatex zref.dtx
%makeindex -s gind.ist zref.idx
%pdflatex zref.dtx
%makeindex -s gind.ist zref.idx
%pdflatex zref.dtx
%\end{verbatim}
% \end{quote}
%
% \begin{thebibliography}{99}
% \bibpackage{footmisc}{Robin Fairbairns}{2004/01/23 v5.3a}
%   {\CTANpkg{footmisc}}
% \bibpackage{hyperref}{Sebastian Rahtz, Heiko Oberdiek}{2006/08/16 v6.75c}
%   {\CTANpkg{hyperref}}
% \bibpackage{lastpage}{Jeff Goldberg}{1994/06/25 v0.1b}
%   {\CTANpkg{lastpage}}
% \bibpackage{nameref}{Sebastian Rahtz, Heiko Oberdiek}{2006/02/12 v2.24}
%   {\CTANpkg{nameref}}
% \bibpackage{perpage}{David Kastrup}{2002/12/20 v1.0}
%   {\CTANpkg{perpage}}
% \bibpackage{titleref}{Donald Arsenau}{2001/04/05 v3.1}
%   {\CTANpkg{titleref}}
% \bibpackage{totpages}{Wilhelm M\"uller}{1999/07/14 v1.00}
%   {\CTANpkg{totpages}}
% \bibpackage{xr}{David Carlisle}{1994/05/28 v5.02}
%   {\CTANpkg{xr}}
% \bibpackage{xr-hyper}{David Carlisle}{2000/03/22 v6.00beta4}
%   {\CTANpkg{xr-hyper}}
% \end{thebibliography}
%
% \begin{History}
%   \begin{Version}{2006/02/20 v1.0}
%   \item
%     First version.
%   \end{Version}
%   \begin{Version}{2006/05/03 v1.1}
%   \item
%     Module \xmodule{perpage} added.
%   \item
%     Module redesign as packages.
%   \end{Version}
%   \begin{Version}{2006/05/25 v1.2}
%   \item
%     Module \xmodule{dotfillmin} added.
%   \item
%     Module \xmodule{base}: macros \cs{zref@require@uniqe}
%     and \cs{thezref@unique} added
%     (used by modules \xmodule{titleref} and \xmodule{dotfillmin}).
%   \end{Version}
%   \begin{Version}{2006/09/08 v1.3}
%   \item
%     Typo fixes and English cleanup by Per Starback.
%   \end{Version}
%   \begin{Version}{2007/01/23 v1.4}
%   \item
%     Typo in macro name fixed in documentation.
%   \end{Version}
%   \begin{Version}{2007/02/18 v1.5}
%   \item
%     \cs{zref@getcurrent} added (suggestion of Igor Akkerman).
%   \item
%     Module \xmodule{savepos} also supports \hologo{XeTeX}.
%   \end{Version}
%   \begin{Version}{2007/04/06 v1.6}
%   \item
%     Fix in modules \xmodule{abspage} and \xmodule{base}:
%     Now counter |abspage| and |zref@unique|
%     are not remembered by \cs{include}.
%   \item
%     Beamer support for module \xmodule{titleref}.
%   \end{Version}
%   \begin{Version}{2007/04/17 v1.7}
%   \item
%     Package \xpackage{atbegshi} replaces \xpackage{everyshi}.
%   \end{Version}
%   \begin{Version}{2007/04/22 v1.8}
%   \item
%     \cs{zref@wrapper@babel} and \cs{zref@refused} are now expandable
%     if babel is not used or \cs{if@safe@actives} is already set to true.
%     (Feature request of Josselin Noirel)
%   \end{Version}
%   \begin{Version}{2007/05/02 v1.9}
%   \item
%     Module \xmodule{titleref}: Some support for \cs{caption} of package
%     \xpackage{longtable}, but only if \cs{label} is given after
%     \cs{caption}.
%   \end{Version}
%   \begin{Version}{2007/05/06 v2.0}
%   \item
%     Uses package \xpackage{etexcmds} for accessing \hologo{eTeX}'s \cs{unexpanded}.
%   \end{Version}
%   \begin{Version}{2007/05/28 v2.1}
%   \item
%     Module \xmodule{titleref} supports caption of package \xpackage{listings}.
%   \item
%     Fixes in module \xmodule{titleref} for support
%     of packages \xpackage{titlesec} and \xpackage{longtable}.
%   \end{Version}
%   \begin{Version}{2008/09/21 v2.2}
%   \item
%     Module \xmodule{base}: \cs{zref@iflistcontainsprop} is documented, but
%     a broken \cs{zref@listcontainsprop} implemented. Name and
%     implementation fixed (thanks Ohad Kammar).
%   \end{Version}
%   \begin{Version}{2008/10/01 v2.3}
%   \item
%     \cs{zref@localaddprop} added (feature request of Ohad Kammar).
%   \item
%     Module \xmodule{lastpage}: list `LastPage' added. Label `LastPage' will use
%     the properties of this list (default is empty) along with the
%     properties of the main list.
%   \end{Version}
%   \begin{Version}{2009/08/07 v2.4}
%   \item
%     Module \xmodule{runs} added.
%   \end{Version}
%   \begin{Version}{2009/12/06 v2.5}
%   \item
%     Module \xmodule{lastpage}: Uses package \xpackage{atveryend}.
%   \item
%     Module \xmodule{titleref}: Further commands are disabled
%     during string expansion, imported from package \xpackage{nameref}.
%   \end{Version}
%   \begin{Version}{2009/12/07 v2.6}
%   \item
%     Version date added for package \xpackage{atveryend}.
%   \end{Version}
%   \begin{Version}{2009/12/08 v2.7}
%   \item
%     Module \xmodule{titleref}: Use of package \xpackage{gettitlestring}.
%   \end{Version}
%   \begin{Version}{2010/03/26 v2.8}
%   \item
%     \cs{zifrefundefined} added.
%   \item
%     Module \xmodule{lastpage}: Macros \cs{zref@iflastpage}
%     and \cs{ziflastpage} added.
%   \item
%     Module \xmodule{thepage} added.
%   \item
%     Module \xmodule{nextpage} added.
%   \end{Version}
%   \begin{Version}{2010/03/29 v2.9}
%   \item
%     Module \xmodule{marks} added (without documentation).
%   \item
%     \cs{zref@addprop} now adds expanded property to list.
%   \item
%     Useless \cs{ZREF@ErrorNoLine} removed.
%   \end{Version}
%   \begin{Version}{2010/04/08 v2.10}
%   \item
%     Module \xmodule{xr} remembers the external document name
%     in property `externaldocument'.
%   \end{Version}
%   \begin{Version}{2010/04/15 v2.11}
%   \item
%     Module \xmodule{titleref}: Better support of class \xclass{memoir}.
%   \item
%     Module \xmodule{titleref}: Support of theorems.
%   \end{Version}
%   \begin{Version}{2010/04/17 v2.12}
%   \item
%     Module \xmodule{base}: \cs{zref@newprop} ensures global
%     empty default.
%   \item
%     Module \xmodule{xr}: Setup options \xoption{tozreflabel}
%     and \xoption{toltxlabel} added.
%   \end{Version}
%   \begin{Version}{2010/04/19 v2.13}
%   \item
%     \cs{zref@setcurrent} throws an error if the property
%     does not exist (Florent Chervet).
%   \item
%     \cs{zref@getcurrent} the documentation is fixed (Florent Chervet).
%     Also it returns the empty string in case of errors.
%   \item
%     \cs{zref@addprop} and \cs{zref@localaddprop} now take a list of
%     property names (feature request of Florent Chervet).
%   \item
%     Example for \cs{zref@wrapper@unexpanded} corrected (Florent Chervet).
%   \end{Version}
%   \begin{Version}{2010/04/22 v2.14}
%   \item
%     Bug fix for \cs{zref@getcurrent} second argument wasn't eaten in
%     case of unknown property.
%   \item
%     \cs{zref@getcurrent} supports \cs{zref@wrapper@unexpanded}.
%   \item
%     \cs{zref@wrapper@unexpanded} added for \cs{ZREF@xr@tolabel}.
%   \item
%     \cs{zref@extract}, \cs{zref@extractdefault}, \cs{zref@getcurrent}
%     are expandable in exact two steps
%     except inside \cs{zref@wrapper@unexpanded}.
%   \end{Version}
%   \begin{Version}{2010/04/23 v2.15}
%   \item
%     \cs{zexternaldocument} fixed for property `url' when
%     importing \cs{new@label} (bug found by Victor Ivrii).
%   \item
%     Two expansion steps also in \cs{zref@wrapper@unexpanded}.
%   \item
%     Nested calls of \cs{zref@wrapper@unexpanded} possible.
%   \end{Version}
%   \begin{Version}{2010/04/28 v2.16}
%   \item
%     More consequent use of package `ltxcmds' and `hologo'.
%   \item
%     Module \xmodule{pagelayout} added.
%   \item
%     Module \xmodule{pageattr} added.
%   \item
%     Robustness introduced for non-expandable interface macros.
%   \item
%     Internal change of the data format of property lists
%     (suggestion of Florent Chervet).
%   \item
%     Module \xmodule{titleref}: Support of environment \textsf{description}.
%   \end{Version}
%   \begin{Version}{2010/05/01 v2.17}
%   \item
%     \cs{zref@newprop} throws an error if the property already exists.
%   \item
%     Module \xmodule{xr}: Bug fix for the case of several \xext{aux} files
%     (bug found by Victor Ivrii).
%   \item
%     Module \xmodule{xr}: Property `urluse' and option \xoption{urluse} added.
%   \end{Version}
%   \begin{Version}{2010/05/13 v2.18}
%   \item
%     Module \xmodule{env} added.
%   \item
%     Module \xmodule{savepos}: \cs{zref@savepos} added.
%   \end{Version}
%   \begin{Version}{2010/10/22 v2.19}
%   \item
%     \cs{zref@addprop} and \cs{zref@localaddprop} are limited
%     to one property only
%     (incompatibility to versions v2.13 to v2.18).
%   \item
%     \cs{zref@addprops} and \cs{zref@localaddprops} added.
%   \item
%     \cs{zref@delprop} and \cs{zref@localdelprop} added.
%   \item
%     \cs{zref@labelbykv} and \cs{zkvlabel} (module \xmodule{user})
%     with keys \xoption{prop}, \xoption{list}, \xoption{delprop},
%     \xoption{immediate}, \xoption{values} added.
%   \end{Version}
%   \begin{Version}{2011/02/12 v2.20}
%   \item
%     Fix for warning in \xpackage{zref-xr}.
%   \end{Version}
%   \begin{Version}{2011/03/18 v2.21}
%   \item
%     Fix in module \xmodule{pagelayout} for \cs{zlistpagelayout}.
%   \item
%     Fix for \cs{zref@localaddprop} (probably since v2.19).
%   \end{Version}
%   \begin{Version}{2011/10/05 v2.22}
%   \item
%     Documentation fixed for \cs{zref@(local)addprop(s)}.
%   \item
%     Module \xmodule{base}: \cs{zref@def@extract}, \cs{zref@def@extractdefault}
%     added.
%   \item
%     Fix in module \xmodule{pagelayout}: Because of missing \cs{noexpand}
%     commands the values of the pagelayout properties on all pages were
%     the values at package loading.
%   \item
%     Module \xmodule{base}: \cs{zref@showprop} added.
%   \end{Version}
%   \begin{Version}{2011/12/05 v2.23}
%   \item
%     Module \xmodule{savepos}: \cs{zsaveposx} and \cs{zsaveposy} added.
%   \end{Version}
%   \begin{Version}{2012/04/04 v2.24}
%   \item
%     Module \xmodule{titleref}, package \xpackage{titlesec}:
%     some support for class `straight' (\cs{ttl@straight@i}) added.
%   \end{Version}
%   \begin{Version}{2016/05/16 v2.25}
%   \item
%     Documentation updates.
%   \end{Version}
%   \begin{Version}{2016/05/21 v2.26}
%   \item
%     update zref-savepos for new luatex
%   \end{Version}
%   \begin{Version}{2018/11/21 v2.27}
%   \item
%     adapted zref-perpage, see issue https://github.com/ho-tex/zref/issues/2
%   \end{Version}
%   \begin{Version}{2019/11/29 v2.28}
%   \item
%     Documentation updates.
%   \item
%     Use \xpackage{iftex} directly.
%   \end{Version}
%   \begin{Version}{2020-03-03 v2.29}
%   \item
%     adapted in module zref-pagelayout the properties pdfhorigin, pdfvorigin,
%     pdfpagewidth, pdfpageheight for luatex to the right primitives.
%   \item
%     Removed no longer needed code for older lualatex versions.
%   \item added some documentation of the abspos module.
%   \item adapted the abspos module to the new luatex primitives.
%   \item adapted pageattr module to the new luatex primitives.
%   \item added short documentation for pageattr module
%   \item use luatex command names directly in zref-savepos rather than defining pdftex compatibility names.
%   \item allow zref-abspos to use \verb|\pdf[vh]origin| in dvi mode.
%   \end{Version}
%   \begin{Version}{2020-03-04 v2.30}
%   \item
%     add pagevalue property to savepos in the abspos module (issue 1)
%   \end{Version}
%   \begin{Version}{2020-05-28 v2.31}
%   \item Adapted module zref-counter to use \cs{@currentcounter} in the next
%   \LaTeX version (issue 5)
%   \end{Version}
%   \begin{Version}{2020-07-03 v2.32}
%   \item Changed in zref-pagelayout the names of the shipout box dimensions
%    to adapt to the new hook management.
%   \end{Version}
% \end{History}
%
% \PrintIndex
%
% \Finale
\endinput
|
% \end{quote}
% Do not forget to quote the argument according to the demands
% of your shell.
%
% \paragraph{Generating the documentation.}
% You can use both the \xfile{.dtx} or the \xfile{.drv} to generate
% the documentation. The process can be configured by the
% configuration file \xfile{ltxdoc.cfg}. For instance, put this
% line into this file, if you want to have A4 as paper format:
% \begin{quote}
%   \verb|\PassOptionsToClass{a4paper}{article}|
% \end{quote}
% An example follows how to generate the
% documentation with pdf\LaTeX:
% \begin{quote}
%\begin{verbatim}
%pdflatex zref.dtx
%makeindex -s gind.ist zref.idx
%pdflatex zref.dtx
%makeindex -s gind.ist zref.idx
%pdflatex zref.dtx
%\end{verbatim}
% \end{quote}
%
% \begin{thebibliography}{99}
% \bibpackage{footmisc}{Robin Fairbairns}{2004/01/23 v5.3a}
%   {\CTANpkg{footmisc}}
% \bibpackage{hyperref}{Sebastian Rahtz, Heiko Oberdiek}{2006/08/16 v6.75c}
%   {\CTANpkg{hyperref}}
% \bibpackage{lastpage}{Jeff Goldberg}{1994/06/25 v0.1b}
%   {\CTANpkg{lastpage}}
% \bibpackage{nameref}{Sebastian Rahtz, Heiko Oberdiek}{2006/02/12 v2.24}
%   {\CTANpkg{nameref}}
% \bibpackage{perpage}{David Kastrup}{2002/12/20 v1.0}
%   {\CTANpkg{perpage}}
% \bibpackage{titleref}{Donald Arsenau}{2001/04/05 v3.1}
%   {\CTANpkg{titleref}}
% \bibpackage{totpages}{Wilhelm M\"uller}{1999/07/14 v1.00}
%   {\CTANpkg{totpages}}
% \bibpackage{xr}{David Carlisle}{1994/05/28 v5.02}
%   {\CTANpkg{xr}}
% \bibpackage{xr-hyper}{David Carlisle}{2000/03/22 v6.00beta4}
%   {\CTANpkg{xr-hyper}}
% \end{thebibliography}
%
% \begin{History}
%   \begin{Version}{2006/02/20 v1.0}
%   \item
%     First version.
%   \end{Version}
%   \begin{Version}{2006/05/03 v1.1}
%   \item
%     Module \xmodule{perpage} added.
%   \item
%     Module redesign as packages.
%   \end{Version}
%   \begin{Version}{2006/05/25 v1.2}
%   \item
%     Module \xmodule{dotfillmin} added.
%   \item
%     Module \xmodule{base}: macros \cs{zref@require@uniqe}
%     and \cs{thezref@unique} added
%     (used by modules \xmodule{titleref} and \xmodule{dotfillmin}).
%   \end{Version}
%   \begin{Version}{2006/09/08 v1.3}
%   \item
%     Typo fixes and English cleanup by Per Starback.
%   \end{Version}
%   \begin{Version}{2007/01/23 v1.4}
%   \item
%     Typo in macro name fixed in documentation.
%   \end{Version}
%   \begin{Version}{2007/02/18 v1.5}
%   \item
%     \cs{zref@getcurrent} added (suggestion of Igor Akkerman).
%   \item
%     Module \xmodule{savepos} also supports \hologo{XeTeX}.
%   \end{Version}
%   \begin{Version}{2007/04/06 v1.6}
%   \item
%     Fix in modules \xmodule{abspage} and \xmodule{base}:
%     Now counter |abspage| and |zref@unique|
%     are not remembered by \cs{include}.
%   \item
%     Beamer support for module \xmodule{titleref}.
%   \end{Version}
%   \begin{Version}{2007/04/17 v1.7}
%   \item
%     Package \xpackage{atbegshi} replaces \xpackage{everyshi}.
%   \end{Version}
%   \begin{Version}{2007/04/22 v1.8}
%   \item
%     \cs{zref@wrapper@babel} and \cs{zref@refused} are now expandable
%     if babel is not used or \cs{if@safe@actives} is already set to true.
%     (Feature request of Josselin Noirel)
%   \end{Version}
%   \begin{Version}{2007/05/02 v1.9}
%   \item
%     Module \xmodule{titleref}: Some support for \cs{caption} of package
%     \xpackage{longtable}, but only if \cs{label} is given after
%     \cs{caption}.
%   \end{Version}
%   \begin{Version}{2007/05/06 v2.0}
%   \item
%     Uses package \xpackage{etexcmds} for accessing \hologo{eTeX}'s \cs{unexpanded}.
%   \end{Version}
%   \begin{Version}{2007/05/28 v2.1}
%   \item
%     Module \xmodule{titleref} supports caption of package \xpackage{listings}.
%   \item
%     Fixes in module \xmodule{titleref} for support
%     of packages \xpackage{titlesec} and \xpackage{longtable}.
%   \end{Version}
%   \begin{Version}{2008/09/21 v2.2}
%   \item
%     Module \xmodule{base}: \cs{zref@iflistcontainsprop} is documented, but
%     a broken \cs{zref@listcontainsprop} implemented. Name and
%     implementation fixed (thanks Ohad Kammar).
%   \end{Version}
%   \begin{Version}{2008/10/01 v2.3}
%   \item
%     \cs{zref@localaddprop} added (feature request of Ohad Kammar).
%   \item
%     Module \xmodule{lastpage}: list `LastPage' added. Label `LastPage' will use
%     the properties of this list (default is empty) along with the
%     properties of the main list.
%   \end{Version}
%   \begin{Version}{2009/08/07 v2.4}
%   \item
%     Module \xmodule{runs} added.
%   \end{Version}
%   \begin{Version}{2009/12/06 v2.5}
%   \item
%     Module \xmodule{lastpage}: Uses package \xpackage{atveryend}.
%   \item
%     Module \xmodule{titleref}: Further commands are disabled
%     during string expansion, imported from package \xpackage{nameref}.
%   \end{Version}
%   \begin{Version}{2009/12/07 v2.6}
%   \item
%     Version date added for package \xpackage{atveryend}.
%   \end{Version}
%   \begin{Version}{2009/12/08 v2.7}
%   \item
%     Module \xmodule{titleref}: Use of package \xpackage{gettitlestring}.
%   \end{Version}
%   \begin{Version}{2010/03/26 v2.8}
%   \item
%     \cs{zifrefundefined} added.
%   \item
%     Module \xmodule{lastpage}: Macros \cs{zref@iflastpage}
%     and \cs{ziflastpage} added.
%   \item
%     Module \xmodule{thepage} added.
%   \item
%     Module \xmodule{nextpage} added.
%   \end{Version}
%   \begin{Version}{2010/03/29 v2.9}
%   \item
%     Module \xmodule{marks} added (without documentation).
%   \item
%     \cs{zref@addprop} now adds expanded property to list.
%   \item
%     Useless \cs{ZREF@ErrorNoLine} removed.
%   \end{Version}
%   \begin{Version}{2010/04/08 v2.10}
%   \item
%     Module \xmodule{xr} remembers the external document name
%     in property `externaldocument'.
%   \end{Version}
%   \begin{Version}{2010/04/15 v2.11}
%   \item
%     Module \xmodule{titleref}: Better support of class \xclass{memoir}.
%   \item
%     Module \xmodule{titleref}: Support of theorems.
%   \end{Version}
%   \begin{Version}{2010/04/17 v2.12}
%   \item
%     Module \xmodule{base}: \cs{zref@newprop} ensures global
%     empty default.
%   \item
%     Module \xmodule{xr}: Setup options \xoption{tozreflabel}
%     and \xoption{toltxlabel} added.
%   \end{Version}
%   \begin{Version}{2010/04/19 v2.13}
%   \item
%     \cs{zref@setcurrent} throws an error if the property
%     does not exist (Florent Chervet).
%   \item
%     \cs{zref@getcurrent} the documentation is fixed (Florent Chervet).
%     Also it returns the empty string in case of errors.
%   \item
%     \cs{zref@addprop} and \cs{zref@localaddprop} now take a list of
%     property names (feature request of Florent Chervet).
%   \item
%     Example for \cs{zref@wrapper@unexpanded} corrected (Florent Chervet).
%   \end{Version}
%   \begin{Version}{2010/04/22 v2.14}
%   \item
%     Bug fix for \cs{zref@getcurrent} second argument wasn't eaten in
%     case of unknown property.
%   \item
%     \cs{zref@getcurrent} supports \cs{zref@wrapper@unexpanded}.
%   \item
%     \cs{zref@wrapper@unexpanded} added for \cs{ZREF@xr@tolabel}.
%   \item
%     \cs{zref@extract}, \cs{zref@extractdefault}, \cs{zref@getcurrent}
%     are expandable in exact two steps
%     except inside \cs{zref@wrapper@unexpanded}.
%   \end{Version}
%   \begin{Version}{2010/04/23 v2.15}
%   \item
%     \cs{zexternaldocument} fixed for property `url' when
%     importing \cs{new@label} (bug found by Victor Ivrii).
%   \item
%     Two expansion steps also in \cs{zref@wrapper@unexpanded}.
%   \item
%     Nested calls of \cs{zref@wrapper@unexpanded} possible.
%   \end{Version}
%   \begin{Version}{2010/04/28 v2.16}
%   \item
%     More consequent use of package `ltxcmds' and `hologo'.
%   \item
%     Module \xmodule{pagelayout} added.
%   \item
%     Module \xmodule{pageattr} added.
%   \item
%     Robustness introduced for non-expandable interface macros.
%   \item
%     Internal change of the data format of property lists
%     (suggestion of Florent Chervet).
%   \item
%     Module \xmodule{titleref}: Support of environment \textsf{description}.
%   \end{Version}
%   \begin{Version}{2010/05/01 v2.17}
%   \item
%     \cs{zref@newprop} throws an error if the property already exists.
%   \item
%     Module \xmodule{xr}: Bug fix for the case of several \xext{aux} files
%     (bug found by Victor Ivrii).
%   \item
%     Module \xmodule{xr}: Property `urluse' and option \xoption{urluse} added.
%   \end{Version}
%   \begin{Version}{2010/05/13 v2.18}
%   \item
%     Module \xmodule{env} added.
%   \item
%     Module \xmodule{savepos}: \cs{zref@savepos} added.
%   \end{Version}
%   \begin{Version}{2010/10/22 v2.19}
%   \item
%     \cs{zref@addprop} and \cs{zref@localaddprop} are limited
%     to one property only
%     (incompatibility to versions v2.13 to v2.18).
%   \item
%     \cs{zref@addprops} and \cs{zref@localaddprops} added.
%   \item
%     \cs{zref@delprop} and \cs{zref@localdelprop} added.
%   \item
%     \cs{zref@labelbykv} and \cs{zkvlabel} (module \xmodule{user})
%     with keys \xoption{prop}, \xoption{list}, \xoption{delprop},
%     \xoption{immediate}, \xoption{values} added.
%   \end{Version}
%   \begin{Version}{2011/02/12 v2.20}
%   \item
%     Fix for warning in \xpackage{zref-xr}.
%   \end{Version}
%   \begin{Version}{2011/03/18 v2.21}
%   \item
%     Fix in module \xmodule{pagelayout} for \cs{zlistpagelayout}.
%   \item
%     Fix for \cs{zref@localaddprop} (probably since v2.19).
%   \end{Version}
%   \begin{Version}{2011/10/05 v2.22}
%   \item
%     Documentation fixed for \cs{zref@(local)addprop(s)}.
%   \item
%     Module \xmodule{base}: \cs{zref@def@extract}, \cs{zref@def@extractdefault}
%     added.
%   \item
%     Fix in module \xmodule{pagelayout}: Because of missing \cs{noexpand}
%     commands the values of the pagelayout properties on all pages were
%     the values at package loading.
%   \item
%     Module \xmodule{base}: \cs{zref@showprop} added.
%   \end{Version}
%   \begin{Version}{2011/12/05 v2.23}
%   \item
%     Module \xmodule{savepos}: \cs{zsaveposx} and \cs{zsaveposy} added.
%   \end{Version}
%   \begin{Version}{2012/04/04 v2.24}
%   \item
%     Module \xmodule{titleref}, package \xpackage{titlesec}:
%     some support for class `straight' (\cs{ttl@straight@i}) added.
%   \end{Version}
%   \begin{Version}{2016/05/16 v2.25}
%   \item
%     Documentation updates.
%   \end{Version}
%   \begin{Version}{2016/05/21 v2.26}
%   \item
%     update zref-savepos for new luatex
%   \end{Version}
%   \begin{Version}{2018/11/21 v2.27}
%   \item
%     adapted zref-perpage, see issue https://github.com/ho-tex/zref/issues/2
%   \end{Version}
%   \begin{Version}{2019/11/29 v2.28}
%   \item
%     Documentation updates.
%   \item
%     Use \xpackage{iftex} directly.
%   \end{Version}
%   \begin{Version}{2020-03-03 v2.29}
%   \item
%     adapted in module zref-pagelayout the properties pdfhorigin, pdfvorigin,
%     pdfpagewidth, pdfpageheight for luatex to the right primitives.
%   \item
%     Removed no longer needed code for older lualatex versions.
%   \item added some documentation of the abspos module.
%   \item adapted the abspos module to the new luatex primitives.
%   \item adapted pageattr module to the new luatex primitives.
%   \item added short documentation for pageattr module
%   \item use luatex command names directly in zref-savepos rather than defining pdftex compatibility names.
%   \item allow zref-abspos to use \verb|\pdf[vh]origin| in dvi mode.
%   \end{Version}
%   \begin{Version}{2020-03-04 v2.30}
%   \item
%     add pagevalue property to savepos in the abspos module (issue 1)
%   \end{Version}
%   \begin{Version}{2020-05-28 v2.31}
%   \item Adapted module zref-counter to use \cs{@currentcounter} in the next
%   \LaTeX version (issue 5)
%   \end{Version}
%   \begin{Version}{2020-07-03 v2.32}
%   \item Changed in zref-pagelayout the names of the shipout box dimensions
%    to adapt to the new hook management.
%   \end{Version}
% \end{History}
%
% \PrintIndex
%
% \Finale
\endinput
