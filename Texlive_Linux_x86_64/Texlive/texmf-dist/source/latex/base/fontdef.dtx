% \iffalse meta-comment
%
% Copyright (C) 1993-2020
% The LaTeX3 Project and any individual authors listed elsewhere
% in this file.
%
% This file is part of the LaTeX base system.
% -------------------------------------------
%
% It may be distributed and/or modified under the
% conditions of the LaTeX Project Public License, either version 1.3c
% of this license or (at your option) any later version.
% The latest version of this license is in
%    https://www.latex-project.org/lppl.txt
% and version 1.3c or later is part of all distributions of LaTeX
% version 2008 or later.
%
% This file has the LPPL maintenance status "maintained".
%
% The list of all files belonging to the LaTeX base distribution is
% given in the file `manifest.txt'. See also `legal.txt' for additional
% information.
%
% The list of derived (unpacked) files belonging to the distribution
% and covered by LPPL is defined by the unpacking scripts (with
% extension .ins) which are part of the distribution.
%
% \fi
% \iffalse
%%% From File: fontdef.dtx
%<*dtx>
           \ProvidesFile{fontdef.dtx}
%</dtx>
%<text,   >\ProvidesFile{fonttext.ltx}
%<math,   >\ProvidesFile{fontmath.ltx}
%<+cfgtext>\ProvidesFile{fonttext.cfg}
%<+cfgmath>\ProvidesFile{fontmath.cfg}
%<+cfgprel>\ProvidesFile{preload.cfg}
%<driver, >\ProvidesFile{fontdef.drv}
% \fi
%          \ProvidesFile{fontdef.dtx}
%<-latexrelease>           [2020/08/01 v3.0i LaTeX Kernel
% \iftrue  (\else
%<text,   >(Text
%<math,   >(Math
%<+cfgtext>(Uncustomised text
%<+cfgmath>(Uncustomised math
%<+cfgprel>(Uncustomised preload
% \fi
%<-latexrelease>           font setup)]
%
%
%\iffalse        This is a META comment
%
% File `fontdef.dtx'.
% Copyright (C) 1989-1999 Frank Mittelbach and Rainer Sch\"opf,
% All rights reserved.
%
%\fi
%
% \changes{v2.1a}{1993/12/01}{Update for LaTeX2e}
% \changes{v2.2a}{1994/10/14}{New coding}
% \changes{v2.2i}{1994/12/02}{Commented out \cs{ldots}. ASAJ.}
% \changes{v2.2j}{1995/05/11}{Updates to some plain macros}
% \changes{v2.2l}{1995/10/03}{\cs{@@sqrt} from patch file for /1701}
% \changes{v2.2o}{1996/05/17}{\cs{@@sqrt} removed, at last}
% \changes{v2.2p}{1996/11/20}{lowercase fd and enc.def file names /1044}
% \changes{v2.2x}{1999/01/05}{Need special protection for character
%     \texttt{\char62} in \cs{changes} entry.}
%
% \title{The \texttt{fontdef.dtx} file\thanks
%         {This file has version number \fileversion, dated \filedate}}
% \author{Frank Mittelbach \and Rainer Sch\"opf}
%
% \def\dst{{\normalfont\scshape docstrip}}
% \setcounter{StandardModuleDepth}{1}
%
%
% \MaintainedByLaTeXTeam{latex}
% \maketitle
%
% \section{Introduction}
%
% This file is used to generate the files \texttt{fonttext.ltx} (text
% font declarations) and \texttt{fontmath.ltx} (math font
% declarations), which are used during the format generation.  It
% contains the declaration of the standard text encodings used at the
% site as well as a minimal subset of font shape groups that NFSS will
% look at to ensure that the specified encodings are valid.
%
% The math part contains the setup for math encodings as well as the
% default math symbol declarations that belong to the encoding.
%
% It is possible to change this setup (by using other fonts, or
% defaults) without losing the ability to
% process documents written at other sites. Portability in this sense
% means that a document will compile without errors. It does not mean,
% however, that identical output will be produced. For this it is
% necessary that the distributed setup is used at both installations.
%
% \section{Customization}
%
% You are not allowed to change this source file!  If you want to
% change the default encodings and/or the font shape groups preloaded
% you should create a copy of \texttt{fonttext.ltx}
% under the name \texttt{fonttext.cfg} and change this copy. If
% \LaTeXe{} finds a file of this name it will use it, otherwise it
% uses the standard file which is \texttt{fontdef.ltx}.
%
% If you don't plan to use Computer Modern much or at all, it might
% (!)  be a good idea to make your own \texttt{fonttext.cfg}. Look at
% the comments below (docstrip module `text') to see what should
% should go into such a file.
%
% To change the math font setup use a copy of \texttt{fontmath.ltx}
% under the name \texttt{fontmath.cfg} and change this copy. However,
% dealing with this interface is even more a job for an expert than
% changing the text font setup --- in short, we don't encourage either.
%
% \begin{quote}
%   \textbf{Warning:} please note that we don't support customised
%   \LaTeX{} versions. Thus, before sending in a bug report please try
%   your test file with a \LaTeX{} format which is not customised and
%   send in the log from that version (unless the problem goes away).
% \end{quote}
%
% Please note: the following standard encodings  have to
% be defined in all local variants of \texttt{font....cfg} to guarantee
% that all \LaTeX{} installations behave in the same way.
% \begin{center}
% \begin{tabular}{ll}
%   |T1|      &    Cork \TeX{} text encoding \\
%   |OT1|      &   old \TeX{} text encoding \\
%   |U|        &   unknown encoding \\
%   |OML|      &   old \TeX{} math letters encoding \\
%   |OMS|      &   old \TeX{} math symbols encoding \\
%   |OMX|      &   old \TeX{} math extension symbols encoding\\
%   |TU|      &   Unicode
% \end{tabular}
% \end{center}
% Notice that some of these encodings are `old' in the sense that we
% hope that they will be superseded soon by encoding standards defined
% by the \TeX{} user community. Therefore this set of default encodings
% may change in the future.
%
% The first candidate is |OT1| which will soon be replaced by |T1|, the
% official \TeX{} text encoding.
%
% \begin{quote}\textbf{Warning:}
% If you add additional encodings to this file there is no guarantee
% any longer that files processable at your installation will also be
% processable at other installations. Thus, if you make use of
% such an encoding in your document, e.g.~if you intend to typeset in
% Cyrillic (|OT2| encoding), you need to specify this encoding in the
% preamble of your document prior to sending it to another
% installation. Once the encoding is specified in that place in your
% document, the document is processable at all \LaTeX{} installations
% (provided they have suitable fonts installed).
%
% For this reason we suggest that you define a short package file that
% sets up an additional encoding used at your site (rather than
% putting the encoding into this file) since this package can easily
% be shipped with your document.
% \end{quote}
%
%
% \StopEventually{}
%
% \section{The \texttt{docstrip} modules}
%
% The following modules are used to direct \texttt{docstrip} in
% generating external files:
% \begin{center}
% \begin{tabular}{ll}
%   driver & produce a documentation driver file \\
%   text   & produce the file \texttt{fonttext.ltx}\\
%   math   & produce the file \texttt{fontmath.ltx}\\
%   cfgtext   & produce a dummy \texttt{fonttext.cfg} file\\
%   cfgmath   & produce a dummy \texttt{fontmath.cfg} file\\
% \end{tabular}
% \end{center}
% A typical \texttt{docstrip} command file would then have entries like:
% \begin{verbatim}
%\generateFile{fonttext.ltx}{t}{\from{fontdef.dtx}{text}}
%\end{verbatim}
%
%
% \section{A driver for this document}
%
% The next bit of code contains the documentation driver file for
% \TeX{}, i.e.~the file that will produce the documentation you are
% currently reading. It will be extracted from this file by the
% \dst{} program.
%    \begin{macrocode}
%<*driver>
\documentclass{ltxdoc}
\GetFileInfo{fontdef.dtx}
\begin{document}
   \DocInput{fontdef.dtx}
\end{document}
%</driver>
%    \end{macrocode}
%
%
%
% \section{The \texttt{fonttext.ltx} file}
%
%    The identification is done earlier on with a |\ProvidesFile|
%    declaration.
%    \begin{macrocode}
%<*text>
\typeout{=== Don't modify this file, use a .cfg file instead ===^^J}
%    \end{macrocode}
%
%  \subsection{Encodings}
%
%    This file declares the standard encodings for text and math
%    fonts. All others should be declared in packages or in the
%    documents directly.
%
%    For every text encoding there are normally a number of encoding
%    specific commands, e.g.~accents, special characters, etc.  (The
%    definition for such a command might have to change when the
%    encoding is changed, because the character is in a different
%    position, or not available at all, or the accent is produced in a
%    different way.)  This is handled by a general mechanism which is
%    described in \texttt{ltoutenc.dtx}.
%
%    By convention, text  encoding specific declarations, including the
%    declaration |\DeclareFontEncoding|, are kept in separate file of
%    the form \meta{enc}\texttt{enc.def}, e.g.~\texttt{ot1enc.def}. This
%    allows other applications to make use of the declarations as
%    well.
%
%    Similar to the default encoding, the loading of the encoding
%    files for the two major text encodings shouldn't be changed.
%    In particular, the \texttt{inputenc} package depends on this.
% \changes{v2.2s}{1997/12/20}{Added documentation}
%
% \changes{v2.1d}{1994/01/05}{Removed nf prefix from file names.}
% \changes{v2.1f}{1994/05/14}{Removed .def files.}
% \changes{v2.1g}{1994/05/16}{Removed \cs{DeclareFontEncoding} for ot1
%                             and t1 and input .def files instead}
% \changes{v2.2c}{1994/10/25}{Added OMSenc.def}
% \changes{v2.2d}{1994/10/31}{Added OMLenc.def ...}
% \changes{v2.2e}{1994/10/31}{... and moved further down}
% \changes{v2.2f}{1994/11/07}{(DPC) Updated to use \cs{ProvidesFile}}
% \changes{v2.2h}{1994/11/16}{(DPC) Removed \cmd\{ and \cmd\}}
% \changes{v3.0a}{2016/12/03}{(DPC) Default to TU encoding for Unicode TeX engines}
%    \begin{macrocode}
%%
%% This is file `omlenc.def',
%% generated with the docstrip utility.
%%
%% The original source files were:
%%
%% ltoutenc.dtx  (with options: `OML')
%% 
%% This is a generated file.
%% 
%% The source is maintained by the LaTeX Project team and bug
%% reports for it can be opened at https://latex-project.org/bugs.html
%% (but please observe conditions on bug reports sent to that address!)
%% 
%% 
%% Copyright (C) 1993-2020
%% The LaTeX3 Project and any individual authors listed elsewhere
%% in this file.
%% 
%% This file was generated from file(s) of the LaTeX base system.
%% --------------------------------------------------------------
%% 
%% It may be distributed and/or modified under the
%% conditions of the LaTeX Project Public License, either version 1.3c
%% of this license or (at your option) any later version.
%% The latest version of this license is in
%%    https://www.latex-project.org/lppl.txt
%% and version 1.3c or later is part of all distributions of LaTeX
%% version 2008 or later.
%% 
%% This file has the LPPL maintenance status "maintained".
%% 
%% This file may only be distributed together with a copy of the LaTeX
%% base system. You may however distribute the LaTeX base system without
%% such generated files.
%% 
%% The list of all files belonging to the LaTeX base distribution is
%% given in the file `manifest.txt'. See also `legal.txt' for additional
%% information.
%% 
%% The list of derived (unpacked) files belonging to the distribution
%% and covered by LPPL is defined by the unpacking scripts (with
%% extension .ins) which are part of the distribution.
%%% From File: ltoutenc.dtx
\ProvidesFile{omlenc.def}
 [2020/08/10 v2.0s
      Standard LaTeX file]
\DeclareFontEncoding{OML}{}{}
\DeclareTextSymbol{\textless}{OML}{`\<}
\DeclareTextSymbol{\textgreater}{OML}{`\>}
\DeclareTextAccent{\t}{OML}{127}  % "7F
\endinput
%%
%% End of file `omlenc.def'.

%%
%% This is file `omsenc.def',
%% generated with the docstrip utility.
%%
%% The original source files were:
%%
%% ltoutenc.dtx  (with options: `OMS')
%% 
%% This is a generated file.
%% 
%% The source is maintained by the LaTeX Project team and bug
%% reports for it can be opened at https://latex-project.org/bugs.html
%% (but please observe conditions on bug reports sent to that address!)
%% 
%% 
%% Copyright (C) 1993-2020
%% The LaTeX3 Project and any individual authors listed elsewhere
%% in this file.
%% 
%% This file was generated from file(s) of the LaTeX base system.
%% --------------------------------------------------------------
%% 
%% It may be distributed and/or modified under the
%% conditions of the LaTeX Project Public License, either version 1.3c
%% of this license or (at your option) any later version.
%% The latest version of this license is in
%%    https://www.latex-project.org/lppl.txt
%% and version 1.3c or later is part of all distributions of LaTeX
%% version 2008 or later.
%% 
%% This file has the LPPL maintenance status "maintained".
%% 
%% This file may only be distributed together with a copy of the LaTeX
%% base system. You may however distribute the LaTeX base system without
%% such generated files.
%% 
%% The list of all files belonging to the LaTeX base distribution is
%% given in the file `manifest.txt'. See also `legal.txt' for additional
%% information.
%% 
%% The list of derived (unpacked) files belonging to the distribution
%% and covered by LPPL is defined by the unpacking scripts (with
%% extension .ins) which are part of the distribution.
%%% From File: ltoutenc.dtx
\ProvidesFile{omsenc.def}
 [2020/08/10 v2.0s
      Standard LaTeX file]
\DeclareFontEncoding{OMS}{}{}
\DeclareTextSymbol{\textasteriskcentered}{OMS}{3}   % "03
\DeclareTextSymbol{\textbackslash}{OMS}{110}        % "6E
\DeclareTextSymbol{\textbar}{OMS}{106}              % "6A
\DeclareTextSymbol{\textbardbl}{OMS}{107}           % "6B
\DeclareTextSymbol{\textbraceleft}{OMS}{102}        % "66
\DeclareTextSymbol{\textbraceright}{OMS}{103}       % "67
\DeclareTextSymbol{\textbullet}{OMS}{15}            % "0F
\DeclareTextSymbol{\textdaggerdbl}{OMS}{122}        % "7A
\DeclareTextSymbol{\textdagger}{OMS}{121}           % "79
\DeclareTextSymbol{\textparagraph}{OMS}{123}        % "7B
\DeclareTextSymbol{\textperiodcentered}{OMS}{1}     % "01
\DeclareTextSymbol{\textsection}{OMS}{120}          % "78
\DeclareTextSymbol{\textbigcircle}{OMS}{13}         % "0D
\DeclareTextCommand{\textcircled}{OMS}[1]{\hmode@bgroup
   \ooalign{%
      \hfil \raise .07ex\hbox {\upshape#1}\hfil \crcr
      \char 13 % "0D
   }%
 \egroup}
\endinput
%%
%% End of file `omsenc.def'.

%    \end{macrocode}
%    Documents containing a lot of accented characters should really
%    be using T1 fonts. We therefore load this last so that T1 encoding
%    specific commands are executed as fast as possible (encoding
%    files are no longer reloaded in \texttt{fontenc}.
% \changes{v3.0f}{2020/01/25}{Load t1enc.def last (gh/255)}
%    \begin{macrocode}
%%
%% This is file `ot1enc.def',
%% generated with the docstrip utility.
%%
%% The original source files were:
%%
%% ltoutenc.dtx  (with options: `OT1')
%% 
%% This is a generated file.
%% 
%% The source is maintained by the LaTeX Project team and bug
%% reports for it can be opened at https://latex-project.org/bugs.html
%% (but please observe conditions on bug reports sent to that address!)
%% 
%% 
%% Copyright (C) 1993-2020
%% The LaTeX3 Project and any individual authors listed elsewhere
%% in this file.
%% 
%% This file was generated from file(s) of the LaTeX base system.
%% --------------------------------------------------------------
%% 
%% It may be distributed and/or modified under the
%% conditions of the LaTeX Project Public License, either version 1.3c
%% of this license or (at your option) any later version.
%% The latest version of this license is in
%%    https://www.latex-project.org/lppl.txt
%% and version 1.3c or later is part of all distributions of LaTeX
%% version 2008 or later.
%% 
%% This file has the LPPL maintenance status "maintained".
%% 
%% This file may only be distributed together with a copy of the LaTeX
%% base system. You may however distribute the LaTeX base system without
%% such generated files.
%% 
%% The list of all files belonging to the LaTeX base distribution is
%% given in the file `manifest.txt'. See also `legal.txt' for additional
%% information.
%% 
%% The list of derived (unpacked) files belonging to the distribution
%% and covered by LPPL is defined by the unpacking scripts (with
%% extension .ins) which are part of the distribution.
%%% From File: ltoutenc.dtx
\ProvidesFile{ot1enc.def}
 [2020/08/10 v2.0s
      Standard LaTeX file]
\DeclareFontEncoding{OT1}{}{}
\DeclareTextAccent{\"}{OT1}{127}
\DeclareTextAccent{\'}{OT1}{19}
\DeclareTextAccent{\.}{OT1}{95}
\DeclareTextAccent{\=}{OT1}{22}
\DeclareTextAccent{\^}{OT1}{94}
\DeclareTextAccent{\`}{OT1}{18}
\DeclareTextAccent{\~}{OT1}{126}
\DeclareTextAccent{\H}{OT1}{125}
\DeclareTextAccent{\u}{OT1}{21}
\DeclareTextAccent{\v}{OT1}{20}
\DeclareTextAccent{\r}{OT1}{23}
\DeclareTextCommand{\b}{OT1}[1]
   {\hmode@bgroup\o@lign{\relax#1\crcr\hidewidth\ltx@sh@ft{-3ex}%
     \vbox to.2ex{\hbox{\char22}\vss}\hidewidth}\egroup}
\DeclareTextCommand{\c}{OT1}[1]
   {\leavevmode\setbox\z@\hbox{#1}\ifdim\ht\z@=1ex\accent24 #1%
    \else{\ooalign{\unhbox\z@\crcr\hidewidth\char24\hidewidth}}\fi}
\DeclareTextCommand{\d}{OT1}[1]
   {\hmode@bgroup
    \o@lign{\relax#1\crcr\hidewidth\ltx@sh@ft{-1ex}.\hidewidth}\egroup}
\DeclareTextSymbol{\AE}{OT1}{29}
\DeclareTextSymbol{\OE}{OT1}{30}
\DeclareTextSymbol{\O}{OT1}{31}
\DeclareTextSymbol{\ae}{OT1}{26}
\DeclareTextSymbol{\i}{OT1}{16}
\DeclareTextSymbol{\j}{OT1}{17}
\DeclareTextSymbol{\oe}{OT1}{27}
\DeclareTextSymbol{\o}{OT1}{28}
\DeclareTextSymbol{\ss}{OT1}{25}
\DeclareTextSymbol{\textemdash}{OT1}{124}
\DeclareTextSymbol{\textendash}{OT1}{123}
\DeclareTextCommand{\textexclamdown}{OT1}{!`}
\DeclareTextCommand{\textquestiondown}{OT1}{?`}
\DeclareTextSymbol{\textquotedblleft}{OT1}{92}
\DeclareTextSymbol{\textquotedblright}{OT1}{`\"}
\DeclareTextSymbol{\textquoteleft}{OT1}{`\`}
\DeclareTextSymbol{\textquoteright}{OT1}{`\'}
\DeclareTextCommand{\L}{OT1}
   {\leavevmode\setbox\z@\hbox{L}\hb@xt@\wd\z@{\hss\@xxxii L}}
\DeclareTextCommand{\l}{OT1}
   {\hmode@bgroup\@xxxii l\egroup}
\DeclareTextCompositeCommand{\r}{OT1}{A}
   {\leavevmode\setbox\z@\hbox{!}\dimen@\ht\z@\advance\dimen@-1ex%
    \rlap{\raise.67\dimen@\hbox{\char23}}A}
\DeclareTextCommand{\ij}{OT1}{%
  \nobreak\hskip\z@skip i\kern-0.02em j\nobreak\hskip\z@skip}
\DeclareTextCommand{\IJ}{OT1}{%
  \nobreak\hskip\z@skip I\kern-0.02em J\nobreak\hskip\z@skip}
\DeclareTextCommand{\textdollar}{OT1}{\hmode@bgroup
   \ifdim \fontdimen\@ne\font >\z@
      \slshape
   \else
      \upshape
   \fi
   \char`\$\egroup}
\DeclareTextCommand{\textsterling}{OT1}{\hmode@bgroup
   \ifdim \fontdimen\@ne\font >\z@
      \itshape
   \else
      \fontshape{ui}\selectfont
   \fi
   \char`\$\egroup}
\DeclareTextComposite{\.}{OT1}{i}{`\i}
\DeclareTextComposite{\.}{OT1}{\i}{`\i}
\DeclareTextCompositeCommand{\`}{OT1}{i}{\@tabacckludge`\i}
\DeclareTextCompositeCommand{\'}{OT1}{i}{\@tabacckludge'\i}
\DeclareTextCompositeCommand{\^}{OT1}{i}{\^\i}
\DeclareTextCompositeCommand{\"}{OT1}{i}{\"\i}
\ifx\textcommaabove\@undefined\else
\DeclareTextCompositeCommand{\c}{OT1}{g}{\textcommaabove{g}}
\fi
\endinput
%%
%% End of file `ot1enc.def'.

%%
%% This is file `t1enc.def',
%% generated with the docstrip utility.
%%
%% The original source files were:
%%
%% ltoutenc.dtx  (with options: `T1')
%% 
%% This is a generated file.
%% 
%% The source is maintained by the LaTeX Project team and bug
%% reports for it can be opened at https://latex-project.org/bugs.html
%% (but please observe conditions on bug reports sent to that address!)
%% 
%% 
%% Copyright (C) 1993-2020
%% The LaTeX3 Project and any individual authors listed elsewhere
%% in this file.
%% 
%% This file was generated from file(s) of the LaTeX base system.
%% --------------------------------------------------------------
%% 
%% It may be distributed and/or modified under the
%% conditions of the LaTeX Project Public License, either version 1.3c
%% of this license or (at your option) any later version.
%% The latest version of this license is in
%%    https://www.latex-project.org/lppl.txt
%% and version 1.3c or later is part of all distributions of LaTeX
%% version 2008 or later.
%% 
%% This file has the LPPL maintenance status "maintained".
%% 
%% This file may only be distributed together with a copy of the LaTeX
%% base system. You may however distribute the LaTeX base system without
%% such generated files.
%% 
%% The list of all files belonging to the LaTeX base distribution is
%% given in the file `manifest.txt'. See also `legal.txt' for additional
%% information.
%% 
%% The list of derived (unpacked) files belonging to the distribution
%% and covered by LPPL is defined by the unpacking scripts (with
%% extension .ins) which are part of the distribution.
%%% From File: ltoutenc.dtx
\ProvidesFile{t1enc.def}
 [2020/08/10 v2.0s
      Standard LaTeX file]
\DeclareFontEncoding{T1}{}{}
\DeclareTextAccent{\`}{T1}{0}
\DeclareTextAccent{\'}{T1}{1}
\DeclareTextAccent{\^}{T1}{2}
\DeclareTextAccent{\~}{T1}{3}
\DeclareTextAccent{\"}{T1}{4}
\DeclareTextAccent{\H}{T1}{5}
\DeclareTextAccent{\r}{T1}{6}
\DeclareTextAccent{\v}{T1}{7}
\DeclareTextAccent{\u}{T1}{8}
\DeclareTextAccent{\=}{T1}{9}
\DeclareTextAccent{\.}{T1}{10}
\DeclareTextCommand{\b}{T1}[1]
   {\hmode@bgroup\o@lign{\relax#1\crcr\hidewidth\ltx@sh@ft{-3ex}%
     \vbox to.2ex{\hbox{\char9}\vss}\hidewidth}\egroup}
\DeclareTextCommand{\c}{T1}[1]
   {\leavevmode\setbox\z@\hbox{#1}\ifdim\ht\z@=1ex\accent11 #1%
     \else{\ooalign{\unhbox\z@\crcr
        \hidewidth\char11\hidewidth}}\fi}
\DeclareTextCommand{\d}{T1}[1]
   {\hmode@bgroup
    \o@lign{\relax#1\crcr\hidewidth\ltx@sh@ft{-1ex}.\hidewidth}\egroup}
\DeclareTextCommand{\k}{T1}[1]
   {\hmode@bgroup\ooalign{\null#1\crcr\hidewidth\char12}\egroup}
\DeclareTextCommand{\textogonekcentered}{T1}[1]
   {\hmode@bgroup\ooalign{%
                \null#1\crcr\hidewidth\char12\hidewidth}\egroup}
\DeclareTextCommand{\textperthousand}{T1}
   {\%\char 24 }          % space or `relax as delimiter?
\DeclareTextCommand{\textpertenthousand}{T1}
   {\%\char 24\char 24 }  % space or `relax as delimiter?
\DeclareTextCommand{\Hwithstroke}{T1}
   {%
    \hmode@bgroup
    \vphantom{H}%
    \sbox\z@{H}%
    \ooalign{%
      H\cr
      \hidewidth
      \vrule
        height \dimexpr 0.7\ht\z@+0.1ex\relax
        depth  -0.7\ht\z@
        width  0.8\wd\z@
      \hidewidth\cr
    }%
    \egroup
   }
\DeclareTextCommand{\hwithstroke}{T1}
   {%
    \hmode@bgroup
    \vphantom{h}%
    \sbox\z@{h}%
    \ooalign{%
      h\cr
      \kern0.075\wd\z@
      \vrule
        height \dimexpr 0.7\ht\z@+0.1ex\relax
        depth  -0.7\ht\z@
        width  0.4\wd\z@
      \hidewidth\cr
    }%
    \egroup
   }
\DeclareTextSymbol{\AE}{T1}{198}
\DeclareTextSymbol{\DH}{T1}{208}
\DeclareTextSymbol{\DJ}{T1}{208}
\DeclareTextSymbol{\L}{T1}{138}
\DeclareTextSymbol{\NG}{T1}{141}
\DeclareTextSymbol{\OE}{T1}{215}
\DeclareTextSymbol{\O}{T1}{216}
\DeclareTextSymbol{\SS}{T1}{223}
\DeclareTextSymbol{\TH}{T1}{222}
\DeclareTextSymbol{\ae}{T1}{230}
\DeclareTextSymbol{\dh}{T1}{240}
\DeclareTextSymbol{\dj}{T1}{158}
\DeclareTextSymbol{\guillemetleft}{T1}{19}
\DeclareTextSymbol{\guillemetright}{T1}{20}
\DeclareTextSymbol{\guillemotleft}{T1}{19}
\DeclareTextSymbol{\guillemotright}{T1}{20}
\DeclareTextSymbol{\guilsinglleft}{T1}{14}
\DeclareTextSymbol{\guilsinglright}{T1}{15}
\DeclareTextSymbol{\i}{T1}{25}
\DeclareTextSymbol{\j}{T1}{26}
\DeclareTextSymbol{\ij}{T1}{188}
\DeclareTextSymbol{\IJ}{T1}{156}
\DeclareTextSymbol{\l}{T1}{170}
\DeclareTextSymbol{\ng}{T1}{173}
\DeclareTextSymbol{\oe}{T1}{247}
\DeclareTextSymbol{\o}{T1}{248}
\DeclareTextSymbol{\quotedblbase}{T1}{18}
\DeclareTextSymbol{\quotesinglbase}{T1}{13}
\DeclareTextSymbol{\ss}{T1}{255}
\DeclareTextSymbol{\textasciicircum}{T1}{`\^}
\DeclareTextSymbol{\textasciitilde}{T1}{`\~}
\DeclareTextSymbol{\textbackslash}{T1}{`\\}
\DeclareTextSymbol{\textbar}{T1}{`\|}
\DeclareTextSymbol{\textbraceleft}{T1}{`\{}
\DeclareTextSymbol{\textbraceright}{T1}{`\}}
\DeclareTextSymbol{\textcompwordmark}{T1}{23}
\DeclareTextSymbol{\textdollar}{T1}{`\$}
\DeclareTextSymbol{\textemdash}{T1}{22}
\DeclareTextSymbol{\textendash}{T1}{21}
\DeclareTextSymbol{\textexclamdown}{T1}{189}
\DeclareTextSymbol{\textgreater}{T1}{`\>}
\DeclareTextSymbol{\textless}{T1}{`\<}
\DeclareTextSymbol{\textquestiondown}{T1}{190}
\DeclareTextSymbol{\textquotedblleft}{T1}{16}
\DeclareTextSymbol{\textquotedblright}{T1}{17}
\DeclareTextSymbol{\textquotedbl}{T1}{`\"}
\DeclareTextSymbol{\textquoteleft}{T1}{`\`}
\DeclareTextSymbol{\textquoteright}{T1}{`\'}
\DeclareTextSymbol{\textsection}{T1}{159}
\DeclareTextSymbol{\textsterling}{T1}{191}
\DeclareTextSymbol{\textunderscore}{T1}{95}
\DeclareTextSymbol{\textvisiblespace}{T1}{32}
\DeclareTextSymbol{\th}{T1}{254}
\DeclareTextComposite{\.}{T1}{i}{`\i}
\DeclareTextComposite{\.}{T1}{\i}{`\i}
\DeclareTextComposite{\u}{T1}{A}{128}
\DeclareTextComposite{\k}{T1}{A}{129}
\DeclareTextComposite{\'}{T1}{C}{130}
\DeclareTextComposite{\v}{T1}{C}{131}
\DeclareTextComposite{\v}{T1}{D}{132}
\DeclareTextComposite{\v}{T1}{E}{133}
\DeclareTextComposite{\k}{T1}{E}{134}
\DeclareTextComposite{\u}{T1}{G}{135}
\DeclareTextComposite{\'}{T1}{L}{136}
\DeclareTextComposite{\v}{T1}{L}{137}
\DeclareTextComposite{\'}{T1}{N}{139}
\DeclareTextComposite{\v}{T1}{N}{140}
\DeclareTextComposite{\H}{T1}{O}{142}
\DeclareTextComposite{\'}{T1}{R}{143}
\DeclareTextComposite{\v}{T1}{R}{144}
\DeclareTextComposite{\'}{T1}{S}{145}
\DeclareTextComposite{\v}{T1}{S}{146}
\DeclareTextComposite{\c}{T1}{S}{147}
\DeclareTextComposite{\v}{T1}{T}{148}
\DeclareTextComposite{\c}{T1}{T}{149}
\DeclareTextComposite{\H}{T1}{U}{150}
\DeclareTextComposite{\r}{T1}{U}{151}
\DeclareTextComposite{\"}{T1}{Y}{152}
\DeclareTextComposite{\'}{T1}{Z}{153}
\DeclareTextComposite{\v}{T1}{Z}{154}
\DeclareTextComposite{\.}{T1}{Z}{155}
\DeclareTextComposite{\.}{T1}{I}{157}
\DeclareTextComposite{\u}{T1}{a}{160}
\DeclareTextComposite{\k}{T1}{a}{161}
\DeclareTextComposite{\'}{T1}{c}{162}
\DeclareTextComposite{\v}{T1}{c}{163}
\DeclareTextComposite{\v}{T1}{d}{164}
\DeclareTextComposite{\v}{T1}{e}{165}
\DeclareTextComposite{\k}{T1}{e}{166}
\DeclareTextComposite{\u}{T1}{g}{167}
\DeclareTextComposite{\'}{T1}{l}{168}
\DeclareTextComposite{\v}{T1}{l}{169}
\DeclareTextComposite{\'}{T1}{n}{171}
\DeclareTextComposite{\v}{T1}{n}{172}
\DeclareTextComposite{\H}{T1}{o}{174}
\DeclareTextComposite{\'}{T1}{r}{175}
\DeclareTextComposite{\v}{T1}{r}{176}
\DeclareTextComposite{\'}{T1}{s}{177}
\DeclareTextComposite{\v}{T1}{s}{178}
\DeclareTextComposite{\c}{T1}{s}{179}
\DeclareTextComposite{\v}{T1}{t}{180}
\DeclareTextComposite{\c}{T1}{t}{181}
\DeclareTextComposite{\H}{T1}{u}{182}
\DeclareTextComposite{\r}{T1}{u}{183}
\DeclareTextComposite{\"}{T1}{y}{184}
\DeclareTextComposite{\'}{T1}{z}{185}
\DeclareTextComposite{\v}{T1}{z}{186}
\DeclareTextComposite{\.}{T1}{z}{187}
\DeclareTextComposite{\`}{T1}{A}{192}
\DeclareTextComposite{\'}{T1}{A}{193}
\DeclareTextComposite{\^}{T1}{A}{194}
\DeclareTextComposite{\~}{T1}{A}{195}
\DeclareTextComposite{\"}{T1}{A}{196}
\DeclareTextComposite{\r}{T1}{A}{197}
\DeclareTextComposite{\c}{T1}{C}{199}
\DeclareTextComposite{\`}{T1}{E}{200}
\DeclareTextComposite{\'}{T1}{E}{201}
\DeclareTextComposite{\^}{T1}{E}{202}
\DeclareTextComposite{\"}{T1}{E}{203}
\DeclareTextComposite{\`}{T1}{I}{204}
\DeclareTextComposite{\'}{T1}{I}{205}
\DeclareTextComposite{\^}{T1}{I}{206}
\DeclareTextComposite{\"}{T1}{I}{207}
\DeclareTextComposite{\~}{T1}{N}{209}
\DeclareTextComposite{\`}{T1}{O}{210}
\DeclareTextComposite{\'}{T1}{O}{211}
\DeclareTextComposite{\^}{T1}{O}{212}
\DeclareTextComposite{\~}{T1}{O}{213}
\DeclareTextComposite{\"}{T1}{O}{214}
\DeclareTextComposite{\`}{T1}{U}{217}
\DeclareTextComposite{\'}{T1}{U}{218}
\DeclareTextComposite{\^}{T1}{U}{219}
\DeclareTextComposite{\"}{T1}{U}{220}
\DeclareTextComposite{\'}{T1}{Y}{221}
\DeclareTextComposite{\`}{T1}{a}{224}
\DeclareTextComposite{\'}{T1}{a}{225}
\DeclareTextComposite{\^}{T1}{a}{226}
\DeclareTextComposite{\~}{T1}{a}{227}
\DeclareTextComposite{\"}{T1}{a}{228}
\DeclareTextComposite{\r}{T1}{a}{229}
\DeclareTextComposite{\c}{T1}{c}{231}
\DeclareTextComposite{\`}{T1}{e}{232}
\DeclareTextComposite{\'}{T1}{e}{233}
\DeclareTextComposite{\^}{T1}{e}{234}
\DeclareTextComposite{\"}{T1}{e}{235}
\DeclareTextComposite{\`}{T1}{i}{236}
\DeclareTextComposite{\`}{T1}{\i}{236}
\DeclareTextComposite{\'}{T1}{i}{237}
\DeclareTextComposite{\'}{T1}{\i}{237}
\DeclareTextComposite{\^}{T1}{i}{238}
\DeclareTextComposite{\^}{T1}{\i}{238}
\DeclareTextComposite{\"}{T1}{i}{239}
\DeclareTextComposite{\"}{T1}{\i}{239}
\DeclareTextComposite{\~}{T1}{n}{241}
\DeclareTextComposite{\`}{T1}{o}{242}
\DeclareTextComposite{\'}{T1}{o}{243}
\DeclareTextComposite{\^}{T1}{o}{244}
\DeclareTextComposite{\~}{T1}{o}{245}
\DeclareTextComposite{\"}{T1}{o}{246}
\DeclareTextComposite{\`}{T1}{u}{249}
\DeclareTextComposite{\'}{T1}{u}{250}
\DeclareTextComposite{\^}{T1}{u}{251}
\DeclareTextComposite{\"}{T1}{u}{252}
\DeclareTextComposite{\'}{T1}{y}{253}
\DeclareTextCompositeCommand{\k}{T1}{o}{\textogonekcentered{o}}
\DeclareTextCompositeCommand{\k}{T1}{O}{\textogonekcentered{O}}
\ifx\textcommaabove\@undefined\else
\DeclareTextCompositeCommand{\c}{T1}{g}{\textcommaabove{g}}
\fi
\ifx\textcommabelow\@undefined\else
\DeclareTextCompositeCommand{\c}{T1}{G}{\textcommabelow{G}}
\DeclareTextCompositeCommand{\c}{T1}{K}{\textcommabelow{K}}
\DeclareTextCompositeCommand{\c}{T1}{k}{\textcommabelow{k}}
\DeclareTextCompositeCommand{\c}{T1}{L}{\textcommabelow{L}}
\DeclareTextCompositeCommand{\c}{T1}{l}{\textcommabelow{l}}
\DeclareTextCompositeCommand{\c}{T1}{N}{\textcommabelow{N}}
\DeclareTextCompositeCommand{\c}{T1}{n}{\textcommabelow{n}}
\DeclareTextCompositeCommand{\c}{T1}{R}{\textcommabelow{R}}
\DeclareTextCompositeCommand{\c}{T1}{r}{\textcommabelow{r}}
\fi
\endinput
%%
%% End of file `t1enc.def'.

%    \end{macrocode}
%
%    \begin{macrocode}
%%
%% This is file `ts1enc.def',
%% generated with the docstrip utility.
%%
%% The original source files were:
%%
%% ltoutenc.dtx  (with options: `TS1')
%% 
%% This is a generated file.
%% 
%% The source is maintained by the LaTeX Project team and bug
%% reports for it can be opened at https://latex-project.org/bugs.html
%% (but please observe conditions on bug reports sent to that address!)
%% 
%% 
%% Copyright (C) 1993-2020
%% The LaTeX3 Project and any individual authors listed elsewhere
%% in this file.
%% 
%% This file was generated from file(s) of the LaTeX base system.
%% --------------------------------------------------------------
%% 
%% It may be distributed and/or modified under the
%% conditions of the LaTeX Project Public License, either version 1.3c
%% of this license or (at your option) any later version.
%% The latest version of this license is in
%%    https://www.latex-project.org/lppl.txt
%% and version 1.3c or later is part of all distributions of LaTeX
%% version 2008 or later.
%% 
%% This file has the LPPL maintenance status "maintained".
%% 
%% This file may only be distributed together with a copy of the LaTeX
%% base system. You may however distribute the LaTeX base system without
%% such generated files.
%% 
%% The list of all files belonging to the LaTeX base distribution is
%% given in the file `manifest.txt'. See also `legal.txt' for additional
%% information.
%% 
%% The list of derived (unpacked) files belonging to the distribution
%% and covered by LPPL is defined by the unpacking scripts (with
%% extension .ins) which are part of the distribution.
%%% From File: ltoutenc.dtx
\ProvidesFile{ts1enc.def}[2001/06/05 v3.0e (jk/car/fm)
      Standard LaTeX file]
\DeclareFontEncoding{TS1}{}{}
\DeclareFontSubstitution{TS1}{cmr}{m}{n}
\DeclareTextCommand{\capitalcedilla}{TS1}[1]
   {\hmode@bgroup
    \ooalign{\null#1\crcr\hidewidth\char11\hidewidth}\egroup}
\DeclareTextCommand{\capitalogonek}{TS1}[1]
   {\hmode@bgroup
    \ooalign{\null#1\crcr\hidewidth\char12\hidewidth}\egroup}
\DeclareTextAccent{\capitalgrave}{TS1}{0}
\DeclareTextAccent{\capitalacute}{TS1}{1}
\DeclareTextAccent{\capitalcircumflex}{TS1}{2}
\DeclareTextAccent{\capitaltilde}{TS1}{3}
\DeclareTextAccent{\capitaldieresis}{TS1}{4}
\DeclareTextAccent{\capitalhungarumlaut}{TS1}{5}
\DeclareTextAccent{\capitalring}{TS1}{6}
\DeclareTextAccent{\capitalcaron}{TS1}{7}
\DeclareTextAccent{\capitalbreve}{TS1}{8}
\DeclareTextAccent{\capitalmacron}{TS1}{9}
\DeclareTextAccent{\capitaldotaccent}{TS1}{10}
\DeclareTextAccent{\t}{TS1}{26}
\DeclareTextAccent{\capitaltie}{TS1}{27}
\DeclareTextAccent{\newtie}{TS1}{28}
\DeclareTextAccent{\capitalnewtie}{TS1}{29}
\DeclareTextSymbol{\textcapitalcompwordmark}{TS1}{23}
\DeclareTextSymbol{\textascendercompwordmark}{TS1}{31}
\DeclareTextSymbol{\textquotestraightbase}{TS1}{13}
\DeclareTextSymbol{\textquotestraightdblbase}{TS1}{18}
\DeclareTextSymbol{\texttwelveudash}{TS1}{21}
\DeclareTextSymbol{\textthreequartersemdash}{TS1}{22}
\DeclareTextSymbol{\textleftarrow}{TS1}{24}
\DeclareTextSymbol{\textrightarrow}{TS1}{25}
\DeclareTextSymbol{\textblank}{TS1}{32}
\DeclareTextSymbol{\textdollar}{TS1}{36}
\DeclareTextSymbol{\textquotesingle}{TS1}{39}
\DeclareTextSymbol{\textasteriskcentered}{TS1}{42}
\DeclareTextSymbol{\textdblhyphen}{TS1}{45}
\DeclareTextSymbol{\textfractionsolidus}{TS1}{47}
\DeclareTextSymbol{\textzerooldstyle}{TS1}{48}
\DeclareTextSymbol{\textoneoldstyle}{TS1}{49}
\DeclareTextSymbol{\texttwooldstyle}{TS1}{50}
\DeclareTextSymbol{\textthreeoldstyle}{TS1}{51}
\DeclareTextSymbol{\textfouroldstyle}{TS1}{52}
\DeclareTextSymbol{\textfiveoldstyle}{TS1}{53}
\DeclareTextSymbol{\textsixoldstyle}{TS1}{54}
\DeclareTextSymbol{\textsevenoldstyle}{TS1}{55}
\DeclareTextSymbol{\texteightoldstyle}{TS1}{56}
\DeclareTextSymbol{\textnineoldstyle}{TS1}{57}
\DeclareTextSymbol{\textlangle}{TS1}{60}
\DeclareTextSymbol{\textminus}{TS1}{61}
\DeclareTextSymbol{\textrangle}{TS1}{62}
\DeclareTextSymbol{\textmho}{TS1}{77}
\DeclareTextSymbol{\textbigcircle}{TS1}{79}
\DeclareTextCommand{\textcircled}{TS1}[1]{\hmode@bgroup
   \ooalign{%
      \hfil \raise .07ex\hbox {\upshape#1}\hfil \crcr
      \char 79   % '117 = "4F
   }%
 \egroup}
\DeclareTextSymbol{\textohm}{TS1}{87}
\DeclareTextSymbol{\textlbrackdbl}{TS1}{91}
\DeclareTextSymbol{\textrbrackdbl}{TS1}{93}
\DeclareTextSymbol{\textuparrow}{TS1}{94}
\DeclareTextSymbol{\textdownarrow}{TS1}{95}
\DeclareTextSymbol{\textasciigrave}{TS1}{96}
\DeclareTextSymbol{\textborn}{TS1}{98}
\DeclareTextSymbol{\textdivorced}{TS1}{99}
\DeclareTextSymbol{\textdied}{TS1}{100}
\DeclareTextSymbol{\textleaf}{TS1}{108}
\DeclareTextSymbol{\textmarried}{TS1}{109}
\DeclareTextSymbol{\textmusicalnote}{TS1}{110}
\DeclareTextSymbol{\texttildelow}{TS1}{126}
\DeclareTextSymbol{\textdblhyphenchar}{TS1}{127}
\DeclareTextSymbol{\textasciibreve}{TS1}{128}
\DeclareTextSymbol{\textasciicaron}{TS1}{129}
\DeclareTextSymbol{\textacutedbl}{TS1}{130}
\DeclareTextSymbol{\textgravedbl}{TS1}{131}
\DeclareTextSymbol{\textdagger}{TS1}{132}
\DeclareTextSymbol{\textdaggerdbl}{TS1}{133}
\DeclareTextSymbol{\textbardbl}{TS1}{134}
\DeclareTextSymbol{\textperthousand}{TS1}{135}
\DeclareTextSymbol{\textbullet}{TS1}{136}
\DeclareTextSymbol{\textcelsius}{TS1}{137}
\DeclareTextSymbol{\textdollaroldstyle}{TS1}{138}
\DeclareTextSymbol{\textcentoldstyle}{TS1}{139}
\DeclareTextSymbol{\textflorin}{TS1}{140}
\DeclareTextSymbol{\textcolonmonetary}{TS1}{141}
\DeclareTextSymbol{\textwon}{TS1}{142}
\DeclareTextSymbol{\textnaira}{TS1}{143}
\DeclareTextSymbol{\textguarani}{TS1}{144}
\DeclareTextSymbol{\textpeso}{TS1}{145}
\DeclareTextSymbol{\textlira}{TS1}{146}
\DeclareTextSymbol{\textrecipe}{TS1}{147}
\DeclareTextSymbol{\textinterrobang}{TS1}{148}
\DeclareTextSymbol{\textinterrobangdown}{TS1}{149}
\DeclareTextSymbol{\textdong}{TS1}{150}
\DeclareTextSymbol{\texttrademark}{TS1}{151}
\DeclareTextSymbol{\textpertenthousand}{TS1}{152}
\DeclareTextSymbol{\textpilcrow}{TS1}{153}
\DeclareTextSymbol{\textbaht}{TS1}{154}
\DeclareTextSymbol{\textnumero}{TS1}{155}
\DeclareTextSymbol{\textdiscount}{TS1}{156}
\DeclareTextSymbol{\textestimated}{TS1}{157}
\DeclareTextSymbol{\textopenbullet}{TS1}{158}
\DeclareTextSymbol{\textservicemark}{TS1}{159}
\DeclareTextSymbol{\textlquill}{TS1}{160}
\DeclareTextSymbol{\textrquill}{TS1}{161}
\DeclareTextSymbol{\textcent}{TS1}{162}
\DeclareTextSymbol{\textsterling}{TS1}{163}
\DeclareTextSymbol{\textcurrency}{TS1}{164}
\DeclareTextSymbol{\textyen}{TS1}{165}
\DeclareTextSymbol{\textbrokenbar}{TS1}{166}
\DeclareTextSymbol{\textsection}{TS1}{167}
\DeclareTextSymbol{\textasciidieresis}{TS1}{168}
\DeclareTextSymbol{\textcopyright}{TS1}{169}
\DeclareTextSymbol{\textordfeminine}{TS1}{170}
\DeclareTextSymbol{\textcopyleft}{TS1}{171}
\DeclareTextSymbol{\textlnot}{TS1}{172}
\DeclareTextSymbol{\textcircledP}{TS1}{173}
\DeclareTextSymbol{\textregistered}{TS1}{174}
\DeclareTextSymbol{\textasciimacron}{TS1}{175}
\DeclareTextSymbol{\textdegree}{TS1}{176}
\DeclareTextSymbol{\textpm}{TS1}{177}
\DeclareTextSymbol{\texttwosuperior}{TS1}{178}
\DeclareTextSymbol{\textthreesuperior}{TS1}{179}
\DeclareTextSymbol{\textasciiacute}{TS1}{180}
\DeclareTextSymbol{\textmu}{TS1}{181} % micro sign
\DeclareTextSymbol{\textparagraph}{TS1}{182}
\DeclareTextSymbol{\textperiodcentered}{TS1}{183}
\DeclareTextSymbol{\textreferencemark}{TS1}{184}
\DeclareTextSymbol{\textonesuperior}{TS1}{185}
\DeclareTextSymbol{\textordmasculine}{TS1}{186}
\DeclareTextSymbol{\textsurd}{TS1}{187}
\DeclareTextSymbol{\textonequarter}{TS1}{188}
\DeclareTextSymbol{\textonehalf}{TS1}{189}
\DeclareTextSymbol{\textthreequarters}{TS1}{190}
\DeclareTextSymbol{\texteuro}{TS1}{191}
\DeclareTextSymbol{\texttimes}{TS1}{214}
\DeclareTextSymbol{\textdiv}{TS1}{246}
\endinput
%%
%% End of file `ts1enc.def'.

%    \end{macrocode}
%    
%
%
% \changes{v3.0a}{2016/12/03}{(DPC) Default to TU encoding for Unicode TeX engines}
%    \begin{macrocode}
\ifx\Umathcode\@undefined
%    \end{macrocode}
%
%    We then set the default text font encoding. This will
%    hopefully change some day to |T1|. This setting should \emph{not}
%    be changed to produce a portable format.
%    \begin{macrocode}
\fontencoding{OT1}
%    \end{macrocode}
%
%    The initial \texttt{fontenc} package load list if an 8-bit \TeX{}
%    engine is used:
% \changes{v3.0g}{2020/02/11}{Provide value for \cs{@fontenc@load@list} (gh/273)}
%    \begin{macrocode}
\def\@fontenc@load@list{\@elt{T1,OT1}}
%    \end{macrocode}
%
%
%
%
%    \begin{macrocode}
\def\rmsubstdefault{cmr}
\def\sfsubstdefault{cmss}
\def\ttsubstdefault{cmtt}
\LoadFontDefinitionFile{TS1}{cmr}
%    \end{macrocode}
%
%    \begin{macrocode}
\else
%    \end{macrocode}
% Unicode.
%    \begin{macrocode}
%%
%% This is file `tuenc.def',
%% generated with the docstrip utility.
%%
%% The original source files were:
%%
%% ltoutenc.dtx  (with options: `TU')
%% 
%% This is a generated file.
%% 
%% The source is maintained by the LaTeX Project team and bug
%% reports for it can be opened at https://latex-project.org/bugs.html
%% (but please observe conditions on bug reports sent to that address!)
%% 
%% 
%% Copyright (C) 1993-2020
%% The LaTeX3 Project and any individual authors listed elsewhere
%% in this file.
%% 
%% This file was generated from file(s) of the LaTeX base system.
%% --------------------------------------------------------------
%% 
%% It may be distributed and/or modified under the
%% conditions of the LaTeX Project Public License, either version 1.3c
%% of this license or (at your option) any later version.
%% The latest version of this license is in
%%    https://www.latex-project.org/lppl.txt
%% and version 1.3c or later is part of all distributions of LaTeX
%% version 2008 or later.
%% 
%% This file has the LPPL maintenance status "maintained".
%% 
%% This file may only be distributed together with a copy of the LaTeX
%% base system. You may however distribute the LaTeX base system without
%% such generated files.
%% 
%% The list of all files belonging to the LaTeX base distribution is
%% given in the file `manifest.txt'. See also `legal.txt' for additional
%% information.
%% 
%% The list of derived (unpacked) files belonging to the distribution
%% and covered by LPPL is defined by the unpacking scripts (with
%% extension .ins) which are part of the distribution.
%%% From File: ltoutenc.dtx
\ProvidesFile{tuenc.def}
 [2020/08/10 v2.0s
      Standard LaTeX file]
\providecommand\UnicodeEncodingName{TU}
\begingroup\expandafter\expandafter\expandafter\endgroup
\expandafter\ifx\csname XeTeXrevision\endcsname\relax
  \begingroup\expandafter\expandafter\expandafter\endgroup
  \expandafter\ifx\csname directlua\endcsname\relax
    \PackageWarningNoLine{fontenc}
      {\UnicodeEncodingName\space
       encoding is only available with XeTeX and LuaTeX.\MessageBreak
       Defaulting to T1 encoding}
      \def\encodingdefault{T1}
    \expandafter\expandafter\expandafter\endinput
  \else
    \def\UnicodeFontTeXLigatures{+tlig;}
    \ifnum\luatexversion<110
      \def\reserved@a#1{%
        \def\@remove@tlig##1{\@remove@tlig@##1\@nil#1\@nil\relax}
        \def\@remove@tlig@##1#1{\@remove@tlig@@##1}}
      \edef\reserved@b{\detokenize{+tlig;}}
      \expandafter\reserved@a\expandafter{\reserved@b}
      \def\@remove@tlig@@#1\@nil#2\relax{#1}
      \def\remove@tlig#1{%
        \begingroup
        \font\remove@tlig
        \expandafter\@remove@tlig\expandafter{\fontname\font}%
        \remove@tlig
        \char#1\relax
        \endgroup
      }
    \else
      \newluafunction\@remove@tlig@@@@
      \now@and@everyjob{\directlua{
        local rawchar_func = token.create'@remove@tlig@@@@'.index
        local forcehmode = tex.forcehmode
        local put_next = token.put_next
        local glyph_id = node.id'glyph'
        local rawchar_token = token.new(rawchar_func, token.command_id'lua_call')
        lua.get_functions_table()[rawchar_func] = function()
          local mode = tex.nest.top.mode
          if mode == 1 or mode == -1 then
            put_next(rawchar_token)
            return forcehmode(true)
          end
          local n = node.new(glyph_id, 256)
          n.font = font.current()
          n.char = token.scan_int()
          return node.write(n)
        end
        token.set_lua('@remove@tlig@@@', rawchar_func, 'global', 'protected')
      }}
      \def\remove@tlig#1{\@remove@tlig@@@#1\relax}
    \fi
  \fi
\else
  \def\UnicodeFontTeXLigatures{mapping=tex-text;}
  \def\remove@tlig#1{\XeTeXglyph\numexpr\XeTeXcharglyph#1\relax}
\fi
\def\UnicodeFontFile#1#2{"[#1]:#2"}
\def\UnicodeFontName#1#2{"#1:#2"}
\DeclareFontEncoding\UnicodeEncodingName{}{}
\def\add@unicode@accent#1#2{%
  \if\relax\detokenize{#2}\relax^^a0\else#2\fi
  \char#1\relax}
\def\DeclareUnicodeAccent#1#2#3{%
  \DeclareTextCommand{#1}{#2}{\add@unicode@accent{#3}}%
}
{
\catcode\z@=11\relax
\gdef\DeclareUnicodeComposite#1#2#3{%
   \def\reserved@a##1##2{%
     \DeclareTextCompositeCommand#1\UnicodeEncodingName{#2}{%
   \iffontchar\font#3 ##2%
      \else ##1\fi}}%
    \expandafter\expandafter\expandafter\extract@default@composite
    \csname\UnicodeEncodingName\string#1\endcsname{#2}\@nil
   \bgroup
      \lccode\z@#3 %
      \lowercase{\egroup
      \expandafter\reserved@a\expandafter{\reserved@b}{^^@}}}%
}
\def\extract@default@composite#1{%
 \ifx\@text@composite#1%
   \expandafter\extract@default@composite@a
 \else
   \expandafter\extract@default@composite@b\expandafter#1%
 \fi}
\def\extract@default@composite@a#1\@text@composite#2\@nil{%
  \def\reserved@b{#2}}
\def\extract@default@composite@b#1#2\@nil{%
  \def\reserved@b{#1#2}}
\DeclareTextCommand\textquotesingle \UnicodeEncodingName{%
                                                \remove@tlig{"0027}}
\DeclareTextCommand\textasciigrave  \UnicodeEncodingName{%
                                                \remove@tlig{"0060}}
\DeclareTextCommand\textquotedbl    \UnicodeEncodingName{%
                                                \remove@tlig{"0022}}
\DeclareTextSymbol{\textdollar}          \UnicodeEncodingName{"0024}
\DeclareTextSymbol{\textless}            \UnicodeEncodingName{"003C}
\DeclareTextSymbol{\textgreater}         \UnicodeEncodingName{"003E}
\DeclareTextSymbol{\textbackslash}       \UnicodeEncodingName{"005C}
\DeclareTextSymbol{\textasciicircum}     \UnicodeEncodingName{"005E}
\DeclareTextSymbol{\textunderscore}      \UnicodeEncodingName{"005F}
\DeclareTextSymbol{\textbraceleft}       \UnicodeEncodingName{"007B}
\DeclareTextSymbol{\textbar}             \UnicodeEncodingName{"007C}
\DeclareTextSymbol{\textbraceright}      \UnicodeEncodingName{"007D}
\DeclareTextSymbol{\textasciitilde}      \UnicodeEncodingName{"007E}
\DeclareTextSymbol{\textexclamdown}      \UnicodeEncodingName{"00A1}
\DeclareTextSymbol{\textcent}            \UnicodeEncodingName{"00A2}
\DeclareTextSymbol{\textsterling}        \UnicodeEncodingName{"00A3}
\DeclareTextSymbol{\textcurrency}        \UnicodeEncodingName{"00A4}
\DeclareTextSymbol{\textyen}             \UnicodeEncodingName{"00A5}
\DeclareTextSymbol{\textbrokenbar}       \UnicodeEncodingName{"00A6}
\DeclareTextSymbol{\textsection}         \UnicodeEncodingName{"00A7}
\DeclareTextSymbol{\textasciidieresis}   \UnicodeEncodingName{"00A8}
\DeclareTextSymbol{\textcopyright}       \UnicodeEncodingName{"00A9}
\DeclareTextSymbol{\textordfeminine}     \UnicodeEncodingName{"00AA}
\DeclareTextSymbol{\guillemetleft}       \UnicodeEncodingName{"00AB}
\DeclareTextSymbol{\guillemotleft}       \UnicodeEncodingName{"00AB}
\DeclareTextSymbol{\textlnot}            \UnicodeEncodingName{"00AC}
\DeclareTextSymbol{\textregistered}      \UnicodeEncodingName{"00AE}
\DeclareTextSymbol{\textasciimacron}     \UnicodeEncodingName{"00AF}
\DeclareTextSymbol{\textdegree}          \UnicodeEncodingName{"00B0}
\DeclareTextSymbol{\textpm}              \UnicodeEncodingName{"00B1}
\DeclareTextSymbol{\texttwosuperior}     \UnicodeEncodingName{"00B2}
\DeclareTextSymbol{\textthreesuperior}   \UnicodeEncodingName{"00B3}
\DeclareTextSymbol{\textasciiacute}      \UnicodeEncodingName{"00B4}
\DeclareTextSymbol{\textmu}              \UnicodeEncodingName{"00B5}
\DeclareTextSymbol{\textparagraph}       \UnicodeEncodingName{"00B6}
\DeclareTextSymbol{\textperiodcentered}  \UnicodeEncodingName{"00B7}
\DeclareTextSymbol{\textonesuperior}     \UnicodeEncodingName{"00B9}
\DeclareTextSymbol{\textordmasculine}    \UnicodeEncodingName{"00BA}
\DeclareTextSymbol{\guillemetright}      \UnicodeEncodingName{"00BB}
\DeclareTextSymbol{\guillemotright}      \UnicodeEncodingName{"00BB}
\DeclareTextSymbol{\textonequarter}      \UnicodeEncodingName{"00BC}
\DeclareTextSymbol{\textonehalf}         \UnicodeEncodingName{"00BD}
\DeclareTextSymbol{\textthreequarters}   \UnicodeEncodingName{"00BE}
\DeclareTextSymbol{\textquestiondown}    \UnicodeEncodingName{"00BF}
\DeclareTextSymbol{\AE}                  \UnicodeEncodingName{"00C6}
\DeclareTextSymbol{\DH}                  \UnicodeEncodingName{"00D0}
\DeclareTextSymbol{\texttimes}           \UnicodeEncodingName{"00D7}
\DeclareTextSymbol{\O}                   \UnicodeEncodingName{"00D8}
\DeclareTextSymbol{\TH}                  \UnicodeEncodingName{"00DE}
\DeclareTextSymbol{\ss}                  \UnicodeEncodingName{"00DF}
\DeclareTextSymbol{\ae}                  \UnicodeEncodingName{"00E6}
\DeclareTextSymbol{\dh}                  \UnicodeEncodingName{"00F0}
\DeclareTextSymbol{\textdiv}             \UnicodeEncodingName{"00F7}
\DeclareTextSymbol{\o}                   \UnicodeEncodingName{"00F8}
\DeclareTextSymbol{\th}                  \UnicodeEncodingName{"00FE}
\DeclareTextSymbol{\DJ}                  \UnicodeEncodingName{"0110}
\DeclareTextSymbol{\dj}                  \UnicodeEncodingName{"0111}
\DeclareTextSymbol{\i}                   \UnicodeEncodingName{"0131}
\DeclareTextSymbol{\IJ}                  \UnicodeEncodingName{"0132}
\DeclareTextSymbol{\ij}                  \UnicodeEncodingName{"0133}
\DeclareTextSymbol{\L}                   \UnicodeEncodingName{"0141}
\DeclareTextSymbol{\l}                   \UnicodeEncodingName{"0142}
\DeclareTextSymbol{\NG}                  \UnicodeEncodingName{"014A}
\DeclareTextSymbol{\ng}                  \UnicodeEncodingName{"014B}
\DeclareTextSymbol{\OE}                  \UnicodeEncodingName{"0152}
\DeclareTextSymbol{\oe}                  \UnicodeEncodingName{"0153}
\DeclareTextSymbol{\textflorin}          \UnicodeEncodingName{"0192}
\DeclareTextSymbol{\j}                   \UnicodeEncodingName{"0237}
\DeclareTextSymbol{\textasciicaron}      \UnicodeEncodingName{"02C7}
\DeclareTextSymbol{\textasciibreve}      \UnicodeEncodingName{"02D8}
\DeclareTextSymbol{\textacutedbl}        \UnicodeEncodingName{"02DD}
\DeclareTextSymbol{\textgravedbl}        \UnicodeEncodingName{"02F5}
\DeclareTextSymbol{\texttildelow}        \UnicodeEncodingName{"02F7}
\DeclareTextSymbol{\textbaht}            \UnicodeEncodingName{"0E3F}
\DeclareTextSymbol{\SS}                  \UnicodeEncodingName{"1E9E}
\DeclareTextSymbol{\textcompwordmark}    \UnicodeEncodingName{"200C}
\DeclareTextSymbol{\textendash}          \UnicodeEncodingName{"2013}
\DeclareTextSymbol{\textemdash}          \UnicodeEncodingName{"2014}
\DeclareTextSymbol{\textbardbl}          \UnicodeEncodingName{"2016}
\DeclareTextSymbol{\textquoteleft}       \UnicodeEncodingName{"2018}
\DeclareTextSymbol{\textquoteright}      \UnicodeEncodingName{"2019}
\DeclareTextSymbol{\quotesinglbase}      \UnicodeEncodingName{"201A}
\DeclareTextSymbol{\textquotedblleft}    \UnicodeEncodingName{"201C}
\DeclareTextSymbol{\textquotedblright}   \UnicodeEncodingName{"201D}
\DeclareTextSymbol{\quotedblbase}        \UnicodeEncodingName{"201E}
\DeclareTextSymbol{\textdagger}          \UnicodeEncodingName{"2020}
\DeclareTextSymbol{\textdaggerdbl}       \UnicodeEncodingName{"2021}
\DeclareTextSymbol{\textbullet}          \UnicodeEncodingName{"2022}
\DeclareTextSymbol{\textellipsis}        \UnicodeEncodingName{"2026}
\DeclareTextSymbol{\textperthousand}     \UnicodeEncodingName{"2030}
\DeclareTextSymbol{\textpertenthousand}  \UnicodeEncodingName{"2031}
\DeclareTextSymbol{\guilsinglleft}       \UnicodeEncodingName{"2039}
\DeclareTextSymbol{\guilsinglright}      \UnicodeEncodingName{"203A}
\DeclareTextSymbol{\textreferencemark}   \UnicodeEncodingName{"203B}
\DeclareTextSymbol{\textinterrobang}     \UnicodeEncodingName{"203D}
\DeclareTextSymbol{\textfractionsolidus} \UnicodeEncodingName{"2044}
\DeclareTextSymbol{\textlquill}          \UnicodeEncodingName{"2045}
\DeclareTextSymbol{\textrquill}          \UnicodeEncodingName{"2046}
\DeclareTextSymbol{\textdiscount}        \UnicodeEncodingName{"2052}
\DeclareTextSymbol{\textcolonmonetary}   \UnicodeEncodingName{"20A1}
\DeclareTextSymbol{\textlira}            \UnicodeEncodingName{"20A4}
\DeclareTextSymbol{\textnaira}           \UnicodeEncodingName{"20A6}
\DeclareTextSymbol{\textwon}             \UnicodeEncodingName{"20A9}
\DeclareTextSymbol{\textdong}            \UnicodeEncodingName{"20AB}
\DeclareTextSymbol{\texteuro}            \UnicodeEncodingName{"20AC}
\DeclareTextSymbol{\textpeso}            \UnicodeEncodingName{"20B1}
\DeclareTextSymbol{\textcelsius}         \UnicodeEncodingName{"2103}
\DeclareTextSymbol{\textnumero}          \UnicodeEncodingName{"2116}
\DeclareTextSymbol{\textcircledP}        \UnicodeEncodingName{"2117}
\DeclareTextSymbol{\textrecipe}          \UnicodeEncodingName{"211E}
\DeclareTextSymbol{\textservicemark}     \UnicodeEncodingName{"2120}
\DeclareTextSymbol{\texttrademark}       \UnicodeEncodingName{"2122}
\DeclareTextSymbol{\textohm}             \UnicodeEncodingName{"2126}
\DeclareTextSymbol{\textmho}             \UnicodeEncodingName{"2127}
\DeclareTextSymbol{\textestimated}       \UnicodeEncodingName{"212E}
\DeclareTextSymbol{\textleftarrow}       \UnicodeEncodingName{"2190}
\DeclareTextSymbol{\textuparrow}         \UnicodeEncodingName{"2191}
\DeclareTextSymbol{\textrightarrow}      \UnicodeEncodingName{"2192}
\DeclareTextSymbol{\textdownarrow}       \UnicodeEncodingName{"2193}
\DeclareTextSymbol{\textminus}           \UnicodeEncodingName{"2212}

\DeclareTextSymbol{\Hwithstroke}         \UnicodeEncodingName{"0126}
\DeclareTextSymbol{\hwithstroke}         \UnicodeEncodingName{"0127}
\DeclareTextCommand{\textasteriskcentered}\UnicodeEncodingName{%
  \iffontchar\font"2217 \char"2217 \else
    \begingroup
      \fontsize
       {\the\dimexpr1.2\dimexpr\f@size pt\relax}%
       {\f@baselineskip}%
      \selectfont
      \raisebox{-0.6ex}[\dimexpr\height-0.6ex][0pt]{*}%
    \endgroup
  \fi
}
\DeclareTextSymbol{\textsurd}            \UnicodeEncodingName{"221A}
\DeclareTextSymbol{\textlangle}          \UnicodeEncodingName{"2329}
\DeclareTextSymbol{\textrangle}          \UnicodeEncodingName{"232A}
\DeclareTextSymbol{\textblank}           \UnicodeEncodingName{"2422}
\DeclareTextSymbol{\textvisiblespace}    \UnicodeEncodingName{"2423}
\DeclareTextSymbol{\textopenbullet}      \UnicodeEncodingName{"25E6}
\DeclareTextSymbol{\textbigcircle}       \UnicodeEncodingName{"25EF}
\DeclareTextSymbol{\textmusicalnote}     \UnicodeEncodingName{"266A}
\DeclareTextSymbol{\textmarried}         \UnicodeEncodingName{"26AD}
\DeclareTextSymbol{\textdivorced}        \UnicodeEncodingName{"26AE}
\DeclareTextSymbol{\textinterrobangdown} \UnicodeEncodingName{"2E18}
\DeclareUnicodeAccent{\`}                \UnicodeEncodingName{"0300}
\DeclareUnicodeAccent{\'}                \UnicodeEncodingName{"0301}
\DeclareUnicodeAccent{\^}                \UnicodeEncodingName{"0302}
\DeclareUnicodeAccent{\~}                \UnicodeEncodingName{"0303}
\DeclareUnicodeAccent{\=}                \UnicodeEncodingName{"0304}
\DeclareUnicodeAccent{\u}                \UnicodeEncodingName{"0306}
\DeclareUnicodeAccent{\.}                \UnicodeEncodingName{"0307}
\DeclareUnicodeAccent{\"}                \UnicodeEncodingName{"0308}
\DeclareUnicodeAccent{\r}                \UnicodeEncodingName{"030A}
\DeclareUnicodeAccent{\H}                \UnicodeEncodingName{"030B}
\DeclareUnicodeAccent{\v}                \UnicodeEncodingName{"030C}
\DeclareUnicodeAccent{\b}                \UnicodeEncodingName{"0332}
\DeclareUnicodeAccent{\d}                \UnicodeEncodingName{"0323}
\DeclareUnicodeAccent{\c}                \UnicodeEncodingName{"0327}
\DeclareUnicodeAccent{\k}                \UnicodeEncodingName{"0328}
\DeclareTextCommand\textcommabelow       \UnicodeEncodingName[1]
  {\hmode@bgroup\ooalign{\null#1\crcr\hidewidth\raise-.31ex
   \hbox{\check@mathfonts\fontsize\ssf@size\z@
   \math@fontsfalse\selectfont,}\hidewidth}\egroup}
\DeclareUnicodeComposite{\^}              {}{"005E}
\DeclareUnicodeComposite{\~}              {}{"007E}
\DeclareUnicodeComposite{\`}             {A}{"00C0}
\DeclareUnicodeComposite{\'}             {A}{"00C1}
\DeclareUnicodeComposite{\^}             {A}{"00C2}
\DeclareUnicodeComposite{\~}             {A}{"00C3}
\DeclareUnicodeComposite{\"}             {A}{"00C4}
\DeclareUnicodeComposite{\r}             {A}{"00C5}
\DeclareUnicodeComposite{\c}             {C}{"00C7}
\DeclareUnicodeComposite{\`}             {E}{"00C8}
\DeclareUnicodeComposite{\'}             {E}{"00C9}
\DeclareUnicodeComposite{\^}             {E}{"00CA}
\DeclareUnicodeComposite{\"}             {E}{"00CB}
\DeclareUnicodeComposite{\`}             {I}{"00CC}
\DeclareUnicodeComposite{\'}             {I}{"00CD}
\DeclareUnicodeComposite{\^}             {I}{"00CE}
\DeclareUnicodeComposite{\"}             {I}{"00CF}
\DeclareUnicodeComposite{\~}             {N}{"00D1}
\DeclareUnicodeComposite{\`}             {O}{"00D2}
\DeclareUnicodeComposite{\'}             {O}{"00D3}
\DeclareUnicodeComposite{\^}             {O}{"00D4}
\DeclareUnicodeComposite{\~}             {O}{"00D5}
\DeclareUnicodeComposite{\"}             {O}{"00D6}
\DeclareUnicodeComposite{\`}             {U}{"00D9}
\DeclareUnicodeComposite{\'}             {U}{"00DA}
\DeclareUnicodeComposite{\^}             {U}{"00DB}
\DeclareUnicodeComposite{\"}             {U}{"00DC}
\DeclareUnicodeComposite{\'}             {Y}{"00DD}
\DeclareUnicodeComposite{\`}             {a}{"00E0}
\DeclareUnicodeComposite{\'}             {a}{"00E1}
\DeclareUnicodeComposite{\^}             {a}{"00E2}
\DeclareUnicodeComposite{\~}             {a}{"00E3}
\DeclareUnicodeComposite{\"}             {a}{"00E4}
\DeclareUnicodeComposite{\r}             {a}{"00E5}
\DeclareUnicodeComposite{\c}             {c}{"00E7}
\DeclareUnicodeComposite{\`}             {e}{"00E8}
\DeclareUnicodeComposite{\'}             {e}{"00E9}
\DeclareUnicodeComposite{\^}             {e}{"00EA}
\DeclareUnicodeComposite{\"}             {e}{"00EB}
\DeclareUnicodeComposite{\`}             \i {"00EC}
\DeclareUnicodeComposite{\`}             {i}{"00EC}
\DeclareUnicodeComposite{\'}             \i {"00ED}
\DeclareUnicodeComposite{\'}             {i}{"00ED}
\DeclareUnicodeComposite{\^}             \i {"00EE}
\DeclareUnicodeComposite{\^}             {i}{"00EE}
\DeclareUnicodeComposite{\"}             \i {"00EF}
\DeclareUnicodeComposite{\"}             {i}{"00EF}
\DeclareUnicodeComposite{\~}             {n}{"00F1}
\DeclareUnicodeComposite{\`}             {o}{"00F2}
\DeclareUnicodeComposite{\'}             {o}{"00F3}
\DeclareUnicodeComposite{\^}             {o}{"00F4}
\DeclareUnicodeComposite{\~}             {o}{"00F5}
\DeclareUnicodeComposite{\"}             {o}{"00F6}
\DeclareUnicodeComposite{\`}             {u}{"00F9}
\DeclareUnicodeComposite{\'}             {u}{"00FA}
\DeclareUnicodeComposite{\^}             {u}{"00FB}
\DeclareUnicodeComposite{\"}             {u}{"00FC}
\DeclareUnicodeComposite{\'}             {y}{"00FD}
\DeclareUnicodeComposite{\"}             {y}{"00FF}
\DeclareUnicodeComposite{\=}             {A}{"0100}
\DeclareUnicodeComposite{\=}             {a}{"0101}
\DeclareUnicodeComposite{\u}             {A}{"0102}
\DeclareUnicodeComposite{\u}             {a}{"0103}
\DeclareUnicodeComposite{\k}             {A}{"0104}
\DeclareUnicodeComposite{\k}             {a}{"0105}
\DeclareUnicodeComposite{\'}             {C}{"0106}
\DeclareUnicodeComposite{\'}             {c}{"0107}
\DeclareUnicodeComposite{\^}             {C}{"0108}
\DeclareUnicodeComposite{\^}             {c}{"0109}
\DeclareUnicodeComposite{\.}             {C}{"010A}
\DeclareUnicodeComposite{\.}             {c}{"010B}
\DeclareUnicodeComposite{\v}             {C}{"010C}
\DeclareUnicodeComposite{\v}             {c}{"010D}
\DeclareUnicodeComposite{\v}             {D}{"010E}
\DeclareUnicodeComposite{\v}             {d}{"010F}
\DeclareUnicodeComposite{\=}             {E}{"0112}
\DeclareUnicodeComposite{\=}             {e}{"0113}
\DeclareUnicodeComposite{\u}             {E}{"0114}
\DeclareUnicodeComposite{\u}             {e}{"0115}
\DeclareUnicodeComposite{\.}             {E}{"0116}
\DeclareUnicodeComposite{\.}             {e}{"0117}
\DeclareUnicodeComposite{\k}             {E}{"0118}
\DeclareUnicodeComposite{\k}             {e}{"0119}
\DeclareUnicodeComposite{\v}             {E}{"011A}
\DeclareUnicodeComposite{\v}             {e}{"011B}
\DeclareUnicodeComposite{\^}             {G}{"011C}
\DeclareUnicodeComposite{\^}             {g}{"011D}
\DeclareUnicodeComposite{\u}             {G}{"011E}
\DeclareUnicodeComposite{\u}             {g}{"011F}
\DeclareUnicodeComposite{\.}             {G}{"0120}
\DeclareUnicodeComposite{\.}             {g}{"0121}
\DeclareUnicodeComposite{\c}             {G}{"0122}
\DeclareUnicodeComposite{\c}             {g}{"0123}
\DeclareUnicodeComposite{\^}             {H}{"0124}
\DeclareUnicodeComposite{\^}             {h}{"0125}
\DeclareUnicodeComposite{\~}             {I}{"0128}
\DeclareUnicodeComposite{\~}             \i {"0129}
\DeclareUnicodeComposite{\~}             {i}{"0129}
\DeclareUnicodeComposite{\=}             {I}{"012A}
\DeclareUnicodeComposite{\=}             \i {"012B}
\DeclareUnicodeComposite{\=}             {i}{"012B}
\DeclareUnicodeComposite{\u}             {I}{"012C}
\DeclareUnicodeComposite{\u}             \i {"012D}
\DeclareUnicodeComposite{\u}             {i}{"012D}
\DeclareUnicodeComposite{\k}             {I}{"012E}
\DeclareUnicodeComposite{\k}             \i {"012F}
\DeclareUnicodeComposite{\k}             {i}{"012F}
\DeclareUnicodeComposite{\.}             {I}{"0130}
\DeclareUnicodeComposite{\^}             {J}{"0134}
\DeclareUnicodeComposite{\^}             \j {"0135}
\DeclareUnicodeComposite{\^}             {j}{"0135}
\DeclareUnicodeComposite{\c}             {K}{"0136}
\DeclareUnicodeComposite{\c}             {k}{"0137}
\DeclareUnicodeComposite{\'}             {L}{"0139}
\DeclareUnicodeComposite{\'}             {l}{"013A}
\DeclareUnicodeComposite{\c}             {L}{"013B}
\DeclareUnicodeComposite{\c}             {l}{"013C}
\DeclareUnicodeComposite{\v}             {L}{"013D}
\DeclareUnicodeComposite{\v}             {l}{"013E}
\DeclareUnicodeComposite{\'}             {N}{"0143}
\DeclareUnicodeComposite{\'}             {n}{"0144}
\DeclareUnicodeComposite{\c}             {N}{"0145}
\DeclareUnicodeComposite{\c}             {n}{"0146}
\DeclareUnicodeComposite{\v}             {N}{"0147}
\DeclareUnicodeComposite{\v}             {n}{"0148}
\DeclareUnicodeComposite{\=}             {O}{"014C}
\DeclareUnicodeComposite{\=}             {o}{"014D}
\DeclareUnicodeComposite{\u}             {O}{"014E}
\DeclareUnicodeComposite{\u}             {o}{"014F}
\DeclareUnicodeComposite{\H}             {O}{"0150}
\DeclareUnicodeComposite{\H}             {o}{"0151}
\DeclareUnicodeComposite{\'}             {R}{"0154}
\DeclareUnicodeComposite{\'}             {r}{"0155}
\DeclareUnicodeComposite{\c}             {R}{"0156}
\DeclareUnicodeComposite{\c}             {r}{"0157}
\DeclareUnicodeComposite{\v}             {R}{"0158}
\DeclareUnicodeComposite{\v}             {r}{"0159}
\DeclareUnicodeComposite{\'}             {S}{"015A}
\DeclareUnicodeComposite{\'}             {s}{"015B}
\DeclareUnicodeComposite{\^}             {S}{"015C}
\DeclareUnicodeComposite{\^}             {s}{"015D}
\DeclareUnicodeComposite{\c}             {S}{"015E}
\DeclareUnicodeComposite{\c}             {s}{"015F}
\DeclareUnicodeComposite{\v}             {S}{"0160}
\DeclareUnicodeComposite{\v}             {s}{"0161}
\DeclareUnicodeComposite{\c}             {T}{"0162}
\DeclareUnicodeComposite{\c}             {t}{"0163}
\DeclareUnicodeComposite{\v}             {T}{"0164}
\DeclareUnicodeComposite{\v}             {t}{"0165}
\DeclareUnicodeComposite{\~}             {U}{"0168}
\DeclareUnicodeComposite{\~}             {u}{"0169}
\DeclareUnicodeComposite{\=}             {U}{"016A}
\DeclareUnicodeComposite{\=}             {u}{"016B}
\DeclareUnicodeComposite{\u}             {U}{"016C}
\DeclareUnicodeComposite{\u}             {u}{"016D}
\DeclareUnicodeComposite{\r}             {U}{"016E}
\DeclareUnicodeComposite{\r}             {u}{"016F}
\DeclareUnicodeComposite{\H}             {U}{"0170}
\DeclareUnicodeComposite{\H}             {u}{"0171}
\DeclareUnicodeComposite{\k}             {U}{"0172}
\DeclareUnicodeComposite{\k}             {u}{"0173}
\DeclareUnicodeComposite{\^}             {W}{"0174}
\DeclareUnicodeComposite{\^}             {w}{"0175}
\DeclareUnicodeComposite{\^}             {Y}{"0176}
\DeclareUnicodeComposite{\^}             {y}{"0177}
\DeclareUnicodeComposite{\"}             {Y}{"0178}
\DeclareUnicodeComposite{\'}             {Z}{"0179}
\DeclareUnicodeComposite{\'}             {z}{"017A}
\DeclareUnicodeComposite{\.}             {Z}{"017B}
\DeclareUnicodeComposite{\.}             {z}{"017C}
\DeclareUnicodeComposite{\v}             {Z}{"017D}
\DeclareUnicodeComposite{\v}             {z}{"017E}
\DeclareUnicodeComposite{\v}             {A}{"01CD}
\DeclareUnicodeComposite{\v}             {a}{"01CE}
\DeclareUnicodeComposite{\v}             {I}{"01CF}
\DeclareUnicodeComposite{\v}             \i {"01D0}
\DeclareUnicodeComposite{\v}             {i}{"01D0}
\DeclareUnicodeComposite{\v}             {O}{"01D1}
\DeclareUnicodeComposite{\v}             {o}{"01D2}
\DeclareUnicodeComposite{\v}             {U}{"01D3}
\DeclareUnicodeComposite{\v}             {u}{"01D4}
\DeclareUnicodeComposite{\=}             \AE{"01E2}
\DeclareUnicodeComposite{\=}             \ae{"01E3}
\DeclareUnicodeComposite{\v}             {G}{"01E6}
\DeclareUnicodeComposite{\v}             {g}{"01E7}
\DeclareUnicodeComposite{\v}             {K}{"01E8}
\DeclareUnicodeComposite{\v}             {k}{"01E9}
\DeclareUnicodeComposite{\k}             {O}{"01EA}
\DeclareUnicodeComposite{\k}             {o}{"01EB}
\DeclareUnicodeComposite{\v}             \j {"01F0}
\DeclareUnicodeComposite{\v}             {j}{"01F0}
\DeclareUnicodeComposite{\'}             {G}{"01F4}
\DeclareUnicodeComposite{\'}             {g}{"01F5}
\DeclareUnicodeComposite{\textcommabelow}{S}{"0218}
\DeclareUnicodeComposite{\textcommabelow}{s}{"0219}
\DeclareUnicodeComposite{\textcommabelow}{T}{"021A}
\DeclareUnicodeComposite{\textcommabelow}{t}{"021B}
\DeclareUnicodeComposite{\=}             {Y}{"0232}
\DeclareUnicodeComposite{\=}             {y}{"0233}
\DeclareUnicodeComposite{\.}             {B}{"1E02}
\DeclareUnicodeComposite{\.}             {b}{"1E03}
\DeclareUnicodeComposite{\d}             {B}{"1E04}
\DeclareUnicodeComposite{\d}             {b}{"1E05}
\DeclareUnicodeComposite{\d}             {D}{"1E0C}
\DeclareUnicodeComposite{\d}             {d}{"1E0D}
\DeclareUnicodeComposite{\=}             {G}{"1E20}
\DeclareUnicodeComposite{\=}             {g}{"1E21}
\DeclareUnicodeComposite{\d}             {H}{"1E24}
\DeclareUnicodeComposite{\d}             {h}{"1E25}
\DeclareUnicodeComposite{\d}             {K}{"1E32}
\DeclareUnicodeComposite{\d}             {k}{"1E33}
\DeclareUnicodeComposite{\d}             {L}{"1E36}
\DeclareUnicodeComposite{\d}             {l}{"1E37}
\DeclareUnicodeComposite{\d}             {M}{"1E42}
\DeclareUnicodeComposite{\d}             {m}{"1E43}
\DeclareUnicodeComposite{\d}             {N}{"1E46}
\DeclareUnicodeComposite{\d}             {n}{"1E47}
\DeclareUnicodeComposite{\d}             {R}{"1E5A}
\DeclareUnicodeComposite{\d}             {r}{"1E5B}
\DeclareUnicodeComposite{\d}             {S}{"1E62}
\DeclareUnicodeComposite{\d}             {s}{"1E63}
\DeclareUnicodeComposite{\d}             {T}{"1E6C}
\DeclareUnicodeComposite{\d}             {t}{"1E6D}
\DeclareUnicodeComposite{\d}             {V}{"1E7E}
\DeclareUnicodeComposite{\d}             {v}{"1E7F}
\DeclareUnicodeComposite{\d}             {W}{"1E88}
\DeclareUnicodeComposite{\d}             {w}{"1E89}
\DeclareUnicodeComposite{\d}             {Z}{"1E92}
\DeclareUnicodeComposite{\d}             {z}{"1E93}
\DeclareUnicodeComposite{\d}             {A}{"1EA0}
\DeclareUnicodeComposite{\d}             {a}{"1EA1}
\DeclareUnicodeComposite{\d}             {E}{"1EB8}
\DeclareUnicodeComposite{\d}             {e}{"1EB9}
\DeclareUnicodeComposite{\d}             {I}{"1ECA}
\DeclareUnicodeComposite{\d}             {i}{"1ECB}
\DeclareUnicodeComposite{\d}             {O}{"1ECC}
\DeclareUnicodeComposite{\d}             {o}{"1ECD}
\DeclareUnicodeComposite{\d}             {U}{"1EE4}
\DeclareUnicodeComposite{\d}             {u}{"1EE5}
\DeclareUnicodeComposite{\d}             {Y}{"1EF4}
\DeclareUnicodeComposite{\d}             {y}{"1EF5}
\endinput
%%
%% End of file `tuenc.def'.

\fontencoding{TU}
%    \end{macrocode}
%
%    The initial \texttt{fontenc} package load list if a Unicode
%    engine is used:
% \changes{v3.0g}{2020/02/11}{Provide value for \cs{@fontenc@load@list} (gh/273)}
%    \begin{macrocode}
\def\@fontenc@load@list{\@elt{TU}}
%    \end{macrocode}
%
%    \begin{macrocode}
\DeclareFontSubstitution{TU}{lmr}{m}{n}
\LoadFontDefinitionFile{TU}{lmr}
\LoadFontDefinitionFile{TU}{lmss}
\LoadFontDefinitionFile{TU}{lmtt}
%    \end{macrocode}
%
%    \begin{macrocode}
\def\rmsubstdefault{lmr}
\def\sfsubstdefault{lmss}
\def\ttsubstdefault{lmtt}
\LoadFontDefinitionFile{TS1}{lmr}
%    \end{macrocode}
%
%    \begin{macrocode}
\DeclareFontSubstitution{TU}{lmr}{m}{n}
%    \end{macrocode}
% End of Unicode branch.
%    \begin{macrocode}
\fi
%    \end{macrocode}
%
%    If different encodings for text fonts are in use one could put
%    the common setup into |\DeclareFontEncodingDefaults|. There is
%    now a better mechanism so using this interface is discouraged!
%    \begin{macrocode}
\DeclareFontEncodingDefaults{}{}
%    \end{macrocode}
%
%    Then we define the default substitution for every encoding.
%    This release of \LaTeXe{} assumes that the ec fonts are
%    available. It is possible to change this to point to some other
%    font family (e.g., Times with the appropriate encoding if it is
%    available) without making documents non-portable. However, in
%    such a case documents will produce different page breaks at other
%    sites. The substitution defaults can all be changed without
%    losing portability as long as there are font shape definitions
%    for the selected substitutions.
%    \begin{macrocode}
\DeclareFontSubstitution{T1}{cmr}{m}{n}
\DeclareFontSubstitution{OT1}{cmr}{m}{n}
%    \end{macrocode}
%
%    For every encoding declaration, \LaTeXe{} will try to verify that
%    the given substitution information makes sense, i.e.~that it is
%    impossible to go into an endless loop if font substitution
%    happens. This is done at the moment the |\begin{document}| is
%    encountered. \LaTeXe{} will then check that for every encoding the
%    substitution defaults form a valid font shape group, which means
%    that it will check if there is a |\DeclareFontShape| declaration
%    for this combination. We will therefore load the corresponding
%    |.fd| files now. If we don't do this they would be loaded at
%    verification time (i.e.~at |\begin{document}| which would delay
%    processing unnecessarily.
%
%    \begin{quote}
%       \textbf{Warning:} Please note that this means that you have to
%       regenerate the format whenever you change any of these
%       \texttt{.fd} files since \LaTeXe{} will not read \texttt{.fd}
%       files if it already knows about the encoding/family
%       combination.
%    \end{quote}
%
% \changes{v2.2m}{1995/11/01}{add \cs{nfss@catcodes} for internal/1932}
% The |\nfss@catcodes| ensures that white space is ignored in any
% definitions made in the fd files.
%    \begin{macrocode}
\begingroup
\nfss@catcodes
%%
%% This is file `t1cmr.fd',
%% generated with the docstrip utility.
%%
%% The original source files were:
%%
%% cmfonts.fdd  (with options: `fd,T1cmr,ec')
%% 
%% This is a generated file.
%% 
%% The source is maintained by the LaTeX Project team and bug
%% reports for it can be opened at https://latex-project.org/bugs.html
%% (but please observe conditions on bug reports sent to that address!)
%% 
%% 
%% Copyright (C) 1993-2020
%% The LaTeX3 Project and any individual authors listed elsewhere
%% in this file.
%% 
%% This file was generated from file(s) of the LaTeX base system.
%% --------------------------------------------------------------
%% 
%% It may be distributed and/or modified under the
%% conditions of the LaTeX Project Public License, either version 1.3c
%% of this license or (at your option) any later version.
%% The latest version of this license is in
%%    https://www.latex-project.org/lppl.txt
%% and version 1.3c or later is part of all distributions of LaTeX
%% version 2008 or later.
%% 
%% This file may only be distributed together with a copy of the LaTeX
%% base system. You may however distribute the LaTeX base system without
%% such generated files.
%% 
%% The list of all files belonging to the LaTeX base distribution is
%% given in the file `manifest.txt'. See also `legal.txt' for additional
%% information.
%% 
%% In particular, permission is granted to customize the declarations in
%% this file to serve the needs of your installation.
%% 
%% However, a modified version of this file under its original name
%% should not be distributed as part of a standard LaTeX distribution, as
%% the modification will be nontransparent for the user of that
%% distribution (and thus violating clause 6a of the LPPL license),
%% making successful document exchange impossible.
%% 
\ProvidesFile{t1cmr.fd}
        [2019/12/16 v2.5j Standard LaTeX font definitions]
\providecommand{\EC@family}[5]{%
  \DeclareFontShape{#1}{#2}{#3}{#4}%
  {<5><6><7><8><9><10><10.95><12><14.4>%
   <17.28><20.74><24.88><29.86><35.83>genb*#5}{}}
\DeclareFontFamily{T1}{cmr}{}
\EC@family{T1}{cmr}{m}{n}{ecrm}
\EC@family{T1}{cmr}{m}{sl}{ecsl}
\EC@family{T1}{cmr}{m}{it}{ecti}
\EC@family{T1}{cmr}{m}{sc}{eccc}
\EC@family{T1}{cmr}{bx}{n}{ecbx}
\EC@family{T1}{cmr}{b}{n}{ecrb}
\EC@family{T1}{cmr}{bx}{it}{ecbi}
\EC@family{T1}{cmr}{bx}{sl}{ecbl}
\EC@family{T1}{cmr}{bx}{sc}{ecxc}
\EC@family{T1}{cmr}{m}{ui}{ecui}
\endinput
%%
%% End of file `t1cmr.fd'.

%%
%% This is file `ot1cmr.fd',
%% generated with the docstrip utility.
%%
%% The original source files were:
%%
%% cmfonts.fdd  (with options: `OT1cmr')
%% 
%% This is a generated file.
%% 
%% The source is maintained by the LaTeX Project team and bug
%% reports for it can be opened at https://latex-project.org/bugs.html
%% (but please observe conditions on bug reports sent to that address!)
%% 
%% 
%% Copyright (C) 1993-2020
%% The LaTeX3 Project and any individual authors listed elsewhere
%% in this file.
%% 
%% This file was generated from file(s) of the LaTeX base system.
%% --------------------------------------------------------------
%% 
%% It may be distributed and/or modified under the
%% conditions of the LaTeX Project Public License, either version 1.3c
%% of this license or (at your option) any later version.
%% The latest version of this license is in
%%    https://www.latex-project.org/lppl.txt
%% and version 1.3c or later is part of all distributions of LaTeX
%% version 2008 or later.
%% 
%% This file may only be distributed together with a copy of the LaTeX
%% base system. You may however distribute the LaTeX base system without
%% such generated files.
%% 
%% The list of all files belonging to the LaTeX base distribution is
%% given in the file `manifest.txt'. See also `legal.txt' for additional
%% information.
%% 
%% In particular, permission is granted to customize the declarations in
%% this file to serve the needs of your installation.
%% 
%% However, a modified version of this file under its original name
%% should not be distributed as part of a standard LaTeX distribution, as
%% the modification will be nontransparent for the user of that
%% distribution (and thus violating clause 6a of the LPPL license),
%% making successful document exchange impossible.
%% 
\ProvidesFile{ot1cmr.fd}
        [2019/12/16 v2.5j Standard LaTeX font definitions]
\DeclareFontFamily{OT1}{cmr}{\hyphenchar\font45 }
\DeclareFontShape{OT1}{cmr}{m}{n}%
     {<5><6><7><8><9><10><12>gen*cmr%
      <10.95>cmr10%
      <14.4>cmr12%
      <17.28><20.74><24.88>cmr17}{}
\DeclareFontShape{OT1}{cmr}{m}{sl}%
     {%
      <5><6><7>cmsl8%
      <8><9>gen*cmsl%
      <10><10.95>cmsl10%
      <12><14.4><17.28><20.74><24.88>cmsl12%
      }{}
\DeclareFontShape{OT1}{cmr}{m}{it}%
     {%
      <5><6><7>cmti7%
      <8>cmti8%
      <9>cmti9%
      <10><10.95>cmti10%
      <12><14.4><17.28><20.74><24.88>cmti12%
      }{}
\DeclareFontShape{OT1}{cmr}{m}{sc}%
     {%
      <5><6><7><8><9><10><10.95><12>%
      <14.4><17.28><20.74><24.88>cmcsc10%
      }{}
% Warning: please note that the upright shape below is
%          used for the \pounds symbol of LaTeX. So this
%          font definition shouldn't be removed.
%
\DeclareFontShape{OT1}{cmr}{m}{ui}
   {
      <5><6><7><8><9><10><10.95><12>%
      <14.4><17.28><20.74><24.88>cmu10%
      }{}
%%%%%%% bold series
\DeclareFontShape{OT1}{cmr}{b}{n}
     {%
      <5><6><7><8><9><10><10.95><12>%
      <14.4><17.28><20.74><24.88>cmb10%
      }{}
%%%%%%%% bold extended series
\DeclareFontShape{OT1}{cmr}{bx}{n}
   {%
      <5><6><7><8><9>gen*cmbx%
      <10><10.95>cmbx10%
      <12><14.4><17.28><20.74><24.88>cmbx12%
      }{}
\DeclareFontShape{OT1}{cmr}{bx}{sl}
      {%
      <5><6><7><8><9>%
      <10><10.95><12><14.4><17.28><20.74><24.88>cmbxsl10%
      }{}
\DeclareFontShape{OT1}{cmr}{bx}{it}
      {%
      <5><6><7><8><9>%
      <10><10.95><12><14.4><17.28><20.74><24.88>cmbxti10%
      }{}
% Again this is necessary for a correct \pounds symbol in
% the cmr fonts Hopefully the dc/ec font layout will take
% over soon.
%
\DeclareFontShape{OT1}{cmr}{bx}{ui}
      {<->sub*cmr/m/ui}{}
\endinput
%%
%% End of file `ot1cmr.fd'.

\endgroup
%    \end{macrocode}
%
%    We also load some other font definition files which are normally
%    needed in a document. This is only done for processing speed and
%    you can comment the next two lines out to save some memory. If
%    necessary these files are then loaded when your document is
%    processed. (Loading |.fd| files is a less drastic step compared
%    to preloading fonts because the number of fonts is limited 255 at
%    (nearly) every \TeX{} installation, while the amount of main memory
%    is not a limiting factor at most installations.)
%
%    \begin{macrocode}
\begingroup
\nfss@catcodes
%%
%% This is file `ot1cmss.fd',
%% generated with the docstrip utility.
%%
%% The original source files were:
%%
%% cmfonts.fdd  (with options: `OT1cmss')
%% 
%% This is a generated file.
%% 
%% The source is maintained by the LaTeX Project team and bug
%% reports for it can be opened at https://latex-project.org/bugs.html
%% (but please observe conditions on bug reports sent to that address!)
%% 
%% 
%% Copyright (C) 1993-2020
%% The LaTeX3 Project and any individual authors listed elsewhere
%% in this file.
%% 
%% This file was generated from file(s) of the LaTeX base system.
%% --------------------------------------------------------------
%% 
%% It may be distributed and/or modified under the
%% conditions of the LaTeX Project Public License, either version 1.3c
%% of this license or (at your option) any later version.
%% The latest version of this license is in
%%    https://www.latex-project.org/lppl.txt
%% and version 1.3c or later is part of all distributions of LaTeX
%% version 2008 or later.
%% 
%% This file may only be distributed together with a copy of the LaTeX
%% base system. You may however distribute the LaTeX base system without
%% such generated files.
%% 
%% The list of all files belonging to the LaTeX base distribution is
%% given in the file `manifest.txt'. See also `legal.txt' for additional
%% information.
%% 
%% In particular, permission is granted to customize the declarations in
%% this file to serve the needs of your installation.
%% 
%% However, a modified version of this file under its original name
%% should not be distributed as part of a standard LaTeX distribution, as
%% the modification will be nontransparent for the user of that
%% distribution (and thus violating clause 6a of the LPPL license),
%% making successful document exchange impossible.
%% 
\ProvidesFile{ot1cmss.fd}
        [2019/12/16 v2.5j Standard LaTeX font definitions]
\DeclareFontFamily{OT1}{cmss}{\hyphenchar\font45 }
\DeclareFontShape{OT1}{cmss}{m}{n}
     {%
      <5><6><7><8>cmss8%
      <9>cmss9%
      <10><10.95>cmss10%
      <12><14.4>cmss12%
      <17.28><20.74><24.88>cmss17%
      }{}
% Font undefined, therefore substituted
\DeclareFontShape{OT1}{cmss}{m}{it}
    {<->ssub*cmss/m/sl}{}
\DeclareFontShape{OT1}{cmss}{m}{sl}
    {%
      <5><6><7><8>cmssi8<9>cmssi9%
      <10><10.95>cmssi10%
      <12><14.4>cmssi12%
      <17.28><20.74><24.88>cmssi17%
      }{}
%%%%%%% Font/shape undefined, therefore substituted
\DeclareFontShape{OT1}{cmss}{m}{sc}
       {<->sub*cmr/m/sc}{}
%%%%%%% Font/shape undefined, therefore substituted
\DeclareFontShape{OT1}{cmss}{m}{ui}
       {<->sub*cmr/m/ui}{}
%%%%%%%% semibold condensed series
\DeclareFontShape{OT1}{cmss}{sbc}{n}
     {%
      <5><6><7><8><9>cmssdc10%
       <10><10.95><12><14.4><17.28><20.74><24.88>cmssdc10%
       }{}

%%%%%%%%% bold extended series
\DeclareFontShape{OT1}{cmss}{bx}{n}
     {%
      <5><6><7><8><9>cmssbx10%
      <10><10.95><12><14.4><17.28><20.74><24.88>cmssbx10%
      }{}
%%%%%%% Font/shape undefined, therefore substituted
\DeclareFontShape{OT1}{cmss}{bx}{ui}
       {<->sub*cmr/bx/ui}{}
\endinput
%%
%% End of file `ot1cmss.fd'.

%%
%% This is file `ot1cmtt.fd',
%% generated with the docstrip utility.
%%
%% The original source files were:
%%
%% cmfonts.fdd  (with options: `OT1cmtt,nowarn')
%% 
%% This is a generated file.
%% 
%% The source is maintained by the LaTeX Project team and bug
%% reports for it can be opened at https://latex-project.org/bugs.html
%% (but please observe conditions on bug reports sent to that address!)
%% 
%% 
%% Copyright (C) 1993-2020
%% The LaTeX3 Project and any individual authors listed elsewhere
%% in this file.
%% 
%% This file was generated from file(s) of the LaTeX base system.
%% --------------------------------------------------------------
%% 
%% It may be distributed and/or modified under the
%% conditions of the LaTeX Project Public License, either version 1.3c
%% of this license or (at your option) any later version.
%% The latest version of this license is in
%%    https://www.latex-project.org/lppl.txt
%% and version 1.3c or later is part of all distributions of LaTeX
%% version 2008 or later.
%% 
%% This file may only be distributed together with a copy of the LaTeX
%% base system. You may however distribute the LaTeX base system without
%% such generated files.
%% 
%% The list of all files belonging to the LaTeX base distribution is
%% given in the file `manifest.txt'. See also `legal.txt' for additional
%% information.
%% 
%% In particular, permission is granted to customize the declarations in
%% this file to serve the needs of your installation.
%% 
%% However, a modified version of this file under its original name
%% should not be distributed as part of a standard LaTeX distribution, as
%% the modification will be nontransparent for the user of that
%% distribution (and thus violating clause 6a of the LPPL license),
%% making successful document exchange impossible.
%% 
\ProvidesFile{ot1cmtt.fd}
        [2019/12/16 v2.5j Standard LaTeX font definitions]
\DeclareFontFamily{OT1}{cmtt}{\hyphenchar \font\m@ne}
\DeclareFontShape{OT1}{cmtt}{m}{n}
     {%
      <5><6><7><8>cmtt8<9>cmtt9%
      <10><10.95>cmtt10%
      <12><14.4><17.28><20.74><24.88>cmtt12%
      }{}
%%%%%% make sure subst shapes are available
\DeclareFontShape{OT1}{cmtt}{m}{it}
     {%
      <5><6><7><8><9>%
      <10><10.95><12><14.4><17.28><20.74><24.88>cmitt10%
      }{}
\DeclareFontShape{OT1}{cmtt}{m}{sl}
     {%
      <5><6><7><8><9>%
      <10><10.95><12><14.4><17.28><20.74><24.88>cmsltt10%
      }{}
\DeclareFontShape{OT1}{cmtt}{m}{sc}
     {%
      <5><6><7><8><9>%
      <10><10.95><12><14.4><17.28><20.74><24.88>cmtcsc10%
      }{}
\DeclareFontShape{OT1}{cmtt}{m}{ui}
  {<->ssub*cmtt/m/it}{}
\DeclareFontShape{OT1}{cmtt}{bx}{n}
  {<->ssub*cmtt/m/n}{}
\DeclareFontShape{OT1}{cmtt}{bx}{it}
  {<->ssub*cmtt/m/it}{}
\DeclareFontShape{OT1}{cmtt}{bx}{sl}
  {<->ssub*cmtt/m/n}{}
\DeclareFontShape{OT1}{cmtt}{bx}{ui}
  {<->ssub*cmtt/m/it}{}
\endinput
%%
%% End of file `ot1cmtt.fd'.

\endgroup
%    \end{macrocode}
%
%    Even with all the precautions it is still possible that NFSS will
%    run into problems, for example, when a |.fd| file contains
%    corrupted data. To guard against such cases NFSS has a very
%    low-level fallback font that is installed with the following line.
%    \begin{macrocode}
\DeclareErrorFont{OT1}{cmr}{m}{n}{10}
%    \end{macrocode}
%    This means, ``if everything else fails use Computer Modern Roman
%    normal shape at 10pt in the old text encoding''.
%    You can change the font used but the encoding should be the same
%    as the one specified with |\fontencoding| above.
%
%
% \subsection{Defaults}
%
%    To allow the use of |\rmfamily|, |\sffamily|, etc.\ in documents
%    even if non-standard families are used we provide nine macros
%    which hold the name of the corresponding families, series, and so
%    on. This makes it easy to use other font families (like Times
%    Roman, etc.). One simply has to redefine these defaults.
%
%    All these hooks have to be defined in this file but you can
%    change their meaning (except for |\encodingdefault|) without
%    making documents non-portable.
%
%
% \begin{macro}{\encodingdefault}
% \begin{macro}{\rmdefault}
% \begin{macro}{\sfdefault}
% \begin{macro}{\ttdefault}
%    The following three definitions set up the meaning for
%    |\rmfamily|, |\sffamily|, and |\ttfamily|.
%    \begin{macrocode}
\ifx\Umathcode\@undefined
\newcommand\encodingdefault{OT1}
\newcommand\rmdefault{cmr}
\newcommand\sfdefault{cmss}
\newcommand\ttdefault{cmtt}
\else
\newcommand\encodingdefault{TU}
\newcommand\rmdefault{lmr}
\fontfamily{\rmdefault}
\newcommand\sfdefault{lmss}
\newcommand\ttdefault{lmtt}
\fi
%</text>
%<latexrelease>\IncludeInRelease{2017/01/01}%
%<latexrelease>                 {\encodingdefault}{TU encoding default}%
%<latexrelease>\ifx\Umathcode\@undefined
%<latexrelease>\renewcommand\encodingdefault{OT1}
%<latexrelease>\fontencoding{\encodingdefault}
%<latexrelease>\renewcommand\rmdefault{cmr}
%<latexrelease>\fontfamily{\rmdefault}
%<latexrelease>\renewcommand\sfdefault{cmss}
%<latexrelease>\renewcommand\ttdefault{cmtt}
%<latexrelease>\else
%<latexrelease>\renewcommand\encodingdefault{TU}
%<latexrelease>%done in everyjob\fontencoding{\encodingdefault}
%<latexrelease>\renewcommand\rmdefault{lmr}
%<latexrelease>\fontfamily{\rmdefault}
%<latexrelease>\renewcommand\sfdefault{lmss}
%<latexrelease>\renewcommand\ttdefault{lmtt}
%<latexrelease>\fi
%<latexrelease>\EndIncludeInRelease
%<latexrelease>\IncludeInRelease{0000/00/00}%
%<latexrelease>                 {\encodingdefault}{TU encoding default}%
%<latexrelease>\fontencoding{OT1}
%<latexrelease>\renewcommand\encodingdefault{OT1}
%<latexrelease>\fontencoding{\encodingdefault}
%<latexrelease>\renewcommand\rmdefault{cmr}
%<latexrelease>\fontfamily{\rmdefault}
%<latexrelease>\renewcommand\sfdefault{cmss}
%<latexrelease>\renewcommand\ttdefault{cmtt}
%<latexrelease>\EndIncludeInRelease
%<*text>
%    \end{macrocode}
% \end{macro}
% \end{macro}
% \end{macro}
% \end{macro}
%
% \begin{macro}{\bfdefault}
% \begin{macro}{\mddefault}
%    Series changing commands are influenced by the following hooks.
% \changes{v3.0e}{2019/12/17}{Set \cs{bfdefault} to ``b''}
%    \begin{macrocode}
\newcommand\bfdefault{b}  % overwritten below (for rollback)
\newcommand\mddefault{m}  % overwritten below (for rollback)
%    \end{macrocode}
% \end{macro}
% \end{macro}
%
% \begin{macro}{\itdefault}
% \begin{macro}{\sldefault}
% \begin{macro}{\scdefault}
% \begin{macro}{\updefault}
%
%    Shape changing commands use the following hooks.
% \changes{v3.0e}{2019/12/17}{Set \cs{updefault} to ``up''}
%    \begin{macrocode}
\newcommand\itdefault{it}
\newcommand\sldefault{sl}
\newcommand\scdefault{sc}
\newcommand\updefault{up}  % overwritten below (for rollback)
%    \end{macrocode}
% \end{macro}
% \end{macro}
% \end{macro}
% \end{macro}
%

%    \begin{macrocode}
%</text>
%<*text|latexrelease>
%<latexrelease>\IncludeInRelease{2020/02/02}%
%<latexrelease>                 {\updefault}{font defaults change}%
%    \begin{macrocode}
\renewcommand\updefault{up}
%    \end{macrocode}
%    We append \cs{@empty} to the series value so that we can detect
%    if it got changed via \cs{def} or \cs{renewcommand} later.
% \changes{v3.0h}{2020/03/19}{Support legacy use of \cs{bfdefault}
%        and \cs{mddefault} (gh/306)}
%    \begin{macrocode}
\renewcommand\bfdefault{b\@empty}
\renewcommand\mddefault{m\@empty}
%    \end{macrocode}
%    
%    \begin{macrocode}
\let\bfdefault@previous\bfdefault
\let\mddefault@previous\mddefault
%</text|latexrelease>
%<latexrelease>\EndIncludeInRelease
%<latexrelease>\IncludeInRelease{0000/00/00}%
%<latexrelease>                 {\updefault}{font defaults change}%
%<latexrelease>
%<latexrelease>\renewcommand\updefault{n}
%<latexrelease>\renewcommand\bfdefault{bx}
%<latexrelease>
%<latexrelease>\let\bfdefault@previous\undefined
%<latexrelease>\let\mddefault@previous\undefined
%<latexrelease>\EndIncludeInRelease
%<*text>
%    \end{macrocode}
%
% \begin{macro}{\familydefault}
% \begin{macro}{\seriesdefault}
% \begin{macro}{\shapedefault}
%    Finally we have the hooks that describe the behaviour of
%    the |\normalfont| command. To stay portable, the definition of
%    |\encodingdefault| should \emph{not} be changed and should match
%    the setting above for |\fontencoding|. All other values can be
%    set according to your taste.
% \changes{v3.0a}{2016/12/03}{(DPC) Default to TU encoding for Unicode TeX engines}
%    \begin{macrocode}
\newcommand\familydefault{\rmdefault}
\newcommand\seriesdefault{\mddefault}
%    \end{macrocode}
%    In previous releases \cs{shapedefault} pointed to \cs{updefault}
%    which resolved to \texttt{n}, but these days that is no longer
%    the case (and \texttt{up} is wrong when you want to do a
%    reset. So we now use \texttt{n} explicitly.
% \changes{v3.0e}{2019/12/17}{Set \cs{shapedefault} explicitly to ``n''}
%    \begin{macrocode}
\newcommand\shapedefault{n}
%    \end{macrocode}
% \end{macro}
% \end{macro}
% \end{macro}
%
%
%    This finishes the low-level setup in \texttt{fonttext.ltx}.
%    \begin{macrocode}
%</text>
%    \end{macrocode}
%
%
%
%
% \section{The \texttt{fontmath.ltx} file}
%
%    The identification is done earlier on with a |\ProvidesFile|
%    declaration.
%    \begin{macrocode}
%<*math>
\typeout{=== Don't modify this file, use a .cfg file instead ===^^J}
%    \end{macrocode}
%
% \subsection{The font encodings used}
%
%    \begin{macrocode}
\DeclareFontEncoding{OML}{}{}
\DeclareFontEncoding{OMS}{}{}
\DeclareFontEncoding{OMX}{}{}
%    \end{macrocode}
%    Finally a declaration for |U| encoding which serves for all fonts
%    that do not fit standard encodings. For math this sets up
%    |\noaccents@| providing for AMS-\LaTeX{}. This macro is used
%    therein to handle accented characters if they are not supported
%    by the font. In other words, if fonts with |U| encoding are used
%    in math, all accents (like from |\breve|) are obtained from some
%    other font that has them.
%    \begin{macrocode}
\DeclareFontEncoding{U}{}{\noaccents@}
%    \end{macrocode}
%    The encodings for math are next:
%    \begin{macrocode}
\DeclareFontSubstitution{OML}{cmm}{m}{it}
\DeclareFontSubstitution{OMS}{cmsy}{m}{n}
\DeclareFontSubstitution{OMX}{cmex}{m}{n}
\DeclareFontSubstitution{U}{cmr}{m}{n}
%    \end{macrocode}
%
%    \begin{macrocode}
\begingroup
\nfss@catcodes
%%
%% This is file `omlcmm.fd',
%% generated with the docstrip utility.
%%
%% The original source files were:
%%
%% cmfonts.fdd  (with options: `OMLcmm')
%% 
%% This is a generated file.
%% 
%% The source is maintained by the LaTeX Project team and bug
%% reports for it can be opened at https://latex-project.org/bugs.html
%% (but please observe conditions on bug reports sent to that address!)
%% 
%% 
%% Copyright (C) 1993-2020
%% The LaTeX3 Project and any individual authors listed elsewhere
%% in this file.
%% 
%% This file was generated from file(s) of the LaTeX base system.
%% --------------------------------------------------------------
%% 
%% It may be distributed and/or modified under the
%% conditions of the LaTeX Project Public License, either version 1.3c
%% of this license or (at your option) any later version.
%% The latest version of this license is in
%%    https://www.latex-project.org/lppl.txt
%% and version 1.3c or later is part of all distributions of LaTeX
%% version 2008 or later.
%% 
%% This file may only be distributed together with a copy of the LaTeX
%% base system. You may however distribute the LaTeX base system without
%% such generated files.
%% 
%% The list of all files belonging to the LaTeX base distribution is
%% given in the file `manifest.txt'. See also `legal.txt' for additional
%% information.
%% 
%% In particular, permission is granted to customize the declarations in
%% this file to serve the needs of your installation.
%% 
%% However, a modified version of this file under its original name
%% should not be distributed as part of a standard LaTeX distribution, as
%% the modification will be nontransparent for the user of that
%% distribution (and thus violating clause 6a of the LPPL license),
%% making successful document exchange impossible.
%% 
\ProvidesFile{omlcmm.fd}
        [2019/12/16 v2.5j Standard LaTeX font definitions]
\DeclareFontFamily{OML}{cmm}{\skewchar\font127 }
\DeclareFontShape{OML}{cmm}{m}{it}%
     {<5><6><7><8><9>gen*cmmi%
      <10><10.95>cmmi10%
      <12><14.4><17.28><20.74><24.88>cmmi12%
      }{}
\DeclareFontShape{OML}{cmm}{b}{it}{%
      <5><6><7><8><9>gen*cmmib%
      <10><10.95><12><14.4><17.28><20.74><24.88>cmmib10%
      }{}
\DeclareFontShape{OML}{cmm}{bx}{it}%
   {<->ssub*cmm/b/it}{}
\endinput
%%
%% End of file `omlcmm.fd'.

%%
%% This is file `omscmsy.fd',
%% generated with the docstrip utility.
%%
%% The original source files were:
%%
%% cmfonts.fdd  (with options: `OMScmsy')
%% 
%% This is a generated file.
%% 
%% The source is maintained by the LaTeX Project team and bug
%% reports for it can be opened at https://latex-project.org/bugs.html
%% (but please observe conditions on bug reports sent to that address!)
%% 
%% 
%% Copyright (C) 1993-2020
%% The LaTeX3 Project and any individual authors listed elsewhere
%% in this file.
%% 
%% This file was generated from file(s) of the LaTeX base system.
%% --------------------------------------------------------------
%% 
%% It may be distributed and/or modified under the
%% conditions of the LaTeX Project Public License, either version 1.3c
%% of this license or (at your option) any later version.
%% The latest version of this license is in
%%    https://www.latex-project.org/lppl.txt
%% and version 1.3c or later is part of all distributions of LaTeX
%% version 2008 or later.
%% 
%% This file may only be distributed together with a copy of the LaTeX
%% base system. You may however distribute the LaTeX base system without
%% such generated files.
%% 
%% The list of all files belonging to the LaTeX base distribution is
%% given in the file `manifest.txt'. See also `legal.txt' for additional
%% information.
%% 
%% In particular, permission is granted to customize the declarations in
%% this file to serve the needs of your installation.
%% 
%% However, a modified version of this file under its original name
%% should not be distributed as part of a standard LaTeX distribution, as
%% the modification will be nontransparent for the user of that
%% distribution (and thus violating clause 6a of the LPPL license),
%% making successful document exchange impossible.
%% 
\ProvidesFile{omscmsy.fd}
        [2019/12/16 v2.5j Standard LaTeX font definitions]
\DeclareFontFamily{OMS}{cmsy}{\skewchar\font48 }
\DeclareFontShape{OMS}{cmsy}{m}{n}{%
      <5><6><7><8><9><10>gen*cmsy%
      <10.95><12><14.4><17.28><20.74><24.88>cmsy10%
      }{}
\DeclareFontShape{OMS}{cmsy}{b}{n}{%
      <5><6><7><8><9>gen*cmbsy%
      <10><10.95><12><14.4><17.28><20.74><24.88>cmbsy10%
      }{}
\endinput
%%
%% End of file `omscmsy.fd'.

%%
%% This is file `omxcmex.fd',
%% generated with the docstrip utility.
%%
%% The original source files were:
%%
%% cmfonts.fdd  (with options: `OMXcmex')
%% 
%% This is a generated file.
%% 
%% The source is maintained by the LaTeX Project team and bug
%% reports for it can be opened at https://latex-project.org/bugs.html
%% (but please observe conditions on bug reports sent to that address!)
%% 
%% 
%% Copyright (C) 1993-2020
%% The LaTeX3 Project and any individual authors listed elsewhere
%% in this file.
%% 
%% This file was generated from file(s) of the LaTeX base system.
%% --------------------------------------------------------------
%% 
%% It may be distributed and/or modified under the
%% conditions of the LaTeX Project Public License, either version 1.3c
%% of this license or (at your option) any later version.
%% The latest version of this license is in
%%    https://www.latex-project.org/lppl.txt
%% and version 1.3c or later is part of all distributions of LaTeX
%% version 2008 or later.
%% 
%% This file may only be distributed together with a copy of the LaTeX
%% base system. You may however distribute the LaTeX base system without
%% such generated files.
%% 
%% The list of all files belonging to the LaTeX base distribution is
%% given in the file `manifest.txt'. See also `legal.txt' for additional
%% information.
%% 
%% In particular, permission is granted to customize the declarations in
%% this file to serve the needs of your installation.
%% 
%% However, a modified version of this file under its original name
%% should not be distributed as part of a standard LaTeX distribution, as
%% the modification will be nontransparent for the user of that
%% distribution (and thus violating clause 6a of the LPPL license),
%% making successful document exchange impossible.
%% 
\ProvidesFile{omxcmex.fd}
        [2019/12/16 v2.5j Standard LaTeX font definitions]
\DeclareFontFamily{OMX}{cmex}{}
\DeclareFontShape{OMX}{cmex}{m}{n}{%
   <->sfixed*cmex10%
   }{}
\endinput
%%
%% End of file `omxcmex.fd'.

%%
%% This is file `ucmr.fd',
%% generated with the docstrip utility.
%%
%% The original source files were:
%%
%% cmfonts.fdd  (with options: `Ucmr')
%% 
%% This is a generated file.
%% 
%% The source is maintained by the LaTeX Project team and bug
%% reports for it can be opened at https://latex-project.org/bugs.html
%% (but please observe conditions on bug reports sent to that address!)
%% 
%% 
%% Copyright (C) 1993-2020
%% The LaTeX3 Project and any individual authors listed elsewhere
%% in this file.
%% 
%% This file was generated from file(s) of the LaTeX base system.
%% --------------------------------------------------------------
%% 
%% It may be distributed and/or modified under the
%% conditions of the LaTeX Project Public License, either version 1.3c
%% of this license or (at your option) any later version.
%% The latest version of this license is in
%%    https://www.latex-project.org/lppl.txt
%% and version 1.3c or later is part of all distributions of LaTeX
%% version 2008 or later.
%% 
%% This file may only be distributed together with a copy of the LaTeX
%% base system. You may however distribute the LaTeX base system without
%% such generated files.
%% 
%% The list of all files belonging to the LaTeX base distribution is
%% given in the file `manifest.txt'. See also `legal.txt' for additional
%% information.
%% 
%% In particular, permission is granted to customize the declarations in
%% this file to serve the needs of your installation.
%% 
%% However, a modified version of this file under its original name
%% should not be distributed as part of a standard LaTeX distribution, as
%% the modification will be nontransparent for the user of that
%% distribution (and thus violating clause 6a of the LPPL license),
%% making successful document exchange impossible.
%% 
\ProvidesFile{ucmr.fd}
        [2019/12/16 v2.5j Standard LaTeX font definitions]
\DeclareFontFamily{U}{cmr}{\hyphenchar\font45 }
\DeclareFontShape{U}{cmr}{m}{n}%
     {<5><6><7><8><9><10><12>gen*cmr%
      <10.95>cmr10%
      <14.4>cmr12%
      <17.28><20.74><24.88>cmr17}{}
\DeclareFontShape{U}{cmr}{m}{sl}%
     {%
      <5><6><7>cmsl8%
      <8><9>gen*cmsl%
      <10><10.95>cmsl10%
      <12><14.4><17.28><20.74><24.88>cmsl12%
      }{}
\DeclareFontShape{U}{cmr}{m}{it}%
     {%
      <5><6><7>cmti7%
      <8>cmti8%
      <9>cmti9%
      <10><10.95>cmti10%
      <12><14.4><17.28><20.74><24.88>cmti12%
      }{}
\DeclareFontShape{U}{cmr}{m}{sc}%
     {%
      <5><6><7><8><9><10><10.95><12>%
      <14.4><17.28><20.74><24.88>cmcsc10%
      }{}
% Warning: please note that the upright shape below is
%          used for the \pounds symbol of LaTeX. So this
%          font definition shouldn't be removed.
%
\DeclareFontShape{U}{cmr}{m}{ui}
   {
      <5><6><7><8><9><10><10.95><12>%
      <14.4><17.28><20.74><24.88>cmu10%
      }{}
%%%%%%% bold series
\DeclareFontShape{U}{cmr}{b}{n}%
     {%
      <5><6><7><8><9><10><10.95><12>%
      <14.4><17.28><20.74><24.88>cmb10%
      }{}
%%%%%%%% bold extended series
\DeclareFontShape{U}{cmr}{bx}{n}%
   {%
      <5><6><7><8><9>gen*cmbx%
      <10><10.95>cmbx10%
      <12><14.4><17.28><20.74><24.88>cmbx12%
      }{}
\DeclareFontShape{U}{cmr}{bx}{sl}%
      {%
      <5><6><7><8><9>%
      <10><10.95><12><14.4><17.28><20.74><24.88>cmbxsl10%
      }{}
\DeclareFontShape{U}{cmr}{bx}{it}%
      {%
      <5><6><7><8><9>%
      <10><10.95><12><14.4><17.28><20.74><24.88>cmbxti10%
      }{}
% Again this is necessary for a correct \pounds symbol in
% the cmr fonts Hopefully the dc/ec font layout will take
% over soon.
%
\DeclareFontShape{U}{cmr}{bx}{ui}%
      {<->sub*cmr/m/ui}{}
\endinput
%%
%% End of file `ucmr.fd'.

\endgroup
%    \end{macrocode}
%
%  \subsubsection{Symbolfont and Alphabet declarations}
%
%    We now define the basic symbol fonts used by \LaTeX{}.
%    These four symbol fonts must be defined by this file.
%
%    It is possible to make the symbol fonts point to other external
%    fonts without losing the ability to process  documents written
%    at other sites, as long as one defines the same symbol font names
%    with the same encodings, e.g.~|operators| with |OT1| etc.
%    If other encodings are used documents become non-portable.
%    Such a change should therefore be done in a package file.
%
% \changes{v2.1e}{1994/01/19}{Added missing setting for symbols in
%                             bold version.}
%    \begin{macrocode}
\DeclareSymbolFont{operators}   {OT1}{cmr} {m}{n}
\DeclareSymbolFont{letters}     {OML}{cmm} {m}{it}
\DeclareSymbolFont{symbols}     {OMS}{cmsy}{m}{n}
\DeclareSymbolFont{largesymbols}{OMX}{cmex}{m}{n}
%    \end{macrocode}
%
%    \begin{macrocode}
\SetSymbolFont{operators}{bold}{OT1}{cmr} {bx}{n}
\SetSymbolFont{letters}  {bold}{OML}{cmm} {b}{it}
\SetSymbolFont{symbols}  {bold}{OMS}{cmsy}{b}{n}
%    \end{macrocode}
%
%    Below are the seven math alphabets which are defined by NFSS.
%    Again they must be defined by this file.
%    However, as before you can change the fonts used without losing
%    portability, but you should be careful when changing the encoding
%    since that may make documents come out wrong.
%    \begin{macrocode}
\DeclareSymbolFontAlphabet{\mathrm}    {operators}
\DeclareSymbolFontAlphabet{\mathnormal}{letters}
\DeclareSymbolFontAlphabet{\mathcal}   {symbols}
\DeclareMathAlphabet      {\mathbf}{OT1}{cmr}{bx}{n}
\DeclareMathAlphabet      {\mathsf}{OT1}{cmss}{m}{n}
\DeclareMathAlphabet      {\mathit}{OT1}{cmr}{m}{it}
\DeclareMathAlphabet      {\mathtt}{OT1}{cmtt}{m}{n}
%    \end{macrocode}
%    Given the currently available fonts we cannot bold-en |\mathbf|
%    and |\mathtt| but in principle one could use `ultra bold' or
%    something. The alphabets defined via |\DeclareSymbolFontAlphabet|
%    will change automatically in a new math version if the
%    corresponding symbol font changes.
%    \begin{macrocode}
\SetMathAlphabet\mathsf{bold}{OT1}{cmss}{bx}{n}
\SetMathAlphabet\mathit{bold}{OT1}{cmr}{bx}{it}
%    \end{macrocode}
%
%
% \subsection{Math font sizes}
% \changes{v2.2f}{1994/11/07}
%     {(DPC) Add \cs{DeclareMathSizes} declarations}
%
%    The declarations below declare the text, script and scriptscript
%    size to be used for each text font size.
%
%    All occurrences of sizes longer than a single character are replaced
%    with the macro name that holds them, saving a number of
%    tokens (but losing a bit of speed, so this may not stay this way).
%    \begin{macrocode}
 \DeclareMathSizes{5}{5}{5}{5}
 \DeclareMathSizes{6}{6}{5}{5}
 \DeclareMathSizes{7}{7}{5}{5}
 \DeclareMathSizes{8}{8}{6}{5}
 \DeclareMathSizes{9}{9}{6}{5}
 \DeclareMathSizes{\@xpt}{\@xpt}{7}{5}
 \DeclareMathSizes{\@xipt}{\@xipt}{8}{6}
 \DeclareMathSizes{\@xiipt}{\@xiipt}{8}{6}
 \DeclareMathSizes{\@xivpt}{\@xivpt}{\@xpt}{7}
 \DeclareMathSizes{\@xviipt}{\@xviipt}{\@xiipt}{\@xpt}
 \DeclareMathSizes{\@xxpt}{\@xxpt}{\@xivpt}{\@xiipt}
 \DeclareMathSizes{\@xxvpt}{\@xxvpt}{\@xxpt}{\@xviipt}
%    \end{macrocode}
%
% \subsection{The math symbol assignments}
%
%    We start by setting up math codes for most of the characters
%    typed in directly from the keyboard. Most of them are normally
%    already setup up in the same way by Ini\TeX{}. However, we repeat
%    them here to have a complete setup which can be exchanged with
%    another if desired.
%
% \subsubsection{The letters}
%    \begin{macrocode}
\DeclareMathSymbol{a}{\mathalpha}{letters}{`a}
\DeclareMathSymbol{b}{\mathalpha}{letters}{`b}
\DeclareMathSymbol{c}{\mathalpha}{letters}{`c}
\DeclareMathSymbol{d}{\mathalpha}{letters}{`d}
\DeclareMathSymbol{e}{\mathalpha}{letters}{`e}
\DeclareMathSymbol{f}{\mathalpha}{letters}{`f}
\DeclareMathSymbol{g}{\mathalpha}{letters}{`g}
\DeclareMathSymbol{h}{\mathalpha}{letters}{`h}
\DeclareMathSymbol{i}{\mathalpha}{letters}{`i}
\DeclareMathSymbol{j}{\mathalpha}{letters}{`j}
\DeclareMathSymbol{k}{\mathalpha}{letters}{`k}
\DeclareMathSymbol{l}{\mathalpha}{letters}{`l}
\DeclareMathSymbol{m}{\mathalpha}{letters}{`m}
\DeclareMathSymbol{n}{\mathalpha}{letters}{`n}
\DeclareMathSymbol{o}{\mathalpha}{letters}{`o}
\DeclareMathSymbol{p}{\mathalpha}{letters}{`p}
\DeclareMathSymbol{q}{\mathalpha}{letters}{`q}
\DeclareMathSymbol{r}{\mathalpha}{letters}{`r}
\DeclareMathSymbol{s}{\mathalpha}{letters}{`s}
\DeclareMathSymbol{t}{\mathalpha}{letters}{`t}
\DeclareMathSymbol{u}{\mathalpha}{letters}{`u}
\DeclareMathSymbol{v}{\mathalpha}{letters}{`v}
\DeclareMathSymbol{w}{\mathalpha}{letters}{`w}
\DeclareMathSymbol{x}{\mathalpha}{letters}{`x}
\DeclareMathSymbol{y}{\mathalpha}{letters}{`y}
\DeclareMathSymbol{z}{\mathalpha}{letters}{`z}
%    \end{macrocode}
%
%    \begin{macrocode}
\DeclareMathSymbol{A}{\mathalpha}{letters}{`A}
\DeclareMathSymbol{B}{\mathalpha}{letters}{`B}
\DeclareMathSymbol{C}{\mathalpha}{letters}{`C}
\DeclareMathSymbol{D}{\mathalpha}{letters}{`D}
\DeclareMathSymbol{E}{\mathalpha}{letters}{`E}
\DeclareMathSymbol{F}{\mathalpha}{letters}{`F}
\DeclareMathSymbol{G}{\mathalpha}{letters}{`G}
\DeclareMathSymbol{H}{\mathalpha}{letters}{`H}
\DeclareMathSymbol{I}{\mathalpha}{letters}{`I}
\DeclareMathSymbol{J}{\mathalpha}{letters}{`J}
\DeclareMathSymbol{K}{\mathalpha}{letters}{`K}
\DeclareMathSymbol{L}{\mathalpha}{letters}{`L}
\DeclareMathSymbol{M}{\mathalpha}{letters}{`M}
\DeclareMathSymbol{N}{\mathalpha}{letters}{`N}
\DeclareMathSymbol{O}{\mathalpha}{letters}{`O}
\DeclareMathSymbol{P}{\mathalpha}{letters}{`P}
\DeclareMathSymbol{Q}{\mathalpha}{letters}{`Q}
\DeclareMathSymbol{R}{\mathalpha}{letters}{`R}
\DeclareMathSymbol{S}{\mathalpha}{letters}{`S}
\DeclareMathSymbol{T}{\mathalpha}{letters}{`T}
\DeclareMathSymbol{U}{\mathalpha}{letters}{`U}
\DeclareMathSymbol{V}{\mathalpha}{letters}{`V}
\DeclareMathSymbol{W}{\mathalpha}{letters}{`W}
\DeclareMathSymbol{X}{\mathalpha}{letters}{`X}
\DeclareMathSymbol{Y}{\mathalpha}{letters}{`Y}
\DeclareMathSymbol{Z}{\mathalpha}{letters}{`Z}
%    \end{macrocode}
%
% \subsubsection{The digits}
%
%    \begin{macrocode}
\DeclareMathSymbol{0}{\mathalpha}{operators}{`0}
\DeclareMathSymbol{1}{\mathalpha}{operators}{`1}
\DeclareMathSymbol{2}{\mathalpha}{operators}{`2}
\DeclareMathSymbol{3}{\mathalpha}{operators}{`3}
\DeclareMathSymbol{4}{\mathalpha}{operators}{`4}
\DeclareMathSymbol{5}{\mathalpha}{operators}{`5}
\DeclareMathSymbol{6}{\mathalpha}{operators}{`6}
\DeclareMathSymbol{7}{\mathalpha}{operators}{`7}
\DeclareMathSymbol{8}{\mathalpha}{operators}{`8}
\DeclareMathSymbol{9}{\mathalpha}{operators}{`9}
%    \end{macrocode}
%
%
% \subsubsection{Punctuation, brace, etc. keys}
%
%    \begin{macrocode}
\DeclareMathSymbol{!}{\mathclose}{operators}{"21}
\DeclareMathSymbol{*}{\mathbin}{symbols}{"03} % \ast
\DeclareMathSymbol{+}{\mathbin}{operators}{"2B}
\DeclareMathSymbol{,}{\mathpunct}{letters}{"3B}
\DeclareMathSymbol{-}{\mathbin}{symbols}{"00}
\DeclareMathSymbol{.}{\mathord}{letters}{"3A}
\DeclareMathSymbol{:}{\mathrel}{operators}{"3A}
\DeclareMathSymbol{;}{\mathpunct}{operators}{"3B}
\DeclareMathSymbol{=}{\mathrel}{operators}{"3D}
\DeclareMathSymbol{?}{\mathclose}{operators}{"3F}
%    \end{macrocode}
% The following symbols are defined as delimiters below
% which automatically defines them as math symbols.
%    \begin{macrocode}
%\DeclareMathSymbol{(}{\mathopen}{operators}{"28}
%\DeclareMathSymbol{)}{\mathclose}{operators}{"29}
%\DeclareMathSymbol{/}{\mathord}{letters}{"3D}
%\DeclareMathSymbol{[}{\mathopen}{operators}{"5B}
%\DeclareMathSymbol{]}{\mathclose}{operators}{"5D}
%\DeclareMathSymbol{|}{\mathord}{symbols}{"6A}
%\DeclareMathSymbol{<}{\mathrel}{letters}{"3C}
%\DeclareMathSymbol{>}{\mathrel}{letters}{"3E}
%    \end{macrocode}
%
%    Should all of the following being activated by default? Probably
%    not.
%    \begin{macrocode}
%\DeclareMathSymbol{`\{}{\mathopen}{symbols}{"66}
%\DeclareMathSymbol{`\}}{\mathclose}{symbols}{"67}
%\DeclareMathSymbol{`\\}{\mathord}{symbols}{"6E} % \backslash
\mathcode`\ ="8000 % \space
\mathcode`\'="8000 % ^\prime
\mathcode`\_="8000 % \_
%    \end{macrocode}
%
%
% \subsubsection{Delimitercodes for characters}
% \changes{v2.2q}{1997/01/08}
%     {Use \cs{DeclareMathDelimiter} to set delimiter codes}
% \changes{v2.2u}{1998/04/15}
%     {Use new syntax for \cs{DeclareMathDelimiter}}
%    [to be completed]
%
%    Finally, Ini\TeX{} sets all |\delcode| values to -1, except
%    |\delcode`.=0|
%    \begin{macrocode}
\DeclareMathDelimiter{(}{\mathopen} {operators}{"28}{largesymbols}{"00}
\DeclareMathDelimiter{)}{\mathclose}{operators}{"29}{largesymbols}{"01}
\DeclareMathDelimiter{[}{\mathopen} {operators}{"5B}{largesymbols}{"02}
\DeclareMathDelimiter{]}{\mathclose}{operators}{"5D}{largesymbols}{"03}
%    \end{macrocode}
%
% The next two are considered to be relations when not used in the context
% of a delimiter! And worse, they do even represent different glyphs when
% being used as delimiter and not as delimiter. This is a user level syntax
% inherited from plain \TeX{}. Therefore we explicitly redefine the math
% symbol definitions for these symbols afterwards.
% \changes{v2.2v}{1998/04/17}
%     {Reinsert symbol defs for \texttt{<} and \texttt{\char62} chars.}
%    \begin{macrocode}
\DeclareMathDelimiter{<}{\mathopen}{symbols}{"68}{largesymbols}{"0A}
\DeclareMathDelimiter{>}{\mathclose}{symbols}{"69}{largesymbols}{"0B}
\DeclareMathSymbol{<}{\mathrel}{letters}{"3C}
\DeclareMathSymbol{>}{\mathrel}{letters}{"3E}
%    \end{macrocode}
% And here is another case where the non-delimiter version produces a
% glyph different from the delimiter version.
% \changes{v2.2w}{1998/04/18}
%     {Reinsert symbol def for \texttt{/} char.}
%    \begin{macrocode}
\DeclareMathDelimiter{/}{\mathord}{operators}{"2F}{largesymbols}{"0E}
\DeclareMathSymbol{/}{\mathord}{letters}{"3D}
%    \end{macrocode}
%
%    \begin{macrocode}
\DeclareMathDelimiter{|}{\mathord}{symbols}{"6A}{largesymbols}{"0C}
%    \end{macrocode}
%
%    \begin{macrocode}
\expandafter\DeclareMathDelimiter\@backslashchar
                        {\mathord}{symbols}{"6E}{largesymbols}{"0F}
%    \end{macrocode}
% N.B. |{| and |}| should NOT get delcodes;
% otherwise parameter grouping fails!
%
%
% \subsection{Symbols accessed via control sequences}
%
% \subsubsection{Greek letters}
%
%    \begin{macrocode}
\DeclareMathSymbol{\alpha}{\mathord}{letters}{"0B}
\DeclareMathSymbol{\beta}{\mathord}{letters}{"0C}
\DeclareMathSymbol{\gamma}{\mathord}{letters}{"0D}
\DeclareMathSymbol{\delta}{\mathord}{letters}{"0E}
\DeclareMathSymbol{\epsilon}{\mathord}{letters}{"0F}
\DeclareMathSymbol{\zeta}{\mathord}{letters}{"10}
\DeclareMathSymbol{\eta}{\mathord}{letters}{"11}
\DeclareMathSymbol{\theta}{\mathord}{letters}{"12}
\DeclareMathSymbol{\iota}{\mathord}{letters}{"13}
\DeclareMathSymbol{\kappa}{\mathord}{letters}{"14}
\DeclareMathSymbol{\lambda}{\mathord}{letters}{"15}
\DeclareMathSymbol{\mu}{\mathord}{letters}{"16}
\DeclareMathSymbol{\nu}{\mathord}{letters}{"17}
\DeclareMathSymbol{\xi}{\mathord}{letters}{"18}
\DeclareMathSymbol{\pi}{\mathord}{letters}{"19}
\DeclareMathSymbol{\rho}{\mathord}{letters}{"1A}
\DeclareMathSymbol{\sigma}{\mathord}{letters}{"1B}
\DeclareMathSymbol{\tau}{\mathord}{letters}{"1C}
\DeclareMathSymbol{\upsilon}{\mathord}{letters}{"1D}
\DeclareMathSymbol{\phi}{\mathord}{letters}{"1E}
\DeclareMathSymbol{\chi}{\mathord}{letters}{"1F}
\DeclareMathSymbol{\psi}{\mathord}{letters}{"20}
\DeclareMathSymbol{\omega}{\mathord}{letters}{"21}
\DeclareMathSymbol{\varepsilon}{\mathord}{letters}{"22}
\DeclareMathSymbol{\vartheta}{\mathord}{letters}{"23}
\DeclareMathSymbol{\varpi}{\mathord}{letters}{"24}
\DeclareMathSymbol{\varrho}{\mathord}{letters}{"25}
\DeclareMathSymbol{\varsigma}{\mathord}{letters}{"26}
\DeclareMathSymbol{\varphi}{\mathord}{letters}{"27}
\DeclareMathSymbol{\Gamma}{\mathalpha}{operators}{"00}
\DeclareMathSymbol{\Delta}{\mathalpha}{operators}{"01}
\DeclareMathSymbol{\Theta}{\mathalpha}{operators}{"02}
\DeclareMathSymbol{\Lambda}{\mathalpha}{operators}{"03}
\DeclareMathSymbol{\Xi}{\mathalpha}{operators}{"04}
\DeclareMathSymbol{\Pi}{\mathalpha}{operators}{"05}
\DeclareMathSymbol{\Sigma}{\mathalpha}{operators}{"06}
\DeclareMathSymbol{\Upsilon}{\mathalpha}{operators}{"07}
\DeclareMathSymbol{\Phi}{\mathalpha}{operators}{"08}
\DeclareMathSymbol{\Psi}{\mathalpha}{operators}{"09}
\DeclareMathSymbol{\Omega}{\mathalpha}{operators}{"0A}
%    \end{macrocode}
%
%
% \subsubsection{Ordinary symbols}
%
%    \begin{macrocode}
\DeclareMathSymbol{\aleph}{\mathord}{symbols}{"40}
\DeclareMathSymbol{\imath}{\mathord}{letters}{"7B}
\DeclareMathSymbol{\jmath}{\mathord}{letters}{"7C}
\DeclareMathSymbol{\ell}{\mathord}{letters}{"60}
\DeclareMathSymbol{\wp}{\mathord}{letters}{"7D}
\DeclareMathSymbol{\Re}{\mathord}{symbols}{"3C}
\DeclareMathSymbol{\Im}{\mathord}{symbols}{"3D}
\DeclareMathSymbol{\partial}{\mathord}{letters}{"40}
\DeclareMathSymbol{\infty}{\mathord}{symbols}{"31}
\DeclareMathSymbol{\prime}{\mathord}{symbols}{"30}
\DeclareMathSymbol{\emptyset}{\mathord}{symbols}{"3B}
\DeclareMathSymbol{\nabla}{\mathord}{symbols}{"72}
\DeclareMathSymbol{\top}{\mathord}{symbols}{"3E}
\DeclareMathSymbol{\bot}{\mathord}{symbols}{"3F}
\DeclareMathSymbol{\triangle}{\mathord}{symbols}{"34}
\DeclareMathSymbol{\forall}{\mathord}{symbols}{"38}
\DeclareMathSymbol{\exists}{\mathord}{symbols}{"39}
\DeclareMathSymbol{\neg}{\mathord}{symbols}{"3A}
%    \end{macrocode}
%    Alias:
% \changes{v3.0e}{2019/12/21}{Distangle alias (gh/184)}
%    \begin{macrocode}
%    \let\lnot=\neg
\DeclareMathSymbol{\lnot}{\mathord}{symbols}{"3A}
%    \end{macrocode}
%
%    \begin{macrocode}
\DeclareMathSymbol{\flat}{\mathord}{letters}{"5B}
\DeclareMathSymbol{\natural}{\mathord}{letters}{"5C}
\DeclareMathSymbol{\sharp}{\mathord}{letters}{"5D}
\DeclareMathSymbol{\clubsuit}{\mathord}{symbols}{"7C}
\DeclareMathSymbol{\diamondsuit}{\mathord}{symbols}{"7D}
\DeclareMathSymbol{\heartsuit}{\mathord}{symbols}{"7E}
\DeclareMathSymbol{\spadesuit}{\mathord}{symbols}{"7F}
%    \end{macrocode}
%
% \changes{v3.0c}{2019/08/27}{Various commands made robust throughout
%   the file}
%    \begin{macrocode}
\DeclareRobustCommand\hbar{{\mathchar'26\mkern-9muh}}
\DeclareRobustCommand\surd{{\mathchar"1270}}
\DeclareRobustCommand\angle{{\vbox{\ialign{$\m@th\scriptstyle##$\crcr
      \not\mathrel{\mkern14mu}\crcr
      \noalign{\nointerlineskip}
      \mkern2.5mu\leaders\hrule \@height.34pt\hfill\mkern2.5mu\crcr}}}}
%    \end{macrocode}
%
%
% \subsubsection{Large Operators}
%
%    \begin{macrocode}
\DeclareMathSymbol{\coprod}{\mathop}{largesymbols}{"60}
\DeclareMathSymbol{\bigvee}{\mathop}{largesymbols}{"57}
\DeclareMathSymbol{\bigwedge}{\mathop}{largesymbols}{"56}
\DeclareMathSymbol{\biguplus}{\mathop}{largesymbols}{"55}
\DeclareMathSymbol{\bigcap}{\mathop}{largesymbols}{"54}
\DeclareMathSymbol{\bigcup}{\mathop}{largesymbols}{"53}
\DeclareMathSymbol{\intop}{\mathop}{largesymbols}{"52}
    \DeclareRobustCommand\int{\intop\nolimits}
\DeclareMathSymbol{\prod}{\mathop}{largesymbols}{"51}
\DeclareMathSymbol{\sum}{\mathop}{largesymbols}{"50}
\DeclareMathSymbol{\bigotimes}{\mathop}{largesymbols}{"4E}
\DeclareMathSymbol{\bigoplus}{\mathop}{largesymbols}{"4C}
\DeclareMathSymbol{\bigodot}{\mathop}{largesymbols}{"4A}
\DeclareMathSymbol{\ointop}{\mathop}{largesymbols}{"48}
    \DeclareRobustCommand\oint{\ointop\nolimits}
\DeclareMathSymbol{\bigsqcup}{\mathop}{largesymbols}{"46}
\DeclareMathSymbol{\smallint}{\mathop}{symbols}{"73}
%    \end{macrocode}
%
%
% \subsubsection{Binary symbols}
%
% \changes{v2.3a}{2004/02/04}
%     {Added bigtriangle synonyms for stmaryrd}
%    \begin{macrocode}
\DeclareMathSymbol{\triangleleft}{\mathbin}{letters}{"2F}
\DeclareMathSymbol{\triangleright}{\mathbin}{letters}{"2E}
\DeclareMathSymbol{\bigtriangleup}{\mathbin}{symbols}{"34}
\DeclareMathSymbol{\bigtriangledown}{\mathbin}{symbols}{"35}
%    \end{macrocode}
%    Alias:
% \changes{v3.0e}{2019/12/21}{Distangle alias (gh/184)}
%    \begin{macrocode}
%   \let \varbigtriangledown \bigtriangledown
%   \let \varbigtriangleup \bigtriangleup
\DeclareMathSymbol{\varbigtriangleup}{\mathbin}{symbols}{"34}
\DeclareMathSymbol{\varbigtriangledown}{\mathbin}{symbols}{"35}
%    \end{macrocode}
%
% These last two synonyms are needed because the \textsf{stmaryrd}
% package redefines them as Operators.
%
%    \begin{macrocode}
\DeclareMathSymbol{\wedge}{\mathbin}{symbols}{"5E}
\DeclareMathSymbol{\vee}{\mathbin}{symbols}{"5F}
%    \end{macrocode}
%    Alias:
% \changes{v3.0e}{2019/12/21}{Distangle alias (gh/184)}
%    \begin{macrocode}
%   \let\land=\wedge
%   \let\lor=\vee
\DeclareMathSymbol{\land}{\mathbin}{symbols}{"5E}
\DeclareMathSymbol{\lor}{\mathbin}{symbols}{"5F}
%    \end{macrocode}
%
%    \begin{macrocode}
\DeclareMathSymbol{\cap}{\mathbin}{symbols}{"5C}
\DeclareMathSymbol{\cup}{\mathbin}{symbols}{"5B}
\DeclareMathSymbol{\ddagger}{\mathbin}{symbols}{"7A}
\DeclareMathSymbol{\dagger}{\mathbin}{symbols}{"79}
\DeclareMathSymbol{\sqcap}{\mathbin}{symbols}{"75}
\DeclareMathSymbol{\sqcup}{\mathbin}{symbols}{"74}
\DeclareMathSymbol{\uplus}{\mathbin}{symbols}{"5D}
\DeclareMathSymbol{\amalg}{\mathbin}{symbols}{"71}
\DeclareMathSymbol{\diamond}{\mathbin}{symbols}{"05}
\DeclareMathSymbol{\bullet}{\mathbin}{symbols}{"0F}
\DeclareMathSymbol{\wr}{\mathbin}{symbols}{"6F}
\DeclareMathSymbol{\div}{\mathbin}{symbols}{"04}
\DeclareMathSymbol{\odot}{\mathbin}{symbols}{"0C}
\DeclareMathSymbol{\oslash}{\mathbin}{symbols}{"0B}
\DeclareMathSymbol{\otimes}{\mathbin}{symbols}{"0A}
\DeclareMathSymbol{\ominus}{\mathbin}{symbols}{"09}
\DeclareMathSymbol{\oplus}{\mathbin}{symbols}{"08}
\DeclareMathSymbol{\mp}{\mathbin}{symbols}{"07}
\DeclareMathSymbol{\pm}{\mathbin}{symbols}{"06}
\DeclareMathSymbol{\circ}{\mathbin}{symbols}{"0E}
\DeclareMathSymbol{\bigcirc}{\mathbin}{symbols}{"0D}
\DeclareMathSymbol{\setminus}{\mathbin}{symbols}{"6E}
\DeclareMathSymbol{\cdot}{\mathbin}{symbols}{"01}
\DeclareMathSymbol{\ast}{\mathbin}{symbols}{"03}
\DeclareMathSymbol{\times}{\mathbin}{symbols}{"02}
\DeclareMathSymbol{\star}{\mathbin}{letters}{"3F}
%    \end{macrocode}
%
%
% \subsubsection{Relations}
%
%    \begin{macrocode}
\DeclareMathSymbol{\propto}{\mathrel}{symbols}{"2F}
\DeclareMathSymbol{\sqsubseteq}{\mathrel}{symbols}{"76}
\DeclareMathSymbol{\sqsupseteq}{\mathrel}{symbols}{"77}
\DeclareMathSymbol{\parallel}{\mathrel}{symbols}{"6B}
\DeclareMathSymbol{\mid}{\mathrel}{symbols}{"6A}
\DeclareMathSymbol{\dashv}{\mathrel}{symbols}{"61}
\DeclareMathSymbol{\vdash}{\mathrel}{symbols}{"60}
\DeclareMathSymbol{\nearrow}{\mathrel}{symbols}{"25}
\DeclareMathSymbol{\searrow}{\mathrel}{symbols}{"26}
\DeclareMathSymbol{\nwarrow}{\mathrel}{symbols}{"2D}
\DeclareMathSymbol{\swarrow}{\mathrel}{symbols}{"2E}
\DeclareMathSymbol{\Leftrightarrow}{\mathrel}{symbols}{"2C}
\DeclareMathSymbol{\Leftarrow}{\mathrel}{symbols}{"28}
\DeclareMathSymbol{\Rightarrow}{\mathrel}{symbols}{"29}
   \DeclareRobustCommand\neq{\not=}
%    \end{macrocode}
%    As \cs{neq} is robust we should not use \cs{let} to define
%    \cs{ne} as then it would change if \cs{neq} changes.
% \changes{v3.0d}{2019/09/21}{Distangle alias (gh/184)}
%    \begin{macrocode}
   \DeclareRobustCommand\ne{\not=}
%    \end{macrocode}
%    It would ok to use \cs{let} for those declared by
%    \cs{DeclareMathSymbol} but for a cleaner interface we avoid it
%    always (just in case the internals change).
%    \begin{macrocode}
\DeclareMathSymbol{\leq}{\mathrel}{symbols}{"14}
\DeclareMathSymbol{\geq}{\mathrel}{symbols}{"15}
%    \end{macrocode}
%    Alias:
% \changes{v3.0e}{2019/12/21}{Distangle alias (gh/184)}
%    \begin{macrocode}
%   \let\le=\leq
%   \let\ge=\geq
\DeclareMathSymbol{\le}{\mathrel}{symbols}{"14}
\DeclareMathSymbol{\ge}{\mathrel}{symbols}{"15}
%    \end{macrocode}
%
%    \begin{macrocode}
\DeclareMathSymbol{\succ}{\mathrel}{symbols}{"1F}
\DeclareMathSymbol{\prec}{\mathrel}{symbols}{"1E}
\DeclareMathSymbol{\approx}{\mathrel}{symbols}{"19}
\DeclareMathSymbol{\succeq}{\mathrel}{symbols}{"17}
\DeclareMathSymbol{\preceq}{\mathrel}{symbols}{"16}
\DeclareMathSymbol{\supset}{\mathrel}{symbols}{"1B}
\DeclareMathSymbol{\subset}{\mathrel}{symbols}{"1A}
\DeclareMathSymbol{\supseteq}{\mathrel}{symbols}{"13}
\DeclareMathSymbol{\subseteq}{\mathrel}{symbols}{"12}
\DeclareMathSymbol{\in}{\mathrel}{symbols}{"32}
\DeclareMathSymbol{\ni}{\mathrel}{symbols}{"33}
%    \end{macrocode}
%    Alias:
% \changes{v3.0e}{2019/12/21}{Distangle alias (gh/184)}
%    \begin{macrocode}
%    \let\owns=\ni
\DeclareMathSymbol{\owns}{\mathrel}{symbols}{"33}
%    \end{macrocode}
%
%    \begin{macrocode}
\DeclareMathSymbol{\gg}{\mathrel}{symbols}{"1D}
\DeclareMathSymbol{\ll}{\mathrel}{symbols}{"1C}
\DeclareMathSymbol{\not}{\mathrel}{symbols}{"36}
\DeclareMathSymbol{\leftrightarrow}{\mathrel}{symbols}{"24}
\DeclareMathSymbol{\leftarrow}{\mathrel}{symbols}{"20}
\DeclareMathSymbol{\rightarrow}{\mathrel}{symbols}{"21}
%    \end{macrocode}
%    Alias:
% \changes{v3.0e}{2019/12/21}{Distangle alias (gh/184)}
%    \begin{macrocode}
%   \let\gets=\leftarrow
%   \let\to=\rightarrow
\DeclareMathSymbol{\gets}{\mathrel}{symbols}{"20}
\DeclareMathSymbol{\to}{\mathrel}{symbols}{"21}
%    \end{macrocode}
%
%    \begin{macrocode}
\DeclareMathSymbol{\mapstochar}{\mathrel}{symbols}{"37}
   \DeclareRobustCommand\mapsto{\mapstochar\rightarrow}
\DeclareMathSymbol{\sim}{\mathrel}{symbols}{"18}
\DeclareMathSymbol{\simeq}{\mathrel}{symbols}{"27}
\DeclareMathSymbol{\perp}{\mathrel}{symbols}{"3F}
\DeclareMathSymbol{\equiv}{\mathrel}{symbols}{"11}
\DeclareMathSymbol{\asymp}{\mathrel}{symbols}{"10}
\DeclareMathSymbol{\smile}{\mathrel}{letters}{"5E}
\DeclareMathSymbol{\frown}{\mathrel}{letters}{"5F}
\DeclareMathSymbol{\leftharpoonup}{\mathrel}{letters}{"28}
\DeclareMathSymbol{\leftharpoondown}{\mathrel}{letters}{"29}
\DeclareMathSymbol{\rightharpoonup}{\mathrel}{letters}{"2A}
\DeclareMathSymbol{\rightharpoondown}{\mathrel}{letters}{"2B}
%    \end{macrocode}
%
%    Here cometh much profligate robustification of math constructs.
%    Warning: some of these commands may become non-robust if an
%    AMS package is loaded.
%
%    Further potential problems: some math font packages may make
%    unfortunate assumptions about some of these definitions that are
%    not true of the robust versions we need.
% \changes{v2.3}{2004/02/02}
%     {Many things from here on made robust}
%    \begin{macrocode}
\DeclareRobustCommand
  \cong{\mathrel{\mathpalette\@vereq\sim}} % congruence sign
\def\@vereq#1#2{\lower.5\p@\vbox{\lineskiplimit\maxdimen\lineskip-.5\p@
    \ialign{$\m@th#1\hfil##\hfil$\crcr#2\crcr=\crcr}}}
\DeclareRobustCommand
  \notin{\mathrel{\m@th\mathpalette\c@ncel\in}}
\def\c@ncel#1#2{\m@th\ooalign{$\hfil#1\mkern1mu/\hfil$\crcr$#1#2$}}
\DeclareRobustCommand
  \rightleftharpoons{\mathrel{\mathpalette\rlh@{}}}
\def\rlh@#1{\vcenter{\m@th\hbox{\ooalign{\raise2pt
          \hbox{$#1\rightharpoonup$}\crcr
        $#1\leftharpoondown$}}}}
\DeclareRobustCommand
  \doteq{\buildrel\textstyle.\over=}
%    \end{macrocode}
%
% \subsubsection{Arrows}
%
%    \begin{macrocode}
\DeclareRobustCommand
  \joinrel{\mathrel{\mkern-3mu}}
\DeclareRobustCommand
  \relbar{\mathrel{\smash-}} % \smash, because -
                               % has the same height as +
%    \end{macrocode}
%    In contrast to \texttt{plain.tex} |\Relbar| got braces around the
%    equal sign to guard against it being ``math active'' expanding to
%    |\futurelet...|. This might be the case when packages are
%    implementing shorthands for math, e.g. |=>| meaning |\Rightarrow|
%    etc. It would actually be better not to use |=| in such
%    definitions but instead define something like |\mathequalsign|
%    and use this. However we can't do this now as it would break
%    other math layouts where characters are in different places
%    (since those wouldn't know about the need for a new command name).
% \changes{v2.2z}{2001/06/04}{Guard against math active equal sign in
%    \cs{Relbar} (pr/3333)}
%    \begin{macrocode}
\DeclareRobustCommand
  \Relbar{\mathrel{=}}
\DeclareMathSymbol{\lhook}{\mathrel}{letters}{"2C}
   \DeclareRobustCommand\hookrightarrow{\lhook\joinrel\rightarrow}
\DeclareMathSymbol{\rhook}{\mathrel}{letters}{"2D}
   \DeclareRobustCommand\hookleftarrow{\leftarrow\joinrel\rhook}
\DeclareRobustCommand
  \bowtie{\mathrel\triangleright\joinrel\mathrel\triangleleft}
%    \end{macrocode}
%
% \changes{v2.2z}{2001/06/04}{Guard against math active equal and pipe
%    sign in \cs{models} (pr/3333)}
%    \begin{macrocode}
\DeclareRobustCommand
  \models{\mathrel{|}\joinrel\Relbar}
\DeclareRobustCommand
  \Longrightarrow{\Relbar\joinrel\Rightarrow}
%    \end{macrocode}
%
% LaTeX Change: |\longrightarrow| and |\longleftarrow| redefined to make
%   then robust.
%    \begin{macrocode}
\DeclareRobustCommand\longrightarrow
     {\relbar\joinrel\rightarrow}
\DeclareRobustCommand\longleftarrow
     {\leftarrow\joinrel\relbar}
%    \end{macrocode}
%
%    \begin{macrocode}
\DeclareRobustCommand
  \Longleftarrow{\Leftarrow\joinrel\Relbar}
\DeclareRobustCommand
  \longmapsto{\mapstochar\longrightarrow}
\DeclareRobustCommand
  \longleftrightarrow{\leftarrow\joinrel\rightarrow}
\DeclareRobustCommand
  \Longleftrightarrow{\Leftarrow\joinrel\Rightarrow}
\DeclareRobustCommand
  \iff{\;\Longleftrightarrow\;}
%    \end{macrocode}
%
%
% \subsubsection{Punctuation symbols}
%
%    \begin{macrocode}
\DeclareMathSymbol{\ldotp}{\mathpunct}{letters}{"3A}
\DeclareMathSymbol{\cdotp}{\mathpunct}{symbols}{"01}
\DeclareMathSymbol{\colon}{\mathpunct}{operators}{"3A}
%    \end{macrocode}
%
%
% This is commented out, since |\ldots| is now defined in ltoutenc.dtx.
%    \begin{macrocode}
%\def\@ldots{\mathinner{\ldotp\ldotp\ldotp}}
%\DeclareRobustCommand\ldots
%          {\relax\ifmmode\@ldots\else\mbox{$\m@th\@ldots\,$}\fi}
%    \end{macrocode}
%
%    \begin{macrocode}
\DeclareRobustCommand
  \cdots{\mathinner{\cdotp\cdotp\cdotp}}
\DeclareRobustCommand
  \vdots{\vbox{\baselineskip4\p@ \lineskiplimit\z@
    \kern6\p@\hbox{.}\hbox{.}\hbox{.}}}
\DeclareRobustCommand
  \ddots{\mathinner{\mkern1mu\raise7\p@
    \vbox{\kern7\p@\hbox{.}}\mkern2mu
    \raise4\p@\hbox{.}\mkern2mu\raise\p@\hbox{.}\mkern1mu}}
%    \end{macrocode}
%
%
% \subsubsection{Math accents}
%
%    \begin{macrocode}
\DeclareMathAccent{\acute}{\mathalpha}{operators}{"13}
\DeclareMathAccent{\grave}{\mathalpha}{operators}{"12}
\DeclareMathAccent{\ddot}{\mathalpha}{operators}{"7F}
\DeclareMathAccent{\tilde}{\mathalpha}{operators}{"7E}
\DeclareMathAccent{\bar}{\mathalpha}{operators}{"16}
\DeclareMathAccent{\breve}{\mathalpha}{operators}{"15}
\DeclareMathAccent{\check}{\mathalpha}{operators}{"14}
\DeclareMathAccent{\hat}{\mathalpha}{operators}{"5E}
\DeclareMathAccent{\vec}{\mathord}{letters}{"7E}
\DeclareMathAccent{\dot}{\mathalpha}{operators}{"5F}
\DeclareMathAccent{\widetilde}{\mathord}{largesymbols}{"65}
\DeclareMathAccent{\widehat}{\mathord}{largesymbols}{"62}
%    \end{macrocode}
%    For some reason plain \TeX{} never bothered to provide
%    a ring accent in math (although it is available in the fonts),
%    but since we got a request for it here we go:
% \changes{v2.2t}{1998/04/11}{Added \cs{mathring} accent (pr2785)}
%    \begin{macrocode}
\DeclareMathAccent{\mathring}{\mathalpha}{operators}{"17}
%    \end{macrocode}
%
%
% \subsubsection{Radicals}
%
% \changes{v2.2o}{1996/05/17}{\cs{@@sqrt} removed, at last}
%    \begin{macrocode}
\DeclareMathRadical{\sqrtsign}{symbols}{"70}{largesymbols}{"70}
%    \end{macrocode}
%
%
% \subsubsection{Over and under something, etc}
%
%    \begin{macrocode}
\DeclareRobustCommand\overrightarrow[1]{\vbox{\m@th\ialign{##\crcr
      \rightarrowfill\crcr\noalign{\kern-\p@\nointerlineskip}
      $\hfil\displaystyle{#1}\hfil$\crcr}}}
\DeclareRobustCommand\overleftarrow[1]{\vbox{\m@th\ialign{##\crcr
      \leftarrowfill\crcr\noalign{\kern-\p@\nointerlineskip}%
      $\hfil\displaystyle{#1}\hfil$\crcr}}}
\DeclareRobustCommand\overbrace[1]
     {\mathop{\vbox{\m@th\ialign{##\crcr\noalign{\kern3\p@}%
      \downbracefill\crcr\noalign{\kern3\p@\nointerlineskip}%
      $\hfil\displaystyle{#1}\hfil$\crcr}}}\limits}
\DeclareRobustCommand\underbrace[1]{\mathop{\vtop{\m@th\ialign{##\crcr
   $\hfil\displaystyle{#1}\hfil$\crcr
   \noalign{\kern3\p@\nointerlineskip}%
   \upbracefill\crcr\noalign{\kern3\p@}}}}\limits}
%    \end{macrocode}
%    (quite a waste of tokens, IMHO --- Frank)
%    \begin{macrocode}
\DeclareRobustCommand\skew[3]
  {{\muskip\z@#1mu\divide\muskip\z@\tw@ \mkern\muskip\z@
    #2{\mkern-\muskip\z@{#3}\mkern\muskip\z@}\mkern-\muskip\z@}{}}
%    \end{macrocode}
%
% \changes{v2.2n}{1995/11/21}{Incorporate changed figures,
%                              as in plain.tex}
%    \begin{macrocode}
\DeclareRobustCommand\rightarrowfill{$\m@th\smash-\mkern-7mu%
  \cleaders\hbox{$\mkern-2mu\smash-\mkern-2mu$}\hfill
  \mkern-7mu\mathord\rightarrow$}
\DeclareRobustCommand\leftarrowfill{$\m@th\mathord\leftarrow\mkern-7mu%
  \cleaders\hbox{$\mkern-2mu\smash-\mkern-2mu$}\hfill
  \mkern-7mu\smash-$}
\DeclareMathSymbol{\braceld}{\mathord}{largesymbols}{"7A}
\DeclareMathSymbol{\bracerd}{\mathord}{largesymbols}{"7B}
\DeclareMathSymbol{\bracelu}{\mathord}{largesymbols}{"7C}
\DeclareMathSymbol{\braceru}{\mathord}{largesymbols}{"7D}
\DeclareRobustCommand\downbracefill{$\m@th \setbox\z@\hbox{$\braceld$}%
  \braceld\leaders\vrule \@height\ht\z@ \@depth\z@\hfill\braceru
  \bracelu\leaders\vrule \@height\ht\z@ \@depth\z@\hfill\bracerd$}
\DeclareRobustCommand\upbracefill{$\m@th \setbox\z@\hbox{$\braceld$}%
  \bracelu\leaders\vrule \@height\ht\z@ \@depth\z@\hfill\bracerd
  \braceld\leaders\vrule \@height\ht\z@ \@depth\z@\hfill\braceru$}
%    \end{macrocode}
%
% \subsubsection{Delimiters}
%
%    \begin{macrocode}
\DeclareMathDelimiter{\lmoustache}   % top from (, bottom from )
   {\mathopen}{largesymbols}{"7A}{largesymbols}{"40}
\DeclareMathDelimiter{\rmoustache}   % top from ), bottom from (
   {\mathclose}{largesymbols}{"7B}{largesymbols}{"41}
\DeclareMathDelimiter{\arrowvert}    % arrow without arrowheads
   {\mathord}{symbols}{"6A}{largesymbols}{"3C}
\DeclareMathDelimiter{\Arrowvert}    % double arrow without arrowheads
   {\mathord}{symbols}{"6B}{largesymbols}{"3D}
\DeclareMathDelimiter{\Vert}
   {\mathord}{symbols}{"6B}{largesymbols}{"0D}
%    \end{macrocode}
%    \cs{DeclareMathDelimiter} produces a command that is robust (with
%    an internal macro containing the payload) so we should not use
%    \cs{let} for making an alias
% \changes{v3.0d}{2019/09/21}{Distangle alias (gh/184)}
%    \begin{macrocode}
%\let\|=\Vert
\DeclareMathDelimiter{\|}
   {\mathord}{symbols}{"6B}{largesymbols}{"0D}
%    \end{macrocode}
%
%    \begin{macrocode}
\DeclareMathDelimiter{\vert}
   {\mathord}{symbols}{"6A}{largesymbols}{"0C}
\DeclareMathDelimiter{\uparrow}
   {\mathrel}{symbols}{"22}{largesymbols}{"78}
\DeclareMathDelimiter{\downarrow}
   {\mathrel}{symbols}{"23}{largesymbols}{"79}
\DeclareMathDelimiter{\updownarrow}
   {\mathrel}{symbols}{"6C}{largesymbols}{"3F}
\DeclareMathDelimiter{\Uparrow}
   {\mathrel}{symbols}{"2A}{largesymbols}{"7E}
\DeclareMathDelimiter{\Downarrow}
   {\mathrel}{symbols}{"2B}{largesymbols}{"7F}
\DeclareMathDelimiter{\Updownarrow}
   {\mathrel}{symbols}{"6D}{largesymbols}{"77}
\DeclareMathDelimiter{\backslash}    % for double coset G\backslash H
   {\mathord}{symbols}{"6E}{largesymbols}{"0F}
\DeclareMathDelimiter{\rangle}
   {\mathclose}{symbols}{"69}{largesymbols}{"0B}
\DeclareMathDelimiter{\langle}
   {\mathopen}{symbols}{"68}{largesymbols}{"0A}
\DeclareMathDelimiter{\rbrace}
   {\mathclose}{symbols}{"67}{largesymbols}{"09}
\DeclareMathDelimiter{\lbrace}
   {\mathopen}{symbols}{"66}{largesymbols}{"08}
\DeclareMathDelimiter{\rceil}
   {\mathclose}{symbols}{"65}{largesymbols}{"07}
\DeclareMathDelimiter{\lceil}
   {\mathopen}{symbols}{"64}{largesymbols}{"06}
\DeclareMathDelimiter{\rfloor}
   {\mathclose}{symbols}{"63}{largesymbols}{"05}
\DeclareMathDelimiter{\lfloor}
   {\mathopen}{symbols}{"62}{largesymbols}{"04}
%    \end{macrocode}
%
%  \begin{macro}{\lgroup}
%  \begin{macro}{\rgroup}
%  \begin{macro}{\bracevert}
%    There are three plain \TeX{} delimiters which are not fully
%    supported by NFSS, since they partly point into a bold cmr font.
%    Allocating a full symbol font, just to have three delimiters
%    seems a bit too much given the limited space available.  For this
%    reason only the extensible sizes are supported.  If this is not
%    desired one can use, without losing portability, define |\mathbf|
%    and |\mathtt| as font symbol alphabet (setting up
%    \texttt{cmr/bx/n} and \texttt{cmtt/m/n} as symbol fonts first)
%    and modify the delimiter declarations to point with their
%    small variant to those symbol fonts. (This is done in
%    \texttt{oldlfont.dtx} so look there for examples.)
%    \begin{macrocode}
\DeclareMathDelimiter{\lgroup} % extensible ( with sharper tips
     {\mathopen}{largesymbols}{"3A}{largesymbols}{"3A}
\DeclareMathDelimiter{\rgroup} % extensible ) with sharper tips
     {\mathclose}{largesymbols}{"3B}{largesymbols}{"3B}
\DeclareMathDelimiter{\bracevert} % the vertical bar that extends braces
     {\mathord}{largesymbols}{"3E}{largesymbols}{"3E}
%    \end{macrocode}
%  \end{macro}
%  \end{macro}
%  \end{macro}
%
% \subsection{Math versions of text commands}
%
% \changes{v2.2k}{1995/06/05}{Moved math commands from ltoutenc.dtx.}
%
% The |\mathunderscore| here is really a text definition, so it has
% been put back into |ltoutenc.dtx| (by Chris, 30/04/97) and should
% be removed from here.
%
% These symbols are the math versions of text commands such as |\P|,
% |\$|, etc.
% \begin{macro}{\mathparagraph}
% \changes{v2.2q}{1997/01/08}
%     {Define using \cs{DeclareMathSymbol}}
% \begin{macro}{\mathsection}
% \begin{macro}{\mathdollar}
% \begin{macro}{\mathsterling}
% \begin{macro}{\mathunderscore}
%    These math symbols are not in plain \TeX.
%    \begin{macrocode}
\DeclareMathSymbol{\mathparagraph}{\mathord}{symbols}{"7B}
\DeclareMathSymbol{\mathsection}{\mathord}{symbols}{"78}
\DeclareMathSymbol{\mathdollar}{\mathord}{operators}{"24}
%    \end{macrocode}
%
%    \begin{macrocode}
\DeclareRobustCommand\mathsterling{\mathit{\mathchar"7024}}
\DeclareRobustCommand\mathunderscore{\kern.06em\vbox{\hrule\@width.3em}}
%    \end{macrocode}
% \end{macro}
% \end{macro}
% \end{macro}
% \end{macro}
% \end{macro}
%
% \begin{macro}{\mathellipsis}
%    This is plain \TeX's |\ldots|.
%    \begin{macrocode}
\DeclareRobustCommand\mathellipsis{\mathinner{\ldotp\ldotp\ldotp}}%
%    \end{macrocode}
% \end{macro}
%
% \subsection{Other special functions and parameters}
%
% \subsubsection{Biggggg}
%
% \changes{v3.0b}{2018/09/24}{Start LR-mode if necessary (git/49)}
%    \begin{macrocode}
%</math>
%<*math|latexrelease>
%<latexrelease>\IncludeInRelease{2018/12/01}%
%<latexrelease>                 {\Big}{Start LR-mode}%
\DeclareRobustCommand\big[1]{\leavevmode@ifvmode
   {\hbox{$\left#1\vbox to8.5\p@{}\right.\n@space$}}}
\DeclareRobustCommand\Big[1]{\leavevmode@ifvmode
   {\hbox{$\left#1\vbox to11.5\p@{}\right.\n@space$}}}
\DeclareRobustCommand\bigg[1]{\leavevmode@ifvmode
   {\hbox{$\left#1\vbox to14.5\p@{}\right.\n@space$}}}
\DeclareRobustCommand\Bigg[1]{\leavevmode@ifvmode
   {\hbox{$\left#1\vbox to17.5\p@{}\right.\n@space$}}}
%</math|latexrelease>
%<latexrelease>\EndIncludeInRelease
%<latexrelease>\IncludeInRelease{0000/00/00}%
%<latexrelease>                 {\Big}{Start LR-mode}%
%<latexrelease>\def\big#1{{\hbox{$\left#1\vbox to8.5\p@{}\right.\n@space$}}}
%<latexrelease>\def\Big#1{{\hbox{$\left#1\vbox to11.5\p@{}\right.\n@space$}}}
%<latexrelease>\def\bigg#1{{\hbox{$\left#1\vbox to14.5\p@{}\right.\n@space$}}}
%<latexrelease>\def\Bigg#1{{\hbox{$\left#1\vbox to17.5\p@{}\right.\n@space$}}}
%<latexrelease>\EndIncludeInRelease
%<*math>
%    \end{macrocode}
%
%    \begin{macrocode}
\def\n@space{\nulldelimiterspace\z@ \m@th}
%    \end{macrocode}
%
%
%
% \subsubsection{The log-like functions}
%
% \begin{macro}{\operator@font}
%    The |\operator@font| determines the symbol font used for log-like
%    functions.
%    \begin{macrocode}
\def\operator@font{\mathgroup\symoperators}
%    \end{macrocode}
%  \end{macro}
%
%
% \subsubsection{Parameters}
%
%    \begin{macrocode}
\thinmuskip=3mu
\medmuskip=4mu plus 2mu minus 4mu
\thickmuskip=5mu plus 5mu
%    \end{macrocode}
%
%
%    This finishes the low-level setup in \texttt{fontmath.ltx}.
%    \begin{macrocode}
%</math>
%    \end{macrocode}
%
%
% \section{Default cfg files}
%
%    We provide default \texttt{cfg} files here to ensure that
%    on installations that search large file trees we do not pick up
%    some strange customisation files from somewhere.
% \changes{v2.2y}{2001/06/02}{Provide default cfg files (pr/3264)}
%    \begin{macrocode}
%<*cfgtext|cfgmath|cfgprel>
%%
%%
%%
%% Load the standard setup:
%%
%<+cfgtext>%%
%% This is file `fonttext.cfg',
%% generated with the docstrip utility.
%%
%% The original source files were:
%%
%% fontdef.dtx  (with options: `cfgtext')
%% 
%% This is a generated file.
%% 
%% Copyright (C) 1993-2020
%% The LaTeX3 Project and any individual authors listed elsewhere
%% in this file.
%% 
%% This file was generated from file(s) of the LaTeX base system.
%% --------------------------------------------------------------
%% 
%% It may be distributed and/or modified under the
%% conditions of the LaTeX Project Public License, either version 1.3c
%% of this license or (at your option) any later version.
%% The latest version of this license is in
%%    https://www.latex-project.org/lppl.txt
%% and version 1.3c or later is part of all distributions of LaTeX
%% version 2008 or later.
%% 
%% This file may only be distributed together with a copy of the LaTeX
%% base system. You may however distribute the LaTeX base system without
%% such generated files.
%% 
%% The list of all files belonging to the LaTeX base distribution is
%% given in the file `manifest.txt'. See also `legal.txt' for additional
%% information.
%% 
%% Details of how to use a configuration file to modify this part of
%% the system are in the document `cfgguide.tex'.
%% 
%% 
%%% From File: fontdef.dtx
\ProvidesFile{fonttext.cfg}
           [2020/08/01 v3.0i LaTeX Kernel
(Uncustomised text
           font setup)]
%%
%%
%%
%% Load the standard setup:
%%
%%
%% This is file `fonttext.cfg',
%% generated with the docstrip utility.
%%
%% The original source files were:
%%
%% fontdef.dtx  (with options: `cfgtext')
%% 
%% This is a generated file.
%% 
%% Copyright (C) 1993-2020
%% The LaTeX3 Project and any individual authors listed elsewhere
%% in this file.
%% 
%% This file was generated from file(s) of the LaTeX base system.
%% --------------------------------------------------------------
%% 
%% It may be distributed and/or modified under the
%% conditions of the LaTeX Project Public License, either version 1.3c
%% of this license or (at your option) any later version.
%% The latest version of this license is in
%%    https://www.latex-project.org/lppl.txt
%% and version 1.3c or later is part of all distributions of LaTeX
%% version 2008 or later.
%% 
%% This file may only be distributed together with a copy of the LaTeX
%% base system. You may however distribute the LaTeX base system without
%% such generated files.
%% 
%% The list of all files belonging to the LaTeX base distribution is
%% given in the file `manifest.txt'. See also `legal.txt' for additional
%% information.
%% 
%% Details of how to use a configuration file to modify this part of
%% the system are in the document `cfgguide.tex'.
%% 
%% 
%%% From File: fontdef.dtx
\ProvidesFile{fonttext.cfg}
           [2020/08/01 v3.0i LaTeX Kernel
(Uncustomised text
           font setup)]
%%
%%
%%
%% Load the standard setup:
%%
%%
%% This is file `fonttext.cfg',
%% generated with the docstrip utility.
%%
%% The original source files were:
%%
%% fontdef.dtx  (with options: `cfgtext')
%% 
%% This is a generated file.
%% 
%% Copyright (C) 1993-2020
%% The LaTeX3 Project and any individual authors listed elsewhere
%% in this file.
%% 
%% This file was generated from file(s) of the LaTeX base system.
%% --------------------------------------------------------------
%% 
%% It may be distributed and/or modified under the
%% conditions of the LaTeX Project Public License, either version 1.3c
%% of this license or (at your option) any later version.
%% The latest version of this license is in
%%    https://www.latex-project.org/lppl.txt
%% and version 1.3c or later is part of all distributions of LaTeX
%% version 2008 or later.
%% 
%% This file may only be distributed together with a copy of the LaTeX
%% base system. You may however distribute the LaTeX base system without
%% such generated files.
%% 
%% The list of all files belonging to the LaTeX base distribution is
%% given in the file `manifest.txt'. See also `legal.txt' for additional
%% information.
%% 
%% Details of how to use a configuration file to modify this part of
%% the system are in the document `cfgguide.tex'.
%% 
%% 
%%% From File: fontdef.dtx
\ProvidesFile{fonttext.cfg}
           [2020/08/01 v3.0i LaTeX Kernel
(Uncustomised text
           font setup)]
%%
%%
%%
%% Load the standard setup:
%%
\input{fonttext.ltx}
%%
%% Small changes could go here; see documentation in cfgguide.tex for
%% allowed modifications.
%%
%% In particular it is not allowed to misuse this configuration file
%% to modify internal LaTeX commands!
%%
%% If you use this file as the basis for configuration please change
%% the \ProvidesFile lines to clearly identify your modification, e.g.,
%%
%%  \ProvidesFile{fonttext.cfg}[2001/06/01
%%                              Customised local font setup]
%%
%%
\endinput
%%
%% End of file `fonttext.cfg'.

%%
%% Small changes could go here; see documentation in cfgguide.tex for
%% allowed modifications.
%%
%% In particular it is not allowed to misuse this configuration file
%% to modify internal LaTeX commands!
%%
%% If you use this file as the basis for configuration please change
%% the \ProvidesFile lines to clearly identify your modification, e.g.,
%%
%%  \ProvidesFile{fonttext.cfg}[2001/06/01
%%                              Customised local font setup]
%%
%%
\endinput
%%
%% End of file `fonttext.cfg'.

%%
%% Small changes could go here; see documentation in cfgguide.tex for
%% allowed modifications.
%%
%% In particular it is not allowed to misuse this configuration file
%% to modify internal LaTeX commands!
%%
%% If you use this file as the basis for configuration please change
%% the \ProvidesFile lines to clearly identify your modification, e.g.,
%%
%%  \ProvidesFile{fonttext.cfg}[2001/06/01
%%                              Customised local font setup]
%%
%%
\endinput
%%
%% End of file `fonttext.cfg'.

%<+cfgmath>%%
%% This is file `fontmath.cfg',
%% generated with the docstrip utility.
%%
%% The original source files were:
%%
%% fontdef.dtx  (with options: `cfgmath')
%% 
%% This is a generated file.
%% 
%% Copyright (C) 1993-2020
%% The LaTeX3 Project and any individual authors listed elsewhere
%% in this file.
%% 
%% This file was generated from file(s) of the LaTeX base system.
%% --------------------------------------------------------------
%% 
%% It may be distributed and/or modified under the
%% conditions of the LaTeX Project Public License, either version 1.3c
%% of this license or (at your option) any later version.
%% The latest version of this license is in
%%    https://www.latex-project.org/lppl.txt
%% and version 1.3c or later is part of all distributions of LaTeX
%% version 2008 or later.
%% 
%% This file may only be distributed together with a copy of the LaTeX
%% base system. You may however distribute the LaTeX base system without
%% such generated files.
%% 
%% The list of all files belonging to the LaTeX base distribution is
%% given in the file `manifest.txt'. See also `legal.txt' for additional
%% information.
%% 
%% Details of how to use a configuration file to modify this part of
%% the system are in the document `cfgguide.tex'.
%% 
%% 
%%% From File: fontdef.dtx
\ProvidesFile{fontmath.cfg}
           [2020/08/01 v3.0i LaTeX Kernel
(Uncustomised math
           font setup)]
%%
%%
%%
%% Load the standard setup:
%%
%%
%% This is file `fontmath.cfg',
%% generated with the docstrip utility.
%%
%% The original source files were:
%%
%% fontdef.dtx  (with options: `cfgmath')
%% 
%% This is a generated file.
%% 
%% Copyright (C) 1993-2020
%% The LaTeX3 Project and any individual authors listed elsewhere
%% in this file.
%% 
%% This file was generated from file(s) of the LaTeX base system.
%% --------------------------------------------------------------
%% 
%% It may be distributed and/or modified under the
%% conditions of the LaTeX Project Public License, either version 1.3c
%% of this license or (at your option) any later version.
%% The latest version of this license is in
%%    https://www.latex-project.org/lppl.txt
%% and version 1.3c or later is part of all distributions of LaTeX
%% version 2008 or later.
%% 
%% This file may only be distributed together with a copy of the LaTeX
%% base system. You may however distribute the LaTeX base system without
%% such generated files.
%% 
%% The list of all files belonging to the LaTeX base distribution is
%% given in the file `manifest.txt'. See also `legal.txt' for additional
%% information.
%% 
%% Details of how to use a configuration file to modify this part of
%% the system are in the document `cfgguide.tex'.
%% 
%% 
%%% From File: fontdef.dtx
\ProvidesFile{fontmath.cfg}
           [2020/08/01 v3.0i LaTeX Kernel
(Uncustomised math
           font setup)]
%%
%%
%%
%% Load the standard setup:
%%
%%
%% This is file `fontmath.cfg',
%% generated with the docstrip utility.
%%
%% The original source files were:
%%
%% fontdef.dtx  (with options: `cfgmath')
%% 
%% This is a generated file.
%% 
%% Copyright (C) 1993-2020
%% The LaTeX3 Project and any individual authors listed elsewhere
%% in this file.
%% 
%% This file was generated from file(s) of the LaTeX base system.
%% --------------------------------------------------------------
%% 
%% It may be distributed and/or modified under the
%% conditions of the LaTeX Project Public License, either version 1.3c
%% of this license or (at your option) any later version.
%% The latest version of this license is in
%%    https://www.latex-project.org/lppl.txt
%% and version 1.3c or later is part of all distributions of LaTeX
%% version 2008 or later.
%% 
%% This file may only be distributed together with a copy of the LaTeX
%% base system. You may however distribute the LaTeX base system without
%% such generated files.
%% 
%% The list of all files belonging to the LaTeX base distribution is
%% given in the file `manifest.txt'. See also `legal.txt' for additional
%% information.
%% 
%% Details of how to use a configuration file to modify this part of
%% the system are in the document `cfgguide.tex'.
%% 
%% 
%%% From File: fontdef.dtx
\ProvidesFile{fontmath.cfg}
           [2020/08/01 v3.0i LaTeX Kernel
(Uncustomised math
           font setup)]
%%
%%
%%
%% Load the standard setup:
%%
\input{fontmath.ltx}
%%
%% Small changes could go here; see documentation in cfgguide.tex for
%% allowed modifications.
%%
%% In particular it is not allowed to misuse this configuration file
%% to modify internal LaTeX commands!
%%
%% If you use this file as the basis for configuration please change
%% the \ProvidesFile lines to clearly identify your modification, e.g.,
%%
%%  \ProvidesFile{fonttext.cfg}[2001/06/01
%%                              Customised local font setup]
%%
%%
\endinput
%%
%% End of file `fontmath.cfg'.

%%
%% Small changes could go here; see documentation in cfgguide.tex for
%% allowed modifications.
%%
%% In particular it is not allowed to misuse this configuration file
%% to modify internal LaTeX commands!
%%
%% If you use this file as the basis for configuration please change
%% the \ProvidesFile lines to clearly identify your modification, e.g.,
%%
%%  \ProvidesFile{fonttext.cfg}[2001/06/01
%%                              Customised local font setup]
%%
%%
\endinput
%%
%% End of file `fontmath.cfg'.

%%
%% Small changes could go here; see documentation in cfgguide.tex for
%% allowed modifications.
%%
%% In particular it is not allowed to misuse this configuration file
%% to modify internal LaTeX commands!
%%
%% If you use this file as the basis for configuration please change
%% the \ProvidesFile lines to clearly identify your modification, e.g.,
%%
%%  \ProvidesFile{fonttext.cfg}[2001/06/01
%%                              Customised local font setup]
%%
%%
\endinput
%%
%% End of file `fontmath.cfg'.

%<+cfgprel>%%
%% This is file `preload.cfg',
%% generated with the docstrip utility.
%%
%% The original source files were:
%%
%% fontdef.dtx  (with options: `cfgprel')
%% 
%% This is a generated file.
%% 
%% Copyright (C) 1993-2020
%% The LaTeX3 Project and any individual authors listed elsewhere
%% in this file.
%% 
%% This file was generated from file(s) of the LaTeX base system.
%% --------------------------------------------------------------
%% 
%% It may be distributed and/or modified under the
%% conditions of the LaTeX Project Public License, either version 1.3c
%% of this license or (at your option) any later version.
%% The latest version of this license is in
%%    https://www.latex-project.org/lppl.txt
%% and version 1.3c or later is part of all distributions of LaTeX
%% version 2008 or later.
%% 
%% This file may only be distributed together with a copy of the LaTeX
%% base system. You may however distribute the LaTeX base system without
%% such generated files.
%% 
%% The list of all files belonging to the LaTeX base distribution is
%% given in the file `manifest.txt'. See also `legal.txt' for additional
%% information.
%% 
%% Details of how to use a configuration file to modify this part of
%% the system are in the document `cfgguide.tex'.
%% 
%% 
%%% From File: fontdef.dtx
\ProvidesFile{preload.cfg}
           [2020/08/01 v3.0i LaTeX Kernel
(Uncustomised preload
           font setup)]
%%
%%
%%
%% Load the standard setup:
%%
%%
%% This is file `preload.cfg',
%% generated with the docstrip utility.
%%
%% The original source files were:
%%
%% fontdef.dtx  (with options: `cfgprel')
%% 
%% This is a generated file.
%% 
%% Copyright (C) 1993-2020
%% The LaTeX3 Project and any individual authors listed elsewhere
%% in this file.
%% 
%% This file was generated from file(s) of the LaTeX base system.
%% --------------------------------------------------------------
%% 
%% It may be distributed and/or modified under the
%% conditions of the LaTeX Project Public License, either version 1.3c
%% of this license or (at your option) any later version.
%% The latest version of this license is in
%%    https://www.latex-project.org/lppl.txt
%% and version 1.3c or later is part of all distributions of LaTeX
%% version 2008 or later.
%% 
%% This file may only be distributed together with a copy of the LaTeX
%% base system. You may however distribute the LaTeX base system without
%% such generated files.
%% 
%% The list of all files belonging to the LaTeX base distribution is
%% given in the file `manifest.txt'. See also `legal.txt' for additional
%% information.
%% 
%% Details of how to use a configuration file to modify this part of
%% the system are in the document `cfgguide.tex'.
%% 
%% 
%%% From File: fontdef.dtx
\ProvidesFile{preload.cfg}
           [2020/08/01 v3.0i LaTeX Kernel
(Uncustomised preload
           font setup)]
%%
%%
%%
%% Load the standard setup:
%%
%%
%% This is file `preload.cfg',
%% generated with the docstrip utility.
%%
%% The original source files were:
%%
%% fontdef.dtx  (with options: `cfgprel')
%% 
%% This is a generated file.
%% 
%% Copyright (C) 1993-2020
%% The LaTeX3 Project and any individual authors listed elsewhere
%% in this file.
%% 
%% This file was generated from file(s) of the LaTeX base system.
%% --------------------------------------------------------------
%% 
%% It may be distributed and/or modified under the
%% conditions of the LaTeX Project Public License, either version 1.3c
%% of this license or (at your option) any later version.
%% The latest version of this license is in
%%    https://www.latex-project.org/lppl.txt
%% and version 1.3c or later is part of all distributions of LaTeX
%% version 2008 or later.
%% 
%% This file may only be distributed together with a copy of the LaTeX
%% base system. You may however distribute the LaTeX base system without
%% such generated files.
%% 
%% The list of all files belonging to the LaTeX base distribution is
%% given in the file `manifest.txt'. See also `legal.txt' for additional
%% information.
%% 
%% Details of how to use a configuration file to modify this part of
%% the system are in the document `cfgguide.tex'.
%% 
%% 
%%% From File: fontdef.dtx
\ProvidesFile{preload.cfg}
           [2020/08/01 v3.0i LaTeX Kernel
(Uncustomised preload
           font setup)]
%%
%%
%%
%% Load the standard setup:
%%
\input{preload.ltx}
%%
%% Small changes could go here; see documentation in cfgguide.tex for
%% allowed modifications.
%%
%% In particular it is not allowed to misuse this configuration file
%% to modify internal LaTeX commands!
%%
%% If you use this file as the basis for configuration please change
%% the \ProvidesFile lines to clearly identify your modification, e.g.,
%%
%%   \ProvidesFile{preload.cfg}[2001/06/01
%%                              Customised local font setup]
%%
%%
\endinput
%%
%% End of file `preload.cfg'.

%%
%% Small changes could go here; see documentation in cfgguide.tex for
%% allowed modifications.
%%
%% In particular it is not allowed to misuse this configuration file
%% to modify internal LaTeX commands!
%%
%% If you use this file as the basis for configuration please change
%% the \ProvidesFile lines to clearly identify your modification, e.g.,
%%
%%   \ProvidesFile{preload.cfg}[2001/06/01
%%                              Customised local font setup]
%%
%%
\endinput
%%
%% End of file `preload.cfg'.

%%
%% Small changes could go here; see documentation in cfgguide.tex for
%% allowed modifications.
%%
%% In particular it is not allowed to misuse this configuration file
%% to modify internal LaTeX commands!
%%
%% If you use this file as the basis for configuration please change
%% the \ProvidesFile lines to clearly identify your modification, e.g.,
%%
%%   \ProvidesFile{preload.cfg}[2001/06/01
%%                              Customised local font setup]
%%
%%
\endinput
%%
%% End of file `preload.cfg'.

%%
%% Small changes could go here; see documentation in cfgguide.tex for
%% allowed modifications.
%%
%% In particular it is not allowed to misuse this configuration file
%% to modify internal LaTeX commands!
%%
%% If you use this file as the basis for configuration please change
%% the \ProvidesFile lines to clearly identify your modification, e.g.,
%%
%<+cfgtext>%%  \ProvidesFile{fonttext.cfg}[2001/06/01
%<+cfgmath>%%  \ProvidesFile{fonttext.cfg}[2001/06/01
%<+cfgprel>%%   \ProvidesFile{preload.cfg}[2001/06/01
%%                              Customised local font setup]
%%
%%
%</cfgtext|cfgmath|cfgprel>
%    \end{macrocode}
%
% \Finale
%
\endinput
