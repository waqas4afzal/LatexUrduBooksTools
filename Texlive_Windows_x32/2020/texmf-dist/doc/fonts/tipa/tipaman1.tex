% tipaman1.tex
% Copyright 2002 FUKUI Rei
%
% This program may be distributed and/or modified under the
% conditions of the LaTeX Project Public License, either version 1.2
% of this license or (at your option) any later version.
% The latest version of this license is in
%   http://www.latex-project.org/lppl.txt
% and version 1.2 or later is part of all distributions of LaTeX 
% version 1999/12/01 or later.
%
% This program consists of all files listed in Manifest.txt.
%

\chapter{Introduction}

\tipa{}\footnote{\tipa{} stands for \emph{\TeX\ IPA} or \emph{Tokyo
    IPA}. The primary ftp site in which the latest version of \tipa{}
  is placed is \texttt{ftp://tooyoo.L.u-tokyo.ac.jp/pub/TeX/tipa}, and
  also it is mirrored onto the directory \texttt{fonts/tipa} of the
  CTAN archives.} is a system for processing IPA (International
Phonetic Alphabet) symbols in \LaTeX. It is based on
\tsipa{}\footnote{\tsipa{} was made in 1992 by Kobayashi Hajime, Fukui
  Rei and Shirakawa Shun. It is available from a CTAN archive.

  One problem with \tsipa{} was that symbols already included in
  \texttt{OT1}, \texttt{T1} or Math fonts are excluded, because of the
  limitation of its 128 character encoding. As a result, a string of
  phonetic representation had to be often composed of symbols from
  different fonts, disabling the possibility of automatic inter-word
  kerning, and also too many symbols had to be realized as macros.}
but both \MF{} source codes and \LaTeX{} macros have been thoroughly
rewritten so it can be considered as a new system.

Among many features of \tipa{}, the following are the new features 
as compared with \tsipa{} or any other existing systems for processing
IPA symbols.

\begin{itemize}
\itemsep0pt
\item A new 256 character encoding for phonetic symbols
  (`\texttt{T3}'), which includes all the symbols and diacritics found
  in the recent versions of IPA and some non-IPA symbols.
\item Complete support of \LaTeXe.
\item A variety of font styles including roman, slanted, bold, bold
  extended, sans serif and typewriter.
\item Easy input method in the IPA environment.
\item Extended macros for accents and diacritics.\footnote{These
    macros are now defined in a separate file called
    `\texttt{exaccent.sty}' in order for the authors of other
    packages to be able to make use of them. The idea of separating
    these macros from other ones was suggested by Frank Mittelbach.}
\item A flexible system of macros for `tone letters'.
\item An optional package (\texttt{vowel.sty}) for drawing vowel
  diagrams. This package can be used independently from the \tipa{}
  package.\footnote{Documentation is also made separately in
    `\texttt{vowel.tex}' so that no further mention will be made
    here.}
\item A slightly modified set of fonts that go well when used with
  Times Roman and Helvetica fonts.
\end{itemize}


\section{Installation}

\subsection{Basics}

In a CTAN site or any other sites that have a copy of the \tipa{}
package, the directory structure of \tipa{} looks as follows.

\begin{quote}
  \verb|sty| --- containing \verb|*.sty|, \verb|*.fd|, \verb|*.def|
  files.\\
  \verb|mf | --- containing \MF{} source files.\\
  \verb|tfm| --- containing font metric files.\\
  \verb|doc| --- containing document files.\\
  \verb|dvips| --- containing tipa.map file.\\
  \verb|type1| --- containing PostScript type1 fonts.
\end{quote}

If you are using a recent set of \LaTeX2e distribution, all you need
to do is basically only two things.

\begin{itemize}
\item Copy all the files in the \verb|sty| directory into an
  appropriate place.
\item Copy all the files in the \verb|mf| directory into an
  appropriate place.
\end{itemize}

In the case of a popular Unix-like OS, the actual installation
procedure will look like the following.

(\verb|$texmf| stands for your \TeX{} system directory; %$
\verb|/usr/local/share/texmf|, for example).

\begin{quote}
  \verb|mkdir $texmf/tex/latex/tipa| %$
    --- create a directory for style files.\\
  \verb|cp sty/* $texmf/tex/latex/tipa| %$
    --- copy all the files in \verb|sty|.\\
  \verb|mkdir $texmf/fonts/source/fkr| %$
    --- create a directory for\\
  \verb|mkdir $texmf/fonts/source/fkr/tipa| %$
    \hspace{1em}  mf files.\\
  \verb|cp mf/* $texmf/fonts/source/fkr/tipa| %$
    --- copy all the \verb|mf| files.\\
  \verb|mktexlsr| --- update the kpathsea database.
\end{quote}

If you are using Windows or Mac, follow the equivalent steps: i.e.,
create a directory/folder for style files and copy the contents of the 
\tipa{} \verb|sty| directory/folder; then, create a directory/folder
for \MF{} source files and copy the contents of the 
\tipa{} \verb|mf| directory/folder.

If you are going to run \tipa{} on the basis of \texttt{pk} files, all 
other things such as \texttt{tfm} files and \texttt{pk} files will be
generated automatically. That's all for the installation.

You may optionally copy all the \texttt{tfm} files into an appropriate
directory which \TeX{} and device driver programs can find. This will
save time for the automatic font generation.

\begin{quote}
  \verb|mkdir $texmf/fonts/tfm/fkr| \\%$
  \verb|mkdir $texmf/fonts/tfm/fkr/tipa| \\%$
  \verb|cp tfm/* $texmf/fonts/tfm/fkr/tipa| %$
\end{quote}

If your \TeX{} system is not equiped with the automatic font
generation mechanism, you may have to create and install \texttt{pk}
files by yourself. For example:

\begin{quote}
  (generate pk font files; please ask someone how to do this.)\\
  \verb|mkdir $texmf/fonts/pk/ljfour/fkr|\\ %$
  \verb|mkdir $texmf/fonts/pk/ljfour/fkr/tipa|\\ %$
  \verb|cp *pk $texmf/fonts/pk/ljfour/fkr/tipa| %$
\end{quote}


\subsection{Installing Type1 fonts}\label{sec:pdf}

If you want to create a PDF document, you need to install Type1 fonts.
First, copy the contents of the directories \texttt{dvips} and
\texttt{type1} onto appropriate directories. For example:

\begin{quote}
  \verb|cp dvips/tipa.map $texmf/dvips/config|\\ %$
  \verb|mkdir $texmf/fonts/type1/fkr|\\ %$
  \verb|mkdir $texmf/fonts/type1/fkr/tipa|\\ %$
  \verb|cp type1/* $texmf/fonts/type1/fkr/tipa|\\ %$
  \verb|mktexlsr| --- update the kpathsea database.
\end{quote}

Then, edit config files for your device driver. In the case
of \texttt{dvips}, edit \texttt{config.ps} and/or \texttt{config.pdf},
for example, and insert a line containing:

\begin{quote}
  \texttt{p +tipa.map}
\end{quote}

There are several ways to make PDF documents. The author of this
document usually uses \texttt{dvips}. For example:

\begin{quote}
  \texttt{dvips -Ppdf tipaman}
\end{quote}

\noindent will produce \texttt{tipaman.ps}. In this case, \texttt{config.pdf}
has to be modified as explained above. Then, by using Acrobat
Distiller (this is not free software), you can convert it to a PDF
file. Alternatively, you can use free software such as
\texttt{dvipdfm}, \texttt{dvipdf}, \texttt{pdflatex}, and so on.

In the case of \texttt{pdflatex}, for example, you have to copy the file
\texttt{tipa.map} onto the following directory.

\begin{quote}
  \verb|$texmf/pdftex/config|%$
\end{quote}

Then, edit \texttt{pdftex.cfg} and insert a line containing:

\begin{quote}
  \texttt{map +tipa.map}
\end{quote}


\section{\tipa{} font families}

This version of \tipa{} includes two families of IPA fonts,
\texttt{tipa} and \texttt{xipa}. The former family of fonts is for
normal use with \LaTeX, and the latter family is intended to be used
with `\texttt{times.sty}'(PSNFSS). They all have the same \texttt{T3}
encoding as explained in the previous section.

\begin{itemize}
\item \texttt{tipa}
  \begin{description}
  \item[Roman:]
    \texttt{tipa8}, \texttt{tipa9}, \texttt{tipa10}, \texttt{tipa12},
    \texttt{tipa17}
  \item[Slanted:]
    \texttt{tipasl8}, \texttt{tipasl9}, \texttt{tipasl10},
    \texttt{tipasl12}
  \item[Bold extended:]
    \texttt{tipabx8}, \texttt{tipabx9},
    \texttt{tipabx10}, \texttt{tipabx12}
  \item[Bold extended Slanted:] \texttt{tipabs10}
  \item[Sans serif:]
    \texttt{tipass8}, \texttt{tipass9}, \texttt{tipass10},
    \texttt{tipass12}, \texttt{tipass17}
  \item[Sans serif Bold extended:] \texttt{tipasb10}
  \item[Sans serif Slanted:] \texttt{tipasi10}
  \item[Bold:] \texttt{tipab10}
  \item[Typewriter Text:]
    \texttt{tipatt8}, \texttt{tipatt9}, \texttt{tipatt10},
    \texttt{tipatt12}
  \item[Typewriter Text Slanted:] \texttt{tipats10}
  \end{description}
\item \texttt{xipa}
  \begin{description}
  \item[Roman:] \texttt{xipa10}
  \item[Slanted:] \texttt{xipasl10}
  \item[Bold:] \texttt{xipab10}
  \item[Bold Slanted:] \texttt{xipabs10}
  \item[Sans serif Bold:] \texttt{xipasb10}
  \item[Sans serif Slanted:] \texttt{xipasi10}
  \end{description}
\end{itemize}

All these fonts are made by \MF{}, based on the Computer Modern font
series. In the case of the \texttt{xipa} series, parameters are
adjusted so as to look fine when used with Times Roman (in the cases
of \texttt{xipa10}, \texttt{xipasl10}, \texttt{xipab10}) and Helvetica
(in the case of \texttt{xipass10}).


\clearemptydoublepage
\chapter{TIPA Encoding}

\section{Selection of symbols}\label{sec:selection}

\subsection{IPA symbols}

When the first version of \tipa{} (version 1.0) was released, the
selection of IPA phonetic symbols was made based on the following
works.

\begin{itemize}
\item \emph{Phonetic Symbol Guide} \citep{PSG:I}.
\item The official IPA charts of '49, '79, '89 and '93 versions.
\item Articles published in the
  \emph{JIPA}\footnote{\emph{Journal of the International Phonetic
      Association.}}, such as \cite{IPA:KielConvention},
  \cite{IPA:FurtherReport}, \cite{Esling:ComputerCodes},
  \cite{IPA:CouncilActions}, and so on.
\item An unpublished paper by J.\ C.\ Wells:
 ``Computer-coding the IPA: a proposed extension of SAM\-PA'' \citep{SAMPA}.
\item Popular textbooks on phonetics.
\end{itemize}

More specifically, this first version tried to incorporate all the
symbols and diacritics defined in the '79, '89 and '93 versions of IPA
and some non-IPA symbols. And in the case of the '49 version of
IPA, as was described in the \textit{Principles} \citep{Principles},
there were too many obsolete symbols and only those symbols that had
had some popularity at least for some time or for some group of people
were included.

Then, soon after the first release, several important works were
published.

\begin{itemize}
\item The second edition of \emph{Phonetic Symbol Guide}
  \citep{PSG:II}. (henceforth abbreviated as \PSG.)
\item The official IPA chart of '96 version.
\item ``Preview of the IPA Handbook'' \citep{IPA:Preview}.
\item \emph{Handbook of the International Phonetic Association}
  \citep{Handbook}. (henceforth abbreviated as \Handbook.)
\end{itemize}

The differences between '93 and '96 versions of IPA are very
few. However, the second edition of \PSG\ contains much more symbols
than before.

The current version of tipa{} (version 1.1) is a result of an effort
to increase the number of symbols as much as possible and to cover
almost all the symbols included in \PSG.  However, the 256 character
encoding (see next section for detail) used in \tipa{} has been
already filled with symbols assigned in the first release. Therefore,
it was necessary to create a set of new auxiliary fonts to include
new symbols, and the new set of fonts is now called \tipx{}.

It should be also noted that \tipa{} includes all the necessary
elements of `tone letters', enabling all the theoretically possible
combinations of the tone letter system. This system was devised by
Yuen-Ren Chao \citep{Chao:ToneLetters} and it is now admitted as an
official way of representing tones in the recent publication of the
International Phonetic Association.

But the treatment of tone letters is quite insufficient in that only a
limited number of combinations is allowed. This is apparently due to
the fact that there has been no `portable' way of combining symbols
that can be used across various computer environments. Therefore
\TeX's productive system of macro is an ideal tool for handling a
system like tone letters.

In the process of writing \MF{} source codes for \tipa{} phonetic
symbols there have been many problems besides the one with the
selection of symbols.  One such problem was that sometimes the
exact shape of a symbol was unclear.  For example, the shapes of the
symbols such as \textstretchc\ (Stretched C), and \textctj\ (Curly-tail J)
differ according to sources. This is partly due to the fact that the
IPA has been continuously revised for the past few decades, and partly
due to the fact that different ways of computerizing phonetic symbols
on different systems have resulted in a diversity of the shapes of
phonetic symbols.

Although there is no definite answer to such a problem yet, it seems
to me that it is a privilege of those working with \MF{} to have a
systematic way of controlling the shapes of phonetic symbols.

\subsection{Non-IPA symbols}

Besides IPA symbols, \tipa{} also contains symbols that are useful for
the following areas of phonetics and linguistics.

\begin{itemize}
\itemsep0pt
\item Symbols used in the American phonetics. (e.g., \textturncelig,
  \textsce, \textscomega, \textlambda, etc.)
\item Symbols used in the historical study of Indo-European languages.
  (e.g., \textthorn, \textwynn, \texthvlig, \textcommatailz,
  \textsoftsign, \texthardsign, and accents such as \textipa{\'=a},
  \textipa{\.'e}, etc.)
\item Symbols used in the phonetic description of languages in East Asia.
  (e.g., \textlhtlongi, \textvibyi, \textctd, \textctn, \textctt, etc.)
\item Diacritics used in `ExtIPA Symbols for Disordered
  Speech' \citep{extIPA} and `VoQS (Voice Quality Symbols)' \citep{VoQS}.
  (e.g., \bibridge{n}, \subdoublevert{f}, \crtilde{m}, etc.)
\end{itemize}


\section{Encoding}

\begin{table}[t]
\begin{center}\tabcolsep1em
\def\zerobox#1{\hbox to0pt{\hss #1\hss}}
\def\MCL#1#2{\multicolumn{#1}{|c|}{#2}}
\begin{tabular}{|c|c|c|c|c|c|c|c|c|}
\hline
  &{\it'0}&{\it'1}&{\it'2}&{\it'3}&{\it'4}&{\it'5}&{\it'6}&{\it'7}\\
\hline
  {\it'00x}& \MCL{8}{} \\
           & \MCL{8}{Accents and diacritics} \\
  {\it'04x}& \MCL{8}{} \\
\hline
  {\it'05x}& \MCL{8}{Punctuation marks} \\
\hline
  {\it'06x}& \MCL{8}{Basic IPA symbols I (vowels)} \\
\cline{4-9}
  {\it'07x}& \MCL{2}{} & \MCL{6}{Diacritics, etc.} \\
\hline
  {\it'10x}& \MCL{8}{} \\
           & \MCL{8}{Basic IPA symbols II} \\
           & \MCL{8}{} \\
\cline{5-9}
  {\it'13x}& \MCL{3}{} & \MCL{5}{Diacritics, etc.} \\
\hline
  {\it'14x}& \MCL{1}{\zerobox{Punct\kern-.1em.}} & \MCL{7}{} \\
\cline{2-2}
           & \MCL{8}{Basic IPA symbols III} \\
           & \MCL{8}{(lowercase letters)} \\
\cline{8-9}
  {\it'17x}& \MCL{6}{} & \MCL{2}{\zerobox{Diacritics}}\\
\hline
  {\it'20x}& \MCL{8}{} \\
           & \MCL{8}{Tone letters and other suprasegmentals} \\
  {\it'23x}& \MCL{8}{} \\
\hline
  {\it'24x}& \MCL{8}{} \\
           & \MCL{8}{Old IPA, non-IPA symbols} \\
  {\it'27x}& \MCL{8}{} \\
\hline
  {\it'30x}& \MCL{8}{} \\
           & \MCL{8}{Extended IPA symbols} \\
\cline{8-9}
  {\it'33x}& \MCL{6}{} & \MCL{2}{\zerobox{Germanic}}\\
\hline
  {\it'34x}& \MCL{8}{} \\
           & \MCL{8}{Basic IPA symbols IV} \\
\cline{8-9}
  {\it'37x}& \MCL{6}{} & \MCL{2}{\zerobox{Germanic}}\\
\hline
\end{tabular}
\end{center}
\caption{Layout of the {\tt T3} encoding}\label{tab:TIPAenc}
\end{table}


The 256 character encoding of \tipa{} is now officially called the
`\texttt{T3}' encoding.\footnote{In a discussion with the \LaTeXe{}
  team it was suggested that the 128 character encoding used in WSUIPA
  would be referred to as the \texttt{OT3} encoding.} In deciding this
new encoding, care is taken to harmonize with other existing 
encodings, especially with the \texttt{T1} encoding. Also the easiness
of inputting phonetic symbols is taken into consideration in such a
way that frequently used symbols can be inputted with small number of
keystrokes.

Table~\ref{tab:TIPAenc} shows the layout of the \texttt{T3} encoding.

The basic structure of the encoding found in the first half of the
table (character codes \texttt{'000-'177}) is based on normal text
encodings (ASCII, \texttt{OT1} and \texttt{T1}) in that sectioning of
this area into several groups, such as the section for accents and
diacritics, the section for punctuation marks, the section for
numerals, and the sections for uppercase and lowercase letters, is
basically the same with these encodings.

Note also that the \texttt{T3} encoding contains not only phonetic
symbols but also usual punctuation marks that are used with phonetic
symbols, and in such cases the same codes are assigned as the normal
text encodings. However, it is a matter of trade-off to decide which
punctuation marks are to be included. For example `:' and `;' might
have been preserved in \texttt{T3} but in this case `:' has been
traditionally used as a substitute for the length mark `\textipa{:}' so
that I decided to exclude `:' in favor of the easiness of inputting the
length mark by a single keystroke.

The encoding of the section for accents and diacritics is closely
related to \texttt{T1} in that the accents commonly included in
\texttt{T1} and \texttt{T3} have the same encoding.

The sections for numerals and uppercase letters are filled with
phonetic symbols that are used frequently in many languages, because
numerals and uppercase letters are usually not used as phonetic
symbols. Also, the assignments made here are used as the `shortcut
characters', which will be explained in section~\ref{sec:OrdinarySymbol}.

As for the section for uppercase letters in the usual text encoding, a
series of discussion among the members of the \texttt{ling-tex}
mailing list revealed that there seem to be a certain amount of
consensus on what symbols are to be assigned to each code. For example,
they were almost unanimous for the assignments such as \textipa{A} for
\texttt{A}, \textipa{B} for \texttt{B}, \textipa{D} for \texttt{D},
\textipa{S} for \texttt{S}, \textipa{T} for \texttt{T}, etc. For more
details, see table~\ref{tab:shortcut}.

\begin{table}[t]
\begin{center}
\let\:\texttt \let\;\textipa
\begin{tabular}{l|cccccccccc}
\hline
\textit{ASCII}&\:: &\:; &\:" & & & & & & &\\
\textit{TIPA} &\;: &\;; &\;" & & & & & & &\\
\hline
\textit{ASCII}&\:0 &\:1 &\:2 &\:3 &\:4 &\:5 &\:6 &\:7 &\:8 &\:9\\
\textit{TIPA} &\;0 &\;1 &\;2 &\;3 &\;4 &\;5 &\;6 &\;7 &\;8 &\;9\\
\hline
\textit{ASCII}&\:@ &\:A &\:B &\:C &\:D &\:E &\:F &\:G &\:H &\:I\\
\textit{TIPA} &\;@ &\;A &\;B &\;C &\;D &\;E &\;F &\;G &\;H &\;I\\
\hline
\textit{ASCII}&\:J &\:K &\:L &\:M &\:N &\:O &\:P &\:Q &\:R &\:S\\
\textit{TIPA} &\;J &\;K &\;L &\;M &\;N &\;O &\;P &\;Q &\;R &\;S\\
\hline
\textit{ASCII}&\:T &\:U &\:V &\:W &\:X &\:Y &\:Z &\:| &  &\\
\textit{TIPA} &\;T &\;U &\;V &\;W &\;X &\;Y &\;Z &\;| &  &\\
\hline
\end{tabular}
\end{center}
\caption{\tipa{} shortcut characters}\label{tab:shortcut}
\end{table}

The encoding of the section for numerals was more difficult than the
above case. One of the possibilities was to assign symbols based on the
resemblance of shapes. One can easily think of assignments such as
\textipa{3} for \texttt{3}, \texthtb{} for \texttt{6}, etc. But the
resemblance of shape alone does not serve as a criteria for all the
assignments. So I decided to assign basic vowel symbols to this
section.\footnote{This idea was influenced by the above mentioned
  article by J.\ C.\ Wells \citep{SAMPA}.} Fortunately the resemblance
of shape is to some extent maintained as is shown in
table~\ref{tab:shortcut}.

The encoding of the section for lowercase letters poses no problem
since they are all used as phonetic symbols. Only one symbol, namely
`\textipa{g}', needs some attention because its shape should be
`\textipa{g}', rather than `g', as a phonetic symbol.\footnote{%
  However, it was declared that these two symbols are equivalent in
  the most recent version of the IPA.  Anyway, alternative shape
  `\textg' is preserved in another section and can be used as
  \texttt{\tbs textg}. }

The second half of the table (character codes \texttt{'200-'377}) is
divided into four sections. The first section is devoted to the
elements of tone letters and other suprasegmental symbols.

Among the remaining three sections the last section \texttt{'340-'377}
contains more basic symbols than the other two sections. This is a
result of assigning the same character codes as latin-1 (ISO8859-1)
and T1 encodings to the symbols that are commonly included in \tipa{},
latin-1 and T1 encoded fonts.\footnote{This is based on a suggestion
  by J\"org Knappen.} These are the cases of \ae{}, \o, \oe{}, \c{c}
and \textthorn. And within each section, symbols are arranged largely
in alphabetical order.

For a table of the \texttt{T3} encoding, see Appendix
\ref{sec:FontLayout}.



\clearemptydoublepage
\chapter{Usage}

\section{Declaration of \tipa{} package}

In order to use \tipa{}, first declare \texttt{tipa.sty} package at
the preamble of a document.

\begin{verbatim}
    \documentclass{article}
    \usepackage{tipa}
\end{verbatim}

If you want to use an additional set of phonetic symbols, declare
\texttt{tipx.sty} after the declaration of \texttt{tipa.sty}.

\begin{verbatim}
    \documentclass{article}
    \usepackage{tipa}
    \usepackage{tipx}
\end{verbatim}

\subsection{Encoding options}

The above declaration uses \texttt{OT1} as the default text encoding. If 
you want to use \tipa{} symbols with \texttt{T1}, specify the option
`\texttt{T1}'.

\begin{verbatim}
    \documentclass{article}
    \usepackage[T1]{tipa}
\end{verbatim}

If you want to use a more complex form of encoding, declare the use of
\texttt{fontenc} package by yourself and specify the option
`\texttt{noenc}'. In this case the option `\texttt{T3}', which represents
the \tipa{} encoding, must be included as an option to the
\texttt{fontenc} package. For example, if you want to use \tipa{} and
the University Washington Cyrillic (\texttt{OT2}) with the \texttt{T1}
text encoding, the following command will do this.

\begin{verbatim}
    \documentclass{article}
    \usepackage[T3,OT2,T1]{fontenc}
    \usepackage[noenc]{tipa}
\end{verbatim}

By default, \tipa{} includes the \texttt{fontenc} package internally but 
the option \texttt{noenc} suppresses this.

\subsection{Using \tipa{} with PSNFSS}

In order to use \tipa{} with \texttt{times.sty}, declare the use of
\texttt{times.sty} before declaring \texttt{tipa} packages.

\begin{verbatim}
    \documentclass{article}
    \usepackage{times}
    \usepackage{tipa}
\end{verbatim}

Font description files \texttt{t3ptm.fd} and \texttt{t3phv.fd} 
are automatically loaded by the above declaration.

This manual can be typeset with Times Roman and \textsf{XIPA} fonts by
uncommenting a few lines that appear near the top of the file
\texttt{tipaman.tex}.

\subsection{Other options}\label{sec:otheroptions}

\tipa{} can be extended by the options \texttt{tone}, \texttt{extra}.

If you want to use the optional package for `tone letters', add
`\texttt{tone}' option to the \verb|\usepackage| command that declares
\texttt{tipa} package.

\begin{verbatim}
    \usepackage[tone]{tipa}
\end{verbatim}

And if you want to use diacritics for extIPA and Vo\-QS, specify
`\texttt{extra}' option.

\begin{verbatim}
    \usepackage[extra]{tipa}
\end{verbatim}

Finally, there is one more option called `\texttt{safe}', which is used
to suppress definitions of some possibly `dangerous' commands of
\tipa.

\begin{verbatim}
    \usepackage[safe]{tipa}
\end{verbatim}

More specifically, the following commands are suppressed by declaring
the \texttt{safe} option.  Explanation on the function of each command
will be given later.

\begin{itemize}
\itemsep0pt
\item \verb|\s| \quad Equivalent to \verb|\textsyllabic|; maybe harmless
  but too short for a control sequence name.
\item \verb|\*| \quad Already defined in plain \TeX;  however, many consider
  its redefinition harmless.
\item \verb+\|+, \verb|\:|, \verb|\;|, \verb|\!| \quad Already defined in
  \LaTeX; these redefinitions are obviously the most dangerous
  ones. However, remedies are prepared even in `unsafe' mode.
  There is a command called \verb|\Vert| which has the same meaning as 
  \verb+\|+ and can be used in `unsafe' mode. For the remaining three
  commands, \tipa{} provides commands called \verb|\tipamedspace|,
  \verb|\tipathickspace| and \verb|\tipanegthinspace| which retain the
  meanings of \verb|\:|, \verb|\;| and \verb|\!|, respectively, even in
  the `unsafe mode'.
\end{itemize}\label{unsafemode}

Despite the above `remedies', you may sometimes want to use the above
commands with their original names in the `unsafe' mode. In such
cases, a command called \texttt{\tbs tipasafemode} can be used. For
example:

\begin{tipaexample}
  \yitem
    \verb+\textipa{[\!b] [\:r] [\;B]}\quad{\tipasafemode+\\
    \verb+  $ a\:a\quad b\;b\quad c\!c\quad\| $}\quad+\\
    \verb+\textipa{[\!b] [\:r] [\;B] (back again!)}+
  \yitem
    \textipa{[\!b] [\:r] [\;B]}\quad
    {\tipasafemode $ a\:a\quad b\;b\quad c\!c\quad\| $}\quad
    \textipa{[\!b] [\:r] [\;B] (back again!)}
\end{tipaexample}

As is shown in this example, \texttt{\tbs tipasafemode} must be used
within a group. Otherwise, the meanings of \tipa{}'s special macros
are lost.

Finally, more than one options can be specified at the same
time, by separating a comma. For example:

\begin{verbatim}
    \usepackage[tone,extra,safe]{tipa}
\end{verbatim}

\section{Input commands for phonetic symbols}
\subsection{Ordinary phonetic symbols}\label{sec:OrdinarySymbol}

\tipa{} phonetic symbols can be inputted by the following two ways.

\begin{enumerate}
\itemsep0pt
\item Input macro names in the normal text environment.
\item Input macro names or \emph{shortcut characters} with\-in the
  following groups or environment.
  \begin{itemize}
  \item \verb|\textipa{...}|\footnote{I personally prefer a slightly
      shorter name like \texttt{\tbs ipa} rather than \texttt{\tbs
        textipa} so that I usually put a command \texttt{\tbs let\tbs
        ipa\tbs textipa} somewhere in my style file. However, this
      command was named after the general convention of \LaTeXe. The
      same can be said for all the symbol names beginning with
      \texttt{\tbs text}.}
  \item \verb|{\tipaencoding ...}|
  \item \verb|\begin{IPA} ... \end{IPA}|
  \end{itemize}
  (These groups and environment will be henceforth referred to as the
  \textsl{IPA environment}.)
\end{enumerate}

A shortcut character refers to a single character that is assigned to
a specific phonetic symbol and that can be directly inputted by an
ordinary keyboard. In \tipa{} fonts, the character codes for numerals
and uppercase letters in the normal ASCII encoding are assigned to such
shortcut characters, because numerals and uppercase letters are
usually not used as phonetic symbols.  Additional shortcut
characters for symbols such as \ae{}, \oe{}, \o{} may also be used if
you are using a T1 encoded font and an appropriate input system for
it.

The following pair of examples show the same phonetic transcription of 
an English word that are inputted by the above mentioned two input methods.

\begin{tipaexample}
  \xitem
    \verb|[\textsecstress\textepsilon kspl\textschwa|\\
    \verb|\textprimstress ne\textsci\textesh\textschwa n]|
  \xitem {[\textsecstress\textepsilon kspl\textschwa\textprimstress
    ne\textsci\textesh\textschwa n]}
  \xitem \verb|\textipa{[""Ekspl@"neIS@n]}|
  \xitem \textipa{[""Ekspl@"neIS@n]}
\end{tipaexample}

It is apparent that inputting shortcut characters in the IPA
environment is far easier than inputting lengthy symbol names in the
normal text environment.

Moreover, although the outputs of the above examples look
almost the same, they are \textsl{not\/} identical, exactly
speaking. This is because in the IPA environment automatic kerning
between neighboring symbols is enabled, as is illustrated by the
following pair of examples.

\begin{tipaexample}
  \xitem \verb|v\textturnv v w\textsca w |
         \verb|y\textturny y [\textesh]|
  \xitem v\textturnv v w\textsca w y\textturny y [\textesh]
  \xitem \verb|\textipa{v2v w\textsca w yLy [S]}|
  \xitem \textipa{v2v w\textsca w yLy [S]}
\end{tipaexample}

In the next example, \emph{Input 2} is far better theen \emph{Input 1}, 
for the same reason.

\begin{tipaexample}
  \xitem \verb|[\textipa{S}]|
  \xitem [\textipa{S}]
  \xitem \verb|\textipa{[S]}|
  \xitem \textipa{[S]}
\end{tipaexample}

Therefore, it is recommended to use \texttt{\tbs textipa} or other IPA
environments as much as possible.

Table~\ref{tab:shortcut} shows most of the shortcut characters that
can be used in the IPA environment, together with the corresponding
characters in the ASCII encoding.


\begin{table}
\begin{center}\tabcolsep1em
\begin{tabular}{llc}
\hline
  \emph{Symbol name}  & \emph{Macro name}  & \emph{Symbol}\\
\hline
  Turned A            & \verb|\textturna|      & \textturna  \\
  Glottal stop        & \verb|\textglotstop|   & \textglotstop \\
  Right-tail D        & \verb|\textrtaild|     & \textrtaild \\
  Small capital G     & \verb|\textscg|        & \textscg \\
  Hooktop B           & \verb|\texthtb|        & \texthtb \\
  Curly-tail C        & \verb|\textctc|        & \textctc \\
  Crossed H           & \verb|\textcrh|        & \textcrh \\
  Old L-Yogh ligature & \verb|\textOlyoghlig|  & \textOlyoghlig \\
  Beta                & \verb|\textbeta|       & \textbeta \\
\hline
\end{tabular}
\end{center}
\caption{Naming of \tipa{} symbols}\label{tab:naming}
\end{table}

\subsection{Naming of phonetic symbols}

Every \tipa{} phonetic symbol has a unique symbol name, such as
\textsl{Turned A, Hooktop B, Schwa}. Also each symbol has a
corresponding macro name, such as \verb|\textturna|, \verb|\texthtb|,
\verb|\textschwa|. The naming was made based on the literature listed
in section~\ref{sec:selection}.  Among them, \PSG{} is particularly
important because it gives several explicit principles on naming. As
an example, the three terms `turned', `inverted' and `reversed' are
distinguished in the following way (p.\ xxvii):

\begin{quote}
\begin{description}\itemsep0pt
\item[Turned] rotated by 180 degrees (e.g., \textipa{t} vs. \textipa{\*t})
\item[Inverted] vertical mirror image (e.g., \textipa{\;R} vs. \textipa{K})
\item[Reversed] horizontal mirror image (e.g., \textipa{P} vs. \textipa{Q})
\end{description}
\end{quote}

The name used as a control sequence is usually an abbreviated form of
the corresponding symbol name with a prefix \verb|\text|. The
conventions used in the abbreviation can be summarized as follows.

\begin{itemize}
\itemsep0pt
\item Suffixes and endings such as `-ive', `-al', `-ed' are omitted.
\item `right', `left' are abbreviated to \texttt{r}, \texttt{l} respectively.
\item For `small capital' symbols, prefix \texttt{sc} is added.
\item A symbol with a hooktop is abbreviated as \texttt{ht}...
\item A symbol with a curly-tail is abbreviated as \texttt{ct}...
\item A `crossed' symbol is abbreviated as \texttt{cr}...
\item A ligature is abbreviated as ...\texttt{lig}.
\item For an old version of a symbol, prefix \texttt{O} is added.
\end{itemize}

Note that the prefix \texttt{O} (old) should be given in uppercase
letter.

Table~\ref{tab:naming} shows some examples of correspondence between
symbol names and control sequence names.

\subsection{Ligatures}

Just like the symbols such as ``, '', --, ---, fi, ff are realized as
ligatures by inputting \verb|``|, \verb|''|, \verb|--|, \verb|---|,
\verb|fi|, \verb|ff| in \TeX{}, two of the \tipa{} symbols, namely
\textsl{Secondary Stress} and \textsl{Double Pipe}, and double
quotation marks\footnote{Although \tipa{} fonts do not include the
  symbols `` and '', a negative value of kerning is automatically
  inserted between ` and `, ' and ', so that the same results can be
  obtained as in the case of the normal text font.} can be inputted as
ligatures in the IPA environment.

\begin{tipaexample}
  \yitem \verb+\textipa{" "" | || `` ''}+
  \yitem \textipa{" "" | || `` ''}
\end{tipaexample}

\subsection{Special macros {\tt\tbs*}, {\tt\tbs;}, {\tt\tbs:} and
  {\tt\tbs!}}\label{sec:specialmacros}

\tipa{} defines {\tt\tbs*}, {\tt\tbs:}, {\tt\tbs;} and {\tt\tbs!} as
special macros in order to easily input phonetic symbols that do not
have a shortcut character explained above.  Before explaining how to
use these macros, it is necessary to note that these macros are
primarily intended to be used by linguists who usually do not care
about things in math mode. And they can be `dangerous' in that they
override existing \LaTeX{} commands used in the math mode. So if you
want to preserve the original meaning of these commands, declare the
option `\texttt{safe}' at the preamble.
(However, \tipa{} provides cammands called \verb|\tipamedspace|,
\verb|\tipathickspace| and \verb|\tipanegthinspace|, having the same
meanings as {\tt\tbs:}, {\tt\tbs;} and {\tt\tbs!}, respectively. These
can be used even in `unsafe' mode.)


The macro \verb|\*| is used in three different ways.  First, when this
macro is followed by one of the letters f, k, r, t or w, it results in
a turned symbol.\footnote{This idea was pointed out by J\"org
  Knappen.}

\begin{tipaexample}
  \yitem \verb|\textipa{\*f \*k \*r \*t \*w}|
  \yitem \textipa{\*f \*k \*r \*t \*w}
\end{tipaexample}

Secondly, when this macro is followed by one of the letters j, n, h,
l or z, it results in a frequently used symbol that otherwise has no
easy way to input.

\begin{tipaexample}
  \yitem \verb|\textipa{\*j \*n \*h \*l \*z}|
  \yitem \textipa{\*j \*n \*h \*l \*z}
\end{tipaexample}

Thirdly, when this macro is followed by letters other than the above
cases, they are turned into the symbols of the default text font. This
is useful in the IPA environment to select symbols temporarily from
the normal text font.

\begin{tipaexample}
  \yitem \verb|\textipa{\*A dOg, \*B k\ae{}t, |
    \verb|ma\super{\*{214}}}|
  \yitem \textipa{\*A dOg, \*B k\ae{}t, ma\super{\*{214}}}
\end{tipaexample}

The remaining macros {\tt\tbs;}, {\tt\tbs:} and {\tt\tbs!} are
used to make small capital symbols, retroflex symbols, and implosives
or clicks, respectively.

\begin{tipaexample}
  \yitem \verb|\textipa{\;B \;E \;A \;H \;L \;R}|
  \yitem \textipa{\;B \;E \;A \;H \;L \;R}
  \yitem \verb|\textipa{\:d \:l \:n \:r \:s \:z}|
  \yitem \textipa{\:d \:l \:n \:r \:s \:z}
  \yitem \verb|\textipa{\!b \!d \!g \!j \!G \!o}|
  \yitem \textipa{\!b \!d \!g \!j \!G \!o}
\end{tipaexample}

\subsection{Punctuation marks}

The following punctuation marks and text symbols that are
normally included in the text encoding are also included in the
\texttt{T3} encoding so that they can be directly inputted in the IPA
environment.

\begin{tipaexample}
 \yitem \verb|\textipa{! ' ( ) * + , - . / = ? [ ] `}|
 \yitem \textipa{! ' ( ) * + , - .\ / = ? [ ] `}
\end{tipaexample}

All the other punctuation marks and text symbols that are not included
in \texttt{T3} need to be inputted with a prefix \verb|\*| explained in
the last section when they appear in the IPA environment.

\begin{tipaexample}
 \yitem
   \verb|\textipa{\*; \*: \*@ \*\# \*\$ \*\& \*\% \*\{ \*\}}|
 \yitem \textipa{\*; \*: \*@ \*\# \*\$ \*\& \*\% \*\{ \*\}}
\end{tipaexample}

\subsection{Accents and diacritics}

Table~\ref{tab:accent} shows how to input accents and diacritics in
\tipa{} with some examples. Here again, there are two kinds of input
methods; one for the normal text environment, and the other for the
IPA environment.

\begin{table}
\begin{center}\tabcolsep1em
\def\TblShrt#1{\hbox to 2em{#1\hss}}
\def\TblMvRt#1{\kern 2em #1}
\begin{tabular}{llc}
\hline
\textit{Input in the normal}& \textit{Input in the IPA} & \textit{Output} \\
\textit{text environment}   & \textit{environment}      & \\
\hline
\TblShrt{\Tt{'a}}               & \TblMvRt{\Tt{'a}}        &\textipa{\'a} \\
\TblShrt{\Tt{"a}}               & \TblMvRt{\Tt{"a}}        &\textipa{\"a} \\
\TblShrt{\Tt{~a}}               & \TblMvRt{\Tt{\ttilde a}} &\textipa{\~a} \\
\TblShrt{\Td{r}{a}}             & \TblMvRt{\Td{r}{a}}      &\textipa{\r{a}}\\
\TblShrt{\Td{textsyllabic}{m}}  & \TblMvRt{\Td{s}{m}}      &\textipa{\s{m}}\\
\TblShrt{\Td{textsubumlaut}{a}} & \TblMvRt{\Tt{"*a}}       &\textipa{\"*a}\\
\TblShrt{\Td{textsubtilde}{a}}  & \TblMvRt{\Tt{\ttilde*a}} &\textipa{\~*a}\\
\TblShrt{\Td{textsubring}{a}}   & \TblMvRt{\Tt{r*a}}       &\textipa{\r*a}\\
\TblShrt{\Td{textdotacute}{e}}  & \TblMvRt{\Tt{.'e}}       &\textipa{\.'e}\\
\TblShrt{\Td{textgravedot}{e}}  & \TblMvRt{\Tt{`.e}}       &\textipa{\`.e}\\
\TblShrt{\Td{textacutemacron}{a}}& \TblMvRt{\Tt{'=a}}      &\textipa{\'=a}\\
\TblShrt{\Td{textcircumdot}{a}} & \TblMvRt{\Tt{\tcircum.a}}&\textipa{\^.a}\\
\TblShrt{\Td{texttildedot}{a}}  & \TblMvRt{\Tt{\ttilde.a}} &\textipa{\~.a}\\
\TblShrt{\Td{textbrevemacron}{a}}& \TblMvRt{\Tt{u=a}}      &\textipa{\u=a}\\
%\Td{}{a}                    & \Tt{}              &\textipa{}\\
\hline
\end{tabular}
\end{center}
\caption{Examples of inputting accents and diacritics}\label{tab:accent}
\end{table}

In the IPA environment, most of the accents and diacritics can be
inputted more easily than in the normal text environment, especially in
the cases of subscript symbols that are normally placed over a symbol
and in the cases of combined accents, as shown in the table.

As can be seen by the above examples, most of the accents that are
normally placed over a symbol can be placed under a symbol by adding
an \texttt{*} to the corresponding accent command in the IPA environment.

The advantage of IPA environment is further exemplified by the
all-purpose accent \verb+\|+, which is used as a macro prefix to
provide shortcut inputs for the diacritics that otherwise have to be
inputted by lengthy macro names. Table~\ref{tab:accprefix} shows examples
of such accents. Note that the macro \verb+\|+ is also `dangerous' in
that it has been already defined as a math symbol of \LaTeX{}. So if
you want to preserve the original meaning of this macro, declare
`\texttt{safe}' option at the preamble.
(However, there is an alternative command called \verb+\Vert+
(originally defined in plain \TeX) which has the same meaning as
\verb+\|+ and can be used even if the \texttt{safe} option is not
specified.)

\begin{table}
\begin{center}\tabcolsep1em
\def\TblShrt#1{\hbox to 2em{#1\hss}}
\def\TblMvRt#1{\kern 2em #1}
\begin{tabular}{llc}
\hline
\textit{Input in the normal}& \textit{Input in the IPA} & \textit{Output} \\
\textit{text environment}   & \textit{environment}      & \\
\hline
\TblShrt{\Td{textsubbridge}{t}}   & \TblMvRt{\Tt{|[t}}   & \textipa{\|[t}\\
\TblShrt{\Td{textinvsubbridge}{t}}& \TblMvRt{\Tt{|]t}}   & \textipa{\|]t}\\
\TblShrt{\Td{textsublhalfring}{a}}& \TblMvRt{\Tt{|(a}}   & \textipa{\|(a}\\
\TblShrt{\Td{textsubrhalfring}{a}}& \TblMvRt{\Tt{|)a}}   & \textipa{\|)a}\\
\TblShrt{\Td{textroundcap}{k}}    & \TblMvRt{\Td{|c}{k}} & \textipa{\|c{k}}\\
\TblShrt{\Td{textsubplus}{o}}     & \TblMvRt{\Tt{|+o}}   & \textipa{\|+o}\\
\TblShrt{\Td{textraising}{e}}     & \TblMvRt{\Tt{|'e}}   & \textipa{\|'e}\\
\TblShrt{\Td{textlowering}{e}}    & \TblMvRt{\Tt{|`e}}   & \textipa{\|`e}\\
\TblShrt{\Td{textadvancing}{o}}   & \TblMvRt{\Tt{|<o}}   & \textipa{\|<o}\\
\TblShrt{\Td{textretracting}{a}}  & \TblMvRt{\Tt{|>a}}   & \textipa{\|>a}\\
\TblShrt{\Td{textovercross}{e}}   & \TblMvRt{\Td{|x}{e}} & \textipa{\|x{e}}\\
\TblShrt{\Td{textsubw}{k}}        & \TblMvRt{\Td{|w}{k}} & \textipa{\|w{k}}\\
\TblShrt{\Td{textseagull}{t}}     & \TblMvRt{\Td{|m}{t}} & \textipa{\|m{t}}\\
\hline
\end{tabular}
\end{center}
\caption{Examples of the accent prefix {\tt \tbs|}}\label{tab:accprefix}
\end{table}

Finally, examples of words with complex accents that are inputted in the
IPA environment are shown below.

\begin{tipaexample}
  \yitem \verb+\textipa{*\|c{k}\r*mt\'om +
    \verb+*bhr\'=at\=er}+
  \yitem \textipa{*\|c{k}\r*mt\'om *bhr\'=at\=er}
\end{tipaexample}

For a full list of accents and diacritics, see Appendix~A.

\subsection{Superscript symbols}

In the normal text environment, superscript symbols can be inputted by a
\LaTeX{} macro called \verb|\textsuperscript|.  This macro takes one
argument which can be either a symbol or a string of symbols, and can
be nested.

Since the name of this macro is too long, \tipa{} provides an
abbreviated form of this macro called \verb|\super|.

\begin{tipaexample}
  \xitem
    \verb|t\textsuperscript h |
    \verb|k\textsuperscript w|\\
    \verb|a\textsuperscript{bc}|\\
    \verb|a\textsuperscript{b\textsuperscript{c}}|
  \xitem
    t\textsuperscript h k\textsuperscript w a\textsuperscript{bc} 
      a\textsuperscript{b\textsuperscript{c}}
  \xitem
    \verb|\textipa{t\super{h} k\super{w}|\\
    \verb|  a\super{bc} a\super{b\super{c}}}|
  \xitem
    \textipa{\textipa{t\super{h} k\super{w} a\super{bc}
      a\super{b\super{c}}}}
\end{tipaexample}

\noindent
(A quiz question: A careful reader may have noticed that the above
\textit{Output1} and \textit{Output2} slightly differ. Explain the
reason.)

\medskip

These macros automatically select the correct size of superscript font 
no matter what size of text font is used.

\subsection{Tone letters}\label{sec:tone}

\tipa{} provides a flexible system of macros for `tone letters'. A
tone letter is represented by a macro called `\verb|\tone|', which
takes one argument consisting of a string of numbers ranging from 1 to
5. These numbers denote pitch levels, 1 being the lowest and 5, the
highest. Within this range, any combination is allowed and there is no
limit in the length of combination.

As an example of the usage of the tone letter macro, the four tones of
Chinese are shown below.

\begin{tipaexample}
  \yitem
    \verb|\tone{55}ma ``mother'', |
    \verb|\tone{35}ma ``hemp'',|\\
    \verb|\tone{214}ma ``horse'', |
    \verb|\tone{51}ma ``scold''|
  \yitem
    ma\tone{55} ``mother'', ma\tone{35} ``hemp'',
    ma\tone{214} ``horse'', ma\tone{51} ``scold''
\end{tipaexample}

The next example looks ridiculous but shows capabilities of the tone
letter macro.

\begin{tipaexample}
  \yitem
    \verb|\tone{15253545}|
  \yitem
    \tone{15253545}
\end{tipaexample}

\subsubsection{{\tt\tbs stone}}

In some languages, length distinctions accompany the tone letter
description. In such cases a command called \verb|\stone| can be used
to represent a tone letter that is shorter than a usual one. The next
example from Cantonese illustrates this (look at the examples for
\emph{entering tones}).

\begin{center}
\begin{tabular}{l|c|c}
\emph{Tone name} & \emph{Input} & \emph{Output} \\
\hline
high level & \verb|\tone{53}| or \verb|\tone{55}|&\tone{53} or \tone{55}\\
low level  & \verb|\tone{21}| or \verb|\tone{22}|&\tone{21} or \tone{22}\\
\hline
high rising     & \verb|\tone{35}|  & \tone{35} \\
low rising      & \verb|\tone{24}|  & \tone{24} \\
\hline
high departing  & \verb|\tone{44}|  & \tone{44} \\
low departing   & \verb|\tone{33}|  & \tone{33} \\
\hline
high entering   & \verb|\stone{55}| & \stone{55} \\
mid entering    & \verb|\stone{44}| & \stone{44} \\
low entering    & \verb|\stone{33}| & \stone{33} \\
\end{tabular}
\end{center}

\subsubsection{{\tt\tbs rtone}}

In some languages, the level/contour bars are placed at the right hand
side of the vertical bar. In such cases a command called \verb|\rtone|
is used instead of \verb|\tone|. The next example is from the Kyoto
Japanese.

\begin{tipaexample}
  \yitem
    \verb|\textipa{[\rtone{11}a\rtone{53}me]} `rain'|
  \yitem
    \textipa{[\rtone{11}a\rtone{53}me]} `rain'
\end{tipaexample}


\section{How easy is it to input phonetic symbols?}

Let us briefly estimate here how easy (or difficult) it is to input
phonetic symbols with \tipa{} in terms of the number of keystrokes.

The following table shows statistics for all the phonetic symbols that
appear in the '93 version of IPA chart (diacritics and symbols for
su\-pra\-seg\-men\-tals excluded). It is assumed here that each symbol
is inputted within the IPA environment and the \texttt{safe} option is
not specified.

\begin{center}
\begin{tabular}{c|c|l}
\emph{keystrokes} & \emph{number} & \emph{examples} \\
\hline
1            & 65 & \textipa{a, b, @, A, B, etc.} \\
2            & 2  & \textipa{\o, ||} \\
3            & 30 & \textipa{\ae, \:t, \;B, \!b, etc.} \\
5            & 1  & \textipa{\c{c}} \\
more than 5  & 7  & \textipa{\textcloseepsilon, \textbarglotstop,
  \textdoublebarpipe, \textturnmrleg, etc.}
\end{tabular}
\end{center}

As is shown in the table, about 92\% of the symbols can be inputted
within three keystrokes.

\section{Changing font styles}

\begin{table}
\begin{center}
\begin{tabular}{@{\hspace{0pt}}lll}
\hline
\textit{Font style}& \textit{Input in the IPA environment}& \textit{Output} \\
\hline
\emph{Roman}&\verb|\textipa{f@"nEtIks}|        &\textipa{f@"nEtIks} \\
\emph{Slanted}&\verb|\textipa{\slshape f@"nEtIks}|&\textipa{\slshape f@"nEtIks}\\ 
or          &\verb|\textipa{\textsl{f@"nEtIks}|&\textipa{\textsl{f@"nEtIks}}\\ 
or          &\verb|\textsl{\textipa{f@"nEtIks}|&\textsl{\textipa{f@"nEtIks}}\\ 
\emph{Bold extended} & \verb|\textipa{\bfseries f@"nEtIks}| &
  \textipa{\bfseries f@"nEtIks}\\
or          &\verb|\textipa{\textbf{f@"nEtIks}|&\textipa{\textbf{f@"nEtIks}}\\
or          &\verb|\textbf{\textipa{f@"nEtIks}|&\textbf{\textipa{f@"nEtIks}}\\
\emph{Sans serif}& \verb|\textipa{\sffamily f@"nEtIks}| &
  \textipa{\sffamily f@"nEtIks}\\ 
or          &\verb|\textipa{\textsf{f@"nEtIks}|&\textipa{\textsf{f@"nEtIks}}\\ 
or          &\verb|\textsf{\textipa{f@"nEtIks}|&\textsf{\textipa{f@"nEtIks}}\\ 
\emph{Typewriter Text}& \verb|\textipa{\ttfamily f@"nEtIks}| &
  \textipa{\ttfamily f@"nEtIks}\\ 
or          &\verb|\textipa{\texttt{f@"nEtIks}|&\textipa{\texttt{f@"nEtIks}}\\ 
or          &\verb|\texttt{\textipa{f@"nEtIks}|&\texttt{\textipa{f@"nEtIks}}\\ 
\hline
\end{tabular}
\end{center}
\caption{Examples of font switching}\label{tab:fontswitch}
\end{table}

This version of \tipa{} includes five styles of fonts, i.e., roman,
slanted, bold, bold extended, sans serif and typewriter. These styles can be
switched in much the same way as in the normal text fonts (see
table~\ref{tab:fontswitch}).

The bold fonts are usually not used within the standard \LaTeX{} class 
packages so that if you want to use them, it is necessary to use
low-level font selection commands of \LaTeXe.

\begin{tipaexample}
  \yitem \verb|{\fontseries{b}\selectfont|
    \verb|abcdefg \textipa{ABCDEFG}}|
  \yitem {\fontseries{b}\selectfont abcdefg \textipa{ABCDEFG}}
\end{tipaexample}

Note also that slanting of \tipa{} symbols should correctly work even
in the cases of nested accents and in the cases of symbols made up
by macros.

\begin{tipaexample}
  \yitem \verb|\textsl{\textipa{\'{\"{\u*{e}}}}}|
  \yitem \textsl{\textipa{\'{\"{\u*{e}}}}}
  \yitem \verb|\textsl{\textdoublebaresh}|
  \yitem \textsl{\textdoublebaresh\/} {\small(This symbol is
    composed by a macro.)}
\end{tipaexample}


\clearemptydoublepage
\chapter{Customizing TIPA}


\section{Internal commands}

Some of the internal commands of \tipa{} are defined without the
letter \texttt{@} in order to allow a user to extend the capability of 
\tipa{}.

\subsection{{\tt\tbs ipabar}}

Some \tipa{} symbols such as \verb|\textbarb| \textbarb,
\verb|\textcrtwo| \textcrtwo{} are defined by using an internal macro
command \verb|\ipabar|. This command is useful when you want to make
barred or crossed symbols not defined in \tipa{}.

This command requires the following five parameters to control the
position and length of the bar.

\begin{itemize}
\itemsep0pt
\item {\tt\#1} the symbol to be barred
\item {\tt\#2} the height of the bar (in dimen)
\item {\tt\#3} bar width
\item {\tt\#4} left kern added to the bar
\item {\tt\#5} right kern added to the bar
\end{itemize}

Parameters \texttt{\#3}, \texttt{\#4}, \texttt{\#5} are to be given in
a scaling factor to the width of the symbol, which is equal to 1 if
the bar has the same width as the symbol in question. For example,
the following command states a barred b (\textbarb) of which the bar
position in the y-coordinate is \verb|.5ex| and the width of the bar
is slightly larger than that of the letter b.

\begin{verbatim}
  % Barred B
  \newcommand\textbarb{%
    \ipabar{{\tipaencoding b}}{.5ex}{1.1}{}{}}
\end{verbatim}

Note that the parameters \texttt{\#4} and \texttt{\#5} can be left
blank if the value is equal to 0.

And the next example declares a barred c (\textbarc) of which the bar
width is a little more than half as large as the letter c and it has
the same amount of kerning at the right.

\begin{verbatim}
  % Barred C
  \newcommand\textbarc{%
    \ipabar{{\tipaencoding c}}{.5ex}{.55}{}{.55}}
\end{verbatim}

More complex examples with the \verb|\ipabar| command are found in
\texttt{T3enc.def}.

\subsection{{\tt\tbs tipaloweraccent}, {\tt\tbs tipaupperaccent}}

These two commands are used in the definitions of \tipa{} accents and
diacritics. They are special forms of the commands \verb|\loweraccent|
and \verb|\upperaccent| that are defined in \texttt{exaccent.sty}. The
difference between the commands with the prefix \verb|tipa| and the
ones without it is that the former commands select accents from a T3
encoded font while the latter ones do so from the current text font.

These commands take two parameters, the code of the accent (in
decimal, octal or hexadecimal number) and the
symbol to be accented, as shown below.

\begin{tipaexample}
  \yitem \verb|\tipaupperaccent{0}{a}|
  \yitem \tipaupperaccent{0}{a}
\end{tipaexample}

Optionally, these commands can take an extra parameter to adjust the
vertical position of the accent. Such an adjustment is sometimes
necessary in the definition of a nested accent. The next example shows
\tipa's definition of the `Circumflex Dot Accent' (e.g., 
\textcircumdot{a}).

\begin{verbatim}
  % Circumflex Dot Accent
  \newcommand\textcircumdot[1]{\tipaupperaccent[-.2ex]{2}%
    {\tipaupperaccent[-.1ex]{10}{#1}}}
\end{verbatim}

This definition states that a dot accent is placed over a symbol
thereby reducing the vertical distance between the symbol and the dot
by \texttt{.1ex}, and a circumflex accent is placed over the dot and
the distance between the two accents is reduced by \texttt{.2ex}.

If you want to make a combined accent not included in \tipa{}, you can
do so fairly easily by using these two commands together with the
optional parameter. For more examples of these commands, see
\verb|tipa.sty| and \verb|extraipa.sty|.

\subsection{{\tt\tbs tipaLoweraccent}, {\tt\tbs tipaUpperaccent}}

These two commands differ from the two commands explained above in that
the first parameter should be a symbol (or any other thing, typically
an \verb|\hbox|), rather than the code of the accent. They are special
cases of the commands \verb|\Loweraccent| and \verb|\Upperaccent| and
the difference between the two pairs of commands is the same as before.

The next example makes a schwa an accent.

\begin{tipaexample}
  \yitem
    \verb|\tipaUpperaccent[.2ex]%|\\
    \verb|  {\lower.8ex\hbox{\textipa{\super@}}}{a}|
  \yitem
    \tipaUpperaccent[.2ex]{\lower.8ex\hbox{\textipa{\super@}}}{a}
\end{tipaexample}

The next example is an interesting application of this command for the
Middle High German (This macro and the example below are provided
by Christian Folini and now included in \texttt{tipa.sty}).

\begin{verbatim}
  \newcommand{\sups}[2]{\textipa{\tipaUpperaccent[.2ex]{%
    \lower.8ex\hbox{\super{#2}}}{#1}}}
\end{verbatim}

\begin{quote}
Und swer dc mit flis t\sups{u}{o}t, so stat das gelt und \sups{o}{v}ch
d\`{u} g\sups{u}{e}ter in deste bessere behabn\`{u}sse und
beh\`{u}gde.  (1330 AD.\ Translation: And if this is done with
diligence, the money and the affairs will be in better shape.)
\end{quote}

In this example, t\sups{u}{o}t is inputted as \verb|t\sups{u}{o}t| and so 
on.

\subsection{{\tt\tbs ipaclap}}

This command is useful if you need to compose a new symbol by
overlapping two symbols. This command is different from \TeX's
commands \texttt{\tbs llap} and \texttt{\tbs rlap} in that the
alignment is made at the center of each symbol.

The next example shows how to make a Slashed B.

\begin{tipaexample}
  \yitem \verb|\ipaclap{\textipa{b}}{\textipa{/}}|
  \yitem \ipaclap{\textipa{b}}{\textipa{/}}
\end{tipaexample}


\section{Manual Kerning}

The shapes of phonetic symbols are sometimes \emph{nasty} in the sense
that they can have a leftward or rightward protrusion that cannot
be found in the case of normal text fonts. In such cases it is
sometimes necessary to input kerning commands manually.

One way to do this is to prepare a set of kerning commands like the
following:

\begin{verbatim}
  \newcommand\K{\kern.05em} % small amount of kerning
  \newcommand\KK{\kern.1em} % middle amount of kerning
  \newcommand\KKK{\kern.2em} % big amount of kerning
\end{verbatim}

\newcommand\K{\kern.05em} % small amount of kerning
\newcommand\KK{\kern.1em} % middle amount of kerning
\newcommand\KKK{\kern.2em} % big amount of kerning

And then to put these commands whenever necessary. For example:

\begin{tipaexample}
  \yitem \verb|\textipa{[\textrhooke r]}|
  \yitem \textipa{[\textrhooke r]} --- This is OK but
  \yitem \verb|\textipa{[\textrhooke]}|
  \yitem \textipa{[\textrhooke]} --- this doesn't look good so that
  \yitem \verb|\textipa{[\textrhooke\KK]}|
  \yitem \textipa{[\textrhooke\KK]} --- manually fixed like this.
\end{tipaexample}

\clearemptydoublepage
\chapter*{Acknowledgments}
\addcontentsline{toc}{chapter}{Acknowledgments}

First of all, many thanks are due to the co-authors of \tsipa{},
Kobayashi Hajime and Shirakawa Shun. Ko\-ba\-yashi Hajime was the main
font designer of \tsipa{}. Shirakawa Shun worked very hard in deciding
encoding, checking the shapes of symbols and writing the Japanese
version of the document. \tipa{} was impossible without \tsipa{}.

I would like to thank also J\"org Knappen whose insightful comments
greatly helped the development of \tipa{} in many ways. I was also
helped and encouraged by Christina Thiele, Martin Haase, Kirk Sullivan
and many other members of the \texttt{ling-tex} mailing list.

At the last stage of the development of \tipa{}, Frank Mittelbach gave
me precious comments on how to incorporate various \tipa{} commands
into the NFSS. I would like to thank also Barbara Beeton who kindly
read over the preliminary draft of this document and gave me useful
comments.

After the first release, I also received useful comments from:
Dominique Unruh, Peter Zimmermann, Rafael Laboissi\`ere, Yoshinari
Fujino, Walter Schmidt, Dirk Janssen, Joachim Becker, Christian
Folini, Conrado Badenas, Alexis Dimitriadis, John Frampton and
probably from many others. I am very sorry that I haven't recorded all
the names. Please let me know if your name is missing. I am also sorry
that I sometimes have been unable to follow all the advice simply
because of lack of time. Finally, Donna Erickson kindly read over the
draft of the present version and corrected my English. All remaining
errors are, of course, my own.

\clearemptydoublepage
\bibliographystyle{plainnat}
\addcontentsline{toc}{chapter}{Bibliography}
\bibliography{tipa}

%%% Local Variables: 
%%% mode: latex
%%% TeX-master: "tipaman"
%%% End: 
