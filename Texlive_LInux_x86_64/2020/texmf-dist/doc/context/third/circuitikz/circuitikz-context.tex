% Copyright 2018-2021 by Romano Giannetti
% Copyright 2015-2021 by Stefan Lindner
% Copyright 2013-2021 by Stefan Erhardt
% Copyright 2007-2021 by Massimo Redaelli
%
% This file may be distributed and/or modified
%
% 1. under the LaTeX Project Public License and/or
% 2. under the GNU Public License.
%
% See the files gpl-3.0_license.txt and lppl-1-3c_license.txt for more details.


\usemodule[circuitikz]

\starttext

A simple example to test the installation.

\startcircuitikz[scale=1.2]
	\draw
  (0,2) to[I=1\milli\ampere] (2,2)
        to[R, l_=2\kilo\ohm, *-*] (0,0)
        to[R, l_=2\kilo\ohm] (2,0)
        to[V, v_=2\volt] (2,2)
        to[cspst, l=$t_0$] (4,2) -- (4,1.5)
        to [generic, i=$i_1$, v=$v_1$] (4,-.5) -- (4,-1.5)
  (0,2) -- (0,-1.5) to[V, v_=4\volt] (2,-1.5)
        to [R, l=1\kilo\ohm] (4,-1.5)
  (5,2) node[dipchip, anchor=pin 1]{}
  (5,-2) node[flipflop JK, anchor=pin 1]{};

\stopcircuitikz

\stoptext
