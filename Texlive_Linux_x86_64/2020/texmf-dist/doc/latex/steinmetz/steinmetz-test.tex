%%
%% This is file `steinmetz-test.tex',
%% generated with the docstrip utility.
%%
%% The original source files were:
%%
%% steinmetz.dtx  (with options: `example')
%% 
%% This is a generated file.
%% 
%% Copyright (C) 2009 by Enrico Gregorio <Enrico dot Gregorio (at) univr dot it>
%% 
%% This file may be distributed and/or modified under the conditions of
%% the LaTeX Project Public License, either version 1.3 of this license
%% or (at your option) any later version.  The latest version of this
%% license is in:
%% 
%%    http://www.latex-project.org/lppl.txt
%% 
%% and version 1.3 or later is part of all distributions of LaTeX version
%% 2005/12/01 or later.
%% 
\documentclass[a4paper]{article}
\usepackage{steinmetz}

\begin{document}

We will indicate the amplitude and phase relationship through the use
of complex notation: a complex number is used to indicate only the
amplitude and phase of voltages and currents in the circuit (since the
sinusoidal time variation factor is common to all terms).  For
example, a circuit described by the equation
\[
i(t)=I_{0}\cos(\omega t+\theta)
\]
can be written in complex exponential form as
\[
i(t)=\Re\{I_{0}e^{j(\omega t+\theta)}\}=
\Re\{I_{0}e^{j\theta}e^{j\omega t}\},
\]
and in polar complex form as
\[
i(t)=\Re\{I_{0}\phase{\theta}e^{j\omega t}\}.
\]
Finally, we can simplify the notation by dropping the implied
$e^{j\omega t}$ term and the $\Re\{~\}$ operator, leaving the phasor
notation:
\[
\mathbf{I}=I_{0}\phase{\theta}=I_{0}e^{j\theta}=
I_{0}(\cos\theta+j\sin\theta),
\]
where the boldface $\mathbf{I}$ reminds us that the phasor quantity
$\mathbf{I}$ is a complex number.  The important advantage of this
approach is that the mathematics involves mostly simple algebraic
operations on the magnitudes and phases.

It is interesting to look at the complex ratio of phasor voltage and
phasor current, $\mathbf{V}\!/\mathbf{I}$, which is called the
\emph{impedance}~$\mathbf{Z}$.  For the basic circuit elements we
find:
\begin{itemize}
\item Resistor, $R$: $\mathbf{Z}=R$,
\item Inductor, $L$: $\mathbf{Z}=j\omega L=
  \omega L\phase{90^{\circ}}$,
\item Capacitor, $C$: $\displaystyle\mathbf{Z}=\frac{1}{j\omega C}=
  -j\frac{1}{\omega C}=\frac{1}{\omega C}\phase{-90^{\circ}}$.
\end{itemize}
\end{document}
\endinput
%%
%% End of file `steinmetz-test.tex'.
