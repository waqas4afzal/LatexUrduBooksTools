\ProvidesFile{fnlineno.tex}[2011/02/14 documenting fnlineno.sty (UL)]
\title{\textsf{fnlineno.sty}\\---\\Numbering Footnote Lines\thanks{This 
       document %%% manual %% 2010/12/28
       describes version 
       \textcolor{blue}{\UseVersionOf{fnlineno.sty}} 
       of \textsf{fnlineno.sty} as of \UseDateOf{fnlineno.sty}.}}
% \listfiles                                          %% 2010/12/22
{ \RequirePackage{makedoc}[2010/12/20] \ProcessLineMessage{} 
  \MakeJobDoc{19}{\SectionLevelThreeParseInput}     %% 2010/12/16
}
\documentclass{article}%% TODO paper dimensions!?
\input{makedoc.cfg} %% shared formatting settings
\newcommand*{\lt}{<} \newcommand*{\gt}{>}           %% 2010/12/22
\providecommand*{\strong}{\textbf}                  %% 2010/12/15
\ReadPackageInfos{fnlineno}
\usepackage{color}
% \hypersetup{bookmarksopen}    %% rm. 2010/12/21, cf. .cfg
%% 2010/12/21: %% 2010/12/26 not sure, splits code
% \makeatletter \@beginparpenalty\@highpenalty \makeatother
%% 2010/12/27:
\makeatletter \@beginparpenalty\@lowpenalty \makeatother
\sloppy
\begin{document}
\maketitle
\begin{abstract}\noindent
'fnlineno.sty' extends 
% Stephan~I. B\"ottcher's 
\CtanPkgRef{lineno}{lineno.sty}\urlpkgfoot{lineno}
(created by Stephan~I. B\"ottcher) 
such that even 
`\footnote'                                 %% `\' 2010/12/09
lines are numbered and can be referred to 
using `\linelabel', `\ref', etc. 
%% rm. 2011/02/09:
% Version v0.5 aims at working as a user expects 
% (just cf.~``Limitations"), otherwise please complain!

Making the package was motivated as support for 
\emph{critical editions}
% of scientific work from an age when footnotes 
% were a standard in publishing in print, 
%% <- 2011/02/09 ->
of \emph{printed works with footnotes} 
as opposed to scholarly critical editions of \emph{manuscripts.} 
For this purpose, an extension 'edfnotes' of the \ctanpkgref{ednotes}
package for critical editions, building on 'fnlineno', is provided 
by the \textit{ednotes} bundle.\urlfoot{CtanPkgRef}{ednotes}

'lineno.sty' has also been used for the revision process 
of \emph{submissions.} 
With 'fnlineno.sty', reference to footnotes 
in the submitted work may become possible. 

%% rm. 2011/02/09:
% Another standalone package 'finstrut' is described. 
As to \emph{implementation:}    %% 2011/02/14
1.~Some included tools for 
\emph{storing and restoring global settings} 
may be ``exported" as standalone packages later. 
2.~The method of typesetting footnotes on the main vertical list 
may later lead to applying the line numbering method to 
several \emph{parallel} texts (with footnotes) and to 
`inner' material such as table cells.
%% <- 2011/02/14 ->
% \dots

%% new 2011/02/09:
  \smallskip\noindent
\strong{Keywords:}\quad line numbers; footnotes, pagewise, 
critical editions, revision
\end{abstract}
\tableofcontents

%   \newpage                    %% rm. 2011/02/09
\section{Usage and Features}
\subsection{Package File Header (Legalize)}
\NeedsTeXFormat{LaTeX2e}[1994/12/01]
\ProvidesPackage{fnlineno}[2011/01/07 v0.55
                           numbers to footnote lines (UL)]

%% Copyright (C) 2010 Uwe Lueck, 
%% http://www.contact-ednotes.sty.de.vu 
%% -- author-maintained in the sense of LPPL below -- 
%%
%% This file can be redistributed and/or modified under 
%% the terms of the LaTeX Project Public License; either 
%% version 1.3c of the License, or any later version.
%% The latest version of this license is in
%%     http://www.latex-project.org/lppl.txt
%% We did our best to help you, but there is NO WARRANTY. 
%%
%% Please report bugs, problems, and suggestions via 
%% 
%%   http://www.contact-ednotes.sty.de.vu 
%%
%% {\it This work has been supported 
%%  % and is owned      %% wrong, owner is RUB -- 2011/02/09
%% by the Deutsche Forschungsgemeinschaft (DFG), 
%% organized                        %% 2010/12/18 TODO!?
%% by Prof.~Dr.\ Dr.\ Christian Tapp 
%% at Ruhr-Universit\"at Bochum, Germany.}
%% %% 2010/12/18:
%% Christian also has constructed some critical tests.
%% 
%% === Installing and Calling ===
%% The file 'fnlineno.sty' is provided ready, installation only requires 
%% putting it somewhere where \TeX\ finds it 
%% (which may need updating the filename data 
%%  base).\footnote{\url{http://www.tex.ac.uk/cgi-bin/texfaq2html?label=inst-wlcf}}
%% 
%% As usually, 'fnlineno.sty' is loaded by 
%% % \begin{verbatim}
%%     `\usepackage{fnlineno}'
%% % \end{verbatim}
%% below the `\documentclass' line and before `\begin{document}'.
%%
%% === Limitations ===
%% v0.55 %%% v0.5 %% 2011/02/09
%% should really work 
%% the way users expect,    %% 2011/02/09
%% but please consider:
%% % In the \strong{present version (v0.4)}, 'flnineno.sty' is just 
%% % ``work in progress"---with at least the following issues:
%% \begin{enumerate}
%% %   \item Lines are numbered in an order 
%% %         that you may not like, in the presence of 
%% %         footnotes. 
%% %         Overcoming this failure by automagic 
%% %         will be the major further work on the package! 
%% %         (v0.5!?)
%% % 
%% %   \item The package is not expected to work with 'manyfoot' 
%% %         (neither 'ednotes' ... future work!).
%%
%%   \item Nothing is known about compatibility with 
%%         packages (other than 'manyfoot' and 'bigfoot')
%%         providing footnote features beyond standard \LaTeX.
%% 
%%   \item `\lipsum[<opt-arg>]' in main text produces 
%%         a different number of paragraphs ...
%% 
%% %% 2010/12/18:
%% %   \item `\pagebreak' (without optional argument, or 
%% %         `\pagebreak[4]') in footnotes prints footnote parts 
%% %         (splits) in wrong order.\footnote{TODO:
%% %              (i)~extra case (using flag) in &\output 
%% %              with $&\outputpenalty=-10000$; 
%% %              \emph{or}
%% %              (ii)~redefine &\pagebreak in footnotes ...}
%% %% 2010/12/19, "tried" "assumed" 2010/12/21:
%%   \item v0.41 tried supporting `\pagebreak' in footnotes 
%%         for manual control of splitting footnotes. 
%% %% 2010/12/20f.:
%%         However, it wrongly assumed that `\pagebreak[4]'
%%         forces a footnote split, cf. Section~\ref{sec:sw}; 
%%         users better still don't use `\pagebreak'
%%         in footnotes!
%% 
%%   \item Much of the code is ``guessed" without complete 
%%         knowledge of \TeX\ internals and without having 
%%         tested many possible cases.
%% %% 2010/12/27:
%%   \item \emph{Local} switching to ``pagewise" numbering
%%         won't be possible for a while; we rather assume 
%%         that you \emph{always} want ``pagewise" numbering. 
%%
%%   \item Nothing has been tried to offer choices about the 
%%         \emph{style} of numbering footnotes.
%% \end{enumerate}
%%
%%  \pagebreak              %% 2011/02/09
%%  == Implementation ==
%% === Terms ===
%% ``\strong{OTR}" is short for \textit{``output routine"}, 
%% ``\strong{MVL}" is short for \textit{``main vertical list"}.
%% %%% TODO "acro"
%%
%% === Basic Strategy ===
%% \LaTeX's |\@footnotetext| writes the footnote text into the 
%% insertion register. For numbering the footnote lines, 
%% \emph{we here} do not execute this `\@footnotetext' 
%% immediately after placing |\@footnotemark|, 
%% but postpone its |\insert| a little so it 
%% is executed only after the main text paragraph has been 
%% broken into lines. Right below the line that contains the 
%% footnote mark, a special new ``slot" of the \strong{OTR} 
%% is called that interchanges ``the page so far" with the 
%% footnote text. When the latter has been typeset, 
%% another ``slot" of the OTR puts ``the page so far" 
%% back to the MVL and immediately after that fills the footnote 
%% text as just typeset on the MVL into the `\insert' register. 
%% 
%% Passing footnotes from horizontal mode to vertical mode 
%% resembles 'lineno''s `\PostponeVadjust', but a 
%% different \strong{list} |\FNLN@list|
%% must store code 
%% (\textit{a})~for the footnote \strong{mark} and 
%% (\textit{b})~for the footnote \strong{text}. 
%% %%% (TODO: move this paragraph)
%%
%% === Package Options ===
%% A package option |[check-latex]|     %% typo article 2010/12/31
%% for checking vital \LaTeX\ internals may once be offered 
%% (TODO 2010/12/12) ...
\newif\if@FNLN@check@
\DeclareOption{check-latex}{\@FNLN@check@true}
\ProcessOptions
%%
%% === Footnote Commands ===
%% ==== Standard Footnotes ====
%% The following macro |\FNLN@ltx@fntext|           %% more 2010/12/16
%% is a copy of \LaTeX's `\@footnotetext' 
%% that we are varying. It may be used for a check 
%% if the `\@footnotetext' that 'fnlineno.sty' encounters 
%% is the one expected (TODO).
%% In line numbering mode, this code may never be needed 
%% all at once, rather we will have to see which material 
%% must be used at which point of our unusual way of 
%% processeing footnotes.
\if@FNLN@check@
  \long\def\FNLN@ltx@fntext#1{\insert\footins{%
      \reset@font\footnotesize
      \interlinepenalty\interfootnotelinepenalty
      \splittopskip\footnotesep
      \splitmaxdepth \dp\strutbox \floatingpenalty \@MM
      \hsize\columnwidth \@parboxrestore
      \protected@edef\@currentlabel{%
         \csname p@footnote\endcsname\@thefnmark
      }%
      \color@begingroup
        \@makefntext{%
          \rule\z@\footnotesep\ignorespaces#1\@finalstrut\strutbox}%
      \color@endgroup}}%
\fi
%%
%% ==== Modifying Footnote Commands ====
%% In order to number |\footnote| lines and make 
%% `\linelabel' available in footnotes, it seems to suffice 
%% (with standard \LaTeX) to \strong{redefine} 
%% the internal |\@footnotetext|.
%% In line numbering mode, `\@footnotetext' will act as 
%% |\FNLN@text|, 
%% % \begin{itemize}
%% %  \item[(i)] 
%% (\textit{i})~%
%%        placing a ``signal" \strong{output penalty}
%%        below the current line via `\vadjust' and
%% %  \item[(ii)] 
%% (\textit{ii})~%
%%        appending the footnote text to the 
%%        \strong{list} |\FNLN@list| of \strong{footnote texts}. 
%% % \end{itemize}
%% 
%% |\FNLN@@text| stores the `\@footnotetext' found, 
%% we might check if it is `\FNLN@ltx@fntext' ... %% TODO
\let\FNLN@@text\@footnotetext
\def\@footnotetext{%
    \ifLineNumbers  \expandafter \FNLN@text
    \else           \expandafter \FNLN@@text
    \fi}
\def \FNLN@text {%                      %% 2010/12/31 arg read later
    \vadjust{\penalty-\FNLN@M@swap@codepen}%
%% Standard \LaTeX's `\@footnotetext' expands `\@thefnmark' 
%% to produce the footnote mark at the page bottom, 
%% right after it has been determined for the mark 
%% in the main text. \emph{Here} the footnote text 
%% will be typeset only when \emph{other} footnote marks
%% may have been formed for typesetting the main text 
%% paragraph before. 
%% %%% (TODO clearer wording)
%% In the \strong{footnote list} 
%% macro |\FNLN@list|, the (\dqtd{`&\protect'ed})
%% \emph{current} expansion <mark> of `\@thefnmark' 
%% is stored as an item preceding the footnote text 
%% <text>. One footnote entry in `\FNLN@list' 
%% thus has the form \lq`<mark>\@lt<text>\@lt'\rq.
%% \LaTeX's internal `\g@addto@macro' is used to \emph{append} 
%% an entry to the list (at the right). The OTR will later 
%% take the entries from the left of the list. 
%% 
%% The argument of the auxiliary/temporary `\@tempa' 
%% will contain the footnote text and thus must be able to 
%% carry `\par' tokens. We therefore need a `\long' version of 
%% `\protected@edef':
   \let\@@protect\protect
   \let\protect\@unexpandable@protect
   \afterassignment\restore@protect
   \long \edef \@tempa ##1{%
        \noexpand\g@addto@macro \noexpand\FNLN@list {%
            \@thefnmark \noexpand\@lt ##1\noexpand \@lt}}%
%% ... issuing 
%%     \lq`\g@addto@macro\FNLN@list{<mark>\elt<text>\@lt}'\rq\ ...
   \@tempa                              %% reads arg
}
%% Here we initialize |\FNLN@list|:
\let\FNLN@list\@empty
%%
%% === Output Routines ===
%% ==== 'lineno''s Output Routine ====
%% The following is a copy of 'lineno''s OTR
%% that we are varying. It may be used for a check 
%% if the OTR that 'fnlineno.sty' encounters 
%% is the one expected (TODO).
\if@FNLN@check@
  \def\FNLN@lno@output {%
    \LineNoTest
    \if@tempswa
      \ifnum\outputpenalty=-\@Mllbcodepen 
        \WriteLineNo 
      \else 
        \ifnum\outputpenalty=-\@Mppvacodepen 
          \PassVadjustList 
        \else 
          \LineNoLaTeXOutput 
        \fi 
      \fi 
    \else  
      \MakeLineNo
    \fi
  }
%% The ``signal penalties" used here are
  \mathchardef\FNLN@M@llbl@codepen=11111 
  \mathchardef\FNLN@M@ppva@codepen=11112 
\fi
%% Their names should mean ```\linelabel' code penalty" and 
%% ```\PostponeVadjust' code penalty."
%%
%% |\TheLineNoLaTeXOutput|:
%% It turns out to be inconvenient here 
%% that 'lineno' sacrifices access 
%% to the \emph{primitive} `\output' (```\@tempa'"; 
%%  TODO: auxiliary package before loading 'lineno'!?; 
%%  later change 'lineno.sty' indeed). 
%% So to change the OTR we use `\LineNoLaTeXOutput' as a hook 
%% for adding additional cases of `\outputpenalt'ies. 
%% We take a copy of `\LineNoLaTeXOutput' here. 
\let\TheLineNoLaTeXOutput\LineNoLaTeXOutput
%%
%% ==== Tools for Temporary Parameter Changes ====
%% |\GStoreReg{<register>}|\quad 
%% (or\quad |\GStoreReg<register>|\\[\smallskipamount]
%%  when <register> is a single token---\lq`\count0'\rq\
%%  being a counterexample ...)\quad 
%% stores the current content of <register> 
%% (\emph{globally}) as an internal macro so that it can be 
%% restored later by\\[\smallskipamount] 
%% |\RestoreReg{<register>}|\quad 
%%   \mbox{(or\quad }|\RestoreReg<register>|)\\[\smallskipamount]
%% or \emph{globally} by\\[\smallskipamount] 
%% |\GRestoreReg{<register>}|\quad (|\GRestoreReg<register>|)
%% \\[\smallskipamount]
%% (The OTR runs in a local group!---Recall that assignments 
%%  to ``special dimens"---\TeX book p.\,271---are 
%%  automatically global.) 
%% <register> is something that can be prefixed by `\the'
%% to read its content and to which you can assign a value 
%% <value> by \lq<register><value>\rq.
%% (TODO: could also be some `\catcode'!)
%% %% and TODO allow changing name space!
\newcommand*{\GStoreReg}[1]{%
    \expandafter \xdef \csname GS\string#1\endcsname {\the #1}}
\newcommand*{\RestoreReg}[1]{#1\csname GS\string#1\endcsname \relax}
\newcommand*{\GRestoreReg}{\global\RestoreReg}
%% |\GStoreSetReg{<register>}{<value>}| assigns %%% the new value 
%% <value> to <register> (locally) after executing `\GStoreSet', 
%% |\GStoreGSetReg| does the same \emph{globally} 
%% (and still argument braces aren't needed when a single 
%%  token refers to the register). 
\newcommand*{\g@storesetreg}[3]{\GStoreReg{#2}#1#2#3\relax}
\newcommand*{\GStoreSetReg} {\g@storesetreg\relax}
\newcommand*{\GStoreGSetReg}{\g@storesetreg\global}
%% (These preliminaries might go into an own new package, TODO! 
%%  + loop on list of <register>s ...) %% 2010/12/19
%%
%% ==== The basic hook ====
%% \label{sec:sw}
%% We use two more penalties triggering the ``MVL swaps:"
\mathchardef\FNLN@M@swap@codepen  =11113
\mathchardef\FNLN@M@insert@codepen=11114
%% v0.41 deals with |\pagebreak| in footnote texts, 
%% using a flag |\if@FNLN@sw@| that must be set globally. 
%% %% 2011/01/07:
%% It turned out not to work properly; however, the new 
%% switch has served a different purpose for ``continuous 
%% line numbering," cf. section~\ref{sec:cont}.
\newif\if@FNLN@sw@  \global\@FNLN@sw@false  %% v0.41
%% When a `\pagebreak' triggers the OTR while typesetting 
%% the footnote text, the page content is collected 
%% in a box |\FNLN@holdft|:
\newsavebox\FNLN@holdft                     %% v0.41
%% Using |\LineNoLaTeXOutput| for hooking into the OTR:
\renewcommand*{\LineNoLaTeXOutput}{%
  \ifnum\outputpenalty=-\FNLN@M@swap@codepen
    \SwapFootnoteMain
  \else
    \ifnum\outputpenalty=-\FNLN@M@insert@codepen
      \InsertFootnote
    \else
      \if@FNLN@sw@                          %% v0.41
%        \showthe\outputpenalty     %% 2010/12/20
        \global\setbox \FNLN@holdft \vbox{%
          \unvbox\FNLN@holdft 
%% %% moved here, clarified 2010/12/21:
%% TODO from v0.41: `\pagebreak[4]' does not seem to force (reliably) 
%% splitting a footnote; if the footnote is not split here, 
%% at present the `\baselineskip' is lost,
%% see the footnote paragraph starting 
%% with \qtd{C} in `edfndemo.pdf' as of 2010/12/21.
%% We would need some measuring ... `\pagebreak'
%% might be redefined ... resembling \LaTeX's `\@specialoutput'!
          \unvbox\@cclv 
%% TODO same problem here, see the footnote paragraph starting 
%% with \qtd{D} in `edfndemo.pdf' as of 2010/12/21.
          \penalty\outputpenalty}%
        %% TODO reset page book-keeping!?   %% v0.41
      \else
        \TheLineNoLaTeXOutput       %% "the real \LineNoLaTeXOuput"
      \fi
    \fi
  \fi
}
%% \strong{An idea:}\quad Instead of so many `\ifnum', use 
%% \[`\csname<chars>\the\outputpenalty\endcsname'\]
%% ... in 'lineno.sty', when you really have a broad 
%% range of `\outputpenalties' useful to be 
%% described by `\ifnum' range checks ...
%%
%% ==== Typesetting the Footnote Text ====
%% |\SwapFootnoteMain| is the slot of the OTR that our 
%% modified `\@footnotetext' calls with 
%% $`\outputpenalty'=-`\FNLN@M@swap@codepen'$.
%% The ``column so far" is stored in a new box register 
%% |\FLNL@holdcol|.
\newsavebox\FNLN@holdcol
\newcommand*{\SwapFootnoteMain}{%
    \global \setbox\FNLN@holdcol \vbox{\unvbox\@cclv}%
%% (... cf. `\@holdpg' in \LaTeX.)
%%
%% The entire text of a footnote is typeset on top of 
%% the MVL. `\vsize' is maximized temporarily to avoid 
%% that the footnote text is broken across pages. 
    \GStoreGSetReg\vsize\maxdimen
%% However, the user may want to use `\pagebreak'
%% in a footnote in order to control manually where 
%% a ``long" footnote is split. v0.41 tries to support 
%% this:
    \global\@FNLN@sw@true                   %% v0.41
%% ... cf. Section~\ref{sec:sw}.
%%
%% There shouldn't be any `\topskip', the space 
%% on top of a footnote is controlled by `\footnotesep'
%% entirely:
    \GStoreGSetReg\topskip\z@skip
%% (`\nointerlineskip' as well as setting `\topskip' 
%%  locally instead fails ... according to `\showlists' ...)
%%
%% Resetting |\pagegoal| 
%% (why doesn't it switch to $`\vsize'=`\maxdimen'$ automatically?), 
%% |\pagetotal|, and the other ``special dimens" 
%% (\TeX book p.\,271; rather experimental ... 
%%  I think it is important to restore them later ...)
    \GStoreSetReg\pagegoal \vsize
    \GStoreSetReg\pagetotal\z@
    \GStoreSetReg\pagestretch\z@
    \GStoreSetReg\pagefilstretch\z@
    \GStoreSetReg\pagefillstretch\z@
    \GStoreSetReg\pagefilllstretch\z@
    \GStoreSetReg\pageshrink\z@
    \GStoreSetReg\pagedepth\z@
%% We must choose certain settings from `\@footnotetext'
%% such as font:
    \reset@font\footnotesize
    \interlinepenalty\interfootnotelinepenalty
%% \LaTeX's `split' things here are relevant at 
%% `\insert\footins' only: (TODO!?)
%     \splittopskip\footnotesep
%     \splitmaxdepth \dp\strutbox \floatingpenalty \@MM
    \hsize\columnwidth \@parboxrestore
%% The previous lines were from \LaTeX's `\@footnotetext'.
%% Now we need to restore the `\@thefnmark' that belongs 
%% to the current footnote text. We use a macro that 
%% tears two items from |\FNLN@list| and executes the 
%% rest of \LaTeX's `\@footnotetext':
    \expandafter \FNLN@typeset \FNLN@list \@@
%     \showthe\vsize
%% ... so a `\vsize' assignment without `\global' 
%% is noted here, and an analogous `\topskip' assignment 
%% is not!? TODO ...
}
%% |\FNLN@typeset| first removes something %% reworded 2010/12/26
%% from the list of footnotes, similarly to \LaTeX's |\@xnext|
%% and 'lineno''s `\@LN@xnext', then executes a 
%% remaining portion of \LaTeX's `\@footnotetext'. 
%% The footnote text may contain `\par' tokens, 
%% so the definition must be `\long':
\long\def \FNLN@typeset #1\@lt #2\@lt #3\@@{%
    \gdef\FNLN@list{#3}%
    \def\@thefnmark{#1}%
%% This was our own, and next \LaTeX\ continues:
    \protected@edef\@currentlabel{%
       \csname p@footnote\endcsname\@thefnmark
    }%
    \color@begingroup
%% We insert starting the 'lineno' settings ...
      \linenumbers
      \setfootnotelinenumbers                   %% 2010/12/25
%% ... \LaTeX\ again (v0.41 exports dealing with 
%%  closing `\par' to `finstrut.sty'): 
      \@makefntext{%
        \rule\z@\footnotesep\ignorespaces
%% We replace `#1' by `#2\par' (`\linenumberpar'), 
%% so we really need `finstrut.sty':
        #2\par
        \@finalstrut\strutbox}%
    \color@endgroup
%% Now we trigger the ``swap back slot" of the OTR:
    \penalty-\FNLN@M@insert@codepen
}
\RequirePackage{finstrut}
%% 
%% ==== &\insert\ the Footnote Text ====
%% |\InsertFootnote| is the slot of the OTR that executes
%% `\insert\footins' with the numbered footnote text.
%% The ``column so far" stored in `\FNLN@holdcol' 
%% is put onto the top of the MVL, and then parts of 
%% \LaTeX's `\@footnotetext' are performed that haven't
%% been done earlier, applied to the footnote text 
%% that the OTR should have found in `\box255'. 
%% Before however, the previous `\topskip', `\vsize', and the 
%% `\page'... book-keeping parameters are restored:
\newcommand*{\InsertFootnote}{%
    \GRestoreReg\topskip    \GRestoreReg\vsize
%% (... \emph{global} restoring of `\vsize' proved vital with 
%%      'edfndemo' 2010/12/17 ...)
    \RestoreReg \pagegoal     \RestoreReg\pagetotal
    \RestoreReg \pagestretch
    \RestoreReg \pagefilstretch 
    \RestoreReg \pagefillstretch 
    \RestoreReg \pagefilllstretch
    \RestoreReg \pageshrink   \RestoreReg\pagedepth
    \unvbox\FNLN@holdcol
    \insert\footins{%
        \splittopskip\footnotesep
        \splitmaxdepth \dp\strutbox \floatingpenalty \@MM
%% Support of `\pagebreak' with v0.41:
        \unvbox\FNLN@holdft                 %% v0.41
        \unvbox\@cclv}%
    \global\@FNLN@sw@false                  %% v0.41
%% With v0.5, global settings for ``pagewise" numbering
%% must be restored:                        %% 2010/12/26
    \unsetfootnotelinenumbers
}
%%
%% === ``Continuous'' Numbering ===
%% \label{sec:cont}
%% ==== Goal ====
%% With v0.5, for the first time we try to get a 
%% ``pagewise" numbering such that, if a main text line
%% has a footnote, 
%% (\textit{i})~its printed number is just the natural 
%% successor of the printed number of the previous main 
%% text line 
%% (instead of continuing previous numbering with the 
%%  lines of the footnote first), 
%% and (\textit{ii})~the printed numbers of footnote lines 
%% just continue the printed numbers of the main text lines.
%% This ``obvious'' desirement is not easy to achieve; 
%% already pagewise numbering of main text lines, 
%% without numbering footnote lines, has been somewhat 
%% ingenious.
%%
%% ==== How to Number Lines Pagewise ====
%% The basic idea of 'lineno''s pagewise numbering is:
%% \begin{enumerate}
%%   \item Each numbered line of the document is identified 
%%         by a unique counter value, an ``absolute" number.
%%   \item For each page (and column), the range of absolute 
%%         line numbers occurring on them is recorded 
%%         (or actually: the first and the last number). 
%%   \item The ``public," ``human-readable" (``pagewise") 
%%         format of a given absolute line number $l$ is generated by 
%%         (\textit{i})~finding the page (and column) with 
%%         first number $n$ and last number $k$ such that 
%%         $n\leq l\leq k$, 
%%         (\textit{ii}) ``printing" $l-n+1$ in ``columnwise"
%%         mode, otherwise $l-m+1$ where $m$ is the first 
%%         absolute line number in the left-hand column of the 
%%         same page.
%% \end{enumerate}
%% \strong{Generating} the ``pagewise" representation for a 
%% given absolute line number $l$ thus may be summarized as 
%% \emph{finding the corresponding \emph{\strong{offset}} value} 
%% to be subtracted ($n$, $n+1$, $m$, or $m+1$ ...).
%%
%% When \emph{footnote} lines are to be numbered as well, 
%% a little problem is the order in which main text and 
%% footnote lines increment the absolute counter. 
%% 'lineno''s mechanism for this is started immediately 
%% after a paragraph has been broken into lines. 
%% Each line of the paragraph then calls a macro 
%% generating the line number. 'fnlineno' now interrupts 
%% numbering of main text lines at a line issuing a footnote. 
%% The footnote text is typeset, including numbering its 
%% lines at each end of a footnote paragraph. When the 
%% footnote text has been sent into the `\insert' register, 
%% numbering of main text lines is resumed. 
%% 
%% Up to v0.4 (a development version), we used the \emph{same} 
%% absolute counter for main text and footnote lines. 
%% When a page $p$ has more than one main text line and the first 
%% one has a long footnote continued on the next page~$p+1$, 
%% there is no ``range" of absolute line numbers 
%% characterizing page~$p$ any more, 
%% because the greatest absolute line number of page~$p$ 
%% exceeds the absolute line numbers of the footnote continued 
%% on page~$p+1$.
%%
%% 'lineno''s procedure can be revived by numbering 
%% main text lines and footnote lines independently 
%% from each other. We use \emph{two} absolute counters, 
%% one is incremented with main text lines only, 
%% the other with footnote lines only. 
%% Numbering of main text lines just will not be affected 
%% by numbering of the footnote lines. 
%%
%% Almost the same will hold for footnote lines. 
%% Each page (and column) will have a characteristic 
%% ``range" of absolute footnote line numbers $\{n,\dots,k\}$. 
%% The only notable difference will be that for footnote line~$l$ 
%% we print $(l-n+1)+(K-N+1) = (K+l)-(N+n)+2$ instead of 
%% $l-n+1$---where $\{N,\dots,K\}$ is the range of \emph{main text}
%% line numbers of the page (and column). 
%% 
%% The previous discussion of \strong{generating} the printed 
%% line number from its absolute version 
%% has assumed that corresponding \strong{offset} values 
%% have been given somehow, or that the ``line number ranges" 
%% for pages are known from somewhere. 
%% In fact, these ranges are \strong{computed} at the 
%% \strong{start} of a \LaTeX\ run \emph{before} typesetting, 
%% when reading the `.aux' file for the first 
%% time. They are used in the entire document. 
%% While typesetting, each numbered line of main text leaves 
%% a record of its absolute number and page number in the new 
%% version of the `.aux' file that the run creates, 
%% a two-parameter macro `\@LN'. With 'fnlineno.sty', 
%% there will be new `\@FLN' entries of the same type. 
%% These `.aux' entries are used for building 
%% the page range data for the next run. 
%% When the document source has been changed, at least 
%% \emph{two runs} will usually be required 
%% to get correct line numbers in page margins, 
%% and \emph{another} run will be needed so references 
%% to line numbers by `\ref' and `\linelabel' are correct. 
%% 
%% ==== Summary of Changes ====
%% Variants of 'lineno.sty''s code for ``pagewise" 
%% numbering are following. Sometimes we generalize
%% `pagewise' stuff from 'lineno' and re-implement pagewise 
%% numbering of main text lines as a special case, 
%% the other special case being numbering of footnote lines.
%% 
%% Five things need modifications:
%% \begin{description}
%%   \item[Building page info macros:] 
%%      Processing `\@LN' and `\@FLN' `.aux' entries will 
%%      use shared building macros, the difference is 
%%      obtained by switching \strong{name spaces}. 
%%      (It may be notable that a page may get one 
%%       info macro for main text and another for 
%%       footnote text, if it contains footnote text.)
%%   \item[Logging:]
%%      While typesetting, the shared logging macro 
%%      is switched to write either `\@LN' or `\@FLN' 
%%      to the `.aux' file. Also, `\c@linenumber' 
%%      may refer to either the main text or to the 
%%      footnote text counter. 
%%   \item[Generating ``pagewise" format:]
%%      The choice of `\c@linenumber' also determines 
%%      which counter is incremented, and again 
%%      name spaces for page info macros are switched. 
%%      For footnote lines, 
%%      a tail macro for adding the number of main text lines 
%%      will be activated.
%%   \item[Referencing:]
%%      The `.aux' file may have entries from `\linelabel'
%%      containing large numbers from an ``absolute" counter.
%%      In generating the ``human-readable" number, 
%%      it must be known whether it is a main text or a 
%%      footnote line number. 
%%      %% 2010/12/28:
%%      An additional complication is referring to a 
%%      main text line from a footnote and vice versa---thinking 
%%      of global changes in generating the number.
%%      Or even think of the case referring from unnumbered text 
%%      to numbered text!
%%      (I have wondered before if the entry couldn't be 
%%       the ready human-readable number, TODO!)
%%   %% 2010/12/27:
%%   \item[Lists of ``vertical tasks":]
%%      'lineno.sty' (v4) has introduced two lists of 
%%      tasks that were issued in horizontal mode
%%      but only can be completed after breaking a paragraph 
%%      into lines: one for `\linelabel's and one for `\vadjust' 
%%      items that must wait until the line number has been 
%%      attached. It is essential that the tasks are processed 
%%      in the same order in vertical mode as they were 
%%      issued in horizontal mode. As we are now interrupting 
%%      processing of main text paragraphs for processing 
%%      footnotes, tasks for footnote text must be lined up 
%%      in separate lists than tasks for main text. 
%%      This is indeed essential for the previous issue 
%%      of getting `\linelabel' work in footnotes as well 
%%      as in main text.
%% \end{description}
%% 
%% ==== Info Building ====
%% % These macros 
%% |\@LN|, |\@FLN|, and |\@FNLN|
%% are processed at reading the `.aux' file before typesetting only. 
%% The \strong{interface} to \strong{generating} 
%% ``pagewise" and footnote line numbers just are |\LN@Pfirst|
%% and |\FLN@Pfirst|, eventually pointing to the first page/column 
%% with numbered main text lines or footnote lines, resp.
\def \FLN@Pfirst {\nextLN\relax}
%% This initialization of `\FLN@Pfirst' is just the same 
%% as the one of `\LN@Pfirst' in 'lineno.sty'; 
%% their expansions are changed as soon 
%% as such a page is found, replacing the `\relax' by the
%% corresponding page info macro. 
%% 
%% `\LN@Pfirst' and `\FLN@Pfirst' are passed to 
%% |\testFirstNumberedPage| via the hook |\FNLN@first@numbered|
%% that by default is the same as `\LN@first':
\def \FNLN@first@numbered {\LN@Pfirst}
%% (oh, it must be `\def' here to recognize the change ...).
%% This must be changed by `\setfootnotelinenumbers' 
%% (`\let' then, as when called the change will have happened). 
%%
%% Moreover, they are passed to |\NumberedPageCache| 
%% (the page info macro where a search starts, 
%%  ``current" page/column) as its initialization; 
%% the ``generating" macros then change the 
%% latter macro following `\nextLN' in the page info macros.
%%
%% In this sense, no other ``name space switching" is needed 
%% for communication with other functions.
%% 
%% 'lineno.sty' has changed |\LastNumberedPage| globally ...
%% the last page with numbered \emph{footnote} lines 
%% may well be another one than the last page with 
%% numbered \emph{main} text lines ... 
%% But fortunately, also `\LastNumberedPage'
%% is needed in reading the `.aux' before typesetting only
%% (`\@onlypreamble' is \LaTeX's disabling command):
\@onlypreamble\LastNumberedPage
%% In 'lineno.sty', we have `\def\LastNumberedPage{first}'. 
%% We need the same for the footnote variant |\FNLN@last@numbered| 
%% (to be handled globally!): 
\global   \let \FNLN@last@numbered \LastNumberedPage
\@onlypreamble \FNLN@last@numbered
%% |\@FNLN{<names>}<last-numbered>{<line>}{<page>}|
%% \\[\smallskipamount]
%% generalizes 'lineno.sty''s `\@LN{<line>}{<page>}' 
%% to re-implement it. 
%% There is an additional parameter argument <names> 
%% for choosing name spaces
%% and a parameter <last-numbered> for choosing the macro 
%% storing the ``last numbered page."
%% (An argument without braces expects a macro name.)
\newcommand* \@FNLN [4]{{%
    \expandafter\@@LN
        \csname #1#4C\@LN@column \expandafter\endcsname
        \csname #1O#4\endcsname
        {#3}{#4}{#1}{#2}}}
\@onlypreamble\@FNLN
%% As in 'lineno.sty' `\@LN' calls `\@@LN', a new variant 
%% of `\@@LN' is called by `\@FLN' here, but it gets one
%% additional parameter for passing <names> 
%% and another for passing <last-numbered> from `\@FLN'. 
%% So the new syntax is\\[\smallskipamount] 
%% |\@@LN<info><first-page-line>{<line>}{<page>}{<names>}<last-numbered>|:
\renewcommand* \@@LN [6]{% 
  \ifx#1\relax
    \ifx#2\relax\gdef#2{#3}\fi
    \expandafter\@@@LN\csname #5#6\endcsname#1% 
    \xdef#1{\lastLN{#3}\firstLN{#3}% 
            \pageLN{#4}{\@LN@column}{#2}\nextLN\relax}%
  \else
    \def\lastLN##1{\noexpand\lastLN{#3}}%
    \xdef#1{#1}%
  \fi
  \xdef#6{#4C\@LN@column}}
\@onlypreamble\@@LN
%% 'lineno.sty''s |\@@@LN| does not need any adjustment.
%% 
%% 'lineno.sty''s |\@LN{<line>}{<page>}| is reimplemented as 
\def \@LN {\@FNLN{LN@P}\LastNumberedPage} 
%% ---so `\@LN' really does the same as before, including name spaces. 
%% 
%% |\@FLN{<line>}{<page>}| is the other special case of the 
%% new `\@FNLN'---an `F' precedes the earlier names, and 
%% |\FNLN@last@numbered| is the storing macro initialized above:
\def \@FLN {\@FNLN{FLN@P}\FNLN@last@numbered}
%% For logging, we make both unexpandable: %% 2010/12/27 
% \AtBeginDocument{\let\@LN\relax \let\@FLN\relax}
%% ... but this way nothing appears in the file!? TODO ...
\@onlypreamble\@LN  \@onlypreamble\@FLN
%% For reading the `.aux' finally, we do what 
%% 'lineno' does with `\@LN':
\AtEndDocument{\let\@FLN\@gobbletwo}
%%
%% ==== Tool for Reusing Global Operations with Macros ==== 
%% %% 2010/12/28
%% 'lineno.sty' v4 provides list handling (changing lists globally) 
%% and global changes of `\NumberedPageCache'. 
%% We want to use them in ``main text" mode 
%% as well as in ``footnote" mode. 
%% To use such an operation on <ln-macro> for <fln-macro>, 
%% we `\global\let<ln-macro><fln-macro>', apply the operations, 
%% and finally `\global\let<fln-macro><ln-macro>'. 
%% However, we are not only interested in how <fln-macro> 
%% is changed this way, rather <ln-macro> also is used as input 
%% for some operations, and we can choose which <fln-macro> 
%% should be used as input. To switch from working on/with 
%% <fln-1> to <fln-2>  using <ln-macro> with an option 
%% to use <fln-1> later again, a tool 
%% |\GStoreUse<ln-macro><fln-1><fln-2>| is provided 
%% (should render later switchings much better readable):
\newcommand* \GStoreUse [3]{\global\let#2#1\global\let#1#3}
%% I.e., current content of #1 is stored in #2, then 
%% #1 attains the content of #3.
%% 
%% ==== General Settings for Typesetting Stage ====
%% % %% 2011/01/01: %% 2011/01/02: wrong!
%% % With 'ednotes', it turns out that the heading is not quite 
%% % correct, 'ednotes'' special `\newlabel' mechanism evaluates 
%% % `\getpagewiselinelabel' already for building its note 
%% % info macros at the first reading of the `.aux' file. 
%%
%% %% 2010/12/27:
%% Oh my dear, it seems that all the switching 
%% for the footnote variant of `pagewise' must be global 
%% (I can't find something useful using `\aftergroup' quickly).
%% Therefore, I render 'lineno''s |\setpagewisenumbers|
%% acting globally:
\renewcommand*\setpagewiselinenumbers{%
   \global\let \theLineNumber  \thePagewiseLineNumber
   \global\let \c@linenumber   \c@pagewiselinenumber
   \global\let \makeLineNumber \makePagewiseLineNumber
}
%% I just force this, hehe ...
\setpagewiselinenumbers
%% As a counterpart to `\c@pagelinenumber', 
%% |\c@footnotelinenumber| is reserved for the 
%% absolute footnote line numbers:
\newcount\c@footnotelinenumber
%% %% 2011/01/01 severe-bug fix, why didn't show up?:
%% |\FNLN@@cache| stores `\NumberedPageCache' 
%% as from ``main" mode:
\let \FNLN@@cache \NumberedPageCache
%% |\FNLN@cache| stores `\NumberedPageCache' 
%% as from ``footnote" mode; its initial content is the 
%% counterpart or analogue to `\LN@Pfirst':
\def \FNLN@cache {\FLN@Pfirst}
%% %% 2010/12/28:
%% |\FNLN@foot@cache| and |\FNLN@main@cache|
%% switch |\NumberedPageCache|:
\def \FNLN@foot@cache {%
    \GStoreUse \NumberedPageCache \FNLN@@cache \FNLN@cache}
\def \FNLN@main@cache {%
    \GStoreUse \NumberedPageCache \FNLN@cache \FNLN@@cache}
%% |\FNLN@labels| will be the counterpart to
%% 'lineno.sty''s `\@LN@labellist':
\global\let \FNLN@labels \@empty
%% |\FNLN@vadjusts| will be the counterpart to
%% 'lineno''s `\@LN@vadjustlist':
\global\let \FNLN@vadjusts \@empty
%% %% separation for referencing 2010/12/28:
%% Settings for footnote line numbers first resemble
%% `\setpagewiselinenumbers'; but more changes 
%% are needed, and results from main text numbering 
%% must be stored. 
%% Some of the settings are needed \emph{locally} for 
%% generating numbers for labels, collected in 
%% |\setgetfootnotelinenumbers|; for this purpose
%% nothing must be stored explicitly:
\newcommand* \setgetfootnotelinenumbers {% 
%% Change of `\theLineNumber' is omitted as we are \emph{reading},
%% not writing a label. 
    \let\c@linenumber\c@footnotelinenumber
%     \let\makeLineNumber\makeFootnoteLineNumber
%% But in fact, |\makeFootnoteLineNumber| and
%% |\makePagewiseLineNumber| will be the same. 
%% The difference is made by the choice of 
%% |\FNLN@first@numbered| and |\NumberedPageCache| 
%% for the line range searches. 
    \let \FNLN@first@numbered \FLN@Pfirst
    \let \FNLN@finish \FNLN@add
}
%% |\setfootnotelinenumbers| performs all the settings 
%% for typesetting footnotes in line numbering mode 
%% \emph{globally}, 
%% including storing results from typesetting main text:
\newcommand* \setfootnotelinenumbers {% 
  \globaldefs\@ne
%% The previous line also renders `\setgetfootnotelinenumbers' global:
    \setgetfootnotelinenumbers
%% |\theLineNumber| is used for `\linelabel' entries. 
%% `\thePagewiseLineNumber' is replaced by
%% |\theFootnoteLineNumber|:
    \let\theLineNumber\theFootnoteLineNumber
%% Logging to `.aux':
    \def \FNLN@log {\string\@FLN}% 
%% %% 2010/12/28:
%% Starting range search: |\NumberedPageCache|
    \FNLN@foot@cache
%% Reusing 'lineno''s task list operations:
    \GStoreUse \@LN@labellist \FNLN@@labels \FNLN@labels
    \GStoreUse \@LN@vadjustlist \FNLN@@vadjusts \FNLN@vadjusts
  \globaldefs\z@
}
%% For switching back to ``main text mode," 
%% again some settings may need a local variant---for 
%% processing line references from footnotes to main text!
%% This is the purpose of |\setgetpagewiselinenumbers|:
\newcommand* \setgetpagewiselinenumbers {%
    \let \FNLN@first@numbered \LN@Pfirst
    \let \FNLN@finish         \@gobbletwo
}
%% |\unsetfootnotelinenumbers| stores the ``current" 
%% page with footnote lines and loads the ``most recent" 
%% page with main text lines---and more ...:
\newcommand* \unsetfootnotelinenumbers {%
    \gdef \FNLN@log {\string\@LN}%
    \FNLN@main@cache
%% Task lists: %% 2010/12/28
    \GStoreUse \@LN@labellist \FNLN@labels \FNLN@@labels
    \GStoreUse \@LN@vadjustlist \FNLN@vadjusts \FNLN@@vadjusts
    \globaldefs\@ne \setgetpagewiselinenumbers \globaldefs\z@ %% v0.53
    \setpagewiselinenumbers
}
%% |\makeFootnoteLineNumber| actually only copies 
%% |\makePagewiseLineNumber|, different results are obtained 
%% be changing hooks. The command first calls 
%% logging---|\logtheLineNumber|, 
%% then generating the ``public" line number---|\getLineNumber| 
%% (which in turn only is a copy of |\testNumberedPage| 
%%  in 'lineno.sty').
\@ifdefinable\makeFootnoteLineNumber
    {\let \makeFootnoteLineNumber \makePagewiseLineNumber}
%%
%% ==== Logging ====
%% |\logtheLineNumber| is redefined to log both 
%% main text and footnote line numbers. 
\def \logtheLineNumber {%
    \protected@write\@auxout{}{%
        \FNLN@log{\the\c@linenumber}{\noexpand\the\c@LN@truepage}}} 
%% |\FNLN@log| is the hook for the difference, 
%% its default expansion |\@LN| is made for 
%% \emph{main text} line numbers:
\gdef \FNLN@log {\string\@LN}
%%
%% ==== ``Public" Line Numbers ====
%% Fortunately, these commands don't need to know much about 
%% name spaces. The interfaces to them are 
%% |\NumberedPageCache|---changing globally---and 
%% |\FNLN@first@numbered|. Our 
%% |\FNLN@cache| is initialized 
%% by analogy to its counterpart `\NumberedPageCache'
%% (a minute name space change):
\def \FNLN@cache {\FLN@Pfirst}
%% |\testFirstNumberedPage{<integer>}| from 'lineno.sty'
%% is modified by replacing `\LN@Pfirst' only:
\renewcommand* \testFirstNumberedPage [1]{%
  \ifnum#1>\c@linenumber
     \def\nextLN##1{%
        \testNextNumberedPage\FNLN@first@numbered}%
  \else
     \let\nextLN\@gobble
     \def\pageLN{\gotNumberedPage{#1}}%
  \fi}
%% |\testNumberedPage| and |\testNextNumberedPage| from 'lineno'
%% don't need any modification. |\testLastNumberedPage|
%% is modified in 'edfnotes.sty'.   %% 2011/01/07
%%
%% |\gotNumberedPage| just needs a closing hook 
%% `\FNLN@finish' to allow for footnote lines.
\renewcommand* \gotNumberedPage [4]{%
  \oddNumberedPagefalse
  \ifodd \if@twocolumn #3\else #2\fi\relax\oddNumberedPagetrue\fi
  \advance\c@linenumber\@ne 
  \ifcolumnwiselinenumbers
     \subtractlinenumberoffset{#1}%
  \else
     \subtractlinenumberoffset{#4}%
  \fi
%  \show\FNLN@finish
  \FNLN@finish{#2}{#3}%
}
%% |\FNLN@finish{<page>}{<column>}| gobbles both 
%% arguments with \emph{main} text lines, 
%% but will add the number of main text lines to 
%% \emph{footnote} line numbers:
\global\let \FNLN@finish \@gobbletwo
%% Then it will act as |\FNLN@add|. We run the page info macro 
%% for the same page (column; if defined).
\newcommand* \FNLN@add [2]{%
  \expandafter \let\expandafter \@tempa\csname LN@P#1C#2\endcsname
  \ifx\@tempa\relax
  \else
    \advance\c@linenumber\@ne
    \ifcolumnwiselinenumbers
      \let\firstLN\subtractlinenumberoffset
%% ... rather assuming `\realpagewiselinenumbers'.
      \let\pageLN\@gobblethree
    \else
      \let\firstLN\@gobble
      \def\pageLN##1##2##3{\subtractlinenumberoffset{##3}}%
    \fi
    \def\lastLN##1{\subtractlinenumberoffset{-##1}}%
    \let\nextLN\@gobble
%% ... TODO all needed?
    \@tempa
  \fi
}
%%
%% ==== Referencing ====
%% Now that we are using two separate counters for main text lines 
%% and footnote lines (v0.5), correct references to footnote lines 
%% using |\linelabel| and `\ref' need further adjustments.
%% %
%% 'lineno.sty''s `\thePagewiseLineNumber'
%% and `\getpagewiselinenumber{<integer>}'
%% are generalized and re-implemented
%% % by |\theWiseLineNumber| and |\getwiselinenumber|
%% by macros that then serve to implement 
%% referring to footnote line numbers. 
%%
%% |\theWiseLineNumber{<trans>}| leaves a `\protect'ed call 
%% to a one-parameter macro <trans> in the `.aux' file:
\newcommand* \theWiseLineNumber [1]{\protect #1{\the\c@linenumber}} 
%% |\getwiselinenumber{<choice>}{<integer>}| executes <choice> 
%% before applying `\testNumberedPage' to <integer>---within 
%% a local group:
\newcommand* \getwiselinenumber [2]{{%
%% Some wisdom is needed to take account of the current 
%% ``numbering state" from which `\ref' was called.
%% \begin{description}
%%   \item[Referring to main text line:]\leavevmode
%%     \begin{itemize}
%%       \item Unless called from numbered footnote, 
%%             no extra care is needed.
%%       \item If called from numbered footnote, 
%%             `\setgetpagewiselinenumbers'
%%             and temporary switching of `\NumberedPageCache'
%%             is needed.
%%     \end{itemize}
%%   \item[Referring to footnote line:]\leavevmode
%%     \begin{itemize}
%%       \item If called from numbered footnote, 
%%             no extra care is needed. 
%%       \item Otherwise, `\setgetfootnotelinenumbers'
%%             and temporary switching of `\NumberedPageCache'
%%             is needed.
%%     \end{itemize}
%% \end{description}
  \ifx#1\relax          %% to main text
    \if@FNLN@sw@        %% from footnote
      \setgetpagewiselinenumbers
      \FNLN@main@cache
      \let \FNLN@restore@cache \FNLN@foot@cache
    \fi
  \else                 %% to footnote
    \if@FNLN@sw@ \else  %% from elsewhere
      #1%
      \FNLN@foot@cache
      \let \FNLN@restore@cache \FNLN@main@cache
    \fi
  \fi
  \c@linenumber #2\relax\testNumberedPage
  \thelinenumber
  \FNLN@restore@cache
}}
\let \FNLN@restore@cache \relax
%% |\getpagewiselinenumber| doesn't need any <choice>---we 
%% assume that the label was written in the default `pagewise' 
%% mode (but it is difficult, though, `\relax' is essential!):
% \renewcommand* \getpagewiselinenumber {\getwiselinenumber\relax} %!! 
%% 2010/12/31, a compatibility problem with 'ednotes'' `\newlabel' 
%% mechanism shows up. 'ednotes' ``undefines" `\getpagewiselinenumber' 
%% and restores it only `\AtBeginDocument'. We must ensure that 
%% 'ednotes' will not override our new version of 
%% `\getpagewiselinenumber'. 
%% (TODO in my view another motivation to write ``ready" numbers 
%%  without `\getpagewiselinenumbers' directly.)
%%
%% We might assume that 'ednotes' (if at all) is loaded directly 
%% and loads 'lineno.sty' 
%% (that is the usual and recommended way of using 'ednotes') 
%% and that this will happen before 'fnlineno.sty' is loaded. 
%% But now that we have spent some time understanding the 
%% situation, we can deal with the case as well that 'lineno.sty' 
%% is loaded first, then 'fnlineno.sty' is loaded, and then 'ednotes'. 
%% (I have assumed earlier that 'fnlineno.sty' is loaded after 
%%  'lineno.sty' ...)
\AtBeginDocument{%
    \def \getpagewiselinenumber {\getwiselinenumber\relax}% sic!
    \let \@EN@getpagewiselno \getpagewiselinenumber}
%% For |\thePagewiseLineNumber|, <trans> is `\getpagewiselinenumber':
\renewcommand* \thePagewiseLineNumber {%
    \theWiseLineNumber\getpagewiselinenumber}
%% |\getfootnotelinenumber{<integer>}| considers <integer> 
%% the absolute number of a \emph{footnote} line. The <choice>
%% therefore is `\setgetfootnotelinenumbers':
\newcommand* \getfootnotelinenumber {%
    \getwiselinenumber\setgetfootnotelinenumbers}
%% Finally, |\theFootnoteLineNumber| is how `\linelabel'
%% refers to a \emph{footnote} line. `\theWiseLineNumber'
%% is called with <trans> being `\getfootnotelinenumber':
\newcommand* \theFootnoteLineNumber {%
    \theWiseLineNumber\getfootnotelinenumber}
%% 
%% === Leaving the Package File ===
\endinput
%%
%%  == Acknowledgements == 
%% %% 2010/12/19
%% On the 'texhax' mailing list, Boris Veytsman recommended 
%% using Victor \mbox{Eijk}\-hout's \textit{\TeX\ by Topic} to me, 
%% and Andrej Lapshin pointed me to David Salomon's work 
%% on output routines 
%% (TUGboat 1990 and 1994, also available as a book, 
%%  as Ulrich Dirr tells me).
%% It helped me a lot to read about output routines 
%% in these works, beyond the \TeX book. 
%% The abbreviations \lq OTR\rq\ and \lq MVL\rq\ are 
%% Salomon's.---And %% added 2010/12/27
%% recall Christian's work and support 
%% by the DFG named at the start of the package file.---And 
%% ... the ideas of how to implement 
%% (\textit i)~attaching line numbers, 
%% (\textit{ii})~`\linelabel', and 
%% (\textit{iii})~numbering lines ``pagewise"---so flexibly, 
%% compatibly with many other \LaTeX\ packages, still 
%% are Stephan's ...
%% 
%%  == VERSION HISTORY ==
v0.1    2010/12/08  very first, \linelabel works in footnote 
        SENT TO Christian, problems with "long" footnotes

v0.2    2010/12/08  corr. "manifoot"
        2010/12/09  moving doc. from .tex to here, 
                    different doc. sectioning;
                    \@footnotetext modified (user feature!); 
                    \@doclearpage NOT modified!; \if@FNLN@placing@
        2010/12/10  ignore dummy footnote split; 
                    \FNLNpar, \AutoPars, \ExplicitPars, 
                    more on limitations
        2010/12/11  more trying, almost anew ...
        JUST STORED

v0.3    2010/12/12  new approach, removed much before proceeding
        2010/12/13  -- this was putting \box\footins onto MVL, 
                    bad with those penalties
        JUST STORED

v0.4    2010/12/14  another new approach: 
                    typeset footnote on MVL immediately -- 
                    described strategy
        2010/12/15  ... continued, choice of hooking into \output 
                    (...swap...)
        2010/12/16  ... continued; rearranged sections ... 
                    \FNLN@@fntext vs. ...ltx...
        2010/12/17  success with \pagegoal ...; \GStoreReg etc.; 
                    ...@fntext shortened 
        2010/12/18  another two limitations: \pagebreak in fn., 
                    guessed/tested; another note to <register>; 
                    ack. Christian; directed -> organized!?
        SENT TO Christian/Stephan

v0.41   2010/12/19  support of \pagebreak with \if@FNLN@sw@ etc.; 
                    TODO on lists of <register>s
        2010/12/20  debugging: \if...true; \setbox...ft;
                    \@finalstrut in vmode exported to finstrut.sty; 
                    notes on how v0.41 still fails with \pagebreak
        2010/12/21  additional notes on *two* \pagebreak's

v0.5    2010/12/21  restructuring doc., check@latex@ -> check@,
                    own account of lineno's pagewise mode
        2010/12/22  ... continued ...
        2010/12/23  ... continued ...
        2010/12/24  ... continued ...
        2010/12/25  moved this to pwlineno, replaced ...
                    more on \FNLN@typeset, + \setfootnotelinenumbers
        2010/12/26  new summary of implementation, 
                    rearranged code sections; logging settled
v0.51   2010/12/27  "build" settled, typesetting, logging reformated;
                    ack.s: "recall"; all settings global, 
                    "public" works
        JUST STORED, MARGINAL NUMBERS OK, 
        \linelabel in footnote broken
        [2010/12/28] 
v0.52   2010/12/28  own label and vadjust lists for footnotes; 
                    local settings for referencing, 
                    tool and care for global changes (...Cache)
                    (TODO write ready in .aux? needs another run)
        \linelabel's ok, MARGINAL NOTES MAIN BROKEN
v0.53   2010/12/28  debugging; OK; minor doc. modifications; 
                    less "limitations"; \\[\smallskipamount]
        TO CHRISTIAN 2010-12-29
v0.54   2010/12/31  typo options; \FNLN@text without arg, 
                    \getpagewiselinenumber with ednotes
        2011/01/01  \FNLN@cache, \FNLN@@cache initialized; 
                    doc. "Typesetting Stage" qualification
        2011/01/02  that qualification was wrong
        2011/01/03  samepage@hook
        TO CHRISTIAN SAME DAY
v0.55   2011/01/04  samepage@hook emptied here as well; 
        2011/01/06  edited version history
        2011/01/07  note on \if@FNLN@sw@ with v0.5; 
                    finally without support for samepage@hook!
                    note on \testLastNumberedPage
        PART OF EDFN RELEASE r0.5 (together with edfnotes v0.2)
v0.55a  2011/02/09  corr. owner; "Limitations" updated; \pagebreak 

\end{document}

VERSION HISTORY

2010/12/08  for v0.1    very first
2010/12/09  for v0.2    moved much to .sty
2010/12/15  for v0.4    \strong
2010/12/16              \SectionLevelThree... 
2010/12/22ff.           beginparpenalty varied
2010/12/28  for v0.5    abstract extended
2011/02/09              removing `finstrut'; mention `edfnotes'
2011/02/10              using \urlpkgfoot etc.
2011/02/14              abstract modified
