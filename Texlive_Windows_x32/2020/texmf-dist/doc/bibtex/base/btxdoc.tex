% Copyright (C) 1988, 2010 Oren Patashnik.
% Unlimited copying and redistribution of this file are permitted if it
% is unmodified.  Modifications (and their redistribution) are also
% permitted, as long as the resulting file is renamed.

\def\BibTeX{{\rm B\kern-.05em{\sc i\kern-.025em b}\kern-.08em
    T\kern-.1667em\lower.7ex\hbox{E}\kern-.125emX}}

\title{B\kern-.05em{\large I}\kern-.025em{\large B}\kern-.08em\TeX ing}
\author{Oren Patashnik}
\date{February 8, 1988}

\documentstyle{article}
\begin{document}

\maketitle

\section{Overview}

[This document will be expanded when \BibTeX\ version 1.00 comes out.
Please report typos, omissions, inaccuracies,
and especially unclear explanations
to {\tt biblio@tug.org} ({\tt http://lists.tug.org/biblio}).
Suggestions for improvements are wanted and welcome.]

This documentation, for \BibTeX\ version 0.99b,
is meant for general \BibTeX\ users;
bibliography-style designers should read this document
and then read ``Designing \BibTeX\ Styles''~\cite{btxhak},
which is meant for just them.

This document has three parts:
Section~\ref{differences}
describes the differences between versions 0.98i and 0.99b
of \BibTeX\ and between the corresponding versions of the standard styles;
Section~\ref{latex-appendix}
updates Appendix~B.2 of the \LaTeX\ book~\cite{latex};
and Section~\ref{odds-and-ends}
gives some general and specific tips
that aren't documented elsewhere.
It's assumed throughout that you're familiar with
the relevant sections of the \LaTeX\ book.

This documentation also serves as sample input to help
\BibTeX\ implementors get it running.
For most documents, this one included, you produce the reference list by:
running \LaTeX\ on the document (to produce the {\tt aux} file(s)),
then running \BibTeX\ (to produce the {\tt bbl} file),
then \LaTeX\ twice more (first to find the information in the {\tt bbl} file
and then to get the forward references correct).
In very rare circumstances you may need an extra \BibTeX/\LaTeX\ run.

\BibTeX\ version 0.99b should be used with \LaTeX\ version 2.09,
for which the closed bibliography format is the default;
to get the open format, use the optional document style {\tt openbib}
(in an open format there's a line break between major blocks of a
reference-list entry; in a closed format the blocks run together).]

Note: \BibTeX\ 0.99b is not compatible with the old style files;
nor is \BibTeX\ 0.98i compatible with the new ones
(the new \BibTeX, however, is compatible with old database files).

Note for implementors: \BibTeX\ provides logical-area names
\hbox{\tt TEXINPUTS:} for bibliography-style files and
\hbox{\tt TEXBIB:} for database files it can't otherwise find.


\section{Changes}
\label{differences}

This section describes the differences between
\BibTeX\ versions 0.98i and 0.99b, and also between
the corresponding standard styles.
There were a lot of differences;
there will be a lot fewer between 0.99 and 1.00.


\subsection{New \BibTeX\ features}
\label{features}

The following list explains \BibTeX's new features and how to use them.
\begin{enumerate}

\item
With the single command `\hbox{\verb|\nocite{*}|}'
you can now include in the reference list
every entry in the database files, without having to explicitly
\verb|\cite| or \hbox{\verb|\nocite|} each entry.
Giving this command, in essence,
\hbox{\verb|\nocite|}s
all the enties in the database, in database order,
at the very spot in your document
where you give the command.

\item
\label{concat}
You can now have as a field value (or an {\tt @STRING} definition)
the concatenation of several strings.
For example if you've defined
\begin{verbatim}
    @STRING( WGA = " World Gnus Almanac" )
\end{verbatim}
then it's easy to produce nearly-identical
{\tt title} fields for different entries:
\begin{verbatim}
  @BOOK(almanac-66,
    title = 1966 # WGA,
    .  .  .
  @BOOK(almanac-67,
    title = 1967 # WGA,
\end{verbatim}
and so on.  Or, you could have a field like
\begin{verbatim}
    month = "1~" # jan,
\end{verbatim}
which would come out something like
`\hbox{\verb|1~January|}' or `\hbox{\verb|1~Jan.|}' in the {\tt bbl} file,
depending on how your bibliography style defines
the {\tt jan} abbreviation.
You may concatenate as many strings as you like
(except that there's a limit to the overall length
of the resulting field);
just be sure to put the concatenation character `{\tt\#}'$\!$,
surrounded by optional spaces or newlines,
between each successive pair of strings.

\item
\BibTeX\ has a new cross-referencing feature,
explained by an example.
Suppose you say \hbox{\verb|\cite{no-gnats}|} in your document,
and suppose you have these two entries in your database file:
\begin{verbatim}
  @INPROCEEDINGS(no-gnats,
    crossref = "gg-proceedings",
    author = "Rocky Gneisser",
    title = "No Gnats Are Taken for Granite",
    pages = "133-139")
  .  .  .
  @PROCEEDINGS(gg-proceedings,
    editor = "Gerald Ford and Jimmy Carter",
    title = "The Gnats and Gnus 1988 Proceedings",
    booktitle = "The Gnats and Gnus 1988 Proceedings")
\end{verbatim}
Two things happen.
First, the special \hbox{\tt crossref} field tells \BibTeX\
that the \hbox{\tt no-gnats} entry should inherit
any fields it's missing from
the entry it cross references, \hbox{\tt gg-proceedings}.
In this case it in inherits the two fields
\hbox{\tt editor} and \hbox{\tt booktitle}.
Note that, in the standard styles at least,
the \hbox{\tt booktitle} field is irrelevant
for the \hbox{\tt PROCEEDINGS} entry type.
The \hbox{\tt booktitle} field appears here
in the \hbox{\tt gg-proceedings} entry
only so that the entries that cross reference it
may inherit the field.
No matter how many papers from this meeting exist in the database,
this \hbox{\tt booktitle} field need only appear once.

The second thing that happens:
\BibTeX\ automatically puts the entry \hbox{\tt gg-proceedings}
into the reference list if it's cross
referenced by two or more entries that you
\verb|\cite| or \hbox{\verb|\nocite|},
even if you don't \verb|\cite| or \hbox{\verb|\nocite|}
the \hbox{\tt gg-proceedings} entry itself.
So \hbox{\tt gg-proceedings} will automatically appear
on the reference list if one other entry
besides \hbox{\tt no-gnats} cross references it.

To guarantee that this scheme works, however,
a cross-referenced entry must occur later in the database files
than every entry that cross-references it.
Thus, putting all cross-referenced entries at the end makes sense.
(Moreover, you may not reliably nest cross references;
that is, a cross-referenced entry may
not itself reliably cross reference an entry.
This is almost certainly not something you'd
want to do, though.)

One final note:
This cross-referencing feature is completely unrelated
to the old \BibTeX's cross referencing,
which is still allowed.
Thus, having a field like
\begin{verbatim}
    note = "Jones \cite{jones-proof} improves the result"
\end{verbatim}
is not affected by the new feature.

\item
\BibTeX\ now handles accented characters.
For example if you have an entry with the two fields
\begin{verbatim}
    author = "Kurt G{\"o}del",
    year = 1931,
\end{verbatim}
and if you're using the \hbox{\tt alpha} bibliography style,
then \BibTeX\ will construct the label
\hbox{[G{\"o}d31]} for this entry, which is what you'd want.
To get this feature to work you must place the entire accented
character in braces;
in this case either \hbox{\verb|{\"o}|}
or \hbox{\verb|{\"{o}}|} will do.
Furthermore these braces must not themselves be
enclosed in braces (other than the ones that might delimit
the entire field or the entire entry);
and there must be a backslash
as the very first character inside the braces.
Thus neither \hbox{\verb|{G{\"{o}}del}|}
nor \hbox{\verb|{G\"{o}del}|} will work for this example.

This feature handles all the accented characters and
all but the nonbackslashed foreign symbols found in Tables
3.1 and~3.2 of the \LaTeX\ book.
This feature behaves similarly for ``accents'' you might define;
we'll see an example shortly.
For the purposes of counting letters in labels,
\BibTeX\ considers everything contained inside the braces
as a single letter.

\item
\BibTeX\ also handles hyphenated names.
For example if you have an entry with
\begin{verbatim}
    author = "Jean-Paul Sartre",
\end{verbatim}
and if you're using the \hbox{\tt abbrv} style,
then the result is `J.-P. Sartre'$\!$.

\item
\label{preamble}
There's now an \hbox{\verb|@PREAMBLE|} command
for the database files.
This command's syntax is just like \hbox{\verb|@STRING|}'s,
except that there is no name or equals-sign, just the string.
Here's an example:
\begin{verbatim}
    @PREAMBLE{ "\newcommand{\noopsort}[1]{} "
             # "\newcommand{\singleletter}[1]{#1} " }
\end{verbatim}
(note the use of concatenation here, too).
The standard styles output whatever information you give this command
(\LaTeX\ macros most likely) directly to the {\tt bbl} file.
We'll look at one possible use of this command,
based on the \hbox{\verb|\noopsort|} command just defined.

The issue here is sorting (alphabetizing).
\BibTeX\ does a pretty good job,
but occasionally weird circumstances conspire to confuse \BibTeX:
Suppose that you have entries in your database for
the two books in a two-volume set by the same author,
and that you'd like volume~1 to appear
just before volume~2 in your reference list.
Further suppose that there's now a second edition of volume~1,
which came out in 1973, say,
but that there's still just one edition of volume~2,
which came out in 1971.
Since the {\tt plain} standard style
sorts by author and then year,
it will place volume~2 first
(because its edition came out two years earlier)
unless you help \BibTeX.
You can do this by using the {\tt year} fields below
for the two volumes:
\begin{verbatim}
    year = "{\noopsort{a}}1973"
    .  .  .
    year = "{\noopsort{b}}1971"
\end{verbatim}
According to the definition of \hbox{\verb|\noopsort|},
\LaTeX\ will print nothing but the true year for these fields.
But \BibTeX\ will be perfectly happy pretending that
\hbox{\verb|\noopsort|} specifies some fancy accent
that's supposed to adorn the `a' and the~`b';
thus when \BibTeX\ sorts it will pretend that
`a1973' and `b1971' are the real years,
and since `a' comes before~`b'$\!$, it will place volume~1 before volume~2,
just what you wanted.
By the way, if this author has any other works included
in your database, you'd probably want to use instead something like
\hbox{\verb|{\noopsort{1968a}}1973|} and
\hbox{\verb|{\noopsort{1968b}}1971|},
so that these two books would come out in a reasonable spot
relative to the author's other works
(this assumes that 1968 results in a reasonable spot,
say because that's when the first edition of volume~1 appeared).

There is a limit to the number of \hbox{\verb|@PREAMBLE|} commands
you may use, but you'll never exceed this limit if
you restrict yourself to one per database file;
this is not a serious restriction,
given the concatenation feature (item~\ref{concat}).

\item
\BibTeX's sorting algorithm is now stable.
This means that if two entries have identical sort keys,
those two entries will appear in citation order.
(The bibliography styles construct these sort keys---%
usually the author information followed by the year and the title.)

\item
\BibTeX\ no longer does case conversion for file names;
this will make \BibTeX\ easier to install on Unix systems, for example.

\item
It's now easier to add code for processing a
command-line {\tt aux}-file name.

\end{enumerate}


\subsection{Changes to the standard styles}

This section describes changes to the standard styles
({\tt plain}, {\tt unsrt}, {\tt alpha}, {\tt abbrv})
that affect ordinary users.
Changes that affect style designers appear in
the document ``Designing \BibTeX\ Styles''~\cite{btxhak}.
\begin{enumerate}

\item
In general, sorting is now by ``author''$\!$, then year, then title---%
the old versions didn't use the year field.
(The {\tt alpha} style, however, sorts first by label,
then ``author''$\!$, year, and title.)
The quotes around author mean that some entry types
might use something besides the author, like the editor or organization.

\item
Many unnecessary ties (\verb|~|) have been removed.
\LaTeX\ thus will produce slightly fewer
`\hbox{\tt Underfull} \verb|\hbox|' messages
when it's formatting the reference list.

\item
Emphasizing (\hbox{\verb|{\em ...}|})
has replaced italicizing (\hbox{\verb|{\it ...}|}).
This will almost never result in a difference
between the old output and the new.

\item
The {\tt alpha} style now uses a superscripted~`$^{+}$' instead of a~`*'
to represent names omitted in constructing the label.
If you really liked it the way it was, however,
or if you want to omit the character entirely,
you don't have to modify the style file---%
you can override the~`$^{+}$' by
redefining the \hbox{\verb|\etalchar|} command
that the {\tt alpha} style writes onto the {\tt bbl} file
(just preceding the \hbox{\verb|\thebibliography|} environment);
use \LaTeX's \hbox{\verb|\renewcommand|} inside
a database \hbox{\tt @PREAMBLE} command,
described in the previous subsection's item~\ref{preamble}.

\item
The {\tt abbrv} style now uses `Mar.' and `Sept.'\
for those months rather than `March' and `Sep.'

\item
The standard styles use \BibTeX's new cross-referencing feature
by giving a \verb|\cite| of the cross-referenced entry and by
omitting from the cross-referencing entry
(most of the) information that appears
in the cross-referenced entry.
These styles do this when
a titled thing (the cross-referencing entry)
is part of a larger titled thing (the cross-referenced entry).
There are five such situations:
when (1)~an \hbox{\tt INPROCEEDINGS}
(or \hbox{\tt CONFERENCE}, which is the same)
cross references a \hbox{\tt PROCEEDINGS};
when (2)~a {\tt BOOK}, (3)~an \hbox{\tt INBOOK},
or (4)~an \hbox{\tt INCOLLECTION}
cross references a {\tt BOOK}
(in these cases, the cross-referencing entry is a single
volume in a multi-volume work);
and when (5)~an \hbox{\tt ARTICLE}
cross references an \hbox{\tt ARTICLE}
(in this case, the cross-referenced entry is really a journal,
but there's no \hbox{\tt JOURNAL} entry type;
this will result in warning messages about
an empty \hbox{\tt author} and \hbox{\tt title} for the journal---%
you should just ignore these warnings).

\item
The \hbox{\tt MASTERSTHESIS} and \hbox{\tt PHDTHESIS}
entry types now take an optional {\tt type} field.
For example you can get the standard styles to
call your reference a `Ph.D.\ dissertation'
instead of the default `PhD thesis' by including a
\begin{verbatim}
    type = "{Ph.D.} dissertation"
\end{verbatim}
in your database entry.

\item
Similarly, the \hbox{\tt INBOOK} and \hbox{\tt INCOLLECTION}
entry types now take an optional {\tt type} field,
allowing `section~1.2' instead of the default `chapter~1.2'$\!$.
You get this by putting
\begin{verbatim}
    chapter = "1.2",
    type = "Section"
\end{verbatim}
in your database entry.

\item
The \hbox{\tt BOOKLET}, \hbox{\tt MASTERSTHESIS},
and \hbox{\tt TECHREPORT} entry types now format
their \hbox{\tt title} fields as if they were
\hbox{\tt ARTICLE} \hbox{\tt title}s
rather than \hbox{\tt BOOK} \hbox{\tt title}s.

\item
The \hbox{\tt PROCEEDINGS} and \hbox{\tt INPROCEEDINGS}
entry types now use the \hbox{\tt address} field
to tell where a conference was held,
rather than to give the address
of the publisher or organization.
If you want to include the
publisher's or organization's address,
put it in the \hbox{\tt publisher}
or \hbox{\tt organization} field.

\item
The \hbox{\tt BOOK}, \hbox{\tt INBOOK}, \hbox{\tt INCOLLECTION},
and \hbox{\tt PROCEEDINGS} entry types now allow either
\hbox{\tt volume} or \hbox{\tt number} (but not both),
rather than just \hbox{\tt volume}.

\item
The \hbox{\tt INCOLLECTION} entry type now allows
a \hbox{\tt series} and an \hbox{\tt edition} field.

\item
The \hbox{\tt INPROCEEDINGS} and \hbox{\tt PROCEEDINGS}
entry types now allow either a \hbox{\tt volume} or \hbox{\tt number},
and also a \hbox{\tt series} field.

\item
The \hbox{\tt UNPUBLISHED} entry type now outputs,
in one block, the \hbox{\tt note} field
followed by the date information.

\item
The \hbox{\tt MANUAL} entry type now prints out
the \hbox{\tt organization} in the first block
if the \hbox{\tt author} field is empty.

\item
The {\tt MISC} entry type now issues a warning
if all the optional fields are empty
(that is, if the entire entry is empty).

\end{enumerate}


\section{The Entries}
\label{latex-appendix}

This section is simply a corrected version of
Appendix~B.2 of the \LaTeX\ book~\cite{latex},
\copyright~1986, by Addison-Wesley.
The basic scheme is the same, only a few details have changed.


\subsection{Entry Types}

When entering a reference in the database, the first thing to decide
is what type of entry it is.  No fixed classification scheme can be
complete, but \BibTeX\ provides enough entry types to handle almost
any reference reasonably well.

References to different types of publications contain different
information; a reference to a journal article might include the volume
and number of the journal, which is usually not meaningful for a book.
Therefore, database entries of different types have different fields.
For each entry type, the fields are divided into three classes:
\begin{description}

\item[required]
Omitting the field will produce a warning message
and, rarely, a badly formatted bibliography entry.
If the required information is not meaningful,
you are using the wrong entry type.
However, if the required information is meaningful
but, say, already included is some other field,
simply ignore the warning.

\item[optional]
The field's information will be used if present,
but can be omitted without causing any formatting problems.
You should include the optional field if it will help the reader.

\item[ignored]
The field is ignored.
\BibTeX\ ignores any field that is not required or optional, so you can include
any fields you want in a \hbox{\tt bib} file entry.  It's a good idea
to put all relevant information about
a reference in its \hbox{\tt bib} file entry---even information that
may never appear in the bibliography.  For example, if you want to
keep an abstract of a paper in a computer file, put it in an \hbox{\tt
abstract} field in the paper's \hbox{\tt bib} file entry.  The
\hbox{\tt bib} file is likely to be as good a place as any for the
abstract, and it is possible to design a bibliography style for
printing selected abstracts.
Note: Misspelling a field name will
result in its being ignored,
so watch out for typos
(especially for optional fields,
since \BibTeX\ won't warn you when those are missing).

\end{description}

The following are the standard entry types, along with their required
and optional fields, that are used by the standard bibliography styles.
The fields within each class (required or optional)
are listed in order of occurrence in the output,
except that a few entry types may perturb the order slightly,
depending on what fields are missing.
These entry types are similar to those adapted by Brian Reid
from the classification scheme of van~Leunen~\cite{van-leunen}
for use in the {\em Scribe\/} system.
The meanings of the individual fields are explained in the next section.
Some nonstandard bibliography styles may ignore some optional fields
in creating the reference.
Remember that, when used in the \hbox{\tt bib}
file, the entry-type name is preceded by an \hbox{\tt @} character.

\begin{description}
\sloppy

\item[article\hfill] An article from a journal or magazine.
Required fields: \hbox{\tt author}, \hbox{\tt title}, \hbox{\tt journal},
\hbox{\tt year}.
Optional fields: \hbox{\tt volume}, \hbox{\tt number},
\hbox{\tt pages}, \hbox{\tt month}, \hbox{\tt note}.

\item[book\hfill] A book with an explicit publisher.
Required fields: \hbox{\tt author} or \hbox{\tt editor},
\hbox{\tt title}, \hbox{\tt publisher}, \hbox{\tt year}.
Optional fields: \hbox{\tt volume} or \hbox{\tt number}, \hbox{\tt series},
\hbox{\tt address}, \hbox{\tt edition}, \hbox{\tt month},
\hbox{\tt note}.

\item[booklet\hfill] A work that is printed and bound,
but without a named publisher or sponsoring institution.
Required field: \hbox{\tt title}.
Optional fields: \hbox{\tt author}, \hbox{\tt howpublished},
\hbox{\tt address}, \hbox{\tt month}, \hbox{\tt year}, \hbox{\tt note}.

\item[conference\hfill] The same as {\tt INPROCEEDINGS},
included for {\em Scribe\/} compatibility.

\item[inbook\hfill] A part of a book,
which may be a chapter (or section or whatever) and/or a range of pages.
Required fields: \hbox{\tt author} or \hbox{\tt editor}, \hbox{\tt title},
\hbox{\tt chapter} and/or \hbox{\tt pages}, \hbox{\tt publisher},
\hbox{\tt year}.
Optional fields: \hbox{\tt volume} or \hbox{\tt number}, \hbox{\tt series},
\hbox{\tt type}, \hbox{\tt address},
\hbox{\tt edition}, \hbox{\tt month}, \hbox{\tt note}.

\item[incollection\hfill] A part of a book having its own title.
Required fields: \hbox{\tt author}, \hbox{\tt title}, \hbox{\tt booktitle},
\hbox{\tt publisher}, \hbox{\tt year}.
Optional fields: \hbox{\tt editor}, \hbox{\tt volume} or \hbox{\tt number},
\hbox{\tt series}, \hbox{\tt type}, \hbox{\tt chapter}, \hbox{\tt pages},
\hbox{\tt address}, \hbox{\tt edition}, \hbox{\tt month}, \hbox{\tt note}.

\item[inproceedings\hfill] An article in a conference proceedings.
Required fields: \hbox{\tt author}, \hbox{\tt title}, \hbox{\tt booktitle},
\hbox{\tt year}.
Optional fields: \hbox{\tt editor}, \hbox{\tt volume} or \hbox{\tt number},
\hbox{\tt series}, \hbox{\tt pages}, \hbox{\tt address}, \hbox{\tt month},
\hbox{\tt organization}, \hbox{\tt publisher}, \hbox{\tt note}.

\item[manual\hfill] Technical documentation.  Required field: \hbox{\tt title}.
Optional fields: \hbox{\tt author}, \hbox{\tt organization},
\hbox{\tt address}, \hbox{\tt edition}, \hbox{\tt month}, \hbox{\tt year},
\hbox{\tt note}.

\item[mastersthesis\hfill] A Master's thesis.
Required fields: \hbox{\tt author}, \hbox{\tt title}, \hbox{\tt school},
\hbox{\tt year}.
Optional fields: \hbox{\tt type}, \hbox{\tt address}, \hbox{\tt month},
\hbox{\tt note}.

\item[misc\hfill] Use this type when nothing else fits.
Required fields: none.
Optional fields: \hbox{\tt author}, \hbox{\tt title}, \hbox{\tt howpublished},
\hbox{\tt month}, \hbox{\tt year}, \hbox{\tt note}.

\item[phdthesis\hfill] A PhD thesis.
Required fields: \hbox{\tt author}, \hbox{\tt title}, \hbox{\tt school},
\hbox{\tt year}.
Optional fields: \hbox{\tt type}, \hbox{\tt address}, \hbox{\tt month},
\hbox{\tt note}.

\item[proceedings\hfill] The proceedings of a conference.
Required fields: \hbox{\tt title}, \hbox{\tt year}.
Optional fields: \hbox{\tt editor}, \hbox{\tt volume} or \hbox{\tt number},
\hbox{\tt series}, \hbox{\tt address}, \hbox{\tt month},
\hbox{\tt organization}, \hbox{\tt publisher}, \hbox{\tt note}.


\item[techreport\hfill] A report published by a school or other institution,
usually numbered within a series.
Required fields: \hbox{\tt author},
\hbox{\tt title}, \hbox{\tt institution}, \hbox{\tt year}.
Optional fields: \hbox{\tt type}, \hbox{\tt number}, \hbox{\tt address},
\hbox{\tt month}, \hbox{\tt note}.

\item[unpublished\hfill] A document having an author and title,
but not formally published.
Required fields: \hbox{\tt author}, \hbox{\tt title}, \hbox{\tt note}.
Optional fields: \hbox{\tt month}, \hbox{\tt year}.

\end{description}

In addition to the fields listed above, each entry type also has an
optional \hbox{\tt key} field, used in some styles
for alphabetizing, for cross referencing,
or for forming a \hbox{\verb|\bibitem|} label.
You should include a \hbox{\tt key} field for any entry whose
``author'' information is missing;
the ``author'' information is usually the \hbox{\tt author} field,
but for some entry types it can be the \hbox{\tt editor}
or even the \hbox{\tt organization} field
(Section~\ref{odds-and-ends} describes this in more detail).
Do not confuse the \hbox{\tt key} field with the key that appears in the
\hbox{\verb|\cite|} command and at the beginning of the database entry;
this field is named ``key'' only for compatibility with {\it Scribe}.


\subsection{Fields}

Below is a description of all fields
recognized by the standard bibliography styles.
An entry can also contain other fields, which are ignored by those styles.
\begin{description}

\item[address\hfill]
Usually the address of the \hbox{\tt publisher} or other type
of institution.
For major publishing houses,
van~Leunen recommends omitting the information entirely.
For small publishers, on the other hand, you can help the
reader by giving the complete address.

\item[annote\hfill]
An annotation.
It is not used by the standard bibliography styles,
but may be used by others that produce an annotated bibliography.

\item[author\hfill]
The name(s) of the author(s),
in the format described in the \LaTeX\ book.

\item[booktitle\hfill]
Title of a book, part of which is being cited.
See the \LaTeX\ book for how to type titles.
For book entries, use the \hbox{\tt title} field instead.

\item[chapter\hfill]
A chapter (or section or whatever) number.

\item[crossref\hfill]
The database key of the entry being cross referenced.

\item[edition\hfill]
The edition of a book---for example, ``Second''$\!$.
This should be an ordinal, and
should have the first letter capitalized, as shown here;
the standard styles convert to lower case when necessary.

\item[editor\hfill]
Name(s) of editor(s), typed as indicated in the \LaTeX\ book.
If there is also an \hbox{\tt author} field, then
the \hbox{\tt editor} field gives the editor of the book or collection
in which the reference appears.

\item[howpublished\hfill]
How something strange has been published.
The first word should be capitalized.

\item[institution\hfill]
The sponsoring institution of a technical report.

\item[journal\hfill]
A journal name.
Abbreviations are provided for many journals; see the {\it Local Guide}.

\item[key\hfill]
Used for alphabetizing, cross referencing, and creating a label when
the ``author'' information
(described in Section~\ref{odds-and-ends}) is missing.
This field should not be confused with the key that appears in the
\hbox{\verb|\cite|} command and at the beginning of the database entry.

\item[month\hfill]
The month in which the work was
published or, for an unpublished work, in which it was written.
You should use the standard three-letter abbreviation,
as described in Appendix B.1.3 of the \LaTeX\ book.

\item[note\hfill]
Any additional information that can help the reader.
The first word should be capitalized.

\item[number\hfill]
The number of a journal, magazine, technical report,
or of a work in a series.
An issue of a journal or magazine is usually
identified by its volume and number;
the organization that issues a
technical report usually gives it a number;
and sometimes books are given numbers in a named series.

\item[organization\hfill]
The organization that sponsors a conference or that publishes a \hbox{manual}.

\item[pages\hfill]
One or more page numbers or range of numbers,
such as \hbox{\tt 42--111} or \hbox{\tt 7,41,73--97} or \hbox{\tt 43+}
(the `{\tt +}' in this last example indicates pages following
that don't form a simple range).
To make it easier to maintain {\em Scribe\/}-compatible databases,
the standard styles convert a single dash (as in \hbox{\tt 7-33})
to the double dash used in \TeX\ to denote number ranges
(as in \hbox{\tt 7--33}).

\item[publisher\hfill]
The publisher's name.

\item[school\hfill]
The name of the school where a thesis was written.

\item[series\hfill]
The name of a series or set of books.
When citing an entire book, the the \hbox{\tt title} field
gives its title and an optional \hbox{\tt series} field gives the
name of a series or multi-volume set
in which the book is published.

\item[title\hfill]
The work's title, typed as explained in the \LaTeX\ book.

\item[type\hfill]
The type of a technical report---for example,
``Research Note''$\!$.

\item[volume\hfill]
The volume of a journal or multivolume book.

\item[year\hfill]
The year of publication or, for
an unpublished work, the year it was written.
Generally it should consist of four numerals, such as {\tt 1984},
although the standard styles can handle any {\tt year} whose
last four nonpunctuation characters are numerals,
such as `\hbox{(about 1984)}'$\!$.

\end{description}


\section{Helpful Hints}
\label{odds-and-ends}

This section gives some random tips
that aren't documented elsewhere,
at least not in this detail.
They are, roughly, in order
of least esoteric to most.
First, however, a brief spiel.

I understand that there's often little choice in choosing
a bibliography style---journal~$X$ says you must use style~$Y$
and that's that.
If you have a choice, however, I strongly recommend that you
choose something like the {\tt plain} standard style.
Such a style, van~Leunen~\cite{van-leunen} argues convincingly,
encourages better writing than the alternatives---%
more concrete, more vivid.

{\em The Chicago Manual of Style\/}~\cite{chicago},
on the other hand,
espouse the author-date system,
in which the citation might appear in the text as `(Jones, 1986)'$\!$.
I argue that this system,
besides cluttering up the
text with information that may or may not be relevant,
encourages the passive voice and vague writing.
Furthermore the strongest arguments for
using the author-date system---like ``it's the most practical''---%
fall flat on their face with the advent
of computer-typesetting technology.
For instance the {\em Chicago Manual\/} contains,
right in the middle of page~401, this anachronism:
``The chief disadvantage of [a style like {\tt plain}] is that additions
or deletions cannot be made after the manuscript is typed without changing
numbers in both text references and list.''
\LaTeX, obviously, sidesteps the disadvantage.

Finally, the logical deficiencies of the author-date style
are quite evident once you've written a program to implement it.
For example, in a large bibliography,
using the standard alphabetizing scheme,
the entry for `(Aho et~al., 1983b)'
might be half a page later than the one for `(Aho et~al., 1983a)'$\!$.
Fixing this problem results in even worse ones.
What a mess.
(I have, unfortunately, programmed such a style,
and if you're saddled with an unenlightened publisher
or if you don't buy my propaganda,
it's available from the Rochester style collection.)

Ok, so the spiel wasn't very brief;
but it made me feel better,
and now my blood pressure is back to normal.
Here are the tips for using \BibTeX\
with the standard styles
(although many of them hold for nonstandard styles, too).
\begin{enumerate}

\item
With \BibTeX's style-designing language
you can program general database manipulations,
in addition to bibliography styles.
For example it's a fairly easy task for someone familiar with the language
to produce a database-key/author index of all the entries in a database.
Consult the {\em Local Guide\/} to see
what tools are available on your system.

\item
The standard style's thirteen entry types
do reasonably well at formatting most entries,
but no scheme with just thirteen formats
can do everything perfectly.
Thus, you should feel free to be creative
in how you use these entry types
(but if you have to be too creative,
there's a good chance you're using the wrong entry type).

\item
Don't take the field names too seriously.
Sometimes, for instance, you might have to include
the publisher's address along with the publisher's name
in the \hbox{\tt publisher} field,
rather than putting it in the \hbox{\tt address} field.
Or sometimes, difficult entries work best when you
make judicious use of the {\tt note} field.

\item
Don't take the warning messages too seriously.
Sometimes, for instance, the year appears in the title,
as in {\em The 1966 World Gnus Almanac}.
In this case it's best to omit the {\tt year} field
and to ignore \BibTeX's warning message.

\item
If you have too many names to list in an
\hbox{\tt author} or \hbox{\tt editor} field,
you can end the list with ``and others'';
the standard styles appropriately append an ``et~al.''

\item
In general, if you want to keep \BibTeX\ from changing
something to lower case, you enclose it in braces.
You might not get the effect you want, however,
if the very first character after the left brace is a backslash.
The ``special characters'' item later in this section explains.

\item
For {\em Scribe\/} compatibility, the database files
allow an \hbox{\tt @COMMENT} command; it's not really
needed because \BibTeX\ allows in the database files
any comment that's not within an entry.
If you want to comment out an entry,
simply remove the `{\tt @}' character preceding the entry type.

\item
The standard styles have journal abbreviations that are
computer-science oriented;
these are in the style files primarily for the example.
If you have a different set of journal abbreviations,
it's sensible to put them in \hbox{\tt @STRING} commands
in their own database file and to list this database file
as an argument to \LaTeX's \hbox{\verb|\bibliography|} command
(but you should list this argument before the ones that
specify real database entries).

\item
It's best to use the three-letter abbreviations for the month,
rather than spelling out the month yourself.
This lets the bibliography style be consistent.
And if you want to include information for the day of the month,
the {\tt month} field is usually the best place.
For example
\begin{verbatim}
    month = jul # "~4,"
\end{verbatim}
will probably produce just what you want.

\item
If you're using the \hbox{\tt unsrt} style
(references are listed in order of citation)
along with the \hbox{\verb|\nocite{*}|} feature
(all entries in the database are included),
the placement of the \hbox{\verb|\nocite{*}|} command
within your document file will determine the reference order.
According to the rule given in Section~\ref{features}:
If the command is placed at the beginning of the document,
the entries will be listed in exactly the order
they occur in the database;
if it's placed at the end,
the entries that you explicitly
\hbox{\verb|\cite|} or \hbox{\verb|\nocite|}
will occur in citation order,
and the remaining database entries will be in database order.

\item
For theses, van Leunen recommends not giving
the school's department after the name of the degree,
since schools, not departments, issue degrees.
If you really think that giving the department information
will help the reader find the thesis,
put that information in the \hbox{\tt address} field.

\item
The \hbox{\tt MASTERSTHESIS} and \hbox{\tt PHDTHESIS} entry types
are so named for {\em Scribe\/} compatibility;
\hbox{\tt MINORTHESIS} and \hbox{\tt MAJORTHESIS}
probably would have been better names.
Keep this in mind when trying to classify
a non-U.S.\ thesis.

\item
Here's yet another suggestion for what to do when an author's
name appears slightly differently in two publications.
Suppose, for example, two journals articles use these fields.
\begin{verbatim}
    author = "Donald E. Knuth"
    .  .  .
    author = "D. E. Knuth"
\end{verbatim}
There are two possibilities.
You could (1)~simply leave them as is,
or (2)~assuming you know for sure that
these authors are one and the same person,
you could list both in the form that the author prefers
(say, `Donald~E.\ Knuth').
In the first case, the entries might be alphabetized incorrectly,
and in the second, the slightly altered name might
foul up somebody's electronic library search.
But there's a third possibility, which is the one I prefer.
You could convert the second journal's field to
\begin{verbatim}
    author = "D[onald] E. Knuth"
\end{verbatim}
This avoids the pitfalls of the previous two solutions,
since \BibTeX\ alphabetizes this as if the brackets weren't there,
and since the brackets clue the reader in that a full first name
was missing from the original.
Of course it introduces another pitfall---`D[onald]~E.\ Knuth' looks ugly---%
but in this case I think the increase in accuracy outweighs
the loss in aesthetics.

\item
\LaTeX's comment character `{\tt\%}' is not a comment character
in the database files.

\item
Here's a more complete description of
the ``author'' information referred to in previous sections.
For most entry types the ``author'' information
is simply the \hbox{\tt author} field.
However:
For the \hbox{\tt BOOK} and \hbox{\tt INBOOK} entry types
it's the \hbox{\tt author} field, but if there's no author
then it's the \hbox{\tt editor} field;
for the \hbox{\tt MANUAL} entry type
it's the \hbox{\tt author} field, but if there's no author
then it's the \hbox{\tt organization} field;
and for the \hbox{\tt PROCEEDINGS} entry type
it's the \hbox{\tt editor} field, but if there's no editor
then it's the \hbox{\tt organization} field.

\item
When creating a label,
the \hbox{\tt alpha} style uses the ``author'' information described above,
but with a slight change---%
for the \hbox{\tt MANUAL} and \hbox{\tt PROCEEDINGS} entry types,
the {\tt key} field takes precedence over the \hbox{\tt organization} field.
Here's a situation where this is useful.
\begin{verbatim}
   organization = "The Association for Computing Machinery",
   key = "ACM"
\end{verbatim}
Without the {\tt key} field, the \hbox{\tt alpha} style
would make a label from the first three letters of information
in the \hbox{\tt organization} field;
\hbox{\tt alpha} knows to strip off the `\hbox{\tt The }'$\!$,
but it would still form a label like `\hbox{[Ass86]}'$\!$,
which, however intriguing, is uninformative.
Including the {\tt key} field, as above,
would yield the better label `\hbox{[ACM86]}'$\!$.

You won't always need the {\tt key} field to override the
\hbox{\tt organization}, though:
With
\begin{verbatim}
    organization = "Unilogic, Ltd.",
\end{verbatim}
for instance, the \hbox{\tt alpha} style would
form the perfectly reasonable label `\hbox{[Uni86]}'$\!$.

\item
Section~\ref{features} discusses accented characters.
To \BibTeX, an accented character is really a special case
of a ``special character''$\!$,
which consists of everything from a left brace at the top-most level,
immediately followed by a backslash,
up through the matching right brace.
For example in the field
\begin{verbatim}
    author = "\AA{ke} {Jos{\'{e}} {\'{E}douard} G{\"o}del"
\end{verbatim}
there are just two special characters,
`\hbox{\verb|{\'{E}douard}|}' and `\hbox{\verb|{\"o}|}'
(the same would be true if the pair of double quotes
delimiting the field were braces instead).
In general, \BibTeX\ will not do any processing
of a \TeX\ or \LaTeX\ control sequence inside a special character,
but it {\em will\/} process other characters.
Thus a style that converts all titles to lower case
would convert
\begin{verbatim}
    The {\TeX BOOK\NOOP} Experience
\end{verbatim}
to
\begin{verbatim}
    The {\TeX book\NOOP} experience
\end{verbatim}
(the `{\tt The}' is still capitalized
because it's the first word of the title).

This special-character scheme is useful for handling accented characters,
for getting \BibTeX's alphabetizing to do what you want,
and, since \BibTeX\ counts an entire special character as just one letter,
for stuffing extra characters inside labels.
The file \hbox{\tt XAMPL.BIB} distributed with \BibTeX\
gives examples of all three uses.

\item
This final item of the section describes \BibTeX's names
(which appear in the \hbox{\tt author} or \hbox{\tt editor} field)
in slightly more detail than what
appears in Appendix~B of the \LaTeX\ book.
In what follows, a ``name'' corresponds to a person.
(Recall that you separate multiple names in a single field
with the word ``and''$\!$, surrounded by spaces,
and not enclosed in braces.
This item concerns itself with the structure of a single name.)

Each name consists of four parts: First, von, Last, and~Jr;
each part consists of a (possibly empty) list of name-tokens.
The Last part will be nonempty if any part is,
so if there's just one token, it's always a Last token.

Recall that Per Brinch~Hansen's name should be typed
\begin{verbatim}
    "Brinch Hansen, Per"
\end{verbatim}
The First part of his name has the single token ``Per'';
the Last part has two tokens, ``Brinch'' and ``Hansen'';
and the von and Jr parts are empty.
If you had typed
\begin{verbatim}
    "Per Brinch Hansen"
\end{verbatim}
instead, \BibTeX\ would (erroneously) think ``Brinch'' were a First-part token,
just as ``Paul'' is a First-part token in ``John~Paul Jones''$\!$,
so this erroneous form would have two First tokens and one Last token.

Here's another example:
\begin{verbatim}
    "Charles Louis Xavier Joseph de la Vall{\'e}e Poussin"
\end{verbatim}
This name has four tokens in the First part, two in the von, and
two in the Last.
Here \BibTeX\ knows where one part ends and the other begins because
the tokens in the von part begin with lower-case letters.

In general, it's a von token if the first letter at brace-level~0
is in lower case.
Since technically everything
in a ``special character'' is at brace-level~0,
you can trick \BibTeX\ into thinking that
a token is or is not a von token by prepending a dummy
special character whose first letter past the \TeX\ control sequence
is in the desired case, upper or lower.

To summarize, \BibTeX\ allows three possible forms for the name:
\begin{verbatim}
    "First von Last"
    "von Last, First"
    "von Last, Jr, First"
\end{verbatim}
You may almost always use the first form;
you shouldn't if either there's a Jr part,
or the Last part has multiple tokens but there's no von part.

\end{enumerate}

\bibliography{btxdoc}
\bibliographystyle{plain}
\end{document}
