% \iffalse meta-comment
%
% File: letltxmacro.dtx
% Version: 2019/12/03 v1.6
% Info: Let assignment for LaTeX macros
%
% Copyright (C)
%    2008, 2010 Heiko Oberdiek
%    2016-2019 Oberdiek Package Support Group
%    https://github.com/ho-tex/letltxmacro/issues
%
% This work may be distributed and/or modified under the
% conditions of the LaTeX Project Public License, either
% version 1.3c of this license or (at your option) any later
% version. This version of this license is in
%    https://www.latex-project.org/lppl/lppl-1-3c.txt
% and the latest version of this license is in
%    https://www.latex-project.org/lppl.txt
% and version 1.3 or later is part of all distributions of
% LaTeX version 2005/12/01 or later.
%
% This work has the LPPL maintenance status "maintained".
%
% The Current Maintainers of this work are
% Heiko Oberdiek and the Oberdiek Package Support Group
% https://github.com/ho-tex/letltxmacro/issues
%
% This work consists of the main source file letltxmacro.dtx
% and the derived files
%    letltxmacro.sty, letltxmacro.pdf, letltxmacro.ins, letltxmacro.drv,
%    letltxmacro-showcases.tex, letltxmacro-test1.tex,
%    letltxmacro-test2.tex.
%
% Distribution:
%    CTAN:macros/latex/contrib/letltxmacro/letltxmacro.dtx
%    CTAN:macros/latex/contrib/letltxmacro/letltxmacro.pdf
%
% Unpacking:
%    (a) If letltxmacro.ins is present:
%           tex letltxmacro.ins
%    (b) Without letltxmacro.ins:
%           tex letltxmacro.dtx
%    (c) If you insist on using LaTeX
%           latex \let\install=y% \iffalse meta-comment
%
% File: letltxmacro.dtx
% Version: 2019/12/03 v1.6
% Info: Let assignment for LaTeX macros
%
% Copyright (C)
%    2008, 2010 Heiko Oberdiek
%    2016-2019 Oberdiek Package Support Group
%    https://github.com/ho-tex/letltxmacro/issues
%
% This work may be distributed and/or modified under the
% conditions of the LaTeX Project Public License, either
% version 1.3c of this license or (at your option) any later
% version. This version of this license is in
%    https://www.latex-project.org/lppl/lppl-1-3c.txt
% and the latest version of this license is in
%    https://www.latex-project.org/lppl.txt
% and version 1.3 or later is part of all distributions of
% LaTeX version 2005/12/01 or later.
%
% This work has the LPPL maintenance status "maintained".
%
% The Current Maintainers of this work are
% Heiko Oberdiek and the Oberdiek Package Support Group
% https://github.com/ho-tex/letltxmacro/issues
%
% This work consists of the main source file letltxmacro.dtx
% and the derived files
%    letltxmacro.sty, letltxmacro.pdf, letltxmacro.ins, letltxmacro.drv,
%    letltxmacro-showcases.tex, letltxmacro-test1.tex,
%    letltxmacro-test2.tex.
%
% Distribution:
%    CTAN:macros/latex/contrib/letltxmacro/letltxmacro.dtx
%    CTAN:macros/latex/contrib/letltxmacro/letltxmacro.pdf
%
% Unpacking:
%    (a) If letltxmacro.ins is present:
%           tex letltxmacro.ins
%    (b) Without letltxmacro.ins:
%           tex letltxmacro.dtx
%    (c) If you insist on using LaTeX
%           latex \let\install=y% \iffalse meta-comment
%
% File: letltxmacro.dtx
% Version: 2019/12/03 v1.6
% Info: Let assignment for LaTeX macros
%
% Copyright (C)
%    2008, 2010 Heiko Oberdiek
%    2016-2019 Oberdiek Package Support Group
%    https://github.com/ho-tex/letltxmacro/issues
%
% This work may be distributed and/or modified under the
% conditions of the LaTeX Project Public License, either
% version 1.3c of this license or (at your option) any later
% version. This version of this license is in
%    https://www.latex-project.org/lppl/lppl-1-3c.txt
% and the latest version of this license is in
%    https://www.latex-project.org/lppl.txt
% and version 1.3 or later is part of all distributions of
% LaTeX version 2005/12/01 or later.
%
% This work has the LPPL maintenance status "maintained".
%
% The Current Maintainers of this work are
% Heiko Oberdiek and the Oberdiek Package Support Group
% https://github.com/ho-tex/letltxmacro/issues
%
% This work consists of the main source file letltxmacro.dtx
% and the derived files
%    letltxmacro.sty, letltxmacro.pdf, letltxmacro.ins, letltxmacro.drv,
%    letltxmacro-showcases.tex, letltxmacro-test1.tex,
%    letltxmacro-test2.tex.
%
% Distribution:
%    CTAN:macros/latex/contrib/letltxmacro/letltxmacro.dtx
%    CTAN:macros/latex/contrib/letltxmacro/letltxmacro.pdf
%
% Unpacking:
%    (a) If letltxmacro.ins is present:
%           tex letltxmacro.ins
%    (b) Without letltxmacro.ins:
%           tex letltxmacro.dtx
%    (c) If you insist on using LaTeX
%           latex \let\install=y% \iffalse meta-comment
%
% File: letltxmacro.dtx
% Version: 2019/12/03 v1.6
% Info: Let assignment for LaTeX macros
%
% Copyright (C)
%    2008, 2010 Heiko Oberdiek
%    2016-2019 Oberdiek Package Support Group
%    https://github.com/ho-tex/letltxmacro/issues
%
% This work may be distributed and/or modified under the
% conditions of the LaTeX Project Public License, either
% version 1.3c of this license or (at your option) any later
% version. This version of this license is in
%    https://www.latex-project.org/lppl/lppl-1-3c.txt
% and the latest version of this license is in
%    https://www.latex-project.org/lppl.txt
% and version 1.3 or later is part of all distributions of
% LaTeX version 2005/12/01 or later.
%
% This work has the LPPL maintenance status "maintained".
%
% The Current Maintainers of this work are
% Heiko Oberdiek and the Oberdiek Package Support Group
% https://github.com/ho-tex/letltxmacro/issues
%
% This work consists of the main source file letltxmacro.dtx
% and the derived files
%    letltxmacro.sty, letltxmacro.pdf, letltxmacro.ins, letltxmacro.drv,
%    letltxmacro-showcases.tex, letltxmacro-test1.tex,
%    letltxmacro-test2.tex.
%
% Distribution:
%    CTAN:macros/latex/contrib/letltxmacro/letltxmacro.dtx
%    CTAN:macros/latex/contrib/letltxmacro/letltxmacro.pdf
%
% Unpacking:
%    (a) If letltxmacro.ins is present:
%           tex letltxmacro.ins
%    (b) Without letltxmacro.ins:
%           tex letltxmacro.dtx
%    (c) If you insist on using LaTeX
%           latex \let\install=y\input{letltxmacro.dtx}
%        (quote the arguments according to the demands of your shell)
%
% Documentation:
%    (a) If letltxmacro.drv is present:
%           latex letltxmacro.drv
%    (b) Without letltxmacro.drv:
%           latex letltxmacro.dtx; ...
%    The class ltxdoc loads the configuration file ltxdoc.cfg
%    if available. Here you can specify further options, e.g.
%    use A4 as paper format:
%       \PassOptionsToClass{a4paper}{article}
%
%    Programm calls to get the documentation (example):
%       pdflatex letltxmacro.dtx
%       makeindex -s gind.ist letltxmacro.idx
%       pdflatex letltxmacro.dtx
%       makeindex -s gind.ist letltxmacro.idx
%       pdflatex letltxmacro.dtx
%
% Installation:
%    TDS:tex/latex/letltxmacro/letltxmacro.sty
%    TDS:doc/latex/letltxmacro/letltxmacro.pdf
%    TDS:doc/latex/letltxmacro/letltxmacro-showcases.tex
%    TDS:source/latex/letltxmacro/letltxmacro.dtx
%
%<*ignore>
\begingroup
  \catcode123=1 %
  \catcode125=2 %
  \def\x{LaTeX2e}%
\expandafter\endgroup
\ifcase 0\ifx\install y1\fi\expandafter
         \ifx\csname processbatchFile\endcsname\relax\else1\fi
         \ifx\fmtname\x\else 1\fi\relax
\else\csname fi\endcsname
%</ignore>
%<*install>
\input docstrip.tex
\Msg{************************************************************************}
\Msg{* Installation}
\Msg{* Package: letltxmacro 2019/12/03 v1.6 Let assignment for LaTeX macros (HO)}
\Msg{************************************************************************}

\keepsilent
\askforoverwritefalse

\let\MetaPrefix\relax
\preamble

This is a generated file.

Project: letltxmacro
Version: 2019/12/03 v1.6

Copyright (C)
   2008, 2010 Heiko Oberdiek
   2016-2019 Oberdiek Package Support Group

This work may be distributed and/or modified under the
conditions of the LaTeX Project Public License, either
version 1.3c of this license or (at your option) any later
version. This version of this license is in
   https://www.latex-project.org/lppl/lppl-1-3c.txt
and the latest version of this license is in
   https://www.latex-project.org/lppl.txt
and version 1.3 or later is part of all distributions of
LaTeX version 2005/12/01 or later.

This work has the LPPL maintenance status "maintained".

The Current Maintainers of this work are
Heiko Oberdiek and the Oberdiek Package Support Group
https://github.com/ho-tex/letltxmacro/issues


This work consists of the main source file letltxmacro.dtx
and the derived files
   letltxmacro.sty, letltxmacro.pdf, letltxmacro.ins, letltxmacro.drv,
   letltxmacro-showcases.tex, letltxmacro-test1.tex,
   letltxmacro-test2.tex.

\endpreamble
\let\MetaPrefix\DoubleperCent

\generate{%
  \file{letltxmacro.ins}{\from{letltxmacro.dtx}{install}}%
  \file{letltxmacro.drv}{\from{letltxmacro.dtx}{driver}}%
  \usedir{tex/latex/letltxmacro}%
  \file{letltxmacro.sty}{\from{letltxmacro.dtx}{package}}%
  \usedir{doc/latex/letltxmacro}%
  \file{letltxmacro-showcases.tex}{\from{letltxmacro.dtx}{showcases}}%
%  \usedir{doc/latex/letltxmacro/test}%
%  \file{letltxmacro-test1.tex}{\from{letltxmacro.dtx}{test1}}%
%  \file{letltxmacro-test2.tex}{\from{letltxmacro.dtx}{test2}}%
}

\catcode32=13\relax% active space
\let =\space%
\Msg{************************************************************************}
\Msg{*}
\Msg{* To finish the installation you have to move the following}
\Msg{* file into a directory searched by TeX:}
\Msg{*}
\Msg{*     letltxmacro.sty}
\Msg{*}
\Msg{* To produce the documentation run the file `letltxmacro.drv'}
\Msg{* through LaTeX.}
\Msg{*}
\Msg{* Happy TeXing!}
\Msg{*}
\Msg{************************************************************************}

\endbatchfile
%</install>
%<*ignore>
\fi
%</ignore>
%<*driver>
\NeedsTeXFormat{LaTeX2e}
\ProvidesFile{letltxmacro.drv}%
  [2019/12/03 v1.6 Let assignment for LaTeX macros (HO)]%
\documentclass{ltxdoc}
\usepackage{holtxdoc}[2011/11/22]
\begin{document}
  \DocInput{letltxmacro.dtx}%
\end{document}
%</driver>
% \fi
%
%
%
% \GetFileInfo{letltxmacro.drv}
%
% \title{The \xpackage{letltxmacro} package}
% \date{2019/12/03 v1.6}
% \author{Heiko Oberdiek\thanks
% {Please report any issues at \url{https://github.com/ho-tex/letltxmacro/issues}}}
%
% \maketitle
%
% \begin{abstract}
% \TeX's \cs{let} assignment does not work for \LaTeX\ macros
% with optional arguments or for macros that are defined
% as robust macros by \cs{DeclareRobustCommand}. This package
% defines \cs{LetLtxMacro} that also takes care of the involved
% internal macros.
% \end{abstract}
%
% \tableofcontents
%
% \section{Documentation}
%
% If someone wants to redefine a macro with using the old
% meaning, then one method is \TeX's command \cs{let}:
%\begin{quote}
%\begin{verbatim}
%\newcommand{\Macro}{\typeout{Test Macro}}
%\let\SavedMacro=\Macro
%\renewcommand{\Macro}{%
%  \typeout{Begin}%
%  \SavedMacro
%  \typeout{End}%
%}
%\end{verbatim}
%\end{quote}
% However, this method fails, if \cs{Macro} is defined
% by \cs{DeclareRobustCommand} and/or has an optional argument.
% In both cases \LaTeX\ defines an additional internal macro
% that is forgotten in the simple \cs{let} assignment of
% the example above.
%
% \begin{declcs}{LetLtxMacro} \M{new macro} \M{old macro}
% \end{declcs}
% Macro \cs{LetLtxMacro} behaves similar to \TeX's \cs{let}
% assignment, but it takes care of macros that are
% defined by \cs{DeclareRobustCommand} and/or have optional
% arguments. Example:
%\begin{quote}
%\begin{verbatim}
%\DeclareRobustCommand{\Macro}[1][default]{...}
%\LetLtxMacro{\SavedMacro}{\Macro}
%\end{verbatim}
%\end{quote}
% Then macro \cs{SavedMacro} only uses internal macro names
% that are derived from \cs{SavedMacro}'s macro name. Macro \cs{Macro}
% can now be redefined without affecting \cs{SavedMacro}.
%
% \begin{declcs}{GlobalLetLtxMacro} \M{new macro} \M{old macro}
% \end{declcs}
% Like \cs{LetLtxMacro}, but the \meta{new macro} is defined globally.
% Since version 2019/12/03~v1.4.
%
% \subsection{Supported macro definition commands}
%
% \begin{quote}
%   \begin{tabular}{@{}ll@{}}
%     \cs{newcommand}, \cs{renewcommand} & latex/base\\
%     \cs{newenvironment}, \cs{renewenvironment} & latex/base\\
%     \cs{DeclareRobustCommand}& latex/base\\
%     \cs{newrobustcmd}, \cs{renewrobustcmd} & etoolbox\\
%     \cs{robustify} & etoolbox 2008/06/22 v1.6\\
%   \end{tabular}
% \end{quote}
%
% \StopEventually{
% }
%
% \section{Implementation}
%
% \subsection{Show cases}
%
% \subsubsection{\xfile{letltxmacro-showcases.tex}}
%
%    \begin{macrocode}
%<*showcases>
\NeedsTeXFormat{LaTeX2e}
\makeatletter
%    \end{macrocode}
%    \begin{macro}{\Line}
%    The result is displayed by macro \cs{Line}. The percent symbol
%    at line start allows easy grepping and inserting into the DTX
%    file.
%    \begin{macrocode}
\newcommand*{\Line}[1]{%
  \typeout{\@percentchar#1}%
}
%    \end{macrocode}
%    \end{macro}
%    \begin{macrocode}
\newcommand*{\ShowCmdName}[1]{%
  \@ifundefined{#1}{}{%
    \Line{%
      \space\space(\expandafter\string\csname#1\endcsname) = %
      (\expandafter\meaning\csname#1\endcsname)%
    }%
  }%
}
\newcommand*{\ShowCmds}[1]{%
  \ShowCmdName{#1}%
  \ShowCmdName{#1 }%
  \ShowCmdName{\\#1}%
  \ShowCmdName{\\#1 }%
}
\let\\\@backslashchar
%    \end{macrocode}
%    \begin{macro}{\ShowDef}
%    \begin{macrocode}
\newcommand*{\ShowDef}[2]{%
  \begingroup
    \Line{}%
    \newcommand*{\DefString}{#2}%
    \@onelevel@sanitize\DefString
    \Line{\DefString}%
    #2%
    \ShowCmds{#1}%
  \endgroup
}
%    \end{macrocode}
%    \end{macro}
%    \begin{macrocode}
\typeout{}
\Line{* LaTeX definitions:}
\ShowDef{cmd}{%
  \newcommand{\cmd}[2][default]{}%
}
\ShowDef{cmd}{%
  \DeclareRobustCommand{\cmd}{}%
}
\ShowDef{cmd}{%
  \DeclareRobustCommand{\cmd}[2][default]{}%
}
\typeout{}
%    \end{macrocode}
% The minimal version of package \xpackage{etoolbox} is 2008/06/12 v1.6a
% because it fixes \cs{robustify}.
%    \begin{macrocode}
\RequirePackage{etoolbox}[2008/06/12]%
\Line{}
\Line{* etoolbox's robust definitions:}
\ShowDef{cmd}{%
  \newrobustcmd{\cmd}{}%
}
\ShowDef{cmd}{%
  \newrobustcmd{\cmd}[2][default]{}%
}
\Line{}
\Line{* etoolbox's \string\robustify:}
\ShowDef{cmd}{%
  \newcommand{\cmd}[2][default]{} %
  \robustify{\cmd}%
}
\ShowDef{cmd}{%
  \DeclareRobustCommand{\cmd}{} %
  \robustify{\cmd}%
}
\ShowDef{cmd}{%
  \DeclareRobustCommand{\cmd}[2][default]{} %
  \robustify{\cmd}%
}
\typeout{}
\@@end
%</showcases>
%    \end{macrocode}
%
% \subsubsection{Result}
%
% \begingroup
%   \makeatletter
%   \let\org@verbatim\@verbatim
%   \def\@verbatim{^^A
%     \org@verbatim
%     \catcode`\~=\active
%   }^^A
%   \let~\textvisiblespace
%\begin{verbatim}
%* LaTeX definitions:
%
%\newcommand {\cmd }[2][default]{}
%  (\cmd) = (macro:->\@protected@testopt \cmd \\cmd {default})
%  (\\cmd) = (\long macro:[#1]#2->)
%
%\DeclareRobustCommand {\cmd }{}
%  (\cmd) = (macro:->\protect \cmd~ )
%  (\cmd~) = (\long macro:->)
%
%\DeclareRobustCommand {\cmd }[2][default]{}
%  (\cmd) = (macro:->\protect \cmd~ )
%  (\cmd~) = (macro:->\@protected@testopt \cmd~ \\cmd~ {default})
%  (\\cmd~) = (\long macro:[#1]#2->)
%
%* etoolbox's robust definitions:
%
%\newrobustcmd {\cmd }{}
%  (\cmd) = (\protected\long macro:->)
%
%\newrobustcmd {\cmd }[2][default]{}
%  (\cmd) = (\protected macro:->\@testopt \\cmd {default})
%  (\\cmd) = (\long macro:[#1]#2->)
%
%* etoolbox's \robustify:
%
%\newcommand {\cmd }[2][default]{} \robustify {\cmd }
%  (\cmd) = (\protected macro:->\@protected@testopt \cmd \\cmd {default})
%  (\\cmd) = (\long macro:[#1]#2->)
%
%\DeclareRobustCommand {\cmd }{} \robustify {\cmd }
%  (\cmd) = (\protected macro:->)
%
%\DeclareRobustCommand {\cmd }[2][default]{} \robustify {\cmd }
%  (\cmd) = (\protected macro:->\@protected@testopt \cmd~ \\cmd~ {default})
%  (\cmd~) = (macro:->\@protected@testopt \cmd~ \\cmd~ {default})
%  (\\cmd~) = (\long macro:[#1]#2->)
%\end{verbatim}
% \endgroup
%
% \subsection{Package}
%
%    \begin{macrocode}
%<*package>
%    \end{macrocode}
%
% \subsubsection{Catcodes and identification}
%
%    \begin{macrocode}
\begingroup\catcode61\catcode48\catcode32=10\relax%
  \catcode13=5 % ^^M
  \endlinechar=13 %
  \catcode123=1 % {
  \catcode125=2 % }
  \catcode64=11 % @
  \def\x{\endgroup
    \expandafter\edef\csname llm@AtEnd\endcsname{%
      \endlinechar=\the\endlinechar\relax
      \catcode13=\the\catcode13\relax
      \catcode32=\the\catcode32\relax
      \catcode35=\the\catcode35\relax
      \catcode61=\the\catcode61\relax
      \catcode64=\the\catcode64\relax
      \catcode123=\the\catcode123\relax
      \catcode125=\the\catcode125\relax
    }%
  }%
\x\catcode61\catcode48\catcode32=10\relax%
\catcode13=5 % ^^M
\endlinechar=13 %
\catcode35=6 % #
\catcode64=11 % @
\catcode123=1 % {
\catcode125=2 % }
\def\TMP@EnsureCode#1#2{%
  \edef\llm@AtEnd{%
    \llm@AtEnd
    \catcode#1=\the\catcode#1\relax
  }%
  \catcode#1=#2\relax
}
\TMP@EnsureCode{40}{12}% (
\TMP@EnsureCode{41}{12}% )
\TMP@EnsureCode{42}{12}% *
\TMP@EnsureCode{45}{12}% -
\TMP@EnsureCode{46}{12}% .
\TMP@EnsureCode{47}{12}% /
\TMP@EnsureCode{58}{12}% :
\TMP@EnsureCode{62}{12}% >
\TMP@EnsureCode{91}{12}% [
\TMP@EnsureCode{93}{12}% ]
\edef\llm@AtEnd{%
  \llm@AtEnd
  \escapechar\the\escapechar\relax
  \noexpand\endinput
}
\escapechar=92 % `\\
%    \end{macrocode}
%
%    Package identification.
%    \begin{macrocode}
\NeedsTeXFormat{LaTeX2e}
\ProvidesPackage{letltxmacro}%
  [2019/12/03 v1.6 Let assignment for LaTeX macros (HO)]
%    \end{macrocode}
%
% \subsubsection{Main macros}
%
%    \begin{macro}{\LetLtxMacro}
%    \begin{macrocode}
\newcommand*{\LetLtxMacro}{%
  \llm@ModeLetLtxMacro{}%
}
%    \end{macrocode}
%    \end{macro}
%    \begin{macro}{\GlobalLetLtxMacro}
%    \begin{macrocode}
\newcommand*{\GlobalLetLtxMacro}{%
  \llm@ModeLetLtxMacro\global
}
%    \end{macrocode}
%    \end{macro}
%
%    \begin{macro}{\llm@ModeLetLtxMacro}
%    \begin{macrocode}
\newcommand*{\llm@ModeLetLtxMacro}[3]{%
  \edef\llm@escapechar{\the\escapechar}%
  \escapechar=-1 %
  \edef\reserved@a{%
    \noexpand\protect
    \expandafter\noexpand
    \csname\string#3 \endcsname
  }%
  \ifx\reserved@a#3\relax
    #1\edef#2{%
      \noexpand\protect
      \expandafter\noexpand
      \csname\string#2 \endcsname
    }%
    #1\expandafter\let
    \csname\string#2 \expandafter\endcsname
    \csname\string#3 \endcsname
    \expandafter\llm@LetLtxMacro
        \csname\string#2 \expandafter\endcsname
        \csname\string#3 \endcsname{#1}%
  \else
    \llm@LetLtxMacro{#2}{#3}{#1}%
  \fi
  \escapechar=\llm@escapechar\relax
}
%    \end{macrocode}
%    \end{macro}
%    \begin{macro}{\llm@LetLtxMacro}
%    \begin{macrocode}
\def\llm@LetLtxMacro#1#2#3{%
  \escapechar=92 %
  \expandafter\llm@CheckParams\meaning#2:->\@nil{%
    \begingroup
      \def\@protected@testopt{%
        \expandafter\@testopt\@gobble
      }%
      \def\@testopt##1##2{%
        \toks@={##2}%
      }%
      \let\llm@testopt\@empty
      \edef\x{%
        \noexpand\@protected@testopt
        \noexpand#2%
        \expandafter\noexpand\csname\string#2\endcsname
      }%
      \expandafter\expandafter\expandafter\def
      \expandafter\expandafter\expandafter\y
      \expandafter\expandafter\expandafter{%
        \expandafter\llm@CarThree#2{}{}{}\llm@nil
      }%
      \ifx\x\y
        #2%
        \def\llm@testopt{%
          \noexpand\@protected@testopt
          \noexpand#1%
        }%
      \else
        \edef\x{%
          \noexpand\@testopt
          \expandafter\noexpand
          \csname\string#2\endcsname
        }%
        \expandafter\expandafter\expandafter\def
        \expandafter\expandafter\expandafter\y
        \expandafter\expandafter\expandafter{%
          \expandafter\llm@CarTwo#2{}{}\llm@nil
        }%
        \ifx\x\y
          #2%
          \def\llm@testopt{%
            \noexpand\@testopt
          }%
        \fi
      \fi
      \ifx\llm@testopt\@empty
      \else
        \llm@protected\xdef\llm@GlobalTemp{%
          \llm@testopt
          \expandafter\noexpand
          \csname\string#1\endcsname
          {\the\toks@}%
        }%
      \fi
    \expandafter\endgroup\ifx\llm@testopt\@empty
      #3\let#1=#2\relax
    \else
      #3\let#1=\llm@GlobalTemp
      #3\expandafter\let
          \csname\string#1\expandafter\endcsname
          \csname\string#2\endcsname
    \fi
  }{%
    #3\let#1=#2\relax
  }%
}
%    \end{macrocode}
%    \end{macro}
%    \begin{macro}{\llm@CheckParams}
%    \begin{macrocode}
\def\llm@CheckParams#1:->#2\@nil{%
  \begingroup
    \def\x{#1}%
  \ifx\x\llm@macro
    \endgroup
    \def\llm@protected{}%
    \expandafter\@firstoftwo
  \else
    \ifx\x\llm@protectedmacro
      \endgroup
      \def\llm@protected{\protected}%
      \expandafter\expandafter\expandafter\@firstoftwo
    \else
      \endgroup
      \expandafter\expandafter\expandafter\@secondoftwo
    \fi
  \fi
}
%    \end{macrocode}
%    \end{macro}
%    \begin{macro}{\llm@macro}
%    \begin{macrocode}
\def\llm@macro{macro}
\@onelevel@sanitize\llm@macro
%    \end{macrocode}
%    \end{macro}
%    \begin{macro}{\llm@protectedmacro}
%    \begin{macrocode}
\def\llm@protectedmacro{\protected macro}
\@onelevel@sanitize\llm@protectedmacro
%    \end{macrocode}
%    \end{macro}
%    \begin{macro}{\llm@CarThree}
%    \begin{macrocode}
\def\llm@CarThree#1#2#3#4\llm@nil{#1#2#3}%
%    \end{macrocode}
%    \end{macro}
%    \begin{macro}{\llm@CarTwo}
%    \begin{macrocode}
\def\llm@CarTwo#1#2#3\llm@nil{#1#2}%
%    \end{macrocode}
%    \end{macro}
%
%    \begin{macrocode}
\llm@AtEnd%
%</package>
%    \end{macrocode}
% \section{Installation}
%
% \subsection{Download}
%
% \paragraph{Package.} This package is available on
% CTAN\footnote{\CTANpkg{letltxmacro}}:
% \begin{description}
% \item[\CTAN{macros/latex/contrib/letltxmacro/letltxmacro.dtx}] The source file.
% \item[\CTAN{macros/latex/contrib/letltxmacro/letltxmacro.pdf}] Documentation.
% \end{description}
%
%
% \paragraph{Bundle.} All the packages of the bundle `letltxmacro'
% are also available in a TDS compliant ZIP archive. There
% the packages are already unpacked and the documentation files
% are generated. The files and directories obey the TDS standard.
% \begin{description}
% \item[\CTANinstall{install/macros/latex/contrib/letltxmacro.tds.zip}]
% \end{description}
% \emph{TDS} refers to the standard ``A Directory Structure
% for \TeX\ Files'' (\CTANpkg{tds}). Directories
% with \xfile{texmf} in their name are usually organized this way.
%
% \subsection{Bundle installation}
%
% \paragraph{Unpacking.} Unpack the \xfile{letltxmacro.tds.zip} in the
% TDS tree (also known as \xfile{texmf} tree) of your choice.
% Example (linux):
% \begin{quote}
%   |unzip letltxmacro.tds.zip -d ~/texmf|
% \end{quote}
%
% \subsection{Package installation}
%
% \paragraph{Unpacking.} The \xfile{.dtx} file is a self-extracting
% \docstrip\ archive. The files are extracted by running the
% \xfile{.dtx} through \plainTeX:
% \begin{quote}
%   \verb|tex letltxmacro.dtx|
% \end{quote}
%
% \paragraph{TDS.} Now the different files must be moved into
% the different directories in your installation TDS tree
% (also known as \xfile{texmf} tree):
% \begin{quote}
% \def\t{^^A
% \begin{tabular}{@{}>{\ttfamily}l@{ $\rightarrow$ }>{\ttfamily}l@{}}
%   letltxmacro.sty & tex/latex/letltxmacro/letltxmacro.sty\\
%   letltxmacro.pdf & doc/latex/letltxmacro/letltxmacro.pdf\\
%   letltxmacro-showcases.tex & doc/latex/letltxmacro/letltxmacro-showcases.tex\\
%   letltxmacro.dtx & source/latex/letltxmacro/letltxmacro.dtx\\
% \end{tabular}^^A
% }^^A
% \sbox0{\t}^^A
% \ifdim\wd0>\linewidth
%   \begingroup
%     \advance\linewidth by\leftmargin
%     \advance\linewidth by\rightmargin
%   \edef\x{\endgroup
%     \def\noexpand\lw{\the\linewidth}^^A
%   }\x
%   \def\lwbox{^^A
%     \leavevmode
%     \hbox to \linewidth{^^A
%       \kern-\leftmargin\relax
%       \hss
%       \usebox0
%       \hss
%       \kern-\rightmargin\relax
%     }^^A
%   }^^A
%   \ifdim\wd0>\lw
%     \sbox0{\small\t}^^A
%     \ifdim\wd0>\linewidth
%       \ifdim\wd0>\lw
%         \sbox0{\footnotesize\t}^^A
%         \ifdim\wd0>\linewidth
%           \ifdim\wd0>\lw
%             \sbox0{\scriptsize\t}^^A
%             \ifdim\wd0>\linewidth
%               \ifdim\wd0>\lw
%                 \sbox0{\tiny\t}^^A
%                 \ifdim\wd0>\linewidth
%                   \lwbox
%                 \else
%                   \usebox0
%                 \fi
%               \else
%                 \lwbox
%               \fi
%             \else
%               \usebox0
%             \fi
%           \else
%             \lwbox
%           \fi
%         \else
%           \usebox0
%         \fi
%       \else
%         \lwbox
%       \fi
%     \else
%       \usebox0
%     \fi
%   \else
%     \lwbox
%   \fi
% \else
%   \usebox0
% \fi
% \end{quote}
% If you have a \xfile{docstrip.cfg} that configures and enables \docstrip's
% TDS installing feature, then some files can already be in the right
% place, see the documentation of \docstrip.
%
% \subsection{Refresh file name databases}
%
% If your \TeX~distribution
% (\TeX\,Live, \mikTeX, \dots) relies on file name databases, you must refresh
% these. For example, \TeX\,Live\ users run \verb|texhash| or
% \verb|mktexlsr|.
%
% \subsection{Some details for the interested}
%
% \paragraph{Unpacking with \LaTeX.}
% The \xfile{.dtx} chooses its action depending on the format:
% \begin{description}
% \item[\plainTeX:] Run \docstrip\ and extract the files.
% \item[\LaTeX:] Generate the documentation.
% \end{description}
% If you insist on using \LaTeX\ for \docstrip\ (really,
% \docstrip\ does not need \LaTeX), then inform the autodetect routine
% about your intention:
% \begin{quote}
%   \verb|latex \let\install=y\input{letltxmacro.dtx}|
% \end{quote}
% Do not forget to quote the argument according to the demands
% of your shell.
%
% \paragraph{Generating the documentation.}
% You can use both the \xfile{.dtx} or the \xfile{.drv} to generate
% the documentation. The process can be configured by the
% configuration file \xfile{ltxdoc.cfg}. For instance, put this
% line into this file, if you want to have A4 as paper format:
% \begin{quote}
%   \verb|\PassOptionsToClass{a4paper}{article}|
% \end{quote}
% An example follows how to generate the
% documentation with pdf\LaTeX:
% \begin{quote}
%\begin{verbatim}
%pdflatex letltxmacro.dtx
%makeindex -s gind.ist letltxmacro.idx
%pdflatex letltxmacro.dtx
%makeindex -s gind.ist letltxmacro.idx
%pdflatex letltxmacro.dtx
%\end{verbatim}
% \end{quote}
%
% \begin{History}
%   \begin{Version}{2008/06/09 v1.0}
%   \item
%     First version.
%   \end{Version}
%   \begin{Version}{2008/06/12 v1.1}
%   \item
%     Support for \xpackage{etoolbox}'s \cs{newrobustcmd} added.
%   \end{Version}
%   \begin{Version}{2008/06/13 v1.2}
%   \item
%     Support for \xpackage{etoolbox}'s \cs{robustify} added.
%   \end{Version}
%   \begin{Version}{2008/06/24 v1.3}
%   \item
%     Test file adapted for etoolbox 2008/06/22 v1.6.
%   \end{Version}
%   \begin{Version}{2010/09/02 v1.4}
%   \item
%     \cs{GlobalLetLtxMacro} added.
%   \end{Version}
%   \begin{Version}{2016/05/16 v1.5}
%   \item
%     Documentation updates.
%   \end{Version}
%   \begin{Version}{2019/12/03 v1.6}
%   \item
%     Documentation updates.
%   \end{Version}
% \end{History}
%
% \PrintIndex
%
% \Finale
\endinput

%        (quote the arguments according to the demands of your shell)
%
% Documentation:
%    (a) If letltxmacro.drv is present:
%           latex letltxmacro.drv
%    (b) Without letltxmacro.drv:
%           latex letltxmacro.dtx; ...
%    The class ltxdoc loads the configuration file ltxdoc.cfg
%    if available. Here you can specify further options, e.g.
%    use A4 as paper format:
%       \PassOptionsToClass{a4paper}{article}
%
%    Programm calls to get the documentation (example):
%       pdflatex letltxmacro.dtx
%       makeindex -s gind.ist letltxmacro.idx
%       pdflatex letltxmacro.dtx
%       makeindex -s gind.ist letltxmacro.idx
%       pdflatex letltxmacro.dtx
%
% Installation:
%    TDS:tex/latex/letltxmacro/letltxmacro.sty
%    TDS:doc/latex/letltxmacro/letltxmacro.pdf
%    TDS:doc/latex/letltxmacro/letltxmacro-showcases.tex
%    TDS:source/latex/letltxmacro/letltxmacro.dtx
%
%<*ignore>
\begingroup
  \catcode123=1 %
  \catcode125=2 %
  \def\x{LaTeX2e}%
\expandafter\endgroup
\ifcase 0\ifx\install y1\fi\expandafter
         \ifx\csname processbatchFile\endcsname\relax\else1\fi
         \ifx\fmtname\x\else 1\fi\relax
\else\csname fi\endcsname
%</ignore>
%<*install>
\input docstrip.tex
\Msg{************************************************************************}
\Msg{* Installation}
\Msg{* Package: letltxmacro 2019/12/03 v1.6 Let assignment for LaTeX macros (HO)}
\Msg{************************************************************************}

\keepsilent
\askforoverwritefalse

\let\MetaPrefix\relax
\preamble

This is a generated file.

Project: letltxmacro
Version: 2019/12/03 v1.6

Copyright (C)
   2008, 2010 Heiko Oberdiek
   2016-2019 Oberdiek Package Support Group

This work may be distributed and/or modified under the
conditions of the LaTeX Project Public License, either
version 1.3c of this license or (at your option) any later
version. This version of this license is in
   https://www.latex-project.org/lppl/lppl-1-3c.txt
and the latest version of this license is in
   https://www.latex-project.org/lppl.txt
and version 1.3 or later is part of all distributions of
LaTeX version 2005/12/01 or later.

This work has the LPPL maintenance status "maintained".

The Current Maintainers of this work are
Heiko Oberdiek and the Oberdiek Package Support Group
https://github.com/ho-tex/letltxmacro/issues


This work consists of the main source file letltxmacro.dtx
and the derived files
   letltxmacro.sty, letltxmacro.pdf, letltxmacro.ins, letltxmacro.drv,
   letltxmacro-showcases.tex, letltxmacro-test1.tex,
   letltxmacro-test2.tex.

\endpreamble
\let\MetaPrefix\DoubleperCent

\generate{%
  \file{letltxmacro.ins}{\from{letltxmacro.dtx}{install}}%
  \file{letltxmacro.drv}{\from{letltxmacro.dtx}{driver}}%
  \usedir{tex/latex/letltxmacro}%
  \file{letltxmacro.sty}{\from{letltxmacro.dtx}{package}}%
  \usedir{doc/latex/letltxmacro}%
  \file{letltxmacro-showcases.tex}{\from{letltxmacro.dtx}{showcases}}%
%  \usedir{doc/latex/letltxmacro/test}%
%  \file{letltxmacro-test1.tex}{\from{letltxmacro.dtx}{test1}}%
%  \file{letltxmacro-test2.tex}{\from{letltxmacro.dtx}{test2}}%
}

\catcode32=13\relax% active space
\let =\space%
\Msg{************************************************************************}
\Msg{*}
\Msg{* To finish the installation you have to move the following}
\Msg{* file into a directory searched by TeX:}
\Msg{*}
\Msg{*     letltxmacro.sty}
\Msg{*}
\Msg{* To produce the documentation run the file `letltxmacro.drv'}
\Msg{* through LaTeX.}
\Msg{*}
\Msg{* Happy TeXing!}
\Msg{*}
\Msg{************************************************************************}

\endbatchfile
%</install>
%<*ignore>
\fi
%</ignore>
%<*driver>
\NeedsTeXFormat{LaTeX2e}
\ProvidesFile{letltxmacro.drv}%
  [2019/12/03 v1.6 Let assignment for LaTeX macros (HO)]%
\documentclass{ltxdoc}
\usepackage{holtxdoc}[2011/11/22]
\begin{document}
  \DocInput{letltxmacro.dtx}%
\end{document}
%</driver>
% \fi
%
%
%
% \GetFileInfo{letltxmacro.drv}
%
% \title{The \xpackage{letltxmacro} package}
% \date{2019/12/03 v1.6}
% \author{Heiko Oberdiek\thanks
% {Please report any issues at \url{https://github.com/ho-tex/letltxmacro/issues}}}
%
% \maketitle
%
% \begin{abstract}
% \TeX's \cs{let} assignment does not work for \LaTeX\ macros
% with optional arguments or for macros that are defined
% as robust macros by \cs{DeclareRobustCommand}. This package
% defines \cs{LetLtxMacro} that also takes care of the involved
% internal macros.
% \end{abstract}
%
% \tableofcontents
%
% \section{Documentation}
%
% If someone wants to redefine a macro with using the old
% meaning, then one method is \TeX's command \cs{let}:
%\begin{quote}
%\begin{verbatim}
%\newcommand{\Macro}{\typeout{Test Macro}}
%\let\SavedMacro=\Macro
%\renewcommand{\Macro}{%
%  \typeout{Begin}%
%  \SavedMacro
%  \typeout{End}%
%}
%\end{verbatim}
%\end{quote}
% However, this method fails, if \cs{Macro} is defined
% by \cs{DeclareRobustCommand} and/or has an optional argument.
% In both cases \LaTeX\ defines an additional internal macro
% that is forgotten in the simple \cs{let} assignment of
% the example above.
%
% \begin{declcs}{LetLtxMacro} \M{new macro} \M{old macro}
% \end{declcs}
% Macro \cs{LetLtxMacro} behaves similar to \TeX's \cs{let}
% assignment, but it takes care of macros that are
% defined by \cs{DeclareRobustCommand} and/or have optional
% arguments. Example:
%\begin{quote}
%\begin{verbatim}
%\DeclareRobustCommand{\Macro}[1][default]{...}
%\LetLtxMacro{\SavedMacro}{\Macro}
%\end{verbatim}
%\end{quote}
% Then macro \cs{SavedMacro} only uses internal macro names
% that are derived from \cs{SavedMacro}'s macro name. Macro \cs{Macro}
% can now be redefined without affecting \cs{SavedMacro}.
%
% \begin{declcs}{GlobalLetLtxMacro} \M{new macro} \M{old macro}
% \end{declcs}
% Like \cs{LetLtxMacro}, but the \meta{new macro} is defined globally.
% Since version 2019/12/03~v1.4.
%
% \subsection{Supported macro definition commands}
%
% \begin{quote}
%   \begin{tabular}{@{}ll@{}}
%     \cs{newcommand}, \cs{renewcommand} & latex/base\\
%     \cs{newenvironment}, \cs{renewenvironment} & latex/base\\
%     \cs{DeclareRobustCommand}& latex/base\\
%     \cs{newrobustcmd}, \cs{renewrobustcmd} & etoolbox\\
%     \cs{robustify} & etoolbox 2008/06/22 v1.6\\
%   \end{tabular}
% \end{quote}
%
% \StopEventually{
% }
%
% \section{Implementation}
%
% \subsection{Show cases}
%
% \subsubsection{\xfile{letltxmacro-showcases.tex}}
%
%    \begin{macrocode}
%<*showcases>
\NeedsTeXFormat{LaTeX2e}
\makeatletter
%    \end{macrocode}
%    \begin{macro}{\Line}
%    The result is displayed by macro \cs{Line}. The percent symbol
%    at line start allows easy grepping and inserting into the DTX
%    file.
%    \begin{macrocode}
\newcommand*{\Line}[1]{%
  \typeout{\@percentchar#1}%
}
%    \end{macrocode}
%    \end{macro}
%    \begin{macrocode}
\newcommand*{\ShowCmdName}[1]{%
  \@ifundefined{#1}{}{%
    \Line{%
      \space\space(\expandafter\string\csname#1\endcsname) = %
      (\expandafter\meaning\csname#1\endcsname)%
    }%
  }%
}
\newcommand*{\ShowCmds}[1]{%
  \ShowCmdName{#1}%
  \ShowCmdName{#1 }%
  \ShowCmdName{\\#1}%
  \ShowCmdName{\\#1 }%
}
\let\\\@backslashchar
%    \end{macrocode}
%    \begin{macro}{\ShowDef}
%    \begin{macrocode}
\newcommand*{\ShowDef}[2]{%
  \begingroup
    \Line{}%
    \newcommand*{\DefString}{#2}%
    \@onelevel@sanitize\DefString
    \Line{\DefString}%
    #2%
    \ShowCmds{#1}%
  \endgroup
}
%    \end{macrocode}
%    \end{macro}
%    \begin{macrocode}
\typeout{}
\Line{* LaTeX definitions:}
\ShowDef{cmd}{%
  \newcommand{\cmd}[2][default]{}%
}
\ShowDef{cmd}{%
  \DeclareRobustCommand{\cmd}{}%
}
\ShowDef{cmd}{%
  \DeclareRobustCommand{\cmd}[2][default]{}%
}
\typeout{}
%    \end{macrocode}
% The minimal version of package \xpackage{etoolbox} is 2008/06/12 v1.6a
% because it fixes \cs{robustify}.
%    \begin{macrocode}
\RequirePackage{etoolbox}[2008/06/12]%
\Line{}
\Line{* etoolbox's robust definitions:}
\ShowDef{cmd}{%
  \newrobustcmd{\cmd}{}%
}
\ShowDef{cmd}{%
  \newrobustcmd{\cmd}[2][default]{}%
}
\Line{}
\Line{* etoolbox's \string\robustify:}
\ShowDef{cmd}{%
  \newcommand{\cmd}[2][default]{} %
  \robustify{\cmd}%
}
\ShowDef{cmd}{%
  \DeclareRobustCommand{\cmd}{} %
  \robustify{\cmd}%
}
\ShowDef{cmd}{%
  \DeclareRobustCommand{\cmd}[2][default]{} %
  \robustify{\cmd}%
}
\typeout{}
\@@end
%</showcases>
%    \end{macrocode}
%
% \subsubsection{Result}
%
% \begingroup
%   \makeatletter
%   \let\org@verbatim\@verbatim
%   \def\@verbatim{^^A
%     \org@verbatim
%     \catcode`\~=\active
%   }^^A
%   \let~\textvisiblespace
%\begin{verbatim}
%* LaTeX definitions:
%
%\newcommand {\cmd }[2][default]{}
%  (\cmd) = (macro:->\@protected@testopt \cmd \\cmd {default})
%  (\\cmd) = (\long macro:[#1]#2->)
%
%\DeclareRobustCommand {\cmd }{}
%  (\cmd) = (macro:->\protect \cmd~ )
%  (\cmd~) = (\long macro:->)
%
%\DeclareRobustCommand {\cmd }[2][default]{}
%  (\cmd) = (macro:->\protect \cmd~ )
%  (\cmd~) = (macro:->\@protected@testopt \cmd~ \\cmd~ {default})
%  (\\cmd~) = (\long macro:[#1]#2->)
%
%* etoolbox's robust definitions:
%
%\newrobustcmd {\cmd }{}
%  (\cmd) = (\protected\long macro:->)
%
%\newrobustcmd {\cmd }[2][default]{}
%  (\cmd) = (\protected macro:->\@testopt \\cmd {default})
%  (\\cmd) = (\long macro:[#1]#2->)
%
%* etoolbox's \robustify:
%
%\newcommand {\cmd }[2][default]{} \robustify {\cmd }
%  (\cmd) = (\protected macro:->\@protected@testopt \cmd \\cmd {default})
%  (\\cmd) = (\long macro:[#1]#2->)
%
%\DeclareRobustCommand {\cmd }{} \robustify {\cmd }
%  (\cmd) = (\protected macro:->)
%
%\DeclareRobustCommand {\cmd }[2][default]{} \robustify {\cmd }
%  (\cmd) = (\protected macro:->\@protected@testopt \cmd~ \\cmd~ {default})
%  (\cmd~) = (macro:->\@protected@testopt \cmd~ \\cmd~ {default})
%  (\\cmd~) = (\long macro:[#1]#2->)
%\end{verbatim}
% \endgroup
%
% \subsection{Package}
%
%    \begin{macrocode}
%<*package>
%    \end{macrocode}
%
% \subsubsection{Catcodes and identification}
%
%    \begin{macrocode}
\begingroup\catcode61\catcode48\catcode32=10\relax%
  \catcode13=5 % ^^M
  \endlinechar=13 %
  \catcode123=1 % {
  \catcode125=2 % }
  \catcode64=11 % @
  \def\x{\endgroup
    \expandafter\edef\csname llm@AtEnd\endcsname{%
      \endlinechar=\the\endlinechar\relax
      \catcode13=\the\catcode13\relax
      \catcode32=\the\catcode32\relax
      \catcode35=\the\catcode35\relax
      \catcode61=\the\catcode61\relax
      \catcode64=\the\catcode64\relax
      \catcode123=\the\catcode123\relax
      \catcode125=\the\catcode125\relax
    }%
  }%
\x\catcode61\catcode48\catcode32=10\relax%
\catcode13=5 % ^^M
\endlinechar=13 %
\catcode35=6 % #
\catcode64=11 % @
\catcode123=1 % {
\catcode125=2 % }
\def\TMP@EnsureCode#1#2{%
  \edef\llm@AtEnd{%
    \llm@AtEnd
    \catcode#1=\the\catcode#1\relax
  }%
  \catcode#1=#2\relax
}
\TMP@EnsureCode{40}{12}% (
\TMP@EnsureCode{41}{12}% )
\TMP@EnsureCode{42}{12}% *
\TMP@EnsureCode{45}{12}% -
\TMP@EnsureCode{46}{12}% .
\TMP@EnsureCode{47}{12}% /
\TMP@EnsureCode{58}{12}% :
\TMP@EnsureCode{62}{12}% >
\TMP@EnsureCode{91}{12}% [
\TMP@EnsureCode{93}{12}% ]
\edef\llm@AtEnd{%
  \llm@AtEnd
  \escapechar\the\escapechar\relax
  \noexpand\endinput
}
\escapechar=92 % `\\
%    \end{macrocode}
%
%    Package identification.
%    \begin{macrocode}
\NeedsTeXFormat{LaTeX2e}
\ProvidesPackage{letltxmacro}%
  [2019/12/03 v1.6 Let assignment for LaTeX macros (HO)]
%    \end{macrocode}
%
% \subsubsection{Main macros}
%
%    \begin{macro}{\LetLtxMacro}
%    \begin{macrocode}
\newcommand*{\LetLtxMacro}{%
  \llm@ModeLetLtxMacro{}%
}
%    \end{macrocode}
%    \end{macro}
%    \begin{macro}{\GlobalLetLtxMacro}
%    \begin{macrocode}
\newcommand*{\GlobalLetLtxMacro}{%
  \llm@ModeLetLtxMacro\global
}
%    \end{macrocode}
%    \end{macro}
%
%    \begin{macro}{\llm@ModeLetLtxMacro}
%    \begin{macrocode}
\newcommand*{\llm@ModeLetLtxMacro}[3]{%
  \edef\llm@escapechar{\the\escapechar}%
  \escapechar=-1 %
  \edef\reserved@a{%
    \noexpand\protect
    \expandafter\noexpand
    \csname\string#3 \endcsname
  }%
  \ifx\reserved@a#3\relax
    #1\edef#2{%
      \noexpand\protect
      \expandafter\noexpand
      \csname\string#2 \endcsname
    }%
    #1\expandafter\let
    \csname\string#2 \expandafter\endcsname
    \csname\string#3 \endcsname
    \expandafter\llm@LetLtxMacro
        \csname\string#2 \expandafter\endcsname
        \csname\string#3 \endcsname{#1}%
  \else
    \llm@LetLtxMacro{#2}{#3}{#1}%
  \fi
  \escapechar=\llm@escapechar\relax
}
%    \end{macrocode}
%    \end{macro}
%    \begin{macro}{\llm@LetLtxMacro}
%    \begin{macrocode}
\def\llm@LetLtxMacro#1#2#3{%
  \escapechar=92 %
  \expandafter\llm@CheckParams\meaning#2:->\@nil{%
    \begingroup
      \def\@protected@testopt{%
        \expandafter\@testopt\@gobble
      }%
      \def\@testopt##1##2{%
        \toks@={##2}%
      }%
      \let\llm@testopt\@empty
      \edef\x{%
        \noexpand\@protected@testopt
        \noexpand#2%
        \expandafter\noexpand\csname\string#2\endcsname
      }%
      \expandafter\expandafter\expandafter\def
      \expandafter\expandafter\expandafter\y
      \expandafter\expandafter\expandafter{%
        \expandafter\llm@CarThree#2{}{}{}\llm@nil
      }%
      \ifx\x\y
        #2%
        \def\llm@testopt{%
          \noexpand\@protected@testopt
          \noexpand#1%
        }%
      \else
        \edef\x{%
          \noexpand\@testopt
          \expandafter\noexpand
          \csname\string#2\endcsname
        }%
        \expandafter\expandafter\expandafter\def
        \expandafter\expandafter\expandafter\y
        \expandafter\expandafter\expandafter{%
          \expandafter\llm@CarTwo#2{}{}\llm@nil
        }%
        \ifx\x\y
          #2%
          \def\llm@testopt{%
            \noexpand\@testopt
          }%
        \fi
      \fi
      \ifx\llm@testopt\@empty
      \else
        \llm@protected\xdef\llm@GlobalTemp{%
          \llm@testopt
          \expandafter\noexpand
          \csname\string#1\endcsname
          {\the\toks@}%
        }%
      \fi
    \expandafter\endgroup\ifx\llm@testopt\@empty
      #3\let#1=#2\relax
    \else
      #3\let#1=\llm@GlobalTemp
      #3\expandafter\let
          \csname\string#1\expandafter\endcsname
          \csname\string#2\endcsname
    \fi
  }{%
    #3\let#1=#2\relax
  }%
}
%    \end{macrocode}
%    \end{macro}
%    \begin{macro}{\llm@CheckParams}
%    \begin{macrocode}
\def\llm@CheckParams#1:->#2\@nil{%
  \begingroup
    \def\x{#1}%
  \ifx\x\llm@macro
    \endgroup
    \def\llm@protected{}%
    \expandafter\@firstoftwo
  \else
    \ifx\x\llm@protectedmacro
      \endgroup
      \def\llm@protected{\protected}%
      \expandafter\expandafter\expandafter\@firstoftwo
    \else
      \endgroup
      \expandafter\expandafter\expandafter\@secondoftwo
    \fi
  \fi
}
%    \end{macrocode}
%    \end{macro}
%    \begin{macro}{\llm@macro}
%    \begin{macrocode}
\def\llm@macro{macro}
\@onelevel@sanitize\llm@macro
%    \end{macrocode}
%    \end{macro}
%    \begin{macro}{\llm@protectedmacro}
%    \begin{macrocode}
\def\llm@protectedmacro{\protected macro}
\@onelevel@sanitize\llm@protectedmacro
%    \end{macrocode}
%    \end{macro}
%    \begin{macro}{\llm@CarThree}
%    \begin{macrocode}
\def\llm@CarThree#1#2#3#4\llm@nil{#1#2#3}%
%    \end{macrocode}
%    \end{macro}
%    \begin{macro}{\llm@CarTwo}
%    \begin{macrocode}
\def\llm@CarTwo#1#2#3\llm@nil{#1#2}%
%    \end{macrocode}
%    \end{macro}
%
%    \begin{macrocode}
\llm@AtEnd%
%</package>
%    \end{macrocode}
% \section{Installation}
%
% \subsection{Download}
%
% \paragraph{Package.} This package is available on
% CTAN\footnote{\CTANpkg{letltxmacro}}:
% \begin{description}
% \item[\CTAN{macros/latex/contrib/letltxmacro/letltxmacro.dtx}] The source file.
% \item[\CTAN{macros/latex/contrib/letltxmacro/letltxmacro.pdf}] Documentation.
% \end{description}
%
%
% \paragraph{Bundle.} All the packages of the bundle `letltxmacro'
% are also available in a TDS compliant ZIP archive. There
% the packages are already unpacked and the documentation files
% are generated. The files and directories obey the TDS standard.
% \begin{description}
% \item[\CTANinstall{install/macros/latex/contrib/letltxmacro.tds.zip}]
% \end{description}
% \emph{TDS} refers to the standard ``A Directory Structure
% for \TeX\ Files'' (\CTANpkg{tds}). Directories
% with \xfile{texmf} in their name are usually organized this way.
%
% \subsection{Bundle installation}
%
% \paragraph{Unpacking.} Unpack the \xfile{letltxmacro.tds.zip} in the
% TDS tree (also known as \xfile{texmf} tree) of your choice.
% Example (linux):
% \begin{quote}
%   |unzip letltxmacro.tds.zip -d ~/texmf|
% \end{quote}
%
% \subsection{Package installation}
%
% \paragraph{Unpacking.} The \xfile{.dtx} file is a self-extracting
% \docstrip\ archive. The files are extracted by running the
% \xfile{.dtx} through \plainTeX:
% \begin{quote}
%   \verb|tex letltxmacro.dtx|
% \end{quote}
%
% \paragraph{TDS.} Now the different files must be moved into
% the different directories in your installation TDS tree
% (also known as \xfile{texmf} tree):
% \begin{quote}
% \def\t{^^A
% \begin{tabular}{@{}>{\ttfamily}l@{ $\rightarrow$ }>{\ttfamily}l@{}}
%   letltxmacro.sty & tex/latex/letltxmacro/letltxmacro.sty\\
%   letltxmacro.pdf & doc/latex/letltxmacro/letltxmacro.pdf\\
%   letltxmacro-showcases.tex & doc/latex/letltxmacro/letltxmacro-showcases.tex\\
%   letltxmacro.dtx & source/latex/letltxmacro/letltxmacro.dtx\\
% \end{tabular}^^A
% }^^A
% \sbox0{\t}^^A
% \ifdim\wd0>\linewidth
%   \begingroup
%     \advance\linewidth by\leftmargin
%     \advance\linewidth by\rightmargin
%   \edef\x{\endgroup
%     \def\noexpand\lw{\the\linewidth}^^A
%   }\x
%   \def\lwbox{^^A
%     \leavevmode
%     \hbox to \linewidth{^^A
%       \kern-\leftmargin\relax
%       \hss
%       \usebox0
%       \hss
%       \kern-\rightmargin\relax
%     }^^A
%   }^^A
%   \ifdim\wd0>\lw
%     \sbox0{\small\t}^^A
%     \ifdim\wd0>\linewidth
%       \ifdim\wd0>\lw
%         \sbox0{\footnotesize\t}^^A
%         \ifdim\wd0>\linewidth
%           \ifdim\wd0>\lw
%             \sbox0{\scriptsize\t}^^A
%             \ifdim\wd0>\linewidth
%               \ifdim\wd0>\lw
%                 \sbox0{\tiny\t}^^A
%                 \ifdim\wd0>\linewidth
%                   \lwbox
%                 \else
%                   \usebox0
%                 \fi
%               \else
%                 \lwbox
%               \fi
%             \else
%               \usebox0
%             \fi
%           \else
%             \lwbox
%           \fi
%         \else
%           \usebox0
%         \fi
%       \else
%         \lwbox
%       \fi
%     \else
%       \usebox0
%     \fi
%   \else
%     \lwbox
%   \fi
% \else
%   \usebox0
% \fi
% \end{quote}
% If you have a \xfile{docstrip.cfg} that configures and enables \docstrip's
% TDS installing feature, then some files can already be in the right
% place, see the documentation of \docstrip.
%
% \subsection{Refresh file name databases}
%
% If your \TeX~distribution
% (\TeX\,Live, \mikTeX, \dots) relies on file name databases, you must refresh
% these. For example, \TeX\,Live\ users run \verb|texhash| or
% \verb|mktexlsr|.
%
% \subsection{Some details for the interested}
%
% \paragraph{Unpacking with \LaTeX.}
% The \xfile{.dtx} chooses its action depending on the format:
% \begin{description}
% \item[\plainTeX:] Run \docstrip\ and extract the files.
% \item[\LaTeX:] Generate the documentation.
% \end{description}
% If you insist on using \LaTeX\ for \docstrip\ (really,
% \docstrip\ does not need \LaTeX), then inform the autodetect routine
% about your intention:
% \begin{quote}
%   \verb|latex \let\install=y% \iffalse meta-comment
%
% File: letltxmacro.dtx
% Version: 2019/12/03 v1.6
% Info: Let assignment for LaTeX macros
%
% Copyright (C)
%    2008, 2010 Heiko Oberdiek
%    2016-2019 Oberdiek Package Support Group
%    https://github.com/ho-tex/letltxmacro/issues
%
% This work may be distributed and/or modified under the
% conditions of the LaTeX Project Public License, either
% version 1.3c of this license or (at your option) any later
% version. This version of this license is in
%    https://www.latex-project.org/lppl/lppl-1-3c.txt
% and the latest version of this license is in
%    https://www.latex-project.org/lppl.txt
% and version 1.3 or later is part of all distributions of
% LaTeX version 2005/12/01 or later.
%
% This work has the LPPL maintenance status "maintained".
%
% The Current Maintainers of this work are
% Heiko Oberdiek and the Oberdiek Package Support Group
% https://github.com/ho-tex/letltxmacro/issues
%
% This work consists of the main source file letltxmacro.dtx
% and the derived files
%    letltxmacro.sty, letltxmacro.pdf, letltxmacro.ins, letltxmacro.drv,
%    letltxmacro-showcases.tex, letltxmacro-test1.tex,
%    letltxmacro-test2.tex.
%
% Distribution:
%    CTAN:macros/latex/contrib/letltxmacro/letltxmacro.dtx
%    CTAN:macros/latex/contrib/letltxmacro/letltxmacro.pdf
%
% Unpacking:
%    (a) If letltxmacro.ins is present:
%           tex letltxmacro.ins
%    (b) Without letltxmacro.ins:
%           tex letltxmacro.dtx
%    (c) If you insist on using LaTeX
%           latex \let\install=y\input{letltxmacro.dtx}
%        (quote the arguments according to the demands of your shell)
%
% Documentation:
%    (a) If letltxmacro.drv is present:
%           latex letltxmacro.drv
%    (b) Without letltxmacro.drv:
%           latex letltxmacro.dtx; ...
%    The class ltxdoc loads the configuration file ltxdoc.cfg
%    if available. Here you can specify further options, e.g.
%    use A4 as paper format:
%       \PassOptionsToClass{a4paper}{article}
%
%    Programm calls to get the documentation (example):
%       pdflatex letltxmacro.dtx
%       makeindex -s gind.ist letltxmacro.idx
%       pdflatex letltxmacro.dtx
%       makeindex -s gind.ist letltxmacro.idx
%       pdflatex letltxmacro.dtx
%
% Installation:
%    TDS:tex/latex/letltxmacro/letltxmacro.sty
%    TDS:doc/latex/letltxmacro/letltxmacro.pdf
%    TDS:doc/latex/letltxmacro/letltxmacro-showcases.tex
%    TDS:source/latex/letltxmacro/letltxmacro.dtx
%
%<*ignore>
\begingroup
  \catcode123=1 %
  \catcode125=2 %
  \def\x{LaTeX2e}%
\expandafter\endgroup
\ifcase 0\ifx\install y1\fi\expandafter
         \ifx\csname processbatchFile\endcsname\relax\else1\fi
         \ifx\fmtname\x\else 1\fi\relax
\else\csname fi\endcsname
%</ignore>
%<*install>
\input docstrip.tex
\Msg{************************************************************************}
\Msg{* Installation}
\Msg{* Package: letltxmacro 2019/12/03 v1.6 Let assignment for LaTeX macros (HO)}
\Msg{************************************************************************}

\keepsilent
\askforoverwritefalse

\let\MetaPrefix\relax
\preamble

This is a generated file.

Project: letltxmacro
Version: 2019/12/03 v1.6

Copyright (C)
   2008, 2010 Heiko Oberdiek
   2016-2019 Oberdiek Package Support Group

This work may be distributed and/or modified under the
conditions of the LaTeX Project Public License, either
version 1.3c of this license or (at your option) any later
version. This version of this license is in
   https://www.latex-project.org/lppl/lppl-1-3c.txt
and the latest version of this license is in
   https://www.latex-project.org/lppl.txt
and version 1.3 or later is part of all distributions of
LaTeX version 2005/12/01 or later.

This work has the LPPL maintenance status "maintained".

The Current Maintainers of this work are
Heiko Oberdiek and the Oberdiek Package Support Group
https://github.com/ho-tex/letltxmacro/issues


This work consists of the main source file letltxmacro.dtx
and the derived files
   letltxmacro.sty, letltxmacro.pdf, letltxmacro.ins, letltxmacro.drv,
   letltxmacro-showcases.tex, letltxmacro-test1.tex,
   letltxmacro-test2.tex.

\endpreamble
\let\MetaPrefix\DoubleperCent

\generate{%
  \file{letltxmacro.ins}{\from{letltxmacro.dtx}{install}}%
  \file{letltxmacro.drv}{\from{letltxmacro.dtx}{driver}}%
  \usedir{tex/latex/letltxmacro}%
  \file{letltxmacro.sty}{\from{letltxmacro.dtx}{package}}%
  \usedir{doc/latex/letltxmacro}%
  \file{letltxmacro-showcases.tex}{\from{letltxmacro.dtx}{showcases}}%
%  \usedir{doc/latex/letltxmacro/test}%
%  \file{letltxmacro-test1.tex}{\from{letltxmacro.dtx}{test1}}%
%  \file{letltxmacro-test2.tex}{\from{letltxmacro.dtx}{test2}}%
}

\catcode32=13\relax% active space
\let =\space%
\Msg{************************************************************************}
\Msg{*}
\Msg{* To finish the installation you have to move the following}
\Msg{* file into a directory searched by TeX:}
\Msg{*}
\Msg{*     letltxmacro.sty}
\Msg{*}
\Msg{* To produce the documentation run the file `letltxmacro.drv'}
\Msg{* through LaTeX.}
\Msg{*}
\Msg{* Happy TeXing!}
\Msg{*}
\Msg{************************************************************************}

\endbatchfile
%</install>
%<*ignore>
\fi
%</ignore>
%<*driver>
\NeedsTeXFormat{LaTeX2e}
\ProvidesFile{letltxmacro.drv}%
  [2019/12/03 v1.6 Let assignment for LaTeX macros (HO)]%
\documentclass{ltxdoc}
\usepackage{holtxdoc}[2011/11/22]
\begin{document}
  \DocInput{letltxmacro.dtx}%
\end{document}
%</driver>
% \fi
%
%
%
% \GetFileInfo{letltxmacro.drv}
%
% \title{The \xpackage{letltxmacro} package}
% \date{2019/12/03 v1.6}
% \author{Heiko Oberdiek\thanks
% {Please report any issues at \url{https://github.com/ho-tex/letltxmacro/issues}}}
%
% \maketitle
%
% \begin{abstract}
% \TeX's \cs{let} assignment does not work for \LaTeX\ macros
% with optional arguments or for macros that are defined
% as robust macros by \cs{DeclareRobustCommand}. This package
% defines \cs{LetLtxMacro} that also takes care of the involved
% internal macros.
% \end{abstract}
%
% \tableofcontents
%
% \section{Documentation}
%
% If someone wants to redefine a macro with using the old
% meaning, then one method is \TeX's command \cs{let}:
%\begin{quote}
%\begin{verbatim}
%\newcommand{\Macro}{\typeout{Test Macro}}
%\let\SavedMacro=\Macro
%\renewcommand{\Macro}{%
%  \typeout{Begin}%
%  \SavedMacro
%  \typeout{End}%
%}
%\end{verbatim}
%\end{quote}
% However, this method fails, if \cs{Macro} is defined
% by \cs{DeclareRobustCommand} and/or has an optional argument.
% In both cases \LaTeX\ defines an additional internal macro
% that is forgotten in the simple \cs{let} assignment of
% the example above.
%
% \begin{declcs}{LetLtxMacro} \M{new macro} \M{old macro}
% \end{declcs}
% Macro \cs{LetLtxMacro} behaves similar to \TeX's \cs{let}
% assignment, but it takes care of macros that are
% defined by \cs{DeclareRobustCommand} and/or have optional
% arguments. Example:
%\begin{quote}
%\begin{verbatim}
%\DeclareRobustCommand{\Macro}[1][default]{...}
%\LetLtxMacro{\SavedMacro}{\Macro}
%\end{verbatim}
%\end{quote}
% Then macro \cs{SavedMacro} only uses internal macro names
% that are derived from \cs{SavedMacro}'s macro name. Macro \cs{Macro}
% can now be redefined without affecting \cs{SavedMacro}.
%
% \begin{declcs}{GlobalLetLtxMacro} \M{new macro} \M{old macro}
% \end{declcs}
% Like \cs{LetLtxMacro}, but the \meta{new macro} is defined globally.
% Since version 2019/12/03~v1.4.
%
% \subsection{Supported macro definition commands}
%
% \begin{quote}
%   \begin{tabular}{@{}ll@{}}
%     \cs{newcommand}, \cs{renewcommand} & latex/base\\
%     \cs{newenvironment}, \cs{renewenvironment} & latex/base\\
%     \cs{DeclareRobustCommand}& latex/base\\
%     \cs{newrobustcmd}, \cs{renewrobustcmd} & etoolbox\\
%     \cs{robustify} & etoolbox 2008/06/22 v1.6\\
%   \end{tabular}
% \end{quote}
%
% \StopEventually{
% }
%
% \section{Implementation}
%
% \subsection{Show cases}
%
% \subsubsection{\xfile{letltxmacro-showcases.tex}}
%
%    \begin{macrocode}
%<*showcases>
\NeedsTeXFormat{LaTeX2e}
\makeatletter
%    \end{macrocode}
%    \begin{macro}{\Line}
%    The result is displayed by macro \cs{Line}. The percent symbol
%    at line start allows easy grepping and inserting into the DTX
%    file.
%    \begin{macrocode}
\newcommand*{\Line}[1]{%
  \typeout{\@percentchar#1}%
}
%    \end{macrocode}
%    \end{macro}
%    \begin{macrocode}
\newcommand*{\ShowCmdName}[1]{%
  \@ifundefined{#1}{}{%
    \Line{%
      \space\space(\expandafter\string\csname#1\endcsname) = %
      (\expandafter\meaning\csname#1\endcsname)%
    }%
  }%
}
\newcommand*{\ShowCmds}[1]{%
  \ShowCmdName{#1}%
  \ShowCmdName{#1 }%
  \ShowCmdName{\\#1}%
  \ShowCmdName{\\#1 }%
}
\let\\\@backslashchar
%    \end{macrocode}
%    \begin{macro}{\ShowDef}
%    \begin{macrocode}
\newcommand*{\ShowDef}[2]{%
  \begingroup
    \Line{}%
    \newcommand*{\DefString}{#2}%
    \@onelevel@sanitize\DefString
    \Line{\DefString}%
    #2%
    \ShowCmds{#1}%
  \endgroup
}
%    \end{macrocode}
%    \end{macro}
%    \begin{macrocode}
\typeout{}
\Line{* LaTeX definitions:}
\ShowDef{cmd}{%
  \newcommand{\cmd}[2][default]{}%
}
\ShowDef{cmd}{%
  \DeclareRobustCommand{\cmd}{}%
}
\ShowDef{cmd}{%
  \DeclareRobustCommand{\cmd}[2][default]{}%
}
\typeout{}
%    \end{macrocode}
% The minimal version of package \xpackage{etoolbox} is 2008/06/12 v1.6a
% because it fixes \cs{robustify}.
%    \begin{macrocode}
\RequirePackage{etoolbox}[2008/06/12]%
\Line{}
\Line{* etoolbox's robust definitions:}
\ShowDef{cmd}{%
  \newrobustcmd{\cmd}{}%
}
\ShowDef{cmd}{%
  \newrobustcmd{\cmd}[2][default]{}%
}
\Line{}
\Line{* etoolbox's \string\robustify:}
\ShowDef{cmd}{%
  \newcommand{\cmd}[2][default]{} %
  \robustify{\cmd}%
}
\ShowDef{cmd}{%
  \DeclareRobustCommand{\cmd}{} %
  \robustify{\cmd}%
}
\ShowDef{cmd}{%
  \DeclareRobustCommand{\cmd}[2][default]{} %
  \robustify{\cmd}%
}
\typeout{}
\@@end
%</showcases>
%    \end{macrocode}
%
% \subsubsection{Result}
%
% \begingroup
%   \makeatletter
%   \let\org@verbatim\@verbatim
%   \def\@verbatim{^^A
%     \org@verbatim
%     \catcode`\~=\active
%   }^^A
%   \let~\textvisiblespace
%\begin{verbatim}
%* LaTeX definitions:
%
%\newcommand {\cmd }[2][default]{}
%  (\cmd) = (macro:->\@protected@testopt \cmd \\cmd {default})
%  (\\cmd) = (\long macro:[#1]#2->)
%
%\DeclareRobustCommand {\cmd }{}
%  (\cmd) = (macro:->\protect \cmd~ )
%  (\cmd~) = (\long macro:->)
%
%\DeclareRobustCommand {\cmd }[2][default]{}
%  (\cmd) = (macro:->\protect \cmd~ )
%  (\cmd~) = (macro:->\@protected@testopt \cmd~ \\cmd~ {default})
%  (\\cmd~) = (\long macro:[#1]#2->)
%
%* etoolbox's robust definitions:
%
%\newrobustcmd {\cmd }{}
%  (\cmd) = (\protected\long macro:->)
%
%\newrobustcmd {\cmd }[2][default]{}
%  (\cmd) = (\protected macro:->\@testopt \\cmd {default})
%  (\\cmd) = (\long macro:[#1]#2->)
%
%* etoolbox's \robustify:
%
%\newcommand {\cmd }[2][default]{} \robustify {\cmd }
%  (\cmd) = (\protected macro:->\@protected@testopt \cmd \\cmd {default})
%  (\\cmd) = (\long macro:[#1]#2->)
%
%\DeclareRobustCommand {\cmd }{} \robustify {\cmd }
%  (\cmd) = (\protected macro:->)
%
%\DeclareRobustCommand {\cmd }[2][default]{} \robustify {\cmd }
%  (\cmd) = (\protected macro:->\@protected@testopt \cmd~ \\cmd~ {default})
%  (\cmd~) = (macro:->\@protected@testopt \cmd~ \\cmd~ {default})
%  (\\cmd~) = (\long macro:[#1]#2->)
%\end{verbatim}
% \endgroup
%
% \subsection{Package}
%
%    \begin{macrocode}
%<*package>
%    \end{macrocode}
%
% \subsubsection{Catcodes and identification}
%
%    \begin{macrocode}
\begingroup\catcode61\catcode48\catcode32=10\relax%
  \catcode13=5 % ^^M
  \endlinechar=13 %
  \catcode123=1 % {
  \catcode125=2 % }
  \catcode64=11 % @
  \def\x{\endgroup
    \expandafter\edef\csname llm@AtEnd\endcsname{%
      \endlinechar=\the\endlinechar\relax
      \catcode13=\the\catcode13\relax
      \catcode32=\the\catcode32\relax
      \catcode35=\the\catcode35\relax
      \catcode61=\the\catcode61\relax
      \catcode64=\the\catcode64\relax
      \catcode123=\the\catcode123\relax
      \catcode125=\the\catcode125\relax
    }%
  }%
\x\catcode61\catcode48\catcode32=10\relax%
\catcode13=5 % ^^M
\endlinechar=13 %
\catcode35=6 % #
\catcode64=11 % @
\catcode123=1 % {
\catcode125=2 % }
\def\TMP@EnsureCode#1#2{%
  \edef\llm@AtEnd{%
    \llm@AtEnd
    \catcode#1=\the\catcode#1\relax
  }%
  \catcode#1=#2\relax
}
\TMP@EnsureCode{40}{12}% (
\TMP@EnsureCode{41}{12}% )
\TMP@EnsureCode{42}{12}% *
\TMP@EnsureCode{45}{12}% -
\TMP@EnsureCode{46}{12}% .
\TMP@EnsureCode{47}{12}% /
\TMP@EnsureCode{58}{12}% :
\TMP@EnsureCode{62}{12}% >
\TMP@EnsureCode{91}{12}% [
\TMP@EnsureCode{93}{12}% ]
\edef\llm@AtEnd{%
  \llm@AtEnd
  \escapechar\the\escapechar\relax
  \noexpand\endinput
}
\escapechar=92 % `\\
%    \end{macrocode}
%
%    Package identification.
%    \begin{macrocode}
\NeedsTeXFormat{LaTeX2e}
\ProvidesPackage{letltxmacro}%
  [2019/12/03 v1.6 Let assignment for LaTeX macros (HO)]
%    \end{macrocode}
%
% \subsubsection{Main macros}
%
%    \begin{macro}{\LetLtxMacro}
%    \begin{macrocode}
\newcommand*{\LetLtxMacro}{%
  \llm@ModeLetLtxMacro{}%
}
%    \end{macrocode}
%    \end{macro}
%    \begin{macro}{\GlobalLetLtxMacro}
%    \begin{macrocode}
\newcommand*{\GlobalLetLtxMacro}{%
  \llm@ModeLetLtxMacro\global
}
%    \end{macrocode}
%    \end{macro}
%
%    \begin{macro}{\llm@ModeLetLtxMacro}
%    \begin{macrocode}
\newcommand*{\llm@ModeLetLtxMacro}[3]{%
  \edef\llm@escapechar{\the\escapechar}%
  \escapechar=-1 %
  \edef\reserved@a{%
    \noexpand\protect
    \expandafter\noexpand
    \csname\string#3 \endcsname
  }%
  \ifx\reserved@a#3\relax
    #1\edef#2{%
      \noexpand\protect
      \expandafter\noexpand
      \csname\string#2 \endcsname
    }%
    #1\expandafter\let
    \csname\string#2 \expandafter\endcsname
    \csname\string#3 \endcsname
    \expandafter\llm@LetLtxMacro
        \csname\string#2 \expandafter\endcsname
        \csname\string#3 \endcsname{#1}%
  \else
    \llm@LetLtxMacro{#2}{#3}{#1}%
  \fi
  \escapechar=\llm@escapechar\relax
}
%    \end{macrocode}
%    \end{macro}
%    \begin{macro}{\llm@LetLtxMacro}
%    \begin{macrocode}
\def\llm@LetLtxMacro#1#2#3{%
  \escapechar=92 %
  \expandafter\llm@CheckParams\meaning#2:->\@nil{%
    \begingroup
      \def\@protected@testopt{%
        \expandafter\@testopt\@gobble
      }%
      \def\@testopt##1##2{%
        \toks@={##2}%
      }%
      \let\llm@testopt\@empty
      \edef\x{%
        \noexpand\@protected@testopt
        \noexpand#2%
        \expandafter\noexpand\csname\string#2\endcsname
      }%
      \expandafter\expandafter\expandafter\def
      \expandafter\expandafter\expandafter\y
      \expandafter\expandafter\expandafter{%
        \expandafter\llm@CarThree#2{}{}{}\llm@nil
      }%
      \ifx\x\y
        #2%
        \def\llm@testopt{%
          \noexpand\@protected@testopt
          \noexpand#1%
        }%
      \else
        \edef\x{%
          \noexpand\@testopt
          \expandafter\noexpand
          \csname\string#2\endcsname
        }%
        \expandafter\expandafter\expandafter\def
        \expandafter\expandafter\expandafter\y
        \expandafter\expandafter\expandafter{%
          \expandafter\llm@CarTwo#2{}{}\llm@nil
        }%
        \ifx\x\y
          #2%
          \def\llm@testopt{%
            \noexpand\@testopt
          }%
        \fi
      \fi
      \ifx\llm@testopt\@empty
      \else
        \llm@protected\xdef\llm@GlobalTemp{%
          \llm@testopt
          \expandafter\noexpand
          \csname\string#1\endcsname
          {\the\toks@}%
        }%
      \fi
    \expandafter\endgroup\ifx\llm@testopt\@empty
      #3\let#1=#2\relax
    \else
      #3\let#1=\llm@GlobalTemp
      #3\expandafter\let
          \csname\string#1\expandafter\endcsname
          \csname\string#2\endcsname
    \fi
  }{%
    #3\let#1=#2\relax
  }%
}
%    \end{macrocode}
%    \end{macro}
%    \begin{macro}{\llm@CheckParams}
%    \begin{macrocode}
\def\llm@CheckParams#1:->#2\@nil{%
  \begingroup
    \def\x{#1}%
  \ifx\x\llm@macro
    \endgroup
    \def\llm@protected{}%
    \expandafter\@firstoftwo
  \else
    \ifx\x\llm@protectedmacro
      \endgroup
      \def\llm@protected{\protected}%
      \expandafter\expandafter\expandafter\@firstoftwo
    \else
      \endgroup
      \expandafter\expandafter\expandafter\@secondoftwo
    \fi
  \fi
}
%    \end{macrocode}
%    \end{macro}
%    \begin{macro}{\llm@macro}
%    \begin{macrocode}
\def\llm@macro{macro}
\@onelevel@sanitize\llm@macro
%    \end{macrocode}
%    \end{macro}
%    \begin{macro}{\llm@protectedmacro}
%    \begin{macrocode}
\def\llm@protectedmacro{\protected macro}
\@onelevel@sanitize\llm@protectedmacro
%    \end{macrocode}
%    \end{macro}
%    \begin{macro}{\llm@CarThree}
%    \begin{macrocode}
\def\llm@CarThree#1#2#3#4\llm@nil{#1#2#3}%
%    \end{macrocode}
%    \end{macro}
%    \begin{macro}{\llm@CarTwo}
%    \begin{macrocode}
\def\llm@CarTwo#1#2#3\llm@nil{#1#2}%
%    \end{macrocode}
%    \end{macro}
%
%    \begin{macrocode}
\llm@AtEnd%
%</package>
%    \end{macrocode}
% \section{Installation}
%
% \subsection{Download}
%
% \paragraph{Package.} This package is available on
% CTAN\footnote{\CTANpkg{letltxmacro}}:
% \begin{description}
% \item[\CTAN{macros/latex/contrib/letltxmacro/letltxmacro.dtx}] The source file.
% \item[\CTAN{macros/latex/contrib/letltxmacro/letltxmacro.pdf}] Documentation.
% \end{description}
%
%
% \paragraph{Bundle.} All the packages of the bundle `letltxmacro'
% are also available in a TDS compliant ZIP archive. There
% the packages are already unpacked and the documentation files
% are generated. The files and directories obey the TDS standard.
% \begin{description}
% \item[\CTANinstall{install/macros/latex/contrib/letltxmacro.tds.zip}]
% \end{description}
% \emph{TDS} refers to the standard ``A Directory Structure
% for \TeX\ Files'' (\CTANpkg{tds}). Directories
% with \xfile{texmf} in their name are usually organized this way.
%
% \subsection{Bundle installation}
%
% \paragraph{Unpacking.} Unpack the \xfile{letltxmacro.tds.zip} in the
% TDS tree (also known as \xfile{texmf} tree) of your choice.
% Example (linux):
% \begin{quote}
%   |unzip letltxmacro.tds.zip -d ~/texmf|
% \end{quote}
%
% \subsection{Package installation}
%
% \paragraph{Unpacking.} The \xfile{.dtx} file is a self-extracting
% \docstrip\ archive. The files are extracted by running the
% \xfile{.dtx} through \plainTeX:
% \begin{quote}
%   \verb|tex letltxmacro.dtx|
% \end{quote}
%
% \paragraph{TDS.} Now the different files must be moved into
% the different directories in your installation TDS tree
% (also known as \xfile{texmf} tree):
% \begin{quote}
% \def\t{^^A
% \begin{tabular}{@{}>{\ttfamily}l@{ $\rightarrow$ }>{\ttfamily}l@{}}
%   letltxmacro.sty & tex/latex/letltxmacro/letltxmacro.sty\\
%   letltxmacro.pdf & doc/latex/letltxmacro/letltxmacro.pdf\\
%   letltxmacro-showcases.tex & doc/latex/letltxmacro/letltxmacro-showcases.tex\\
%   letltxmacro.dtx & source/latex/letltxmacro/letltxmacro.dtx\\
% \end{tabular}^^A
% }^^A
% \sbox0{\t}^^A
% \ifdim\wd0>\linewidth
%   \begingroup
%     \advance\linewidth by\leftmargin
%     \advance\linewidth by\rightmargin
%   \edef\x{\endgroup
%     \def\noexpand\lw{\the\linewidth}^^A
%   }\x
%   \def\lwbox{^^A
%     \leavevmode
%     \hbox to \linewidth{^^A
%       \kern-\leftmargin\relax
%       \hss
%       \usebox0
%       \hss
%       \kern-\rightmargin\relax
%     }^^A
%   }^^A
%   \ifdim\wd0>\lw
%     \sbox0{\small\t}^^A
%     \ifdim\wd0>\linewidth
%       \ifdim\wd0>\lw
%         \sbox0{\footnotesize\t}^^A
%         \ifdim\wd0>\linewidth
%           \ifdim\wd0>\lw
%             \sbox0{\scriptsize\t}^^A
%             \ifdim\wd0>\linewidth
%               \ifdim\wd0>\lw
%                 \sbox0{\tiny\t}^^A
%                 \ifdim\wd0>\linewidth
%                   \lwbox
%                 \else
%                   \usebox0
%                 \fi
%               \else
%                 \lwbox
%               \fi
%             \else
%               \usebox0
%             \fi
%           \else
%             \lwbox
%           \fi
%         \else
%           \usebox0
%         \fi
%       \else
%         \lwbox
%       \fi
%     \else
%       \usebox0
%     \fi
%   \else
%     \lwbox
%   \fi
% \else
%   \usebox0
% \fi
% \end{quote}
% If you have a \xfile{docstrip.cfg} that configures and enables \docstrip's
% TDS installing feature, then some files can already be in the right
% place, see the documentation of \docstrip.
%
% \subsection{Refresh file name databases}
%
% If your \TeX~distribution
% (\TeX\,Live, \mikTeX, \dots) relies on file name databases, you must refresh
% these. For example, \TeX\,Live\ users run \verb|texhash| or
% \verb|mktexlsr|.
%
% \subsection{Some details for the interested}
%
% \paragraph{Unpacking with \LaTeX.}
% The \xfile{.dtx} chooses its action depending on the format:
% \begin{description}
% \item[\plainTeX:] Run \docstrip\ and extract the files.
% \item[\LaTeX:] Generate the documentation.
% \end{description}
% If you insist on using \LaTeX\ for \docstrip\ (really,
% \docstrip\ does not need \LaTeX), then inform the autodetect routine
% about your intention:
% \begin{quote}
%   \verb|latex \let\install=y\input{letltxmacro.dtx}|
% \end{quote}
% Do not forget to quote the argument according to the demands
% of your shell.
%
% \paragraph{Generating the documentation.}
% You can use both the \xfile{.dtx} or the \xfile{.drv} to generate
% the documentation. The process can be configured by the
% configuration file \xfile{ltxdoc.cfg}. For instance, put this
% line into this file, if you want to have A4 as paper format:
% \begin{quote}
%   \verb|\PassOptionsToClass{a4paper}{article}|
% \end{quote}
% An example follows how to generate the
% documentation with pdf\LaTeX:
% \begin{quote}
%\begin{verbatim}
%pdflatex letltxmacro.dtx
%makeindex -s gind.ist letltxmacro.idx
%pdflatex letltxmacro.dtx
%makeindex -s gind.ist letltxmacro.idx
%pdflatex letltxmacro.dtx
%\end{verbatim}
% \end{quote}
%
% \begin{History}
%   \begin{Version}{2008/06/09 v1.0}
%   \item
%     First version.
%   \end{Version}
%   \begin{Version}{2008/06/12 v1.1}
%   \item
%     Support for \xpackage{etoolbox}'s \cs{newrobustcmd} added.
%   \end{Version}
%   \begin{Version}{2008/06/13 v1.2}
%   \item
%     Support for \xpackage{etoolbox}'s \cs{robustify} added.
%   \end{Version}
%   \begin{Version}{2008/06/24 v1.3}
%   \item
%     Test file adapted for etoolbox 2008/06/22 v1.6.
%   \end{Version}
%   \begin{Version}{2010/09/02 v1.4}
%   \item
%     \cs{GlobalLetLtxMacro} added.
%   \end{Version}
%   \begin{Version}{2016/05/16 v1.5}
%   \item
%     Documentation updates.
%   \end{Version}
%   \begin{Version}{2019/12/03 v1.6}
%   \item
%     Documentation updates.
%   \end{Version}
% \end{History}
%
% \PrintIndex
%
% \Finale
\endinput
|
% \end{quote}
% Do not forget to quote the argument according to the demands
% of your shell.
%
% \paragraph{Generating the documentation.}
% You can use both the \xfile{.dtx} or the \xfile{.drv} to generate
% the documentation. The process can be configured by the
% configuration file \xfile{ltxdoc.cfg}. For instance, put this
% line into this file, if you want to have A4 as paper format:
% \begin{quote}
%   \verb|\PassOptionsToClass{a4paper}{article}|
% \end{quote}
% An example follows how to generate the
% documentation with pdf\LaTeX:
% \begin{quote}
%\begin{verbatim}
%pdflatex letltxmacro.dtx
%makeindex -s gind.ist letltxmacro.idx
%pdflatex letltxmacro.dtx
%makeindex -s gind.ist letltxmacro.idx
%pdflatex letltxmacro.dtx
%\end{verbatim}
% \end{quote}
%
% \begin{History}
%   \begin{Version}{2008/06/09 v1.0}
%   \item
%     First version.
%   \end{Version}
%   \begin{Version}{2008/06/12 v1.1}
%   \item
%     Support for \xpackage{etoolbox}'s \cs{newrobustcmd} added.
%   \end{Version}
%   \begin{Version}{2008/06/13 v1.2}
%   \item
%     Support for \xpackage{etoolbox}'s \cs{robustify} added.
%   \end{Version}
%   \begin{Version}{2008/06/24 v1.3}
%   \item
%     Test file adapted for etoolbox 2008/06/22 v1.6.
%   \end{Version}
%   \begin{Version}{2010/09/02 v1.4}
%   \item
%     \cs{GlobalLetLtxMacro} added.
%   \end{Version}
%   \begin{Version}{2016/05/16 v1.5}
%   \item
%     Documentation updates.
%   \end{Version}
%   \begin{Version}{2019/12/03 v1.6}
%   \item
%     Documentation updates.
%   \end{Version}
% \end{History}
%
% \PrintIndex
%
% \Finale
\endinput

%        (quote the arguments according to the demands of your shell)
%
% Documentation:
%    (a) If letltxmacro.drv is present:
%           latex letltxmacro.drv
%    (b) Without letltxmacro.drv:
%           latex letltxmacro.dtx; ...
%    The class ltxdoc loads the configuration file ltxdoc.cfg
%    if available. Here you can specify further options, e.g.
%    use A4 as paper format:
%       \PassOptionsToClass{a4paper}{article}
%
%    Programm calls to get the documentation (example):
%       pdflatex letltxmacro.dtx
%       makeindex -s gind.ist letltxmacro.idx
%       pdflatex letltxmacro.dtx
%       makeindex -s gind.ist letltxmacro.idx
%       pdflatex letltxmacro.dtx
%
% Installation:
%    TDS:tex/latex/letltxmacro/letltxmacro.sty
%    TDS:doc/latex/letltxmacro/letltxmacro.pdf
%    TDS:doc/latex/letltxmacro/letltxmacro-showcases.tex
%    TDS:source/latex/letltxmacro/letltxmacro.dtx
%
%<*ignore>
\begingroup
  \catcode123=1 %
  \catcode125=2 %
  \def\x{LaTeX2e}%
\expandafter\endgroup
\ifcase 0\ifx\install y1\fi\expandafter
         \ifx\csname processbatchFile\endcsname\relax\else1\fi
         \ifx\fmtname\x\else 1\fi\relax
\else\csname fi\endcsname
%</ignore>
%<*install>
\input docstrip.tex
\Msg{************************************************************************}
\Msg{* Installation}
\Msg{* Package: letltxmacro 2019/12/03 v1.6 Let assignment for LaTeX macros (HO)}
\Msg{************************************************************************}

\keepsilent
\askforoverwritefalse

\let\MetaPrefix\relax
\preamble

This is a generated file.

Project: letltxmacro
Version: 2019/12/03 v1.6

Copyright (C)
   2008, 2010 Heiko Oberdiek
   2016-2019 Oberdiek Package Support Group

This work may be distributed and/or modified under the
conditions of the LaTeX Project Public License, either
version 1.3c of this license or (at your option) any later
version. This version of this license is in
   https://www.latex-project.org/lppl/lppl-1-3c.txt
and the latest version of this license is in
   https://www.latex-project.org/lppl.txt
and version 1.3 or later is part of all distributions of
LaTeX version 2005/12/01 or later.

This work has the LPPL maintenance status "maintained".

The Current Maintainers of this work are
Heiko Oberdiek and the Oberdiek Package Support Group
https://github.com/ho-tex/letltxmacro/issues


This work consists of the main source file letltxmacro.dtx
and the derived files
   letltxmacro.sty, letltxmacro.pdf, letltxmacro.ins, letltxmacro.drv,
   letltxmacro-showcases.tex, letltxmacro-test1.tex,
   letltxmacro-test2.tex.

\endpreamble
\let\MetaPrefix\DoubleperCent

\generate{%
  \file{letltxmacro.ins}{\from{letltxmacro.dtx}{install}}%
  \file{letltxmacro.drv}{\from{letltxmacro.dtx}{driver}}%
  \usedir{tex/latex/letltxmacro}%
  \file{letltxmacro.sty}{\from{letltxmacro.dtx}{package}}%
  \usedir{doc/latex/letltxmacro}%
  \file{letltxmacro-showcases.tex}{\from{letltxmacro.dtx}{showcases}}%
%  \usedir{doc/latex/letltxmacro/test}%
%  \file{letltxmacro-test1.tex}{\from{letltxmacro.dtx}{test1}}%
%  \file{letltxmacro-test2.tex}{\from{letltxmacro.dtx}{test2}}%
}

\catcode32=13\relax% active space
\let =\space%
\Msg{************************************************************************}
\Msg{*}
\Msg{* To finish the installation you have to move the following}
\Msg{* file into a directory searched by TeX:}
\Msg{*}
\Msg{*     letltxmacro.sty}
\Msg{*}
\Msg{* To produce the documentation run the file `letltxmacro.drv'}
\Msg{* through LaTeX.}
\Msg{*}
\Msg{* Happy TeXing!}
\Msg{*}
\Msg{************************************************************************}

\endbatchfile
%</install>
%<*ignore>
\fi
%</ignore>
%<*driver>
\NeedsTeXFormat{LaTeX2e}
\ProvidesFile{letltxmacro.drv}%
  [2019/12/03 v1.6 Let assignment for LaTeX macros (HO)]%
\documentclass{ltxdoc}
\usepackage{holtxdoc}[2011/11/22]
\begin{document}
  \DocInput{letltxmacro.dtx}%
\end{document}
%</driver>
% \fi
%
%
%
% \GetFileInfo{letltxmacro.drv}
%
% \title{The \xpackage{letltxmacro} package}
% \date{2019/12/03 v1.6}
% \author{Heiko Oberdiek\thanks
% {Please report any issues at \url{https://github.com/ho-tex/letltxmacro/issues}}}
%
% \maketitle
%
% \begin{abstract}
% \TeX's \cs{let} assignment does not work for \LaTeX\ macros
% with optional arguments or for macros that are defined
% as robust macros by \cs{DeclareRobustCommand}. This package
% defines \cs{LetLtxMacro} that also takes care of the involved
% internal macros.
% \end{abstract}
%
% \tableofcontents
%
% \section{Documentation}
%
% If someone wants to redefine a macro with using the old
% meaning, then one method is \TeX's command \cs{let}:
%\begin{quote}
%\begin{verbatim}
%\newcommand{\Macro}{\typeout{Test Macro}}
%\let\SavedMacro=\Macro
%\renewcommand{\Macro}{%
%  \typeout{Begin}%
%  \SavedMacro
%  \typeout{End}%
%}
%\end{verbatim}
%\end{quote}
% However, this method fails, if \cs{Macro} is defined
% by \cs{DeclareRobustCommand} and/or has an optional argument.
% In both cases \LaTeX\ defines an additional internal macro
% that is forgotten in the simple \cs{let} assignment of
% the example above.
%
% \begin{declcs}{LetLtxMacro} \M{new macro} \M{old macro}
% \end{declcs}
% Macro \cs{LetLtxMacro} behaves similar to \TeX's \cs{let}
% assignment, but it takes care of macros that are
% defined by \cs{DeclareRobustCommand} and/or have optional
% arguments. Example:
%\begin{quote}
%\begin{verbatim}
%\DeclareRobustCommand{\Macro}[1][default]{...}
%\LetLtxMacro{\SavedMacro}{\Macro}
%\end{verbatim}
%\end{quote}
% Then macro \cs{SavedMacro} only uses internal macro names
% that are derived from \cs{SavedMacro}'s macro name. Macro \cs{Macro}
% can now be redefined without affecting \cs{SavedMacro}.
%
% \begin{declcs}{GlobalLetLtxMacro} \M{new macro} \M{old macro}
% \end{declcs}
% Like \cs{LetLtxMacro}, but the \meta{new macro} is defined globally.
% Since version 2019/12/03~v1.4.
%
% \subsection{Supported macro definition commands}
%
% \begin{quote}
%   \begin{tabular}{@{}ll@{}}
%     \cs{newcommand}, \cs{renewcommand} & latex/base\\
%     \cs{newenvironment}, \cs{renewenvironment} & latex/base\\
%     \cs{DeclareRobustCommand}& latex/base\\
%     \cs{newrobustcmd}, \cs{renewrobustcmd} & etoolbox\\
%     \cs{robustify} & etoolbox 2008/06/22 v1.6\\
%   \end{tabular}
% \end{quote}
%
% \StopEventually{
% }
%
% \section{Implementation}
%
% \subsection{Show cases}
%
% \subsubsection{\xfile{letltxmacro-showcases.tex}}
%
%    \begin{macrocode}
%<*showcases>
\NeedsTeXFormat{LaTeX2e}
\makeatletter
%    \end{macrocode}
%    \begin{macro}{\Line}
%    The result is displayed by macro \cs{Line}. The percent symbol
%    at line start allows easy grepping and inserting into the DTX
%    file.
%    \begin{macrocode}
\newcommand*{\Line}[1]{%
  \typeout{\@percentchar#1}%
}
%    \end{macrocode}
%    \end{macro}
%    \begin{macrocode}
\newcommand*{\ShowCmdName}[1]{%
  \@ifundefined{#1}{}{%
    \Line{%
      \space\space(\expandafter\string\csname#1\endcsname) = %
      (\expandafter\meaning\csname#1\endcsname)%
    }%
  }%
}
\newcommand*{\ShowCmds}[1]{%
  \ShowCmdName{#1}%
  \ShowCmdName{#1 }%
  \ShowCmdName{\\#1}%
  \ShowCmdName{\\#1 }%
}
\let\\\@backslashchar
%    \end{macrocode}
%    \begin{macro}{\ShowDef}
%    \begin{macrocode}
\newcommand*{\ShowDef}[2]{%
  \begingroup
    \Line{}%
    \newcommand*{\DefString}{#2}%
    \@onelevel@sanitize\DefString
    \Line{\DefString}%
    #2%
    \ShowCmds{#1}%
  \endgroup
}
%    \end{macrocode}
%    \end{macro}
%    \begin{macrocode}
\typeout{}
\Line{* LaTeX definitions:}
\ShowDef{cmd}{%
  \newcommand{\cmd}[2][default]{}%
}
\ShowDef{cmd}{%
  \DeclareRobustCommand{\cmd}{}%
}
\ShowDef{cmd}{%
  \DeclareRobustCommand{\cmd}[2][default]{}%
}
\typeout{}
%    \end{macrocode}
% The minimal version of package \xpackage{etoolbox} is 2008/06/12 v1.6a
% because it fixes \cs{robustify}.
%    \begin{macrocode}
\RequirePackage{etoolbox}[2008/06/12]%
\Line{}
\Line{* etoolbox's robust definitions:}
\ShowDef{cmd}{%
  \newrobustcmd{\cmd}{}%
}
\ShowDef{cmd}{%
  \newrobustcmd{\cmd}[2][default]{}%
}
\Line{}
\Line{* etoolbox's \string\robustify:}
\ShowDef{cmd}{%
  \newcommand{\cmd}[2][default]{} %
  \robustify{\cmd}%
}
\ShowDef{cmd}{%
  \DeclareRobustCommand{\cmd}{} %
  \robustify{\cmd}%
}
\ShowDef{cmd}{%
  \DeclareRobustCommand{\cmd}[2][default]{} %
  \robustify{\cmd}%
}
\typeout{}
\@@end
%</showcases>
%    \end{macrocode}
%
% \subsubsection{Result}
%
% \begingroup
%   \makeatletter
%   \let\org@verbatim\@verbatim
%   \def\@verbatim{^^A
%     \org@verbatim
%     \catcode`\~=\active
%   }^^A
%   \let~\textvisiblespace
%\begin{verbatim}
%* LaTeX definitions:
%
%\newcommand {\cmd }[2][default]{}
%  (\cmd) = (macro:->\@protected@testopt \cmd \\cmd {default})
%  (\\cmd) = (\long macro:[#1]#2->)
%
%\DeclareRobustCommand {\cmd }{}
%  (\cmd) = (macro:->\protect \cmd~ )
%  (\cmd~) = (\long macro:->)
%
%\DeclareRobustCommand {\cmd }[2][default]{}
%  (\cmd) = (macro:->\protect \cmd~ )
%  (\cmd~) = (macro:->\@protected@testopt \cmd~ \\cmd~ {default})
%  (\\cmd~) = (\long macro:[#1]#2->)
%
%* etoolbox's robust definitions:
%
%\newrobustcmd {\cmd }{}
%  (\cmd) = (\protected\long macro:->)
%
%\newrobustcmd {\cmd }[2][default]{}
%  (\cmd) = (\protected macro:->\@testopt \\cmd {default})
%  (\\cmd) = (\long macro:[#1]#2->)
%
%* etoolbox's \robustify:
%
%\newcommand {\cmd }[2][default]{} \robustify {\cmd }
%  (\cmd) = (\protected macro:->\@protected@testopt \cmd \\cmd {default})
%  (\\cmd) = (\long macro:[#1]#2->)
%
%\DeclareRobustCommand {\cmd }{} \robustify {\cmd }
%  (\cmd) = (\protected macro:->)
%
%\DeclareRobustCommand {\cmd }[2][default]{} \robustify {\cmd }
%  (\cmd) = (\protected macro:->\@protected@testopt \cmd~ \\cmd~ {default})
%  (\cmd~) = (macro:->\@protected@testopt \cmd~ \\cmd~ {default})
%  (\\cmd~) = (\long macro:[#1]#2->)
%\end{verbatim}
% \endgroup
%
% \subsection{Package}
%
%    \begin{macrocode}
%<*package>
%    \end{macrocode}
%
% \subsubsection{Catcodes and identification}
%
%    \begin{macrocode}
\begingroup\catcode61\catcode48\catcode32=10\relax%
  \catcode13=5 % ^^M
  \endlinechar=13 %
  \catcode123=1 % {
  \catcode125=2 % }
  \catcode64=11 % @
  \def\x{\endgroup
    \expandafter\edef\csname llm@AtEnd\endcsname{%
      \endlinechar=\the\endlinechar\relax
      \catcode13=\the\catcode13\relax
      \catcode32=\the\catcode32\relax
      \catcode35=\the\catcode35\relax
      \catcode61=\the\catcode61\relax
      \catcode64=\the\catcode64\relax
      \catcode123=\the\catcode123\relax
      \catcode125=\the\catcode125\relax
    }%
  }%
\x\catcode61\catcode48\catcode32=10\relax%
\catcode13=5 % ^^M
\endlinechar=13 %
\catcode35=6 % #
\catcode64=11 % @
\catcode123=1 % {
\catcode125=2 % }
\def\TMP@EnsureCode#1#2{%
  \edef\llm@AtEnd{%
    \llm@AtEnd
    \catcode#1=\the\catcode#1\relax
  }%
  \catcode#1=#2\relax
}
\TMP@EnsureCode{40}{12}% (
\TMP@EnsureCode{41}{12}% )
\TMP@EnsureCode{42}{12}% *
\TMP@EnsureCode{45}{12}% -
\TMP@EnsureCode{46}{12}% .
\TMP@EnsureCode{47}{12}% /
\TMP@EnsureCode{58}{12}% :
\TMP@EnsureCode{62}{12}% >
\TMP@EnsureCode{91}{12}% [
\TMP@EnsureCode{93}{12}% ]
\edef\llm@AtEnd{%
  \llm@AtEnd
  \escapechar\the\escapechar\relax
  \noexpand\endinput
}
\escapechar=92 % `\\
%    \end{macrocode}
%
%    Package identification.
%    \begin{macrocode}
\NeedsTeXFormat{LaTeX2e}
\ProvidesPackage{letltxmacro}%
  [2019/12/03 v1.6 Let assignment for LaTeX macros (HO)]
%    \end{macrocode}
%
% \subsubsection{Main macros}
%
%    \begin{macro}{\LetLtxMacro}
%    \begin{macrocode}
\newcommand*{\LetLtxMacro}{%
  \llm@ModeLetLtxMacro{}%
}
%    \end{macrocode}
%    \end{macro}
%    \begin{macro}{\GlobalLetLtxMacro}
%    \begin{macrocode}
\newcommand*{\GlobalLetLtxMacro}{%
  \llm@ModeLetLtxMacro\global
}
%    \end{macrocode}
%    \end{macro}
%
%    \begin{macro}{\llm@ModeLetLtxMacro}
%    \begin{macrocode}
\newcommand*{\llm@ModeLetLtxMacro}[3]{%
  \edef\llm@escapechar{\the\escapechar}%
  \escapechar=-1 %
  \edef\reserved@a{%
    \noexpand\protect
    \expandafter\noexpand
    \csname\string#3 \endcsname
  }%
  \ifx\reserved@a#3\relax
    #1\edef#2{%
      \noexpand\protect
      \expandafter\noexpand
      \csname\string#2 \endcsname
    }%
    #1\expandafter\let
    \csname\string#2 \expandafter\endcsname
    \csname\string#3 \endcsname
    \expandafter\llm@LetLtxMacro
        \csname\string#2 \expandafter\endcsname
        \csname\string#3 \endcsname{#1}%
  \else
    \llm@LetLtxMacro{#2}{#3}{#1}%
  \fi
  \escapechar=\llm@escapechar\relax
}
%    \end{macrocode}
%    \end{macro}
%    \begin{macro}{\llm@LetLtxMacro}
%    \begin{macrocode}
\def\llm@LetLtxMacro#1#2#3{%
  \escapechar=92 %
  \expandafter\llm@CheckParams\meaning#2:->\@nil{%
    \begingroup
      \def\@protected@testopt{%
        \expandafter\@testopt\@gobble
      }%
      \def\@testopt##1##2{%
        \toks@={##2}%
      }%
      \let\llm@testopt\@empty
      \edef\x{%
        \noexpand\@protected@testopt
        \noexpand#2%
        \expandafter\noexpand\csname\string#2\endcsname
      }%
      \expandafter\expandafter\expandafter\def
      \expandafter\expandafter\expandafter\y
      \expandafter\expandafter\expandafter{%
        \expandafter\llm@CarThree#2{}{}{}\llm@nil
      }%
      \ifx\x\y
        #2%
        \def\llm@testopt{%
          \noexpand\@protected@testopt
          \noexpand#1%
        }%
      \else
        \edef\x{%
          \noexpand\@testopt
          \expandafter\noexpand
          \csname\string#2\endcsname
        }%
        \expandafter\expandafter\expandafter\def
        \expandafter\expandafter\expandafter\y
        \expandafter\expandafter\expandafter{%
          \expandafter\llm@CarTwo#2{}{}\llm@nil
        }%
        \ifx\x\y
          #2%
          \def\llm@testopt{%
            \noexpand\@testopt
          }%
        \fi
      \fi
      \ifx\llm@testopt\@empty
      \else
        \llm@protected\xdef\llm@GlobalTemp{%
          \llm@testopt
          \expandafter\noexpand
          \csname\string#1\endcsname
          {\the\toks@}%
        }%
      \fi
    \expandafter\endgroup\ifx\llm@testopt\@empty
      #3\let#1=#2\relax
    \else
      #3\let#1=\llm@GlobalTemp
      #3\expandafter\let
          \csname\string#1\expandafter\endcsname
          \csname\string#2\endcsname
    \fi
  }{%
    #3\let#1=#2\relax
  }%
}
%    \end{macrocode}
%    \end{macro}
%    \begin{macro}{\llm@CheckParams}
%    \begin{macrocode}
\def\llm@CheckParams#1:->#2\@nil{%
  \begingroup
    \def\x{#1}%
  \ifx\x\llm@macro
    \endgroup
    \def\llm@protected{}%
    \expandafter\@firstoftwo
  \else
    \ifx\x\llm@protectedmacro
      \endgroup
      \def\llm@protected{\protected}%
      \expandafter\expandafter\expandafter\@firstoftwo
    \else
      \endgroup
      \expandafter\expandafter\expandafter\@secondoftwo
    \fi
  \fi
}
%    \end{macrocode}
%    \end{macro}
%    \begin{macro}{\llm@macro}
%    \begin{macrocode}
\def\llm@macro{macro}
\@onelevel@sanitize\llm@macro
%    \end{macrocode}
%    \end{macro}
%    \begin{macro}{\llm@protectedmacro}
%    \begin{macrocode}
\def\llm@protectedmacro{\protected macro}
\@onelevel@sanitize\llm@protectedmacro
%    \end{macrocode}
%    \end{macro}
%    \begin{macro}{\llm@CarThree}
%    \begin{macrocode}
\def\llm@CarThree#1#2#3#4\llm@nil{#1#2#3}%
%    \end{macrocode}
%    \end{macro}
%    \begin{macro}{\llm@CarTwo}
%    \begin{macrocode}
\def\llm@CarTwo#1#2#3\llm@nil{#1#2}%
%    \end{macrocode}
%    \end{macro}
%
%    \begin{macrocode}
\llm@AtEnd%
%</package>
%    \end{macrocode}
% \section{Installation}
%
% \subsection{Download}
%
% \paragraph{Package.} This package is available on
% CTAN\footnote{\CTANpkg{letltxmacro}}:
% \begin{description}
% \item[\CTAN{macros/latex/contrib/letltxmacro/letltxmacro.dtx}] The source file.
% \item[\CTAN{macros/latex/contrib/letltxmacro/letltxmacro.pdf}] Documentation.
% \end{description}
%
%
% \paragraph{Bundle.} All the packages of the bundle `letltxmacro'
% are also available in a TDS compliant ZIP archive. There
% the packages are already unpacked and the documentation files
% are generated. The files and directories obey the TDS standard.
% \begin{description}
% \item[\CTANinstall{install/macros/latex/contrib/letltxmacro.tds.zip}]
% \end{description}
% \emph{TDS} refers to the standard ``A Directory Structure
% for \TeX\ Files'' (\CTANpkg{tds}). Directories
% with \xfile{texmf} in their name are usually organized this way.
%
% \subsection{Bundle installation}
%
% \paragraph{Unpacking.} Unpack the \xfile{letltxmacro.tds.zip} in the
% TDS tree (also known as \xfile{texmf} tree) of your choice.
% Example (linux):
% \begin{quote}
%   |unzip letltxmacro.tds.zip -d ~/texmf|
% \end{quote}
%
% \subsection{Package installation}
%
% \paragraph{Unpacking.} The \xfile{.dtx} file is a self-extracting
% \docstrip\ archive. The files are extracted by running the
% \xfile{.dtx} through \plainTeX:
% \begin{quote}
%   \verb|tex letltxmacro.dtx|
% \end{quote}
%
% \paragraph{TDS.} Now the different files must be moved into
% the different directories in your installation TDS tree
% (also known as \xfile{texmf} tree):
% \begin{quote}
% \def\t{^^A
% \begin{tabular}{@{}>{\ttfamily}l@{ $\rightarrow$ }>{\ttfamily}l@{}}
%   letltxmacro.sty & tex/latex/letltxmacro/letltxmacro.sty\\
%   letltxmacro.pdf & doc/latex/letltxmacro/letltxmacro.pdf\\
%   letltxmacro-showcases.tex & doc/latex/letltxmacro/letltxmacro-showcases.tex\\
%   letltxmacro.dtx & source/latex/letltxmacro/letltxmacro.dtx\\
% \end{tabular}^^A
% }^^A
% \sbox0{\t}^^A
% \ifdim\wd0>\linewidth
%   \begingroup
%     \advance\linewidth by\leftmargin
%     \advance\linewidth by\rightmargin
%   \edef\x{\endgroup
%     \def\noexpand\lw{\the\linewidth}^^A
%   }\x
%   \def\lwbox{^^A
%     \leavevmode
%     \hbox to \linewidth{^^A
%       \kern-\leftmargin\relax
%       \hss
%       \usebox0
%       \hss
%       \kern-\rightmargin\relax
%     }^^A
%   }^^A
%   \ifdim\wd0>\lw
%     \sbox0{\small\t}^^A
%     \ifdim\wd0>\linewidth
%       \ifdim\wd0>\lw
%         \sbox0{\footnotesize\t}^^A
%         \ifdim\wd0>\linewidth
%           \ifdim\wd0>\lw
%             \sbox0{\scriptsize\t}^^A
%             \ifdim\wd0>\linewidth
%               \ifdim\wd0>\lw
%                 \sbox0{\tiny\t}^^A
%                 \ifdim\wd0>\linewidth
%                   \lwbox
%                 \else
%                   \usebox0
%                 \fi
%               \else
%                 \lwbox
%               \fi
%             \else
%               \usebox0
%             \fi
%           \else
%             \lwbox
%           \fi
%         \else
%           \usebox0
%         \fi
%       \else
%         \lwbox
%       \fi
%     \else
%       \usebox0
%     \fi
%   \else
%     \lwbox
%   \fi
% \else
%   \usebox0
% \fi
% \end{quote}
% If you have a \xfile{docstrip.cfg} that configures and enables \docstrip's
% TDS installing feature, then some files can already be in the right
% place, see the documentation of \docstrip.
%
% \subsection{Refresh file name databases}
%
% If your \TeX~distribution
% (\TeX\,Live, \mikTeX, \dots) relies on file name databases, you must refresh
% these. For example, \TeX\,Live\ users run \verb|texhash| or
% \verb|mktexlsr|.
%
% \subsection{Some details for the interested}
%
% \paragraph{Unpacking with \LaTeX.}
% The \xfile{.dtx} chooses its action depending on the format:
% \begin{description}
% \item[\plainTeX:] Run \docstrip\ and extract the files.
% \item[\LaTeX:] Generate the documentation.
% \end{description}
% If you insist on using \LaTeX\ for \docstrip\ (really,
% \docstrip\ does not need \LaTeX), then inform the autodetect routine
% about your intention:
% \begin{quote}
%   \verb|latex \let\install=y% \iffalse meta-comment
%
% File: letltxmacro.dtx
% Version: 2019/12/03 v1.6
% Info: Let assignment for LaTeX macros
%
% Copyright (C)
%    2008, 2010 Heiko Oberdiek
%    2016-2019 Oberdiek Package Support Group
%    https://github.com/ho-tex/letltxmacro/issues
%
% This work may be distributed and/or modified under the
% conditions of the LaTeX Project Public License, either
% version 1.3c of this license or (at your option) any later
% version. This version of this license is in
%    https://www.latex-project.org/lppl/lppl-1-3c.txt
% and the latest version of this license is in
%    https://www.latex-project.org/lppl.txt
% and version 1.3 or later is part of all distributions of
% LaTeX version 2005/12/01 or later.
%
% This work has the LPPL maintenance status "maintained".
%
% The Current Maintainers of this work are
% Heiko Oberdiek and the Oberdiek Package Support Group
% https://github.com/ho-tex/letltxmacro/issues
%
% This work consists of the main source file letltxmacro.dtx
% and the derived files
%    letltxmacro.sty, letltxmacro.pdf, letltxmacro.ins, letltxmacro.drv,
%    letltxmacro-showcases.tex, letltxmacro-test1.tex,
%    letltxmacro-test2.tex.
%
% Distribution:
%    CTAN:macros/latex/contrib/letltxmacro/letltxmacro.dtx
%    CTAN:macros/latex/contrib/letltxmacro/letltxmacro.pdf
%
% Unpacking:
%    (a) If letltxmacro.ins is present:
%           tex letltxmacro.ins
%    (b) Without letltxmacro.ins:
%           tex letltxmacro.dtx
%    (c) If you insist on using LaTeX
%           latex \let\install=y% \iffalse meta-comment
%
% File: letltxmacro.dtx
% Version: 2019/12/03 v1.6
% Info: Let assignment for LaTeX macros
%
% Copyright (C)
%    2008, 2010 Heiko Oberdiek
%    2016-2019 Oberdiek Package Support Group
%    https://github.com/ho-tex/letltxmacro/issues
%
% This work may be distributed and/or modified under the
% conditions of the LaTeX Project Public License, either
% version 1.3c of this license or (at your option) any later
% version. This version of this license is in
%    https://www.latex-project.org/lppl/lppl-1-3c.txt
% and the latest version of this license is in
%    https://www.latex-project.org/lppl.txt
% and version 1.3 or later is part of all distributions of
% LaTeX version 2005/12/01 or later.
%
% This work has the LPPL maintenance status "maintained".
%
% The Current Maintainers of this work are
% Heiko Oberdiek and the Oberdiek Package Support Group
% https://github.com/ho-tex/letltxmacro/issues
%
% This work consists of the main source file letltxmacro.dtx
% and the derived files
%    letltxmacro.sty, letltxmacro.pdf, letltxmacro.ins, letltxmacro.drv,
%    letltxmacro-showcases.tex, letltxmacro-test1.tex,
%    letltxmacro-test2.tex.
%
% Distribution:
%    CTAN:macros/latex/contrib/letltxmacro/letltxmacro.dtx
%    CTAN:macros/latex/contrib/letltxmacro/letltxmacro.pdf
%
% Unpacking:
%    (a) If letltxmacro.ins is present:
%           tex letltxmacro.ins
%    (b) Without letltxmacro.ins:
%           tex letltxmacro.dtx
%    (c) If you insist on using LaTeX
%           latex \let\install=y\input{letltxmacro.dtx}
%        (quote the arguments according to the demands of your shell)
%
% Documentation:
%    (a) If letltxmacro.drv is present:
%           latex letltxmacro.drv
%    (b) Without letltxmacro.drv:
%           latex letltxmacro.dtx; ...
%    The class ltxdoc loads the configuration file ltxdoc.cfg
%    if available. Here you can specify further options, e.g.
%    use A4 as paper format:
%       \PassOptionsToClass{a4paper}{article}
%
%    Programm calls to get the documentation (example):
%       pdflatex letltxmacro.dtx
%       makeindex -s gind.ist letltxmacro.idx
%       pdflatex letltxmacro.dtx
%       makeindex -s gind.ist letltxmacro.idx
%       pdflatex letltxmacro.dtx
%
% Installation:
%    TDS:tex/latex/letltxmacro/letltxmacro.sty
%    TDS:doc/latex/letltxmacro/letltxmacro.pdf
%    TDS:doc/latex/letltxmacro/letltxmacro-showcases.tex
%    TDS:source/latex/letltxmacro/letltxmacro.dtx
%
%<*ignore>
\begingroup
  \catcode123=1 %
  \catcode125=2 %
  \def\x{LaTeX2e}%
\expandafter\endgroup
\ifcase 0\ifx\install y1\fi\expandafter
         \ifx\csname processbatchFile\endcsname\relax\else1\fi
         \ifx\fmtname\x\else 1\fi\relax
\else\csname fi\endcsname
%</ignore>
%<*install>
\input docstrip.tex
\Msg{************************************************************************}
\Msg{* Installation}
\Msg{* Package: letltxmacro 2019/12/03 v1.6 Let assignment for LaTeX macros (HO)}
\Msg{************************************************************************}

\keepsilent
\askforoverwritefalse

\let\MetaPrefix\relax
\preamble

This is a generated file.

Project: letltxmacro
Version: 2019/12/03 v1.6

Copyright (C)
   2008, 2010 Heiko Oberdiek
   2016-2019 Oberdiek Package Support Group

This work may be distributed and/or modified under the
conditions of the LaTeX Project Public License, either
version 1.3c of this license or (at your option) any later
version. This version of this license is in
   https://www.latex-project.org/lppl/lppl-1-3c.txt
and the latest version of this license is in
   https://www.latex-project.org/lppl.txt
and version 1.3 or later is part of all distributions of
LaTeX version 2005/12/01 or later.

This work has the LPPL maintenance status "maintained".

The Current Maintainers of this work are
Heiko Oberdiek and the Oberdiek Package Support Group
https://github.com/ho-tex/letltxmacro/issues


This work consists of the main source file letltxmacro.dtx
and the derived files
   letltxmacro.sty, letltxmacro.pdf, letltxmacro.ins, letltxmacro.drv,
   letltxmacro-showcases.tex, letltxmacro-test1.tex,
   letltxmacro-test2.tex.

\endpreamble
\let\MetaPrefix\DoubleperCent

\generate{%
  \file{letltxmacro.ins}{\from{letltxmacro.dtx}{install}}%
  \file{letltxmacro.drv}{\from{letltxmacro.dtx}{driver}}%
  \usedir{tex/latex/letltxmacro}%
  \file{letltxmacro.sty}{\from{letltxmacro.dtx}{package}}%
  \usedir{doc/latex/letltxmacro}%
  \file{letltxmacro-showcases.tex}{\from{letltxmacro.dtx}{showcases}}%
%  \usedir{doc/latex/letltxmacro/test}%
%  \file{letltxmacro-test1.tex}{\from{letltxmacro.dtx}{test1}}%
%  \file{letltxmacro-test2.tex}{\from{letltxmacro.dtx}{test2}}%
}

\catcode32=13\relax% active space
\let =\space%
\Msg{************************************************************************}
\Msg{*}
\Msg{* To finish the installation you have to move the following}
\Msg{* file into a directory searched by TeX:}
\Msg{*}
\Msg{*     letltxmacro.sty}
\Msg{*}
\Msg{* To produce the documentation run the file `letltxmacro.drv'}
\Msg{* through LaTeX.}
\Msg{*}
\Msg{* Happy TeXing!}
\Msg{*}
\Msg{************************************************************************}

\endbatchfile
%</install>
%<*ignore>
\fi
%</ignore>
%<*driver>
\NeedsTeXFormat{LaTeX2e}
\ProvidesFile{letltxmacro.drv}%
  [2019/12/03 v1.6 Let assignment for LaTeX macros (HO)]%
\documentclass{ltxdoc}
\usepackage{holtxdoc}[2011/11/22]
\begin{document}
  \DocInput{letltxmacro.dtx}%
\end{document}
%</driver>
% \fi
%
%
%
% \GetFileInfo{letltxmacro.drv}
%
% \title{The \xpackage{letltxmacro} package}
% \date{2019/12/03 v1.6}
% \author{Heiko Oberdiek\thanks
% {Please report any issues at \url{https://github.com/ho-tex/letltxmacro/issues}}}
%
% \maketitle
%
% \begin{abstract}
% \TeX's \cs{let} assignment does not work for \LaTeX\ macros
% with optional arguments or for macros that are defined
% as robust macros by \cs{DeclareRobustCommand}. This package
% defines \cs{LetLtxMacro} that also takes care of the involved
% internal macros.
% \end{abstract}
%
% \tableofcontents
%
% \section{Documentation}
%
% If someone wants to redefine a macro with using the old
% meaning, then one method is \TeX's command \cs{let}:
%\begin{quote}
%\begin{verbatim}
%\newcommand{\Macro}{\typeout{Test Macro}}
%\let\SavedMacro=\Macro
%\renewcommand{\Macro}{%
%  \typeout{Begin}%
%  \SavedMacro
%  \typeout{End}%
%}
%\end{verbatim}
%\end{quote}
% However, this method fails, if \cs{Macro} is defined
% by \cs{DeclareRobustCommand} and/or has an optional argument.
% In both cases \LaTeX\ defines an additional internal macro
% that is forgotten in the simple \cs{let} assignment of
% the example above.
%
% \begin{declcs}{LetLtxMacro} \M{new macro} \M{old macro}
% \end{declcs}
% Macro \cs{LetLtxMacro} behaves similar to \TeX's \cs{let}
% assignment, but it takes care of macros that are
% defined by \cs{DeclareRobustCommand} and/or have optional
% arguments. Example:
%\begin{quote}
%\begin{verbatim}
%\DeclareRobustCommand{\Macro}[1][default]{...}
%\LetLtxMacro{\SavedMacro}{\Macro}
%\end{verbatim}
%\end{quote}
% Then macro \cs{SavedMacro} only uses internal macro names
% that are derived from \cs{SavedMacro}'s macro name. Macro \cs{Macro}
% can now be redefined without affecting \cs{SavedMacro}.
%
% \begin{declcs}{GlobalLetLtxMacro} \M{new macro} \M{old macro}
% \end{declcs}
% Like \cs{LetLtxMacro}, but the \meta{new macro} is defined globally.
% Since version 2019/12/03~v1.4.
%
% \subsection{Supported macro definition commands}
%
% \begin{quote}
%   \begin{tabular}{@{}ll@{}}
%     \cs{newcommand}, \cs{renewcommand} & latex/base\\
%     \cs{newenvironment}, \cs{renewenvironment} & latex/base\\
%     \cs{DeclareRobustCommand}& latex/base\\
%     \cs{newrobustcmd}, \cs{renewrobustcmd} & etoolbox\\
%     \cs{robustify} & etoolbox 2008/06/22 v1.6\\
%   \end{tabular}
% \end{quote}
%
% \StopEventually{
% }
%
% \section{Implementation}
%
% \subsection{Show cases}
%
% \subsubsection{\xfile{letltxmacro-showcases.tex}}
%
%    \begin{macrocode}
%<*showcases>
\NeedsTeXFormat{LaTeX2e}
\makeatletter
%    \end{macrocode}
%    \begin{macro}{\Line}
%    The result is displayed by macro \cs{Line}. The percent symbol
%    at line start allows easy grepping and inserting into the DTX
%    file.
%    \begin{macrocode}
\newcommand*{\Line}[1]{%
  \typeout{\@percentchar#1}%
}
%    \end{macrocode}
%    \end{macro}
%    \begin{macrocode}
\newcommand*{\ShowCmdName}[1]{%
  \@ifundefined{#1}{}{%
    \Line{%
      \space\space(\expandafter\string\csname#1\endcsname) = %
      (\expandafter\meaning\csname#1\endcsname)%
    }%
  }%
}
\newcommand*{\ShowCmds}[1]{%
  \ShowCmdName{#1}%
  \ShowCmdName{#1 }%
  \ShowCmdName{\\#1}%
  \ShowCmdName{\\#1 }%
}
\let\\\@backslashchar
%    \end{macrocode}
%    \begin{macro}{\ShowDef}
%    \begin{macrocode}
\newcommand*{\ShowDef}[2]{%
  \begingroup
    \Line{}%
    \newcommand*{\DefString}{#2}%
    \@onelevel@sanitize\DefString
    \Line{\DefString}%
    #2%
    \ShowCmds{#1}%
  \endgroup
}
%    \end{macrocode}
%    \end{macro}
%    \begin{macrocode}
\typeout{}
\Line{* LaTeX definitions:}
\ShowDef{cmd}{%
  \newcommand{\cmd}[2][default]{}%
}
\ShowDef{cmd}{%
  \DeclareRobustCommand{\cmd}{}%
}
\ShowDef{cmd}{%
  \DeclareRobustCommand{\cmd}[2][default]{}%
}
\typeout{}
%    \end{macrocode}
% The minimal version of package \xpackage{etoolbox} is 2008/06/12 v1.6a
% because it fixes \cs{robustify}.
%    \begin{macrocode}
\RequirePackage{etoolbox}[2008/06/12]%
\Line{}
\Line{* etoolbox's robust definitions:}
\ShowDef{cmd}{%
  \newrobustcmd{\cmd}{}%
}
\ShowDef{cmd}{%
  \newrobustcmd{\cmd}[2][default]{}%
}
\Line{}
\Line{* etoolbox's \string\robustify:}
\ShowDef{cmd}{%
  \newcommand{\cmd}[2][default]{} %
  \robustify{\cmd}%
}
\ShowDef{cmd}{%
  \DeclareRobustCommand{\cmd}{} %
  \robustify{\cmd}%
}
\ShowDef{cmd}{%
  \DeclareRobustCommand{\cmd}[2][default]{} %
  \robustify{\cmd}%
}
\typeout{}
\@@end
%</showcases>
%    \end{macrocode}
%
% \subsubsection{Result}
%
% \begingroup
%   \makeatletter
%   \let\org@verbatim\@verbatim
%   \def\@verbatim{^^A
%     \org@verbatim
%     \catcode`\~=\active
%   }^^A
%   \let~\textvisiblespace
%\begin{verbatim}
%* LaTeX definitions:
%
%\newcommand {\cmd }[2][default]{}
%  (\cmd) = (macro:->\@protected@testopt \cmd \\cmd {default})
%  (\\cmd) = (\long macro:[#1]#2->)
%
%\DeclareRobustCommand {\cmd }{}
%  (\cmd) = (macro:->\protect \cmd~ )
%  (\cmd~) = (\long macro:->)
%
%\DeclareRobustCommand {\cmd }[2][default]{}
%  (\cmd) = (macro:->\protect \cmd~ )
%  (\cmd~) = (macro:->\@protected@testopt \cmd~ \\cmd~ {default})
%  (\\cmd~) = (\long macro:[#1]#2->)
%
%* etoolbox's robust definitions:
%
%\newrobustcmd {\cmd }{}
%  (\cmd) = (\protected\long macro:->)
%
%\newrobustcmd {\cmd }[2][default]{}
%  (\cmd) = (\protected macro:->\@testopt \\cmd {default})
%  (\\cmd) = (\long macro:[#1]#2->)
%
%* etoolbox's \robustify:
%
%\newcommand {\cmd }[2][default]{} \robustify {\cmd }
%  (\cmd) = (\protected macro:->\@protected@testopt \cmd \\cmd {default})
%  (\\cmd) = (\long macro:[#1]#2->)
%
%\DeclareRobustCommand {\cmd }{} \robustify {\cmd }
%  (\cmd) = (\protected macro:->)
%
%\DeclareRobustCommand {\cmd }[2][default]{} \robustify {\cmd }
%  (\cmd) = (\protected macro:->\@protected@testopt \cmd~ \\cmd~ {default})
%  (\cmd~) = (macro:->\@protected@testopt \cmd~ \\cmd~ {default})
%  (\\cmd~) = (\long macro:[#1]#2->)
%\end{verbatim}
% \endgroup
%
% \subsection{Package}
%
%    \begin{macrocode}
%<*package>
%    \end{macrocode}
%
% \subsubsection{Catcodes and identification}
%
%    \begin{macrocode}
\begingroup\catcode61\catcode48\catcode32=10\relax%
  \catcode13=5 % ^^M
  \endlinechar=13 %
  \catcode123=1 % {
  \catcode125=2 % }
  \catcode64=11 % @
  \def\x{\endgroup
    \expandafter\edef\csname llm@AtEnd\endcsname{%
      \endlinechar=\the\endlinechar\relax
      \catcode13=\the\catcode13\relax
      \catcode32=\the\catcode32\relax
      \catcode35=\the\catcode35\relax
      \catcode61=\the\catcode61\relax
      \catcode64=\the\catcode64\relax
      \catcode123=\the\catcode123\relax
      \catcode125=\the\catcode125\relax
    }%
  }%
\x\catcode61\catcode48\catcode32=10\relax%
\catcode13=5 % ^^M
\endlinechar=13 %
\catcode35=6 % #
\catcode64=11 % @
\catcode123=1 % {
\catcode125=2 % }
\def\TMP@EnsureCode#1#2{%
  \edef\llm@AtEnd{%
    \llm@AtEnd
    \catcode#1=\the\catcode#1\relax
  }%
  \catcode#1=#2\relax
}
\TMP@EnsureCode{40}{12}% (
\TMP@EnsureCode{41}{12}% )
\TMP@EnsureCode{42}{12}% *
\TMP@EnsureCode{45}{12}% -
\TMP@EnsureCode{46}{12}% .
\TMP@EnsureCode{47}{12}% /
\TMP@EnsureCode{58}{12}% :
\TMP@EnsureCode{62}{12}% >
\TMP@EnsureCode{91}{12}% [
\TMP@EnsureCode{93}{12}% ]
\edef\llm@AtEnd{%
  \llm@AtEnd
  \escapechar\the\escapechar\relax
  \noexpand\endinput
}
\escapechar=92 % `\\
%    \end{macrocode}
%
%    Package identification.
%    \begin{macrocode}
\NeedsTeXFormat{LaTeX2e}
\ProvidesPackage{letltxmacro}%
  [2019/12/03 v1.6 Let assignment for LaTeX macros (HO)]
%    \end{macrocode}
%
% \subsubsection{Main macros}
%
%    \begin{macro}{\LetLtxMacro}
%    \begin{macrocode}
\newcommand*{\LetLtxMacro}{%
  \llm@ModeLetLtxMacro{}%
}
%    \end{macrocode}
%    \end{macro}
%    \begin{macro}{\GlobalLetLtxMacro}
%    \begin{macrocode}
\newcommand*{\GlobalLetLtxMacro}{%
  \llm@ModeLetLtxMacro\global
}
%    \end{macrocode}
%    \end{macro}
%
%    \begin{macro}{\llm@ModeLetLtxMacro}
%    \begin{macrocode}
\newcommand*{\llm@ModeLetLtxMacro}[3]{%
  \edef\llm@escapechar{\the\escapechar}%
  \escapechar=-1 %
  \edef\reserved@a{%
    \noexpand\protect
    \expandafter\noexpand
    \csname\string#3 \endcsname
  }%
  \ifx\reserved@a#3\relax
    #1\edef#2{%
      \noexpand\protect
      \expandafter\noexpand
      \csname\string#2 \endcsname
    }%
    #1\expandafter\let
    \csname\string#2 \expandafter\endcsname
    \csname\string#3 \endcsname
    \expandafter\llm@LetLtxMacro
        \csname\string#2 \expandafter\endcsname
        \csname\string#3 \endcsname{#1}%
  \else
    \llm@LetLtxMacro{#2}{#3}{#1}%
  \fi
  \escapechar=\llm@escapechar\relax
}
%    \end{macrocode}
%    \end{macro}
%    \begin{macro}{\llm@LetLtxMacro}
%    \begin{macrocode}
\def\llm@LetLtxMacro#1#2#3{%
  \escapechar=92 %
  \expandafter\llm@CheckParams\meaning#2:->\@nil{%
    \begingroup
      \def\@protected@testopt{%
        \expandafter\@testopt\@gobble
      }%
      \def\@testopt##1##2{%
        \toks@={##2}%
      }%
      \let\llm@testopt\@empty
      \edef\x{%
        \noexpand\@protected@testopt
        \noexpand#2%
        \expandafter\noexpand\csname\string#2\endcsname
      }%
      \expandafter\expandafter\expandafter\def
      \expandafter\expandafter\expandafter\y
      \expandafter\expandafter\expandafter{%
        \expandafter\llm@CarThree#2{}{}{}\llm@nil
      }%
      \ifx\x\y
        #2%
        \def\llm@testopt{%
          \noexpand\@protected@testopt
          \noexpand#1%
        }%
      \else
        \edef\x{%
          \noexpand\@testopt
          \expandafter\noexpand
          \csname\string#2\endcsname
        }%
        \expandafter\expandafter\expandafter\def
        \expandafter\expandafter\expandafter\y
        \expandafter\expandafter\expandafter{%
          \expandafter\llm@CarTwo#2{}{}\llm@nil
        }%
        \ifx\x\y
          #2%
          \def\llm@testopt{%
            \noexpand\@testopt
          }%
        \fi
      \fi
      \ifx\llm@testopt\@empty
      \else
        \llm@protected\xdef\llm@GlobalTemp{%
          \llm@testopt
          \expandafter\noexpand
          \csname\string#1\endcsname
          {\the\toks@}%
        }%
      \fi
    \expandafter\endgroup\ifx\llm@testopt\@empty
      #3\let#1=#2\relax
    \else
      #3\let#1=\llm@GlobalTemp
      #3\expandafter\let
          \csname\string#1\expandafter\endcsname
          \csname\string#2\endcsname
    \fi
  }{%
    #3\let#1=#2\relax
  }%
}
%    \end{macrocode}
%    \end{macro}
%    \begin{macro}{\llm@CheckParams}
%    \begin{macrocode}
\def\llm@CheckParams#1:->#2\@nil{%
  \begingroup
    \def\x{#1}%
  \ifx\x\llm@macro
    \endgroup
    \def\llm@protected{}%
    \expandafter\@firstoftwo
  \else
    \ifx\x\llm@protectedmacro
      \endgroup
      \def\llm@protected{\protected}%
      \expandafter\expandafter\expandafter\@firstoftwo
    \else
      \endgroup
      \expandafter\expandafter\expandafter\@secondoftwo
    \fi
  \fi
}
%    \end{macrocode}
%    \end{macro}
%    \begin{macro}{\llm@macro}
%    \begin{macrocode}
\def\llm@macro{macro}
\@onelevel@sanitize\llm@macro
%    \end{macrocode}
%    \end{macro}
%    \begin{macro}{\llm@protectedmacro}
%    \begin{macrocode}
\def\llm@protectedmacro{\protected macro}
\@onelevel@sanitize\llm@protectedmacro
%    \end{macrocode}
%    \end{macro}
%    \begin{macro}{\llm@CarThree}
%    \begin{macrocode}
\def\llm@CarThree#1#2#3#4\llm@nil{#1#2#3}%
%    \end{macrocode}
%    \end{macro}
%    \begin{macro}{\llm@CarTwo}
%    \begin{macrocode}
\def\llm@CarTwo#1#2#3\llm@nil{#1#2}%
%    \end{macrocode}
%    \end{macro}
%
%    \begin{macrocode}
\llm@AtEnd%
%</package>
%    \end{macrocode}
% \section{Installation}
%
% \subsection{Download}
%
% \paragraph{Package.} This package is available on
% CTAN\footnote{\CTANpkg{letltxmacro}}:
% \begin{description}
% \item[\CTAN{macros/latex/contrib/letltxmacro/letltxmacro.dtx}] The source file.
% \item[\CTAN{macros/latex/contrib/letltxmacro/letltxmacro.pdf}] Documentation.
% \end{description}
%
%
% \paragraph{Bundle.} All the packages of the bundle `letltxmacro'
% are also available in a TDS compliant ZIP archive. There
% the packages are already unpacked and the documentation files
% are generated. The files and directories obey the TDS standard.
% \begin{description}
% \item[\CTANinstall{install/macros/latex/contrib/letltxmacro.tds.zip}]
% \end{description}
% \emph{TDS} refers to the standard ``A Directory Structure
% for \TeX\ Files'' (\CTANpkg{tds}). Directories
% with \xfile{texmf} in their name are usually organized this way.
%
% \subsection{Bundle installation}
%
% \paragraph{Unpacking.} Unpack the \xfile{letltxmacro.tds.zip} in the
% TDS tree (also known as \xfile{texmf} tree) of your choice.
% Example (linux):
% \begin{quote}
%   |unzip letltxmacro.tds.zip -d ~/texmf|
% \end{quote}
%
% \subsection{Package installation}
%
% \paragraph{Unpacking.} The \xfile{.dtx} file is a self-extracting
% \docstrip\ archive. The files are extracted by running the
% \xfile{.dtx} through \plainTeX:
% \begin{quote}
%   \verb|tex letltxmacro.dtx|
% \end{quote}
%
% \paragraph{TDS.} Now the different files must be moved into
% the different directories in your installation TDS tree
% (also known as \xfile{texmf} tree):
% \begin{quote}
% \def\t{^^A
% \begin{tabular}{@{}>{\ttfamily}l@{ $\rightarrow$ }>{\ttfamily}l@{}}
%   letltxmacro.sty & tex/latex/letltxmacro/letltxmacro.sty\\
%   letltxmacro.pdf & doc/latex/letltxmacro/letltxmacro.pdf\\
%   letltxmacro-showcases.tex & doc/latex/letltxmacro/letltxmacro-showcases.tex\\
%   letltxmacro.dtx & source/latex/letltxmacro/letltxmacro.dtx\\
% \end{tabular}^^A
% }^^A
% \sbox0{\t}^^A
% \ifdim\wd0>\linewidth
%   \begingroup
%     \advance\linewidth by\leftmargin
%     \advance\linewidth by\rightmargin
%   \edef\x{\endgroup
%     \def\noexpand\lw{\the\linewidth}^^A
%   }\x
%   \def\lwbox{^^A
%     \leavevmode
%     \hbox to \linewidth{^^A
%       \kern-\leftmargin\relax
%       \hss
%       \usebox0
%       \hss
%       \kern-\rightmargin\relax
%     }^^A
%   }^^A
%   \ifdim\wd0>\lw
%     \sbox0{\small\t}^^A
%     \ifdim\wd0>\linewidth
%       \ifdim\wd0>\lw
%         \sbox0{\footnotesize\t}^^A
%         \ifdim\wd0>\linewidth
%           \ifdim\wd0>\lw
%             \sbox0{\scriptsize\t}^^A
%             \ifdim\wd0>\linewidth
%               \ifdim\wd0>\lw
%                 \sbox0{\tiny\t}^^A
%                 \ifdim\wd0>\linewidth
%                   \lwbox
%                 \else
%                   \usebox0
%                 \fi
%               \else
%                 \lwbox
%               \fi
%             \else
%               \usebox0
%             \fi
%           \else
%             \lwbox
%           \fi
%         \else
%           \usebox0
%         \fi
%       \else
%         \lwbox
%       \fi
%     \else
%       \usebox0
%     \fi
%   \else
%     \lwbox
%   \fi
% \else
%   \usebox0
% \fi
% \end{quote}
% If you have a \xfile{docstrip.cfg} that configures and enables \docstrip's
% TDS installing feature, then some files can already be in the right
% place, see the documentation of \docstrip.
%
% \subsection{Refresh file name databases}
%
% If your \TeX~distribution
% (\TeX\,Live, \mikTeX, \dots) relies on file name databases, you must refresh
% these. For example, \TeX\,Live\ users run \verb|texhash| or
% \verb|mktexlsr|.
%
% \subsection{Some details for the interested}
%
% \paragraph{Unpacking with \LaTeX.}
% The \xfile{.dtx} chooses its action depending on the format:
% \begin{description}
% \item[\plainTeX:] Run \docstrip\ and extract the files.
% \item[\LaTeX:] Generate the documentation.
% \end{description}
% If you insist on using \LaTeX\ for \docstrip\ (really,
% \docstrip\ does not need \LaTeX), then inform the autodetect routine
% about your intention:
% \begin{quote}
%   \verb|latex \let\install=y\input{letltxmacro.dtx}|
% \end{quote}
% Do not forget to quote the argument according to the demands
% of your shell.
%
% \paragraph{Generating the documentation.}
% You can use both the \xfile{.dtx} or the \xfile{.drv} to generate
% the documentation. The process can be configured by the
% configuration file \xfile{ltxdoc.cfg}. For instance, put this
% line into this file, if you want to have A4 as paper format:
% \begin{quote}
%   \verb|\PassOptionsToClass{a4paper}{article}|
% \end{quote}
% An example follows how to generate the
% documentation with pdf\LaTeX:
% \begin{quote}
%\begin{verbatim}
%pdflatex letltxmacro.dtx
%makeindex -s gind.ist letltxmacro.idx
%pdflatex letltxmacro.dtx
%makeindex -s gind.ist letltxmacro.idx
%pdflatex letltxmacro.dtx
%\end{verbatim}
% \end{quote}
%
% \begin{History}
%   \begin{Version}{2008/06/09 v1.0}
%   \item
%     First version.
%   \end{Version}
%   \begin{Version}{2008/06/12 v1.1}
%   \item
%     Support for \xpackage{etoolbox}'s \cs{newrobustcmd} added.
%   \end{Version}
%   \begin{Version}{2008/06/13 v1.2}
%   \item
%     Support for \xpackage{etoolbox}'s \cs{robustify} added.
%   \end{Version}
%   \begin{Version}{2008/06/24 v1.3}
%   \item
%     Test file adapted for etoolbox 2008/06/22 v1.6.
%   \end{Version}
%   \begin{Version}{2010/09/02 v1.4}
%   \item
%     \cs{GlobalLetLtxMacro} added.
%   \end{Version}
%   \begin{Version}{2016/05/16 v1.5}
%   \item
%     Documentation updates.
%   \end{Version}
%   \begin{Version}{2019/12/03 v1.6}
%   \item
%     Documentation updates.
%   \end{Version}
% \end{History}
%
% \PrintIndex
%
% \Finale
\endinput

%        (quote the arguments according to the demands of your shell)
%
% Documentation:
%    (a) If letltxmacro.drv is present:
%           latex letltxmacro.drv
%    (b) Without letltxmacro.drv:
%           latex letltxmacro.dtx; ...
%    The class ltxdoc loads the configuration file ltxdoc.cfg
%    if available. Here you can specify further options, e.g.
%    use A4 as paper format:
%       \PassOptionsToClass{a4paper}{article}
%
%    Programm calls to get the documentation (example):
%       pdflatex letltxmacro.dtx
%       makeindex -s gind.ist letltxmacro.idx
%       pdflatex letltxmacro.dtx
%       makeindex -s gind.ist letltxmacro.idx
%       pdflatex letltxmacro.dtx
%
% Installation:
%    TDS:tex/latex/letltxmacro/letltxmacro.sty
%    TDS:doc/latex/letltxmacro/letltxmacro.pdf
%    TDS:doc/latex/letltxmacro/letltxmacro-showcases.tex
%    TDS:source/latex/letltxmacro/letltxmacro.dtx
%
%<*ignore>
\begingroup
  \catcode123=1 %
  \catcode125=2 %
  \def\x{LaTeX2e}%
\expandafter\endgroup
\ifcase 0\ifx\install y1\fi\expandafter
         \ifx\csname processbatchFile\endcsname\relax\else1\fi
         \ifx\fmtname\x\else 1\fi\relax
\else\csname fi\endcsname
%</ignore>
%<*install>
\input docstrip.tex
\Msg{************************************************************************}
\Msg{* Installation}
\Msg{* Package: letltxmacro 2019/12/03 v1.6 Let assignment for LaTeX macros (HO)}
\Msg{************************************************************************}

\keepsilent
\askforoverwritefalse

\let\MetaPrefix\relax
\preamble

This is a generated file.

Project: letltxmacro
Version: 2019/12/03 v1.6

Copyright (C)
   2008, 2010 Heiko Oberdiek
   2016-2019 Oberdiek Package Support Group

This work may be distributed and/or modified under the
conditions of the LaTeX Project Public License, either
version 1.3c of this license or (at your option) any later
version. This version of this license is in
   https://www.latex-project.org/lppl/lppl-1-3c.txt
and the latest version of this license is in
   https://www.latex-project.org/lppl.txt
and version 1.3 or later is part of all distributions of
LaTeX version 2005/12/01 or later.

This work has the LPPL maintenance status "maintained".

The Current Maintainers of this work are
Heiko Oberdiek and the Oberdiek Package Support Group
https://github.com/ho-tex/letltxmacro/issues


This work consists of the main source file letltxmacro.dtx
and the derived files
   letltxmacro.sty, letltxmacro.pdf, letltxmacro.ins, letltxmacro.drv,
   letltxmacro-showcases.tex, letltxmacro-test1.tex,
   letltxmacro-test2.tex.

\endpreamble
\let\MetaPrefix\DoubleperCent

\generate{%
  \file{letltxmacro.ins}{\from{letltxmacro.dtx}{install}}%
  \file{letltxmacro.drv}{\from{letltxmacro.dtx}{driver}}%
  \usedir{tex/latex/letltxmacro}%
  \file{letltxmacro.sty}{\from{letltxmacro.dtx}{package}}%
  \usedir{doc/latex/letltxmacro}%
  \file{letltxmacro-showcases.tex}{\from{letltxmacro.dtx}{showcases}}%
%  \usedir{doc/latex/letltxmacro/test}%
%  \file{letltxmacro-test1.tex}{\from{letltxmacro.dtx}{test1}}%
%  \file{letltxmacro-test2.tex}{\from{letltxmacro.dtx}{test2}}%
}

\catcode32=13\relax% active space
\let =\space%
\Msg{************************************************************************}
\Msg{*}
\Msg{* To finish the installation you have to move the following}
\Msg{* file into a directory searched by TeX:}
\Msg{*}
\Msg{*     letltxmacro.sty}
\Msg{*}
\Msg{* To produce the documentation run the file `letltxmacro.drv'}
\Msg{* through LaTeX.}
\Msg{*}
\Msg{* Happy TeXing!}
\Msg{*}
\Msg{************************************************************************}

\endbatchfile
%</install>
%<*ignore>
\fi
%</ignore>
%<*driver>
\NeedsTeXFormat{LaTeX2e}
\ProvidesFile{letltxmacro.drv}%
  [2019/12/03 v1.6 Let assignment for LaTeX macros (HO)]%
\documentclass{ltxdoc}
\usepackage{holtxdoc}[2011/11/22]
\begin{document}
  \DocInput{letltxmacro.dtx}%
\end{document}
%</driver>
% \fi
%
%
%
% \GetFileInfo{letltxmacro.drv}
%
% \title{The \xpackage{letltxmacro} package}
% \date{2019/12/03 v1.6}
% \author{Heiko Oberdiek\thanks
% {Please report any issues at \url{https://github.com/ho-tex/letltxmacro/issues}}}
%
% \maketitle
%
% \begin{abstract}
% \TeX's \cs{let} assignment does not work for \LaTeX\ macros
% with optional arguments or for macros that are defined
% as robust macros by \cs{DeclareRobustCommand}. This package
% defines \cs{LetLtxMacro} that also takes care of the involved
% internal macros.
% \end{abstract}
%
% \tableofcontents
%
% \section{Documentation}
%
% If someone wants to redefine a macro with using the old
% meaning, then one method is \TeX's command \cs{let}:
%\begin{quote}
%\begin{verbatim}
%\newcommand{\Macro}{\typeout{Test Macro}}
%\let\SavedMacro=\Macro
%\renewcommand{\Macro}{%
%  \typeout{Begin}%
%  \SavedMacro
%  \typeout{End}%
%}
%\end{verbatim}
%\end{quote}
% However, this method fails, if \cs{Macro} is defined
% by \cs{DeclareRobustCommand} and/or has an optional argument.
% In both cases \LaTeX\ defines an additional internal macro
% that is forgotten in the simple \cs{let} assignment of
% the example above.
%
% \begin{declcs}{LetLtxMacro} \M{new macro} \M{old macro}
% \end{declcs}
% Macro \cs{LetLtxMacro} behaves similar to \TeX's \cs{let}
% assignment, but it takes care of macros that are
% defined by \cs{DeclareRobustCommand} and/or have optional
% arguments. Example:
%\begin{quote}
%\begin{verbatim}
%\DeclareRobustCommand{\Macro}[1][default]{...}
%\LetLtxMacro{\SavedMacro}{\Macro}
%\end{verbatim}
%\end{quote}
% Then macro \cs{SavedMacro} only uses internal macro names
% that are derived from \cs{SavedMacro}'s macro name. Macro \cs{Macro}
% can now be redefined without affecting \cs{SavedMacro}.
%
% \begin{declcs}{GlobalLetLtxMacro} \M{new macro} \M{old macro}
% \end{declcs}
% Like \cs{LetLtxMacro}, but the \meta{new macro} is defined globally.
% Since version 2019/12/03~v1.4.
%
% \subsection{Supported macro definition commands}
%
% \begin{quote}
%   \begin{tabular}{@{}ll@{}}
%     \cs{newcommand}, \cs{renewcommand} & latex/base\\
%     \cs{newenvironment}, \cs{renewenvironment} & latex/base\\
%     \cs{DeclareRobustCommand}& latex/base\\
%     \cs{newrobustcmd}, \cs{renewrobustcmd} & etoolbox\\
%     \cs{robustify} & etoolbox 2008/06/22 v1.6\\
%   \end{tabular}
% \end{quote}
%
% \StopEventually{
% }
%
% \section{Implementation}
%
% \subsection{Show cases}
%
% \subsubsection{\xfile{letltxmacro-showcases.tex}}
%
%    \begin{macrocode}
%<*showcases>
\NeedsTeXFormat{LaTeX2e}
\makeatletter
%    \end{macrocode}
%    \begin{macro}{\Line}
%    The result is displayed by macro \cs{Line}. The percent symbol
%    at line start allows easy grepping and inserting into the DTX
%    file.
%    \begin{macrocode}
\newcommand*{\Line}[1]{%
  \typeout{\@percentchar#1}%
}
%    \end{macrocode}
%    \end{macro}
%    \begin{macrocode}
\newcommand*{\ShowCmdName}[1]{%
  \@ifundefined{#1}{}{%
    \Line{%
      \space\space(\expandafter\string\csname#1\endcsname) = %
      (\expandafter\meaning\csname#1\endcsname)%
    }%
  }%
}
\newcommand*{\ShowCmds}[1]{%
  \ShowCmdName{#1}%
  \ShowCmdName{#1 }%
  \ShowCmdName{\\#1}%
  \ShowCmdName{\\#1 }%
}
\let\\\@backslashchar
%    \end{macrocode}
%    \begin{macro}{\ShowDef}
%    \begin{macrocode}
\newcommand*{\ShowDef}[2]{%
  \begingroup
    \Line{}%
    \newcommand*{\DefString}{#2}%
    \@onelevel@sanitize\DefString
    \Line{\DefString}%
    #2%
    \ShowCmds{#1}%
  \endgroup
}
%    \end{macrocode}
%    \end{macro}
%    \begin{macrocode}
\typeout{}
\Line{* LaTeX definitions:}
\ShowDef{cmd}{%
  \newcommand{\cmd}[2][default]{}%
}
\ShowDef{cmd}{%
  \DeclareRobustCommand{\cmd}{}%
}
\ShowDef{cmd}{%
  \DeclareRobustCommand{\cmd}[2][default]{}%
}
\typeout{}
%    \end{macrocode}
% The minimal version of package \xpackage{etoolbox} is 2008/06/12 v1.6a
% because it fixes \cs{robustify}.
%    \begin{macrocode}
\RequirePackage{etoolbox}[2008/06/12]%
\Line{}
\Line{* etoolbox's robust definitions:}
\ShowDef{cmd}{%
  \newrobustcmd{\cmd}{}%
}
\ShowDef{cmd}{%
  \newrobustcmd{\cmd}[2][default]{}%
}
\Line{}
\Line{* etoolbox's \string\robustify:}
\ShowDef{cmd}{%
  \newcommand{\cmd}[2][default]{} %
  \robustify{\cmd}%
}
\ShowDef{cmd}{%
  \DeclareRobustCommand{\cmd}{} %
  \robustify{\cmd}%
}
\ShowDef{cmd}{%
  \DeclareRobustCommand{\cmd}[2][default]{} %
  \robustify{\cmd}%
}
\typeout{}
\@@end
%</showcases>
%    \end{macrocode}
%
% \subsubsection{Result}
%
% \begingroup
%   \makeatletter
%   \let\org@verbatim\@verbatim
%   \def\@verbatim{^^A
%     \org@verbatim
%     \catcode`\~=\active
%   }^^A
%   \let~\textvisiblespace
%\begin{verbatim}
%* LaTeX definitions:
%
%\newcommand {\cmd }[2][default]{}
%  (\cmd) = (macro:->\@protected@testopt \cmd \\cmd {default})
%  (\\cmd) = (\long macro:[#1]#2->)
%
%\DeclareRobustCommand {\cmd }{}
%  (\cmd) = (macro:->\protect \cmd~ )
%  (\cmd~) = (\long macro:->)
%
%\DeclareRobustCommand {\cmd }[2][default]{}
%  (\cmd) = (macro:->\protect \cmd~ )
%  (\cmd~) = (macro:->\@protected@testopt \cmd~ \\cmd~ {default})
%  (\\cmd~) = (\long macro:[#1]#2->)
%
%* etoolbox's robust definitions:
%
%\newrobustcmd {\cmd }{}
%  (\cmd) = (\protected\long macro:->)
%
%\newrobustcmd {\cmd }[2][default]{}
%  (\cmd) = (\protected macro:->\@testopt \\cmd {default})
%  (\\cmd) = (\long macro:[#1]#2->)
%
%* etoolbox's \robustify:
%
%\newcommand {\cmd }[2][default]{} \robustify {\cmd }
%  (\cmd) = (\protected macro:->\@protected@testopt \cmd \\cmd {default})
%  (\\cmd) = (\long macro:[#1]#2->)
%
%\DeclareRobustCommand {\cmd }{} \robustify {\cmd }
%  (\cmd) = (\protected macro:->)
%
%\DeclareRobustCommand {\cmd }[2][default]{} \robustify {\cmd }
%  (\cmd) = (\protected macro:->\@protected@testopt \cmd~ \\cmd~ {default})
%  (\cmd~) = (macro:->\@protected@testopt \cmd~ \\cmd~ {default})
%  (\\cmd~) = (\long macro:[#1]#2->)
%\end{verbatim}
% \endgroup
%
% \subsection{Package}
%
%    \begin{macrocode}
%<*package>
%    \end{macrocode}
%
% \subsubsection{Catcodes and identification}
%
%    \begin{macrocode}
\begingroup\catcode61\catcode48\catcode32=10\relax%
  \catcode13=5 % ^^M
  \endlinechar=13 %
  \catcode123=1 % {
  \catcode125=2 % }
  \catcode64=11 % @
  \def\x{\endgroup
    \expandafter\edef\csname llm@AtEnd\endcsname{%
      \endlinechar=\the\endlinechar\relax
      \catcode13=\the\catcode13\relax
      \catcode32=\the\catcode32\relax
      \catcode35=\the\catcode35\relax
      \catcode61=\the\catcode61\relax
      \catcode64=\the\catcode64\relax
      \catcode123=\the\catcode123\relax
      \catcode125=\the\catcode125\relax
    }%
  }%
\x\catcode61\catcode48\catcode32=10\relax%
\catcode13=5 % ^^M
\endlinechar=13 %
\catcode35=6 % #
\catcode64=11 % @
\catcode123=1 % {
\catcode125=2 % }
\def\TMP@EnsureCode#1#2{%
  \edef\llm@AtEnd{%
    \llm@AtEnd
    \catcode#1=\the\catcode#1\relax
  }%
  \catcode#1=#2\relax
}
\TMP@EnsureCode{40}{12}% (
\TMP@EnsureCode{41}{12}% )
\TMP@EnsureCode{42}{12}% *
\TMP@EnsureCode{45}{12}% -
\TMP@EnsureCode{46}{12}% .
\TMP@EnsureCode{47}{12}% /
\TMP@EnsureCode{58}{12}% :
\TMP@EnsureCode{62}{12}% >
\TMP@EnsureCode{91}{12}% [
\TMP@EnsureCode{93}{12}% ]
\edef\llm@AtEnd{%
  \llm@AtEnd
  \escapechar\the\escapechar\relax
  \noexpand\endinput
}
\escapechar=92 % `\\
%    \end{macrocode}
%
%    Package identification.
%    \begin{macrocode}
\NeedsTeXFormat{LaTeX2e}
\ProvidesPackage{letltxmacro}%
  [2019/12/03 v1.6 Let assignment for LaTeX macros (HO)]
%    \end{macrocode}
%
% \subsubsection{Main macros}
%
%    \begin{macro}{\LetLtxMacro}
%    \begin{macrocode}
\newcommand*{\LetLtxMacro}{%
  \llm@ModeLetLtxMacro{}%
}
%    \end{macrocode}
%    \end{macro}
%    \begin{macro}{\GlobalLetLtxMacro}
%    \begin{macrocode}
\newcommand*{\GlobalLetLtxMacro}{%
  \llm@ModeLetLtxMacro\global
}
%    \end{macrocode}
%    \end{macro}
%
%    \begin{macro}{\llm@ModeLetLtxMacro}
%    \begin{macrocode}
\newcommand*{\llm@ModeLetLtxMacro}[3]{%
  \edef\llm@escapechar{\the\escapechar}%
  \escapechar=-1 %
  \edef\reserved@a{%
    \noexpand\protect
    \expandafter\noexpand
    \csname\string#3 \endcsname
  }%
  \ifx\reserved@a#3\relax
    #1\edef#2{%
      \noexpand\protect
      \expandafter\noexpand
      \csname\string#2 \endcsname
    }%
    #1\expandafter\let
    \csname\string#2 \expandafter\endcsname
    \csname\string#3 \endcsname
    \expandafter\llm@LetLtxMacro
        \csname\string#2 \expandafter\endcsname
        \csname\string#3 \endcsname{#1}%
  \else
    \llm@LetLtxMacro{#2}{#3}{#1}%
  \fi
  \escapechar=\llm@escapechar\relax
}
%    \end{macrocode}
%    \end{macro}
%    \begin{macro}{\llm@LetLtxMacro}
%    \begin{macrocode}
\def\llm@LetLtxMacro#1#2#3{%
  \escapechar=92 %
  \expandafter\llm@CheckParams\meaning#2:->\@nil{%
    \begingroup
      \def\@protected@testopt{%
        \expandafter\@testopt\@gobble
      }%
      \def\@testopt##1##2{%
        \toks@={##2}%
      }%
      \let\llm@testopt\@empty
      \edef\x{%
        \noexpand\@protected@testopt
        \noexpand#2%
        \expandafter\noexpand\csname\string#2\endcsname
      }%
      \expandafter\expandafter\expandafter\def
      \expandafter\expandafter\expandafter\y
      \expandafter\expandafter\expandafter{%
        \expandafter\llm@CarThree#2{}{}{}\llm@nil
      }%
      \ifx\x\y
        #2%
        \def\llm@testopt{%
          \noexpand\@protected@testopt
          \noexpand#1%
        }%
      \else
        \edef\x{%
          \noexpand\@testopt
          \expandafter\noexpand
          \csname\string#2\endcsname
        }%
        \expandafter\expandafter\expandafter\def
        \expandafter\expandafter\expandafter\y
        \expandafter\expandafter\expandafter{%
          \expandafter\llm@CarTwo#2{}{}\llm@nil
        }%
        \ifx\x\y
          #2%
          \def\llm@testopt{%
            \noexpand\@testopt
          }%
        \fi
      \fi
      \ifx\llm@testopt\@empty
      \else
        \llm@protected\xdef\llm@GlobalTemp{%
          \llm@testopt
          \expandafter\noexpand
          \csname\string#1\endcsname
          {\the\toks@}%
        }%
      \fi
    \expandafter\endgroup\ifx\llm@testopt\@empty
      #3\let#1=#2\relax
    \else
      #3\let#1=\llm@GlobalTemp
      #3\expandafter\let
          \csname\string#1\expandafter\endcsname
          \csname\string#2\endcsname
    \fi
  }{%
    #3\let#1=#2\relax
  }%
}
%    \end{macrocode}
%    \end{macro}
%    \begin{macro}{\llm@CheckParams}
%    \begin{macrocode}
\def\llm@CheckParams#1:->#2\@nil{%
  \begingroup
    \def\x{#1}%
  \ifx\x\llm@macro
    \endgroup
    \def\llm@protected{}%
    \expandafter\@firstoftwo
  \else
    \ifx\x\llm@protectedmacro
      \endgroup
      \def\llm@protected{\protected}%
      \expandafter\expandafter\expandafter\@firstoftwo
    \else
      \endgroup
      \expandafter\expandafter\expandafter\@secondoftwo
    \fi
  \fi
}
%    \end{macrocode}
%    \end{macro}
%    \begin{macro}{\llm@macro}
%    \begin{macrocode}
\def\llm@macro{macro}
\@onelevel@sanitize\llm@macro
%    \end{macrocode}
%    \end{macro}
%    \begin{macro}{\llm@protectedmacro}
%    \begin{macrocode}
\def\llm@protectedmacro{\protected macro}
\@onelevel@sanitize\llm@protectedmacro
%    \end{macrocode}
%    \end{macro}
%    \begin{macro}{\llm@CarThree}
%    \begin{macrocode}
\def\llm@CarThree#1#2#3#4\llm@nil{#1#2#3}%
%    \end{macrocode}
%    \end{macro}
%    \begin{macro}{\llm@CarTwo}
%    \begin{macrocode}
\def\llm@CarTwo#1#2#3\llm@nil{#1#2}%
%    \end{macrocode}
%    \end{macro}
%
%    \begin{macrocode}
\llm@AtEnd%
%</package>
%    \end{macrocode}
% \section{Installation}
%
% \subsection{Download}
%
% \paragraph{Package.} This package is available on
% CTAN\footnote{\CTANpkg{letltxmacro}}:
% \begin{description}
% \item[\CTAN{macros/latex/contrib/letltxmacro/letltxmacro.dtx}] The source file.
% \item[\CTAN{macros/latex/contrib/letltxmacro/letltxmacro.pdf}] Documentation.
% \end{description}
%
%
% \paragraph{Bundle.} All the packages of the bundle `letltxmacro'
% are also available in a TDS compliant ZIP archive. There
% the packages are already unpacked and the documentation files
% are generated. The files and directories obey the TDS standard.
% \begin{description}
% \item[\CTANinstall{install/macros/latex/contrib/letltxmacro.tds.zip}]
% \end{description}
% \emph{TDS} refers to the standard ``A Directory Structure
% for \TeX\ Files'' (\CTANpkg{tds}). Directories
% with \xfile{texmf} in their name are usually organized this way.
%
% \subsection{Bundle installation}
%
% \paragraph{Unpacking.} Unpack the \xfile{letltxmacro.tds.zip} in the
% TDS tree (also known as \xfile{texmf} tree) of your choice.
% Example (linux):
% \begin{quote}
%   |unzip letltxmacro.tds.zip -d ~/texmf|
% \end{quote}
%
% \subsection{Package installation}
%
% \paragraph{Unpacking.} The \xfile{.dtx} file is a self-extracting
% \docstrip\ archive. The files are extracted by running the
% \xfile{.dtx} through \plainTeX:
% \begin{quote}
%   \verb|tex letltxmacro.dtx|
% \end{quote}
%
% \paragraph{TDS.} Now the different files must be moved into
% the different directories in your installation TDS tree
% (also known as \xfile{texmf} tree):
% \begin{quote}
% \def\t{^^A
% \begin{tabular}{@{}>{\ttfamily}l@{ $\rightarrow$ }>{\ttfamily}l@{}}
%   letltxmacro.sty & tex/latex/letltxmacro/letltxmacro.sty\\
%   letltxmacro.pdf & doc/latex/letltxmacro/letltxmacro.pdf\\
%   letltxmacro-showcases.tex & doc/latex/letltxmacro/letltxmacro-showcases.tex\\
%   letltxmacro.dtx & source/latex/letltxmacro/letltxmacro.dtx\\
% \end{tabular}^^A
% }^^A
% \sbox0{\t}^^A
% \ifdim\wd0>\linewidth
%   \begingroup
%     \advance\linewidth by\leftmargin
%     \advance\linewidth by\rightmargin
%   \edef\x{\endgroup
%     \def\noexpand\lw{\the\linewidth}^^A
%   }\x
%   \def\lwbox{^^A
%     \leavevmode
%     \hbox to \linewidth{^^A
%       \kern-\leftmargin\relax
%       \hss
%       \usebox0
%       \hss
%       \kern-\rightmargin\relax
%     }^^A
%   }^^A
%   \ifdim\wd0>\lw
%     \sbox0{\small\t}^^A
%     \ifdim\wd0>\linewidth
%       \ifdim\wd0>\lw
%         \sbox0{\footnotesize\t}^^A
%         \ifdim\wd0>\linewidth
%           \ifdim\wd0>\lw
%             \sbox0{\scriptsize\t}^^A
%             \ifdim\wd0>\linewidth
%               \ifdim\wd0>\lw
%                 \sbox0{\tiny\t}^^A
%                 \ifdim\wd0>\linewidth
%                   \lwbox
%                 \else
%                   \usebox0
%                 \fi
%               \else
%                 \lwbox
%               \fi
%             \else
%               \usebox0
%             \fi
%           \else
%             \lwbox
%           \fi
%         \else
%           \usebox0
%         \fi
%       \else
%         \lwbox
%       \fi
%     \else
%       \usebox0
%     \fi
%   \else
%     \lwbox
%   \fi
% \else
%   \usebox0
% \fi
% \end{quote}
% If you have a \xfile{docstrip.cfg} that configures and enables \docstrip's
% TDS installing feature, then some files can already be in the right
% place, see the documentation of \docstrip.
%
% \subsection{Refresh file name databases}
%
% If your \TeX~distribution
% (\TeX\,Live, \mikTeX, \dots) relies on file name databases, you must refresh
% these. For example, \TeX\,Live\ users run \verb|texhash| or
% \verb|mktexlsr|.
%
% \subsection{Some details for the interested}
%
% \paragraph{Unpacking with \LaTeX.}
% The \xfile{.dtx} chooses its action depending on the format:
% \begin{description}
% \item[\plainTeX:] Run \docstrip\ and extract the files.
% \item[\LaTeX:] Generate the documentation.
% \end{description}
% If you insist on using \LaTeX\ for \docstrip\ (really,
% \docstrip\ does not need \LaTeX), then inform the autodetect routine
% about your intention:
% \begin{quote}
%   \verb|latex \let\install=y% \iffalse meta-comment
%
% File: letltxmacro.dtx
% Version: 2019/12/03 v1.6
% Info: Let assignment for LaTeX macros
%
% Copyright (C)
%    2008, 2010 Heiko Oberdiek
%    2016-2019 Oberdiek Package Support Group
%    https://github.com/ho-tex/letltxmacro/issues
%
% This work may be distributed and/or modified under the
% conditions of the LaTeX Project Public License, either
% version 1.3c of this license or (at your option) any later
% version. This version of this license is in
%    https://www.latex-project.org/lppl/lppl-1-3c.txt
% and the latest version of this license is in
%    https://www.latex-project.org/lppl.txt
% and version 1.3 or later is part of all distributions of
% LaTeX version 2005/12/01 or later.
%
% This work has the LPPL maintenance status "maintained".
%
% The Current Maintainers of this work are
% Heiko Oberdiek and the Oberdiek Package Support Group
% https://github.com/ho-tex/letltxmacro/issues
%
% This work consists of the main source file letltxmacro.dtx
% and the derived files
%    letltxmacro.sty, letltxmacro.pdf, letltxmacro.ins, letltxmacro.drv,
%    letltxmacro-showcases.tex, letltxmacro-test1.tex,
%    letltxmacro-test2.tex.
%
% Distribution:
%    CTAN:macros/latex/contrib/letltxmacro/letltxmacro.dtx
%    CTAN:macros/latex/contrib/letltxmacro/letltxmacro.pdf
%
% Unpacking:
%    (a) If letltxmacro.ins is present:
%           tex letltxmacro.ins
%    (b) Without letltxmacro.ins:
%           tex letltxmacro.dtx
%    (c) If you insist on using LaTeX
%           latex \let\install=y\input{letltxmacro.dtx}
%        (quote the arguments according to the demands of your shell)
%
% Documentation:
%    (a) If letltxmacro.drv is present:
%           latex letltxmacro.drv
%    (b) Without letltxmacro.drv:
%           latex letltxmacro.dtx; ...
%    The class ltxdoc loads the configuration file ltxdoc.cfg
%    if available. Here you can specify further options, e.g.
%    use A4 as paper format:
%       \PassOptionsToClass{a4paper}{article}
%
%    Programm calls to get the documentation (example):
%       pdflatex letltxmacro.dtx
%       makeindex -s gind.ist letltxmacro.idx
%       pdflatex letltxmacro.dtx
%       makeindex -s gind.ist letltxmacro.idx
%       pdflatex letltxmacro.dtx
%
% Installation:
%    TDS:tex/latex/letltxmacro/letltxmacro.sty
%    TDS:doc/latex/letltxmacro/letltxmacro.pdf
%    TDS:doc/latex/letltxmacro/letltxmacro-showcases.tex
%    TDS:source/latex/letltxmacro/letltxmacro.dtx
%
%<*ignore>
\begingroup
  \catcode123=1 %
  \catcode125=2 %
  \def\x{LaTeX2e}%
\expandafter\endgroup
\ifcase 0\ifx\install y1\fi\expandafter
         \ifx\csname processbatchFile\endcsname\relax\else1\fi
         \ifx\fmtname\x\else 1\fi\relax
\else\csname fi\endcsname
%</ignore>
%<*install>
\input docstrip.tex
\Msg{************************************************************************}
\Msg{* Installation}
\Msg{* Package: letltxmacro 2019/12/03 v1.6 Let assignment for LaTeX macros (HO)}
\Msg{************************************************************************}

\keepsilent
\askforoverwritefalse

\let\MetaPrefix\relax
\preamble

This is a generated file.

Project: letltxmacro
Version: 2019/12/03 v1.6

Copyright (C)
   2008, 2010 Heiko Oberdiek
   2016-2019 Oberdiek Package Support Group

This work may be distributed and/or modified under the
conditions of the LaTeX Project Public License, either
version 1.3c of this license or (at your option) any later
version. This version of this license is in
   https://www.latex-project.org/lppl/lppl-1-3c.txt
and the latest version of this license is in
   https://www.latex-project.org/lppl.txt
and version 1.3 or later is part of all distributions of
LaTeX version 2005/12/01 or later.

This work has the LPPL maintenance status "maintained".

The Current Maintainers of this work are
Heiko Oberdiek and the Oberdiek Package Support Group
https://github.com/ho-tex/letltxmacro/issues


This work consists of the main source file letltxmacro.dtx
and the derived files
   letltxmacro.sty, letltxmacro.pdf, letltxmacro.ins, letltxmacro.drv,
   letltxmacro-showcases.tex, letltxmacro-test1.tex,
   letltxmacro-test2.tex.

\endpreamble
\let\MetaPrefix\DoubleperCent

\generate{%
  \file{letltxmacro.ins}{\from{letltxmacro.dtx}{install}}%
  \file{letltxmacro.drv}{\from{letltxmacro.dtx}{driver}}%
  \usedir{tex/latex/letltxmacro}%
  \file{letltxmacro.sty}{\from{letltxmacro.dtx}{package}}%
  \usedir{doc/latex/letltxmacro}%
  \file{letltxmacro-showcases.tex}{\from{letltxmacro.dtx}{showcases}}%
%  \usedir{doc/latex/letltxmacro/test}%
%  \file{letltxmacro-test1.tex}{\from{letltxmacro.dtx}{test1}}%
%  \file{letltxmacro-test2.tex}{\from{letltxmacro.dtx}{test2}}%
}

\catcode32=13\relax% active space
\let =\space%
\Msg{************************************************************************}
\Msg{*}
\Msg{* To finish the installation you have to move the following}
\Msg{* file into a directory searched by TeX:}
\Msg{*}
\Msg{*     letltxmacro.sty}
\Msg{*}
\Msg{* To produce the documentation run the file `letltxmacro.drv'}
\Msg{* through LaTeX.}
\Msg{*}
\Msg{* Happy TeXing!}
\Msg{*}
\Msg{************************************************************************}

\endbatchfile
%</install>
%<*ignore>
\fi
%</ignore>
%<*driver>
\NeedsTeXFormat{LaTeX2e}
\ProvidesFile{letltxmacro.drv}%
  [2019/12/03 v1.6 Let assignment for LaTeX macros (HO)]%
\documentclass{ltxdoc}
\usepackage{holtxdoc}[2011/11/22]
\begin{document}
  \DocInput{letltxmacro.dtx}%
\end{document}
%</driver>
% \fi
%
%
%
% \GetFileInfo{letltxmacro.drv}
%
% \title{The \xpackage{letltxmacro} package}
% \date{2019/12/03 v1.6}
% \author{Heiko Oberdiek\thanks
% {Please report any issues at \url{https://github.com/ho-tex/letltxmacro/issues}}}
%
% \maketitle
%
% \begin{abstract}
% \TeX's \cs{let} assignment does not work for \LaTeX\ macros
% with optional arguments or for macros that are defined
% as robust macros by \cs{DeclareRobustCommand}. This package
% defines \cs{LetLtxMacro} that also takes care of the involved
% internal macros.
% \end{abstract}
%
% \tableofcontents
%
% \section{Documentation}
%
% If someone wants to redefine a macro with using the old
% meaning, then one method is \TeX's command \cs{let}:
%\begin{quote}
%\begin{verbatim}
%\newcommand{\Macro}{\typeout{Test Macro}}
%\let\SavedMacro=\Macro
%\renewcommand{\Macro}{%
%  \typeout{Begin}%
%  \SavedMacro
%  \typeout{End}%
%}
%\end{verbatim}
%\end{quote}
% However, this method fails, if \cs{Macro} is defined
% by \cs{DeclareRobustCommand} and/or has an optional argument.
% In both cases \LaTeX\ defines an additional internal macro
% that is forgotten in the simple \cs{let} assignment of
% the example above.
%
% \begin{declcs}{LetLtxMacro} \M{new macro} \M{old macro}
% \end{declcs}
% Macro \cs{LetLtxMacro} behaves similar to \TeX's \cs{let}
% assignment, but it takes care of macros that are
% defined by \cs{DeclareRobustCommand} and/or have optional
% arguments. Example:
%\begin{quote}
%\begin{verbatim}
%\DeclareRobustCommand{\Macro}[1][default]{...}
%\LetLtxMacro{\SavedMacro}{\Macro}
%\end{verbatim}
%\end{quote}
% Then macro \cs{SavedMacro} only uses internal macro names
% that are derived from \cs{SavedMacro}'s macro name. Macro \cs{Macro}
% can now be redefined without affecting \cs{SavedMacro}.
%
% \begin{declcs}{GlobalLetLtxMacro} \M{new macro} \M{old macro}
% \end{declcs}
% Like \cs{LetLtxMacro}, but the \meta{new macro} is defined globally.
% Since version 2019/12/03~v1.4.
%
% \subsection{Supported macro definition commands}
%
% \begin{quote}
%   \begin{tabular}{@{}ll@{}}
%     \cs{newcommand}, \cs{renewcommand} & latex/base\\
%     \cs{newenvironment}, \cs{renewenvironment} & latex/base\\
%     \cs{DeclareRobustCommand}& latex/base\\
%     \cs{newrobustcmd}, \cs{renewrobustcmd} & etoolbox\\
%     \cs{robustify} & etoolbox 2008/06/22 v1.6\\
%   \end{tabular}
% \end{quote}
%
% \StopEventually{
% }
%
% \section{Implementation}
%
% \subsection{Show cases}
%
% \subsubsection{\xfile{letltxmacro-showcases.tex}}
%
%    \begin{macrocode}
%<*showcases>
\NeedsTeXFormat{LaTeX2e}
\makeatletter
%    \end{macrocode}
%    \begin{macro}{\Line}
%    The result is displayed by macro \cs{Line}. The percent symbol
%    at line start allows easy grepping and inserting into the DTX
%    file.
%    \begin{macrocode}
\newcommand*{\Line}[1]{%
  \typeout{\@percentchar#1}%
}
%    \end{macrocode}
%    \end{macro}
%    \begin{macrocode}
\newcommand*{\ShowCmdName}[1]{%
  \@ifundefined{#1}{}{%
    \Line{%
      \space\space(\expandafter\string\csname#1\endcsname) = %
      (\expandafter\meaning\csname#1\endcsname)%
    }%
  }%
}
\newcommand*{\ShowCmds}[1]{%
  \ShowCmdName{#1}%
  \ShowCmdName{#1 }%
  \ShowCmdName{\\#1}%
  \ShowCmdName{\\#1 }%
}
\let\\\@backslashchar
%    \end{macrocode}
%    \begin{macro}{\ShowDef}
%    \begin{macrocode}
\newcommand*{\ShowDef}[2]{%
  \begingroup
    \Line{}%
    \newcommand*{\DefString}{#2}%
    \@onelevel@sanitize\DefString
    \Line{\DefString}%
    #2%
    \ShowCmds{#1}%
  \endgroup
}
%    \end{macrocode}
%    \end{macro}
%    \begin{macrocode}
\typeout{}
\Line{* LaTeX definitions:}
\ShowDef{cmd}{%
  \newcommand{\cmd}[2][default]{}%
}
\ShowDef{cmd}{%
  \DeclareRobustCommand{\cmd}{}%
}
\ShowDef{cmd}{%
  \DeclareRobustCommand{\cmd}[2][default]{}%
}
\typeout{}
%    \end{macrocode}
% The minimal version of package \xpackage{etoolbox} is 2008/06/12 v1.6a
% because it fixes \cs{robustify}.
%    \begin{macrocode}
\RequirePackage{etoolbox}[2008/06/12]%
\Line{}
\Line{* etoolbox's robust definitions:}
\ShowDef{cmd}{%
  \newrobustcmd{\cmd}{}%
}
\ShowDef{cmd}{%
  \newrobustcmd{\cmd}[2][default]{}%
}
\Line{}
\Line{* etoolbox's \string\robustify:}
\ShowDef{cmd}{%
  \newcommand{\cmd}[2][default]{} %
  \robustify{\cmd}%
}
\ShowDef{cmd}{%
  \DeclareRobustCommand{\cmd}{} %
  \robustify{\cmd}%
}
\ShowDef{cmd}{%
  \DeclareRobustCommand{\cmd}[2][default]{} %
  \robustify{\cmd}%
}
\typeout{}
\@@end
%</showcases>
%    \end{macrocode}
%
% \subsubsection{Result}
%
% \begingroup
%   \makeatletter
%   \let\org@verbatim\@verbatim
%   \def\@verbatim{^^A
%     \org@verbatim
%     \catcode`\~=\active
%   }^^A
%   \let~\textvisiblespace
%\begin{verbatim}
%* LaTeX definitions:
%
%\newcommand {\cmd }[2][default]{}
%  (\cmd) = (macro:->\@protected@testopt \cmd \\cmd {default})
%  (\\cmd) = (\long macro:[#1]#2->)
%
%\DeclareRobustCommand {\cmd }{}
%  (\cmd) = (macro:->\protect \cmd~ )
%  (\cmd~) = (\long macro:->)
%
%\DeclareRobustCommand {\cmd }[2][default]{}
%  (\cmd) = (macro:->\protect \cmd~ )
%  (\cmd~) = (macro:->\@protected@testopt \cmd~ \\cmd~ {default})
%  (\\cmd~) = (\long macro:[#1]#2->)
%
%* etoolbox's robust definitions:
%
%\newrobustcmd {\cmd }{}
%  (\cmd) = (\protected\long macro:->)
%
%\newrobustcmd {\cmd }[2][default]{}
%  (\cmd) = (\protected macro:->\@testopt \\cmd {default})
%  (\\cmd) = (\long macro:[#1]#2->)
%
%* etoolbox's \robustify:
%
%\newcommand {\cmd }[2][default]{} \robustify {\cmd }
%  (\cmd) = (\protected macro:->\@protected@testopt \cmd \\cmd {default})
%  (\\cmd) = (\long macro:[#1]#2->)
%
%\DeclareRobustCommand {\cmd }{} \robustify {\cmd }
%  (\cmd) = (\protected macro:->)
%
%\DeclareRobustCommand {\cmd }[2][default]{} \robustify {\cmd }
%  (\cmd) = (\protected macro:->\@protected@testopt \cmd~ \\cmd~ {default})
%  (\cmd~) = (macro:->\@protected@testopt \cmd~ \\cmd~ {default})
%  (\\cmd~) = (\long macro:[#1]#2->)
%\end{verbatim}
% \endgroup
%
% \subsection{Package}
%
%    \begin{macrocode}
%<*package>
%    \end{macrocode}
%
% \subsubsection{Catcodes and identification}
%
%    \begin{macrocode}
\begingroup\catcode61\catcode48\catcode32=10\relax%
  \catcode13=5 % ^^M
  \endlinechar=13 %
  \catcode123=1 % {
  \catcode125=2 % }
  \catcode64=11 % @
  \def\x{\endgroup
    \expandafter\edef\csname llm@AtEnd\endcsname{%
      \endlinechar=\the\endlinechar\relax
      \catcode13=\the\catcode13\relax
      \catcode32=\the\catcode32\relax
      \catcode35=\the\catcode35\relax
      \catcode61=\the\catcode61\relax
      \catcode64=\the\catcode64\relax
      \catcode123=\the\catcode123\relax
      \catcode125=\the\catcode125\relax
    }%
  }%
\x\catcode61\catcode48\catcode32=10\relax%
\catcode13=5 % ^^M
\endlinechar=13 %
\catcode35=6 % #
\catcode64=11 % @
\catcode123=1 % {
\catcode125=2 % }
\def\TMP@EnsureCode#1#2{%
  \edef\llm@AtEnd{%
    \llm@AtEnd
    \catcode#1=\the\catcode#1\relax
  }%
  \catcode#1=#2\relax
}
\TMP@EnsureCode{40}{12}% (
\TMP@EnsureCode{41}{12}% )
\TMP@EnsureCode{42}{12}% *
\TMP@EnsureCode{45}{12}% -
\TMP@EnsureCode{46}{12}% .
\TMP@EnsureCode{47}{12}% /
\TMP@EnsureCode{58}{12}% :
\TMP@EnsureCode{62}{12}% >
\TMP@EnsureCode{91}{12}% [
\TMP@EnsureCode{93}{12}% ]
\edef\llm@AtEnd{%
  \llm@AtEnd
  \escapechar\the\escapechar\relax
  \noexpand\endinput
}
\escapechar=92 % `\\
%    \end{macrocode}
%
%    Package identification.
%    \begin{macrocode}
\NeedsTeXFormat{LaTeX2e}
\ProvidesPackage{letltxmacro}%
  [2019/12/03 v1.6 Let assignment for LaTeX macros (HO)]
%    \end{macrocode}
%
% \subsubsection{Main macros}
%
%    \begin{macro}{\LetLtxMacro}
%    \begin{macrocode}
\newcommand*{\LetLtxMacro}{%
  \llm@ModeLetLtxMacro{}%
}
%    \end{macrocode}
%    \end{macro}
%    \begin{macro}{\GlobalLetLtxMacro}
%    \begin{macrocode}
\newcommand*{\GlobalLetLtxMacro}{%
  \llm@ModeLetLtxMacro\global
}
%    \end{macrocode}
%    \end{macro}
%
%    \begin{macro}{\llm@ModeLetLtxMacro}
%    \begin{macrocode}
\newcommand*{\llm@ModeLetLtxMacro}[3]{%
  \edef\llm@escapechar{\the\escapechar}%
  \escapechar=-1 %
  \edef\reserved@a{%
    \noexpand\protect
    \expandafter\noexpand
    \csname\string#3 \endcsname
  }%
  \ifx\reserved@a#3\relax
    #1\edef#2{%
      \noexpand\protect
      \expandafter\noexpand
      \csname\string#2 \endcsname
    }%
    #1\expandafter\let
    \csname\string#2 \expandafter\endcsname
    \csname\string#3 \endcsname
    \expandafter\llm@LetLtxMacro
        \csname\string#2 \expandafter\endcsname
        \csname\string#3 \endcsname{#1}%
  \else
    \llm@LetLtxMacro{#2}{#3}{#1}%
  \fi
  \escapechar=\llm@escapechar\relax
}
%    \end{macrocode}
%    \end{macro}
%    \begin{macro}{\llm@LetLtxMacro}
%    \begin{macrocode}
\def\llm@LetLtxMacro#1#2#3{%
  \escapechar=92 %
  \expandafter\llm@CheckParams\meaning#2:->\@nil{%
    \begingroup
      \def\@protected@testopt{%
        \expandafter\@testopt\@gobble
      }%
      \def\@testopt##1##2{%
        \toks@={##2}%
      }%
      \let\llm@testopt\@empty
      \edef\x{%
        \noexpand\@protected@testopt
        \noexpand#2%
        \expandafter\noexpand\csname\string#2\endcsname
      }%
      \expandafter\expandafter\expandafter\def
      \expandafter\expandafter\expandafter\y
      \expandafter\expandafter\expandafter{%
        \expandafter\llm@CarThree#2{}{}{}\llm@nil
      }%
      \ifx\x\y
        #2%
        \def\llm@testopt{%
          \noexpand\@protected@testopt
          \noexpand#1%
        }%
      \else
        \edef\x{%
          \noexpand\@testopt
          \expandafter\noexpand
          \csname\string#2\endcsname
        }%
        \expandafter\expandafter\expandafter\def
        \expandafter\expandafter\expandafter\y
        \expandafter\expandafter\expandafter{%
          \expandafter\llm@CarTwo#2{}{}\llm@nil
        }%
        \ifx\x\y
          #2%
          \def\llm@testopt{%
            \noexpand\@testopt
          }%
        \fi
      \fi
      \ifx\llm@testopt\@empty
      \else
        \llm@protected\xdef\llm@GlobalTemp{%
          \llm@testopt
          \expandafter\noexpand
          \csname\string#1\endcsname
          {\the\toks@}%
        }%
      \fi
    \expandafter\endgroup\ifx\llm@testopt\@empty
      #3\let#1=#2\relax
    \else
      #3\let#1=\llm@GlobalTemp
      #3\expandafter\let
          \csname\string#1\expandafter\endcsname
          \csname\string#2\endcsname
    \fi
  }{%
    #3\let#1=#2\relax
  }%
}
%    \end{macrocode}
%    \end{macro}
%    \begin{macro}{\llm@CheckParams}
%    \begin{macrocode}
\def\llm@CheckParams#1:->#2\@nil{%
  \begingroup
    \def\x{#1}%
  \ifx\x\llm@macro
    \endgroup
    \def\llm@protected{}%
    \expandafter\@firstoftwo
  \else
    \ifx\x\llm@protectedmacro
      \endgroup
      \def\llm@protected{\protected}%
      \expandafter\expandafter\expandafter\@firstoftwo
    \else
      \endgroup
      \expandafter\expandafter\expandafter\@secondoftwo
    \fi
  \fi
}
%    \end{macrocode}
%    \end{macro}
%    \begin{macro}{\llm@macro}
%    \begin{macrocode}
\def\llm@macro{macro}
\@onelevel@sanitize\llm@macro
%    \end{macrocode}
%    \end{macro}
%    \begin{macro}{\llm@protectedmacro}
%    \begin{macrocode}
\def\llm@protectedmacro{\protected macro}
\@onelevel@sanitize\llm@protectedmacro
%    \end{macrocode}
%    \end{macro}
%    \begin{macro}{\llm@CarThree}
%    \begin{macrocode}
\def\llm@CarThree#1#2#3#4\llm@nil{#1#2#3}%
%    \end{macrocode}
%    \end{macro}
%    \begin{macro}{\llm@CarTwo}
%    \begin{macrocode}
\def\llm@CarTwo#1#2#3\llm@nil{#1#2}%
%    \end{macrocode}
%    \end{macro}
%
%    \begin{macrocode}
\llm@AtEnd%
%</package>
%    \end{macrocode}
% \section{Installation}
%
% \subsection{Download}
%
% \paragraph{Package.} This package is available on
% CTAN\footnote{\CTANpkg{letltxmacro}}:
% \begin{description}
% \item[\CTAN{macros/latex/contrib/letltxmacro/letltxmacro.dtx}] The source file.
% \item[\CTAN{macros/latex/contrib/letltxmacro/letltxmacro.pdf}] Documentation.
% \end{description}
%
%
% \paragraph{Bundle.} All the packages of the bundle `letltxmacro'
% are also available in a TDS compliant ZIP archive. There
% the packages are already unpacked and the documentation files
% are generated. The files and directories obey the TDS standard.
% \begin{description}
% \item[\CTANinstall{install/macros/latex/contrib/letltxmacro.tds.zip}]
% \end{description}
% \emph{TDS} refers to the standard ``A Directory Structure
% for \TeX\ Files'' (\CTANpkg{tds}). Directories
% with \xfile{texmf} in their name are usually organized this way.
%
% \subsection{Bundle installation}
%
% \paragraph{Unpacking.} Unpack the \xfile{letltxmacro.tds.zip} in the
% TDS tree (also known as \xfile{texmf} tree) of your choice.
% Example (linux):
% \begin{quote}
%   |unzip letltxmacro.tds.zip -d ~/texmf|
% \end{quote}
%
% \subsection{Package installation}
%
% \paragraph{Unpacking.} The \xfile{.dtx} file is a self-extracting
% \docstrip\ archive. The files are extracted by running the
% \xfile{.dtx} through \plainTeX:
% \begin{quote}
%   \verb|tex letltxmacro.dtx|
% \end{quote}
%
% \paragraph{TDS.} Now the different files must be moved into
% the different directories in your installation TDS tree
% (also known as \xfile{texmf} tree):
% \begin{quote}
% \def\t{^^A
% \begin{tabular}{@{}>{\ttfamily}l@{ $\rightarrow$ }>{\ttfamily}l@{}}
%   letltxmacro.sty & tex/latex/letltxmacro/letltxmacro.sty\\
%   letltxmacro.pdf & doc/latex/letltxmacro/letltxmacro.pdf\\
%   letltxmacro-showcases.tex & doc/latex/letltxmacro/letltxmacro-showcases.tex\\
%   letltxmacro.dtx & source/latex/letltxmacro/letltxmacro.dtx\\
% \end{tabular}^^A
% }^^A
% \sbox0{\t}^^A
% \ifdim\wd0>\linewidth
%   \begingroup
%     \advance\linewidth by\leftmargin
%     \advance\linewidth by\rightmargin
%   \edef\x{\endgroup
%     \def\noexpand\lw{\the\linewidth}^^A
%   }\x
%   \def\lwbox{^^A
%     \leavevmode
%     \hbox to \linewidth{^^A
%       \kern-\leftmargin\relax
%       \hss
%       \usebox0
%       \hss
%       \kern-\rightmargin\relax
%     }^^A
%   }^^A
%   \ifdim\wd0>\lw
%     \sbox0{\small\t}^^A
%     \ifdim\wd0>\linewidth
%       \ifdim\wd0>\lw
%         \sbox0{\footnotesize\t}^^A
%         \ifdim\wd0>\linewidth
%           \ifdim\wd0>\lw
%             \sbox0{\scriptsize\t}^^A
%             \ifdim\wd0>\linewidth
%               \ifdim\wd0>\lw
%                 \sbox0{\tiny\t}^^A
%                 \ifdim\wd0>\linewidth
%                   \lwbox
%                 \else
%                   \usebox0
%                 \fi
%               \else
%                 \lwbox
%               \fi
%             \else
%               \usebox0
%             \fi
%           \else
%             \lwbox
%           \fi
%         \else
%           \usebox0
%         \fi
%       \else
%         \lwbox
%       \fi
%     \else
%       \usebox0
%     \fi
%   \else
%     \lwbox
%   \fi
% \else
%   \usebox0
% \fi
% \end{quote}
% If you have a \xfile{docstrip.cfg} that configures and enables \docstrip's
% TDS installing feature, then some files can already be in the right
% place, see the documentation of \docstrip.
%
% \subsection{Refresh file name databases}
%
% If your \TeX~distribution
% (\TeX\,Live, \mikTeX, \dots) relies on file name databases, you must refresh
% these. For example, \TeX\,Live\ users run \verb|texhash| or
% \verb|mktexlsr|.
%
% \subsection{Some details for the interested}
%
% \paragraph{Unpacking with \LaTeX.}
% The \xfile{.dtx} chooses its action depending on the format:
% \begin{description}
% \item[\plainTeX:] Run \docstrip\ and extract the files.
% \item[\LaTeX:] Generate the documentation.
% \end{description}
% If you insist on using \LaTeX\ for \docstrip\ (really,
% \docstrip\ does not need \LaTeX), then inform the autodetect routine
% about your intention:
% \begin{quote}
%   \verb|latex \let\install=y\input{letltxmacro.dtx}|
% \end{quote}
% Do not forget to quote the argument according to the demands
% of your shell.
%
% \paragraph{Generating the documentation.}
% You can use both the \xfile{.dtx} or the \xfile{.drv} to generate
% the documentation. The process can be configured by the
% configuration file \xfile{ltxdoc.cfg}. For instance, put this
% line into this file, if you want to have A4 as paper format:
% \begin{quote}
%   \verb|\PassOptionsToClass{a4paper}{article}|
% \end{quote}
% An example follows how to generate the
% documentation with pdf\LaTeX:
% \begin{quote}
%\begin{verbatim}
%pdflatex letltxmacro.dtx
%makeindex -s gind.ist letltxmacro.idx
%pdflatex letltxmacro.dtx
%makeindex -s gind.ist letltxmacro.idx
%pdflatex letltxmacro.dtx
%\end{verbatim}
% \end{quote}
%
% \begin{History}
%   \begin{Version}{2008/06/09 v1.0}
%   \item
%     First version.
%   \end{Version}
%   \begin{Version}{2008/06/12 v1.1}
%   \item
%     Support for \xpackage{etoolbox}'s \cs{newrobustcmd} added.
%   \end{Version}
%   \begin{Version}{2008/06/13 v1.2}
%   \item
%     Support for \xpackage{etoolbox}'s \cs{robustify} added.
%   \end{Version}
%   \begin{Version}{2008/06/24 v1.3}
%   \item
%     Test file adapted for etoolbox 2008/06/22 v1.6.
%   \end{Version}
%   \begin{Version}{2010/09/02 v1.4}
%   \item
%     \cs{GlobalLetLtxMacro} added.
%   \end{Version}
%   \begin{Version}{2016/05/16 v1.5}
%   \item
%     Documentation updates.
%   \end{Version}
%   \begin{Version}{2019/12/03 v1.6}
%   \item
%     Documentation updates.
%   \end{Version}
% \end{History}
%
% \PrintIndex
%
% \Finale
\endinput
|
% \end{quote}
% Do not forget to quote the argument according to the demands
% of your shell.
%
% \paragraph{Generating the documentation.}
% You can use both the \xfile{.dtx} or the \xfile{.drv} to generate
% the documentation. The process can be configured by the
% configuration file \xfile{ltxdoc.cfg}. For instance, put this
% line into this file, if you want to have A4 as paper format:
% \begin{quote}
%   \verb|\PassOptionsToClass{a4paper}{article}|
% \end{quote}
% An example follows how to generate the
% documentation with pdf\LaTeX:
% \begin{quote}
%\begin{verbatim}
%pdflatex letltxmacro.dtx
%makeindex -s gind.ist letltxmacro.idx
%pdflatex letltxmacro.dtx
%makeindex -s gind.ist letltxmacro.idx
%pdflatex letltxmacro.dtx
%\end{verbatim}
% \end{quote}
%
% \begin{History}
%   \begin{Version}{2008/06/09 v1.0}
%   \item
%     First version.
%   \end{Version}
%   \begin{Version}{2008/06/12 v1.1}
%   \item
%     Support for \xpackage{etoolbox}'s \cs{newrobustcmd} added.
%   \end{Version}
%   \begin{Version}{2008/06/13 v1.2}
%   \item
%     Support for \xpackage{etoolbox}'s \cs{robustify} added.
%   \end{Version}
%   \begin{Version}{2008/06/24 v1.3}
%   \item
%     Test file adapted for etoolbox 2008/06/22 v1.6.
%   \end{Version}
%   \begin{Version}{2010/09/02 v1.4}
%   \item
%     \cs{GlobalLetLtxMacro} added.
%   \end{Version}
%   \begin{Version}{2016/05/16 v1.5}
%   \item
%     Documentation updates.
%   \end{Version}
%   \begin{Version}{2019/12/03 v1.6}
%   \item
%     Documentation updates.
%   \end{Version}
% \end{History}
%
% \PrintIndex
%
% \Finale
\endinput
|
% \end{quote}
% Do not forget to quote the argument according to the demands
% of your shell.
%
% \paragraph{Generating the documentation.}
% You can use both the \xfile{.dtx} or the \xfile{.drv} to generate
% the documentation. The process can be configured by the
% configuration file \xfile{ltxdoc.cfg}. For instance, put this
% line into this file, if you want to have A4 as paper format:
% \begin{quote}
%   \verb|\PassOptionsToClass{a4paper}{article}|
% \end{quote}
% An example follows how to generate the
% documentation with pdf\LaTeX:
% \begin{quote}
%\begin{verbatim}
%pdflatex letltxmacro.dtx
%makeindex -s gind.ist letltxmacro.idx
%pdflatex letltxmacro.dtx
%makeindex -s gind.ist letltxmacro.idx
%pdflatex letltxmacro.dtx
%\end{verbatim}
% \end{quote}
%
% \begin{History}
%   \begin{Version}{2008/06/09 v1.0}
%   \item
%     First version.
%   \end{Version}
%   \begin{Version}{2008/06/12 v1.1}
%   \item
%     Support for \xpackage{etoolbox}'s \cs{newrobustcmd} added.
%   \end{Version}
%   \begin{Version}{2008/06/13 v1.2}
%   \item
%     Support for \xpackage{etoolbox}'s \cs{robustify} added.
%   \end{Version}
%   \begin{Version}{2008/06/24 v1.3}
%   \item
%     Test file adapted for etoolbox 2008/06/22 v1.6.
%   \end{Version}
%   \begin{Version}{2010/09/02 v1.4}
%   \item
%     \cs{GlobalLetLtxMacro} added.
%   \end{Version}
%   \begin{Version}{2016/05/16 v1.5}
%   \item
%     Documentation updates.
%   \end{Version}
%   \begin{Version}{2019/12/03 v1.6}
%   \item
%     Documentation updates.
%   \end{Version}
% \end{History}
%
% \PrintIndex
%
% \Finale
\endinput

%        (quote the arguments according to the demands of your shell)
%
% Documentation:
%    (a) If letltxmacro.drv is present:
%           latex letltxmacro.drv
%    (b) Without letltxmacro.drv:
%           latex letltxmacro.dtx; ...
%    The class ltxdoc loads the configuration file ltxdoc.cfg
%    if available. Here you can specify further options, e.g.
%    use A4 as paper format:
%       \PassOptionsToClass{a4paper}{article}
%
%    Programm calls to get the documentation (example):
%       pdflatex letltxmacro.dtx
%       makeindex -s gind.ist letltxmacro.idx
%       pdflatex letltxmacro.dtx
%       makeindex -s gind.ist letltxmacro.idx
%       pdflatex letltxmacro.dtx
%
% Installation:
%    TDS:tex/latex/letltxmacro/letltxmacro.sty
%    TDS:doc/latex/letltxmacro/letltxmacro.pdf
%    TDS:doc/latex/letltxmacro/letltxmacro-showcases.tex
%    TDS:source/latex/letltxmacro/letltxmacro.dtx
%
%<*ignore>
\begingroup
  \catcode123=1 %
  \catcode125=2 %
  \def\x{LaTeX2e}%
\expandafter\endgroup
\ifcase 0\ifx\install y1\fi\expandafter
         \ifx\csname processbatchFile\endcsname\relax\else1\fi
         \ifx\fmtname\x\else 1\fi\relax
\else\csname fi\endcsname
%</ignore>
%<*install>
\input docstrip.tex
\Msg{************************************************************************}
\Msg{* Installation}
\Msg{* Package: letltxmacro 2019/12/03 v1.6 Let assignment for LaTeX macros (HO)}
\Msg{************************************************************************}

\keepsilent
\askforoverwritefalse

\let\MetaPrefix\relax
\preamble

This is a generated file.

Project: letltxmacro
Version: 2019/12/03 v1.6

Copyright (C)
   2008, 2010 Heiko Oberdiek
   2016-2019 Oberdiek Package Support Group

This work may be distributed and/or modified under the
conditions of the LaTeX Project Public License, either
version 1.3c of this license or (at your option) any later
version. This version of this license is in
   https://www.latex-project.org/lppl/lppl-1-3c.txt
and the latest version of this license is in
   https://www.latex-project.org/lppl.txt
and version 1.3 or later is part of all distributions of
LaTeX version 2005/12/01 or later.

This work has the LPPL maintenance status "maintained".

The Current Maintainers of this work are
Heiko Oberdiek and the Oberdiek Package Support Group
https://github.com/ho-tex/letltxmacro/issues


This work consists of the main source file letltxmacro.dtx
and the derived files
   letltxmacro.sty, letltxmacro.pdf, letltxmacro.ins, letltxmacro.drv,
   letltxmacro-showcases.tex, letltxmacro-test1.tex,
   letltxmacro-test2.tex.

\endpreamble
\let\MetaPrefix\DoubleperCent

\generate{%
  \file{letltxmacro.ins}{\from{letltxmacro.dtx}{install}}%
  \file{letltxmacro.drv}{\from{letltxmacro.dtx}{driver}}%
  \usedir{tex/latex/letltxmacro}%
  \file{letltxmacro.sty}{\from{letltxmacro.dtx}{package}}%
  \usedir{doc/latex/letltxmacro}%
  \file{letltxmacro-showcases.tex}{\from{letltxmacro.dtx}{showcases}}%
%  \usedir{doc/latex/letltxmacro/test}%
%  \file{letltxmacro-test1.tex}{\from{letltxmacro.dtx}{test1}}%
%  \file{letltxmacro-test2.tex}{\from{letltxmacro.dtx}{test2}}%
}

\catcode32=13\relax% active space
\let =\space%
\Msg{************************************************************************}
\Msg{*}
\Msg{* To finish the installation you have to move the following}
\Msg{* file into a directory searched by TeX:}
\Msg{*}
\Msg{*     letltxmacro.sty}
\Msg{*}
\Msg{* To produce the documentation run the file `letltxmacro.drv'}
\Msg{* through LaTeX.}
\Msg{*}
\Msg{* Happy TeXing!}
\Msg{*}
\Msg{************************************************************************}

\endbatchfile
%</install>
%<*ignore>
\fi
%</ignore>
%<*driver>
\NeedsTeXFormat{LaTeX2e}
\ProvidesFile{letltxmacro.drv}%
  [2019/12/03 v1.6 Let assignment for LaTeX macros (HO)]%
\documentclass{ltxdoc}
\usepackage{holtxdoc}[2011/11/22]
\begin{document}
  \DocInput{letltxmacro.dtx}%
\end{document}
%</driver>
% \fi
%
%
%
% \GetFileInfo{letltxmacro.drv}
%
% \title{The \xpackage{letltxmacro} package}
% \date{2019/12/03 v1.6}
% \author{Heiko Oberdiek\thanks
% {Please report any issues at \url{https://github.com/ho-tex/letltxmacro/issues}}}
%
% \maketitle
%
% \begin{abstract}
% \TeX's \cs{let} assignment does not work for \LaTeX\ macros
% with optional arguments or for macros that are defined
% as robust macros by \cs{DeclareRobustCommand}. This package
% defines \cs{LetLtxMacro} that also takes care of the involved
% internal macros.
% \end{abstract}
%
% \tableofcontents
%
% \section{Documentation}
%
% If someone wants to redefine a macro with using the old
% meaning, then one method is \TeX's command \cs{let}:
%\begin{quote}
%\begin{verbatim}
%\newcommand{\Macro}{\typeout{Test Macro}}
%\let\SavedMacro=\Macro
%\renewcommand{\Macro}{%
%  \typeout{Begin}%
%  \SavedMacro
%  \typeout{End}%
%}
%\end{verbatim}
%\end{quote}
% However, this method fails, if \cs{Macro} is defined
% by \cs{DeclareRobustCommand} and/or has an optional argument.
% In both cases \LaTeX\ defines an additional internal macro
% that is forgotten in the simple \cs{let} assignment of
% the example above.
%
% \begin{declcs}{LetLtxMacro} \M{new macro} \M{old macro}
% \end{declcs}
% Macro \cs{LetLtxMacro} behaves similar to \TeX's \cs{let}
% assignment, but it takes care of macros that are
% defined by \cs{DeclareRobustCommand} and/or have optional
% arguments. Example:
%\begin{quote}
%\begin{verbatim}
%\DeclareRobustCommand{\Macro}[1][default]{...}
%\LetLtxMacro{\SavedMacro}{\Macro}
%\end{verbatim}
%\end{quote}
% Then macro \cs{SavedMacro} only uses internal macro names
% that are derived from \cs{SavedMacro}'s macro name. Macro \cs{Macro}
% can now be redefined without affecting \cs{SavedMacro}.
%
% \begin{declcs}{GlobalLetLtxMacro} \M{new macro} \M{old macro}
% \end{declcs}
% Like \cs{LetLtxMacro}, but the \meta{new macro} is defined globally.
% Since version 2019/12/03~v1.4.
%
% \subsection{Supported macro definition commands}
%
% \begin{quote}
%   \begin{tabular}{@{}ll@{}}
%     \cs{newcommand}, \cs{renewcommand} & latex/base\\
%     \cs{newenvironment}, \cs{renewenvironment} & latex/base\\
%     \cs{DeclareRobustCommand}& latex/base\\
%     \cs{newrobustcmd}, \cs{renewrobustcmd} & etoolbox\\
%     \cs{robustify} & etoolbox 2008/06/22 v1.6\\
%   \end{tabular}
% \end{quote}
%
% \StopEventually{
% }
%
% \section{Implementation}
%
% \subsection{Show cases}
%
% \subsubsection{\xfile{letltxmacro-showcases.tex}}
%
%    \begin{macrocode}
%<*showcases>
\NeedsTeXFormat{LaTeX2e}
\makeatletter
%    \end{macrocode}
%    \begin{macro}{\Line}
%    The result is displayed by macro \cs{Line}. The percent symbol
%    at line start allows easy grepping and inserting into the DTX
%    file.
%    \begin{macrocode}
\newcommand*{\Line}[1]{%
  \typeout{\@percentchar#1}%
}
%    \end{macrocode}
%    \end{macro}
%    \begin{macrocode}
\newcommand*{\ShowCmdName}[1]{%
  \@ifundefined{#1}{}{%
    \Line{%
      \space\space(\expandafter\string\csname#1\endcsname) = %
      (\expandafter\meaning\csname#1\endcsname)%
    }%
  }%
}
\newcommand*{\ShowCmds}[1]{%
  \ShowCmdName{#1}%
  \ShowCmdName{#1 }%
  \ShowCmdName{\\#1}%
  \ShowCmdName{\\#1 }%
}
\let\\\@backslashchar
%    \end{macrocode}
%    \begin{macro}{\ShowDef}
%    \begin{macrocode}
\newcommand*{\ShowDef}[2]{%
  \begingroup
    \Line{}%
    \newcommand*{\DefString}{#2}%
    \@onelevel@sanitize\DefString
    \Line{\DefString}%
    #2%
    \ShowCmds{#1}%
  \endgroup
}
%    \end{macrocode}
%    \end{macro}
%    \begin{macrocode}
\typeout{}
\Line{* LaTeX definitions:}
\ShowDef{cmd}{%
  \newcommand{\cmd}[2][default]{}%
}
\ShowDef{cmd}{%
  \DeclareRobustCommand{\cmd}{}%
}
\ShowDef{cmd}{%
  \DeclareRobustCommand{\cmd}[2][default]{}%
}
\typeout{}
%    \end{macrocode}
% The minimal version of package \xpackage{etoolbox} is 2008/06/12 v1.6a
% because it fixes \cs{robustify}.
%    \begin{macrocode}
\RequirePackage{etoolbox}[2008/06/12]%
\Line{}
\Line{* etoolbox's robust definitions:}
\ShowDef{cmd}{%
  \newrobustcmd{\cmd}{}%
}
\ShowDef{cmd}{%
  \newrobustcmd{\cmd}[2][default]{}%
}
\Line{}
\Line{* etoolbox's \string\robustify:}
\ShowDef{cmd}{%
  \newcommand{\cmd}[2][default]{} %
  \robustify{\cmd}%
}
\ShowDef{cmd}{%
  \DeclareRobustCommand{\cmd}{} %
  \robustify{\cmd}%
}
\ShowDef{cmd}{%
  \DeclareRobustCommand{\cmd}[2][default]{} %
  \robustify{\cmd}%
}
\typeout{}
\@@end
%</showcases>
%    \end{macrocode}
%
% \subsubsection{Result}
%
% \begingroup
%   \makeatletter
%   \let\org@verbatim\@verbatim
%   \def\@verbatim{^^A
%     \org@verbatim
%     \catcode`\~=\active
%   }^^A
%   \let~\textvisiblespace
%\begin{verbatim}
%* LaTeX definitions:
%
%\newcommand {\cmd }[2][default]{}
%  (\cmd) = (macro:->\@protected@testopt \cmd \\cmd {default})
%  (\\cmd) = (\long macro:[#1]#2->)
%
%\DeclareRobustCommand {\cmd }{}
%  (\cmd) = (macro:->\protect \cmd~ )
%  (\cmd~) = (\long macro:->)
%
%\DeclareRobustCommand {\cmd }[2][default]{}
%  (\cmd) = (macro:->\protect \cmd~ )
%  (\cmd~) = (macro:->\@protected@testopt \cmd~ \\cmd~ {default})
%  (\\cmd~) = (\long macro:[#1]#2->)
%
%* etoolbox's robust definitions:
%
%\newrobustcmd {\cmd }{}
%  (\cmd) = (\protected\long macro:->)
%
%\newrobustcmd {\cmd }[2][default]{}
%  (\cmd) = (\protected macro:->\@testopt \\cmd {default})
%  (\\cmd) = (\long macro:[#1]#2->)
%
%* etoolbox's \robustify:
%
%\newcommand {\cmd }[2][default]{} \robustify {\cmd }
%  (\cmd) = (\protected macro:->\@protected@testopt \cmd \\cmd {default})
%  (\\cmd) = (\long macro:[#1]#2->)
%
%\DeclareRobustCommand {\cmd }{} \robustify {\cmd }
%  (\cmd) = (\protected macro:->)
%
%\DeclareRobustCommand {\cmd }[2][default]{} \robustify {\cmd }
%  (\cmd) = (\protected macro:->\@protected@testopt \cmd~ \\cmd~ {default})
%  (\cmd~) = (macro:->\@protected@testopt \cmd~ \\cmd~ {default})
%  (\\cmd~) = (\long macro:[#1]#2->)
%\end{verbatim}
% \endgroup
%
% \subsection{Package}
%
%    \begin{macrocode}
%<*package>
%    \end{macrocode}
%
% \subsubsection{Catcodes and identification}
%
%    \begin{macrocode}
\begingroup\catcode61\catcode48\catcode32=10\relax%
  \catcode13=5 % ^^M
  \endlinechar=13 %
  \catcode123=1 % {
  \catcode125=2 % }
  \catcode64=11 % @
  \def\x{\endgroup
    \expandafter\edef\csname llm@AtEnd\endcsname{%
      \endlinechar=\the\endlinechar\relax
      \catcode13=\the\catcode13\relax
      \catcode32=\the\catcode32\relax
      \catcode35=\the\catcode35\relax
      \catcode61=\the\catcode61\relax
      \catcode64=\the\catcode64\relax
      \catcode123=\the\catcode123\relax
      \catcode125=\the\catcode125\relax
    }%
  }%
\x\catcode61\catcode48\catcode32=10\relax%
\catcode13=5 % ^^M
\endlinechar=13 %
\catcode35=6 % #
\catcode64=11 % @
\catcode123=1 % {
\catcode125=2 % }
\def\TMP@EnsureCode#1#2{%
  \edef\llm@AtEnd{%
    \llm@AtEnd
    \catcode#1=\the\catcode#1\relax
  }%
  \catcode#1=#2\relax
}
\TMP@EnsureCode{40}{12}% (
\TMP@EnsureCode{41}{12}% )
\TMP@EnsureCode{42}{12}% *
\TMP@EnsureCode{45}{12}% -
\TMP@EnsureCode{46}{12}% .
\TMP@EnsureCode{47}{12}% /
\TMP@EnsureCode{58}{12}% :
\TMP@EnsureCode{62}{12}% >
\TMP@EnsureCode{91}{12}% [
\TMP@EnsureCode{93}{12}% ]
\edef\llm@AtEnd{%
  \llm@AtEnd
  \escapechar\the\escapechar\relax
  \noexpand\endinput
}
\escapechar=92 % `\\
%    \end{macrocode}
%
%    Package identification.
%    \begin{macrocode}
\NeedsTeXFormat{LaTeX2e}
\ProvidesPackage{letltxmacro}%
  [2019/12/03 v1.6 Let assignment for LaTeX macros (HO)]
%    \end{macrocode}
%
% \subsubsection{Main macros}
%
%    \begin{macro}{\LetLtxMacro}
%    \begin{macrocode}
\newcommand*{\LetLtxMacro}{%
  \llm@ModeLetLtxMacro{}%
}
%    \end{macrocode}
%    \end{macro}
%    \begin{macro}{\GlobalLetLtxMacro}
%    \begin{macrocode}
\newcommand*{\GlobalLetLtxMacro}{%
  \llm@ModeLetLtxMacro\global
}
%    \end{macrocode}
%    \end{macro}
%
%    \begin{macro}{\llm@ModeLetLtxMacro}
%    \begin{macrocode}
\newcommand*{\llm@ModeLetLtxMacro}[3]{%
  \edef\llm@escapechar{\the\escapechar}%
  \escapechar=-1 %
  \edef\reserved@a{%
    \noexpand\protect
    \expandafter\noexpand
    \csname\string#3 \endcsname
  }%
  \ifx\reserved@a#3\relax
    #1\edef#2{%
      \noexpand\protect
      \expandafter\noexpand
      \csname\string#2 \endcsname
    }%
    #1\expandafter\let
    \csname\string#2 \expandafter\endcsname
    \csname\string#3 \endcsname
    \expandafter\llm@LetLtxMacro
        \csname\string#2 \expandafter\endcsname
        \csname\string#3 \endcsname{#1}%
  \else
    \llm@LetLtxMacro{#2}{#3}{#1}%
  \fi
  \escapechar=\llm@escapechar\relax
}
%    \end{macrocode}
%    \end{macro}
%    \begin{macro}{\llm@LetLtxMacro}
%    \begin{macrocode}
\def\llm@LetLtxMacro#1#2#3{%
  \escapechar=92 %
  \expandafter\llm@CheckParams\meaning#2:->\@nil{%
    \begingroup
      \def\@protected@testopt{%
        \expandafter\@testopt\@gobble
      }%
      \def\@testopt##1##2{%
        \toks@={##2}%
      }%
      \let\llm@testopt\@empty
      \edef\x{%
        \noexpand\@protected@testopt
        \noexpand#2%
        \expandafter\noexpand\csname\string#2\endcsname
      }%
      \expandafter\expandafter\expandafter\def
      \expandafter\expandafter\expandafter\y
      \expandafter\expandafter\expandafter{%
        \expandafter\llm@CarThree#2{}{}{}\llm@nil
      }%
      \ifx\x\y
        #2%
        \def\llm@testopt{%
          \noexpand\@protected@testopt
          \noexpand#1%
        }%
      \else
        \edef\x{%
          \noexpand\@testopt
          \expandafter\noexpand
          \csname\string#2\endcsname
        }%
        \expandafter\expandafter\expandafter\def
        \expandafter\expandafter\expandafter\y
        \expandafter\expandafter\expandafter{%
          \expandafter\llm@CarTwo#2{}{}\llm@nil
        }%
        \ifx\x\y
          #2%
          \def\llm@testopt{%
            \noexpand\@testopt
          }%
        \fi
      \fi
      \ifx\llm@testopt\@empty
      \else
        \llm@protected\xdef\llm@GlobalTemp{%
          \llm@testopt
          \expandafter\noexpand
          \csname\string#1\endcsname
          {\the\toks@}%
        }%
      \fi
    \expandafter\endgroup\ifx\llm@testopt\@empty
      #3\let#1=#2\relax
    \else
      #3\let#1=\llm@GlobalTemp
      #3\expandafter\let
          \csname\string#1\expandafter\endcsname
          \csname\string#2\endcsname
    \fi
  }{%
    #3\let#1=#2\relax
  }%
}
%    \end{macrocode}
%    \end{macro}
%    \begin{macro}{\llm@CheckParams}
%    \begin{macrocode}
\def\llm@CheckParams#1:->#2\@nil{%
  \begingroup
    \def\x{#1}%
  \ifx\x\llm@macro
    \endgroup
    \def\llm@protected{}%
    \expandafter\@firstoftwo
  \else
    \ifx\x\llm@protectedmacro
      \endgroup
      \def\llm@protected{\protected}%
      \expandafter\expandafter\expandafter\@firstoftwo
    \else
      \endgroup
      \expandafter\expandafter\expandafter\@secondoftwo
    \fi
  \fi
}
%    \end{macrocode}
%    \end{macro}
%    \begin{macro}{\llm@macro}
%    \begin{macrocode}
\def\llm@macro{macro}
\@onelevel@sanitize\llm@macro
%    \end{macrocode}
%    \end{macro}
%    \begin{macro}{\llm@protectedmacro}
%    \begin{macrocode}
\def\llm@protectedmacro{\protected macro}
\@onelevel@sanitize\llm@protectedmacro
%    \end{macrocode}
%    \end{macro}
%    \begin{macro}{\llm@CarThree}
%    \begin{macrocode}
\def\llm@CarThree#1#2#3#4\llm@nil{#1#2#3}%
%    \end{macrocode}
%    \end{macro}
%    \begin{macro}{\llm@CarTwo}
%    \begin{macrocode}
\def\llm@CarTwo#1#2#3\llm@nil{#1#2}%
%    \end{macrocode}
%    \end{macro}
%
%    \begin{macrocode}
\llm@AtEnd%
%</package>
%    \end{macrocode}
% \section{Installation}
%
% \subsection{Download}
%
% \paragraph{Package.} This package is available on
% CTAN\footnote{\CTANpkg{letltxmacro}}:
% \begin{description}
% \item[\CTAN{macros/latex/contrib/letltxmacro/letltxmacro.dtx}] The source file.
% \item[\CTAN{macros/latex/contrib/letltxmacro/letltxmacro.pdf}] Documentation.
% \end{description}
%
%
% \paragraph{Bundle.} All the packages of the bundle `letltxmacro'
% are also available in a TDS compliant ZIP archive. There
% the packages are already unpacked and the documentation files
% are generated. The files and directories obey the TDS standard.
% \begin{description}
% \item[\CTANinstall{install/macros/latex/contrib/letltxmacro.tds.zip}]
% \end{description}
% \emph{TDS} refers to the standard ``A Directory Structure
% for \TeX\ Files'' (\CTANpkg{tds}). Directories
% with \xfile{texmf} in their name are usually organized this way.
%
% \subsection{Bundle installation}
%
% \paragraph{Unpacking.} Unpack the \xfile{letltxmacro.tds.zip} in the
% TDS tree (also known as \xfile{texmf} tree) of your choice.
% Example (linux):
% \begin{quote}
%   |unzip letltxmacro.tds.zip -d ~/texmf|
% \end{quote}
%
% \subsection{Package installation}
%
% \paragraph{Unpacking.} The \xfile{.dtx} file is a self-extracting
% \docstrip\ archive. The files are extracted by running the
% \xfile{.dtx} through \plainTeX:
% \begin{quote}
%   \verb|tex letltxmacro.dtx|
% \end{quote}
%
% \paragraph{TDS.} Now the different files must be moved into
% the different directories in your installation TDS tree
% (also known as \xfile{texmf} tree):
% \begin{quote}
% \def\t{^^A
% \begin{tabular}{@{}>{\ttfamily}l@{ $\rightarrow$ }>{\ttfamily}l@{}}
%   letltxmacro.sty & tex/latex/letltxmacro/letltxmacro.sty\\
%   letltxmacro.pdf & doc/latex/letltxmacro/letltxmacro.pdf\\
%   letltxmacro-showcases.tex & doc/latex/letltxmacro/letltxmacro-showcases.tex\\
%   letltxmacro.dtx & source/latex/letltxmacro/letltxmacro.dtx\\
% \end{tabular}^^A
% }^^A
% \sbox0{\t}^^A
% \ifdim\wd0>\linewidth
%   \begingroup
%     \advance\linewidth by\leftmargin
%     \advance\linewidth by\rightmargin
%   \edef\x{\endgroup
%     \def\noexpand\lw{\the\linewidth}^^A
%   }\x
%   \def\lwbox{^^A
%     \leavevmode
%     \hbox to \linewidth{^^A
%       \kern-\leftmargin\relax
%       \hss
%       \usebox0
%       \hss
%       \kern-\rightmargin\relax
%     }^^A
%   }^^A
%   \ifdim\wd0>\lw
%     \sbox0{\small\t}^^A
%     \ifdim\wd0>\linewidth
%       \ifdim\wd0>\lw
%         \sbox0{\footnotesize\t}^^A
%         \ifdim\wd0>\linewidth
%           \ifdim\wd0>\lw
%             \sbox0{\scriptsize\t}^^A
%             \ifdim\wd0>\linewidth
%               \ifdim\wd0>\lw
%                 \sbox0{\tiny\t}^^A
%                 \ifdim\wd0>\linewidth
%                   \lwbox
%                 \else
%                   \usebox0
%                 \fi
%               \else
%                 \lwbox
%               \fi
%             \else
%               \usebox0
%             \fi
%           \else
%             \lwbox
%           \fi
%         \else
%           \usebox0
%         \fi
%       \else
%         \lwbox
%       \fi
%     \else
%       \usebox0
%     \fi
%   \else
%     \lwbox
%   \fi
% \else
%   \usebox0
% \fi
% \end{quote}
% If you have a \xfile{docstrip.cfg} that configures and enables \docstrip's
% TDS installing feature, then some files can already be in the right
% place, see the documentation of \docstrip.
%
% \subsection{Refresh file name databases}
%
% If your \TeX~distribution
% (\TeX\,Live, \mikTeX, \dots) relies on file name databases, you must refresh
% these. For example, \TeX\,Live\ users run \verb|texhash| or
% \verb|mktexlsr|.
%
% \subsection{Some details for the interested}
%
% \paragraph{Unpacking with \LaTeX.}
% The \xfile{.dtx} chooses its action depending on the format:
% \begin{description}
% \item[\plainTeX:] Run \docstrip\ and extract the files.
% \item[\LaTeX:] Generate the documentation.
% \end{description}
% If you insist on using \LaTeX\ for \docstrip\ (really,
% \docstrip\ does not need \LaTeX), then inform the autodetect routine
% about your intention:
% \begin{quote}
%   \verb|latex \let\install=y% \iffalse meta-comment
%
% File: letltxmacro.dtx
% Version: 2019/12/03 v1.6
% Info: Let assignment for LaTeX macros
%
% Copyright (C)
%    2008, 2010 Heiko Oberdiek
%    2016-2019 Oberdiek Package Support Group
%    https://github.com/ho-tex/letltxmacro/issues
%
% This work may be distributed and/or modified under the
% conditions of the LaTeX Project Public License, either
% version 1.3c of this license or (at your option) any later
% version. This version of this license is in
%    https://www.latex-project.org/lppl/lppl-1-3c.txt
% and the latest version of this license is in
%    https://www.latex-project.org/lppl.txt
% and version 1.3 or later is part of all distributions of
% LaTeX version 2005/12/01 or later.
%
% This work has the LPPL maintenance status "maintained".
%
% The Current Maintainers of this work are
% Heiko Oberdiek and the Oberdiek Package Support Group
% https://github.com/ho-tex/letltxmacro/issues
%
% This work consists of the main source file letltxmacro.dtx
% and the derived files
%    letltxmacro.sty, letltxmacro.pdf, letltxmacro.ins, letltxmacro.drv,
%    letltxmacro-showcases.tex, letltxmacro-test1.tex,
%    letltxmacro-test2.tex.
%
% Distribution:
%    CTAN:macros/latex/contrib/letltxmacro/letltxmacro.dtx
%    CTAN:macros/latex/contrib/letltxmacro/letltxmacro.pdf
%
% Unpacking:
%    (a) If letltxmacro.ins is present:
%           tex letltxmacro.ins
%    (b) Without letltxmacro.ins:
%           tex letltxmacro.dtx
%    (c) If you insist on using LaTeX
%           latex \let\install=y% \iffalse meta-comment
%
% File: letltxmacro.dtx
% Version: 2019/12/03 v1.6
% Info: Let assignment for LaTeX macros
%
% Copyright (C)
%    2008, 2010 Heiko Oberdiek
%    2016-2019 Oberdiek Package Support Group
%    https://github.com/ho-tex/letltxmacro/issues
%
% This work may be distributed and/or modified under the
% conditions of the LaTeX Project Public License, either
% version 1.3c of this license or (at your option) any later
% version. This version of this license is in
%    https://www.latex-project.org/lppl/lppl-1-3c.txt
% and the latest version of this license is in
%    https://www.latex-project.org/lppl.txt
% and version 1.3 or later is part of all distributions of
% LaTeX version 2005/12/01 or later.
%
% This work has the LPPL maintenance status "maintained".
%
% The Current Maintainers of this work are
% Heiko Oberdiek and the Oberdiek Package Support Group
% https://github.com/ho-tex/letltxmacro/issues
%
% This work consists of the main source file letltxmacro.dtx
% and the derived files
%    letltxmacro.sty, letltxmacro.pdf, letltxmacro.ins, letltxmacro.drv,
%    letltxmacro-showcases.tex, letltxmacro-test1.tex,
%    letltxmacro-test2.tex.
%
% Distribution:
%    CTAN:macros/latex/contrib/letltxmacro/letltxmacro.dtx
%    CTAN:macros/latex/contrib/letltxmacro/letltxmacro.pdf
%
% Unpacking:
%    (a) If letltxmacro.ins is present:
%           tex letltxmacro.ins
%    (b) Without letltxmacro.ins:
%           tex letltxmacro.dtx
%    (c) If you insist on using LaTeX
%           latex \let\install=y% \iffalse meta-comment
%
% File: letltxmacro.dtx
% Version: 2019/12/03 v1.6
% Info: Let assignment for LaTeX macros
%
% Copyright (C)
%    2008, 2010 Heiko Oberdiek
%    2016-2019 Oberdiek Package Support Group
%    https://github.com/ho-tex/letltxmacro/issues
%
% This work may be distributed and/or modified under the
% conditions of the LaTeX Project Public License, either
% version 1.3c of this license or (at your option) any later
% version. This version of this license is in
%    https://www.latex-project.org/lppl/lppl-1-3c.txt
% and the latest version of this license is in
%    https://www.latex-project.org/lppl.txt
% and version 1.3 or later is part of all distributions of
% LaTeX version 2005/12/01 or later.
%
% This work has the LPPL maintenance status "maintained".
%
% The Current Maintainers of this work are
% Heiko Oberdiek and the Oberdiek Package Support Group
% https://github.com/ho-tex/letltxmacro/issues
%
% This work consists of the main source file letltxmacro.dtx
% and the derived files
%    letltxmacro.sty, letltxmacro.pdf, letltxmacro.ins, letltxmacro.drv,
%    letltxmacro-showcases.tex, letltxmacro-test1.tex,
%    letltxmacro-test2.tex.
%
% Distribution:
%    CTAN:macros/latex/contrib/letltxmacro/letltxmacro.dtx
%    CTAN:macros/latex/contrib/letltxmacro/letltxmacro.pdf
%
% Unpacking:
%    (a) If letltxmacro.ins is present:
%           tex letltxmacro.ins
%    (b) Without letltxmacro.ins:
%           tex letltxmacro.dtx
%    (c) If you insist on using LaTeX
%           latex \let\install=y\input{letltxmacro.dtx}
%        (quote the arguments according to the demands of your shell)
%
% Documentation:
%    (a) If letltxmacro.drv is present:
%           latex letltxmacro.drv
%    (b) Without letltxmacro.drv:
%           latex letltxmacro.dtx; ...
%    The class ltxdoc loads the configuration file ltxdoc.cfg
%    if available. Here you can specify further options, e.g.
%    use A4 as paper format:
%       \PassOptionsToClass{a4paper}{article}
%
%    Programm calls to get the documentation (example):
%       pdflatex letltxmacro.dtx
%       makeindex -s gind.ist letltxmacro.idx
%       pdflatex letltxmacro.dtx
%       makeindex -s gind.ist letltxmacro.idx
%       pdflatex letltxmacro.dtx
%
% Installation:
%    TDS:tex/latex/letltxmacro/letltxmacro.sty
%    TDS:doc/latex/letltxmacro/letltxmacro.pdf
%    TDS:doc/latex/letltxmacro/letltxmacro-showcases.tex
%    TDS:source/latex/letltxmacro/letltxmacro.dtx
%
%<*ignore>
\begingroup
  \catcode123=1 %
  \catcode125=2 %
  \def\x{LaTeX2e}%
\expandafter\endgroup
\ifcase 0\ifx\install y1\fi\expandafter
         \ifx\csname processbatchFile\endcsname\relax\else1\fi
         \ifx\fmtname\x\else 1\fi\relax
\else\csname fi\endcsname
%</ignore>
%<*install>
\input docstrip.tex
\Msg{************************************************************************}
\Msg{* Installation}
\Msg{* Package: letltxmacro 2019/12/03 v1.6 Let assignment for LaTeX macros (HO)}
\Msg{************************************************************************}

\keepsilent
\askforoverwritefalse

\let\MetaPrefix\relax
\preamble

This is a generated file.

Project: letltxmacro
Version: 2019/12/03 v1.6

Copyright (C)
   2008, 2010 Heiko Oberdiek
   2016-2019 Oberdiek Package Support Group

This work may be distributed and/or modified under the
conditions of the LaTeX Project Public License, either
version 1.3c of this license or (at your option) any later
version. This version of this license is in
   https://www.latex-project.org/lppl/lppl-1-3c.txt
and the latest version of this license is in
   https://www.latex-project.org/lppl.txt
and version 1.3 or later is part of all distributions of
LaTeX version 2005/12/01 or later.

This work has the LPPL maintenance status "maintained".

The Current Maintainers of this work are
Heiko Oberdiek and the Oberdiek Package Support Group
https://github.com/ho-tex/letltxmacro/issues


This work consists of the main source file letltxmacro.dtx
and the derived files
   letltxmacro.sty, letltxmacro.pdf, letltxmacro.ins, letltxmacro.drv,
   letltxmacro-showcases.tex, letltxmacro-test1.tex,
   letltxmacro-test2.tex.

\endpreamble
\let\MetaPrefix\DoubleperCent

\generate{%
  \file{letltxmacro.ins}{\from{letltxmacro.dtx}{install}}%
  \file{letltxmacro.drv}{\from{letltxmacro.dtx}{driver}}%
  \usedir{tex/latex/letltxmacro}%
  \file{letltxmacro.sty}{\from{letltxmacro.dtx}{package}}%
  \usedir{doc/latex/letltxmacro}%
  \file{letltxmacro-showcases.tex}{\from{letltxmacro.dtx}{showcases}}%
%  \usedir{doc/latex/letltxmacro/test}%
%  \file{letltxmacro-test1.tex}{\from{letltxmacro.dtx}{test1}}%
%  \file{letltxmacro-test2.tex}{\from{letltxmacro.dtx}{test2}}%
}

\catcode32=13\relax% active space
\let =\space%
\Msg{************************************************************************}
\Msg{*}
\Msg{* To finish the installation you have to move the following}
\Msg{* file into a directory searched by TeX:}
\Msg{*}
\Msg{*     letltxmacro.sty}
\Msg{*}
\Msg{* To produce the documentation run the file `letltxmacro.drv'}
\Msg{* through LaTeX.}
\Msg{*}
\Msg{* Happy TeXing!}
\Msg{*}
\Msg{************************************************************************}

\endbatchfile
%</install>
%<*ignore>
\fi
%</ignore>
%<*driver>
\NeedsTeXFormat{LaTeX2e}
\ProvidesFile{letltxmacro.drv}%
  [2019/12/03 v1.6 Let assignment for LaTeX macros (HO)]%
\documentclass{ltxdoc}
\usepackage{holtxdoc}[2011/11/22]
\begin{document}
  \DocInput{letltxmacro.dtx}%
\end{document}
%</driver>
% \fi
%
%
%
% \GetFileInfo{letltxmacro.drv}
%
% \title{The \xpackage{letltxmacro} package}
% \date{2019/12/03 v1.6}
% \author{Heiko Oberdiek\thanks
% {Please report any issues at \url{https://github.com/ho-tex/letltxmacro/issues}}}
%
% \maketitle
%
% \begin{abstract}
% \TeX's \cs{let} assignment does not work for \LaTeX\ macros
% with optional arguments or for macros that are defined
% as robust macros by \cs{DeclareRobustCommand}. This package
% defines \cs{LetLtxMacro} that also takes care of the involved
% internal macros.
% \end{abstract}
%
% \tableofcontents
%
% \section{Documentation}
%
% If someone wants to redefine a macro with using the old
% meaning, then one method is \TeX's command \cs{let}:
%\begin{quote}
%\begin{verbatim}
%\newcommand{\Macro}{\typeout{Test Macro}}
%\let\SavedMacro=\Macro
%\renewcommand{\Macro}{%
%  \typeout{Begin}%
%  \SavedMacro
%  \typeout{End}%
%}
%\end{verbatim}
%\end{quote}
% However, this method fails, if \cs{Macro} is defined
% by \cs{DeclareRobustCommand} and/or has an optional argument.
% In both cases \LaTeX\ defines an additional internal macro
% that is forgotten in the simple \cs{let} assignment of
% the example above.
%
% \begin{declcs}{LetLtxMacro} \M{new macro} \M{old macro}
% \end{declcs}
% Macro \cs{LetLtxMacro} behaves similar to \TeX's \cs{let}
% assignment, but it takes care of macros that are
% defined by \cs{DeclareRobustCommand} and/or have optional
% arguments. Example:
%\begin{quote}
%\begin{verbatim}
%\DeclareRobustCommand{\Macro}[1][default]{...}
%\LetLtxMacro{\SavedMacro}{\Macro}
%\end{verbatim}
%\end{quote}
% Then macro \cs{SavedMacro} only uses internal macro names
% that are derived from \cs{SavedMacro}'s macro name. Macro \cs{Macro}
% can now be redefined without affecting \cs{SavedMacro}.
%
% \begin{declcs}{GlobalLetLtxMacro} \M{new macro} \M{old macro}
% \end{declcs}
% Like \cs{LetLtxMacro}, but the \meta{new macro} is defined globally.
% Since version 2019/12/03~v1.4.
%
% \subsection{Supported macro definition commands}
%
% \begin{quote}
%   \begin{tabular}{@{}ll@{}}
%     \cs{newcommand}, \cs{renewcommand} & latex/base\\
%     \cs{newenvironment}, \cs{renewenvironment} & latex/base\\
%     \cs{DeclareRobustCommand}& latex/base\\
%     \cs{newrobustcmd}, \cs{renewrobustcmd} & etoolbox\\
%     \cs{robustify} & etoolbox 2008/06/22 v1.6\\
%   \end{tabular}
% \end{quote}
%
% \StopEventually{
% }
%
% \section{Implementation}
%
% \subsection{Show cases}
%
% \subsubsection{\xfile{letltxmacro-showcases.tex}}
%
%    \begin{macrocode}
%<*showcases>
\NeedsTeXFormat{LaTeX2e}
\makeatletter
%    \end{macrocode}
%    \begin{macro}{\Line}
%    The result is displayed by macro \cs{Line}. The percent symbol
%    at line start allows easy grepping and inserting into the DTX
%    file.
%    \begin{macrocode}
\newcommand*{\Line}[1]{%
  \typeout{\@percentchar#1}%
}
%    \end{macrocode}
%    \end{macro}
%    \begin{macrocode}
\newcommand*{\ShowCmdName}[1]{%
  \@ifundefined{#1}{}{%
    \Line{%
      \space\space(\expandafter\string\csname#1\endcsname) = %
      (\expandafter\meaning\csname#1\endcsname)%
    }%
  }%
}
\newcommand*{\ShowCmds}[1]{%
  \ShowCmdName{#1}%
  \ShowCmdName{#1 }%
  \ShowCmdName{\\#1}%
  \ShowCmdName{\\#1 }%
}
\let\\\@backslashchar
%    \end{macrocode}
%    \begin{macro}{\ShowDef}
%    \begin{macrocode}
\newcommand*{\ShowDef}[2]{%
  \begingroup
    \Line{}%
    \newcommand*{\DefString}{#2}%
    \@onelevel@sanitize\DefString
    \Line{\DefString}%
    #2%
    \ShowCmds{#1}%
  \endgroup
}
%    \end{macrocode}
%    \end{macro}
%    \begin{macrocode}
\typeout{}
\Line{* LaTeX definitions:}
\ShowDef{cmd}{%
  \newcommand{\cmd}[2][default]{}%
}
\ShowDef{cmd}{%
  \DeclareRobustCommand{\cmd}{}%
}
\ShowDef{cmd}{%
  \DeclareRobustCommand{\cmd}[2][default]{}%
}
\typeout{}
%    \end{macrocode}
% The minimal version of package \xpackage{etoolbox} is 2008/06/12 v1.6a
% because it fixes \cs{robustify}.
%    \begin{macrocode}
\RequirePackage{etoolbox}[2008/06/12]%
\Line{}
\Line{* etoolbox's robust definitions:}
\ShowDef{cmd}{%
  \newrobustcmd{\cmd}{}%
}
\ShowDef{cmd}{%
  \newrobustcmd{\cmd}[2][default]{}%
}
\Line{}
\Line{* etoolbox's \string\robustify:}
\ShowDef{cmd}{%
  \newcommand{\cmd}[2][default]{} %
  \robustify{\cmd}%
}
\ShowDef{cmd}{%
  \DeclareRobustCommand{\cmd}{} %
  \robustify{\cmd}%
}
\ShowDef{cmd}{%
  \DeclareRobustCommand{\cmd}[2][default]{} %
  \robustify{\cmd}%
}
\typeout{}
\@@end
%</showcases>
%    \end{macrocode}
%
% \subsubsection{Result}
%
% \begingroup
%   \makeatletter
%   \let\org@verbatim\@verbatim
%   \def\@verbatim{^^A
%     \org@verbatim
%     \catcode`\~=\active
%   }^^A
%   \let~\textvisiblespace
%\begin{verbatim}
%* LaTeX definitions:
%
%\newcommand {\cmd }[2][default]{}
%  (\cmd) = (macro:->\@protected@testopt \cmd \\cmd {default})
%  (\\cmd) = (\long macro:[#1]#2->)
%
%\DeclareRobustCommand {\cmd }{}
%  (\cmd) = (macro:->\protect \cmd~ )
%  (\cmd~) = (\long macro:->)
%
%\DeclareRobustCommand {\cmd }[2][default]{}
%  (\cmd) = (macro:->\protect \cmd~ )
%  (\cmd~) = (macro:->\@protected@testopt \cmd~ \\cmd~ {default})
%  (\\cmd~) = (\long macro:[#1]#2->)
%
%* etoolbox's robust definitions:
%
%\newrobustcmd {\cmd }{}
%  (\cmd) = (\protected\long macro:->)
%
%\newrobustcmd {\cmd }[2][default]{}
%  (\cmd) = (\protected macro:->\@testopt \\cmd {default})
%  (\\cmd) = (\long macro:[#1]#2->)
%
%* etoolbox's \robustify:
%
%\newcommand {\cmd }[2][default]{} \robustify {\cmd }
%  (\cmd) = (\protected macro:->\@protected@testopt \cmd \\cmd {default})
%  (\\cmd) = (\long macro:[#1]#2->)
%
%\DeclareRobustCommand {\cmd }{} \robustify {\cmd }
%  (\cmd) = (\protected macro:->)
%
%\DeclareRobustCommand {\cmd }[2][default]{} \robustify {\cmd }
%  (\cmd) = (\protected macro:->\@protected@testopt \cmd~ \\cmd~ {default})
%  (\cmd~) = (macro:->\@protected@testopt \cmd~ \\cmd~ {default})
%  (\\cmd~) = (\long macro:[#1]#2->)
%\end{verbatim}
% \endgroup
%
% \subsection{Package}
%
%    \begin{macrocode}
%<*package>
%    \end{macrocode}
%
% \subsubsection{Catcodes and identification}
%
%    \begin{macrocode}
\begingroup\catcode61\catcode48\catcode32=10\relax%
  \catcode13=5 % ^^M
  \endlinechar=13 %
  \catcode123=1 % {
  \catcode125=2 % }
  \catcode64=11 % @
  \def\x{\endgroup
    \expandafter\edef\csname llm@AtEnd\endcsname{%
      \endlinechar=\the\endlinechar\relax
      \catcode13=\the\catcode13\relax
      \catcode32=\the\catcode32\relax
      \catcode35=\the\catcode35\relax
      \catcode61=\the\catcode61\relax
      \catcode64=\the\catcode64\relax
      \catcode123=\the\catcode123\relax
      \catcode125=\the\catcode125\relax
    }%
  }%
\x\catcode61\catcode48\catcode32=10\relax%
\catcode13=5 % ^^M
\endlinechar=13 %
\catcode35=6 % #
\catcode64=11 % @
\catcode123=1 % {
\catcode125=2 % }
\def\TMP@EnsureCode#1#2{%
  \edef\llm@AtEnd{%
    \llm@AtEnd
    \catcode#1=\the\catcode#1\relax
  }%
  \catcode#1=#2\relax
}
\TMP@EnsureCode{40}{12}% (
\TMP@EnsureCode{41}{12}% )
\TMP@EnsureCode{42}{12}% *
\TMP@EnsureCode{45}{12}% -
\TMP@EnsureCode{46}{12}% .
\TMP@EnsureCode{47}{12}% /
\TMP@EnsureCode{58}{12}% :
\TMP@EnsureCode{62}{12}% >
\TMP@EnsureCode{91}{12}% [
\TMP@EnsureCode{93}{12}% ]
\edef\llm@AtEnd{%
  \llm@AtEnd
  \escapechar\the\escapechar\relax
  \noexpand\endinput
}
\escapechar=92 % `\\
%    \end{macrocode}
%
%    Package identification.
%    \begin{macrocode}
\NeedsTeXFormat{LaTeX2e}
\ProvidesPackage{letltxmacro}%
  [2019/12/03 v1.6 Let assignment for LaTeX macros (HO)]
%    \end{macrocode}
%
% \subsubsection{Main macros}
%
%    \begin{macro}{\LetLtxMacro}
%    \begin{macrocode}
\newcommand*{\LetLtxMacro}{%
  \llm@ModeLetLtxMacro{}%
}
%    \end{macrocode}
%    \end{macro}
%    \begin{macro}{\GlobalLetLtxMacro}
%    \begin{macrocode}
\newcommand*{\GlobalLetLtxMacro}{%
  \llm@ModeLetLtxMacro\global
}
%    \end{macrocode}
%    \end{macro}
%
%    \begin{macro}{\llm@ModeLetLtxMacro}
%    \begin{macrocode}
\newcommand*{\llm@ModeLetLtxMacro}[3]{%
  \edef\llm@escapechar{\the\escapechar}%
  \escapechar=-1 %
  \edef\reserved@a{%
    \noexpand\protect
    \expandafter\noexpand
    \csname\string#3 \endcsname
  }%
  \ifx\reserved@a#3\relax
    #1\edef#2{%
      \noexpand\protect
      \expandafter\noexpand
      \csname\string#2 \endcsname
    }%
    #1\expandafter\let
    \csname\string#2 \expandafter\endcsname
    \csname\string#3 \endcsname
    \expandafter\llm@LetLtxMacro
        \csname\string#2 \expandafter\endcsname
        \csname\string#3 \endcsname{#1}%
  \else
    \llm@LetLtxMacro{#2}{#3}{#1}%
  \fi
  \escapechar=\llm@escapechar\relax
}
%    \end{macrocode}
%    \end{macro}
%    \begin{macro}{\llm@LetLtxMacro}
%    \begin{macrocode}
\def\llm@LetLtxMacro#1#2#3{%
  \escapechar=92 %
  \expandafter\llm@CheckParams\meaning#2:->\@nil{%
    \begingroup
      \def\@protected@testopt{%
        \expandafter\@testopt\@gobble
      }%
      \def\@testopt##1##2{%
        \toks@={##2}%
      }%
      \let\llm@testopt\@empty
      \edef\x{%
        \noexpand\@protected@testopt
        \noexpand#2%
        \expandafter\noexpand\csname\string#2\endcsname
      }%
      \expandafter\expandafter\expandafter\def
      \expandafter\expandafter\expandafter\y
      \expandafter\expandafter\expandafter{%
        \expandafter\llm@CarThree#2{}{}{}\llm@nil
      }%
      \ifx\x\y
        #2%
        \def\llm@testopt{%
          \noexpand\@protected@testopt
          \noexpand#1%
        }%
      \else
        \edef\x{%
          \noexpand\@testopt
          \expandafter\noexpand
          \csname\string#2\endcsname
        }%
        \expandafter\expandafter\expandafter\def
        \expandafter\expandafter\expandafter\y
        \expandafter\expandafter\expandafter{%
          \expandafter\llm@CarTwo#2{}{}\llm@nil
        }%
        \ifx\x\y
          #2%
          \def\llm@testopt{%
            \noexpand\@testopt
          }%
        \fi
      \fi
      \ifx\llm@testopt\@empty
      \else
        \llm@protected\xdef\llm@GlobalTemp{%
          \llm@testopt
          \expandafter\noexpand
          \csname\string#1\endcsname
          {\the\toks@}%
        }%
      \fi
    \expandafter\endgroup\ifx\llm@testopt\@empty
      #3\let#1=#2\relax
    \else
      #3\let#1=\llm@GlobalTemp
      #3\expandafter\let
          \csname\string#1\expandafter\endcsname
          \csname\string#2\endcsname
    \fi
  }{%
    #3\let#1=#2\relax
  }%
}
%    \end{macrocode}
%    \end{macro}
%    \begin{macro}{\llm@CheckParams}
%    \begin{macrocode}
\def\llm@CheckParams#1:->#2\@nil{%
  \begingroup
    \def\x{#1}%
  \ifx\x\llm@macro
    \endgroup
    \def\llm@protected{}%
    \expandafter\@firstoftwo
  \else
    \ifx\x\llm@protectedmacro
      \endgroup
      \def\llm@protected{\protected}%
      \expandafter\expandafter\expandafter\@firstoftwo
    \else
      \endgroup
      \expandafter\expandafter\expandafter\@secondoftwo
    \fi
  \fi
}
%    \end{macrocode}
%    \end{macro}
%    \begin{macro}{\llm@macro}
%    \begin{macrocode}
\def\llm@macro{macro}
\@onelevel@sanitize\llm@macro
%    \end{macrocode}
%    \end{macro}
%    \begin{macro}{\llm@protectedmacro}
%    \begin{macrocode}
\def\llm@protectedmacro{\protected macro}
\@onelevel@sanitize\llm@protectedmacro
%    \end{macrocode}
%    \end{macro}
%    \begin{macro}{\llm@CarThree}
%    \begin{macrocode}
\def\llm@CarThree#1#2#3#4\llm@nil{#1#2#3}%
%    \end{macrocode}
%    \end{macro}
%    \begin{macro}{\llm@CarTwo}
%    \begin{macrocode}
\def\llm@CarTwo#1#2#3\llm@nil{#1#2}%
%    \end{macrocode}
%    \end{macro}
%
%    \begin{macrocode}
\llm@AtEnd%
%</package>
%    \end{macrocode}
% \section{Installation}
%
% \subsection{Download}
%
% \paragraph{Package.} This package is available on
% CTAN\footnote{\CTANpkg{letltxmacro}}:
% \begin{description}
% \item[\CTAN{macros/latex/contrib/letltxmacro/letltxmacro.dtx}] The source file.
% \item[\CTAN{macros/latex/contrib/letltxmacro/letltxmacro.pdf}] Documentation.
% \end{description}
%
%
% \paragraph{Bundle.} All the packages of the bundle `letltxmacro'
% are also available in a TDS compliant ZIP archive. There
% the packages are already unpacked and the documentation files
% are generated. The files and directories obey the TDS standard.
% \begin{description}
% \item[\CTANinstall{install/macros/latex/contrib/letltxmacro.tds.zip}]
% \end{description}
% \emph{TDS} refers to the standard ``A Directory Structure
% for \TeX\ Files'' (\CTANpkg{tds}). Directories
% with \xfile{texmf} in their name are usually organized this way.
%
% \subsection{Bundle installation}
%
% \paragraph{Unpacking.} Unpack the \xfile{letltxmacro.tds.zip} in the
% TDS tree (also known as \xfile{texmf} tree) of your choice.
% Example (linux):
% \begin{quote}
%   |unzip letltxmacro.tds.zip -d ~/texmf|
% \end{quote}
%
% \subsection{Package installation}
%
% \paragraph{Unpacking.} The \xfile{.dtx} file is a self-extracting
% \docstrip\ archive. The files are extracted by running the
% \xfile{.dtx} through \plainTeX:
% \begin{quote}
%   \verb|tex letltxmacro.dtx|
% \end{quote}
%
% \paragraph{TDS.} Now the different files must be moved into
% the different directories in your installation TDS tree
% (also known as \xfile{texmf} tree):
% \begin{quote}
% \def\t{^^A
% \begin{tabular}{@{}>{\ttfamily}l@{ $\rightarrow$ }>{\ttfamily}l@{}}
%   letltxmacro.sty & tex/latex/letltxmacro/letltxmacro.sty\\
%   letltxmacro.pdf & doc/latex/letltxmacro/letltxmacro.pdf\\
%   letltxmacro-showcases.tex & doc/latex/letltxmacro/letltxmacro-showcases.tex\\
%   letltxmacro.dtx & source/latex/letltxmacro/letltxmacro.dtx\\
% \end{tabular}^^A
% }^^A
% \sbox0{\t}^^A
% \ifdim\wd0>\linewidth
%   \begingroup
%     \advance\linewidth by\leftmargin
%     \advance\linewidth by\rightmargin
%   \edef\x{\endgroup
%     \def\noexpand\lw{\the\linewidth}^^A
%   }\x
%   \def\lwbox{^^A
%     \leavevmode
%     \hbox to \linewidth{^^A
%       \kern-\leftmargin\relax
%       \hss
%       \usebox0
%       \hss
%       \kern-\rightmargin\relax
%     }^^A
%   }^^A
%   \ifdim\wd0>\lw
%     \sbox0{\small\t}^^A
%     \ifdim\wd0>\linewidth
%       \ifdim\wd0>\lw
%         \sbox0{\footnotesize\t}^^A
%         \ifdim\wd0>\linewidth
%           \ifdim\wd0>\lw
%             \sbox0{\scriptsize\t}^^A
%             \ifdim\wd0>\linewidth
%               \ifdim\wd0>\lw
%                 \sbox0{\tiny\t}^^A
%                 \ifdim\wd0>\linewidth
%                   \lwbox
%                 \else
%                   \usebox0
%                 \fi
%               \else
%                 \lwbox
%               \fi
%             \else
%               \usebox0
%             \fi
%           \else
%             \lwbox
%           \fi
%         \else
%           \usebox0
%         \fi
%       \else
%         \lwbox
%       \fi
%     \else
%       \usebox0
%     \fi
%   \else
%     \lwbox
%   \fi
% \else
%   \usebox0
% \fi
% \end{quote}
% If you have a \xfile{docstrip.cfg} that configures and enables \docstrip's
% TDS installing feature, then some files can already be in the right
% place, see the documentation of \docstrip.
%
% \subsection{Refresh file name databases}
%
% If your \TeX~distribution
% (\TeX\,Live, \mikTeX, \dots) relies on file name databases, you must refresh
% these. For example, \TeX\,Live\ users run \verb|texhash| or
% \verb|mktexlsr|.
%
% \subsection{Some details for the interested}
%
% \paragraph{Unpacking with \LaTeX.}
% The \xfile{.dtx} chooses its action depending on the format:
% \begin{description}
% \item[\plainTeX:] Run \docstrip\ and extract the files.
% \item[\LaTeX:] Generate the documentation.
% \end{description}
% If you insist on using \LaTeX\ for \docstrip\ (really,
% \docstrip\ does not need \LaTeX), then inform the autodetect routine
% about your intention:
% \begin{quote}
%   \verb|latex \let\install=y\input{letltxmacro.dtx}|
% \end{quote}
% Do not forget to quote the argument according to the demands
% of your shell.
%
% \paragraph{Generating the documentation.}
% You can use both the \xfile{.dtx} or the \xfile{.drv} to generate
% the documentation. The process can be configured by the
% configuration file \xfile{ltxdoc.cfg}. For instance, put this
% line into this file, if you want to have A4 as paper format:
% \begin{quote}
%   \verb|\PassOptionsToClass{a4paper}{article}|
% \end{quote}
% An example follows how to generate the
% documentation with pdf\LaTeX:
% \begin{quote}
%\begin{verbatim}
%pdflatex letltxmacro.dtx
%makeindex -s gind.ist letltxmacro.idx
%pdflatex letltxmacro.dtx
%makeindex -s gind.ist letltxmacro.idx
%pdflatex letltxmacro.dtx
%\end{verbatim}
% \end{quote}
%
% \begin{History}
%   \begin{Version}{2008/06/09 v1.0}
%   \item
%     First version.
%   \end{Version}
%   \begin{Version}{2008/06/12 v1.1}
%   \item
%     Support for \xpackage{etoolbox}'s \cs{newrobustcmd} added.
%   \end{Version}
%   \begin{Version}{2008/06/13 v1.2}
%   \item
%     Support for \xpackage{etoolbox}'s \cs{robustify} added.
%   \end{Version}
%   \begin{Version}{2008/06/24 v1.3}
%   \item
%     Test file adapted for etoolbox 2008/06/22 v1.6.
%   \end{Version}
%   \begin{Version}{2010/09/02 v1.4}
%   \item
%     \cs{GlobalLetLtxMacro} added.
%   \end{Version}
%   \begin{Version}{2016/05/16 v1.5}
%   \item
%     Documentation updates.
%   \end{Version}
%   \begin{Version}{2019/12/03 v1.6}
%   \item
%     Documentation updates.
%   \end{Version}
% \end{History}
%
% \PrintIndex
%
% \Finale
\endinput

%        (quote the arguments according to the demands of your shell)
%
% Documentation:
%    (a) If letltxmacro.drv is present:
%           latex letltxmacro.drv
%    (b) Without letltxmacro.drv:
%           latex letltxmacro.dtx; ...
%    The class ltxdoc loads the configuration file ltxdoc.cfg
%    if available. Here you can specify further options, e.g.
%    use A4 as paper format:
%       \PassOptionsToClass{a4paper}{article}
%
%    Programm calls to get the documentation (example):
%       pdflatex letltxmacro.dtx
%       makeindex -s gind.ist letltxmacro.idx
%       pdflatex letltxmacro.dtx
%       makeindex -s gind.ist letltxmacro.idx
%       pdflatex letltxmacro.dtx
%
% Installation:
%    TDS:tex/latex/letltxmacro/letltxmacro.sty
%    TDS:doc/latex/letltxmacro/letltxmacro.pdf
%    TDS:doc/latex/letltxmacro/letltxmacro-showcases.tex
%    TDS:source/latex/letltxmacro/letltxmacro.dtx
%
%<*ignore>
\begingroup
  \catcode123=1 %
  \catcode125=2 %
  \def\x{LaTeX2e}%
\expandafter\endgroup
\ifcase 0\ifx\install y1\fi\expandafter
         \ifx\csname processbatchFile\endcsname\relax\else1\fi
         \ifx\fmtname\x\else 1\fi\relax
\else\csname fi\endcsname
%</ignore>
%<*install>
\input docstrip.tex
\Msg{************************************************************************}
\Msg{* Installation}
\Msg{* Package: letltxmacro 2019/12/03 v1.6 Let assignment for LaTeX macros (HO)}
\Msg{************************************************************************}

\keepsilent
\askforoverwritefalse

\let\MetaPrefix\relax
\preamble

This is a generated file.

Project: letltxmacro
Version: 2019/12/03 v1.6

Copyright (C)
   2008, 2010 Heiko Oberdiek
   2016-2019 Oberdiek Package Support Group

This work may be distributed and/or modified under the
conditions of the LaTeX Project Public License, either
version 1.3c of this license or (at your option) any later
version. This version of this license is in
   https://www.latex-project.org/lppl/lppl-1-3c.txt
and the latest version of this license is in
   https://www.latex-project.org/lppl.txt
and version 1.3 or later is part of all distributions of
LaTeX version 2005/12/01 or later.

This work has the LPPL maintenance status "maintained".

The Current Maintainers of this work are
Heiko Oberdiek and the Oberdiek Package Support Group
https://github.com/ho-tex/letltxmacro/issues


This work consists of the main source file letltxmacro.dtx
and the derived files
   letltxmacro.sty, letltxmacro.pdf, letltxmacro.ins, letltxmacro.drv,
   letltxmacro-showcases.tex, letltxmacro-test1.tex,
   letltxmacro-test2.tex.

\endpreamble
\let\MetaPrefix\DoubleperCent

\generate{%
  \file{letltxmacro.ins}{\from{letltxmacro.dtx}{install}}%
  \file{letltxmacro.drv}{\from{letltxmacro.dtx}{driver}}%
  \usedir{tex/latex/letltxmacro}%
  \file{letltxmacro.sty}{\from{letltxmacro.dtx}{package}}%
  \usedir{doc/latex/letltxmacro}%
  \file{letltxmacro-showcases.tex}{\from{letltxmacro.dtx}{showcases}}%
%  \usedir{doc/latex/letltxmacro/test}%
%  \file{letltxmacro-test1.tex}{\from{letltxmacro.dtx}{test1}}%
%  \file{letltxmacro-test2.tex}{\from{letltxmacro.dtx}{test2}}%
}

\catcode32=13\relax% active space
\let =\space%
\Msg{************************************************************************}
\Msg{*}
\Msg{* To finish the installation you have to move the following}
\Msg{* file into a directory searched by TeX:}
\Msg{*}
\Msg{*     letltxmacro.sty}
\Msg{*}
\Msg{* To produce the documentation run the file `letltxmacro.drv'}
\Msg{* through LaTeX.}
\Msg{*}
\Msg{* Happy TeXing!}
\Msg{*}
\Msg{************************************************************************}

\endbatchfile
%</install>
%<*ignore>
\fi
%</ignore>
%<*driver>
\NeedsTeXFormat{LaTeX2e}
\ProvidesFile{letltxmacro.drv}%
  [2019/12/03 v1.6 Let assignment for LaTeX macros (HO)]%
\documentclass{ltxdoc}
\usepackage{holtxdoc}[2011/11/22]
\begin{document}
  \DocInput{letltxmacro.dtx}%
\end{document}
%</driver>
% \fi
%
%
%
% \GetFileInfo{letltxmacro.drv}
%
% \title{The \xpackage{letltxmacro} package}
% \date{2019/12/03 v1.6}
% \author{Heiko Oberdiek\thanks
% {Please report any issues at \url{https://github.com/ho-tex/letltxmacro/issues}}}
%
% \maketitle
%
% \begin{abstract}
% \TeX's \cs{let} assignment does not work for \LaTeX\ macros
% with optional arguments or for macros that are defined
% as robust macros by \cs{DeclareRobustCommand}. This package
% defines \cs{LetLtxMacro} that also takes care of the involved
% internal macros.
% \end{abstract}
%
% \tableofcontents
%
% \section{Documentation}
%
% If someone wants to redefine a macro with using the old
% meaning, then one method is \TeX's command \cs{let}:
%\begin{quote}
%\begin{verbatim}
%\newcommand{\Macro}{\typeout{Test Macro}}
%\let\SavedMacro=\Macro
%\renewcommand{\Macro}{%
%  \typeout{Begin}%
%  \SavedMacro
%  \typeout{End}%
%}
%\end{verbatim}
%\end{quote}
% However, this method fails, if \cs{Macro} is defined
% by \cs{DeclareRobustCommand} and/or has an optional argument.
% In both cases \LaTeX\ defines an additional internal macro
% that is forgotten in the simple \cs{let} assignment of
% the example above.
%
% \begin{declcs}{LetLtxMacro} \M{new macro} \M{old macro}
% \end{declcs}
% Macro \cs{LetLtxMacro} behaves similar to \TeX's \cs{let}
% assignment, but it takes care of macros that are
% defined by \cs{DeclareRobustCommand} and/or have optional
% arguments. Example:
%\begin{quote}
%\begin{verbatim}
%\DeclareRobustCommand{\Macro}[1][default]{...}
%\LetLtxMacro{\SavedMacro}{\Macro}
%\end{verbatim}
%\end{quote}
% Then macro \cs{SavedMacro} only uses internal macro names
% that are derived from \cs{SavedMacro}'s macro name. Macro \cs{Macro}
% can now be redefined without affecting \cs{SavedMacro}.
%
% \begin{declcs}{GlobalLetLtxMacro} \M{new macro} \M{old macro}
% \end{declcs}
% Like \cs{LetLtxMacro}, but the \meta{new macro} is defined globally.
% Since version 2019/12/03~v1.4.
%
% \subsection{Supported macro definition commands}
%
% \begin{quote}
%   \begin{tabular}{@{}ll@{}}
%     \cs{newcommand}, \cs{renewcommand} & latex/base\\
%     \cs{newenvironment}, \cs{renewenvironment} & latex/base\\
%     \cs{DeclareRobustCommand}& latex/base\\
%     \cs{newrobustcmd}, \cs{renewrobustcmd} & etoolbox\\
%     \cs{robustify} & etoolbox 2008/06/22 v1.6\\
%   \end{tabular}
% \end{quote}
%
% \StopEventually{
% }
%
% \section{Implementation}
%
% \subsection{Show cases}
%
% \subsubsection{\xfile{letltxmacro-showcases.tex}}
%
%    \begin{macrocode}
%<*showcases>
\NeedsTeXFormat{LaTeX2e}
\makeatletter
%    \end{macrocode}
%    \begin{macro}{\Line}
%    The result is displayed by macro \cs{Line}. The percent symbol
%    at line start allows easy grepping and inserting into the DTX
%    file.
%    \begin{macrocode}
\newcommand*{\Line}[1]{%
  \typeout{\@percentchar#1}%
}
%    \end{macrocode}
%    \end{macro}
%    \begin{macrocode}
\newcommand*{\ShowCmdName}[1]{%
  \@ifundefined{#1}{}{%
    \Line{%
      \space\space(\expandafter\string\csname#1\endcsname) = %
      (\expandafter\meaning\csname#1\endcsname)%
    }%
  }%
}
\newcommand*{\ShowCmds}[1]{%
  \ShowCmdName{#1}%
  \ShowCmdName{#1 }%
  \ShowCmdName{\\#1}%
  \ShowCmdName{\\#1 }%
}
\let\\\@backslashchar
%    \end{macrocode}
%    \begin{macro}{\ShowDef}
%    \begin{macrocode}
\newcommand*{\ShowDef}[2]{%
  \begingroup
    \Line{}%
    \newcommand*{\DefString}{#2}%
    \@onelevel@sanitize\DefString
    \Line{\DefString}%
    #2%
    \ShowCmds{#1}%
  \endgroup
}
%    \end{macrocode}
%    \end{macro}
%    \begin{macrocode}
\typeout{}
\Line{* LaTeX definitions:}
\ShowDef{cmd}{%
  \newcommand{\cmd}[2][default]{}%
}
\ShowDef{cmd}{%
  \DeclareRobustCommand{\cmd}{}%
}
\ShowDef{cmd}{%
  \DeclareRobustCommand{\cmd}[2][default]{}%
}
\typeout{}
%    \end{macrocode}
% The minimal version of package \xpackage{etoolbox} is 2008/06/12 v1.6a
% because it fixes \cs{robustify}.
%    \begin{macrocode}
\RequirePackage{etoolbox}[2008/06/12]%
\Line{}
\Line{* etoolbox's robust definitions:}
\ShowDef{cmd}{%
  \newrobustcmd{\cmd}{}%
}
\ShowDef{cmd}{%
  \newrobustcmd{\cmd}[2][default]{}%
}
\Line{}
\Line{* etoolbox's \string\robustify:}
\ShowDef{cmd}{%
  \newcommand{\cmd}[2][default]{} %
  \robustify{\cmd}%
}
\ShowDef{cmd}{%
  \DeclareRobustCommand{\cmd}{} %
  \robustify{\cmd}%
}
\ShowDef{cmd}{%
  \DeclareRobustCommand{\cmd}[2][default]{} %
  \robustify{\cmd}%
}
\typeout{}
\@@end
%</showcases>
%    \end{macrocode}
%
% \subsubsection{Result}
%
% \begingroup
%   \makeatletter
%   \let\org@verbatim\@verbatim
%   \def\@verbatim{^^A
%     \org@verbatim
%     \catcode`\~=\active
%   }^^A
%   \let~\textvisiblespace
%\begin{verbatim}
%* LaTeX definitions:
%
%\newcommand {\cmd }[2][default]{}
%  (\cmd) = (macro:->\@protected@testopt \cmd \\cmd {default})
%  (\\cmd) = (\long macro:[#1]#2->)
%
%\DeclareRobustCommand {\cmd }{}
%  (\cmd) = (macro:->\protect \cmd~ )
%  (\cmd~) = (\long macro:->)
%
%\DeclareRobustCommand {\cmd }[2][default]{}
%  (\cmd) = (macro:->\protect \cmd~ )
%  (\cmd~) = (macro:->\@protected@testopt \cmd~ \\cmd~ {default})
%  (\\cmd~) = (\long macro:[#1]#2->)
%
%* etoolbox's robust definitions:
%
%\newrobustcmd {\cmd }{}
%  (\cmd) = (\protected\long macro:->)
%
%\newrobustcmd {\cmd }[2][default]{}
%  (\cmd) = (\protected macro:->\@testopt \\cmd {default})
%  (\\cmd) = (\long macro:[#1]#2->)
%
%* etoolbox's \robustify:
%
%\newcommand {\cmd }[2][default]{} \robustify {\cmd }
%  (\cmd) = (\protected macro:->\@protected@testopt \cmd \\cmd {default})
%  (\\cmd) = (\long macro:[#1]#2->)
%
%\DeclareRobustCommand {\cmd }{} \robustify {\cmd }
%  (\cmd) = (\protected macro:->)
%
%\DeclareRobustCommand {\cmd }[2][default]{} \robustify {\cmd }
%  (\cmd) = (\protected macro:->\@protected@testopt \cmd~ \\cmd~ {default})
%  (\cmd~) = (macro:->\@protected@testopt \cmd~ \\cmd~ {default})
%  (\\cmd~) = (\long macro:[#1]#2->)
%\end{verbatim}
% \endgroup
%
% \subsection{Package}
%
%    \begin{macrocode}
%<*package>
%    \end{macrocode}
%
% \subsubsection{Catcodes and identification}
%
%    \begin{macrocode}
\begingroup\catcode61\catcode48\catcode32=10\relax%
  \catcode13=5 % ^^M
  \endlinechar=13 %
  \catcode123=1 % {
  \catcode125=2 % }
  \catcode64=11 % @
  \def\x{\endgroup
    \expandafter\edef\csname llm@AtEnd\endcsname{%
      \endlinechar=\the\endlinechar\relax
      \catcode13=\the\catcode13\relax
      \catcode32=\the\catcode32\relax
      \catcode35=\the\catcode35\relax
      \catcode61=\the\catcode61\relax
      \catcode64=\the\catcode64\relax
      \catcode123=\the\catcode123\relax
      \catcode125=\the\catcode125\relax
    }%
  }%
\x\catcode61\catcode48\catcode32=10\relax%
\catcode13=5 % ^^M
\endlinechar=13 %
\catcode35=6 % #
\catcode64=11 % @
\catcode123=1 % {
\catcode125=2 % }
\def\TMP@EnsureCode#1#2{%
  \edef\llm@AtEnd{%
    \llm@AtEnd
    \catcode#1=\the\catcode#1\relax
  }%
  \catcode#1=#2\relax
}
\TMP@EnsureCode{40}{12}% (
\TMP@EnsureCode{41}{12}% )
\TMP@EnsureCode{42}{12}% *
\TMP@EnsureCode{45}{12}% -
\TMP@EnsureCode{46}{12}% .
\TMP@EnsureCode{47}{12}% /
\TMP@EnsureCode{58}{12}% :
\TMP@EnsureCode{62}{12}% >
\TMP@EnsureCode{91}{12}% [
\TMP@EnsureCode{93}{12}% ]
\edef\llm@AtEnd{%
  \llm@AtEnd
  \escapechar\the\escapechar\relax
  \noexpand\endinput
}
\escapechar=92 % `\\
%    \end{macrocode}
%
%    Package identification.
%    \begin{macrocode}
\NeedsTeXFormat{LaTeX2e}
\ProvidesPackage{letltxmacro}%
  [2019/12/03 v1.6 Let assignment for LaTeX macros (HO)]
%    \end{macrocode}
%
% \subsubsection{Main macros}
%
%    \begin{macro}{\LetLtxMacro}
%    \begin{macrocode}
\newcommand*{\LetLtxMacro}{%
  \llm@ModeLetLtxMacro{}%
}
%    \end{macrocode}
%    \end{macro}
%    \begin{macro}{\GlobalLetLtxMacro}
%    \begin{macrocode}
\newcommand*{\GlobalLetLtxMacro}{%
  \llm@ModeLetLtxMacro\global
}
%    \end{macrocode}
%    \end{macro}
%
%    \begin{macro}{\llm@ModeLetLtxMacro}
%    \begin{macrocode}
\newcommand*{\llm@ModeLetLtxMacro}[3]{%
  \edef\llm@escapechar{\the\escapechar}%
  \escapechar=-1 %
  \edef\reserved@a{%
    \noexpand\protect
    \expandafter\noexpand
    \csname\string#3 \endcsname
  }%
  \ifx\reserved@a#3\relax
    #1\edef#2{%
      \noexpand\protect
      \expandafter\noexpand
      \csname\string#2 \endcsname
    }%
    #1\expandafter\let
    \csname\string#2 \expandafter\endcsname
    \csname\string#3 \endcsname
    \expandafter\llm@LetLtxMacro
        \csname\string#2 \expandafter\endcsname
        \csname\string#3 \endcsname{#1}%
  \else
    \llm@LetLtxMacro{#2}{#3}{#1}%
  \fi
  \escapechar=\llm@escapechar\relax
}
%    \end{macrocode}
%    \end{macro}
%    \begin{macro}{\llm@LetLtxMacro}
%    \begin{macrocode}
\def\llm@LetLtxMacro#1#2#3{%
  \escapechar=92 %
  \expandafter\llm@CheckParams\meaning#2:->\@nil{%
    \begingroup
      \def\@protected@testopt{%
        \expandafter\@testopt\@gobble
      }%
      \def\@testopt##1##2{%
        \toks@={##2}%
      }%
      \let\llm@testopt\@empty
      \edef\x{%
        \noexpand\@protected@testopt
        \noexpand#2%
        \expandafter\noexpand\csname\string#2\endcsname
      }%
      \expandafter\expandafter\expandafter\def
      \expandafter\expandafter\expandafter\y
      \expandafter\expandafter\expandafter{%
        \expandafter\llm@CarThree#2{}{}{}\llm@nil
      }%
      \ifx\x\y
        #2%
        \def\llm@testopt{%
          \noexpand\@protected@testopt
          \noexpand#1%
        }%
      \else
        \edef\x{%
          \noexpand\@testopt
          \expandafter\noexpand
          \csname\string#2\endcsname
        }%
        \expandafter\expandafter\expandafter\def
        \expandafter\expandafter\expandafter\y
        \expandafter\expandafter\expandafter{%
          \expandafter\llm@CarTwo#2{}{}\llm@nil
        }%
        \ifx\x\y
          #2%
          \def\llm@testopt{%
            \noexpand\@testopt
          }%
        \fi
      \fi
      \ifx\llm@testopt\@empty
      \else
        \llm@protected\xdef\llm@GlobalTemp{%
          \llm@testopt
          \expandafter\noexpand
          \csname\string#1\endcsname
          {\the\toks@}%
        }%
      \fi
    \expandafter\endgroup\ifx\llm@testopt\@empty
      #3\let#1=#2\relax
    \else
      #3\let#1=\llm@GlobalTemp
      #3\expandafter\let
          \csname\string#1\expandafter\endcsname
          \csname\string#2\endcsname
    \fi
  }{%
    #3\let#1=#2\relax
  }%
}
%    \end{macrocode}
%    \end{macro}
%    \begin{macro}{\llm@CheckParams}
%    \begin{macrocode}
\def\llm@CheckParams#1:->#2\@nil{%
  \begingroup
    \def\x{#1}%
  \ifx\x\llm@macro
    \endgroup
    \def\llm@protected{}%
    \expandafter\@firstoftwo
  \else
    \ifx\x\llm@protectedmacro
      \endgroup
      \def\llm@protected{\protected}%
      \expandafter\expandafter\expandafter\@firstoftwo
    \else
      \endgroup
      \expandafter\expandafter\expandafter\@secondoftwo
    \fi
  \fi
}
%    \end{macrocode}
%    \end{macro}
%    \begin{macro}{\llm@macro}
%    \begin{macrocode}
\def\llm@macro{macro}
\@onelevel@sanitize\llm@macro
%    \end{macrocode}
%    \end{macro}
%    \begin{macro}{\llm@protectedmacro}
%    \begin{macrocode}
\def\llm@protectedmacro{\protected macro}
\@onelevel@sanitize\llm@protectedmacro
%    \end{macrocode}
%    \end{macro}
%    \begin{macro}{\llm@CarThree}
%    \begin{macrocode}
\def\llm@CarThree#1#2#3#4\llm@nil{#1#2#3}%
%    \end{macrocode}
%    \end{macro}
%    \begin{macro}{\llm@CarTwo}
%    \begin{macrocode}
\def\llm@CarTwo#1#2#3\llm@nil{#1#2}%
%    \end{macrocode}
%    \end{macro}
%
%    \begin{macrocode}
\llm@AtEnd%
%</package>
%    \end{macrocode}
% \section{Installation}
%
% \subsection{Download}
%
% \paragraph{Package.} This package is available on
% CTAN\footnote{\CTANpkg{letltxmacro}}:
% \begin{description}
% \item[\CTAN{macros/latex/contrib/letltxmacro/letltxmacro.dtx}] The source file.
% \item[\CTAN{macros/latex/contrib/letltxmacro/letltxmacro.pdf}] Documentation.
% \end{description}
%
%
% \paragraph{Bundle.} All the packages of the bundle `letltxmacro'
% are also available in a TDS compliant ZIP archive. There
% the packages are already unpacked and the documentation files
% are generated. The files and directories obey the TDS standard.
% \begin{description}
% \item[\CTANinstall{install/macros/latex/contrib/letltxmacro.tds.zip}]
% \end{description}
% \emph{TDS} refers to the standard ``A Directory Structure
% for \TeX\ Files'' (\CTANpkg{tds}). Directories
% with \xfile{texmf} in their name are usually organized this way.
%
% \subsection{Bundle installation}
%
% \paragraph{Unpacking.} Unpack the \xfile{letltxmacro.tds.zip} in the
% TDS tree (also known as \xfile{texmf} tree) of your choice.
% Example (linux):
% \begin{quote}
%   |unzip letltxmacro.tds.zip -d ~/texmf|
% \end{quote}
%
% \subsection{Package installation}
%
% \paragraph{Unpacking.} The \xfile{.dtx} file is a self-extracting
% \docstrip\ archive. The files are extracted by running the
% \xfile{.dtx} through \plainTeX:
% \begin{quote}
%   \verb|tex letltxmacro.dtx|
% \end{quote}
%
% \paragraph{TDS.} Now the different files must be moved into
% the different directories in your installation TDS tree
% (also known as \xfile{texmf} tree):
% \begin{quote}
% \def\t{^^A
% \begin{tabular}{@{}>{\ttfamily}l@{ $\rightarrow$ }>{\ttfamily}l@{}}
%   letltxmacro.sty & tex/latex/letltxmacro/letltxmacro.sty\\
%   letltxmacro.pdf & doc/latex/letltxmacro/letltxmacro.pdf\\
%   letltxmacro-showcases.tex & doc/latex/letltxmacro/letltxmacro-showcases.tex\\
%   letltxmacro.dtx & source/latex/letltxmacro/letltxmacro.dtx\\
% \end{tabular}^^A
% }^^A
% \sbox0{\t}^^A
% \ifdim\wd0>\linewidth
%   \begingroup
%     \advance\linewidth by\leftmargin
%     \advance\linewidth by\rightmargin
%   \edef\x{\endgroup
%     \def\noexpand\lw{\the\linewidth}^^A
%   }\x
%   \def\lwbox{^^A
%     \leavevmode
%     \hbox to \linewidth{^^A
%       \kern-\leftmargin\relax
%       \hss
%       \usebox0
%       \hss
%       \kern-\rightmargin\relax
%     }^^A
%   }^^A
%   \ifdim\wd0>\lw
%     \sbox0{\small\t}^^A
%     \ifdim\wd0>\linewidth
%       \ifdim\wd0>\lw
%         \sbox0{\footnotesize\t}^^A
%         \ifdim\wd0>\linewidth
%           \ifdim\wd0>\lw
%             \sbox0{\scriptsize\t}^^A
%             \ifdim\wd0>\linewidth
%               \ifdim\wd0>\lw
%                 \sbox0{\tiny\t}^^A
%                 \ifdim\wd0>\linewidth
%                   \lwbox
%                 \else
%                   \usebox0
%                 \fi
%               \else
%                 \lwbox
%               \fi
%             \else
%               \usebox0
%             \fi
%           \else
%             \lwbox
%           \fi
%         \else
%           \usebox0
%         \fi
%       \else
%         \lwbox
%       \fi
%     \else
%       \usebox0
%     \fi
%   \else
%     \lwbox
%   \fi
% \else
%   \usebox0
% \fi
% \end{quote}
% If you have a \xfile{docstrip.cfg} that configures and enables \docstrip's
% TDS installing feature, then some files can already be in the right
% place, see the documentation of \docstrip.
%
% \subsection{Refresh file name databases}
%
% If your \TeX~distribution
% (\TeX\,Live, \mikTeX, \dots) relies on file name databases, you must refresh
% these. For example, \TeX\,Live\ users run \verb|texhash| or
% \verb|mktexlsr|.
%
% \subsection{Some details for the interested}
%
% \paragraph{Unpacking with \LaTeX.}
% The \xfile{.dtx} chooses its action depending on the format:
% \begin{description}
% \item[\plainTeX:] Run \docstrip\ and extract the files.
% \item[\LaTeX:] Generate the documentation.
% \end{description}
% If you insist on using \LaTeX\ for \docstrip\ (really,
% \docstrip\ does not need \LaTeX), then inform the autodetect routine
% about your intention:
% \begin{quote}
%   \verb|latex \let\install=y% \iffalse meta-comment
%
% File: letltxmacro.dtx
% Version: 2019/12/03 v1.6
% Info: Let assignment for LaTeX macros
%
% Copyright (C)
%    2008, 2010 Heiko Oberdiek
%    2016-2019 Oberdiek Package Support Group
%    https://github.com/ho-tex/letltxmacro/issues
%
% This work may be distributed and/or modified under the
% conditions of the LaTeX Project Public License, either
% version 1.3c of this license or (at your option) any later
% version. This version of this license is in
%    https://www.latex-project.org/lppl/lppl-1-3c.txt
% and the latest version of this license is in
%    https://www.latex-project.org/lppl.txt
% and version 1.3 or later is part of all distributions of
% LaTeX version 2005/12/01 or later.
%
% This work has the LPPL maintenance status "maintained".
%
% The Current Maintainers of this work are
% Heiko Oberdiek and the Oberdiek Package Support Group
% https://github.com/ho-tex/letltxmacro/issues
%
% This work consists of the main source file letltxmacro.dtx
% and the derived files
%    letltxmacro.sty, letltxmacro.pdf, letltxmacro.ins, letltxmacro.drv,
%    letltxmacro-showcases.tex, letltxmacro-test1.tex,
%    letltxmacro-test2.tex.
%
% Distribution:
%    CTAN:macros/latex/contrib/letltxmacro/letltxmacro.dtx
%    CTAN:macros/latex/contrib/letltxmacro/letltxmacro.pdf
%
% Unpacking:
%    (a) If letltxmacro.ins is present:
%           tex letltxmacro.ins
%    (b) Without letltxmacro.ins:
%           tex letltxmacro.dtx
%    (c) If you insist on using LaTeX
%           latex \let\install=y\input{letltxmacro.dtx}
%        (quote the arguments according to the demands of your shell)
%
% Documentation:
%    (a) If letltxmacro.drv is present:
%           latex letltxmacro.drv
%    (b) Without letltxmacro.drv:
%           latex letltxmacro.dtx; ...
%    The class ltxdoc loads the configuration file ltxdoc.cfg
%    if available. Here you can specify further options, e.g.
%    use A4 as paper format:
%       \PassOptionsToClass{a4paper}{article}
%
%    Programm calls to get the documentation (example):
%       pdflatex letltxmacro.dtx
%       makeindex -s gind.ist letltxmacro.idx
%       pdflatex letltxmacro.dtx
%       makeindex -s gind.ist letltxmacro.idx
%       pdflatex letltxmacro.dtx
%
% Installation:
%    TDS:tex/latex/letltxmacro/letltxmacro.sty
%    TDS:doc/latex/letltxmacro/letltxmacro.pdf
%    TDS:doc/latex/letltxmacro/letltxmacro-showcases.tex
%    TDS:source/latex/letltxmacro/letltxmacro.dtx
%
%<*ignore>
\begingroup
  \catcode123=1 %
  \catcode125=2 %
  \def\x{LaTeX2e}%
\expandafter\endgroup
\ifcase 0\ifx\install y1\fi\expandafter
         \ifx\csname processbatchFile\endcsname\relax\else1\fi
         \ifx\fmtname\x\else 1\fi\relax
\else\csname fi\endcsname
%</ignore>
%<*install>
\input docstrip.tex
\Msg{************************************************************************}
\Msg{* Installation}
\Msg{* Package: letltxmacro 2019/12/03 v1.6 Let assignment for LaTeX macros (HO)}
\Msg{************************************************************************}

\keepsilent
\askforoverwritefalse

\let\MetaPrefix\relax
\preamble

This is a generated file.

Project: letltxmacro
Version: 2019/12/03 v1.6

Copyright (C)
   2008, 2010 Heiko Oberdiek
   2016-2019 Oberdiek Package Support Group

This work may be distributed and/or modified under the
conditions of the LaTeX Project Public License, either
version 1.3c of this license or (at your option) any later
version. This version of this license is in
   https://www.latex-project.org/lppl/lppl-1-3c.txt
and the latest version of this license is in
   https://www.latex-project.org/lppl.txt
and version 1.3 or later is part of all distributions of
LaTeX version 2005/12/01 or later.

This work has the LPPL maintenance status "maintained".

The Current Maintainers of this work are
Heiko Oberdiek and the Oberdiek Package Support Group
https://github.com/ho-tex/letltxmacro/issues


This work consists of the main source file letltxmacro.dtx
and the derived files
   letltxmacro.sty, letltxmacro.pdf, letltxmacro.ins, letltxmacro.drv,
   letltxmacro-showcases.tex, letltxmacro-test1.tex,
   letltxmacro-test2.tex.

\endpreamble
\let\MetaPrefix\DoubleperCent

\generate{%
  \file{letltxmacro.ins}{\from{letltxmacro.dtx}{install}}%
  \file{letltxmacro.drv}{\from{letltxmacro.dtx}{driver}}%
  \usedir{tex/latex/letltxmacro}%
  \file{letltxmacro.sty}{\from{letltxmacro.dtx}{package}}%
  \usedir{doc/latex/letltxmacro}%
  \file{letltxmacro-showcases.tex}{\from{letltxmacro.dtx}{showcases}}%
%  \usedir{doc/latex/letltxmacro/test}%
%  \file{letltxmacro-test1.tex}{\from{letltxmacro.dtx}{test1}}%
%  \file{letltxmacro-test2.tex}{\from{letltxmacro.dtx}{test2}}%
}

\catcode32=13\relax% active space
\let =\space%
\Msg{************************************************************************}
\Msg{*}
\Msg{* To finish the installation you have to move the following}
\Msg{* file into a directory searched by TeX:}
\Msg{*}
\Msg{*     letltxmacro.sty}
\Msg{*}
\Msg{* To produce the documentation run the file `letltxmacro.drv'}
\Msg{* through LaTeX.}
\Msg{*}
\Msg{* Happy TeXing!}
\Msg{*}
\Msg{************************************************************************}

\endbatchfile
%</install>
%<*ignore>
\fi
%</ignore>
%<*driver>
\NeedsTeXFormat{LaTeX2e}
\ProvidesFile{letltxmacro.drv}%
  [2019/12/03 v1.6 Let assignment for LaTeX macros (HO)]%
\documentclass{ltxdoc}
\usepackage{holtxdoc}[2011/11/22]
\begin{document}
  \DocInput{letltxmacro.dtx}%
\end{document}
%</driver>
% \fi
%
%
%
% \GetFileInfo{letltxmacro.drv}
%
% \title{The \xpackage{letltxmacro} package}
% \date{2019/12/03 v1.6}
% \author{Heiko Oberdiek\thanks
% {Please report any issues at \url{https://github.com/ho-tex/letltxmacro/issues}}}
%
% \maketitle
%
% \begin{abstract}
% \TeX's \cs{let} assignment does not work for \LaTeX\ macros
% with optional arguments or for macros that are defined
% as robust macros by \cs{DeclareRobustCommand}. This package
% defines \cs{LetLtxMacro} that also takes care of the involved
% internal macros.
% \end{abstract}
%
% \tableofcontents
%
% \section{Documentation}
%
% If someone wants to redefine a macro with using the old
% meaning, then one method is \TeX's command \cs{let}:
%\begin{quote}
%\begin{verbatim}
%\newcommand{\Macro}{\typeout{Test Macro}}
%\let\SavedMacro=\Macro
%\renewcommand{\Macro}{%
%  \typeout{Begin}%
%  \SavedMacro
%  \typeout{End}%
%}
%\end{verbatim}
%\end{quote}
% However, this method fails, if \cs{Macro} is defined
% by \cs{DeclareRobustCommand} and/or has an optional argument.
% In both cases \LaTeX\ defines an additional internal macro
% that is forgotten in the simple \cs{let} assignment of
% the example above.
%
% \begin{declcs}{LetLtxMacro} \M{new macro} \M{old macro}
% \end{declcs}
% Macro \cs{LetLtxMacro} behaves similar to \TeX's \cs{let}
% assignment, but it takes care of macros that are
% defined by \cs{DeclareRobustCommand} and/or have optional
% arguments. Example:
%\begin{quote}
%\begin{verbatim}
%\DeclareRobustCommand{\Macro}[1][default]{...}
%\LetLtxMacro{\SavedMacro}{\Macro}
%\end{verbatim}
%\end{quote}
% Then macro \cs{SavedMacro} only uses internal macro names
% that are derived from \cs{SavedMacro}'s macro name. Macro \cs{Macro}
% can now be redefined without affecting \cs{SavedMacro}.
%
% \begin{declcs}{GlobalLetLtxMacro} \M{new macro} \M{old macro}
% \end{declcs}
% Like \cs{LetLtxMacro}, but the \meta{new macro} is defined globally.
% Since version 2019/12/03~v1.4.
%
% \subsection{Supported macro definition commands}
%
% \begin{quote}
%   \begin{tabular}{@{}ll@{}}
%     \cs{newcommand}, \cs{renewcommand} & latex/base\\
%     \cs{newenvironment}, \cs{renewenvironment} & latex/base\\
%     \cs{DeclareRobustCommand}& latex/base\\
%     \cs{newrobustcmd}, \cs{renewrobustcmd} & etoolbox\\
%     \cs{robustify} & etoolbox 2008/06/22 v1.6\\
%   \end{tabular}
% \end{quote}
%
% \StopEventually{
% }
%
% \section{Implementation}
%
% \subsection{Show cases}
%
% \subsubsection{\xfile{letltxmacro-showcases.tex}}
%
%    \begin{macrocode}
%<*showcases>
\NeedsTeXFormat{LaTeX2e}
\makeatletter
%    \end{macrocode}
%    \begin{macro}{\Line}
%    The result is displayed by macro \cs{Line}. The percent symbol
%    at line start allows easy grepping and inserting into the DTX
%    file.
%    \begin{macrocode}
\newcommand*{\Line}[1]{%
  \typeout{\@percentchar#1}%
}
%    \end{macrocode}
%    \end{macro}
%    \begin{macrocode}
\newcommand*{\ShowCmdName}[1]{%
  \@ifundefined{#1}{}{%
    \Line{%
      \space\space(\expandafter\string\csname#1\endcsname) = %
      (\expandafter\meaning\csname#1\endcsname)%
    }%
  }%
}
\newcommand*{\ShowCmds}[1]{%
  \ShowCmdName{#1}%
  \ShowCmdName{#1 }%
  \ShowCmdName{\\#1}%
  \ShowCmdName{\\#1 }%
}
\let\\\@backslashchar
%    \end{macrocode}
%    \begin{macro}{\ShowDef}
%    \begin{macrocode}
\newcommand*{\ShowDef}[2]{%
  \begingroup
    \Line{}%
    \newcommand*{\DefString}{#2}%
    \@onelevel@sanitize\DefString
    \Line{\DefString}%
    #2%
    \ShowCmds{#1}%
  \endgroup
}
%    \end{macrocode}
%    \end{macro}
%    \begin{macrocode}
\typeout{}
\Line{* LaTeX definitions:}
\ShowDef{cmd}{%
  \newcommand{\cmd}[2][default]{}%
}
\ShowDef{cmd}{%
  \DeclareRobustCommand{\cmd}{}%
}
\ShowDef{cmd}{%
  \DeclareRobustCommand{\cmd}[2][default]{}%
}
\typeout{}
%    \end{macrocode}
% The minimal version of package \xpackage{etoolbox} is 2008/06/12 v1.6a
% because it fixes \cs{robustify}.
%    \begin{macrocode}
\RequirePackage{etoolbox}[2008/06/12]%
\Line{}
\Line{* etoolbox's robust definitions:}
\ShowDef{cmd}{%
  \newrobustcmd{\cmd}{}%
}
\ShowDef{cmd}{%
  \newrobustcmd{\cmd}[2][default]{}%
}
\Line{}
\Line{* etoolbox's \string\robustify:}
\ShowDef{cmd}{%
  \newcommand{\cmd}[2][default]{} %
  \robustify{\cmd}%
}
\ShowDef{cmd}{%
  \DeclareRobustCommand{\cmd}{} %
  \robustify{\cmd}%
}
\ShowDef{cmd}{%
  \DeclareRobustCommand{\cmd}[2][default]{} %
  \robustify{\cmd}%
}
\typeout{}
\@@end
%</showcases>
%    \end{macrocode}
%
% \subsubsection{Result}
%
% \begingroup
%   \makeatletter
%   \let\org@verbatim\@verbatim
%   \def\@verbatim{^^A
%     \org@verbatim
%     \catcode`\~=\active
%   }^^A
%   \let~\textvisiblespace
%\begin{verbatim}
%* LaTeX definitions:
%
%\newcommand {\cmd }[2][default]{}
%  (\cmd) = (macro:->\@protected@testopt \cmd \\cmd {default})
%  (\\cmd) = (\long macro:[#1]#2->)
%
%\DeclareRobustCommand {\cmd }{}
%  (\cmd) = (macro:->\protect \cmd~ )
%  (\cmd~) = (\long macro:->)
%
%\DeclareRobustCommand {\cmd }[2][default]{}
%  (\cmd) = (macro:->\protect \cmd~ )
%  (\cmd~) = (macro:->\@protected@testopt \cmd~ \\cmd~ {default})
%  (\\cmd~) = (\long macro:[#1]#2->)
%
%* etoolbox's robust definitions:
%
%\newrobustcmd {\cmd }{}
%  (\cmd) = (\protected\long macro:->)
%
%\newrobustcmd {\cmd }[2][default]{}
%  (\cmd) = (\protected macro:->\@testopt \\cmd {default})
%  (\\cmd) = (\long macro:[#1]#2->)
%
%* etoolbox's \robustify:
%
%\newcommand {\cmd }[2][default]{} \robustify {\cmd }
%  (\cmd) = (\protected macro:->\@protected@testopt \cmd \\cmd {default})
%  (\\cmd) = (\long macro:[#1]#2->)
%
%\DeclareRobustCommand {\cmd }{} \robustify {\cmd }
%  (\cmd) = (\protected macro:->)
%
%\DeclareRobustCommand {\cmd }[2][default]{} \robustify {\cmd }
%  (\cmd) = (\protected macro:->\@protected@testopt \cmd~ \\cmd~ {default})
%  (\cmd~) = (macro:->\@protected@testopt \cmd~ \\cmd~ {default})
%  (\\cmd~) = (\long macro:[#1]#2->)
%\end{verbatim}
% \endgroup
%
% \subsection{Package}
%
%    \begin{macrocode}
%<*package>
%    \end{macrocode}
%
% \subsubsection{Catcodes and identification}
%
%    \begin{macrocode}
\begingroup\catcode61\catcode48\catcode32=10\relax%
  \catcode13=5 % ^^M
  \endlinechar=13 %
  \catcode123=1 % {
  \catcode125=2 % }
  \catcode64=11 % @
  \def\x{\endgroup
    \expandafter\edef\csname llm@AtEnd\endcsname{%
      \endlinechar=\the\endlinechar\relax
      \catcode13=\the\catcode13\relax
      \catcode32=\the\catcode32\relax
      \catcode35=\the\catcode35\relax
      \catcode61=\the\catcode61\relax
      \catcode64=\the\catcode64\relax
      \catcode123=\the\catcode123\relax
      \catcode125=\the\catcode125\relax
    }%
  }%
\x\catcode61\catcode48\catcode32=10\relax%
\catcode13=5 % ^^M
\endlinechar=13 %
\catcode35=6 % #
\catcode64=11 % @
\catcode123=1 % {
\catcode125=2 % }
\def\TMP@EnsureCode#1#2{%
  \edef\llm@AtEnd{%
    \llm@AtEnd
    \catcode#1=\the\catcode#1\relax
  }%
  \catcode#1=#2\relax
}
\TMP@EnsureCode{40}{12}% (
\TMP@EnsureCode{41}{12}% )
\TMP@EnsureCode{42}{12}% *
\TMP@EnsureCode{45}{12}% -
\TMP@EnsureCode{46}{12}% .
\TMP@EnsureCode{47}{12}% /
\TMP@EnsureCode{58}{12}% :
\TMP@EnsureCode{62}{12}% >
\TMP@EnsureCode{91}{12}% [
\TMP@EnsureCode{93}{12}% ]
\edef\llm@AtEnd{%
  \llm@AtEnd
  \escapechar\the\escapechar\relax
  \noexpand\endinput
}
\escapechar=92 % `\\
%    \end{macrocode}
%
%    Package identification.
%    \begin{macrocode}
\NeedsTeXFormat{LaTeX2e}
\ProvidesPackage{letltxmacro}%
  [2019/12/03 v1.6 Let assignment for LaTeX macros (HO)]
%    \end{macrocode}
%
% \subsubsection{Main macros}
%
%    \begin{macro}{\LetLtxMacro}
%    \begin{macrocode}
\newcommand*{\LetLtxMacro}{%
  \llm@ModeLetLtxMacro{}%
}
%    \end{macrocode}
%    \end{macro}
%    \begin{macro}{\GlobalLetLtxMacro}
%    \begin{macrocode}
\newcommand*{\GlobalLetLtxMacro}{%
  \llm@ModeLetLtxMacro\global
}
%    \end{macrocode}
%    \end{macro}
%
%    \begin{macro}{\llm@ModeLetLtxMacro}
%    \begin{macrocode}
\newcommand*{\llm@ModeLetLtxMacro}[3]{%
  \edef\llm@escapechar{\the\escapechar}%
  \escapechar=-1 %
  \edef\reserved@a{%
    \noexpand\protect
    \expandafter\noexpand
    \csname\string#3 \endcsname
  }%
  \ifx\reserved@a#3\relax
    #1\edef#2{%
      \noexpand\protect
      \expandafter\noexpand
      \csname\string#2 \endcsname
    }%
    #1\expandafter\let
    \csname\string#2 \expandafter\endcsname
    \csname\string#3 \endcsname
    \expandafter\llm@LetLtxMacro
        \csname\string#2 \expandafter\endcsname
        \csname\string#3 \endcsname{#1}%
  \else
    \llm@LetLtxMacro{#2}{#3}{#1}%
  \fi
  \escapechar=\llm@escapechar\relax
}
%    \end{macrocode}
%    \end{macro}
%    \begin{macro}{\llm@LetLtxMacro}
%    \begin{macrocode}
\def\llm@LetLtxMacro#1#2#3{%
  \escapechar=92 %
  \expandafter\llm@CheckParams\meaning#2:->\@nil{%
    \begingroup
      \def\@protected@testopt{%
        \expandafter\@testopt\@gobble
      }%
      \def\@testopt##1##2{%
        \toks@={##2}%
      }%
      \let\llm@testopt\@empty
      \edef\x{%
        \noexpand\@protected@testopt
        \noexpand#2%
        \expandafter\noexpand\csname\string#2\endcsname
      }%
      \expandafter\expandafter\expandafter\def
      \expandafter\expandafter\expandafter\y
      \expandafter\expandafter\expandafter{%
        \expandafter\llm@CarThree#2{}{}{}\llm@nil
      }%
      \ifx\x\y
        #2%
        \def\llm@testopt{%
          \noexpand\@protected@testopt
          \noexpand#1%
        }%
      \else
        \edef\x{%
          \noexpand\@testopt
          \expandafter\noexpand
          \csname\string#2\endcsname
        }%
        \expandafter\expandafter\expandafter\def
        \expandafter\expandafter\expandafter\y
        \expandafter\expandafter\expandafter{%
          \expandafter\llm@CarTwo#2{}{}\llm@nil
        }%
        \ifx\x\y
          #2%
          \def\llm@testopt{%
            \noexpand\@testopt
          }%
        \fi
      \fi
      \ifx\llm@testopt\@empty
      \else
        \llm@protected\xdef\llm@GlobalTemp{%
          \llm@testopt
          \expandafter\noexpand
          \csname\string#1\endcsname
          {\the\toks@}%
        }%
      \fi
    \expandafter\endgroup\ifx\llm@testopt\@empty
      #3\let#1=#2\relax
    \else
      #3\let#1=\llm@GlobalTemp
      #3\expandafter\let
          \csname\string#1\expandafter\endcsname
          \csname\string#2\endcsname
    \fi
  }{%
    #3\let#1=#2\relax
  }%
}
%    \end{macrocode}
%    \end{macro}
%    \begin{macro}{\llm@CheckParams}
%    \begin{macrocode}
\def\llm@CheckParams#1:->#2\@nil{%
  \begingroup
    \def\x{#1}%
  \ifx\x\llm@macro
    \endgroup
    \def\llm@protected{}%
    \expandafter\@firstoftwo
  \else
    \ifx\x\llm@protectedmacro
      \endgroup
      \def\llm@protected{\protected}%
      \expandafter\expandafter\expandafter\@firstoftwo
    \else
      \endgroup
      \expandafter\expandafter\expandafter\@secondoftwo
    \fi
  \fi
}
%    \end{macrocode}
%    \end{macro}
%    \begin{macro}{\llm@macro}
%    \begin{macrocode}
\def\llm@macro{macro}
\@onelevel@sanitize\llm@macro
%    \end{macrocode}
%    \end{macro}
%    \begin{macro}{\llm@protectedmacro}
%    \begin{macrocode}
\def\llm@protectedmacro{\protected macro}
\@onelevel@sanitize\llm@protectedmacro
%    \end{macrocode}
%    \end{macro}
%    \begin{macro}{\llm@CarThree}
%    \begin{macrocode}
\def\llm@CarThree#1#2#3#4\llm@nil{#1#2#3}%
%    \end{macrocode}
%    \end{macro}
%    \begin{macro}{\llm@CarTwo}
%    \begin{macrocode}
\def\llm@CarTwo#1#2#3\llm@nil{#1#2}%
%    \end{macrocode}
%    \end{macro}
%
%    \begin{macrocode}
\llm@AtEnd%
%</package>
%    \end{macrocode}
% \section{Installation}
%
% \subsection{Download}
%
% \paragraph{Package.} This package is available on
% CTAN\footnote{\CTANpkg{letltxmacro}}:
% \begin{description}
% \item[\CTAN{macros/latex/contrib/letltxmacro/letltxmacro.dtx}] The source file.
% \item[\CTAN{macros/latex/contrib/letltxmacro/letltxmacro.pdf}] Documentation.
% \end{description}
%
%
% \paragraph{Bundle.} All the packages of the bundle `letltxmacro'
% are also available in a TDS compliant ZIP archive. There
% the packages are already unpacked and the documentation files
% are generated. The files and directories obey the TDS standard.
% \begin{description}
% \item[\CTANinstall{install/macros/latex/contrib/letltxmacro.tds.zip}]
% \end{description}
% \emph{TDS} refers to the standard ``A Directory Structure
% for \TeX\ Files'' (\CTANpkg{tds}). Directories
% with \xfile{texmf} in their name are usually organized this way.
%
% \subsection{Bundle installation}
%
% \paragraph{Unpacking.} Unpack the \xfile{letltxmacro.tds.zip} in the
% TDS tree (also known as \xfile{texmf} tree) of your choice.
% Example (linux):
% \begin{quote}
%   |unzip letltxmacro.tds.zip -d ~/texmf|
% \end{quote}
%
% \subsection{Package installation}
%
% \paragraph{Unpacking.} The \xfile{.dtx} file is a self-extracting
% \docstrip\ archive. The files are extracted by running the
% \xfile{.dtx} through \plainTeX:
% \begin{quote}
%   \verb|tex letltxmacro.dtx|
% \end{quote}
%
% \paragraph{TDS.} Now the different files must be moved into
% the different directories in your installation TDS tree
% (also known as \xfile{texmf} tree):
% \begin{quote}
% \def\t{^^A
% \begin{tabular}{@{}>{\ttfamily}l@{ $\rightarrow$ }>{\ttfamily}l@{}}
%   letltxmacro.sty & tex/latex/letltxmacro/letltxmacro.sty\\
%   letltxmacro.pdf & doc/latex/letltxmacro/letltxmacro.pdf\\
%   letltxmacro-showcases.tex & doc/latex/letltxmacro/letltxmacro-showcases.tex\\
%   letltxmacro.dtx & source/latex/letltxmacro/letltxmacro.dtx\\
% \end{tabular}^^A
% }^^A
% \sbox0{\t}^^A
% \ifdim\wd0>\linewidth
%   \begingroup
%     \advance\linewidth by\leftmargin
%     \advance\linewidth by\rightmargin
%   \edef\x{\endgroup
%     \def\noexpand\lw{\the\linewidth}^^A
%   }\x
%   \def\lwbox{^^A
%     \leavevmode
%     \hbox to \linewidth{^^A
%       \kern-\leftmargin\relax
%       \hss
%       \usebox0
%       \hss
%       \kern-\rightmargin\relax
%     }^^A
%   }^^A
%   \ifdim\wd0>\lw
%     \sbox0{\small\t}^^A
%     \ifdim\wd0>\linewidth
%       \ifdim\wd0>\lw
%         \sbox0{\footnotesize\t}^^A
%         \ifdim\wd0>\linewidth
%           \ifdim\wd0>\lw
%             \sbox0{\scriptsize\t}^^A
%             \ifdim\wd0>\linewidth
%               \ifdim\wd0>\lw
%                 \sbox0{\tiny\t}^^A
%                 \ifdim\wd0>\linewidth
%                   \lwbox
%                 \else
%                   \usebox0
%                 \fi
%               \else
%                 \lwbox
%               \fi
%             \else
%               \usebox0
%             \fi
%           \else
%             \lwbox
%           \fi
%         \else
%           \usebox0
%         \fi
%       \else
%         \lwbox
%       \fi
%     \else
%       \usebox0
%     \fi
%   \else
%     \lwbox
%   \fi
% \else
%   \usebox0
% \fi
% \end{quote}
% If you have a \xfile{docstrip.cfg} that configures and enables \docstrip's
% TDS installing feature, then some files can already be in the right
% place, see the documentation of \docstrip.
%
% \subsection{Refresh file name databases}
%
% If your \TeX~distribution
% (\TeX\,Live, \mikTeX, \dots) relies on file name databases, you must refresh
% these. For example, \TeX\,Live\ users run \verb|texhash| or
% \verb|mktexlsr|.
%
% \subsection{Some details for the interested}
%
% \paragraph{Unpacking with \LaTeX.}
% The \xfile{.dtx} chooses its action depending on the format:
% \begin{description}
% \item[\plainTeX:] Run \docstrip\ and extract the files.
% \item[\LaTeX:] Generate the documentation.
% \end{description}
% If you insist on using \LaTeX\ for \docstrip\ (really,
% \docstrip\ does not need \LaTeX), then inform the autodetect routine
% about your intention:
% \begin{quote}
%   \verb|latex \let\install=y\input{letltxmacro.dtx}|
% \end{quote}
% Do not forget to quote the argument according to the demands
% of your shell.
%
% \paragraph{Generating the documentation.}
% You can use both the \xfile{.dtx} or the \xfile{.drv} to generate
% the documentation. The process can be configured by the
% configuration file \xfile{ltxdoc.cfg}. For instance, put this
% line into this file, if you want to have A4 as paper format:
% \begin{quote}
%   \verb|\PassOptionsToClass{a4paper}{article}|
% \end{quote}
% An example follows how to generate the
% documentation with pdf\LaTeX:
% \begin{quote}
%\begin{verbatim}
%pdflatex letltxmacro.dtx
%makeindex -s gind.ist letltxmacro.idx
%pdflatex letltxmacro.dtx
%makeindex -s gind.ist letltxmacro.idx
%pdflatex letltxmacro.dtx
%\end{verbatim}
% \end{quote}
%
% \begin{History}
%   \begin{Version}{2008/06/09 v1.0}
%   \item
%     First version.
%   \end{Version}
%   \begin{Version}{2008/06/12 v1.1}
%   \item
%     Support for \xpackage{etoolbox}'s \cs{newrobustcmd} added.
%   \end{Version}
%   \begin{Version}{2008/06/13 v1.2}
%   \item
%     Support for \xpackage{etoolbox}'s \cs{robustify} added.
%   \end{Version}
%   \begin{Version}{2008/06/24 v1.3}
%   \item
%     Test file adapted for etoolbox 2008/06/22 v1.6.
%   \end{Version}
%   \begin{Version}{2010/09/02 v1.4}
%   \item
%     \cs{GlobalLetLtxMacro} added.
%   \end{Version}
%   \begin{Version}{2016/05/16 v1.5}
%   \item
%     Documentation updates.
%   \end{Version}
%   \begin{Version}{2019/12/03 v1.6}
%   \item
%     Documentation updates.
%   \end{Version}
% \end{History}
%
% \PrintIndex
%
% \Finale
\endinput
|
% \end{quote}
% Do not forget to quote the argument according to the demands
% of your shell.
%
% \paragraph{Generating the documentation.}
% You can use both the \xfile{.dtx} or the \xfile{.drv} to generate
% the documentation. The process can be configured by the
% configuration file \xfile{ltxdoc.cfg}. For instance, put this
% line into this file, if you want to have A4 as paper format:
% \begin{quote}
%   \verb|\PassOptionsToClass{a4paper}{article}|
% \end{quote}
% An example follows how to generate the
% documentation with pdf\LaTeX:
% \begin{quote}
%\begin{verbatim}
%pdflatex letltxmacro.dtx
%makeindex -s gind.ist letltxmacro.idx
%pdflatex letltxmacro.dtx
%makeindex -s gind.ist letltxmacro.idx
%pdflatex letltxmacro.dtx
%\end{verbatim}
% \end{quote}
%
% \begin{History}
%   \begin{Version}{2008/06/09 v1.0}
%   \item
%     First version.
%   \end{Version}
%   \begin{Version}{2008/06/12 v1.1}
%   \item
%     Support for \xpackage{etoolbox}'s \cs{newrobustcmd} added.
%   \end{Version}
%   \begin{Version}{2008/06/13 v1.2}
%   \item
%     Support for \xpackage{etoolbox}'s \cs{robustify} added.
%   \end{Version}
%   \begin{Version}{2008/06/24 v1.3}
%   \item
%     Test file adapted for etoolbox 2008/06/22 v1.6.
%   \end{Version}
%   \begin{Version}{2010/09/02 v1.4}
%   \item
%     \cs{GlobalLetLtxMacro} added.
%   \end{Version}
%   \begin{Version}{2016/05/16 v1.5}
%   \item
%     Documentation updates.
%   \end{Version}
%   \begin{Version}{2019/12/03 v1.6}
%   \item
%     Documentation updates.
%   \end{Version}
% \end{History}
%
% \PrintIndex
%
% \Finale
\endinput

%        (quote the arguments according to the demands of your shell)
%
% Documentation:
%    (a) If letltxmacro.drv is present:
%           latex letltxmacro.drv
%    (b) Without letltxmacro.drv:
%           latex letltxmacro.dtx; ...
%    The class ltxdoc loads the configuration file ltxdoc.cfg
%    if available. Here you can specify further options, e.g.
%    use A4 as paper format:
%       \PassOptionsToClass{a4paper}{article}
%
%    Programm calls to get the documentation (example):
%       pdflatex letltxmacro.dtx
%       makeindex -s gind.ist letltxmacro.idx
%       pdflatex letltxmacro.dtx
%       makeindex -s gind.ist letltxmacro.idx
%       pdflatex letltxmacro.dtx
%
% Installation:
%    TDS:tex/latex/letltxmacro/letltxmacro.sty
%    TDS:doc/latex/letltxmacro/letltxmacro.pdf
%    TDS:doc/latex/letltxmacro/letltxmacro-showcases.tex
%    TDS:source/latex/letltxmacro/letltxmacro.dtx
%
%<*ignore>
\begingroup
  \catcode123=1 %
  \catcode125=2 %
  \def\x{LaTeX2e}%
\expandafter\endgroup
\ifcase 0\ifx\install y1\fi\expandafter
         \ifx\csname processbatchFile\endcsname\relax\else1\fi
         \ifx\fmtname\x\else 1\fi\relax
\else\csname fi\endcsname
%</ignore>
%<*install>
\input docstrip.tex
\Msg{************************************************************************}
\Msg{* Installation}
\Msg{* Package: letltxmacro 2019/12/03 v1.6 Let assignment for LaTeX macros (HO)}
\Msg{************************************************************************}

\keepsilent
\askforoverwritefalse

\let\MetaPrefix\relax
\preamble

This is a generated file.

Project: letltxmacro
Version: 2019/12/03 v1.6

Copyright (C)
   2008, 2010 Heiko Oberdiek
   2016-2019 Oberdiek Package Support Group

This work may be distributed and/or modified under the
conditions of the LaTeX Project Public License, either
version 1.3c of this license or (at your option) any later
version. This version of this license is in
   https://www.latex-project.org/lppl/lppl-1-3c.txt
and the latest version of this license is in
   https://www.latex-project.org/lppl.txt
and version 1.3 or later is part of all distributions of
LaTeX version 2005/12/01 or later.

This work has the LPPL maintenance status "maintained".

The Current Maintainers of this work are
Heiko Oberdiek and the Oberdiek Package Support Group
https://github.com/ho-tex/letltxmacro/issues


This work consists of the main source file letltxmacro.dtx
and the derived files
   letltxmacro.sty, letltxmacro.pdf, letltxmacro.ins, letltxmacro.drv,
   letltxmacro-showcases.tex, letltxmacro-test1.tex,
   letltxmacro-test2.tex.

\endpreamble
\let\MetaPrefix\DoubleperCent

\generate{%
  \file{letltxmacro.ins}{\from{letltxmacro.dtx}{install}}%
  \file{letltxmacro.drv}{\from{letltxmacro.dtx}{driver}}%
  \usedir{tex/latex/letltxmacro}%
  \file{letltxmacro.sty}{\from{letltxmacro.dtx}{package}}%
  \usedir{doc/latex/letltxmacro}%
  \file{letltxmacro-showcases.tex}{\from{letltxmacro.dtx}{showcases}}%
%  \usedir{doc/latex/letltxmacro/test}%
%  \file{letltxmacro-test1.tex}{\from{letltxmacro.dtx}{test1}}%
%  \file{letltxmacro-test2.tex}{\from{letltxmacro.dtx}{test2}}%
}

\catcode32=13\relax% active space
\let =\space%
\Msg{************************************************************************}
\Msg{*}
\Msg{* To finish the installation you have to move the following}
\Msg{* file into a directory searched by TeX:}
\Msg{*}
\Msg{*     letltxmacro.sty}
\Msg{*}
\Msg{* To produce the documentation run the file `letltxmacro.drv'}
\Msg{* through LaTeX.}
\Msg{*}
\Msg{* Happy TeXing!}
\Msg{*}
\Msg{************************************************************************}

\endbatchfile
%</install>
%<*ignore>
\fi
%</ignore>
%<*driver>
\NeedsTeXFormat{LaTeX2e}
\ProvidesFile{letltxmacro.drv}%
  [2019/12/03 v1.6 Let assignment for LaTeX macros (HO)]%
\documentclass{ltxdoc}
\usepackage{holtxdoc}[2011/11/22]
\begin{document}
  \DocInput{letltxmacro.dtx}%
\end{document}
%</driver>
% \fi
%
%
%
% \GetFileInfo{letltxmacro.drv}
%
% \title{The \xpackage{letltxmacro} package}
% \date{2019/12/03 v1.6}
% \author{Heiko Oberdiek\thanks
% {Please report any issues at \url{https://github.com/ho-tex/letltxmacro/issues}}}
%
% \maketitle
%
% \begin{abstract}
% \TeX's \cs{let} assignment does not work for \LaTeX\ macros
% with optional arguments or for macros that are defined
% as robust macros by \cs{DeclareRobustCommand}. This package
% defines \cs{LetLtxMacro} that also takes care of the involved
% internal macros.
% \end{abstract}
%
% \tableofcontents
%
% \section{Documentation}
%
% If someone wants to redefine a macro with using the old
% meaning, then one method is \TeX's command \cs{let}:
%\begin{quote}
%\begin{verbatim}
%\newcommand{\Macro}{\typeout{Test Macro}}
%\let\SavedMacro=\Macro
%\renewcommand{\Macro}{%
%  \typeout{Begin}%
%  \SavedMacro
%  \typeout{End}%
%}
%\end{verbatim}
%\end{quote}
% However, this method fails, if \cs{Macro} is defined
% by \cs{DeclareRobustCommand} and/or has an optional argument.
% In both cases \LaTeX\ defines an additional internal macro
% that is forgotten in the simple \cs{let} assignment of
% the example above.
%
% \begin{declcs}{LetLtxMacro} \M{new macro} \M{old macro}
% \end{declcs}
% Macro \cs{LetLtxMacro} behaves similar to \TeX's \cs{let}
% assignment, but it takes care of macros that are
% defined by \cs{DeclareRobustCommand} and/or have optional
% arguments. Example:
%\begin{quote}
%\begin{verbatim}
%\DeclareRobustCommand{\Macro}[1][default]{...}
%\LetLtxMacro{\SavedMacro}{\Macro}
%\end{verbatim}
%\end{quote}
% Then macro \cs{SavedMacro} only uses internal macro names
% that are derived from \cs{SavedMacro}'s macro name. Macro \cs{Macro}
% can now be redefined without affecting \cs{SavedMacro}.
%
% \begin{declcs}{GlobalLetLtxMacro} \M{new macro} \M{old macro}
% \end{declcs}
% Like \cs{LetLtxMacro}, but the \meta{new macro} is defined globally.
% Since version 2019/12/03~v1.4.
%
% \subsection{Supported macro definition commands}
%
% \begin{quote}
%   \begin{tabular}{@{}ll@{}}
%     \cs{newcommand}, \cs{renewcommand} & latex/base\\
%     \cs{newenvironment}, \cs{renewenvironment} & latex/base\\
%     \cs{DeclareRobustCommand}& latex/base\\
%     \cs{newrobustcmd}, \cs{renewrobustcmd} & etoolbox\\
%     \cs{robustify} & etoolbox 2008/06/22 v1.6\\
%   \end{tabular}
% \end{quote}
%
% \StopEventually{
% }
%
% \section{Implementation}
%
% \subsection{Show cases}
%
% \subsubsection{\xfile{letltxmacro-showcases.tex}}
%
%    \begin{macrocode}
%<*showcases>
\NeedsTeXFormat{LaTeX2e}
\makeatletter
%    \end{macrocode}
%    \begin{macro}{\Line}
%    The result is displayed by macro \cs{Line}. The percent symbol
%    at line start allows easy grepping and inserting into the DTX
%    file.
%    \begin{macrocode}
\newcommand*{\Line}[1]{%
  \typeout{\@percentchar#1}%
}
%    \end{macrocode}
%    \end{macro}
%    \begin{macrocode}
\newcommand*{\ShowCmdName}[1]{%
  \@ifundefined{#1}{}{%
    \Line{%
      \space\space(\expandafter\string\csname#1\endcsname) = %
      (\expandafter\meaning\csname#1\endcsname)%
    }%
  }%
}
\newcommand*{\ShowCmds}[1]{%
  \ShowCmdName{#1}%
  \ShowCmdName{#1 }%
  \ShowCmdName{\\#1}%
  \ShowCmdName{\\#1 }%
}
\let\\\@backslashchar
%    \end{macrocode}
%    \begin{macro}{\ShowDef}
%    \begin{macrocode}
\newcommand*{\ShowDef}[2]{%
  \begingroup
    \Line{}%
    \newcommand*{\DefString}{#2}%
    \@onelevel@sanitize\DefString
    \Line{\DefString}%
    #2%
    \ShowCmds{#1}%
  \endgroup
}
%    \end{macrocode}
%    \end{macro}
%    \begin{macrocode}
\typeout{}
\Line{* LaTeX definitions:}
\ShowDef{cmd}{%
  \newcommand{\cmd}[2][default]{}%
}
\ShowDef{cmd}{%
  \DeclareRobustCommand{\cmd}{}%
}
\ShowDef{cmd}{%
  \DeclareRobustCommand{\cmd}[2][default]{}%
}
\typeout{}
%    \end{macrocode}
% The minimal version of package \xpackage{etoolbox} is 2008/06/12 v1.6a
% because it fixes \cs{robustify}.
%    \begin{macrocode}
\RequirePackage{etoolbox}[2008/06/12]%
\Line{}
\Line{* etoolbox's robust definitions:}
\ShowDef{cmd}{%
  \newrobustcmd{\cmd}{}%
}
\ShowDef{cmd}{%
  \newrobustcmd{\cmd}[2][default]{}%
}
\Line{}
\Line{* etoolbox's \string\robustify:}
\ShowDef{cmd}{%
  \newcommand{\cmd}[2][default]{} %
  \robustify{\cmd}%
}
\ShowDef{cmd}{%
  \DeclareRobustCommand{\cmd}{} %
  \robustify{\cmd}%
}
\ShowDef{cmd}{%
  \DeclareRobustCommand{\cmd}[2][default]{} %
  \robustify{\cmd}%
}
\typeout{}
\@@end
%</showcases>
%    \end{macrocode}
%
% \subsubsection{Result}
%
% \begingroup
%   \makeatletter
%   \let\org@verbatim\@verbatim
%   \def\@verbatim{^^A
%     \org@verbatim
%     \catcode`\~=\active
%   }^^A
%   \let~\textvisiblespace
%\begin{verbatim}
%* LaTeX definitions:
%
%\newcommand {\cmd }[2][default]{}
%  (\cmd) = (macro:->\@protected@testopt \cmd \\cmd {default})
%  (\\cmd) = (\long macro:[#1]#2->)
%
%\DeclareRobustCommand {\cmd }{}
%  (\cmd) = (macro:->\protect \cmd~ )
%  (\cmd~) = (\long macro:->)
%
%\DeclareRobustCommand {\cmd }[2][default]{}
%  (\cmd) = (macro:->\protect \cmd~ )
%  (\cmd~) = (macro:->\@protected@testopt \cmd~ \\cmd~ {default})
%  (\\cmd~) = (\long macro:[#1]#2->)
%
%* etoolbox's robust definitions:
%
%\newrobustcmd {\cmd }{}
%  (\cmd) = (\protected\long macro:->)
%
%\newrobustcmd {\cmd }[2][default]{}
%  (\cmd) = (\protected macro:->\@testopt \\cmd {default})
%  (\\cmd) = (\long macro:[#1]#2->)
%
%* etoolbox's \robustify:
%
%\newcommand {\cmd }[2][default]{} \robustify {\cmd }
%  (\cmd) = (\protected macro:->\@protected@testopt \cmd \\cmd {default})
%  (\\cmd) = (\long macro:[#1]#2->)
%
%\DeclareRobustCommand {\cmd }{} \robustify {\cmd }
%  (\cmd) = (\protected macro:->)
%
%\DeclareRobustCommand {\cmd }[2][default]{} \robustify {\cmd }
%  (\cmd) = (\protected macro:->\@protected@testopt \cmd~ \\cmd~ {default})
%  (\cmd~) = (macro:->\@protected@testopt \cmd~ \\cmd~ {default})
%  (\\cmd~) = (\long macro:[#1]#2->)
%\end{verbatim}
% \endgroup
%
% \subsection{Package}
%
%    \begin{macrocode}
%<*package>
%    \end{macrocode}
%
% \subsubsection{Catcodes and identification}
%
%    \begin{macrocode}
\begingroup\catcode61\catcode48\catcode32=10\relax%
  \catcode13=5 % ^^M
  \endlinechar=13 %
  \catcode123=1 % {
  \catcode125=2 % }
  \catcode64=11 % @
  \def\x{\endgroup
    \expandafter\edef\csname llm@AtEnd\endcsname{%
      \endlinechar=\the\endlinechar\relax
      \catcode13=\the\catcode13\relax
      \catcode32=\the\catcode32\relax
      \catcode35=\the\catcode35\relax
      \catcode61=\the\catcode61\relax
      \catcode64=\the\catcode64\relax
      \catcode123=\the\catcode123\relax
      \catcode125=\the\catcode125\relax
    }%
  }%
\x\catcode61\catcode48\catcode32=10\relax%
\catcode13=5 % ^^M
\endlinechar=13 %
\catcode35=6 % #
\catcode64=11 % @
\catcode123=1 % {
\catcode125=2 % }
\def\TMP@EnsureCode#1#2{%
  \edef\llm@AtEnd{%
    \llm@AtEnd
    \catcode#1=\the\catcode#1\relax
  }%
  \catcode#1=#2\relax
}
\TMP@EnsureCode{40}{12}% (
\TMP@EnsureCode{41}{12}% )
\TMP@EnsureCode{42}{12}% *
\TMP@EnsureCode{45}{12}% -
\TMP@EnsureCode{46}{12}% .
\TMP@EnsureCode{47}{12}% /
\TMP@EnsureCode{58}{12}% :
\TMP@EnsureCode{62}{12}% >
\TMP@EnsureCode{91}{12}% [
\TMP@EnsureCode{93}{12}% ]
\edef\llm@AtEnd{%
  \llm@AtEnd
  \escapechar\the\escapechar\relax
  \noexpand\endinput
}
\escapechar=92 % `\\
%    \end{macrocode}
%
%    Package identification.
%    \begin{macrocode}
\NeedsTeXFormat{LaTeX2e}
\ProvidesPackage{letltxmacro}%
  [2019/12/03 v1.6 Let assignment for LaTeX macros (HO)]
%    \end{macrocode}
%
% \subsubsection{Main macros}
%
%    \begin{macro}{\LetLtxMacro}
%    \begin{macrocode}
\newcommand*{\LetLtxMacro}{%
  \llm@ModeLetLtxMacro{}%
}
%    \end{macrocode}
%    \end{macro}
%    \begin{macro}{\GlobalLetLtxMacro}
%    \begin{macrocode}
\newcommand*{\GlobalLetLtxMacro}{%
  \llm@ModeLetLtxMacro\global
}
%    \end{macrocode}
%    \end{macro}
%
%    \begin{macro}{\llm@ModeLetLtxMacro}
%    \begin{macrocode}
\newcommand*{\llm@ModeLetLtxMacro}[3]{%
  \edef\llm@escapechar{\the\escapechar}%
  \escapechar=-1 %
  \edef\reserved@a{%
    \noexpand\protect
    \expandafter\noexpand
    \csname\string#3 \endcsname
  }%
  \ifx\reserved@a#3\relax
    #1\edef#2{%
      \noexpand\protect
      \expandafter\noexpand
      \csname\string#2 \endcsname
    }%
    #1\expandafter\let
    \csname\string#2 \expandafter\endcsname
    \csname\string#3 \endcsname
    \expandafter\llm@LetLtxMacro
        \csname\string#2 \expandafter\endcsname
        \csname\string#3 \endcsname{#1}%
  \else
    \llm@LetLtxMacro{#2}{#3}{#1}%
  \fi
  \escapechar=\llm@escapechar\relax
}
%    \end{macrocode}
%    \end{macro}
%    \begin{macro}{\llm@LetLtxMacro}
%    \begin{macrocode}
\def\llm@LetLtxMacro#1#2#3{%
  \escapechar=92 %
  \expandafter\llm@CheckParams\meaning#2:->\@nil{%
    \begingroup
      \def\@protected@testopt{%
        \expandafter\@testopt\@gobble
      }%
      \def\@testopt##1##2{%
        \toks@={##2}%
      }%
      \let\llm@testopt\@empty
      \edef\x{%
        \noexpand\@protected@testopt
        \noexpand#2%
        \expandafter\noexpand\csname\string#2\endcsname
      }%
      \expandafter\expandafter\expandafter\def
      \expandafter\expandafter\expandafter\y
      \expandafter\expandafter\expandafter{%
        \expandafter\llm@CarThree#2{}{}{}\llm@nil
      }%
      \ifx\x\y
        #2%
        \def\llm@testopt{%
          \noexpand\@protected@testopt
          \noexpand#1%
        }%
      \else
        \edef\x{%
          \noexpand\@testopt
          \expandafter\noexpand
          \csname\string#2\endcsname
        }%
        \expandafter\expandafter\expandafter\def
        \expandafter\expandafter\expandafter\y
        \expandafter\expandafter\expandafter{%
          \expandafter\llm@CarTwo#2{}{}\llm@nil
        }%
        \ifx\x\y
          #2%
          \def\llm@testopt{%
            \noexpand\@testopt
          }%
        \fi
      \fi
      \ifx\llm@testopt\@empty
      \else
        \llm@protected\xdef\llm@GlobalTemp{%
          \llm@testopt
          \expandafter\noexpand
          \csname\string#1\endcsname
          {\the\toks@}%
        }%
      \fi
    \expandafter\endgroup\ifx\llm@testopt\@empty
      #3\let#1=#2\relax
    \else
      #3\let#1=\llm@GlobalTemp
      #3\expandafter\let
          \csname\string#1\expandafter\endcsname
          \csname\string#2\endcsname
    \fi
  }{%
    #3\let#1=#2\relax
  }%
}
%    \end{macrocode}
%    \end{macro}
%    \begin{macro}{\llm@CheckParams}
%    \begin{macrocode}
\def\llm@CheckParams#1:->#2\@nil{%
  \begingroup
    \def\x{#1}%
  \ifx\x\llm@macro
    \endgroup
    \def\llm@protected{}%
    \expandafter\@firstoftwo
  \else
    \ifx\x\llm@protectedmacro
      \endgroup
      \def\llm@protected{\protected}%
      \expandafter\expandafter\expandafter\@firstoftwo
    \else
      \endgroup
      \expandafter\expandafter\expandafter\@secondoftwo
    \fi
  \fi
}
%    \end{macrocode}
%    \end{macro}
%    \begin{macro}{\llm@macro}
%    \begin{macrocode}
\def\llm@macro{macro}
\@onelevel@sanitize\llm@macro
%    \end{macrocode}
%    \end{macro}
%    \begin{macro}{\llm@protectedmacro}
%    \begin{macrocode}
\def\llm@protectedmacro{\protected macro}
\@onelevel@sanitize\llm@protectedmacro
%    \end{macrocode}
%    \end{macro}
%    \begin{macro}{\llm@CarThree}
%    \begin{macrocode}
\def\llm@CarThree#1#2#3#4\llm@nil{#1#2#3}%
%    \end{macrocode}
%    \end{macro}
%    \begin{macro}{\llm@CarTwo}
%    \begin{macrocode}
\def\llm@CarTwo#1#2#3\llm@nil{#1#2}%
%    \end{macrocode}
%    \end{macro}
%
%    \begin{macrocode}
\llm@AtEnd%
%</package>
%    \end{macrocode}
% \section{Installation}
%
% \subsection{Download}
%
% \paragraph{Package.} This package is available on
% CTAN\footnote{\CTANpkg{letltxmacro}}:
% \begin{description}
% \item[\CTAN{macros/latex/contrib/letltxmacro/letltxmacro.dtx}] The source file.
% \item[\CTAN{macros/latex/contrib/letltxmacro/letltxmacro.pdf}] Documentation.
% \end{description}
%
%
% \paragraph{Bundle.} All the packages of the bundle `letltxmacro'
% are also available in a TDS compliant ZIP archive. There
% the packages are already unpacked and the documentation files
% are generated. The files and directories obey the TDS standard.
% \begin{description}
% \item[\CTANinstall{install/macros/latex/contrib/letltxmacro.tds.zip}]
% \end{description}
% \emph{TDS} refers to the standard ``A Directory Structure
% for \TeX\ Files'' (\CTANpkg{tds}). Directories
% with \xfile{texmf} in their name are usually organized this way.
%
% \subsection{Bundle installation}
%
% \paragraph{Unpacking.} Unpack the \xfile{letltxmacro.tds.zip} in the
% TDS tree (also known as \xfile{texmf} tree) of your choice.
% Example (linux):
% \begin{quote}
%   |unzip letltxmacro.tds.zip -d ~/texmf|
% \end{quote}
%
% \subsection{Package installation}
%
% \paragraph{Unpacking.} The \xfile{.dtx} file is a self-extracting
% \docstrip\ archive. The files are extracted by running the
% \xfile{.dtx} through \plainTeX:
% \begin{quote}
%   \verb|tex letltxmacro.dtx|
% \end{quote}
%
% \paragraph{TDS.} Now the different files must be moved into
% the different directories in your installation TDS tree
% (also known as \xfile{texmf} tree):
% \begin{quote}
% \def\t{^^A
% \begin{tabular}{@{}>{\ttfamily}l@{ $\rightarrow$ }>{\ttfamily}l@{}}
%   letltxmacro.sty & tex/latex/letltxmacro/letltxmacro.sty\\
%   letltxmacro.pdf & doc/latex/letltxmacro/letltxmacro.pdf\\
%   letltxmacro-showcases.tex & doc/latex/letltxmacro/letltxmacro-showcases.tex\\
%   letltxmacro.dtx & source/latex/letltxmacro/letltxmacro.dtx\\
% \end{tabular}^^A
% }^^A
% \sbox0{\t}^^A
% \ifdim\wd0>\linewidth
%   \begingroup
%     \advance\linewidth by\leftmargin
%     \advance\linewidth by\rightmargin
%   \edef\x{\endgroup
%     \def\noexpand\lw{\the\linewidth}^^A
%   }\x
%   \def\lwbox{^^A
%     \leavevmode
%     \hbox to \linewidth{^^A
%       \kern-\leftmargin\relax
%       \hss
%       \usebox0
%       \hss
%       \kern-\rightmargin\relax
%     }^^A
%   }^^A
%   \ifdim\wd0>\lw
%     \sbox0{\small\t}^^A
%     \ifdim\wd0>\linewidth
%       \ifdim\wd0>\lw
%         \sbox0{\footnotesize\t}^^A
%         \ifdim\wd0>\linewidth
%           \ifdim\wd0>\lw
%             \sbox0{\scriptsize\t}^^A
%             \ifdim\wd0>\linewidth
%               \ifdim\wd0>\lw
%                 \sbox0{\tiny\t}^^A
%                 \ifdim\wd0>\linewidth
%                   \lwbox
%                 \else
%                   \usebox0
%                 \fi
%               \else
%                 \lwbox
%               \fi
%             \else
%               \usebox0
%             \fi
%           \else
%             \lwbox
%           \fi
%         \else
%           \usebox0
%         \fi
%       \else
%         \lwbox
%       \fi
%     \else
%       \usebox0
%     \fi
%   \else
%     \lwbox
%   \fi
% \else
%   \usebox0
% \fi
% \end{quote}
% If you have a \xfile{docstrip.cfg} that configures and enables \docstrip's
% TDS installing feature, then some files can already be in the right
% place, see the documentation of \docstrip.
%
% \subsection{Refresh file name databases}
%
% If your \TeX~distribution
% (\TeX\,Live, \mikTeX, \dots) relies on file name databases, you must refresh
% these. For example, \TeX\,Live\ users run \verb|texhash| or
% \verb|mktexlsr|.
%
% \subsection{Some details for the interested}
%
% \paragraph{Unpacking with \LaTeX.}
% The \xfile{.dtx} chooses its action depending on the format:
% \begin{description}
% \item[\plainTeX:] Run \docstrip\ and extract the files.
% \item[\LaTeX:] Generate the documentation.
% \end{description}
% If you insist on using \LaTeX\ for \docstrip\ (really,
% \docstrip\ does not need \LaTeX), then inform the autodetect routine
% about your intention:
% \begin{quote}
%   \verb|latex \let\install=y% \iffalse meta-comment
%
% File: letltxmacro.dtx
% Version: 2019/12/03 v1.6
% Info: Let assignment for LaTeX macros
%
% Copyright (C)
%    2008, 2010 Heiko Oberdiek
%    2016-2019 Oberdiek Package Support Group
%    https://github.com/ho-tex/letltxmacro/issues
%
% This work may be distributed and/or modified under the
% conditions of the LaTeX Project Public License, either
% version 1.3c of this license or (at your option) any later
% version. This version of this license is in
%    https://www.latex-project.org/lppl/lppl-1-3c.txt
% and the latest version of this license is in
%    https://www.latex-project.org/lppl.txt
% and version 1.3 or later is part of all distributions of
% LaTeX version 2005/12/01 or later.
%
% This work has the LPPL maintenance status "maintained".
%
% The Current Maintainers of this work are
% Heiko Oberdiek and the Oberdiek Package Support Group
% https://github.com/ho-tex/letltxmacro/issues
%
% This work consists of the main source file letltxmacro.dtx
% and the derived files
%    letltxmacro.sty, letltxmacro.pdf, letltxmacro.ins, letltxmacro.drv,
%    letltxmacro-showcases.tex, letltxmacro-test1.tex,
%    letltxmacro-test2.tex.
%
% Distribution:
%    CTAN:macros/latex/contrib/letltxmacro/letltxmacro.dtx
%    CTAN:macros/latex/contrib/letltxmacro/letltxmacro.pdf
%
% Unpacking:
%    (a) If letltxmacro.ins is present:
%           tex letltxmacro.ins
%    (b) Without letltxmacro.ins:
%           tex letltxmacro.dtx
%    (c) If you insist on using LaTeX
%           latex \let\install=y% \iffalse meta-comment
%
% File: letltxmacro.dtx
% Version: 2019/12/03 v1.6
% Info: Let assignment for LaTeX macros
%
% Copyright (C)
%    2008, 2010 Heiko Oberdiek
%    2016-2019 Oberdiek Package Support Group
%    https://github.com/ho-tex/letltxmacro/issues
%
% This work may be distributed and/or modified under the
% conditions of the LaTeX Project Public License, either
% version 1.3c of this license or (at your option) any later
% version. This version of this license is in
%    https://www.latex-project.org/lppl/lppl-1-3c.txt
% and the latest version of this license is in
%    https://www.latex-project.org/lppl.txt
% and version 1.3 or later is part of all distributions of
% LaTeX version 2005/12/01 or later.
%
% This work has the LPPL maintenance status "maintained".
%
% The Current Maintainers of this work are
% Heiko Oberdiek and the Oberdiek Package Support Group
% https://github.com/ho-tex/letltxmacro/issues
%
% This work consists of the main source file letltxmacro.dtx
% and the derived files
%    letltxmacro.sty, letltxmacro.pdf, letltxmacro.ins, letltxmacro.drv,
%    letltxmacro-showcases.tex, letltxmacro-test1.tex,
%    letltxmacro-test2.tex.
%
% Distribution:
%    CTAN:macros/latex/contrib/letltxmacro/letltxmacro.dtx
%    CTAN:macros/latex/contrib/letltxmacro/letltxmacro.pdf
%
% Unpacking:
%    (a) If letltxmacro.ins is present:
%           tex letltxmacro.ins
%    (b) Without letltxmacro.ins:
%           tex letltxmacro.dtx
%    (c) If you insist on using LaTeX
%           latex \let\install=y\input{letltxmacro.dtx}
%        (quote the arguments according to the demands of your shell)
%
% Documentation:
%    (a) If letltxmacro.drv is present:
%           latex letltxmacro.drv
%    (b) Without letltxmacro.drv:
%           latex letltxmacro.dtx; ...
%    The class ltxdoc loads the configuration file ltxdoc.cfg
%    if available. Here you can specify further options, e.g.
%    use A4 as paper format:
%       \PassOptionsToClass{a4paper}{article}
%
%    Programm calls to get the documentation (example):
%       pdflatex letltxmacro.dtx
%       makeindex -s gind.ist letltxmacro.idx
%       pdflatex letltxmacro.dtx
%       makeindex -s gind.ist letltxmacro.idx
%       pdflatex letltxmacro.dtx
%
% Installation:
%    TDS:tex/latex/letltxmacro/letltxmacro.sty
%    TDS:doc/latex/letltxmacro/letltxmacro.pdf
%    TDS:doc/latex/letltxmacro/letltxmacro-showcases.tex
%    TDS:source/latex/letltxmacro/letltxmacro.dtx
%
%<*ignore>
\begingroup
  \catcode123=1 %
  \catcode125=2 %
  \def\x{LaTeX2e}%
\expandafter\endgroup
\ifcase 0\ifx\install y1\fi\expandafter
         \ifx\csname processbatchFile\endcsname\relax\else1\fi
         \ifx\fmtname\x\else 1\fi\relax
\else\csname fi\endcsname
%</ignore>
%<*install>
\input docstrip.tex
\Msg{************************************************************************}
\Msg{* Installation}
\Msg{* Package: letltxmacro 2019/12/03 v1.6 Let assignment for LaTeX macros (HO)}
\Msg{************************************************************************}

\keepsilent
\askforoverwritefalse

\let\MetaPrefix\relax
\preamble

This is a generated file.

Project: letltxmacro
Version: 2019/12/03 v1.6

Copyright (C)
   2008, 2010 Heiko Oberdiek
   2016-2019 Oberdiek Package Support Group

This work may be distributed and/or modified under the
conditions of the LaTeX Project Public License, either
version 1.3c of this license or (at your option) any later
version. This version of this license is in
   https://www.latex-project.org/lppl/lppl-1-3c.txt
and the latest version of this license is in
   https://www.latex-project.org/lppl.txt
and version 1.3 or later is part of all distributions of
LaTeX version 2005/12/01 or later.

This work has the LPPL maintenance status "maintained".

The Current Maintainers of this work are
Heiko Oberdiek and the Oberdiek Package Support Group
https://github.com/ho-tex/letltxmacro/issues


This work consists of the main source file letltxmacro.dtx
and the derived files
   letltxmacro.sty, letltxmacro.pdf, letltxmacro.ins, letltxmacro.drv,
   letltxmacro-showcases.tex, letltxmacro-test1.tex,
   letltxmacro-test2.tex.

\endpreamble
\let\MetaPrefix\DoubleperCent

\generate{%
  \file{letltxmacro.ins}{\from{letltxmacro.dtx}{install}}%
  \file{letltxmacro.drv}{\from{letltxmacro.dtx}{driver}}%
  \usedir{tex/latex/letltxmacro}%
  \file{letltxmacro.sty}{\from{letltxmacro.dtx}{package}}%
  \usedir{doc/latex/letltxmacro}%
  \file{letltxmacro-showcases.tex}{\from{letltxmacro.dtx}{showcases}}%
%  \usedir{doc/latex/letltxmacro/test}%
%  \file{letltxmacro-test1.tex}{\from{letltxmacro.dtx}{test1}}%
%  \file{letltxmacro-test2.tex}{\from{letltxmacro.dtx}{test2}}%
}

\catcode32=13\relax% active space
\let =\space%
\Msg{************************************************************************}
\Msg{*}
\Msg{* To finish the installation you have to move the following}
\Msg{* file into a directory searched by TeX:}
\Msg{*}
\Msg{*     letltxmacro.sty}
\Msg{*}
\Msg{* To produce the documentation run the file `letltxmacro.drv'}
\Msg{* through LaTeX.}
\Msg{*}
\Msg{* Happy TeXing!}
\Msg{*}
\Msg{************************************************************************}

\endbatchfile
%</install>
%<*ignore>
\fi
%</ignore>
%<*driver>
\NeedsTeXFormat{LaTeX2e}
\ProvidesFile{letltxmacro.drv}%
  [2019/12/03 v1.6 Let assignment for LaTeX macros (HO)]%
\documentclass{ltxdoc}
\usepackage{holtxdoc}[2011/11/22]
\begin{document}
  \DocInput{letltxmacro.dtx}%
\end{document}
%</driver>
% \fi
%
%
%
% \GetFileInfo{letltxmacro.drv}
%
% \title{The \xpackage{letltxmacro} package}
% \date{2019/12/03 v1.6}
% \author{Heiko Oberdiek\thanks
% {Please report any issues at \url{https://github.com/ho-tex/letltxmacro/issues}}}
%
% \maketitle
%
% \begin{abstract}
% \TeX's \cs{let} assignment does not work for \LaTeX\ macros
% with optional arguments or for macros that are defined
% as robust macros by \cs{DeclareRobustCommand}. This package
% defines \cs{LetLtxMacro} that also takes care of the involved
% internal macros.
% \end{abstract}
%
% \tableofcontents
%
% \section{Documentation}
%
% If someone wants to redefine a macro with using the old
% meaning, then one method is \TeX's command \cs{let}:
%\begin{quote}
%\begin{verbatim}
%\newcommand{\Macro}{\typeout{Test Macro}}
%\let\SavedMacro=\Macro
%\renewcommand{\Macro}{%
%  \typeout{Begin}%
%  \SavedMacro
%  \typeout{End}%
%}
%\end{verbatim}
%\end{quote}
% However, this method fails, if \cs{Macro} is defined
% by \cs{DeclareRobustCommand} and/or has an optional argument.
% In both cases \LaTeX\ defines an additional internal macro
% that is forgotten in the simple \cs{let} assignment of
% the example above.
%
% \begin{declcs}{LetLtxMacro} \M{new macro} \M{old macro}
% \end{declcs}
% Macro \cs{LetLtxMacro} behaves similar to \TeX's \cs{let}
% assignment, but it takes care of macros that are
% defined by \cs{DeclareRobustCommand} and/or have optional
% arguments. Example:
%\begin{quote}
%\begin{verbatim}
%\DeclareRobustCommand{\Macro}[1][default]{...}
%\LetLtxMacro{\SavedMacro}{\Macro}
%\end{verbatim}
%\end{quote}
% Then macro \cs{SavedMacro} only uses internal macro names
% that are derived from \cs{SavedMacro}'s macro name. Macro \cs{Macro}
% can now be redefined without affecting \cs{SavedMacro}.
%
% \begin{declcs}{GlobalLetLtxMacro} \M{new macro} \M{old macro}
% \end{declcs}
% Like \cs{LetLtxMacro}, but the \meta{new macro} is defined globally.
% Since version 2019/12/03~v1.4.
%
% \subsection{Supported macro definition commands}
%
% \begin{quote}
%   \begin{tabular}{@{}ll@{}}
%     \cs{newcommand}, \cs{renewcommand} & latex/base\\
%     \cs{newenvironment}, \cs{renewenvironment} & latex/base\\
%     \cs{DeclareRobustCommand}& latex/base\\
%     \cs{newrobustcmd}, \cs{renewrobustcmd} & etoolbox\\
%     \cs{robustify} & etoolbox 2008/06/22 v1.6\\
%   \end{tabular}
% \end{quote}
%
% \StopEventually{
% }
%
% \section{Implementation}
%
% \subsection{Show cases}
%
% \subsubsection{\xfile{letltxmacro-showcases.tex}}
%
%    \begin{macrocode}
%<*showcases>
\NeedsTeXFormat{LaTeX2e}
\makeatletter
%    \end{macrocode}
%    \begin{macro}{\Line}
%    The result is displayed by macro \cs{Line}. The percent symbol
%    at line start allows easy grepping and inserting into the DTX
%    file.
%    \begin{macrocode}
\newcommand*{\Line}[1]{%
  \typeout{\@percentchar#1}%
}
%    \end{macrocode}
%    \end{macro}
%    \begin{macrocode}
\newcommand*{\ShowCmdName}[1]{%
  \@ifundefined{#1}{}{%
    \Line{%
      \space\space(\expandafter\string\csname#1\endcsname) = %
      (\expandafter\meaning\csname#1\endcsname)%
    }%
  }%
}
\newcommand*{\ShowCmds}[1]{%
  \ShowCmdName{#1}%
  \ShowCmdName{#1 }%
  \ShowCmdName{\\#1}%
  \ShowCmdName{\\#1 }%
}
\let\\\@backslashchar
%    \end{macrocode}
%    \begin{macro}{\ShowDef}
%    \begin{macrocode}
\newcommand*{\ShowDef}[2]{%
  \begingroup
    \Line{}%
    \newcommand*{\DefString}{#2}%
    \@onelevel@sanitize\DefString
    \Line{\DefString}%
    #2%
    \ShowCmds{#1}%
  \endgroup
}
%    \end{macrocode}
%    \end{macro}
%    \begin{macrocode}
\typeout{}
\Line{* LaTeX definitions:}
\ShowDef{cmd}{%
  \newcommand{\cmd}[2][default]{}%
}
\ShowDef{cmd}{%
  \DeclareRobustCommand{\cmd}{}%
}
\ShowDef{cmd}{%
  \DeclareRobustCommand{\cmd}[2][default]{}%
}
\typeout{}
%    \end{macrocode}
% The minimal version of package \xpackage{etoolbox} is 2008/06/12 v1.6a
% because it fixes \cs{robustify}.
%    \begin{macrocode}
\RequirePackage{etoolbox}[2008/06/12]%
\Line{}
\Line{* etoolbox's robust definitions:}
\ShowDef{cmd}{%
  \newrobustcmd{\cmd}{}%
}
\ShowDef{cmd}{%
  \newrobustcmd{\cmd}[2][default]{}%
}
\Line{}
\Line{* etoolbox's \string\robustify:}
\ShowDef{cmd}{%
  \newcommand{\cmd}[2][default]{} %
  \robustify{\cmd}%
}
\ShowDef{cmd}{%
  \DeclareRobustCommand{\cmd}{} %
  \robustify{\cmd}%
}
\ShowDef{cmd}{%
  \DeclareRobustCommand{\cmd}[2][default]{} %
  \robustify{\cmd}%
}
\typeout{}
\@@end
%</showcases>
%    \end{macrocode}
%
% \subsubsection{Result}
%
% \begingroup
%   \makeatletter
%   \let\org@verbatim\@verbatim
%   \def\@verbatim{^^A
%     \org@verbatim
%     \catcode`\~=\active
%   }^^A
%   \let~\textvisiblespace
%\begin{verbatim}
%* LaTeX definitions:
%
%\newcommand {\cmd }[2][default]{}
%  (\cmd) = (macro:->\@protected@testopt \cmd \\cmd {default})
%  (\\cmd) = (\long macro:[#1]#2->)
%
%\DeclareRobustCommand {\cmd }{}
%  (\cmd) = (macro:->\protect \cmd~ )
%  (\cmd~) = (\long macro:->)
%
%\DeclareRobustCommand {\cmd }[2][default]{}
%  (\cmd) = (macro:->\protect \cmd~ )
%  (\cmd~) = (macro:->\@protected@testopt \cmd~ \\cmd~ {default})
%  (\\cmd~) = (\long macro:[#1]#2->)
%
%* etoolbox's robust definitions:
%
%\newrobustcmd {\cmd }{}
%  (\cmd) = (\protected\long macro:->)
%
%\newrobustcmd {\cmd }[2][default]{}
%  (\cmd) = (\protected macro:->\@testopt \\cmd {default})
%  (\\cmd) = (\long macro:[#1]#2->)
%
%* etoolbox's \robustify:
%
%\newcommand {\cmd }[2][default]{} \robustify {\cmd }
%  (\cmd) = (\protected macro:->\@protected@testopt \cmd \\cmd {default})
%  (\\cmd) = (\long macro:[#1]#2->)
%
%\DeclareRobustCommand {\cmd }{} \robustify {\cmd }
%  (\cmd) = (\protected macro:->)
%
%\DeclareRobustCommand {\cmd }[2][default]{} \robustify {\cmd }
%  (\cmd) = (\protected macro:->\@protected@testopt \cmd~ \\cmd~ {default})
%  (\cmd~) = (macro:->\@protected@testopt \cmd~ \\cmd~ {default})
%  (\\cmd~) = (\long macro:[#1]#2->)
%\end{verbatim}
% \endgroup
%
% \subsection{Package}
%
%    \begin{macrocode}
%<*package>
%    \end{macrocode}
%
% \subsubsection{Catcodes and identification}
%
%    \begin{macrocode}
\begingroup\catcode61\catcode48\catcode32=10\relax%
  \catcode13=5 % ^^M
  \endlinechar=13 %
  \catcode123=1 % {
  \catcode125=2 % }
  \catcode64=11 % @
  \def\x{\endgroup
    \expandafter\edef\csname llm@AtEnd\endcsname{%
      \endlinechar=\the\endlinechar\relax
      \catcode13=\the\catcode13\relax
      \catcode32=\the\catcode32\relax
      \catcode35=\the\catcode35\relax
      \catcode61=\the\catcode61\relax
      \catcode64=\the\catcode64\relax
      \catcode123=\the\catcode123\relax
      \catcode125=\the\catcode125\relax
    }%
  }%
\x\catcode61\catcode48\catcode32=10\relax%
\catcode13=5 % ^^M
\endlinechar=13 %
\catcode35=6 % #
\catcode64=11 % @
\catcode123=1 % {
\catcode125=2 % }
\def\TMP@EnsureCode#1#2{%
  \edef\llm@AtEnd{%
    \llm@AtEnd
    \catcode#1=\the\catcode#1\relax
  }%
  \catcode#1=#2\relax
}
\TMP@EnsureCode{40}{12}% (
\TMP@EnsureCode{41}{12}% )
\TMP@EnsureCode{42}{12}% *
\TMP@EnsureCode{45}{12}% -
\TMP@EnsureCode{46}{12}% .
\TMP@EnsureCode{47}{12}% /
\TMP@EnsureCode{58}{12}% :
\TMP@EnsureCode{62}{12}% >
\TMP@EnsureCode{91}{12}% [
\TMP@EnsureCode{93}{12}% ]
\edef\llm@AtEnd{%
  \llm@AtEnd
  \escapechar\the\escapechar\relax
  \noexpand\endinput
}
\escapechar=92 % `\\
%    \end{macrocode}
%
%    Package identification.
%    \begin{macrocode}
\NeedsTeXFormat{LaTeX2e}
\ProvidesPackage{letltxmacro}%
  [2019/12/03 v1.6 Let assignment for LaTeX macros (HO)]
%    \end{macrocode}
%
% \subsubsection{Main macros}
%
%    \begin{macro}{\LetLtxMacro}
%    \begin{macrocode}
\newcommand*{\LetLtxMacro}{%
  \llm@ModeLetLtxMacro{}%
}
%    \end{macrocode}
%    \end{macro}
%    \begin{macro}{\GlobalLetLtxMacro}
%    \begin{macrocode}
\newcommand*{\GlobalLetLtxMacro}{%
  \llm@ModeLetLtxMacro\global
}
%    \end{macrocode}
%    \end{macro}
%
%    \begin{macro}{\llm@ModeLetLtxMacro}
%    \begin{macrocode}
\newcommand*{\llm@ModeLetLtxMacro}[3]{%
  \edef\llm@escapechar{\the\escapechar}%
  \escapechar=-1 %
  \edef\reserved@a{%
    \noexpand\protect
    \expandafter\noexpand
    \csname\string#3 \endcsname
  }%
  \ifx\reserved@a#3\relax
    #1\edef#2{%
      \noexpand\protect
      \expandafter\noexpand
      \csname\string#2 \endcsname
    }%
    #1\expandafter\let
    \csname\string#2 \expandafter\endcsname
    \csname\string#3 \endcsname
    \expandafter\llm@LetLtxMacro
        \csname\string#2 \expandafter\endcsname
        \csname\string#3 \endcsname{#1}%
  \else
    \llm@LetLtxMacro{#2}{#3}{#1}%
  \fi
  \escapechar=\llm@escapechar\relax
}
%    \end{macrocode}
%    \end{macro}
%    \begin{macro}{\llm@LetLtxMacro}
%    \begin{macrocode}
\def\llm@LetLtxMacro#1#2#3{%
  \escapechar=92 %
  \expandafter\llm@CheckParams\meaning#2:->\@nil{%
    \begingroup
      \def\@protected@testopt{%
        \expandafter\@testopt\@gobble
      }%
      \def\@testopt##1##2{%
        \toks@={##2}%
      }%
      \let\llm@testopt\@empty
      \edef\x{%
        \noexpand\@protected@testopt
        \noexpand#2%
        \expandafter\noexpand\csname\string#2\endcsname
      }%
      \expandafter\expandafter\expandafter\def
      \expandafter\expandafter\expandafter\y
      \expandafter\expandafter\expandafter{%
        \expandafter\llm@CarThree#2{}{}{}\llm@nil
      }%
      \ifx\x\y
        #2%
        \def\llm@testopt{%
          \noexpand\@protected@testopt
          \noexpand#1%
        }%
      \else
        \edef\x{%
          \noexpand\@testopt
          \expandafter\noexpand
          \csname\string#2\endcsname
        }%
        \expandafter\expandafter\expandafter\def
        \expandafter\expandafter\expandafter\y
        \expandafter\expandafter\expandafter{%
          \expandafter\llm@CarTwo#2{}{}\llm@nil
        }%
        \ifx\x\y
          #2%
          \def\llm@testopt{%
            \noexpand\@testopt
          }%
        \fi
      \fi
      \ifx\llm@testopt\@empty
      \else
        \llm@protected\xdef\llm@GlobalTemp{%
          \llm@testopt
          \expandafter\noexpand
          \csname\string#1\endcsname
          {\the\toks@}%
        }%
      \fi
    \expandafter\endgroup\ifx\llm@testopt\@empty
      #3\let#1=#2\relax
    \else
      #3\let#1=\llm@GlobalTemp
      #3\expandafter\let
          \csname\string#1\expandafter\endcsname
          \csname\string#2\endcsname
    \fi
  }{%
    #3\let#1=#2\relax
  }%
}
%    \end{macrocode}
%    \end{macro}
%    \begin{macro}{\llm@CheckParams}
%    \begin{macrocode}
\def\llm@CheckParams#1:->#2\@nil{%
  \begingroup
    \def\x{#1}%
  \ifx\x\llm@macro
    \endgroup
    \def\llm@protected{}%
    \expandafter\@firstoftwo
  \else
    \ifx\x\llm@protectedmacro
      \endgroup
      \def\llm@protected{\protected}%
      \expandafter\expandafter\expandafter\@firstoftwo
    \else
      \endgroup
      \expandafter\expandafter\expandafter\@secondoftwo
    \fi
  \fi
}
%    \end{macrocode}
%    \end{macro}
%    \begin{macro}{\llm@macro}
%    \begin{macrocode}
\def\llm@macro{macro}
\@onelevel@sanitize\llm@macro
%    \end{macrocode}
%    \end{macro}
%    \begin{macro}{\llm@protectedmacro}
%    \begin{macrocode}
\def\llm@protectedmacro{\protected macro}
\@onelevel@sanitize\llm@protectedmacro
%    \end{macrocode}
%    \end{macro}
%    \begin{macro}{\llm@CarThree}
%    \begin{macrocode}
\def\llm@CarThree#1#2#3#4\llm@nil{#1#2#3}%
%    \end{macrocode}
%    \end{macro}
%    \begin{macro}{\llm@CarTwo}
%    \begin{macrocode}
\def\llm@CarTwo#1#2#3\llm@nil{#1#2}%
%    \end{macrocode}
%    \end{macro}
%
%    \begin{macrocode}
\llm@AtEnd%
%</package>
%    \end{macrocode}
% \section{Installation}
%
% \subsection{Download}
%
% \paragraph{Package.} This package is available on
% CTAN\footnote{\CTANpkg{letltxmacro}}:
% \begin{description}
% \item[\CTAN{macros/latex/contrib/letltxmacro/letltxmacro.dtx}] The source file.
% \item[\CTAN{macros/latex/contrib/letltxmacro/letltxmacro.pdf}] Documentation.
% \end{description}
%
%
% \paragraph{Bundle.} All the packages of the bundle `letltxmacro'
% are also available in a TDS compliant ZIP archive. There
% the packages are already unpacked and the documentation files
% are generated. The files and directories obey the TDS standard.
% \begin{description}
% \item[\CTANinstall{install/macros/latex/contrib/letltxmacro.tds.zip}]
% \end{description}
% \emph{TDS} refers to the standard ``A Directory Structure
% for \TeX\ Files'' (\CTANpkg{tds}). Directories
% with \xfile{texmf} in their name are usually organized this way.
%
% \subsection{Bundle installation}
%
% \paragraph{Unpacking.} Unpack the \xfile{letltxmacro.tds.zip} in the
% TDS tree (also known as \xfile{texmf} tree) of your choice.
% Example (linux):
% \begin{quote}
%   |unzip letltxmacro.tds.zip -d ~/texmf|
% \end{quote}
%
% \subsection{Package installation}
%
% \paragraph{Unpacking.} The \xfile{.dtx} file is a self-extracting
% \docstrip\ archive. The files are extracted by running the
% \xfile{.dtx} through \plainTeX:
% \begin{quote}
%   \verb|tex letltxmacro.dtx|
% \end{quote}
%
% \paragraph{TDS.} Now the different files must be moved into
% the different directories in your installation TDS tree
% (also known as \xfile{texmf} tree):
% \begin{quote}
% \def\t{^^A
% \begin{tabular}{@{}>{\ttfamily}l@{ $\rightarrow$ }>{\ttfamily}l@{}}
%   letltxmacro.sty & tex/latex/letltxmacro/letltxmacro.sty\\
%   letltxmacro.pdf & doc/latex/letltxmacro/letltxmacro.pdf\\
%   letltxmacro-showcases.tex & doc/latex/letltxmacro/letltxmacro-showcases.tex\\
%   letltxmacro.dtx & source/latex/letltxmacro/letltxmacro.dtx\\
% \end{tabular}^^A
% }^^A
% \sbox0{\t}^^A
% \ifdim\wd0>\linewidth
%   \begingroup
%     \advance\linewidth by\leftmargin
%     \advance\linewidth by\rightmargin
%   \edef\x{\endgroup
%     \def\noexpand\lw{\the\linewidth}^^A
%   }\x
%   \def\lwbox{^^A
%     \leavevmode
%     \hbox to \linewidth{^^A
%       \kern-\leftmargin\relax
%       \hss
%       \usebox0
%       \hss
%       \kern-\rightmargin\relax
%     }^^A
%   }^^A
%   \ifdim\wd0>\lw
%     \sbox0{\small\t}^^A
%     \ifdim\wd0>\linewidth
%       \ifdim\wd0>\lw
%         \sbox0{\footnotesize\t}^^A
%         \ifdim\wd0>\linewidth
%           \ifdim\wd0>\lw
%             \sbox0{\scriptsize\t}^^A
%             \ifdim\wd0>\linewidth
%               \ifdim\wd0>\lw
%                 \sbox0{\tiny\t}^^A
%                 \ifdim\wd0>\linewidth
%                   \lwbox
%                 \else
%                   \usebox0
%                 \fi
%               \else
%                 \lwbox
%               \fi
%             \else
%               \usebox0
%             \fi
%           \else
%             \lwbox
%           \fi
%         \else
%           \usebox0
%         \fi
%       \else
%         \lwbox
%       \fi
%     \else
%       \usebox0
%     \fi
%   \else
%     \lwbox
%   \fi
% \else
%   \usebox0
% \fi
% \end{quote}
% If you have a \xfile{docstrip.cfg} that configures and enables \docstrip's
% TDS installing feature, then some files can already be in the right
% place, see the documentation of \docstrip.
%
% \subsection{Refresh file name databases}
%
% If your \TeX~distribution
% (\TeX\,Live, \mikTeX, \dots) relies on file name databases, you must refresh
% these. For example, \TeX\,Live\ users run \verb|texhash| or
% \verb|mktexlsr|.
%
% \subsection{Some details for the interested}
%
% \paragraph{Unpacking with \LaTeX.}
% The \xfile{.dtx} chooses its action depending on the format:
% \begin{description}
% \item[\plainTeX:] Run \docstrip\ and extract the files.
% \item[\LaTeX:] Generate the documentation.
% \end{description}
% If you insist on using \LaTeX\ for \docstrip\ (really,
% \docstrip\ does not need \LaTeX), then inform the autodetect routine
% about your intention:
% \begin{quote}
%   \verb|latex \let\install=y\input{letltxmacro.dtx}|
% \end{quote}
% Do not forget to quote the argument according to the demands
% of your shell.
%
% \paragraph{Generating the documentation.}
% You can use both the \xfile{.dtx} or the \xfile{.drv} to generate
% the documentation. The process can be configured by the
% configuration file \xfile{ltxdoc.cfg}. For instance, put this
% line into this file, if you want to have A4 as paper format:
% \begin{quote}
%   \verb|\PassOptionsToClass{a4paper}{article}|
% \end{quote}
% An example follows how to generate the
% documentation with pdf\LaTeX:
% \begin{quote}
%\begin{verbatim}
%pdflatex letltxmacro.dtx
%makeindex -s gind.ist letltxmacro.idx
%pdflatex letltxmacro.dtx
%makeindex -s gind.ist letltxmacro.idx
%pdflatex letltxmacro.dtx
%\end{verbatim}
% \end{quote}
%
% \begin{History}
%   \begin{Version}{2008/06/09 v1.0}
%   \item
%     First version.
%   \end{Version}
%   \begin{Version}{2008/06/12 v1.1}
%   \item
%     Support for \xpackage{etoolbox}'s \cs{newrobustcmd} added.
%   \end{Version}
%   \begin{Version}{2008/06/13 v1.2}
%   \item
%     Support for \xpackage{etoolbox}'s \cs{robustify} added.
%   \end{Version}
%   \begin{Version}{2008/06/24 v1.3}
%   \item
%     Test file adapted for etoolbox 2008/06/22 v1.6.
%   \end{Version}
%   \begin{Version}{2010/09/02 v1.4}
%   \item
%     \cs{GlobalLetLtxMacro} added.
%   \end{Version}
%   \begin{Version}{2016/05/16 v1.5}
%   \item
%     Documentation updates.
%   \end{Version}
%   \begin{Version}{2019/12/03 v1.6}
%   \item
%     Documentation updates.
%   \end{Version}
% \end{History}
%
% \PrintIndex
%
% \Finale
\endinput

%        (quote the arguments according to the demands of your shell)
%
% Documentation:
%    (a) If letltxmacro.drv is present:
%           latex letltxmacro.drv
%    (b) Without letltxmacro.drv:
%           latex letltxmacro.dtx; ...
%    The class ltxdoc loads the configuration file ltxdoc.cfg
%    if available. Here you can specify further options, e.g.
%    use A4 as paper format:
%       \PassOptionsToClass{a4paper}{article}
%
%    Programm calls to get the documentation (example):
%       pdflatex letltxmacro.dtx
%       makeindex -s gind.ist letltxmacro.idx
%       pdflatex letltxmacro.dtx
%       makeindex -s gind.ist letltxmacro.idx
%       pdflatex letltxmacro.dtx
%
% Installation:
%    TDS:tex/latex/letltxmacro/letltxmacro.sty
%    TDS:doc/latex/letltxmacro/letltxmacro.pdf
%    TDS:doc/latex/letltxmacro/letltxmacro-showcases.tex
%    TDS:source/latex/letltxmacro/letltxmacro.dtx
%
%<*ignore>
\begingroup
  \catcode123=1 %
  \catcode125=2 %
  \def\x{LaTeX2e}%
\expandafter\endgroup
\ifcase 0\ifx\install y1\fi\expandafter
         \ifx\csname processbatchFile\endcsname\relax\else1\fi
         \ifx\fmtname\x\else 1\fi\relax
\else\csname fi\endcsname
%</ignore>
%<*install>
\input docstrip.tex
\Msg{************************************************************************}
\Msg{* Installation}
\Msg{* Package: letltxmacro 2019/12/03 v1.6 Let assignment for LaTeX macros (HO)}
\Msg{************************************************************************}

\keepsilent
\askforoverwritefalse

\let\MetaPrefix\relax
\preamble

This is a generated file.

Project: letltxmacro
Version: 2019/12/03 v1.6

Copyright (C)
   2008, 2010 Heiko Oberdiek
   2016-2019 Oberdiek Package Support Group

This work may be distributed and/or modified under the
conditions of the LaTeX Project Public License, either
version 1.3c of this license or (at your option) any later
version. This version of this license is in
   https://www.latex-project.org/lppl/lppl-1-3c.txt
and the latest version of this license is in
   https://www.latex-project.org/lppl.txt
and version 1.3 or later is part of all distributions of
LaTeX version 2005/12/01 or later.

This work has the LPPL maintenance status "maintained".

The Current Maintainers of this work are
Heiko Oberdiek and the Oberdiek Package Support Group
https://github.com/ho-tex/letltxmacro/issues


This work consists of the main source file letltxmacro.dtx
and the derived files
   letltxmacro.sty, letltxmacro.pdf, letltxmacro.ins, letltxmacro.drv,
   letltxmacro-showcases.tex, letltxmacro-test1.tex,
   letltxmacro-test2.tex.

\endpreamble
\let\MetaPrefix\DoubleperCent

\generate{%
  \file{letltxmacro.ins}{\from{letltxmacro.dtx}{install}}%
  \file{letltxmacro.drv}{\from{letltxmacro.dtx}{driver}}%
  \usedir{tex/latex/letltxmacro}%
  \file{letltxmacro.sty}{\from{letltxmacro.dtx}{package}}%
  \usedir{doc/latex/letltxmacro}%
  \file{letltxmacro-showcases.tex}{\from{letltxmacro.dtx}{showcases}}%
%  \usedir{doc/latex/letltxmacro/test}%
%  \file{letltxmacro-test1.tex}{\from{letltxmacro.dtx}{test1}}%
%  \file{letltxmacro-test2.tex}{\from{letltxmacro.dtx}{test2}}%
}

\catcode32=13\relax% active space
\let =\space%
\Msg{************************************************************************}
\Msg{*}
\Msg{* To finish the installation you have to move the following}
\Msg{* file into a directory searched by TeX:}
\Msg{*}
\Msg{*     letltxmacro.sty}
\Msg{*}
\Msg{* To produce the documentation run the file `letltxmacro.drv'}
\Msg{* through LaTeX.}
\Msg{*}
\Msg{* Happy TeXing!}
\Msg{*}
\Msg{************************************************************************}

\endbatchfile
%</install>
%<*ignore>
\fi
%</ignore>
%<*driver>
\NeedsTeXFormat{LaTeX2e}
\ProvidesFile{letltxmacro.drv}%
  [2019/12/03 v1.6 Let assignment for LaTeX macros (HO)]%
\documentclass{ltxdoc}
\usepackage{holtxdoc}[2011/11/22]
\begin{document}
  \DocInput{letltxmacro.dtx}%
\end{document}
%</driver>
% \fi
%
%
%
% \GetFileInfo{letltxmacro.drv}
%
% \title{The \xpackage{letltxmacro} package}
% \date{2019/12/03 v1.6}
% \author{Heiko Oberdiek\thanks
% {Please report any issues at \url{https://github.com/ho-tex/letltxmacro/issues}}}
%
% \maketitle
%
% \begin{abstract}
% \TeX's \cs{let} assignment does not work for \LaTeX\ macros
% with optional arguments or for macros that are defined
% as robust macros by \cs{DeclareRobustCommand}. This package
% defines \cs{LetLtxMacro} that also takes care of the involved
% internal macros.
% \end{abstract}
%
% \tableofcontents
%
% \section{Documentation}
%
% If someone wants to redefine a macro with using the old
% meaning, then one method is \TeX's command \cs{let}:
%\begin{quote}
%\begin{verbatim}
%\newcommand{\Macro}{\typeout{Test Macro}}
%\let\SavedMacro=\Macro
%\renewcommand{\Macro}{%
%  \typeout{Begin}%
%  \SavedMacro
%  \typeout{End}%
%}
%\end{verbatim}
%\end{quote}
% However, this method fails, if \cs{Macro} is defined
% by \cs{DeclareRobustCommand} and/or has an optional argument.
% In both cases \LaTeX\ defines an additional internal macro
% that is forgotten in the simple \cs{let} assignment of
% the example above.
%
% \begin{declcs}{LetLtxMacro} \M{new macro} \M{old macro}
% \end{declcs}
% Macro \cs{LetLtxMacro} behaves similar to \TeX's \cs{let}
% assignment, but it takes care of macros that are
% defined by \cs{DeclareRobustCommand} and/or have optional
% arguments. Example:
%\begin{quote}
%\begin{verbatim}
%\DeclareRobustCommand{\Macro}[1][default]{...}
%\LetLtxMacro{\SavedMacro}{\Macro}
%\end{verbatim}
%\end{quote}
% Then macro \cs{SavedMacro} only uses internal macro names
% that are derived from \cs{SavedMacro}'s macro name. Macro \cs{Macro}
% can now be redefined without affecting \cs{SavedMacro}.
%
% \begin{declcs}{GlobalLetLtxMacro} \M{new macro} \M{old macro}
% \end{declcs}
% Like \cs{LetLtxMacro}, but the \meta{new macro} is defined globally.
% Since version 2019/12/03~v1.4.
%
% \subsection{Supported macro definition commands}
%
% \begin{quote}
%   \begin{tabular}{@{}ll@{}}
%     \cs{newcommand}, \cs{renewcommand} & latex/base\\
%     \cs{newenvironment}, \cs{renewenvironment} & latex/base\\
%     \cs{DeclareRobustCommand}& latex/base\\
%     \cs{newrobustcmd}, \cs{renewrobustcmd} & etoolbox\\
%     \cs{robustify} & etoolbox 2008/06/22 v1.6\\
%   \end{tabular}
% \end{quote}
%
% \StopEventually{
% }
%
% \section{Implementation}
%
% \subsection{Show cases}
%
% \subsubsection{\xfile{letltxmacro-showcases.tex}}
%
%    \begin{macrocode}
%<*showcases>
\NeedsTeXFormat{LaTeX2e}
\makeatletter
%    \end{macrocode}
%    \begin{macro}{\Line}
%    The result is displayed by macro \cs{Line}. The percent symbol
%    at line start allows easy grepping and inserting into the DTX
%    file.
%    \begin{macrocode}
\newcommand*{\Line}[1]{%
  \typeout{\@percentchar#1}%
}
%    \end{macrocode}
%    \end{macro}
%    \begin{macrocode}
\newcommand*{\ShowCmdName}[1]{%
  \@ifundefined{#1}{}{%
    \Line{%
      \space\space(\expandafter\string\csname#1\endcsname) = %
      (\expandafter\meaning\csname#1\endcsname)%
    }%
  }%
}
\newcommand*{\ShowCmds}[1]{%
  \ShowCmdName{#1}%
  \ShowCmdName{#1 }%
  \ShowCmdName{\\#1}%
  \ShowCmdName{\\#1 }%
}
\let\\\@backslashchar
%    \end{macrocode}
%    \begin{macro}{\ShowDef}
%    \begin{macrocode}
\newcommand*{\ShowDef}[2]{%
  \begingroup
    \Line{}%
    \newcommand*{\DefString}{#2}%
    \@onelevel@sanitize\DefString
    \Line{\DefString}%
    #2%
    \ShowCmds{#1}%
  \endgroup
}
%    \end{macrocode}
%    \end{macro}
%    \begin{macrocode}
\typeout{}
\Line{* LaTeX definitions:}
\ShowDef{cmd}{%
  \newcommand{\cmd}[2][default]{}%
}
\ShowDef{cmd}{%
  \DeclareRobustCommand{\cmd}{}%
}
\ShowDef{cmd}{%
  \DeclareRobustCommand{\cmd}[2][default]{}%
}
\typeout{}
%    \end{macrocode}
% The minimal version of package \xpackage{etoolbox} is 2008/06/12 v1.6a
% because it fixes \cs{robustify}.
%    \begin{macrocode}
\RequirePackage{etoolbox}[2008/06/12]%
\Line{}
\Line{* etoolbox's robust definitions:}
\ShowDef{cmd}{%
  \newrobustcmd{\cmd}{}%
}
\ShowDef{cmd}{%
  \newrobustcmd{\cmd}[2][default]{}%
}
\Line{}
\Line{* etoolbox's \string\robustify:}
\ShowDef{cmd}{%
  \newcommand{\cmd}[2][default]{} %
  \robustify{\cmd}%
}
\ShowDef{cmd}{%
  \DeclareRobustCommand{\cmd}{} %
  \robustify{\cmd}%
}
\ShowDef{cmd}{%
  \DeclareRobustCommand{\cmd}[2][default]{} %
  \robustify{\cmd}%
}
\typeout{}
\@@end
%</showcases>
%    \end{macrocode}
%
% \subsubsection{Result}
%
% \begingroup
%   \makeatletter
%   \let\org@verbatim\@verbatim
%   \def\@verbatim{^^A
%     \org@verbatim
%     \catcode`\~=\active
%   }^^A
%   \let~\textvisiblespace
%\begin{verbatim}
%* LaTeX definitions:
%
%\newcommand {\cmd }[2][default]{}
%  (\cmd) = (macro:->\@protected@testopt \cmd \\cmd {default})
%  (\\cmd) = (\long macro:[#1]#2->)
%
%\DeclareRobustCommand {\cmd }{}
%  (\cmd) = (macro:->\protect \cmd~ )
%  (\cmd~) = (\long macro:->)
%
%\DeclareRobustCommand {\cmd }[2][default]{}
%  (\cmd) = (macro:->\protect \cmd~ )
%  (\cmd~) = (macro:->\@protected@testopt \cmd~ \\cmd~ {default})
%  (\\cmd~) = (\long macro:[#1]#2->)
%
%* etoolbox's robust definitions:
%
%\newrobustcmd {\cmd }{}
%  (\cmd) = (\protected\long macro:->)
%
%\newrobustcmd {\cmd }[2][default]{}
%  (\cmd) = (\protected macro:->\@testopt \\cmd {default})
%  (\\cmd) = (\long macro:[#1]#2->)
%
%* etoolbox's \robustify:
%
%\newcommand {\cmd }[2][default]{} \robustify {\cmd }
%  (\cmd) = (\protected macro:->\@protected@testopt \cmd \\cmd {default})
%  (\\cmd) = (\long macro:[#1]#2->)
%
%\DeclareRobustCommand {\cmd }{} \robustify {\cmd }
%  (\cmd) = (\protected macro:->)
%
%\DeclareRobustCommand {\cmd }[2][default]{} \robustify {\cmd }
%  (\cmd) = (\protected macro:->\@protected@testopt \cmd~ \\cmd~ {default})
%  (\cmd~) = (macro:->\@protected@testopt \cmd~ \\cmd~ {default})
%  (\\cmd~) = (\long macro:[#1]#2->)
%\end{verbatim}
% \endgroup
%
% \subsection{Package}
%
%    \begin{macrocode}
%<*package>
%    \end{macrocode}
%
% \subsubsection{Catcodes and identification}
%
%    \begin{macrocode}
\begingroup\catcode61\catcode48\catcode32=10\relax%
  \catcode13=5 % ^^M
  \endlinechar=13 %
  \catcode123=1 % {
  \catcode125=2 % }
  \catcode64=11 % @
  \def\x{\endgroup
    \expandafter\edef\csname llm@AtEnd\endcsname{%
      \endlinechar=\the\endlinechar\relax
      \catcode13=\the\catcode13\relax
      \catcode32=\the\catcode32\relax
      \catcode35=\the\catcode35\relax
      \catcode61=\the\catcode61\relax
      \catcode64=\the\catcode64\relax
      \catcode123=\the\catcode123\relax
      \catcode125=\the\catcode125\relax
    }%
  }%
\x\catcode61\catcode48\catcode32=10\relax%
\catcode13=5 % ^^M
\endlinechar=13 %
\catcode35=6 % #
\catcode64=11 % @
\catcode123=1 % {
\catcode125=2 % }
\def\TMP@EnsureCode#1#2{%
  \edef\llm@AtEnd{%
    \llm@AtEnd
    \catcode#1=\the\catcode#1\relax
  }%
  \catcode#1=#2\relax
}
\TMP@EnsureCode{40}{12}% (
\TMP@EnsureCode{41}{12}% )
\TMP@EnsureCode{42}{12}% *
\TMP@EnsureCode{45}{12}% -
\TMP@EnsureCode{46}{12}% .
\TMP@EnsureCode{47}{12}% /
\TMP@EnsureCode{58}{12}% :
\TMP@EnsureCode{62}{12}% >
\TMP@EnsureCode{91}{12}% [
\TMP@EnsureCode{93}{12}% ]
\edef\llm@AtEnd{%
  \llm@AtEnd
  \escapechar\the\escapechar\relax
  \noexpand\endinput
}
\escapechar=92 % `\\
%    \end{macrocode}
%
%    Package identification.
%    \begin{macrocode}
\NeedsTeXFormat{LaTeX2e}
\ProvidesPackage{letltxmacro}%
  [2019/12/03 v1.6 Let assignment for LaTeX macros (HO)]
%    \end{macrocode}
%
% \subsubsection{Main macros}
%
%    \begin{macro}{\LetLtxMacro}
%    \begin{macrocode}
\newcommand*{\LetLtxMacro}{%
  \llm@ModeLetLtxMacro{}%
}
%    \end{macrocode}
%    \end{macro}
%    \begin{macro}{\GlobalLetLtxMacro}
%    \begin{macrocode}
\newcommand*{\GlobalLetLtxMacro}{%
  \llm@ModeLetLtxMacro\global
}
%    \end{macrocode}
%    \end{macro}
%
%    \begin{macro}{\llm@ModeLetLtxMacro}
%    \begin{macrocode}
\newcommand*{\llm@ModeLetLtxMacro}[3]{%
  \edef\llm@escapechar{\the\escapechar}%
  \escapechar=-1 %
  \edef\reserved@a{%
    \noexpand\protect
    \expandafter\noexpand
    \csname\string#3 \endcsname
  }%
  \ifx\reserved@a#3\relax
    #1\edef#2{%
      \noexpand\protect
      \expandafter\noexpand
      \csname\string#2 \endcsname
    }%
    #1\expandafter\let
    \csname\string#2 \expandafter\endcsname
    \csname\string#3 \endcsname
    \expandafter\llm@LetLtxMacro
        \csname\string#2 \expandafter\endcsname
        \csname\string#3 \endcsname{#1}%
  \else
    \llm@LetLtxMacro{#2}{#3}{#1}%
  \fi
  \escapechar=\llm@escapechar\relax
}
%    \end{macrocode}
%    \end{macro}
%    \begin{macro}{\llm@LetLtxMacro}
%    \begin{macrocode}
\def\llm@LetLtxMacro#1#2#3{%
  \escapechar=92 %
  \expandafter\llm@CheckParams\meaning#2:->\@nil{%
    \begingroup
      \def\@protected@testopt{%
        \expandafter\@testopt\@gobble
      }%
      \def\@testopt##1##2{%
        \toks@={##2}%
      }%
      \let\llm@testopt\@empty
      \edef\x{%
        \noexpand\@protected@testopt
        \noexpand#2%
        \expandafter\noexpand\csname\string#2\endcsname
      }%
      \expandafter\expandafter\expandafter\def
      \expandafter\expandafter\expandafter\y
      \expandafter\expandafter\expandafter{%
        \expandafter\llm@CarThree#2{}{}{}\llm@nil
      }%
      \ifx\x\y
        #2%
        \def\llm@testopt{%
          \noexpand\@protected@testopt
          \noexpand#1%
        }%
      \else
        \edef\x{%
          \noexpand\@testopt
          \expandafter\noexpand
          \csname\string#2\endcsname
        }%
        \expandafter\expandafter\expandafter\def
        \expandafter\expandafter\expandafter\y
        \expandafter\expandafter\expandafter{%
          \expandafter\llm@CarTwo#2{}{}\llm@nil
        }%
        \ifx\x\y
          #2%
          \def\llm@testopt{%
            \noexpand\@testopt
          }%
        \fi
      \fi
      \ifx\llm@testopt\@empty
      \else
        \llm@protected\xdef\llm@GlobalTemp{%
          \llm@testopt
          \expandafter\noexpand
          \csname\string#1\endcsname
          {\the\toks@}%
        }%
      \fi
    \expandafter\endgroup\ifx\llm@testopt\@empty
      #3\let#1=#2\relax
    \else
      #3\let#1=\llm@GlobalTemp
      #3\expandafter\let
          \csname\string#1\expandafter\endcsname
          \csname\string#2\endcsname
    \fi
  }{%
    #3\let#1=#2\relax
  }%
}
%    \end{macrocode}
%    \end{macro}
%    \begin{macro}{\llm@CheckParams}
%    \begin{macrocode}
\def\llm@CheckParams#1:->#2\@nil{%
  \begingroup
    \def\x{#1}%
  \ifx\x\llm@macro
    \endgroup
    \def\llm@protected{}%
    \expandafter\@firstoftwo
  \else
    \ifx\x\llm@protectedmacro
      \endgroup
      \def\llm@protected{\protected}%
      \expandafter\expandafter\expandafter\@firstoftwo
    \else
      \endgroup
      \expandafter\expandafter\expandafter\@secondoftwo
    \fi
  \fi
}
%    \end{macrocode}
%    \end{macro}
%    \begin{macro}{\llm@macro}
%    \begin{macrocode}
\def\llm@macro{macro}
\@onelevel@sanitize\llm@macro
%    \end{macrocode}
%    \end{macro}
%    \begin{macro}{\llm@protectedmacro}
%    \begin{macrocode}
\def\llm@protectedmacro{\protected macro}
\@onelevel@sanitize\llm@protectedmacro
%    \end{macrocode}
%    \end{macro}
%    \begin{macro}{\llm@CarThree}
%    \begin{macrocode}
\def\llm@CarThree#1#2#3#4\llm@nil{#1#2#3}%
%    \end{macrocode}
%    \end{macro}
%    \begin{macro}{\llm@CarTwo}
%    \begin{macrocode}
\def\llm@CarTwo#1#2#3\llm@nil{#1#2}%
%    \end{macrocode}
%    \end{macro}
%
%    \begin{macrocode}
\llm@AtEnd%
%</package>
%    \end{macrocode}
% \section{Installation}
%
% \subsection{Download}
%
% \paragraph{Package.} This package is available on
% CTAN\footnote{\CTANpkg{letltxmacro}}:
% \begin{description}
% \item[\CTAN{macros/latex/contrib/letltxmacro/letltxmacro.dtx}] The source file.
% \item[\CTAN{macros/latex/contrib/letltxmacro/letltxmacro.pdf}] Documentation.
% \end{description}
%
%
% \paragraph{Bundle.} All the packages of the bundle `letltxmacro'
% are also available in a TDS compliant ZIP archive. There
% the packages are already unpacked and the documentation files
% are generated. The files and directories obey the TDS standard.
% \begin{description}
% \item[\CTANinstall{install/macros/latex/contrib/letltxmacro.tds.zip}]
% \end{description}
% \emph{TDS} refers to the standard ``A Directory Structure
% for \TeX\ Files'' (\CTANpkg{tds}). Directories
% with \xfile{texmf} in their name are usually organized this way.
%
% \subsection{Bundle installation}
%
% \paragraph{Unpacking.} Unpack the \xfile{letltxmacro.tds.zip} in the
% TDS tree (also known as \xfile{texmf} tree) of your choice.
% Example (linux):
% \begin{quote}
%   |unzip letltxmacro.tds.zip -d ~/texmf|
% \end{quote}
%
% \subsection{Package installation}
%
% \paragraph{Unpacking.} The \xfile{.dtx} file is a self-extracting
% \docstrip\ archive. The files are extracted by running the
% \xfile{.dtx} through \plainTeX:
% \begin{quote}
%   \verb|tex letltxmacro.dtx|
% \end{quote}
%
% \paragraph{TDS.} Now the different files must be moved into
% the different directories in your installation TDS tree
% (also known as \xfile{texmf} tree):
% \begin{quote}
% \def\t{^^A
% \begin{tabular}{@{}>{\ttfamily}l@{ $\rightarrow$ }>{\ttfamily}l@{}}
%   letltxmacro.sty & tex/latex/letltxmacro/letltxmacro.sty\\
%   letltxmacro.pdf & doc/latex/letltxmacro/letltxmacro.pdf\\
%   letltxmacro-showcases.tex & doc/latex/letltxmacro/letltxmacro-showcases.tex\\
%   letltxmacro.dtx & source/latex/letltxmacro/letltxmacro.dtx\\
% \end{tabular}^^A
% }^^A
% \sbox0{\t}^^A
% \ifdim\wd0>\linewidth
%   \begingroup
%     \advance\linewidth by\leftmargin
%     \advance\linewidth by\rightmargin
%   \edef\x{\endgroup
%     \def\noexpand\lw{\the\linewidth}^^A
%   }\x
%   \def\lwbox{^^A
%     \leavevmode
%     \hbox to \linewidth{^^A
%       \kern-\leftmargin\relax
%       \hss
%       \usebox0
%       \hss
%       \kern-\rightmargin\relax
%     }^^A
%   }^^A
%   \ifdim\wd0>\lw
%     \sbox0{\small\t}^^A
%     \ifdim\wd0>\linewidth
%       \ifdim\wd0>\lw
%         \sbox0{\footnotesize\t}^^A
%         \ifdim\wd0>\linewidth
%           \ifdim\wd0>\lw
%             \sbox0{\scriptsize\t}^^A
%             \ifdim\wd0>\linewidth
%               \ifdim\wd0>\lw
%                 \sbox0{\tiny\t}^^A
%                 \ifdim\wd0>\linewidth
%                   \lwbox
%                 \else
%                   \usebox0
%                 \fi
%               \else
%                 \lwbox
%               \fi
%             \else
%               \usebox0
%             \fi
%           \else
%             \lwbox
%           \fi
%         \else
%           \usebox0
%         \fi
%       \else
%         \lwbox
%       \fi
%     \else
%       \usebox0
%     \fi
%   \else
%     \lwbox
%   \fi
% \else
%   \usebox0
% \fi
% \end{quote}
% If you have a \xfile{docstrip.cfg} that configures and enables \docstrip's
% TDS installing feature, then some files can already be in the right
% place, see the documentation of \docstrip.
%
% \subsection{Refresh file name databases}
%
% If your \TeX~distribution
% (\TeX\,Live, \mikTeX, \dots) relies on file name databases, you must refresh
% these. For example, \TeX\,Live\ users run \verb|texhash| or
% \verb|mktexlsr|.
%
% \subsection{Some details for the interested}
%
% \paragraph{Unpacking with \LaTeX.}
% The \xfile{.dtx} chooses its action depending on the format:
% \begin{description}
% \item[\plainTeX:] Run \docstrip\ and extract the files.
% \item[\LaTeX:] Generate the documentation.
% \end{description}
% If you insist on using \LaTeX\ for \docstrip\ (really,
% \docstrip\ does not need \LaTeX), then inform the autodetect routine
% about your intention:
% \begin{quote}
%   \verb|latex \let\install=y% \iffalse meta-comment
%
% File: letltxmacro.dtx
% Version: 2019/12/03 v1.6
% Info: Let assignment for LaTeX macros
%
% Copyright (C)
%    2008, 2010 Heiko Oberdiek
%    2016-2019 Oberdiek Package Support Group
%    https://github.com/ho-tex/letltxmacro/issues
%
% This work may be distributed and/or modified under the
% conditions of the LaTeX Project Public License, either
% version 1.3c of this license or (at your option) any later
% version. This version of this license is in
%    https://www.latex-project.org/lppl/lppl-1-3c.txt
% and the latest version of this license is in
%    https://www.latex-project.org/lppl.txt
% and version 1.3 or later is part of all distributions of
% LaTeX version 2005/12/01 or later.
%
% This work has the LPPL maintenance status "maintained".
%
% The Current Maintainers of this work are
% Heiko Oberdiek and the Oberdiek Package Support Group
% https://github.com/ho-tex/letltxmacro/issues
%
% This work consists of the main source file letltxmacro.dtx
% and the derived files
%    letltxmacro.sty, letltxmacro.pdf, letltxmacro.ins, letltxmacro.drv,
%    letltxmacro-showcases.tex, letltxmacro-test1.tex,
%    letltxmacro-test2.tex.
%
% Distribution:
%    CTAN:macros/latex/contrib/letltxmacro/letltxmacro.dtx
%    CTAN:macros/latex/contrib/letltxmacro/letltxmacro.pdf
%
% Unpacking:
%    (a) If letltxmacro.ins is present:
%           tex letltxmacro.ins
%    (b) Without letltxmacro.ins:
%           tex letltxmacro.dtx
%    (c) If you insist on using LaTeX
%           latex \let\install=y\input{letltxmacro.dtx}
%        (quote the arguments according to the demands of your shell)
%
% Documentation:
%    (a) If letltxmacro.drv is present:
%           latex letltxmacro.drv
%    (b) Without letltxmacro.drv:
%           latex letltxmacro.dtx; ...
%    The class ltxdoc loads the configuration file ltxdoc.cfg
%    if available. Here you can specify further options, e.g.
%    use A4 as paper format:
%       \PassOptionsToClass{a4paper}{article}
%
%    Programm calls to get the documentation (example):
%       pdflatex letltxmacro.dtx
%       makeindex -s gind.ist letltxmacro.idx
%       pdflatex letltxmacro.dtx
%       makeindex -s gind.ist letltxmacro.idx
%       pdflatex letltxmacro.dtx
%
% Installation:
%    TDS:tex/latex/letltxmacro/letltxmacro.sty
%    TDS:doc/latex/letltxmacro/letltxmacro.pdf
%    TDS:doc/latex/letltxmacro/letltxmacro-showcases.tex
%    TDS:source/latex/letltxmacro/letltxmacro.dtx
%
%<*ignore>
\begingroup
  \catcode123=1 %
  \catcode125=2 %
  \def\x{LaTeX2e}%
\expandafter\endgroup
\ifcase 0\ifx\install y1\fi\expandafter
         \ifx\csname processbatchFile\endcsname\relax\else1\fi
         \ifx\fmtname\x\else 1\fi\relax
\else\csname fi\endcsname
%</ignore>
%<*install>
\input docstrip.tex
\Msg{************************************************************************}
\Msg{* Installation}
\Msg{* Package: letltxmacro 2019/12/03 v1.6 Let assignment for LaTeX macros (HO)}
\Msg{************************************************************************}

\keepsilent
\askforoverwritefalse

\let\MetaPrefix\relax
\preamble

This is a generated file.

Project: letltxmacro
Version: 2019/12/03 v1.6

Copyright (C)
   2008, 2010 Heiko Oberdiek
   2016-2019 Oberdiek Package Support Group

This work may be distributed and/or modified under the
conditions of the LaTeX Project Public License, either
version 1.3c of this license or (at your option) any later
version. This version of this license is in
   https://www.latex-project.org/lppl/lppl-1-3c.txt
and the latest version of this license is in
   https://www.latex-project.org/lppl.txt
and version 1.3 or later is part of all distributions of
LaTeX version 2005/12/01 or later.

This work has the LPPL maintenance status "maintained".

The Current Maintainers of this work are
Heiko Oberdiek and the Oberdiek Package Support Group
https://github.com/ho-tex/letltxmacro/issues


This work consists of the main source file letltxmacro.dtx
and the derived files
   letltxmacro.sty, letltxmacro.pdf, letltxmacro.ins, letltxmacro.drv,
   letltxmacro-showcases.tex, letltxmacro-test1.tex,
   letltxmacro-test2.tex.

\endpreamble
\let\MetaPrefix\DoubleperCent

\generate{%
  \file{letltxmacro.ins}{\from{letltxmacro.dtx}{install}}%
  \file{letltxmacro.drv}{\from{letltxmacro.dtx}{driver}}%
  \usedir{tex/latex/letltxmacro}%
  \file{letltxmacro.sty}{\from{letltxmacro.dtx}{package}}%
  \usedir{doc/latex/letltxmacro}%
  \file{letltxmacro-showcases.tex}{\from{letltxmacro.dtx}{showcases}}%
%  \usedir{doc/latex/letltxmacro/test}%
%  \file{letltxmacro-test1.tex}{\from{letltxmacro.dtx}{test1}}%
%  \file{letltxmacro-test2.tex}{\from{letltxmacro.dtx}{test2}}%
}

\catcode32=13\relax% active space
\let =\space%
\Msg{************************************************************************}
\Msg{*}
\Msg{* To finish the installation you have to move the following}
\Msg{* file into a directory searched by TeX:}
\Msg{*}
\Msg{*     letltxmacro.sty}
\Msg{*}
\Msg{* To produce the documentation run the file `letltxmacro.drv'}
\Msg{* through LaTeX.}
\Msg{*}
\Msg{* Happy TeXing!}
\Msg{*}
\Msg{************************************************************************}

\endbatchfile
%</install>
%<*ignore>
\fi
%</ignore>
%<*driver>
\NeedsTeXFormat{LaTeX2e}
\ProvidesFile{letltxmacro.drv}%
  [2019/12/03 v1.6 Let assignment for LaTeX macros (HO)]%
\documentclass{ltxdoc}
\usepackage{holtxdoc}[2011/11/22]
\begin{document}
  \DocInput{letltxmacro.dtx}%
\end{document}
%</driver>
% \fi
%
%
%
% \GetFileInfo{letltxmacro.drv}
%
% \title{The \xpackage{letltxmacro} package}
% \date{2019/12/03 v1.6}
% \author{Heiko Oberdiek\thanks
% {Please report any issues at \url{https://github.com/ho-tex/letltxmacro/issues}}}
%
% \maketitle
%
% \begin{abstract}
% \TeX's \cs{let} assignment does not work for \LaTeX\ macros
% with optional arguments or for macros that are defined
% as robust macros by \cs{DeclareRobustCommand}. This package
% defines \cs{LetLtxMacro} that also takes care of the involved
% internal macros.
% \end{abstract}
%
% \tableofcontents
%
% \section{Documentation}
%
% If someone wants to redefine a macro with using the old
% meaning, then one method is \TeX's command \cs{let}:
%\begin{quote}
%\begin{verbatim}
%\newcommand{\Macro}{\typeout{Test Macro}}
%\let\SavedMacro=\Macro
%\renewcommand{\Macro}{%
%  \typeout{Begin}%
%  \SavedMacro
%  \typeout{End}%
%}
%\end{verbatim}
%\end{quote}
% However, this method fails, if \cs{Macro} is defined
% by \cs{DeclareRobustCommand} and/or has an optional argument.
% In both cases \LaTeX\ defines an additional internal macro
% that is forgotten in the simple \cs{let} assignment of
% the example above.
%
% \begin{declcs}{LetLtxMacro} \M{new macro} \M{old macro}
% \end{declcs}
% Macro \cs{LetLtxMacro} behaves similar to \TeX's \cs{let}
% assignment, but it takes care of macros that are
% defined by \cs{DeclareRobustCommand} and/or have optional
% arguments. Example:
%\begin{quote}
%\begin{verbatim}
%\DeclareRobustCommand{\Macro}[1][default]{...}
%\LetLtxMacro{\SavedMacro}{\Macro}
%\end{verbatim}
%\end{quote}
% Then macro \cs{SavedMacro} only uses internal macro names
% that are derived from \cs{SavedMacro}'s macro name. Macro \cs{Macro}
% can now be redefined without affecting \cs{SavedMacro}.
%
% \begin{declcs}{GlobalLetLtxMacro} \M{new macro} \M{old macro}
% \end{declcs}
% Like \cs{LetLtxMacro}, but the \meta{new macro} is defined globally.
% Since version 2019/12/03~v1.4.
%
% \subsection{Supported macro definition commands}
%
% \begin{quote}
%   \begin{tabular}{@{}ll@{}}
%     \cs{newcommand}, \cs{renewcommand} & latex/base\\
%     \cs{newenvironment}, \cs{renewenvironment} & latex/base\\
%     \cs{DeclareRobustCommand}& latex/base\\
%     \cs{newrobustcmd}, \cs{renewrobustcmd} & etoolbox\\
%     \cs{robustify} & etoolbox 2008/06/22 v1.6\\
%   \end{tabular}
% \end{quote}
%
% \StopEventually{
% }
%
% \section{Implementation}
%
% \subsection{Show cases}
%
% \subsubsection{\xfile{letltxmacro-showcases.tex}}
%
%    \begin{macrocode}
%<*showcases>
\NeedsTeXFormat{LaTeX2e}
\makeatletter
%    \end{macrocode}
%    \begin{macro}{\Line}
%    The result is displayed by macro \cs{Line}. The percent symbol
%    at line start allows easy grepping and inserting into the DTX
%    file.
%    \begin{macrocode}
\newcommand*{\Line}[1]{%
  \typeout{\@percentchar#1}%
}
%    \end{macrocode}
%    \end{macro}
%    \begin{macrocode}
\newcommand*{\ShowCmdName}[1]{%
  \@ifundefined{#1}{}{%
    \Line{%
      \space\space(\expandafter\string\csname#1\endcsname) = %
      (\expandafter\meaning\csname#1\endcsname)%
    }%
  }%
}
\newcommand*{\ShowCmds}[1]{%
  \ShowCmdName{#1}%
  \ShowCmdName{#1 }%
  \ShowCmdName{\\#1}%
  \ShowCmdName{\\#1 }%
}
\let\\\@backslashchar
%    \end{macrocode}
%    \begin{macro}{\ShowDef}
%    \begin{macrocode}
\newcommand*{\ShowDef}[2]{%
  \begingroup
    \Line{}%
    \newcommand*{\DefString}{#2}%
    \@onelevel@sanitize\DefString
    \Line{\DefString}%
    #2%
    \ShowCmds{#1}%
  \endgroup
}
%    \end{macrocode}
%    \end{macro}
%    \begin{macrocode}
\typeout{}
\Line{* LaTeX definitions:}
\ShowDef{cmd}{%
  \newcommand{\cmd}[2][default]{}%
}
\ShowDef{cmd}{%
  \DeclareRobustCommand{\cmd}{}%
}
\ShowDef{cmd}{%
  \DeclareRobustCommand{\cmd}[2][default]{}%
}
\typeout{}
%    \end{macrocode}
% The minimal version of package \xpackage{etoolbox} is 2008/06/12 v1.6a
% because it fixes \cs{robustify}.
%    \begin{macrocode}
\RequirePackage{etoolbox}[2008/06/12]%
\Line{}
\Line{* etoolbox's robust definitions:}
\ShowDef{cmd}{%
  \newrobustcmd{\cmd}{}%
}
\ShowDef{cmd}{%
  \newrobustcmd{\cmd}[2][default]{}%
}
\Line{}
\Line{* etoolbox's \string\robustify:}
\ShowDef{cmd}{%
  \newcommand{\cmd}[2][default]{} %
  \robustify{\cmd}%
}
\ShowDef{cmd}{%
  \DeclareRobustCommand{\cmd}{} %
  \robustify{\cmd}%
}
\ShowDef{cmd}{%
  \DeclareRobustCommand{\cmd}[2][default]{} %
  \robustify{\cmd}%
}
\typeout{}
\@@end
%</showcases>
%    \end{macrocode}
%
% \subsubsection{Result}
%
% \begingroup
%   \makeatletter
%   \let\org@verbatim\@verbatim
%   \def\@verbatim{^^A
%     \org@verbatim
%     \catcode`\~=\active
%   }^^A
%   \let~\textvisiblespace
%\begin{verbatim}
%* LaTeX definitions:
%
%\newcommand {\cmd }[2][default]{}
%  (\cmd) = (macro:->\@protected@testopt \cmd \\cmd {default})
%  (\\cmd) = (\long macro:[#1]#2->)
%
%\DeclareRobustCommand {\cmd }{}
%  (\cmd) = (macro:->\protect \cmd~ )
%  (\cmd~) = (\long macro:->)
%
%\DeclareRobustCommand {\cmd }[2][default]{}
%  (\cmd) = (macro:->\protect \cmd~ )
%  (\cmd~) = (macro:->\@protected@testopt \cmd~ \\cmd~ {default})
%  (\\cmd~) = (\long macro:[#1]#2->)
%
%* etoolbox's robust definitions:
%
%\newrobustcmd {\cmd }{}
%  (\cmd) = (\protected\long macro:->)
%
%\newrobustcmd {\cmd }[2][default]{}
%  (\cmd) = (\protected macro:->\@testopt \\cmd {default})
%  (\\cmd) = (\long macro:[#1]#2->)
%
%* etoolbox's \robustify:
%
%\newcommand {\cmd }[2][default]{} \robustify {\cmd }
%  (\cmd) = (\protected macro:->\@protected@testopt \cmd \\cmd {default})
%  (\\cmd) = (\long macro:[#1]#2->)
%
%\DeclareRobustCommand {\cmd }{} \robustify {\cmd }
%  (\cmd) = (\protected macro:->)
%
%\DeclareRobustCommand {\cmd }[2][default]{} \robustify {\cmd }
%  (\cmd) = (\protected macro:->\@protected@testopt \cmd~ \\cmd~ {default})
%  (\cmd~) = (macro:->\@protected@testopt \cmd~ \\cmd~ {default})
%  (\\cmd~) = (\long macro:[#1]#2->)
%\end{verbatim}
% \endgroup
%
% \subsection{Package}
%
%    \begin{macrocode}
%<*package>
%    \end{macrocode}
%
% \subsubsection{Catcodes and identification}
%
%    \begin{macrocode}
\begingroup\catcode61\catcode48\catcode32=10\relax%
  \catcode13=5 % ^^M
  \endlinechar=13 %
  \catcode123=1 % {
  \catcode125=2 % }
  \catcode64=11 % @
  \def\x{\endgroup
    \expandafter\edef\csname llm@AtEnd\endcsname{%
      \endlinechar=\the\endlinechar\relax
      \catcode13=\the\catcode13\relax
      \catcode32=\the\catcode32\relax
      \catcode35=\the\catcode35\relax
      \catcode61=\the\catcode61\relax
      \catcode64=\the\catcode64\relax
      \catcode123=\the\catcode123\relax
      \catcode125=\the\catcode125\relax
    }%
  }%
\x\catcode61\catcode48\catcode32=10\relax%
\catcode13=5 % ^^M
\endlinechar=13 %
\catcode35=6 % #
\catcode64=11 % @
\catcode123=1 % {
\catcode125=2 % }
\def\TMP@EnsureCode#1#2{%
  \edef\llm@AtEnd{%
    \llm@AtEnd
    \catcode#1=\the\catcode#1\relax
  }%
  \catcode#1=#2\relax
}
\TMP@EnsureCode{40}{12}% (
\TMP@EnsureCode{41}{12}% )
\TMP@EnsureCode{42}{12}% *
\TMP@EnsureCode{45}{12}% -
\TMP@EnsureCode{46}{12}% .
\TMP@EnsureCode{47}{12}% /
\TMP@EnsureCode{58}{12}% :
\TMP@EnsureCode{62}{12}% >
\TMP@EnsureCode{91}{12}% [
\TMP@EnsureCode{93}{12}% ]
\edef\llm@AtEnd{%
  \llm@AtEnd
  \escapechar\the\escapechar\relax
  \noexpand\endinput
}
\escapechar=92 % `\\
%    \end{macrocode}
%
%    Package identification.
%    \begin{macrocode}
\NeedsTeXFormat{LaTeX2e}
\ProvidesPackage{letltxmacro}%
  [2019/12/03 v1.6 Let assignment for LaTeX macros (HO)]
%    \end{macrocode}
%
% \subsubsection{Main macros}
%
%    \begin{macro}{\LetLtxMacro}
%    \begin{macrocode}
\newcommand*{\LetLtxMacro}{%
  \llm@ModeLetLtxMacro{}%
}
%    \end{macrocode}
%    \end{macro}
%    \begin{macro}{\GlobalLetLtxMacro}
%    \begin{macrocode}
\newcommand*{\GlobalLetLtxMacro}{%
  \llm@ModeLetLtxMacro\global
}
%    \end{macrocode}
%    \end{macro}
%
%    \begin{macro}{\llm@ModeLetLtxMacro}
%    \begin{macrocode}
\newcommand*{\llm@ModeLetLtxMacro}[3]{%
  \edef\llm@escapechar{\the\escapechar}%
  \escapechar=-1 %
  \edef\reserved@a{%
    \noexpand\protect
    \expandafter\noexpand
    \csname\string#3 \endcsname
  }%
  \ifx\reserved@a#3\relax
    #1\edef#2{%
      \noexpand\protect
      \expandafter\noexpand
      \csname\string#2 \endcsname
    }%
    #1\expandafter\let
    \csname\string#2 \expandafter\endcsname
    \csname\string#3 \endcsname
    \expandafter\llm@LetLtxMacro
        \csname\string#2 \expandafter\endcsname
        \csname\string#3 \endcsname{#1}%
  \else
    \llm@LetLtxMacro{#2}{#3}{#1}%
  \fi
  \escapechar=\llm@escapechar\relax
}
%    \end{macrocode}
%    \end{macro}
%    \begin{macro}{\llm@LetLtxMacro}
%    \begin{macrocode}
\def\llm@LetLtxMacro#1#2#3{%
  \escapechar=92 %
  \expandafter\llm@CheckParams\meaning#2:->\@nil{%
    \begingroup
      \def\@protected@testopt{%
        \expandafter\@testopt\@gobble
      }%
      \def\@testopt##1##2{%
        \toks@={##2}%
      }%
      \let\llm@testopt\@empty
      \edef\x{%
        \noexpand\@protected@testopt
        \noexpand#2%
        \expandafter\noexpand\csname\string#2\endcsname
      }%
      \expandafter\expandafter\expandafter\def
      \expandafter\expandafter\expandafter\y
      \expandafter\expandafter\expandafter{%
        \expandafter\llm@CarThree#2{}{}{}\llm@nil
      }%
      \ifx\x\y
        #2%
        \def\llm@testopt{%
          \noexpand\@protected@testopt
          \noexpand#1%
        }%
      \else
        \edef\x{%
          \noexpand\@testopt
          \expandafter\noexpand
          \csname\string#2\endcsname
        }%
        \expandafter\expandafter\expandafter\def
        \expandafter\expandafter\expandafter\y
        \expandafter\expandafter\expandafter{%
          \expandafter\llm@CarTwo#2{}{}\llm@nil
        }%
        \ifx\x\y
          #2%
          \def\llm@testopt{%
            \noexpand\@testopt
          }%
        \fi
      \fi
      \ifx\llm@testopt\@empty
      \else
        \llm@protected\xdef\llm@GlobalTemp{%
          \llm@testopt
          \expandafter\noexpand
          \csname\string#1\endcsname
          {\the\toks@}%
        }%
      \fi
    \expandafter\endgroup\ifx\llm@testopt\@empty
      #3\let#1=#2\relax
    \else
      #3\let#1=\llm@GlobalTemp
      #3\expandafter\let
          \csname\string#1\expandafter\endcsname
          \csname\string#2\endcsname
    \fi
  }{%
    #3\let#1=#2\relax
  }%
}
%    \end{macrocode}
%    \end{macro}
%    \begin{macro}{\llm@CheckParams}
%    \begin{macrocode}
\def\llm@CheckParams#1:->#2\@nil{%
  \begingroup
    \def\x{#1}%
  \ifx\x\llm@macro
    \endgroup
    \def\llm@protected{}%
    \expandafter\@firstoftwo
  \else
    \ifx\x\llm@protectedmacro
      \endgroup
      \def\llm@protected{\protected}%
      \expandafter\expandafter\expandafter\@firstoftwo
    \else
      \endgroup
      \expandafter\expandafter\expandafter\@secondoftwo
    \fi
  \fi
}
%    \end{macrocode}
%    \end{macro}
%    \begin{macro}{\llm@macro}
%    \begin{macrocode}
\def\llm@macro{macro}
\@onelevel@sanitize\llm@macro
%    \end{macrocode}
%    \end{macro}
%    \begin{macro}{\llm@protectedmacro}
%    \begin{macrocode}
\def\llm@protectedmacro{\protected macro}
\@onelevel@sanitize\llm@protectedmacro
%    \end{macrocode}
%    \end{macro}
%    \begin{macro}{\llm@CarThree}
%    \begin{macrocode}
\def\llm@CarThree#1#2#3#4\llm@nil{#1#2#3}%
%    \end{macrocode}
%    \end{macro}
%    \begin{macro}{\llm@CarTwo}
%    \begin{macrocode}
\def\llm@CarTwo#1#2#3\llm@nil{#1#2}%
%    \end{macrocode}
%    \end{macro}
%
%    \begin{macrocode}
\llm@AtEnd%
%</package>
%    \end{macrocode}
% \section{Installation}
%
% \subsection{Download}
%
% \paragraph{Package.} This package is available on
% CTAN\footnote{\CTANpkg{letltxmacro}}:
% \begin{description}
% \item[\CTAN{macros/latex/contrib/letltxmacro/letltxmacro.dtx}] The source file.
% \item[\CTAN{macros/latex/contrib/letltxmacro/letltxmacro.pdf}] Documentation.
% \end{description}
%
%
% \paragraph{Bundle.} All the packages of the bundle `letltxmacro'
% are also available in a TDS compliant ZIP archive. There
% the packages are already unpacked and the documentation files
% are generated. The files and directories obey the TDS standard.
% \begin{description}
% \item[\CTANinstall{install/macros/latex/contrib/letltxmacro.tds.zip}]
% \end{description}
% \emph{TDS} refers to the standard ``A Directory Structure
% for \TeX\ Files'' (\CTANpkg{tds}). Directories
% with \xfile{texmf} in their name are usually organized this way.
%
% \subsection{Bundle installation}
%
% \paragraph{Unpacking.} Unpack the \xfile{letltxmacro.tds.zip} in the
% TDS tree (also known as \xfile{texmf} tree) of your choice.
% Example (linux):
% \begin{quote}
%   |unzip letltxmacro.tds.zip -d ~/texmf|
% \end{quote}
%
% \subsection{Package installation}
%
% \paragraph{Unpacking.} The \xfile{.dtx} file is a self-extracting
% \docstrip\ archive. The files are extracted by running the
% \xfile{.dtx} through \plainTeX:
% \begin{quote}
%   \verb|tex letltxmacro.dtx|
% \end{quote}
%
% \paragraph{TDS.} Now the different files must be moved into
% the different directories in your installation TDS tree
% (also known as \xfile{texmf} tree):
% \begin{quote}
% \def\t{^^A
% \begin{tabular}{@{}>{\ttfamily}l@{ $\rightarrow$ }>{\ttfamily}l@{}}
%   letltxmacro.sty & tex/latex/letltxmacro/letltxmacro.sty\\
%   letltxmacro.pdf & doc/latex/letltxmacro/letltxmacro.pdf\\
%   letltxmacro-showcases.tex & doc/latex/letltxmacro/letltxmacro-showcases.tex\\
%   letltxmacro.dtx & source/latex/letltxmacro/letltxmacro.dtx\\
% \end{tabular}^^A
% }^^A
% \sbox0{\t}^^A
% \ifdim\wd0>\linewidth
%   \begingroup
%     \advance\linewidth by\leftmargin
%     \advance\linewidth by\rightmargin
%   \edef\x{\endgroup
%     \def\noexpand\lw{\the\linewidth}^^A
%   }\x
%   \def\lwbox{^^A
%     \leavevmode
%     \hbox to \linewidth{^^A
%       \kern-\leftmargin\relax
%       \hss
%       \usebox0
%       \hss
%       \kern-\rightmargin\relax
%     }^^A
%   }^^A
%   \ifdim\wd0>\lw
%     \sbox0{\small\t}^^A
%     \ifdim\wd0>\linewidth
%       \ifdim\wd0>\lw
%         \sbox0{\footnotesize\t}^^A
%         \ifdim\wd0>\linewidth
%           \ifdim\wd0>\lw
%             \sbox0{\scriptsize\t}^^A
%             \ifdim\wd0>\linewidth
%               \ifdim\wd0>\lw
%                 \sbox0{\tiny\t}^^A
%                 \ifdim\wd0>\linewidth
%                   \lwbox
%                 \else
%                   \usebox0
%                 \fi
%               \else
%                 \lwbox
%               \fi
%             \else
%               \usebox0
%             \fi
%           \else
%             \lwbox
%           \fi
%         \else
%           \usebox0
%         \fi
%       \else
%         \lwbox
%       \fi
%     \else
%       \usebox0
%     \fi
%   \else
%     \lwbox
%   \fi
% \else
%   \usebox0
% \fi
% \end{quote}
% If you have a \xfile{docstrip.cfg} that configures and enables \docstrip's
% TDS installing feature, then some files can already be in the right
% place, see the documentation of \docstrip.
%
% \subsection{Refresh file name databases}
%
% If your \TeX~distribution
% (\TeX\,Live, \mikTeX, \dots) relies on file name databases, you must refresh
% these. For example, \TeX\,Live\ users run \verb|texhash| or
% \verb|mktexlsr|.
%
% \subsection{Some details for the interested}
%
% \paragraph{Unpacking with \LaTeX.}
% The \xfile{.dtx} chooses its action depending on the format:
% \begin{description}
% \item[\plainTeX:] Run \docstrip\ and extract the files.
% \item[\LaTeX:] Generate the documentation.
% \end{description}
% If you insist on using \LaTeX\ for \docstrip\ (really,
% \docstrip\ does not need \LaTeX), then inform the autodetect routine
% about your intention:
% \begin{quote}
%   \verb|latex \let\install=y\input{letltxmacro.dtx}|
% \end{quote}
% Do not forget to quote the argument according to the demands
% of your shell.
%
% \paragraph{Generating the documentation.}
% You can use both the \xfile{.dtx} or the \xfile{.drv} to generate
% the documentation. The process can be configured by the
% configuration file \xfile{ltxdoc.cfg}. For instance, put this
% line into this file, if you want to have A4 as paper format:
% \begin{quote}
%   \verb|\PassOptionsToClass{a4paper}{article}|
% \end{quote}
% An example follows how to generate the
% documentation with pdf\LaTeX:
% \begin{quote}
%\begin{verbatim}
%pdflatex letltxmacro.dtx
%makeindex -s gind.ist letltxmacro.idx
%pdflatex letltxmacro.dtx
%makeindex -s gind.ist letltxmacro.idx
%pdflatex letltxmacro.dtx
%\end{verbatim}
% \end{quote}
%
% \begin{History}
%   \begin{Version}{2008/06/09 v1.0}
%   \item
%     First version.
%   \end{Version}
%   \begin{Version}{2008/06/12 v1.1}
%   \item
%     Support for \xpackage{etoolbox}'s \cs{newrobustcmd} added.
%   \end{Version}
%   \begin{Version}{2008/06/13 v1.2}
%   \item
%     Support for \xpackage{etoolbox}'s \cs{robustify} added.
%   \end{Version}
%   \begin{Version}{2008/06/24 v1.3}
%   \item
%     Test file adapted for etoolbox 2008/06/22 v1.6.
%   \end{Version}
%   \begin{Version}{2010/09/02 v1.4}
%   \item
%     \cs{GlobalLetLtxMacro} added.
%   \end{Version}
%   \begin{Version}{2016/05/16 v1.5}
%   \item
%     Documentation updates.
%   \end{Version}
%   \begin{Version}{2019/12/03 v1.6}
%   \item
%     Documentation updates.
%   \end{Version}
% \end{History}
%
% \PrintIndex
%
% \Finale
\endinput
|
% \end{quote}
% Do not forget to quote the argument according to the demands
% of your shell.
%
% \paragraph{Generating the documentation.}
% You can use both the \xfile{.dtx} or the \xfile{.drv} to generate
% the documentation. The process can be configured by the
% configuration file \xfile{ltxdoc.cfg}. For instance, put this
% line into this file, if you want to have A4 as paper format:
% \begin{quote}
%   \verb|\PassOptionsToClass{a4paper}{article}|
% \end{quote}
% An example follows how to generate the
% documentation with pdf\LaTeX:
% \begin{quote}
%\begin{verbatim}
%pdflatex letltxmacro.dtx
%makeindex -s gind.ist letltxmacro.idx
%pdflatex letltxmacro.dtx
%makeindex -s gind.ist letltxmacro.idx
%pdflatex letltxmacro.dtx
%\end{verbatim}
% \end{quote}
%
% \begin{History}
%   \begin{Version}{2008/06/09 v1.0}
%   \item
%     First version.
%   \end{Version}
%   \begin{Version}{2008/06/12 v1.1}
%   \item
%     Support for \xpackage{etoolbox}'s \cs{newrobustcmd} added.
%   \end{Version}
%   \begin{Version}{2008/06/13 v1.2}
%   \item
%     Support for \xpackage{etoolbox}'s \cs{robustify} added.
%   \end{Version}
%   \begin{Version}{2008/06/24 v1.3}
%   \item
%     Test file adapted for etoolbox 2008/06/22 v1.6.
%   \end{Version}
%   \begin{Version}{2010/09/02 v1.4}
%   \item
%     \cs{GlobalLetLtxMacro} added.
%   \end{Version}
%   \begin{Version}{2016/05/16 v1.5}
%   \item
%     Documentation updates.
%   \end{Version}
%   \begin{Version}{2019/12/03 v1.6}
%   \item
%     Documentation updates.
%   \end{Version}
% \end{History}
%
% \PrintIndex
%
% \Finale
\endinput
|
% \end{quote}
% Do not forget to quote the argument according to the demands
% of your shell.
%
% \paragraph{Generating the documentation.}
% You can use both the \xfile{.dtx} or the \xfile{.drv} to generate
% the documentation. The process can be configured by the
% configuration file \xfile{ltxdoc.cfg}. For instance, put this
% line into this file, if you want to have A4 as paper format:
% \begin{quote}
%   \verb|\PassOptionsToClass{a4paper}{article}|
% \end{quote}
% An example follows how to generate the
% documentation with pdf\LaTeX:
% \begin{quote}
%\begin{verbatim}
%pdflatex letltxmacro.dtx
%makeindex -s gind.ist letltxmacro.idx
%pdflatex letltxmacro.dtx
%makeindex -s gind.ist letltxmacro.idx
%pdflatex letltxmacro.dtx
%\end{verbatim}
% \end{quote}
%
% \begin{History}
%   \begin{Version}{2008/06/09 v1.0}
%   \item
%     First version.
%   \end{Version}
%   \begin{Version}{2008/06/12 v1.1}
%   \item
%     Support for \xpackage{etoolbox}'s \cs{newrobustcmd} added.
%   \end{Version}
%   \begin{Version}{2008/06/13 v1.2}
%   \item
%     Support for \xpackage{etoolbox}'s \cs{robustify} added.
%   \end{Version}
%   \begin{Version}{2008/06/24 v1.3}
%   \item
%     Test file adapted for etoolbox 2008/06/22 v1.6.
%   \end{Version}
%   \begin{Version}{2010/09/02 v1.4}
%   \item
%     \cs{GlobalLetLtxMacro} added.
%   \end{Version}
%   \begin{Version}{2016/05/16 v1.5}
%   \item
%     Documentation updates.
%   \end{Version}
%   \begin{Version}{2019/12/03 v1.6}
%   \item
%     Documentation updates.
%   \end{Version}
% \end{History}
%
% \PrintIndex
%
% \Finale
\endinput
|
% \end{quote}
% Do not forget to quote the argument according to the demands
% of your shell.
%
% \paragraph{Generating the documentation.}
% You can use both the \xfile{.dtx} or the \xfile{.drv} to generate
% the documentation. The process can be configured by the
% configuration file \xfile{ltxdoc.cfg}. For instance, put this
% line into this file, if you want to have A4 as paper format:
% \begin{quote}
%   \verb|\PassOptionsToClass{a4paper}{article}|
% \end{quote}
% An example follows how to generate the
% documentation with pdf\LaTeX:
% \begin{quote}
%\begin{verbatim}
%pdflatex letltxmacro.dtx
%makeindex -s gind.ist letltxmacro.idx
%pdflatex letltxmacro.dtx
%makeindex -s gind.ist letltxmacro.idx
%pdflatex letltxmacro.dtx
%\end{verbatim}
% \end{quote}
%
% \begin{History}
%   \begin{Version}{2008/06/09 v1.0}
%   \item
%     First version.
%   \end{Version}
%   \begin{Version}{2008/06/12 v1.1}
%   \item
%     Support for \xpackage{etoolbox}'s \cs{newrobustcmd} added.
%   \end{Version}
%   \begin{Version}{2008/06/13 v1.2}
%   \item
%     Support for \xpackage{etoolbox}'s \cs{robustify} added.
%   \end{Version}
%   \begin{Version}{2008/06/24 v1.3}
%   \item
%     Test file adapted for etoolbox 2008/06/22 v1.6.
%   \end{Version}
%   \begin{Version}{2010/09/02 v1.4}
%   \item
%     \cs{GlobalLetLtxMacro} added.
%   \end{Version}
%   \begin{Version}{2016/05/16 v1.5}
%   \item
%     Documentation updates.
%   \end{Version}
%   \begin{Version}{2019/12/03 v1.6}
%   \item
%     Documentation updates.
%   \end{Version}
% \end{History}
%
% \PrintIndex
%
% \Finale
\endinput
