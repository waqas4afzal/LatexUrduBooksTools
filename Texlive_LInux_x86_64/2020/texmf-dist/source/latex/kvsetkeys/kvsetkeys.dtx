% \iffalse meta-comment
%
% File: kvsetkeys.dtx
% Version: 2019/12/15 v1.18
% Info: Key value parser
%
% Copyright (C)
%    2006, 2007, 2009-2012 Heiko Oberdiek
%    2016-2019 Oberdiek Package Support Group
%    https://github.com/ho-tex/kvsetkeys/issues
%
% This work may be distributed and/or modified under the
% conditions of the LaTeX Project Public License, either
% version 1.3c of this license or (at your option) any later
% version. This version of this license is in
%    https://www.latex-project.org/lppl/lppl-1-3c.txt
% and the latest version of this license is in
%    https://www.latex-project.org/lppl.txt
% and version 1.3 or later is part of all distributions of
% LaTeX version 2005/12/01 or later.
%
% This work has the LPPL maintenance status "maintained".
%
% The Current Maintainers of this work are
% Heiko Oberdiek and the Oberdiek Package Support Group
% https://github.com/ho-tex/kvsetkeys/issues
%
% The Base Interpreter refers to any `TeX-Format',
% because some files are installed in TDS:tex/generic//.
%
% This work consists of the main source file kvsetkeys.dtx
% and the derived files
%    kvsetkeys.sty, kvsetkeys.pdf, kvsetkeys.ins, kvsetkeys.drv,
%    kvsetkeys-example.tex, kvsetkeys-test1.tex,
%    kvsetkeys-test2.tex, kvsetkeys-test3.tex,
%    kvsetkeys-test4.tex.
%
% Distribution:
%    CTAN:macros/latex/contrib/kvsetkeys/kvsetkeys.dtx
%    CTAN:macros/latex/contrib/kvsetkeys/kvsetkeys.pdf
%
% Unpacking:
%    (a) If kvsetkeys.ins is present:
%           tex kvsetkeys.ins
%    (b) Without kvsetkeys.ins:
%           tex kvsetkeys.dtx
%    (c) If you insist on using LaTeX
%           latex \let\install=y% \iffalse meta-comment
%
% File: kvsetkeys.dtx
% Version: 2019/12/15 v1.18
% Info: Key value parser
%
% Copyright (C)
%    2006, 2007, 2009-2012 Heiko Oberdiek
%    2016-2019 Oberdiek Package Support Group
%    https://github.com/ho-tex/kvsetkeys/issues
%
% This work may be distributed and/or modified under the
% conditions of the LaTeX Project Public License, either
% version 1.3c of this license or (at your option) any later
% version. This version of this license is in
%    https://www.latex-project.org/lppl/lppl-1-3c.txt
% and the latest version of this license is in
%    https://www.latex-project.org/lppl.txt
% and version 1.3 or later is part of all distributions of
% LaTeX version 2005/12/01 or later.
%
% This work has the LPPL maintenance status "maintained".
%
% The Current Maintainers of this work are
% Heiko Oberdiek and the Oberdiek Package Support Group
% https://github.com/ho-tex/kvsetkeys/issues
%
% The Base Interpreter refers to any `TeX-Format',
% because some files are installed in TDS:tex/generic//.
%
% This work consists of the main source file kvsetkeys.dtx
% and the derived files
%    kvsetkeys.sty, kvsetkeys.pdf, kvsetkeys.ins, kvsetkeys.drv,
%    kvsetkeys-example.tex, kvsetkeys-test1.tex,
%    kvsetkeys-test2.tex, kvsetkeys-test3.tex,
%    kvsetkeys-test4.tex.
%
% Distribution:
%    CTAN:macros/latex/contrib/kvsetkeys/kvsetkeys.dtx
%    CTAN:macros/latex/contrib/kvsetkeys/kvsetkeys.pdf
%
% Unpacking:
%    (a) If kvsetkeys.ins is present:
%           tex kvsetkeys.ins
%    (b) Without kvsetkeys.ins:
%           tex kvsetkeys.dtx
%    (c) If you insist on using LaTeX
%           latex \let\install=y% \iffalse meta-comment
%
% File: kvsetkeys.dtx
% Version: 2019/12/15 v1.18
% Info: Key value parser
%
% Copyright (C)
%    2006, 2007, 2009-2012 Heiko Oberdiek
%    2016-2019 Oberdiek Package Support Group
%    https://github.com/ho-tex/kvsetkeys/issues
%
% This work may be distributed and/or modified under the
% conditions of the LaTeX Project Public License, either
% version 1.3c of this license or (at your option) any later
% version. This version of this license is in
%    https://www.latex-project.org/lppl/lppl-1-3c.txt
% and the latest version of this license is in
%    https://www.latex-project.org/lppl.txt
% and version 1.3 or later is part of all distributions of
% LaTeX version 2005/12/01 or later.
%
% This work has the LPPL maintenance status "maintained".
%
% The Current Maintainers of this work are
% Heiko Oberdiek and the Oberdiek Package Support Group
% https://github.com/ho-tex/kvsetkeys/issues
%
% The Base Interpreter refers to any `TeX-Format',
% because some files are installed in TDS:tex/generic//.
%
% This work consists of the main source file kvsetkeys.dtx
% and the derived files
%    kvsetkeys.sty, kvsetkeys.pdf, kvsetkeys.ins, kvsetkeys.drv,
%    kvsetkeys-example.tex, kvsetkeys-test1.tex,
%    kvsetkeys-test2.tex, kvsetkeys-test3.tex,
%    kvsetkeys-test4.tex.
%
% Distribution:
%    CTAN:macros/latex/contrib/kvsetkeys/kvsetkeys.dtx
%    CTAN:macros/latex/contrib/kvsetkeys/kvsetkeys.pdf
%
% Unpacking:
%    (a) If kvsetkeys.ins is present:
%           tex kvsetkeys.ins
%    (b) Without kvsetkeys.ins:
%           tex kvsetkeys.dtx
%    (c) If you insist on using LaTeX
%           latex \let\install=y% \iffalse meta-comment
%
% File: kvsetkeys.dtx
% Version: 2019/12/15 v1.18
% Info: Key value parser
%
% Copyright (C)
%    2006, 2007, 2009-2012 Heiko Oberdiek
%    2016-2019 Oberdiek Package Support Group
%    https://github.com/ho-tex/kvsetkeys/issues
%
% This work may be distributed and/or modified under the
% conditions of the LaTeX Project Public License, either
% version 1.3c of this license or (at your option) any later
% version. This version of this license is in
%    https://www.latex-project.org/lppl/lppl-1-3c.txt
% and the latest version of this license is in
%    https://www.latex-project.org/lppl.txt
% and version 1.3 or later is part of all distributions of
% LaTeX version 2005/12/01 or later.
%
% This work has the LPPL maintenance status "maintained".
%
% The Current Maintainers of this work are
% Heiko Oberdiek and the Oberdiek Package Support Group
% https://github.com/ho-tex/kvsetkeys/issues
%
% The Base Interpreter refers to any `TeX-Format',
% because some files are installed in TDS:tex/generic//.
%
% This work consists of the main source file kvsetkeys.dtx
% and the derived files
%    kvsetkeys.sty, kvsetkeys.pdf, kvsetkeys.ins, kvsetkeys.drv,
%    kvsetkeys-example.tex, kvsetkeys-test1.tex,
%    kvsetkeys-test2.tex, kvsetkeys-test3.tex,
%    kvsetkeys-test4.tex.
%
% Distribution:
%    CTAN:macros/latex/contrib/kvsetkeys/kvsetkeys.dtx
%    CTAN:macros/latex/contrib/kvsetkeys/kvsetkeys.pdf
%
% Unpacking:
%    (a) If kvsetkeys.ins is present:
%           tex kvsetkeys.ins
%    (b) Without kvsetkeys.ins:
%           tex kvsetkeys.dtx
%    (c) If you insist on using LaTeX
%           latex \let\install=y\input{kvsetkeys.dtx}
%        (quote the arguments according to the demands of your shell)
%
% Documentation:
%    (a) If kvsetkeys.drv is present:
%           latex kvsetkeys.drv
%    (b) Without kvsetkeys.drv:
%           latex kvsetkeys.dtx; ...
%    The class ltxdoc loads the configuration file ltxdoc.cfg
%    if available. Here you can specify further options, e.g.
%    use A4 as paper format:
%       \PassOptionsToClass{a4paper}{article}
%
%    Programm calls to get the documentation (example):
%       pdflatex kvsetkeys.dtx
%       makeindex -s gind.ist kvsetkeys.idx
%       pdflatex kvsetkeys.dtx
%       makeindex -s gind.ist kvsetkeys.idx
%       pdflatex kvsetkeys.dtx
%
% Installation:
%    TDS:tex/generic/kvsetkeys/kvsetkeys.sty
%    TDS:doc/latex/kvsetkeys/kvsetkeys.pdf
%    TDS:doc/latex/kvsetkeys/kvsetkeys-example.tex
%    TDS:source/latex/kvsetkeys/kvsetkeys.dtx
%
%<*ignore>
\begingroup
  \catcode123=1 %
  \catcode125=2 %
  \def\x{LaTeX2e}%
\expandafter\endgroup
\ifcase 0\ifx\install y1\fi\expandafter
         \ifx\csname processbatchFile\endcsname\relax\else1\fi
         \ifx\fmtname\x\else 1\fi\relax
\else\csname fi\endcsname
%</ignore>
%<*install>
\input docstrip.tex
\Msg{************************************************************************}
\Msg{* Installation}
\Msg{* Package: kvsetkeys 2019/12/15 v1.18 Key value parser (HO)}
\Msg{************************************************************************}

\keepsilent
\askforoverwritefalse

\let\MetaPrefix\relax
\preamble

This is a generated file.

Project: kvsetkeys
Version: 2019/12/15 v1.18

Copyright (C)
   2006, 2007, 2009-2012 Heiko Oberdiek
   2016-2019 Oberdiek Package Support Group

This work may be distributed and/or modified under the
conditions of the LaTeX Project Public License, either
version 1.3c of this license or (at your option) any later
version. This version of this license is in
   https://www.latex-project.org/lppl/lppl-1-3c.txt
and the latest version of this license is in
   https://www.latex-project.org/lppl.txt
and version 1.3 or later is part of all distributions of
LaTeX version 2005/12/01 or later.

This work has the LPPL maintenance status "maintained".

The Current Maintainers of this work are
Heiko Oberdiek and the Oberdiek Package Support Group
https://github.com/ho-tex/kvsetkeys/issues


The Base Interpreter refers to any `TeX-Format',
because some files are installed in TDS:tex/generic//.

This work consists of the main source file kvsetkeys.dtx
and the derived files
   kvsetkeys.sty, kvsetkeys.pdf, kvsetkeys.ins, kvsetkeys.drv,
   kvsetkeys-example.tex, kvsetkeys-test1.tex,
   kvsetkeys-test2.tex, kvsetkeys-test3.tex,
   kvsetkeys-test4.tex.

\endpreamble
\let\MetaPrefix\DoubleperCent

\generate{%
  \file{kvsetkeys.ins}{\from{kvsetkeys.dtx}{install}}%
  \file{kvsetkeys.drv}{\from{kvsetkeys.dtx}{driver}}%
  \usedir{tex/generic/kvsetkeys}%
  \file{kvsetkeys.sty}{\from{kvsetkeys.dtx}{package}}%
  \usedir{doc/latex/kvsetkeys}%
  \file{kvsetkeys-example.tex}{\from{kvsetkeys.dtx}{example}}%
}

\catcode32=13\relax% active space
\let =\space%
\Msg{************************************************************************}
\Msg{*}
\Msg{* To finish the installation you have to move the following}
\Msg{* file into a directory searched by TeX:}
\Msg{*}
\Msg{*     kvsetkeys.sty}
\Msg{*}
\Msg{* To produce the documentation run the file `kvsetkeys.drv'}
\Msg{* through LaTeX.}
\Msg{*}
\Msg{* Happy TeXing!}
\Msg{*}
\Msg{************************************************************************}

\endbatchfile
%</install>
%<*ignore>
\fi
%</ignore>
%<*driver>
\NeedsTeXFormat{LaTeX2e}
\ProvidesFile{kvsetkeys.drv}%
  [2019/12/15 v1.18 Key value parser (HO)]%
\documentclass{ltxdoc}
\usepackage{holtxdoc}[2011/11/22]
\begin{document}
  \DocInput{kvsetkeys.dtx}%
\end{document}
%</driver>
% \fi
%
%
%
% \GetFileInfo{kvsetkeys.drv}
%
% \title{The \xpackage{kvsetkeys} package}
% \date{2019/12/15 v1.18}
% \author{Heiko Oberdiek\thanks
% {Please report any issues at \url{https://github.com/ho-tex/kvsetkeys/issues}}}
%
% \maketitle
%
% \begin{abstract}
% Package \xpackage{kvsetkeys} provides \cs{kvsetkeys}, a variant
% of package \xpackage{keyval}'s \cs{setkeys}. It allows to specify
% a handler that deals with unknown options. Active commas and equal
% signs may be used (e.g. see \xpackage{babel}'s shorthands) and
% only one level of curly braces is removed from the values.
% \end{abstract}
%
% \tableofcontents
%
% \def\M#1{\texttt{\{}\meta{#1}\texttt{\}}}
%
% \section{Documentation}
%
% First I want to recommend the very good review article
% ``A guide to key-value methods'' by Joseph Wright \cite{tb94wright}.
% It introduces the different key-value packages and compares them.
%
% \subsection{Motivation}
%
% \cs{kvsetkeys} serves as replacement for \xpackage{keyval}'s
% \cs{setkeys}. It basically uses the same syntax. But the
% implementation is more robust and predictable:
% \begin{description}
% \item[Active syntax characters:]
%   Comma `|,|' and the equals sign `|=|' are used inside
%   key value lists as syntax characters. Package \xpackage{keyval}
%   uses the catcode of the characters that is active during
%   package loading, usually this is catcode 12 (other).
%   But it can happen that the catcode setting of the syntax characters
%   changes. Especially active characters are of interest, because
%   some language adaptations uses them. For example, option \xoption{turkish}
%   of package \xpackage{babel} uses the equals sign as active shorthand
%   character. Therefore package \xpackage{kvsetkeys} deals with
%   both catcode settings 12 (other) and 13 (active).
% \item[Brace removal:]
%   Package \xpackage{keyval}'s \cs{setkeys} removes up to two
%   levels of curly braces around the value in some unpredictable way:
%\begin{quote}
%|\setkeys{fam}{key={{value}}} || || | $\rightarrow$ |value|\\
%|\setkeys{fam}{key={{{value}}}} | $\rightarrow$ |{value}|\\
%|\setkeys{fam}{key= {{{value}}}}| $\rightarrow$ |{{value}}|
%\end{quote}
%   This package \xpackage{kvsetkeys} follows a much stronger rule:
%   Exactly one level of braces are removed from an item, if the
%   item is surrounded by curly braces. An item can be a
%   the key value pair, the key or the value.
%\begin{quote}
%|\kvsetkeys{fam}{key={value}} || | $\rightarrow$ |value|\\
%|\kvsetkeys{fam}{key={{value}} | $\rightarrow$ |{value}|\\
%|\kvsetkeys{fam}{key= {{value}}| $\rightarrow$ |{value}|
%\end{quote}
% \item[Arbitrary values:] Unmatched conditionals are supported.
% \end{description}
%
% Before I describe
% \cs{kvsetkeys} in more detail, first I want to explain, how
% this package deals with key value lists. For the package also
% provides low level interfaces that can be used by package authors.
%
% \subsection{Normalizing key value lists}
%
% \begin{declcs}{kv@normalize}\,\M{key value list}
% \end{declcs}
% If the user specifies key value lists, he usually prefers
% nice formatted source code, e.g.:
% \begin{quote}
%\begin{verbatim}
%\hypersetup{
%  pdftitle    = {...},
%  pdfsubject  = {...},
%  pdfauthor   = {...},
%  pdfkeywords = {...},
%  ...
%}
%\end{verbatim}
% \end{quote}
% Thus there can be spaces around keys, around |=| or around the value.
% Also empty entries are possible by too many commas. Therefore these
% spaces and empty entries are silently removed by package \xpackage{keyval}
% and this package. Whereas the contents of the value can be protected
% by curly braces, especially if spaces or commas are used inside,
% a key name must not use spaces or other syntax characters.
%
% \cs{kv@normalize} takes a key value list and performs the cleanup:
% \begin{itemize}
% \item Spaces are removed.
% \item Syntax characters (comma and equal sign) that are active
%   are replaced by the same characters with standard catcode.
%   (Example: \xpackage{babel}'s language option \xoption{turkish}
%   uses the equal sign as active shorthand character.)
% \end{itemize}
% The result is stored in \cs{kv@list}, e.g.:
% \begin{quote}
% |\kv@list| $\rightarrow$ |,pdftitle={...},pdfsubject={...},...,|
% \end{quote}
% Curly braces around values (or keys) remain untouched.
% \begin{description}
% \item[v1.3+:]
% One comma is added in front of the list and each pair ends with
% a comma. Thus an empty list consists of one comma, otherwise
% two commas encloses the list. Empty entries other than the first
% are removed.
% \item[v1.0 -- v1.2:]
% Empty entries are removed later. In fact it adds a comma at the begin
% and end to protect the last value and an easier implementation.
% \end{description}
%
% \subsection{Parsing key value lists}
%
% \begin{declcs}{kv@parse}\,\M{key value list}\,\M{processor}
% \end{declcs}
% It is easier to parse a normalized list, thus \cs{kv@parse}
% normalizes the list and calls \cs{kv@parse@normalized}.
%
% \begin{declcs}{kv@parse@normalized}\,\M{key value list}%
%   \,\M{processor}
% \end{declcs}
% Now the key value list is split into single key value pairs.
% For further processing the key and value are given as arguments
% for the \meta{processor}:
% \begin{quote}
%   \meta{processor}\,\M{key}\,\M{value}
% \end{quote}
% Also key and value are stored in macro names:
% \begin{itemize}
% \item \cs{kv@key} stores the key.
% \item \cs{kv@value} stores the value or if the value was not
%   specified it has the meaning \cs{relax}.
% \end{itemize}
% The behaviour in pseudo code:
% \begin{quote}
%   foreach (\meta{key}, \meta{value}) in (\meta{key value list})\\
%   \hspace*{1.5em}\cs{kv@key} := \meta{key}\\
%   \hspace*{1.5em}\cs{kv@value} := \meta{value}\\
%   \hspace*{1.5em}\meta{processor}\,\M{key}\,\M{value}
% \end{quote}
%
% \begin{declcs}{kv@break}
% \end{declcs}
% Since version 2011/03/03 v1.11 \cs{kv@break} can be
% called inside the \meta{processor}
% of \cs{kv@parse} or \cs{kv@parse@normalized}, then
% the processing is stopped and the following entries discarded.
%
% \subsection{Processing key value pairs}
%
% Key value pairs can be processed in many different ways.
% For example, the processor for \cs{kvsetkeys} works similar
% to \cs{setkeys} of package \xpackage{keyval}. There unknown
% keys raise an error.
%
% Package \xpackage{xkeyval} also knows a star form of \cs{setkeys}
% that stores unknown keys in an internal macro for further processing
% with \cs{setrmkeys} and similar macros. This feature is covered
% by processor \cs{kv@processor@known}.
%
% \subsubsection{Processing similar to \xpackage{keyval}}
%
% \begin{declcs}{kv@processor@default}\,\M{family}\,\M{key}\,\M{value}
% \end{declcs}
% There are many possiblities to process key value pairs.
% \cs{kv@processor@default} is the processor used in \cs{kvsetkeys}.
% It reimplements and extends the behaviour of
% \xpackage{keyval}'s \cs{setkeys}.
% In case of unknown keys \cs{setkeys} raise an error.
% This processer, however, calls a handler instead, if it
% is provided by the family. Both \meta{family} and \meta{key}
% may contain package \xpackage{babel}'s shorthands
% (since 2011/04/07 v1.13).
%
% Since 2011/10/18 v1.15 the family handler can reject the
% successful handling of a key by calling \cs{kv@handled@false}.
%
% Since 2019/12/15 v1.18 \cs{kv@processor@default} also defines
% macro \cs{kv@fam} with meaning \meta{family} for convenience.
%
% \subsubsection{Processing similar to \cs{setkeys*} of package \xpackage{xkeyval}}
%
% \begin{declcs}{kv@processor@known}\,\M{family}\,^^A
%   \M{cmd}\,\M{key}\,\M{value}
% \end{declcs}
% The key value processor \cs{kv@processor@known} behaves similar
% to \cs{kv@processor@default}. If the \meta{key} exists in the
% \meta{family} its code is called, otherwise the family handler
% is tried. If the family handler is not set or cannot handle the
% key, the unknown key value pair is added to the macro \meta{cmd}.
% Since 2011/10/18 v1.15.
%
% The behaviour in pseudo code:
% \begin{quote}
%   if \meta{key} exists\\
%   \hspace*{1.5em}call the keyval code of \meta{key}\\
%   else\\
%   \hspace*{1.5em}if \meta{handler} for \meta{family} exists\\
%   \hspace*{3em}handled = true\\
%   \hspace*{3em}\meta{handler}\,\M{key}\,\M{value}\\
%   \hspace*{3em}if handled\\
%   \hspace*{3em}else\\
%   \hspace*{4.5em}add \texttt{"}\M{key}\texttt{=}\M{value}\texttt{"}
%     to \M{cmd}\\
%   \hspace*{3em}fi\\
%   \hspace*{1.5em}else\\
%   \hspace*{3em}add \texttt{"}\M{key}\texttt{=}\M{value}\texttt{"}
%     to \M{cmd}\\
%   \hspace*{3em}raise unknown key error\\
%   \hspace*{1.5em}fi\\
%   fi
% \end{quote}
%
% Since 2019/12/15 v1.18 \cs{kv@processor@known} also defines
% macro \cs{kv@fam} with meaning \meta{family} for convenience.
%
% \subsection{Default family handler}
%
% \cs{kv@processor@default} calls \meta{handler}, the default
% handler for the family, if the key does not exist in the family.
% The handler is called with two arguments, the key and the value.
% It can be defined with \cs{kv@set@family@hander}:
%
% \begin{declcs}{kv@set@family@handler}\,\M{family}\,\M{handler definition}
% \end{declcs}
% This sets the default family handler for the keyval family
% \meta{family}. Inside \meta{handler definition} |#1| stands for
% the key and |#2| is the value. Also \cs{kv@key} and \cs{kv@value}
% can be used for the key and the value. If the value is not given,
% \cs{kv@value} has the meaning \cs{relax}.
%
% \begin{declcs}{kv@unset@family@handler}\,\M{family}
% \end{declcs}
% It removes the family handler for \meta{family}.
% Since 2011/10/18 v1.15.
%
% \subsection{Put it all together}
%
% \begin{declcs}{kvsetkeys}\,\M{family}\,\M{key value list}
% \end{declcs}
% Macro \cs{kvsetkeys} processes the \meta{key value list} with
% the standard processor \cs{kv@processor@default}:
% \begin{quote}
%   \cs{kv@parse}\,\M{key value list}^^A
%   \texttt{\{}\cs{kv@processor@default}\,\M{family}\texttt{\}}
% \end{quote}
%
% \begin{declcs}{kvsetknownkeys}\,\M{family}\,\M{cmd}\,\M{key value list}
% \end{declcs}
% Macro \cs{kvsetknownkeys} processes the \meta{key value list}
% with processor \cs{kv@processor@known}. All key value pairs
% with keys that are not known in \meta{family} are stored
% in macro \meta{cmd}. A previous contents of macro \meta{cmd}
% will be overwritten. If all keys can be handled, \meta{cmd}
% will be empty, otherwise it contains a key value list of
% unhandled key value pairs.
% Since 2011/10/18 v1.15.
%
% Pseudo code:
% \begin{quote}
%   create macro \meta{cmdaux} with unique name (inside the current group)\\
%   \cs{def}\meta{cmdaux}\{\}\\
%   \cs{kv@parse}\,\M{key value list}^^A
%   \texttt{\{}\cs{kv@processor@known}\,\M{family}\,\M{cmdaux}\texttt{\}}\\
%   \cs{let}\meta{cmd}=\meta{cmdaux}
% \end{quote}
%
% \begin{declcs}{kvsetkeys@expandafter}\,\M{family}\,\M{list cmd}\\
%   \cs{kvsetknownkeys@expandafter}\,\M{family}\,%
%   \M{cmd}\,\M{list cmd}
% \end{declcs}
% Both macros behave like the counterparts without suffix
% |@expandafter|. The difference is that the key value list is
% given as macro that is expanded once.
% Since 2011/10/18 v1.15.
%
% Thus you can replace \cs{setkeys} of package \xpackage{keyval}
% by the key value parser of this package:
% \begin{quote}
%   |\renewcommand*{\setkeys}{\kvsetkeys}|\\
%   or\\
%   |\let\setkeys\kvsetkeys|
% \end{quote}
%
% \subsection{Comma separated lists}
%
% Since version 2007/09/29 v1.3 this package also supports the normalizing
% and parsing of general comma separated lists.
%
% \begin{declcs}{comma@normalize}\,\M{comma list}
% \end{declcs}
% Macro \cs{comma@normalize} normalizes the comma separated list,
% removes spaces around commas. The result is put in macro \cs{comma@list}.
%
% \begin{declcs}{comma@parse}\,\M{comma list}\,\M{processor}
% \end{declcs}
% Macro \cs{comma@parse} first normalizes the comma separated list
% and then parses the list by calling \cs{comma@parse@normalized}.
%
% \begin{declcs}{comma@parse@normalized}\,\M{normalized comma list}^^A
%   \,\M{processor}
% \end{declcs}
% The list is parsed. Empty entries are ignored. \meta{processor}
% is called for each non-empty entry with the entry as argument:
% \begin{quote}
%   \meta{processor}|{|\meta{entry}|}|
% \end{quote}
% Also the entry is stored in the macro \cs{comma@entry}.
%
% \begin{declcs}{comma@break}
% \end{declcs}
% Since version 2011/03/03 v1.11 \cs{comma@break} can be
% called inside the \meta{processor}
% of \cs{comma@parse} or \cs{comma@parse@normalized}, then
% the processing is stopped and the following entries discarded.
%
% \section{Example}
%
% The following example prints a short piece of HTML code using
% the tabbing environment for indenting purpose and a key value
% syntax for specifying the attributes of an HTML tag.
% The example illustrates the use of a default family handler.
%    \begin{macrocode}
%<*example>
\documentclass{article}
\usepackage[T1]{fontenc}
\usepackage{kvsetkeys}
\usepackage{keyval}

\makeatletter
\newcommand*{\tag}[2][]{%
  % #1: attributes
  % #2: tag name
  \begingroup
    \toks@={}%
    \let\@endslash\@empty
    \kvsetkeys{tag}{#1}%
    \texttt{%
      \textless #2\the\toks@\@endslash\textgreater
    }%
  \endgroup
}
\kv@set@family@handler{tag}{%
  % #1: key
  % #2: value
  \toks@\expandafter{%
    \the\toks@
    \space
    #1=\string"#2\string"%
  }%
}
\define@key{tag}{/}[]{%
  \def\@endslash{/}%
}
\makeatother

\begin{document}
\begin{tabbing}
  \mbox{}\qquad\=\qquad\=\kill
  \tag{html}\\
  \>\dots\\
  \>\tag[border=1]{table}\\
  \>\>\tag[width=200, span=3, /]{colgroup}\\
  \>\>\dots\\
  \>\tag{/table}\\
  \>\dots\\
  \tag{/html}\\
\end{tabbing}
\end{document}
%</example>
%    \end{macrocode}
%
% \StopEventually{
% }
%
% \section{Implementation}
%
% \subsection{Identification}
%
%    \begin{macrocode}
%<*package>
%    \end{macrocode}
%    Reload check, especially if the package is not used with \LaTeX.
%    \begin{macrocode}
\begingroup\catcode61\catcode48\catcode32=10\relax%
  \catcode13=5 % ^^M
  \endlinechar=13 %
  \catcode35=6 % #
  \catcode39=12 % '
  \catcode44=12 % ,
  \catcode45=12 % -
  \catcode46=12 % .
  \catcode58=12 % :
  \catcode64=11 % @
  \catcode123=1 % {
  \catcode125=2 % }
  \expandafter\let\expandafter\x\csname ver@kvsetkeys.sty\endcsname
  \ifx\x\relax % plain-TeX, first loading
  \else
    \def\empty{}%
    \ifx\x\empty % LaTeX, first loading,
      % variable is initialized, but \ProvidesPackage not yet seen
    \else
      \expandafter\ifx\csname PackageInfo\endcsname\relax
        \def\x#1#2{%
          \immediate\write-1{Package #1 Info: #2.}%
        }%
      \else
        \def\x#1#2{\PackageInfo{#1}{#2, stopped}}%
      \fi
      \x{kvsetkeys}{The package is already loaded}%
      \aftergroup\endinput
    \fi
  \fi
\endgroup%
%    \end{macrocode}
%    Package identification:
%    \begin{macrocode}
\begingroup\catcode61\catcode48\catcode32=10\relax%
  \catcode13=5 % ^^M
  \endlinechar=13 %
  \catcode35=6 % #
  \catcode39=12 % '
  \catcode40=12 % (
  \catcode41=12 % )
  \catcode44=12 % ,
  \catcode45=12 % -
  \catcode46=12 % .
  \catcode47=12 % /
  \catcode58=12 % :
  \catcode64=11 % @
  \catcode91=12 % [
  \catcode93=12 % ]
  \catcode123=1 % {
  \catcode125=2 % }
  \expandafter\ifx\csname ProvidesPackage\endcsname\relax
    \def\x#1#2#3[#4]{\endgroup
      \immediate\write-1{Package: #3 #4}%
      \xdef#1{#4}%
    }%
  \else
    \def\x#1#2[#3]{\endgroup
      #2[{#3}]%
      \ifx#1\@undefined
        \xdef#1{#3}%
      \fi
      \ifx#1\relax
        \xdef#1{#3}%
      \fi
    }%
  \fi
\expandafter\x\csname ver@kvsetkeys.sty\endcsname
\ProvidesPackage{kvsetkeys}%
  [2019/12/15 v1.18 Key value parser (HO)]%
%    \end{macrocode}
%
%    \begin{macrocode}
\begingroup\catcode61\catcode48\catcode32=10\relax%
  \catcode13=5 % ^^M
  \endlinechar=13 %
  \catcode123=1 % {
  \catcode125=2 % }
  \catcode64=11 % @
  \def\x{\endgroup
    \expandafter\edef\csname KVS@AtEnd\endcsname{%
      \endlinechar=\the\endlinechar\relax
      \catcode13=\the\catcode13\relax
      \catcode32=\the\catcode32\relax
      \catcode35=\the\catcode35\relax
      \catcode61=\the\catcode61\relax
      \catcode64=\the\catcode64\relax
      \catcode123=\the\catcode123\relax
      \catcode125=\the\catcode125\relax
    }%
  }%
\x\catcode61\catcode48\catcode32=10\relax%
\catcode13=5 % ^^M
\endlinechar=13 %
\catcode35=6 % #
\catcode64=11 % @
\catcode123=1 % {
\catcode125=2 % }
\def\TMP@EnsureCode#1#2{%
  \edef\KVS@AtEnd{%
    \KVS@AtEnd
    \catcode#1=\the\catcode#1\relax
  }%
  \catcode#1=#2\relax
}
\TMP@EnsureCode{36}{3}% $
\TMP@EnsureCode{38}{4}% &
\TMP@EnsureCode{39}{12}% '
\TMP@EnsureCode{43}{12}% +
\TMP@EnsureCode{44}{12}% ,
\TMP@EnsureCode{45}{12}% -
\TMP@EnsureCode{46}{12}% .
\TMP@EnsureCode{47}{12}% /
\TMP@EnsureCode{91}{12}% [
\TMP@EnsureCode{93}{12}% ]
\TMP@EnsureCode{94}{7}% ^ (superscript)
\TMP@EnsureCode{96}{12}% `
\TMP@EnsureCode{126}{13}% ~ (active)
\edef\KVS@AtEnd{\KVS@AtEnd\noexpand\endinput}
%    \end{macrocode}
%
% \subsection{Package loading}
%
%    \begin{macrocode}
\begingroup\expandafter\expandafter\expandafter\endgroup
\expandafter\ifx\csname RequirePackage\endcsname\relax
  \def\TMP@RequirePackage#1[#2]{%
    \begingroup\expandafter\expandafter\expandafter\endgroup
    \expandafter\ifx\csname ver@#1.sty\endcsname\relax
      \input #1.sty\relax
    \fi
  }%
  \TMP@RequirePackage{infwarerr}[2007/09/09]%
  \let\PackageError\@PackageError
\else
\fi
%    \end{macrocode}
%
%    \begin{macrocode}
\expandafter\ifx\csname toks@\endcsname\relax
  \toksdef\toks@=0 %
\fi
%    \end{macrocode}
%
% \subsection{Check for \eTeX}
%
%    \cs{unexpanded}, \cs{ifcsname}, and \cs{unless} are used if found.
%    \begin{macrocode}
\ifx\numexpr\@undefined
  \catcode`\$=9 % ignore
  \catcode`\&=14 % comment
\else % e-TeX
  \catcode`\$=14 % comment
  \catcode`\&=9  % ignore
\fi
%    \end{macrocode}
%
% \subsection{Generic help macros}
%
%    \begin{macro}{\KVS@Empty}
%    \begin{macrocode}
\def\KVS@Empty{}
%    \end{macrocode}
%    \end{macro}
%    \begin{macro}{\KVS@FirstOfTwo}
%    \begin{macrocode}
\long\def\KVS@FirstOfTwo#1#2{#1}
%    \end{macrocode}
%    \end{macro}
%    \begin{macro}{\KVS@SecondOfTwo}
%    \begin{macrocode}
\long\def\KVS@SecondOfTwo#1#2{#2}
%    \end{macrocode}
%    \end{macro}
%
%    \begin{macro}{\KVS@IfEmpty}
%    \begin{macrocode}
\long\def\KVS@IfEmpty#1{%
& \edef\KVS@Temp{\unexpanded{#1}}%
$ \begingroup
$   \toks@{#1}%
$   \edef\KVS@Temp{\the\toks@}%
$ \expandafter\endgroup
  \ifx\KVS@Temp\KVS@Empty
    \expandafter\KVS@FirstOfTwo
  \else
    \expandafter\KVS@SecondOfTwo
  \fi
}
%    \end{macrocode}
%    \end{macro}
%
% \subsection{Normalizing}
%
%    \begin{macro}{\kv@normalize}
%    \begin{macrocode}
\long\def\kv@normalize#1{%
  \begingroup
    \toks@{,#1,}%
    \KVS@Comma
    \KVS@SpaceComma
    \KVS@CommaSpace
    \KVS@CommaComma
    \KVS@Equals
    \KVS@SpaceEquals
    \KVS@EqualsSpace
    \xdef\KVS@Global{\the\toks@}%
  \endgroup
  \let\kv@list\KVS@Global
}
%    \end{macrocode}
%    \end{macro}
%    \begin{macro}{\comma@normalize}
%    \begin{macrocode}
\def\comma@normalize#1{%
  \begingroup
    \toks@{,#1,}%
    \KVS@Comma
    \KVS@SpaceComma
    \KVS@CommaSpace
    \KVS@CommaComma
    \xdef\KVS@Global{\the\toks@}%
  \endgroup
  \let\comma@list\KVS@Global
}
%    \end{macrocode}
%    \end{macro}
%    \begin{macro}{\KVS@Comma}
%    Converts active commas into comma with catcode other.
%    Also adds a comma at the end to protect the last value
%    for next cleanup steps.
%    \begin{macrocode}
\begingroup
  \lccode`\,=`\,%
  \lccode`\~=`\,%
\lowercase{\endgroup
  \def\KVS@Comma{%
    \toks@\expandafter{\expandafter}\expandafter
    \KVS@@Comma\the\toks@~\KVS@Nil
  }%
  \long\def\KVS@@Comma#1~#2\KVS@Nil{%
    \toks@\expandafter{\the\toks@#1}%
    \KVS@IfEmpty{#2}{%
    }{%
      \KVS@@Comma,#2\KVS@Nil
    }%
  }%
}
%    \end{macrocode}
%    \end{macro}
%    \begin{macro}{\KVS@SpaceComma}
%    Removes spaces before the comma, may add commas at the end.
%    \begin{macrocode}
\def\KVS@SpaceComma#1{%
  \def\KVS@SpaceComma{%
    \expandafter\KVS@@SpaceComma\the\toks@#1,\KVS@Nil
  }%
}
\KVS@SpaceComma{ }
%    \end{macrocode}
%    \end{macro}
%    \begin{macro}{\KVS@@SpaceComma}
%    \begin{macrocode}
\long\def\KVS@@SpaceComma#1 ,#2\KVS@Nil{%
  \KVS@IfEmpty{#2}{%
    \toks@{#1}%
  }{%
    \KVS@@SpaceComma#1,#2\KVS@Nil
  }%
}
%    \end{macrocode}
%    \end{macro}
%    \begin{macro}{\KVS@CommaSpace}
%    Removes spaces after the comma, may add commas at the end.
%    \begin{macrocode}
\def\KVS@CommaSpace{%
  \expandafter\KVS@@CommaSpace\the\toks@, \KVS@Nil
}
%    \end{macrocode}
%    \end{macro}
%    \begin{macro}{\KVS@@CommaSpace}
%    \begin{macrocode}
\long\def\KVS@@CommaSpace#1, #2\KVS@Nil{%
  \KVS@IfEmpty{#2}{%
    \toks@{#1}%
  }{%
    \KVS@@CommaSpace#1,#2\KVS@Nil
  }%
}
%    \end{macrocode}
%    \end{macro}
%    \begin{macro}{\KVS@CommaComma}
%    Replaces multiple commas by one comma.
%    \begin{macrocode}
\def\KVS@CommaComma{%
  \expandafter\KVS@@CommaComma\the\toks@,\KVS@Nil
}
%    \end{macrocode}
%    \end{macro}
%    \begin{macro}{\KVS@@CommaComma}
%    \begin{macrocode}
\long\def\KVS@@CommaComma#1,,#2\KVS@Nil{%
  \KVS@IfEmpty{#2}{%
    \toks@{#1,}% (!)
  }{%
    \KVS@@CommaComma#1,#2\KVS@Nil
  }%
}
%    \end{macrocode}
%    \end{macro}
%
%    \begin{macro}{\KVS@Equals}
%    Converts active equals signs into catcode other characters.
%    \begin{macrocode}
\begingroup
  \lccode`\==`\=%
  \lccode`\~=`\=%
\lowercase{\endgroup
  \def\KVS@Equals{%
    \toks@\expandafter{\expandafter}\expandafter
    \KVS@@Equals\the\toks@~\KVS@Nil
  }%
  \long\def\KVS@@Equals#1~#2\KVS@Nil{%
    \edef\KVS@Temp{\the\toks@}%
    \ifx\KVS@Temp\KVS@Empty
      \expandafter\KVS@FirstOfTwo
    \else
      \expandafter\KVS@SecondOfTwo
    \fi
    {%
      \toks@{#1}%
    }{%
      \toks@\expandafter{\the\toks@=#1}%
    }%
    \KVS@IfEmpty{#2}{%
    }{%
      \KVS@@Equals#2\KVS@Nil
    }%
  }%
}
%    \end{macrocode}
%    \end{macro}
%    \begin{macro}{\KVS@SpaceEquals}
%    Removes spaces before the equals sign.
%    \begin{macrocode}
\def\KVS@SpaceEquals#1{%
  \def\KVS@SpaceEquals{%
    \expandafter\KVS@@SpaceEquals\the\toks@#1=\KVS@Nil
  }%
}
\KVS@SpaceEquals{ }
%    \end{macrocode}
%    \end{macro}
%    \begin{macro}{\KVS@@SpaceEquals}
%    \begin{macrocode}
\long\def\KVS@@SpaceEquals#1 =#2\KVS@Nil{%
  \KVS@IfEmpty{#2}{%
    \toks@{#1}%
  }{%
    \KVS@@SpaceEquals#1=#2\KVS@Nil
  }%
}
%    \end{macrocode}
%    \end{macro}
%    \begin{macro}{\KVS@EqualsSpace}
%    Removes spaces after the equals sign.
%    \begin{macrocode}
\def\KVS@EqualsSpace{%
  \expandafter\KVS@@EqualsSpace\the\toks@= \KVS@Nil
}
%    \end{macrocode}
%    \end{macro}
%    \begin{macro}{\KVS@@EqualsSpace}
%    \begin{macrocode}
\long\def\KVS@@EqualsSpace#1= #2\KVS@Nil{%
  \KVS@IfEmpty{#2}{%
    \toks@{#1}%
  }{%
    \KVS@@EqualsSpace#1=#2\KVS@Nil
  }%
}
%    \end{macrocode}
%    \end{macro}
%
% \subsection{Parsing key value lists}
%
%    \begin{macro}{\kv@parse}
%    Normalizes and parses the key value list. Also sets \cs{kv@list}.
%    \begin{macrocode}
\long\def\kv@parse#1{%
  \kv@normalize{#1}%
  \expandafter\kv@parse@normalized\expandafter{\kv@list}%
}
%    \end{macrocode}
%    \end{macro}
%    \begin{macro}{\kv@parse@normalized}
%    |#1|: key value list\\
%    |#2|: processor
%    \begin{macrocode}
\long\def\kv@parse@normalized#1#2{%
  \KVS@Parse#1,\KVS@Nil{#2}%
}
%    \end{macrocode}
%    \end{macro}
%    \begin{macro}{\KVS@Parse}
%    |#1,#2|: key value list\\
%    |#3|: processor
%    \begin{macrocode}
\long\def\KVS@Parse#1,#2\KVS@Nil#3{%
  \KVS@IfEmpty{#1}{%
  }{%
    \KVS@Process#1=\KVS@Nil{#3}%
  }%
  \KVS@MaybeBreak
  \KVS@IfEmpty{#2}{%
  }{%
    \KVS@Parse#2\KVS@Nil{#3}%
  }%
}
%    \end{macrocode}
%    \end{macro}
%    \begin{macro}{\KVS@Process}
%    |#1|: key\\
%    |#2|: value, |=|\\
%    |#3|: processor
%    \begin{macrocode}
\long\def\KVS@Process#1=#2\KVS@Nil#3{%
  \let\KVS@MaybeBreak\relax
  \def\kv@key{#1}%
  \KVS@IfEmpty{#2}{%
    \let\kv@value\relax
    #3{#1}{}%
  }{%
    \KVS@@Process{#1}#2\KVS@Nil{#3}%
  }%
}
%    \end{macrocode}
%    \end{macro}
%    \begin{macro}{\KVS@@Process}
%    |#1|: key\\
%    |#2|: value\\
%    |#3|: processor
%    \begin{macrocode}
\long\def\KVS@@Process#1#2=\KVS@Nil#3{%
& \edef\kv@value{\unexpanded{#2}}%
$ \begingroup
$   \toks@{#2}%
$   \xdef\KVS@Global{\the\toks@}%
$ \endgroup
$ \let\kv@value\KVS@Global
  #3{#1}{#2}%
}
%    \end{macrocode}
%    \end{macro}
%    \begin{macro}{\KVS@MaybeBreak}
%    \begin{macrocode}
\let\KVS@MaybeBreak\relax
%    \end{macrocode}
%    \end{macro}
%    \begin{macro}{\KVS@break}
%    \begin{macrocode}
\def\KVS@break#1#2#3#4{%
  \let\KVS@MaybeBreak\relax
}
%    \end{macrocode}
%    \end{macro}
%    \begin{macro}{\kv@break}
%    \begin{macrocode}
\def\kv@break{%
  \let\KVS@MaybeBreak\KVS@break
}
%    \end{macrocode}
%    \end{macro}
%
% \subsection{Parsing comma lists}
%
%    \begin{macro}{\comma@parse}
%    Normalizes and parses the key value list. Also sets \cs{comma@list}.
%    \begin{macrocode}
\def\comma@parse#1{%
  \comma@normalize{#1}%
  \expandafter\comma@parse@normalized\expandafter{\comma@list}%
}
%    \end{macrocode}
%    \end{macro}
%
%    \begin{macro}{\comma@parse@normalized}
%    |#1|: comma list\\
%    |#2|: processor
%    \begin{macrocode}
\def\comma@parse@normalized#1#2{%
  \KVS@CommaParse#1,\KVS@Nil{#2}%
}
%    \end{macrocode}
%    \end{macro}
%    \begin{macro}{\KVS@CommaParse}
%    |#1,#2|: comma list\\
%    |#3|: processor
%    \begin{macrocode}
\def\KVS@CommaParse#1,#2\KVS@Nil#3{%
  \KVS@IfEmpty{#1}{%
  }{%
    \def\comma@entry{#1}%
    #3{#1}%
  }%
  \KVS@MaybeBreak
  \KVS@IfEmpty{#2}{%
  }{%
    \KVS@CommaParse#2\KVS@Nil{#3}%
  }%
}
%    \end{macrocode}
%    \end{macro}
%    \begin{macro}{\comma@break}
%    \begin{macrocode}
\def\comma@break{%
  \let\KVS@MaybeBreak\KVS@break
}
%    \end{macrocode}
%    \end{macro}
%
% \subsection{Processing key value pairs}
%
%    \begin{macro}{\kv@handled@false}
%    The handler can call \cs{kv@handled@false} or
%    \cs{kv@handled@true} so report failure or success.
%    The default is success (compatibility for versions
%    before 2011/10/18 v1.15).
%    \begin{macrocode}
\def\kv@handled@false{%
  \let\ifkv@handled@\iffalse
}
%    \end{macrocode}
%    \end{macro}
%    \begin{macro}{\kv@handled@true}
%    \begin{macrocode}
\def\kv@handled@true{%
  \let\ifkv@handled@\iftrue
}
%    \end{macrocode}
%    \end{macro}
%    \begin{macro}{\ifkv@handled@}
%    \begin{macrocode}
\kv@handled@true
%    \end{macrocode}
%    \end{macro}
%
%    \begin{macro}{\kv@processor@default}
%    \begin{macrocode}
\def\kv@processor@default#1#2{%
  \begingroup
    \csname @safe@activestrue\endcsname
    \let\ifincsname\iftrue
    \edef\KVS@temp{\endgroup
      \noexpand\KVS@ProcessorDefault{#1}{#2}%
    }%
  \KVS@temp
}
%    \end{macrocode}
%    \end{macro}
%    \begin{macro}{\KVS@ProcessorDefault}
%    \begin{macrocode}
\long\def\KVS@ProcessorDefault#1#2#3{%
  \def\kv@fam{#1}%
& \unless\ifcsname KV@#1@#2\endcsname
$ \begingroup\expandafter\expandafter\expandafter\endgroup
$ \expandafter\ifx\csname KV@#1@#2\endcsname\relax
&   \unless\ifcsname KVS@#1@handler\endcsname
$   \begingroup\expandafter\expandafter\expandafter\endgroup
$   \expandafter\ifx\csname KVS@#1@handler\endcsname\relax
      \kv@error@unknownkey{#1}{#2}%
    \else
      \kv@handled@true
      \csname KVS@#1@handler\endcsname{#2}{#3}%
      \relax
      \ifkv@handled@
      \else
        \kv@error@unknownkey{#1}{#2}%
      \fi
    \fi
  \else
    \ifx\kv@value\relax
&     \unless\ifcsname KV@#1@#2@default\endcsname
$     \begingroup\expandafter\expandafter\expandafter\endgroup
$     \expandafter\ifx\csname KV@#1@#2@default\endcsname\relax
        \kv@error@novalue{#1}{#2}%
      \else
        \csname KV@#1@#2@default\endcsname
        \relax
      \fi
    \else
      \csname KV@#1@#2\endcsname{#3}%
    \fi
  \fi
}
%    \end{macrocode}
%    \end{macro}
%
%    \begin{macro}{\kv@processor@known}
%    \begin{macrocode}
\def\kv@processor@known#1#2#3{%
  \begingroup
    \csname @safe@activestrue\endcsname
    \let\ifincsname\iftrue
    \edef\KVS@temp{\endgroup
      \noexpand\KVS@ProcessorKnown{#1}\noexpand#2{#3}%
    }%
  \KVS@temp
}
%    \end{macrocode}
%    \end{macro}
%    \begin{macro}{\KVS@ProcessorKnown}
%    \begin{macrocode}
\long\def\KVS@ProcessorKnown#1#2#3#4{%
  \def\kv@fam{#1}%
& \unless\ifcsname KV@#1@#3\endcsname
$ \begingroup\expandafter\expandafter\expandafter\endgroup
$ \expandafter\ifx\csname KV@#1@#3\endcsname\relax
&   \unless\ifcsname KVS@#1@handler\endcsname
$   \begingroup\expandafter\expandafter\expandafter\endgroup
$   \expandafter\ifx\csname KVS@#1@handler\endcsname\relax
      \KVS@AddUnhandled#2{#3}{#4}%
    \else
      \kv@handled@true
      \csname KVS@#1@handler\endcsname{#3}{#4}%
      \relax
      \ifkv@handled@
      \else
        \KVS@AddUnhandled#2{#3}{#4}%
      \fi
    \fi
  \else
    \ifx\kv@value\relax
&     \unless\ifcsname KV@#1@#2@default\endcsname
$     \begingroup\expandafter\expandafter\expandafter\endgroup
$     \expandafter\ifx\csname KV@#1@#3@default\endcsname\relax
        \kv@error@novalue{#1}{#3}%
      \else
        \csname KV@#1@#3@default\endcsname
        \relax
      \fi
    \else
      \csname KV@#1@#3\endcsname{#4}%
    \fi
  \fi
}
%    \end{macrocode}
%    \end{macro}
%    \begin{macro}{\KVS@AddUnhandled}
%    \begin{macrocode}
\long\def\KVS@AddUnhandled#1#2#3{%
& \edef#1{%
&   \ifx#1\KVS@empty
&   \else
&     \unexpanded{#1},%
&   \fi
&   \unexpanded{{#2}={#3}}%
& }%
$ \begingroup
$   \ifx#1\KVS@empty
$     \toks@{{#2}={#3}}%
$   \else
$     \toks@\expandafter{#1,{#2}={#3}}%
$   \fi
$   \xdef\KVS@Global{\the\toks@}%
$ \endgroup
$ \let#1\KVS@Global
}
%    \end{macrocode}
%    \end{macro}
%
%    \begin{macro}{\kv@set@family@handler}
%    \begin{macrocode}
\long\def\kv@set@family@handler#1#2{%
  \begingroup
    \csname @safe@activestrue\endcsname
    \let\ifincsname\iftrue
  \expandafter\endgroup
  \expandafter\def\csname KVS@#1@handler\endcsname##1##2{#2}%
}
%    \end{macrocode}
%    \end{macro}
%
%    \begin{macro}{\kv@unset@family@handler}
%    \begin{macrocode}
\long\def\kv@unset@family@handler#1#2{%
  \begingroup
    \csname @safe@activestrue\endcsname
    \let\ifincsname\iftrue
  \expandafter\endgroup
  \expandafter\let\csname KVS@#1@handler\endcsname\@UnDeFiNeD
}
%    \end{macrocode}
%    \end{macro}
%
% \subsection{Error handling}
%
%    \begin{macro}{\kv@error@novalue}
%    \begin{macrocode}
\def\kv@error@novalue{%
  \kv@error@generic{No value specified for}%
}
%    \end{macrocode}
%    \end{macro}
%    \begin{macro}{\kv@error@unknownkey}
%    \begin{macrocode}
\def\kv@error@unknownkey{%
  \kv@error@generic{Undefined}%
}
%    \end{macrocode}
%    \end{macro}
%    \begin{macro}{\kv@error@generic}
%    \begin{macrocode}
\def\kv@error@generic#1#2#3{%
  \PackageError{kvsetkeys}{%
    #1 key `#3'%
  }{%
    The keyval family of the key `#3' is `#2'.\MessageBreak
    The setting of the key is ignored because of the error.\MessageBreak
    \MessageBreak
    \@ehc
  }%
}
%    \end{macrocode}
%    \end{macro}
%
% \subsection{Do it all}
%
%    \begin{macro}{\kvsetkeys}
%    \begin{macrocode}
\long\def\kvsetkeys#1#2{%
  \kv@parse{#2}{\kv@processor@default{#1}}%
}
%    \end{macrocode}
%    \end{macro}
%
%    \begin{macro}{\kvsetkeys@expandafter}
%    \begin{macrocode}
\def\kvsetkeys@expandafter#1#2{%
  \expandafter\kv@parse\expandafter{#2}{%
    \kv@processor@default{#1}%
  }%
}
%    \end{macrocode}
%    \end{macro}
%
%    \begin{macro}{\KVS@cmd}
%    \begin{macrocode}
\def\KVS@cmd{0}%
%    \end{macrocode}
%    \end{macro}
%    \begin{macro}{\KVS@cmd@inc}
%    \begin{macrocode}
\def\KVS@cmd@inc{%
& \edef\KVS@cmd{\the\numexpr\KVS@cmd+1}%
$ \begingroup
$   \count255=\KVS@cmd\relax
$   \advance\count255 by 1\relax
$ \edef\x{\endgroup
$   \noexpand\def\noexpand\KVS@cmd{\number\count255}%
$ }%
$ \x
}
%    \end{macrocode}
%    \end{macro}
%    \begin{macro}{\KVS@cmd@dec}
%    \begin{macrocode}
\def\KVS@cmd@dec{%
& \edef\KVS@cmd{\the\numexpr\KVS@cmd-1}%
$ \begingroup
$   \count255=\KVS@cmd\relax
$   \advance\count255 by -1\relax
$ \edef\x{\endgroup
$   \noexpand\def\noexpand\KVS@cmd{\number\count255}%
$ }%
$ \x
}
%    \end{macrocode}
%    \end{macro}
%    \begin{macro}{\KVS@empty}
%    \begin{macrocode}
\def\KVS@empty{}
%    \end{macrocode}
%    \end{macro}
%
%    \begin{macro}{\kvsetknownkeys}
%    \begin{macrocode}
\def\kvsetknownkeys{%
  \expandafter
  \KVS@setknownkeys\csname KVS@cmd\KVS@cmd\endcsname{}%
}
%    \end{macrocode}
%    \end{macro}
%    \begin{macro}{\KVS@setknownkeys}
%    \begin{macrocode}
\long\def\KVS@setknownkeys#1#2#3#4#5{%
  \let#1\KVS@empty
  \KVS@cmd@inc
  #2\kv@parse#2{#5}{\kv@processor@known{#3}#1}%
  \KVS@cmd@dec
  \let#4=#1%
}
%    \end{macrocode}
%    \end{macro}
%
%    \begin{macro}{\kvsetknownkeys@expandafter}
%    \begin{macrocode}
\def\kvsetknownkeys@expandafter{%
  \expandafter
  \KVS@setknownkeys
      \csname KVS@cmd\KVS@cmd\endcsname\expandafter
}
%    \end{macrocode}
%    \end{macro}
%
%    \begin{macrocode}
\KVS@AtEnd%
%</package>
%    \end{macrocode}
%% \section{Installation}
%
% \subsection{Download}
%
% \paragraph{Package.} This package is available on
% CTAN\footnote{\CTANpkg{kvsetkeys}}:
% \begin{description}
% \item[\CTAN{macros/latex/contrib/kvsetkeys/kvsetkeys.dtx}] The source file.
% \item[\CTAN{macros/latex/contrib/kvsetkeys/kvsetkeys.pdf}] Documentation.
% \end{description}
%
%
% \paragraph{Bundle.} All the packages of the bundle `kvsetkeys'
% are also available in a TDS compliant ZIP archive. There
% the packages are already unpacked and the documentation files
% are generated. The files and directories obey the TDS standard.
% \begin{description}
% \item[\CTANinstall{install/macros/latex/contrib/kvsetkeys.tds.zip}]
% \end{description}
% \emph{TDS} refers to the standard ``A Directory Structure
% for \TeX\ Files'' (\CTANpkg{tds}). Directories
% with \xfile{texmf} in their name are usually organized this way.
%
% \subsection{Bundle installation}
%
% \paragraph{Unpacking.} Unpack the \xfile{kvsetkeys.tds.zip} in the
% TDS tree (also known as \xfile{texmf} tree) of your choice.
% Example (linux):
% \begin{quote}
%   |unzip kvsetkeys.tds.zip -d ~/texmf|
% \end{quote}
%
% \subsection{Package installation}
%
% \paragraph{Unpacking.} The \xfile{.dtx} file is a self-extracting
% \docstrip\ archive. The files are extracted by running the
% \xfile{.dtx} through \plainTeX:
% \begin{quote}
%   \verb|tex kvsetkeys.dtx|
% \end{quote}
%
% \paragraph{TDS.} Now the different files must be moved into
% the different directories in your installation TDS tree
% (also known as \xfile{texmf} tree):
% \begin{quote}
% \def\t{^^A
% \begin{tabular}{@{}>{\ttfamily}l@{ $\rightarrow$ }>{\ttfamily}l@{}}
%   kvsetkeys.sty & tex/generic/kvsetkeys/kvsetkeys.sty\\
%   kvsetkeys.pdf & doc/latex/kvsetkeys/kvsetkeys.pdf\\
%   kvsetkeys-example.tex & doc/latex/kvsetkeys/kvsetkeys-example.tex\\
%   kvsetkeys.dtx & source/latex/kvsetkeys/kvsetkeys.dtx\\
% \end{tabular}^^A
% }^^A
% \sbox0{\t}^^A
% \ifdim\wd0>\linewidth
%   \begingroup
%     \advance\linewidth by\leftmargin
%     \advance\linewidth by\rightmargin
%   \edef\x{\endgroup
%     \def\noexpand\lw{\the\linewidth}^^A
%   }\x
%   \def\lwbox{^^A
%     \leavevmode
%     \hbox to \linewidth{^^A
%       \kern-\leftmargin\relax
%       \hss
%       \usebox0
%       \hss
%       \kern-\rightmargin\relax
%     }^^A
%   }^^A
%   \ifdim\wd0>\lw
%     \sbox0{\small\t}^^A
%     \ifdim\wd0>\linewidth
%       \ifdim\wd0>\lw
%         \sbox0{\footnotesize\t}^^A
%         \ifdim\wd0>\linewidth
%           \ifdim\wd0>\lw
%             \sbox0{\scriptsize\t}^^A
%             \ifdim\wd0>\linewidth
%               \ifdim\wd0>\lw
%                 \sbox0{\tiny\t}^^A
%                 \ifdim\wd0>\linewidth
%                   \lwbox
%                 \else
%                   \usebox0
%                 \fi
%               \else
%                 \lwbox
%               \fi
%             \else
%               \usebox0
%             \fi
%           \else
%             \lwbox
%           \fi
%         \else
%           \usebox0
%         \fi
%       \else
%         \lwbox
%       \fi
%     \else
%       \usebox0
%     \fi
%   \else
%     \lwbox
%   \fi
% \else
%   \usebox0
% \fi
% \end{quote}
% If you have a \xfile{docstrip.cfg} that configures and enables \docstrip's
% TDS installing feature, then some files can already be in the right
% place, see the documentation of \docstrip.
%
% \subsection{Refresh file name databases}
%
% If your \TeX~distribution
% (\TeX\,Live, \mikTeX, \dots) relies on file name databases, you must refresh
% these. For example, \TeX\,Live\ users run \verb|texhash| or
% \verb|mktexlsr|.
%
% \subsection{Some details for the interested}
%
% \paragraph{Unpacking with \LaTeX.}
% The \xfile{.dtx} chooses its action depending on the format:
% \begin{description}
% \item[\plainTeX:] Run \docstrip\ and extract the files.
% \item[\LaTeX:] Generate the documentation.
% \end{description}
% If you insist on using \LaTeX\ for \docstrip\ (really,
% \docstrip\ does not need \LaTeX), then inform the autodetect routine
% about your intention:
% \begin{quote}
%   \verb|latex \let\install=y\input{kvsetkeys.dtx}|
% \end{quote}
% Do not forget to quote the argument according to the demands
% of your shell.
%
% \paragraph{Generating the documentation.}
% You can use both the \xfile{.dtx} or the \xfile{.drv} to generate
% the documentation. The process can be configured by the
% configuration file \xfile{ltxdoc.cfg}. For instance, put this
% line into this file, if you want to have A4 as paper format:
% \begin{quote}
%   \verb|\PassOptionsToClass{a4paper}{article}|
% \end{quote}
% An example follows how to generate the
% documentation with pdf\LaTeX:
% \begin{quote}
%\begin{verbatim}
%pdflatex kvsetkeys.dtx
%makeindex -s gind.ist kvsetkeys.idx
%pdflatex kvsetkeys.dtx
%makeindex -s gind.ist kvsetkeys.idx
%pdflatex kvsetkeys.dtx
%\end{verbatim}
% \end{quote}
%
% \begin{thebibliography}{9}
% \bibitem{tb94wright}
%   A guide to key-value methods, Joseph Wright, second draft for
%   \href{https://www.tug.org/tugboat}{TUGBoat}, 2009-03-17.
%   \url{https://www.texdev.net/uploads/2009/03/keyval.pdf}
%
% \bibitem{keyval}
%   David Carlisle:
%   \textit{The \xpackage{keyval} package};
%   1999/03/16 v1.13;
%   \CTANpkg{keyval}.
%
% \end{thebibliography}
%
% \begin{History}
%   \begin{Version}{2006/03/06 v1.0}
%   \item
%     First version.
%   \end{Version}
%   \begin{Version}{2006/10/19 v1.1}
%   \item
%     Fix of \cs{kv@set@family@handler}.
%   \item
%     Example added.
%   \end{Version}
%   \begin{Version}{2007/09/09 v1.2}
%   \item
%     Using package \xpackage{infwarerr} for error messages.
%   \item
%     Catcode section rewritten.
%   \end{Version}
%   \begin{Version}{2007/09/29 v1.3}
%   \item
%     Normalizing and parsing of comma separated lists added.
%   \item
%     \cs{kv@normalize} rewritten.
%   \item
%     Robustness increased for normalizing and parsing,
%     e.g. for values with unmatched conditionals.
%   \item
%     \eTeX\ is used if available.
%   \item
%     Tests added for normalizing and parsing.
%   \end{Version}
%   \begin{Version}{2009/07/19 v1.4}
%   \item
%     Bug fix for \cs{kv@normalize}: unwanted space removed (Florent Chervet).
%   \end{Version}
%   \begin{Version}{2009/07/30 v1.5}
%   \item
%     Documentation addition: recommendation for Joseph Wright's
%     review article.
%   \end{Version}
%   \begin{Version}{2009/12/12 v1.6}
%   \item
%     Short info shortened.
%   \end{Version}
%   \begin{Version}{2009/12/22 v1.7}
%   \item
%     Internal optimization (\cs{KVS@CommaSpace}, \dots, \cs{KVS@EqualsSpace}).
%   \end{Version}
%   \begin{Version}{2010/01/28 v1.8}
%   \item
%     Compatibility to ini\TeX\ added.
%   \end{Version}
%   \begin{Version}{2010/03/01 v1.9}
%   \item
%     Support of \cs{par} inside values.
%   \end{Version}
%   \begin{Version}{2011/01/30 v1.10}
%   \item
%     Already loaded package files are not input in \hologo{plainTeX}.
%   \end{Version}
%   \begin{Version}{2011/03/03 v1.11}
%   \item
%     \cs{kv@break} and \cs{comma@break} added.
%   \end{Version}
%   \begin{Version}{2011/04/05 v1.12}
%   \item
%     Error message with recovery action in help message
%     (request by GL).
%   \end{Version}
%   \begin{Version}{2011/04/07 v1.13}
%   \item
%     \cs{kv@processor@default} supports package \xpackage{babel}'s
%     shorthands.
%   \item
%     \cs{kv@set@family@handler} with shorthand support.
%   \end{Version}
%   \begin{Version}{2011/06/15 v1.14}
%   \item
%     Some optimizations in token register uses (GL, HO).
%   \end{Version}
%   \begin{Version}{2011/10/18 v1.15}
%   \item
%     \cs{kv@processor@known} and \cs{kvsetknownkeys} added.
%   \item
%     \cs{kvsetkeys@expandafter} and \cs{kvsetknownkeys@expandafter} added.
%   \item
%     Family handler can report success or failure by \cs{kv@handled@true}
%     or \cs{kv@handled@false}.
%   \item
%     \cs{kv@unset@family@handler} added.
%   \end{Version}
%   \begin{Version}{2012/04/25 v1.16}
%   \item
%     \cs{kv@processor@default} and \cs{kv@processor@known} define
%     macro \cs{kv@fam} for convenience.
%   \item
%     Catcode section: Catcode setting for \texttt{+} added for \hologo{eTeX}.
%   \end{Version}
%   \begin{Version}{2016/05/16 v1.17}
%   \item
%     Documentation updates.
%   \end{Version}
%   \begin{Version}{2019/12/15 v1.18}
%   \item
%     Documentation updates.
%   \item
%     Avoid \xpackage{etexcmds} and \xpackage{infwarwerr} in \LaTeX.
%   \end{Version}
% \end{History}
%
% \PrintIndex
%
% \Finale
\endinput

%        (quote the arguments according to the demands of your shell)
%
% Documentation:
%    (a) If kvsetkeys.drv is present:
%           latex kvsetkeys.drv
%    (b) Without kvsetkeys.drv:
%           latex kvsetkeys.dtx; ...
%    The class ltxdoc loads the configuration file ltxdoc.cfg
%    if available. Here you can specify further options, e.g.
%    use A4 as paper format:
%       \PassOptionsToClass{a4paper}{article}
%
%    Programm calls to get the documentation (example):
%       pdflatex kvsetkeys.dtx
%       makeindex -s gind.ist kvsetkeys.idx
%       pdflatex kvsetkeys.dtx
%       makeindex -s gind.ist kvsetkeys.idx
%       pdflatex kvsetkeys.dtx
%
% Installation:
%    TDS:tex/generic/kvsetkeys/kvsetkeys.sty
%    TDS:doc/latex/kvsetkeys/kvsetkeys.pdf
%    TDS:doc/latex/kvsetkeys/kvsetkeys-example.tex
%    TDS:source/latex/kvsetkeys/kvsetkeys.dtx
%
%<*ignore>
\begingroup
  \catcode123=1 %
  \catcode125=2 %
  \def\x{LaTeX2e}%
\expandafter\endgroup
\ifcase 0\ifx\install y1\fi\expandafter
         \ifx\csname processbatchFile\endcsname\relax\else1\fi
         \ifx\fmtname\x\else 1\fi\relax
\else\csname fi\endcsname
%</ignore>
%<*install>
\input docstrip.tex
\Msg{************************************************************************}
\Msg{* Installation}
\Msg{* Package: kvsetkeys 2019/12/15 v1.18 Key value parser (HO)}
\Msg{************************************************************************}

\keepsilent
\askforoverwritefalse

\let\MetaPrefix\relax
\preamble

This is a generated file.

Project: kvsetkeys
Version: 2019/12/15 v1.18

Copyright (C)
   2006, 2007, 2009-2012 Heiko Oberdiek
   2016-2019 Oberdiek Package Support Group

This work may be distributed and/or modified under the
conditions of the LaTeX Project Public License, either
version 1.3c of this license or (at your option) any later
version. This version of this license is in
   https://www.latex-project.org/lppl/lppl-1-3c.txt
and the latest version of this license is in
   https://www.latex-project.org/lppl.txt
and version 1.3 or later is part of all distributions of
LaTeX version 2005/12/01 or later.

This work has the LPPL maintenance status "maintained".

The Current Maintainers of this work are
Heiko Oberdiek and the Oberdiek Package Support Group
https://github.com/ho-tex/kvsetkeys/issues


The Base Interpreter refers to any `TeX-Format',
because some files are installed in TDS:tex/generic//.

This work consists of the main source file kvsetkeys.dtx
and the derived files
   kvsetkeys.sty, kvsetkeys.pdf, kvsetkeys.ins, kvsetkeys.drv,
   kvsetkeys-example.tex, kvsetkeys-test1.tex,
   kvsetkeys-test2.tex, kvsetkeys-test3.tex,
   kvsetkeys-test4.tex.

\endpreamble
\let\MetaPrefix\DoubleperCent

\generate{%
  \file{kvsetkeys.ins}{\from{kvsetkeys.dtx}{install}}%
  \file{kvsetkeys.drv}{\from{kvsetkeys.dtx}{driver}}%
  \usedir{tex/generic/kvsetkeys}%
  \file{kvsetkeys.sty}{\from{kvsetkeys.dtx}{package}}%
  \usedir{doc/latex/kvsetkeys}%
  \file{kvsetkeys-example.tex}{\from{kvsetkeys.dtx}{example}}%
}

\catcode32=13\relax% active space
\let =\space%
\Msg{************************************************************************}
\Msg{*}
\Msg{* To finish the installation you have to move the following}
\Msg{* file into a directory searched by TeX:}
\Msg{*}
\Msg{*     kvsetkeys.sty}
\Msg{*}
\Msg{* To produce the documentation run the file `kvsetkeys.drv'}
\Msg{* through LaTeX.}
\Msg{*}
\Msg{* Happy TeXing!}
\Msg{*}
\Msg{************************************************************************}

\endbatchfile
%</install>
%<*ignore>
\fi
%</ignore>
%<*driver>
\NeedsTeXFormat{LaTeX2e}
\ProvidesFile{kvsetkeys.drv}%
  [2019/12/15 v1.18 Key value parser (HO)]%
\documentclass{ltxdoc}
\usepackage{holtxdoc}[2011/11/22]
\begin{document}
  \DocInput{kvsetkeys.dtx}%
\end{document}
%</driver>
% \fi
%
%
%
% \GetFileInfo{kvsetkeys.drv}
%
% \title{The \xpackage{kvsetkeys} package}
% \date{2019/12/15 v1.18}
% \author{Heiko Oberdiek\thanks
% {Please report any issues at \url{https://github.com/ho-tex/kvsetkeys/issues}}}
%
% \maketitle
%
% \begin{abstract}
% Package \xpackage{kvsetkeys} provides \cs{kvsetkeys}, a variant
% of package \xpackage{keyval}'s \cs{setkeys}. It allows to specify
% a handler that deals with unknown options. Active commas and equal
% signs may be used (e.g. see \xpackage{babel}'s shorthands) and
% only one level of curly braces is removed from the values.
% \end{abstract}
%
% \tableofcontents
%
% \def\M#1{\texttt{\{}\meta{#1}\texttt{\}}}
%
% \section{Documentation}
%
% First I want to recommend the very good review article
% ``A guide to key-value methods'' by Joseph Wright \cite{tb94wright}.
% It introduces the different key-value packages and compares them.
%
% \subsection{Motivation}
%
% \cs{kvsetkeys} serves as replacement for \xpackage{keyval}'s
% \cs{setkeys}. It basically uses the same syntax. But the
% implementation is more robust and predictable:
% \begin{description}
% \item[Active syntax characters:]
%   Comma `|,|' and the equals sign `|=|' are used inside
%   key value lists as syntax characters. Package \xpackage{keyval}
%   uses the catcode of the characters that is active during
%   package loading, usually this is catcode 12 (other).
%   But it can happen that the catcode setting of the syntax characters
%   changes. Especially active characters are of interest, because
%   some language adaptations uses them. For example, option \xoption{turkish}
%   of package \xpackage{babel} uses the equals sign as active shorthand
%   character. Therefore package \xpackage{kvsetkeys} deals with
%   both catcode settings 12 (other) and 13 (active).
% \item[Brace removal:]
%   Package \xpackage{keyval}'s \cs{setkeys} removes up to two
%   levels of curly braces around the value in some unpredictable way:
%\begin{quote}
%|\setkeys{fam}{key={{value}}} || || | $\rightarrow$ |value|\\
%|\setkeys{fam}{key={{{value}}}} | $\rightarrow$ |{value}|\\
%|\setkeys{fam}{key= {{{value}}}}| $\rightarrow$ |{{value}}|
%\end{quote}
%   This package \xpackage{kvsetkeys} follows a much stronger rule:
%   Exactly one level of braces are removed from an item, if the
%   item is surrounded by curly braces. An item can be a
%   the key value pair, the key or the value.
%\begin{quote}
%|\kvsetkeys{fam}{key={value}} || | $\rightarrow$ |value|\\
%|\kvsetkeys{fam}{key={{value}} | $\rightarrow$ |{value}|\\
%|\kvsetkeys{fam}{key= {{value}}| $\rightarrow$ |{value}|
%\end{quote}
% \item[Arbitrary values:] Unmatched conditionals are supported.
% \end{description}
%
% Before I describe
% \cs{kvsetkeys} in more detail, first I want to explain, how
% this package deals with key value lists. For the package also
% provides low level interfaces that can be used by package authors.
%
% \subsection{Normalizing key value lists}
%
% \begin{declcs}{kv@normalize}\,\M{key value list}
% \end{declcs}
% If the user specifies key value lists, he usually prefers
% nice formatted source code, e.g.:
% \begin{quote}
%\begin{verbatim}
%\hypersetup{
%  pdftitle    = {...},
%  pdfsubject  = {...},
%  pdfauthor   = {...},
%  pdfkeywords = {...},
%  ...
%}
%\end{verbatim}
% \end{quote}
% Thus there can be spaces around keys, around |=| or around the value.
% Also empty entries are possible by too many commas. Therefore these
% spaces and empty entries are silently removed by package \xpackage{keyval}
% and this package. Whereas the contents of the value can be protected
% by curly braces, especially if spaces or commas are used inside,
% a key name must not use spaces or other syntax characters.
%
% \cs{kv@normalize} takes a key value list and performs the cleanup:
% \begin{itemize}
% \item Spaces are removed.
% \item Syntax characters (comma and equal sign) that are active
%   are replaced by the same characters with standard catcode.
%   (Example: \xpackage{babel}'s language option \xoption{turkish}
%   uses the equal sign as active shorthand character.)
% \end{itemize}
% The result is stored in \cs{kv@list}, e.g.:
% \begin{quote}
% |\kv@list| $\rightarrow$ |,pdftitle={...},pdfsubject={...},...,|
% \end{quote}
% Curly braces around values (or keys) remain untouched.
% \begin{description}
% \item[v1.3+:]
% One comma is added in front of the list and each pair ends with
% a comma. Thus an empty list consists of one comma, otherwise
% two commas encloses the list. Empty entries other than the first
% are removed.
% \item[v1.0 -- v1.2:]
% Empty entries are removed later. In fact it adds a comma at the begin
% and end to protect the last value and an easier implementation.
% \end{description}
%
% \subsection{Parsing key value lists}
%
% \begin{declcs}{kv@parse}\,\M{key value list}\,\M{processor}
% \end{declcs}
% It is easier to parse a normalized list, thus \cs{kv@parse}
% normalizes the list and calls \cs{kv@parse@normalized}.
%
% \begin{declcs}{kv@parse@normalized}\,\M{key value list}%
%   \,\M{processor}
% \end{declcs}
% Now the key value list is split into single key value pairs.
% For further processing the key and value are given as arguments
% for the \meta{processor}:
% \begin{quote}
%   \meta{processor}\,\M{key}\,\M{value}
% \end{quote}
% Also key and value are stored in macro names:
% \begin{itemize}
% \item \cs{kv@key} stores the key.
% \item \cs{kv@value} stores the value or if the value was not
%   specified it has the meaning \cs{relax}.
% \end{itemize}
% The behaviour in pseudo code:
% \begin{quote}
%   foreach (\meta{key}, \meta{value}) in (\meta{key value list})\\
%   \hspace*{1.5em}\cs{kv@key} := \meta{key}\\
%   \hspace*{1.5em}\cs{kv@value} := \meta{value}\\
%   \hspace*{1.5em}\meta{processor}\,\M{key}\,\M{value}
% \end{quote}
%
% \begin{declcs}{kv@break}
% \end{declcs}
% Since version 2011/03/03 v1.11 \cs{kv@break} can be
% called inside the \meta{processor}
% of \cs{kv@parse} or \cs{kv@parse@normalized}, then
% the processing is stopped and the following entries discarded.
%
% \subsection{Processing key value pairs}
%
% Key value pairs can be processed in many different ways.
% For example, the processor for \cs{kvsetkeys} works similar
% to \cs{setkeys} of package \xpackage{keyval}. There unknown
% keys raise an error.
%
% Package \xpackage{xkeyval} also knows a star form of \cs{setkeys}
% that stores unknown keys in an internal macro for further processing
% with \cs{setrmkeys} and similar macros. This feature is covered
% by processor \cs{kv@processor@known}.
%
% \subsubsection{Processing similar to \xpackage{keyval}}
%
% \begin{declcs}{kv@processor@default}\,\M{family}\,\M{key}\,\M{value}
% \end{declcs}
% There are many possiblities to process key value pairs.
% \cs{kv@processor@default} is the processor used in \cs{kvsetkeys}.
% It reimplements and extends the behaviour of
% \xpackage{keyval}'s \cs{setkeys}.
% In case of unknown keys \cs{setkeys} raise an error.
% This processer, however, calls a handler instead, if it
% is provided by the family. Both \meta{family} and \meta{key}
% may contain package \xpackage{babel}'s shorthands
% (since 2011/04/07 v1.13).
%
% Since 2011/10/18 v1.15 the family handler can reject the
% successful handling of a key by calling \cs{kv@handled@false}.
%
% Since 2019/12/15 v1.18 \cs{kv@processor@default} also defines
% macro \cs{kv@fam} with meaning \meta{family} for convenience.
%
% \subsubsection{Processing similar to \cs{setkeys*} of package \xpackage{xkeyval}}
%
% \begin{declcs}{kv@processor@known}\,\M{family}\,^^A
%   \M{cmd}\,\M{key}\,\M{value}
% \end{declcs}
% The key value processor \cs{kv@processor@known} behaves similar
% to \cs{kv@processor@default}. If the \meta{key} exists in the
% \meta{family} its code is called, otherwise the family handler
% is tried. If the family handler is not set or cannot handle the
% key, the unknown key value pair is added to the macro \meta{cmd}.
% Since 2011/10/18 v1.15.
%
% The behaviour in pseudo code:
% \begin{quote}
%   if \meta{key} exists\\
%   \hspace*{1.5em}call the keyval code of \meta{key}\\
%   else\\
%   \hspace*{1.5em}if \meta{handler} for \meta{family} exists\\
%   \hspace*{3em}handled = true\\
%   \hspace*{3em}\meta{handler}\,\M{key}\,\M{value}\\
%   \hspace*{3em}if handled\\
%   \hspace*{3em}else\\
%   \hspace*{4.5em}add \texttt{"}\M{key}\texttt{=}\M{value}\texttt{"}
%     to \M{cmd}\\
%   \hspace*{3em}fi\\
%   \hspace*{1.5em}else\\
%   \hspace*{3em}add \texttt{"}\M{key}\texttt{=}\M{value}\texttt{"}
%     to \M{cmd}\\
%   \hspace*{3em}raise unknown key error\\
%   \hspace*{1.5em}fi\\
%   fi
% \end{quote}
%
% Since 2019/12/15 v1.18 \cs{kv@processor@known} also defines
% macro \cs{kv@fam} with meaning \meta{family} for convenience.
%
% \subsection{Default family handler}
%
% \cs{kv@processor@default} calls \meta{handler}, the default
% handler for the family, if the key does not exist in the family.
% The handler is called with two arguments, the key and the value.
% It can be defined with \cs{kv@set@family@hander}:
%
% \begin{declcs}{kv@set@family@handler}\,\M{family}\,\M{handler definition}
% \end{declcs}
% This sets the default family handler for the keyval family
% \meta{family}. Inside \meta{handler definition} |#1| stands for
% the key and |#2| is the value. Also \cs{kv@key} and \cs{kv@value}
% can be used for the key and the value. If the value is not given,
% \cs{kv@value} has the meaning \cs{relax}.
%
% \begin{declcs}{kv@unset@family@handler}\,\M{family}
% \end{declcs}
% It removes the family handler for \meta{family}.
% Since 2011/10/18 v1.15.
%
% \subsection{Put it all together}
%
% \begin{declcs}{kvsetkeys}\,\M{family}\,\M{key value list}
% \end{declcs}
% Macro \cs{kvsetkeys} processes the \meta{key value list} with
% the standard processor \cs{kv@processor@default}:
% \begin{quote}
%   \cs{kv@parse}\,\M{key value list}^^A
%   \texttt{\{}\cs{kv@processor@default}\,\M{family}\texttt{\}}
% \end{quote}
%
% \begin{declcs}{kvsetknownkeys}\,\M{family}\,\M{cmd}\,\M{key value list}
% \end{declcs}
% Macro \cs{kvsetknownkeys} processes the \meta{key value list}
% with processor \cs{kv@processor@known}. All key value pairs
% with keys that are not known in \meta{family} are stored
% in macro \meta{cmd}. A previous contents of macro \meta{cmd}
% will be overwritten. If all keys can be handled, \meta{cmd}
% will be empty, otherwise it contains a key value list of
% unhandled key value pairs.
% Since 2011/10/18 v1.15.
%
% Pseudo code:
% \begin{quote}
%   create macro \meta{cmdaux} with unique name (inside the current group)\\
%   \cs{def}\meta{cmdaux}\{\}\\
%   \cs{kv@parse}\,\M{key value list}^^A
%   \texttt{\{}\cs{kv@processor@known}\,\M{family}\,\M{cmdaux}\texttt{\}}\\
%   \cs{let}\meta{cmd}=\meta{cmdaux}
% \end{quote}
%
% \begin{declcs}{kvsetkeys@expandafter}\,\M{family}\,\M{list cmd}\\
%   \cs{kvsetknownkeys@expandafter}\,\M{family}\,%
%   \M{cmd}\,\M{list cmd}
% \end{declcs}
% Both macros behave like the counterparts without suffix
% |@expandafter|. The difference is that the key value list is
% given as macro that is expanded once.
% Since 2011/10/18 v1.15.
%
% Thus you can replace \cs{setkeys} of package \xpackage{keyval}
% by the key value parser of this package:
% \begin{quote}
%   |\renewcommand*{\setkeys}{\kvsetkeys}|\\
%   or\\
%   |\let\setkeys\kvsetkeys|
% \end{quote}
%
% \subsection{Comma separated lists}
%
% Since version 2007/09/29 v1.3 this package also supports the normalizing
% and parsing of general comma separated lists.
%
% \begin{declcs}{comma@normalize}\,\M{comma list}
% \end{declcs}
% Macro \cs{comma@normalize} normalizes the comma separated list,
% removes spaces around commas. The result is put in macro \cs{comma@list}.
%
% \begin{declcs}{comma@parse}\,\M{comma list}\,\M{processor}
% \end{declcs}
% Macro \cs{comma@parse} first normalizes the comma separated list
% and then parses the list by calling \cs{comma@parse@normalized}.
%
% \begin{declcs}{comma@parse@normalized}\,\M{normalized comma list}^^A
%   \,\M{processor}
% \end{declcs}
% The list is parsed. Empty entries are ignored. \meta{processor}
% is called for each non-empty entry with the entry as argument:
% \begin{quote}
%   \meta{processor}|{|\meta{entry}|}|
% \end{quote}
% Also the entry is stored in the macro \cs{comma@entry}.
%
% \begin{declcs}{comma@break}
% \end{declcs}
% Since version 2011/03/03 v1.11 \cs{comma@break} can be
% called inside the \meta{processor}
% of \cs{comma@parse} or \cs{comma@parse@normalized}, then
% the processing is stopped and the following entries discarded.
%
% \section{Example}
%
% The following example prints a short piece of HTML code using
% the tabbing environment for indenting purpose and a key value
% syntax for specifying the attributes of an HTML tag.
% The example illustrates the use of a default family handler.
%    \begin{macrocode}
%<*example>
\documentclass{article}
\usepackage[T1]{fontenc}
\usepackage{kvsetkeys}
\usepackage{keyval}

\makeatletter
\newcommand*{\tag}[2][]{%
  % #1: attributes
  % #2: tag name
  \begingroup
    \toks@={}%
    \let\@endslash\@empty
    \kvsetkeys{tag}{#1}%
    \texttt{%
      \textless #2\the\toks@\@endslash\textgreater
    }%
  \endgroup
}
\kv@set@family@handler{tag}{%
  % #1: key
  % #2: value
  \toks@\expandafter{%
    \the\toks@
    \space
    #1=\string"#2\string"%
  }%
}
\define@key{tag}{/}[]{%
  \def\@endslash{/}%
}
\makeatother

\begin{document}
\begin{tabbing}
  \mbox{}\qquad\=\qquad\=\kill
  \tag{html}\\
  \>\dots\\
  \>\tag[border=1]{table}\\
  \>\>\tag[width=200, span=3, /]{colgroup}\\
  \>\>\dots\\
  \>\tag{/table}\\
  \>\dots\\
  \tag{/html}\\
\end{tabbing}
\end{document}
%</example>
%    \end{macrocode}
%
% \StopEventually{
% }
%
% \section{Implementation}
%
% \subsection{Identification}
%
%    \begin{macrocode}
%<*package>
%    \end{macrocode}
%    Reload check, especially if the package is not used with \LaTeX.
%    \begin{macrocode}
\begingroup\catcode61\catcode48\catcode32=10\relax%
  \catcode13=5 % ^^M
  \endlinechar=13 %
  \catcode35=6 % #
  \catcode39=12 % '
  \catcode44=12 % ,
  \catcode45=12 % -
  \catcode46=12 % .
  \catcode58=12 % :
  \catcode64=11 % @
  \catcode123=1 % {
  \catcode125=2 % }
  \expandafter\let\expandafter\x\csname ver@kvsetkeys.sty\endcsname
  \ifx\x\relax % plain-TeX, first loading
  \else
    \def\empty{}%
    \ifx\x\empty % LaTeX, first loading,
      % variable is initialized, but \ProvidesPackage not yet seen
    \else
      \expandafter\ifx\csname PackageInfo\endcsname\relax
        \def\x#1#2{%
          \immediate\write-1{Package #1 Info: #2.}%
        }%
      \else
        \def\x#1#2{\PackageInfo{#1}{#2, stopped}}%
      \fi
      \x{kvsetkeys}{The package is already loaded}%
      \aftergroup\endinput
    \fi
  \fi
\endgroup%
%    \end{macrocode}
%    Package identification:
%    \begin{macrocode}
\begingroup\catcode61\catcode48\catcode32=10\relax%
  \catcode13=5 % ^^M
  \endlinechar=13 %
  \catcode35=6 % #
  \catcode39=12 % '
  \catcode40=12 % (
  \catcode41=12 % )
  \catcode44=12 % ,
  \catcode45=12 % -
  \catcode46=12 % .
  \catcode47=12 % /
  \catcode58=12 % :
  \catcode64=11 % @
  \catcode91=12 % [
  \catcode93=12 % ]
  \catcode123=1 % {
  \catcode125=2 % }
  \expandafter\ifx\csname ProvidesPackage\endcsname\relax
    \def\x#1#2#3[#4]{\endgroup
      \immediate\write-1{Package: #3 #4}%
      \xdef#1{#4}%
    }%
  \else
    \def\x#1#2[#3]{\endgroup
      #2[{#3}]%
      \ifx#1\@undefined
        \xdef#1{#3}%
      \fi
      \ifx#1\relax
        \xdef#1{#3}%
      \fi
    }%
  \fi
\expandafter\x\csname ver@kvsetkeys.sty\endcsname
\ProvidesPackage{kvsetkeys}%
  [2019/12/15 v1.18 Key value parser (HO)]%
%    \end{macrocode}
%
%    \begin{macrocode}
\begingroup\catcode61\catcode48\catcode32=10\relax%
  \catcode13=5 % ^^M
  \endlinechar=13 %
  \catcode123=1 % {
  \catcode125=2 % }
  \catcode64=11 % @
  \def\x{\endgroup
    \expandafter\edef\csname KVS@AtEnd\endcsname{%
      \endlinechar=\the\endlinechar\relax
      \catcode13=\the\catcode13\relax
      \catcode32=\the\catcode32\relax
      \catcode35=\the\catcode35\relax
      \catcode61=\the\catcode61\relax
      \catcode64=\the\catcode64\relax
      \catcode123=\the\catcode123\relax
      \catcode125=\the\catcode125\relax
    }%
  }%
\x\catcode61\catcode48\catcode32=10\relax%
\catcode13=5 % ^^M
\endlinechar=13 %
\catcode35=6 % #
\catcode64=11 % @
\catcode123=1 % {
\catcode125=2 % }
\def\TMP@EnsureCode#1#2{%
  \edef\KVS@AtEnd{%
    \KVS@AtEnd
    \catcode#1=\the\catcode#1\relax
  }%
  \catcode#1=#2\relax
}
\TMP@EnsureCode{36}{3}% $
\TMP@EnsureCode{38}{4}% &
\TMP@EnsureCode{39}{12}% '
\TMP@EnsureCode{43}{12}% +
\TMP@EnsureCode{44}{12}% ,
\TMP@EnsureCode{45}{12}% -
\TMP@EnsureCode{46}{12}% .
\TMP@EnsureCode{47}{12}% /
\TMP@EnsureCode{91}{12}% [
\TMP@EnsureCode{93}{12}% ]
\TMP@EnsureCode{94}{7}% ^ (superscript)
\TMP@EnsureCode{96}{12}% `
\TMP@EnsureCode{126}{13}% ~ (active)
\edef\KVS@AtEnd{\KVS@AtEnd\noexpand\endinput}
%    \end{macrocode}
%
% \subsection{Package loading}
%
%    \begin{macrocode}
\begingroup\expandafter\expandafter\expandafter\endgroup
\expandafter\ifx\csname RequirePackage\endcsname\relax
  \def\TMP@RequirePackage#1[#2]{%
    \begingroup\expandafter\expandafter\expandafter\endgroup
    \expandafter\ifx\csname ver@#1.sty\endcsname\relax
      \input #1.sty\relax
    \fi
  }%
  \TMP@RequirePackage{infwarerr}[2007/09/09]%
  \let\PackageError\@PackageError
\else
\fi
%    \end{macrocode}
%
%    \begin{macrocode}
\expandafter\ifx\csname toks@\endcsname\relax
  \toksdef\toks@=0 %
\fi
%    \end{macrocode}
%
% \subsection{Check for \eTeX}
%
%    \cs{unexpanded}, \cs{ifcsname}, and \cs{unless} are used if found.
%    \begin{macrocode}
\ifx\numexpr\@undefined
  \catcode`\$=9 % ignore
  \catcode`\&=14 % comment
\else % e-TeX
  \catcode`\$=14 % comment
  \catcode`\&=9  % ignore
\fi
%    \end{macrocode}
%
% \subsection{Generic help macros}
%
%    \begin{macro}{\KVS@Empty}
%    \begin{macrocode}
\def\KVS@Empty{}
%    \end{macrocode}
%    \end{macro}
%    \begin{macro}{\KVS@FirstOfTwo}
%    \begin{macrocode}
\long\def\KVS@FirstOfTwo#1#2{#1}
%    \end{macrocode}
%    \end{macro}
%    \begin{macro}{\KVS@SecondOfTwo}
%    \begin{macrocode}
\long\def\KVS@SecondOfTwo#1#2{#2}
%    \end{macrocode}
%    \end{macro}
%
%    \begin{macro}{\KVS@IfEmpty}
%    \begin{macrocode}
\long\def\KVS@IfEmpty#1{%
& \edef\KVS@Temp{\unexpanded{#1}}%
$ \begingroup
$   \toks@{#1}%
$   \edef\KVS@Temp{\the\toks@}%
$ \expandafter\endgroup
  \ifx\KVS@Temp\KVS@Empty
    \expandafter\KVS@FirstOfTwo
  \else
    \expandafter\KVS@SecondOfTwo
  \fi
}
%    \end{macrocode}
%    \end{macro}
%
% \subsection{Normalizing}
%
%    \begin{macro}{\kv@normalize}
%    \begin{macrocode}
\long\def\kv@normalize#1{%
  \begingroup
    \toks@{,#1,}%
    \KVS@Comma
    \KVS@SpaceComma
    \KVS@CommaSpace
    \KVS@CommaComma
    \KVS@Equals
    \KVS@SpaceEquals
    \KVS@EqualsSpace
    \xdef\KVS@Global{\the\toks@}%
  \endgroup
  \let\kv@list\KVS@Global
}
%    \end{macrocode}
%    \end{macro}
%    \begin{macro}{\comma@normalize}
%    \begin{macrocode}
\def\comma@normalize#1{%
  \begingroup
    \toks@{,#1,}%
    \KVS@Comma
    \KVS@SpaceComma
    \KVS@CommaSpace
    \KVS@CommaComma
    \xdef\KVS@Global{\the\toks@}%
  \endgroup
  \let\comma@list\KVS@Global
}
%    \end{macrocode}
%    \end{macro}
%    \begin{macro}{\KVS@Comma}
%    Converts active commas into comma with catcode other.
%    Also adds a comma at the end to protect the last value
%    for next cleanup steps.
%    \begin{macrocode}
\begingroup
  \lccode`\,=`\,%
  \lccode`\~=`\,%
\lowercase{\endgroup
  \def\KVS@Comma{%
    \toks@\expandafter{\expandafter}\expandafter
    \KVS@@Comma\the\toks@~\KVS@Nil
  }%
  \long\def\KVS@@Comma#1~#2\KVS@Nil{%
    \toks@\expandafter{\the\toks@#1}%
    \KVS@IfEmpty{#2}{%
    }{%
      \KVS@@Comma,#2\KVS@Nil
    }%
  }%
}
%    \end{macrocode}
%    \end{macro}
%    \begin{macro}{\KVS@SpaceComma}
%    Removes spaces before the comma, may add commas at the end.
%    \begin{macrocode}
\def\KVS@SpaceComma#1{%
  \def\KVS@SpaceComma{%
    \expandafter\KVS@@SpaceComma\the\toks@#1,\KVS@Nil
  }%
}
\KVS@SpaceComma{ }
%    \end{macrocode}
%    \end{macro}
%    \begin{macro}{\KVS@@SpaceComma}
%    \begin{macrocode}
\long\def\KVS@@SpaceComma#1 ,#2\KVS@Nil{%
  \KVS@IfEmpty{#2}{%
    \toks@{#1}%
  }{%
    \KVS@@SpaceComma#1,#2\KVS@Nil
  }%
}
%    \end{macrocode}
%    \end{macro}
%    \begin{macro}{\KVS@CommaSpace}
%    Removes spaces after the comma, may add commas at the end.
%    \begin{macrocode}
\def\KVS@CommaSpace{%
  \expandafter\KVS@@CommaSpace\the\toks@, \KVS@Nil
}
%    \end{macrocode}
%    \end{macro}
%    \begin{macro}{\KVS@@CommaSpace}
%    \begin{macrocode}
\long\def\KVS@@CommaSpace#1, #2\KVS@Nil{%
  \KVS@IfEmpty{#2}{%
    \toks@{#1}%
  }{%
    \KVS@@CommaSpace#1,#2\KVS@Nil
  }%
}
%    \end{macrocode}
%    \end{macro}
%    \begin{macro}{\KVS@CommaComma}
%    Replaces multiple commas by one comma.
%    \begin{macrocode}
\def\KVS@CommaComma{%
  \expandafter\KVS@@CommaComma\the\toks@,\KVS@Nil
}
%    \end{macrocode}
%    \end{macro}
%    \begin{macro}{\KVS@@CommaComma}
%    \begin{macrocode}
\long\def\KVS@@CommaComma#1,,#2\KVS@Nil{%
  \KVS@IfEmpty{#2}{%
    \toks@{#1,}% (!)
  }{%
    \KVS@@CommaComma#1,#2\KVS@Nil
  }%
}
%    \end{macrocode}
%    \end{macro}
%
%    \begin{macro}{\KVS@Equals}
%    Converts active equals signs into catcode other characters.
%    \begin{macrocode}
\begingroup
  \lccode`\==`\=%
  \lccode`\~=`\=%
\lowercase{\endgroup
  \def\KVS@Equals{%
    \toks@\expandafter{\expandafter}\expandafter
    \KVS@@Equals\the\toks@~\KVS@Nil
  }%
  \long\def\KVS@@Equals#1~#2\KVS@Nil{%
    \edef\KVS@Temp{\the\toks@}%
    \ifx\KVS@Temp\KVS@Empty
      \expandafter\KVS@FirstOfTwo
    \else
      \expandafter\KVS@SecondOfTwo
    \fi
    {%
      \toks@{#1}%
    }{%
      \toks@\expandafter{\the\toks@=#1}%
    }%
    \KVS@IfEmpty{#2}{%
    }{%
      \KVS@@Equals#2\KVS@Nil
    }%
  }%
}
%    \end{macrocode}
%    \end{macro}
%    \begin{macro}{\KVS@SpaceEquals}
%    Removes spaces before the equals sign.
%    \begin{macrocode}
\def\KVS@SpaceEquals#1{%
  \def\KVS@SpaceEquals{%
    \expandafter\KVS@@SpaceEquals\the\toks@#1=\KVS@Nil
  }%
}
\KVS@SpaceEquals{ }
%    \end{macrocode}
%    \end{macro}
%    \begin{macro}{\KVS@@SpaceEquals}
%    \begin{macrocode}
\long\def\KVS@@SpaceEquals#1 =#2\KVS@Nil{%
  \KVS@IfEmpty{#2}{%
    \toks@{#1}%
  }{%
    \KVS@@SpaceEquals#1=#2\KVS@Nil
  }%
}
%    \end{macrocode}
%    \end{macro}
%    \begin{macro}{\KVS@EqualsSpace}
%    Removes spaces after the equals sign.
%    \begin{macrocode}
\def\KVS@EqualsSpace{%
  \expandafter\KVS@@EqualsSpace\the\toks@= \KVS@Nil
}
%    \end{macrocode}
%    \end{macro}
%    \begin{macro}{\KVS@@EqualsSpace}
%    \begin{macrocode}
\long\def\KVS@@EqualsSpace#1= #2\KVS@Nil{%
  \KVS@IfEmpty{#2}{%
    \toks@{#1}%
  }{%
    \KVS@@EqualsSpace#1=#2\KVS@Nil
  }%
}
%    \end{macrocode}
%    \end{macro}
%
% \subsection{Parsing key value lists}
%
%    \begin{macro}{\kv@parse}
%    Normalizes and parses the key value list. Also sets \cs{kv@list}.
%    \begin{macrocode}
\long\def\kv@parse#1{%
  \kv@normalize{#1}%
  \expandafter\kv@parse@normalized\expandafter{\kv@list}%
}
%    \end{macrocode}
%    \end{macro}
%    \begin{macro}{\kv@parse@normalized}
%    |#1|: key value list\\
%    |#2|: processor
%    \begin{macrocode}
\long\def\kv@parse@normalized#1#2{%
  \KVS@Parse#1,\KVS@Nil{#2}%
}
%    \end{macrocode}
%    \end{macro}
%    \begin{macro}{\KVS@Parse}
%    |#1,#2|: key value list\\
%    |#3|: processor
%    \begin{macrocode}
\long\def\KVS@Parse#1,#2\KVS@Nil#3{%
  \KVS@IfEmpty{#1}{%
  }{%
    \KVS@Process#1=\KVS@Nil{#3}%
  }%
  \KVS@MaybeBreak
  \KVS@IfEmpty{#2}{%
  }{%
    \KVS@Parse#2\KVS@Nil{#3}%
  }%
}
%    \end{macrocode}
%    \end{macro}
%    \begin{macro}{\KVS@Process}
%    |#1|: key\\
%    |#2|: value, |=|\\
%    |#3|: processor
%    \begin{macrocode}
\long\def\KVS@Process#1=#2\KVS@Nil#3{%
  \let\KVS@MaybeBreak\relax
  \def\kv@key{#1}%
  \KVS@IfEmpty{#2}{%
    \let\kv@value\relax
    #3{#1}{}%
  }{%
    \KVS@@Process{#1}#2\KVS@Nil{#3}%
  }%
}
%    \end{macrocode}
%    \end{macro}
%    \begin{macro}{\KVS@@Process}
%    |#1|: key\\
%    |#2|: value\\
%    |#3|: processor
%    \begin{macrocode}
\long\def\KVS@@Process#1#2=\KVS@Nil#3{%
& \edef\kv@value{\unexpanded{#2}}%
$ \begingroup
$   \toks@{#2}%
$   \xdef\KVS@Global{\the\toks@}%
$ \endgroup
$ \let\kv@value\KVS@Global
  #3{#1}{#2}%
}
%    \end{macrocode}
%    \end{macro}
%    \begin{macro}{\KVS@MaybeBreak}
%    \begin{macrocode}
\let\KVS@MaybeBreak\relax
%    \end{macrocode}
%    \end{macro}
%    \begin{macro}{\KVS@break}
%    \begin{macrocode}
\def\KVS@break#1#2#3#4{%
  \let\KVS@MaybeBreak\relax
}
%    \end{macrocode}
%    \end{macro}
%    \begin{macro}{\kv@break}
%    \begin{macrocode}
\def\kv@break{%
  \let\KVS@MaybeBreak\KVS@break
}
%    \end{macrocode}
%    \end{macro}
%
% \subsection{Parsing comma lists}
%
%    \begin{macro}{\comma@parse}
%    Normalizes and parses the key value list. Also sets \cs{comma@list}.
%    \begin{macrocode}
\def\comma@parse#1{%
  \comma@normalize{#1}%
  \expandafter\comma@parse@normalized\expandafter{\comma@list}%
}
%    \end{macrocode}
%    \end{macro}
%
%    \begin{macro}{\comma@parse@normalized}
%    |#1|: comma list\\
%    |#2|: processor
%    \begin{macrocode}
\def\comma@parse@normalized#1#2{%
  \KVS@CommaParse#1,\KVS@Nil{#2}%
}
%    \end{macrocode}
%    \end{macro}
%    \begin{macro}{\KVS@CommaParse}
%    |#1,#2|: comma list\\
%    |#3|: processor
%    \begin{macrocode}
\def\KVS@CommaParse#1,#2\KVS@Nil#3{%
  \KVS@IfEmpty{#1}{%
  }{%
    \def\comma@entry{#1}%
    #3{#1}%
  }%
  \KVS@MaybeBreak
  \KVS@IfEmpty{#2}{%
  }{%
    \KVS@CommaParse#2\KVS@Nil{#3}%
  }%
}
%    \end{macrocode}
%    \end{macro}
%    \begin{macro}{\comma@break}
%    \begin{macrocode}
\def\comma@break{%
  \let\KVS@MaybeBreak\KVS@break
}
%    \end{macrocode}
%    \end{macro}
%
% \subsection{Processing key value pairs}
%
%    \begin{macro}{\kv@handled@false}
%    The handler can call \cs{kv@handled@false} or
%    \cs{kv@handled@true} so report failure or success.
%    The default is success (compatibility for versions
%    before 2011/10/18 v1.15).
%    \begin{macrocode}
\def\kv@handled@false{%
  \let\ifkv@handled@\iffalse
}
%    \end{macrocode}
%    \end{macro}
%    \begin{macro}{\kv@handled@true}
%    \begin{macrocode}
\def\kv@handled@true{%
  \let\ifkv@handled@\iftrue
}
%    \end{macrocode}
%    \end{macro}
%    \begin{macro}{\ifkv@handled@}
%    \begin{macrocode}
\kv@handled@true
%    \end{macrocode}
%    \end{macro}
%
%    \begin{macro}{\kv@processor@default}
%    \begin{macrocode}
\def\kv@processor@default#1#2{%
  \begingroup
    \csname @safe@activestrue\endcsname
    \let\ifincsname\iftrue
    \edef\KVS@temp{\endgroup
      \noexpand\KVS@ProcessorDefault{#1}{#2}%
    }%
  \KVS@temp
}
%    \end{macrocode}
%    \end{macro}
%    \begin{macro}{\KVS@ProcessorDefault}
%    \begin{macrocode}
\long\def\KVS@ProcessorDefault#1#2#3{%
  \def\kv@fam{#1}%
& \unless\ifcsname KV@#1@#2\endcsname
$ \begingroup\expandafter\expandafter\expandafter\endgroup
$ \expandafter\ifx\csname KV@#1@#2\endcsname\relax
&   \unless\ifcsname KVS@#1@handler\endcsname
$   \begingroup\expandafter\expandafter\expandafter\endgroup
$   \expandafter\ifx\csname KVS@#1@handler\endcsname\relax
      \kv@error@unknownkey{#1}{#2}%
    \else
      \kv@handled@true
      \csname KVS@#1@handler\endcsname{#2}{#3}%
      \relax
      \ifkv@handled@
      \else
        \kv@error@unknownkey{#1}{#2}%
      \fi
    \fi
  \else
    \ifx\kv@value\relax
&     \unless\ifcsname KV@#1@#2@default\endcsname
$     \begingroup\expandafter\expandafter\expandafter\endgroup
$     \expandafter\ifx\csname KV@#1@#2@default\endcsname\relax
        \kv@error@novalue{#1}{#2}%
      \else
        \csname KV@#1@#2@default\endcsname
        \relax
      \fi
    \else
      \csname KV@#1@#2\endcsname{#3}%
    \fi
  \fi
}
%    \end{macrocode}
%    \end{macro}
%
%    \begin{macro}{\kv@processor@known}
%    \begin{macrocode}
\def\kv@processor@known#1#2#3{%
  \begingroup
    \csname @safe@activestrue\endcsname
    \let\ifincsname\iftrue
    \edef\KVS@temp{\endgroup
      \noexpand\KVS@ProcessorKnown{#1}\noexpand#2{#3}%
    }%
  \KVS@temp
}
%    \end{macrocode}
%    \end{macro}
%    \begin{macro}{\KVS@ProcessorKnown}
%    \begin{macrocode}
\long\def\KVS@ProcessorKnown#1#2#3#4{%
  \def\kv@fam{#1}%
& \unless\ifcsname KV@#1@#3\endcsname
$ \begingroup\expandafter\expandafter\expandafter\endgroup
$ \expandafter\ifx\csname KV@#1@#3\endcsname\relax
&   \unless\ifcsname KVS@#1@handler\endcsname
$   \begingroup\expandafter\expandafter\expandafter\endgroup
$   \expandafter\ifx\csname KVS@#1@handler\endcsname\relax
      \KVS@AddUnhandled#2{#3}{#4}%
    \else
      \kv@handled@true
      \csname KVS@#1@handler\endcsname{#3}{#4}%
      \relax
      \ifkv@handled@
      \else
        \KVS@AddUnhandled#2{#3}{#4}%
      \fi
    \fi
  \else
    \ifx\kv@value\relax
&     \unless\ifcsname KV@#1@#2@default\endcsname
$     \begingroup\expandafter\expandafter\expandafter\endgroup
$     \expandafter\ifx\csname KV@#1@#3@default\endcsname\relax
        \kv@error@novalue{#1}{#3}%
      \else
        \csname KV@#1@#3@default\endcsname
        \relax
      \fi
    \else
      \csname KV@#1@#3\endcsname{#4}%
    \fi
  \fi
}
%    \end{macrocode}
%    \end{macro}
%    \begin{macro}{\KVS@AddUnhandled}
%    \begin{macrocode}
\long\def\KVS@AddUnhandled#1#2#3{%
& \edef#1{%
&   \ifx#1\KVS@empty
&   \else
&     \unexpanded{#1},%
&   \fi
&   \unexpanded{{#2}={#3}}%
& }%
$ \begingroup
$   \ifx#1\KVS@empty
$     \toks@{{#2}={#3}}%
$   \else
$     \toks@\expandafter{#1,{#2}={#3}}%
$   \fi
$   \xdef\KVS@Global{\the\toks@}%
$ \endgroup
$ \let#1\KVS@Global
}
%    \end{macrocode}
%    \end{macro}
%
%    \begin{macro}{\kv@set@family@handler}
%    \begin{macrocode}
\long\def\kv@set@family@handler#1#2{%
  \begingroup
    \csname @safe@activestrue\endcsname
    \let\ifincsname\iftrue
  \expandafter\endgroup
  \expandafter\def\csname KVS@#1@handler\endcsname##1##2{#2}%
}
%    \end{macrocode}
%    \end{macro}
%
%    \begin{macro}{\kv@unset@family@handler}
%    \begin{macrocode}
\long\def\kv@unset@family@handler#1#2{%
  \begingroup
    \csname @safe@activestrue\endcsname
    \let\ifincsname\iftrue
  \expandafter\endgroup
  \expandafter\let\csname KVS@#1@handler\endcsname\@UnDeFiNeD
}
%    \end{macrocode}
%    \end{macro}
%
% \subsection{Error handling}
%
%    \begin{macro}{\kv@error@novalue}
%    \begin{macrocode}
\def\kv@error@novalue{%
  \kv@error@generic{No value specified for}%
}
%    \end{macrocode}
%    \end{macro}
%    \begin{macro}{\kv@error@unknownkey}
%    \begin{macrocode}
\def\kv@error@unknownkey{%
  \kv@error@generic{Undefined}%
}
%    \end{macrocode}
%    \end{macro}
%    \begin{macro}{\kv@error@generic}
%    \begin{macrocode}
\def\kv@error@generic#1#2#3{%
  \PackageError{kvsetkeys}{%
    #1 key `#3'%
  }{%
    The keyval family of the key `#3' is `#2'.\MessageBreak
    The setting of the key is ignored because of the error.\MessageBreak
    \MessageBreak
    \@ehc
  }%
}
%    \end{macrocode}
%    \end{macro}
%
% \subsection{Do it all}
%
%    \begin{macro}{\kvsetkeys}
%    \begin{macrocode}
\long\def\kvsetkeys#1#2{%
  \kv@parse{#2}{\kv@processor@default{#1}}%
}
%    \end{macrocode}
%    \end{macro}
%
%    \begin{macro}{\kvsetkeys@expandafter}
%    \begin{macrocode}
\def\kvsetkeys@expandafter#1#2{%
  \expandafter\kv@parse\expandafter{#2}{%
    \kv@processor@default{#1}%
  }%
}
%    \end{macrocode}
%    \end{macro}
%
%    \begin{macro}{\KVS@cmd}
%    \begin{macrocode}
\def\KVS@cmd{0}%
%    \end{macrocode}
%    \end{macro}
%    \begin{macro}{\KVS@cmd@inc}
%    \begin{macrocode}
\def\KVS@cmd@inc{%
& \edef\KVS@cmd{\the\numexpr\KVS@cmd+1}%
$ \begingroup
$   \count255=\KVS@cmd\relax
$   \advance\count255 by 1\relax
$ \edef\x{\endgroup
$   \noexpand\def\noexpand\KVS@cmd{\number\count255}%
$ }%
$ \x
}
%    \end{macrocode}
%    \end{macro}
%    \begin{macro}{\KVS@cmd@dec}
%    \begin{macrocode}
\def\KVS@cmd@dec{%
& \edef\KVS@cmd{\the\numexpr\KVS@cmd-1}%
$ \begingroup
$   \count255=\KVS@cmd\relax
$   \advance\count255 by -1\relax
$ \edef\x{\endgroup
$   \noexpand\def\noexpand\KVS@cmd{\number\count255}%
$ }%
$ \x
}
%    \end{macrocode}
%    \end{macro}
%    \begin{macro}{\KVS@empty}
%    \begin{macrocode}
\def\KVS@empty{}
%    \end{macrocode}
%    \end{macro}
%
%    \begin{macro}{\kvsetknownkeys}
%    \begin{macrocode}
\def\kvsetknownkeys{%
  \expandafter
  \KVS@setknownkeys\csname KVS@cmd\KVS@cmd\endcsname{}%
}
%    \end{macrocode}
%    \end{macro}
%    \begin{macro}{\KVS@setknownkeys}
%    \begin{macrocode}
\long\def\KVS@setknownkeys#1#2#3#4#5{%
  \let#1\KVS@empty
  \KVS@cmd@inc
  #2\kv@parse#2{#5}{\kv@processor@known{#3}#1}%
  \KVS@cmd@dec
  \let#4=#1%
}
%    \end{macrocode}
%    \end{macro}
%
%    \begin{macro}{\kvsetknownkeys@expandafter}
%    \begin{macrocode}
\def\kvsetknownkeys@expandafter{%
  \expandafter
  \KVS@setknownkeys
      \csname KVS@cmd\KVS@cmd\endcsname\expandafter
}
%    \end{macrocode}
%    \end{macro}
%
%    \begin{macrocode}
\KVS@AtEnd%
%</package>
%    \end{macrocode}
%% \section{Installation}
%
% \subsection{Download}
%
% \paragraph{Package.} This package is available on
% CTAN\footnote{\CTANpkg{kvsetkeys}}:
% \begin{description}
% \item[\CTAN{macros/latex/contrib/kvsetkeys/kvsetkeys.dtx}] The source file.
% \item[\CTAN{macros/latex/contrib/kvsetkeys/kvsetkeys.pdf}] Documentation.
% \end{description}
%
%
% \paragraph{Bundle.} All the packages of the bundle `kvsetkeys'
% are also available in a TDS compliant ZIP archive. There
% the packages are already unpacked and the documentation files
% are generated. The files and directories obey the TDS standard.
% \begin{description}
% \item[\CTANinstall{install/macros/latex/contrib/kvsetkeys.tds.zip}]
% \end{description}
% \emph{TDS} refers to the standard ``A Directory Structure
% for \TeX\ Files'' (\CTANpkg{tds}). Directories
% with \xfile{texmf} in their name are usually organized this way.
%
% \subsection{Bundle installation}
%
% \paragraph{Unpacking.} Unpack the \xfile{kvsetkeys.tds.zip} in the
% TDS tree (also known as \xfile{texmf} tree) of your choice.
% Example (linux):
% \begin{quote}
%   |unzip kvsetkeys.tds.zip -d ~/texmf|
% \end{quote}
%
% \subsection{Package installation}
%
% \paragraph{Unpacking.} The \xfile{.dtx} file is a self-extracting
% \docstrip\ archive. The files are extracted by running the
% \xfile{.dtx} through \plainTeX:
% \begin{quote}
%   \verb|tex kvsetkeys.dtx|
% \end{quote}
%
% \paragraph{TDS.} Now the different files must be moved into
% the different directories in your installation TDS tree
% (also known as \xfile{texmf} tree):
% \begin{quote}
% \def\t{^^A
% \begin{tabular}{@{}>{\ttfamily}l@{ $\rightarrow$ }>{\ttfamily}l@{}}
%   kvsetkeys.sty & tex/generic/kvsetkeys/kvsetkeys.sty\\
%   kvsetkeys.pdf & doc/latex/kvsetkeys/kvsetkeys.pdf\\
%   kvsetkeys-example.tex & doc/latex/kvsetkeys/kvsetkeys-example.tex\\
%   kvsetkeys.dtx & source/latex/kvsetkeys/kvsetkeys.dtx\\
% \end{tabular}^^A
% }^^A
% \sbox0{\t}^^A
% \ifdim\wd0>\linewidth
%   \begingroup
%     \advance\linewidth by\leftmargin
%     \advance\linewidth by\rightmargin
%   \edef\x{\endgroup
%     \def\noexpand\lw{\the\linewidth}^^A
%   }\x
%   \def\lwbox{^^A
%     \leavevmode
%     \hbox to \linewidth{^^A
%       \kern-\leftmargin\relax
%       \hss
%       \usebox0
%       \hss
%       \kern-\rightmargin\relax
%     }^^A
%   }^^A
%   \ifdim\wd0>\lw
%     \sbox0{\small\t}^^A
%     \ifdim\wd0>\linewidth
%       \ifdim\wd0>\lw
%         \sbox0{\footnotesize\t}^^A
%         \ifdim\wd0>\linewidth
%           \ifdim\wd0>\lw
%             \sbox0{\scriptsize\t}^^A
%             \ifdim\wd0>\linewidth
%               \ifdim\wd0>\lw
%                 \sbox0{\tiny\t}^^A
%                 \ifdim\wd0>\linewidth
%                   \lwbox
%                 \else
%                   \usebox0
%                 \fi
%               \else
%                 \lwbox
%               \fi
%             \else
%               \usebox0
%             \fi
%           \else
%             \lwbox
%           \fi
%         \else
%           \usebox0
%         \fi
%       \else
%         \lwbox
%       \fi
%     \else
%       \usebox0
%     \fi
%   \else
%     \lwbox
%   \fi
% \else
%   \usebox0
% \fi
% \end{quote}
% If you have a \xfile{docstrip.cfg} that configures and enables \docstrip's
% TDS installing feature, then some files can already be in the right
% place, see the documentation of \docstrip.
%
% \subsection{Refresh file name databases}
%
% If your \TeX~distribution
% (\TeX\,Live, \mikTeX, \dots) relies on file name databases, you must refresh
% these. For example, \TeX\,Live\ users run \verb|texhash| or
% \verb|mktexlsr|.
%
% \subsection{Some details for the interested}
%
% \paragraph{Unpacking with \LaTeX.}
% The \xfile{.dtx} chooses its action depending on the format:
% \begin{description}
% \item[\plainTeX:] Run \docstrip\ and extract the files.
% \item[\LaTeX:] Generate the documentation.
% \end{description}
% If you insist on using \LaTeX\ for \docstrip\ (really,
% \docstrip\ does not need \LaTeX), then inform the autodetect routine
% about your intention:
% \begin{quote}
%   \verb|latex \let\install=y% \iffalse meta-comment
%
% File: kvsetkeys.dtx
% Version: 2019/12/15 v1.18
% Info: Key value parser
%
% Copyright (C)
%    2006, 2007, 2009-2012 Heiko Oberdiek
%    2016-2019 Oberdiek Package Support Group
%    https://github.com/ho-tex/kvsetkeys/issues
%
% This work may be distributed and/or modified under the
% conditions of the LaTeX Project Public License, either
% version 1.3c of this license or (at your option) any later
% version. This version of this license is in
%    https://www.latex-project.org/lppl/lppl-1-3c.txt
% and the latest version of this license is in
%    https://www.latex-project.org/lppl.txt
% and version 1.3 or later is part of all distributions of
% LaTeX version 2005/12/01 or later.
%
% This work has the LPPL maintenance status "maintained".
%
% The Current Maintainers of this work are
% Heiko Oberdiek and the Oberdiek Package Support Group
% https://github.com/ho-tex/kvsetkeys/issues
%
% The Base Interpreter refers to any `TeX-Format',
% because some files are installed in TDS:tex/generic//.
%
% This work consists of the main source file kvsetkeys.dtx
% and the derived files
%    kvsetkeys.sty, kvsetkeys.pdf, kvsetkeys.ins, kvsetkeys.drv,
%    kvsetkeys-example.tex, kvsetkeys-test1.tex,
%    kvsetkeys-test2.tex, kvsetkeys-test3.tex,
%    kvsetkeys-test4.tex.
%
% Distribution:
%    CTAN:macros/latex/contrib/kvsetkeys/kvsetkeys.dtx
%    CTAN:macros/latex/contrib/kvsetkeys/kvsetkeys.pdf
%
% Unpacking:
%    (a) If kvsetkeys.ins is present:
%           tex kvsetkeys.ins
%    (b) Without kvsetkeys.ins:
%           tex kvsetkeys.dtx
%    (c) If you insist on using LaTeX
%           latex \let\install=y\input{kvsetkeys.dtx}
%        (quote the arguments according to the demands of your shell)
%
% Documentation:
%    (a) If kvsetkeys.drv is present:
%           latex kvsetkeys.drv
%    (b) Without kvsetkeys.drv:
%           latex kvsetkeys.dtx; ...
%    The class ltxdoc loads the configuration file ltxdoc.cfg
%    if available. Here you can specify further options, e.g.
%    use A4 as paper format:
%       \PassOptionsToClass{a4paper}{article}
%
%    Programm calls to get the documentation (example):
%       pdflatex kvsetkeys.dtx
%       makeindex -s gind.ist kvsetkeys.idx
%       pdflatex kvsetkeys.dtx
%       makeindex -s gind.ist kvsetkeys.idx
%       pdflatex kvsetkeys.dtx
%
% Installation:
%    TDS:tex/generic/kvsetkeys/kvsetkeys.sty
%    TDS:doc/latex/kvsetkeys/kvsetkeys.pdf
%    TDS:doc/latex/kvsetkeys/kvsetkeys-example.tex
%    TDS:source/latex/kvsetkeys/kvsetkeys.dtx
%
%<*ignore>
\begingroup
  \catcode123=1 %
  \catcode125=2 %
  \def\x{LaTeX2e}%
\expandafter\endgroup
\ifcase 0\ifx\install y1\fi\expandafter
         \ifx\csname processbatchFile\endcsname\relax\else1\fi
         \ifx\fmtname\x\else 1\fi\relax
\else\csname fi\endcsname
%</ignore>
%<*install>
\input docstrip.tex
\Msg{************************************************************************}
\Msg{* Installation}
\Msg{* Package: kvsetkeys 2019/12/15 v1.18 Key value parser (HO)}
\Msg{************************************************************************}

\keepsilent
\askforoverwritefalse

\let\MetaPrefix\relax
\preamble

This is a generated file.

Project: kvsetkeys
Version: 2019/12/15 v1.18

Copyright (C)
   2006, 2007, 2009-2012 Heiko Oberdiek
   2016-2019 Oberdiek Package Support Group

This work may be distributed and/or modified under the
conditions of the LaTeX Project Public License, either
version 1.3c of this license or (at your option) any later
version. This version of this license is in
   https://www.latex-project.org/lppl/lppl-1-3c.txt
and the latest version of this license is in
   https://www.latex-project.org/lppl.txt
and version 1.3 or later is part of all distributions of
LaTeX version 2005/12/01 or later.

This work has the LPPL maintenance status "maintained".

The Current Maintainers of this work are
Heiko Oberdiek and the Oberdiek Package Support Group
https://github.com/ho-tex/kvsetkeys/issues


The Base Interpreter refers to any `TeX-Format',
because some files are installed in TDS:tex/generic//.

This work consists of the main source file kvsetkeys.dtx
and the derived files
   kvsetkeys.sty, kvsetkeys.pdf, kvsetkeys.ins, kvsetkeys.drv,
   kvsetkeys-example.tex, kvsetkeys-test1.tex,
   kvsetkeys-test2.tex, kvsetkeys-test3.tex,
   kvsetkeys-test4.tex.

\endpreamble
\let\MetaPrefix\DoubleperCent

\generate{%
  \file{kvsetkeys.ins}{\from{kvsetkeys.dtx}{install}}%
  \file{kvsetkeys.drv}{\from{kvsetkeys.dtx}{driver}}%
  \usedir{tex/generic/kvsetkeys}%
  \file{kvsetkeys.sty}{\from{kvsetkeys.dtx}{package}}%
  \usedir{doc/latex/kvsetkeys}%
  \file{kvsetkeys-example.tex}{\from{kvsetkeys.dtx}{example}}%
}

\catcode32=13\relax% active space
\let =\space%
\Msg{************************************************************************}
\Msg{*}
\Msg{* To finish the installation you have to move the following}
\Msg{* file into a directory searched by TeX:}
\Msg{*}
\Msg{*     kvsetkeys.sty}
\Msg{*}
\Msg{* To produce the documentation run the file `kvsetkeys.drv'}
\Msg{* through LaTeX.}
\Msg{*}
\Msg{* Happy TeXing!}
\Msg{*}
\Msg{************************************************************************}

\endbatchfile
%</install>
%<*ignore>
\fi
%</ignore>
%<*driver>
\NeedsTeXFormat{LaTeX2e}
\ProvidesFile{kvsetkeys.drv}%
  [2019/12/15 v1.18 Key value parser (HO)]%
\documentclass{ltxdoc}
\usepackage{holtxdoc}[2011/11/22]
\begin{document}
  \DocInput{kvsetkeys.dtx}%
\end{document}
%</driver>
% \fi
%
%
%
% \GetFileInfo{kvsetkeys.drv}
%
% \title{The \xpackage{kvsetkeys} package}
% \date{2019/12/15 v1.18}
% \author{Heiko Oberdiek\thanks
% {Please report any issues at \url{https://github.com/ho-tex/kvsetkeys/issues}}}
%
% \maketitle
%
% \begin{abstract}
% Package \xpackage{kvsetkeys} provides \cs{kvsetkeys}, a variant
% of package \xpackage{keyval}'s \cs{setkeys}. It allows to specify
% a handler that deals with unknown options. Active commas and equal
% signs may be used (e.g. see \xpackage{babel}'s shorthands) and
% only one level of curly braces is removed from the values.
% \end{abstract}
%
% \tableofcontents
%
% \def\M#1{\texttt{\{}\meta{#1}\texttt{\}}}
%
% \section{Documentation}
%
% First I want to recommend the very good review article
% ``A guide to key-value methods'' by Joseph Wright \cite{tb94wright}.
% It introduces the different key-value packages and compares them.
%
% \subsection{Motivation}
%
% \cs{kvsetkeys} serves as replacement for \xpackage{keyval}'s
% \cs{setkeys}. It basically uses the same syntax. But the
% implementation is more robust and predictable:
% \begin{description}
% \item[Active syntax characters:]
%   Comma `|,|' and the equals sign `|=|' are used inside
%   key value lists as syntax characters. Package \xpackage{keyval}
%   uses the catcode of the characters that is active during
%   package loading, usually this is catcode 12 (other).
%   But it can happen that the catcode setting of the syntax characters
%   changes. Especially active characters are of interest, because
%   some language adaptations uses them. For example, option \xoption{turkish}
%   of package \xpackage{babel} uses the equals sign as active shorthand
%   character. Therefore package \xpackage{kvsetkeys} deals with
%   both catcode settings 12 (other) and 13 (active).
% \item[Brace removal:]
%   Package \xpackage{keyval}'s \cs{setkeys} removes up to two
%   levels of curly braces around the value in some unpredictable way:
%\begin{quote}
%|\setkeys{fam}{key={{value}}} || || | $\rightarrow$ |value|\\
%|\setkeys{fam}{key={{{value}}}} | $\rightarrow$ |{value}|\\
%|\setkeys{fam}{key= {{{value}}}}| $\rightarrow$ |{{value}}|
%\end{quote}
%   This package \xpackage{kvsetkeys} follows a much stronger rule:
%   Exactly one level of braces are removed from an item, if the
%   item is surrounded by curly braces. An item can be a
%   the key value pair, the key or the value.
%\begin{quote}
%|\kvsetkeys{fam}{key={value}} || | $\rightarrow$ |value|\\
%|\kvsetkeys{fam}{key={{value}} | $\rightarrow$ |{value}|\\
%|\kvsetkeys{fam}{key= {{value}}| $\rightarrow$ |{value}|
%\end{quote}
% \item[Arbitrary values:] Unmatched conditionals are supported.
% \end{description}
%
% Before I describe
% \cs{kvsetkeys} in more detail, first I want to explain, how
% this package deals with key value lists. For the package also
% provides low level interfaces that can be used by package authors.
%
% \subsection{Normalizing key value lists}
%
% \begin{declcs}{kv@normalize}\,\M{key value list}
% \end{declcs}
% If the user specifies key value lists, he usually prefers
% nice formatted source code, e.g.:
% \begin{quote}
%\begin{verbatim}
%\hypersetup{
%  pdftitle    = {...},
%  pdfsubject  = {...},
%  pdfauthor   = {...},
%  pdfkeywords = {...},
%  ...
%}
%\end{verbatim}
% \end{quote}
% Thus there can be spaces around keys, around |=| or around the value.
% Also empty entries are possible by too many commas. Therefore these
% spaces and empty entries are silently removed by package \xpackage{keyval}
% and this package. Whereas the contents of the value can be protected
% by curly braces, especially if spaces or commas are used inside,
% a key name must not use spaces or other syntax characters.
%
% \cs{kv@normalize} takes a key value list and performs the cleanup:
% \begin{itemize}
% \item Spaces are removed.
% \item Syntax characters (comma and equal sign) that are active
%   are replaced by the same characters with standard catcode.
%   (Example: \xpackage{babel}'s language option \xoption{turkish}
%   uses the equal sign as active shorthand character.)
% \end{itemize}
% The result is stored in \cs{kv@list}, e.g.:
% \begin{quote}
% |\kv@list| $\rightarrow$ |,pdftitle={...},pdfsubject={...},...,|
% \end{quote}
% Curly braces around values (or keys) remain untouched.
% \begin{description}
% \item[v1.3+:]
% One comma is added in front of the list and each pair ends with
% a comma. Thus an empty list consists of one comma, otherwise
% two commas encloses the list. Empty entries other than the first
% are removed.
% \item[v1.0 -- v1.2:]
% Empty entries are removed later. In fact it adds a comma at the begin
% and end to protect the last value and an easier implementation.
% \end{description}
%
% \subsection{Parsing key value lists}
%
% \begin{declcs}{kv@parse}\,\M{key value list}\,\M{processor}
% \end{declcs}
% It is easier to parse a normalized list, thus \cs{kv@parse}
% normalizes the list and calls \cs{kv@parse@normalized}.
%
% \begin{declcs}{kv@parse@normalized}\,\M{key value list}%
%   \,\M{processor}
% \end{declcs}
% Now the key value list is split into single key value pairs.
% For further processing the key and value are given as arguments
% for the \meta{processor}:
% \begin{quote}
%   \meta{processor}\,\M{key}\,\M{value}
% \end{quote}
% Also key and value are stored in macro names:
% \begin{itemize}
% \item \cs{kv@key} stores the key.
% \item \cs{kv@value} stores the value or if the value was not
%   specified it has the meaning \cs{relax}.
% \end{itemize}
% The behaviour in pseudo code:
% \begin{quote}
%   foreach (\meta{key}, \meta{value}) in (\meta{key value list})\\
%   \hspace*{1.5em}\cs{kv@key} := \meta{key}\\
%   \hspace*{1.5em}\cs{kv@value} := \meta{value}\\
%   \hspace*{1.5em}\meta{processor}\,\M{key}\,\M{value}
% \end{quote}
%
% \begin{declcs}{kv@break}
% \end{declcs}
% Since version 2011/03/03 v1.11 \cs{kv@break} can be
% called inside the \meta{processor}
% of \cs{kv@parse} or \cs{kv@parse@normalized}, then
% the processing is stopped and the following entries discarded.
%
% \subsection{Processing key value pairs}
%
% Key value pairs can be processed in many different ways.
% For example, the processor for \cs{kvsetkeys} works similar
% to \cs{setkeys} of package \xpackage{keyval}. There unknown
% keys raise an error.
%
% Package \xpackage{xkeyval} also knows a star form of \cs{setkeys}
% that stores unknown keys in an internal macro for further processing
% with \cs{setrmkeys} and similar macros. This feature is covered
% by processor \cs{kv@processor@known}.
%
% \subsubsection{Processing similar to \xpackage{keyval}}
%
% \begin{declcs}{kv@processor@default}\,\M{family}\,\M{key}\,\M{value}
% \end{declcs}
% There are many possiblities to process key value pairs.
% \cs{kv@processor@default} is the processor used in \cs{kvsetkeys}.
% It reimplements and extends the behaviour of
% \xpackage{keyval}'s \cs{setkeys}.
% In case of unknown keys \cs{setkeys} raise an error.
% This processer, however, calls a handler instead, if it
% is provided by the family. Both \meta{family} and \meta{key}
% may contain package \xpackage{babel}'s shorthands
% (since 2011/04/07 v1.13).
%
% Since 2011/10/18 v1.15 the family handler can reject the
% successful handling of a key by calling \cs{kv@handled@false}.
%
% Since 2019/12/15 v1.18 \cs{kv@processor@default} also defines
% macro \cs{kv@fam} with meaning \meta{family} for convenience.
%
% \subsubsection{Processing similar to \cs{setkeys*} of package \xpackage{xkeyval}}
%
% \begin{declcs}{kv@processor@known}\,\M{family}\,^^A
%   \M{cmd}\,\M{key}\,\M{value}
% \end{declcs}
% The key value processor \cs{kv@processor@known} behaves similar
% to \cs{kv@processor@default}. If the \meta{key} exists in the
% \meta{family} its code is called, otherwise the family handler
% is tried. If the family handler is not set or cannot handle the
% key, the unknown key value pair is added to the macro \meta{cmd}.
% Since 2011/10/18 v1.15.
%
% The behaviour in pseudo code:
% \begin{quote}
%   if \meta{key} exists\\
%   \hspace*{1.5em}call the keyval code of \meta{key}\\
%   else\\
%   \hspace*{1.5em}if \meta{handler} for \meta{family} exists\\
%   \hspace*{3em}handled = true\\
%   \hspace*{3em}\meta{handler}\,\M{key}\,\M{value}\\
%   \hspace*{3em}if handled\\
%   \hspace*{3em}else\\
%   \hspace*{4.5em}add \texttt{"}\M{key}\texttt{=}\M{value}\texttt{"}
%     to \M{cmd}\\
%   \hspace*{3em}fi\\
%   \hspace*{1.5em}else\\
%   \hspace*{3em}add \texttt{"}\M{key}\texttt{=}\M{value}\texttt{"}
%     to \M{cmd}\\
%   \hspace*{3em}raise unknown key error\\
%   \hspace*{1.5em}fi\\
%   fi
% \end{quote}
%
% Since 2019/12/15 v1.18 \cs{kv@processor@known} also defines
% macro \cs{kv@fam} with meaning \meta{family} for convenience.
%
% \subsection{Default family handler}
%
% \cs{kv@processor@default} calls \meta{handler}, the default
% handler for the family, if the key does not exist in the family.
% The handler is called with two arguments, the key and the value.
% It can be defined with \cs{kv@set@family@hander}:
%
% \begin{declcs}{kv@set@family@handler}\,\M{family}\,\M{handler definition}
% \end{declcs}
% This sets the default family handler for the keyval family
% \meta{family}. Inside \meta{handler definition} |#1| stands for
% the key and |#2| is the value. Also \cs{kv@key} and \cs{kv@value}
% can be used for the key and the value. If the value is not given,
% \cs{kv@value} has the meaning \cs{relax}.
%
% \begin{declcs}{kv@unset@family@handler}\,\M{family}
% \end{declcs}
% It removes the family handler for \meta{family}.
% Since 2011/10/18 v1.15.
%
% \subsection{Put it all together}
%
% \begin{declcs}{kvsetkeys}\,\M{family}\,\M{key value list}
% \end{declcs}
% Macro \cs{kvsetkeys} processes the \meta{key value list} with
% the standard processor \cs{kv@processor@default}:
% \begin{quote}
%   \cs{kv@parse}\,\M{key value list}^^A
%   \texttt{\{}\cs{kv@processor@default}\,\M{family}\texttt{\}}
% \end{quote}
%
% \begin{declcs}{kvsetknownkeys}\,\M{family}\,\M{cmd}\,\M{key value list}
% \end{declcs}
% Macro \cs{kvsetknownkeys} processes the \meta{key value list}
% with processor \cs{kv@processor@known}. All key value pairs
% with keys that are not known in \meta{family} are stored
% in macro \meta{cmd}. A previous contents of macro \meta{cmd}
% will be overwritten. If all keys can be handled, \meta{cmd}
% will be empty, otherwise it contains a key value list of
% unhandled key value pairs.
% Since 2011/10/18 v1.15.
%
% Pseudo code:
% \begin{quote}
%   create macro \meta{cmdaux} with unique name (inside the current group)\\
%   \cs{def}\meta{cmdaux}\{\}\\
%   \cs{kv@parse}\,\M{key value list}^^A
%   \texttt{\{}\cs{kv@processor@known}\,\M{family}\,\M{cmdaux}\texttt{\}}\\
%   \cs{let}\meta{cmd}=\meta{cmdaux}
% \end{quote}
%
% \begin{declcs}{kvsetkeys@expandafter}\,\M{family}\,\M{list cmd}\\
%   \cs{kvsetknownkeys@expandafter}\,\M{family}\,%
%   \M{cmd}\,\M{list cmd}
% \end{declcs}
% Both macros behave like the counterparts without suffix
% |@expandafter|. The difference is that the key value list is
% given as macro that is expanded once.
% Since 2011/10/18 v1.15.
%
% Thus you can replace \cs{setkeys} of package \xpackage{keyval}
% by the key value parser of this package:
% \begin{quote}
%   |\renewcommand*{\setkeys}{\kvsetkeys}|\\
%   or\\
%   |\let\setkeys\kvsetkeys|
% \end{quote}
%
% \subsection{Comma separated lists}
%
% Since version 2007/09/29 v1.3 this package also supports the normalizing
% and parsing of general comma separated lists.
%
% \begin{declcs}{comma@normalize}\,\M{comma list}
% \end{declcs}
% Macro \cs{comma@normalize} normalizes the comma separated list,
% removes spaces around commas. The result is put in macro \cs{comma@list}.
%
% \begin{declcs}{comma@parse}\,\M{comma list}\,\M{processor}
% \end{declcs}
% Macro \cs{comma@parse} first normalizes the comma separated list
% and then parses the list by calling \cs{comma@parse@normalized}.
%
% \begin{declcs}{comma@parse@normalized}\,\M{normalized comma list}^^A
%   \,\M{processor}
% \end{declcs}
% The list is parsed. Empty entries are ignored. \meta{processor}
% is called for each non-empty entry with the entry as argument:
% \begin{quote}
%   \meta{processor}|{|\meta{entry}|}|
% \end{quote}
% Also the entry is stored in the macro \cs{comma@entry}.
%
% \begin{declcs}{comma@break}
% \end{declcs}
% Since version 2011/03/03 v1.11 \cs{comma@break} can be
% called inside the \meta{processor}
% of \cs{comma@parse} or \cs{comma@parse@normalized}, then
% the processing is stopped and the following entries discarded.
%
% \section{Example}
%
% The following example prints a short piece of HTML code using
% the tabbing environment for indenting purpose and a key value
% syntax for specifying the attributes of an HTML tag.
% The example illustrates the use of a default family handler.
%    \begin{macrocode}
%<*example>
\documentclass{article}
\usepackage[T1]{fontenc}
\usepackage{kvsetkeys}
\usepackage{keyval}

\makeatletter
\newcommand*{\tag}[2][]{%
  % #1: attributes
  % #2: tag name
  \begingroup
    \toks@={}%
    \let\@endslash\@empty
    \kvsetkeys{tag}{#1}%
    \texttt{%
      \textless #2\the\toks@\@endslash\textgreater
    }%
  \endgroup
}
\kv@set@family@handler{tag}{%
  % #1: key
  % #2: value
  \toks@\expandafter{%
    \the\toks@
    \space
    #1=\string"#2\string"%
  }%
}
\define@key{tag}{/}[]{%
  \def\@endslash{/}%
}
\makeatother

\begin{document}
\begin{tabbing}
  \mbox{}\qquad\=\qquad\=\kill
  \tag{html}\\
  \>\dots\\
  \>\tag[border=1]{table}\\
  \>\>\tag[width=200, span=3, /]{colgroup}\\
  \>\>\dots\\
  \>\tag{/table}\\
  \>\dots\\
  \tag{/html}\\
\end{tabbing}
\end{document}
%</example>
%    \end{macrocode}
%
% \StopEventually{
% }
%
% \section{Implementation}
%
% \subsection{Identification}
%
%    \begin{macrocode}
%<*package>
%    \end{macrocode}
%    Reload check, especially if the package is not used with \LaTeX.
%    \begin{macrocode}
\begingroup\catcode61\catcode48\catcode32=10\relax%
  \catcode13=5 % ^^M
  \endlinechar=13 %
  \catcode35=6 % #
  \catcode39=12 % '
  \catcode44=12 % ,
  \catcode45=12 % -
  \catcode46=12 % .
  \catcode58=12 % :
  \catcode64=11 % @
  \catcode123=1 % {
  \catcode125=2 % }
  \expandafter\let\expandafter\x\csname ver@kvsetkeys.sty\endcsname
  \ifx\x\relax % plain-TeX, first loading
  \else
    \def\empty{}%
    \ifx\x\empty % LaTeX, first loading,
      % variable is initialized, but \ProvidesPackage not yet seen
    \else
      \expandafter\ifx\csname PackageInfo\endcsname\relax
        \def\x#1#2{%
          \immediate\write-1{Package #1 Info: #2.}%
        }%
      \else
        \def\x#1#2{\PackageInfo{#1}{#2, stopped}}%
      \fi
      \x{kvsetkeys}{The package is already loaded}%
      \aftergroup\endinput
    \fi
  \fi
\endgroup%
%    \end{macrocode}
%    Package identification:
%    \begin{macrocode}
\begingroup\catcode61\catcode48\catcode32=10\relax%
  \catcode13=5 % ^^M
  \endlinechar=13 %
  \catcode35=6 % #
  \catcode39=12 % '
  \catcode40=12 % (
  \catcode41=12 % )
  \catcode44=12 % ,
  \catcode45=12 % -
  \catcode46=12 % .
  \catcode47=12 % /
  \catcode58=12 % :
  \catcode64=11 % @
  \catcode91=12 % [
  \catcode93=12 % ]
  \catcode123=1 % {
  \catcode125=2 % }
  \expandafter\ifx\csname ProvidesPackage\endcsname\relax
    \def\x#1#2#3[#4]{\endgroup
      \immediate\write-1{Package: #3 #4}%
      \xdef#1{#4}%
    }%
  \else
    \def\x#1#2[#3]{\endgroup
      #2[{#3}]%
      \ifx#1\@undefined
        \xdef#1{#3}%
      \fi
      \ifx#1\relax
        \xdef#1{#3}%
      \fi
    }%
  \fi
\expandafter\x\csname ver@kvsetkeys.sty\endcsname
\ProvidesPackage{kvsetkeys}%
  [2019/12/15 v1.18 Key value parser (HO)]%
%    \end{macrocode}
%
%    \begin{macrocode}
\begingroup\catcode61\catcode48\catcode32=10\relax%
  \catcode13=5 % ^^M
  \endlinechar=13 %
  \catcode123=1 % {
  \catcode125=2 % }
  \catcode64=11 % @
  \def\x{\endgroup
    \expandafter\edef\csname KVS@AtEnd\endcsname{%
      \endlinechar=\the\endlinechar\relax
      \catcode13=\the\catcode13\relax
      \catcode32=\the\catcode32\relax
      \catcode35=\the\catcode35\relax
      \catcode61=\the\catcode61\relax
      \catcode64=\the\catcode64\relax
      \catcode123=\the\catcode123\relax
      \catcode125=\the\catcode125\relax
    }%
  }%
\x\catcode61\catcode48\catcode32=10\relax%
\catcode13=5 % ^^M
\endlinechar=13 %
\catcode35=6 % #
\catcode64=11 % @
\catcode123=1 % {
\catcode125=2 % }
\def\TMP@EnsureCode#1#2{%
  \edef\KVS@AtEnd{%
    \KVS@AtEnd
    \catcode#1=\the\catcode#1\relax
  }%
  \catcode#1=#2\relax
}
\TMP@EnsureCode{36}{3}% $
\TMP@EnsureCode{38}{4}% &
\TMP@EnsureCode{39}{12}% '
\TMP@EnsureCode{43}{12}% +
\TMP@EnsureCode{44}{12}% ,
\TMP@EnsureCode{45}{12}% -
\TMP@EnsureCode{46}{12}% .
\TMP@EnsureCode{47}{12}% /
\TMP@EnsureCode{91}{12}% [
\TMP@EnsureCode{93}{12}% ]
\TMP@EnsureCode{94}{7}% ^ (superscript)
\TMP@EnsureCode{96}{12}% `
\TMP@EnsureCode{126}{13}% ~ (active)
\edef\KVS@AtEnd{\KVS@AtEnd\noexpand\endinput}
%    \end{macrocode}
%
% \subsection{Package loading}
%
%    \begin{macrocode}
\begingroup\expandafter\expandafter\expandafter\endgroup
\expandafter\ifx\csname RequirePackage\endcsname\relax
  \def\TMP@RequirePackage#1[#2]{%
    \begingroup\expandafter\expandafter\expandafter\endgroup
    \expandafter\ifx\csname ver@#1.sty\endcsname\relax
      \input #1.sty\relax
    \fi
  }%
  \TMP@RequirePackage{infwarerr}[2007/09/09]%
  \let\PackageError\@PackageError
\else
\fi
%    \end{macrocode}
%
%    \begin{macrocode}
\expandafter\ifx\csname toks@\endcsname\relax
  \toksdef\toks@=0 %
\fi
%    \end{macrocode}
%
% \subsection{Check for \eTeX}
%
%    \cs{unexpanded}, \cs{ifcsname}, and \cs{unless} are used if found.
%    \begin{macrocode}
\ifx\numexpr\@undefined
  \catcode`\$=9 % ignore
  \catcode`\&=14 % comment
\else % e-TeX
  \catcode`\$=14 % comment
  \catcode`\&=9  % ignore
\fi
%    \end{macrocode}
%
% \subsection{Generic help macros}
%
%    \begin{macro}{\KVS@Empty}
%    \begin{macrocode}
\def\KVS@Empty{}
%    \end{macrocode}
%    \end{macro}
%    \begin{macro}{\KVS@FirstOfTwo}
%    \begin{macrocode}
\long\def\KVS@FirstOfTwo#1#2{#1}
%    \end{macrocode}
%    \end{macro}
%    \begin{macro}{\KVS@SecondOfTwo}
%    \begin{macrocode}
\long\def\KVS@SecondOfTwo#1#2{#2}
%    \end{macrocode}
%    \end{macro}
%
%    \begin{macro}{\KVS@IfEmpty}
%    \begin{macrocode}
\long\def\KVS@IfEmpty#1{%
& \edef\KVS@Temp{\unexpanded{#1}}%
$ \begingroup
$   \toks@{#1}%
$   \edef\KVS@Temp{\the\toks@}%
$ \expandafter\endgroup
  \ifx\KVS@Temp\KVS@Empty
    \expandafter\KVS@FirstOfTwo
  \else
    \expandafter\KVS@SecondOfTwo
  \fi
}
%    \end{macrocode}
%    \end{macro}
%
% \subsection{Normalizing}
%
%    \begin{macro}{\kv@normalize}
%    \begin{macrocode}
\long\def\kv@normalize#1{%
  \begingroup
    \toks@{,#1,}%
    \KVS@Comma
    \KVS@SpaceComma
    \KVS@CommaSpace
    \KVS@CommaComma
    \KVS@Equals
    \KVS@SpaceEquals
    \KVS@EqualsSpace
    \xdef\KVS@Global{\the\toks@}%
  \endgroup
  \let\kv@list\KVS@Global
}
%    \end{macrocode}
%    \end{macro}
%    \begin{macro}{\comma@normalize}
%    \begin{macrocode}
\def\comma@normalize#1{%
  \begingroup
    \toks@{,#1,}%
    \KVS@Comma
    \KVS@SpaceComma
    \KVS@CommaSpace
    \KVS@CommaComma
    \xdef\KVS@Global{\the\toks@}%
  \endgroup
  \let\comma@list\KVS@Global
}
%    \end{macrocode}
%    \end{macro}
%    \begin{macro}{\KVS@Comma}
%    Converts active commas into comma with catcode other.
%    Also adds a comma at the end to protect the last value
%    for next cleanup steps.
%    \begin{macrocode}
\begingroup
  \lccode`\,=`\,%
  \lccode`\~=`\,%
\lowercase{\endgroup
  \def\KVS@Comma{%
    \toks@\expandafter{\expandafter}\expandafter
    \KVS@@Comma\the\toks@~\KVS@Nil
  }%
  \long\def\KVS@@Comma#1~#2\KVS@Nil{%
    \toks@\expandafter{\the\toks@#1}%
    \KVS@IfEmpty{#2}{%
    }{%
      \KVS@@Comma,#2\KVS@Nil
    }%
  }%
}
%    \end{macrocode}
%    \end{macro}
%    \begin{macro}{\KVS@SpaceComma}
%    Removes spaces before the comma, may add commas at the end.
%    \begin{macrocode}
\def\KVS@SpaceComma#1{%
  \def\KVS@SpaceComma{%
    \expandafter\KVS@@SpaceComma\the\toks@#1,\KVS@Nil
  }%
}
\KVS@SpaceComma{ }
%    \end{macrocode}
%    \end{macro}
%    \begin{macro}{\KVS@@SpaceComma}
%    \begin{macrocode}
\long\def\KVS@@SpaceComma#1 ,#2\KVS@Nil{%
  \KVS@IfEmpty{#2}{%
    \toks@{#1}%
  }{%
    \KVS@@SpaceComma#1,#2\KVS@Nil
  }%
}
%    \end{macrocode}
%    \end{macro}
%    \begin{macro}{\KVS@CommaSpace}
%    Removes spaces after the comma, may add commas at the end.
%    \begin{macrocode}
\def\KVS@CommaSpace{%
  \expandafter\KVS@@CommaSpace\the\toks@, \KVS@Nil
}
%    \end{macrocode}
%    \end{macro}
%    \begin{macro}{\KVS@@CommaSpace}
%    \begin{macrocode}
\long\def\KVS@@CommaSpace#1, #2\KVS@Nil{%
  \KVS@IfEmpty{#2}{%
    \toks@{#1}%
  }{%
    \KVS@@CommaSpace#1,#2\KVS@Nil
  }%
}
%    \end{macrocode}
%    \end{macro}
%    \begin{macro}{\KVS@CommaComma}
%    Replaces multiple commas by one comma.
%    \begin{macrocode}
\def\KVS@CommaComma{%
  \expandafter\KVS@@CommaComma\the\toks@,\KVS@Nil
}
%    \end{macrocode}
%    \end{macro}
%    \begin{macro}{\KVS@@CommaComma}
%    \begin{macrocode}
\long\def\KVS@@CommaComma#1,,#2\KVS@Nil{%
  \KVS@IfEmpty{#2}{%
    \toks@{#1,}% (!)
  }{%
    \KVS@@CommaComma#1,#2\KVS@Nil
  }%
}
%    \end{macrocode}
%    \end{macro}
%
%    \begin{macro}{\KVS@Equals}
%    Converts active equals signs into catcode other characters.
%    \begin{macrocode}
\begingroup
  \lccode`\==`\=%
  \lccode`\~=`\=%
\lowercase{\endgroup
  \def\KVS@Equals{%
    \toks@\expandafter{\expandafter}\expandafter
    \KVS@@Equals\the\toks@~\KVS@Nil
  }%
  \long\def\KVS@@Equals#1~#2\KVS@Nil{%
    \edef\KVS@Temp{\the\toks@}%
    \ifx\KVS@Temp\KVS@Empty
      \expandafter\KVS@FirstOfTwo
    \else
      \expandafter\KVS@SecondOfTwo
    \fi
    {%
      \toks@{#1}%
    }{%
      \toks@\expandafter{\the\toks@=#1}%
    }%
    \KVS@IfEmpty{#2}{%
    }{%
      \KVS@@Equals#2\KVS@Nil
    }%
  }%
}
%    \end{macrocode}
%    \end{macro}
%    \begin{macro}{\KVS@SpaceEquals}
%    Removes spaces before the equals sign.
%    \begin{macrocode}
\def\KVS@SpaceEquals#1{%
  \def\KVS@SpaceEquals{%
    \expandafter\KVS@@SpaceEquals\the\toks@#1=\KVS@Nil
  }%
}
\KVS@SpaceEquals{ }
%    \end{macrocode}
%    \end{macro}
%    \begin{macro}{\KVS@@SpaceEquals}
%    \begin{macrocode}
\long\def\KVS@@SpaceEquals#1 =#2\KVS@Nil{%
  \KVS@IfEmpty{#2}{%
    \toks@{#1}%
  }{%
    \KVS@@SpaceEquals#1=#2\KVS@Nil
  }%
}
%    \end{macrocode}
%    \end{macro}
%    \begin{macro}{\KVS@EqualsSpace}
%    Removes spaces after the equals sign.
%    \begin{macrocode}
\def\KVS@EqualsSpace{%
  \expandafter\KVS@@EqualsSpace\the\toks@= \KVS@Nil
}
%    \end{macrocode}
%    \end{macro}
%    \begin{macro}{\KVS@@EqualsSpace}
%    \begin{macrocode}
\long\def\KVS@@EqualsSpace#1= #2\KVS@Nil{%
  \KVS@IfEmpty{#2}{%
    \toks@{#1}%
  }{%
    \KVS@@EqualsSpace#1=#2\KVS@Nil
  }%
}
%    \end{macrocode}
%    \end{macro}
%
% \subsection{Parsing key value lists}
%
%    \begin{macro}{\kv@parse}
%    Normalizes and parses the key value list. Also sets \cs{kv@list}.
%    \begin{macrocode}
\long\def\kv@parse#1{%
  \kv@normalize{#1}%
  \expandafter\kv@parse@normalized\expandafter{\kv@list}%
}
%    \end{macrocode}
%    \end{macro}
%    \begin{macro}{\kv@parse@normalized}
%    |#1|: key value list\\
%    |#2|: processor
%    \begin{macrocode}
\long\def\kv@parse@normalized#1#2{%
  \KVS@Parse#1,\KVS@Nil{#2}%
}
%    \end{macrocode}
%    \end{macro}
%    \begin{macro}{\KVS@Parse}
%    |#1,#2|: key value list\\
%    |#3|: processor
%    \begin{macrocode}
\long\def\KVS@Parse#1,#2\KVS@Nil#3{%
  \KVS@IfEmpty{#1}{%
  }{%
    \KVS@Process#1=\KVS@Nil{#3}%
  }%
  \KVS@MaybeBreak
  \KVS@IfEmpty{#2}{%
  }{%
    \KVS@Parse#2\KVS@Nil{#3}%
  }%
}
%    \end{macrocode}
%    \end{macro}
%    \begin{macro}{\KVS@Process}
%    |#1|: key\\
%    |#2|: value, |=|\\
%    |#3|: processor
%    \begin{macrocode}
\long\def\KVS@Process#1=#2\KVS@Nil#3{%
  \let\KVS@MaybeBreak\relax
  \def\kv@key{#1}%
  \KVS@IfEmpty{#2}{%
    \let\kv@value\relax
    #3{#1}{}%
  }{%
    \KVS@@Process{#1}#2\KVS@Nil{#3}%
  }%
}
%    \end{macrocode}
%    \end{macro}
%    \begin{macro}{\KVS@@Process}
%    |#1|: key\\
%    |#2|: value\\
%    |#3|: processor
%    \begin{macrocode}
\long\def\KVS@@Process#1#2=\KVS@Nil#3{%
& \edef\kv@value{\unexpanded{#2}}%
$ \begingroup
$   \toks@{#2}%
$   \xdef\KVS@Global{\the\toks@}%
$ \endgroup
$ \let\kv@value\KVS@Global
  #3{#1}{#2}%
}
%    \end{macrocode}
%    \end{macro}
%    \begin{macro}{\KVS@MaybeBreak}
%    \begin{macrocode}
\let\KVS@MaybeBreak\relax
%    \end{macrocode}
%    \end{macro}
%    \begin{macro}{\KVS@break}
%    \begin{macrocode}
\def\KVS@break#1#2#3#4{%
  \let\KVS@MaybeBreak\relax
}
%    \end{macrocode}
%    \end{macro}
%    \begin{macro}{\kv@break}
%    \begin{macrocode}
\def\kv@break{%
  \let\KVS@MaybeBreak\KVS@break
}
%    \end{macrocode}
%    \end{macro}
%
% \subsection{Parsing comma lists}
%
%    \begin{macro}{\comma@parse}
%    Normalizes and parses the key value list. Also sets \cs{comma@list}.
%    \begin{macrocode}
\def\comma@parse#1{%
  \comma@normalize{#1}%
  \expandafter\comma@parse@normalized\expandafter{\comma@list}%
}
%    \end{macrocode}
%    \end{macro}
%
%    \begin{macro}{\comma@parse@normalized}
%    |#1|: comma list\\
%    |#2|: processor
%    \begin{macrocode}
\def\comma@parse@normalized#1#2{%
  \KVS@CommaParse#1,\KVS@Nil{#2}%
}
%    \end{macrocode}
%    \end{macro}
%    \begin{macro}{\KVS@CommaParse}
%    |#1,#2|: comma list\\
%    |#3|: processor
%    \begin{macrocode}
\def\KVS@CommaParse#1,#2\KVS@Nil#3{%
  \KVS@IfEmpty{#1}{%
  }{%
    \def\comma@entry{#1}%
    #3{#1}%
  }%
  \KVS@MaybeBreak
  \KVS@IfEmpty{#2}{%
  }{%
    \KVS@CommaParse#2\KVS@Nil{#3}%
  }%
}
%    \end{macrocode}
%    \end{macro}
%    \begin{macro}{\comma@break}
%    \begin{macrocode}
\def\comma@break{%
  \let\KVS@MaybeBreak\KVS@break
}
%    \end{macrocode}
%    \end{macro}
%
% \subsection{Processing key value pairs}
%
%    \begin{macro}{\kv@handled@false}
%    The handler can call \cs{kv@handled@false} or
%    \cs{kv@handled@true} so report failure or success.
%    The default is success (compatibility for versions
%    before 2011/10/18 v1.15).
%    \begin{macrocode}
\def\kv@handled@false{%
  \let\ifkv@handled@\iffalse
}
%    \end{macrocode}
%    \end{macro}
%    \begin{macro}{\kv@handled@true}
%    \begin{macrocode}
\def\kv@handled@true{%
  \let\ifkv@handled@\iftrue
}
%    \end{macrocode}
%    \end{macro}
%    \begin{macro}{\ifkv@handled@}
%    \begin{macrocode}
\kv@handled@true
%    \end{macrocode}
%    \end{macro}
%
%    \begin{macro}{\kv@processor@default}
%    \begin{macrocode}
\def\kv@processor@default#1#2{%
  \begingroup
    \csname @safe@activestrue\endcsname
    \let\ifincsname\iftrue
    \edef\KVS@temp{\endgroup
      \noexpand\KVS@ProcessorDefault{#1}{#2}%
    }%
  \KVS@temp
}
%    \end{macrocode}
%    \end{macro}
%    \begin{macro}{\KVS@ProcessorDefault}
%    \begin{macrocode}
\long\def\KVS@ProcessorDefault#1#2#3{%
  \def\kv@fam{#1}%
& \unless\ifcsname KV@#1@#2\endcsname
$ \begingroup\expandafter\expandafter\expandafter\endgroup
$ \expandafter\ifx\csname KV@#1@#2\endcsname\relax
&   \unless\ifcsname KVS@#1@handler\endcsname
$   \begingroup\expandafter\expandafter\expandafter\endgroup
$   \expandafter\ifx\csname KVS@#1@handler\endcsname\relax
      \kv@error@unknownkey{#1}{#2}%
    \else
      \kv@handled@true
      \csname KVS@#1@handler\endcsname{#2}{#3}%
      \relax
      \ifkv@handled@
      \else
        \kv@error@unknownkey{#1}{#2}%
      \fi
    \fi
  \else
    \ifx\kv@value\relax
&     \unless\ifcsname KV@#1@#2@default\endcsname
$     \begingroup\expandafter\expandafter\expandafter\endgroup
$     \expandafter\ifx\csname KV@#1@#2@default\endcsname\relax
        \kv@error@novalue{#1}{#2}%
      \else
        \csname KV@#1@#2@default\endcsname
        \relax
      \fi
    \else
      \csname KV@#1@#2\endcsname{#3}%
    \fi
  \fi
}
%    \end{macrocode}
%    \end{macro}
%
%    \begin{macro}{\kv@processor@known}
%    \begin{macrocode}
\def\kv@processor@known#1#2#3{%
  \begingroup
    \csname @safe@activestrue\endcsname
    \let\ifincsname\iftrue
    \edef\KVS@temp{\endgroup
      \noexpand\KVS@ProcessorKnown{#1}\noexpand#2{#3}%
    }%
  \KVS@temp
}
%    \end{macrocode}
%    \end{macro}
%    \begin{macro}{\KVS@ProcessorKnown}
%    \begin{macrocode}
\long\def\KVS@ProcessorKnown#1#2#3#4{%
  \def\kv@fam{#1}%
& \unless\ifcsname KV@#1@#3\endcsname
$ \begingroup\expandafter\expandafter\expandafter\endgroup
$ \expandafter\ifx\csname KV@#1@#3\endcsname\relax
&   \unless\ifcsname KVS@#1@handler\endcsname
$   \begingroup\expandafter\expandafter\expandafter\endgroup
$   \expandafter\ifx\csname KVS@#1@handler\endcsname\relax
      \KVS@AddUnhandled#2{#3}{#4}%
    \else
      \kv@handled@true
      \csname KVS@#1@handler\endcsname{#3}{#4}%
      \relax
      \ifkv@handled@
      \else
        \KVS@AddUnhandled#2{#3}{#4}%
      \fi
    \fi
  \else
    \ifx\kv@value\relax
&     \unless\ifcsname KV@#1@#2@default\endcsname
$     \begingroup\expandafter\expandafter\expandafter\endgroup
$     \expandafter\ifx\csname KV@#1@#3@default\endcsname\relax
        \kv@error@novalue{#1}{#3}%
      \else
        \csname KV@#1@#3@default\endcsname
        \relax
      \fi
    \else
      \csname KV@#1@#3\endcsname{#4}%
    \fi
  \fi
}
%    \end{macrocode}
%    \end{macro}
%    \begin{macro}{\KVS@AddUnhandled}
%    \begin{macrocode}
\long\def\KVS@AddUnhandled#1#2#3{%
& \edef#1{%
&   \ifx#1\KVS@empty
&   \else
&     \unexpanded{#1},%
&   \fi
&   \unexpanded{{#2}={#3}}%
& }%
$ \begingroup
$   \ifx#1\KVS@empty
$     \toks@{{#2}={#3}}%
$   \else
$     \toks@\expandafter{#1,{#2}={#3}}%
$   \fi
$   \xdef\KVS@Global{\the\toks@}%
$ \endgroup
$ \let#1\KVS@Global
}
%    \end{macrocode}
%    \end{macro}
%
%    \begin{macro}{\kv@set@family@handler}
%    \begin{macrocode}
\long\def\kv@set@family@handler#1#2{%
  \begingroup
    \csname @safe@activestrue\endcsname
    \let\ifincsname\iftrue
  \expandafter\endgroup
  \expandafter\def\csname KVS@#1@handler\endcsname##1##2{#2}%
}
%    \end{macrocode}
%    \end{macro}
%
%    \begin{macro}{\kv@unset@family@handler}
%    \begin{macrocode}
\long\def\kv@unset@family@handler#1#2{%
  \begingroup
    \csname @safe@activestrue\endcsname
    \let\ifincsname\iftrue
  \expandafter\endgroup
  \expandafter\let\csname KVS@#1@handler\endcsname\@UnDeFiNeD
}
%    \end{macrocode}
%    \end{macro}
%
% \subsection{Error handling}
%
%    \begin{macro}{\kv@error@novalue}
%    \begin{macrocode}
\def\kv@error@novalue{%
  \kv@error@generic{No value specified for}%
}
%    \end{macrocode}
%    \end{macro}
%    \begin{macro}{\kv@error@unknownkey}
%    \begin{macrocode}
\def\kv@error@unknownkey{%
  \kv@error@generic{Undefined}%
}
%    \end{macrocode}
%    \end{macro}
%    \begin{macro}{\kv@error@generic}
%    \begin{macrocode}
\def\kv@error@generic#1#2#3{%
  \PackageError{kvsetkeys}{%
    #1 key `#3'%
  }{%
    The keyval family of the key `#3' is `#2'.\MessageBreak
    The setting of the key is ignored because of the error.\MessageBreak
    \MessageBreak
    \@ehc
  }%
}
%    \end{macrocode}
%    \end{macro}
%
% \subsection{Do it all}
%
%    \begin{macro}{\kvsetkeys}
%    \begin{macrocode}
\long\def\kvsetkeys#1#2{%
  \kv@parse{#2}{\kv@processor@default{#1}}%
}
%    \end{macrocode}
%    \end{macro}
%
%    \begin{macro}{\kvsetkeys@expandafter}
%    \begin{macrocode}
\def\kvsetkeys@expandafter#1#2{%
  \expandafter\kv@parse\expandafter{#2}{%
    \kv@processor@default{#1}%
  }%
}
%    \end{macrocode}
%    \end{macro}
%
%    \begin{macro}{\KVS@cmd}
%    \begin{macrocode}
\def\KVS@cmd{0}%
%    \end{macrocode}
%    \end{macro}
%    \begin{macro}{\KVS@cmd@inc}
%    \begin{macrocode}
\def\KVS@cmd@inc{%
& \edef\KVS@cmd{\the\numexpr\KVS@cmd+1}%
$ \begingroup
$   \count255=\KVS@cmd\relax
$   \advance\count255 by 1\relax
$ \edef\x{\endgroup
$   \noexpand\def\noexpand\KVS@cmd{\number\count255}%
$ }%
$ \x
}
%    \end{macrocode}
%    \end{macro}
%    \begin{macro}{\KVS@cmd@dec}
%    \begin{macrocode}
\def\KVS@cmd@dec{%
& \edef\KVS@cmd{\the\numexpr\KVS@cmd-1}%
$ \begingroup
$   \count255=\KVS@cmd\relax
$   \advance\count255 by -1\relax
$ \edef\x{\endgroup
$   \noexpand\def\noexpand\KVS@cmd{\number\count255}%
$ }%
$ \x
}
%    \end{macrocode}
%    \end{macro}
%    \begin{macro}{\KVS@empty}
%    \begin{macrocode}
\def\KVS@empty{}
%    \end{macrocode}
%    \end{macro}
%
%    \begin{macro}{\kvsetknownkeys}
%    \begin{macrocode}
\def\kvsetknownkeys{%
  \expandafter
  \KVS@setknownkeys\csname KVS@cmd\KVS@cmd\endcsname{}%
}
%    \end{macrocode}
%    \end{macro}
%    \begin{macro}{\KVS@setknownkeys}
%    \begin{macrocode}
\long\def\KVS@setknownkeys#1#2#3#4#5{%
  \let#1\KVS@empty
  \KVS@cmd@inc
  #2\kv@parse#2{#5}{\kv@processor@known{#3}#1}%
  \KVS@cmd@dec
  \let#4=#1%
}
%    \end{macrocode}
%    \end{macro}
%
%    \begin{macro}{\kvsetknownkeys@expandafter}
%    \begin{macrocode}
\def\kvsetknownkeys@expandafter{%
  \expandafter
  \KVS@setknownkeys
      \csname KVS@cmd\KVS@cmd\endcsname\expandafter
}
%    \end{macrocode}
%    \end{macro}
%
%    \begin{macrocode}
\KVS@AtEnd%
%</package>
%    \end{macrocode}
%% \section{Installation}
%
% \subsection{Download}
%
% \paragraph{Package.} This package is available on
% CTAN\footnote{\CTANpkg{kvsetkeys}}:
% \begin{description}
% \item[\CTAN{macros/latex/contrib/kvsetkeys/kvsetkeys.dtx}] The source file.
% \item[\CTAN{macros/latex/contrib/kvsetkeys/kvsetkeys.pdf}] Documentation.
% \end{description}
%
%
% \paragraph{Bundle.} All the packages of the bundle `kvsetkeys'
% are also available in a TDS compliant ZIP archive. There
% the packages are already unpacked and the documentation files
% are generated. The files and directories obey the TDS standard.
% \begin{description}
% \item[\CTANinstall{install/macros/latex/contrib/kvsetkeys.tds.zip}]
% \end{description}
% \emph{TDS} refers to the standard ``A Directory Structure
% for \TeX\ Files'' (\CTANpkg{tds}). Directories
% with \xfile{texmf} in their name are usually organized this way.
%
% \subsection{Bundle installation}
%
% \paragraph{Unpacking.} Unpack the \xfile{kvsetkeys.tds.zip} in the
% TDS tree (also known as \xfile{texmf} tree) of your choice.
% Example (linux):
% \begin{quote}
%   |unzip kvsetkeys.tds.zip -d ~/texmf|
% \end{quote}
%
% \subsection{Package installation}
%
% \paragraph{Unpacking.} The \xfile{.dtx} file is a self-extracting
% \docstrip\ archive. The files are extracted by running the
% \xfile{.dtx} through \plainTeX:
% \begin{quote}
%   \verb|tex kvsetkeys.dtx|
% \end{quote}
%
% \paragraph{TDS.} Now the different files must be moved into
% the different directories in your installation TDS tree
% (also known as \xfile{texmf} tree):
% \begin{quote}
% \def\t{^^A
% \begin{tabular}{@{}>{\ttfamily}l@{ $\rightarrow$ }>{\ttfamily}l@{}}
%   kvsetkeys.sty & tex/generic/kvsetkeys/kvsetkeys.sty\\
%   kvsetkeys.pdf & doc/latex/kvsetkeys/kvsetkeys.pdf\\
%   kvsetkeys-example.tex & doc/latex/kvsetkeys/kvsetkeys-example.tex\\
%   kvsetkeys.dtx & source/latex/kvsetkeys/kvsetkeys.dtx\\
% \end{tabular}^^A
% }^^A
% \sbox0{\t}^^A
% \ifdim\wd0>\linewidth
%   \begingroup
%     \advance\linewidth by\leftmargin
%     \advance\linewidth by\rightmargin
%   \edef\x{\endgroup
%     \def\noexpand\lw{\the\linewidth}^^A
%   }\x
%   \def\lwbox{^^A
%     \leavevmode
%     \hbox to \linewidth{^^A
%       \kern-\leftmargin\relax
%       \hss
%       \usebox0
%       \hss
%       \kern-\rightmargin\relax
%     }^^A
%   }^^A
%   \ifdim\wd0>\lw
%     \sbox0{\small\t}^^A
%     \ifdim\wd0>\linewidth
%       \ifdim\wd0>\lw
%         \sbox0{\footnotesize\t}^^A
%         \ifdim\wd0>\linewidth
%           \ifdim\wd0>\lw
%             \sbox0{\scriptsize\t}^^A
%             \ifdim\wd0>\linewidth
%               \ifdim\wd0>\lw
%                 \sbox0{\tiny\t}^^A
%                 \ifdim\wd0>\linewidth
%                   \lwbox
%                 \else
%                   \usebox0
%                 \fi
%               \else
%                 \lwbox
%               \fi
%             \else
%               \usebox0
%             \fi
%           \else
%             \lwbox
%           \fi
%         \else
%           \usebox0
%         \fi
%       \else
%         \lwbox
%       \fi
%     \else
%       \usebox0
%     \fi
%   \else
%     \lwbox
%   \fi
% \else
%   \usebox0
% \fi
% \end{quote}
% If you have a \xfile{docstrip.cfg} that configures and enables \docstrip's
% TDS installing feature, then some files can already be in the right
% place, see the documentation of \docstrip.
%
% \subsection{Refresh file name databases}
%
% If your \TeX~distribution
% (\TeX\,Live, \mikTeX, \dots) relies on file name databases, you must refresh
% these. For example, \TeX\,Live\ users run \verb|texhash| or
% \verb|mktexlsr|.
%
% \subsection{Some details for the interested}
%
% \paragraph{Unpacking with \LaTeX.}
% The \xfile{.dtx} chooses its action depending on the format:
% \begin{description}
% \item[\plainTeX:] Run \docstrip\ and extract the files.
% \item[\LaTeX:] Generate the documentation.
% \end{description}
% If you insist on using \LaTeX\ for \docstrip\ (really,
% \docstrip\ does not need \LaTeX), then inform the autodetect routine
% about your intention:
% \begin{quote}
%   \verb|latex \let\install=y\input{kvsetkeys.dtx}|
% \end{quote}
% Do not forget to quote the argument according to the demands
% of your shell.
%
% \paragraph{Generating the documentation.}
% You can use both the \xfile{.dtx} or the \xfile{.drv} to generate
% the documentation. The process can be configured by the
% configuration file \xfile{ltxdoc.cfg}. For instance, put this
% line into this file, if you want to have A4 as paper format:
% \begin{quote}
%   \verb|\PassOptionsToClass{a4paper}{article}|
% \end{quote}
% An example follows how to generate the
% documentation with pdf\LaTeX:
% \begin{quote}
%\begin{verbatim}
%pdflatex kvsetkeys.dtx
%makeindex -s gind.ist kvsetkeys.idx
%pdflatex kvsetkeys.dtx
%makeindex -s gind.ist kvsetkeys.idx
%pdflatex kvsetkeys.dtx
%\end{verbatim}
% \end{quote}
%
% \begin{thebibliography}{9}
% \bibitem{tb94wright}
%   A guide to key-value methods, Joseph Wright, second draft for
%   \href{https://www.tug.org/tugboat}{TUGBoat}, 2009-03-17.
%   \url{https://www.texdev.net/uploads/2009/03/keyval.pdf}
%
% \bibitem{keyval}
%   David Carlisle:
%   \textit{The \xpackage{keyval} package};
%   1999/03/16 v1.13;
%   \CTANpkg{keyval}.
%
% \end{thebibliography}
%
% \begin{History}
%   \begin{Version}{2006/03/06 v1.0}
%   \item
%     First version.
%   \end{Version}
%   \begin{Version}{2006/10/19 v1.1}
%   \item
%     Fix of \cs{kv@set@family@handler}.
%   \item
%     Example added.
%   \end{Version}
%   \begin{Version}{2007/09/09 v1.2}
%   \item
%     Using package \xpackage{infwarerr} for error messages.
%   \item
%     Catcode section rewritten.
%   \end{Version}
%   \begin{Version}{2007/09/29 v1.3}
%   \item
%     Normalizing and parsing of comma separated lists added.
%   \item
%     \cs{kv@normalize} rewritten.
%   \item
%     Robustness increased for normalizing and parsing,
%     e.g. for values with unmatched conditionals.
%   \item
%     \eTeX\ is used if available.
%   \item
%     Tests added for normalizing and parsing.
%   \end{Version}
%   \begin{Version}{2009/07/19 v1.4}
%   \item
%     Bug fix for \cs{kv@normalize}: unwanted space removed (Florent Chervet).
%   \end{Version}
%   \begin{Version}{2009/07/30 v1.5}
%   \item
%     Documentation addition: recommendation for Joseph Wright's
%     review article.
%   \end{Version}
%   \begin{Version}{2009/12/12 v1.6}
%   \item
%     Short info shortened.
%   \end{Version}
%   \begin{Version}{2009/12/22 v1.7}
%   \item
%     Internal optimization (\cs{KVS@CommaSpace}, \dots, \cs{KVS@EqualsSpace}).
%   \end{Version}
%   \begin{Version}{2010/01/28 v1.8}
%   \item
%     Compatibility to ini\TeX\ added.
%   \end{Version}
%   \begin{Version}{2010/03/01 v1.9}
%   \item
%     Support of \cs{par} inside values.
%   \end{Version}
%   \begin{Version}{2011/01/30 v1.10}
%   \item
%     Already loaded package files are not input in \hologo{plainTeX}.
%   \end{Version}
%   \begin{Version}{2011/03/03 v1.11}
%   \item
%     \cs{kv@break} and \cs{comma@break} added.
%   \end{Version}
%   \begin{Version}{2011/04/05 v1.12}
%   \item
%     Error message with recovery action in help message
%     (request by GL).
%   \end{Version}
%   \begin{Version}{2011/04/07 v1.13}
%   \item
%     \cs{kv@processor@default} supports package \xpackage{babel}'s
%     shorthands.
%   \item
%     \cs{kv@set@family@handler} with shorthand support.
%   \end{Version}
%   \begin{Version}{2011/06/15 v1.14}
%   \item
%     Some optimizations in token register uses (GL, HO).
%   \end{Version}
%   \begin{Version}{2011/10/18 v1.15}
%   \item
%     \cs{kv@processor@known} and \cs{kvsetknownkeys} added.
%   \item
%     \cs{kvsetkeys@expandafter} and \cs{kvsetknownkeys@expandafter} added.
%   \item
%     Family handler can report success or failure by \cs{kv@handled@true}
%     or \cs{kv@handled@false}.
%   \item
%     \cs{kv@unset@family@handler} added.
%   \end{Version}
%   \begin{Version}{2012/04/25 v1.16}
%   \item
%     \cs{kv@processor@default} and \cs{kv@processor@known} define
%     macro \cs{kv@fam} for convenience.
%   \item
%     Catcode section: Catcode setting for \texttt{+} added for \hologo{eTeX}.
%   \end{Version}
%   \begin{Version}{2016/05/16 v1.17}
%   \item
%     Documentation updates.
%   \end{Version}
%   \begin{Version}{2019/12/15 v1.18}
%   \item
%     Documentation updates.
%   \item
%     Avoid \xpackage{etexcmds} and \xpackage{infwarwerr} in \LaTeX.
%   \end{Version}
% \end{History}
%
% \PrintIndex
%
% \Finale
\endinput
|
% \end{quote}
% Do not forget to quote the argument according to the demands
% of your shell.
%
% \paragraph{Generating the documentation.}
% You can use both the \xfile{.dtx} or the \xfile{.drv} to generate
% the documentation. The process can be configured by the
% configuration file \xfile{ltxdoc.cfg}. For instance, put this
% line into this file, if you want to have A4 as paper format:
% \begin{quote}
%   \verb|\PassOptionsToClass{a4paper}{article}|
% \end{quote}
% An example follows how to generate the
% documentation with pdf\LaTeX:
% \begin{quote}
%\begin{verbatim}
%pdflatex kvsetkeys.dtx
%makeindex -s gind.ist kvsetkeys.idx
%pdflatex kvsetkeys.dtx
%makeindex -s gind.ist kvsetkeys.idx
%pdflatex kvsetkeys.dtx
%\end{verbatim}
% \end{quote}
%
% \begin{thebibliography}{9}
% \bibitem{tb94wright}
%   A guide to key-value methods, Joseph Wright, second draft for
%   \href{https://www.tug.org/tugboat}{TUGBoat}, 2009-03-17.
%   \url{https://www.texdev.net/uploads/2009/03/keyval.pdf}
%
% \bibitem{keyval}
%   David Carlisle:
%   \textit{The \xpackage{keyval} package};
%   1999/03/16 v1.13;
%   \CTANpkg{keyval}.
%
% \end{thebibliography}
%
% \begin{History}
%   \begin{Version}{2006/03/06 v1.0}
%   \item
%     First version.
%   \end{Version}
%   \begin{Version}{2006/10/19 v1.1}
%   \item
%     Fix of \cs{kv@set@family@handler}.
%   \item
%     Example added.
%   \end{Version}
%   \begin{Version}{2007/09/09 v1.2}
%   \item
%     Using package \xpackage{infwarerr} for error messages.
%   \item
%     Catcode section rewritten.
%   \end{Version}
%   \begin{Version}{2007/09/29 v1.3}
%   \item
%     Normalizing and parsing of comma separated lists added.
%   \item
%     \cs{kv@normalize} rewritten.
%   \item
%     Robustness increased for normalizing and parsing,
%     e.g. for values with unmatched conditionals.
%   \item
%     \eTeX\ is used if available.
%   \item
%     Tests added for normalizing and parsing.
%   \end{Version}
%   \begin{Version}{2009/07/19 v1.4}
%   \item
%     Bug fix for \cs{kv@normalize}: unwanted space removed (Florent Chervet).
%   \end{Version}
%   \begin{Version}{2009/07/30 v1.5}
%   \item
%     Documentation addition: recommendation for Joseph Wright's
%     review article.
%   \end{Version}
%   \begin{Version}{2009/12/12 v1.6}
%   \item
%     Short info shortened.
%   \end{Version}
%   \begin{Version}{2009/12/22 v1.7}
%   \item
%     Internal optimization (\cs{KVS@CommaSpace}, \dots, \cs{KVS@EqualsSpace}).
%   \end{Version}
%   \begin{Version}{2010/01/28 v1.8}
%   \item
%     Compatibility to ini\TeX\ added.
%   \end{Version}
%   \begin{Version}{2010/03/01 v1.9}
%   \item
%     Support of \cs{par} inside values.
%   \end{Version}
%   \begin{Version}{2011/01/30 v1.10}
%   \item
%     Already loaded package files are not input in \hologo{plainTeX}.
%   \end{Version}
%   \begin{Version}{2011/03/03 v1.11}
%   \item
%     \cs{kv@break} and \cs{comma@break} added.
%   \end{Version}
%   \begin{Version}{2011/04/05 v1.12}
%   \item
%     Error message with recovery action in help message
%     (request by GL).
%   \end{Version}
%   \begin{Version}{2011/04/07 v1.13}
%   \item
%     \cs{kv@processor@default} supports package \xpackage{babel}'s
%     shorthands.
%   \item
%     \cs{kv@set@family@handler} with shorthand support.
%   \end{Version}
%   \begin{Version}{2011/06/15 v1.14}
%   \item
%     Some optimizations in token register uses (GL, HO).
%   \end{Version}
%   \begin{Version}{2011/10/18 v1.15}
%   \item
%     \cs{kv@processor@known} and \cs{kvsetknownkeys} added.
%   \item
%     \cs{kvsetkeys@expandafter} and \cs{kvsetknownkeys@expandafter} added.
%   \item
%     Family handler can report success or failure by \cs{kv@handled@true}
%     or \cs{kv@handled@false}.
%   \item
%     \cs{kv@unset@family@handler} added.
%   \end{Version}
%   \begin{Version}{2012/04/25 v1.16}
%   \item
%     \cs{kv@processor@default} and \cs{kv@processor@known} define
%     macro \cs{kv@fam} for convenience.
%   \item
%     Catcode section: Catcode setting for \texttt{+} added for \hologo{eTeX}.
%   \end{Version}
%   \begin{Version}{2016/05/16 v1.17}
%   \item
%     Documentation updates.
%   \end{Version}
%   \begin{Version}{2019/12/15 v1.18}
%   \item
%     Documentation updates.
%   \item
%     Avoid \xpackage{etexcmds} and \xpackage{infwarwerr} in \LaTeX.
%   \end{Version}
% \end{History}
%
% \PrintIndex
%
% \Finale
\endinput

%        (quote the arguments according to the demands of your shell)
%
% Documentation:
%    (a) If kvsetkeys.drv is present:
%           latex kvsetkeys.drv
%    (b) Without kvsetkeys.drv:
%           latex kvsetkeys.dtx; ...
%    The class ltxdoc loads the configuration file ltxdoc.cfg
%    if available. Here you can specify further options, e.g.
%    use A4 as paper format:
%       \PassOptionsToClass{a4paper}{article}
%
%    Programm calls to get the documentation (example):
%       pdflatex kvsetkeys.dtx
%       makeindex -s gind.ist kvsetkeys.idx
%       pdflatex kvsetkeys.dtx
%       makeindex -s gind.ist kvsetkeys.idx
%       pdflatex kvsetkeys.dtx
%
% Installation:
%    TDS:tex/generic/kvsetkeys/kvsetkeys.sty
%    TDS:doc/latex/kvsetkeys/kvsetkeys.pdf
%    TDS:doc/latex/kvsetkeys/kvsetkeys-example.tex
%    TDS:source/latex/kvsetkeys/kvsetkeys.dtx
%
%<*ignore>
\begingroup
  \catcode123=1 %
  \catcode125=2 %
  \def\x{LaTeX2e}%
\expandafter\endgroup
\ifcase 0\ifx\install y1\fi\expandafter
         \ifx\csname processbatchFile\endcsname\relax\else1\fi
         \ifx\fmtname\x\else 1\fi\relax
\else\csname fi\endcsname
%</ignore>
%<*install>
\input docstrip.tex
\Msg{************************************************************************}
\Msg{* Installation}
\Msg{* Package: kvsetkeys 2019/12/15 v1.18 Key value parser (HO)}
\Msg{************************************************************************}

\keepsilent
\askforoverwritefalse

\let\MetaPrefix\relax
\preamble

This is a generated file.

Project: kvsetkeys
Version: 2019/12/15 v1.18

Copyright (C)
   2006, 2007, 2009-2012 Heiko Oberdiek
   2016-2019 Oberdiek Package Support Group

This work may be distributed and/or modified under the
conditions of the LaTeX Project Public License, either
version 1.3c of this license or (at your option) any later
version. This version of this license is in
   https://www.latex-project.org/lppl/lppl-1-3c.txt
and the latest version of this license is in
   https://www.latex-project.org/lppl.txt
and version 1.3 or later is part of all distributions of
LaTeX version 2005/12/01 or later.

This work has the LPPL maintenance status "maintained".

The Current Maintainers of this work are
Heiko Oberdiek and the Oberdiek Package Support Group
https://github.com/ho-tex/kvsetkeys/issues


The Base Interpreter refers to any `TeX-Format',
because some files are installed in TDS:tex/generic//.

This work consists of the main source file kvsetkeys.dtx
and the derived files
   kvsetkeys.sty, kvsetkeys.pdf, kvsetkeys.ins, kvsetkeys.drv,
   kvsetkeys-example.tex, kvsetkeys-test1.tex,
   kvsetkeys-test2.tex, kvsetkeys-test3.tex,
   kvsetkeys-test4.tex.

\endpreamble
\let\MetaPrefix\DoubleperCent

\generate{%
  \file{kvsetkeys.ins}{\from{kvsetkeys.dtx}{install}}%
  \file{kvsetkeys.drv}{\from{kvsetkeys.dtx}{driver}}%
  \usedir{tex/generic/kvsetkeys}%
  \file{kvsetkeys.sty}{\from{kvsetkeys.dtx}{package}}%
  \usedir{doc/latex/kvsetkeys}%
  \file{kvsetkeys-example.tex}{\from{kvsetkeys.dtx}{example}}%
}

\catcode32=13\relax% active space
\let =\space%
\Msg{************************************************************************}
\Msg{*}
\Msg{* To finish the installation you have to move the following}
\Msg{* file into a directory searched by TeX:}
\Msg{*}
\Msg{*     kvsetkeys.sty}
\Msg{*}
\Msg{* To produce the documentation run the file `kvsetkeys.drv'}
\Msg{* through LaTeX.}
\Msg{*}
\Msg{* Happy TeXing!}
\Msg{*}
\Msg{************************************************************************}

\endbatchfile
%</install>
%<*ignore>
\fi
%</ignore>
%<*driver>
\NeedsTeXFormat{LaTeX2e}
\ProvidesFile{kvsetkeys.drv}%
  [2019/12/15 v1.18 Key value parser (HO)]%
\documentclass{ltxdoc}
\usepackage{holtxdoc}[2011/11/22]
\begin{document}
  \DocInput{kvsetkeys.dtx}%
\end{document}
%</driver>
% \fi
%
%
%
% \GetFileInfo{kvsetkeys.drv}
%
% \title{The \xpackage{kvsetkeys} package}
% \date{2019/12/15 v1.18}
% \author{Heiko Oberdiek\thanks
% {Please report any issues at \url{https://github.com/ho-tex/kvsetkeys/issues}}}
%
% \maketitle
%
% \begin{abstract}
% Package \xpackage{kvsetkeys} provides \cs{kvsetkeys}, a variant
% of package \xpackage{keyval}'s \cs{setkeys}. It allows to specify
% a handler that deals with unknown options. Active commas and equal
% signs may be used (e.g. see \xpackage{babel}'s shorthands) and
% only one level of curly braces is removed from the values.
% \end{abstract}
%
% \tableofcontents
%
% \def\M#1{\texttt{\{}\meta{#1}\texttt{\}}}
%
% \section{Documentation}
%
% First I want to recommend the very good review article
% ``A guide to key-value methods'' by Joseph Wright \cite{tb94wright}.
% It introduces the different key-value packages and compares them.
%
% \subsection{Motivation}
%
% \cs{kvsetkeys} serves as replacement for \xpackage{keyval}'s
% \cs{setkeys}. It basically uses the same syntax. But the
% implementation is more robust and predictable:
% \begin{description}
% \item[Active syntax characters:]
%   Comma `|,|' and the equals sign `|=|' are used inside
%   key value lists as syntax characters. Package \xpackage{keyval}
%   uses the catcode of the characters that is active during
%   package loading, usually this is catcode 12 (other).
%   But it can happen that the catcode setting of the syntax characters
%   changes. Especially active characters are of interest, because
%   some language adaptations uses them. For example, option \xoption{turkish}
%   of package \xpackage{babel} uses the equals sign as active shorthand
%   character. Therefore package \xpackage{kvsetkeys} deals with
%   both catcode settings 12 (other) and 13 (active).
% \item[Brace removal:]
%   Package \xpackage{keyval}'s \cs{setkeys} removes up to two
%   levels of curly braces around the value in some unpredictable way:
%\begin{quote}
%|\setkeys{fam}{key={{value}}} || || | $\rightarrow$ |value|\\
%|\setkeys{fam}{key={{{value}}}} | $\rightarrow$ |{value}|\\
%|\setkeys{fam}{key= {{{value}}}}| $\rightarrow$ |{{value}}|
%\end{quote}
%   This package \xpackage{kvsetkeys} follows a much stronger rule:
%   Exactly one level of braces are removed from an item, if the
%   item is surrounded by curly braces. An item can be a
%   the key value pair, the key or the value.
%\begin{quote}
%|\kvsetkeys{fam}{key={value}} || | $\rightarrow$ |value|\\
%|\kvsetkeys{fam}{key={{value}} | $\rightarrow$ |{value}|\\
%|\kvsetkeys{fam}{key= {{value}}| $\rightarrow$ |{value}|
%\end{quote}
% \item[Arbitrary values:] Unmatched conditionals are supported.
% \end{description}
%
% Before I describe
% \cs{kvsetkeys} in more detail, first I want to explain, how
% this package deals with key value lists. For the package also
% provides low level interfaces that can be used by package authors.
%
% \subsection{Normalizing key value lists}
%
% \begin{declcs}{kv@normalize}\,\M{key value list}
% \end{declcs}
% If the user specifies key value lists, he usually prefers
% nice formatted source code, e.g.:
% \begin{quote}
%\begin{verbatim}
%\hypersetup{
%  pdftitle    = {...},
%  pdfsubject  = {...},
%  pdfauthor   = {...},
%  pdfkeywords = {...},
%  ...
%}
%\end{verbatim}
% \end{quote}
% Thus there can be spaces around keys, around |=| or around the value.
% Also empty entries are possible by too many commas. Therefore these
% spaces and empty entries are silently removed by package \xpackage{keyval}
% and this package. Whereas the contents of the value can be protected
% by curly braces, especially if spaces or commas are used inside,
% a key name must not use spaces or other syntax characters.
%
% \cs{kv@normalize} takes a key value list and performs the cleanup:
% \begin{itemize}
% \item Spaces are removed.
% \item Syntax characters (comma and equal sign) that are active
%   are replaced by the same characters with standard catcode.
%   (Example: \xpackage{babel}'s language option \xoption{turkish}
%   uses the equal sign as active shorthand character.)
% \end{itemize}
% The result is stored in \cs{kv@list}, e.g.:
% \begin{quote}
% |\kv@list| $\rightarrow$ |,pdftitle={...},pdfsubject={...},...,|
% \end{quote}
% Curly braces around values (or keys) remain untouched.
% \begin{description}
% \item[v1.3+:]
% One comma is added in front of the list and each pair ends with
% a comma. Thus an empty list consists of one comma, otherwise
% two commas encloses the list. Empty entries other than the first
% are removed.
% \item[v1.0 -- v1.2:]
% Empty entries are removed later. In fact it adds a comma at the begin
% and end to protect the last value and an easier implementation.
% \end{description}
%
% \subsection{Parsing key value lists}
%
% \begin{declcs}{kv@parse}\,\M{key value list}\,\M{processor}
% \end{declcs}
% It is easier to parse a normalized list, thus \cs{kv@parse}
% normalizes the list and calls \cs{kv@parse@normalized}.
%
% \begin{declcs}{kv@parse@normalized}\,\M{key value list}%
%   \,\M{processor}
% \end{declcs}
% Now the key value list is split into single key value pairs.
% For further processing the key and value are given as arguments
% for the \meta{processor}:
% \begin{quote}
%   \meta{processor}\,\M{key}\,\M{value}
% \end{quote}
% Also key and value are stored in macro names:
% \begin{itemize}
% \item \cs{kv@key} stores the key.
% \item \cs{kv@value} stores the value or if the value was not
%   specified it has the meaning \cs{relax}.
% \end{itemize}
% The behaviour in pseudo code:
% \begin{quote}
%   foreach (\meta{key}, \meta{value}) in (\meta{key value list})\\
%   \hspace*{1.5em}\cs{kv@key} := \meta{key}\\
%   \hspace*{1.5em}\cs{kv@value} := \meta{value}\\
%   \hspace*{1.5em}\meta{processor}\,\M{key}\,\M{value}
% \end{quote}
%
% \begin{declcs}{kv@break}
% \end{declcs}
% Since version 2011/03/03 v1.11 \cs{kv@break} can be
% called inside the \meta{processor}
% of \cs{kv@parse} or \cs{kv@parse@normalized}, then
% the processing is stopped and the following entries discarded.
%
% \subsection{Processing key value pairs}
%
% Key value pairs can be processed in many different ways.
% For example, the processor for \cs{kvsetkeys} works similar
% to \cs{setkeys} of package \xpackage{keyval}. There unknown
% keys raise an error.
%
% Package \xpackage{xkeyval} also knows a star form of \cs{setkeys}
% that stores unknown keys in an internal macro for further processing
% with \cs{setrmkeys} and similar macros. This feature is covered
% by processor \cs{kv@processor@known}.
%
% \subsubsection{Processing similar to \xpackage{keyval}}
%
% \begin{declcs}{kv@processor@default}\,\M{family}\,\M{key}\,\M{value}
% \end{declcs}
% There are many possiblities to process key value pairs.
% \cs{kv@processor@default} is the processor used in \cs{kvsetkeys}.
% It reimplements and extends the behaviour of
% \xpackage{keyval}'s \cs{setkeys}.
% In case of unknown keys \cs{setkeys} raise an error.
% This processer, however, calls a handler instead, if it
% is provided by the family. Both \meta{family} and \meta{key}
% may contain package \xpackage{babel}'s shorthands
% (since 2011/04/07 v1.13).
%
% Since 2011/10/18 v1.15 the family handler can reject the
% successful handling of a key by calling \cs{kv@handled@false}.
%
% Since 2019/12/15 v1.18 \cs{kv@processor@default} also defines
% macro \cs{kv@fam} with meaning \meta{family} for convenience.
%
% \subsubsection{Processing similar to \cs{setkeys*} of package \xpackage{xkeyval}}
%
% \begin{declcs}{kv@processor@known}\,\M{family}\,^^A
%   \M{cmd}\,\M{key}\,\M{value}
% \end{declcs}
% The key value processor \cs{kv@processor@known} behaves similar
% to \cs{kv@processor@default}. If the \meta{key} exists in the
% \meta{family} its code is called, otherwise the family handler
% is tried. If the family handler is not set or cannot handle the
% key, the unknown key value pair is added to the macro \meta{cmd}.
% Since 2011/10/18 v1.15.
%
% The behaviour in pseudo code:
% \begin{quote}
%   if \meta{key} exists\\
%   \hspace*{1.5em}call the keyval code of \meta{key}\\
%   else\\
%   \hspace*{1.5em}if \meta{handler} for \meta{family} exists\\
%   \hspace*{3em}handled = true\\
%   \hspace*{3em}\meta{handler}\,\M{key}\,\M{value}\\
%   \hspace*{3em}if handled\\
%   \hspace*{3em}else\\
%   \hspace*{4.5em}add \texttt{"}\M{key}\texttt{=}\M{value}\texttt{"}
%     to \M{cmd}\\
%   \hspace*{3em}fi\\
%   \hspace*{1.5em}else\\
%   \hspace*{3em}add \texttt{"}\M{key}\texttt{=}\M{value}\texttt{"}
%     to \M{cmd}\\
%   \hspace*{3em}raise unknown key error\\
%   \hspace*{1.5em}fi\\
%   fi
% \end{quote}
%
% Since 2019/12/15 v1.18 \cs{kv@processor@known} also defines
% macro \cs{kv@fam} with meaning \meta{family} for convenience.
%
% \subsection{Default family handler}
%
% \cs{kv@processor@default} calls \meta{handler}, the default
% handler for the family, if the key does not exist in the family.
% The handler is called with two arguments, the key and the value.
% It can be defined with \cs{kv@set@family@hander}:
%
% \begin{declcs}{kv@set@family@handler}\,\M{family}\,\M{handler definition}
% \end{declcs}
% This sets the default family handler for the keyval family
% \meta{family}. Inside \meta{handler definition} |#1| stands for
% the key and |#2| is the value. Also \cs{kv@key} and \cs{kv@value}
% can be used for the key and the value. If the value is not given,
% \cs{kv@value} has the meaning \cs{relax}.
%
% \begin{declcs}{kv@unset@family@handler}\,\M{family}
% \end{declcs}
% It removes the family handler for \meta{family}.
% Since 2011/10/18 v1.15.
%
% \subsection{Put it all together}
%
% \begin{declcs}{kvsetkeys}\,\M{family}\,\M{key value list}
% \end{declcs}
% Macro \cs{kvsetkeys} processes the \meta{key value list} with
% the standard processor \cs{kv@processor@default}:
% \begin{quote}
%   \cs{kv@parse}\,\M{key value list}^^A
%   \texttt{\{}\cs{kv@processor@default}\,\M{family}\texttt{\}}
% \end{quote}
%
% \begin{declcs}{kvsetknownkeys}\,\M{family}\,\M{cmd}\,\M{key value list}
% \end{declcs}
% Macro \cs{kvsetknownkeys} processes the \meta{key value list}
% with processor \cs{kv@processor@known}. All key value pairs
% with keys that are not known in \meta{family} are stored
% in macro \meta{cmd}. A previous contents of macro \meta{cmd}
% will be overwritten. If all keys can be handled, \meta{cmd}
% will be empty, otherwise it contains a key value list of
% unhandled key value pairs.
% Since 2011/10/18 v1.15.
%
% Pseudo code:
% \begin{quote}
%   create macro \meta{cmdaux} with unique name (inside the current group)\\
%   \cs{def}\meta{cmdaux}\{\}\\
%   \cs{kv@parse}\,\M{key value list}^^A
%   \texttt{\{}\cs{kv@processor@known}\,\M{family}\,\M{cmdaux}\texttt{\}}\\
%   \cs{let}\meta{cmd}=\meta{cmdaux}
% \end{quote}
%
% \begin{declcs}{kvsetkeys@expandafter}\,\M{family}\,\M{list cmd}\\
%   \cs{kvsetknownkeys@expandafter}\,\M{family}\,%
%   \M{cmd}\,\M{list cmd}
% \end{declcs}
% Both macros behave like the counterparts without suffix
% |@expandafter|. The difference is that the key value list is
% given as macro that is expanded once.
% Since 2011/10/18 v1.15.
%
% Thus you can replace \cs{setkeys} of package \xpackage{keyval}
% by the key value parser of this package:
% \begin{quote}
%   |\renewcommand*{\setkeys}{\kvsetkeys}|\\
%   or\\
%   |\let\setkeys\kvsetkeys|
% \end{quote}
%
% \subsection{Comma separated lists}
%
% Since version 2007/09/29 v1.3 this package also supports the normalizing
% and parsing of general comma separated lists.
%
% \begin{declcs}{comma@normalize}\,\M{comma list}
% \end{declcs}
% Macro \cs{comma@normalize} normalizes the comma separated list,
% removes spaces around commas. The result is put in macro \cs{comma@list}.
%
% \begin{declcs}{comma@parse}\,\M{comma list}\,\M{processor}
% \end{declcs}
% Macro \cs{comma@parse} first normalizes the comma separated list
% and then parses the list by calling \cs{comma@parse@normalized}.
%
% \begin{declcs}{comma@parse@normalized}\,\M{normalized comma list}^^A
%   \,\M{processor}
% \end{declcs}
% The list is parsed. Empty entries are ignored. \meta{processor}
% is called for each non-empty entry with the entry as argument:
% \begin{quote}
%   \meta{processor}|{|\meta{entry}|}|
% \end{quote}
% Also the entry is stored in the macro \cs{comma@entry}.
%
% \begin{declcs}{comma@break}
% \end{declcs}
% Since version 2011/03/03 v1.11 \cs{comma@break} can be
% called inside the \meta{processor}
% of \cs{comma@parse} or \cs{comma@parse@normalized}, then
% the processing is stopped and the following entries discarded.
%
% \section{Example}
%
% The following example prints a short piece of HTML code using
% the tabbing environment for indenting purpose and a key value
% syntax for specifying the attributes of an HTML tag.
% The example illustrates the use of a default family handler.
%    \begin{macrocode}
%<*example>
\documentclass{article}
\usepackage[T1]{fontenc}
\usepackage{kvsetkeys}
\usepackage{keyval}

\makeatletter
\newcommand*{\tag}[2][]{%
  % #1: attributes
  % #2: tag name
  \begingroup
    \toks@={}%
    \let\@endslash\@empty
    \kvsetkeys{tag}{#1}%
    \texttt{%
      \textless #2\the\toks@\@endslash\textgreater
    }%
  \endgroup
}
\kv@set@family@handler{tag}{%
  % #1: key
  % #2: value
  \toks@\expandafter{%
    \the\toks@
    \space
    #1=\string"#2\string"%
  }%
}
\define@key{tag}{/}[]{%
  \def\@endslash{/}%
}
\makeatother

\begin{document}
\begin{tabbing}
  \mbox{}\qquad\=\qquad\=\kill
  \tag{html}\\
  \>\dots\\
  \>\tag[border=1]{table}\\
  \>\>\tag[width=200, span=3, /]{colgroup}\\
  \>\>\dots\\
  \>\tag{/table}\\
  \>\dots\\
  \tag{/html}\\
\end{tabbing}
\end{document}
%</example>
%    \end{macrocode}
%
% \StopEventually{
% }
%
% \section{Implementation}
%
% \subsection{Identification}
%
%    \begin{macrocode}
%<*package>
%    \end{macrocode}
%    Reload check, especially if the package is not used with \LaTeX.
%    \begin{macrocode}
\begingroup\catcode61\catcode48\catcode32=10\relax%
  \catcode13=5 % ^^M
  \endlinechar=13 %
  \catcode35=6 % #
  \catcode39=12 % '
  \catcode44=12 % ,
  \catcode45=12 % -
  \catcode46=12 % .
  \catcode58=12 % :
  \catcode64=11 % @
  \catcode123=1 % {
  \catcode125=2 % }
  \expandafter\let\expandafter\x\csname ver@kvsetkeys.sty\endcsname
  \ifx\x\relax % plain-TeX, first loading
  \else
    \def\empty{}%
    \ifx\x\empty % LaTeX, first loading,
      % variable is initialized, but \ProvidesPackage not yet seen
    \else
      \expandafter\ifx\csname PackageInfo\endcsname\relax
        \def\x#1#2{%
          \immediate\write-1{Package #1 Info: #2.}%
        }%
      \else
        \def\x#1#2{\PackageInfo{#1}{#2, stopped}}%
      \fi
      \x{kvsetkeys}{The package is already loaded}%
      \aftergroup\endinput
    \fi
  \fi
\endgroup%
%    \end{macrocode}
%    Package identification:
%    \begin{macrocode}
\begingroup\catcode61\catcode48\catcode32=10\relax%
  \catcode13=5 % ^^M
  \endlinechar=13 %
  \catcode35=6 % #
  \catcode39=12 % '
  \catcode40=12 % (
  \catcode41=12 % )
  \catcode44=12 % ,
  \catcode45=12 % -
  \catcode46=12 % .
  \catcode47=12 % /
  \catcode58=12 % :
  \catcode64=11 % @
  \catcode91=12 % [
  \catcode93=12 % ]
  \catcode123=1 % {
  \catcode125=2 % }
  \expandafter\ifx\csname ProvidesPackage\endcsname\relax
    \def\x#1#2#3[#4]{\endgroup
      \immediate\write-1{Package: #3 #4}%
      \xdef#1{#4}%
    }%
  \else
    \def\x#1#2[#3]{\endgroup
      #2[{#3}]%
      \ifx#1\@undefined
        \xdef#1{#3}%
      \fi
      \ifx#1\relax
        \xdef#1{#3}%
      \fi
    }%
  \fi
\expandafter\x\csname ver@kvsetkeys.sty\endcsname
\ProvidesPackage{kvsetkeys}%
  [2019/12/15 v1.18 Key value parser (HO)]%
%    \end{macrocode}
%
%    \begin{macrocode}
\begingroup\catcode61\catcode48\catcode32=10\relax%
  \catcode13=5 % ^^M
  \endlinechar=13 %
  \catcode123=1 % {
  \catcode125=2 % }
  \catcode64=11 % @
  \def\x{\endgroup
    \expandafter\edef\csname KVS@AtEnd\endcsname{%
      \endlinechar=\the\endlinechar\relax
      \catcode13=\the\catcode13\relax
      \catcode32=\the\catcode32\relax
      \catcode35=\the\catcode35\relax
      \catcode61=\the\catcode61\relax
      \catcode64=\the\catcode64\relax
      \catcode123=\the\catcode123\relax
      \catcode125=\the\catcode125\relax
    }%
  }%
\x\catcode61\catcode48\catcode32=10\relax%
\catcode13=5 % ^^M
\endlinechar=13 %
\catcode35=6 % #
\catcode64=11 % @
\catcode123=1 % {
\catcode125=2 % }
\def\TMP@EnsureCode#1#2{%
  \edef\KVS@AtEnd{%
    \KVS@AtEnd
    \catcode#1=\the\catcode#1\relax
  }%
  \catcode#1=#2\relax
}
\TMP@EnsureCode{36}{3}% $
\TMP@EnsureCode{38}{4}% &
\TMP@EnsureCode{39}{12}% '
\TMP@EnsureCode{43}{12}% +
\TMP@EnsureCode{44}{12}% ,
\TMP@EnsureCode{45}{12}% -
\TMP@EnsureCode{46}{12}% .
\TMP@EnsureCode{47}{12}% /
\TMP@EnsureCode{91}{12}% [
\TMP@EnsureCode{93}{12}% ]
\TMP@EnsureCode{94}{7}% ^ (superscript)
\TMP@EnsureCode{96}{12}% `
\TMP@EnsureCode{126}{13}% ~ (active)
\edef\KVS@AtEnd{\KVS@AtEnd\noexpand\endinput}
%    \end{macrocode}
%
% \subsection{Package loading}
%
%    \begin{macrocode}
\begingroup\expandafter\expandafter\expandafter\endgroup
\expandafter\ifx\csname RequirePackage\endcsname\relax
  \def\TMP@RequirePackage#1[#2]{%
    \begingroup\expandafter\expandafter\expandafter\endgroup
    \expandafter\ifx\csname ver@#1.sty\endcsname\relax
      \input #1.sty\relax
    \fi
  }%
  \TMP@RequirePackage{infwarerr}[2007/09/09]%
  \let\PackageError\@PackageError
\else
\fi
%    \end{macrocode}
%
%    \begin{macrocode}
\expandafter\ifx\csname toks@\endcsname\relax
  \toksdef\toks@=0 %
\fi
%    \end{macrocode}
%
% \subsection{Check for \eTeX}
%
%    \cs{unexpanded}, \cs{ifcsname}, and \cs{unless} are used if found.
%    \begin{macrocode}
\ifx\numexpr\@undefined
  \catcode`\$=9 % ignore
  \catcode`\&=14 % comment
\else % e-TeX
  \catcode`\$=14 % comment
  \catcode`\&=9  % ignore
\fi
%    \end{macrocode}
%
% \subsection{Generic help macros}
%
%    \begin{macro}{\KVS@Empty}
%    \begin{macrocode}
\def\KVS@Empty{}
%    \end{macrocode}
%    \end{macro}
%    \begin{macro}{\KVS@FirstOfTwo}
%    \begin{macrocode}
\long\def\KVS@FirstOfTwo#1#2{#1}
%    \end{macrocode}
%    \end{macro}
%    \begin{macro}{\KVS@SecondOfTwo}
%    \begin{macrocode}
\long\def\KVS@SecondOfTwo#1#2{#2}
%    \end{macrocode}
%    \end{macro}
%
%    \begin{macro}{\KVS@IfEmpty}
%    \begin{macrocode}
\long\def\KVS@IfEmpty#1{%
& \edef\KVS@Temp{\unexpanded{#1}}%
$ \begingroup
$   \toks@{#1}%
$   \edef\KVS@Temp{\the\toks@}%
$ \expandafter\endgroup
  \ifx\KVS@Temp\KVS@Empty
    \expandafter\KVS@FirstOfTwo
  \else
    \expandafter\KVS@SecondOfTwo
  \fi
}
%    \end{macrocode}
%    \end{macro}
%
% \subsection{Normalizing}
%
%    \begin{macro}{\kv@normalize}
%    \begin{macrocode}
\long\def\kv@normalize#1{%
  \begingroup
    \toks@{,#1,}%
    \KVS@Comma
    \KVS@SpaceComma
    \KVS@CommaSpace
    \KVS@CommaComma
    \KVS@Equals
    \KVS@SpaceEquals
    \KVS@EqualsSpace
    \xdef\KVS@Global{\the\toks@}%
  \endgroup
  \let\kv@list\KVS@Global
}
%    \end{macrocode}
%    \end{macro}
%    \begin{macro}{\comma@normalize}
%    \begin{macrocode}
\def\comma@normalize#1{%
  \begingroup
    \toks@{,#1,}%
    \KVS@Comma
    \KVS@SpaceComma
    \KVS@CommaSpace
    \KVS@CommaComma
    \xdef\KVS@Global{\the\toks@}%
  \endgroup
  \let\comma@list\KVS@Global
}
%    \end{macrocode}
%    \end{macro}
%    \begin{macro}{\KVS@Comma}
%    Converts active commas into comma with catcode other.
%    Also adds a comma at the end to protect the last value
%    for next cleanup steps.
%    \begin{macrocode}
\begingroup
  \lccode`\,=`\,%
  \lccode`\~=`\,%
\lowercase{\endgroup
  \def\KVS@Comma{%
    \toks@\expandafter{\expandafter}\expandafter
    \KVS@@Comma\the\toks@~\KVS@Nil
  }%
  \long\def\KVS@@Comma#1~#2\KVS@Nil{%
    \toks@\expandafter{\the\toks@#1}%
    \KVS@IfEmpty{#2}{%
    }{%
      \KVS@@Comma,#2\KVS@Nil
    }%
  }%
}
%    \end{macrocode}
%    \end{macro}
%    \begin{macro}{\KVS@SpaceComma}
%    Removes spaces before the comma, may add commas at the end.
%    \begin{macrocode}
\def\KVS@SpaceComma#1{%
  \def\KVS@SpaceComma{%
    \expandafter\KVS@@SpaceComma\the\toks@#1,\KVS@Nil
  }%
}
\KVS@SpaceComma{ }
%    \end{macrocode}
%    \end{macro}
%    \begin{macro}{\KVS@@SpaceComma}
%    \begin{macrocode}
\long\def\KVS@@SpaceComma#1 ,#2\KVS@Nil{%
  \KVS@IfEmpty{#2}{%
    \toks@{#1}%
  }{%
    \KVS@@SpaceComma#1,#2\KVS@Nil
  }%
}
%    \end{macrocode}
%    \end{macro}
%    \begin{macro}{\KVS@CommaSpace}
%    Removes spaces after the comma, may add commas at the end.
%    \begin{macrocode}
\def\KVS@CommaSpace{%
  \expandafter\KVS@@CommaSpace\the\toks@, \KVS@Nil
}
%    \end{macrocode}
%    \end{macro}
%    \begin{macro}{\KVS@@CommaSpace}
%    \begin{macrocode}
\long\def\KVS@@CommaSpace#1, #2\KVS@Nil{%
  \KVS@IfEmpty{#2}{%
    \toks@{#1}%
  }{%
    \KVS@@CommaSpace#1,#2\KVS@Nil
  }%
}
%    \end{macrocode}
%    \end{macro}
%    \begin{macro}{\KVS@CommaComma}
%    Replaces multiple commas by one comma.
%    \begin{macrocode}
\def\KVS@CommaComma{%
  \expandafter\KVS@@CommaComma\the\toks@,\KVS@Nil
}
%    \end{macrocode}
%    \end{macro}
%    \begin{macro}{\KVS@@CommaComma}
%    \begin{macrocode}
\long\def\KVS@@CommaComma#1,,#2\KVS@Nil{%
  \KVS@IfEmpty{#2}{%
    \toks@{#1,}% (!)
  }{%
    \KVS@@CommaComma#1,#2\KVS@Nil
  }%
}
%    \end{macrocode}
%    \end{macro}
%
%    \begin{macro}{\KVS@Equals}
%    Converts active equals signs into catcode other characters.
%    \begin{macrocode}
\begingroup
  \lccode`\==`\=%
  \lccode`\~=`\=%
\lowercase{\endgroup
  \def\KVS@Equals{%
    \toks@\expandafter{\expandafter}\expandafter
    \KVS@@Equals\the\toks@~\KVS@Nil
  }%
  \long\def\KVS@@Equals#1~#2\KVS@Nil{%
    \edef\KVS@Temp{\the\toks@}%
    \ifx\KVS@Temp\KVS@Empty
      \expandafter\KVS@FirstOfTwo
    \else
      \expandafter\KVS@SecondOfTwo
    \fi
    {%
      \toks@{#1}%
    }{%
      \toks@\expandafter{\the\toks@=#1}%
    }%
    \KVS@IfEmpty{#2}{%
    }{%
      \KVS@@Equals#2\KVS@Nil
    }%
  }%
}
%    \end{macrocode}
%    \end{macro}
%    \begin{macro}{\KVS@SpaceEquals}
%    Removes spaces before the equals sign.
%    \begin{macrocode}
\def\KVS@SpaceEquals#1{%
  \def\KVS@SpaceEquals{%
    \expandafter\KVS@@SpaceEquals\the\toks@#1=\KVS@Nil
  }%
}
\KVS@SpaceEquals{ }
%    \end{macrocode}
%    \end{macro}
%    \begin{macro}{\KVS@@SpaceEquals}
%    \begin{macrocode}
\long\def\KVS@@SpaceEquals#1 =#2\KVS@Nil{%
  \KVS@IfEmpty{#2}{%
    \toks@{#1}%
  }{%
    \KVS@@SpaceEquals#1=#2\KVS@Nil
  }%
}
%    \end{macrocode}
%    \end{macro}
%    \begin{macro}{\KVS@EqualsSpace}
%    Removes spaces after the equals sign.
%    \begin{macrocode}
\def\KVS@EqualsSpace{%
  \expandafter\KVS@@EqualsSpace\the\toks@= \KVS@Nil
}
%    \end{macrocode}
%    \end{macro}
%    \begin{macro}{\KVS@@EqualsSpace}
%    \begin{macrocode}
\long\def\KVS@@EqualsSpace#1= #2\KVS@Nil{%
  \KVS@IfEmpty{#2}{%
    \toks@{#1}%
  }{%
    \KVS@@EqualsSpace#1=#2\KVS@Nil
  }%
}
%    \end{macrocode}
%    \end{macro}
%
% \subsection{Parsing key value lists}
%
%    \begin{macro}{\kv@parse}
%    Normalizes and parses the key value list. Also sets \cs{kv@list}.
%    \begin{macrocode}
\long\def\kv@parse#1{%
  \kv@normalize{#1}%
  \expandafter\kv@parse@normalized\expandafter{\kv@list}%
}
%    \end{macrocode}
%    \end{macro}
%    \begin{macro}{\kv@parse@normalized}
%    |#1|: key value list\\
%    |#2|: processor
%    \begin{macrocode}
\long\def\kv@parse@normalized#1#2{%
  \KVS@Parse#1,\KVS@Nil{#2}%
}
%    \end{macrocode}
%    \end{macro}
%    \begin{macro}{\KVS@Parse}
%    |#1,#2|: key value list\\
%    |#3|: processor
%    \begin{macrocode}
\long\def\KVS@Parse#1,#2\KVS@Nil#3{%
  \KVS@IfEmpty{#1}{%
  }{%
    \KVS@Process#1=\KVS@Nil{#3}%
  }%
  \KVS@MaybeBreak
  \KVS@IfEmpty{#2}{%
  }{%
    \KVS@Parse#2\KVS@Nil{#3}%
  }%
}
%    \end{macrocode}
%    \end{macro}
%    \begin{macro}{\KVS@Process}
%    |#1|: key\\
%    |#2|: value, |=|\\
%    |#3|: processor
%    \begin{macrocode}
\long\def\KVS@Process#1=#2\KVS@Nil#3{%
  \let\KVS@MaybeBreak\relax
  \def\kv@key{#1}%
  \KVS@IfEmpty{#2}{%
    \let\kv@value\relax
    #3{#1}{}%
  }{%
    \KVS@@Process{#1}#2\KVS@Nil{#3}%
  }%
}
%    \end{macrocode}
%    \end{macro}
%    \begin{macro}{\KVS@@Process}
%    |#1|: key\\
%    |#2|: value\\
%    |#3|: processor
%    \begin{macrocode}
\long\def\KVS@@Process#1#2=\KVS@Nil#3{%
& \edef\kv@value{\unexpanded{#2}}%
$ \begingroup
$   \toks@{#2}%
$   \xdef\KVS@Global{\the\toks@}%
$ \endgroup
$ \let\kv@value\KVS@Global
  #3{#1}{#2}%
}
%    \end{macrocode}
%    \end{macro}
%    \begin{macro}{\KVS@MaybeBreak}
%    \begin{macrocode}
\let\KVS@MaybeBreak\relax
%    \end{macrocode}
%    \end{macro}
%    \begin{macro}{\KVS@break}
%    \begin{macrocode}
\def\KVS@break#1#2#3#4{%
  \let\KVS@MaybeBreak\relax
}
%    \end{macrocode}
%    \end{macro}
%    \begin{macro}{\kv@break}
%    \begin{macrocode}
\def\kv@break{%
  \let\KVS@MaybeBreak\KVS@break
}
%    \end{macrocode}
%    \end{macro}
%
% \subsection{Parsing comma lists}
%
%    \begin{macro}{\comma@parse}
%    Normalizes and parses the key value list. Also sets \cs{comma@list}.
%    \begin{macrocode}
\def\comma@parse#1{%
  \comma@normalize{#1}%
  \expandafter\comma@parse@normalized\expandafter{\comma@list}%
}
%    \end{macrocode}
%    \end{macro}
%
%    \begin{macro}{\comma@parse@normalized}
%    |#1|: comma list\\
%    |#2|: processor
%    \begin{macrocode}
\def\comma@parse@normalized#1#2{%
  \KVS@CommaParse#1,\KVS@Nil{#2}%
}
%    \end{macrocode}
%    \end{macro}
%    \begin{macro}{\KVS@CommaParse}
%    |#1,#2|: comma list\\
%    |#3|: processor
%    \begin{macrocode}
\def\KVS@CommaParse#1,#2\KVS@Nil#3{%
  \KVS@IfEmpty{#1}{%
  }{%
    \def\comma@entry{#1}%
    #3{#1}%
  }%
  \KVS@MaybeBreak
  \KVS@IfEmpty{#2}{%
  }{%
    \KVS@CommaParse#2\KVS@Nil{#3}%
  }%
}
%    \end{macrocode}
%    \end{macro}
%    \begin{macro}{\comma@break}
%    \begin{macrocode}
\def\comma@break{%
  \let\KVS@MaybeBreak\KVS@break
}
%    \end{macrocode}
%    \end{macro}
%
% \subsection{Processing key value pairs}
%
%    \begin{macro}{\kv@handled@false}
%    The handler can call \cs{kv@handled@false} or
%    \cs{kv@handled@true} so report failure or success.
%    The default is success (compatibility for versions
%    before 2011/10/18 v1.15).
%    \begin{macrocode}
\def\kv@handled@false{%
  \let\ifkv@handled@\iffalse
}
%    \end{macrocode}
%    \end{macro}
%    \begin{macro}{\kv@handled@true}
%    \begin{macrocode}
\def\kv@handled@true{%
  \let\ifkv@handled@\iftrue
}
%    \end{macrocode}
%    \end{macro}
%    \begin{macro}{\ifkv@handled@}
%    \begin{macrocode}
\kv@handled@true
%    \end{macrocode}
%    \end{macro}
%
%    \begin{macro}{\kv@processor@default}
%    \begin{macrocode}
\def\kv@processor@default#1#2{%
  \begingroup
    \csname @safe@activestrue\endcsname
    \let\ifincsname\iftrue
    \edef\KVS@temp{\endgroup
      \noexpand\KVS@ProcessorDefault{#1}{#2}%
    }%
  \KVS@temp
}
%    \end{macrocode}
%    \end{macro}
%    \begin{macro}{\KVS@ProcessorDefault}
%    \begin{macrocode}
\long\def\KVS@ProcessorDefault#1#2#3{%
  \def\kv@fam{#1}%
& \unless\ifcsname KV@#1@#2\endcsname
$ \begingroup\expandafter\expandafter\expandafter\endgroup
$ \expandafter\ifx\csname KV@#1@#2\endcsname\relax
&   \unless\ifcsname KVS@#1@handler\endcsname
$   \begingroup\expandafter\expandafter\expandafter\endgroup
$   \expandafter\ifx\csname KVS@#1@handler\endcsname\relax
      \kv@error@unknownkey{#1}{#2}%
    \else
      \kv@handled@true
      \csname KVS@#1@handler\endcsname{#2}{#3}%
      \relax
      \ifkv@handled@
      \else
        \kv@error@unknownkey{#1}{#2}%
      \fi
    \fi
  \else
    \ifx\kv@value\relax
&     \unless\ifcsname KV@#1@#2@default\endcsname
$     \begingroup\expandafter\expandafter\expandafter\endgroup
$     \expandafter\ifx\csname KV@#1@#2@default\endcsname\relax
        \kv@error@novalue{#1}{#2}%
      \else
        \csname KV@#1@#2@default\endcsname
        \relax
      \fi
    \else
      \csname KV@#1@#2\endcsname{#3}%
    \fi
  \fi
}
%    \end{macrocode}
%    \end{macro}
%
%    \begin{macro}{\kv@processor@known}
%    \begin{macrocode}
\def\kv@processor@known#1#2#3{%
  \begingroup
    \csname @safe@activestrue\endcsname
    \let\ifincsname\iftrue
    \edef\KVS@temp{\endgroup
      \noexpand\KVS@ProcessorKnown{#1}\noexpand#2{#3}%
    }%
  \KVS@temp
}
%    \end{macrocode}
%    \end{macro}
%    \begin{macro}{\KVS@ProcessorKnown}
%    \begin{macrocode}
\long\def\KVS@ProcessorKnown#1#2#3#4{%
  \def\kv@fam{#1}%
& \unless\ifcsname KV@#1@#3\endcsname
$ \begingroup\expandafter\expandafter\expandafter\endgroup
$ \expandafter\ifx\csname KV@#1@#3\endcsname\relax
&   \unless\ifcsname KVS@#1@handler\endcsname
$   \begingroup\expandafter\expandafter\expandafter\endgroup
$   \expandafter\ifx\csname KVS@#1@handler\endcsname\relax
      \KVS@AddUnhandled#2{#3}{#4}%
    \else
      \kv@handled@true
      \csname KVS@#1@handler\endcsname{#3}{#4}%
      \relax
      \ifkv@handled@
      \else
        \KVS@AddUnhandled#2{#3}{#4}%
      \fi
    \fi
  \else
    \ifx\kv@value\relax
&     \unless\ifcsname KV@#1@#2@default\endcsname
$     \begingroup\expandafter\expandafter\expandafter\endgroup
$     \expandafter\ifx\csname KV@#1@#3@default\endcsname\relax
        \kv@error@novalue{#1}{#3}%
      \else
        \csname KV@#1@#3@default\endcsname
        \relax
      \fi
    \else
      \csname KV@#1@#3\endcsname{#4}%
    \fi
  \fi
}
%    \end{macrocode}
%    \end{macro}
%    \begin{macro}{\KVS@AddUnhandled}
%    \begin{macrocode}
\long\def\KVS@AddUnhandled#1#2#3{%
& \edef#1{%
&   \ifx#1\KVS@empty
&   \else
&     \unexpanded{#1},%
&   \fi
&   \unexpanded{{#2}={#3}}%
& }%
$ \begingroup
$   \ifx#1\KVS@empty
$     \toks@{{#2}={#3}}%
$   \else
$     \toks@\expandafter{#1,{#2}={#3}}%
$   \fi
$   \xdef\KVS@Global{\the\toks@}%
$ \endgroup
$ \let#1\KVS@Global
}
%    \end{macrocode}
%    \end{macro}
%
%    \begin{macro}{\kv@set@family@handler}
%    \begin{macrocode}
\long\def\kv@set@family@handler#1#2{%
  \begingroup
    \csname @safe@activestrue\endcsname
    \let\ifincsname\iftrue
  \expandafter\endgroup
  \expandafter\def\csname KVS@#1@handler\endcsname##1##2{#2}%
}
%    \end{macrocode}
%    \end{macro}
%
%    \begin{macro}{\kv@unset@family@handler}
%    \begin{macrocode}
\long\def\kv@unset@family@handler#1#2{%
  \begingroup
    \csname @safe@activestrue\endcsname
    \let\ifincsname\iftrue
  \expandafter\endgroup
  \expandafter\let\csname KVS@#1@handler\endcsname\@UnDeFiNeD
}
%    \end{macrocode}
%    \end{macro}
%
% \subsection{Error handling}
%
%    \begin{macro}{\kv@error@novalue}
%    \begin{macrocode}
\def\kv@error@novalue{%
  \kv@error@generic{No value specified for}%
}
%    \end{macrocode}
%    \end{macro}
%    \begin{macro}{\kv@error@unknownkey}
%    \begin{macrocode}
\def\kv@error@unknownkey{%
  \kv@error@generic{Undefined}%
}
%    \end{macrocode}
%    \end{macro}
%    \begin{macro}{\kv@error@generic}
%    \begin{macrocode}
\def\kv@error@generic#1#2#3{%
  \PackageError{kvsetkeys}{%
    #1 key `#3'%
  }{%
    The keyval family of the key `#3' is `#2'.\MessageBreak
    The setting of the key is ignored because of the error.\MessageBreak
    \MessageBreak
    \@ehc
  }%
}
%    \end{macrocode}
%    \end{macro}
%
% \subsection{Do it all}
%
%    \begin{macro}{\kvsetkeys}
%    \begin{macrocode}
\long\def\kvsetkeys#1#2{%
  \kv@parse{#2}{\kv@processor@default{#1}}%
}
%    \end{macrocode}
%    \end{macro}
%
%    \begin{macro}{\kvsetkeys@expandafter}
%    \begin{macrocode}
\def\kvsetkeys@expandafter#1#2{%
  \expandafter\kv@parse\expandafter{#2}{%
    \kv@processor@default{#1}%
  }%
}
%    \end{macrocode}
%    \end{macro}
%
%    \begin{macro}{\KVS@cmd}
%    \begin{macrocode}
\def\KVS@cmd{0}%
%    \end{macrocode}
%    \end{macro}
%    \begin{macro}{\KVS@cmd@inc}
%    \begin{macrocode}
\def\KVS@cmd@inc{%
& \edef\KVS@cmd{\the\numexpr\KVS@cmd+1}%
$ \begingroup
$   \count255=\KVS@cmd\relax
$   \advance\count255 by 1\relax
$ \edef\x{\endgroup
$   \noexpand\def\noexpand\KVS@cmd{\number\count255}%
$ }%
$ \x
}
%    \end{macrocode}
%    \end{macro}
%    \begin{macro}{\KVS@cmd@dec}
%    \begin{macrocode}
\def\KVS@cmd@dec{%
& \edef\KVS@cmd{\the\numexpr\KVS@cmd-1}%
$ \begingroup
$   \count255=\KVS@cmd\relax
$   \advance\count255 by -1\relax
$ \edef\x{\endgroup
$   \noexpand\def\noexpand\KVS@cmd{\number\count255}%
$ }%
$ \x
}
%    \end{macrocode}
%    \end{macro}
%    \begin{macro}{\KVS@empty}
%    \begin{macrocode}
\def\KVS@empty{}
%    \end{macrocode}
%    \end{macro}
%
%    \begin{macro}{\kvsetknownkeys}
%    \begin{macrocode}
\def\kvsetknownkeys{%
  \expandafter
  \KVS@setknownkeys\csname KVS@cmd\KVS@cmd\endcsname{}%
}
%    \end{macrocode}
%    \end{macro}
%    \begin{macro}{\KVS@setknownkeys}
%    \begin{macrocode}
\long\def\KVS@setknownkeys#1#2#3#4#5{%
  \let#1\KVS@empty
  \KVS@cmd@inc
  #2\kv@parse#2{#5}{\kv@processor@known{#3}#1}%
  \KVS@cmd@dec
  \let#4=#1%
}
%    \end{macrocode}
%    \end{macro}
%
%    \begin{macro}{\kvsetknownkeys@expandafter}
%    \begin{macrocode}
\def\kvsetknownkeys@expandafter{%
  \expandafter
  \KVS@setknownkeys
      \csname KVS@cmd\KVS@cmd\endcsname\expandafter
}
%    \end{macrocode}
%    \end{macro}
%
%    \begin{macrocode}
\KVS@AtEnd%
%</package>
%    \end{macrocode}
%% \section{Installation}
%
% \subsection{Download}
%
% \paragraph{Package.} This package is available on
% CTAN\footnote{\CTANpkg{kvsetkeys}}:
% \begin{description}
% \item[\CTAN{macros/latex/contrib/kvsetkeys/kvsetkeys.dtx}] The source file.
% \item[\CTAN{macros/latex/contrib/kvsetkeys/kvsetkeys.pdf}] Documentation.
% \end{description}
%
%
% \paragraph{Bundle.} All the packages of the bundle `kvsetkeys'
% are also available in a TDS compliant ZIP archive. There
% the packages are already unpacked and the documentation files
% are generated. The files and directories obey the TDS standard.
% \begin{description}
% \item[\CTANinstall{install/macros/latex/contrib/kvsetkeys.tds.zip}]
% \end{description}
% \emph{TDS} refers to the standard ``A Directory Structure
% for \TeX\ Files'' (\CTANpkg{tds}). Directories
% with \xfile{texmf} in their name are usually organized this way.
%
% \subsection{Bundle installation}
%
% \paragraph{Unpacking.} Unpack the \xfile{kvsetkeys.tds.zip} in the
% TDS tree (also known as \xfile{texmf} tree) of your choice.
% Example (linux):
% \begin{quote}
%   |unzip kvsetkeys.tds.zip -d ~/texmf|
% \end{quote}
%
% \subsection{Package installation}
%
% \paragraph{Unpacking.} The \xfile{.dtx} file is a self-extracting
% \docstrip\ archive. The files are extracted by running the
% \xfile{.dtx} through \plainTeX:
% \begin{quote}
%   \verb|tex kvsetkeys.dtx|
% \end{quote}
%
% \paragraph{TDS.} Now the different files must be moved into
% the different directories in your installation TDS tree
% (also known as \xfile{texmf} tree):
% \begin{quote}
% \def\t{^^A
% \begin{tabular}{@{}>{\ttfamily}l@{ $\rightarrow$ }>{\ttfamily}l@{}}
%   kvsetkeys.sty & tex/generic/kvsetkeys/kvsetkeys.sty\\
%   kvsetkeys.pdf & doc/latex/kvsetkeys/kvsetkeys.pdf\\
%   kvsetkeys-example.tex & doc/latex/kvsetkeys/kvsetkeys-example.tex\\
%   kvsetkeys.dtx & source/latex/kvsetkeys/kvsetkeys.dtx\\
% \end{tabular}^^A
% }^^A
% \sbox0{\t}^^A
% \ifdim\wd0>\linewidth
%   \begingroup
%     \advance\linewidth by\leftmargin
%     \advance\linewidth by\rightmargin
%   \edef\x{\endgroup
%     \def\noexpand\lw{\the\linewidth}^^A
%   }\x
%   \def\lwbox{^^A
%     \leavevmode
%     \hbox to \linewidth{^^A
%       \kern-\leftmargin\relax
%       \hss
%       \usebox0
%       \hss
%       \kern-\rightmargin\relax
%     }^^A
%   }^^A
%   \ifdim\wd0>\lw
%     \sbox0{\small\t}^^A
%     \ifdim\wd0>\linewidth
%       \ifdim\wd0>\lw
%         \sbox0{\footnotesize\t}^^A
%         \ifdim\wd0>\linewidth
%           \ifdim\wd0>\lw
%             \sbox0{\scriptsize\t}^^A
%             \ifdim\wd0>\linewidth
%               \ifdim\wd0>\lw
%                 \sbox0{\tiny\t}^^A
%                 \ifdim\wd0>\linewidth
%                   \lwbox
%                 \else
%                   \usebox0
%                 \fi
%               \else
%                 \lwbox
%               \fi
%             \else
%               \usebox0
%             \fi
%           \else
%             \lwbox
%           \fi
%         \else
%           \usebox0
%         \fi
%       \else
%         \lwbox
%       \fi
%     \else
%       \usebox0
%     \fi
%   \else
%     \lwbox
%   \fi
% \else
%   \usebox0
% \fi
% \end{quote}
% If you have a \xfile{docstrip.cfg} that configures and enables \docstrip's
% TDS installing feature, then some files can already be in the right
% place, see the documentation of \docstrip.
%
% \subsection{Refresh file name databases}
%
% If your \TeX~distribution
% (\TeX\,Live, \mikTeX, \dots) relies on file name databases, you must refresh
% these. For example, \TeX\,Live\ users run \verb|texhash| or
% \verb|mktexlsr|.
%
% \subsection{Some details for the interested}
%
% \paragraph{Unpacking with \LaTeX.}
% The \xfile{.dtx} chooses its action depending on the format:
% \begin{description}
% \item[\plainTeX:] Run \docstrip\ and extract the files.
% \item[\LaTeX:] Generate the documentation.
% \end{description}
% If you insist on using \LaTeX\ for \docstrip\ (really,
% \docstrip\ does not need \LaTeX), then inform the autodetect routine
% about your intention:
% \begin{quote}
%   \verb|latex \let\install=y% \iffalse meta-comment
%
% File: kvsetkeys.dtx
% Version: 2019/12/15 v1.18
% Info: Key value parser
%
% Copyright (C)
%    2006, 2007, 2009-2012 Heiko Oberdiek
%    2016-2019 Oberdiek Package Support Group
%    https://github.com/ho-tex/kvsetkeys/issues
%
% This work may be distributed and/or modified under the
% conditions of the LaTeX Project Public License, either
% version 1.3c of this license or (at your option) any later
% version. This version of this license is in
%    https://www.latex-project.org/lppl/lppl-1-3c.txt
% and the latest version of this license is in
%    https://www.latex-project.org/lppl.txt
% and version 1.3 or later is part of all distributions of
% LaTeX version 2005/12/01 or later.
%
% This work has the LPPL maintenance status "maintained".
%
% The Current Maintainers of this work are
% Heiko Oberdiek and the Oberdiek Package Support Group
% https://github.com/ho-tex/kvsetkeys/issues
%
% The Base Interpreter refers to any `TeX-Format',
% because some files are installed in TDS:tex/generic//.
%
% This work consists of the main source file kvsetkeys.dtx
% and the derived files
%    kvsetkeys.sty, kvsetkeys.pdf, kvsetkeys.ins, kvsetkeys.drv,
%    kvsetkeys-example.tex, kvsetkeys-test1.tex,
%    kvsetkeys-test2.tex, kvsetkeys-test3.tex,
%    kvsetkeys-test4.tex.
%
% Distribution:
%    CTAN:macros/latex/contrib/kvsetkeys/kvsetkeys.dtx
%    CTAN:macros/latex/contrib/kvsetkeys/kvsetkeys.pdf
%
% Unpacking:
%    (a) If kvsetkeys.ins is present:
%           tex kvsetkeys.ins
%    (b) Without kvsetkeys.ins:
%           tex kvsetkeys.dtx
%    (c) If you insist on using LaTeX
%           latex \let\install=y% \iffalse meta-comment
%
% File: kvsetkeys.dtx
% Version: 2019/12/15 v1.18
% Info: Key value parser
%
% Copyright (C)
%    2006, 2007, 2009-2012 Heiko Oberdiek
%    2016-2019 Oberdiek Package Support Group
%    https://github.com/ho-tex/kvsetkeys/issues
%
% This work may be distributed and/or modified under the
% conditions of the LaTeX Project Public License, either
% version 1.3c of this license or (at your option) any later
% version. This version of this license is in
%    https://www.latex-project.org/lppl/lppl-1-3c.txt
% and the latest version of this license is in
%    https://www.latex-project.org/lppl.txt
% and version 1.3 or later is part of all distributions of
% LaTeX version 2005/12/01 or later.
%
% This work has the LPPL maintenance status "maintained".
%
% The Current Maintainers of this work are
% Heiko Oberdiek and the Oberdiek Package Support Group
% https://github.com/ho-tex/kvsetkeys/issues
%
% The Base Interpreter refers to any `TeX-Format',
% because some files are installed in TDS:tex/generic//.
%
% This work consists of the main source file kvsetkeys.dtx
% and the derived files
%    kvsetkeys.sty, kvsetkeys.pdf, kvsetkeys.ins, kvsetkeys.drv,
%    kvsetkeys-example.tex, kvsetkeys-test1.tex,
%    kvsetkeys-test2.tex, kvsetkeys-test3.tex,
%    kvsetkeys-test4.tex.
%
% Distribution:
%    CTAN:macros/latex/contrib/kvsetkeys/kvsetkeys.dtx
%    CTAN:macros/latex/contrib/kvsetkeys/kvsetkeys.pdf
%
% Unpacking:
%    (a) If kvsetkeys.ins is present:
%           tex kvsetkeys.ins
%    (b) Without kvsetkeys.ins:
%           tex kvsetkeys.dtx
%    (c) If you insist on using LaTeX
%           latex \let\install=y\input{kvsetkeys.dtx}
%        (quote the arguments according to the demands of your shell)
%
% Documentation:
%    (a) If kvsetkeys.drv is present:
%           latex kvsetkeys.drv
%    (b) Without kvsetkeys.drv:
%           latex kvsetkeys.dtx; ...
%    The class ltxdoc loads the configuration file ltxdoc.cfg
%    if available. Here you can specify further options, e.g.
%    use A4 as paper format:
%       \PassOptionsToClass{a4paper}{article}
%
%    Programm calls to get the documentation (example):
%       pdflatex kvsetkeys.dtx
%       makeindex -s gind.ist kvsetkeys.idx
%       pdflatex kvsetkeys.dtx
%       makeindex -s gind.ist kvsetkeys.idx
%       pdflatex kvsetkeys.dtx
%
% Installation:
%    TDS:tex/generic/kvsetkeys/kvsetkeys.sty
%    TDS:doc/latex/kvsetkeys/kvsetkeys.pdf
%    TDS:doc/latex/kvsetkeys/kvsetkeys-example.tex
%    TDS:source/latex/kvsetkeys/kvsetkeys.dtx
%
%<*ignore>
\begingroup
  \catcode123=1 %
  \catcode125=2 %
  \def\x{LaTeX2e}%
\expandafter\endgroup
\ifcase 0\ifx\install y1\fi\expandafter
         \ifx\csname processbatchFile\endcsname\relax\else1\fi
         \ifx\fmtname\x\else 1\fi\relax
\else\csname fi\endcsname
%</ignore>
%<*install>
\input docstrip.tex
\Msg{************************************************************************}
\Msg{* Installation}
\Msg{* Package: kvsetkeys 2019/12/15 v1.18 Key value parser (HO)}
\Msg{************************************************************************}

\keepsilent
\askforoverwritefalse

\let\MetaPrefix\relax
\preamble

This is a generated file.

Project: kvsetkeys
Version: 2019/12/15 v1.18

Copyright (C)
   2006, 2007, 2009-2012 Heiko Oberdiek
   2016-2019 Oberdiek Package Support Group

This work may be distributed and/or modified under the
conditions of the LaTeX Project Public License, either
version 1.3c of this license or (at your option) any later
version. This version of this license is in
   https://www.latex-project.org/lppl/lppl-1-3c.txt
and the latest version of this license is in
   https://www.latex-project.org/lppl.txt
and version 1.3 or later is part of all distributions of
LaTeX version 2005/12/01 or later.

This work has the LPPL maintenance status "maintained".

The Current Maintainers of this work are
Heiko Oberdiek and the Oberdiek Package Support Group
https://github.com/ho-tex/kvsetkeys/issues


The Base Interpreter refers to any `TeX-Format',
because some files are installed in TDS:tex/generic//.

This work consists of the main source file kvsetkeys.dtx
and the derived files
   kvsetkeys.sty, kvsetkeys.pdf, kvsetkeys.ins, kvsetkeys.drv,
   kvsetkeys-example.tex, kvsetkeys-test1.tex,
   kvsetkeys-test2.tex, kvsetkeys-test3.tex,
   kvsetkeys-test4.tex.

\endpreamble
\let\MetaPrefix\DoubleperCent

\generate{%
  \file{kvsetkeys.ins}{\from{kvsetkeys.dtx}{install}}%
  \file{kvsetkeys.drv}{\from{kvsetkeys.dtx}{driver}}%
  \usedir{tex/generic/kvsetkeys}%
  \file{kvsetkeys.sty}{\from{kvsetkeys.dtx}{package}}%
  \usedir{doc/latex/kvsetkeys}%
  \file{kvsetkeys-example.tex}{\from{kvsetkeys.dtx}{example}}%
}

\catcode32=13\relax% active space
\let =\space%
\Msg{************************************************************************}
\Msg{*}
\Msg{* To finish the installation you have to move the following}
\Msg{* file into a directory searched by TeX:}
\Msg{*}
\Msg{*     kvsetkeys.sty}
\Msg{*}
\Msg{* To produce the documentation run the file `kvsetkeys.drv'}
\Msg{* through LaTeX.}
\Msg{*}
\Msg{* Happy TeXing!}
\Msg{*}
\Msg{************************************************************************}

\endbatchfile
%</install>
%<*ignore>
\fi
%</ignore>
%<*driver>
\NeedsTeXFormat{LaTeX2e}
\ProvidesFile{kvsetkeys.drv}%
  [2019/12/15 v1.18 Key value parser (HO)]%
\documentclass{ltxdoc}
\usepackage{holtxdoc}[2011/11/22]
\begin{document}
  \DocInput{kvsetkeys.dtx}%
\end{document}
%</driver>
% \fi
%
%
%
% \GetFileInfo{kvsetkeys.drv}
%
% \title{The \xpackage{kvsetkeys} package}
% \date{2019/12/15 v1.18}
% \author{Heiko Oberdiek\thanks
% {Please report any issues at \url{https://github.com/ho-tex/kvsetkeys/issues}}}
%
% \maketitle
%
% \begin{abstract}
% Package \xpackage{kvsetkeys} provides \cs{kvsetkeys}, a variant
% of package \xpackage{keyval}'s \cs{setkeys}. It allows to specify
% a handler that deals with unknown options. Active commas and equal
% signs may be used (e.g. see \xpackage{babel}'s shorthands) and
% only one level of curly braces is removed from the values.
% \end{abstract}
%
% \tableofcontents
%
% \def\M#1{\texttt{\{}\meta{#1}\texttt{\}}}
%
% \section{Documentation}
%
% First I want to recommend the very good review article
% ``A guide to key-value methods'' by Joseph Wright \cite{tb94wright}.
% It introduces the different key-value packages and compares them.
%
% \subsection{Motivation}
%
% \cs{kvsetkeys} serves as replacement for \xpackage{keyval}'s
% \cs{setkeys}. It basically uses the same syntax. But the
% implementation is more robust and predictable:
% \begin{description}
% \item[Active syntax characters:]
%   Comma `|,|' and the equals sign `|=|' are used inside
%   key value lists as syntax characters. Package \xpackage{keyval}
%   uses the catcode of the characters that is active during
%   package loading, usually this is catcode 12 (other).
%   But it can happen that the catcode setting of the syntax characters
%   changes. Especially active characters are of interest, because
%   some language adaptations uses them. For example, option \xoption{turkish}
%   of package \xpackage{babel} uses the equals sign as active shorthand
%   character. Therefore package \xpackage{kvsetkeys} deals with
%   both catcode settings 12 (other) and 13 (active).
% \item[Brace removal:]
%   Package \xpackage{keyval}'s \cs{setkeys} removes up to two
%   levels of curly braces around the value in some unpredictable way:
%\begin{quote}
%|\setkeys{fam}{key={{value}}} || || | $\rightarrow$ |value|\\
%|\setkeys{fam}{key={{{value}}}} | $\rightarrow$ |{value}|\\
%|\setkeys{fam}{key= {{{value}}}}| $\rightarrow$ |{{value}}|
%\end{quote}
%   This package \xpackage{kvsetkeys} follows a much stronger rule:
%   Exactly one level of braces are removed from an item, if the
%   item is surrounded by curly braces. An item can be a
%   the key value pair, the key or the value.
%\begin{quote}
%|\kvsetkeys{fam}{key={value}} || | $\rightarrow$ |value|\\
%|\kvsetkeys{fam}{key={{value}} | $\rightarrow$ |{value}|\\
%|\kvsetkeys{fam}{key= {{value}}| $\rightarrow$ |{value}|
%\end{quote}
% \item[Arbitrary values:] Unmatched conditionals are supported.
% \end{description}
%
% Before I describe
% \cs{kvsetkeys} in more detail, first I want to explain, how
% this package deals with key value lists. For the package also
% provides low level interfaces that can be used by package authors.
%
% \subsection{Normalizing key value lists}
%
% \begin{declcs}{kv@normalize}\,\M{key value list}
% \end{declcs}
% If the user specifies key value lists, he usually prefers
% nice formatted source code, e.g.:
% \begin{quote}
%\begin{verbatim}
%\hypersetup{
%  pdftitle    = {...},
%  pdfsubject  = {...},
%  pdfauthor   = {...},
%  pdfkeywords = {...},
%  ...
%}
%\end{verbatim}
% \end{quote}
% Thus there can be spaces around keys, around |=| or around the value.
% Also empty entries are possible by too many commas. Therefore these
% spaces and empty entries are silently removed by package \xpackage{keyval}
% and this package. Whereas the contents of the value can be protected
% by curly braces, especially if spaces or commas are used inside,
% a key name must not use spaces or other syntax characters.
%
% \cs{kv@normalize} takes a key value list and performs the cleanup:
% \begin{itemize}
% \item Spaces are removed.
% \item Syntax characters (comma and equal sign) that are active
%   are replaced by the same characters with standard catcode.
%   (Example: \xpackage{babel}'s language option \xoption{turkish}
%   uses the equal sign as active shorthand character.)
% \end{itemize}
% The result is stored in \cs{kv@list}, e.g.:
% \begin{quote}
% |\kv@list| $\rightarrow$ |,pdftitle={...},pdfsubject={...},...,|
% \end{quote}
% Curly braces around values (or keys) remain untouched.
% \begin{description}
% \item[v1.3+:]
% One comma is added in front of the list and each pair ends with
% a comma. Thus an empty list consists of one comma, otherwise
% two commas encloses the list. Empty entries other than the first
% are removed.
% \item[v1.0 -- v1.2:]
% Empty entries are removed later. In fact it adds a comma at the begin
% and end to protect the last value and an easier implementation.
% \end{description}
%
% \subsection{Parsing key value lists}
%
% \begin{declcs}{kv@parse}\,\M{key value list}\,\M{processor}
% \end{declcs}
% It is easier to parse a normalized list, thus \cs{kv@parse}
% normalizes the list and calls \cs{kv@parse@normalized}.
%
% \begin{declcs}{kv@parse@normalized}\,\M{key value list}%
%   \,\M{processor}
% \end{declcs}
% Now the key value list is split into single key value pairs.
% For further processing the key and value are given as arguments
% for the \meta{processor}:
% \begin{quote}
%   \meta{processor}\,\M{key}\,\M{value}
% \end{quote}
% Also key and value are stored in macro names:
% \begin{itemize}
% \item \cs{kv@key} stores the key.
% \item \cs{kv@value} stores the value or if the value was not
%   specified it has the meaning \cs{relax}.
% \end{itemize}
% The behaviour in pseudo code:
% \begin{quote}
%   foreach (\meta{key}, \meta{value}) in (\meta{key value list})\\
%   \hspace*{1.5em}\cs{kv@key} := \meta{key}\\
%   \hspace*{1.5em}\cs{kv@value} := \meta{value}\\
%   \hspace*{1.5em}\meta{processor}\,\M{key}\,\M{value}
% \end{quote}
%
% \begin{declcs}{kv@break}
% \end{declcs}
% Since version 2011/03/03 v1.11 \cs{kv@break} can be
% called inside the \meta{processor}
% of \cs{kv@parse} or \cs{kv@parse@normalized}, then
% the processing is stopped and the following entries discarded.
%
% \subsection{Processing key value pairs}
%
% Key value pairs can be processed in many different ways.
% For example, the processor for \cs{kvsetkeys} works similar
% to \cs{setkeys} of package \xpackage{keyval}. There unknown
% keys raise an error.
%
% Package \xpackage{xkeyval} also knows a star form of \cs{setkeys}
% that stores unknown keys in an internal macro for further processing
% with \cs{setrmkeys} and similar macros. This feature is covered
% by processor \cs{kv@processor@known}.
%
% \subsubsection{Processing similar to \xpackage{keyval}}
%
% \begin{declcs}{kv@processor@default}\,\M{family}\,\M{key}\,\M{value}
% \end{declcs}
% There are many possiblities to process key value pairs.
% \cs{kv@processor@default} is the processor used in \cs{kvsetkeys}.
% It reimplements and extends the behaviour of
% \xpackage{keyval}'s \cs{setkeys}.
% In case of unknown keys \cs{setkeys} raise an error.
% This processer, however, calls a handler instead, if it
% is provided by the family. Both \meta{family} and \meta{key}
% may contain package \xpackage{babel}'s shorthands
% (since 2011/04/07 v1.13).
%
% Since 2011/10/18 v1.15 the family handler can reject the
% successful handling of a key by calling \cs{kv@handled@false}.
%
% Since 2019/12/15 v1.18 \cs{kv@processor@default} also defines
% macro \cs{kv@fam} with meaning \meta{family} for convenience.
%
% \subsubsection{Processing similar to \cs{setkeys*} of package \xpackage{xkeyval}}
%
% \begin{declcs}{kv@processor@known}\,\M{family}\,^^A
%   \M{cmd}\,\M{key}\,\M{value}
% \end{declcs}
% The key value processor \cs{kv@processor@known} behaves similar
% to \cs{kv@processor@default}. If the \meta{key} exists in the
% \meta{family} its code is called, otherwise the family handler
% is tried. If the family handler is not set or cannot handle the
% key, the unknown key value pair is added to the macro \meta{cmd}.
% Since 2011/10/18 v1.15.
%
% The behaviour in pseudo code:
% \begin{quote}
%   if \meta{key} exists\\
%   \hspace*{1.5em}call the keyval code of \meta{key}\\
%   else\\
%   \hspace*{1.5em}if \meta{handler} for \meta{family} exists\\
%   \hspace*{3em}handled = true\\
%   \hspace*{3em}\meta{handler}\,\M{key}\,\M{value}\\
%   \hspace*{3em}if handled\\
%   \hspace*{3em}else\\
%   \hspace*{4.5em}add \texttt{"}\M{key}\texttt{=}\M{value}\texttt{"}
%     to \M{cmd}\\
%   \hspace*{3em}fi\\
%   \hspace*{1.5em}else\\
%   \hspace*{3em}add \texttt{"}\M{key}\texttt{=}\M{value}\texttt{"}
%     to \M{cmd}\\
%   \hspace*{3em}raise unknown key error\\
%   \hspace*{1.5em}fi\\
%   fi
% \end{quote}
%
% Since 2019/12/15 v1.18 \cs{kv@processor@known} also defines
% macro \cs{kv@fam} with meaning \meta{family} for convenience.
%
% \subsection{Default family handler}
%
% \cs{kv@processor@default} calls \meta{handler}, the default
% handler for the family, if the key does not exist in the family.
% The handler is called with two arguments, the key and the value.
% It can be defined with \cs{kv@set@family@hander}:
%
% \begin{declcs}{kv@set@family@handler}\,\M{family}\,\M{handler definition}
% \end{declcs}
% This sets the default family handler for the keyval family
% \meta{family}. Inside \meta{handler definition} |#1| stands for
% the key and |#2| is the value. Also \cs{kv@key} and \cs{kv@value}
% can be used for the key and the value. If the value is not given,
% \cs{kv@value} has the meaning \cs{relax}.
%
% \begin{declcs}{kv@unset@family@handler}\,\M{family}
% \end{declcs}
% It removes the family handler for \meta{family}.
% Since 2011/10/18 v1.15.
%
% \subsection{Put it all together}
%
% \begin{declcs}{kvsetkeys}\,\M{family}\,\M{key value list}
% \end{declcs}
% Macro \cs{kvsetkeys} processes the \meta{key value list} with
% the standard processor \cs{kv@processor@default}:
% \begin{quote}
%   \cs{kv@parse}\,\M{key value list}^^A
%   \texttt{\{}\cs{kv@processor@default}\,\M{family}\texttt{\}}
% \end{quote}
%
% \begin{declcs}{kvsetknownkeys}\,\M{family}\,\M{cmd}\,\M{key value list}
% \end{declcs}
% Macro \cs{kvsetknownkeys} processes the \meta{key value list}
% with processor \cs{kv@processor@known}. All key value pairs
% with keys that are not known in \meta{family} are stored
% in macro \meta{cmd}. A previous contents of macro \meta{cmd}
% will be overwritten. If all keys can be handled, \meta{cmd}
% will be empty, otherwise it contains a key value list of
% unhandled key value pairs.
% Since 2011/10/18 v1.15.
%
% Pseudo code:
% \begin{quote}
%   create macro \meta{cmdaux} with unique name (inside the current group)\\
%   \cs{def}\meta{cmdaux}\{\}\\
%   \cs{kv@parse}\,\M{key value list}^^A
%   \texttt{\{}\cs{kv@processor@known}\,\M{family}\,\M{cmdaux}\texttt{\}}\\
%   \cs{let}\meta{cmd}=\meta{cmdaux}
% \end{quote}
%
% \begin{declcs}{kvsetkeys@expandafter}\,\M{family}\,\M{list cmd}\\
%   \cs{kvsetknownkeys@expandafter}\,\M{family}\,%
%   \M{cmd}\,\M{list cmd}
% \end{declcs}
% Both macros behave like the counterparts without suffix
% |@expandafter|. The difference is that the key value list is
% given as macro that is expanded once.
% Since 2011/10/18 v1.15.
%
% Thus you can replace \cs{setkeys} of package \xpackage{keyval}
% by the key value parser of this package:
% \begin{quote}
%   |\renewcommand*{\setkeys}{\kvsetkeys}|\\
%   or\\
%   |\let\setkeys\kvsetkeys|
% \end{quote}
%
% \subsection{Comma separated lists}
%
% Since version 2007/09/29 v1.3 this package also supports the normalizing
% and parsing of general comma separated lists.
%
% \begin{declcs}{comma@normalize}\,\M{comma list}
% \end{declcs}
% Macro \cs{comma@normalize} normalizes the comma separated list,
% removes spaces around commas. The result is put in macro \cs{comma@list}.
%
% \begin{declcs}{comma@parse}\,\M{comma list}\,\M{processor}
% \end{declcs}
% Macro \cs{comma@parse} first normalizes the comma separated list
% and then parses the list by calling \cs{comma@parse@normalized}.
%
% \begin{declcs}{comma@parse@normalized}\,\M{normalized comma list}^^A
%   \,\M{processor}
% \end{declcs}
% The list is parsed. Empty entries are ignored. \meta{processor}
% is called for each non-empty entry with the entry as argument:
% \begin{quote}
%   \meta{processor}|{|\meta{entry}|}|
% \end{quote}
% Also the entry is stored in the macro \cs{comma@entry}.
%
% \begin{declcs}{comma@break}
% \end{declcs}
% Since version 2011/03/03 v1.11 \cs{comma@break} can be
% called inside the \meta{processor}
% of \cs{comma@parse} or \cs{comma@parse@normalized}, then
% the processing is stopped and the following entries discarded.
%
% \section{Example}
%
% The following example prints a short piece of HTML code using
% the tabbing environment for indenting purpose and a key value
% syntax for specifying the attributes of an HTML tag.
% The example illustrates the use of a default family handler.
%    \begin{macrocode}
%<*example>
\documentclass{article}
\usepackage[T1]{fontenc}
\usepackage{kvsetkeys}
\usepackage{keyval}

\makeatletter
\newcommand*{\tag}[2][]{%
  % #1: attributes
  % #2: tag name
  \begingroup
    \toks@={}%
    \let\@endslash\@empty
    \kvsetkeys{tag}{#1}%
    \texttt{%
      \textless #2\the\toks@\@endslash\textgreater
    }%
  \endgroup
}
\kv@set@family@handler{tag}{%
  % #1: key
  % #2: value
  \toks@\expandafter{%
    \the\toks@
    \space
    #1=\string"#2\string"%
  }%
}
\define@key{tag}{/}[]{%
  \def\@endslash{/}%
}
\makeatother

\begin{document}
\begin{tabbing}
  \mbox{}\qquad\=\qquad\=\kill
  \tag{html}\\
  \>\dots\\
  \>\tag[border=1]{table}\\
  \>\>\tag[width=200, span=3, /]{colgroup}\\
  \>\>\dots\\
  \>\tag{/table}\\
  \>\dots\\
  \tag{/html}\\
\end{tabbing}
\end{document}
%</example>
%    \end{macrocode}
%
% \StopEventually{
% }
%
% \section{Implementation}
%
% \subsection{Identification}
%
%    \begin{macrocode}
%<*package>
%    \end{macrocode}
%    Reload check, especially if the package is not used with \LaTeX.
%    \begin{macrocode}
\begingroup\catcode61\catcode48\catcode32=10\relax%
  \catcode13=5 % ^^M
  \endlinechar=13 %
  \catcode35=6 % #
  \catcode39=12 % '
  \catcode44=12 % ,
  \catcode45=12 % -
  \catcode46=12 % .
  \catcode58=12 % :
  \catcode64=11 % @
  \catcode123=1 % {
  \catcode125=2 % }
  \expandafter\let\expandafter\x\csname ver@kvsetkeys.sty\endcsname
  \ifx\x\relax % plain-TeX, first loading
  \else
    \def\empty{}%
    \ifx\x\empty % LaTeX, first loading,
      % variable is initialized, but \ProvidesPackage not yet seen
    \else
      \expandafter\ifx\csname PackageInfo\endcsname\relax
        \def\x#1#2{%
          \immediate\write-1{Package #1 Info: #2.}%
        }%
      \else
        \def\x#1#2{\PackageInfo{#1}{#2, stopped}}%
      \fi
      \x{kvsetkeys}{The package is already loaded}%
      \aftergroup\endinput
    \fi
  \fi
\endgroup%
%    \end{macrocode}
%    Package identification:
%    \begin{macrocode}
\begingroup\catcode61\catcode48\catcode32=10\relax%
  \catcode13=5 % ^^M
  \endlinechar=13 %
  \catcode35=6 % #
  \catcode39=12 % '
  \catcode40=12 % (
  \catcode41=12 % )
  \catcode44=12 % ,
  \catcode45=12 % -
  \catcode46=12 % .
  \catcode47=12 % /
  \catcode58=12 % :
  \catcode64=11 % @
  \catcode91=12 % [
  \catcode93=12 % ]
  \catcode123=1 % {
  \catcode125=2 % }
  \expandafter\ifx\csname ProvidesPackage\endcsname\relax
    \def\x#1#2#3[#4]{\endgroup
      \immediate\write-1{Package: #3 #4}%
      \xdef#1{#4}%
    }%
  \else
    \def\x#1#2[#3]{\endgroup
      #2[{#3}]%
      \ifx#1\@undefined
        \xdef#1{#3}%
      \fi
      \ifx#1\relax
        \xdef#1{#3}%
      \fi
    }%
  \fi
\expandafter\x\csname ver@kvsetkeys.sty\endcsname
\ProvidesPackage{kvsetkeys}%
  [2019/12/15 v1.18 Key value parser (HO)]%
%    \end{macrocode}
%
%    \begin{macrocode}
\begingroup\catcode61\catcode48\catcode32=10\relax%
  \catcode13=5 % ^^M
  \endlinechar=13 %
  \catcode123=1 % {
  \catcode125=2 % }
  \catcode64=11 % @
  \def\x{\endgroup
    \expandafter\edef\csname KVS@AtEnd\endcsname{%
      \endlinechar=\the\endlinechar\relax
      \catcode13=\the\catcode13\relax
      \catcode32=\the\catcode32\relax
      \catcode35=\the\catcode35\relax
      \catcode61=\the\catcode61\relax
      \catcode64=\the\catcode64\relax
      \catcode123=\the\catcode123\relax
      \catcode125=\the\catcode125\relax
    }%
  }%
\x\catcode61\catcode48\catcode32=10\relax%
\catcode13=5 % ^^M
\endlinechar=13 %
\catcode35=6 % #
\catcode64=11 % @
\catcode123=1 % {
\catcode125=2 % }
\def\TMP@EnsureCode#1#2{%
  \edef\KVS@AtEnd{%
    \KVS@AtEnd
    \catcode#1=\the\catcode#1\relax
  }%
  \catcode#1=#2\relax
}
\TMP@EnsureCode{36}{3}% $
\TMP@EnsureCode{38}{4}% &
\TMP@EnsureCode{39}{12}% '
\TMP@EnsureCode{43}{12}% +
\TMP@EnsureCode{44}{12}% ,
\TMP@EnsureCode{45}{12}% -
\TMP@EnsureCode{46}{12}% .
\TMP@EnsureCode{47}{12}% /
\TMP@EnsureCode{91}{12}% [
\TMP@EnsureCode{93}{12}% ]
\TMP@EnsureCode{94}{7}% ^ (superscript)
\TMP@EnsureCode{96}{12}% `
\TMP@EnsureCode{126}{13}% ~ (active)
\edef\KVS@AtEnd{\KVS@AtEnd\noexpand\endinput}
%    \end{macrocode}
%
% \subsection{Package loading}
%
%    \begin{macrocode}
\begingroup\expandafter\expandafter\expandafter\endgroup
\expandafter\ifx\csname RequirePackage\endcsname\relax
  \def\TMP@RequirePackage#1[#2]{%
    \begingroup\expandafter\expandafter\expandafter\endgroup
    \expandafter\ifx\csname ver@#1.sty\endcsname\relax
      \input #1.sty\relax
    \fi
  }%
  \TMP@RequirePackage{infwarerr}[2007/09/09]%
  \let\PackageError\@PackageError
\else
\fi
%    \end{macrocode}
%
%    \begin{macrocode}
\expandafter\ifx\csname toks@\endcsname\relax
  \toksdef\toks@=0 %
\fi
%    \end{macrocode}
%
% \subsection{Check for \eTeX}
%
%    \cs{unexpanded}, \cs{ifcsname}, and \cs{unless} are used if found.
%    \begin{macrocode}
\ifx\numexpr\@undefined
  \catcode`\$=9 % ignore
  \catcode`\&=14 % comment
\else % e-TeX
  \catcode`\$=14 % comment
  \catcode`\&=9  % ignore
\fi
%    \end{macrocode}
%
% \subsection{Generic help macros}
%
%    \begin{macro}{\KVS@Empty}
%    \begin{macrocode}
\def\KVS@Empty{}
%    \end{macrocode}
%    \end{macro}
%    \begin{macro}{\KVS@FirstOfTwo}
%    \begin{macrocode}
\long\def\KVS@FirstOfTwo#1#2{#1}
%    \end{macrocode}
%    \end{macro}
%    \begin{macro}{\KVS@SecondOfTwo}
%    \begin{macrocode}
\long\def\KVS@SecondOfTwo#1#2{#2}
%    \end{macrocode}
%    \end{macro}
%
%    \begin{macro}{\KVS@IfEmpty}
%    \begin{macrocode}
\long\def\KVS@IfEmpty#1{%
& \edef\KVS@Temp{\unexpanded{#1}}%
$ \begingroup
$   \toks@{#1}%
$   \edef\KVS@Temp{\the\toks@}%
$ \expandafter\endgroup
  \ifx\KVS@Temp\KVS@Empty
    \expandafter\KVS@FirstOfTwo
  \else
    \expandafter\KVS@SecondOfTwo
  \fi
}
%    \end{macrocode}
%    \end{macro}
%
% \subsection{Normalizing}
%
%    \begin{macro}{\kv@normalize}
%    \begin{macrocode}
\long\def\kv@normalize#1{%
  \begingroup
    \toks@{,#1,}%
    \KVS@Comma
    \KVS@SpaceComma
    \KVS@CommaSpace
    \KVS@CommaComma
    \KVS@Equals
    \KVS@SpaceEquals
    \KVS@EqualsSpace
    \xdef\KVS@Global{\the\toks@}%
  \endgroup
  \let\kv@list\KVS@Global
}
%    \end{macrocode}
%    \end{macro}
%    \begin{macro}{\comma@normalize}
%    \begin{macrocode}
\def\comma@normalize#1{%
  \begingroup
    \toks@{,#1,}%
    \KVS@Comma
    \KVS@SpaceComma
    \KVS@CommaSpace
    \KVS@CommaComma
    \xdef\KVS@Global{\the\toks@}%
  \endgroup
  \let\comma@list\KVS@Global
}
%    \end{macrocode}
%    \end{macro}
%    \begin{macro}{\KVS@Comma}
%    Converts active commas into comma with catcode other.
%    Also adds a comma at the end to protect the last value
%    for next cleanup steps.
%    \begin{macrocode}
\begingroup
  \lccode`\,=`\,%
  \lccode`\~=`\,%
\lowercase{\endgroup
  \def\KVS@Comma{%
    \toks@\expandafter{\expandafter}\expandafter
    \KVS@@Comma\the\toks@~\KVS@Nil
  }%
  \long\def\KVS@@Comma#1~#2\KVS@Nil{%
    \toks@\expandafter{\the\toks@#1}%
    \KVS@IfEmpty{#2}{%
    }{%
      \KVS@@Comma,#2\KVS@Nil
    }%
  }%
}
%    \end{macrocode}
%    \end{macro}
%    \begin{macro}{\KVS@SpaceComma}
%    Removes spaces before the comma, may add commas at the end.
%    \begin{macrocode}
\def\KVS@SpaceComma#1{%
  \def\KVS@SpaceComma{%
    \expandafter\KVS@@SpaceComma\the\toks@#1,\KVS@Nil
  }%
}
\KVS@SpaceComma{ }
%    \end{macrocode}
%    \end{macro}
%    \begin{macro}{\KVS@@SpaceComma}
%    \begin{macrocode}
\long\def\KVS@@SpaceComma#1 ,#2\KVS@Nil{%
  \KVS@IfEmpty{#2}{%
    \toks@{#1}%
  }{%
    \KVS@@SpaceComma#1,#2\KVS@Nil
  }%
}
%    \end{macrocode}
%    \end{macro}
%    \begin{macro}{\KVS@CommaSpace}
%    Removes spaces after the comma, may add commas at the end.
%    \begin{macrocode}
\def\KVS@CommaSpace{%
  \expandafter\KVS@@CommaSpace\the\toks@, \KVS@Nil
}
%    \end{macrocode}
%    \end{macro}
%    \begin{macro}{\KVS@@CommaSpace}
%    \begin{macrocode}
\long\def\KVS@@CommaSpace#1, #2\KVS@Nil{%
  \KVS@IfEmpty{#2}{%
    \toks@{#1}%
  }{%
    \KVS@@CommaSpace#1,#2\KVS@Nil
  }%
}
%    \end{macrocode}
%    \end{macro}
%    \begin{macro}{\KVS@CommaComma}
%    Replaces multiple commas by one comma.
%    \begin{macrocode}
\def\KVS@CommaComma{%
  \expandafter\KVS@@CommaComma\the\toks@,\KVS@Nil
}
%    \end{macrocode}
%    \end{macro}
%    \begin{macro}{\KVS@@CommaComma}
%    \begin{macrocode}
\long\def\KVS@@CommaComma#1,,#2\KVS@Nil{%
  \KVS@IfEmpty{#2}{%
    \toks@{#1,}% (!)
  }{%
    \KVS@@CommaComma#1,#2\KVS@Nil
  }%
}
%    \end{macrocode}
%    \end{macro}
%
%    \begin{macro}{\KVS@Equals}
%    Converts active equals signs into catcode other characters.
%    \begin{macrocode}
\begingroup
  \lccode`\==`\=%
  \lccode`\~=`\=%
\lowercase{\endgroup
  \def\KVS@Equals{%
    \toks@\expandafter{\expandafter}\expandafter
    \KVS@@Equals\the\toks@~\KVS@Nil
  }%
  \long\def\KVS@@Equals#1~#2\KVS@Nil{%
    \edef\KVS@Temp{\the\toks@}%
    \ifx\KVS@Temp\KVS@Empty
      \expandafter\KVS@FirstOfTwo
    \else
      \expandafter\KVS@SecondOfTwo
    \fi
    {%
      \toks@{#1}%
    }{%
      \toks@\expandafter{\the\toks@=#1}%
    }%
    \KVS@IfEmpty{#2}{%
    }{%
      \KVS@@Equals#2\KVS@Nil
    }%
  }%
}
%    \end{macrocode}
%    \end{macro}
%    \begin{macro}{\KVS@SpaceEquals}
%    Removes spaces before the equals sign.
%    \begin{macrocode}
\def\KVS@SpaceEquals#1{%
  \def\KVS@SpaceEquals{%
    \expandafter\KVS@@SpaceEquals\the\toks@#1=\KVS@Nil
  }%
}
\KVS@SpaceEquals{ }
%    \end{macrocode}
%    \end{macro}
%    \begin{macro}{\KVS@@SpaceEquals}
%    \begin{macrocode}
\long\def\KVS@@SpaceEquals#1 =#2\KVS@Nil{%
  \KVS@IfEmpty{#2}{%
    \toks@{#1}%
  }{%
    \KVS@@SpaceEquals#1=#2\KVS@Nil
  }%
}
%    \end{macrocode}
%    \end{macro}
%    \begin{macro}{\KVS@EqualsSpace}
%    Removes spaces after the equals sign.
%    \begin{macrocode}
\def\KVS@EqualsSpace{%
  \expandafter\KVS@@EqualsSpace\the\toks@= \KVS@Nil
}
%    \end{macrocode}
%    \end{macro}
%    \begin{macro}{\KVS@@EqualsSpace}
%    \begin{macrocode}
\long\def\KVS@@EqualsSpace#1= #2\KVS@Nil{%
  \KVS@IfEmpty{#2}{%
    \toks@{#1}%
  }{%
    \KVS@@EqualsSpace#1=#2\KVS@Nil
  }%
}
%    \end{macrocode}
%    \end{macro}
%
% \subsection{Parsing key value lists}
%
%    \begin{macro}{\kv@parse}
%    Normalizes and parses the key value list. Also sets \cs{kv@list}.
%    \begin{macrocode}
\long\def\kv@parse#1{%
  \kv@normalize{#1}%
  \expandafter\kv@parse@normalized\expandafter{\kv@list}%
}
%    \end{macrocode}
%    \end{macro}
%    \begin{macro}{\kv@parse@normalized}
%    |#1|: key value list\\
%    |#2|: processor
%    \begin{macrocode}
\long\def\kv@parse@normalized#1#2{%
  \KVS@Parse#1,\KVS@Nil{#2}%
}
%    \end{macrocode}
%    \end{macro}
%    \begin{macro}{\KVS@Parse}
%    |#1,#2|: key value list\\
%    |#3|: processor
%    \begin{macrocode}
\long\def\KVS@Parse#1,#2\KVS@Nil#3{%
  \KVS@IfEmpty{#1}{%
  }{%
    \KVS@Process#1=\KVS@Nil{#3}%
  }%
  \KVS@MaybeBreak
  \KVS@IfEmpty{#2}{%
  }{%
    \KVS@Parse#2\KVS@Nil{#3}%
  }%
}
%    \end{macrocode}
%    \end{macro}
%    \begin{macro}{\KVS@Process}
%    |#1|: key\\
%    |#2|: value, |=|\\
%    |#3|: processor
%    \begin{macrocode}
\long\def\KVS@Process#1=#2\KVS@Nil#3{%
  \let\KVS@MaybeBreak\relax
  \def\kv@key{#1}%
  \KVS@IfEmpty{#2}{%
    \let\kv@value\relax
    #3{#1}{}%
  }{%
    \KVS@@Process{#1}#2\KVS@Nil{#3}%
  }%
}
%    \end{macrocode}
%    \end{macro}
%    \begin{macro}{\KVS@@Process}
%    |#1|: key\\
%    |#2|: value\\
%    |#3|: processor
%    \begin{macrocode}
\long\def\KVS@@Process#1#2=\KVS@Nil#3{%
& \edef\kv@value{\unexpanded{#2}}%
$ \begingroup
$   \toks@{#2}%
$   \xdef\KVS@Global{\the\toks@}%
$ \endgroup
$ \let\kv@value\KVS@Global
  #3{#1}{#2}%
}
%    \end{macrocode}
%    \end{macro}
%    \begin{macro}{\KVS@MaybeBreak}
%    \begin{macrocode}
\let\KVS@MaybeBreak\relax
%    \end{macrocode}
%    \end{macro}
%    \begin{macro}{\KVS@break}
%    \begin{macrocode}
\def\KVS@break#1#2#3#4{%
  \let\KVS@MaybeBreak\relax
}
%    \end{macrocode}
%    \end{macro}
%    \begin{macro}{\kv@break}
%    \begin{macrocode}
\def\kv@break{%
  \let\KVS@MaybeBreak\KVS@break
}
%    \end{macrocode}
%    \end{macro}
%
% \subsection{Parsing comma lists}
%
%    \begin{macro}{\comma@parse}
%    Normalizes and parses the key value list. Also sets \cs{comma@list}.
%    \begin{macrocode}
\def\comma@parse#1{%
  \comma@normalize{#1}%
  \expandafter\comma@parse@normalized\expandafter{\comma@list}%
}
%    \end{macrocode}
%    \end{macro}
%
%    \begin{macro}{\comma@parse@normalized}
%    |#1|: comma list\\
%    |#2|: processor
%    \begin{macrocode}
\def\comma@parse@normalized#1#2{%
  \KVS@CommaParse#1,\KVS@Nil{#2}%
}
%    \end{macrocode}
%    \end{macro}
%    \begin{macro}{\KVS@CommaParse}
%    |#1,#2|: comma list\\
%    |#3|: processor
%    \begin{macrocode}
\def\KVS@CommaParse#1,#2\KVS@Nil#3{%
  \KVS@IfEmpty{#1}{%
  }{%
    \def\comma@entry{#1}%
    #3{#1}%
  }%
  \KVS@MaybeBreak
  \KVS@IfEmpty{#2}{%
  }{%
    \KVS@CommaParse#2\KVS@Nil{#3}%
  }%
}
%    \end{macrocode}
%    \end{macro}
%    \begin{macro}{\comma@break}
%    \begin{macrocode}
\def\comma@break{%
  \let\KVS@MaybeBreak\KVS@break
}
%    \end{macrocode}
%    \end{macro}
%
% \subsection{Processing key value pairs}
%
%    \begin{macro}{\kv@handled@false}
%    The handler can call \cs{kv@handled@false} or
%    \cs{kv@handled@true} so report failure or success.
%    The default is success (compatibility for versions
%    before 2011/10/18 v1.15).
%    \begin{macrocode}
\def\kv@handled@false{%
  \let\ifkv@handled@\iffalse
}
%    \end{macrocode}
%    \end{macro}
%    \begin{macro}{\kv@handled@true}
%    \begin{macrocode}
\def\kv@handled@true{%
  \let\ifkv@handled@\iftrue
}
%    \end{macrocode}
%    \end{macro}
%    \begin{macro}{\ifkv@handled@}
%    \begin{macrocode}
\kv@handled@true
%    \end{macrocode}
%    \end{macro}
%
%    \begin{macro}{\kv@processor@default}
%    \begin{macrocode}
\def\kv@processor@default#1#2{%
  \begingroup
    \csname @safe@activestrue\endcsname
    \let\ifincsname\iftrue
    \edef\KVS@temp{\endgroup
      \noexpand\KVS@ProcessorDefault{#1}{#2}%
    }%
  \KVS@temp
}
%    \end{macrocode}
%    \end{macro}
%    \begin{macro}{\KVS@ProcessorDefault}
%    \begin{macrocode}
\long\def\KVS@ProcessorDefault#1#2#3{%
  \def\kv@fam{#1}%
& \unless\ifcsname KV@#1@#2\endcsname
$ \begingroup\expandafter\expandafter\expandafter\endgroup
$ \expandafter\ifx\csname KV@#1@#2\endcsname\relax
&   \unless\ifcsname KVS@#1@handler\endcsname
$   \begingroup\expandafter\expandafter\expandafter\endgroup
$   \expandafter\ifx\csname KVS@#1@handler\endcsname\relax
      \kv@error@unknownkey{#1}{#2}%
    \else
      \kv@handled@true
      \csname KVS@#1@handler\endcsname{#2}{#3}%
      \relax
      \ifkv@handled@
      \else
        \kv@error@unknownkey{#1}{#2}%
      \fi
    \fi
  \else
    \ifx\kv@value\relax
&     \unless\ifcsname KV@#1@#2@default\endcsname
$     \begingroup\expandafter\expandafter\expandafter\endgroup
$     \expandafter\ifx\csname KV@#1@#2@default\endcsname\relax
        \kv@error@novalue{#1}{#2}%
      \else
        \csname KV@#1@#2@default\endcsname
        \relax
      \fi
    \else
      \csname KV@#1@#2\endcsname{#3}%
    \fi
  \fi
}
%    \end{macrocode}
%    \end{macro}
%
%    \begin{macro}{\kv@processor@known}
%    \begin{macrocode}
\def\kv@processor@known#1#2#3{%
  \begingroup
    \csname @safe@activestrue\endcsname
    \let\ifincsname\iftrue
    \edef\KVS@temp{\endgroup
      \noexpand\KVS@ProcessorKnown{#1}\noexpand#2{#3}%
    }%
  \KVS@temp
}
%    \end{macrocode}
%    \end{macro}
%    \begin{macro}{\KVS@ProcessorKnown}
%    \begin{macrocode}
\long\def\KVS@ProcessorKnown#1#2#3#4{%
  \def\kv@fam{#1}%
& \unless\ifcsname KV@#1@#3\endcsname
$ \begingroup\expandafter\expandafter\expandafter\endgroup
$ \expandafter\ifx\csname KV@#1@#3\endcsname\relax
&   \unless\ifcsname KVS@#1@handler\endcsname
$   \begingroup\expandafter\expandafter\expandafter\endgroup
$   \expandafter\ifx\csname KVS@#1@handler\endcsname\relax
      \KVS@AddUnhandled#2{#3}{#4}%
    \else
      \kv@handled@true
      \csname KVS@#1@handler\endcsname{#3}{#4}%
      \relax
      \ifkv@handled@
      \else
        \KVS@AddUnhandled#2{#3}{#4}%
      \fi
    \fi
  \else
    \ifx\kv@value\relax
&     \unless\ifcsname KV@#1@#2@default\endcsname
$     \begingroup\expandafter\expandafter\expandafter\endgroup
$     \expandafter\ifx\csname KV@#1@#3@default\endcsname\relax
        \kv@error@novalue{#1}{#3}%
      \else
        \csname KV@#1@#3@default\endcsname
        \relax
      \fi
    \else
      \csname KV@#1@#3\endcsname{#4}%
    \fi
  \fi
}
%    \end{macrocode}
%    \end{macro}
%    \begin{macro}{\KVS@AddUnhandled}
%    \begin{macrocode}
\long\def\KVS@AddUnhandled#1#2#3{%
& \edef#1{%
&   \ifx#1\KVS@empty
&   \else
&     \unexpanded{#1},%
&   \fi
&   \unexpanded{{#2}={#3}}%
& }%
$ \begingroup
$   \ifx#1\KVS@empty
$     \toks@{{#2}={#3}}%
$   \else
$     \toks@\expandafter{#1,{#2}={#3}}%
$   \fi
$   \xdef\KVS@Global{\the\toks@}%
$ \endgroup
$ \let#1\KVS@Global
}
%    \end{macrocode}
%    \end{macro}
%
%    \begin{macro}{\kv@set@family@handler}
%    \begin{macrocode}
\long\def\kv@set@family@handler#1#2{%
  \begingroup
    \csname @safe@activestrue\endcsname
    \let\ifincsname\iftrue
  \expandafter\endgroup
  \expandafter\def\csname KVS@#1@handler\endcsname##1##2{#2}%
}
%    \end{macrocode}
%    \end{macro}
%
%    \begin{macro}{\kv@unset@family@handler}
%    \begin{macrocode}
\long\def\kv@unset@family@handler#1#2{%
  \begingroup
    \csname @safe@activestrue\endcsname
    \let\ifincsname\iftrue
  \expandafter\endgroup
  \expandafter\let\csname KVS@#1@handler\endcsname\@UnDeFiNeD
}
%    \end{macrocode}
%    \end{macro}
%
% \subsection{Error handling}
%
%    \begin{macro}{\kv@error@novalue}
%    \begin{macrocode}
\def\kv@error@novalue{%
  \kv@error@generic{No value specified for}%
}
%    \end{macrocode}
%    \end{macro}
%    \begin{macro}{\kv@error@unknownkey}
%    \begin{macrocode}
\def\kv@error@unknownkey{%
  \kv@error@generic{Undefined}%
}
%    \end{macrocode}
%    \end{macro}
%    \begin{macro}{\kv@error@generic}
%    \begin{macrocode}
\def\kv@error@generic#1#2#3{%
  \PackageError{kvsetkeys}{%
    #1 key `#3'%
  }{%
    The keyval family of the key `#3' is `#2'.\MessageBreak
    The setting of the key is ignored because of the error.\MessageBreak
    \MessageBreak
    \@ehc
  }%
}
%    \end{macrocode}
%    \end{macro}
%
% \subsection{Do it all}
%
%    \begin{macro}{\kvsetkeys}
%    \begin{macrocode}
\long\def\kvsetkeys#1#2{%
  \kv@parse{#2}{\kv@processor@default{#1}}%
}
%    \end{macrocode}
%    \end{macro}
%
%    \begin{macro}{\kvsetkeys@expandafter}
%    \begin{macrocode}
\def\kvsetkeys@expandafter#1#2{%
  \expandafter\kv@parse\expandafter{#2}{%
    \kv@processor@default{#1}%
  }%
}
%    \end{macrocode}
%    \end{macro}
%
%    \begin{macro}{\KVS@cmd}
%    \begin{macrocode}
\def\KVS@cmd{0}%
%    \end{macrocode}
%    \end{macro}
%    \begin{macro}{\KVS@cmd@inc}
%    \begin{macrocode}
\def\KVS@cmd@inc{%
& \edef\KVS@cmd{\the\numexpr\KVS@cmd+1}%
$ \begingroup
$   \count255=\KVS@cmd\relax
$   \advance\count255 by 1\relax
$ \edef\x{\endgroup
$   \noexpand\def\noexpand\KVS@cmd{\number\count255}%
$ }%
$ \x
}
%    \end{macrocode}
%    \end{macro}
%    \begin{macro}{\KVS@cmd@dec}
%    \begin{macrocode}
\def\KVS@cmd@dec{%
& \edef\KVS@cmd{\the\numexpr\KVS@cmd-1}%
$ \begingroup
$   \count255=\KVS@cmd\relax
$   \advance\count255 by -1\relax
$ \edef\x{\endgroup
$   \noexpand\def\noexpand\KVS@cmd{\number\count255}%
$ }%
$ \x
}
%    \end{macrocode}
%    \end{macro}
%    \begin{macro}{\KVS@empty}
%    \begin{macrocode}
\def\KVS@empty{}
%    \end{macrocode}
%    \end{macro}
%
%    \begin{macro}{\kvsetknownkeys}
%    \begin{macrocode}
\def\kvsetknownkeys{%
  \expandafter
  \KVS@setknownkeys\csname KVS@cmd\KVS@cmd\endcsname{}%
}
%    \end{macrocode}
%    \end{macro}
%    \begin{macro}{\KVS@setknownkeys}
%    \begin{macrocode}
\long\def\KVS@setknownkeys#1#2#3#4#5{%
  \let#1\KVS@empty
  \KVS@cmd@inc
  #2\kv@parse#2{#5}{\kv@processor@known{#3}#1}%
  \KVS@cmd@dec
  \let#4=#1%
}
%    \end{macrocode}
%    \end{macro}
%
%    \begin{macro}{\kvsetknownkeys@expandafter}
%    \begin{macrocode}
\def\kvsetknownkeys@expandafter{%
  \expandafter
  \KVS@setknownkeys
      \csname KVS@cmd\KVS@cmd\endcsname\expandafter
}
%    \end{macrocode}
%    \end{macro}
%
%    \begin{macrocode}
\KVS@AtEnd%
%</package>
%    \end{macrocode}
%% \section{Installation}
%
% \subsection{Download}
%
% \paragraph{Package.} This package is available on
% CTAN\footnote{\CTANpkg{kvsetkeys}}:
% \begin{description}
% \item[\CTAN{macros/latex/contrib/kvsetkeys/kvsetkeys.dtx}] The source file.
% \item[\CTAN{macros/latex/contrib/kvsetkeys/kvsetkeys.pdf}] Documentation.
% \end{description}
%
%
% \paragraph{Bundle.} All the packages of the bundle `kvsetkeys'
% are also available in a TDS compliant ZIP archive. There
% the packages are already unpacked and the documentation files
% are generated. The files and directories obey the TDS standard.
% \begin{description}
% \item[\CTANinstall{install/macros/latex/contrib/kvsetkeys.tds.zip}]
% \end{description}
% \emph{TDS} refers to the standard ``A Directory Structure
% for \TeX\ Files'' (\CTANpkg{tds}). Directories
% with \xfile{texmf} in their name are usually organized this way.
%
% \subsection{Bundle installation}
%
% \paragraph{Unpacking.} Unpack the \xfile{kvsetkeys.tds.zip} in the
% TDS tree (also known as \xfile{texmf} tree) of your choice.
% Example (linux):
% \begin{quote}
%   |unzip kvsetkeys.tds.zip -d ~/texmf|
% \end{quote}
%
% \subsection{Package installation}
%
% \paragraph{Unpacking.} The \xfile{.dtx} file is a self-extracting
% \docstrip\ archive. The files are extracted by running the
% \xfile{.dtx} through \plainTeX:
% \begin{quote}
%   \verb|tex kvsetkeys.dtx|
% \end{quote}
%
% \paragraph{TDS.} Now the different files must be moved into
% the different directories in your installation TDS tree
% (also known as \xfile{texmf} tree):
% \begin{quote}
% \def\t{^^A
% \begin{tabular}{@{}>{\ttfamily}l@{ $\rightarrow$ }>{\ttfamily}l@{}}
%   kvsetkeys.sty & tex/generic/kvsetkeys/kvsetkeys.sty\\
%   kvsetkeys.pdf & doc/latex/kvsetkeys/kvsetkeys.pdf\\
%   kvsetkeys-example.tex & doc/latex/kvsetkeys/kvsetkeys-example.tex\\
%   kvsetkeys.dtx & source/latex/kvsetkeys/kvsetkeys.dtx\\
% \end{tabular}^^A
% }^^A
% \sbox0{\t}^^A
% \ifdim\wd0>\linewidth
%   \begingroup
%     \advance\linewidth by\leftmargin
%     \advance\linewidth by\rightmargin
%   \edef\x{\endgroup
%     \def\noexpand\lw{\the\linewidth}^^A
%   }\x
%   \def\lwbox{^^A
%     \leavevmode
%     \hbox to \linewidth{^^A
%       \kern-\leftmargin\relax
%       \hss
%       \usebox0
%       \hss
%       \kern-\rightmargin\relax
%     }^^A
%   }^^A
%   \ifdim\wd0>\lw
%     \sbox0{\small\t}^^A
%     \ifdim\wd0>\linewidth
%       \ifdim\wd0>\lw
%         \sbox0{\footnotesize\t}^^A
%         \ifdim\wd0>\linewidth
%           \ifdim\wd0>\lw
%             \sbox0{\scriptsize\t}^^A
%             \ifdim\wd0>\linewidth
%               \ifdim\wd0>\lw
%                 \sbox0{\tiny\t}^^A
%                 \ifdim\wd0>\linewidth
%                   \lwbox
%                 \else
%                   \usebox0
%                 \fi
%               \else
%                 \lwbox
%               \fi
%             \else
%               \usebox0
%             \fi
%           \else
%             \lwbox
%           \fi
%         \else
%           \usebox0
%         \fi
%       \else
%         \lwbox
%       \fi
%     \else
%       \usebox0
%     \fi
%   \else
%     \lwbox
%   \fi
% \else
%   \usebox0
% \fi
% \end{quote}
% If you have a \xfile{docstrip.cfg} that configures and enables \docstrip's
% TDS installing feature, then some files can already be in the right
% place, see the documentation of \docstrip.
%
% \subsection{Refresh file name databases}
%
% If your \TeX~distribution
% (\TeX\,Live, \mikTeX, \dots) relies on file name databases, you must refresh
% these. For example, \TeX\,Live\ users run \verb|texhash| or
% \verb|mktexlsr|.
%
% \subsection{Some details for the interested}
%
% \paragraph{Unpacking with \LaTeX.}
% The \xfile{.dtx} chooses its action depending on the format:
% \begin{description}
% \item[\plainTeX:] Run \docstrip\ and extract the files.
% \item[\LaTeX:] Generate the documentation.
% \end{description}
% If you insist on using \LaTeX\ for \docstrip\ (really,
% \docstrip\ does not need \LaTeX), then inform the autodetect routine
% about your intention:
% \begin{quote}
%   \verb|latex \let\install=y\input{kvsetkeys.dtx}|
% \end{quote}
% Do not forget to quote the argument according to the demands
% of your shell.
%
% \paragraph{Generating the documentation.}
% You can use both the \xfile{.dtx} or the \xfile{.drv} to generate
% the documentation. The process can be configured by the
% configuration file \xfile{ltxdoc.cfg}. For instance, put this
% line into this file, if you want to have A4 as paper format:
% \begin{quote}
%   \verb|\PassOptionsToClass{a4paper}{article}|
% \end{quote}
% An example follows how to generate the
% documentation with pdf\LaTeX:
% \begin{quote}
%\begin{verbatim}
%pdflatex kvsetkeys.dtx
%makeindex -s gind.ist kvsetkeys.idx
%pdflatex kvsetkeys.dtx
%makeindex -s gind.ist kvsetkeys.idx
%pdflatex kvsetkeys.dtx
%\end{verbatim}
% \end{quote}
%
% \begin{thebibliography}{9}
% \bibitem{tb94wright}
%   A guide to key-value methods, Joseph Wright, second draft for
%   \href{https://www.tug.org/tugboat}{TUGBoat}, 2009-03-17.
%   \url{https://www.texdev.net/uploads/2009/03/keyval.pdf}
%
% \bibitem{keyval}
%   David Carlisle:
%   \textit{The \xpackage{keyval} package};
%   1999/03/16 v1.13;
%   \CTANpkg{keyval}.
%
% \end{thebibliography}
%
% \begin{History}
%   \begin{Version}{2006/03/06 v1.0}
%   \item
%     First version.
%   \end{Version}
%   \begin{Version}{2006/10/19 v1.1}
%   \item
%     Fix of \cs{kv@set@family@handler}.
%   \item
%     Example added.
%   \end{Version}
%   \begin{Version}{2007/09/09 v1.2}
%   \item
%     Using package \xpackage{infwarerr} for error messages.
%   \item
%     Catcode section rewritten.
%   \end{Version}
%   \begin{Version}{2007/09/29 v1.3}
%   \item
%     Normalizing and parsing of comma separated lists added.
%   \item
%     \cs{kv@normalize} rewritten.
%   \item
%     Robustness increased for normalizing and parsing,
%     e.g. for values with unmatched conditionals.
%   \item
%     \eTeX\ is used if available.
%   \item
%     Tests added for normalizing and parsing.
%   \end{Version}
%   \begin{Version}{2009/07/19 v1.4}
%   \item
%     Bug fix for \cs{kv@normalize}: unwanted space removed (Florent Chervet).
%   \end{Version}
%   \begin{Version}{2009/07/30 v1.5}
%   \item
%     Documentation addition: recommendation for Joseph Wright's
%     review article.
%   \end{Version}
%   \begin{Version}{2009/12/12 v1.6}
%   \item
%     Short info shortened.
%   \end{Version}
%   \begin{Version}{2009/12/22 v1.7}
%   \item
%     Internal optimization (\cs{KVS@CommaSpace}, \dots, \cs{KVS@EqualsSpace}).
%   \end{Version}
%   \begin{Version}{2010/01/28 v1.8}
%   \item
%     Compatibility to ini\TeX\ added.
%   \end{Version}
%   \begin{Version}{2010/03/01 v1.9}
%   \item
%     Support of \cs{par} inside values.
%   \end{Version}
%   \begin{Version}{2011/01/30 v1.10}
%   \item
%     Already loaded package files are not input in \hologo{plainTeX}.
%   \end{Version}
%   \begin{Version}{2011/03/03 v1.11}
%   \item
%     \cs{kv@break} and \cs{comma@break} added.
%   \end{Version}
%   \begin{Version}{2011/04/05 v1.12}
%   \item
%     Error message with recovery action in help message
%     (request by GL).
%   \end{Version}
%   \begin{Version}{2011/04/07 v1.13}
%   \item
%     \cs{kv@processor@default} supports package \xpackage{babel}'s
%     shorthands.
%   \item
%     \cs{kv@set@family@handler} with shorthand support.
%   \end{Version}
%   \begin{Version}{2011/06/15 v1.14}
%   \item
%     Some optimizations in token register uses (GL, HO).
%   \end{Version}
%   \begin{Version}{2011/10/18 v1.15}
%   \item
%     \cs{kv@processor@known} and \cs{kvsetknownkeys} added.
%   \item
%     \cs{kvsetkeys@expandafter} and \cs{kvsetknownkeys@expandafter} added.
%   \item
%     Family handler can report success or failure by \cs{kv@handled@true}
%     or \cs{kv@handled@false}.
%   \item
%     \cs{kv@unset@family@handler} added.
%   \end{Version}
%   \begin{Version}{2012/04/25 v1.16}
%   \item
%     \cs{kv@processor@default} and \cs{kv@processor@known} define
%     macro \cs{kv@fam} for convenience.
%   \item
%     Catcode section: Catcode setting for \texttt{+} added for \hologo{eTeX}.
%   \end{Version}
%   \begin{Version}{2016/05/16 v1.17}
%   \item
%     Documentation updates.
%   \end{Version}
%   \begin{Version}{2019/12/15 v1.18}
%   \item
%     Documentation updates.
%   \item
%     Avoid \xpackage{etexcmds} and \xpackage{infwarwerr} in \LaTeX.
%   \end{Version}
% \end{History}
%
% \PrintIndex
%
% \Finale
\endinput

%        (quote the arguments according to the demands of your shell)
%
% Documentation:
%    (a) If kvsetkeys.drv is present:
%           latex kvsetkeys.drv
%    (b) Without kvsetkeys.drv:
%           latex kvsetkeys.dtx; ...
%    The class ltxdoc loads the configuration file ltxdoc.cfg
%    if available. Here you can specify further options, e.g.
%    use A4 as paper format:
%       \PassOptionsToClass{a4paper}{article}
%
%    Programm calls to get the documentation (example):
%       pdflatex kvsetkeys.dtx
%       makeindex -s gind.ist kvsetkeys.idx
%       pdflatex kvsetkeys.dtx
%       makeindex -s gind.ist kvsetkeys.idx
%       pdflatex kvsetkeys.dtx
%
% Installation:
%    TDS:tex/generic/kvsetkeys/kvsetkeys.sty
%    TDS:doc/latex/kvsetkeys/kvsetkeys.pdf
%    TDS:doc/latex/kvsetkeys/kvsetkeys-example.tex
%    TDS:source/latex/kvsetkeys/kvsetkeys.dtx
%
%<*ignore>
\begingroup
  \catcode123=1 %
  \catcode125=2 %
  \def\x{LaTeX2e}%
\expandafter\endgroup
\ifcase 0\ifx\install y1\fi\expandafter
         \ifx\csname processbatchFile\endcsname\relax\else1\fi
         \ifx\fmtname\x\else 1\fi\relax
\else\csname fi\endcsname
%</ignore>
%<*install>
\input docstrip.tex
\Msg{************************************************************************}
\Msg{* Installation}
\Msg{* Package: kvsetkeys 2019/12/15 v1.18 Key value parser (HO)}
\Msg{************************************************************************}

\keepsilent
\askforoverwritefalse

\let\MetaPrefix\relax
\preamble

This is a generated file.

Project: kvsetkeys
Version: 2019/12/15 v1.18

Copyright (C)
   2006, 2007, 2009-2012 Heiko Oberdiek
   2016-2019 Oberdiek Package Support Group

This work may be distributed and/or modified under the
conditions of the LaTeX Project Public License, either
version 1.3c of this license or (at your option) any later
version. This version of this license is in
   https://www.latex-project.org/lppl/lppl-1-3c.txt
and the latest version of this license is in
   https://www.latex-project.org/lppl.txt
and version 1.3 or later is part of all distributions of
LaTeX version 2005/12/01 or later.

This work has the LPPL maintenance status "maintained".

The Current Maintainers of this work are
Heiko Oberdiek and the Oberdiek Package Support Group
https://github.com/ho-tex/kvsetkeys/issues


The Base Interpreter refers to any `TeX-Format',
because some files are installed in TDS:tex/generic//.

This work consists of the main source file kvsetkeys.dtx
and the derived files
   kvsetkeys.sty, kvsetkeys.pdf, kvsetkeys.ins, kvsetkeys.drv,
   kvsetkeys-example.tex, kvsetkeys-test1.tex,
   kvsetkeys-test2.tex, kvsetkeys-test3.tex,
   kvsetkeys-test4.tex.

\endpreamble
\let\MetaPrefix\DoubleperCent

\generate{%
  \file{kvsetkeys.ins}{\from{kvsetkeys.dtx}{install}}%
  \file{kvsetkeys.drv}{\from{kvsetkeys.dtx}{driver}}%
  \usedir{tex/generic/kvsetkeys}%
  \file{kvsetkeys.sty}{\from{kvsetkeys.dtx}{package}}%
  \usedir{doc/latex/kvsetkeys}%
  \file{kvsetkeys-example.tex}{\from{kvsetkeys.dtx}{example}}%
}

\catcode32=13\relax% active space
\let =\space%
\Msg{************************************************************************}
\Msg{*}
\Msg{* To finish the installation you have to move the following}
\Msg{* file into a directory searched by TeX:}
\Msg{*}
\Msg{*     kvsetkeys.sty}
\Msg{*}
\Msg{* To produce the documentation run the file `kvsetkeys.drv'}
\Msg{* through LaTeX.}
\Msg{*}
\Msg{* Happy TeXing!}
\Msg{*}
\Msg{************************************************************************}

\endbatchfile
%</install>
%<*ignore>
\fi
%</ignore>
%<*driver>
\NeedsTeXFormat{LaTeX2e}
\ProvidesFile{kvsetkeys.drv}%
  [2019/12/15 v1.18 Key value parser (HO)]%
\documentclass{ltxdoc}
\usepackage{holtxdoc}[2011/11/22]
\begin{document}
  \DocInput{kvsetkeys.dtx}%
\end{document}
%</driver>
% \fi
%
%
%
% \GetFileInfo{kvsetkeys.drv}
%
% \title{The \xpackage{kvsetkeys} package}
% \date{2019/12/15 v1.18}
% \author{Heiko Oberdiek\thanks
% {Please report any issues at \url{https://github.com/ho-tex/kvsetkeys/issues}}}
%
% \maketitle
%
% \begin{abstract}
% Package \xpackage{kvsetkeys} provides \cs{kvsetkeys}, a variant
% of package \xpackage{keyval}'s \cs{setkeys}. It allows to specify
% a handler that deals with unknown options. Active commas and equal
% signs may be used (e.g. see \xpackage{babel}'s shorthands) and
% only one level of curly braces is removed from the values.
% \end{abstract}
%
% \tableofcontents
%
% \def\M#1{\texttt{\{}\meta{#1}\texttt{\}}}
%
% \section{Documentation}
%
% First I want to recommend the very good review article
% ``A guide to key-value methods'' by Joseph Wright \cite{tb94wright}.
% It introduces the different key-value packages and compares them.
%
% \subsection{Motivation}
%
% \cs{kvsetkeys} serves as replacement for \xpackage{keyval}'s
% \cs{setkeys}. It basically uses the same syntax. But the
% implementation is more robust and predictable:
% \begin{description}
% \item[Active syntax characters:]
%   Comma `|,|' and the equals sign `|=|' are used inside
%   key value lists as syntax characters. Package \xpackage{keyval}
%   uses the catcode of the characters that is active during
%   package loading, usually this is catcode 12 (other).
%   But it can happen that the catcode setting of the syntax characters
%   changes. Especially active characters are of interest, because
%   some language adaptations uses them. For example, option \xoption{turkish}
%   of package \xpackage{babel} uses the equals sign as active shorthand
%   character. Therefore package \xpackage{kvsetkeys} deals with
%   both catcode settings 12 (other) and 13 (active).
% \item[Brace removal:]
%   Package \xpackage{keyval}'s \cs{setkeys} removes up to two
%   levels of curly braces around the value in some unpredictable way:
%\begin{quote}
%|\setkeys{fam}{key={{value}}} || || | $\rightarrow$ |value|\\
%|\setkeys{fam}{key={{{value}}}} | $\rightarrow$ |{value}|\\
%|\setkeys{fam}{key= {{{value}}}}| $\rightarrow$ |{{value}}|
%\end{quote}
%   This package \xpackage{kvsetkeys} follows a much stronger rule:
%   Exactly one level of braces are removed from an item, if the
%   item is surrounded by curly braces. An item can be a
%   the key value pair, the key or the value.
%\begin{quote}
%|\kvsetkeys{fam}{key={value}} || | $\rightarrow$ |value|\\
%|\kvsetkeys{fam}{key={{value}} | $\rightarrow$ |{value}|\\
%|\kvsetkeys{fam}{key= {{value}}| $\rightarrow$ |{value}|
%\end{quote}
% \item[Arbitrary values:] Unmatched conditionals are supported.
% \end{description}
%
% Before I describe
% \cs{kvsetkeys} in more detail, first I want to explain, how
% this package deals with key value lists. For the package also
% provides low level interfaces that can be used by package authors.
%
% \subsection{Normalizing key value lists}
%
% \begin{declcs}{kv@normalize}\,\M{key value list}
% \end{declcs}
% If the user specifies key value lists, he usually prefers
% nice formatted source code, e.g.:
% \begin{quote}
%\begin{verbatim}
%\hypersetup{
%  pdftitle    = {...},
%  pdfsubject  = {...},
%  pdfauthor   = {...},
%  pdfkeywords = {...},
%  ...
%}
%\end{verbatim}
% \end{quote}
% Thus there can be spaces around keys, around |=| or around the value.
% Also empty entries are possible by too many commas. Therefore these
% spaces and empty entries are silently removed by package \xpackage{keyval}
% and this package. Whereas the contents of the value can be protected
% by curly braces, especially if spaces or commas are used inside,
% a key name must not use spaces or other syntax characters.
%
% \cs{kv@normalize} takes a key value list and performs the cleanup:
% \begin{itemize}
% \item Spaces are removed.
% \item Syntax characters (comma and equal sign) that are active
%   are replaced by the same characters with standard catcode.
%   (Example: \xpackage{babel}'s language option \xoption{turkish}
%   uses the equal sign as active shorthand character.)
% \end{itemize}
% The result is stored in \cs{kv@list}, e.g.:
% \begin{quote}
% |\kv@list| $\rightarrow$ |,pdftitle={...},pdfsubject={...},...,|
% \end{quote}
% Curly braces around values (or keys) remain untouched.
% \begin{description}
% \item[v1.3+:]
% One comma is added in front of the list and each pair ends with
% a comma. Thus an empty list consists of one comma, otherwise
% two commas encloses the list. Empty entries other than the first
% are removed.
% \item[v1.0 -- v1.2:]
% Empty entries are removed later. In fact it adds a comma at the begin
% and end to protect the last value and an easier implementation.
% \end{description}
%
% \subsection{Parsing key value lists}
%
% \begin{declcs}{kv@parse}\,\M{key value list}\,\M{processor}
% \end{declcs}
% It is easier to parse a normalized list, thus \cs{kv@parse}
% normalizes the list and calls \cs{kv@parse@normalized}.
%
% \begin{declcs}{kv@parse@normalized}\,\M{key value list}%
%   \,\M{processor}
% \end{declcs}
% Now the key value list is split into single key value pairs.
% For further processing the key and value are given as arguments
% for the \meta{processor}:
% \begin{quote}
%   \meta{processor}\,\M{key}\,\M{value}
% \end{quote}
% Also key and value are stored in macro names:
% \begin{itemize}
% \item \cs{kv@key} stores the key.
% \item \cs{kv@value} stores the value or if the value was not
%   specified it has the meaning \cs{relax}.
% \end{itemize}
% The behaviour in pseudo code:
% \begin{quote}
%   foreach (\meta{key}, \meta{value}) in (\meta{key value list})\\
%   \hspace*{1.5em}\cs{kv@key} := \meta{key}\\
%   \hspace*{1.5em}\cs{kv@value} := \meta{value}\\
%   \hspace*{1.5em}\meta{processor}\,\M{key}\,\M{value}
% \end{quote}
%
% \begin{declcs}{kv@break}
% \end{declcs}
% Since version 2011/03/03 v1.11 \cs{kv@break} can be
% called inside the \meta{processor}
% of \cs{kv@parse} or \cs{kv@parse@normalized}, then
% the processing is stopped and the following entries discarded.
%
% \subsection{Processing key value pairs}
%
% Key value pairs can be processed in many different ways.
% For example, the processor for \cs{kvsetkeys} works similar
% to \cs{setkeys} of package \xpackage{keyval}. There unknown
% keys raise an error.
%
% Package \xpackage{xkeyval} also knows a star form of \cs{setkeys}
% that stores unknown keys in an internal macro for further processing
% with \cs{setrmkeys} and similar macros. This feature is covered
% by processor \cs{kv@processor@known}.
%
% \subsubsection{Processing similar to \xpackage{keyval}}
%
% \begin{declcs}{kv@processor@default}\,\M{family}\,\M{key}\,\M{value}
% \end{declcs}
% There are many possiblities to process key value pairs.
% \cs{kv@processor@default} is the processor used in \cs{kvsetkeys}.
% It reimplements and extends the behaviour of
% \xpackage{keyval}'s \cs{setkeys}.
% In case of unknown keys \cs{setkeys} raise an error.
% This processer, however, calls a handler instead, if it
% is provided by the family. Both \meta{family} and \meta{key}
% may contain package \xpackage{babel}'s shorthands
% (since 2011/04/07 v1.13).
%
% Since 2011/10/18 v1.15 the family handler can reject the
% successful handling of a key by calling \cs{kv@handled@false}.
%
% Since 2019/12/15 v1.18 \cs{kv@processor@default} also defines
% macro \cs{kv@fam} with meaning \meta{family} for convenience.
%
% \subsubsection{Processing similar to \cs{setkeys*} of package \xpackage{xkeyval}}
%
% \begin{declcs}{kv@processor@known}\,\M{family}\,^^A
%   \M{cmd}\,\M{key}\,\M{value}
% \end{declcs}
% The key value processor \cs{kv@processor@known} behaves similar
% to \cs{kv@processor@default}. If the \meta{key} exists in the
% \meta{family} its code is called, otherwise the family handler
% is tried. If the family handler is not set or cannot handle the
% key, the unknown key value pair is added to the macro \meta{cmd}.
% Since 2011/10/18 v1.15.
%
% The behaviour in pseudo code:
% \begin{quote}
%   if \meta{key} exists\\
%   \hspace*{1.5em}call the keyval code of \meta{key}\\
%   else\\
%   \hspace*{1.5em}if \meta{handler} for \meta{family} exists\\
%   \hspace*{3em}handled = true\\
%   \hspace*{3em}\meta{handler}\,\M{key}\,\M{value}\\
%   \hspace*{3em}if handled\\
%   \hspace*{3em}else\\
%   \hspace*{4.5em}add \texttt{"}\M{key}\texttt{=}\M{value}\texttt{"}
%     to \M{cmd}\\
%   \hspace*{3em}fi\\
%   \hspace*{1.5em}else\\
%   \hspace*{3em}add \texttt{"}\M{key}\texttt{=}\M{value}\texttt{"}
%     to \M{cmd}\\
%   \hspace*{3em}raise unknown key error\\
%   \hspace*{1.5em}fi\\
%   fi
% \end{quote}
%
% Since 2019/12/15 v1.18 \cs{kv@processor@known} also defines
% macro \cs{kv@fam} with meaning \meta{family} for convenience.
%
% \subsection{Default family handler}
%
% \cs{kv@processor@default} calls \meta{handler}, the default
% handler for the family, if the key does not exist in the family.
% The handler is called with two arguments, the key and the value.
% It can be defined with \cs{kv@set@family@hander}:
%
% \begin{declcs}{kv@set@family@handler}\,\M{family}\,\M{handler definition}
% \end{declcs}
% This sets the default family handler for the keyval family
% \meta{family}. Inside \meta{handler definition} |#1| stands for
% the key and |#2| is the value. Also \cs{kv@key} and \cs{kv@value}
% can be used for the key and the value. If the value is not given,
% \cs{kv@value} has the meaning \cs{relax}.
%
% \begin{declcs}{kv@unset@family@handler}\,\M{family}
% \end{declcs}
% It removes the family handler for \meta{family}.
% Since 2011/10/18 v1.15.
%
% \subsection{Put it all together}
%
% \begin{declcs}{kvsetkeys}\,\M{family}\,\M{key value list}
% \end{declcs}
% Macro \cs{kvsetkeys} processes the \meta{key value list} with
% the standard processor \cs{kv@processor@default}:
% \begin{quote}
%   \cs{kv@parse}\,\M{key value list}^^A
%   \texttt{\{}\cs{kv@processor@default}\,\M{family}\texttt{\}}
% \end{quote}
%
% \begin{declcs}{kvsetknownkeys}\,\M{family}\,\M{cmd}\,\M{key value list}
% \end{declcs}
% Macro \cs{kvsetknownkeys} processes the \meta{key value list}
% with processor \cs{kv@processor@known}. All key value pairs
% with keys that are not known in \meta{family} are stored
% in macro \meta{cmd}. A previous contents of macro \meta{cmd}
% will be overwritten. If all keys can be handled, \meta{cmd}
% will be empty, otherwise it contains a key value list of
% unhandled key value pairs.
% Since 2011/10/18 v1.15.
%
% Pseudo code:
% \begin{quote}
%   create macro \meta{cmdaux} with unique name (inside the current group)\\
%   \cs{def}\meta{cmdaux}\{\}\\
%   \cs{kv@parse}\,\M{key value list}^^A
%   \texttt{\{}\cs{kv@processor@known}\,\M{family}\,\M{cmdaux}\texttt{\}}\\
%   \cs{let}\meta{cmd}=\meta{cmdaux}
% \end{quote}
%
% \begin{declcs}{kvsetkeys@expandafter}\,\M{family}\,\M{list cmd}\\
%   \cs{kvsetknownkeys@expandafter}\,\M{family}\,%
%   \M{cmd}\,\M{list cmd}
% \end{declcs}
% Both macros behave like the counterparts without suffix
% |@expandafter|. The difference is that the key value list is
% given as macro that is expanded once.
% Since 2011/10/18 v1.15.
%
% Thus you can replace \cs{setkeys} of package \xpackage{keyval}
% by the key value parser of this package:
% \begin{quote}
%   |\renewcommand*{\setkeys}{\kvsetkeys}|\\
%   or\\
%   |\let\setkeys\kvsetkeys|
% \end{quote}
%
% \subsection{Comma separated lists}
%
% Since version 2007/09/29 v1.3 this package also supports the normalizing
% and parsing of general comma separated lists.
%
% \begin{declcs}{comma@normalize}\,\M{comma list}
% \end{declcs}
% Macro \cs{comma@normalize} normalizes the comma separated list,
% removes spaces around commas. The result is put in macro \cs{comma@list}.
%
% \begin{declcs}{comma@parse}\,\M{comma list}\,\M{processor}
% \end{declcs}
% Macro \cs{comma@parse} first normalizes the comma separated list
% and then parses the list by calling \cs{comma@parse@normalized}.
%
% \begin{declcs}{comma@parse@normalized}\,\M{normalized comma list}^^A
%   \,\M{processor}
% \end{declcs}
% The list is parsed. Empty entries are ignored. \meta{processor}
% is called for each non-empty entry with the entry as argument:
% \begin{quote}
%   \meta{processor}|{|\meta{entry}|}|
% \end{quote}
% Also the entry is stored in the macro \cs{comma@entry}.
%
% \begin{declcs}{comma@break}
% \end{declcs}
% Since version 2011/03/03 v1.11 \cs{comma@break} can be
% called inside the \meta{processor}
% of \cs{comma@parse} or \cs{comma@parse@normalized}, then
% the processing is stopped and the following entries discarded.
%
% \section{Example}
%
% The following example prints a short piece of HTML code using
% the tabbing environment for indenting purpose and a key value
% syntax for specifying the attributes of an HTML tag.
% The example illustrates the use of a default family handler.
%    \begin{macrocode}
%<*example>
\documentclass{article}
\usepackage[T1]{fontenc}
\usepackage{kvsetkeys}
\usepackage{keyval}

\makeatletter
\newcommand*{\tag}[2][]{%
  % #1: attributes
  % #2: tag name
  \begingroup
    \toks@={}%
    \let\@endslash\@empty
    \kvsetkeys{tag}{#1}%
    \texttt{%
      \textless #2\the\toks@\@endslash\textgreater
    }%
  \endgroup
}
\kv@set@family@handler{tag}{%
  % #1: key
  % #2: value
  \toks@\expandafter{%
    \the\toks@
    \space
    #1=\string"#2\string"%
  }%
}
\define@key{tag}{/}[]{%
  \def\@endslash{/}%
}
\makeatother

\begin{document}
\begin{tabbing}
  \mbox{}\qquad\=\qquad\=\kill
  \tag{html}\\
  \>\dots\\
  \>\tag[border=1]{table}\\
  \>\>\tag[width=200, span=3, /]{colgroup}\\
  \>\>\dots\\
  \>\tag{/table}\\
  \>\dots\\
  \tag{/html}\\
\end{tabbing}
\end{document}
%</example>
%    \end{macrocode}
%
% \StopEventually{
% }
%
% \section{Implementation}
%
% \subsection{Identification}
%
%    \begin{macrocode}
%<*package>
%    \end{macrocode}
%    Reload check, especially if the package is not used with \LaTeX.
%    \begin{macrocode}
\begingroup\catcode61\catcode48\catcode32=10\relax%
  \catcode13=5 % ^^M
  \endlinechar=13 %
  \catcode35=6 % #
  \catcode39=12 % '
  \catcode44=12 % ,
  \catcode45=12 % -
  \catcode46=12 % .
  \catcode58=12 % :
  \catcode64=11 % @
  \catcode123=1 % {
  \catcode125=2 % }
  \expandafter\let\expandafter\x\csname ver@kvsetkeys.sty\endcsname
  \ifx\x\relax % plain-TeX, first loading
  \else
    \def\empty{}%
    \ifx\x\empty % LaTeX, first loading,
      % variable is initialized, but \ProvidesPackage not yet seen
    \else
      \expandafter\ifx\csname PackageInfo\endcsname\relax
        \def\x#1#2{%
          \immediate\write-1{Package #1 Info: #2.}%
        }%
      \else
        \def\x#1#2{\PackageInfo{#1}{#2, stopped}}%
      \fi
      \x{kvsetkeys}{The package is already loaded}%
      \aftergroup\endinput
    \fi
  \fi
\endgroup%
%    \end{macrocode}
%    Package identification:
%    \begin{macrocode}
\begingroup\catcode61\catcode48\catcode32=10\relax%
  \catcode13=5 % ^^M
  \endlinechar=13 %
  \catcode35=6 % #
  \catcode39=12 % '
  \catcode40=12 % (
  \catcode41=12 % )
  \catcode44=12 % ,
  \catcode45=12 % -
  \catcode46=12 % .
  \catcode47=12 % /
  \catcode58=12 % :
  \catcode64=11 % @
  \catcode91=12 % [
  \catcode93=12 % ]
  \catcode123=1 % {
  \catcode125=2 % }
  \expandafter\ifx\csname ProvidesPackage\endcsname\relax
    \def\x#1#2#3[#4]{\endgroup
      \immediate\write-1{Package: #3 #4}%
      \xdef#1{#4}%
    }%
  \else
    \def\x#1#2[#3]{\endgroup
      #2[{#3}]%
      \ifx#1\@undefined
        \xdef#1{#3}%
      \fi
      \ifx#1\relax
        \xdef#1{#3}%
      \fi
    }%
  \fi
\expandafter\x\csname ver@kvsetkeys.sty\endcsname
\ProvidesPackage{kvsetkeys}%
  [2019/12/15 v1.18 Key value parser (HO)]%
%    \end{macrocode}
%
%    \begin{macrocode}
\begingroup\catcode61\catcode48\catcode32=10\relax%
  \catcode13=5 % ^^M
  \endlinechar=13 %
  \catcode123=1 % {
  \catcode125=2 % }
  \catcode64=11 % @
  \def\x{\endgroup
    \expandafter\edef\csname KVS@AtEnd\endcsname{%
      \endlinechar=\the\endlinechar\relax
      \catcode13=\the\catcode13\relax
      \catcode32=\the\catcode32\relax
      \catcode35=\the\catcode35\relax
      \catcode61=\the\catcode61\relax
      \catcode64=\the\catcode64\relax
      \catcode123=\the\catcode123\relax
      \catcode125=\the\catcode125\relax
    }%
  }%
\x\catcode61\catcode48\catcode32=10\relax%
\catcode13=5 % ^^M
\endlinechar=13 %
\catcode35=6 % #
\catcode64=11 % @
\catcode123=1 % {
\catcode125=2 % }
\def\TMP@EnsureCode#1#2{%
  \edef\KVS@AtEnd{%
    \KVS@AtEnd
    \catcode#1=\the\catcode#1\relax
  }%
  \catcode#1=#2\relax
}
\TMP@EnsureCode{36}{3}% $
\TMP@EnsureCode{38}{4}% &
\TMP@EnsureCode{39}{12}% '
\TMP@EnsureCode{43}{12}% +
\TMP@EnsureCode{44}{12}% ,
\TMP@EnsureCode{45}{12}% -
\TMP@EnsureCode{46}{12}% .
\TMP@EnsureCode{47}{12}% /
\TMP@EnsureCode{91}{12}% [
\TMP@EnsureCode{93}{12}% ]
\TMP@EnsureCode{94}{7}% ^ (superscript)
\TMP@EnsureCode{96}{12}% `
\TMP@EnsureCode{126}{13}% ~ (active)
\edef\KVS@AtEnd{\KVS@AtEnd\noexpand\endinput}
%    \end{macrocode}
%
% \subsection{Package loading}
%
%    \begin{macrocode}
\begingroup\expandafter\expandafter\expandafter\endgroup
\expandafter\ifx\csname RequirePackage\endcsname\relax
  \def\TMP@RequirePackage#1[#2]{%
    \begingroup\expandafter\expandafter\expandafter\endgroup
    \expandafter\ifx\csname ver@#1.sty\endcsname\relax
      \input #1.sty\relax
    \fi
  }%
  \TMP@RequirePackage{infwarerr}[2007/09/09]%
  \let\PackageError\@PackageError
\else
\fi
%    \end{macrocode}
%
%    \begin{macrocode}
\expandafter\ifx\csname toks@\endcsname\relax
  \toksdef\toks@=0 %
\fi
%    \end{macrocode}
%
% \subsection{Check for \eTeX}
%
%    \cs{unexpanded}, \cs{ifcsname}, and \cs{unless} are used if found.
%    \begin{macrocode}
\ifx\numexpr\@undefined
  \catcode`\$=9 % ignore
  \catcode`\&=14 % comment
\else % e-TeX
  \catcode`\$=14 % comment
  \catcode`\&=9  % ignore
\fi
%    \end{macrocode}
%
% \subsection{Generic help macros}
%
%    \begin{macro}{\KVS@Empty}
%    \begin{macrocode}
\def\KVS@Empty{}
%    \end{macrocode}
%    \end{macro}
%    \begin{macro}{\KVS@FirstOfTwo}
%    \begin{macrocode}
\long\def\KVS@FirstOfTwo#1#2{#1}
%    \end{macrocode}
%    \end{macro}
%    \begin{macro}{\KVS@SecondOfTwo}
%    \begin{macrocode}
\long\def\KVS@SecondOfTwo#1#2{#2}
%    \end{macrocode}
%    \end{macro}
%
%    \begin{macro}{\KVS@IfEmpty}
%    \begin{macrocode}
\long\def\KVS@IfEmpty#1{%
& \edef\KVS@Temp{\unexpanded{#1}}%
$ \begingroup
$   \toks@{#1}%
$   \edef\KVS@Temp{\the\toks@}%
$ \expandafter\endgroup
  \ifx\KVS@Temp\KVS@Empty
    \expandafter\KVS@FirstOfTwo
  \else
    \expandafter\KVS@SecondOfTwo
  \fi
}
%    \end{macrocode}
%    \end{macro}
%
% \subsection{Normalizing}
%
%    \begin{macro}{\kv@normalize}
%    \begin{macrocode}
\long\def\kv@normalize#1{%
  \begingroup
    \toks@{,#1,}%
    \KVS@Comma
    \KVS@SpaceComma
    \KVS@CommaSpace
    \KVS@CommaComma
    \KVS@Equals
    \KVS@SpaceEquals
    \KVS@EqualsSpace
    \xdef\KVS@Global{\the\toks@}%
  \endgroup
  \let\kv@list\KVS@Global
}
%    \end{macrocode}
%    \end{macro}
%    \begin{macro}{\comma@normalize}
%    \begin{macrocode}
\def\comma@normalize#1{%
  \begingroup
    \toks@{,#1,}%
    \KVS@Comma
    \KVS@SpaceComma
    \KVS@CommaSpace
    \KVS@CommaComma
    \xdef\KVS@Global{\the\toks@}%
  \endgroup
  \let\comma@list\KVS@Global
}
%    \end{macrocode}
%    \end{macro}
%    \begin{macro}{\KVS@Comma}
%    Converts active commas into comma with catcode other.
%    Also adds a comma at the end to protect the last value
%    for next cleanup steps.
%    \begin{macrocode}
\begingroup
  \lccode`\,=`\,%
  \lccode`\~=`\,%
\lowercase{\endgroup
  \def\KVS@Comma{%
    \toks@\expandafter{\expandafter}\expandafter
    \KVS@@Comma\the\toks@~\KVS@Nil
  }%
  \long\def\KVS@@Comma#1~#2\KVS@Nil{%
    \toks@\expandafter{\the\toks@#1}%
    \KVS@IfEmpty{#2}{%
    }{%
      \KVS@@Comma,#2\KVS@Nil
    }%
  }%
}
%    \end{macrocode}
%    \end{macro}
%    \begin{macro}{\KVS@SpaceComma}
%    Removes spaces before the comma, may add commas at the end.
%    \begin{macrocode}
\def\KVS@SpaceComma#1{%
  \def\KVS@SpaceComma{%
    \expandafter\KVS@@SpaceComma\the\toks@#1,\KVS@Nil
  }%
}
\KVS@SpaceComma{ }
%    \end{macrocode}
%    \end{macro}
%    \begin{macro}{\KVS@@SpaceComma}
%    \begin{macrocode}
\long\def\KVS@@SpaceComma#1 ,#2\KVS@Nil{%
  \KVS@IfEmpty{#2}{%
    \toks@{#1}%
  }{%
    \KVS@@SpaceComma#1,#2\KVS@Nil
  }%
}
%    \end{macrocode}
%    \end{macro}
%    \begin{macro}{\KVS@CommaSpace}
%    Removes spaces after the comma, may add commas at the end.
%    \begin{macrocode}
\def\KVS@CommaSpace{%
  \expandafter\KVS@@CommaSpace\the\toks@, \KVS@Nil
}
%    \end{macrocode}
%    \end{macro}
%    \begin{macro}{\KVS@@CommaSpace}
%    \begin{macrocode}
\long\def\KVS@@CommaSpace#1, #2\KVS@Nil{%
  \KVS@IfEmpty{#2}{%
    \toks@{#1}%
  }{%
    \KVS@@CommaSpace#1,#2\KVS@Nil
  }%
}
%    \end{macrocode}
%    \end{macro}
%    \begin{macro}{\KVS@CommaComma}
%    Replaces multiple commas by one comma.
%    \begin{macrocode}
\def\KVS@CommaComma{%
  \expandafter\KVS@@CommaComma\the\toks@,\KVS@Nil
}
%    \end{macrocode}
%    \end{macro}
%    \begin{macro}{\KVS@@CommaComma}
%    \begin{macrocode}
\long\def\KVS@@CommaComma#1,,#2\KVS@Nil{%
  \KVS@IfEmpty{#2}{%
    \toks@{#1,}% (!)
  }{%
    \KVS@@CommaComma#1,#2\KVS@Nil
  }%
}
%    \end{macrocode}
%    \end{macro}
%
%    \begin{macro}{\KVS@Equals}
%    Converts active equals signs into catcode other characters.
%    \begin{macrocode}
\begingroup
  \lccode`\==`\=%
  \lccode`\~=`\=%
\lowercase{\endgroup
  \def\KVS@Equals{%
    \toks@\expandafter{\expandafter}\expandafter
    \KVS@@Equals\the\toks@~\KVS@Nil
  }%
  \long\def\KVS@@Equals#1~#2\KVS@Nil{%
    \edef\KVS@Temp{\the\toks@}%
    \ifx\KVS@Temp\KVS@Empty
      \expandafter\KVS@FirstOfTwo
    \else
      \expandafter\KVS@SecondOfTwo
    \fi
    {%
      \toks@{#1}%
    }{%
      \toks@\expandafter{\the\toks@=#1}%
    }%
    \KVS@IfEmpty{#2}{%
    }{%
      \KVS@@Equals#2\KVS@Nil
    }%
  }%
}
%    \end{macrocode}
%    \end{macro}
%    \begin{macro}{\KVS@SpaceEquals}
%    Removes spaces before the equals sign.
%    \begin{macrocode}
\def\KVS@SpaceEquals#1{%
  \def\KVS@SpaceEquals{%
    \expandafter\KVS@@SpaceEquals\the\toks@#1=\KVS@Nil
  }%
}
\KVS@SpaceEquals{ }
%    \end{macrocode}
%    \end{macro}
%    \begin{macro}{\KVS@@SpaceEquals}
%    \begin{macrocode}
\long\def\KVS@@SpaceEquals#1 =#2\KVS@Nil{%
  \KVS@IfEmpty{#2}{%
    \toks@{#1}%
  }{%
    \KVS@@SpaceEquals#1=#2\KVS@Nil
  }%
}
%    \end{macrocode}
%    \end{macro}
%    \begin{macro}{\KVS@EqualsSpace}
%    Removes spaces after the equals sign.
%    \begin{macrocode}
\def\KVS@EqualsSpace{%
  \expandafter\KVS@@EqualsSpace\the\toks@= \KVS@Nil
}
%    \end{macrocode}
%    \end{macro}
%    \begin{macro}{\KVS@@EqualsSpace}
%    \begin{macrocode}
\long\def\KVS@@EqualsSpace#1= #2\KVS@Nil{%
  \KVS@IfEmpty{#2}{%
    \toks@{#1}%
  }{%
    \KVS@@EqualsSpace#1=#2\KVS@Nil
  }%
}
%    \end{macrocode}
%    \end{macro}
%
% \subsection{Parsing key value lists}
%
%    \begin{macro}{\kv@parse}
%    Normalizes and parses the key value list. Also sets \cs{kv@list}.
%    \begin{macrocode}
\long\def\kv@parse#1{%
  \kv@normalize{#1}%
  \expandafter\kv@parse@normalized\expandafter{\kv@list}%
}
%    \end{macrocode}
%    \end{macro}
%    \begin{macro}{\kv@parse@normalized}
%    |#1|: key value list\\
%    |#2|: processor
%    \begin{macrocode}
\long\def\kv@parse@normalized#1#2{%
  \KVS@Parse#1,\KVS@Nil{#2}%
}
%    \end{macrocode}
%    \end{macro}
%    \begin{macro}{\KVS@Parse}
%    |#1,#2|: key value list\\
%    |#3|: processor
%    \begin{macrocode}
\long\def\KVS@Parse#1,#2\KVS@Nil#3{%
  \KVS@IfEmpty{#1}{%
  }{%
    \KVS@Process#1=\KVS@Nil{#3}%
  }%
  \KVS@MaybeBreak
  \KVS@IfEmpty{#2}{%
  }{%
    \KVS@Parse#2\KVS@Nil{#3}%
  }%
}
%    \end{macrocode}
%    \end{macro}
%    \begin{macro}{\KVS@Process}
%    |#1|: key\\
%    |#2|: value, |=|\\
%    |#3|: processor
%    \begin{macrocode}
\long\def\KVS@Process#1=#2\KVS@Nil#3{%
  \let\KVS@MaybeBreak\relax
  \def\kv@key{#1}%
  \KVS@IfEmpty{#2}{%
    \let\kv@value\relax
    #3{#1}{}%
  }{%
    \KVS@@Process{#1}#2\KVS@Nil{#3}%
  }%
}
%    \end{macrocode}
%    \end{macro}
%    \begin{macro}{\KVS@@Process}
%    |#1|: key\\
%    |#2|: value\\
%    |#3|: processor
%    \begin{macrocode}
\long\def\KVS@@Process#1#2=\KVS@Nil#3{%
& \edef\kv@value{\unexpanded{#2}}%
$ \begingroup
$   \toks@{#2}%
$   \xdef\KVS@Global{\the\toks@}%
$ \endgroup
$ \let\kv@value\KVS@Global
  #3{#1}{#2}%
}
%    \end{macrocode}
%    \end{macro}
%    \begin{macro}{\KVS@MaybeBreak}
%    \begin{macrocode}
\let\KVS@MaybeBreak\relax
%    \end{macrocode}
%    \end{macro}
%    \begin{macro}{\KVS@break}
%    \begin{macrocode}
\def\KVS@break#1#2#3#4{%
  \let\KVS@MaybeBreak\relax
}
%    \end{macrocode}
%    \end{macro}
%    \begin{macro}{\kv@break}
%    \begin{macrocode}
\def\kv@break{%
  \let\KVS@MaybeBreak\KVS@break
}
%    \end{macrocode}
%    \end{macro}
%
% \subsection{Parsing comma lists}
%
%    \begin{macro}{\comma@parse}
%    Normalizes and parses the key value list. Also sets \cs{comma@list}.
%    \begin{macrocode}
\def\comma@parse#1{%
  \comma@normalize{#1}%
  \expandafter\comma@parse@normalized\expandafter{\comma@list}%
}
%    \end{macrocode}
%    \end{macro}
%
%    \begin{macro}{\comma@parse@normalized}
%    |#1|: comma list\\
%    |#2|: processor
%    \begin{macrocode}
\def\comma@parse@normalized#1#2{%
  \KVS@CommaParse#1,\KVS@Nil{#2}%
}
%    \end{macrocode}
%    \end{macro}
%    \begin{macro}{\KVS@CommaParse}
%    |#1,#2|: comma list\\
%    |#3|: processor
%    \begin{macrocode}
\def\KVS@CommaParse#1,#2\KVS@Nil#3{%
  \KVS@IfEmpty{#1}{%
  }{%
    \def\comma@entry{#1}%
    #3{#1}%
  }%
  \KVS@MaybeBreak
  \KVS@IfEmpty{#2}{%
  }{%
    \KVS@CommaParse#2\KVS@Nil{#3}%
  }%
}
%    \end{macrocode}
%    \end{macro}
%    \begin{macro}{\comma@break}
%    \begin{macrocode}
\def\comma@break{%
  \let\KVS@MaybeBreak\KVS@break
}
%    \end{macrocode}
%    \end{macro}
%
% \subsection{Processing key value pairs}
%
%    \begin{macro}{\kv@handled@false}
%    The handler can call \cs{kv@handled@false} or
%    \cs{kv@handled@true} so report failure or success.
%    The default is success (compatibility for versions
%    before 2011/10/18 v1.15).
%    \begin{macrocode}
\def\kv@handled@false{%
  \let\ifkv@handled@\iffalse
}
%    \end{macrocode}
%    \end{macro}
%    \begin{macro}{\kv@handled@true}
%    \begin{macrocode}
\def\kv@handled@true{%
  \let\ifkv@handled@\iftrue
}
%    \end{macrocode}
%    \end{macro}
%    \begin{macro}{\ifkv@handled@}
%    \begin{macrocode}
\kv@handled@true
%    \end{macrocode}
%    \end{macro}
%
%    \begin{macro}{\kv@processor@default}
%    \begin{macrocode}
\def\kv@processor@default#1#2{%
  \begingroup
    \csname @safe@activestrue\endcsname
    \let\ifincsname\iftrue
    \edef\KVS@temp{\endgroup
      \noexpand\KVS@ProcessorDefault{#1}{#2}%
    }%
  \KVS@temp
}
%    \end{macrocode}
%    \end{macro}
%    \begin{macro}{\KVS@ProcessorDefault}
%    \begin{macrocode}
\long\def\KVS@ProcessorDefault#1#2#3{%
  \def\kv@fam{#1}%
& \unless\ifcsname KV@#1@#2\endcsname
$ \begingroup\expandafter\expandafter\expandafter\endgroup
$ \expandafter\ifx\csname KV@#1@#2\endcsname\relax
&   \unless\ifcsname KVS@#1@handler\endcsname
$   \begingroup\expandafter\expandafter\expandafter\endgroup
$   \expandafter\ifx\csname KVS@#1@handler\endcsname\relax
      \kv@error@unknownkey{#1}{#2}%
    \else
      \kv@handled@true
      \csname KVS@#1@handler\endcsname{#2}{#3}%
      \relax
      \ifkv@handled@
      \else
        \kv@error@unknownkey{#1}{#2}%
      \fi
    \fi
  \else
    \ifx\kv@value\relax
&     \unless\ifcsname KV@#1@#2@default\endcsname
$     \begingroup\expandafter\expandafter\expandafter\endgroup
$     \expandafter\ifx\csname KV@#1@#2@default\endcsname\relax
        \kv@error@novalue{#1}{#2}%
      \else
        \csname KV@#1@#2@default\endcsname
        \relax
      \fi
    \else
      \csname KV@#1@#2\endcsname{#3}%
    \fi
  \fi
}
%    \end{macrocode}
%    \end{macro}
%
%    \begin{macro}{\kv@processor@known}
%    \begin{macrocode}
\def\kv@processor@known#1#2#3{%
  \begingroup
    \csname @safe@activestrue\endcsname
    \let\ifincsname\iftrue
    \edef\KVS@temp{\endgroup
      \noexpand\KVS@ProcessorKnown{#1}\noexpand#2{#3}%
    }%
  \KVS@temp
}
%    \end{macrocode}
%    \end{macro}
%    \begin{macro}{\KVS@ProcessorKnown}
%    \begin{macrocode}
\long\def\KVS@ProcessorKnown#1#2#3#4{%
  \def\kv@fam{#1}%
& \unless\ifcsname KV@#1@#3\endcsname
$ \begingroup\expandafter\expandafter\expandafter\endgroup
$ \expandafter\ifx\csname KV@#1@#3\endcsname\relax
&   \unless\ifcsname KVS@#1@handler\endcsname
$   \begingroup\expandafter\expandafter\expandafter\endgroup
$   \expandafter\ifx\csname KVS@#1@handler\endcsname\relax
      \KVS@AddUnhandled#2{#3}{#4}%
    \else
      \kv@handled@true
      \csname KVS@#1@handler\endcsname{#3}{#4}%
      \relax
      \ifkv@handled@
      \else
        \KVS@AddUnhandled#2{#3}{#4}%
      \fi
    \fi
  \else
    \ifx\kv@value\relax
&     \unless\ifcsname KV@#1@#2@default\endcsname
$     \begingroup\expandafter\expandafter\expandafter\endgroup
$     \expandafter\ifx\csname KV@#1@#3@default\endcsname\relax
        \kv@error@novalue{#1}{#3}%
      \else
        \csname KV@#1@#3@default\endcsname
        \relax
      \fi
    \else
      \csname KV@#1@#3\endcsname{#4}%
    \fi
  \fi
}
%    \end{macrocode}
%    \end{macro}
%    \begin{macro}{\KVS@AddUnhandled}
%    \begin{macrocode}
\long\def\KVS@AddUnhandled#1#2#3{%
& \edef#1{%
&   \ifx#1\KVS@empty
&   \else
&     \unexpanded{#1},%
&   \fi
&   \unexpanded{{#2}={#3}}%
& }%
$ \begingroup
$   \ifx#1\KVS@empty
$     \toks@{{#2}={#3}}%
$   \else
$     \toks@\expandafter{#1,{#2}={#3}}%
$   \fi
$   \xdef\KVS@Global{\the\toks@}%
$ \endgroup
$ \let#1\KVS@Global
}
%    \end{macrocode}
%    \end{macro}
%
%    \begin{macro}{\kv@set@family@handler}
%    \begin{macrocode}
\long\def\kv@set@family@handler#1#2{%
  \begingroup
    \csname @safe@activestrue\endcsname
    \let\ifincsname\iftrue
  \expandafter\endgroup
  \expandafter\def\csname KVS@#1@handler\endcsname##1##2{#2}%
}
%    \end{macrocode}
%    \end{macro}
%
%    \begin{macro}{\kv@unset@family@handler}
%    \begin{macrocode}
\long\def\kv@unset@family@handler#1#2{%
  \begingroup
    \csname @safe@activestrue\endcsname
    \let\ifincsname\iftrue
  \expandafter\endgroup
  \expandafter\let\csname KVS@#1@handler\endcsname\@UnDeFiNeD
}
%    \end{macrocode}
%    \end{macro}
%
% \subsection{Error handling}
%
%    \begin{macro}{\kv@error@novalue}
%    \begin{macrocode}
\def\kv@error@novalue{%
  \kv@error@generic{No value specified for}%
}
%    \end{macrocode}
%    \end{macro}
%    \begin{macro}{\kv@error@unknownkey}
%    \begin{macrocode}
\def\kv@error@unknownkey{%
  \kv@error@generic{Undefined}%
}
%    \end{macrocode}
%    \end{macro}
%    \begin{macro}{\kv@error@generic}
%    \begin{macrocode}
\def\kv@error@generic#1#2#3{%
  \PackageError{kvsetkeys}{%
    #1 key `#3'%
  }{%
    The keyval family of the key `#3' is `#2'.\MessageBreak
    The setting of the key is ignored because of the error.\MessageBreak
    \MessageBreak
    \@ehc
  }%
}
%    \end{macrocode}
%    \end{macro}
%
% \subsection{Do it all}
%
%    \begin{macro}{\kvsetkeys}
%    \begin{macrocode}
\long\def\kvsetkeys#1#2{%
  \kv@parse{#2}{\kv@processor@default{#1}}%
}
%    \end{macrocode}
%    \end{macro}
%
%    \begin{macro}{\kvsetkeys@expandafter}
%    \begin{macrocode}
\def\kvsetkeys@expandafter#1#2{%
  \expandafter\kv@parse\expandafter{#2}{%
    \kv@processor@default{#1}%
  }%
}
%    \end{macrocode}
%    \end{macro}
%
%    \begin{macro}{\KVS@cmd}
%    \begin{macrocode}
\def\KVS@cmd{0}%
%    \end{macrocode}
%    \end{macro}
%    \begin{macro}{\KVS@cmd@inc}
%    \begin{macrocode}
\def\KVS@cmd@inc{%
& \edef\KVS@cmd{\the\numexpr\KVS@cmd+1}%
$ \begingroup
$   \count255=\KVS@cmd\relax
$   \advance\count255 by 1\relax
$ \edef\x{\endgroup
$   \noexpand\def\noexpand\KVS@cmd{\number\count255}%
$ }%
$ \x
}
%    \end{macrocode}
%    \end{macro}
%    \begin{macro}{\KVS@cmd@dec}
%    \begin{macrocode}
\def\KVS@cmd@dec{%
& \edef\KVS@cmd{\the\numexpr\KVS@cmd-1}%
$ \begingroup
$   \count255=\KVS@cmd\relax
$   \advance\count255 by -1\relax
$ \edef\x{\endgroup
$   \noexpand\def\noexpand\KVS@cmd{\number\count255}%
$ }%
$ \x
}
%    \end{macrocode}
%    \end{macro}
%    \begin{macro}{\KVS@empty}
%    \begin{macrocode}
\def\KVS@empty{}
%    \end{macrocode}
%    \end{macro}
%
%    \begin{macro}{\kvsetknownkeys}
%    \begin{macrocode}
\def\kvsetknownkeys{%
  \expandafter
  \KVS@setknownkeys\csname KVS@cmd\KVS@cmd\endcsname{}%
}
%    \end{macrocode}
%    \end{macro}
%    \begin{macro}{\KVS@setknownkeys}
%    \begin{macrocode}
\long\def\KVS@setknownkeys#1#2#3#4#5{%
  \let#1\KVS@empty
  \KVS@cmd@inc
  #2\kv@parse#2{#5}{\kv@processor@known{#3}#1}%
  \KVS@cmd@dec
  \let#4=#1%
}
%    \end{macrocode}
%    \end{macro}
%
%    \begin{macro}{\kvsetknownkeys@expandafter}
%    \begin{macrocode}
\def\kvsetknownkeys@expandafter{%
  \expandafter
  \KVS@setknownkeys
      \csname KVS@cmd\KVS@cmd\endcsname\expandafter
}
%    \end{macrocode}
%    \end{macro}
%
%    \begin{macrocode}
\KVS@AtEnd%
%</package>
%    \end{macrocode}
%% \section{Installation}
%
% \subsection{Download}
%
% \paragraph{Package.} This package is available on
% CTAN\footnote{\CTANpkg{kvsetkeys}}:
% \begin{description}
% \item[\CTAN{macros/latex/contrib/kvsetkeys/kvsetkeys.dtx}] The source file.
% \item[\CTAN{macros/latex/contrib/kvsetkeys/kvsetkeys.pdf}] Documentation.
% \end{description}
%
%
% \paragraph{Bundle.} All the packages of the bundle `kvsetkeys'
% are also available in a TDS compliant ZIP archive. There
% the packages are already unpacked and the documentation files
% are generated. The files and directories obey the TDS standard.
% \begin{description}
% \item[\CTANinstall{install/macros/latex/contrib/kvsetkeys.tds.zip}]
% \end{description}
% \emph{TDS} refers to the standard ``A Directory Structure
% for \TeX\ Files'' (\CTANpkg{tds}). Directories
% with \xfile{texmf} in their name are usually organized this way.
%
% \subsection{Bundle installation}
%
% \paragraph{Unpacking.} Unpack the \xfile{kvsetkeys.tds.zip} in the
% TDS tree (also known as \xfile{texmf} tree) of your choice.
% Example (linux):
% \begin{quote}
%   |unzip kvsetkeys.tds.zip -d ~/texmf|
% \end{quote}
%
% \subsection{Package installation}
%
% \paragraph{Unpacking.} The \xfile{.dtx} file is a self-extracting
% \docstrip\ archive. The files are extracted by running the
% \xfile{.dtx} through \plainTeX:
% \begin{quote}
%   \verb|tex kvsetkeys.dtx|
% \end{quote}
%
% \paragraph{TDS.} Now the different files must be moved into
% the different directories in your installation TDS tree
% (also known as \xfile{texmf} tree):
% \begin{quote}
% \def\t{^^A
% \begin{tabular}{@{}>{\ttfamily}l@{ $\rightarrow$ }>{\ttfamily}l@{}}
%   kvsetkeys.sty & tex/generic/kvsetkeys/kvsetkeys.sty\\
%   kvsetkeys.pdf & doc/latex/kvsetkeys/kvsetkeys.pdf\\
%   kvsetkeys-example.tex & doc/latex/kvsetkeys/kvsetkeys-example.tex\\
%   kvsetkeys.dtx & source/latex/kvsetkeys/kvsetkeys.dtx\\
% \end{tabular}^^A
% }^^A
% \sbox0{\t}^^A
% \ifdim\wd0>\linewidth
%   \begingroup
%     \advance\linewidth by\leftmargin
%     \advance\linewidth by\rightmargin
%   \edef\x{\endgroup
%     \def\noexpand\lw{\the\linewidth}^^A
%   }\x
%   \def\lwbox{^^A
%     \leavevmode
%     \hbox to \linewidth{^^A
%       \kern-\leftmargin\relax
%       \hss
%       \usebox0
%       \hss
%       \kern-\rightmargin\relax
%     }^^A
%   }^^A
%   \ifdim\wd0>\lw
%     \sbox0{\small\t}^^A
%     \ifdim\wd0>\linewidth
%       \ifdim\wd0>\lw
%         \sbox0{\footnotesize\t}^^A
%         \ifdim\wd0>\linewidth
%           \ifdim\wd0>\lw
%             \sbox0{\scriptsize\t}^^A
%             \ifdim\wd0>\linewidth
%               \ifdim\wd0>\lw
%                 \sbox0{\tiny\t}^^A
%                 \ifdim\wd0>\linewidth
%                   \lwbox
%                 \else
%                   \usebox0
%                 \fi
%               \else
%                 \lwbox
%               \fi
%             \else
%               \usebox0
%             \fi
%           \else
%             \lwbox
%           \fi
%         \else
%           \usebox0
%         \fi
%       \else
%         \lwbox
%       \fi
%     \else
%       \usebox0
%     \fi
%   \else
%     \lwbox
%   \fi
% \else
%   \usebox0
% \fi
% \end{quote}
% If you have a \xfile{docstrip.cfg} that configures and enables \docstrip's
% TDS installing feature, then some files can already be in the right
% place, see the documentation of \docstrip.
%
% \subsection{Refresh file name databases}
%
% If your \TeX~distribution
% (\TeX\,Live, \mikTeX, \dots) relies on file name databases, you must refresh
% these. For example, \TeX\,Live\ users run \verb|texhash| or
% \verb|mktexlsr|.
%
% \subsection{Some details for the interested}
%
% \paragraph{Unpacking with \LaTeX.}
% The \xfile{.dtx} chooses its action depending on the format:
% \begin{description}
% \item[\plainTeX:] Run \docstrip\ and extract the files.
% \item[\LaTeX:] Generate the documentation.
% \end{description}
% If you insist on using \LaTeX\ for \docstrip\ (really,
% \docstrip\ does not need \LaTeX), then inform the autodetect routine
% about your intention:
% \begin{quote}
%   \verb|latex \let\install=y% \iffalse meta-comment
%
% File: kvsetkeys.dtx
% Version: 2019/12/15 v1.18
% Info: Key value parser
%
% Copyright (C)
%    2006, 2007, 2009-2012 Heiko Oberdiek
%    2016-2019 Oberdiek Package Support Group
%    https://github.com/ho-tex/kvsetkeys/issues
%
% This work may be distributed and/or modified under the
% conditions of the LaTeX Project Public License, either
% version 1.3c of this license or (at your option) any later
% version. This version of this license is in
%    https://www.latex-project.org/lppl/lppl-1-3c.txt
% and the latest version of this license is in
%    https://www.latex-project.org/lppl.txt
% and version 1.3 or later is part of all distributions of
% LaTeX version 2005/12/01 or later.
%
% This work has the LPPL maintenance status "maintained".
%
% The Current Maintainers of this work are
% Heiko Oberdiek and the Oberdiek Package Support Group
% https://github.com/ho-tex/kvsetkeys/issues
%
% The Base Interpreter refers to any `TeX-Format',
% because some files are installed in TDS:tex/generic//.
%
% This work consists of the main source file kvsetkeys.dtx
% and the derived files
%    kvsetkeys.sty, kvsetkeys.pdf, kvsetkeys.ins, kvsetkeys.drv,
%    kvsetkeys-example.tex, kvsetkeys-test1.tex,
%    kvsetkeys-test2.tex, kvsetkeys-test3.tex,
%    kvsetkeys-test4.tex.
%
% Distribution:
%    CTAN:macros/latex/contrib/kvsetkeys/kvsetkeys.dtx
%    CTAN:macros/latex/contrib/kvsetkeys/kvsetkeys.pdf
%
% Unpacking:
%    (a) If kvsetkeys.ins is present:
%           tex kvsetkeys.ins
%    (b) Without kvsetkeys.ins:
%           tex kvsetkeys.dtx
%    (c) If you insist on using LaTeX
%           latex \let\install=y\input{kvsetkeys.dtx}
%        (quote the arguments according to the demands of your shell)
%
% Documentation:
%    (a) If kvsetkeys.drv is present:
%           latex kvsetkeys.drv
%    (b) Without kvsetkeys.drv:
%           latex kvsetkeys.dtx; ...
%    The class ltxdoc loads the configuration file ltxdoc.cfg
%    if available. Here you can specify further options, e.g.
%    use A4 as paper format:
%       \PassOptionsToClass{a4paper}{article}
%
%    Programm calls to get the documentation (example):
%       pdflatex kvsetkeys.dtx
%       makeindex -s gind.ist kvsetkeys.idx
%       pdflatex kvsetkeys.dtx
%       makeindex -s gind.ist kvsetkeys.idx
%       pdflatex kvsetkeys.dtx
%
% Installation:
%    TDS:tex/generic/kvsetkeys/kvsetkeys.sty
%    TDS:doc/latex/kvsetkeys/kvsetkeys.pdf
%    TDS:doc/latex/kvsetkeys/kvsetkeys-example.tex
%    TDS:source/latex/kvsetkeys/kvsetkeys.dtx
%
%<*ignore>
\begingroup
  \catcode123=1 %
  \catcode125=2 %
  \def\x{LaTeX2e}%
\expandafter\endgroup
\ifcase 0\ifx\install y1\fi\expandafter
         \ifx\csname processbatchFile\endcsname\relax\else1\fi
         \ifx\fmtname\x\else 1\fi\relax
\else\csname fi\endcsname
%</ignore>
%<*install>
\input docstrip.tex
\Msg{************************************************************************}
\Msg{* Installation}
\Msg{* Package: kvsetkeys 2019/12/15 v1.18 Key value parser (HO)}
\Msg{************************************************************************}

\keepsilent
\askforoverwritefalse

\let\MetaPrefix\relax
\preamble

This is a generated file.

Project: kvsetkeys
Version: 2019/12/15 v1.18

Copyright (C)
   2006, 2007, 2009-2012 Heiko Oberdiek
   2016-2019 Oberdiek Package Support Group

This work may be distributed and/or modified under the
conditions of the LaTeX Project Public License, either
version 1.3c of this license or (at your option) any later
version. This version of this license is in
   https://www.latex-project.org/lppl/lppl-1-3c.txt
and the latest version of this license is in
   https://www.latex-project.org/lppl.txt
and version 1.3 or later is part of all distributions of
LaTeX version 2005/12/01 or later.

This work has the LPPL maintenance status "maintained".

The Current Maintainers of this work are
Heiko Oberdiek and the Oberdiek Package Support Group
https://github.com/ho-tex/kvsetkeys/issues


The Base Interpreter refers to any `TeX-Format',
because some files are installed in TDS:tex/generic//.

This work consists of the main source file kvsetkeys.dtx
and the derived files
   kvsetkeys.sty, kvsetkeys.pdf, kvsetkeys.ins, kvsetkeys.drv,
   kvsetkeys-example.tex, kvsetkeys-test1.tex,
   kvsetkeys-test2.tex, kvsetkeys-test3.tex,
   kvsetkeys-test4.tex.

\endpreamble
\let\MetaPrefix\DoubleperCent

\generate{%
  \file{kvsetkeys.ins}{\from{kvsetkeys.dtx}{install}}%
  \file{kvsetkeys.drv}{\from{kvsetkeys.dtx}{driver}}%
  \usedir{tex/generic/kvsetkeys}%
  \file{kvsetkeys.sty}{\from{kvsetkeys.dtx}{package}}%
  \usedir{doc/latex/kvsetkeys}%
  \file{kvsetkeys-example.tex}{\from{kvsetkeys.dtx}{example}}%
}

\catcode32=13\relax% active space
\let =\space%
\Msg{************************************************************************}
\Msg{*}
\Msg{* To finish the installation you have to move the following}
\Msg{* file into a directory searched by TeX:}
\Msg{*}
\Msg{*     kvsetkeys.sty}
\Msg{*}
\Msg{* To produce the documentation run the file `kvsetkeys.drv'}
\Msg{* through LaTeX.}
\Msg{*}
\Msg{* Happy TeXing!}
\Msg{*}
\Msg{************************************************************************}

\endbatchfile
%</install>
%<*ignore>
\fi
%</ignore>
%<*driver>
\NeedsTeXFormat{LaTeX2e}
\ProvidesFile{kvsetkeys.drv}%
  [2019/12/15 v1.18 Key value parser (HO)]%
\documentclass{ltxdoc}
\usepackage{holtxdoc}[2011/11/22]
\begin{document}
  \DocInput{kvsetkeys.dtx}%
\end{document}
%</driver>
% \fi
%
%
%
% \GetFileInfo{kvsetkeys.drv}
%
% \title{The \xpackage{kvsetkeys} package}
% \date{2019/12/15 v1.18}
% \author{Heiko Oberdiek\thanks
% {Please report any issues at \url{https://github.com/ho-tex/kvsetkeys/issues}}}
%
% \maketitle
%
% \begin{abstract}
% Package \xpackage{kvsetkeys} provides \cs{kvsetkeys}, a variant
% of package \xpackage{keyval}'s \cs{setkeys}. It allows to specify
% a handler that deals with unknown options. Active commas and equal
% signs may be used (e.g. see \xpackage{babel}'s shorthands) and
% only one level of curly braces is removed from the values.
% \end{abstract}
%
% \tableofcontents
%
% \def\M#1{\texttt{\{}\meta{#1}\texttt{\}}}
%
% \section{Documentation}
%
% First I want to recommend the very good review article
% ``A guide to key-value methods'' by Joseph Wright \cite{tb94wright}.
% It introduces the different key-value packages and compares them.
%
% \subsection{Motivation}
%
% \cs{kvsetkeys} serves as replacement for \xpackage{keyval}'s
% \cs{setkeys}. It basically uses the same syntax. But the
% implementation is more robust and predictable:
% \begin{description}
% \item[Active syntax characters:]
%   Comma `|,|' and the equals sign `|=|' are used inside
%   key value lists as syntax characters. Package \xpackage{keyval}
%   uses the catcode of the characters that is active during
%   package loading, usually this is catcode 12 (other).
%   But it can happen that the catcode setting of the syntax characters
%   changes. Especially active characters are of interest, because
%   some language adaptations uses them. For example, option \xoption{turkish}
%   of package \xpackage{babel} uses the equals sign as active shorthand
%   character. Therefore package \xpackage{kvsetkeys} deals with
%   both catcode settings 12 (other) and 13 (active).
% \item[Brace removal:]
%   Package \xpackage{keyval}'s \cs{setkeys} removes up to two
%   levels of curly braces around the value in some unpredictable way:
%\begin{quote}
%|\setkeys{fam}{key={{value}}} || || | $\rightarrow$ |value|\\
%|\setkeys{fam}{key={{{value}}}} | $\rightarrow$ |{value}|\\
%|\setkeys{fam}{key= {{{value}}}}| $\rightarrow$ |{{value}}|
%\end{quote}
%   This package \xpackage{kvsetkeys} follows a much stronger rule:
%   Exactly one level of braces are removed from an item, if the
%   item is surrounded by curly braces. An item can be a
%   the key value pair, the key or the value.
%\begin{quote}
%|\kvsetkeys{fam}{key={value}} || | $\rightarrow$ |value|\\
%|\kvsetkeys{fam}{key={{value}} | $\rightarrow$ |{value}|\\
%|\kvsetkeys{fam}{key= {{value}}| $\rightarrow$ |{value}|
%\end{quote}
% \item[Arbitrary values:] Unmatched conditionals are supported.
% \end{description}
%
% Before I describe
% \cs{kvsetkeys} in more detail, first I want to explain, how
% this package deals with key value lists. For the package also
% provides low level interfaces that can be used by package authors.
%
% \subsection{Normalizing key value lists}
%
% \begin{declcs}{kv@normalize}\,\M{key value list}
% \end{declcs}
% If the user specifies key value lists, he usually prefers
% nice formatted source code, e.g.:
% \begin{quote}
%\begin{verbatim}
%\hypersetup{
%  pdftitle    = {...},
%  pdfsubject  = {...},
%  pdfauthor   = {...},
%  pdfkeywords = {...},
%  ...
%}
%\end{verbatim}
% \end{quote}
% Thus there can be spaces around keys, around |=| or around the value.
% Also empty entries are possible by too many commas. Therefore these
% spaces and empty entries are silently removed by package \xpackage{keyval}
% and this package. Whereas the contents of the value can be protected
% by curly braces, especially if spaces or commas are used inside,
% a key name must not use spaces or other syntax characters.
%
% \cs{kv@normalize} takes a key value list and performs the cleanup:
% \begin{itemize}
% \item Spaces are removed.
% \item Syntax characters (comma and equal sign) that are active
%   are replaced by the same characters with standard catcode.
%   (Example: \xpackage{babel}'s language option \xoption{turkish}
%   uses the equal sign as active shorthand character.)
% \end{itemize}
% The result is stored in \cs{kv@list}, e.g.:
% \begin{quote}
% |\kv@list| $\rightarrow$ |,pdftitle={...},pdfsubject={...},...,|
% \end{quote}
% Curly braces around values (or keys) remain untouched.
% \begin{description}
% \item[v1.3+:]
% One comma is added in front of the list and each pair ends with
% a comma. Thus an empty list consists of one comma, otherwise
% two commas encloses the list. Empty entries other than the first
% are removed.
% \item[v1.0 -- v1.2:]
% Empty entries are removed later. In fact it adds a comma at the begin
% and end to protect the last value and an easier implementation.
% \end{description}
%
% \subsection{Parsing key value lists}
%
% \begin{declcs}{kv@parse}\,\M{key value list}\,\M{processor}
% \end{declcs}
% It is easier to parse a normalized list, thus \cs{kv@parse}
% normalizes the list and calls \cs{kv@parse@normalized}.
%
% \begin{declcs}{kv@parse@normalized}\,\M{key value list}%
%   \,\M{processor}
% \end{declcs}
% Now the key value list is split into single key value pairs.
% For further processing the key and value are given as arguments
% for the \meta{processor}:
% \begin{quote}
%   \meta{processor}\,\M{key}\,\M{value}
% \end{quote}
% Also key and value are stored in macro names:
% \begin{itemize}
% \item \cs{kv@key} stores the key.
% \item \cs{kv@value} stores the value or if the value was not
%   specified it has the meaning \cs{relax}.
% \end{itemize}
% The behaviour in pseudo code:
% \begin{quote}
%   foreach (\meta{key}, \meta{value}) in (\meta{key value list})\\
%   \hspace*{1.5em}\cs{kv@key} := \meta{key}\\
%   \hspace*{1.5em}\cs{kv@value} := \meta{value}\\
%   \hspace*{1.5em}\meta{processor}\,\M{key}\,\M{value}
% \end{quote}
%
% \begin{declcs}{kv@break}
% \end{declcs}
% Since version 2011/03/03 v1.11 \cs{kv@break} can be
% called inside the \meta{processor}
% of \cs{kv@parse} or \cs{kv@parse@normalized}, then
% the processing is stopped and the following entries discarded.
%
% \subsection{Processing key value pairs}
%
% Key value pairs can be processed in many different ways.
% For example, the processor for \cs{kvsetkeys} works similar
% to \cs{setkeys} of package \xpackage{keyval}. There unknown
% keys raise an error.
%
% Package \xpackage{xkeyval} also knows a star form of \cs{setkeys}
% that stores unknown keys in an internal macro for further processing
% with \cs{setrmkeys} and similar macros. This feature is covered
% by processor \cs{kv@processor@known}.
%
% \subsubsection{Processing similar to \xpackage{keyval}}
%
% \begin{declcs}{kv@processor@default}\,\M{family}\,\M{key}\,\M{value}
% \end{declcs}
% There are many possiblities to process key value pairs.
% \cs{kv@processor@default} is the processor used in \cs{kvsetkeys}.
% It reimplements and extends the behaviour of
% \xpackage{keyval}'s \cs{setkeys}.
% In case of unknown keys \cs{setkeys} raise an error.
% This processer, however, calls a handler instead, if it
% is provided by the family. Both \meta{family} and \meta{key}
% may contain package \xpackage{babel}'s shorthands
% (since 2011/04/07 v1.13).
%
% Since 2011/10/18 v1.15 the family handler can reject the
% successful handling of a key by calling \cs{kv@handled@false}.
%
% Since 2019/12/15 v1.18 \cs{kv@processor@default} also defines
% macro \cs{kv@fam} with meaning \meta{family} for convenience.
%
% \subsubsection{Processing similar to \cs{setkeys*} of package \xpackage{xkeyval}}
%
% \begin{declcs}{kv@processor@known}\,\M{family}\,^^A
%   \M{cmd}\,\M{key}\,\M{value}
% \end{declcs}
% The key value processor \cs{kv@processor@known} behaves similar
% to \cs{kv@processor@default}. If the \meta{key} exists in the
% \meta{family} its code is called, otherwise the family handler
% is tried. If the family handler is not set or cannot handle the
% key, the unknown key value pair is added to the macro \meta{cmd}.
% Since 2011/10/18 v1.15.
%
% The behaviour in pseudo code:
% \begin{quote}
%   if \meta{key} exists\\
%   \hspace*{1.5em}call the keyval code of \meta{key}\\
%   else\\
%   \hspace*{1.5em}if \meta{handler} for \meta{family} exists\\
%   \hspace*{3em}handled = true\\
%   \hspace*{3em}\meta{handler}\,\M{key}\,\M{value}\\
%   \hspace*{3em}if handled\\
%   \hspace*{3em}else\\
%   \hspace*{4.5em}add \texttt{"}\M{key}\texttt{=}\M{value}\texttt{"}
%     to \M{cmd}\\
%   \hspace*{3em}fi\\
%   \hspace*{1.5em}else\\
%   \hspace*{3em}add \texttt{"}\M{key}\texttt{=}\M{value}\texttt{"}
%     to \M{cmd}\\
%   \hspace*{3em}raise unknown key error\\
%   \hspace*{1.5em}fi\\
%   fi
% \end{quote}
%
% Since 2019/12/15 v1.18 \cs{kv@processor@known} also defines
% macro \cs{kv@fam} with meaning \meta{family} for convenience.
%
% \subsection{Default family handler}
%
% \cs{kv@processor@default} calls \meta{handler}, the default
% handler for the family, if the key does not exist in the family.
% The handler is called with two arguments, the key and the value.
% It can be defined with \cs{kv@set@family@hander}:
%
% \begin{declcs}{kv@set@family@handler}\,\M{family}\,\M{handler definition}
% \end{declcs}
% This sets the default family handler for the keyval family
% \meta{family}. Inside \meta{handler definition} |#1| stands for
% the key and |#2| is the value. Also \cs{kv@key} and \cs{kv@value}
% can be used for the key and the value. If the value is not given,
% \cs{kv@value} has the meaning \cs{relax}.
%
% \begin{declcs}{kv@unset@family@handler}\,\M{family}
% \end{declcs}
% It removes the family handler for \meta{family}.
% Since 2011/10/18 v1.15.
%
% \subsection{Put it all together}
%
% \begin{declcs}{kvsetkeys}\,\M{family}\,\M{key value list}
% \end{declcs}
% Macro \cs{kvsetkeys} processes the \meta{key value list} with
% the standard processor \cs{kv@processor@default}:
% \begin{quote}
%   \cs{kv@parse}\,\M{key value list}^^A
%   \texttt{\{}\cs{kv@processor@default}\,\M{family}\texttt{\}}
% \end{quote}
%
% \begin{declcs}{kvsetknownkeys}\,\M{family}\,\M{cmd}\,\M{key value list}
% \end{declcs}
% Macro \cs{kvsetknownkeys} processes the \meta{key value list}
% with processor \cs{kv@processor@known}. All key value pairs
% with keys that are not known in \meta{family} are stored
% in macro \meta{cmd}. A previous contents of macro \meta{cmd}
% will be overwritten. If all keys can be handled, \meta{cmd}
% will be empty, otherwise it contains a key value list of
% unhandled key value pairs.
% Since 2011/10/18 v1.15.
%
% Pseudo code:
% \begin{quote}
%   create macro \meta{cmdaux} with unique name (inside the current group)\\
%   \cs{def}\meta{cmdaux}\{\}\\
%   \cs{kv@parse}\,\M{key value list}^^A
%   \texttt{\{}\cs{kv@processor@known}\,\M{family}\,\M{cmdaux}\texttt{\}}\\
%   \cs{let}\meta{cmd}=\meta{cmdaux}
% \end{quote}
%
% \begin{declcs}{kvsetkeys@expandafter}\,\M{family}\,\M{list cmd}\\
%   \cs{kvsetknownkeys@expandafter}\,\M{family}\,%
%   \M{cmd}\,\M{list cmd}
% \end{declcs}
% Both macros behave like the counterparts without suffix
% |@expandafter|. The difference is that the key value list is
% given as macro that is expanded once.
% Since 2011/10/18 v1.15.
%
% Thus you can replace \cs{setkeys} of package \xpackage{keyval}
% by the key value parser of this package:
% \begin{quote}
%   |\renewcommand*{\setkeys}{\kvsetkeys}|\\
%   or\\
%   |\let\setkeys\kvsetkeys|
% \end{quote}
%
% \subsection{Comma separated lists}
%
% Since version 2007/09/29 v1.3 this package also supports the normalizing
% and parsing of general comma separated lists.
%
% \begin{declcs}{comma@normalize}\,\M{comma list}
% \end{declcs}
% Macro \cs{comma@normalize} normalizes the comma separated list,
% removes spaces around commas. The result is put in macro \cs{comma@list}.
%
% \begin{declcs}{comma@parse}\,\M{comma list}\,\M{processor}
% \end{declcs}
% Macro \cs{comma@parse} first normalizes the comma separated list
% and then parses the list by calling \cs{comma@parse@normalized}.
%
% \begin{declcs}{comma@parse@normalized}\,\M{normalized comma list}^^A
%   \,\M{processor}
% \end{declcs}
% The list is parsed. Empty entries are ignored. \meta{processor}
% is called for each non-empty entry with the entry as argument:
% \begin{quote}
%   \meta{processor}|{|\meta{entry}|}|
% \end{quote}
% Also the entry is stored in the macro \cs{comma@entry}.
%
% \begin{declcs}{comma@break}
% \end{declcs}
% Since version 2011/03/03 v1.11 \cs{comma@break} can be
% called inside the \meta{processor}
% of \cs{comma@parse} or \cs{comma@parse@normalized}, then
% the processing is stopped and the following entries discarded.
%
% \section{Example}
%
% The following example prints a short piece of HTML code using
% the tabbing environment for indenting purpose and a key value
% syntax for specifying the attributes of an HTML tag.
% The example illustrates the use of a default family handler.
%    \begin{macrocode}
%<*example>
\documentclass{article}
\usepackage[T1]{fontenc}
\usepackage{kvsetkeys}
\usepackage{keyval}

\makeatletter
\newcommand*{\tag}[2][]{%
  % #1: attributes
  % #2: tag name
  \begingroup
    \toks@={}%
    \let\@endslash\@empty
    \kvsetkeys{tag}{#1}%
    \texttt{%
      \textless #2\the\toks@\@endslash\textgreater
    }%
  \endgroup
}
\kv@set@family@handler{tag}{%
  % #1: key
  % #2: value
  \toks@\expandafter{%
    \the\toks@
    \space
    #1=\string"#2\string"%
  }%
}
\define@key{tag}{/}[]{%
  \def\@endslash{/}%
}
\makeatother

\begin{document}
\begin{tabbing}
  \mbox{}\qquad\=\qquad\=\kill
  \tag{html}\\
  \>\dots\\
  \>\tag[border=1]{table}\\
  \>\>\tag[width=200, span=3, /]{colgroup}\\
  \>\>\dots\\
  \>\tag{/table}\\
  \>\dots\\
  \tag{/html}\\
\end{tabbing}
\end{document}
%</example>
%    \end{macrocode}
%
% \StopEventually{
% }
%
% \section{Implementation}
%
% \subsection{Identification}
%
%    \begin{macrocode}
%<*package>
%    \end{macrocode}
%    Reload check, especially if the package is not used with \LaTeX.
%    \begin{macrocode}
\begingroup\catcode61\catcode48\catcode32=10\relax%
  \catcode13=5 % ^^M
  \endlinechar=13 %
  \catcode35=6 % #
  \catcode39=12 % '
  \catcode44=12 % ,
  \catcode45=12 % -
  \catcode46=12 % .
  \catcode58=12 % :
  \catcode64=11 % @
  \catcode123=1 % {
  \catcode125=2 % }
  \expandafter\let\expandafter\x\csname ver@kvsetkeys.sty\endcsname
  \ifx\x\relax % plain-TeX, first loading
  \else
    \def\empty{}%
    \ifx\x\empty % LaTeX, first loading,
      % variable is initialized, but \ProvidesPackage not yet seen
    \else
      \expandafter\ifx\csname PackageInfo\endcsname\relax
        \def\x#1#2{%
          \immediate\write-1{Package #1 Info: #2.}%
        }%
      \else
        \def\x#1#2{\PackageInfo{#1}{#2, stopped}}%
      \fi
      \x{kvsetkeys}{The package is already loaded}%
      \aftergroup\endinput
    \fi
  \fi
\endgroup%
%    \end{macrocode}
%    Package identification:
%    \begin{macrocode}
\begingroup\catcode61\catcode48\catcode32=10\relax%
  \catcode13=5 % ^^M
  \endlinechar=13 %
  \catcode35=6 % #
  \catcode39=12 % '
  \catcode40=12 % (
  \catcode41=12 % )
  \catcode44=12 % ,
  \catcode45=12 % -
  \catcode46=12 % .
  \catcode47=12 % /
  \catcode58=12 % :
  \catcode64=11 % @
  \catcode91=12 % [
  \catcode93=12 % ]
  \catcode123=1 % {
  \catcode125=2 % }
  \expandafter\ifx\csname ProvidesPackage\endcsname\relax
    \def\x#1#2#3[#4]{\endgroup
      \immediate\write-1{Package: #3 #4}%
      \xdef#1{#4}%
    }%
  \else
    \def\x#1#2[#3]{\endgroup
      #2[{#3}]%
      \ifx#1\@undefined
        \xdef#1{#3}%
      \fi
      \ifx#1\relax
        \xdef#1{#3}%
      \fi
    }%
  \fi
\expandafter\x\csname ver@kvsetkeys.sty\endcsname
\ProvidesPackage{kvsetkeys}%
  [2019/12/15 v1.18 Key value parser (HO)]%
%    \end{macrocode}
%
%    \begin{macrocode}
\begingroup\catcode61\catcode48\catcode32=10\relax%
  \catcode13=5 % ^^M
  \endlinechar=13 %
  \catcode123=1 % {
  \catcode125=2 % }
  \catcode64=11 % @
  \def\x{\endgroup
    \expandafter\edef\csname KVS@AtEnd\endcsname{%
      \endlinechar=\the\endlinechar\relax
      \catcode13=\the\catcode13\relax
      \catcode32=\the\catcode32\relax
      \catcode35=\the\catcode35\relax
      \catcode61=\the\catcode61\relax
      \catcode64=\the\catcode64\relax
      \catcode123=\the\catcode123\relax
      \catcode125=\the\catcode125\relax
    }%
  }%
\x\catcode61\catcode48\catcode32=10\relax%
\catcode13=5 % ^^M
\endlinechar=13 %
\catcode35=6 % #
\catcode64=11 % @
\catcode123=1 % {
\catcode125=2 % }
\def\TMP@EnsureCode#1#2{%
  \edef\KVS@AtEnd{%
    \KVS@AtEnd
    \catcode#1=\the\catcode#1\relax
  }%
  \catcode#1=#2\relax
}
\TMP@EnsureCode{36}{3}% $
\TMP@EnsureCode{38}{4}% &
\TMP@EnsureCode{39}{12}% '
\TMP@EnsureCode{43}{12}% +
\TMP@EnsureCode{44}{12}% ,
\TMP@EnsureCode{45}{12}% -
\TMP@EnsureCode{46}{12}% .
\TMP@EnsureCode{47}{12}% /
\TMP@EnsureCode{91}{12}% [
\TMP@EnsureCode{93}{12}% ]
\TMP@EnsureCode{94}{7}% ^ (superscript)
\TMP@EnsureCode{96}{12}% `
\TMP@EnsureCode{126}{13}% ~ (active)
\edef\KVS@AtEnd{\KVS@AtEnd\noexpand\endinput}
%    \end{macrocode}
%
% \subsection{Package loading}
%
%    \begin{macrocode}
\begingroup\expandafter\expandafter\expandafter\endgroup
\expandafter\ifx\csname RequirePackage\endcsname\relax
  \def\TMP@RequirePackage#1[#2]{%
    \begingroup\expandafter\expandafter\expandafter\endgroup
    \expandafter\ifx\csname ver@#1.sty\endcsname\relax
      \input #1.sty\relax
    \fi
  }%
  \TMP@RequirePackage{infwarerr}[2007/09/09]%
  \let\PackageError\@PackageError
\else
\fi
%    \end{macrocode}
%
%    \begin{macrocode}
\expandafter\ifx\csname toks@\endcsname\relax
  \toksdef\toks@=0 %
\fi
%    \end{macrocode}
%
% \subsection{Check for \eTeX}
%
%    \cs{unexpanded}, \cs{ifcsname}, and \cs{unless} are used if found.
%    \begin{macrocode}
\ifx\numexpr\@undefined
  \catcode`\$=9 % ignore
  \catcode`\&=14 % comment
\else % e-TeX
  \catcode`\$=14 % comment
  \catcode`\&=9  % ignore
\fi
%    \end{macrocode}
%
% \subsection{Generic help macros}
%
%    \begin{macro}{\KVS@Empty}
%    \begin{macrocode}
\def\KVS@Empty{}
%    \end{macrocode}
%    \end{macro}
%    \begin{macro}{\KVS@FirstOfTwo}
%    \begin{macrocode}
\long\def\KVS@FirstOfTwo#1#2{#1}
%    \end{macrocode}
%    \end{macro}
%    \begin{macro}{\KVS@SecondOfTwo}
%    \begin{macrocode}
\long\def\KVS@SecondOfTwo#1#2{#2}
%    \end{macrocode}
%    \end{macro}
%
%    \begin{macro}{\KVS@IfEmpty}
%    \begin{macrocode}
\long\def\KVS@IfEmpty#1{%
& \edef\KVS@Temp{\unexpanded{#1}}%
$ \begingroup
$   \toks@{#1}%
$   \edef\KVS@Temp{\the\toks@}%
$ \expandafter\endgroup
  \ifx\KVS@Temp\KVS@Empty
    \expandafter\KVS@FirstOfTwo
  \else
    \expandafter\KVS@SecondOfTwo
  \fi
}
%    \end{macrocode}
%    \end{macro}
%
% \subsection{Normalizing}
%
%    \begin{macro}{\kv@normalize}
%    \begin{macrocode}
\long\def\kv@normalize#1{%
  \begingroup
    \toks@{,#1,}%
    \KVS@Comma
    \KVS@SpaceComma
    \KVS@CommaSpace
    \KVS@CommaComma
    \KVS@Equals
    \KVS@SpaceEquals
    \KVS@EqualsSpace
    \xdef\KVS@Global{\the\toks@}%
  \endgroup
  \let\kv@list\KVS@Global
}
%    \end{macrocode}
%    \end{macro}
%    \begin{macro}{\comma@normalize}
%    \begin{macrocode}
\def\comma@normalize#1{%
  \begingroup
    \toks@{,#1,}%
    \KVS@Comma
    \KVS@SpaceComma
    \KVS@CommaSpace
    \KVS@CommaComma
    \xdef\KVS@Global{\the\toks@}%
  \endgroup
  \let\comma@list\KVS@Global
}
%    \end{macrocode}
%    \end{macro}
%    \begin{macro}{\KVS@Comma}
%    Converts active commas into comma with catcode other.
%    Also adds a comma at the end to protect the last value
%    for next cleanup steps.
%    \begin{macrocode}
\begingroup
  \lccode`\,=`\,%
  \lccode`\~=`\,%
\lowercase{\endgroup
  \def\KVS@Comma{%
    \toks@\expandafter{\expandafter}\expandafter
    \KVS@@Comma\the\toks@~\KVS@Nil
  }%
  \long\def\KVS@@Comma#1~#2\KVS@Nil{%
    \toks@\expandafter{\the\toks@#1}%
    \KVS@IfEmpty{#2}{%
    }{%
      \KVS@@Comma,#2\KVS@Nil
    }%
  }%
}
%    \end{macrocode}
%    \end{macro}
%    \begin{macro}{\KVS@SpaceComma}
%    Removes spaces before the comma, may add commas at the end.
%    \begin{macrocode}
\def\KVS@SpaceComma#1{%
  \def\KVS@SpaceComma{%
    \expandafter\KVS@@SpaceComma\the\toks@#1,\KVS@Nil
  }%
}
\KVS@SpaceComma{ }
%    \end{macrocode}
%    \end{macro}
%    \begin{macro}{\KVS@@SpaceComma}
%    \begin{macrocode}
\long\def\KVS@@SpaceComma#1 ,#2\KVS@Nil{%
  \KVS@IfEmpty{#2}{%
    \toks@{#1}%
  }{%
    \KVS@@SpaceComma#1,#2\KVS@Nil
  }%
}
%    \end{macrocode}
%    \end{macro}
%    \begin{macro}{\KVS@CommaSpace}
%    Removes spaces after the comma, may add commas at the end.
%    \begin{macrocode}
\def\KVS@CommaSpace{%
  \expandafter\KVS@@CommaSpace\the\toks@, \KVS@Nil
}
%    \end{macrocode}
%    \end{macro}
%    \begin{macro}{\KVS@@CommaSpace}
%    \begin{macrocode}
\long\def\KVS@@CommaSpace#1, #2\KVS@Nil{%
  \KVS@IfEmpty{#2}{%
    \toks@{#1}%
  }{%
    \KVS@@CommaSpace#1,#2\KVS@Nil
  }%
}
%    \end{macrocode}
%    \end{macro}
%    \begin{macro}{\KVS@CommaComma}
%    Replaces multiple commas by one comma.
%    \begin{macrocode}
\def\KVS@CommaComma{%
  \expandafter\KVS@@CommaComma\the\toks@,\KVS@Nil
}
%    \end{macrocode}
%    \end{macro}
%    \begin{macro}{\KVS@@CommaComma}
%    \begin{macrocode}
\long\def\KVS@@CommaComma#1,,#2\KVS@Nil{%
  \KVS@IfEmpty{#2}{%
    \toks@{#1,}% (!)
  }{%
    \KVS@@CommaComma#1,#2\KVS@Nil
  }%
}
%    \end{macrocode}
%    \end{macro}
%
%    \begin{macro}{\KVS@Equals}
%    Converts active equals signs into catcode other characters.
%    \begin{macrocode}
\begingroup
  \lccode`\==`\=%
  \lccode`\~=`\=%
\lowercase{\endgroup
  \def\KVS@Equals{%
    \toks@\expandafter{\expandafter}\expandafter
    \KVS@@Equals\the\toks@~\KVS@Nil
  }%
  \long\def\KVS@@Equals#1~#2\KVS@Nil{%
    \edef\KVS@Temp{\the\toks@}%
    \ifx\KVS@Temp\KVS@Empty
      \expandafter\KVS@FirstOfTwo
    \else
      \expandafter\KVS@SecondOfTwo
    \fi
    {%
      \toks@{#1}%
    }{%
      \toks@\expandafter{\the\toks@=#1}%
    }%
    \KVS@IfEmpty{#2}{%
    }{%
      \KVS@@Equals#2\KVS@Nil
    }%
  }%
}
%    \end{macrocode}
%    \end{macro}
%    \begin{macro}{\KVS@SpaceEquals}
%    Removes spaces before the equals sign.
%    \begin{macrocode}
\def\KVS@SpaceEquals#1{%
  \def\KVS@SpaceEquals{%
    \expandafter\KVS@@SpaceEquals\the\toks@#1=\KVS@Nil
  }%
}
\KVS@SpaceEquals{ }
%    \end{macrocode}
%    \end{macro}
%    \begin{macro}{\KVS@@SpaceEquals}
%    \begin{macrocode}
\long\def\KVS@@SpaceEquals#1 =#2\KVS@Nil{%
  \KVS@IfEmpty{#2}{%
    \toks@{#1}%
  }{%
    \KVS@@SpaceEquals#1=#2\KVS@Nil
  }%
}
%    \end{macrocode}
%    \end{macro}
%    \begin{macro}{\KVS@EqualsSpace}
%    Removes spaces after the equals sign.
%    \begin{macrocode}
\def\KVS@EqualsSpace{%
  \expandafter\KVS@@EqualsSpace\the\toks@= \KVS@Nil
}
%    \end{macrocode}
%    \end{macro}
%    \begin{macro}{\KVS@@EqualsSpace}
%    \begin{macrocode}
\long\def\KVS@@EqualsSpace#1= #2\KVS@Nil{%
  \KVS@IfEmpty{#2}{%
    \toks@{#1}%
  }{%
    \KVS@@EqualsSpace#1=#2\KVS@Nil
  }%
}
%    \end{macrocode}
%    \end{macro}
%
% \subsection{Parsing key value lists}
%
%    \begin{macro}{\kv@parse}
%    Normalizes and parses the key value list. Also sets \cs{kv@list}.
%    \begin{macrocode}
\long\def\kv@parse#1{%
  \kv@normalize{#1}%
  \expandafter\kv@parse@normalized\expandafter{\kv@list}%
}
%    \end{macrocode}
%    \end{macro}
%    \begin{macro}{\kv@parse@normalized}
%    |#1|: key value list\\
%    |#2|: processor
%    \begin{macrocode}
\long\def\kv@parse@normalized#1#2{%
  \KVS@Parse#1,\KVS@Nil{#2}%
}
%    \end{macrocode}
%    \end{macro}
%    \begin{macro}{\KVS@Parse}
%    |#1,#2|: key value list\\
%    |#3|: processor
%    \begin{macrocode}
\long\def\KVS@Parse#1,#2\KVS@Nil#3{%
  \KVS@IfEmpty{#1}{%
  }{%
    \KVS@Process#1=\KVS@Nil{#3}%
  }%
  \KVS@MaybeBreak
  \KVS@IfEmpty{#2}{%
  }{%
    \KVS@Parse#2\KVS@Nil{#3}%
  }%
}
%    \end{macrocode}
%    \end{macro}
%    \begin{macro}{\KVS@Process}
%    |#1|: key\\
%    |#2|: value, |=|\\
%    |#3|: processor
%    \begin{macrocode}
\long\def\KVS@Process#1=#2\KVS@Nil#3{%
  \let\KVS@MaybeBreak\relax
  \def\kv@key{#1}%
  \KVS@IfEmpty{#2}{%
    \let\kv@value\relax
    #3{#1}{}%
  }{%
    \KVS@@Process{#1}#2\KVS@Nil{#3}%
  }%
}
%    \end{macrocode}
%    \end{macro}
%    \begin{macro}{\KVS@@Process}
%    |#1|: key\\
%    |#2|: value\\
%    |#3|: processor
%    \begin{macrocode}
\long\def\KVS@@Process#1#2=\KVS@Nil#3{%
& \edef\kv@value{\unexpanded{#2}}%
$ \begingroup
$   \toks@{#2}%
$   \xdef\KVS@Global{\the\toks@}%
$ \endgroup
$ \let\kv@value\KVS@Global
  #3{#1}{#2}%
}
%    \end{macrocode}
%    \end{macro}
%    \begin{macro}{\KVS@MaybeBreak}
%    \begin{macrocode}
\let\KVS@MaybeBreak\relax
%    \end{macrocode}
%    \end{macro}
%    \begin{macro}{\KVS@break}
%    \begin{macrocode}
\def\KVS@break#1#2#3#4{%
  \let\KVS@MaybeBreak\relax
}
%    \end{macrocode}
%    \end{macro}
%    \begin{macro}{\kv@break}
%    \begin{macrocode}
\def\kv@break{%
  \let\KVS@MaybeBreak\KVS@break
}
%    \end{macrocode}
%    \end{macro}
%
% \subsection{Parsing comma lists}
%
%    \begin{macro}{\comma@parse}
%    Normalizes and parses the key value list. Also sets \cs{comma@list}.
%    \begin{macrocode}
\def\comma@parse#1{%
  \comma@normalize{#1}%
  \expandafter\comma@parse@normalized\expandafter{\comma@list}%
}
%    \end{macrocode}
%    \end{macro}
%
%    \begin{macro}{\comma@parse@normalized}
%    |#1|: comma list\\
%    |#2|: processor
%    \begin{macrocode}
\def\comma@parse@normalized#1#2{%
  \KVS@CommaParse#1,\KVS@Nil{#2}%
}
%    \end{macrocode}
%    \end{macro}
%    \begin{macro}{\KVS@CommaParse}
%    |#1,#2|: comma list\\
%    |#3|: processor
%    \begin{macrocode}
\def\KVS@CommaParse#1,#2\KVS@Nil#3{%
  \KVS@IfEmpty{#1}{%
  }{%
    \def\comma@entry{#1}%
    #3{#1}%
  }%
  \KVS@MaybeBreak
  \KVS@IfEmpty{#2}{%
  }{%
    \KVS@CommaParse#2\KVS@Nil{#3}%
  }%
}
%    \end{macrocode}
%    \end{macro}
%    \begin{macro}{\comma@break}
%    \begin{macrocode}
\def\comma@break{%
  \let\KVS@MaybeBreak\KVS@break
}
%    \end{macrocode}
%    \end{macro}
%
% \subsection{Processing key value pairs}
%
%    \begin{macro}{\kv@handled@false}
%    The handler can call \cs{kv@handled@false} or
%    \cs{kv@handled@true} so report failure or success.
%    The default is success (compatibility for versions
%    before 2011/10/18 v1.15).
%    \begin{macrocode}
\def\kv@handled@false{%
  \let\ifkv@handled@\iffalse
}
%    \end{macrocode}
%    \end{macro}
%    \begin{macro}{\kv@handled@true}
%    \begin{macrocode}
\def\kv@handled@true{%
  \let\ifkv@handled@\iftrue
}
%    \end{macrocode}
%    \end{macro}
%    \begin{macro}{\ifkv@handled@}
%    \begin{macrocode}
\kv@handled@true
%    \end{macrocode}
%    \end{macro}
%
%    \begin{macro}{\kv@processor@default}
%    \begin{macrocode}
\def\kv@processor@default#1#2{%
  \begingroup
    \csname @safe@activestrue\endcsname
    \let\ifincsname\iftrue
    \edef\KVS@temp{\endgroup
      \noexpand\KVS@ProcessorDefault{#1}{#2}%
    }%
  \KVS@temp
}
%    \end{macrocode}
%    \end{macro}
%    \begin{macro}{\KVS@ProcessorDefault}
%    \begin{macrocode}
\long\def\KVS@ProcessorDefault#1#2#3{%
  \def\kv@fam{#1}%
& \unless\ifcsname KV@#1@#2\endcsname
$ \begingroup\expandafter\expandafter\expandafter\endgroup
$ \expandafter\ifx\csname KV@#1@#2\endcsname\relax
&   \unless\ifcsname KVS@#1@handler\endcsname
$   \begingroup\expandafter\expandafter\expandafter\endgroup
$   \expandafter\ifx\csname KVS@#1@handler\endcsname\relax
      \kv@error@unknownkey{#1}{#2}%
    \else
      \kv@handled@true
      \csname KVS@#1@handler\endcsname{#2}{#3}%
      \relax
      \ifkv@handled@
      \else
        \kv@error@unknownkey{#1}{#2}%
      \fi
    \fi
  \else
    \ifx\kv@value\relax
&     \unless\ifcsname KV@#1@#2@default\endcsname
$     \begingroup\expandafter\expandafter\expandafter\endgroup
$     \expandafter\ifx\csname KV@#1@#2@default\endcsname\relax
        \kv@error@novalue{#1}{#2}%
      \else
        \csname KV@#1@#2@default\endcsname
        \relax
      \fi
    \else
      \csname KV@#1@#2\endcsname{#3}%
    \fi
  \fi
}
%    \end{macrocode}
%    \end{macro}
%
%    \begin{macro}{\kv@processor@known}
%    \begin{macrocode}
\def\kv@processor@known#1#2#3{%
  \begingroup
    \csname @safe@activestrue\endcsname
    \let\ifincsname\iftrue
    \edef\KVS@temp{\endgroup
      \noexpand\KVS@ProcessorKnown{#1}\noexpand#2{#3}%
    }%
  \KVS@temp
}
%    \end{macrocode}
%    \end{macro}
%    \begin{macro}{\KVS@ProcessorKnown}
%    \begin{macrocode}
\long\def\KVS@ProcessorKnown#1#2#3#4{%
  \def\kv@fam{#1}%
& \unless\ifcsname KV@#1@#3\endcsname
$ \begingroup\expandafter\expandafter\expandafter\endgroup
$ \expandafter\ifx\csname KV@#1@#3\endcsname\relax
&   \unless\ifcsname KVS@#1@handler\endcsname
$   \begingroup\expandafter\expandafter\expandafter\endgroup
$   \expandafter\ifx\csname KVS@#1@handler\endcsname\relax
      \KVS@AddUnhandled#2{#3}{#4}%
    \else
      \kv@handled@true
      \csname KVS@#1@handler\endcsname{#3}{#4}%
      \relax
      \ifkv@handled@
      \else
        \KVS@AddUnhandled#2{#3}{#4}%
      \fi
    \fi
  \else
    \ifx\kv@value\relax
&     \unless\ifcsname KV@#1@#2@default\endcsname
$     \begingroup\expandafter\expandafter\expandafter\endgroup
$     \expandafter\ifx\csname KV@#1@#3@default\endcsname\relax
        \kv@error@novalue{#1}{#3}%
      \else
        \csname KV@#1@#3@default\endcsname
        \relax
      \fi
    \else
      \csname KV@#1@#3\endcsname{#4}%
    \fi
  \fi
}
%    \end{macrocode}
%    \end{macro}
%    \begin{macro}{\KVS@AddUnhandled}
%    \begin{macrocode}
\long\def\KVS@AddUnhandled#1#2#3{%
& \edef#1{%
&   \ifx#1\KVS@empty
&   \else
&     \unexpanded{#1},%
&   \fi
&   \unexpanded{{#2}={#3}}%
& }%
$ \begingroup
$   \ifx#1\KVS@empty
$     \toks@{{#2}={#3}}%
$   \else
$     \toks@\expandafter{#1,{#2}={#3}}%
$   \fi
$   \xdef\KVS@Global{\the\toks@}%
$ \endgroup
$ \let#1\KVS@Global
}
%    \end{macrocode}
%    \end{macro}
%
%    \begin{macro}{\kv@set@family@handler}
%    \begin{macrocode}
\long\def\kv@set@family@handler#1#2{%
  \begingroup
    \csname @safe@activestrue\endcsname
    \let\ifincsname\iftrue
  \expandafter\endgroup
  \expandafter\def\csname KVS@#1@handler\endcsname##1##2{#2}%
}
%    \end{macrocode}
%    \end{macro}
%
%    \begin{macro}{\kv@unset@family@handler}
%    \begin{macrocode}
\long\def\kv@unset@family@handler#1#2{%
  \begingroup
    \csname @safe@activestrue\endcsname
    \let\ifincsname\iftrue
  \expandafter\endgroup
  \expandafter\let\csname KVS@#1@handler\endcsname\@UnDeFiNeD
}
%    \end{macrocode}
%    \end{macro}
%
% \subsection{Error handling}
%
%    \begin{macro}{\kv@error@novalue}
%    \begin{macrocode}
\def\kv@error@novalue{%
  \kv@error@generic{No value specified for}%
}
%    \end{macrocode}
%    \end{macro}
%    \begin{macro}{\kv@error@unknownkey}
%    \begin{macrocode}
\def\kv@error@unknownkey{%
  \kv@error@generic{Undefined}%
}
%    \end{macrocode}
%    \end{macro}
%    \begin{macro}{\kv@error@generic}
%    \begin{macrocode}
\def\kv@error@generic#1#2#3{%
  \PackageError{kvsetkeys}{%
    #1 key `#3'%
  }{%
    The keyval family of the key `#3' is `#2'.\MessageBreak
    The setting of the key is ignored because of the error.\MessageBreak
    \MessageBreak
    \@ehc
  }%
}
%    \end{macrocode}
%    \end{macro}
%
% \subsection{Do it all}
%
%    \begin{macro}{\kvsetkeys}
%    \begin{macrocode}
\long\def\kvsetkeys#1#2{%
  \kv@parse{#2}{\kv@processor@default{#1}}%
}
%    \end{macrocode}
%    \end{macro}
%
%    \begin{macro}{\kvsetkeys@expandafter}
%    \begin{macrocode}
\def\kvsetkeys@expandafter#1#2{%
  \expandafter\kv@parse\expandafter{#2}{%
    \kv@processor@default{#1}%
  }%
}
%    \end{macrocode}
%    \end{macro}
%
%    \begin{macro}{\KVS@cmd}
%    \begin{macrocode}
\def\KVS@cmd{0}%
%    \end{macrocode}
%    \end{macro}
%    \begin{macro}{\KVS@cmd@inc}
%    \begin{macrocode}
\def\KVS@cmd@inc{%
& \edef\KVS@cmd{\the\numexpr\KVS@cmd+1}%
$ \begingroup
$   \count255=\KVS@cmd\relax
$   \advance\count255 by 1\relax
$ \edef\x{\endgroup
$   \noexpand\def\noexpand\KVS@cmd{\number\count255}%
$ }%
$ \x
}
%    \end{macrocode}
%    \end{macro}
%    \begin{macro}{\KVS@cmd@dec}
%    \begin{macrocode}
\def\KVS@cmd@dec{%
& \edef\KVS@cmd{\the\numexpr\KVS@cmd-1}%
$ \begingroup
$   \count255=\KVS@cmd\relax
$   \advance\count255 by -1\relax
$ \edef\x{\endgroup
$   \noexpand\def\noexpand\KVS@cmd{\number\count255}%
$ }%
$ \x
}
%    \end{macrocode}
%    \end{macro}
%    \begin{macro}{\KVS@empty}
%    \begin{macrocode}
\def\KVS@empty{}
%    \end{macrocode}
%    \end{macro}
%
%    \begin{macro}{\kvsetknownkeys}
%    \begin{macrocode}
\def\kvsetknownkeys{%
  \expandafter
  \KVS@setknownkeys\csname KVS@cmd\KVS@cmd\endcsname{}%
}
%    \end{macrocode}
%    \end{macro}
%    \begin{macro}{\KVS@setknownkeys}
%    \begin{macrocode}
\long\def\KVS@setknownkeys#1#2#3#4#5{%
  \let#1\KVS@empty
  \KVS@cmd@inc
  #2\kv@parse#2{#5}{\kv@processor@known{#3}#1}%
  \KVS@cmd@dec
  \let#4=#1%
}
%    \end{macrocode}
%    \end{macro}
%
%    \begin{macro}{\kvsetknownkeys@expandafter}
%    \begin{macrocode}
\def\kvsetknownkeys@expandafter{%
  \expandafter
  \KVS@setknownkeys
      \csname KVS@cmd\KVS@cmd\endcsname\expandafter
}
%    \end{macrocode}
%    \end{macro}
%
%    \begin{macrocode}
\KVS@AtEnd%
%</package>
%    \end{macrocode}
%% \section{Installation}
%
% \subsection{Download}
%
% \paragraph{Package.} This package is available on
% CTAN\footnote{\CTANpkg{kvsetkeys}}:
% \begin{description}
% \item[\CTAN{macros/latex/contrib/kvsetkeys/kvsetkeys.dtx}] The source file.
% \item[\CTAN{macros/latex/contrib/kvsetkeys/kvsetkeys.pdf}] Documentation.
% \end{description}
%
%
% \paragraph{Bundle.} All the packages of the bundle `kvsetkeys'
% are also available in a TDS compliant ZIP archive. There
% the packages are already unpacked and the documentation files
% are generated. The files and directories obey the TDS standard.
% \begin{description}
% \item[\CTANinstall{install/macros/latex/contrib/kvsetkeys.tds.zip}]
% \end{description}
% \emph{TDS} refers to the standard ``A Directory Structure
% for \TeX\ Files'' (\CTANpkg{tds}). Directories
% with \xfile{texmf} in their name are usually organized this way.
%
% \subsection{Bundle installation}
%
% \paragraph{Unpacking.} Unpack the \xfile{kvsetkeys.tds.zip} in the
% TDS tree (also known as \xfile{texmf} tree) of your choice.
% Example (linux):
% \begin{quote}
%   |unzip kvsetkeys.tds.zip -d ~/texmf|
% \end{quote}
%
% \subsection{Package installation}
%
% \paragraph{Unpacking.} The \xfile{.dtx} file is a self-extracting
% \docstrip\ archive. The files are extracted by running the
% \xfile{.dtx} through \plainTeX:
% \begin{quote}
%   \verb|tex kvsetkeys.dtx|
% \end{quote}
%
% \paragraph{TDS.} Now the different files must be moved into
% the different directories in your installation TDS tree
% (also known as \xfile{texmf} tree):
% \begin{quote}
% \def\t{^^A
% \begin{tabular}{@{}>{\ttfamily}l@{ $\rightarrow$ }>{\ttfamily}l@{}}
%   kvsetkeys.sty & tex/generic/kvsetkeys/kvsetkeys.sty\\
%   kvsetkeys.pdf & doc/latex/kvsetkeys/kvsetkeys.pdf\\
%   kvsetkeys-example.tex & doc/latex/kvsetkeys/kvsetkeys-example.tex\\
%   kvsetkeys.dtx & source/latex/kvsetkeys/kvsetkeys.dtx\\
% \end{tabular}^^A
% }^^A
% \sbox0{\t}^^A
% \ifdim\wd0>\linewidth
%   \begingroup
%     \advance\linewidth by\leftmargin
%     \advance\linewidth by\rightmargin
%   \edef\x{\endgroup
%     \def\noexpand\lw{\the\linewidth}^^A
%   }\x
%   \def\lwbox{^^A
%     \leavevmode
%     \hbox to \linewidth{^^A
%       \kern-\leftmargin\relax
%       \hss
%       \usebox0
%       \hss
%       \kern-\rightmargin\relax
%     }^^A
%   }^^A
%   \ifdim\wd0>\lw
%     \sbox0{\small\t}^^A
%     \ifdim\wd0>\linewidth
%       \ifdim\wd0>\lw
%         \sbox0{\footnotesize\t}^^A
%         \ifdim\wd0>\linewidth
%           \ifdim\wd0>\lw
%             \sbox0{\scriptsize\t}^^A
%             \ifdim\wd0>\linewidth
%               \ifdim\wd0>\lw
%                 \sbox0{\tiny\t}^^A
%                 \ifdim\wd0>\linewidth
%                   \lwbox
%                 \else
%                   \usebox0
%                 \fi
%               \else
%                 \lwbox
%               \fi
%             \else
%               \usebox0
%             \fi
%           \else
%             \lwbox
%           \fi
%         \else
%           \usebox0
%         \fi
%       \else
%         \lwbox
%       \fi
%     \else
%       \usebox0
%     \fi
%   \else
%     \lwbox
%   \fi
% \else
%   \usebox0
% \fi
% \end{quote}
% If you have a \xfile{docstrip.cfg} that configures and enables \docstrip's
% TDS installing feature, then some files can already be in the right
% place, see the documentation of \docstrip.
%
% \subsection{Refresh file name databases}
%
% If your \TeX~distribution
% (\TeX\,Live, \mikTeX, \dots) relies on file name databases, you must refresh
% these. For example, \TeX\,Live\ users run \verb|texhash| or
% \verb|mktexlsr|.
%
% \subsection{Some details for the interested}
%
% \paragraph{Unpacking with \LaTeX.}
% The \xfile{.dtx} chooses its action depending on the format:
% \begin{description}
% \item[\plainTeX:] Run \docstrip\ and extract the files.
% \item[\LaTeX:] Generate the documentation.
% \end{description}
% If you insist on using \LaTeX\ for \docstrip\ (really,
% \docstrip\ does not need \LaTeX), then inform the autodetect routine
% about your intention:
% \begin{quote}
%   \verb|latex \let\install=y\input{kvsetkeys.dtx}|
% \end{quote}
% Do not forget to quote the argument according to the demands
% of your shell.
%
% \paragraph{Generating the documentation.}
% You can use both the \xfile{.dtx} or the \xfile{.drv} to generate
% the documentation. The process can be configured by the
% configuration file \xfile{ltxdoc.cfg}. For instance, put this
% line into this file, if you want to have A4 as paper format:
% \begin{quote}
%   \verb|\PassOptionsToClass{a4paper}{article}|
% \end{quote}
% An example follows how to generate the
% documentation with pdf\LaTeX:
% \begin{quote}
%\begin{verbatim}
%pdflatex kvsetkeys.dtx
%makeindex -s gind.ist kvsetkeys.idx
%pdflatex kvsetkeys.dtx
%makeindex -s gind.ist kvsetkeys.idx
%pdflatex kvsetkeys.dtx
%\end{verbatim}
% \end{quote}
%
% \begin{thebibliography}{9}
% \bibitem{tb94wright}
%   A guide to key-value methods, Joseph Wright, second draft for
%   \href{https://www.tug.org/tugboat}{TUGBoat}, 2009-03-17.
%   \url{https://www.texdev.net/uploads/2009/03/keyval.pdf}
%
% \bibitem{keyval}
%   David Carlisle:
%   \textit{The \xpackage{keyval} package};
%   1999/03/16 v1.13;
%   \CTANpkg{keyval}.
%
% \end{thebibliography}
%
% \begin{History}
%   \begin{Version}{2006/03/06 v1.0}
%   \item
%     First version.
%   \end{Version}
%   \begin{Version}{2006/10/19 v1.1}
%   \item
%     Fix of \cs{kv@set@family@handler}.
%   \item
%     Example added.
%   \end{Version}
%   \begin{Version}{2007/09/09 v1.2}
%   \item
%     Using package \xpackage{infwarerr} for error messages.
%   \item
%     Catcode section rewritten.
%   \end{Version}
%   \begin{Version}{2007/09/29 v1.3}
%   \item
%     Normalizing and parsing of comma separated lists added.
%   \item
%     \cs{kv@normalize} rewritten.
%   \item
%     Robustness increased for normalizing and parsing,
%     e.g. for values with unmatched conditionals.
%   \item
%     \eTeX\ is used if available.
%   \item
%     Tests added for normalizing and parsing.
%   \end{Version}
%   \begin{Version}{2009/07/19 v1.4}
%   \item
%     Bug fix for \cs{kv@normalize}: unwanted space removed (Florent Chervet).
%   \end{Version}
%   \begin{Version}{2009/07/30 v1.5}
%   \item
%     Documentation addition: recommendation for Joseph Wright's
%     review article.
%   \end{Version}
%   \begin{Version}{2009/12/12 v1.6}
%   \item
%     Short info shortened.
%   \end{Version}
%   \begin{Version}{2009/12/22 v1.7}
%   \item
%     Internal optimization (\cs{KVS@CommaSpace}, \dots, \cs{KVS@EqualsSpace}).
%   \end{Version}
%   \begin{Version}{2010/01/28 v1.8}
%   \item
%     Compatibility to ini\TeX\ added.
%   \end{Version}
%   \begin{Version}{2010/03/01 v1.9}
%   \item
%     Support of \cs{par} inside values.
%   \end{Version}
%   \begin{Version}{2011/01/30 v1.10}
%   \item
%     Already loaded package files are not input in \hologo{plainTeX}.
%   \end{Version}
%   \begin{Version}{2011/03/03 v1.11}
%   \item
%     \cs{kv@break} and \cs{comma@break} added.
%   \end{Version}
%   \begin{Version}{2011/04/05 v1.12}
%   \item
%     Error message with recovery action in help message
%     (request by GL).
%   \end{Version}
%   \begin{Version}{2011/04/07 v1.13}
%   \item
%     \cs{kv@processor@default} supports package \xpackage{babel}'s
%     shorthands.
%   \item
%     \cs{kv@set@family@handler} with shorthand support.
%   \end{Version}
%   \begin{Version}{2011/06/15 v1.14}
%   \item
%     Some optimizations in token register uses (GL, HO).
%   \end{Version}
%   \begin{Version}{2011/10/18 v1.15}
%   \item
%     \cs{kv@processor@known} and \cs{kvsetknownkeys} added.
%   \item
%     \cs{kvsetkeys@expandafter} and \cs{kvsetknownkeys@expandafter} added.
%   \item
%     Family handler can report success or failure by \cs{kv@handled@true}
%     or \cs{kv@handled@false}.
%   \item
%     \cs{kv@unset@family@handler} added.
%   \end{Version}
%   \begin{Version}{2012/04/25 v1.16}
%   \item
%     \cs{kv@processor@default} and \cs{kv@processor@known} define
%     macro \cs{kv@fam} for convenience.
%   \item
%     Catcode section: Catcode setting for \texttt{+} added for \hologo{eTeX}.
%   \end{Version}
%   \begin{Version}{2016/05/16 v1.17}
%   \item
%     Documentation updates.
%   \end{Version}
%   \begin{Version}{2019/12/15 v1.18}
%   \item
%     Documentation updates.
%   \item
%     Avoid \xpackage{etexcmds} and \xpackage{infwarwerr} in \LaTeX.
%   \end{Version}
% \end{History}
%
% \PrintIndex
%
% \Finale
\endinput
|
% \end{quote}
% Do not forget to quote the argument according to the demands
% of your shell.
%
% \paragraph{Generating the documentation.}
% You can use both the \xfile{.dtx} or the \xfile{.drv} to generate
% the documentation. The process can be configured by the
% configuration file \xfile{ltxdoc.cfg}. For instance, put this
% line into this file, if you want to have A4 as paper format:
% \begin{quote}
%   \verb|\PassOptionsToClass{a4paper}{article}|
% \end{quote}
% An example follows how to generate the
% documentation with pdf\LaTeX:
% \begin{quote}
%\begin{verbatim}
%pdflatex kvsetkeys.dtx
%makeindex -s gind.ist kvsetkeys.idx
%pdflatex kvsetkeys.dtx
%makeindex -s gind.ist kvsetkeys.idx
%pdflatex kvsetkeys.dtx
%\end{verbatim}
% \end{quote}
%
% \begin{thebibliography}{9}
% \bibitem{tb94wright}
%   A guide to key-value methods, Joseph Wright, second draft for
%   \href{https://www.tug.org/tugboat}{TUGBoat}, 2009-03-17.
%   \url{https://www.texdev.net/uploads/2009/03/keyval.pdf}
%
% \bibitem{keyval}
%   David Carlisle:
%   \textit{The \xpackage{keyval} package};
%   1999/03/16 v1.13;
%   \CTANpkg{keyval}.
%
% \end{thebibliography}
%
% \begin{History}
%   \begin{Version}{2006/03/06 v1.0}
%   \item
%     First version.
%   \end{Version}
%   \begin{Version}{2006/10/19 v1.1}
%   \item
%     Fix of \cs{kv@set@family@handler}.
%   \item
%     Example added.
%   \end{Version}
%   \begin{Version}{2007/09/09 v1.2}
%   \item
%     Using package \xpackage{infwarerr} for error messages.
%   \item
%     Catcode section rewritten.
%   \end{Version}
%   \begin{Version}{2007/09/29 v1.3}
%   \item
%     Normalizing and parsing of comma separated lists added.
%   \item
%     \cs{kv@normalize} rewritten.
%   \item
%     Robustness increased for normalizing and parsing,
%     e.g. for values with unmatched conditionals.
%   \item
%     \eTeX\ is used if available.
%   \item
%     Tests added for normalizing and parsing.
%   \end{Version}
%   \begin{Version}{2009/07/19 v1.4}
%   \item
%     Bug fix for \cs{kv@normalize}: unwanted space removed (Florent Chervet).
%   \end{Version}
%   \begin{Version}{2009/07/30 v1.5}
%   \item
%     Documentation addition: recommendation for Joseph Wright's
%     review article.
%   \end{Version}
%   \begin{Version}{2009/12/12 v1.6}
%   \item
%     Short info shortened.
%   \end{Version}
%   \begin{Version}{2009/12/22 v1.7}
%   \item
%     Internal optimization (\cs{KVS@CommaSpace}, \dots, \cs{KVS@EqualsSpace}).
%   \end{Version}
%   \begin{Version}{2010/01/28 v1.8}
%   \item
%     Compatibility to ini\TeX\ added.
%   \end{Version}
%   \begin{Version}{2010/03/01 v1.9}
%   \item
%     Support of \cs{par} inside values.
%   \end{Version}
%   \begin{Version}{2011/01/30 v1.10}
%   \item
%     Already loaded package files are not input in \hologo{plainTeX}.
%   \end{Version}
%   \begin{Version}{2011/03/03 v1.11}
%   \item
%     \cs{kv@break} and \cs{comma@break} added.
%   \end{Version}
%   \begin{Version}{2011/04/05 v1.12}
%   \item
%     Error message with recovery action in help message
%     (request by GL).
%   \end{Version}
%   \begin{Version}{2011/04/07 v1.13}
%   \item
%     \cs{kv@processor@default} supports package \xpackage{babel}'s
%     shorthands.
%   \item
%     \cs{kv@set@family@handler} with shorthand support.
%   \end{Version}
%   \begin{Version}{2011/06/15 v1.14}
%   \item
%     Some optimizations in token register uses (GL, HO).
%   \end{Version}
%   \begin{Version}{2011/10/18 v1.15}
%   \item
%     \cs{kv@processor@known} and \cs{kvsetknownkeys} added.
%   \item
%     \cs{kvsetkeys@expandafter} and \cs{kvsetknownkeys@expandafter} added.
%   \item
%     Family handler can report success or failure by \cs{kv@handled@true}
%     or \cs{kv@handled@false}.
%   \item
%     \cs{kv@unset@family@handler} added.
%   \end{Version}
%   \begin{Version}{2012/04/25 v1.16}
%   \item
%     \cs{kv@processor@default} and \cs{kv@processor@known} define
%     macro \cs{kv@fam} for convenience.
%   \item
%     Catcode section: Catcode setting for \texttt{+} added for \hologo{eTeX}.
%   \end{Version}
%   \begin{Version}{2016/05/16 v1.17}
%   \item
%     Documentation updates.
%   \end{Version}
%   \begin{Version}{2019/12/15 v1.18}
%   \item
%     Documentation updates.
%   \item
%     Avoid \xpackage{etexcmds} and \xpackage{infwarwerr} in \LaTeX.
%   \end{Version}
% \end{History}
%
% \PrintIndex
%
% \Finale
\endinput
|
% \end{quote}
% Do not forget to quote the argument according to the demands
% of your shell.
%
% \paragraph{Generating the documentation.}
% You can use both the \xfile{.dtx} or the \xfile{.drv} to generate
% the documentation. The process can be configured by the
% configuration file \xfile{ltxdoc.cfg}. For instance, put this
% line into this file, if you want to have A4 as paper format:
% \begin{quote}
%   \verb|\PassOptionsToClass{a4paper}{article}|
% \end{quote}
% An example follows how to generate the
% documentation with pdf\LaTeX:
% \begin{quote}
%\begin{verbatim}
%pdflatex kvsetkeys.dtx
%makeindex -s gind.ist kvsetkeys.idx
%pdflatex kvsetkeys.dtx
%makeindex -s gind.ist kvsetkeys.idx
%pdflatex kvsetkeys.dtx
%\end{verbatim}
% \end{quote}
%
% \begin{thebibliography}{9}
% \bibitem{tb94wright}
%   A guide to key-value methods, Joseph Wright, second draft for
%   \href{https://www.tug.org/tugboat}{TUGBoat}, 2009-03-17.
%   \url{https://www.texdev.net/uploads/2009/03/keyval.pdf}
%
% \bibitem{keyval}
%   David Carlisle:
%   \textit{The \xpackage{keyval} package};
%   1999/03/16 v1.13;
%   \CTANpkg{keyval}.
%
% \end{thebibliography}
%
% \begin{History}
%   \begin{Version}{2006/03/06 v1.0}
%   \item
%     First version.
%   \end{Version}
%   \begin{Version}{2006/10/19 v1.1}
%   \item
%     Fix of \cs{kv@set@family@handler}.
%   \item
%     Example added.
%   \end{Version}
%   \begin{Version}{2007/09/09 v1.2}
%   \item
%     Using package \xpackage{infwarerr} for error messages.
%   \item
%     Catcode section rewritten.
%   \end{Version}
%   \begin{Version}{2007/09/29 v1.3}
%   \item
%     Normalizing and parsing of comma separated lists added.
%   \item
%     \cs{kv@normalize} rewritten.
%   \item
%     Robustness increased for normalizing and parsing,
%     e.g. for values with unmatched conditionals.
%   \item
%     \eTeX\ is used if available.
%   \item
%     Tests added for normalizing and parsing.
%   \end{Version}
%   \begin{Version}{2009/07/19 v1.4}
%   \item
%     Bug fix for \cs{kv@normalize}: unwanted space removed (Florent Chervet).
%   \end{Version}
%   \begin{Version}{2009/07/30 v1.5}
%   \item
%     Documentation addition: recommendation for Joseph Wright's
%     review article.
%   \end{Version}
%   \begin{Version}{2009/12/12 v1.6}
%   \item
%     Short info shortened.
%   \end{Version}
%   \begin{Version}{2009/12/22 v1.7}
%   \item
%     Internal optimization (\cs{KVS@CommaSpace}, \dots, \cs{KVS@EqualsSpace}).
%   \end{Version}
%   \begin{Version}{2010/01/28 v1.8}
%   \item
%     Compatibility to ini\TeX\ added.
%   \end{Version}
%   \begin{Version}{2010/03/01 v1.9}
%   \item
%     Support of \cs{par} inside values.
%   \end{Version}
%   \begin{Version}{2011/01/30 v1.10}
%   \item
%     Already loaded package files are not input in \hologo{plainTeX}.
%   \end{Version}
%   \begin{Version}{2011/03/03 v1.11}
%   \item
%     \cs{kv@break} and \cs{comma@break} added.
%   \end{Version}
%   \begin{Version}{2011/04/05 v1.12}
%   \item
%     Error message with recovery action in help message
%     (request by GL).
%   \end{Version}
%   \begin{Version}{2011/04/07 v1.13}
%   \item
%     \cs{kv@processor@default} supports package \xpackage{babel}'s
%     shorthands.
%   \item
%     \cs{kv@set@family@handler} with shorthand support.
%   \end{Version}
%   \begin{Version}{2011/06/15 v1.14}
%   \item
%     Some optimizations in token register uses (GL, HO).
%   \end{Version}
%   \begin{Version}{2011/10/18 v1.15}
%   \item
%     \cs{kv@processor@known} and \cs{kvsetknownkeys} added.
%   \item
%     \cs{kvsetkeys@expandafter} and \cs{kvsetknownkeys@expandafter} added.
%   \item
%     Family handler can report success or failure by \cs{kv@handled@true}
%     or \cs{kv@handled@false}.
%   \item
%     \cs{kv@unset@family@handler} added.
%   \end{Version}
%   \begin{Version}{2012/04/25 v1.16}
%   \item
%     \cs{kv@processor@default} and \cs{kv@processor@known} define
%     macro \cs{kv@fam} for convenience.
%   \item
%     Catcode section: Catcode setting for \texttt{+} added for \hologo{eTeX}.
%   \end{Version}
%   \begin{Version}{2016/05/16 v1.17}
%   \item
%     Documentation updates.
%   \end{Version}
%   \begin{Version}{2019/12/15 v1.18}
%   \item
%     Documentation updates.
%   \item
%     Avoid \xpackage{etexcmds} and \xpackage{infwarwerr} in \LaTeX.
%   \end{Version}
% \end{History}
%
% \PrintIndex
%
% \Finale
\endinput

%        (quote the arguments according to the demands of your shell)
%
% Documentation:
%    (a) If kvsetkeys.drv is present:
%           latex kvsetkeys.drv
%    (b) Without kvsetkeys.drv:
%           latex kvsetkeys.dtx; ...
%    The class ltxdoc loads the configuration file ltxdoc.cfg
%    if available. Here you can specify further options, e.g.
%    use A4 as paper format:
%       \PassOptionsToClass{a4paper}{article}
%
%    Programm calls to get the documentation (example):
%       pdflatex kvsetkeys.dtx
%       makeindex -s gind.ist kvsetkeys.idx
%       pdflatex kvsetkeys.dtx
%       makeindex -s gind.ist kvsetkeys.idx
%       pdflatex kvsetkeys.dtx
%
% Installation:
%    TDS:tex/generic/kvsetkeys/kvsetkeys.sty
%    TDS:doc/latex/kvsetkeys/kvsetkeys.pdf
%    TDS:doc/latex/kvsetkeys/kvsetkeys-example.tex
%    TDS:source/latex/kvsetkeys/kvsetkeys.dtx
%
%<*ignore>
\begingroup
  \catcode123=1 %
  \catcode125=2 %
  \def\x{LaTeX2e}%
\expandafter\endgroup
\ifcase 0\ifx\install y1\fi\expandafter
         \ifx\csname processbatchFile\endcsname\relax\else1\fi
         \ifx\fmtname\x\else 1\fi\relax
\else\csname fi\endcsname
%</ignore>
%<*install>
\input docstrip.tex
\Msg{************************************************************************}
\Msg{* Installation}
\Msg{* Package: kvsetkeys 2019/12/15 v1.18 Key value parser (HO)}
\Msg{************************************************************************}

\keepsilent
\askforoverwritefalse

\let\MetaPrefix\relax
\preamble

This is a generated file.

Project: kvsetkeys
Version: 2019/12/15 v1.18

Copyright (C)
   2006, 2007, 2009-2012 Heiko Oberdiek
   2016-2019 Oberdiek Package Support Group

This work may be distributed and/or modified under the
conditions of the LaTeX Project Public License, either
version 1.3c of this license or (at your option) any later
version. This version of this license is in
   https://www.latex-project.org/lppl/lppl-1-3c.txt
and the latest version of this license is in
   https://www.latex-project.org/lppl.txt
and version 1.3 or later is part of all distributions of
LaTeX version 2005/12/01 or later.

This work has the LPPL maintenance status "maintained".

The Current Maintainers of this work are
Heiko Oberdiek and the Oberdiek Package Support Group
https://github.com/ho-tex/kvsetkeys/issues


The Base Interpreter refers to any `TeX-Format',
because some files are installed in TDS:tex/generic//.

This work consists of the main source file kvsetkeys.dtx
and the derived files
   kvsetkeys.sty, kvsetkeys.pdf, kvsetkeys.ins, kvsetkeys.drv,
   kvsetkeys-example.tex, kvsetkeys-test1.tex,
   kvsetkeys-test2.tex, kvsetkeys-test3.tex,
   kvsetkeys-test4.tex.

\endpreamble
\let\MetaPrefix\DoubleperCent

\generate{%
  \file{kvsetkeys.ins}{\from{kvsetkeys.dtx}{install}}%
  \file{kvsetkeys.drv}{\from{kvsetkeys.dtx}{driver}}%
  \usedir{tex/generic/kvsetkeys}%
  \file{kvsetkeys.sty}{\from{kvsetkeys.dtx}{package}}%
  \usedir{doc/latex/kvsetkeys}%
  \file{kvsetkeys-example.tex}{\from{kvsetkeys.dtx}{example}}%
}

\catcode32=13\relax% active space
\let =\space%
\Msg{************************************************************************}
\Msg{*}
\Msg{* To finish the installation you have to move the following}
\Msg{* file into a directory searched by TeX:}
\Msg{*}
\Msg{*     kvsetkeys.sty}
\Msg{*}
\Msg{* To produce the documentation run the file `kvsetkeys.drv'}
\Msg{* through LaTeX.}
\Msg{*}
\Msg{* Happy TeXing!}
\Msg{*}
\Msg{************************************************************************}

\endbatchfile
%</install>
%<*ignore>
\fi
%</ignore>
%<*driver>
\NeedsTeXFormat{LaTeX2e}
\ProvidesFile{kvsetkeys.drv}%
  [2019/12/15 v1.18 Key value parser (HO)]%
\documentclass{ltxdoc}
\usepackage{holtxdoc}[2011/11/22]
\begin{document}
  \DocInput{kvsetkeys.dtx}%
\end{document}
%</driver>
% \fi
%
%
%
% \GetFileInfo{kvsetkeys.drv}
%
% \title{The \xpackage{kvsetkeys} package}
% \date{2019/12/15 v1.18}
% \author{Heiko Oberdiek\thanks
% {Please report any issues at \url{https://github.com/ho-tex/kvsetkeys/issues}}}
%
% \maketitle
%
% \begin{abstract}
% Package \xpackage{kvsetkeys} provides \cs{kvsetkeys}, a variant
% of package \xpackage{keyval}'s \cs{setkeys}. It allows to specify
% a handler that deals with unknown options. Active commas and equal
% signs may be used (e.g. see \xpackage{babel}'s shorthands) and
% only one level of curly braces is removed from the values.
% \end{abstract}
%
% \tableofcontents
%
% \def\M#1{\texttt{\{}\meta{#1}\texttt{\}}}
%
% \section{Documentation}
%
% First I want to recommend the very good review article
% ``A guide to key-value methods'' by Joseph Wright \cite{tb94wright}.
% It introduces the different key-value packages and compares them.
%
% \subsection{Motivation}
%
% \cs{kvsetkeys} serves as replacement for \xpackage{keyval}'s
% \cs{setkeys}. It basically uses the same syntax. But the
% implementation is more robust and predictable:
% \begin{description}
% \item[Active syntax characters:]
%   Comma `|,|' and the equals sign `|=|' are used inside
%   key value lists as syntax characters. Package \xpackage{keyval}
%   uses the catcode of the characters that is active during
%   package loading, usually this is catcode 12 (other).
%   But it can happen that the catcode setting of the syntax characters
%   changes. Especially active characters are of interest, because
%   some language adaptations uses them. For example, option \xoption{turkish}
%   of package \xpackage{babel} uses the equals sign as active shorthand
%   character. Therefore package \xpackage{kvsetkeys} deals with
%   both catcode settings 12 (other) and 13 (active).
% \item[Brace removal:]
%   Package \xpackage{keyval}'s \cs{setkeys} removes up to two
%   levels of curly braces around the value in some unpredictable way:
%\begin{quote}
%|\setkeys{fam}{key={{value}}} || || | $\rightarrow$ |value|\\
%|\setkeys{fam}{key={{{value}}}} | $\rightarrow$ |{value}|\\
%|\setkeys{fam}{key= {{{value}}}}| $\rightarrow$ |{{value}}|
%\end{quote}
%   This package \xpackage{kvsetkeys} follows a much stronger rule:
%   Exactly one level of braces are removed from an item, if the
%   item is surrounded by curly braces. An item can be a
%   the key value pair, the key or the value.
%\begin{quote}
%|\kvsetkeys{fam}{key={value}} || | $\rightarrow$ |value|\\
%|\kvsetkeys{fam}{key={{value}} | $\rightarrow$ |{value}|\\
%|\kvsetkeys{fam}{key= {{value}}| $\rightarrow$ |{value}|
%\end{quote}
% \item[Arbitrary values:] Unmatched conditionals are supported.
% \end{description}
%
% Before I describe
% \cs{kvsetkeys} in more detail, first I want to explain, how
% this package deals with key value lists. For the package also
% provides low level interfaces that can be used by package authors.
%
% \subsection{Normalizing key value lists}
%
% \begin{declcs}{kv@normalize}\,\M{key value list}
% \end{declcs}
% If the user specifies key value lists, he usually prefers
% nice formatted source code, e.g.:
% \begin{quote}
%\begin{verbatim}
%\hypersetup{
%  pdftitle    = {...},
%  pdfsubject  = {...},
%  pdfauthor   = {...},
%  pdfkeywords = {...},
%  ...
%}
%\end{verbatim}
% \end{quote}
% Thus there can be spaces around keys, around |=| or around the value.
% Also empty entries are possible by too many commas. Therefore these
% spaces and empty entries are silently removed by package \xpackage{keyval}
% and this package. Whereas the contents of the value can be protected
% by curly braces, especially if spaces or commas are used inside,
% a key name must not use spaces or other syntax characters.
%
% \cs{kv@normalize} takes a key value list and performs the cleanup:
% \begin{itemize}
% \item Spaces are removed.
% \item Syntax characters (comma and equal sign) that are active
%   are replaced by the same characters with standard catcode.
%   (Example: \xpackage{babel}'s language option \xoption{turkish}
%   uses the equal sign as active shorthand character.)
% \end{itemize}
% The result is stored in \cs{kv@list}, e.g.:
% \begin{quote}
% |\kv@list| $\rightarrow$ |,pdftitle={...},pdfsubject={...},...,|
% \end{quote}
% Curly braces around values (or keys) remain untouched.
% \begin{description}
% \item[v1.3+:]
% One comma is added in front of the list and each pair ends with
% a comma. Thus an empty list consists of one comma, otherwise
% two commas encloses the list. Empty entries other than the first
% are removed.
% \item[v1.0 -- v1.2:]
% Empty entries are removed later. In fact it adds a comma at the begin
% and end to protect the last value and an easier implementation.
% \end{description}
%
% \subsection{Parsing key value lists}
%
% \begin{declcs}{kv@parse}\,\M{key value list}\,\M{processor}
% \end{declcs}
% It is easier to parse a normalized list, thus \cs{kv@parse}
% normalizes the list and calls \cs{kv@parse@normalized}.
%
% \begin{declcs}{kv@parse@normalized}\,\M{key value list}%
%   \,\M{processor}
% \end{declcs}
% Now the key value list is split into single key value pairs.
% For further processing the key and value are given as arguments
% for the \meta{processor}:
% \begin{quote}
%   \meta{processor}\,\M{key}\,\M{value}
% \end{quote}
% Also key and value are stored in macro names:
% \begin{itemize}
% \item \cs{kv@key} stores the key.
% \item \cs{kv@value} stores the value or if the value was not
%   specified it has the meaning \cs{relax}.
% \end{itemize}
% The behaviour in pseudo code:
% \begin{quote}
%   foreach (\meta{key}, \meta{value}) in (\meta{key value list})\\
%   \hspace*{1.5em}\cs{kv@key} := \meta{key}\\
%   \hspace*{1.5em}\cs{kv@value} := \meta{value}\\
%   \hspace*{1.5em}\meta{processor}\,\M{key}\,\M{value}
% \end{quote}
%
% \begin{declcs}{kv@break}
% \end{declcs}
% Since version 2011/03/03 v1.11 \cs{kv@break} can be
% called inside the \meta{processor}
% of \cs{kv@parse} or \cs{kv@parse@normalized}, then
% the processing is stopped and the following entries discarded.
%
% \subsection{Processing key value pairs}
%
% Key value pairs can be processed in many different ways.
% For example, the processor for \cs{kvsetkeys} works similar
% to \cs{setkeys} of package \xpackage{keyval}. There unknown
% keys raise an error.
%
% Package \xpackage{xkeyval} also knows a star form of \cs{setkeys}
% that stores unknown keys in an internal macro for further processing
% with \cs{setrmkeys} and similar macros. This feature is covered
% by processor \cs{kv@processor@known}.
%
% \subsubsection{Processing similar to \xpackage{keyval}}
%
% \begin{declcs}{kv@processor@default}\,\M{family}\,\M{key}\,\M{value}
% \end{declcs}
% There are many possiblities to process key value pairs.
% \cs{kv@processor@default} is the processor used in \cs{kvsetkeys}.
% It reimplements and extends the behaviour of
% \xpackage{keyval}'s \cs{setkeys}.
% In case of unknown keys \cs{setkeys} raise an error.
% This processer, however, calls a handler instead, if it
% is provided by the family. Both \meta{family} and \meta{key}
% may contain package \xpackage{babel}'s shorthands
% (since 2011/04/07 v1.13).
%
% Since 2011/10/18 v1.15 the family handler can reject the
% successful handling of a key by calling \cs{kv@handled@false}.
%
% Since 2019/12/15 v1.18 \cs{kv@processor@default} also defines
% macro \cs{kv@fam} with meaning \meta{family} for convenience.
%
% \subsubsection{Processing similar to \cs{setkeys*} of package \xpackage{xkeyval}}
%
% \begin{declcs}{kv@processor@known}\,\M{family}\,^^A
%   \M{cmd}\,\M{key}\,\M{value}
% \end{declcs}
% The key value processor \cs{kv@processor@known} behaves similar
% to \cs{kv@processor@default}. If the \meta{key} exists in the
% \meta{family} its code is called, otherwise the family handler
% is tried. If the family handler is not set or cannot handle the
% key, the unknown key value pair is added to the macro \meta{cmd}.
% Since 2011/10/18 v1.15.
%
% The behaviour in pseudo code:
% \begin{quote}
%   if \meta{key} exists\\
%   \hspace*{1.5em}call the keyval code of \meta{key}\\
%   else\\
%   \hspace*{1.5em}if \meta{handler} for \meta{family} exists\\
%   \hspace*{3em}handled = true\\
%   \hspace*{3em}\meta{handler}\,\M{key}\,\M{value}\\
%   \hspace*{3em}if handled\\
%   \hspace*{3em}else\\
%   \hspace*{4.5em}add \texttt{"}\M{key}\texttt{=}\M{value}\texttt{"}
%     to \M{cmd}\\
%   \hspace*{3em}fi\\
%   \hspace*{1.5em}else\\
%   \hspace*{3em}add \texttt{"}\M{key}\texttt{=}\M{value}\texttt{"}
%     to \M{cmd}\\
%   \hspace*{3em}raise unknown key error\\
%   \hspace*{1.5em}fi\\
%   fi
% \end{quote}
%
% Since 2019/12/15 v1.18 \cs{kv@processor@known} also defines
% macro \cs{kv@fam} with meaning \meta{family} for convenience.
%
% \subsection{Default family handler}
%
% \cs{kv@processor@default} calls \meta{handler}, the default
% handler for the family, if the key does not exist in the family.
% The handler is called with two arguments, the key and the value.
% It can be defined with \cs{kv@set@family@hander}:
%
% \begin{declcs}{kv@set@family@handler}\,\M{family}\,\M{handler definition}
% \end{declcs}
% This sets the default family handler for the keyval family
% \meta{family}. Inside \meta{handler definition} |#1| stands for
% the key and |#2| is the value. Also \cs{kv@key} and \cs{kv@value}
% can be used for the key and the value. If the value is not given,
% \cs{kv@value} has the meaning \cs{relax}.
%
% \begin{declcs}{kv@unset@family@handler}\,\M{family}
% \end{declcs}
% It removes the family handler for \meta{family}.
% Since 2011/10/18 v1.15.
%
% \subsection{Put it all together}
%
% \begin{declcs}{kvsetkeys}\,\M{family}\,\M{key value list}
% \end{declcs}
% Macro \cs{kvsetkeys} processes the \meta{key value list} with
% the standard processor \cs{kv@processor@default}:
% \begin{quote}
%   \cs{kv@parse}\,\M{key value list}^^A
%   \texttt{\{}\cs{kv@processor@default}\,\M{family}\texttt{\}}
% \end{quote}
%
% \begin{declcs}{kvsetknownkeys}\,\M{family}\,\M{cmd}\,\M{key value list}
% \end{declcs}
% Macro \cs{kvsetknownkeys} processes the \meta{key value list}
% with processor \cs{kv@processor@known}. All key value pairs
% with keys that are not known in \meta{family} are stored
% in macro \meta{cmd}. A previous contents of macro \meta{cmd}
% will be overwritten. If all keys can be handled, \meta{cmd}
% will be empty, otherwise it contains a key value list of
% unhandled key value pairs.
% Since 2011/10/18 v1.15.
%
% Pseudo code:
% \begin{quote}
%   create macro \meta{cmdaux} with unique name (inside the current group)\\
%   \cs{def}\meta{cmdaux}\{\}\\
%   \cs{kv@parse}\,\M{key value list}^^A
%   \texttt{\{}\cs{kv@processor@known}\,\M{family}\,\M{cmdaux}\texttt{\}}\\
%   \cs{let}\meta{cmd}=\meta{cmdaux}
% \end{quote}
%
% \begin{declcs}{kvsetkeys@expandafter}\,\M{family}\,\M{list cmd}\\
%   \cs{kvsetknownkeys@expandafter}\,\M{family}\,%
%   \M{cmd}\,\M{list cmd}
% \end{declcs}
% Both macros behave like the counterparts without suffix
% |@expandafter|. The difference is that the key value list is
% given as macro that is expanded once.
% Since 2011/10/18 v1.15.
%
% Thus you can replace \cs{setkeys} of package \xpackage{keyval}
% by the key value parser of this package:
% \begin{quote}
%   |\renewcommand*{\setkeys}{\kvsetkeys}|\\
%   or\\
%   |\let\setkeys\kvsetkeys|
% \end{quote}
%
% \subsection{Comma separated lists}
%
% Since version 2007/09/29 v1.3 this package also supports the normalizing
% and parsing of general comma separated lists.
%
% \begin{declcs}{comma@normalize}\,\M{comma list}
% \end{declcs}
% Macro \cs{comma@normalize} normalizes the comma separated list,
% removes spaces around commas. The result is put in macro \cs{comma@list}.
%
% \begin{declcs}{comma@parse}\,\M{comma list}\,\M{processor}
% \end{declcs}
% Macro \cs{comma@parse} first normalizes the comma separated list
% and then parses the list by calling \cs{comma@parse@normalized}.
%
% \begin{declcs}{comma@parse@normalized}\,\M{normalized comma list}^^A
%   \,\M{processor}
% \end{declcs}
% The list is parsed. Empty entries are ignored. \meta{processor}
% is called for each non-empty entry with the entry as argument:
% \begin{quote}
%   \meta{processor}|{|\meta{entry}|}|
% \end{quote}
% Also the entry is stored in the macro \cs{comma@entry}.
%
% \begin{declcs}{comma@break}
% \end{declcs}
% Since version 2011/03/03 v1.11 \cs{comma@break} can be
% called inside the \meta{processor}
% of \cs{comma@parse} or \cs{comma@parse@normalized}, then
% the processing is stopped and the following entries discarded.
%
% \section{Example}
%
% The following example prints a short piece of HTML code using
% the tabbing environment for indenting purpose and a key value
% syntax for specifying the attributes of an HTML tag.
% The example illustrates the use of a default family handler.
%    \begin{macrocode}
%<*example>
\documentclass{article}
\usepackage[T1]{fontenc}
\usepackage{kvsetkeys}
\usepackage{keyval}

\makeatletter
\newcommand*{\tag}[2][]{%
  % #1: attributes
  % #2: tag name
  \begingroup
    \toks@={}%
    \let\@endslash\@empty
    \kvsetkeys{tag}{#1}%
    \texttt{%
      \textless #2\the\toks@\@endslash\textgreater
    }%
  \endgroup
}
\kv@set@family@handler{tag}{%
  % #1: key
  % #2: value
  \toks@\expandafter{%
    \the\toks@
    \space
    #1=\string"#2\string"%
  }%
}
\define@key{tag}{/}[]{%
  \def\@endslash{/}%
}
\makeatother

\begin{document}
\begin{tabbing}
  \mbox{}\qquad\=\qquad\=\kill
  \tag{html}\\
  \>\dots\\
  \>\tag[border=1]{table}\\
  \>\>\tag[width=200, span=3, /]{colgroup}\\
  \>\>\dots\\
  \>\tag{/table}\\
  \>\dots\\
  \tag{/html}\\
\end{tabbing}
\end{document}
%</example>
%    \end{macrocode}
%
% \StopEventually{
% }
%
% \section{Implementation}
%
% \subsection{Identification}
%
%    \begin{macrocode}
%<*package>
%    \end{macrocode}
%    Reload check, especially if the package is not used with \LaTeX.
%    \begin{macrocode}
\begingroup\catcode61\catcode48\catcode32=10\relax%
  \catcode13=5 % ^^M
  \endlinechar=13 %
  \catcode35=6 % #
  \catcode39=12 % '
  \catcode44=12 % ,
  \catcode45=12 % -
  \catcode46=12 % .
  \catcode58=12 % :
  \catcode64=11 % @
  \catcode123=1 % {
  \catcode125=2 % }
  \expandafter\let\expandafter\x\csname ver@kvsetkeys.sty\endcsname
  \ifx\x\relax % plain-TeX, first loading
  \else
    \def\empty{}%
    \ifx\x\empty % LaTeX, first loading,
      % variable is initialized, but \ProvidesPackage not yet seen
    \else
      \expandafter\ifx\csname PackageInfo\endcsname\relax
        \def\x#1#2{%
          \immediate\write-1{Package #1 Info: #2.}%
        }%
      \else
        \def\x#1#2{\PackageInfo{#1}{#2, stopped}}%
      \fi
      \x{kvsetkeys}{The package is already loaded}%
      \aftergroup\endinput
    \fi
  \fi
\endgroup%
%    \end{macrocode}
%    Package identification:
%    \begin{macrocode}
\begingroup\catcode61\catcode48\catcode32=10\relax%
  \catcode13=5 % ^^M
  \endlinechar=13 %
  \catcode35=6 % #
  \catcode39=12 % '
  \catcode40=12 % (
  \catcode41=12 % )
  \catcode44=12 % ,
  \catcode45=12 % -
  \catcode46=12 % .
  \catcode47=12 % /
  \catcode58=12 % :
  \catcode64=11 % @
  \catcode91=12 % [
  \catcode93=12 % ]
  \catcode123=1 % {
  \catcode125=2 % }
  \expandafter\ifx\csname ProvidesPackage\endcsname\relax
    \def\x#1#2#3[#4]{\endgroup
      \immediate\write-1{Package: #3 #4}%
      \xdef#1{#4}%
    }%
  \else
    \def\x#1#2[#3]{\endgroup
      #2[{#3}]%
      \ifx#1\@undefined
        \xdef#1{#3}%
      \fi
      \ifx#1\relax
        \xdef#1{#3}%
      \fi
    }%
  \fi
\expandafter\x\csname ver@kvsetkeys.sty\endcsname
\ProvidesPackage{kvsetkeys}%
  [2019/12/15 v1.18 Key value parser (HO)]%
%    \end{macrocode}
%
%    \begin{macrocode}
\begingroup\catcode61\catcode48\catcode32=10\relax%
  \catcode13=5 % ^^M
  \endlinechar=13 %
  \catcode123=1 % {
  \catcode125=2 % }
  \catcode64=11 % @
  \def\x{\endgroup
    \expandafter\edef\csname KVS@AtEnd\endcsname{%
      \endlinechar=\the\endlinechar\relax
      \catcode13=\the\catcode13\relax
      \catcode32=\the\catcode32\relax
      \catcode35=\the\catcode35\relax
      \catcode61=\the\catcode61\relax
      \catcode64=\the\catcode64\relax
      \catcode123=\the\catcode123\relax
      \catcode125=\the\catcode125\relax
    }%
  }%
\x\catcode61\catcode48\catcode32=10\relax%
\catcode13=5 % ^^M
\endlinechar=13 %
\catcode35=6 % #
\catcode64=11 % @
\catcode123=1 % {
\catcode125=2 % }
\def\TMP@EnsureCode#1#2{%
  \edef\KVS@AtEnd{%
    \KVS@AtEnd
    \catcode#1=\the\catcode#1\relax
  }%
  \catcode#1=#2\relax
}
\TMP@EnsureCode{36}{3}% $
\TMP@EnsureCode{38}{4}% &
\TMP@EnsureCode{39}{12}% '
\TMP@EnsureCode{43}{12}% +
\TMP@EnsureCode{44}{12}% ,
\TMP@EnsureCode{45}{12}% -
\TMP@EnsureCode{46}{12}% .
\TMP@EnsureCode{47}{12}% /
\TMP@EnsureCode{91}{12}% [
\TMP@EnsureCode{93}{12}% ]
\TMP@EnsureCode{94}{7}% ^ (superscript)
\TMP@EnsureCode{96}{12}% `
\TMP@EnsureCode{126}{13}% ~ (active)
\edef\KVS@AtEnd{\KVS@AtEnd\noexpand\endinput}
%    \end{macrocode}
%
% \subsection{Package loading}
%
%    \begin{macrocode}
\begingroup\expandafter\expandafter\expandafter\endgroup
\expandafter\ifx\csname RequirePackage\endcsname\relax
  \def\TMP@RequirePackage#1[#2]{%
    \begingroup\expandafter\expandafter\expandafter\endgroup
    \expandafter\ifx\csname ver@#1.sty\endcsname\relax
      \input #1.sty\relax
    \fi
  }%
  \TMP@RequirePackage{infwarerr}[2007/09/09]%
  \let\PackageError\@PackageError
\else
\fi
%    \end{macrocode}
%
%    \begin{macrocode}
\expandafter\ifx\csname toks@\endcsname\relax
  \toksdef\toks@=0 %
\fi
%    \end{macrocode}
%
% \subsection{Check for \eTeX}
%
%    \cs{unexpanded}, \cs{ifcsname}, and \cs{unless} are used if found.
%    \begin{macrocode}
\ifx\numexpr\@undefined
  \catcode`\$=9 % ignore
  \catcode`\&=14 % comment
\else % e-TeX
  \catcode`\$=14 % comment
  \catcode`\&=9  % ignore
\fi
%    \end{macrocode}
%
% \subsection{Generic help macros}
%
%    \begin{macro}{\KVS@Empty}
%    \begin{macrocode}
\def\KVS@Empty{}
%    \end{macrocode}
%    \end{macro}
%    \begin{macro}{\KVS@FirstOfTwo}
%    \begin{macrocode}
\long\def\KVS@FirstOfTwo#1#2{#1}
%    \end{macrocode}
%    \end{macro}
%    \begin{macro}{\KVS@SecondOfTwo}
%    \begin{macrocode}
\long\def\KVS@SecondOfTwo#1#2{#2}
%    \end{macrocode}
%    \end{macro}
%
%    \begin{macro}{\KVS@IfEmpty}
%    \begin{macrocode}
\long\def\KVS@IfEmpty#1{%
& \edef\KVS@Temp{\unexpanded{#1}}%
$ \begingroup
$   \toks@{#1}%
$   \edef\KVS@Temp{\the\toks@}%
$ \expandafter\endgroup
  \ifx\KVS@Temp\KVS@Empty
    \expandafter\KVS@FirstOfTwo
  \else
    \expandafter\KVS@SecondOfTwo
  \fi
}
%    \end{macrocode}
%    \end{macro}
%
% \subsection{Normalizing}
%
%    \begin{macro}{\kv@normalize}
%    \begin{macrocode}
\long\def\kv@normalize#1{%
  \begingroup
    \toks@{,#1,}%
    \KVS@Comma
    \KVS@SpaceComma
    \KVS@CommaSpace
    \KVS@CommaComma
    \KVS@Equals
    \KVS@SpaceEquals
    \KVS@EqualsSpace
    \xdef\KVS@Global{\the\toks@}%
  \endgroup
  \let\kv@list\KVS@Global
}
%    \end{macrocode}
%    \end{macro}
%    \begin{macro}{\comma@normalize}
%    \begin{macrocode}
\def\comma@normalize#1{%
  \begingroup
    \toks@{,#1,}%
    \KVS@Comma
    \KVS@SpaceComma
    \KVS@CommaSpace
    \KVS@CommaComma
    \xdef\KVS@Global{\the\toks@}%
  \endgroup
  \let\comma@list\KVS@Global
}
%    \end{macrocode}
%    \end{macro}
%    \begin{macro}{\KVS@Comma}
%    Converts active commas into comma with catcode other.
%    Also adds a comma at the end to protect the last value
%    for next cleanup steps.
%    \begin{macrocode}
\begingroup
  \lccode`\,=`\,%
  \lccode`\~=`\,%
\lowercase{\endgroup
  \def\KVS@Comma{%
    \toks@\expandafter{\expandafter}\expandafter
    \KVS@@Comma\the\toks@~\KVS@Nil
  }%
  \long\def\KVS@@Comma#1~#2\KVS@Nil{%
    \toks@\expandafter{\the\toks@#1}%
    \KVS@IfEmpty{#2}{%
    }{%
      \KVS@@Comma,#2\KVS@Nil
    }%
  }%
}
%    \end{macrocode}
%    \end{macro}
%    \begin{macro}{\KVS@SpaceComma}
%    Removes spaces before the comma, may add commas at the end.
%    \begin{macrocode}
\def\KVS@SpaceComma#1{%
  \def\KVS@SpaceComma{%
    \expandafter\KVS@@SpaceComma\the\toks@#1,\KVS@Nil
  }%
}
\KVS@SpaceComma{ }
%    \end{macrocode}
%    \end{macro}
%    \begin{macro}{\KVS@@SpaceComma}
%    \begin{macrocode}
\long\def\KVS@@SpaceComma#1 ,#2\KVS@Nil{%
  \KVS@IfEmpty{#2}{%
    \toks@{#1}%
  }{%
    \KVS@@SpaceComma#1,#2\KVS@Nil
  }%
}
%    \end{macrocode}
%    \end{macro}
%    \begin{macro}{\KVS@CommaSpace}
%    Removes spaces after the comma, may add commas at the end.
%    \begin{macrocode}
\def\KVS@CommaSpace{%
  \expandafter\KVS@@CommaSpace\the\toks@, \KVS@Nil
}
%    \end{macrocode}
%    \end{macro}
%    \begin{macro}{\KVS@@CommaSpace}
%    \begin{macrocode}
\long\def\KVS@@CommaSpace#1, #2\KVS@Nil{%
  \KVS@IfEmpty{#2}{%
    \toks@{#1}%
  }{%
    \KVS@@CommaSpace#1,#2\KVS@Nil
  }%
}
%    \end{macrocode}
%    \end{macro}
%    \begin{macro}{\KVS@CommaComma}
%    Replaces multiple commas by one comma.
%    \begin{macrocode}
\def\KVS@CommaComma{%
  \expandafter\KVS@@CommaComma\the\toks@,\KVS@Nil
}
%    \end{macrocode}
%    \end{macro}
%    \begin{macro}{\KVS@@CommaComma}
%    \begin{macrocode}
\long\def\KVS@@CommaComma#1,,#2\KVS@Nil{%
  \KVS@IfEmpty{#2}{%
    \toks@{#1,}% (!)
  }{%
    \KVS@@CommaComma#1,#2\KVS@Nil
  }%
}
%    \end{macrocode}
%    \end{macro}
%
%    \begin{macro}{\KVS@Equals}
%    Converts active equals signs into catcode other characters.
%    \begin{macrocode}
\begingroup
  \lccode`\==`\=%
  \lccode`\~=`\=%
\lowercase{\endgroup
  \def\KVS@Equals{%
    \toks@\expandafter{\expandafter}\expandafter
    \KVS@@Equals\the\toks@~\KVS@Nil
  }%
  \long\def\KVS@@Equals#1~#2\KVS@Nil{%
    \edef\KVS@Temp{\the\toks@}%
    \ifx\KVS@Temp\KVS@Empty
      \expandafter\KVS@FirstOfTwo
    \else
      \expandafter\KVS@SecondOfTwo
    \fi
    {%
      \toks@{#1}%
    }{%
      \toks@\expandafter{\the\toks@=#1}%
    }%
    \KVS@IfEmpty{#2}{%
    }{%
      \KVS@@Equals#2\KVS@Nil
    }%
  }%
}
%    \end{macrocode}
%    \end{macro}
%    \begin{macro}{\KVS@SpaceEquals}
%    Removes spaces before the equals sign.
%    \begin{macrocode}
\def\KVS@SpaceEquals#1{%
  \def\KVS@SpaceEquals{%
    \expandafter\KVS@@SpaceEquals\the\toks@#1=\KVS@Nil
  }%
}
\KVS@SpaceEquals{ }
%    \end{macrocode}
%    \end{macro}
%    \begin{macro}{\KVS@@SpaceEquals}
%    \begin{macrocode}
\long\def\KVS@@SpaceEquals#1 =#2\KVS@Nil{%
  \KVS@IfEmpty{#2}{%
    \toks@{#1}%
  }{%
    \KVS@@SpaceEquals#1=#2\KVS@Nil
  }%
}
%    \end{macrocode}
%    \end{macro}
%    \begin{macro}{\KVS@EqualsSpace}
%    Removes spaces after the equals sign.
%    \begin{macrocode}
\def\KVS@EqualsSpace{%
  \expandafter\KVS@@EqualsSpace\the\toks@= \KVS@Nil
}
%    \end{macrocode}
%    \end{macro}
%    \begin{macro}{\KVS@@EqualsSpace}
%    \begin{macrocode}
\long\def\KVS@@EqualsSpace#1= #2\KVS@Nil{%
  \KVS@IfEmpty{#2}{%
    \toks@{#1}%
  }{%
    \KVS@@EqualsSpace#1=#2\KVS@Nil
  }%
}
%    \end{macrocode}
%    \end{macro}
%
% \subsection{Parsing key value lists}
%
%    \begin{macro}{\kv@parse}
%    Normalizes and parses the key value list. Also sets \cs{kv@list}.
%    \begin{macrocode}
\long\def\kv@parse#1{%
  \kv@normalize{#1}%
  \expandafter\kv@parse@normalized\expandafter{\kv@list}%
}
%    \end{macrocode}
%    \end{macro}
%    \begin{macro}{\kv@parse@normalized}
%    |#1|: key value list\\
%    |#2|: processor
%    \begin{macrocode}
\long\def\kv@parse@normalized#1#2{%
  \KVS@Parse#1,\KVS@Nil{#2}%
}
%    \end{macrocode}
%    \end{macro}
%    \begin{macro}{\KVS@Parse}
%    |#1,#2|: key value list\\
%    |#3|: processor
%    \begin{macrocode}
\long\def\KVS@Parse#1,#2\KVS@Nil#3{%
  \KVS@IfEmpty{#1}{%
  }{%
    \KVS@Process#1=\KVS@Nil{#3}%
  }%
  \KVS@MaybeBreak
  \KVS@IfEmpty{#2}{%
  }{%
    \KVS@Parse#2\KVS@Nil{#3}%
  }%
}
%    \end{macrocode}
%    \end{macro}
%    \begin{macro}{\KVS@Process}
%    |#1|: key\\
%    |#2|: value, |=|\\
%    |#3|: processor
%    \begin{macrocode}
\long\def\KVS@Process#1=#2\KVS@Nil#3{%
  \let\KVS@MaybeBreak\relax
  \def\kv@key{#1}%
  \KVS@IfEmpty{#2}{%
    \let\kv@value\relax
    #3{#1}{}%
  }{%
    \KVS@@Process{#1}#2\KVS@Nil{#3}%
  }%
}
%    \end{macrocode}
%    \end{macro}
%    \begin{macro}{\KVS@@Process}
%    |#1|: key\\
%    |#2|: value\\
%    |#3|: processor
%    \begin{macrocode}
\long\def\KVS@@Process#1#2=\KVS@Nil#3{%
& \edef\kv@value{\unexpanded{#2}}%
$ \begingroup
$   \toks@{#2}%
$   \xdef\KVS@Global{\the\toks@}%
$ \endgroup
$ \let\kv@value\KVS@Global
  #3{#1}{#2}%
}
%    \end{macrocode}
%    \end{macro}
%    \begin{macro}{\KVS@MaybeBreak}
%    \begin{macrocode}
\let\KVS@MaybeBreak\relax
%    \end{macrocode}
%    \end{macro}
%    \begin{macro}{\KVS@break}
%    \begin{macrocode}
\def\KVS@break#1#2#3#4{%
  \let\KVS@MaybeBreak\relax
}
%    \end{macrocode}
%    \end{macro}
%    \begin{macro}{\kv@break}
%    \begin{macrocode}
\def\kv@break{%
  \let\KVS@MaybeBreak\KVS@break
}
%    \end{macrocode}
%    \end{macro}
%
% \subsection{Parsing comma lists}
%
%    \begin{macro}{\comma@parse}
%    Normalizes and parses the key value list. Also sets \cs{comma@list}.
%    \begin{macrocode}
\def\comma@parse#1{%
  \comma@normalize{#1}%
  \expandafter\comma@parse@normalized\expandafter{\comma@list}%
}
%    \end{macrocode}
%    \end{macro}
%
%    \begin{macro}{\comma@parse@normalized}
%    |#1|: comma list\\
%    |#2|: processor
%    \begin{macrocode}
\def\comma@parse@normalized#1#2{%
  \KVS@CommaParse#1,\KVS@Nil{#2}%
}
%    \end{macrocode}
%    \end{macro}
%    \begin{macro}{\KVS@CommaParse}
%    |#1,#2|: comma list\\
%    |#3|: processor
%    \begin{macrocode}
\def\KVS@CommaParse#1,#2\KVS@Nil#3{%
  \KVS@IfEmpty{#1}{%
  }{%
    \def\comma@entry{#1}%
    #3{#1}%
  }%
  \KVS@MaybeBreak
  \KVS@IfEmpty{#2}{%
  }{%
    \KVS@CommaParse#2\KVS@Nil{#3}%
  }%
}
%    \end{macrocode}
%    \end{macro}
%    \begin{macro}{\comma@break}
%    \begin{macrocode}
\def\comma@break{%
  \let\KVS@MaybeBreak\KVS@break
}
%    \end{macrocode}
%    \end{macro}
%
% \subsection{Processing key value pairs}
%
%    \begin{macro}{\kv@handled@false}
%    The handler can call \cs{kv@handled@false} or
%    \cs{kv@handled@true} so report failure or success.
%    The default is success (compatibility for versions
%    before 2011/10/18 v1.15).
%    \begin{macrocode}
\def\kv@handled@false{%
  \let\ifkv@handled@\iffalse
}
%    \end{macrocode}
%    \end{macro}
%    \begin{macro}{\kv@handled@true}
%    \begin{macrocode}
\def\kv@handled@true{%
  \let\ifkv@handled@\iftrue
}
%    \end{macrocode}
%    \end{macro}
%    \begin{macro}{\ifkv@handled@}
%    \begin{macrocode}
\kv@handled@true
%    \end{macrocode}
%    \end{macro}
%
%    \begin{macro}{\kv@processor@default}
%    \begin{macrocode}
\def\kv@processor@default#1#2{%
  \begingroup
    \csname @safe@activestrue\endcsname
    \let\ifincsname\iftrue
    \edef\KVS@temp{\endgroup
      \noexpand\KVS@ProcessorDefault{#1}{#2}%
    }%
  \KVS@temp
}
%    \end{macrocode}
%    \end{macro}
%    \begin{macro}{\KVS@ProcessorDefault}
%    \begin{macrocode}
\long\def\KVS@ProcessorDefault#1#2#3{%
  \def\kv@fam{#1}%
& \unless\ifcsname KV@#1@#2\endcsname
$ \begingroup\expandafter\expandafter\expandafter\endgroup
$ \expandafter\ifx\csname KV@#1@#2\endcsname\relax
&   \unless\ifcsname KVS@#1@handler\endcsname
$   \begingroup\expandafter\expandafter\expandafter\endgroup
$   \expandafter\ifx\csname KVS@#1@handler\endcsname\relax
      \kv@error@unknownkey{#1}{#2}%
    \else
      \kv@handled@true
      \csname KVS@#1@handler\endcsname{#2}{#3}%
      \relax
      \ifkv@handled@
      \else
        \kv@error@unknownkey{#1}{#2}%
      \fi
    \fi
  \else
    \ifx\kv@value\relax
&     \unless\ifcsname KV@#1@#2@default\endcsname
$     \begingroup\expandafter\expandafter\expandafter\endgroup
$     \expandafter\ifx\csname KV@#1@#2@default\endcsname\relax
        \kv@error@novalue{#1}{#2}%
      \else
        \csname KV@#1@#2@default\endcsname
        \relax
      \fi
    \else
      \csname KV@#1@#2\endcsname{#3}%
    \fi
  \fi
}
%    \end{macrocode}
%    \end{macro}
%
%    \begin{macro}{\kv@processor@known}
%    \begin{macrocode}
\def\kv@processor@known#1#2#3{%
  \begingroup
    \csname @safe@activestrue\endcsname
    \let\ifincsname\iftrue
    \edef\KVS@temp{\endgroup
      \noexpand\KVS@ProcessorKnown{#1}\noexpand#2{#3}%
    }%
  \KVS@temp
}
%    \end{macrocode}
%    \end{macro}
%    \begin{macro}{\KVS@ProcessorKnown}
%    \begin{macrocode}
\long\def\KVS@ProcessorKnown#1#2#3#4{%
  \def\kv@fam{#1}%
& \unless\ifcsname KV@#1@#3\endcsname
$ \begingroup\expandafter\expandafter\expandafter\endgroup
$ \expandafter\ifx\csname KV@#1@#3\endcsname\relax
&   \unless\ifcsname KVS@#1@handler\endcsname
$   \begingroup\expandafter\expandafter\expandafter\endgroup
$   \expandafter\ifx\csname KVS@#1@handler\endcsname\relax
      \KVS@AddUnhandled#2{#3}{#4}%
    \else
      \kv@handled@true
      \csname KVS@#1@handler\endcsname{#3}{#4}%
      \relax
      \ifkv@handled@
      \else
        \KVS@AddUnhandled#2{#3}{#4}%
      \fi
    \fi
  \else
    \ifx\kv@value\relax
&     \unless\ifcsname KV@#1@#2@default\endcsname
$     \begingroup\expandafter\expandafter\expandafter\endgroup
$     \expandafter\ifx\csname KV@#1@#3@default\endcsname\relax
        \kv@error@novalue{#1}{#3}%
      \else
        \csname KV@#1@#3@default\endcsname
        \relax
      \fi
    \else
      \csname KV@#1@#3\endcsname{#4}%
    \fi
  \fi
}
%    \end{macrocode}
%    \end{macro}
%    \begin{macro}{\KVS@AddUnhandled}
%    \begin{macrocode}
\long\def\KVS@AddUnhandled#1#2#3{%
& \edef#1{%
&   \ifx#1\KVS@empty
&   \else
&     \unexpanded{#1},%
&   \fi
&   \unexpanded{{#2}={#3}}%
& }%
$ \begingroup
$   \ifx#1\KVS@empty
$     \toks@{{#2}={#3}}%
$   \else
$     \toks@\expandafter{#1,{#2}={#3}}%
$   \fi
$   \xdef\KVS@Global{\the\toks@}%
$ \endgroup
$ \let#1\KVS@Global
}
%    \end{macrocode}
%    \end{macro}
%
%    \begin{macro}{\kv@set@family@handler}
%    \begin{macrocode}
\long\def\kv@set@family@handler#1#2{%
  \begingroup
    \csname @safe@activestrue\endcsname
    \let\ifincsname\iftrue
  \expandafter\endgroup
  \expandafter\def\csname KVS@#1@handler\endcsname##1##2{#2}%
}
%    \end{macrocode}
%    \end{macro}
%
%    \begin{macro}{\kv@unset@family@handler}
%    \begin{macrocode}
\long\def\kv@unset@family@handler#1#2{%
  \begingroup
    \csname @safe@activestrue\endcsname
    \let\ifincsname\iftrue
  \expandafter\endgroup
  \expandafter\let\csname KVS@#1@handler\endcsname\@UnDeFiNeD
}
%    \end{macrocode}
%    \end{macro}
%
% \subsection{Error handling}
%
%    \begin{macro}{\kv@error@novalue}
%    \begin{macrocode}
\def\kv@error@novalue{%
  \kv@error@generic{No value specified for}%
}
%    \end{macrocode}
%    \end{macro}
%    \begin{macro}{\kv@error@unknownkey}
%    \begin{macrocode}
\def\kv@error@unknownkey{%
  \kv@error@generic{Undefined}%
}
%    \end{macrocode}
%    \end{macro}
%    \begin{macro}{\kv@error@generic}
%    \begin{macrocode}
\def\kv@error@generic#1#2#3{%
  \PackageError{kvsetkeys}{%
    #1 key `#3'%
  }{%
    The keyval family of the key `#3' is `#2'.\MessageBreak
    The setting of the key is ignored because of the error.\MessageBreak
    \MessageBreak
    \@ehc
  }%
}
%    \end{macrocode}
%    \end{macro}
%
% \subsection{Do it all}
%
%    \begin{macro}{\kvsetkeys}
%    \begin{macrocode}
\long\def\kvsetkeys#1#2{%
  \kv@parse{#2}{\kv@processor@default{#1}}%
}
%    \end{macrocode}
%    \end{macro}
%
%    \begin{macro}{\kvsetkeys@expandafter}
%    \begin{macrocode}
\def\kvsetkeys@expandafter#1#2{%
  \expandafter\kv@parse\expandafter{#2}{%
    \kv@processor@default{#1}%
  }%
}
%    \end{macrocode}
%    \end{macro}
%
%    \begin{macro}{\KVS@cmd}
%    \begin{macrocode}
\def\KVS@cmd{0}%
%    \end{macrocode}
%    \end{macro}
%    \begin{macro}{\KVS@cmd@inc}
%    \begin{macrocode}
\def\KVS@cmd@inc{%
& \edef\KVS@cmd{\the\numexpr\KVS@cmd+1}%
$ \begingroup
$   \count255=\KVS@cmd\relax
$   \advance\count255 by 1\relax
$ \edef\x{\endgroup
$   \noexpand\def\noexpand\KVS@cmd{\number\count255}%
$ }%
$ \x
}
%    \end{macrocode}
%    \end{macro}
%    \begin{macro}{\KVS@cmd@dec}
%    \begin{macrocode}
\def\KVS@cmd@dec{%
& \edef\KVS@cmd{\the\numexpr\KVS@cmd-1}%
$ \begingroup
$   \count255=\KVS@cmd\relax
$   \advance\count255 by -1\relax
$ \edef\x{\endgroup
$   \noexpand\def\noexpand\KVS@cmd{\number\count255}%
$ }%
$ \x
}
%    \end{macrocode}
%    \end{macro}
%    \begin{macro}{\KVS@empty}
%    \begin{macrocode}
\def\KVS@empty{}
%    \end{macrocode}
%    \end{macro}
%
%    \begin{macro}{\kvsetknownkeys}
%    \begin{macrocode}
\def\kvsetknownkeys{%
  \expandafter
  \KVS@setknownkeys\csname KVS@cmd\KVS@cmd\endcsname{}%
}
%    \end{macrocode}
%    \end{macro}
%    \begin{macro}{\KVS@setknownkeys}
%    \begin{macrocode}
\long\def\KVS@setknownkeys#1#2#3#4#5{%
  \let#1\KVS@empty
  \KVS@cmd@inc
  #2\kv@parse#2{#5}{\kv@processor@known{#3}#1}%
  \KVS@cmd@dec
  \let#4=#1%
}
%    \end{macrocode}
%    \end{macro}
%
%    \begin{macro}{\kvsetknownkeys@expandafter}
%    \begin{macrocode}
\def\kvsetknownkeys@expandafter{%
  \expandafter
  \KVS@setknownkeys
      \csname KVS@cmd\KVS@cmd\endcsname\expandafter
}
%    \end{macrocode}
%    \end{macro}
%
%    \begin{macrocode}
\KVS@AtEnd%
%</package>
%    \end{macrocode}
%% \section{Installation}
%
% \subsection{Download}
%
% \paragraph{Package.} This package is available on
% CTAN\footnote{\CTANpkg{kvsetkeys}}:
% \begin{description}
% \item[\CTAN{macros/latex/contrib/kvsetkeys/kvsetkeys.dtx}] The source file.
% \item[\CTAN{macros/latex/contrib/kvsetkeys/kvsetkeys.pdf}] Documentation.
% \end{description}
%
%
% \paragraph{Bundle.} All the packages of the bundle `kvsetkeys'
% are also available in a TDS compliant ZIP archive. There
% the packages are already unpacked and the documentation files
% are generated. The files and directories obey the TDS standard.
% \begin{description}
% \item[\CTANinstall{install/macros/latex/contrib/kvsetkeys.tds.zip}]
% \end{description}
% \emph{TDS} refers to the standard ``A Directory Structure
% for \TeX\ Files'' (\CTANpkg{tds}). Directories
% with \xfile{texmf} in their name are usually organized this way.
%
% \subsection{Bundle installation}
%
% \paragraph{Unpacking.} Unpack the \xfile{kvsetkeys.tds.zip} in the
% TDS tree (also known as \xfile{texmf} tree) of your choice.
% Example (linux):
% \begin{quote}
%   |unzip kvsetkeys.tds.zip -d ~/texmf|
% \end{quote}
%
% \subsection{Package installation}
%
% \paragraph{Unpacking.} The \xfile{.dtx} file is a self-extracting
% \docstrip\ archive. The files are extracted by running the
% \xfile{.dtx} through \plainTeX:
% \begin{quote}
%   \verb|tex kvsetkeys.dtx|
% \end{quote}
%
% \paragraph{TDS.} Now the different files must be moved into
% the different directories in your installation TDS tree
% (also known as \xfile{texmf} tree):
% \begin{quote}
% \def\t{^^A
% \begin{tabular}{@{}>{\ttfamily}l@{ $\rightarrow$ }>{\ttfamily}l@{}}
%   kvsetkeys.sty & tex/generic/kvsetkeys/kvsetkeys.sty\\
%   kvsetkeys.pdf & doc/latex/kvsetkeys/kvsetkeys.pdf\\
%   kvsetkeys-example.tex & doc/latex/kvsetkeys/kvsetkeys-example.tex\\
%   kvsetkeys.dtx & source/latex/kvsetkeys/kvsetkeys.dtx\\
% \end{tabular}^^A
% }^^A
% \sbox0{\t}^^A
% \ifdim\wd0>\linewidth
%   \begingroup
%     \advance\linewidth by\leftmargin
%     \advance\linewidth by\rightmargin
%   \edef\x{\endgroup
%     \def\noexpand\lw{\the\linewidth}^^A
%   }\x
%   \def\lwbox{^^A
%     \leavevmode
%     \hbox to \linewidth{^^A
%       \kern-\leftmargin\relax
%       \hss
%       \usebox0
%       \hss
%       \kern-\rightmargin\relax
%     }^^A
%   }^^A
%   \ifdim\wd0>\lw
%     \sbox0{\small\t}^^A
%     \ifdim\wd0>\linewidth
%       \ifdim\wd0>\lw
%         \sbox0{\footnotesize\t}^^A
%         \ifdim\wd0>\linewidth
%           \ifdim\wd0>\lw
%             \sbox0{\scriptsize\t}^^A
%             \ifdim\wd0>\linewidth
%               \ifdim\wd0>\lw
%                 \sbox0{\tiny\t}^^A
%                 \ifdim\wd0>\linewidth
%                   \lwbox
%                 \else
%                   \usebox0
%                 \fi
%               \else
%                 \lwbox
%               \fi
%             \else
%               \usebox0
%             \fi
%           \else
%             \lwbox
%           \fi
%         \else
%           \usebox0
%         \fi
%       \else
%         \lwbox
%       \fi
%     \else
%       \usebox0
%     \fi
%   \else
%     \lwbox
%   \fi
% \else
%   \usebox0
% \fi
% \end{quote}
% If you have a \xfile{docstrip.cfg} that configures and enables \docstrip's
% TDS installing feature, then some files can already be in the right
% place, see the documentation of \docstrip.
%
% \subsection{Refresh file name databases}
%
% If your \TeX~distribution
% (\TeX\,Live, \mikTeX, \dots) relies on file name databases, you must refresh
% these. For example, \TeX\,Live\ users run \verb|texhash| or
% \verb|mktexlsr|.
%
% \subsection{Some details for the interested}
%
% \paragraph{Unpacking with \LaTeX.}
% The \xfile{.dtx} chooses its action depending on the format:
% \begin{description}
% \item[\plainTeX:] Run \docstrip\ and extract the files.
% \item[\LaTeX:] Generate the documentation.
% \end{description}
% If you insist on using \LaTeX\ for \docstrip\ (really,
% \docstrip\ does not need \LaTeX), then inform the autodetect routine
% about your intention:
% \begin{quote}
%   \verb|latex \let\install=y% \iffalse meta-comment
%
% File: kvsetkeys.dtx
% Version: 2019/12/15 v1.18
% Info: Key value parser
%
% Copyright (C)
%    2006, 2007, 2009-2012 Heiko Oberdiek
%    2016-2019 Oberdiek Package Support Group
%    https://github.com/ho-tex/kvsetkeys/issues
%
% This work may be distributed and/or modified under the
% conditions of the LaTeX Project Public License, either
% version 1.3c of this license or (at your option) any later
% version. This version of this license is in
%    https://www.latex-project.org/lppl/lppl-1-3c.txt
% and the latest version of this license is in
%    https://www.latex-project.org/lppl.txt
% and version 1.3 or later is part of all distributions of
% LaTeX version 2005/12/01 or later.
%
% This work has the LPPL maintenance status "maintained".
%
% The Current Maintainers of this work are
% Heiko Oberdiek and the Oberdiek Package Support Group
% https://github.com/ho-tex/kvsetkeys/issues
%
% The Base Interpreter refers to any `TeX-Format',
% because some files are installed in TDS:tex/generic//.
%
% This work consists of the main source file kvsetkeys.dtx
% and the derived files
%    kvsetkeys.sty, kvsetkeys.pdf, kvsetkeys.ins, kvsetkeys.drv,
%    kvsetkeys-example.tex, kvsetkeys-test1.tex,
%    kvsetkeys-test2.tex, kvsetkeys-test3.tex,
%    kvsetkeys-test4.tex.
%
% Distribution:
%    CTAN:macros/latex/contrib/kvsetkeys/kvsetkeys.dtx
%    CTAN:macros/latex/contrib/kvsetkeys/kvsetkeys.pdf
%
% Unpacking:
%    (a) If kvsetkeys.ins is present:
%           tex kvsetkeys.ins
%    (b) Without kvsetkeys.ins:
%           tex kvsetkeys.dtx
%    (c) If you insist on using LaTeX
%           latex \let\install=y% \iffalse meta-comment
%
% File: kvsetkeys.dtx
% Version: 2019/12/15 v1.18
% Info: Key value parser
%
% Copyright (C)
%    2006, 2007, 2009-2012 Heiko Oberdiek
%    2016-2019 Oberdiek Package Support Group
%    https://github.com/ho-tex/kvsetkeys/issues
%
% This work may be distributed and/or modified under the
% conditions of the LaTeX Project Public License, either
% version 1.3c of this license or (at your option) any later
% version. This version of this license is in
%    https://www.latex-project.org/lppl/lppl-1-3c.txt
% and the latest version of this license is in
%    https://www.latex-project.org/lppl.txt
% and version 1.3 or later is part of all distributions of
% LaTeX version 2005/12/01 or later.
%
% This work has the LPPL maintenance status "maintained".
%
% The Current Maintainers of this work are
% Heiko Oberdiek and the Oberdiek Package Support Group
% https://github.com/ho-tex/kvsetkeys/issues
%
% The Base Interpreter refers to any `TeX-Format',
% because some files are installed in TDS:tex/generic//.
%
% This work consists of the main source file kvsetkeys.dtx
% and the derived files
%    kvsetkeys.sty, kvsetkeys.pdf, kvsetkeys.ins, kvsetkeys.drv,
%    kvsetkeys-example.tex, kvsetkeys-test1.tex,
%    kvsetkeys-test2.tex, kvsetkeys-test3.tex,
%    kvsetkeys-test4.tex.
%
% Distribution:
%    CTAN:macros/latex/contrib/kvsetkeys/kvsetkeys.dtx
%    CTAN:macros/latex/contrib/kvsetkeys/kvsetkeys.pdf
%
% Unpacking:
%    (a) If kvsetkeys.ins is present:
%           tex kvsetkeys.ins
%    (b) Without kvsetkeys.ins:
%           tex kvsetkeys.dtx
%    (c) If you insist on using LaTeX
%           latex \let\install=y% \iffalse meta-comment
%
% File: kvsetkeys.dtx
% Version: 2019/12/15 v1.18
% Info: Key value parser
%
% Copyright (C)
%    2006, 2007, 2009-2012 Heiko Oberdiek
%    2016-2019 Oberdiek Package Support Group
%    https://github.com/ho-tex/kvsetkeys/issues
%
% This work may be distributed and/or modified under the
% conditions of the LaTeX Project Public License, either
% version 1.3c of this license or (at your option) any later
% version. This version of this license is in
%    https://www.latex-project.org/lppl/lppl-1-3c.txt
% and the latest version of this license is in
%    https://www.latex-project.org/lppl.txt
% and version 1.3 or later is part of all distributions of
% LaTeX version 2005/12/01 or later.
%
% This work has the LPPL maintenance status "maintained".
%
% The Current Maintainers of this work are
% Heiko Oberdiek and the Oberdiek Package Support Group
% https://github.com/ho-tex/kvsetkeys/issues
%
% The Base Interpreter refers to any `TeX-Format',
% because some files are installed in TDS:tex/generic//.
%
% This work consists of the main source file kvsetkeys.dtx
% and the derived files
%    kvsetkeys.sty, kvsetkeys.pdf, kvsetkeys.ins, kvsetkeys.drv,
%    kvsetkeys-example.tex, kvsetkeys-test1.tex,
%    kvsetkeys-test2.tex, kvsetkeys-test3.tex,
%    kvsetkeys-test4.tex.
%
% Distribution:
%    CTAN:macros/latex/contrib/kvsetkeys/kvsetkeys.dtx
%    CTAN:macros/latex/contrib/kvsetkeys/kvsetkeys.pdf
%
% Unpacking:
%    (a) If kvsetkeys.ins is present:
%           tex kvsetkeys.ins
%    (b) Without kvsetkeys.ins:
%           tex kvsetkeys.dtx
%    (c) If you insist on using LaTeX
%           latex \let\install=y\input{kvsetkeys.dtx}
%        (quote the arguments according to the demands of your shell)
%
% Documentation:
%    (a) If kvsetkeys.drv is present:
%           latex kvsetkeys.drv
%    (b) Without kvsetkeys.drv:
%           latex kvsetkeys.dtx; ...
%    The class ltxdoc loads the configuration file ltxdoc.cfg
%    if available. Here you can specify further options, e.g.
%    use A4 as paper format:
%       \PassOptionsToClass{a4paper}{article}
%
%    Programm calls to get the documentation (example):
%       pdflatex kvsetkeys.dtx
%       makeindex -s gind.ist kvsetkeys.idx
%       pdflatex kvsetkeys.dtx
%       makeindex -s gind.ist kvsetkeys.idx
%       pdflatex kvsetkeys.dtx
%
% Installation:
%    TDS:tex/generic/kvsetkeys/kvsetkeys.sty
%    TDS:doc/latex/kvsetkeys/kvsetkeys.pdf
%    TDS:doc/latex/kvsetkeys/kvsetkeys-example.tex
%    TDS:source/latex/kvsetkeys/kvsetkeys.dtx
%
%<*ignore>
\begingroup
  \catcode123=1 %
  \catcode125=2 %
  \def\x{LaTeX2e}%
\expandafter\endgroup
\ifcase 0\ifx\install y1\fi\expandafter
         \ifx\csname processbatchFile\endcsname\relax\else1\fi
         \ifx\fmtname\x\else 1\fi\relax
\else\csname fi\endcsname
%</ignore>
%<*install>
\input docstrip.tex
\Msg{************************************************************************}
\Msg{* Installation}
\Msg{* Package: kvsetkeys 2019/12/15 v1.18 Key value parser (HO)}
\Msg{************************************************************************}

\keepsilent
\askforoverwritefalse

\let\MetaPrefix\relax
\preamble

This is a generated file.

Project: kvsetkeys
Version: 2019/12/15 v1.18

Copyright (C)
   2006, 2007, 2009-2012 Heiko Oberdiek
   2016-2019 Oberdiek Package Support Group

This work may be distributed and/or modified under the
conditions of the LaTeX Project Public License, either
version 1.3c of this license or (at your option) any later
version. This version of this license is in
   https://www.latex-project.org/lppl/lppl-1-3c.txt
and the latest version of this license is in
   https://www.latex-project.org/lppl.txt
and version 1.3 or later is part of all distributions of
LaTeX version 2005/12/01 or later.

This work has the LPPL maintenance status "maintained".

The Current Maintainers of this work are
Heiko Oberdiek and the Oberdiek Package Support Group
https://github.com/ho-tex/kvsetkeys/issues


The Base Interpreter refers to any `TeX-Format',
because some files are installed in TDS:tex/generic//.

This work consists of the main source file kvsetkeys.dtx
and the derived files
   kvsetkeys.sty, kvsetkeys.pdf, kvsetkeys.ins, kvsetkeys.drv,
   kvsetkeys-example.tex, kvsetkeys-test1.tex,
   kvsetkeys-test2.tex, kvsetkeys-test3.tex,
   kvsetkeys-test4.tex.

\endpreamble
\let\MetaPrefix\DoubleperCent

\generate{%
  \file{kvsetkeys.ins}{\from{kvsetkeys.dtx}{install}}%
  \file{kvsetkeys.drv}{\from{kvsetkeys.dtx}{driver}}%
  \usedir{tex/generic/kvsetkeys}%
  \file{kvsetkeys.sty}{\from{kvsetkeys.dtx}{package}}%
  \usedir{doc/latex/kvsetkeys}%
  \file{kvsetkeys-example.tex}{\from{kvsetkeys.dtx}{example}}%
}

\catcode32=13\relax% active space
\let =\space%
\Msg{************************************************************************}
\Msg{*}
\Msg{* To finish the installation you have to move the following}
\Msg{* file into a directory searched by TeX:}
\Msg{*}
\Msg{*     kvsetkeys.sty}
\Msg{*}
\Msg{* To produce the documentation run the file `kvsetkeys.drv'}
\Msg{* through LaTeX.}
\Msg{*}
\Msg{* Happy TeXing!}
\Msg{*}
\Msg{************************************************************************}

\endbatchfile
%</install>
%<*ignore>
\fi
%</ignore>
%<*driver>
\NeedsTeXFormat{LaTeX2e}
\ProvidesFile{kvsetkeys.drv}%
  [2019/12/15 v1.18 Key value parser (HO)]%
\documentclass{ltxdoc}
\usepackage{holtxdoc}[2011/11/22]
\begin{document}
  \DocInput{kvsetkeys.dtx}%
\end{document}
%</driver>
% \fi
%
%
%
% \GetFileInfo{kvsetkeys.drv}
%
% \title{The \xpackage{kvsetkeys} package}
% \date{2019/12/15 v1.18}
% \author{Heiko Oberdiek\thanks
% {Please report any issues at \url{https://github.com/ho-tex/kvsetkeys/issues}}}
%
% \maketitle
%
% \begin{abstract}
% Package \xpackage{kvsetkeys} provides \cs{kvsetkeys}, a variant
% of package \xpackage{keyval}'s \cs{setkeys}. It allows to specify
% a handler that deals with unknown options. Active commas and equal
% signs may be used (e.g. see \xpackage{babel}'s shorthands) and
% only one level of curly braces is removed from the values.
% \end{abstract}
%
% \tableofcontents
%
% \def\M#1{\texttt{\{}\meta{#1}\texttt{\}}}
%
% \section{Documentation}
%
% First I want to recommend the very good review article
% ``A guide to key-value methods'' by Joseph Wright \cite{tb94wright}.
% It introduces the different key-value packages and compares them.
%
% \subsection{Motivation}
%
% \cs{kvsetkeys} serves as replacement for \xpackage{keyval}'s
% \cs{setkeys}. It basically uses the same syntax. But the
% implementation is more robust and predictable:
% \begin{description}
% \item[Active syntax characters:]
%   Comma `|,|' and the equals sign `|=|' are used inside
%   key value lists as syntax characters. Package \xpackage{keyval}
%   uses the catcode of the characters that is active during
%   package loading, usually this is catcode 12 (other).
%   But it can happen that the catcode setting of the syntax characters
%   changes. Especially active characters are of interest, because
%   some language adaptations uses them. For example, option \xoption{turkish}
%   of package \xpackage{babel} uses the equals sign as active shorthand
%   character. Therefore package \xpackage{kvsetkeys} deals with
%   both catcode settings 12 (other) and 13 (active).
% \item[Brace removal:]
%   Package \xpackage{keyval}'s \cs{setkeys} removes up to two
%   levels of curly braces around the value in some unpredictable way:
%\begin{quote}
%|\setkeys{fam}{key={{value}}} || || | $\rightarrow$ |value|\\
%|\setkeys{fam}{key={{{value}}}} | $\rightarrow$ |{value}|\\
%|\setkeys{fam}{key= {{{value}}}}| $\rightarrow$ |{{value}}|
%\end{quote}
%   This package \xpackage{kvsetkeys} follows a much stronger rule:
%   Exactly one level of braces are removed from an item, if the
%   item is surrounded by curly braces. An item can be a
%   the key value pair, the key or the value.
%\begin{quote}
%|\kvsetkeys{fam}{key={value}} || | $\rightarrow$ |value|\\
%|\kvsetkeys{fam}{key={{value}} | $\rightarrow$ |{value}|\\
%|\kvsetkeys{fam}{key= {{value}}| $\rightarrow$ |{value}|
%\end{quote}
% \item[Arbitrary values:] Unmatched conditionals are supported.
% \end{description}
%
% Before I describe
% \cs{kvsetkeys} in more detail, first I want to explain, how
% this package deals with key value lists. For the package also
% provides low level interfaces that can be used by package authors.
%
% \subsection{Normalizing key value lists}
%
% \begin{declcs}{kv@normalize}\,\M{key value list}
% \end{declcs}
% If the user specifies key value lists, he usually prefers
% nice formatted source code, e.g.:
% \begin{quote}
%\begin{verbatim}
%\hypersetup{
%  pdftitle    = {...},
%  pdfsubject  = {...},
%  pdfauthor   = {...},
%  pdfkeywords = {...},
%  ...
%}
%\end{verbatim}
% \end{quote}
% Thus there can be spaces around keys, around |=| or around the value.
% Also empty entries are possible by too many commas. Therefore these
% spaces and empty entries are silently removed by package \xpackage{keyval}
% and this package. Whereas the contents of the value can be protected
% by curly braces, especially if spaces or commas are used inside,
% a key name must not use spaces or other syntax characters.
%
% \cs{kv@normalize} takes a key value list and performs the cleanup:
% \begin{itemize}
% \item Spaces are removed.
% \item Syntax characters (comma and equal sign) that are active
%   are replaced by the same characters with standard catcode.
%   (Example: \xpackage{babel}'s language option \xoption{turkish}
%   uses the equal sign as active shorthand character.)
% \end{itemize}
% The result is stored in \cs{kv@list}, e.g.:
% \begin{quote}
% |\kv@list| $\rightarrow$ |,pdftitle={...},pdfsubject={...},...,|
% \end{quote}
% Curly braces around values (or keys) remain untouched.
% \begin{description}
% \item[v1.3+:]
% One comma is added in front of the list and each pair ends with
% a comma. Thus an empty list consists of one comma, otherwise
% two commas encloses the list. Empty entries other than the first
% are removed.
% \item[v1.0 -- v1.2:]
% Empty entries are removed later. In fact it adds a comma at the begin
% and end to protect the last value and an easier implementation.
% \end{description}
%
% \subsection{Parsing key value lists}
%
% \begin{declcs}{kv@parse}\,\M{key value list}\,\M{processor}
% \end{declcs}
% It is easier to parse a normalized list, thus \cs{kv@parse}
% normalizes the list and calls \cs{kv@parse@normalized}.
%
% \begin{declcs}{kv@parse@normalized}\,\M{key value list}%
%   \,\M{processor}
% \end{declcs}
% Now the key value list is split into single key value pairs.
% For further processing the key and value are given as arguments
% for the \meta{processor}:
% \begin{quote}
%   \meta{processor}\,\M{key}\,\M{value}
% \end{quote}
% Also key and value are stored in macro names:
% \begin{itemize}
% \item \cs{kv@key} stores the key.
% \item \cs{kv@value} stores the value or if the value was not
%   specified it has the meaning \cs{relax}.
% \end{itemize}
% The behaviour in pseudo code:
% \begin{quote}
%   foreach (\meta{key}, \meta{value}) in (\meta{key value list})\\
%   \hspace*{1.5em}\cs{kv@key} := \meta{key}\\
%   \hspace*{1.5em}\cs{kv@value} := \meta{value}\\
%   \hspace*{1.5em}\meta{processor}\,\M{key}\,\M{value}
% \end{quote}
%
% \begin{declcs}{kv@break}
% \end{declcs}
% Since version 2011/03/03 v1.11 \cs{kv@break} can be
% called inside the \meta{processor}
% of \cs{kv@parse} or \cs{kv@parse@normalized}, then
% the processing is stopped and the following entries discarded.
%
% \subsection{Processing key value pairs}
%
% Key value pairs can be processed in many different ways.
% For example, the processor for \cs{kvsetkeys} works similar
% to \cs{setkeys} of package \xpackage{keyval}. There unknown
% keys raise an error.
%
% Package \xpackage{xkeyval} also knows a star form of \cs{setkeys}
% that stores unknown keys in an internal macro for further processing
% with \cs{setrmkeys} and similar macros. This feature is covered
% by processor \cs{kv@processor@known}.
%
% \subsubsection{Processing similar to \xpackage{keyval}}
%
% \begin{declcs}{kv@processor@default}\,\M{family}\,\M{key}\,\M{value}
% \end{declcs}
% There are many possiblities to process key value pairs.
% \cs{kv@processor@default} is the processor used in \cs{kvsetkeys}.
% It reimplements and extends the behaviour of
% \xpackage{keyval}'s \cs{setkeys}.
% In case of unknown keys \cs{setkeys} raise an error.
% This processer, however, calls a handler instead, if it
% is provided by the family. Both \meta{family} and \meta{key}
% may contain package \xpackage{babel}'s shorthands
% (since 2011/04/07 v1.13).
%
% Since 2011/10/18 v1.15 the family handler can reject the
% successful handling of a key by calling \cs{kv@handled@false}.
%
% Since 2019/12/15 v1.18 \cs{kv@processor@default} also defines
% macro \cs{kv@fam} with meaning \meta{family} for convenience.
%
% \subsubsection{Processing similar to \cs{setkeys*} of package \xpackage{xkeyval}}
%
% \begin{declcs}{kv@processor@known}\,\M{family}\,^^A
%   \M{cmd}\,\M{key}\,\M{value}
% \end{declcs}
% The key value processor \cs{kv@processor@known} behaves similar
% to \cs{kv@processor@default}. If the \meta{key} exists in the
% \meta{family} its code is called, otherwise the family handler
% is tried. If the family handler is not set or cannot handle the
% key, the unknown key value pair is added to the macro \meta{cmd}.
% Since 2011/10/18 v1.15.
%
% The behaviour in pseudo code:
% \begin{quote}
%   if \meta{key} exists\\
%   \hspace*{1.5em}call the keyval code of \meta{key}\\
%   else\\
%   \hspace*{1.5em}if \meta{handler} for \meta{family} exists\\
%   \hspace*{3em}handled = true\\
%   \hspace*{3em}\meta{handler}\,\M{key}\,\M{value}\\
%   \hspace*{3em}if handled\\
%   \hspace*{3em}else\\
%   \hspace*{4.5em}add \texttt{"}\M{key}\texttt{=}\M{value}\texttt{"}
%     to \M{cmd}\\
%   \hspace*{3em}fi\\
%   \hspace*{1.5em}else\\
%   \hspace*{3em}add \texttt{"}\M{key}\texttt{=}\M{value}\texttt{"}
%     to \M{cmd}\\
%   \hspace*{3em}raise unknown key error\\
%   \hspace*{1.5em}fi\\
%   fi
% \end{quote}
%
% Since 2019/12/15 v1.18 \cs{kv@processor@known} also defines
% macro \cs{kv@fam} with meaning \meta{family} for convenience.
%
% \subsection{Default family handler}
%
% \cs{kv@processor@default} calls \meta{handler}, the default
% handler for the family, if the key does not exist in the family.
% The handler is called with two arguments, the key and the value.
% It can be defined with \cs{kv@set@family@hander}:
%
% \begin{declcs}{kv@set@family@handler}\,\M{family}\,\M{handler definition}
% \end{declcs}
% This sets the default family handler for the keyval family
% \meta{family}. Inside \meta{handler definition} |#1| stands for
% the key and |#2| is the value. Also \cs{kv@key} and \cs{kv@value}
% can be used for the key and the value. If the value is not given,
% \cs{kv@value} has the meaning \cs{relax}.
%
% \begin{declcs}{kv@unset@family@handler}\,\M{family}
% \end{declcs}
% It removes the family handler for \meta{family}.
% Since 2011/10/18 v1.15.
%
% \subsection{Put it all together}
%
% \begin{declcs}{kvsetkeys}\,\M{family}\,\M{key value list}
% \end{declcs}
% Macro \cs{kvsetkeys} processes the \meta{key value list} with
% the standard processor \cs{kv@processor@default}:
% \begin{quote}
%   \cs{kv@parse}\,\M{key value list}^^A
%   \texttt{\{}\cs{kv@processor@default}\,\M{family}\texttt{\}}
% \end{quote}
%
% \begin{declcs}{kvsetknownkeys}\,\M{family}\,\M{cmd}\,\M{key value list}
% \end{declcs}
% Macro \cs{kvsetknownkeys} processes the \meta{key value list}
% with processor \cs{kv@processor@known}. All key value pairs
% with keys that are not known in \meta{family} are stored
% in macro \meta{cmd}. A previous contents of macro \meta{cmd}
% will be overwritten. If all keys can be handled, \meta{cmd}
% will be empty, otherwise it contains a key value list of
% unhandled key value pairs.
% Since 2011/10/18 v1.15.
%
% Pseudo code:
% \begin{quote}
%   create macro \meta{cmdaux} with unique name (inside the current group)\\
%   \cs{def}\meta{cmdaux}\{\}\\
%   \cs{kv@parse}\,\M{key value list}^^A
%   \texttt{\{}\cs{kv@processor@known}\,\M{family}\,\M{cmdaux}\texttt{\}}\\
%   \cs{let}\meta{cmd}=\meta{cmdaux}
% \end{quote}
%
% \begin{declcs}{kvsetkeys@expandafter}\,\M{family}\,\M{list cmd}\\
%   \cs{kvsetknownkeys@expandafter}\,\M{family}\,%
%   \M{cmd}\,\M{list cmd}
% \end{declcs}
% Both macros behave like the counterparts without suffix
% |@expandafter|. The difference is that the key value list is
% given as macro that is expanded once.
% Since 2011/10/18 v1.15.
%
% Thus you can replace \cs{setkeys} of package \xpackage{keyval}
% by the key value parser of this package:
% \begin{quote}
%   |\renewcommand*{\setkeys}{\kvsetkeys}|\\
%   or\\
%   |\let\setkeys\kvsetkeys|
% \end{quote}
%
% \subsection{Comma separated lists}
%
% Since version 2007/09/29 v1.3 this package also supports the normalizing
% and parsing of general comma separated lists.
%
% \begin{declcs}{comma@normalize}\,\M{comma list}
% \end{declcs}
% Macro \cs{comma@normalize} normalizes the comma separated list,
% removes spaces around commas. The result is put in macro \cs{comma@list}.
%
% \begin{declcs}{comma@parse}\,\M{comma list}\,\M{processor}
% \end{declcs}
% Macro \cs{comma@parse} first normalizes the comma separated list
% and then parses the list by calling \cs{comma@parse@normalized}.
%
% \begin{declcs}{comma@parse@normalized}\,\M{normalized comma list}^^A
%   \,\M{processor}
% \end{declcs}
% The list is parsed. Empty entries are ignored. \meta{processor}
% is called for each non-empty entry with the entry as argument:
% \begin{quote}
%   \meta{processor}|{|\meta{entry}|}|
% \end{quote}
% Also the entry is stored in the macro \cs{comma@entry}.
%
% \begin{declcs}{comma@break}
% \end{declcs}
% Since version 2011/03/03 v1.11 \cs{comma@break} can be
% called inside the \meta{processor}
% of \cs{comma@parse} or \cs{comma@parse@normalized}, then
% the processing is stopped and the following entries discarded.
%
% \section{Example}
%
% The following example prints a short piece of HTML code using
% the tabbing environment for indenting purpose and a key value
% syntax for specifying the attributes of an HTML tag.
% The example illustrates the use of a default family handler.
%    \begin{macrocode}
%<*example>
\documentclass{article}
\usepackage[T1]{fontenc}
\usepackage{kvsetkeys}
\usepackage{keyval}

\makeatletter
\newcommand*{\tag}[2][]{%
  % #1: attributes
  % #2: tag name
  \begingroup
    \toks@={}%
    \let\@endslash\@empty
    \kvsetkeys{tag}{#1}%
    \texttt{%
      \textless #2\the\toks@\@endslash\textgreater
    }%
  \endgroup
}
\kv@set@family@handler{tag}{%
  % #1: key
  % #2: value
  \toks@\expandafter{%
    \the\toks@
    \space
    #1=\string"#2\string"%
  }%
}
\define@key{tag}{/}[]{%
  \def\@endslash{/}%
}
\makeatother

\begin{document}
\begin{tabbing}
  \mbox{}\qquad\=\qquad\=\kill
  \tag{html}\\
  \>\dots\\
  \>\tag[border=1]{table}\\
  \>\>\tag[width=200, span=3, /]{colgroup}\\
  \>\>\dots\\
  \>\tag{/table}\\
  \>\dots\\
  \tag{/html}\\
\end{tabbing}
\end{document}
%</example>
%    \end{macrocode}
%
% \StopEventually{
% }
%
% \section{Implementation}
%
% \subsection{Identification}
%
%    \begin{macrocode}
%<*package>
%    \end{macrocode}
%    Reload check, especially if the package is not used with \LaTeX.
%    \begin{macrocode}
\begingroup\catcode61\catcode48\catcode32=10\relax%
  \catcode13=5 % ^^M
  \endlinechar=13 %
  \catcode35=6 % #
  \catcode39=12 % '
  \catcode44=12 % ,
  \catcode45=12 % -
  \catcode46=12 % .
  \catcode58=12 % :
  \catcode64=11 % @
  \catcode123=1 % {
  \catcode125=2 % }
  \expandafter\let\expandafter\x\csname ver@kvsetkeys.sty\endcsname
  \ifx\x\relax % plain-TeX, first loading
  \else
    \def\empty{}%
    \ifx\x\empty % LaTeX, first loading,
      % variable is initialized, but \ProvidesPackage not yet seen
    \else
      \expandafter\ifx\csname PackageInfo\endcsname\relax
        \def\x#1#2{%
          \immediate\write-1{Package #1 Info: #2.}%
        }%
      \else
        \def\x#1#2{\PackageInfo{#1}{#2, stopped}}%
      \fi
      \x{kvsetkeys}{The package is already loaded}%
      \aftergroup\endinput
    \fi
  \fi
\endgroup%
%    \end{macrocode}
%    Package identification:
%    \begin{macrocode}
\begingroup\catcode61\catcode48\catcode32=10\relax%
  \catcode13=5 % ^^M
  \endlinechar=13 %
  \catcode35=6 % #
  \catcode39=12 % '
  \catcode40=12 % (
  \catcode41=12 % )
  \catcode44=12 % ,
  \catcode45=12 % -
  \catcode46=12 % .
  \catcode47=12 % /
  \catcode58=12 % :
  \catcode64=11 % @
  \catcode91=12 % [
  \catcode93=12 % ]
  \catcode123=1 % {
  \catcode125=2 % }
  \expandafter\ifx\csname ProvidesPackage\endcsname\relax
    \def\x#1#2#3[#4]{\endgroup
      \immediate\write-1{Package: #3 #4}%
      \xdef#1{#4}%
    }%
  \else
    \def\x#1#2[#3]{\endgroup
      #2[{#3}]%
      \ifx#1\@undefined
        \xdef#1{#3}%
      \fi
      \ifx#1\relax
        \xdef#1{#3}%
      \fi
    }%
  \fi
\expandafter\x\csname ver@kvsetkeys.sty\endcsname
\ProvidesPackage{kvsetkeys}%
  [2019/12/15 v1.18 Key value parser (HO)]%
%    \end{macrocode}
%
%    \begin{macrocode}
\begingroup\catcode61\catcode48\catcode32=10\relax%
  \catcode13=5 % ^^M
  \endlinechar=13 %
  \catcode123=1 % {
  \catcode125=2 % }
  \catcode64=11 % @
  \def\x{\endgroup
    \expandafter\edef\csname KVS@AtEnd\endcsname{%
      \endlinechar=\the\endlinechar\relax
      \catcode13=\the\catcode13\relax
      \catcode32=\the\catcode32\relax
      \catcode35=\the\catcode35\relax
      \catcode61=\the\catcode61\relax
      \catcode64=\the\catcode64\relax
      \catcode123=\the\catcode123\relax
      \catcode125=\the\catcode125\relax
    }%
  }%
\x\catcode61\catcode48\catcode32=10\relax%
\catcode13=5 % ^^M
\endlinechar=13 %
\catcode35=6 % #
\catcode64=11 % @
\catcode123=1 % {
\catcode125=2 % }
\def\TMP@EnsureCode#1#2{%
  \edef\KVS@AtEnd{%
    \KVS@AtEnd
    \catcode#1=\the\catcode#1\relax
  }%
  \catcode#1=#2\relax
}
\TMP@EnsureCode{36}{3}% $
\TMP@EnsureCode{38}{4}% &
\TMP@EnsureCode{39}{12}% '
\TMP@EnsureCode{43}{12}% +
\TMP@EnsureCode{44}{12}% ,
\TMP@EnsureCode{45}{12}% -
\TMP@EnsureCode{46}{12}% .
\TMP@EnsureCode{47}{12}% /
\TMP@EnsureCode{91}{12}% [
\TMP@EnsureCode{93}{12}% ]
\TMP@EnsureCode{94}{7}% ^ (superscript)
\TMP@EnsureCode{96}{12}% `
\TMP@EnsureCode{126}{13}% ~ (active)
\edef\KVS@AtEnd{\KVS@AtEnd\noexpand\endinput}
%    \end{macrocode}
%
% \subsection{Package loading}
%
%    \begin{macrocode}
\begingroup\expandafter\expandafter\expandafter\endgroup
\expandafter\ifx\csname RequirePackage\endcsname\relax
  \def\TMP@RequirePackage#1[#2]{%
    \begingroup\expandafter\expandafter\expandafter\endgroup
    \expandafter\ifx\csname ver@#1.sty\endcsname\relax
      \input #1.sty\relax
    \fi
  }%
  \TMP@RequirePackage{infwarerr}[2007/09/09]%
  \let\PackageError\@PackageError
\else
\fi
%    \end{macrocode}
%
%    \begin{macrocode}
\expandafter\ifx\csname toks@\endcsname\relax
  \toksdef\toks@=0 %
\fi
%    \end{macrocode}
%
% \subsection{Check for \eTeX}
%
%    \cs{unexpanded}, \cs{ifcsname}, and \cs{unless} are used if found.
%    \begin{macrocode}
\ifx\numexpr\@undefined
  \catcode`\$=9 % ignore
  \catcode`\&=14 % comment
\else % e-TeX
  \catcode`\$=14 % comment
  \catcode`\&=9  % ignore
\fi
%    \end{macrocode}
%
% \subsection{Generic help macros}
%
%    \begin{macro}{\KVS@Empty}
%    \begin{macrocode}
\def\KVS@Empty{}
%    \end{macrocode}
%    \end{macro}
%    \begin{macro}{\KVS@FirstOfTwo}
%    \begin{macrocode}
\long\def\KVS@FirstOfTwo#1#2{#1}
%    \end{macrocode}
%    \end{macro}
%    \begin{macro}{\KVS@SecondOfTwo}
%    \begin{macrocode}
\long\def\KVS@SecondOfTwo#1#2{#2}
%    \end{macrocode}
%    \end{macro}
%
%    \begin{macro}{\KVS@IfEmpty}
%    \begin{macrocode}
\long\def\KVS@IfEmpty#1{%
& \edef\KVS@Temp{\unexpanded{#1}}%
$ \begingroup
$   \toks@{#1}%
$   \edef\KVS@Temp{\the\toks@}%
$ \expandafter\endgroup
  \ifx\KVS@Temp\KVS@Empty
    \expandafter\KVS@FirstOfTwo
  \else
    \expandafter\KVS@SecondOfTwo
  \fi
}
%    \end{macrocode}
%    \end{macro}
%
% \subsection{Normalizing}
%
%    \begin{macro}{\kv@normalize}
%    \begin{macrocode}
\long\def\kv@normalize#1{%
  \begingroup
    \toks@{,#1,}%
    \KVS@Comma
    \KVS@SpaceComma
    \KVS@CommaSpace
    \KVS@CommaComma
    \KVS@Equals
    \KVS@SpaceEquals
    \KVS@EqualsSpace
    \xdef\KVS@Global{\the\toks@}%
  \endgroup
  \let\kv@list\KVS@Global
}
%    \end{macrocode}
%    \end{macro}
%    \begin{macro}{\comma@normalize}
%    \begin{macrocode}
\def\comma@normalize#1{%
  \begingroup
    \toks@{,#1,}%
    \KVS@Comma
    \KVS@SpaceComma
    \KVS@CommaSpace
    \KVS@CommaComma
    \xdef\KVS@Global{\the\toks@}%
  \endgroup
  \let\comma@list\KVS@Global
}
%    \end{macrocode}
%    \end{macro}
%    \begin{macro}{\KVS@Comma}
%    Converts active commas into comma with catcode other.
%    Also adds a comma at the end to protect the last value
%    for next cleanup steps.
%    \begin{macrocode}
\begingroup
  \lccode`\,=`\,%
  \lccode`\~=`\,%
\lowercase{\endgroup
  \def\KVS@Comma{%
    \toks@\expandafter{\expandafter}\expandafter
    \KVS@@Comma\the\toks@~\KVS@Nil
  }%
  \long\def\KVS@@Comma#1~#2\KVS@Nil{%
    \toks@\expandafter{\the\toks@#1}%
    \KVS@IfEmpty{#2}{%
    }{%
      \KVS@@Comma,#2\KVS@Nil
    }%
  }%
}
%    \end{macrocode}
%    \end{macro}
%    \begin{macro}{\KVS@SpaceComma}
%    Removes spaces before the comma, may add commas at the end.
%    \begin{macrocode}
\def\KVS@SpaceComma#1{%
  \def\KVS@SpaceComma{%
    \expandafter\KVS@@SpaceComma\the\toks@#1,\KVS@Nil
  }%
}
\KVS@SpaceComma{ }
%    \end{macrocode}
%    \end{macro}
%    \begin{macro}{\KVS@@SpaceComma}
%    \begin{macrocode}
\long\def\KVS@@SpaceComma#1 ,#2\KVS@Nil{%
  \KVS@IfEmpty{#2}{%
    \toks@{#1}%
  }{%
    \KVS@@SpaceComma#1,#2\KVS@Nil
  }%
}
%    \end{macrocode}
%    \end{macro}
%    \begin{macro}{\KVS@CommaSpace}
%    Removes spaces after the comma, may add commas at the end.
%    \begin{macrocode}
\def\KVS@CommaSpace{%
  \expandafter\KVS@@CommaSpace\the\toks@, \KVS@Nil
}
%    \end{macrocode}
%    \end{macro}
%    \begin{macro}{\KVS@@CommaSpace}
%    \begin{macrocode}
\long\def\KVS@@CommaSpace#1, #2\KVS@Nil{%
  \KVS@IfEmpty{#2}{%
    \toks@{#1}%
  }{%
    \KVS@@CommaSpace#1,#2\KVS@Nil
  }%
}
%    \end{macrocode}
%    \end{macro}
%    \begin{macro}{\KVS@CommaComma}
%    Replaces multiple commas by one comma.
%    \begin{macrocode}
\def\KVS@CommaComma{%
  \expandafter\KVS@@CommaComma\the\toks@,\KVS@Nil
}
%    \end{macrocode}
%    \end{macro}
%    \begin{macro}{\KVS@@CommaComma}
%    \begin{macrocode}
\long\def\KVS@@CommaComma#1,,#2\KVS@Nil{%
  \KVS@IfEmpty{#2}{%
    \toks@{#1,}% (!)
  }{%
    \KVS@@CommaComma#1,#2\KVS@Nil
  }%
}
%    \end{macrocode}
%    \end{macro}
%
%    \begin{macro}{\KVS@Equals}
%    Converts active equals signs into catcode other characters.
%    \begin{macrocode}
\begingroup
  \lccode`\==`\=%
  \lccode`\~=`\=%
\lowercase{\endgroup
  \def\KVS@Equals{%
    \toks@\expandafter{\expandafter}\expandafter
    \KVS@@Equals\the\toks@~\KVS@Nil
  }%
  \long\def\KVS@@Equals#1~#2\KVS@Nil{%
    \edef\KVS@Temp{\the\toks@}%
    \ifx\KVS@Temp\KVS@Empty
      \expandafter\KVS@FirstOfTwo
    \else
      \expandafter\KVS@SecondOfTwo
    \fi
    {%
      \toks@{#1}%
    }{%
      \toks@\expandafter{\the\toks@=#1}%
    }%
    \KVS@IfEmpty{#2}{%
    }{%
      \KVS@@Equals#2\KVS@Nil
    }%
  }%
}
%    \end{macrocode}
%    \end{macro}
%    \begin{macro}{\KVS@SpaceEquals}
%    Removes spaces before the equals sign.
%    \begin{macrocode}
\def\KVS@SpaceEquals#1{%
  \def\KVS@SpaceEquals{%
    \expandafter\KVS@@SpaceEquals\the\toks@#1=\KVS@Nil
  }%
}
\KVS@SpaceEquals{ }
%    \end{macrocode}
%    \end{macro}
%    \begin{macro}{\KVS@@SpaceEquals}
%    \begin{macrocode}
\long\def\KVS@@SpaceEquals#1 =#2\KVS@Nil{%
  \KVS@IfEmpty{#2}{%
    \toks@{#1}%
  }{%
    \KVS@@SpaceEquals#1=#2\KVS@Nil
  }%
}
%    \end{macrocode}
%    \end{macro}
%    \begin{macro}{\KVS@EqualsSpace}
%    Removes spaces after the equals sign.
%    \begin{macrocode}
\def\KVS@EqualsSpace{%
  \expandafter\KVS@@EqualsSpace\the\toks@= \KVS@Nil
}
%    \end{macrocode}
%    \end{macro}
%    \begin{macro}{\KVS@@EqualsSpace}
%    \begin{macrocode}
\long\def\KVS@@EqualsSpace#1= #2\KVS@Nil{%
  \KVS@IfEmpty{#2}{%
    \toks@{#1}%
  }{%
    \KVS@@EqualsSpace#1=#2\KVS@Nil
  }%
}
%    \end{macrocode}
%    \end{macro}
%
% \subsection{Parsing key value lists}
%
%    \begin{macro}{\kv@parse}
%    Normalizes and parses the key value list. Also sets \cs{kv@list}.
%    \begin{macrocode}
\long\def\kv@parse#1{%
  \kv@normalize{#1}%
  \expandafter\kv@parse@normalized\expandafter{\kv@list}%
}
%    \end{macrocode}
%    \end{macro}
%    \begin{macro}{\kv@parse@normalized}
%    |#1|: key value list\\
%    |#2|: processor
%    \begin{macrocode}
\long\def\kv@parse@normalized#1#2{%
  \KVS@Parse#1,\KVS@Nil{#2}%
}
%    \end{macrocode}
%    \end{macro}
%    \begin{macro}{\KVS@Parse}
%    |#1,#2|: key value list\\
%    |#3|: processor
%    \begin{macrocode}
\long\def\KVS@Parse#1,#2\KVS@Nil#3{%
  \KVS@IfEmpty{#1}{%
  }{%
    \KVS@Process#1=\KVS@Nil{#3}%
  }%
  \KVS@MaybeBreak
  \KVS@IfEmpty{#2}{%
  }{%
    \KVS@Parse#2\KVS@Nil{#3}%
  }%
}
%    \end{macrocode}
%    \end{macro}
%    \begin{macro}{\KVS@Process}
%    |#1|: key\\
%    |#2|: value, |=|\\
%    |#3|: processor
%    \begin{macrocode}
\long\def\KVS@Process#1=#2\KVS@Nil#3{%
  \let\KVS@MaybeBreak\relax
  \def\kv@key{#1}%
  \KVS@IfEmpty{#2}{%
    \let\kv@value\relax
    #3{#1}{}%
  }{%
    \KVS@@Process{#1}#2\KVS@Nil{#3}%
  }%
}
%    \end{macrocode}
%    \end{macro}
%    \begin{macro}{\KVS@@Process}
%    |#1|: key\\
%    |#2|: value\\
%    |#3|: processor
%    \begin{macrocode}
\long\def\KVS@@Process#1#2=\KVS@Nil#3{%
& \edef\kv@value{\unexpanded{#2}}%
$ \begingroup
$   \toks@{#2}%
$   \xdef\KVS@Global{\the\toks@}%
$ \endgroup
$ \let\kv@value\KVS@Global
  #3{#1}{#2}%
}
%    \end{macrocode}
%    \end{macro}
%    \begin{macro}{\KVS@MaybeBreak}
%    \begin{macrocode}
\let\KVS@MaybeBreak\relax
%    \end{macrocode}
%    \end{macro}
%    \begin{macro}{\KVS@break}
%    \begin{macrocode}
\def\KVS@break#1#2#3#4{%
  \let\KVS@MaybeBreak\relax
}
%    \end{macrocode}
%    \end{macro}
%    \begin{macro}{\kv@break}
%    \begin{macrocode}
\def\kv@break{%
  \let\KVS@MaybeBreak\KVS@break
}
%    \end{macrocode}
%    \end{macro}
%
% \subsection{Parsing comma lists}
%
%    \begin{macro}{\comma@parse}
%    Normalizes and parses the key value list. Also sets \cs{comma@list}.
%    \begin{macrocode}
\def\comma@parse#1{%
  \comma@normalize{#1}%
  \expandafter\comma@parse@normalized\expandafter{\comma@list}%
}
%    \end{macrocode}
%    \end{macro}
%
%    \begin{macro}{\comma@parse@normalized}
%    |#1|: comma list\\
%    |#2|: processor
%    \begin{macrocode}
\def\comma@parse@normalized#1#2{%
  \KVS@CommaParse#1,\KVS@Nil{#2}%
}
%    \end{macrocode}
%    \end{macro}
%    \begin{macro}{\KVS@CommaParse}
%    |#1,#2|: comma list\\
%    |#3|: processor
%    \begin{macrocode}
\def\KVS@CommaParse#1,#2\KVS@Nil#3{%
  \KVS@IfEmpty{#1}{%
  }{%
    \def\comma@entry{#1}%
    #3{#1}%
  }%
  \KVS@MaybeBreak
  \KVS@IfEmpty{#2}{%
  }{%
    \KVS@CommaParse#2\KVS@Nil{#3}%
  }%
}
%    \end{macrocode}
%    \end{macro}
%    \begin{macro}{\comma@break}
%    \begin{macrocode}
\def\comma@break{%
  \let\KVS@MaybeBreak\KVS@break
}
%    \end{macrocode}
%    \end{macro}
%
% \subsection{Processing key value pairs}
%
%    \begin{macro}{\kv@handled@false}
%    The handler can call \cs{kv@handled@false} or
%    \cs{kv@handled@true} so report failure or success.
%    The default is success (compatibility for versions
%    before 2011/10/18 v1.15).
%    \begin{macrocode}
\def\kv@handled@false{%
  \let\ifkv@handled@\iffalse
}
%    \end{macrocode}
%    \end{macro}
%    \begin{macro}{\kv@handled@true}
%    \begin{macrocode}
\def\kv@handled@true{%
  \let\ifkv@handled@\iftrue
}
%    \end{macrocode}
%    \end{macro}
%    \begin{macro}{\ifkv@handled@}
%    \begin{macrocode}
\kv@handled@true
%    \end{macrocode}
%    \end{macro}
%
%    \begin{macro}{\kv@processor@default}
%    \begin{macrocode}
\def\kv@processor@default#1#2{%
  \begingroup
    \csname @safe@activestrue\endcsname
    \let\ifincsname\iftrue
    \edef\KVS@temp{\endgroup
      \noexpand\KVS@ProcessorDefault{#1}{#2}%
    }%
  \KVS@temp
}
%    \end{macrocode}
%    \end{macro}
%    \begin{macro}{\KVS@ProcessorDefault}
%    \begin{macrocode}
\long\def\KVS@ProcessorDefault#1#2#3{%
  \def\kv@fam{#1}%
& \unless\ifcsname KV@#1@#2\endcsname
$ \begingroup\expandafter\expandafter\expandafter\endgroup
$ \expandafter\ifx\csname KV@#1@#2\endcsname\relax
&   \unless\ifcsname KVS@#1@handler\endcsname
$   \begingroup\expandafter\expandafter\expandafter\endgroup
$   \expandafter\ifx\csname KVS@#1@handler\endcsname\relax
      \kv@error@unknownkey{#1}{#2}%
    \else
      \kv@handled@true
      \csname KVS@#1@handler\endcsname{#2}{#3}%
      \relax
      \ifkv@handled@
      \else
        \kv@error@unknownkey{#1}{#2}%
      \fi
    \fi
  \else
    \ifx\kv@value\relax
&     \unless\ifcsname KV@#1@#2@default\endcsname
$     \begingroup\expandafter\expandafter\expandafter\endgroup
$     \expandafter\ifx\csname KV@#1@#2@default\endcsname\relax
        \kv@error@novalue{#1}{#2}%
      \else
        \csname KV@#1@#2@default\endcsname
        \relax
      \fi
    \else
      \csname KV@#1@#2\endcsname{#3}%
    \fi
  \fi
}
%    \end{macrocode}
%    \end{macro}
%
%    \begin{macro}{\kv@processor@known}
%    \begin{macrocode}
\def\kv@processor@known#1#2#3{%
  \begingroup
    \csname @safe@activestrue\endcsname
    \let\ifincsname\iftrue
    \edef\KVS@temp{\endgroup
      \noexpand\KVS@ProcessorKnown{#1}\noexpand#2{#3}%
    }%
  \KVS@temp
}
%    \end{macrocode}
%    \end{macro}
%    \begin{macro}{\KVS@ProcessorKnown}
%    \begin{macrocode}
\long\def\KVS@ProcessorKnown#1#2#3#4{%
  \def\kv@fam{#1}%
& \unless\ifcsname KV@#1@#3\endcsname
$ \begingroup\expandafter\expandafter\expandafter\endgroup
$ \expandafter\ifx\csname KV@#1@#3\endcsname\relax
&   \unless\ifcsname KVS@#1@handler\endcsname
$   \begingroup\expandafter\expandafter\expandafter\endgroup
$   \expandafter\ifx\csname KVS@#1@handler\endcsname\relax
      \KVS@AddUnhandled#2{#3}{#4}%
    \else
      \kv@handled@true
      \csname KVS@#1@handler\endcsname{#3}{#4}%
      \relax
      \ifkv@handled@
      \else
        \KVS@AddUnhandled#2{#3}{#4}%
      \fi
    \fi
  \else
    \ifx\kv@value\relax
&     \unless\ifcsname KV@#1@#2@default\endcsname
$     \begingroup\expandafter\expandafter\expandafter\endgroup
$     \expandafter\ifx\csname KV@#1@#3@default\endcsname\relax
        \kv@error@novalue{#1}{#3}%
      \else
        \csname KV@#1@#3@default\endcsname
        \relax
      \fi
    \else
      \csname KV@#1@#3\endcsname{#4}%
    \fi
  \fi
}
%    \end{macrocode}
%    \end{macro}
%    \begin{macro}{\KVS@AddUnhandled}
%    \begin{macrocode}
\long\def\KVS@AddUnhandled#1#2#3{%
& \edef#1{%
&   \ifx#1\KVS@empty
&   \else
&     \unexpanded{#1},%
&   \fi
&   \unexpanded{{#2}={#3}}%
& }%
$ \begingroup
$   \ifx#1\KVS@empty
$     \toks@{{#2}={#3}}%
$   \else
$     \toks@\expandafter{#1,{#2}={#3}}%
$   \fi
$   \xdef\KVS@Global{\the\toks@}%
$ \endgroup
$ \let#1\KVS@Global
}
%    \end{macrocode}
%    \end{macro}
%
%    \begin{macro}{\kv@set@family@handler}
%    \begin{macrocode}
\long\def\kv@set@family@handler#1#2{%
  \begingroup
    \csname @safe@activestrue\endcsname
    \let\ifincsname\iftrue
  \expandafter\endgroup
  \expandafter\def\csname KVS@#1@handler\endcsname##1##2{#2}%
}
%    \end{macrocode}
%    \end{macro}
%
%    \begin{macro}{\kv@unset@family@handler}
%    \begin{macrocode}
\long\def\kv@unset@family@handler#1#2{%
  \begingroup
    \csname @safe@activestrue\endcsname
    \let\ifincsname\iftrue
  \expandafter\endgroup
  \expandafter\let\csname KVS@#1@handler\endcsname\@UnDeFiNeD
}
%    \end{macrocode}
%    \end{macro}
%
% \subsection{Error handling}
%
%    \begin{macro}{\kv@error@novalue}
%    \begin{macrocode}
\def\kv@error@novalue{%
  \kv@error@generic{No value specified for}%
}
%    \end{macrocode}
%    \end{macro}
%    \begin{macro}{\kv@error@unknownkey}
%    \begin{macrocode}
\def\kv@error@unknownkey{%
  \kv@error@generic{Undefined}%
}
%    \end{macrocode}
%    \end{macro}
%    \begin{macro}{\kv@error@generic}
%    \begin{macrocode}
\def\kv@error@generic#1#2#3{%
  \PackageError{kvsetkeys}{%
    #1 key `#3'%
  }{%
    The keyval family of the key `#3' is `#2'.\MessageBreak
    The setting of the key is ignored because of the error.\MessageBreak
    \MessageBreak
    \@ehc
  }%
}
%    \end{macrocode}
%    \end{macro}
%
% \subsection{Do it all}
%
%    \begin{macro}{\kvsetkeys}
%    \begin{macrocode}
\long\def\kvsetkeys#1#2{%
  \kv@parse{#2}{\kv@processor@default{#1}}%
}
%    \end{macrocode}
%    \end{macro}
%
%    \begin{macro}{\kvsetkeys@expandafter}
%    \begin{macrocode}
\def\kvsetkeys@expandafter#1#2{%
  \expandafter\kv@parse\expandafter{#2}{%
    \kv@processor@default{#1}%
  }%
}
%    \end{macrocode}
%    \end{macro}
%
%    \begin{macro}{\KVS@cmd}
%    \begin{macrocode}
\def\KVS@cmd{0}%
%    \end{macrocode}
%    \end{macro}
%    \begin{macro}{\KVS@cmd@inc}
%    \begin{macrocode}
\def\KVS@cmd@inc{%
& \edef\KVS@cmd{\the\numexpr\KVS@cmd+1}%
$ \begingroup
$   \count255=\KVS@cmd\relax
$   \advance\count255 by 1\relax
$ \edef\x{\endgroup
$   \noexpand\def\noexpand\KVS@cmd{\number\count255}%
$ }%
$ \x
}
%    \end{macrocode}
%    \end{macro}
%    \begin{macro}{\KVS@cmd@dec}
%    \begin{macrocode}
\def\KVS@cmd@dec{%
& \edef\KVS@cmd{\the\numexpr\KVS@cmd-1}%
$ \begingroup
$   \count255=\KVS@cmd\relax
$   \advance\count255 by -1\relax
$ \edef\x{\endgroup
$   \noexpand\def\noexpand\KVS@cmd{\number\count255}%
$ }%
$ \x
}
%    \end{macrocode}
%    \end{macro}
%    \begin{macro}{\KVS@empty}
%    \begin{macrocode}
\def\KVS@empty{}
%    \end{macrocode}
%    \end{macro}
%
%    \begin{macro}{\kvsetknownkeys}
%    \begin{macrocode}
\def\kvsetknownkeys{%
  \expandafter
  \KVS@setknownkeys\csname KVS@cmd\KVS@cmd\endcsname{}%
}
%    \end{macrocode}
%    \end{macro}
%    \begin{macro}{\KVS@setknownkeys}
%    \begin{macrocode}
\long\def\KVS@setknownkeys#1#2#3#4#5{%
  \let#1\KVS@empty
  \KVS@cmd@inc
  #2\kv@parse#2{#5}{\kv@processor@known{#3}#1}%
  \KVS@cmd@dec
  \let#4=#1%
}
%    \end{macrocode}
%    \end{macro}
%
%    \begin{macro}{\kvsetknownkeys@expandafter}
%    \begin{macrocode}
\def\kvsetknownkeys@expandafter{%
  \expandafter
  \KVS@setknownkeys
      \csname KVS@cmd\KVS@cmd\endcsname\expandafter
}
%    \end{macrocode}
%    \end{macro}
%
%    \begin{macrocode}
\KVS@AtEnd%
%</package>
%    \end{macrocode}
%% \section{Installation}
%
% \subsection{Download}
%
% \paragraph{Package.} This package is available on
% CTAN\footnote{\CTANpkg{kvsetkeys}}:
% \begin{description}
% \item[\CTAN{macros/latex/contrib/kvsetkeys/kvsetkeys.dtx}] The source file.
% \item[\CTAN{macros/latex/contrib/kvsetkeys/kvsetkeys.pdf}] Documentation.
% \end{description}
%
%
% \paragraph{Bundle.} All the packages of the bundle `kvsetkeys'
% are also available in a TDS compliant ZIP archive. There
% the packages are already unpacked and the documentation files
% are generated. The files and directories obey the TDS standard.
% \begin{description}
% \item[\CTANinstall{install/macros/latex/contrib/kvsetkeys.tds.zip}]
% \end{description}
% \emph{TDS} refers to the standard ``A Directory Structure
% for \TeX\ Files'' (\CTANpkg{tds}). Directories
% with \xfile{texmf} in their name are usually organized this way.
%
% \subsection{Bundle installation}
%
% \paragraph{Unpacking.} Unpack the \xfile{kvsetkeys.tds.zip} in the
% TDS tree (also known as \xfile{texmf} tree) of your choice.
% Example (linux):
% \begin{quote}
%   |unzip kvsetkeys.tds.zip -d ~/texmf|
% \end{quote}
%
% \subsection{Package installation}
%
% \paragraph{Unpacking.} The \xfile{.dtx} file is a self-extracting
% \docstrip\ archive. The files are extracted by running the
% \xfile{.dtx} through \plainTeX:
% \begin{quote}
%   \verb|tex kvsetkeys.dtx|
% \end{quote}
%
% \paragraph{TDS.} Now the different files must be moved into
% the different directories in your installation TDS tree
% (also known as \xfile{texmf} tree):
% \begin{quote}
% \def\t{^^A
% \begin{tabular}{@{}>{\ttfamily}l@{ $\rightarrow$ }>{\ttfamily}l@{}}
%   kvsetkeys.sty & tex/generic/kvsetkeys/kvsetkeys.sty\\
%   kvsetkeys.pdf & doc/latex/kvsetkeys/kvsetkeys.pdf\\
%   kvsetkeys-example.tex & doc/latex/kvsetkeys/kvsetkeys-example.tex\\
%   kvsetkeys.dtx & source/latex/kvsetkeys/kvsetkeys.dtx\\
% \end{tabular}^^A
% }^^A
% \sbox0{\t}^^A
% \ifdim\wd0>\linewidth
%   \begingroup
%     \advance\linewidth by\leftmargin
%     \advance\linewidth by\rightmargin
%   \edef\x{\endgroup
%     \def\noexpand\lw{\the\linewidth}^^A
%   }\x
%   \def\lwbox{^^A
%     \leavevmode
%     \hbox to \linewidth{^^A
%       \kern-\leftmargin\relax
%       \hss
%       \usebox0
%       \hss
%       \kern-\rightmargin\relax
%     }^^A
%   }^^A
%   \ifdim\wd0>\lw
%     \sbox0{\small\t}^^A
%     \ifdim\wd0>\linewidth
%       \ifdim\wd0>\lw
%         \sbox0{\footnotesize\t}^^A
%         \ifdim\wd0>\linewidth
%           \ifdim\wd0>\lw
%             \sbox0{\scriptsize\t}^^A
%             \ifdim\wd0>\linewidth
%               \ifdim\wd0>\lw
%                 \sbox0{\tiny\t}^^A
%                 \ifdim\wd0>\linewidth
%                   \lwbox
%                 \else
%                   \usebox0
%                 \fi
%               \else
%                 \lwbox
%               \fi
%             \else
%               \usebox0
%             \fi
%           \else
%             \lwbox
%           \fi
%         \else
%           \usebox0
%         \fi
%       \else
%         \lwbox
%       \fi
%     \else
%       \usebox0
%     \fi
%   \else
%     \lwbox
%   \fi
% \else
%   \usebox0
% \fi
% \end{quote}
% If you have a \xfile{docstrip.cfg} that configures and enables \docstrip's
% TDS installing feature, then some files can already be in the right
% place, see the documentation of \docstrip.
%
% \subsection{Refresh file name databases}
%
% If your \TeX~distribution
% (\TeX\,Live, \mikTeX, \dots) relies on file name databases, you must refresh
% these. For example, \TeX\,Live\ users run \verb|texhash| or
% \verb|mktexlsr|.
%
% \subsection{Some details for the interested}
%
% \paragraph{Unpacking with \LaTeX.}
% The \xfile{.dtx} chooses its action depending on the format:
% \begin{description}
% \item[\plainTeX:] Run \docstrip\ and extract the files.
% \item[\LaTeX:] Generate the documentation.
% \end{description}
% If you insist on using \LaTeX\ for \docstrip\ (really,
% \docstrip\ does not need \LaTeX), then inform the autodetect routine
% about your intention:
% \begin{quote}
%   \verb|latex \let\install=y\input{kvsetkeys.dtx}|
% \end{quote}
% Do not forget to quote the argument according to the demands
% of your shell.
%
% \paragraph{Generating the documentation.}
% You can use both the \xfile{.dtx} or the \xfile{.drv} to generate
% the documentation. The process can be configured by the
% configuration file \xfile{ltxdoc.cfg}. For instance, put this
% line into this file, if you want to have A4 as paper format:
% \begin{quote}
%   \verb|\PassOptionsToClass{a4paper}{article}|
% \end{quote}
% An example follows how to generate the
% documentation with pdf\LaTeX:
% \begin{quote}
%\begin{verbatim}
%pdflatex kvsetkeys.dtx
%makeindex -s gind.ist kvsetkeys.idx
%pdflatex kvsetkeys.dtx
%makeindex -s gind.ist kvsetkeys.idx
%pdflatex kvsetkeys.dtx
%\end{verbatim}
% \end{quote}
%
% \begin{thebibliography}{9}
% \bibitem{tb94wright}
%   A guide to key-value methods, Joseph Wright, second draft for
%   \href{https://www.tug.org/tugboat}{TUGBoat}, 2009-03-17.
%   \url{https://www.texdev.net/uploads/2009/03/keyval.pdf}
%
% \bibitem{keyval}
%   David Carlisle:
%   \textit{The \xpackage{keyval} package};
%   1999/03/16 v1.13;
%   \CTANpkg{keyval}.
%
% \end{thebibliography}
%
% \begin{History}
%   \begin{Version}{2006/03/06 v1.0}
%   \item
%     First version.
%   \end{Version}
%   \begin{Version}{2006/10/19 v1.1}
%   \item
%     Fix of \cs{kv@set@family@handler}.
%   \item
%     Example added.
%   \end{Version}
%   \begin{Version}{2007/09/09 v1.2}
%   \item
%     Using package \xpackage{infwarerr} for error messages.
%   \item
%     Catcode section rewritten.
%   \end{Version}
%   \begin{Version}{2007/09/29 v1.3}
%   \item
%     Normalizing and parsing of comma separated lists added.
%   \item
%     \cs{kv@normalize} rewritten.
%   \item
%     Robustness increased for normalizing and parsing,
%     e.g. for values with unmatched conditionals.
%   \item
%     \eTeX\ is used if available.
%   \item
%     Tests added for normalizing and parsing.
%   \end{Version}
%   \begin{Version}{2009/07/19 v1.4}
%   \item
%     Bug fix for \cs{kv@normalize}: unwanted space removed (Florent Chervet).
%   \end{Version}
%   \begin{Version}{2009/07/30 v1.5}
%   \item
%     Documentation addition: recommendation for Joseph Wright's
%     review article.
%   \end{Version}
%   \begin{Version}{2009/12/12 v1.6}
%   \item
%     Short info shortened.
%   \end{Version}
%   \begin{Version}{2009/12/22 v1.7}
%   \item
%     Internal optimization (\cs{KVS@CommaSpace}, \dots, \cs{KVS@EqualsSpace}).
%   \end{Version}
%   \begin{Version}{2010/01/28 v1.8}
%   \item
%     Compatibility to ini\TeX\ added.
%   \end{Version}
%   \begin{Version}{2010/03/01 v1.9}
%   \item
%     Support of \cs{par} inside values.
%   \end{Version}
%   \begin{Version}{2011/01/30 v1.10}
%   \item
%     Already loaded package files are not input in \hologo{plainTeX}.
%   \end{Version}
%   \begin{Version}{2011/03/03 v1.11}
%   \item
%     \cs{kv@break} and \cs{comma@break} added.
%   \end{Version}
%   \begin{Version}{2011/04/05 v1.12}
%   \item
%     Error message with recovery action in help message
%     (request by GL).
%   \end{Version}
%   \begin{Version}{2011/04/07 v1.13}
%   \item
%     \cs{kv@processor@default} supports package \xpackage{babel}'s
%     shorthands.
%   \item
%     \cs{kv@set@family@handler} with shorthand support.
%   \end{Version}
%   \begin{Version}{2011/06/15 v1.14}
%   \item
%     Some optimizations in token register uses (GL, HO).
%   \end{Version}
%   \begin{Version}{2011/10/18 v1.15}
%   \item
%     \cs{kv@processor@known} and \cs{kvsetknownkeys} added.
%   \item
%     \cs{kvsetkeys@expandafter} and \cs{kvsetknownkeys@expandafter} added.
%   \item
%     Family handler can report success or failure by \cs{kv@handled@true}
%     or \cs{kv@handled@false}.
%   \item
%     \cs{kv@unset@family@handler} added.
%   \end{Version}
%   \begin{Version}{2012/04/25 v1.16}
%   \item
%     \cs{kv@processor@default} and \cs{kv@processor@known} define
%     macro \cs{kv@fam} for convenience.
%   \item
%     Catcode section: Catcode setting for \texttt{+} added for \hologo{eTeX}.
%   \end{Version}
%   \begin{Version}{2016/05/16 v1.17}
%   \item
%     Documentation updates.
%   \end{Version}
%   \begin{Version}{2019/12/15 v1.18}
%   \item
%     Documentation updates.
%   \item
%     Avoid \xpackage{etexcmds} and \xpackage{infwarwerr} in \LaTeX.
%   \end{Version}
% \end{History}
%
% \PrintIndex
%
% \Finale
\endinput

%        (quote the arguments according to the demands of your shell)
%
% Documentation:
%    (a) If kvsetkeys.drv is present:
%           latex kvsetkeys.drv
%    (b) Without kvsetkeys.drv:
%           latex kvsetkeys.dtx; ...
%    The class ltxdoc loads the configuration file ltxdoc.cfg
%    if available. Here you can specify further options, e.g.
%    use A4 as paper format:
%       \PassOptionsToClass{a4paper}{article}
%
%    Programm calls to get the documentation (example):
%       pdflatex kvsetkeys.dtx
%       makeindex -s gind.ist kvsetkeys.idx
%       pdflatex kvsetkeys.dtx
%       makeindex -s gind.ist kvsetkeys.idx
%       pdflatex kvsetkeys.dtx
%
% Installation:
%    TDS:tex/generic/kvsetkeys/kvsetkeys.sty
%    TDS:doc/latex/kvsetkeys/kvsetkeys.pdf
%    TDS:doc/latex/kvsetkeys/kvsetkeys-example.tex
%    TDS:source/latex/kvsetkeys/kvsetkeys.dtx
%
%<*ignore>
\begingroup
  \catcode123=1 %
  \catcode125=2 %
  \def\x{LaTeX2e}%
\expandafter\endgroup
\ifcase 0\ifx\install y1\fi\expandafter
         \ifx\csname processbatchFile\endcsname\relax\else1\fi
         \ifx\fmtname\x\else 1\fi\relax
\else\csname fi\endcsname
%</ignore>
%<*install>
\input docstrip.tex
\Msg{************************************************************************}
\Msg{* Installation}
\Msg{* Package: kvsetkeys 2019/12/15 v1.18 Key value parser (HO)}
\Msg{************************************************************************}

\keepsilent
\askforoverwritefalse

\let\MetaPrefix\relax
\preamble

This is a generated file.

Project: kvsetkeys
Version: 2019/12/15 v1.18

Copyright (C)
   2006, 2007, 2009-2012 Heiko Oberdiek
   2016-2019 Oberdiek Package Support Group

This work may be distributed and/or modified under the
conditions of the LaTeX Project Public License, either
version 1.3c of this license or (at your option) any later
version. This version of this license is in
   https://www.latex-project.org/lppl/lppl-1-3c.txt
and the latest version of this license is in
   https://www.latex-project.org/lppl.txt
and version 1.3 or later is part of all distributions of
LaTeX version 2005/12/01 or later.

This work has the LPPL maintenance status "maintained".

The Current Maintainers of this work are
Heiko Oberdiek and the Oberdiek Package Support Group
https://github.com/ho-tex/kvsetkeys/issues


The Base Interpreter refers to any `TeX-Format',
because some files are installed in TDS:tex/generic//.

This work consists of the main source file kvsetkeys.dtx
and the derived files
   kvsetkeys.sty, kvsetkeys.pdf, kvsetkeys.ins, kvsetkeys.drv,
   kvsetkeys-example.tex, kvsetkeys-test1.tex,
   kvsetkeys-test2.tex, kvsetkeys-test3.tex,
   kvsetkeys-test4.tex.

\endpreamble
\let\MetaPrefix\DoubleperCent

\generate{%
  \file{kvsetkeys.ins}{\from{kvsetkeys.dtx}{install}}%
  \file{kvsetkeys.drv}{\from{kvsetkeys.dtx}{driver}}%
  \usedir{tex/generic/kvsetkeys}%
  \file{kvsetkeys.sty}{\from{kvsetkeys.dtx}{package}}%
  \usedir{doc/latex/kvsetkeys}%
  \file{kvsetkeys-example.tex}{\from{kvsetkeys.dtx}{example}}%
}

\catcode32=13\relax% active space
\let =\space%
\Msg{************************************************************************}
\Msg{*}
\Msg{* To finish the installation you have to move the following}
\Msg{* file into a directory searched by TeX:}
\Msg{*}
\Msg{*     kvsetkeys.sty}
\Msg{*}
\Msg{* To produce the documentation run the file `kvsetkeys.drv'}
\Msg{* through LaTeX.}
\Msg{*}
\Msg{* Happy TeXing!}
\Msg{*}
\Msg{************************************************************************}

\endbatchfile
%</install>
%<*ignore>
\fi
%</ignore>
%<*driver>
\NeedsTeXFormat{LaTeX2e}
\ProvidesFile{kvsetkeys.drv}%
  [2019/12/15 v1.18 Key value parser (HO)]%
\documentclass{ltxdoc}
\usepackage{holtxdoc}[2011/11/22]
\begin{document}
  \DocInput{kvsetkeys.dtx}%
\end{document}
%</driver>
% \fi
%
%
%
% \GetFileInfo{kvsetkeys.drv}
%
% \title{The \xpackage{kvsetkeys} package}
% \date{2019/12/15 v1.18}
% \author{Heiko Oberdiek\thanks
% {Please report any issues at \url{https://github.com/ho-tex/kvsetkeys/issues}}}
%
% \maketitle
%
% \begin{abstract}
% Package \xpackage{kvsetkeys} provides \cs{kvsetkeys}, a variant
% of package \xpackage{keyval}'s \cs{setkeys}. It allows to specify
% a handler that deals with unknown options. Active commas and equal
% signs may be used (e.g. see \xpackage{babel}'s shorthands) and
% only one level of curly braces is removed from the values.
% \end{abstract}
%
% \tableofcontents
%
% \def\M#1{\texttt{\{}\meta{#1}\texttt{\}}}
%
% \section{Documentation}
%
% First I want to recommend the very good review article
% ``A guide to key-value methods'' by Joseph Wright \cite{tb94wright}.
% It introduces the different key-value packages and compares them.
%
% \subsection{Motivation}
%
% \cs{kvsetkeys} serves as replacement for \xpackage{keyval}'s
% \cs{setkeys}. It basically uses the same syntax. But the
% implementation is more robust and predictable:
% \begin{description}
% \item[Active syntax characters:]
%   Comma `|,|' and the equals sign `|=|' are used inside
%   key value lists as syntax characters. Package \xpackage{keyval}
%   uses the catcode of the characters that is active during
%   package loading, usually this is catcode 12 (other).
%   But it can happen that the catcode setting of the syntax characters
%   changes. Especially active characters are of interest, because
%   some language adaptations uses them. For example, option \xoption{turkish}
%   of package \xpackage{babel} uses the equals sign as active shorthand
%   character. Therefore package \xpackage{kvsetkeys} deals with
%   both catcode settings 12 (other) and 13 (active).
% \item[Brace removal:]
%   Package \xpackage{keyval}'s \cs{setkeys} removes up to two
%   levels of curly braces around the value in some unpredictable way:
%\begin{quote}
%|\setkeys{fam}{key={{value}}} || || | $\rightarrow$ |value|\\
%|\setkeys{fam}{key={{{value}}}} | $\rightarrow$ |{value}|\\
%|\setkeys{fam}{key= {{{value}}}}| $\rightarrow$ |{{value}}|
%\end{quote}
%   This package \xpackage{kvsetkeys} follows a much stronger rule:
%   Exactly one level of braces are removed from an item, if the
%   item is surrounded by curly braces. An item can be a
%   the key value pair, the key or the value.
%\begin{quote}
%|\kvsetkeys{fam}{key={value}} || | $\rightarrow$ |value|\\
%|\kvsetkeys{fam}{key={{value}} | $\rightarrow$ |{value}|\\
%|\kvsetkeys{fam}{key= {{value}}| $\rightarrow$ |{value}|
%\end{quote}
% \item[Arbitrary values:] Unmatched conditionals are supported.
% \end{description}
%
% Before I describe
% \cs{kvsetkeys} in more detail, first I want to explain, how
% this package deals with key value lists. For the package also
% provides low level interfaces that can be used by package authors.
%
% \subsection{Normalizing key value lists}
%
% \begin{declcs}{kv@normalize}\,\M{key value list}
% \end{declcs}
% If the user specifies key value lists, he usually prefers
% nice formatted source code, e.g.:
% \begin{quote}
%\begin{verbatim}
%\hypersetup{
%  pdftitle    = {...},
%  pdfsubject  = {...},
%  pdfauthor   = {...},
%  pdfkeywords = {...},
%  ...
%}
%\end{verbatim}
% \end{quote}
% Thus there can be spaces around keys, around |=| or around the value.
% Also empty entries are possible by too many commas. Therefore these
% spaces and empty entries are silently removed by package \xpackage{keyval}
% and this package. Whereas the contents of the value can be protected
% by curly braces, especially if spaces or commas are used inside,
% a key name must not use spaces or other syntax characters.
%
% \cs{kv@normalize} takes a key value list and performs the cleanup:
% \begin{itemize}
% \item Spaces are removed.
% \item Syntax characters (comma and equal sign) that are active
%   are replaced by the same characters with standard catcode.
%   (Example: \xpackage{babel}'s language option \xoption{turkish}
%   uses the equal sign as active shorthand character.)
% \end{itemize}
% The result is stored in \cs{kv@list}, e.g.:
% \begin{quote}
% |\kv@list| $\rightarrow$ |,pdftitle={...},pdfsubject={...},...,|
% \end{quote}
% Curly braces around values (or keys) remain untouched.
% \begin{description}
% \item[v1.3+:]
% One comma is added in front of the list and each pair ends with
% a comma. Thus an empty list consists of one comma, otherwise
% two commas encloses the list. Empty entries other than the first
% are removed.
% \item[v1.0 -- v1.2:]
% Empty entries are removed later. In fact it adds a comma at the begin
% and end to protect the last value and an easier implementation.
% \end{description}
%
% \subsection{Parsing key value lists}
%
% \begin{declcs}{kv@parse}\,\M{key value list}\,\M{processor}
% \end{declcs}
% It is easier to parse a normalized list, thus \cs{kv@parse}
% normalizes the list and calls \cs{kv@parse@normalized}.
%
% \begin{declcs}{kv@parse@normalized}\,\M{key value list}%
%   \,\M{processor}
% \end{declcs}
% Now the key value list is split into single key value pairs.
% For further processing the key and value are given as arguments
% for the \meta{processor}:
% \begin{quote}
%   \meta{processor}\,\M{key}\,\M{value}
% \end{quote}
% Also key and value are stored in macro names:
% \begin{itemize}
% \item \cs{kv@key} stores the key.
% \item \cs{kv@value} stores the value or if the value was not
%   specified it has the meaning \cs{relax}.
% \end{itemize}
% The behaviour in pseudo code:
% \begin{quote}
%   foreach (\meta{key}, \meta{value}) in (\meta{key value list})\\
%   \hspace*{1.5em}\cs{kv@key} := \meta{key}\\
%   \hspace*{1.5em}\cs{kv@value} := \meta{value}\\
%   \hspace*{1.5em}\meta{processor}\,\M{key}\,\M{value}
% \end{quote}
%
% \begin{declcs}{kv@break}
% \end{declcs}
% Since version 2011/03/03 v1.11 \cs{kv@break} can be
% called inside the \meta{processor}
% of \cs{kv@parse} or \cs{kv@parse@normalized}, then
% the processing is stopped and the following entries discarded.
%
% \subsection{Processing key value pairs}
%
% Key value pairs can be processed in many different ways.
% For example, the processor for \cs{kvsetkeys} works similar
% to \cs{setkeys} of package \xpackage{keyval}. There unknown
% keys raise an error.
%
% Package \xpackage{xkeyval} also knows a star form of \cs{setkeys}
% that stores unknown keys in an internal macro for further processing
% with \cs{setrmkeys} and similar macros. This feature is covered
% by processor \cs{kv@processor@known}.
%
% \subsubsection{Processing similar to \xpackage{keyval}}
%
% \begin{declcs}{kv@processor@default}\,\M{family}\,\M{key}\,\M{value}
% \end{declcs}
% There are many possiblities to process key value pairs.
% \cs{kv@processor@default} is the processor used in \cs{kvsetkeys}.
% It reimplements and extends the behaviour of
% \xpackage{keyval}'s \cs{setkeys}.
% In case of unknown keys \cs{setkeys} raise an error.
% This processer, however, calls a handler instead, if it
% is provided by the family. Both \meta{family} and \meta{key}
% may contain package \xpackage{babel}'s shorthands
% (since 2011/04/07 v1.13).
%
% Since 2011/10/18 v1.15 the family handler can reject the
% successful handling of a key by calling \cs{kv@handled@false}.
%
% Since 2019/12/15 v1.18 \cs{kv@processor@default} also defines
% macro \cs{kv@fam} with meaning \meta{family} for convenience.
%
% \subsubsection{Processing similar to \cs{setkeys*} of package \xpackage{xkeyval}}
%
% \begin{declcs}{kv@processor@known}\,\M{family}\,^^A
%   \M{cmd}\,\M{key}\,\M{value}
% \end{declcs}
% The key value processor \cs{kv@processor@known} behaves similar
% to \cs{kv@processor@default}. If the \meta{key} exists in the
% \meta{family} its code is called, otherwise the family handler
% is tried. If the family handler is not set or cannot handle the
% key, the unknown key value pair is added to the macro \meta{cmd}.
% Since 2011/10/18 v1.15.
%
% The behaviour in pseudo code:
% \begin{quote}
%   if \meta{key} exists\\
%   \hspace*{1.5em}call the keyval code of \meta{key}\\
%   else\\
%   \hspace*{1.5em}if \meta{handler} for \meta{family} exists\\
%   \hspace*{3em}handled = true\\
%   \hspace*{3em}\meta{handler}\,\M{key}\,\M{value}\\
%   \hspace*{3em}if handled\\
%   \hspace*{3em}else\\
%   \hspace*{4.5em}add \texttt{"}\M{key}\texttt{=}\M{value}\texttt{"}
%     to \M{cmd}\\
%   \hspace*{3em}fi\\
%   \hspace*{1.5em}else\\
%   \hspace*{3em}add \texttt{"}\M{key}\texttt{=}\M{value}\texttt{"}
%     to \M{cmd}\\
%   \hspace*{3em}raise unknown key error\\
%   \hspace*{1.5em}fi\\
%   fi
% \end{quote}
%
% Since 2019/12/15 v1.18 \cs{kv@processor@known} also defines
% macro \cs{kv@fam} with meaning \meta{family} for convenience.
%
% \subsection{Default family handler}
%
% \cs{kv@processor@default} calls \meta{handler}, the default
% handler for the family, if the key does not exist in the family.
% The handler is called with two arguments, the key and the value.
% It can be defined with \cs{kv@set@family@hander}:
%
% \begin{declcs}{kv@set@family@handler}\,\M{family}\,\M{handler definition}
% \end{declcs}
% This sets the default family handler for the keyval family
% \meta{family}. Inside \meta{handler definition} |#1| stands for
% the key and |#2| is the value. Also \cs{kv@key} and \cs{kv@value}
% can be used for the key and the value. If the value is not given,
% \cs{kv@value} has the meaning \cs{relax}.
%
% \begin{declcs}{kv@unset@family@handler}\,\M{family}
% \end{declcs}
% It removes the family handler for \meta{family}.
% Since 2011/10/18 v1.15.
%
% \subsection{Put it all together}
%
% \begin{declcs}{kvsetkeys}\,\M{family}\,\M{key value list}
% \end{declcs}
% Macro \cs{kvsetkeys} processes the \meta{key value list} with
% the standard processor \cs{kv@processor@default}:
% \begin{quote}
%   \cs{kv@parse}\,\M{key value list}^^A
%   \texttt{\{}\cs{kv@processor@default}\,\M{family}\texttt{\}}
% \end{quote}
%
% \begin{declcs}{kvsetknownkeys}\,\M{family}\,\M{cmd}\,\M{key value list}
% \end{declcs}
% Macro \cs{kvsetknownkeys} processes the \meta{key value list}
% with processor \cs{kv@processor@known}. All key value pairs
% with keys that are not known in \meta{family} are stored
% in macro \meta{cmd}. A previous contents of macro \meta{cmd}
% will be overwritten. If all keys can be handled, \meta{cmd}
% will be empty, otherwise it contains a key value list of
% unhandled key value pairs.
% Since 2011/10/18 v1.15.
%
% Pseudo code:
% \begin{quote}
%   create macro \meta{cmdaux} with unique name (inside the current group)\\
%   \cs{def}\meta{cmdaux}\{\}\\
%   \cs{kv@parse}\,\M{key value list}^^A
%   \texttt{\{}\cs{kv@processor@known}\,\M{family}\,\M{cmdaux}\texttt{\}}\\
%   \cs{let}\meta{cmd}=\meta{cmdaux}
% \end{quote}
%
% \begin{declcs}{kvsetkeys@expandafter}\,\M{family}\,\M{list cmd}\\
%   \cs{kvsetknownkeys@expandafter}\,\M{family}\,%
%   \M{cmd}\,\M{list cmd}
% \end{declcs}
% Both macros behave like the counterparts without suffix
% |@expandafter|. The difference is that the key value list is
% given as macro that is expanded once.
% Since 2011/10/18 v1.15.
%
% Thus you can replace \cs{setkeys} of package \xpackage{keyval}
% by the key value parser of this package:
% \begin{quote}
%   |\renewcommand*{\setkeys}{\kvsetkeys}|\\
%   or\\
%   |\let\setkeys\kvsetkeys|
% \end{quote}
%
% \subsection{Comma separated lists}
%
% Since version 2007/09/29 v1.3 this package also supports the normalizing
% and parsing of general comma separated lists.
%
% \begin{declcs}{comma@normalize}\,\M{comma list}
% \end{declcs}
% Macro \cs{comma@normalize} normalizes the comma separated list,
% removes spaces around commas. The result is put in macro \cs{comma@list}.
%
% \begin{declcs}{comma@parse}\,\M{comma list}\,\M{processor}
% \end{declcs}
% Macro \cs{comma@parse} first normalizes the comma separated list
% and then parses the list by calling \cs{comma@parse@normalized}.
%
% \begin{declcs}{comma@parse@normalized}\,\M{normalized comma list}^^A
%   \,\M{processor}
% \end{declcs}
% The list is parsed. Empty entries are ignored. \meta{processor}
% is called for each non-empty entry with the entry as argument:
% \begin{quote}
%   \meta{processor}|{|\meta{entry}|}|
% \end{quote}
% Also the entry is stored in the macro \cs{comma@entry}.
%
% \begin{declcs}{comma@break}
% \end{declcs}
% Since version 2011/03/03 v1.11 \cs{comma@break} can be
% called inside the \meta{processor}
% of \cs{comma@parse} or \cs{comma@parse@normalized}, then
% the processing is stopped and the following entries discarded.
%
% \section{Example}
%
% The following example prints a short piece of HTML code using
% the tabbing environment for indenting purpose and a key value
% syntax for specifying the attributes of an HTML tag.
% The example illustrates the use of a default family handler.
%    \begin{macrocode}
%<*example>
\documentclass{article}
\usepackage[T1]{fontenc}
\usepackage{kvsetkeys}
\usepackage{keyval}

\makeatletter
\newcommand*{\tag}[2][]{%
  % #1: attributes
  % #2: tag name
  \begingroup
    \toks@={}%
    \let\@endslash\@empty
    \kvsetkeys{tag}{#1}%
    \texttt{%
      \textless #2\the\toks@\@endslash\textgreater
    }%
  \endgroup
}
\kv@set@family@handler{tag}{%
  % #1: key
  % #2: value
  \toks@\expandafter{%
    \the\toks@
    \space
    #1=\string"#2\string"%
  }%
}
\define@key{tag}{/}[]{%
  \def\@endslash{/}%
}
\makeatother

\begin{document}
\begin{tabbing}
  \mbox{}\qquad\=\qquad\=\kill
  \tag{html}\\
  \>\dots\\
  \>\tag[border=1]{table}\\
  \>\>\tag[width=200, span=3, /]{colgroup}\\
  \>\>\dots\\
  \>\tag{/table}\\
  \>\dots\\
  \tag{/html}\\
\end{tabbing}
\end{document}
%</example>
%    \end{macrocode}
%
% \StopEventually{
% }
%
% \section{Implementation}
%
% \subsection{Identification}
%
%    \begin{macrocode}
%<*package>
%    \end{macrocode}
%    Reload check, especially if the package is not used with \LaTeX.
%    \begin{macrocode}
\begingroup\catcode61\catcode48\catcode32=10\relax%
  \catcode13=5 % ^^M
  \endlinechar=13 %
  \catcode35=6 % #
  \catcode39=12 % '
  \catcode44=12 % ,
  \catcode45=12 % -
  \catcode46=12 % .
  \catcode58=12 % :
  \catcode64=11 % @
  \catcode123=1 % {
  \catcode125=2 % }
  \expandafter\let\expandafter\x\csname ver@kvsetkeys.sty\endcsname
  \ifx\x\relax % plain-TeX, first loading
  \else
    \def\empty{}%
    \ifx\x\empty % LaTeX, first loading,
      % variable is initialized, but \ProvidesPackage not yet seen
    \else
      \expandafter\ifx\csname PackageInfo\endcsname\relax
        \def\x#1#2{%
          \immediate\write-1{Package #1 Info: #2.}%
        }%
      \else
        \def\x#1#2{\PackageInfo{#1}{#2, stopped}}%
      \fi
      \x{kvsetkeys}{The package is already loaded}%
      \aftergroup\endinput
    \fi
  \fi
\endgroup%
%    \end{macrocode}
%    Package identification:
%    \begin{macrocode}
\begingroup\catcode61\catcode48\catcode32=10\relax%
  \catcode13=5 % ^^M
  \endlinechar=13 %
  \catcode35=6 % #
  \catcode39=12 % '
  \catcode40=12 % (
  \catcode41=12 % )
  \catcode44=12 % ,
  \catcode45=12 % -
  \catcode46=12 % .
  \catcode47=12 % /
  \catcode58=12 % :
  \catcode64=11 % @
  \catcode91=12 % [
  \catcode93=12 % ]
  \catcode123=1 % {
  \catcode125=2 % }
  \expandafter\ifx\csname ProvidesPackage\endcsname\relax
    \def\x#1#2#3[#4]{\endgroup
      \immediate\write-1{Package: #3 #4}%
      \xdef#1{#4}%
    }%
  \else
    \def\x#1#2[#3]{\endgroup
      #2[{#3}]%
      \ifx#1\@undefined
        \xdef#1{#3}%
      \fi
      \ifx#1\relax
        \xdef#1{#3}%
      \fi
    }%
  \fi
\expandafter\x\csname ver@kvsetkeys.sty\endcsname
\ProvidesPackage{kvsetkeys}%
  [2019/12/15 v1.18 Key value parser (HO)]%
%    \end{macrocode}
%
%    \begin{macrocode}
\begingroup\catcode61\catcode48\catcode32=10\relax%
  \catcode13=5 % ^^M
  \endlinechar=13 %
  \catcode123=1 % {
  \catcode125=2 % }
  \catcode64=11 % @
  \def\x{\endgroup
    \expandafter\edef\csname KVS@AtEnd\endcsname{%
      \endlinechar=\the\endlinechar\relax
      \catcode13=\the\catcode13\relax
      \catcode32=\the\catcode32\relax
      \catcode35=\the\catcode35\relax
      \catcode61=\the\catcode61\relax
      \catcode64=\the\catcode64\relax
      \catcode123=\the\catcode123\relax
      \catcode125=\the\catcode125\relax
    }%
  }%
\x\catcode61\catcode48\catcode32=10\relax%
\catcode13=5 % ^^M
\endlinechar=13 %
\catcode35=6 % #
\catcode64=11 % @
\catcode123=1 % {
\catcode125=2 % }
\def\TMP@EnsureCode#1#2{%
  \edef\KVS@AtEnd{%
    \KVS@AtEnd
    \catcode#1=\the\catcode#1\relax
  }%
  \catcode#1=#2\relax
}
\TMP@EnsureCode{36}{3}% $
\TMP@EnsureCode{38}{4}% &
\TMP@EnsureCode{39}{12}% '
\TMP@EnsureCode{43}{12}% +
\TMP@EnsureCode{44}{12}% ,
\TMP@EnsureCode{45}{12}% -
\TMP@EnsureCode{46}{12}% .
\TMP@EnsureCode{47}{12}% /
\TMP@EnsureCode{91}{12}% [
\TMP@EnsureCode{93}{12}% ]
\TMP@EnsureCode{94}{7}% ^ (superscript)
\TMP@EnsureCode{96}{12}% `
\TMP@EnsureCode{126}{13}% ~ (active)
\edef\KVS@AtEnd{\KVS@AtEnd\noexpand\endinput}
%    \end{macrocode}
%
% \subsection{Package loading}
%
%    \begin{macrocode}
\begingroup\expandafter\expandafter\expandafter\endgroup
\expandafter\ifx\csname RequirePackage\endcsname\relax
  \def\TMP@RequirePackage#1[#2]{%
    \begingroup\expandafter\expandafter\expandafter\endgroup
    \expandafter\ifx\csname ver@#1.sty\endcsname\relax
      \input #1.sty\relax
    \fi
  }%
  \TMP@RequirePackage{infwarerr}[2007/09/09]%
  \let\PackageError\@PackageError
\else
\fi
%    \end{macrocode}
%
%    \begin{macrocode}
\expandafter\ifx\csname toks@\endcsname\relax
  \toksdef\toks@=0 %
\fi
%    \end{macrocode}
%
% \subsection{Check for \eTeX}
%
%    \cs{unexpanded}, \cs{ifcsname}, and \cs{unless} are used if found.
%    \begin{macrocode}
\ifx\numexpr\@undefined
  \catcode`\$=9 % ignore
  \catcode`\&=14 % comment
\else % e-TeX
  \catcode`\$=14 % comment
  \catcode`\&=9  % ignore
\fi
%    \end{macrocode}
%
% \subsection{Generic help macros}
%
%    \begin{macro}{\KVS@Empty}
%    \begin{macrocode}
\def\KVS@Empty{}
%    \end{macrocode}
%    \end{macro}
%    \begin{macro}{\KVS@FirstOfTwo}
%    \begin{macrocode}
\long\def\KVS@FirstOfTwo#1#2{#1}
%    \end{macrocode}
%    \end{macro}
%    \begin{macro}{\KVS@SecondOfTwo}
%    \begin{macrocode}
\long\def\KVS@SecondOfTwo#1#2{#2}
%    \end{macrocode}
%    \end{macro}
%
%    \begin{macro}{\KVS@IfEmpty}
%    \begin{macrocode}
\long\def\KVS@IfEmpty#1{%
& \edef\KVS@Temp{\unexpanded{#1}}%
$ \begingroup
$   \toks@{#1}%
$   \edef\KVS@Temp{\the\toks@}%
$ \expandafter\endgroup
  \ifx\KVS@Temp\KVS@Empty
    \expandafter\KVS@FirstOfTwo
  \else
    \expandafter\KVS@SecondOfTwo
  \fi
}
%    \end{macrocode}
%    \end{macro}
%
% \subsection{Normalizing}
%
%    \begin{macro}{\kv@normalize}
%    \begin{macrocode}
\long\def\kv@normalize#1{%
  \begingroup
    \toks@{,#1,}%
    \KVS@Comma
    \KVS@SpaceComma
    \KVS@CommaSpace
    \KVS@CommaComma
    \KVS@Equals
    \KVS@SpaceEquals
    \KVS@EqualsSpace
    \xdef\KVS@Global{\the\toks@}%
  \endgroup
  \let\kv@list\KVS@Global
}
%    \end{macrocode}
%    \end{macro}
%    \begin{macro}{\comma@normalize}
%    \begin{macrocode}
\def\comma@normalize#1{%
  \begingroup
    \toks@{,#1,}%
    \KVS@Comma
    \KVS@SpaceComma
    \KVS@CommaSpace
    \KVS@CommaComma
    \xdef\KVS@Global{\the\toks@}%
  \endgroup
  \let\comma@list\KVS@Global
}
%    \end{macrocode}
%    \end{macro}
%    \begin{macro}{\KVS@Comma}
%    Converts active commas into comma with catcode other.
%    Also adds a comma at the end to protect the last value
%    for next cleanup steps.
%    \begin{macrocode}
\begingroup
  \lccode`\,=`\,%
  \lccode`\~=`\,%
\lowercase{\endgroup
  \def\KVS@Comma{%
    \toks@\expandafter{\expandafter}\expandafter
    \KVS@@Comma\the\toks@~\KVS@Nil
  }%
  \long\def\KVS@@Comma#1~#2\KVS@Nil{%
    \toks@\expandafter{\the\toks@#1}%
    \KVS@IfEmpty{#2}{%
    }{%
      \KVS@@Comma,#2\KVS@Nil
    }%
  }%
}
%    \end{macrocode}
%    \end{macro}
%    \begin{macro}{\KVS@SpaceComma}
%    Removes spaces before the comma, may add commas at the end.
%    \begin{macrocode}
\def\KVS@SpaceComma#1{%
  \def\KVS@SpaceComma{%
    \expandafter\KVS@@SpaceComma\the\toks@#1,\KVS@Nil
  }%
}
\KVS@SpaceComma{ }
%    \end{macrocode}
%    \end{macro}
%    \begin{macro}{\KVS@@SpaceComma}
%    \begin{macrocode}
\long\def\KVS@@SpaceComma#1 ,#2\KVS@Nil{%
  \KVS@IfEmpty{#2}{%
    \toks@{#1}%
  }{%
    \KVS@@SpaceComma#1,#2\KVS@Nil
  }%
}
%    \end{macrocode}
%    \end{macro}
%    \begin{macro}{\KVS@CommaSpace}
%    Removes spaces after the comma, may add commas at the end.
%    \begin{macrocode}
\def\KVS@CommaSpace{%
  \expandafter\KVS@@CommaSpace\the\toks@, \KVS@Nil
}
%    \end{macrocode}
%    \end{macro}
%    \begin{macro}{\KVS@@CommaSpace}
%    \begin{macrocode}
\long\def\KVS@@CommaSpace#1, #2\KVS@Nil{%
  \KVS@IfEmpty{#2}{%
    \toks@{#1}%
  }{%
    \KVS@@CommaSpace#1,#2\KVS@Nil
  }%
}
%    \end{macrocode}
%    \end{macro}
%    \begin{macro}{\KVS@CommaComma}
%    Replaces multiple commas by one comma.
%    \begin{macrocode}
\def\KVS@CommaComma{%
  \expandafter\KVS@@CommaComma\the\toks@,\KVS@Nil
}
%    \end{macrocode}
%    \end{macro}
%    \begin{macro}{\KVS@@CommaComma}
%    \begin{macrocode}
\long\def\KVS@@CommaComma#1,,#2\KVS@Nil{%
  \KVS@IfEmpty{#2}{%
    \toks@{#1,}% (!)
  }{%
    \KVS@@CommaComma#1,#2\KVS@Nil
  }%
}
%    \end{macrocode}
%    \end{macro}
%
%    \begin{macro}{\KVS@Equals}
%    Converts active equals signs into catcode other characters.
%    \begin{macrocode}
\begingroup
  \lccode`\==`\=%
  \lccode`\~=`\=%
\lowercase{\endgroup
  \def\KVS@Equals{%
    \toks@\expandafter{\expandafter}\expandafter
    \KVS@@Equals\the\toks@~\KVS@Nil
  }%
  \long\def\KVS@@Equals#1~#2\KVS@Nil{%
    \edef\KVS@Temp{\the\toks@}%
    \ifx\KVS@Temp\KVS@Empty
      \expandafter\KVS@FirstOfTwo
    \else
      \expandafter\KVS@SecondOfTwo
    \fi
    {%
      \toks@{#1}%
    }{%
      \toks@\expandafter{\the\toks@=#1}%
    }%
    \KVS@IfEmpty{#2}{%
    }{%
      \KVS@@Equals#2\KVS@Nil
    }%
  }%
}
%    \end{macrocode}
%    \end{macro}
%    \begin{macro}{\KVS@SpaceEquals}
%    Removes spaces before the equals sign.
%    \begin{macrocode}
\def\KVS@SpaceEquals#1{%
  \def\KVS@SpaceEquals{%
    \expandafter\KVS@@SpaceEquals\the\toks@#1=\KVS@Nil
  }%
}
\KVS@SpaceEquals{ }
%    \end{macrocode}
%    \end{macro}
%    \begin{macro}{\KVS@@SpaceEquals}
%    \begin{macrocode}
\long\def\KVS@@SpaceEquals#1 =#2\KVS@Nil{%
  \KVS@IfEmpty{#2}{%
    \toks@{#1}%
  }{%
    \KVS@@SpaceEquals#1=#2\KVS@Nil
  }%
}
%    \end{macrocode}
%    \end{macro}
%    \begin{macro}{\KVS@EqualsSpace}
%    Removes spaces after the equals sign.
%    \begin{macrocode}
\def\KVS@EqualsSpace{%
  \expandafter\KVS@@EqualsSpace\the\toks@= \KVS@Nil
}
%    \end{macrocode}
%    \end{macro}
%    \begin{macro}{\KVS@@EqualsSpace}
%    \begin{macrocode}
\long\def\KVS@@EqualsSpace#1= #2\KVS@Nil{%
  \KVS@IfEmpty{#2}{%
    \toks@{#1}%
  }{%
    \KVS@@EqualsSpace#1=#2\KVS@Nil
  }%
}
%    \end{macrocode}
%    \end{macro}
%
% \subsection{Parsing key value lists}
%
%    \begin{macro}{\kv@parse}
%    Normalizes and parses the key value list. Also sets \cs{kv@list}.
%    \begin{macrocode}
\long\def\kv@parse#1{%
  \kv@normalize{#1}%
  \expandafter\kv@parse@normalized\expandafter{\kv@list}%
}
%    \end{macrocode}
%    \end{macro}
%    \begin{macro}{\kv@parse@normalized}
%    |#1|: key value list\\
%    |#2|: processor
%    \begin{macrocode}
\long\def\kv@parse@normalized#1#2{%
  \KVS@Parse#1,\KVS@Nil{#2}%
}
%    \end{macrocode}
%    \end{macro}
%    \begin{macro}{\KVS@Parse}
%    |#1,#2|: key value list\\
%    |#3|: processor
%    \begin{macrocode}
\long\def\KVS@Parse#1,#2\KVS@Nil#3{%
  \KVS@IfEmpty{#1}{%
  }{%
    \KVS@Process#1=\KVS@Nil{#3}%
  }%
  \KVS@MaybeBreak
  \KVS@IfEmpty{#2}{%
  }{%
    \KVS@Parse#2\KVS@Nil{#3}%
  }%
}
%    \end{macrocode}
%    \end{macro}
%    \begin{macro}{\KVS@Process}
%    |#1|: key\\
%    |#2|: value, |=|\\
%    |#3|: processor
%    \begin{macrocode}
\long\def\KVS@Process#1=#2\KVS@Nil#3{%
  \let\KVS@MaybeBreak\relax
  \def\kv@key{#1}%
  \KVS@IfEmpty{#2}{%
    \let\kv@value\relax
    #3{#1}{}%
  }{%
    \KVS@@Process{#1}#2\KVS@Nil{#3}%
  }%
}
%    \end{macrocode}
%    \end{macro}
%    \begin{macro}{\KVS@@Process}
%    |#1|: key\\
%    |#2|: value\\
%    |#3|: processor
%    \begin{macrocode}
\long\def\KVS@@Process#1#2=\KVS@Nil#3{%
& \edef\kv@value{\unexpanded{#2}}%
$ \begingroup
$   \toks@{#2}%
$   \xdef\KVS@Global{\the\toks@}%
$ \endgroup
$ \let\kv@value\KVS@Global
  #3{#1}{#2}%
}
%    \end{macrocode}
%    \end{macro}
%    \begin{macro}{\KVS@MaybeBreak}
%    \begin{macrocode}
\let\KVS@MaybeBreak\relax
%    \end{macrocode}
%    \end{macro}
%    \begin{macro}{\KVS@break}
%    \begin{macrocode}
\def\KVS@break#1#2#3#4{%
  \let\KVS@MaybeBreak\relax
}
%    \end{macrocode}
%    \end{macro}
%    \begin{macro}{\kv@break}
%    \begin{macrocode}
\def\kv@break{%
  \let\KVS@MaybeBreak\KVS@break
}
%    \end{macrocode}
%    \end{macro}
%
% \subsection{Parsing comma lists}
%
%    \begin{macro}{\comma@parse}
%    Normalizes and parses the key value list. Also sets \cs{comma@list}.
%    \begin{macrocode}
\def\comma@parse#1{%
  \comma@normalize{#1}%
  \expandafter\comma@parse@normalized\expandafter{\comma@list}%
}
%    \end{macrocode}
%    \end{macro}
%
%    \begin{macro}{\comma@parse@normalized}
%    |#1|: comma list\\
%    |#2|: processor
%    \begin{macrocode}
\def\comma@parse@normalized#1#2{%
  \KVS@CommaParse#1,\KVS@Nil{#2}%
}
%    \end{macrocode}
%    \end{macro}
%    \begin{macro}{\KVS@CommaParse}
%    |#1,#2|: comma list\\
%    |#3|: processor
%    \begin{macrocode}
\def\KVS@CommaParse#1,#2\KVS@Nil#3{%
  \KVS@IfEmpty{#1}{%
  }{%
    \def\comma@entry{#1}%
    #3{#1}%
  }%
  \KVS@MaybeBreak
  \KVS@IfEmpty{#2}{%
  }{%
    \KVS@CommaParse#2\KVS@Nil{#3}%
  }%
}
%    \end{macrocode}
%    \end{macro}
%    \begin{macro}{\comma@break}
%    \begin{macrocode}
\def\comma@break{%
  \let\KVS@MaybeBreak\KVS@break
}
%    \end{macrocode}
%    \end{macro}
%
% \subsection{Processing key value pairs}
%
%    \begin{macro}{\kv@handled@false}
%    The handler can call \cs{kv@handled@false} or
%    \cs{kv@handled@true} so report failure or success.
%    The default is success (compatibility for versions
%    before 2011/10/18 v1.15).
%    \begin{macrocode}
\def\kv@handled@false{%
  \let\ifkv@handled@\iffalse
}
%    \end{macrocode}
%    \end{macro}
%    \begin{macro}{\kv@handled@true}
%    \begin{macrocode}
\def\kv@handled@true{%
  \let\ifkv@handled@\iftrue
}
%    \end{macrocode}
%    \end{macro}
%    \begin{macro}{\ifkv@handled@}
%    \begin{macrocode}
\kv@handled@true
%    \end{macrocode}
%    \end{macro}
%
%    \begin{macro}{\kv@processor@default}
%    \begin{macrocode}
\def\kv@processor@default#1#2{%
  \begingroup
    \csname @safe@activestrue\endcsname
    \let\ifincsname\iftrue
    \edef\KVS@temp{\endgroup
      \noexpand\KVS@ProcessorDefault{#1}{#2}%
    }%
  \KVS@temp
}
%    \end{macrocode}
%    \end{macro}
%    \begin{macro}{\KVS@ProcessorDefault}
%    \begin{macrocode}
\long\def\KVS@ProcessorDefault#1#2#3{%
  \def\kv@fam{#1}%
& \unless\ifcsname KV@#1@#2\endcsname
$ \begingroup\expandafter\expandafter\expandafter\endgroup
$ \expandafter\ifx\csname KV@#1@#2\endcsname\relax
&   \unless\ifcsname KVS@#1@handler\endcsname
$   \begingroup\expandafter\expandafter\expandafter\endgroup
$   \expandafter\ifx\csname KVS@#1@handler\endcsname\relax
      \kv@error@unknownkey{#1}{#2}%
    \else
      \kv@handled@true
      \csname KVS@#1@handler\endcsname{#2}{#3}%
      \relax
      \ifkv@handled@
      \else
        \kv@error@unknownkey{#1}{#2}%
      \fi
    \fi
  \else
    \ifx\kv@value\relax
&     \unless\ifcsname KV@#1@#2@default\endcsname
$     \begingroup\expandafter\expandafter\expandafter\endgroup
$     \expandafter\ifx\csname KV@#1@#2@default\endcsname\relax
        \kv@error@novalue{#1}{#2}%
      \else
        \csname KV@#1@#2@default\endcsname
        \relax
      \fi
    \else
      \csname KV@#1@#2\endcsname{#3}%
    \fi
  \fi
}
%    \end{macrocode}
%    \end{macro}
%
%    \begin{macro}{\kv@processor@known}
%    \begin{macrocode}
\def\kv@processor@known#1#2#3{%
  \begingroup
    \csname @safe@activestrue\endcsname
    \let\ifincsname\iftrue
    \edef\KVS@temp{\endgroup
      \noexpand\KVS@ProcessorKnown{#1}\noexpand#2{#3}%
    }%
  \KVS@temp
}
%    \end{macrocode}
%    \end{macro}
%    \begin{macro}{\KVS@ProcessorKnown}
%    \begin{macrocode}
\long\def\KVS@ProcessorKnown#1#2#3#4{%
  \def\kv@fam{#1}%
& \unless\ifcsname KV@#1@#3\endcsname
$ \begingroup\expandafter\expandafter\expandafter\endgroup
$ \expandafter\ifx\csname KV@#1@#3\endcsname\relax
&   \unless\ifcsname KVS@#1@handler\endcsname
$   \begingroup\expandafter\expandafter\expandafter\endgroup
$   \expandafter\ifx\csname KVS@#1@handler\endcsname\relax
      \KVS@AddUnhandled#2{#3}{#4}%
    \else
      \kv@handled@true
      \csname KVS@#1@handler\endcsname{#3}{#4}%
      \relax
      \ifkv@handled@
      \else
        \KVS@AddUnhandled#2{#3}{#4}%
      \fi
    \fi
  \else
    \ifx\kv@value\relax
&     \unless\ifcsname KV@#1@#2@default\endcsname
$     \begingroup\expandafter\expandafter\expandafter\endgroup
$     \expandafter\ifx\csname KV@#1@#3@default\endcsname\relax
        \kv@error@novalue{#1}{#3}%
      \else
        \csname KV@#1@#3@default\endcsname
        \relax
      \fi
    \else
      \csname KV@#1@#3\endcsname{#4}%
    \fi
  \fi
}
%    \end{macrocode}
%    \end{macro}
%    \begin{macro}{\KVS@AddUnhandled}
%    \begin{macrocode}
\long\def\KVS@AddUnhandled#1#2#3{%
& \edef#1{%
&   \ifx#1\KVS@empty
&   \else
&     \unexpanded{#1},%
&   \fi
&   \unexpanded{{#2}={#3}}%
& }%
$ \begingroup
$   \ifx#1\KVS@empty
$     \toks@{{#2}={#3}}%
$   \else
$     \toks@\expandafter{#1,{#2}={#3}}%
$   \fi
$   \xdef\KVS@Global{\the\toks@}%
$ \endgroup
$ \let#1\KVS@Global
}
%    \end{macrocode}
%    \end{macro}
%
%    \begin{macro}{\kv@set@family@handler}
%    \begin{macrocode}
\long\def\kv@set@family@handler#1#2{%
  \begingroup
    \csname @safe@activestrue\endcsname
    \let\ifincsname\iftrue
  \expandafter\endgroup
  \expandafter\def\csname KVS@#1@handler\endcsname##1##2{#2}%
}
%    \end{macrocode}
%    \end{macro}
%
%    \begin{macro}{\kv@unset@family@handler}
%    \begin{macrocode}
\long\def\kv@unset@family@handler#1#2{%
  \begingroup
    \csname @safe@activestrue\endcsname
    \let\ifincsname\iftrue
  \expandafter\endgroup
  \expandafter\let\csname KVS@#1@handler\endcsname\@UnDeFiNeD
}
%    \end{macrocode}
%    \end{macro}
%
% \subsection{Error handling}
%
%    \begin{macro}{\kv@error@novalue}
%    \begin{macrocode}
\def\kv@error@novalue{%
  \kv@error@generic{No value specified for}%
}
%    \end{macrocode}
%    \end{macro}
%    \begin{macro}{\kv@error@unknownkey}
%    \begin{macrocode}
\def\kv@error@unknownkey{%
  \kv@error@generic{Undefined}%
}
%    \end{macrocode}
%    \end{macro}
%    \begin{macro}{\kv@error@generic}
%    \begin{macrocode}
\def\kv@error@generic#1#2#3{%
  \PackageError{kvsetkeys}{%
    #1 key `#3'%
  }{%
    The keyval family of the key `#3' is `#2'.\MessageBreak
    The setting of the key is ignored because of the error.\MessageBreak
    \MessageBreak
    \@ehc
  }%
}
%    \end{macrocode}
%    \end{macro}
%
% \subsection{Do it all}
%
%    \begin{macro}{\kvsetkeys}
%    \begin{macrocode}
\long\def\kvsetkeys#1#2{%
  \kv@parse{#2}{\kv@processor@default{#1}}%
}
%    \end{macrocode}
%    \end{macro}
%
%    \begin{macro}{\kvsetkeys@expandafter}
%    \begin{macrocode}
\def\kvsetkeys@expandafter#1#2{%
  \expandafter\kv@parse\expandafter{#2}{%
    \kv@processor@default{#1}%
  }%
}
%    \end{macrocode}
%    \end{macro}
%
%    \begin{macro}{\KVS@cmd}
%    \begin{macrocode}
\def\KVS@cmd{0}%
%    \end{macrocode}
%    \end{macro}
%    \begin{macro}{\KVS@cmd@inc}
%    \begin{macrocode}
\def\KVS@cmd@inc{%
& \edef\KVS@cmd{\the\numexpr\KVS@cmd+1}%
$ \begingroup
$   \count255=\KVS@cmd\relax
$   \advance\count255 by 1\relax
$ \edef\x{\endgroup
$   \noexpand\def\noexpand\KVS@cmd{\number\count255}%
$ }%
$ \x
}
%    \end{macrocode}
%    \end{macro}
%    \begin{macro}{\KVS@cmd@dec}
%    \begin{macrocode}
\def\KVS@cmd@dec{%
& \edef\KVS@cmd{\the\numexpr\KVS@cmd-1}%
$ \begingroup
$   \count255=\KVS@cmd\relax
$   \advance\count255 by -1\relax
$ \edef\x{\endgroup
$   \noexpand\def\noexpand\KVS@cmd{\number\count255}%
$ }%
$ \x
}
%    \end{macrocode}
%    \end{macro}
%    \begin{macro}{\KVS@empty}
%    \begin{macrocode}
\def\KVS@empty{}
%    \end{macrocode}
%    \end{macro}
%
%    \begin{macro}{\kvsetknownkeys}
%    \begin{macrocode}
\def\kvsetknownkeys{%
  \expandafter
  \KVS@setknownkeys\csname KVS@cmd\KVS@cmd\endcsname{}%
}
%    \end{macrocode}
%    \end{macro}
%    \begin{macro}{\KVS@setknownkeys}
%    \begin{macrocode}
\long\def\KVS@setknownkeys#1#2#3#4#5{%
  \let#1\KVS@empty
  \KVS@cmd@inc
  #2\kv@parse#2{#5}{\kv@processor@known{#3}#1}%
  \KVS@cmd@dec
  \let#4=#1%
}
%    \end{macrocode}
%    \end{macro}
%
%    \begin{macro}{\kvsetknownkeys@expandafter}
%    \begin{macrocode}
\def\kvsetknownkeys@expandafter{%
  \expandafter
  \KVS@setknownkeys
      \csname KVS@cmd\KVS@cmd\endcsname\expandafter
}
%    \end{macrocode}
%    \end{macro}
%
%    \begin{macrocode}
\KVS@AtEnd%
%</package>
%    \end{macrocode}
%% \section{Installation}
%
% \subsection{Download}
%
% \paragraph{Package.} This package is available on
% CTAN\footnote{\CTANpkg{kvsetkeys}}:
% \begin{description}
% \item[\CTAN{macros/latex/contrib/kvsetkeys/kvsetkeys.dtx}] The source file.
% \item[\CTAN{macros/latex/contrib/kvsetkeys/kvsetkeys.pdf}] Documentation.
% \end{description}
%
%
% \paragraph{Bundle.} All the packages of the bundle `kvsetkeys'
% are also available in a TDS compliant ZIP archive. There
% the packages are already unpacked and the documentation files
% are generated. The files and directories obey the TDS standard.
% \begin{description}
% \item[\CTANinstall{install/macros/latex/contrib/kvsetkeys.tds.zip}]
% \end{description}
% \emph{TDS} refers to the standard ``A Directory Structure
% for \TeX\ Files'' (\CTANpkg{tds}). Directories
% with \xfile{texmf} in their name are usually organized this way.
%
% \subsection{Bundle installation}
%
% \paragraph{Unpacking.} Unpack the \xfile{kvsetkeys.tds.zip} in the
% TDS tree (also known as \xfile{texmf} tree) of your choice.
% Example (linux):
% \begin{quote}
%   |unzip kvsetkeys.tds.zip -d ~/texmf|
% \end{quote}
%
% \subsection{Package installation}
%
% \paragraph{Unpacking.} The \xfile{.dtx} file is a self-extracting
% \docstrip\ archive. The files are extracted by running the
% \xfile{.dtx} through \plainTeX:
% \begin{quote}
%   \verb|tex kvsetkeys.dtx|
% \end{quote}
%
% \paragraph{TDS.} Now the different files must be moved into
% the different directories in your installation TDS tree
% (also known as \xfile{texmf} tree):
% \begin{quote}
% \def\t{^^A
% \begin{tabular}{@{}>{\ttfamily}l@{ $\rightarrow$ }>{\ttfamily}l@{}}
%   kvsetkeys.sty & tex/generic/kvsetkeys/kvsetkeys.sty\\
%   kvsetkeys.pdf & doc/latex/kvsetkeys/kvsetkeys.pdf\\
%   kvsetkeys-example.tex & doc/latex/kvsetkeys/kvsetkeys-example.tex\\
%   kvsetkeys.dtx & source/latex/kvsetkeys/kvsetkeys.dtx\\
% \end{tabular}^^A
% }^^A
% \sbox0{\t}^^A
% \ifdim\wd0>\linewidth
%   \begingroup
%     \advance\linewidth by\leftmargin
%     \advance\linewidth by\rightmargin
%   \edef\x{\endgroup
%     \def\noexpand\lw{\the\linewidth}^^A
%   }\x
%   \def\lwbox{^^A
%     \leavevmode
%     \hbox to \linewidth{^^A
%       \kern-\leftmargin\relax
%       \hss
%       \usebox0
%       \hss
%       \kern-\rightmargin\relax
%     }^^A
%   }^^A
%   \ifdim\wd0>\lw
%     \sbox0{\small\t}^^A
%     \ifdim\wd0>\linewidth
%       \ifdim\wd0>\lw
%         \sbox0{\footnotesize\t}^^A
%         \ifdim\wd0>\linewidth
%           \ifdim\wd0>\lw
%             \sbox0{\scriptsize\t}^^A
%             \ifdim\wd0>\linewidth
%               \ifdim\wd0>\lw
%                 \sbox0{\tiny\t}^^A
%                 \ifdim\wd0>\linewidth
%                   \lwbox
%                 \else
%                   \usebox0
%                 \fi
%               \else
%                 \lwbox
%               \fi
%             \else
%               \usebox0
%             \fi
%           \else
%             \lwbox
%           \fi
%         \else
%           \usebox0
%         \fi
%       \else
%         \lwbox
%       \fi
%     \else
%       \usebox0
%     \fi
%   \else
%     \lwbox
%   \fi
% \else
%   \usebox0
% \fi
% \end{quote}
% If you have a \xfile{docstrip.cfg} that configures and enables \docstrip's
% TDS installing feature, then some files can already be in the right
% place, see the documentation of \docstrip.
%
% \subsection{Refresh file name databases}
%
% If your \TeX~distribution
% (\TeX\,Live, \mikTeX, \dots) relies on file name databases, you must refresh
% these. For example, \TeX\,Live\ users run \verb|texhash| or
% \verb|mktexlsr|.
%
% \subsection{Some details for the interested}
%
% \paragraph{Unpacking with \LaTeX.}
% The \xfile{.dtx} chooses its action depending on the format:
% \begin{description}
% \item[\plainTeX:] Run \docstrip\ and extract the files.
% \item[\LaTeX:] Generate the documentation.
% \end{description}
% If you insist on using \LaTeX\ for \docstrip\ (really,
% \docstrip\ does not need \LaTeX), then inform the autodetect routine
% about your intention:
% \begin{quote}
%   \verb|latex \let\install=y% \iffalse meta-comment
%
% File: kvsetkeys.dtx
% Version: 2019/12/15 v1.18
% Info: Key value parser
%
% Copyright (C)
%    2006, 2007, 2009-2012 Heiko Oberdiek
%    2016-2019 Oberdiek Package Support Group
%    https://github.com/ho-tex/kvsetkeys/issues
%
% This work may be distributed and/or modified under the
% conditions of the LaTeX Project Public License, either
% version 1.3c of this license or (at your option) any later
% version. This version of this license is in
%    https://www.latex-project.org/lppl/lppl-1-3c.txt
% and the latest version of this license is in
%    https://www.latex-project.org/lppl.txt
% and version 1.3 or later is part of all distributions of
% LaTeX version 2005/12/01 or later.
%
% This work has the LPPL maintenance status "maintained".
%
% The Current Maintainers of this work are
% Heiko Oberdiek and the Oberdiek Package Support Group
% https://github.com/ho-tex/kvsetkeys/issues
%
% The Base Interpreter refers to any `TeX-Format',
% because some files are installed in TDS:tex/generic//.
%
% This work consists of the main source file kvsetkeys.dtx
% and the derived files
%    kvsetkeys.sty, kvsetkeys.pdf, kvsetkeys.ins, kvsetkeys.drv,
%    kvsetkeys-example.tex, kvsetkeys-test1.tex,
%    kvsetkeys-test2.tex, kvsetkeys-test3.tex,
%    kvsetkeys-test4.tex.
%
% Distribution:
%    CTAN:macros/latex/contrib/kvsetkeys/kvsetkeys.dtx
%    CTAN:macros/latex/contrib/kvsetkeys/kvsetkeys.pdf
%
% Unpacking:
%    (a) If kvsetkeys.ins is present:
%           tex kvsetkeys.ins
%    (b) Without kvsetkeys.ins:
%           tex kvsetkeys.dtx
%    (c) If you insist on using LaTeX
%           latex \let\install=y\input{kvsetkeys.dtx}
%        (quote the arguments according to the demands of your shell)
%
% Documentation:
%    (a) If kvsetkeys.drv is present:
%           latex kvsetkeys.drv
%    (b) Without kvsetkeys.drv:
%           latex kvsetkeys.dtx; ...
%    The class ltxdoc loads the configuration file ltxdoc.cfg
%    if available. Here you can specify further options, e.g.
%    use A4 as paper format:
%       \PassOptionsToClass{a4paper}{article}
%
%    Programm calls to get the documentation (example):
%       pdflatex kvsetkeys.dtx
%       makeindex -s gind.ist kvsetkeys.idx
%       pdflatex kvsetkeys.dtx
%       makeindex -s gind.ist kvsetkeys.idx
%       pdflatex kvsetkeys.dtx
%
% Installation:
%    TDS:tex/generic/kvsetkeys/kvsetkeys.sty
%    TDS:doc/latex/kvsetkeys/kvsetkeys.pdf
%    TDS:doc/latex/kvsetkeys/kvsetkeys-example.tex
%    TDS:source/latex/kvsetkeys/kvsetkeys.dtx
%
%<*ignore>
\begingroup
  \catcode123=1 %
  \catcode125=2 %
  \def\x{LaTeX2e}%
\expandafter\endgroup
\ifcase 0\ifx\install y1\fi\expandafter
         \ifx\csname processbatchFile\endcsname\relax\else1\fi
         \ifx\fmtname\x\else 1\fi\relax
\else\csname fi\endcsname
%</ignore>
%<*install>
\input docstrip.tex
\Msg{************************************************************************}
\Msg{* Installation}
\Msg{* Package: kvsetkeys 2019/12/15 v1.18 Key value parser (HO)}
\Msg{************************************************************************}

\keepsilent
\askforoverwritefalse

\let\MetaPrefix\relax
\preamble

This is a generated file.

Project: kvsetkeys
Version: 2019/12/15 v1.18

Copyright (C)
   2006, 2007, 2009-2012 Heiko Oberdiek
   2016-2019 Oberdiek Package Support Group

This work may be distributed and/or modified under the
conditions of the LaTeX Project Public License, either
version 1.3c of this license or (at your option) any later
version. This version of this license is in
   https://www.latex-project.org/lppl/lppl-1-3c.txt
and the latest version of this license is in
   https://www.latex-project.org/lppl.txt
and version 1.3 or later is part of all distributions of
LaTeX version 2005/12/01 or later.

This work has the LPPL maintenance status "maintained".

The Current Maintainers of this work are
Heiko Oberdiek and the Oberdiek Package Support Group
https://github.com/ho-tex/kvsetkeys/issues


The Base Interpreter refers to any `TeX-Format',
because some files are installed in TDS:tex/generic//.

This work consists of the main source file kvsetkeys.dtx
and the derived files
   kvsetkeys.sty, kvsetkeys.pdf, kvsetkeys.ins, kvsetkeys.drv,
   kvsetkeys-example.tex, kvsetkeys-test1.tex,
   kvsetkeys-test2.tex, kvsetkeys-test3.tex,
   kvsetkeys-test4.tex.

\endpreamble
\let\MetaPrefix\DoubleperCent

\generate{%
  \file{kvsetkeys.ins}{\from{kvsetkeys.dtx}{install}}%
  \file{kvsetkeys.drv}{\from{kvsetkeys.dtx}{driver}}%
  \usedir{tex/generic/kvsetkeys}%
  \file{kvsetkeys.sty}{\from{kvsetkeys.dtx}{package}}%
  \usedir{doc/latex/kvsetkeys}%
  \file{kvsetkeys-example.tex}{\from{kvsetkeys.dtx}{example}}%
}

\catcode32=13\relax% active space
\let =\space%
\Msg{************************************************************************}
\Msg{*}
\Msg{* To finish the installation you have to move the following}
\Msg{* file into a directory searched by TeX:}
\Msg{*}
\Msg{*     kvsetkeys.sty}
\Msg{*}
\Msg{* To produce the documentation run the file `kvsetkeys.drv'}
\Msg{* through LaTeX.}
\Msg{*}
\Msg{* Happy TeXing!}
\Msg{*}
\Msg{************************************************************************}

\endbatchfile
%</install>
%<*ignore>
\fi
%</ignore>
%<*driver>
\NeedsTeXFormat{LaTeX2e}
\ProvidesFile{kvsetkeys.drv}%
  [2019/12/15 v1.18 Key value parser (HO)]%
\documentclass{ltxdoc}
\usepackage{holtxdoc}[2011/11/22]
\begin{document}
  \DocInput{kvsetkeys.dtx}%
\end{document}
%</driver>
% \fi
%
%
%
% \GetFileInfo{kvsetkeys.drv}
%
% \title{The \xpackage{kvsetkeys} package}
% \date{2019/12/15 v1.18}
% \author{Heiko Oberdiek\thanks
% {Please report any issues at \url{https://github.com/ho-tex/kvsetkeys/issues}}}
%
% \maketitle
%
% \begin{abstract}
% Package \xpackage{kvsetkeys} provides \cs{kvsetkeys}, a variant
% of package \xpackage{keyval}'s \cs{setkeys}. It allows to specify
% a handler that deals with unknown options. Active commas and equal
% signs may be used (e.g. see \xpackage{babel}'s shorthands) and
% only one level of curly braces is removed from the values.
% \end{abstract}
%
% \tableofcontents
%
% \def\M#1{\texttt{\{}\meta{#1}\texttt{\}}}
%
% \section{Documentation}
%
% First I want to recommend the very good review article
% ``A guide to key-value methods'' by Joseph Wright \cite{tb94wright}.
% It introduces the different key-value packages and compares them.
%
% \subsection{Motivation}
%
% \cs{kvsetkeys} serves as replacement for \xpackage{keyval}'s
% \cs{setkeys}. It basically uses the same syntax. But the
% implementation is more robust and predictable:
% \begin{description}
% \item[Active syntax characters:]
%   Comma `|,|' and the equals sign `|=|' are used inside
%   key value lists as syntax characters. Package \xpackage{keyval}
%   uses the catcode of the characters that is active during
%   package loading, usually this is catcode 12 (other).
%   But it can happen that the catcode setting of the syntax characters
%   changes. Especially active characters are of interest, because
%   some language adaptations uses them. For example, option \xoption{turkish}
%   of package \xpackage{babel} uses the equals sign as active shorthand
%   character. Therefore package \xpackage{kvsetkeys} deals with
%   both catcode settings 12 (other) and 13 (active).
% \item[Brace removal:]
%   Package \xpackage{keyval}'s \cs{setkeys} removes up to two
%   levels of curly braces around the value in some unpredictable way:
%\begin{quote}
%|\setkeys{fam}{key={{value}}} || || | $\rightarrow$ |value|\\
%|\setkeys{fam}{key={{{value}}}} | $\rightarrow$ |{value}|\\
%|\setkeys{fam}{key= {{{value}}}}| $\rightarrow$ |{{value}}|
%\end{quote}
%   This package \xpackage{kvsetkeys} follows a much stronger rule:
%   Exactly one level of braces are removed from an item, if the
%   item is surrounded by curly braces. An item can be a
%   the key value pair, the key or the value.
%\begin{quote}
%|\kvsetkeys{fam}{key={value}} || | $\rightarrow$ |value|\\
%|\kvsetkeys{fam}{key={{value}} | $\rightarrow$ |{value}|\\
%|\kvsetkeys{fam}{key= {{value}}| $\rightarrow$ |{value}|
%\end{quote}
% \item[Arbitrary values:] Unmatched conditionals are supported.
% \end{description}
%
% Before I describe
% \cs{kvsetkeys} in more detail, first I want to explain, how
% this package deals with key value lists. For the package also
% provides low level interfaces that can be used by package authors.
%
% \subsection{Normalizing key value lists}
%
% \begin{declcs}{kv@normalize}\,\M{key value list}
% \end{declcs}
% If the user specifies key value lists, he usually prefers
% nice formatted source code, e.g.:
% \begin{quote}
%\begin{verbatim}
%\hypersetup{
%  pdftitle    = {...},
%  pdfsubject  = {...},
%  pdfauthor   = {...},
%  pdfkeywords = {...},
%  ...
%}
%\end{verbatim}
% \end{quote}
% Thus there can be spaces around keys, around |=| or around the value.
% Also empty entries are possible by too many commas. Therefore these
% spaces and empty entries are silently removed by package \xpackage{keyval}
% and this package. Whereas the contents of the value can be protected
% by curly braces, especially if spaces or commas are used inside,
% a key name must not use spaces or other syntax characters.
%
% \cs{kv@normalize} takes a key value list and performs the cleanup:
% \begin{itemize}
% \item Spaces are removed.
% \item Syntax characters (comma and equal sign) that are active
%   are replaced by the same characters with standard catcode.
%   (Example: \xpackage{babel}'s language option \xoption{turkish}
%   uses the equal sign as active shorthand character.)
% \end{itemize}
% The result is stored in \cs{kv@list}, e.g.:
% \begin{quote}
% |\kv@list| $\rightarrow$ |,pdftitle={...},pdfsubject={...},...,|
% \end{quote}
% Curly braces around values (or keys) remain untouched.
% \begin{description}
% \item[v1.3+:]
% One comma is added in front of the list and each pair ends with
% a comma. Thus an empty list consists of one comma, otherwise
% two commas encloses the list. Empty entries other than the first
% are removed.
% \item[v1.0 -- v1.2:]
% Empty entries are removed later. In fact it adds a comma at the begin
% and end to protect the last value and an easier implementation.
% \end{description}
%
% \subsection{Parsing key value lists}
%
% \begin{declcs}{kv@parse}\,\M{key value list}\,\M{processor}
% \end{declcs}
% It is easier to parse a normalized list, thus \cs{kv@parse}
% normalizes the list and calls \cs{kv@parse@normalized}.
%
% \begin{declcs}{kv@parse@normalized}\,\M{key value list}%
%   \,\M{processor}
% \end{declcs}
% Now the key value list is split into single key value pairs.
% For further processing the key and value are given as arguments
% for the \meta{processor}:
% \begin{quote}
%   \meta{processor}\,\M{key}\,\M{value}
% \end{quote}
% Also key and value are stored in macro names:
% \begin{itemize}
% \item \cs{kv@key} stores the key.
% \item \cs{kv@value} stores the value or if the value was not
%   specified it has the meaning \cs{relax}.
% \end{itemize}
% The behaviour in pseudo code:
% \begin{quote}
%   foreach (\meta{key}, \meta{value}) in (\meta{key value list})\\
%   \hspace*{1.5em}\cs{kv@key} := \meta{key}\\
%   \hspace*{1.5em}\cs{kv@value} := \meta{value}\\
%   \hspace*{1.5em}\meta{processor}\,\M{key}\,\M{value}
% \end{quote}
%
% \begin{declcs}{kv@break}
% \end{declcs}
% Since version 2011/03/03 v1.11 \cs{kv@break} can be
% called inside the \meta{processor}
% of \cs{kv@parse} or \cs{kv@parse@normalized}, then
% the processing is stopped and the following entries discarded.
%
% \subsection{Processing key value pairs}
%
% Key value pairs can be processed in many different ways.
% For example, the processor for \cs{kvsetkeys} works similar
% to \cs{setkeys} of package \xpackage{keyval}. There unknown
% keys raise an error.
%
% Package \xpackage{xkeyval} also knows a star form of \cs{setkeys}
% that stores unknown keys in an internal macro for further processing
% with \cs{setrmkeys} and similar macros. This feature is covered
% by processor \cs{kv@processor@known}.
%
% \subsubsection{Processing similar to \xpackage{keyval}}
%
% \begin{declcs}{kv@processor@default}\,\M{family}\,\M{key}\,\M{value}
% \end{declcs}
% There are many possiblities to process key value pairs.
% \cs{kv@processor@default} is the processor used in \cs{kvsetkeys}.
% It reimplements and extends the behaviour of
% \xpackage{keyval}'s \cs{setkeys}.
% In case of unknown keys \cs{setkeys} raise an error.
% This processer, however, calls a handler instead, if it
% is provided by the family. Both \meta{family} and \meta{key}
% may contain package \xpackage{babel}'s shorthands
% (since 2011/04/07 v1.13).
%
% Since 2011/10/18 v1.15 the family handler can reject the
% successful handling of a key by calling \cs{kv@handled@false}.
%
% Since 2019/12/15 v1.18 \cs{kv@processor@default} also defines
% macro \cs{kv@fam} with meaning \meta{family} for convenience.
%
% \subsubsection{Processing similar to \cs{setkeys*} of package \xpackage{xkeyval}}
%
% \begin{declcs}{kv@processor@known}\,\M{family}\,^^A
%   \M{cmd}\,\M{key}\,\M{value}
% \end{declcs}
% The key value processor \cs{kv@processor@known} behaves similar
% to \cs{kv@processor@default}. If the \meta{key} exists in the
% \meta{family} its code is called, otherwise the family handler
% is tried. If the family handler is not set or cannot handle the
% key, the unknown key value pair is added to the macro \meta{cmd}.
% Since 2011/10/18 v1.15.
%
% The behaviour in pseudo code:
% \begin{quote}
%   if \meta{key} exists\\
%   \hspace*{1.5em}call the keyval code of \meta{key}\\
%   else\\
%   \hspace*{1.5em}if \meta{handler} for \meta{family} exists\\
%   \hspace*{3em}handled = true\\
%   \hspace*{3em}\meta{handler}\,\M{key}\,\M{value}\\
%   \hspace*{3em}if handled\\
%   \hspace*{3em}else\\
%   \hspace*{4.5em}add \texttt{"}\M{key}\texttt{=}\M{value}\texttt{"}
%     to \M{cmd}\\
%   \hspace*{3em}fi\\
%   \hspace*{1.5em}else\\
%   \hspace*{3em}add \texttt{"}\M{key}\texttt{=}\M{value}\texttt{"}
%     to \M{cmd}\\
%   \hspace*{3em}raise unknown key error\\
%   \hspace*{1.5em}fi\\
%   fi
% \end{quote}
%
% Since 2019/12/15 v1.18 \cs{kv@processor@known} also defines
% macro \cs{kv@fam} with meaning \meta{family} for convenience.
%
% \subsection{Default family handler}
%
% \cs{kv@processor@default} calls \meta{handler}, the default
% handler for the family, if the key does not exist in the family.
% The handler is called with two arguments, the key and the value.
% It can be defined with \cs{kv@set@family@hander}:
%
% \begin{declcs}{kv@set@family@handler}\,\M{family}\,\M{handler definition}
% \end{declcs}
% This sets the default family handler for the keyval family
% \meta{family}. Inside \meta{handler definition} |#1| stands for
% the key and |#2| is the value. Also \cs{kv@key} and \cs{kv@value}
% can be used for the key and the value. If the value is not given,
% \cs{kv@value} has the meaning \cs{relax}.
%
% \begin{declcs}{kv@unset@family@handler}\,\M{family}
% \end{declcs}
% It removes the family handler for \meta{family}.
% Since 2011/10/18 v1.15.
%
% \subsection{Put it all together}
%
% \begin{declcs}{kvsetkeys}\,\M{family}\,\M{key value list}
% \end{declcs}
% Macro \cs{kvsetkeys} processes the \meta{key value list} with
% the standard processor \cs{kv@processor@default}:
% \begin{quote}
%   \cs{kv@parse}\,\M{key value list}^^A
%   \texttt{\{}\cs{kv@processor@default}\,\M{family}\texttt{\}}
% \end{quote}
%
% \begin{declcs}{kvsetknownkeys}\,\M{family}\,\M{cmd}\,\M{key value list}
% \end{declcs}
% Macro \cs{kvsetknownkeys} processes the \meta{key value list}
% with processor \cs{kv@processor@known}. All key value pairs
% with keys that are not known in \meta{family} are stored
% in macro \meta{cmd}. A previous contents of macro \meta{cmd}
% will be overwritten. If all keys can be handled, \meta{cmd}
% will be empty, otherwise it contains a key value list of
% unhandled key value pairs.
% Since 2011/10/18 v1.15.
%
% Pseudo code:
% \begin{quote}
%   create macro \meta{cmdaux} with unique name (inside the current group)\\
%   \cs{def}\meta{cmdaux}\{\}\\
%   \cs{kv@parse}\,\M{key value list}^^A
%   \texttt{\{}\cs{kv@processor@known}\,\M{family}\,\M{cmdaux}\texttt{\}}\\
%   \cs{let}\meta{cmd}=\meta{cmdaux}
% \end{quote}
%
% \begin{declcs}{kvsetkeys@expandafter}\,\M{family}\,\M{list cmd}\\
%   \cs{kvsetknownkeys@expandafter}\,\M{family}\,%
%   \M{cmd}\,\M{list cmd}
% \end{declcs}
% Both macros behave like the counterparts without suffix
% |@expandafter|. The difference is that the key value list is
% given as macro that is expanded once.
% Since 2011/10/18 v1.15.
%
% Thus you can replace \cs{setkeys} of package \xpackage{keyval}
% by the key value parser of this package:
% \begin{quote}
%   |\renewcommand*{\setkeys}{\kvsetkeys}|\\
%   or\\
%   |\let\setkeys\kvsetkeys|
% \end{quote}
%
% \subsection{Comma separated lists}
%
% Since version 2007/09/29 v1.3 this package also supports the normalizing
% and parsing of general comma separated lists.
%
% \begin{declcs}{comma@normalize}\,\M{comma list}
% \end{declcs}
% Macro \cs{comma@normalize} normalizes the comma separated list,
% removes spaces around commas. The result is put in macro \cs{comma@list}.
%
% \begin{declcs}{comma@parse}\,\M{comma list}\,\M{processor}
% \end{declcs}
% Macro \cs{comma@parse} first normalizes the comma separated list
% and then parses the list by calling \cs{comma@parse@normalized}.
%
% \begin{declcs}{comma@parse@normalized}\,\M{normalized comma list}^^A
%   \,\M{processor}
% \end{declcs}
% The list is parsed. Empty entries are ignored. \meta{processor}
% is called for each non-empty entry with the entry as argument:
% \begin{quote}
%   \meta{processor}|{|\meta{entry}|}|
% \end{quote}
% Also the entry is stored in the macro \cs{comma@entry}.
%
% \begin{declcs}{comma@break}
% \end{declcs}
% Since version 2011/03/03 v1.11 \cs{comma@break} can be
% called inside the \meta{processor}
% of \cs{comma@parse} or \cs{comma@parse@normalized}, then
% the processing is stopped and the following entries discarded.
%
% \section{Example}
%
% The following example prints a short piece of HTML code using
% the tabbing environment for indenting purpose and a key value
% syntax for specifying the attributes of an HTML tag.
% The example illustrates the use of a default family handler.
%    \begin{macrocode}
%<*example>
\documentclass{article}
\usepackage[T1]{fontenc}
\usepackage{kvsetkeys}
\usepackage{keyval}

\makeatletter
\newcommand*{\tag}[2][]{%
  % #1: attributes
  % #2: tag name
  \begingroup
    \toks@={}%
    \let\@endslash\@empty
    \kvsetkeys{tag}{#1}%
    \texttt{%
      \textless #2\the\toks@\@endslash\textgreater
    }%
  \endgroup
}
\kv@set@family@handler{tag}{%
  % #1: key
  % #2: value
  \toks@\expandafter{%
    \the\toks@
    \space
    #1=\string"#2\string"%
  }%
}
\define@key{tag}{/}[]{%
  \def\@endslash{/}%
}
\makeatother

\begin{document}
\begin{tabbing}
  \mbox{}\qquad\=\qquad\=\kill
  \tag{html}\\
  \>\dots\\
  \>\tag[border=1]{table}\\
  \>\>\tag[width=200, span=3, /]{colgroup}\\
  \>\>\dots\\
  \>\tag{/table}\\
  \>\dots\\
  \tag{/html}\\
\end{tabbing}
\end{document}
%</example>
%    \end{macrocode}
%
% \StopEventually{
% }
%
% \section{Implementation}
%
% \subsection{Identification}
%
%    \begin{macrocode}
%<*package>
%    \end{macrocode}
%    Reload check, especially if the package is not used with \LaTeX.
%    \begin{macrocode}
\begingroup\catcode61\catcode48\catcode32=10\relax%
  \catcode13=5 % ^^M
  \endlinechar=13 %
  \catcode35=6 % #
  \catcode39=12 % '
  \catcode44=12 % ,
  \catcode45=12 % -
  \catcode46=12 % .
  \catcode58=12 % :
  \catcode64=11 % @
  \catcode123=1 % {
  \catcode125=2 % }
  \expandafter\let\expandafter\x\csname ver@kvsetkeys.sty\endcsname
  \ifx\x\relax % plain-TeX, first loading
  \else
    \def\empty{}%
    \ifx\x\empty % LaTeX, first loading,
      % variable is initialized, but \ProvidesPackage not yet seen
    \else
      \expandafter\ifx\csname PackageInfo\endcsname\relax
        \def\x#1#2{%
          \immediate\write-1{Package #1 Info: #2.}%
        }%
      \else
        \def\x#1#2{\PackageInfo{#1}{#2, stopped}}%
      \fi
      \x{kvsetkeys}{The package is already loaded}%
      \aftergroup\endinput
    \fi
  \fi
\endgroup%
%    \end{macrocode}
%    Package identification:
%    \begin{macrocode}
\begingroup\catcode61\catcode48\catcode32=10\relax%
  \catcode13=5 % ^^M
  \endlinechar=13 %
  \catcode35=6 % #
  \catcode39=12 % '
  \catcode40=12 % (
  \catcode41=12 % )
  \catcode44=12 % ,
  \catcode45=12 % -
  \catcode46=12 % .
  \catcode47=12 % /
  \catcode58=12 % :
  \catcode64=11 % @
  \catcode91=12 % [
  \catcode93=12 % ]
  \catcode123=1 % {
  \catcode125=2 % }
  \expandafter\ifx\csname ProvidesPackage\endcsname\relax
    \def\x#1#2#3[#4]{\endgroup
      \immediate\write-1{Package: #3 #4}%
      \xdef#1{#4}%
    }%
  \else
    \def\x#1#2[#3]{\endgroup
      #2[{#3}]%
      \ifx#1\@undefined
        \xdef#1{#3}%
      \fi
      \ifx#1\relax
        \xdef#1{#3}%
      \fi
    }%
  \fi
\expandafter\x\csname ver@kvsetkeys.sty\endcsname
\ProvidesPackage{kvsetkeys}%
  [2019/12/15 v1.18 Key value parser (HO)]%
%    \end{macrocode}
%
%    \begin{macrocode}
\begingroup\catcode61\catcode48\catcode32=10\relax%
  \catcode13=5 % ^^M
  \endlinechar=13 %
  \catcode123=1 % {
  \catcode125=2 % }
  \catcode64=11 % @
  \def\x{\endgroup
    \expandafter\edef\csname KVS@AtEnd\endcsname{%
      \endlinechar=\the\endlinechar\relax
      \catcode13=\the\catcode13\relax
      \catcode32=\the\catcode32\relax
      \catcode35=\the\catcode35\relax
      \catcode61=\the\catcode61\relax
      \catcode64=\the\catcode64\relax
      \catcode123=\the\catcode123\relax
      \catcode125=\the\catcode125\relax
    }%
  }%
\x\catcode61\catcode48\catcode32=10\relax%
\catcode13=5 % ^^M
\endlinechar=13 %
\catcode35=6 % #
\catcode64=11 % @
\catcode123=1 % {
\catcode125=2 % }
\def\TMP@EnsureCode#1#2{%
  \edef\KVS@AtEnd{%
    \KVS@AtEnd
    \catcode#1=\the\catcode#1\relax
  }%
  \catcode#1=#2\relax
}
\TMP@EnsureCode{36}{3}% $
\TMP@EnsureCode{38}{4}% &
\TMP@EnsureCode{39}{12}% '
\TMP@EnsureCode{43}{12}% +
\TMP@EnsureCode{44}{12}% ,
\TMP@EnsureCode{45}{12}% -
\TMP@EnsureCode{46}{12}% .
\TMP@EnsureCode{47}{12}% /
\TMP@EnsureCode{91}{12}% [
\TMP@EnsureCode{93}{12}% ]
\TMP@EnsureCode{94}{7}% ^ (superscript)
\TMP@EnsureCode{96}{12}% `
\TMP@EnsureCode{126}{13}% ~ (active)
\edef\KVS@AtEnd{\KVS@AtEnd\noexpand\endinput}
%    \end{macrocode}
%
% \subsection{Package loading}
%
%    \begin{macrocode}
\begingroup\expandafter\expandafter\expandafter\endgroup
\expandafter\ifx\csname RequirePackage\endcsname\relax
  \def\TMP@RequirePackage#1[#2]{%
    \begingroup\expandafter\expandafter\expandafter\endgroup
    \expandafter\ifx\csname ver@#1.sty\endcsname\relax
      \input #1.sty\relax
    \fi
  }%
  \TMP@RequirePackage{infwarerr}[2007/09/09]%
  \let\PackageError\@PackageError
\else
\fi
%    \end{macrocode}
%
%    \begin{macrocode}
\expandafter\ifx\csname toks@\endcsname\relax
  \toksdef\toks@=0 %
\fi
%    \end{macrocode}
%
% \subsection{Check for \eTeX}
%
%    \cs{unexpanded}, \cs{ifcsname}, and \cs{unless} are used if found.
%    \begin{macrocode}
\ifx\numexpr\@undefined
  \catcode`\$=9 % ignore
  \catcode`\&=14 % comment
\else % e-TeX
  \catcode`\$=14 % comment
  \catcode`\&=9  % ignore
\fi
%    \end{macrocode}
%
% \subsection{Generic help macros}
%
%    \begin{macro}{\KVS@Empty}
%    \begin{macrocode}
\def\KVS@Empty{}
%    \end{macrocode}
%    \end{macro}
%    \begin{macro}{\KVS@FirstOfTwo}
%    \begin{macrocode}
\long\def\KVS@FirstOfTwo#1#2{#1}
%    \end{macrocode}
%    \end{macro}
%    \begin{macro}{\KVS@SecondOfTwo}
%    \begin{macrocode}
\long\def\KVS@SecondOfTwo#1#2{#2}
%    \end{macrocode}
%    \end{macro}
%
%    \begin{macro}{\KVS@IfEmpty}
%    \begin{macrocode}
\long\def\KVS@IfEmpty#1{%
& \edef\KVS@Temp{\unexpanded{#1}}%
$ \begingroup
$   \toks@{#1}%
$   \edef\KVS@Temp{\the\toks@}%
$ \expandafter\endgroup
  \ifx\KVS@Temp\KVS@Empty
    \expandafter\KVS@FirstOfTwo
  \else
    \expandafter\KVS@SecondOfTwo
  \fi
}
%    \end{macrocode}
%    \end{macro}
%
% \subsection{Normalizing}
%
%    \begin{macro}{\kv@normalize}
%    \begin{macrocode}
\long\def\kv@normalize#1{%
  \begingroup
    \toks@{,#1,}%
    \KVS@Comma
    \KVS@SpaceComma
    \KVS@CommaSpace
    \KVS@CommaComma
    \KVS@Equals
    \KVS@SpaceEquals
    \KVS@EqualsSpace
    \xdef\KVS@Global{\the\toks@}%
  \endgroup
  \let\kv@list\KVS@Global
}
%    \end{macrocode}
%    \end{macro}
%    \begin{macro}{\comma@normalize}
%    \begin{macrocode}
\def\comma@normalize#1{%
  \begingroup
    \toks@{,#1,}%
    \KVS@Comma
    \KVS@SpaceComma
    \KVS@CommaSpace
    \KVS@CommaComma
    \xdef\KVS@Global{\the\toks@}%
  \endgroup
  \let\comma@list\KVS@Global
}
%    \end{macrocode}
%    \end{macro}
%    \begin{macro}{\KVS@Comma}
%    Converts active commas into comma with catcode other.
%    Also adds a comma at the end to protect the last value
%    for next cleanup steps.
%    \begin{macrocode}
\begingroup
  \lccode`\,=`\,%
  \lccode`\~=`\,%
\lowercase{\endgroup
  \def\KVS@Comma{%
    \toks@\expandafter{\expandafter}\expandafter
    \KVS@@Comma\the\toks@~\KVS@Nil
  }%
  \long\def\KVS@@Comma#1~#2\KVS@Nil{%
    \toks@\expandafter{\the\toks@#1}%
    \KVS@IfEmpty{#2}{%
    }{%
      \KVS@@Comma,#2\KVS@Nil
    }%
  }%
}
%    \end{macrocode}
%    \end{macro}
%    \begin{macro}{\KVS@SpaceComma}
%    Removes spaces before the comma, may add commas at the end.
%    \begin{macrocode}
\def\KVS@SpaceComma#1{%
  \def\KVS@SpaceComma{%
    \expandafter\KVS@@SpaceComma\the\toks@#1,\KVS@Nil
  }%
}
\KVS@SpaceComma{ }
%    \end{macrocode}
%    \end{macro}
%    \begin{macro}{\KVS@@SpaceComma}
%    \begin{macrocode}
\long\def\KVS@@SpaceComma#1 ,#2\KVS@Nil{%
  \KVS@IfEmpty{#2}{%
    \toks@{#1}%
  }{%
    \KVS@@SpaceComma#1,#2\KVS@Nil
  }%
}
%    \end{macrocode}
%    \end{macro}
%    \begin{macro}{\KVS@CommaSpace}
%    Removes spaces after the comma, may add commas at the end.
%    \begin{macrocode}
\def\KVS@CommaSpace{%
  \expandafter\KVS@@CommaSpace\the\toks@, \KVS@Nil
}
%    \end{macrocode}
%    \end{macro}
%    \begin{macro}{\KVS@@CommaSpace}
%    \begin{macrocode}
\long\def\KVS@@CommaSpace#1, #2\KVS@Nil{%
  \KVS@IfEmpty{#2}{%
    \toks@{#1}%
  }{%
    \KVS@@CommaSpace#1,#2\KVS@Nil
  }%
}
%    \end{macrocode}
%    \end{macro}
%    \begin{macro}{\KVS@CommaComma}
%    Replaces multiple commas by one comma.
%    \begin{macrocode}
\def\KVS@CommaComma{%
  \expandafter\KVS@@CommaComma\the\toks@,\KVS@Nil
}
%    \end{macrocode}
%    \end{macro}
%    \begin{macro}{\KVS@@CommaComma}
%    \begin{macrocode}
\long\def\KVS@@CommaComma#1,,#2\KVS@Nil{%
  \KVS@IfEmpty{#2}{%
    \toks@{#1,}% (!)
  }{%
    \KVS@@CommaComma#1,#2\KVS@Nil
  }%
}
%    \end{macrocode}
%    \end{macro}
%
%    \begin{macro}{\KVS@Equals}
%    Converts active equals signs into catcode other characters.
%    \begin{macrocode}
\begingroup
  \lccode`\==`\=%
  \lccode`\~=`\=%
\lowercase{\endgroup
  \def\KVS@Equals{%
    \toks@\expandafter{\expandafter}\expandafter
    \KVS@@Equals\the\toks@~\KVS@Nil
  }%
  \long\def\KVS@@Equals#1~#2\KVS@Nil{%
    \edef\KVS@Temp{\the\toks@}%
    \ifx\KVS@Temp\KVS@Empty
      \expandafter\KVS@FirstOfTwo
    \else
      \expandafter\KVS@SecondOfTwo
    \fi
    {%
      \toks@{#1}%
    }{%
      \toks@\expandafter{\the\toks@=#1}%
    }%
    \KVS@IfEmpty{#2}{%
    }{%
      \KVS@@Equals#2\KVS@Nil
    }%
  }%
}
%    \end{macrocode}
%    \end{macro}
%    \begin{macro}{\KVS@SpaceEquals}
%    Removes spaces before the equals sign.
%    \begin{macrocode}
\def\KVS@SpaceEquals#1{%
  \def\KVS@SpaceEquals{%
    \expandafter\KVS@@SpaceEquals\the\toks@#1=\KVS@Nil
  }%
}
\KVS@SpaceEquals{ }
%    \end{macrocode}
%    \end{macro}
%    \begin{macro}{\KVS@@SpaceEquals}
%    \begin{macrocode}
\long\def\KVS@@SpaceEquals#1 =#2\KVS@Nil{%
  \KVS@IfEmpty{#2}{%
    \toks@{#1}%
  }{%
    \KVS@@SpaceEquals#1=#2\KVS@Nil
  }%
}
%    \end{macrocode}
%    \end{macro}
%    \begin{macro}{\KVS@EqualsSpace}
%    Removes spaces after the equals sign.
%    \begin{macrocode}
\def\KVS@EqualsSpace{%
  \expandafter\KVS@@EqualsSpace\the\toks@= \KVS@Nil
}
%    \end{macrocode}
%    \end{macro}
%    \begin{macro}{\KVS@@EqualsSpace}
%    \begin{macrocode}
\long\def\KVS@@EqualsSpace#1= #2\KVS@Nil{%
  \KVS@IfEmpty{#2}{%
    \toks@{#1}%
  }{%
    \KVS@@EqualsSpace#1=#2\KVS@Nil
  }%
}
%    \end{macrocode}
%    \end{macro}
%
% \subsection{Parsing key value lists}
%
%    \begin{macro}{\kv@parse}
%    Normalizes and parses the key value list. Also sets \cs{kv@list}.
%    \begin{macrocode}
\long\def\kv@parse#1{%
  \kv@normalize{#1}%
  \expandafter\kv@parse@normalized\expandafter{\kv@list}%
}
%    \end{macrocode}
%    \end{macro}
%    \begin{macro}{\kv@parse@normalized}
%    |#1|: key value list\\
%    |#2|: processor
%    \begin{macrocode}
\long\def\kv@parse@normalized#1#2{%
  \KVS@Parse#1,\KVS@Nil{#2}%
}
%    \end{macrocode}
%    \end{macro}
%    \begin{macro}{\KVS@Parse}
%    |#1,#2|: key value list\\
%    |#3|: processor
%    \begin{macrocode}
\long\def\KVS@Parse#1,#2\KVS@Nil#3{%
  \KVS@IfEmpty{#1}{%
  }{%
    \KVS@Process#1=\KVS@Nil{#3}%
  }%
  \KVS@MaybeBreak
  \KVS@IfEmpty{#2}{%
  }{%
    \KVS@Parse#2\KVS@Nil{#3}%
  }%
}
%    \end{macrocode}
%    \end{macro}
%    \begin{macro}{\KVS@Process}
%    |#1|: key\\
%    |#2|: value, |=|\\
%    |#3|: processor
%    \begin{macrocode}
\long\def\KVS@Process#1=#2\KVS@Nil#3{%
  \let\KVS@MaybeBreak\relax
  \def\kv@key{#1}%
  \KVS@IfEmpty{#2}{%
    \let\kv@value\relax
    #3{#1}{}%
  }{%
    \KVS@@Process{#1}#2\KVS@Nil{#3}%
  }%
}
%    \end{macrocode}
%    \end{macro}
%    \begin{macro}{\KVS@@Process}
%    |#1|: key\\
%    |#2|: value\\
%    |#3|: processor
%    \begin{macrocode}
\long\def\KVS@@Process#1#2=\KVS@Nil#3{%
& \edef\kv@value{\unexpanded{#2}}%
$ \begingroup
$   \toks@{#2}%
$   \xdef\KVS@Global{\the\toks@}%
$ \endgroup
$ \let\kv@value\KVS@Global
  #3{#1}{#2}%
}
%    \end{macrocode}
%    \end{macro}
%    \begin{macro}{\KVS@MaybeBreak}
%    \begin{macrocode}
\let\KVS@MaybeBreak\relax
%    \end{macrocode}
%    \end{macro}
%    \begin{macro}{\KVS@break}
%    \begin{macrocode}
\def\KVS@break#1#2#3#4{%
  \let\KVS@MaybeBreak\relax
}
%    \end{macrocode}
%    \end{macro}
%    \begin{macro}{\kv@break}
%    \begin{macrocode}
\def\kv@break{%
  \let\KVS@MaybeBreak\KVS@break
}
%    \end{macrocode}
%    \end{macro}
%
% \subsection{Parsing comma lists}
%
%    \begin{macro}{\comma@parse}
%    Normalizes and parses the key value list. Also sets \cs{comma@list}.
%    \begin{macrocode}
\def\comma@parse#1{%
  \comma@normalize{#1}%
  \expandafter\comma@parse@normalized\expandafter{\comma@list}%
}
%    \end{macrocode}
%    \end{macro}
%
%    \begin{macro}{\comma@parse@normalized}
%    |#1|: comma list\\
%    |#2|: processor
%    \begin{macrocode}
\def\comma@parse@normalized#1#2{%
  \KVS@CommaParse#1,\KVS@Nil{#2}%
}
%    \end{macrocode}
%    \end{macro}
%    \begin{macro}{\KVS@CommaParse}
%    |#1,#2|: comma list\\
%    |#3|: processor
%    \begin{macrocode}
\def\KVS@CommaParse#1,#2\KVS@Nil#3{%
  \KVS@IfEmpty{#1}{%
  }{%
    \def\comma@entry{#1}%
    #3{#1}%
  }%
  \KVS@MaybeBreak
  \KVS@IfEmpty{#2}{%
  }{%
    \KVS@CommaParse#2\KVS@Nil{#3}%
  }%
}
%    \end{macrocode}
%    \end{macro}
%    \begin{macro}{\comma@break}
%    \begin{macrocode}
\def\comma@break{%
  \let\KVS@MaybeBreak\KVS@break
}
%    \end{macrocode}
%    \end{macro}
%
% \subsection{Processing key value pairs}
%
%    \begin{macro}{\kv@handled@false}
%    The handler can call \cs{kv@handled@false} or
%    \cs{kv@handled@true} so report failure or success.
%    The default is success (compatibility for versions
%    before 2011/10/18 v1.15).
%    \begin{macrocode}
\def\kv@handled@false{%
  \let\ifkv@handled@\iffalse
}
%    \end{macrocode}
%    \end{macro}
%    \begin{macro}{\kv@handled@true}
%    \begin{macrocode}
\def\kv@handled@true{%
  \let\ifkv@handled@\iftrue
}
%    \end{macrocode}
%    \end{macro}
%    \begin{macro}{\ifkv@handled@}
%    \begin{macrocode}
\kv@handled@true
%    \end{macrocode}
%    \end{macro}
%
%    \begin{macro}{\kv@processor@default}
%    \begin{macrocode}
\def\kv@processor@default#1#2{%
  \begingroup
    \csname @safe@activestrue\endcsname
    \let\ifincsname\iftrue
    \edef\KVS@temp{\endgroup
      \noexpand\KVS@ProcessorDefault{#1}{#2}%
    }%
  \KVS@temp
}
%    \end{macrocode}
%    \end{macro}
%    \begin{macro}{\KVS@ProcessorDefault}
%    \begin{macrocode}
\long\def\KVS@ProcessorDefault#1#2#3{%
  \def\kv@fam{#1}%
& \unless\ifcsname KV@#1@#2\endcsname
$ \begingroup\expandafter\expandafter\expandafter\endgroup
$ \expandafter\ifx\csname KV@#1@#2\endcsname\relax
&   \unless\ifcsname KVS@#1@handler\endcsname
$   \begingroup\expandafter\expandafter\expandafter\endgroup
$   \expandafter\ifx\csname KVS@#1@handler\endcsname\relax
      \kv@error@unknownkey{#1}{#2}%
    \else
      \kv@handled@true
      \csname KVS@#1@handler\endcsname{#2}{#3}%
      \relax
      \ifkv@handled@
      \else
        \kv@error@unknownkey{#1}{#2}%
      \fi
    \fi
  \else
    \ifx\kv@value\relax
&     \unless\ifcsname KV@#1@#2@default\endcsname
$     \begingroup\expandafter\expandafter\expandafter\endgroup
$     \expandafter\ifx\csname KV@#1@#2@default\endcsname\relax
        \kv@error@novalue{#1}{#2}%
      \else
        \csname KV@#1@#2@default\endcsname
        \relax
      \fi
    \else
      \csname KV@#1@#2\endcsname{#3}%
    \fi
  \fi
}
%    \end{macrocode}
%    \end{macro}
%
%    \begin{macro}{\kv@processor@known}
%    \begin{macrocode}
\def\kv@processor@known#1#2#3{%
  \begingroup
    \csname @safe@activestrue\endcsname
    \let\ifincsname\iftrue
    \edef\KVS@temp{\endgroup
      \noexpand\KVS@ProcessorKnown{#1}\noexpand#2{#3}%
    }%
  \KVS@temp
}
%    \end{macrocode}
%    \end{macro}
%    \begin{macro}{\KVS@ProcessorKnown}
%    \begin{macrocode}
\long\def\KVS@ProcessorKnown#1#2#3#4{%
  \def\kv@fam{#1}%
& \unless\ifcsname KV@#1@#3\endcsname
$ \begingroup\expandafter\expandafter\expandafter\endgroup
$ \expandafter\ifx\csname KV@#1@#3\endcsname\relax
&   \unless\ifcsname KVS@#1@handler\endcsname
$   \begingroup\expandafter\expandafter\expandafter\endgroup
$   \expandafter\ifx\csname KVS@#1@handler\endcsname\relax
      \KVS@AddUnhandled#2{#3}{#4}%
    \else
      \kv@handled@true
      \csname KVS@#1@handler\endcsname{#3}{#4}%
      \relax
      \ifkv@handled@
      \else
        \KVS@AddUnhandled#2{#3}{#4}%
      \fi
    \fi
  \else
    \ifx\kv@value\relax
&     \unless\ifcsname KV@#1@#2@default\endcsname
$     \begingroup\expandafter\expandafter\expandafter\endgroup
$     \expandafter\ifx\csname KV@#1@#3@default\endcsname\relax
        \kv@error@novalue{#1}{#3}%
      \else
        \csname KV@#1@#3@default\endcsname
        \relax
      \fi
    \else
      \csname KV@#1@#3\endcsname{#4}%
    \fi
  \fi
}
%    \end{macrocode}
%    \end{macro}
%    \begin{macro}{\KVS@AddUnhandled}
%    \begin{macrocode}
\long\def\KVS@AddUnhandled#1#2#3{%
& \edef#1{%
&   \ifx#1\KVS@empty
&   \else
&     \unexpanded{#1},%
&   \fi
&   \unexpanded{{#2}={#3}}%
& }%
$ \begingroup
$   \ifx#1\KVS@empty
$     \toks@{{#2}={#3}}%
$   \else
$     \toks@\expandafter{#1,{#2}={#3}}%
$   \fi
$   \xdef\KVS@Global{\the\toks@}%
$ \endgroup
$ \let#1\KVS@Global
}
%    \end{macrocode}
%    \end{macro}
%
%    \begin{macro}{\kv@set@family@handler}
%    \begin{macrocode}
\long\def\kv@set@family@handler#1#2{%
  \begingroup
    \csname @safe@activestrue\endcsname
    \let\ifincsname\iftrue
  \expandafter\endgroup
  \expandafter\def\csname KVS@#1@handler\endcsname##1##2{#2}%
}
%    \end{macrocode}
%    \end{macro}
%
%    \begin{macro}{\kv@unset@family@handler}
%    \begin{macrocode}
\long\def\kv@unset@family@handler#1#2{%
  \begingroup
    \csname @safe@activestrue\endcsname
    \let\ifincsname\iftrue
  \expandafter\endgroup
  \expandafter\let\csname KVS@#1@handler\endcsname\@UnDeFiNeD
}
%    \end{macrocode}
%    \end{macro}
%
% \subsection{Error handling}
%
%    \begin{macro}{\kv@error@novalue}
%    \begin{macrocode}
\def\kv@error@novalue{%
  \kv@error@generic{No value specified for}%
}
%    \end{macrocode}
%    \end{macro}
%    \begin{macro}{\kv@error@unknownkey}
%    \begin{macrocode}
\def\kv@error@unknownkey{%
  \kv@error@generic{Undefined}%
}
%    \end{macrocode}
%    \end{macro}
%    \begin{macro}{\kv@error@generic}
%    \begin{macrocode}
\def\kv@error@generic#1#2#3{%
  \PackageError{kvsetkeys}{%
    #1 key `#3'%
  }{%
    The keyval family of the key `#3' is `#2'.\MessageBreak
    The setting of the key is ignored because of the error.\MessageBreak
    \MessageBreak
    \@ehc
  }%
}
%    \end{macrocode}
%    \end{macro}
%
% \subsection{Do it all}
%
%    \begin{macro}{\kvsetkeys}
%    \begin{macrocode}
\long\def\kvsetkeys#1#2{%
  \kv@parse{#2}{\kv@processor@default{#1}}%
}
%    \end{macrocode}
%    \end{macro}
%
%    \begin{macro}{\kvsetkeys@expandafter}
%    \begin{macrocode}
\def\kvsetkeys@expandafter#1#2{%
  \expandafter\kv@parse\expandafter{#2}{%
    \kv@processor@default{#1}%
  }%
}
%    \end{macrocode}
%    \end{macro}
%
%    \begin{macro}{\KVS@cmd}
%    \begin{macrocode}
\def\KVS@cmd{0}%
%    \end{macrocode}
%    \end{macro}
%    \begin{macro}{\KVS@cmd@inc}
%    \begin{macrocode}
\def\KVS@cmd@inc{%
& \edef\KVS@cmd{\the\numexpr\KVS@cmd+1}%
$ \begingroup
$   \count255=\KVS@cmd\relax
$   \advance\count255 by 1\relax
$ \edef\x{\endgroup
$   \noexpand\def\noexpand\KVS@cmd{\number\count255}%
$ }%
$ \x
}
%    \end{macrocode}
%    \end{macro}
%    \begin{macro}{\KVS@cmd@dec}
%    \begin{macrocode}
\def\KVS@cmd@dec{%
& \edef\KVS@cmd{\the\numexpr\KVS@cmd-1}%
$ \begingroup
$   \count255=\KVS@cmd\relax
$   \advance\count255 by -1\relax
$ \edef\x{\endgroup
$   \noexpand\def\noexpand\KVS@cmd{\number\count255}%
$ }%
$ \x
}
%    \end{macrocode}
%    \end{macro}
%    \begin{macro}{\KVS@empty}
%    \begin{macrocode}
\def\KVS@empty{}
%    \end{macrocode}
%    \end{macro}
%
%    \begin{macro}{\kvsetknownkeys}
%    \begin{macrocode}
\def\kvsetknownkeys{%
  \expandafter
  \KVS@setknownkeys\csname KVS@cmd\KVS@cmd\endcsname{}%
}
%    \end{macrocode}
%    \end{macro}
%    \begin{macro}{\KVS@setknownkeys}
%    \begin{macrocode}
\long\def\KVS@setknownkeys#1#2#3#4#5{%
  \let#1\KVS@empty
  \KVS@cmd@inc
  #2\kv@parse#2{#5}{\kv@processor@known{#3}#1}%
  \KVS@cmd@dec
  \let#4=#1%
}
%    \end{macrocode}
%    \end{macro}
%
%    \begin{macro}{\kvsetknownkeys@expandafter}
%    \begin{macrocode}
\def\kvsetknownkeys@expandafter{%
  \expandafter
  \KVS@setknownkeys
      \csname KVS@cmd\KVS@cmd\endcsname\expandafter
}
%    \end{macrocode}
%    \end{macro}
%
%    \begin{macrocode}
\KVS@AtEnd%
%</package>
%    \end{macrocode}
%% \section{Installation}
%
% \subsection{Download}
%
% \paragraph{Package.} This package is available on
% CTAN\footnote{\CTANpkg{kvsetkeys}}:
% \begin{description}
% \item[\CTAN{macros/latex/contrib/kvsetkeys/kvsetkeys.dtx}] The source file.
% \item[\CTAN{macros/latex/contrib/kvsetkeys/kvsetkeys.pdf}] Documentation.
% \end{description}
%
%
% \paragraph{Bundle.} All the packages of the bundle `kvsetkeys'
% are also available in a TDS compliant ZIP archive. There
% the packages are already unpacked and the documentation files
% are generated. The files and directories obey the TDS standard.
% \begin{description}
% \item[\CTANinstall{install/macros/latex/contrib/kvsetkeys.tds.zip}]
% \end{description}
% \emph{TDS} refers to the standard ``A Directory Structure
% for \TeX\ Files'' (\CTANpkg{tds}). Directories
% with \xfile{texmf} in their name are usually organized this way.
%
% \subsection{Bundle installation}
%
% \paragraph{Unpacking.} Unpack the \xfile{kvsetkeys.tds.zip} in the
% TDS tree (also known as \xfile{texmf} tree) of your choice.
% Example (linux):
% \begin{quote}
%   |unzip kvsetkeys.tds.zip -d ~/texmf|
% \end{quote}
%
% \subsection{Package installation}
%
% \paragraph{Unpacking.} The \xfile{.dtx} file is a self-extracting
% \docstrip\ archive. The files are extracted by running the
% \xfile{.dtx} through \plainTeX:
% \begin{quote}
%   \verb|tex kvsetkeys.dtx|
% \end{quote}
%
% \paragraph{TDS.} Now the different files must be moved into
% the different directories in your installation TDS tree
% (also known as \xfile{texmf} tree):
% \begin{quote}
% \def\t{^^A
% \begin{tabular}{@{}>{\ttfamily}l@{ $\rightarrow$ }>{\ttfamily}l@{}}
%   kvsetkeys.sty & tex/generic/kvsetkeys/kvsetkeys.sty\\
%   kvsetkeys.pdf & doc/latex/kvsetkeys/kvsetkeys.pdf\\
%   kvsetkeys-example.tex & doc/latex/kvsetkeys/kvsetkeys-example.tex\\
%   kvsetkeys.dtx & source/latex/kvsetkeys/kvsetkeys.dtx\\
% \end{tabular}^^A
% }^^A
% \sbox0{\t}^^A
% \ifdim\wd0>\linewidth
%   \begingroup
%     \advance\linewidth by\leftmargin
%     \advance\linewidth by\rightmargin
%   \edef\x{\endgroup
%     \def\noexpand\lw{\the\linewidth}^^A
%   }\x
%   \def\lwbox{^^A
%     \leavevmode
%     \hbox to \linewidth{^^A
%       \kern-\leftmargin\relax
%       \hss
%       \usebox0
%       \hss
%       \kern-\rightmargin\relax
%     }^^A
%   }^^A
%   \ifdim\wd0>\lw
%     \sbox0{\small\t}^^A
%     \ifdim\wd0>\linewidth
%       \ifdim\wd0>\lw
%         \sbox0{\footnotesize\t}^^A
%         \ifdim\wd0>\linewidth
%           \ifdim\wd0>\lw
%             \sbox0{\scriptsize\t}^^A
%             \ifdim\wd0>\linewidth
%               \ifdim\wd0>\lw
%                 \sbox0{\tiny\t}^^A
%                 \ifdim\wd0>\linewidth
%                   \lwbox
%                 \else
%                   \usebox0
%                 \fi
%               \else
%                 \lwbox
%               \fi
%             \else
%               \usebox0
%             \fi
%           \else
%             \lwbox
%           \fi
%         \else
%           \usebox0
%         \fi
%       \else
%         \lwbox
%       \fi
%     \else
%       \usebox0
%     \fi
%   \else
%     \lwbox
%   \fi
% \else
%   \usebox0
% \fi
% \end{quote}
% If you have a \xfile{docstrip.cfg} that configures and enables \docstrip's
% TDS installing feature, then some files can already be in the right
% place, see the documentation of \docstrip.
%
% \subsection{Refresh file name databases}
%
% If your \TeX~distribution
% (\TeX\,Live, \mikTeX, \dots) relies on file name databases, you must refresh
% these. For example, \TeX\,Live\ users run \verb|texhash| or
% \verb|mktexlsr|.
%
% \subsection{Some details for the interested}
%
% \paragraph{Unpacking with \LaTeX.}
% The \xfile{.dtx} chooses its action depending on the format:
% \begin{description}
% \item[\plainTeX:] Run \docstrip\ and extract the files.
% \item[\LaTeX:] Generate the documentation.
% \end{description}
% If you insist on using \LaTeX\ for \docstrip\ (really,
% \docstrip\ does not need \LaTeX), then inform the autodetect routine
% about your intention:
% \begin{quote}
%   \verb|latex \let\install=y\input{kvsetkeys.dtx}|
% \end{quote}
% Do not forget to quote the argument according to the demands
% of your shell.
%
% \paragraph{Generating the documentation.}
% You can use both the \xfile{.dtx} or the \xfile{.drv} to generate
% the documentation. The process can be configured by the
% configuration file \xfile{ltxdoc.cfg}. For instance, put this
% line into this file, if you want to have A4 as paper format:
% \begin{quote}
%   \verb|\PassOptionsToClass{a4paper}{article}|
% \end{quote}
% An example follows how to generate the
% documentation with pdf\LaTeX:
% \begin{quote}
%\begin{verbatim}
%pdflatex kvsetkeys.dtx
%makeindex -s gind.ist kvsetkeys.idx
%pdflatex kvsetkeys.dtx
%makeindex -s gind.ist kvsetkeys.idx
%pdflatex kvsetkeys.dtx
%\end{verbatim}
% \end{quote}
%
% \begin{thebibliography}{9}
% \bibitem{tb94wright}
%   A guide to key-value methods, Joseph Wright, second draft for
%   \href{https://www.tug.org/tugboat}{TUGBoat}, 2009-03-17.
%   \url{https://www.texdev.net/uploads/2009/03/keyval.pdf}
%
% \bibitem{keyval}
%   David Carlisle:
%   \textit{The \xpackage{keyval} package};
%   1999/03/16 v1.13;
%   \CTANpkg{keyval}.
%
% \end{thebibliography}
%
% \begin{History}
%   \begin{Version}{2006/03/06 v1.0}
%   \item
%     First version.
%   \end{Version}
%   \begin{Version}{2006/10/19 v1.1}
%   \item
%     Fix of \cs{kv@set@family@handler}.
%   \item
%     Example added.
%   \end{Version}
%   \begin{Version}{2007/09/09 v1.2}
%   \item
%     Using package \xpackage{infwarerr} for error messages.
%   \item
%     Catcode section rewritten.
%   \end{Version}
%   \begin{Version}{2007/09/29 v1.3}
%   \item
%     Normalizing and parsing of comma separated lists added.
%   \item
%     \cs{kv@normalize} rewritten.
%   \item
%     Robustness increased for normalizing and parsing,
%     e.g. for values with unmatched conditionals.
%   \item
%     \eTeX\ is used if available.
%   \item
%     Tests added for normalizing and parsing.
%   \end{Version}
%   \begin{Version}{2009/07/19 v1.4}
%   \item
%     Bug fix for \cs{kv@normalize}: unwanted space removed (Florent Chervet).
%   \end{Version}
%   \begin{Version}{2009/07/30 v1.5}
%   \item
%     Documentation addition: recommendation for Joseph Wright's
%     review article.
%   \end{Version}
%   \begin{Version}{2009/12/12 v1.6}
%   \item
%     Short info shortened.
%   \end{Version}
%   \begin{Version}{2009/12/22 v1.7}
%   \item
%     Internal optimization (\cs{KVS@CommaSpace}, \dots, \cs{KVS@EqualsSpace}).
%   \end{Version}
%   \begin{Version}{2010/01/28 v1.8}
%   \item
%     Compatibility to ini\TeX\ added.
%   \end{Version}
%   \begin{Version}{2010/03/01 v1.9}
%   \item
%     Support of \cs{par} inside values.
%   \end{Version}
%   \begin{Version}{2011/01/30 v1.10}
%   \item
%     Already loaded package files are not input in \hologo{plainTeX}.
%   \end{Version}
%   \begin{Version}{2011/03/03 v1.11}
%   \item
%     \cs{kv@break} and \cs{comma@break} added.
%   \end{Version}
%   \begin{Version}{2011/04/05 v1.12}
%   \item
%     Error message with recovery action in help message
%     (request by GL).
%   \end{Version}
%   \begin{Version}{2011/04/07 v1.13}
%   \item
%     \cs{kv@processor@default} supports package \xpackage{babel}'s
%     shorthands.
%   \item
%     \cs{kv@set@family@handler} with shorthand support.
%   \end{Version}
%   \begin{Version}{2011/06/15 v1.14}
%   \item
%     Some optimizations in token register uses (GL, HO).
%   \end{Version}
%   \begin{Version}{2011/10/18 v1.15}
%   \item
%     \cs{kv@processor@known} and \cs{kvsetknownkeys} added.
%   \item
%     \cs{kvsetkeys@expandafter} and \cs{kvsetknownkeys@expandafter} added.
%   \item
%     Family handler can report success or failure by \cs{kv@handled@true}
%     or \cs{kv@handled@false}.
%   \item
%     \cs{kv@unset@family@handler} added.
%   \end{Version}
%   \begin{Version}{2012/04/25 v1.16}
%   \item
%     \cs{kv@processor@default} and \cs{kv@processor@known} define
%     macro \cs{kv@fam} for convenience.
%   \item
%     Catcode section: Catcode setting for \texttt{+} added for \hologo{eTeX}.
%   \end{Version}
%   \begin{Version}{2016/05/16 v1.17}
%   \item
%     Documentation updates.
%   \end{Version}
%   \begin{Version}{2019/12/15 v1.18}
%   \item
%     Documentation updates.
%   \item
%     Avoid \xpackage{etexcmds} and \xpackage{infwarwerr} in \LaTeX.
%   \end{Version}
% \end{History}
%
% \PrintIndex
%
% \Finale
\endinput
|
% \end{quote}
% Do not forget to quote the argument according to the demands
% of your shell.
%
% \paragraph{Generating the documentation.}
% You can use both the \xfile{.dtx} or the \xfile{.drv} to generate
% the documentation. The process can be configured by the
% configuration file \xfile{ltxdoc.cfg}. For instance, put this
% line into this file, if you want to have A4 as paper format:
% \begin{quote}
%   \verb|\PassOptionsToClass{a4paper}{article}|
% \end{quote}
% An example follows how to generate the
% documentation with pdf\LaTeX:
% \begin{quote}
%\begin{verbatim}
%pdflatex kvsetkeys.dtx
%makeindex -s gind.ist kvsetkeys.idx
%pdflatex kvsetkeys.dtx
%makeindex -s gind.ist kvsetkeys.idx
%pdflatex kvsetkeys.dtx
%\end{verbatim}
% \end{quote}
%
% \begin{thebibliography}{9}
% \bibitem{tb94wright}
%   A guide to key-value methods, Joseph Wright, second draft for
%   \href{https://www.tug.org/tugboat}{TUGBoat}, 2009-03-17.
%   \url{https://www.texdev.net/uploads/2009/03/keyval.pdf}
%
% \bibitem{keyval}
%   David Carlisle:
%   \textit{The \xpackage{keyval} package};
%   1999/03/16 v1.13;
%   \CTANpkg{keyval}.
%
% \end{thebibliography}
%
% \begin{History}
%   \begin{Version}{2006/03/06 v1.0}
%   \item
%     First version.
%   \end{Version}
%   \begin{Version}{2006/10/19 v1.1}
%   \item
%     Fix of \cs{kv@set@family@handler}.
%   \item
%     Example added.
%   \end{Version}
%   \begin{Version}{2007/09/09 v1.2}
%   \item
%     Using package \xpackage{infwarerr} for error messages.
%   \item
%     Catcode section rewritten.
%   \end{Version}
%   \begin{Version}{2007/09/29 v1.3}
%   \item
%     Normalizing and parsing of comma separated lists added.
%   \item
%     \cs{kv@normalize} rewritten.
%   \item
%     Robustness increased for normalizing and parsing,
%     e.g. for values with unmatched conditionals.
%   \item
%     \eTeX\ is used if available.
%   \item
%     Tests added for normalizing and parsing.
%   \end{Version}
%   \begin{Version}{2009/07/19 v1.4}
%   \item
%     Bug fix for \cs{kv@normalize}: unwanted space removed (Florent Chervet).
%   \end{Version}
%   \begin{Version}{2009/07/30 v1.5}
%   \item
%     Documentation addition: recommendation for Joseph Wright's
%     review article.
%   \end{Version}
%   \begin{Version}{2009/12/12 v1.6}
%   \item
%     Short info shortened.
%   \end{Version}
%   \begin{Version}{2009/12/22 v1.7}
%   \item
%     Internal optimization (\cs{KVS@CommaSpace}, \dots, \cs{KVS@EqualsSpace}).
%   \end{Version}
%   \begin{Version}{2010/01/28 v1.8}
%   \item
%     Compatibility to ini\TeX\ added.
%   \end{Version}
%   \begin{Version}{2010/03/01 v1.9}
%   \item
%     Support of \cs{par} inside values.
%   \end{Version}
%   \begin{Version}{2011/01/30 v1.10}
%   \item
%     Already loaded package files are not input in \hologo{plainTeX}.
%   \end{Version}
%   \begin{Version}{2011/03/03 v1.11}
%   \item
%     \cs{kv@break} and \cs{comma@break} added.
%   \end{Version}
%   \begin{Version}{2011/04/05 v1.12}
%   \item
%     Error message with recovery action in help message
%     (request by GL).
%   \end{Version}
%   \begin{Version}{2011/04/07 v1.13}
%   \item
%     \cs{kv@processor@default} supports package \xpackage{babel}'s
%     shorthands.
%   \item
%     \cs{kv@set@family@handler} with shorthand support.
%   \end{Version}
%   \begin{Version}{2011/06/15 v1.14}
%   \item
%     Some optimizations in token register uses (GL, HO).
%   \end{Version}
%   \begin{Version}{2011/10/18 v1.15}
%   \item
%     \cs{kv@processor@known} and \cs{kvsetknownkeys} added.
%   \item
%     \cs{kvsetkeys@expandafter} and \cs{kvsetknownkeys@expandafter} added.
%   \item
%     Family handler can report success or failure by \cs{kv@handled@true}
%     or \cs{kv@handled@false}.
%   \item
%     \cs{kv@unset@family@handler} added.
%   \end{Version}
%   \begin{Version}{2012/04/25 v1.16}
%   \item
%     \cs{kv@processor@default} and \cs{kv@processor@known} define
%     macro \cs{kv@fam} for convenience.
%   \item
%     Catcode section: Catcode setting for \texttt{+} added for \hologo{eTeX}.
%   \end{Version}
%   \begin{Version}{2016/05/16 v1.17}
%   \item
%     Documentation updates.
%   \end{Version}
%   \begin{Version}{2019/12/15 v1.18}
%   \item
%     Documentation updates.
%   \item
%     Avoid \xpackage{etexcmds} and \xpackage{infwarwerr} in \LaTeX.
%   \end{Version}
% \end{History}
%
% \PrintIndex
%
% \Finale
\endinput

%        (quote the arguments according to the demands of your shell)
%
% Documentation:
%    (a) If kvsetkeys.drv is present:
%           latex kvsetkeys.drv
%    (b) Without kvsetkeys.drv:
%           latex kvsetkeys.dtx; ...
%    The class ltxdoc loads the configuration file ltxdoc.cfg
%    if available. Here you can specify further options, e.g.
%    use A4 as paper format:
%       \PassOptionsToClass{a4paper}{article}
%
%    Programm calls to get the documentation (example):
%       pdflatex kvsetkeys.dtx
%       makeindex -s gind.ist kvsetkeys.idx
%       pdflatex kvsetkeys.dtx
%       makeindex -s gind.ist kvsetkeys.idx
%       pdflatex kvsetkeys.dtx
%
% Installation:
%    TDS:tex/generic/kvsetkeys/kvsetkeys.sty
%    TDS:doc/latex/kvsetkeys/kvsetkeys.pdf
%    TDS:doc/latex/kvsetkeys/kvsetkeys-example.tex
%    TDS:source/latex/kvsetkeys/kvsetkeys.dtx
%
%<*ignore>
\begingroup
  \catcode123=1 %
  \catcode125=2 %
  \def\x{LaTeX2e}%
\expandafter\endgroup
\ifcase 0\ifx\install y1\fi\expandafter
         \ifx\csname processbatchFile\endcsname\relax\else1\fi
         \ifx\fmtname\x\else 1\fi\relax
\else\csname fi\endcsname
%</ignore>
%<*install>
\input docstrip.tex
\Msg{************************************************************************}
\Msg{* Installation}
\Msg{* Package: kvsetkeys 2019/12/15 v1.18 Key value parser (HO)}
\Msg{************************************************************************}

\keepsilent
\askforoverwritefalse

\let\MetaPrefix\relax
\preamble

This is a generated file.

Project: kvsetkeys
Version: 2019/12/15 v1.18

Copyright (C)
   2006, 2007, 2009-2012 Heiko Oberdiek
   2016-2019 Oberdiek Package Support Group

This work may be distributed and/or modified under the
conditions of the LaTeX Project Public License, either
version 1.3c of this license or (at your option) any later
version. This version of this license is in
   https://www.latex-project.org/lppl/lppl-1-3c.txt
and the latest version of this license is in
   https://www.latex-project.org/lppl.txt
and version 1.3 or later is part of all distributions of
LaTeX version 2005/12/01 or later.

This work has the LPPL maintenance status "maintained".

The Current Maintainers of this work are
Heiko Oberdiek and the Oberdiek Package Support Group
https://github.com/ho-tex/kvsetkeys/issues


The Base Interpreter refers to any `TeX-Format',
because some files are installed in TDS:tex/generic//.

This work consists of the main source file kvsetkeys.dtx
and the derived files
   kvsetkeys.sty, kvsetkeys.pdf, kvsetkeys.ins, kvsetkeys.drv,
   kvsetkeys-example.tex, kvsetkeys-test1.tex,
   kvsetkeys-test2.tex, kvsetkeys-test3.tex,
   kvsetkeys-test4.tex.

\endpreamble
\let\MetaPrefix\DoubleperCent

\generate{%
  \file{kvsetkeys.ins}{\from{kvsetkeys.dtx}{install}}%
  \file{kvsetkeys.drv}{\from{kvsetkeys.dtx}{driver}}%
  \usedir{tex/generic/kvsetkeys}%
  \file{kvsetkeys.sty}{\from{kvsetkeys.dtx}{package}}%
  \usedir{doc/latex/kvsetkeys}%
  \file{kvsetkeys-example.tex}{\from{kvsetkeys.dtx}{example}}%
}

\catcode32=13\relax% active space
\let =\space%
\Msg{************************************************************************}
\Msg{*}
\Msg{* To finish the installation you have to move the following}
\Msg{* file into a directory searched by TeX:}
\Msg{*}
\Msg{*     kvsetkeys.sty}
\Msg{*}
\Msg{* To produce the documentation run the file `kvsetkeys.drv'}
\Msg{* through LaTeX.}
\Msg{*}
\Msg{* Happy TeXing!}
\Msg{*}
\Msg{************************************************************************}

\endbatchfile
%</install>
%<*ignore>
\fi
%</ignore>
%<*driver>
\NeedsTeXFormat{LaTeX2e}
\ProvidesFile{kvsetkeys.drv}%
  [2019/12/15 v1.18 Key value parser (HO)]%
\documentclass{ltxdoc}
\usepackage{holtxdoc}[2011/11/22]
\begin{document}
  \DocInput{kvsetkeys.dtx}%
\end{document}
%</driver>
% \fi
%
%
%
% \GetFileInfo{kvsetkeys.drv}
%
% \title{The \xpackage{kvsetkeys} package}
% \date{2019/12/15 v1.18}
% \author{Heiko Oberdiek\thanks
% {Please report any issues at \url{https://github.com/ho-tex/kvsetkeys/issues}}}
%
% \maketitle
%
% \begin{abstract}
% Package \xpackage{kvsetkeys} provides \cs{kvsetkeys}, a variant
% of package \xpackage{keyval}'s \cs{setkeys}. It allows to specify
% a handler that deals with unknown options. Active commas and equal
% signs may be used (e.g. see \xpackage{babel}'s shorthands) and
% only one level of curly braces is removed from the values.
% \end{abstract}
%
% \tableofcontents
%
% \def\M#1{\texttt{\{}\meta{#1}\texttt{\}}}
%
% \section{Documentation}
%
% First I want to recommend the very good review article
% ``A guide to key-value methods'' by Joseph Wright \cite{tb94wright}.
% It introduces the different key-value packages and compares them.
%
% \subsection{Motivation}
%
% \cs{kvsetkeys} serves as replacement for \xpackage{keyval}'s
% \cs{setkeys}. It basically uses the same syntax. But the
% implementation is more robust and predictable:
% \begin{description}
% \item[Active syntax characters:]
%   Comma `|,|' and the equals sign `|=|' are used inside
%   key value lists as syntax characters. Package \xpackage{keyval}
%   uses the catcode of the characters that is active during
%   package loading, usually this is catcode 12 (other).
%   But it can happen that the catcode setting of the syntax characters
%   changes. Especially active characters are of interest, because
%   some language adaptations uses them. For example, option \xoption{turkish}
%   of package \xpackage{babel} uses the equals sign as active shorthand
%   character. Therefore package \xpackage{kvsetkeys} deals with
%   both catcode settings 12 (other) and 13 (active).
% \item[Brace removal:]
%   Package \xpackage{keyval}'s \cs{setkeys} removes up to two
%   levels of curly braces around the value in some unpredictable way:
%\begin{quote}
%|\setkeys{fam}{key={{value}}} || || | $\rightarrow$ |value|\\
%|\setkeys{fam}{key={{{value}}}} | $\rightarrow$ |{value}|\\
%|\setkeys{fam}{key= {{{value}}}}| $\rightarrow$ |{{value}}|
%\end{quote}
%   This package \xpackage{kvsetkeys} follows a much stronger rule:
%   Exactly one level of braces are removed from an item, if the
%   item is surrounded by curly braces. An item can be a
%   the key value pair, the key or the value.
%\begin{quote}
%|\kvsetkeys{fam}{key={value}} || | $\rightarrow$ |value|\\
%|\kvsetkeys{fam}{key={{value}} | $\rightarrow$ |{value}|\\
%|\kvsetkeys{fam}{key= {{value}}| $\rightarrow$ |{value}|
%\end{quote}
% \item[Arbitrary values:] Unmatched conditionals are supported.
% \end{description}
%
% Before I describe
% \cs{kvsetkeys} in more detail, first I want to explain, how
% this package deals with key value lists. For the package also
% provides low level interfaces that can be used by package authors.
%
% \subsection{Normalizing key value lists}
%
% \begin{declcs}{kv@normalize}\,\M{key value list}
% \end{declcs}
% If the user specifies key value lists, he usually prefers
% nice formatted source code, e.g.:
% \begin{quote}
%\begin{verbatim}
%\hypersetup{
%  pdftitle    = {...},
%  pdfsubject  = {...},
%  pdfauthor   = {...},
%  pdfkeywords = {...},
%  ...
%}
%\end{verbatim}
% \end{quote}
% Thus there can be spaces around keys, around |=| or around the value.
% Also empty entries are possible by too many commas. Therefore these
% spaces and empty entries are silently removed by package \xpackage{keyval}
% and this package. Whereas the contents of the value can be protected
% by curly braces, especially if spaces or commas are used inside,
% a key name must not use spaces or other syntax characters.
%
% \cs{kv@normalize} takes a key value list and performs the cleanup:
% \begin{itemize}
% \item Spaces are removed.
% \item Syntax characters (comma and equal sign) that are active
%   are replaced by the same characters with standard catcode.
%   (Example: \xpackage{babel}'s language option \xoption{turkish}
%   uses the equal sign as active shorthand character.)
% \end{itemize}
% The result is stored in \cs{kv@list}, e.g.:
% \begin{quote}
% |\kv@list| $\rightarrow$ |,pdftitle={...},pdfsubject={...},...,|
% \end{quote}
% Curly braces around values (or keys) remain untouched.
% \begin{description}
% \item[v1.3+:]
% One comma is added in front of the list and each pair ends with
% a comma. Thus an empty list consists of one comma, otherwise
% two commas encloses the list. Empty entries other than the first
% are removed.
% \item[v1.0 -- v1.2:]
% Empty entries are removed later. In fact it adds a comma at the begin
% and end to protect the last value and an easier implementation.
% \end{description}
%
% \subsection{Parsing key value lists}
%
% \begin{declcs}{kv@parse}\,\M{key value list}\,\M{processor}
% \end{declcs}
% It is easier to parse a normalized list, thus \cs{kv@parse}
% normalizes the list and calls \cs{kv@parse@normalized}.
%
% \begin{declcs}{kv@parse@normalized}\,\M{key value list}%
%   \,\M{processor}
% \end{declcs}
% Now the key value list is split into single key value pairs.
% For further processing the key and value are given as arguments
% for the \meta{processor}:
% \begin{quote}
%   \meta{processor}\,\M{key}\,\M{value}
% \end{quote}
% Also key and value are stored in macro names:
% \begin{itemize}
% \item \cs{kv@key} stores the key.
% \item \cs{kv@value} stores the value or if the value was not
%   specified it has the meaning \cs{relax}.
% \end{itemize}
% The behaviour in pseudo code:
% \begin{quote}
%   foreach (\meta{key}, \meta{value}) in (\meta{key value list})\\
%   \hspace*{1.5em}\cs{kv@key} := \meta{key}\\
%   \hspace*{1.5em}\cs{kv@value} := \meta{value}\\
%   \hspace*{1.5em}\meta{processor}\,\M{key}\,\M{value}
% \end{quote}
%
% \begin{declcs}{kv@break}
% \end{declcs}
% Since version 2011/03/03 v1.11 \cs{kv@break} can be
% called inside the \meta{processor}
% of \cs{kv@parse} or \cs{kv@parse@normalized}, then
% the processing is stopped and the following entries discarded.
%
% \subsection{Processing key value pairs}
%
% Key value pairs can be processed in many different ways.
% For example, the processor for \cs{kvsetkeys} works similar
% to \cs{setkeys} of package \xpackage{keyval}. There unknown
% keys raise an error.
%
% Package \xpackage{xkeyval} also knows a star form of \cs{setkeys}
% that stores unknown keys in an internal macro for further processing
% with \cs{setrmkeys} and similar macros. This feature is covered
% by processor \cs{kv@processor@known}.
%
% \subsubsection{Processing similar to \xpackage{keyval}}
%
% \begin{declcs}{kv@processor@default}\,\M{family}\,\M{key}\,\M{value}
% \end{declcs}
% There are many possiblities to process key value pairs.
% \cs{kv@processor@default} is the processor used in \cs{kvsetkeys}.
% It reimplements and extends the behaviour of
% \xpackage{keyval}'s \cs{setkeys}.
% In case of unknown keys \cs{setkeys} raise an error.
% This processer, however, calls a handler instead, if it
% is provided by the family. Both \meta{family} and \meta{key}
% may contain package \xpackage{babel}'s shorthands
% (since 2011/04/07 v1.13).
%
% Since 2011/10/18 v1.15 the family handler can reject the
% successful handling of a key by calling \cs{kv@handled@false}.
%
% Since 2019/12/15 v1.18 \cs{kv@processor@default} also defines
% macro \cs{kv@fam} with meaning \meta{family} for convenience.
%
% \subsubsection{Processing similar to \cs{setkeys*} of package \xpackage{xkeyval}}
%
% \begin{declcs}{kv@processor@known}\,\M{family}\,^^A
%   \M{cmd}\,\M{key}\,\M{value}
% \end{declcs}
% The key value processor \cs{kv@processor@known} behaves similar
% to \cs{kv@processor@default}. If the \meta{key} exists in the
% \meta{family} its code is called, otherwise the family handler
% is tried. If the family handler is not set or cannot handle the
% key, the unknown key value pair is added to the macro \meta{cmd}.
% Since 2011/10/18 v1.15.
%
% The behaviour in pseudo code:
% \begin{quote}
%   if \meta{key} exists\\
%   \hspace*{1.5em}call the keyval code of \meta{key}\\
%   else\\
%   \hspace*{1.5em}if \meta{handler} for \meta{family} exists\\
%   \hspace*{3em}handled = true\\
%   \hspace*{3em}\meta{handler}\,\M{key}\,\M{value}\\
%   \hspace*{3em}if handled\\
%   \hspace*{3em}else\\
%   \hspace*{4.5em}add \texttt{"}\M{key}\texttt{=}\M{value}\texttt{"}
%     to \M{cmd}\\
%   \hspace*{3em}fi\\
%   \hspace*{1.5em}else\\
%   \hspace*{3em}add \texttt{"}\M{key}\texttt{=}\M{value}\texttt{"}
%     to \M{cmd}\\
%   \hspace*{3em}raise unknown key error\\
%   \hspace*{1.5em}fi\\
%   fi
% \end{quote}
%
% Since 2019/12/15 v1.18 \cs{kv@processor@known} also defines
% macro \cs{kv@fam} with meaning \meta{family} for convenience.
%
% \subsection{Default family handler}
%
% \cs{kv@processor@default} calls \meta{handler}, the default
% handler for the family, if the key does not exist in the family.
% The handler is called with two arguments, the key and the value.
% It can be defined with \cs{kv@set@family@hander}:
%
% \begin{declcs}{kv@set@family@handler}\,\M{family}\,\M{handler definition}
% \end{declcs}
% This sets the default family handler for the keyval family
% \meta{family}. Inside \meta{handler definition} |#1| stands for
% the key and |#2| is the value. Also \cs{kv@key} and \cs{kv@value}
% can be used for the key and the value. If the value is not given,
% \cs{kv@value} has the meaning \cs{relax}.
%
% \begin{declcs}{kv@unset@family@handler}\,\M{family}
% \end{declcs}
% It removes the family handler for \meta{family}.
% Since 2011/10/18 v1.15.
%
% \subsection{Put it all together}
%
% \begin{declcs}{kvsetkeys}\,\M{family}\,\M{key value list}
% \end{declcs}
% Macro \cs{kvsetkeys} processes the \meta{key value list} with
% the standard processor \cs{kv@processor@default}:
% \begin{quote}
%   \cs{kv@parse}\,\M{key value list}^^A
%   \texttt{\{}\cs{kv@processor@default}\,\M{family}\texttt{\}}
% \end{quote}
%
% \begin{declcs}{kvsetknownkeys}\,\M{family}\,\M{cmd}\,\M{key value list}
% \end{declcs}
% Macro \cs{kvsetknownkeys} processes the \meta{key value list}
% with processor \cs{kv@processor@known}. All key value pairs
% with keys that are not known in \meta{family} are stored
% in macro \meta{cmd}. A previous contents of macro \meta{cmd}
% will be overwritten. If all keys can be handled, \meta{cmd}
% will be empty, otherwise it contains a key value list of
% unhandled key value pairs.
% Since 2011/10/18 v1.15.
%
% Pseudo code:
% \begin{quote}
%   create macro \meta{cmdaux} with unique name (inside the current group)\\
%   \cs{def}\meta{cmdaux}\{\}\\
%   \cs{kv@parse}\,\M{key value list}^^A
%   \texttt{\{}\cs{kv@processor@known}\,\M{family}\,\M{cmdaux}\texttt{\}}\\
%   \cs{let}\meta{cmd}=\meta{cmdaux}
% \end{quote}
%
% \begin{declcs}{kvsetkeys@expandafter}\,\M{family}\,\M{list cmd}\\
%   \cs{kvsetknownkeys@expandafter}\,\M{family}\,%
%   \M{cmd}\,\M{list cmd}
% \end{declcs}
% Both macros behave like the counterparts without suffix
% |@expandafter|. The difference is that the key value list is
% given as macro that is expanded once.
% Since 2011/10/18 v1.15.
%
% Thus you can replace \cs{setkeys} of package \xpackage{keyval}
% by the key value parser of this package:
% \begin{quote}
%   |\renewcommand*{\setkeys}{\kvsetkeys}|\\
%   or\\
%   |\let\setkeys\kvsetkeys|
% \end{quote}
%
% \subsection{Comma separated lists}
%
% Since version 2007/09/29 v1.3 this package also supports the normalizing
% and parsing of general comma separated lists.
%
% \begin{declcs}{comma@normalize}\,\M{comma list}
% \end{declcs}
% Macro \cs{comma@normalize} normalizes the comma separated list,
% removes spaces around commas. The result is put in macro \cs{comma@list}.
%
% \begin{declcs}{comma@parse}\,\M{comma list}\,\M{processor}
% \end{declcs}
% Macro \cs{comma@parse} first normalizes the comma separated list
% and then parses the list by calling \cs{comma@parse@normalized}.
%
% \begin{declcs}{comma@parse@normalized}\,\M{normalized comma list}^^A
%   \,\M{processor}
% \end{declcs}
% The list is parsed. Empty entries are ignored. \meta{processor}
% is called for each non-empty entry with the entry as argument:
% \begin{quote}
%   \meta{processor}|{|\meta{entry}|}|
% \end{quote}
% Also the entry is stored in the macro \cs{comma@entry}.
%
% \begin{declcs}{comma@break}
% \end{declcs}
% Since version 2011/03/03 v1.11 \cs{comma@break} can be
% called inside the \meta{processor}
% of \cs{comma@parse} or \cs{comma@parse@normalized}, then
% the processing is stopped and the following entries discarded.
%
% \section{Example}
%
% The following example prints a short piece of HTML code using
% the tabbing environment for indenting purpose and a key value
% syntax for specifying the attributes of an HTML tag.
% The example illustrates the use of a default family handler.
%    \begin{macrocode}
%<*example>
\documentclass{article}
\usepackage[T1]{fontenc}
\usepackage{kvsetkeys}
\usepackage{keyval}

\makeatletter
\newcommand*{\tag}[2][]{%
  % #1: attributes
  % #2: tag name
  \begingroup
    \toks@={}%
    \let\@endslash\@empty
    \kvsetkeys{tag}{#1}%
    \texttt{%
      \textless #2\the\toks@\@endslash\textgreater
    }%
  \endgroup
}
\kv@set@family@handler{tag}{%
  % #1: key
  % #2: value
  \toks@\expandafter{%
    \the\toks@
    \space
    #1=\string"#2\string"%
  }%
}
\define@key{tag}{/}[]{%
  \def\@endslash{/}%
}
\makeatother

\begin{document}
\begin{tabbing}
  \mbox{}\qquad\=\qquad\=\kill
  \tag{html}\\
  \>\dots\\
  \>\tag[border=1]{table}\\
  \>\>\tag[width=200, span=3, /]{colgroup}\\
  \>\>\dots\\
  \>\tag{/table}\\
  \>\dots\\
  \tag{/html}\\
\end{tabbing}
\end{document}
%</example>
%    \end{macrocode}
%
% \StopEventually{
% }
%
% \section{Implementation}
%
% \subsection{Identification}
%
%    \begin{macrocode}
%<*package>
%    \end{macrocode}
%    Reload check, especially if the package is not used with \LaTeX.
%    \begin{macrocode}
\begingroup\catcode61\catcode48\catcode32=10\relax%
  \catcode13=5 % ^^M
  \endlinechar=13 %
  \catcode35=6 % #
  \catcode39=12 % '
  \catcode44=12 % ,
  \catcode45=12 % -
  \catcode46=12 % .
  \catcode58=12 % :
  \catcode64=11 % @
  \catcode123=1 % {
  \catcode125=2 % }
  \expandafter\let\expandafter\x\csname ver@kvsetkeys.sty\endcsname
  \ifx\x\relax % plain-TeX, first loading
  \else
    \def\empty{}%
    \ifx\x\empty % LaTeX, first loading,
      % variable is initialized, but \ProvidesPackage not yet seen
    \else
      \expandafter\ifx\csname PackageInfo\endcsname\relax
        \def\x#1#2{%
          \immediate\write-1{Package #1 Info: #2.}%
        }%
      \else
        \def\x#1#2{\PackageInfo{#1}{#2, stopped}}%
      \fi
      \x{kvsetkeys}{The package is already loaded}%
      \aftergroup\endinput
    \fi
  \fi
\endgroup%
%    \end{macrocode}
%    Package identification:
%    \begin{macrocode}
\begingroup\catcode61\catcode48\catcode32=10\relax%
  \catcode13=5 % ^^M
  \endlinechar=13 %
  \catcode35=6 % #
  \catcode39=12 % '
  \catcode40=12 % (
  \catcode41=12 % )
  \catcode44=12 % ,
  \catcode45=12 % -
  \catcode46=12 % .
  \catcode47=12 % /
  \catcode58=12 % :
  \catcode64=11 % @
  \catcode91=12 % [
  \catcode93=12 % ]
  \catcode123=1 % {
  \catcode125=2 % }
  \expandafter\ifx\csname ProvidesPackage\endcsname\relax
    \def\x#1#2#3[#4]{\endgroup
      \immediate\write-1{Package: #3 #4}%
      \xdef#1{#4}%
    }%
  \else
    \def\x#1#2[#3]{\endgroup
      #2[{#3}]%
      \ifx#1\@undefined
        \xdef#1{#3}%
      \fi
      \ifx#1\relax
        \xdef#1{#3}%
      \fi
    }%
  \fi
\expandafter\x\csname ver@kvsetkeys.sty\endcsname
\ProvidesPackage{kvsetkeys}%
  [2019/12/15 v1.18 Key value parser (HO)]%
%    \end{macrocode}
%
%    \begin{macrocode}
\begingroup\catcode61\catcode48\catcode32=10\relax%
  \catcode13=5 % ^^M
  \endlinechar=13 %
  \catcode123=1 % {
  \catcode125=2 % }
  \catcode64=11 % @
  \def\x{\endgroup
    \expandafter\edef\csname KVS@AtEnd\endcsname{%
      \endlinechar=\the\endlinechar\relax
      \catcode13=\the\catcode13\relax
      \catcode32=\the\catcode32\relax
      \catcode35=\the\catcode35\relax
      \catcode61=\the\catcode61\relax
      \catcode64=\the\catcode64\relax
      \catcode123=\the\catcode123\relax
      \catcode125=\the\catcode125\relax
    }%
  }%
\x\catcode61\catcode48\catcode32=10\relax%
\catcode13=5 % ^^M
\endlinechar=13 %
\catcode35=6 % #
\catcode64=11 % @
\catcode123=1 % {
\catcode125=2 % }
\def\TMP@EnsureCode#1#2{%
  \edef\KVS@AtEnd{%
    \KVS@AtEnd
    \catcode#1=\the\catcode#1\relax
  }%
  \catcode#1=#2\relax
}
\TMP@EnsureCode{36}{3}% $
\TMP@EnsureCode{38}{4}% &
\TMP@EnsureCode{39}{12}% '
\TMP@EnsureCode{43}{12}% +
\TMP@EnsureCode{44}{12}% ,
\TMP@EnsureCode{45}{12}% -
\TMP@EnsureCode{46}{12}% .
\TMP@EnsureCode{47}{12}% /
\TMP@EnsureCode{91}{12}% [
\TMP@EnsureCode{93}{12}% ]
\TMP@EnsureCode{94}{7}% ^ (superscript)
\TMP@EnsureCode{96}{12}% `
\TMP@EnsureCode{126}{13}% ~ (active)
\edef\KVS@AtEnd{\KVS@AtEnd\noexpand\endinput}
%    \end{macrocode}
%
% \subsection{Package loading}
%
%    \begin{macrocode}
\begingroup\expandafter\expandafter\expandafter\endgroup
\expandafter\ifx\csname RequirePackage\endcsname\relax
  \def\TMP@RequirePackage#1[#2]{%
    \begingroup\expandafter\expandafter\expandafter\endgroup
    \expandafter\ifx\csname ver@#1.sty\endcsname\relax
      \input #1.sty\relax
    \fi
  }%
  \TMP@RequirePackage{infwarerr}[2007/09/09]%
  \let\PackageError\@PackageError
\else
\fi
%    \end{macrocode}
%
%    \begin{macrocode}
\expandafter\ifx\csname toks@\endcsname\relax
  \toksdef\toks@=0 %
\fi
%    \end{macrocode}
%
% \subsection{Check for \eTeX}
%
%    \cs{unexpanded}, \cs{ifcsname}, and \cs{unless} are used if found.
%    \begin{macrocode}
\ifx\numexpr\@undefined
  \catcode`\$=9 % ignore
  \catcode`\&=14 % comment
\else % e-TeX
  \catcode`\$=14 % comment
  \catcode`\&=9  % ignore
\fi
%    \end{macrocode}
%
% \subsection{Generic help macros}
%
%    \begin{macro}{\KVS@Empty}
%    \begin{macrocode}
\def\KVS@Empty{}
%    \end{macrocode}
%    \end{macro}
%    \begin{macro}{\KVS@FirstOfTwo}
%    \begin{macrocode}
\long\def\KVS@FirstOfTwo#1#2{#1}
%    \end{macrocode}
%    \end{macro}
%    \begin{macro}{\KVS@SecondOfTwo}
%    \begin{macrocode}
\long\def\KVS@SecondOfTwo#1#2{#2}
%    \end{macrocode}
%    \end{macro}
%
%    \begin{macro}{\KVS@IfEmpty}
%    \begin{macrocode}
\long\def\KVS@IfEmpty#1{%
& \edef\KVS@Temp{\unexpanded{#1}}%
$ \begingroup
$   \toks@{#1}%
$   \edef\KVS@Temp{\the\toks@}%
$ \expandafter\endgroup
  \ifx\KVS@Temp\KVS@Empty
    \expandafter\KVS@FirstOfTwo
  \else
    \expandafter\KVS@SecondOfTwo
  \fi
}
%    \end{macrocode}
%    \end{macro}
%
% \subsection{Normalizing}
%
%    \begin{macro}{\kv@normalize}
%    \begin{macrocode}
\long\def\kv@normalize#1{%
  \begingroup
    \toks@{,#1,}%
    \KVS@Comma
    \KVS@SpaceComma
    \KVS@CommaSpace
    \KVS@CommaComma
    \KVS@Equals
    \KVS@SpaceEquals
    \KVS@EqualsSpace
    \xdef\KVS@Global{\the\toks@}%
  \endgroup
  \let\kv@list\KVS@Global
}
%    \end{macrocode}
%    \end{macro}
%    \begin{macro}{\comma@normalize}
%    \begin{macrocode}
\def\comma@normalize#1{%
  \begingroup
    \toks@{,#1,}%
    \KVS@Comma
    \KVS@SpaceComma
    \KVS@CommaSpace
    \KVS@CommaComma
    \xdef\KVS@Global{\the\toks@}%
  \endgroup
  \let\comma@list\KVS@Global
}
%    \end{macrocode}
%    \end{macro}
%    \begin{macro}{\KVS@Comma}
%    Converts active commas into comma with catcode other.
%    Also adds a comma at the end to protect the last value
%    for next cleanup steps.
%    \begin{macrocode}
\begingroup
  \lccode`\,=`\,%
  \lccode`\~=`\,%
\lowercase{\endgroup
  \def\KVS@Comma{%
    \toks@\expandafter{\expandafter}\expandafter
    \KVS@@Comma\the\toks@~\KVS@Nil
  }%
  \long\def\KVS@@Comma#1~#2\KVS@Nil{%
    \toks@\expandafter{\the\toks@#1}%
    \KVS@IfEmpty{#2}{%
    }{%
      \KVS@@Comma,#2\KVS@Nil
    }%
  }%
}
%    \end{macrocode}
%    \end{macro}
%    \begin{macro}{\KVS@SpaceComma}
%    Removes spaces before the comma, may add commas at the end.
%    \begin{macrocode}
\def\KVS@SpaceComma#1{%
  \def\KVS@SpaceComma{%
    \expandafter\KVS@@SpaceComma\the\toks@#1,\KVS@Nil
  }%
}
\KVS@SpaceComma{ }
%    \end{macrocode}
%    \end{macro}
%    \begin{macro}{\KVS@@SpaceComma}
%    \begin{macrocode}
\long\def\KVS@@SpaceComma#1 ,#2\KVS@Nil{%
  \KVS@IfEmpty{#2}{%
    \toks@{#1}%
  }{%
    \KVS@@SpaceComma#1,#2\KVS@Nil
  }%
}
%    \end{macrocode}
%    \end{macro}
%    \begin{macro}{\KVS@CommaSpace}
%    Removes spaces after the comma, may add commas at the end.
%    \begin{macrocode}
\def\KVS@CommaSpace{%
  \expandafter\KVS@@CommaSpace\the\toks@, \KVS@Nil
}
%    \end{macrocode}
%    \end{macro}
%    \begin{macro}{\KVS@@CommaSpace}
%    \begin{macrocode}
\long\def\KVS@@CommaSpace#1, #2\KVS@Nil{%
  \KVS@IfEmpty{#2}{%
    \toks@{#1}%
  }{%
    \KVS@@CommaSpace#1,#2\KVS@Nil
  }%
}
%    \end{macrocode}
%    \end{macro}
%    \begin{macro}{\KVS@CommaComma}
%    Replaces multiple commas by one comma.
%    \begin{macrocode}
\def\KVS@CommaComma{%
  \expandafter\KVS@@CommaComma\the\toks@,\KVS@Nil
}
%    \end{macrocode}
%    \end{macro}
%    \begin{macro}{\KVS@@CommaComma}
%    \begin{macrocode}
\long\def\KVS@@CommaComma#1,,#2\KVS@Nil{%
  \KVS@IfEmpty{#2}{%
    \toks@{#1,}% (!)
  }{%
    \KVS@@CommaComma#1,#2\KVS@Nil
  }%
}
%    \end{macrocode}
%    \end{macro}
%
%    \begin{macro}{\KVS@Equals}
%    Converts active equals signs into catcode other characters.
%    \begin{macrocode}
\begingroup
  \lccode`\==`\=%
  \lccode`\~=`\=%
\lowercase{\endgroup
  \def\KVS@Equals{%
    \toks@\expandafter{\expandafter}\expandafter
    \KVS@@Equals\the\toks@~\KVS@Nil
  }%
  \long\def\KVS@@Equals#1~#2\KVS@Nil{%
    \edef\KVS@Temp{\the\toks@}%
    \ifx\KVS@Temp\KVS@Empty
      \expandafter\KVS@FirstOfTwo
    \else
      \expandafter\KVS@SecondOfTwo
    \fi
    {%
      \toks@{#1}%
    }{%
      \toks@\expandafter{\the\toks@=#1}%
    }%
    \KVS@IfEmpty{#2}{%
    }{%
      \KVS@@Equals#2\KVS@Nil
    }%
  }%
}
%    \end{macrocode}
%    \end{macro}
%    \begin{macro}{\KVS@SpaceEquals}
%    Removes spaces before the equals sign.
%    \begin{macrocode}
\def\KVS@SpaceEquals#1{%
  \def\KVS@SpaceEquals{%
    \expandafter\KVS@@SpaceEquals\the\toks@#1=\KVS@Nil
  }%
}
\KVS@SpaceEquals{ }
%    \end{macrocode}
%    \end{macro}
%    \begin{macro}{\KVS@@SpaceEquals}
%    \begin{macrocode}
\long\def\KVS@@SpaceEquals#1 =#2\KVS@Nil{%
  \KVS@IfEmpty{#2}{%
    \toks@{#1}%
  }{%
    \KVS@@SpaceEquals#1=#2\KVS@Nil
  }%
}
%    \end{macrocode}
%    \end{macro}
%    \begin{macro}{\KVS@EqualsSpace}
%    Removes spaces after the equals sign.
%    \begin{macrocode}
\def\KVS@EqualsSpace{%
  \expandafter\KVS@@EqualsSpace\the\toks@= \KVS@Nil
}
%    \end{macrocode}
%    \end{macro}
%    \begin{macro}{\KVS@@EqualsSpace}
%    \begin{macrocode}
\long\def\KVS@@EqualsSpace#1= #2\KVS@Nil{%
  \KVS@IfEmpty{#2}{%
    \toks@{#1}%
  }{%
    \KVS@@EqualsSpace#1=#2\KVS@Nil
  }%
}
%    \end{macrocode}
%    \end{macro}
%
% \subsection{Parsing key value lists}
%
%    \begin{macro}{\kv@parse}
%    Normalizes and parses the key value list. Also sets \cs{kv@list}.
%    \begin{macrocode}
\long\def\kv@parse#1{%
  \kv@normalize{#1}%
  \expandafter\kv@parse@normalized\expandafter{\kv@list}%
}
%    \end{macrocode}
%    \end{macro}
%    \begin{macro}{\kv@parse@normalized}
%    |#1|: key value list\\
%    |#2|: processor
%    \begin{macrocode}
\long\def\kv@parse@normalized#1#2{%
  \KVS@Parse#1,\KVS@Nil{#2}%
}
%    \end{macrocode}
%    \end{macro}
%    \begin{macro}{\KVS@Parse}
%    |#1,#2|: key value list\\
%    |#3|: processor
%    \begin{macrocode}
\long\def\KVS@Parse#1,#2\KVS@Nil#3{%
  \KVS@IfEmpty{#1}{%
  }{%
    \KVS@Process#1=\KVS@Nil{#3}%
  }%
  \KVS@MaybeBreak
  \KVS@IfEmpty{#2}{%
  }{%
    \KVS@Parse#2\KVS@Nil{#3}%
  }%
}
%    \end{macrocode}
%    \end{macro}
%    \begin{macro}{\KVS@Process}
%    |#1|: key\\
%    |#2|: value, |=|\\
%    |#3|: processor
%    \begin{macrocode}
\long\def\KVS@Process#1=#2\KVS@Nil#3{%
  \let\KVS@MaybeBreak\relax
  \def\kv@key{#1}%
  \KVS@IfEmpty{#2}{%
    \let\kv@value\relax
    #3{#1}{}%
  }{%
    \KVS@@Process{#1}#2\KVS@Nil{#3}%
  }%
}
%    \end{macrocode}
%    \end{macro}
%    \begin{macro}{\KVS@@Process}
%    |#1|: key\\
%    |#2|: value\\
%    |#3|: processor
%    \begin{macrocode}
\long\def\KVS@@Process#1#2=\KVS@Nil#3{%
& \edef\kv@value{\unexpanded{#2}}%
$ \begingroup
$   \toks@{#2}%
$   \xdef\KVS@Global{\the\toks@}%
$ \endgroup
$ \let\kv@value\KVS@Global
  #3{#1}{#2}%
}
%    \end{macrocode}
%    \end{macro}
%    \begin{macro}{\KVS@MaybeBreak}
%    \begin{macrocode}
\let\KVS@MaybeBreak\relax
%    \end{macrocode}
%    \end{macro}
%    \begin{macro}{\KVS@break}
%    \begin{macrocode}
\def\KVS@break#1#2#3#4{%
  \let\KVS@MaybeBreak\relax
}
%    \end{macrocode}
%    \end{macro}
%    \begin{macro}{\kv@break}
%    \begin{macrocode}
\def\kv@break{%
  \let\KVS@MaybeBreak\KVS@break
}
%    \end{macrocode}
%    \end{macro}
%
% \subsection{Parsing comma lists}
%
%    \begin{macro}{\comma@parse}
%    Normalizes and parses the key value list. Also sets \cs{comma@list}.
%    \begin{macrocode}
\def\comma@parse#1{%
  \comma@normalize{#1}%
  \expandafter\comma@parse@normalized\expandafter{\comma@list}%
}
%    \end{macrocode}
%    \end{macro}
%
%    \begin{macro}{\comma@parse@normalized}
%    |#1|: comma list\\
%    |#2|: processor
%    \begin{macrocode}
\def\comma@parse@normalized#1#2{%
  \KVS@CommaParse#1,\KVS@Nil{#2}%
}
%    \end{macrocode}
%    \end{macro}
%    \begin{macro}{\KVS@CommaParse}
%    |#1,#2|: comma list\\
%    |#3|: processor
%    \begin{macrocode}
\def\KVS@CommaParse#1,#2\KVS@Nil#3{%
  \KVS@IfEmpty{#1}{%
  }{%
    \def\comma@entry{#1}%
    #3{#1}%
  }%
  \KVS@MaybeBreak
  \KVS@IfEmpty{#2}{%
  }{%
    \KVS@CommaParse#2\KVS@Nil{#3}%
  }%
}
%    \end{macrocode}
%    \end{macro}
%    \begin{macro}{\comma@break}
%    \begin{macrocode}
\def\comma@break{%
  \let\KVS@MaybeBreak\KVS@break
}
%    \end{macrocode}
%    \end{macro}
%
% \subsection{Processing key value pairs}
%
%    \begin{macro}{\kv@handled@false}
%    The handler can call \cs{kv@handled@false} or
%    \cs{kv@handled@true} so report failure or success.
%    The default is success (compatibility for versions
%    before 2011/10/18 v1.15).
%    \begin{macrocode}
\def\kv@handled@false{%
  \let\ifkv@handled@\iffalse
}
%    \end{macrocode}
%    \end{macro}
%    \begin{macro}{\kv@handled@true}
%    \begin{macrocode}
\def\kv@handled@true{%
  \let\ifkv@handled@\iftrue
}
%    \end{macrocode}
%    \end{macro}
%    \begin{macro}{\ifkv@handled@}
%    \begin{macrocode}
\kv@handled@true
%    \end{macrocode}
%    \end{macro}
%
%    \begin{macro}{\kv@processor@default}
%    \begin{macrocode}
\def\kv@processor@default#1#2{%
  \begingroup
    \csname @safe@activestrue\endcsname
    \let\ifincsname\iftrue
    \edef\KVS@temp{\endgroup
      \noexpand\KVS@ProcessorDefault{#1}{#2}%
    }%
  \KVS@temp
}
%    \end{macrocode}
%    \end{macro}
%    \begin{macro}{\KVS@ProcessorDefault}
%    \begin{macrocode}
\long\def\KVS@ProcessorDefault#1#2#3{%
  \def\kv@fam{#1}%
& \unless\ifcsname KV@#1@#2\endcsname
$ \begingroup\expandafter\expandafter\expandafter\endgroup
$ \expandafter\ifx\csname KV@#1@#2\endcsname\relax
&   \unless\ifcsname KVS@#1@handler\endcsname
$   \begingroup\expandafter\expandafter\expandafter\endgroup
$   \expandafter\ifx\csname KVS@#1@handler\endcsname\relax
      \kv@error@unknownkey{#1}{#2}%
    \else
      \kv@handled@true
      \csname KVS@#1@handler\endcsname{#2}{#3}%
      \relax
      \ifkv@handled@
      \else
        \kv@error@unknownkey{#1}{#2}%
      \fi
    \fi
  \else
    \ifx\kv@value\relax
&     \unless\ifcsname KV@#1@#2@default\endcsname
$     \begingroup\expandafter\expandafter\expandafter\endgroup
$     \expandafter\ifx\csname KV@#1@#2@default\endcsname\relax
        \kv@error@novalue{#1}{#2}%
      \else
        \csname KV@#1@#2@default\endcsname
        \relax
      \fi
    \else
      \csname KV@#1@#2\endcsname{#3}%
    \fi
  \fi
}
%    \end{macrocode}
%    \end{macro}
%
%    \begin{macro}{\kv@processor@known}
%    \begin{macrocode}
\def\kv@processor@known#1#2#3{%
  \begingroup
    \csname @safe@activestrue\endcsname
    \let\ifincsname\iftrue
    \edef\KVS@temp{\endgroup
      \noexpand\KVS@ProcessorKnown{#1}\noexpand#2{#3}%
    }%
  \KVS@temp
}
%    \end{macrocode}
%    \end{macro}
%    \begin{macro}{\KVS@ProcessorKnown}
%    \begin{macrocode}
\long\def\KVS@ProcessorKnown#1#2#3#4{%
  \def\kv@fam{#1}%
& \unless\ifcsname KV@#1@#3\endcsname
$ \begingroup\expandafter\expandafter\expandafter\endgroup
$ \expandafter\ifx\csname KV@#1@#3\endcsname\relax
&   \unless\ifcsname KVS@#1@handler\endcsname
$   \begingroup\expandafter\expandafter\expandafter\endgroup
$   \expandafter\ifx\csname KVS@#1@handler\endcsname\relax
      \KVS@AddUnhandled#2{#3}{#4}%
    \else
      \kv@handled@true
      \csname KVS@#1@handler\endcsname{#3}{#4}%
      \relax
      \ifkv@handled@
      \else
        \KVS@AddUnhandled#2{#3}{#4}%
      \fi
    \fi
  \else
    \ifx\kv@value\relax
&     \unless\ifcsname KV@#1@#2@default\endcsname
$     \begingroup\expandafter\expandafter\expandafter\endgroup
$     \expandafter\ifx\csname KV@#1@#3@default\endcsname\relax
        \kv@error@novalue{#1}{#3}%
      \else
        \csname KV@#1@#3@default\endcsname
        \relax
      \fi
    \else
      \csname KV@#1@#3\endcsname{#4}%
    \fi
  \fi
}
%    \end{macrocode}
%    \end{macro}
%    \begin{macro}{\KVS@AddUnhandled}
%    \begin{macrocode}
\long\def\KVS@AddUnhandled#1#2#3{%
& \edef#1{%
&   \ifx#1\KVS@empty
&   \else
&     \unexpanded{#1},%
&   \fi
&   \unexpanded{{#2}={#3}}%
& }%
$ \begingroup
$   \ifx#1\KVS@empty
$     \toks@{{#2}={#3}}%
$   \else
$     \toks@\expandafter{#1,{#2}={#3}}%
$   \fi
$   \xdef\KVS@Global{\the\toks@}%
$ \endgroup
$ \let#1\KVS@Global
}
%    \end{macrocode}
%    \end{macro}
%
%    \begin{macro}{\kv@set@family@handler}
%    \begin{macrocode}
\long\def\kv@set@family@handler#1#2{%
  \begingroup
    \csname @safe@activestrue\endcsname
    \let\ifincsname\iftrue
  \expandafter\endgroup
  \expandafter\def\csname KVS@#1@handler\endcsname##1##2{#2}%
}
%    \end{macrocode}
%    \end{macro}
%
%    \begin{macro}{\kv@unset@family@handler}
%    \begin{macrocode}
\long\def\kv@unset@family@handler#1#2{%
  \begingroup
    \csname @safe@activestrue\endcsname
    \let\ifincsname\iftrue
  \expandafter\endgroup
  \expandafter\let\csname KVS@#1@handler\endcsname\@UnDeFiNeD
}
%    \end{macrocode}
%    \end{macro}
%
% \subsection{Error handling}
%
%    \begin{macro}{\kv@error@novalue}
%    \begin{macrocode}
\def\kv@error@novalue{%
  \kv@error@generic{No value specified for}%
}
%    \end{macrocode}
%    \end{macro}
%    \begin{macro}{\kv@error@unknownkey}
%    \begin{macrocode}
\def\kv@error@unknownkey{%
  \kv@error@generic{Undefined}%
}
%    \end{macrocode}
%    \end{macro}
%    \begin{macro}{\kv@error@generic}
%    \begin{macrocode}
\def\kv@error@generic#1#2#3{%
  \PackageError{kvsetkeys}{%
    #1 key `#3'%
  }{%
    The keyval family of the key `#3' is `#2'.\MessageBreak
    The setting of the key is ignored because of the error.\MessageBreak
    \MessageBreak
    \@ehc
  }%
}
%    \end{macrocode}
%    \end{macro}
%
% \subsection{Do it all}
%
%    \begin{macro}{\kvsetkeys}
%    \begin{macrocode}
\long\def\kvsetkeys#1#2{%
  \kv@parse{#2}{\kv@processor@default{#1}}%
}
%    \end{macrocode}
%    \end{macro}
%
%    \begin{macro}{\kvsetkeys@expandafter}
%    \begin{macrocode}
\def\kvsetkeys@expandafter#1#2{%
  \expandafter\kv@parse\expandafter{#2}{%
    \kv@processor@default{#1}%
  }%
}
%    \end{macrocode}
%    \end{macro}
%
%    \begin{macro}{\KVS@cmd}
%    \begin{macrocode}
\def\KVS@cmd{0}%
%    \end{macrocode}
%    \end{macro}
%    \begin{macro}{\KVS@cmd@inc}
%    \begin{macrocode}
\def\KVS@cmd@inc{%
& \edef\KVS@cmd{\the\numexpr\KVS@cmd+1}%
$ \begingroup
$   \count255=\KVS@cmd\relax
$   \advance\count255 by 1\relax
$ \edef\x{\endgroup
$   \noexpand\def\noexpand\KVS@cmd{\number\count255}%
$ }%
$ \x
}
%    \end{macrocode}
%    \end{macro}
%    \begin{macro}{\KVS@cmd@dec}
%    \begin{macrocode}
\def\KVS@cmd@dec{%
& \edef\KVS@cmd{\the\numexpr\KVS@cmd-1}%
$ \begingroup
$   \count255=\KVS@cmd\relax
$   \advance\count255 by -1\relax
$ \edef\x{\endgroup
$   \noexpand\def\noexpand\KVS@cmd{\number\count255}%
$ }%
$ \x
}
%    \end{macrocode}
%    \end{macro}
%    \begin{macro}{\KVS@empty}
%    \begin{macrocode}
\def\KVS@empty{}
%    \end{macrocode}
%    \end{macro}
%
%    \begin{macro}{\kvsetknownkeys}
%    \begin{macrocode}
\def\kvsetknownkeys{%
  \expandafter
  \KVS@setknownkeys\csname KVS@cmd\KVS@cmd\endcsname{}%
}
%    \end{macrocode}
%    \end{macro}
%    \begin{macro}{\KVS@setknownkeys}
%    \begin{macrocode}
\long\def\KVS@setknownkeys#1#2#3#4#5{%
  \let#1\KVS@empty
  \KVS@cmd@inc
  #2\kv@parse#2{#5}{\kv@processor@known{#3}#1}%
  \KVS@cmd@dec
  \let#4=#1%
}
%    \end{macrocode}
%    \end{macro}
%
%    \begin{macro}{\kvsetknownkeys@expandafter}
%    \begin{macrocode}
\def\kvsetknownkeys@expandafter{%
  \expandafter
  \KVS@setknownkeys
      \csname KVS@cmd\KVS@cmd\endcsname\expandafter
}
%    \end{macrocode}
%    \end{macro}
%
%    \begin{macrocode}
\KVS@AtEnd%
%</package>
%    \end{macrocode}
%% \section{Installation}
%
% \subsection{Download}
%
% \paragraph{Package.} This package is available on
% CTAN\footnote{\CTANpkg{kvsetkeys}}:
% \begin{description}
% \item[\CTAN{macros/latex/contrib/kvsetkeys/kvsetkeys.dtx}] The source file.
% \item[\CTAN{macros/latex/contrib/kvsetkeys/kvsetkeys.pdf}] Documentation.
% \end{description}
%
%
% \paragraph{Bundle.} All the packages of the bundle `kvsetkeys'
% are also available in a TDS compliant ZIP archive. There
% the packages are already unpacked and the documentation files
% are generated. The files and directories obey the TDS standard.
% \begin{description}
% \item[\CTANinstall{install/macros/latex/contrib/kvsetkeys.tds.zip}]
% \end{description}
% \emph{TDS} refers to the standard ``A Directory Structure
% for \TeX\ Files'' (\CTANpkg{tds}). Directories
% with \xfile{texmf} in their name are usually organized this way.
%
% \subsection{Bundle installation}
%
% \paragraph{Unpacking.} Unpack the \xfile{kvsetkeys.tds.zip} in the
% TDS tree (also known as \xfile{texmf} tree) of your choice.
% Example (linux):
% \begin{quote}
%   |unzip kvsetkeys.tds.zip -d ~/texmf|
% \end{quote}
%
% \subsection{Package installation}
%
% \paragraph{Unpacking.} The \xfile{.dtx} file is a self-extracting
% \docstrip\ archive. The files are extracted by running the
% \xfile{.dtx} through \plainTeX:
% \begin{quote}
%   \verb|tex kvsetkeys.dtx|
% \end{quote}
%
% \paragraph{TDS.} Now the different files must be moved into
% the different directories in your installation TDS tree
% (also known as \xfile{texmf} tree):
% \begin{quote}
% \def\t{^^A
% \begin{tabular}{@{}>{\ttfamily}l@{ $\rightarrow$ }>{\ttfamily}l@{}}
%   kvsetkeys.sty & tex/generic/kvsetkeys/kvsetkeys.sty\\
%   kvsetkeys.pdf & doc/latex/kvsetkeys/kvsetkeys.pdf\\
%   kvsetkeys-example.tex & doc/latex/kvsetkeys/kvsetkeys-example.tex\\
%   kvsetkeys.dtx & source/latex/kvsetkeys/kvsetkeys.dtx\\
% \end{tabular}^^A
% }^^A
% \sbox0{\t}^^A
% \ifdim\wd0>\linewidth
%   \begingroup
%     \advance\linewidth by\leftmargin
%     \advance\linewidth by\rightmargin
%   \edef\x{\endgroup
%     \def\noexpand\lw{\the\linewidth}^^A
%   }\x
%   \def\lwbox{^^A
%     \leavevmode
%     \hbox to \linewidth{^^A
%       \kern-\leftmargin\relax
%       \hss
%       \usebox0
%       \hss
%       \kern-\rightmargin\relax
%     }^^A
%   }^^A
%   \ifdim\wd0>\lw
%     \sbox0{\small\t}^^A
%     \ifdim\wd0>\linewidth
%       \ifdim\wd0>\lw
%         \sbox0{\footnotesize\t}^^A
%         \ifdim\wd0>\linewidth
%           \ifdim\wd0>\lw
%             \sbox0{\scriptsize\t}^^A
%             \ifdim\wd0>\linewidth
%               \ifdim\wd0>\lw
%                 \sbox0{\tiny\t}^^A
%                 \ifdim\wd0>\linewidth
%                   \lwbox
%                 \else
%                   \usebox0
%                 \fi
%               \else
%                 \lwbox
%               \fi
%             \else
%               \usebox0
%             \fi
%           \else
%             \lwbox
%           \fi
%         \else
%           \usebox0
%         \fi
%       \else
%         \lwbox
%       \fi
%     \else
%       \usebox0
%     \fi
%   \else
%     \lwbox
%   \fi
% \else
%   \usebox0
% \fi
% \end{quote}
% If you have a \xfile{docstrip.cfg} that configures and enables \docstrip's
% TDS installing feature, then some files can already be in the right
% place, see the documentation of \docstrip.
%
% \subsection{Refresh file name databases}
%
% If your \TeX~distribution
% (\TeX\,Live, \mikTeX, \dots) relies on file name databases, you must refresh
% these. For example, \TeX\,Live\ users run \verb|texhash| or
% \verb|mktexlsr|.
%
% \subsection{Some details for the interested}
%
% \paragraph{Unpacking with \LaTeX.}
% The \xfile{.dtx} chooses its action depending on the format:
% \begin{description}
% \item[\plainTeX:] Run \docstrip\ and extract the files.
% \item[\LaTeX:] Generate the documentation.
% \end{description}
% If you insist on using \LaTeX\ for \docstrip\ (really,
% \docstrip\ does not need \LaTeX), then inform the autodetect routine
% about your intention:
% \begin{quote}
%   \verb|latex \let\install=y% \iffalse meta-comment
%
% File: kvsetkeys.dtx
% Version: 2019/12/15 v1.18
% Info: Key value parser
%
% Copyright (C)
%    2006, 2007, 2009-2012 Heiko Oberdiek
%    2016-2019 Oberdiek Package Support Group
%    https://github.com/ho-tex/kvsetkeys/issues
%
% This work may be distributed and/or modified under the
% conditions of the LaTeX Project Public License, either
% version 1.3c of this license or (at your option) any later
% version. This version of this license is in
%    https://www.latex-project.org/lppl/lppl-1-3c.txt
% and the latest version of this license is in
%    https://www.latex-project.org/lppl.txt
% and version 1.3 or later is part of all distributions of
% LaTeX version 2005/12/01 or later.
%
% This work has the LPPL maintenance status "maintained".
%
% The Current Maintainers of this work are
% Heiko Oberdiek and the Oberdiek Package Support Group
% https://github.com/ho-tex/kvsetkeys/issues
%
% The Base Interpreter refers to any `TeX-Format',
% because some files are installed in TDS:tex/generic//.
%
% This work consists of the main source file kvsetkeys.dtx
% and the derived files
%    kvsetkeys.sty, kvsetkeys.pdf, kvsetkeys.ins, kvsetkeys.drv,
%    kvsetkeys-example.tex, kvsetkeys-test1.tex,
%    kvsetkeys-test2.tex, kvsetkeys-test3.tex,
%    kvsetkeys-test4.tex.
%
% Distribution:
%    CTAN:macros/latex/contrib/kvsetkeys/kvsetkeys.dtx
%    CTAN:macros/latex/contrib/kvsetkeys/kvsetkeys.pdf
%
% Unpacking:
%    (a) If kvsetkeys.ins is present:
%           tex kvsetkeys.ins
%    (b) Without kvsetkeys.ins:
%           tex kvsetkeys.dtx
%    (c) If you insist on using LaTeX
%           latex \let\install=y% \iffalse meta-comment
%
% File: kvsetkeys.dtx
% Version: 2019/12/15 v1.18
% Info: Key value parser
%
% Copyright (C)
%    2006, 2007, 2009-2012 Heiko Oberdiek
%    2016-2019 Oberdiek Package Support Group
%    https://github.com/ho-tex/kvsetkeys/issues
%
% This work may be distributed and/or modified under the
% conditions of the LaTeX Project Public License, either
% version 1.3c of this license or (at your option) any later
% version. This version of this license is in
%    https://www.latex-project.org/lppl/lppl-1-3c.txt
% and the latest version of this license is in
%    https://www.latex-project.org/lppl.txt
% and version 1.3 or later is part of all distributions of
% LaTeX version 2005/12/01 or later.
%
% This work has the LPPL maintenance status "maintained".
%
% The Current Maintainers of this work are
% Heiko Oberdiek and the Oberdiek Package Support Group
% https://github.com/ho-tex/kvsetkeys/issues
%
% The Base Interpreter refers to any `TeX-Format',
% because some files are installed in TDS:tex/generic//.
%
% This work consists of the main source file kvsetkeys.dtx
% and the derived files
%    kvsetkeys.sty, kvsetkeys.pdf, kvsetkeys.ins, kvsetkeys.drv,
%    kvsetkeys-example.tex, kvsetkeys-test1.tex,
%    kvsetkeys-test2.tex, kvsetkeys-test3.tex,
%    kvsetkeys-test4.tex.
%
% Distribution:
%    CTAN:macros/latex/contrib/kvsetkeys/kvsetkeys.dtx
%    CTAN:macros/latex/contrib/kvsetkeys/kvsetkeys.pdf
%
% Unpacking:
%    (a) If kvsetkeys.ins is present:
%           tex kvsetkeys.ins
%    (b) Without kvsetkeys.ins:
%           tex kvsetkeys.dtx
%    (c) If you insist on using LaTeX
%           latex \let\install=y\input{kvsetkeys.dtx}
%        (quote the arguments according to the demands of your shell)
%
% Documentation:
%    (a) If kvsetkeys.drv is present:
%           latex kvsetkeys.drv
%    (b) Without kvsetkeys.drv:
%           latex kvsetkeys.dtx; ...
%    The class ltxdoc loads the configuration file ltxdoc.cfg
%    if available. Here you can specify further options, e.g.
%    use A4 as paper format:
%       \PassOptionsToClass{a4paper}{article}
%
%    Programm calls to get the documentation (example):
%       pdflatex kvsetkeys.dtx
%       makeindex -s gind.ist kvsetkeys.idx
%       pdflatex kvsetkeys.dtx
%       makeindex -s gind.ist kvsetkeys.idx
%       pdflatex kvsetkeys.dtx
%
% Installation:
%    TDS:tex/generic/kvsetkeys/kvsetkeys.sty
%    TDS:doc/latex/kvsetkeys/kvsetkeys.pdf
%    TDS:doc/latex/kvsetkeys/kvsetkeys-example.tex
%    TDS:source/latex/kvsetkeys/kvsetkeys.dtx
%
%<*ignore>
\begingroup
  \catcode123=1 %
  \catcode125=2 %
  \def\x{LaTeX2e}%
\expandafter\endgroup
\ifcase 0\ifx\install y1\fi\expandafter
         \ifx\csname processbatchFile\endcsname\relax\else1\fi
         \ifx\fmtname\x\else 1\fi\relax
\else\csname fi\endcsname
%</ignore>
%<*install>
\input docstrip.tex
\Msg{************************************************************************}
\Msg{* Installation}
\Msg{* Package: kvsetkeys 2019/12/15 v1.18 Key value parser (HO)}
\Msg{************************************************************************}

\keepsilent
\askforoverwritefalse

\let\MetaPrefix\relax
\preamble

This is a generated file.

Project: kvsetkeys
Version: 2019/12/15 v1.18

Copyright (C)
   2006, 2007, 2009-2012 Heiko Oberdiek
   2016-2019 Oberdiek Package Support Group

This work may be distributed and/or modified under the
conditions of the LaTeX Project Public License, either
version 1.3c of this license or (at your option) any later
version. This version of this license is in
   https://www.latex-project.org/lppl/lppl-1-3c.txt
and the latest version of this license is in
   https://www.latex-project.org/lppl.txt
and version 1.3 or later is part of all distributions of
LaTeX version 2005/12/01 or later.

This work has the LPPL maintenance status "maintained".

The Current Maintainers of this work are
Heiko Oberdiek and the Oberdiek Package Support Group
https://github.com/ho-tex/kvsetkeys/issues


The Base Interpreter refers to any `TeX-Format',
because some files are installed in TDS:tex/generic//.

This work consists of the main source file kvsetkeys.dtx
and the derived files
   kvsetkeys.sty, kvsetkeys.pdf, kvsetkeys.ins, kvsetkeys.drv,
   kvsetkeys-example.tex, kvsetkeys-test1.tex,
   kvsetkeys-test2.tex, kvsetkeys-test3.tex,
   kvsetkeys-test4.tex.

\endpreamble
\let\MetaPrefix\DoubleperCent

\generate{%
  \file{kvsetkeys.ins}{\from{kvsetkeys.dtx}{install}}%
  \file{kvsetkeys.drv}{\from{kvsetkeys.dtx}{driver}}%
  \usedir{tex/generic/kvsetkeys}%
  \file{kvsetkeys.sty}{\from{kvsetkeys.dtx}{package}}%
  \usedir{doc/latex/kvsetkeys}%
  \file{kvsetkeys-example.tex}{\from{kvsetkeys.dtx}{example}}%
}

\catcode32=13\relax% active space
\let =\space%
\Msg{************************************************************************}
\Msg{*}
\Msg{* To finish the installation you have to move the following}
\Msg{* file into a directory searched by TeX:}
\Msg{*}
\Msg{*     kvsetkeys.sty}
\Msg{*}
\Msg{* To produce the documentation run the file `kvsetkeys.drv'}
\Msg{* through LaTeX.}
\Msg{*}
\Msg{* Happy TeXing!}
\Msg{*}
\Msg{************************************************************************}

\endbatchfile
%</install>
%<*ignore>
\fi
%</ignore>
%<*driver>
\NeedsTeXFormat{LaTeX2e}
\ProvidesFile{kvsetkeys.drv}%
  [2019/12/15 v1.18 Key value parser (HO)]%
\documentclass{ltxdoc}
\usepackage{holtxdoc}[2011/11/22]
\begin{document}
  \DocInput{kvsetkeys.dtx}%
\end{document}
%</driver>
% \fi
%
%
%
% \GetFileInfo{kvsetkeys.drv}
%
% \title{The \xpackage{kvsetkeys} package}
% \date{2019/12/15 v1.18}
% \author{Heiko Oberdiek\thanks
% {Please report any issues at \url{https://github.com/ho-tex/kvsetkeys/issues}}}
%
% \maketitle
%
% \begin{abstract}
% Package \xpackage{kvsetkeys} provides \cs{kvsetkeys}, a variant
% of package \xpackage{keyval}'s \cs{setkeys}. It allows to specify
% a handler that deals with unknown options. Active commas and equal
% signs may be used (e.g. see \xpackage{babel}'s shorthands) and
% only one level of curly braces is removed from the values.
% \end{abstract}
%
% \tableofcontents
%
% \def\M#1{\texttt{\{}\meta{#1}\texttt{\}}}
%
% \section{Documentation}
%
% First I want to recommend the very good review article
% ``A guide to key-value methods'' by Joseph Wright \cite{tb94wright}.
% It introduces the different key-value packages and compares them.
%
% \subsection{Motivation}
%
% \cs{kvsetkeys} serves as replacement for \xpackage{keyval}'s
% \cs{setkeys}. It basically uses the same syntax. But the
% implementation is more robust and predictable:
% \begin{description}
% \item[Active syntax characters:]
%   Comma `|,|' and the equals sign `|=|' are used inside
%   key value lists as syntax characters. Package \xpackage{keyval}
%   uses the catcode of the characters that is active during
%   package loading, usually this is catcode 12 (other).
%   But it can happen that the catcode setting of the syntax characters
%   changes. Especially active characters are of interest, because
%   some language adaptations uses them. For example, option \xoption{turkish}
%   of package \xpackage{babel} uses the equals sign as active shorthand
%   character. Therefore package \xpackage{kvsetkeys} deals with
%   both catcode settings 12 (other) and 13 (active).
% \item[Brace removal:]
%   Package \xpackage{keyval}'s \cs{setkeys} removes up to two
%   levels of curly braces around the value in some unpredictable way:
%\begin{quote}
%|\setkeys{fam}{key={{value}}} || || | $\rightarrow$ |value|\\
%|\setkeys{fam}{key={{{value}}}} | $\rightarrow$ |{value}|\\
%|\setkeys{fam}{key= {{{value}}}}| $\rightarrow$ |{{value}}|
%\end{quote}
%   This package \xpackage{kvsetkeys} follows a much stronger rule:
%   Exactly one level of braces are removed from an item, if the
%   item is surrounded by curly braces. An item can be a
%   the key value pair, the key or the value.
%\begin{quote}
%|\kvsetkeys{fam}{key={value}} || | $\rightarrow$ |value|\\
%|\kvsetkeys{fam}{key={{value}} | $\rightarrow$ |{value}|\\
%|\kvsetkeys{fam}{key= {{value}}| $\rightarrow$ |{value}|
%\end{quote}
% \item[Arbitrary values:] Unmatched conditionals are supported.
% \end{description}
%
% Before I describe
% \cs{kvsetkeys} in more detail, first I want to explain, how
% this package deals with key value lists. For the package also
% provides low level interfaces that can be used by package authors.
%
% \subsection{Normalizing key value lists}
%
% \begin{declcs}{kv@normalize}\,\M{key value list}
% \end{declcs}
% If the user specifies key value lists, he usually prefers
% nice formatted source code, e.g.:
% \begin{quote}
%\begin{verbatim}
%\hypersetup{
%  pdftitle    = {...},
%  pdfsubject  = {...},
%  pdfauthor   = {...},
%  pdfkeywords = {...},
%  ...
%}
%\end{verbatim}
% \end{quote}
% Thus there can be spaces around keys, around |=| or around the value.
% Also empty entries are possible by too many commas. Therefore these
% spaces and empty entries are silently removed by package \xpackage{keyval}
% and this package. Whereas the contents of the value can be protected
% by curly braces, especially if spaces or commas are used inside,
% a key name must not use spaces or other syntax characters.
%
% \cs{kv@normalize} takes a key value list and performs the cleanup:
% \begin{itemize}
% \item Spaces are removed.
% \item Syntax characters (comma and equal sign) that are active
%   are replaced by the same characters with standard catcode.
%   (Example: \xpackage{babel}'s language option \xoption{turkish}
%   uses the equal sign as active shorthand character.)
% \end{itemize}
% The result is stored in \cs{kv@list}, e.g.:
% \begin{quote}
% |\kv@list| $\rightarrow$ |,pdftitle={...},pdfsubject={...},...,|
% \end{quote}
% Curly braces around values (or keys) remain untouched.
% \begin{description}
% \item[v1.3+:]
% One comma is added in front of the list and each pair ends with
% a comma. Thus an empty list consists of one comma, otherwise
% two commas encloses the list. Empty entries other than the first
% are removed.
% \item[v1.0 -- v1.2:]
% Empty entries are removed later. In fact it adds a comma at the begin
% and end to protect the last value and an easier implementation.
% \end{description}
%
% \subsection{Parsing key value lists}
%
% \begin{declcs}{kv@parse}\,\M{key value list}\,\M{processor}
% \end{declcs}
% It is easier to parse a normalized list, thus \cs{kv@parse}
% normalizes the list and calls \cs{kv@parse@normalized}.
%
% \begin{declcs}{kv@parse@normalized}\,\M{key value list}%
%   \,\M{processor}
% \end{declcs}
% Now the key value list is split into single key value pairs.
% For further processing the key and value are given as arguments
% for the \meta{processor}:
% \begin{quote}
%   \meta{processor}\,\M{key}\,\M{value}
% \end{quote}
% Also key and value are stored in macro names:
% \begin{itemize}
% \item \cs{kv@key} stores the key.
% \item \cs{kv@value} stores the value or if the value was not
%   specified it has the meaning \cs{relax}.
% \end{itemize}
% The behaviour in pseudo code:
% \begin{quote}
%   foreach (\meta{key}, \meta{value}) in (\meta{key value list})\\
%   \hspace*{1.5em}\cs{kv@key} := \meta{key}\\
%   \hspace*{1.5em}\cs{kv@value} := \meta{value}\\
%   \hspace*{1.5em}\meta{processor}\,\M{key}\,\M{value}
% \end{quote}
%
% \begin{declcs}{kv@break}
% \end{declcs}
% Since version 2011/03/03 v1.11 \cs{kv@break} can be
% called inside the \meta{processor}
% of \cs{kv@parse} or \cs{kv@parse@normalized}, then
% the processing is stopped and the following entries discarded.
%
% \subsection{Processing key value pairs}
%
% Key value pairs can be processed in many different ways.
% For example, the processor for \cs{kvsetkeys} works similar
% to \cs{setkeys} of package \xpackage{keyval}. There unknown
% keys raise an error.
%
% Package \xpackage{xkeyval} also knows a star form of \cs{setkeys}
% that stores unknown keys in an internal macro for further processing
% with \cs{setrmkeys} and similar macros. This feature is covered
% by processor \cs{kv@processor@known}.
%
% \subsubsection{Processing similar to \xpackage{keyval}}
%
% \begin{declcs}{kv@processor@default}\,\M{family}\,\M{key}\,\M{value}
% \end{declcs}
% There are many possiblities to process key value pairs.
% \cs{kv@processor@default} is the processor used in \cs{kvsetkeys}.
% It reimplements and extends the behaviour of
% \xpackage{keyval}'s \cs{setkeys}.
% In case of unknown keys \cs{setkeys} raise an error.
% This processer, however, calls a handler instead, if it
% is provided by the family. Both \meta{family} and \meta{key}
% may contain package \xpackage{babel}'s shorthands
% (since 2011/04/07 v1.13).
%
% Since 2011/10/18 v1.15 the family handler can reject the
% successful handling of a key by calling \cs{kv@handled@false}.
%
% Since 2019/12/15 v1.18 \cs{kv@processor@default} also defines
% macro \cs{kv@fam} with meaning \meta{family} for convenience.
%
% \subsubsection{Processing similar to \cs{setkeys*} of package \xpackage{xkeyval}}
%
% \begin{declcs}{kv@processor@known}\,\M{family}\,^^A
%   \M{cmd}\,\M{key}\,\M{value}
% \end{declcs}
% The key value processor \cs{kv@processor@known} behaves similar
% to \cs{kv@processor@default}. If the \meta{key} exists in the
% \meta{family} its code is called, otherwise the family handler
% is tried. If the family handler is not set or cannot handle the
% key, the unknown key value pair is added to the macro \meta{cmd}.
% Since 2011/10/18 v1.15.
%
% The behaviour in pseudo code:
% \begin{quote}
%   if \meta{key} exists\\
%   \hspace*{1.5em}call the keyval code of \meta{key}\\
%   else\\
%   \hspace*{1.5em}if \meta{handler} for \meta{family} exists\\
%   \hspace*{3em}handled = true\\
%   \hspace*{3em}\meta{handler}\,\M{key}\,\M{value}\\
%   \hspace*{3em}if handled\\
%   \hspace*{3em}else\\
%   \hspace*{4.5em}add \texttt{"}\M{key}\texttt{=}\M{value}\texttt{"}
%     to \M{cmd}\\
%   \hspace*{3em}fi\\
%   \hspace*{1.5em}else\\
%   \hspace*{3em}add \texttt{"}\M{key}\texttt{=}\M{value}\texttt{"}
%     to \M{cmd}\\
%   \hspace*{3em}raise unknown key error\\
%   \hspace*{1.5em}fi\\
%   fi
% \end{quote}
%
% Since 2019/12/15 v1.18 \cs{kv@processor@known} also defines
% macro \cs{kv@fam} with meaning \meta{family} for convenience.
%
% \subsection{Default family handler}
%
% \cs{kv@processor@default} calls \meta{handler}, the default
% handler for the family, if the key does not exist in the family.
% The handler is called with two arguments, the key and the value.
% It can be defined with \cs{kv@set@family@hander}:
%
% \begin{declcs}{kv@set@family@handler}\,\M{family}\,\M{handler definition}
% \end{declcs}
% This sets the default family handler for the keyval family
% \meta{family}. Inside \meta{handler definition} |#1| stands for
% the key and |#2| is the value. Also \cs{kv@key} and \cs{kv@value}
% can be used for the key and the value. If the value is not given,
% \cs{kv@value} has the meaning \cs{relax}.
%
% \begin{declcs}{kv@unset@family@handler}\,\M{family}
% \end{declcs}
% It removes the family handler for \meta{family}.
% Since 2011/10/18 v1.15.
%
% \subsection{Put it all together}
%
% \begin{declcs}{kvsetkeys}\,\M{family}\,\M{key value list}
% \end{declcs}
% Macro \cs{kvsetkeys} processes the \meta{key value list} with
% the standard processor \cs{kv@processor@default}:
% \begin{quote}
%   \cs{kv@parse}\,\M{key value list}^^A
%   \texttt{\{}\cs{kv@processor@default}\,\M{family}\texttt{\}}
% \end{quote}
%
% \begin{declcs}{kvsetknownkeys}\,\M{family}\,\M{cmd}\,\M{key value list}
% \end{declcs}
% Macro \cs{kvsetknownkeys} processes the \meta{key value list}
% with processor \cs{kv@processor@known}. All key value pairs
% with keys that are not known in \meta{family} are stored
% in macro \meta{cmd}. A previous contents of macro \meta{cmd}
% will be overwritten. If all keys can be handled, \meta{cmd}
% will be empty, otherwise it contains a key value list of
% unhandled key value pairs.
% Since 2011/10/18 v1.15.
%
% Pseudo code:
% \begin{quote}
%   create macro \meta{cmdaux} with unique name (inside the current group)\\
%   \cs{def}\meta{cmdaux}\{\}\\
%   \cs{kv@parse}\,\M{key value list}^^A
%   \texttt{\{}\cs{kv@processor@known}\,\M{family}\,\M{cmdaux}\texttt{\}}\\
%   \cs{let}\meta{cmd}=\meta{cmdaux}
% \end{quote}
%
% \begin{declcs}{kvsetkeys@expandafter}\,\M{family}\,\M{list cmd}\\
%   \cs{kvsetknownkeys@expandafter}\,\M{family}\,%
%   \M{cmd}\,\M{list cmd}
% \end{declcs}
% Both macros behave like the counterparts without suffix
% |@expandafter|. The difference is that the key value list is
% given as macro that is expanded once.
% Since 2011/10/18 v1.15.
%
% Thus you can replace \cs{setkeys} of package \xpackage{keyval}
% by the key value parser of this package:
% \begin{quote}
%   |\renewcommand*{\setkeys}{\kvsetkeys}|\\
%   or\\
%   |\let\setkeys\kvsetkeys|
% \end{quote}
%
% \subsection{Comma separated lists}
%
% Since version 2007/09/29 v1.3 this package also supports the normalizing
% and parsing of general comma separated lists.
%
% \begin{declcs}{comma@normalize}\,\M{comma list}
% \end{declcs}
% Macro \cs{comma@normalize} normalizes the comma separated list,
% removes spaces around commas. The result is put in macro \cs{comma@list}.
%
% \begin{declcs}{comma@parse}\,\M{comma list}\,\M{processor}
% \end{declcs}
% Macro \cs{comma@parse} first normalizes the comma separated list
% and then parses the list by calling \cs{comma@parse@normalized}.
%
% \begin{declcs}{comma@parse@normalized}\,\M{normalized comma list}^^A
%   \,\M{processor}
% \end{declcs}
% The list is parsed. Empty entries are ignored. \meta{processor}
% is called for each non-empty entry with the entry as argument:
% \begin{quote}
%   \meta{processor}|{|\meta{entry}|}|
% \end{quote}
% Also the entry is stored in the macro \cs{comma@entry}.
%
% \begin{declcs}{comma@break}
% \end{declcs}
% Since version 2011/03/03 v1.11 \cs{comma@break} can be
% called inside the \meta{processor}
% of \cs{comma@parse} or \cs{comma@parse@normalized}, then
% the processing is stopped and the following entries discarded.
%
% \section{Example}
%
% The following example prints a short piece of HTML code using
% the tabbing environment for indenting purpose and a key value
% syntax for specifying the attributes of an HTML tag.
% The example illustrates the use of a default family handler.
%    \begin{macrocode}
%<*example>
\documentclass{article}
\usepackage[T1]{fontenc}
\usepackage{kvsetkeys}
\usepackage{keyval}

\makeatletter
\newcommand*{\tag}[2][]{%
  % #1: attributes
  % #2: tag name
  \begingroup
    \toks@={}%
    \let\@endslash\@empty
    \kvsetkeys{tag}{#1}%
    \texttt{%
      \textless #2\the\toks@\@endslash\textgreater
    }%
  \endgroup
}
\kv@set@family@handler{tag}{%
  % #1: key
  % #2: value
  \toks@\expandafter{%
    \the\toks@
    \space
    #1=\string"#2\string"%
  }%
}
\define@key{tag}{/}[]{%
  \def\@endslash{/}%
}
\makeatother

\begin{document}
\begin{tabbing}
  \mbox{}\qquad\=\qquad\=\kill
  \tag{html}\\
  \>\dots\\
  \>\tag[border=1]{table}\\
  \>\>\tag[width=200, span=3, /]{colgroup}\\
  \>\>\dots\\
  \>\tag{/table}\\
  \>\dots\\
  \tag{/html}\\
\end{tabbing}
\end{document}
%</example>
%    \end{macrocode}
%
% \StopEventually{
% }
%
% \section{Implementation}
%
% \subsection{Identification}
%
%    \begin{macrocode}
%<*package>
%    \end{macrocode}
%    Reload check, especially if the package is not used with \LaTeX.
%    \begin{macrocode}
\begingroup\catcode61\catcode48\catcode32=10\relax%
  \catcode13=5 % ^^M
  \endlinechar=13 %
  \catcode35=6 % #
  \catcode39=12 % '
  \catcode44=12 % ,
  \catcode45=12 % -
  \catcode46=12 % .
  \catcode58=12 % :
  \catcode64=11 % @
  \catcode123=1 % {
  \catcode125=2 % }
  \expandafter\let\expandafter\x\csname ver@kvsetkeys.sty\endcsname
  \ifx\x\relax % plain-TeX, first loading
  \else
    \def\empty{}%
    \ifx\x\empty % LaTeX, first loading,
      % variable is initialized, but \ProvidesPackage not yet seen
    \else
      \expandafter\ifx\csname PackageInfo\endcsname\relax
        \def\x#1#2{%
          \immediate\write-1{Package #1 Info: #2.}%
        }%
      \else
        \def\x#1#2{\PackageInfo{#1}{#2, stopped}}%
      \fi
      \x{kvsetkeys}{The package is already loaded}%
      \aftergroup\endinput
    \fi
  \fi
\endgroup%
%    \end{macrocode}
%    Package identification:
%    \begin{macrocode}
\begingroup\catcode61\catcode48\catcode32=10\relax%
  \catcode13=5 % ^^M
  \endlinechar=13 %
  \catcode35=6 % #
  \catcode39=12 % '
  \catcode40=12 % (
  \catcode41=12 % )
  \catcode44=12 % ,
  \catcode45=12 % -
  \catcode46=12 % .
  \catcode47=12 % /
  \catcode58=12 % :
  \catcode64=11 % @
  \catcode91=12 % [
  \catcode93=12 % ]
  \catcode123=1 % {
  \catcode125=2 % }
  \expandafter\ifx\csname ProvidesPackage\endcsname\relax
    \def\x#1#2#3[#4]{\endgroup
      \immediate\write-1{Package: #3 #4}%
      \xdef#1{#4}%
    }%
  \else
    \def\x#1#2[#3]{\endgroup
      #2[{#3}]%
      \ifx#1\@undefined
        \xdef#1{#3}%
      \fi
      \ifx#1\relax
        \xdef#1{#3}%
      \fi
    }%
  \fi
\expandafter\x\csname ver@kvsetkeys.sty\endcsname
\ProvidesPackage{kvsetkeys}%
  [2019/12/15 v1.18 Key value parser (HO)]%
%    \end{macrocode}
%
%    \begin{macrocode}
\begingroup\catcode61\catcode48\catcode32=10\relax%
  \catcode13=5 % ^^M
  \endlinechar=13 %
  \catcode123=1 % {
  \catcode125=2 % }
  \catcode64=11 % @
  \def\x{\endgroup
    \expandafter\edef\csname KVS@AtEnd\endcsname{%
      \endlinechar=\the\endlinechar\relax
      \catcode13=\the\catcode13\relax
      \catcode32=\the\catcode32\relax
      \catcode35=\the\catcode35\relax
      \catcode61=\the\catcode61\relax
      \catcode64=\the\catcode64\relax
      \catcode123=\the\catcode123\relax
      \catcode125=\the\catcode125\relax
    }%
  }%
\x\catcode61\catcode48\catcode32=10\relax%
\catcode13=5 % ^^M
\endlinechar=13 %
\catcode35=6 % #
\catcode64=11 % @
\catcode123=1 % {
\catcode125=2 % }
\def\TMP@EnsureCode#1#2{%
  \edef\KVS@AtEnd{%
    \KVS@AtEnd
    \catcode#1=\the\catcode#1\relax
  }%
  \catcode#1=#2\relax
}
\TMP@EnsureCode{36}{3}% $
\TMP@EnsureCode{38}{4}% &
\TMP@EnsureCode{39}{12}% '
\TMP@EnsureCode{43}{12}% +
\TMP@EnsureCode{44}{12}% ,
\TMP@EnsureCode{45}{12}% -
\TMP@EnsureCode{46}{12}% .
\TMP@EnsureCode{47}{12}% /
\TMP@EnsureCode{91}{12}% [
\TMP@EnsureCode{93}{12}% ]
\TMP@EnsureCode{94}{7}% ^ (superscript)
\TMP@EnsureCode{96}{12}% `
\TMP@EnsureCode{126}{13}% ~ (active)
\edef\KVS@AtEnd{\KVS@AtEnd\noexpand\endinput}
%    \end{macrocode}
%
% \subsection{Package loading}
%
%    \begin{macrocode}
\begingroup\expandafter\expandafter\expandafter\endgroup
\expandafter\ifx\csname RequirePackage\endcsname\relax
  \def\TMP@RequirePackage#1[#2]{%
    \begingroup\expandafter\expandafter\expandafter\endgroup
    \expandafter\ifx\csname ver@#1.sty\endcsname\relax
      \input #1.sty\relax
    \fi
  }%
  \TMP@RequirePackage{infwarerr}[2007/09/09]%
  \let\PackageError\@PackageError
\else
\fi
%    \end{macrocode}
%
%    \begin{macrocode}
\expandafter\ifx\csname toks@\endcsname\relax
  \toksdef\toks@=0 %
\fi
%    \end{macrocode}
%
% \subsection{Check for \eTeX}
%
%    \cs{unexpanded}, \cs{ifcsname}, and \cs{unless} are used if found.
%    \begin{macrocode}
\ifx\numexpr\@undefined
  \catcode`\$=9 % ignore
  \catcode`\&=14 % comment
\else % e-TeX
  \catcode`\$=14 % comment
  \catcode`\&=9  % ignore
\fi
%    \end{macrocode}
%
% \subsection{Generic help macros}
%
%    \begin{macro}{\KVS@Empty}
%    \begin{macrocode}
\def\KVS@Empty{}
%    \end{macrocode}
%    \end{macro}
%    \begin{macro}{\KVS@FirstOfTwo}
%    \begin{macrocode}
\long\def\KVS@FirstOfTwo#1#2{#1}
%    \end{macrocode}
%    \end{macro}
%    \begin{macro}{\KVS@SecondOfTwo}
%    \begin{macrocode}
\long\def\KVS@SecondOfTwo#1#2{#2}
%    \end{macrocode}
%    \end{macro}
%
%    \begin{macro}{\KVS@IfEmpty}
%    \begin{macrocode}
\long\def\KVS@IfEmpty#1{%
& \edef\KVS@Temp{\unexpanded{#1}}%
$ \begingroup
$   \toks@{#1}%
$   \edef\KVS@Temp{\the\toks@}%
$ \expandafter\endgroup
  \ifx\KVS@Temp\KVS@Empty
    \expandafter\KVS@FirstOfTwo
  \else
    \expandafter\KVS@SecondOfTwo
  \fi
}
%    \end{macrocode}
%    \end{macro}
%
% \subsection{Normalizing}
%
%    \begin{macro}{\kv@normalize}
%    \begin{macrocode}
\long\def\kv@normalize#1{%
  \begingroup
    \toks@{,#1,}%
    \KVS@Comma
    \KVS@SpaceComma
    \KVS@CommaSpace
    \KVS@CommaComma
    \KVS@Equals
    \KVS@SpaceEquals
    \KVS@EqualsSpace
    \xdef\KVS@Global{\the\toks@}%
  \endgroup
  \let\kv@list\KVS@Global
}
%    \end{macrocode}
%    \end{macro}
%    \begin{macro}{\comma@normalize}
%    \begin{macrocode}
\def\comma@normalize#1{%
  \begingroup
    \toks@{,#1,}%
    \KVS@Comma
    \KVS@SpaceComma
    \KVS@CommaSpace
    \KVS@CommaComma
    \xdef\KVS@Global{\the\toks@}%
  \endgroup
  \let\comma@list\KVS@Global
}
%    \end{macrocode}
%    \end{macro}
%    \begin{macro}{\KVS@Comma}
%    Converts active commas into comma with catcode other.
%    Also adds a comma at the end to protect the last value
%    for next cleanup steps.
%    \begin{macrocode}
\begingroup
  \lccode`\,=`\,%
  \lccode`\~=`\,%
\lowercase{\endgroup
  \def\KVS@Comma{%
    \toks@\expandafter{\expandafter}\expandafter
    \KVS@@Comma\the\toks@~\KVS@Nil
  }%
  \long\def\KVS@@Comma#1~#2\KVS@Nil{%
    \toks@\expandafter{\the\toks@#1}%
    \KVS@IfEmpty{#2}{%
    }{%
      \KVS@@Comma,#2\KVS@Nil
    }%
  }%
}
%    \end{macrocode}
%    \end{macro}
%    \begin{macro}{\KVS@SpaceComma}
%    Removes spaces before the comma, may add commas at the end.
%    \begin{macrocode}
\def\KVS@SpaceComma#1{%
  \def\KVS@SpaceComma{%
    \expandafter\KVS@@SpaceComma\the\toks@#1,\KVS@Nil
  }%
}
\KVS@SpaceComma{ }
%    \end{macrocode}
%    \end{macro}
%    \begin{macro}{\KVS@@SpaceComma}
%    \begin{macrocode}
\long\def\KVS@@SpaceComma#1 ,#2\KVS@Nil{%
  \KVS@IfEmpty{#2}{%
    \toks@{#1}%
  }{%
    \KVS@@SpaceComma#1,#2\KVS@Nil
  }%
}
%    \end{macrocode}
%    \end{macro}
%    \begin{macro}{\KVS@CommaSpace}
%    Removes spaces after the comma, may add commas at the end.
%    \begin{macrocode}
\def\KVS@CommaSpace{%
  \expandafter\KVS@@CommaSpace\the\toks@, \KVS@Nil
}
%    \end{macrocode}
%    \end{macro}
%    \begin{macro}{\KVS@@CommaSpace}
%    \begin{macrocode}
\long\def\KVS@@CommaSpace#1, #2\KVS@Nil{%
  \KVS@IfEmpty{#2}{%
    \toks@{#1}%
  }{%
    \KVS@@CommaSpace#1,#2\KVS@Nil
  }%
}
%    \end{macrocode}
%    \end{macro}
%    \begin{macro}{\KVS@CommaComma}
%    Replaces multiple commas by one comma.
%    \begin{macrocode}
\def\KVS@CommaComma{%
  \expandafter\KVS@@CommaComma\the\toks@,\KVS@Nil
}
%    \end{macrocode}
%    \end{macro}
%    \begin{macro}{\KVS@@CommaComma}
%    \begin{macrocode}
\long\def\KVS@@CommaComma#1,,#2\KVS@Nil{%
  \KVS@IfEmpty{#2}{%
    \toks@{#1,}% (!)
  }{%
    \KVS@@CommaComma#1,#2\KVS@Nil
  }%
}
%    \end{macrocode}
%    \end{macro}
%
%    \begin{macro}{\KVS@Equals}
%    Converts active equals signs into catcode other characters.
%    \begin{macrocode}
\begingroup
  \lccode`\==`\=%
  \lccode`\~=`\=%
\lowercase{\endgroup
  \def\KVS@Equals{%
    \toks@\expandafter{\expandafter}\expandafter
    \KVS@@Equals\the\toks@~\KVS@Nil
  }%
  \long\def\KVS@@Equals#1~#2\KVS@Nil{%
    \edef\KVS@Temp{\the\toks@}%
    \ifx\KVS@Temp\KVS@Empty
      \expandafter\KVS@FirstOfTwo
    \else
      \expandafter\KVS@SecondOfTwo
    \fi
    {%
      \toks@{#1}%
    }{%
      \toks@\expandafter{\the\toks@=#1}%
    }%
    \KVS@IfEmpty{#2}{%
    }{%
      \KVS@@Equals#2\KVS@Nil
    }%
  }%
}
%    \end{macrocode}
%    \end{macro}
%    \begin{macro}{\KVS@SpaceEquals}
%    Removes spaces before the equals sign.
%    \begin{macrocode}
\def\KVS@SpaceEquals#1{%
  \def\KVS@SpaceEquals{%
    \expandafter\KVS@@SpaceEquals\the\toks@#1=\KVS@Nil
  }%
}
\KVS@SpaceEquals{ }
%    \end{macrocode}
%    \end{macro}
%    \begin{macro}{\KVS@@SpaceEquals}
%    \begin{macrocode}
\long\def\KVS@@SpaceEquals#1 =#2\KVS@Nil{%
  \KVS@IfEmpty{#2}{%
    \toks@{#1}%
  }{%
    \KVS@@SpaceEquals#1=#2\KVS@Nil
  }%
}
%    \end{macrocode}
%    \end{macro}
%    \begin{macro}{\KVS@EqualsSpace}
%    Removes spaces after the equals sign.
%    \begin{macrocode}
\def\KVS@EqualsSpace{%
  \expandafter\KVS@@EqualsSpace\the\toks@= \KVS@Nil
}
%    \end{macrocode}
%    \end{macro}
%    \begin{macro}{\KVS@@EqualsSpace}
%    \begin{macrocode}
\long\def\KVS@@EqualsSpace#1= #2\KVS@Nil{%
  \KVS@IfEmpty{#2}{%
    \toks@{#1}%
  }{%
    \KVS@@EqualsSpace#1=#2\KVS@Nil
  }%
}
%    \end{macrocode}
%    \end{macro}
%
% \subsection{Parsing key value lists}
%
%    \begin{macro}{\kv@parse}
%    Normalizes and parses the key value list. Also sets \cs{kv@list}.
%    \begin{macrocode}
\long\def\kv@parse#1{%
  \kv@normalize{#1}%
  \expandafter\kv@parse@normalized\expandafter{\kv@list}%
}
%    \end{macrocode}
%    \end{macro}
%    \begin{macro}{\kv@parse@normalized}
%    |#1|: key value list\\
%    |#2|: processor
%    \begin{macrocode}
\long\def\kv@parse@normalized#1#2{%
  \KVS@Parse#1,\KVS@Nil{#2}%
}
%    \end{macrocode}
%    \end{macro}
%    \begin{macro}{\KVS@Parse}
%    |#1,#2|: key value list\\
%    |#3|: processor
%    \begin{macrocode}
\long\def\KVS@Parse#1,#2\KVS@Nil#3{%
  \KVS@IfEmpty{#1}{%
  }{%
    \KVS@Process#1=\KVS@Nil{#3}%
  }%
  \KVS@MaybeBreak
  \KVS@IfEmpty{#2}{%
  }{%
    \KVS@Parse#2\KVS@Nil{#3}%
  }%
}
%    \end{macrocode}
%    \end{macro}
%    \begin{macro}{\KVS@Process}
%    |#1|: key\\
%    |#2|: value, |=|\\
%    |#3|: processor
%    \begin{macrocode}
\long\def\KVS@Process#1=#2\KVS@Nil#3{%
  \let\KVS@MaybeBreak\relax
  \def\kv@key{#1}%
  \KVS@IfEmpty{#2}{%
    \let\kv@value\relax
    #3{#1}{}%
  }{%
    \KVS@@Process{#1}#2\KVS@Nil{#3}%
  }%
}
%    \end{macrocode}
%    \end{macro}
%    \begin{macro}{\KVS@@Process}
%    |#1|: key\\
%    |#2|: value\\
%    |#3|: processor
%    \begin{macrocode}
\long\def\KVS@@Process#1#2=\KVS@Nil#3{%
& \edef\kv@value{\unexpanded{#2}}%
$ \begingroup
$   \toks@{#2}%
$   \xdef\KVS@Global{\the\toks@}%
$ \endgroup
$ \let\kv@value\KVS@Global
  #3{#1}{#2}%
}
%    \end{macrocode}
%    \end{macro}
%    \begin{macro}{\KVS@MaybeBreak}
%    \begin{macrocode}
\let\KVS@MaybeBreak\relax
%    \end{macrocode}
%    \end{macro}
%    \begin{macro}{\KVS@break}
%    \begin{macrocode}
\def\KVS@break#1#2#3#4{%
  \let\KVS@MaybeBreak\relax
}
%    \end{macrocode}
%    \end{macro}
%    \begin{macro}{\kv@break}
%    \begin{macrocode}
\def\kv@break{%
  \let\KVS@MaybeBreak\KVS@break
}
%    \end{macrocode}
%    \end{macro}
%
% \subsection{Parsing comma lists}
%
%    \begin{macro}{\comma@parse}
%    Normalizes and parses the key value list. Also sets \cs{comma@list}.
%    \begin{macrocode}
\def\comma@parse#1{%
  \comma@normalize{#1}%
  \expandafter\comma@parse@normalized\expandafter{\comma@list}%
}
%    \end{macrocode}
%    \end{macro}
%
%    \begin{macro}{\comma@parse@normalized}
%    |#1|: comma list\\
%    |#2|: processor
%    \begin{macrocode}
\def\comma@parse@normalized#1#2{%
  \KVS@CommaParse#1,\KVS@Nil{#2}%
}
%    \end{macrocode}
%    \end{macro}
%    \begin{macro}{\KVS@CommaParse}
%    |#1,#2|: comma list\\
%    |#3|: processor
%    \begin{macrocode}
\def\KVS@CommaParse#1,#2\KVS@Nil#3{%
  \KVS@IfEmpty{#1}{%
  }{%
    \def\comma@entry{#1}%
    #3{#1}%
  }%
  \KVS@MaybeBreak
  \KVS@IfEmpty{#2}{%
  }{%
    \KVS@CommaParse#2\KVS@Nil{#3}%
  }%
}
%    \end{macrocode}
%    \end{macro}
%    \begin{macro}{\comma@break}
%    \begin{macrocode}
\def\comma@break{%
  \let\KVS@MaybeBreak\KVS@break
}
%    \end{macrocode}
%    \end{macro}
%
% \subsection{Processing key value pairs}
%
%    \begin{macro}{\kv@handled@false}
%    The handler can call \cs{kv@handled@false} or
%    \cs{kv@handled@true} so report failure or success.
%    The default is success (compatibility for versions
%    before 2011/10/18 v1.15).
%    \begin{macrocode}
\def\kv@handled@false{%
  \let\ifkv@handled@\iffalse
}
%    \end{macrocode}
%    \end{macro}
%    \begin{macro}{\kv@handled@true}
%    \begin{macrocode}
\def\kv@handled@true{%
  \let\ifkv@handled@\iftrue
}
%    \end{macrocode}
%    \end{macro}
%    \begin{macro}{\ifkv@handled@}
%    \begin{macrocode}
\kv@handled@true
%    \end{macrocode}
%    \end{macro}
%
%    \begin{macro}{\kv@processor@default}
%    \begin{macrocode}
\def\kv@processor@default#1#2{%
  \begingroup
    \csname @safe@activestrue\endcsname
    \let\ifincsname\iftrue
    \edef\KVS@temp{\endgroup
      \noexpand\KVS@ProcessorDefault{#1}{#2}%
    }%
  \KVS@temp
}
%    \end{macrocode}
%    \end{macro}
%    \begin{macro}{\KVS@ProcessorDefault}
%    \begin{macrocode}
\long\def\KVS@ProcessorDefault#1#2#3{%
  \def\kv@fam{#1}%
& \unless\ifcsname KV@#1@#2\endcsname
$ \begingroup\expandafter\expandafter\expandafter\endgroup
$ \expandafter\ifx\csname KV@#1@#2\endcsname\relax
&   \unless\ifcsname KVS@#1@handler\endcsname
$   \begingroup\expandafter\expandafter\expandafter\endgroup
$   \expandafter\ifx\csname KVS@#1@handler\endcsname\relax
      \kv@error@unknownkey{#1}{#2}%
    \else
      \kv@handled@true
      \csname KVS@#1@handler\endcsname{#2}{#3}%
      \relax
      \ifkv@handled@
      \else
        \kv@error@unknownkey{#1}{#2}%
      \fi
    \fi
  \else
    \ifx\kv@value\relax
&     \unless\ifcsname KV@#1@#2@default\endcsname
$     \begingroup\expandafter\expandafter\expandafter\endgroup
$     \expandafter\ifx\csname KV@#1@#2@default\endcsname\relax
        \kv@error@novalue{#1}{#2}%
      \else
        \csname KV@#1@#2@default\endcsname
        \relax
      \fi
    \else
      \csname KV@#1@#2\endcsname{#3}%
    \fi
  \fi
}
%    \end{macrocode}
%    \end{macro}
%
%    \begin{macro}{\kv@processor@known}
%    \begin{macrocode}
\def\kv@processor@known#1#2#3{%
  \begingroup
    \csname @safe@activestrue\endcsname
    \let\ifincsname\iftrue
    \edef\KVS@temp{\endgroup
      \noexpand\KVS@ProcessorKnown{#1}\noexpand#2{#3}%
    }%
  \KVS@temp
}
%    \end{macrocode}
%    \end{macro}
%    \begin{macro}{\KVS@ProcessorKnown}
%    \begin{macrocode}
\long\def\KVS@ProcessorKnown#1#2#3#4{%
  \def\kv@fam{#1}%
& \unless\ifcsname KV@#1@#3\endcsname
$ \begingroup\expandafter\expandafter\expandafter\endgroup
$ \expandafter\ifx\csname KV@#1@#3\endcsname\relax
&   \unless\ifcsname KVS@#1@handler\endcsname
$   \begingroup\expandafter\expandafter\expandafter\endgroup
$   \expandafter\ifx\csname KVS@#1@handler\endcsname\relax
      \KVS@AddUnhandled#2{#3}{#4}%
    \else
      \kv@handled@true
      \csname KVS@#1@handler\endcsname{#3}{#4}%
      \relax
      \ifkv@handled@
      \else
        \KVS@AddUnhandled#2{#3}{#4}%
      \fi
    \fi
  \else
    \ifx\kv@value\relax
&     \unless\ifcsname KV@#1@#2@default\endcsname
$     \begingroup\expandafter\expandafter\expandafter\endgroup
$     \expandafter\ifx\csname KV@#1@#3@default\endcsname\relax
        \kv@error@novalue{#1}{#3}%
      \else
        \csname KV@#1@#3@default\endcsname
        \relax
      \fi
    \else
      \csname KV@#1@#3\endcsname{#4}%
    \fi
  \fi
}
%    \end{macrocode}
%    \end{macro}
%    \begin{macro}{\KVS@AddUnhandled}
%    \begin{macrocode}
\long\def\KVS@AddUnhandled#1#2#3{%
& \edef#1{%
&   \ifx#1\KVS@empty
&   \else
&     \unexpanded{#1},%
&   \fi
&   \unexpanded{{#2}={#3}}%
& }%
$ \begingroup
$   \ifx#1\KVS@empty
$     \toks@{{#2}={#3}}%
$   \else
$     \toks@\expandafter{#1,{#2}={#3}}%
$   \fi
$   \xdef\KVS@Global{\the\toks@}%
$ \endgroup
$ \let#1\KVS@Global
}
%    \end{macrocode}
%    \end{macro}
%
%    \begin{macro}{\kv@set@family@handler}
%    \begin{macrocode}
\long\def\kv@set@family@handler#1#2{%
  \begingroup
    \csname @safe@activestrue\endcsname
    \let\ifincsname\iftrue
  \expandafter\endgroup
  \expandafter\def\csname KVS@#1@handler\endcsname##1##2{#2}%
}
%    \end{macrocode}
%    \end{macro}
%
%    \begin{macro}{\kv@unset@family@handler}
%    \begin{macrocode}
\long\def\kv@unset@family@handler#1#2{%
  \begingroup
    \csname @safe@activestrue\endcsname
    \let\ifincsname\iftrue
  \expandafter\endgroup
  \expandafter\let\csname KVS@#1@handler\endcsname\@UnDeFiNeD
}
%    \end{macrocode}
%    \end{macro}
%
% \subsection{Error handling}
%
%    \begin{macro}{\kv@error@novalue}
%    \begin{macrocode}
\def\kv@error@novalue{%
  \kv@error@generic{No value specified for}%
}
%    \end{macrocode}
%    \end{macro}
%    \begin{macro}{\kv@error@unknownkey}
%    \begin{macrocode}
\def\kv@error@unknownkey{%
  \kv@error@generic{Undefined}%
}
%    \end{macrocode}
%    \end{macro}
%    \begin{macro}{\kv@error@generic}
%    \begin{macrocode}
\def\kv@error@generic#1#2#3{%
  \PackageError{kvsetkeys}{%
    #1 key `#3'%
  }{%
    The keyval family of the key `#3' is `#2'.\MessageBreak
    The setting of the key is ignored because of the error.\MessageBreak
    \MessageBreak
    \@ehc
  }%
}
%    \end{macrocode}
%    \end{macro}
%
% \subsection{Do it all}
%
%    \begin{macro}{\kvsetkeys}
%    \begin{macrocode}
\long\def\kvsetkeys#1#2{%
  \kv@parse{#2}{\kv@processor@default{#1}}%
}
%    \end{macrocode}
%    \end{macro}
%
%    \begin{macro}{\kvsetkeys@expandafter}
%    \begin{macrocode}
\def\kvsetkeys@expandafter#1#2{%
  \expandafter\kv@parse\expandafter{#2}{%
    \kv@processor@default{#1}%
  }%
}
%    \end{macrocode}
%    \end{macro}
%
%    \begin{macro}{\KVS@cmd}
%    \begin{macrocode}
\def\KVS@cmd{0}%
%    \end{macrocode}
%    \end{macro}
%    \begin{macro}{\KVS@cmd@inc}
%    \begin{macrocode}
\def\KVS@cmd@inc{%
& \edef\KVS@cmd{\the\numexpr\KVS@cmd+1}%
$ \begingroup
$   \count255=\KVS@cmd\relax
$   \advance\count255 by 1\relax
$ \edef\x{\endgroup
$   \noexpand\def\noexpand\KVS@cmd{\number\count255}%
$ }%
$ \x
}
%    \end{macrocode}
%    \end{macro}
%    \begin{macro}{\KVS@cmd@dec}
%    \begin{macrocode}
\def\KVS@cmd@dec{%
& \edef\KVS@cmd{\the\numexpr\KVS@cmd-1}%
$ \begingroup
$   \count255=\KVS@cmd\relax
$   \advance\count255 by -1\relax
$ \edef\x{\endgroup
$   \noexpand\def\noexpand\KVS@cmd{\number\count255}%
$ }%
$ \x
}
%    \end{macrocode}
%    \end{macro}
%    \begin{macro}{\KVS@empty}
%    \begin{macrocode}
\def\KVS@empty{}
%    \end{macrocode}
%    \end{macro}
%
%    \begin{macro}{\kvsetknownkeys}
%    \begin{macrocode}
\def\kvsetknownkeys{%
  \expandafter
  \KVS@setknownkeys\csname KVS@cmd\KVS@cmd\endcsname{}%
}
%    \end{macrocode}
%    \end{macro}
%    \begin{macro}{\KVS@setknownkeys}
%    \begin{macrocode}
\long\def\KVS@setknownkeys#1#2#3#4#5{%
  \let#1\KVS@empty
  \KVS@cmd@inc
  #2\kv@parse#2{#5}{\kv@processor@known{#3}#1}%
  \KVS@cmd@dec
  \let#4=#1%
}
%    \end{macrocode}
%    \end{macro}
%
%    \begin{macro}{\kvsetknownkeys@expandafter}
%    \begin{macrocode}
\def\kvsetknownkeys@expandafter{%
  \expandafter
  \KVS@setknownkeys
      \csname KVS@cmd\KVS@cmd\endcsname\expandafter
}
%    \end{macrocode}
%    \end{macro}
%
%    \begin{macrocode}
\KVS@AtEnd%
%</package>
%    \end{macrocode}
%% \section{Installation}
%
% \subsection{Download}
%
% \paragraph{Package.} This package is available on
% CTAN\footnote{\CTANpkg{kvsetkeys}}:
% \begin{description}
% \item[\CTAN{macros/latex/contrib/kvsetkeys/kvsetkeys.dtx}] The source file.
% \item[\CTAN{macros/latex/contrib/kvsetkeys/kvsetkeys.pdf}] Documentation.
% \end{description}
%
%
% \paragraph{Bundle.} All the packages of the bundle `kvsetkeys'
% are also available in a TDS compliant ZIP archive. There
% the packages are already unpacked and the documentation files
% are generated. The files and directories obey the TDS standard.
% \begin{description}
% \item[\CTANinstall{install/macros/latex/contrib/kvsetkeys.tds.zip}]
% \end{description}
% \emph{TDS} refers to the standard ``A Directory Structure
% for \TeX\ Files'' (\CTANpkg{tds}). Directories
% with \xfile{texmf} in their name are usually organized this way.
%
% \subsection{Bundle installation}
%
% \paragraph{Unpacking.} Unpack the \xfile{kvsetkeys.tds.zip} in the
% TDS tree (also known as \xfile{texmf} tree) of your choice.
% Example (linux):
% \begin{quote}
%   |unzip kvsetkeys.tds.zip -d ~/texmf|
% \end{quote}
%
% \subsection{Package installation}
%
% \paragraph{Unpacking.} The \xfile{.dtx} file is a self-extracting
% \docstrip\ archive. The files are extracted by running the
% \xfile{.dtx} through \plainTeX:
% \begin{quote}
%   \verb|tex kvsetkeys.dtx|
% \end{quote}
%
% \paragraph{TDS.} Now the different files must be moved into
% the different directories in your installation TDS tree
% (also known as \xfile{texmf} tree):
% \begin{quote}
% \def\t{^^A
% \begin{tabular}{@{}>{\ttfamily}l@{ $\rightarrow$ }>{\ttfamily}l@{}}
%   kvsetkeys.sty & tex/generic/kvsetkeys/kvsetkeys.sty\\
%   kvsetkeys.pdf & doc/latex/kvsetkeys/kvsetkeys.pdf\\
%   kvsetkeys-example.tex & doc/latex/kvsetkeys/kvsetkeys-example.tex\\
%   kvsetkeys.dtx & source/latex/kvsetkeys/kvsetkeys.dtx\\
% \end{tabular}^^A
% }^^A
% \sbox0{\t}^^A
% \ifdim\wd0>\linewidth
%   \begingroup
%     \advance\linewidth by\leftmargin
%     \advance\linewidth by\rightmargin
%   \edef\x{\endgroup
%     \def\noexpand\lw{\the\linewidth}^^A
%   }\x
%   \def\lwbox{^^A
%     \leavevmode
%     \hbox to \linewidth{^^A
%       \kern-\leftmargin\relax
%       \hss
%       \usebox0
%       \hss
%       \kern-\rightmargin\relax
%     }^^A
%   }^^A
%   \ifdim\wd0>\lw
%     \sbox0{\small\t}^^A
%     \ifdim\wd0>\linewidth
%       \ifdim\wd0>\lw
%         \sbox0{\footnotesize\t}^^A
%         \ifdim\wd0>\linewidth
%           \ifdim\wd0>\lw
%             \sbox0{\scriptsize\t}^^A
%             \ifdim\wd0>\linewidth
%               \ifdim\wd0>\lw
%                 \sbox0{\tiny\t}^^A
%                 \ifdim\wd0>\linewidth
%                   \lwbox
%                 \else
%                   \usebox0
%                 \fi
%               \else
%                 \lwbox
%               \fi
%             \else
%               \usebox0
%             \fi
%           \else
%             \lwbox
%           \fi
%         \else
%           \usebox0
%         \fi
%       \else
%         \lwbox
%       \fi
%     \else
%       \usebox0
%     \fi
%   \else
%     \lwbox
%   \fi
% \else
%   \usebox0
% \fi
% \end{quote}
% If you have a \xfile{docstrip.cfg} that configures and enables \docstrip's
% TDS installing feature, then some files can already be in the right
% place, see the documentation of \docstrip.
%
% \subsection{Refresh file name databases}
%
% If your \TeX~distribution
% (\TeX\,Live, \mikTeX, \dots) relies on file name databases, you must refresh
% these. For example, \TeX\,Live\ users run \verb|texhash| or
% \verb|mktexlsr|.
%
% \subsection{Some details for the interested}
%
% \paragraph{Unpacking with \LaTeX.}
% The \xfile{.dtx} chooses its action depending on the format:
% \begin{description}
% \item[\plainTeX:] Run \docstrip\ and extract the files.
% \item[\LaTeX:] Generate the documentation.
% \end{description}
% If you insist on using \LaTeX\ for \docstrip\ (really,
% \docstrip\ does not need \LaTeX), then inform the autodetect routine
% about your intention:
% \begin{quote}
%   \verb|latex \let\install=y\input{kvsetkeys.dtx}|
% \end{quote}
% Do not forget to quote the argument according to the demands
% of your shell.
%
% \paragraph{Generating the documentation.}
% You can use both the \xfile{.dtx} or the \xfile{.drv} to generate
% the documentation. The process can be configured by the
% configuration file \xfile{ltxdoc.cfg}. For instance, put this
% line into this file, if you want to have A4 as paper format:
% \begin{quote}
%   \verb|\PassOptionsToClass{a4paper}{article}|
% \end{quote}
% An example follows how to generate the
% documentation with pdf\LaTeX:
% \begin{quote}
%\begin{verbatim}
%pdflatex kvsetkeys.dtx
%makeindex -s gind.ist kvsetkeys.idx
%pdflatex kvsetkeys.dtx
%makeindex -s gind.ist kvsetkeys.idx
%pdflatex kvsetkeys.dtx
%\end{verbatim}
% \end{quote}
%
% \begin{thebibliography}{9}
% \bibitem{tb94wright}
%   A guide to key-value methods, Joseph Wright, second draft for
%   \href{https://www.tug.org/tugboat}{TUGBoat}, 2009-03-17.
%   \url{https://www.texdev.net/uploads/2009/03/keyval.pdf}
%
% \bibitem{keyval}
%   David Carlisle:
%   \textit{The \xpackage{keyval} package};
%   1999/03/16 v1.13;
%   \CTANpkg{keyval}.
%
% \end{thebibliography}
%
% \begin{History}
%   \begin{Version}{2006/03/06 v1.0}
%   \item
%     First version.
%   \end{Version}
%   \begin{Version}{2006/10/19 v1.1}
%   \item
%     Fix of \cs{kv@set@family@handler}.
%   \item
%     Example added.
%   \end{Version}
%   \begin{Version}{2007/09/09 v1.2}
%   \item
%     Using package \xpackage{infwarerr} for error messages.
%   \item
%     Catcode section rewritten.
%   \end{Version}
%   \begin{Version}{2007/09/29 v1.3}
%   \item
%     Normalizing and parsing of comma separated lists added.
%   \item
%     \cs{kv@normalize} rewritten.
%   \item
%     Robustness increased for normalizing and parsing,
%     e.g. for values with unmatched conditionals.
%   \item
%     \eTeX\ is used if available.
%   \item
%     Tests added for normalizing and parsing.
%   \end{Version}
%   \begin{Version}{2009/07/19 v1.4}
%   \item
%     Bug fix for \cs{kv@normalize}: unwanted space removed (Florent Chervet).
%   \end{Version}
%   \begin{Version}{2009/07/30 v1.5}
%   \item
%     Documentation addition: recommendation for Joseph Wright's
%     review article.
%   \end{Version}
%   \begin{Version}{2009/12/12 v1.6}
%   \item
%     Short info shortened.
%   \end{Version}
%   \begin{Version}{2009/12/22 v1.7}
%   \item
%     Internal optimization (\cs{KVS@CommaSpace}, \dots, \cs{KVS@EqualsSpace}).
%   \end{Version}
%   \begin{Version}{2010/01/28 v1.8}
%   \item
%     Compatibility to ini\TeX\ added.
%   \end{Version}
%   \begin{Version}{2010/03/01 v1.9}
%   \item
%     Support of \cs{par} inside values.
%   \end{Version}
%   \begin{Version}{2011/01/30 v1.10}
%   \item
%     Already loaded package files are not input in \hologo{plainTeX}.
%   \end{Version}
%   \begin{Version}{2011/03/03 v1.11}
%   \item
%     \cs{kv@break} and \cs{comma@break} added.
%   \end{Version}
%   \begin{Version}{2011/04/05 v1.12}
%   \item
%     Error message with recovery action in help message
%     (request by GL).
%   \end{Version}
%   \begin{Version}{2011/04/07 v1.13}
%   \item
%     \cs{kv@processor@default} supports package \xpackage{babel}'s
%     shorthands.
%   \item
%     \cs{kv@set@family@handler} with shorthand support.
%   \end{Version}
%   \begin{Version}{2011/06/15 v1.14}
%   \item
%     Some optimizations in token register uses (GL, HO).
%   \end{Version}
%   \begin{Version}{2011/10/18 v1.15}
%   \item
%     \cs{kv@processor@known} and \cs{kvsetknownkeys} added.
%   \item
%     \cs{kvsetkeys@expandafter} and \cs{kvsetknownkeys@expandafter} added.
%   \item
%     Family handler can report success or failure by \cs{kv@handled@true}
%     or \cs{kv@handled@false}.
%   \item
%     \cs{kv@unset@family@handler} added.
%   \end{Version}
%   \begin{Version}{2012/04/25 v1.16}
%   \item
%     \cs{kv@processor@default} and \cs{kv@processor@known} define
%     macro \cs{kv@fam} for convenience.
%   \item
%     Catcode section: Catcode setting for \texttt{+} added for \hologo{eTeX}.
%   \end{Version}
%   \begin{Version}{2016/05/16 v1.17}
%   \item
%     Documentation updates.
%   \end{Version}
%   \begin{Version}{2019/12/15 v1.18}
%   \item
%     Documentation updates.
%   \item
%     Avoid \xpackage{etexcmds} and \xpackage{infwarwerr} in \LaTeX.
%   \end{Version}
% \end{History}
%
% \PrintIndex
%
% \Finale
\endinput

%        (quote the arguments according to the demands of your shell)
%
% Documentation:
%    (a) If kvsetkeys.drv is present:
%           latex kvsetkeys.drv
%    (b) Without kvsetkeys.drv:
%           latex kvsetkeys.dtx; ...
%    The class ltxdoc loads the configuration file ltxdoc.cfg
%    if available. Here you can specify further options, e.g.
%    use A4 as paper format:
%       \PassOptionsToClass{a4paper}{article}
%
%    Programm calls to get the documentation (example):
%       pdflatex kvsetkeys.dtx
%       makeindex -s gind.ist kvsetkeys.idx
%       pdflatex kvsetkeys.dtx
%       makeindex -s gind.ist kvsetkeys.idx
%       pdflatex kvsetkeys.dtx
%
% Installation:
%    TDS:tex/generic/kvsetkeys/kvsetkeys.sty
%    TDS:doc/latex/kvsetkeys/kvsetkeys.pdf
%    TDS:doc/latex/kvsetkeys/kvsetkeys-example.tex
%    TDS:source/latex/kvsetkeys/kvsetkeys.dtx
%
%<*ignore>
\begingroup
  \catcode123=1 %
  \catcode125=2 %
  \def\x{LaTeX2e}%
\expandafter\endgroup
\ifcase 0\ifx\install y1\fi\expandafter
         \ifx\csname processbatchFile\endcsname\relax\else1\fi
         \ifx\fmtname\x\else 1\fi\relax
\else\csname fi\endcsname
%</ignore>
%<*install>
\input docstrip.tex
\Msg{************************************************************************}
\Msg{* Installation}
\Msg{* Package: kvsetkeys 2019/12/15 v1.18 Key value parser (HO)}
\Msg{************************************************************************}

\keepsilent
\askforoverwritefalse

\let\MetaPrefix\relax
\preamble

This is a generated file.

Project: kvsetkeys
Version: 2019/12/15 v1.18

Copyright (C)
   2006, 2007, 2009-2012 Heiko Oberdiek
   2016-2019 Oberdiek Package Support Group

This work may be distributed and/or modified under the
conditions of the LaTeX Project Public License, either
version 1.3c of this license or (at your option) any later
version. This version of this license is in
   https://www.latex-project.org/lppl/lppl-1-3c.txt
and the latest version of this license is in
   https://www.latex-project.org/lppl.txt
and version 1.3 or later is part of all distributions of
LaTeX version 2005/12/01 or later.

This work has the LPPL maintenance status "maintained".

The Current Maintainers of this work are
Heiko Oberdiek and the Oberdiek Package Support Group
https://github.com/ho-tex/kvsetkeys/issues


The Base Interpreter refers to any `TeX-Format',
because some files are installed in TDS:tex/generic//.

This work consists of the main source file kvsetkeys.dtx
and the derived files
   kvsetkeys.sty, kvsetkeys.pdf, kvsetkeys.ins, kvsetkeys.drv,
   kvsetkeys-example.tex, kvsetkeys-test1.tex,
   kvsetkeys-test2.tex, kvsetkeys-test3.tex,
   kvsetkeys-test4.tex.

\endpreamble
\let\MetaPrefix\DoubleperCent

\generate{%
  \file{kvsetkeys.ins}{\from{kvsetkeys.dtx}{install}}%
  \file{kvsetkeys.drv}{\from{kvsetkeys.dtx}{driver}}%
  \usedir{tex/generic/kvsetkeys}%
  \file{kvsetkeys.sty}{\from{kvsetkeys.dtx}{package}}%
  \usedir{doc/latex/kvsetkeys}%
  \file{kvsetkeys-example.tex}{\from{kvsetkeys.dtx}{example}}%
}

\catcode32=13\relax% active space
\let =\space%
\Msg{************************************************************************}
\Msg{*}
\Msg{* To finish the installation you have to move the following}
\Msg{* file into a directory searched by TeX:}
\Msg{*}
\Msg{*     kvsetkeys.sty}
\Msg{*}
\Msg{* To produce the documentation run the file `kvsetkeys.drv'}
\Msg{* through LaTeX.}
\Msg{*}
\Msg{* Happy TeXing!}
\Msg{*}
\Msg{************************************************************************}

\endbatchfile
%</install>
%<*ignore>
\fi
%</ignore>
%<*driver>
\NeedsTeXFormat{LaTeX2e}
\ProvidesFile{kvsetkeys.drv}%
  [2019/12/15 v1.18 Key value parser (HO)]%
\documentclass{ltxdoc}
\usepackage{holtxdoc}[2011/11/22]
\begin{document}
  \DocInput{kvsetkeys.dtx}%
\end{document}
%</driver>
% \fi
%
%
%
% \GetFileInfo{kvsetkeys.drv}
%
% \title{The \xpackage{kvsetkeys} package}
% \date{2019/12/15 v1.18}
% \author{Heiko Oberdiek\thanks
% {Please report any issues at \url{https://github.com/ho-tex/kvsetkeys/issues}}}
%
% \maketitle
%
% \begin{abstract}
% Package \xpackage{kvsetkeys} provides \cs{kvsetkeys}, a variant
% of package \xpackage{keyval}'s \cs{setkeys}. It allows to specify
% a handler that deals with unknown options. Active commas and equal
% signs may be used (e.g. see \xpackage{babel}'s shorthands) and
% only one level of curly braces is removed from the values.
% \end{abstract}
%
% \tableofcontents
%
% \def\M#1{\texttt{\{}\meta{#1}\texttt{\}}}
%
% \section{Documentation}
%
% First I want to recommend the very good review article
% ``A guide to key-value methods'' by Joseph Wright \cite{tb94wright}.
% It introduces the different key-value packages and compares them.
%
% \subsection{Motivation}
%
% \cs{kvsetkeys} serves as replacement for \xpackage{keyval}'s
% \cs{setkeys}. It basically uses the same syntax. But the
% implementation is more robust and predictable:
% \begin{description}
% \item[Active syntax characters:]
%   Comma `|,|' and the equals sign `|=|' are used inside
%   key value lists as syntax characters. Package \xpackage{keyval}
%   uses the catcode of the characters that is active during
%   package loading, usually this is catcode 12 (other).
%   But it can happen that the catcode setting of the syntax characters
%   changes. Especially active characters are of interest, because
%   some language adaptations uses them. For example, option \xoption{turkish}
%   of package \xpackage{babel} uses the equals sign as active shorthand
%   character. Therefore package \xpackage{kvsetkeys} deals with
%   both catcode settings 12 (other) and 13 (active).
% \item[Brace removal:]
%   Package \xpackage{keyval}'s \cs{setkeys} removes up to two
%   levels of curly braces around the value in some unpredictable way:
%\begin{quote}
%|\setkeys{fam}{key={{value}}} || || | $\rightarrow$ |value|\\
%|\setkeys{fam}{key={{{value}}}} | $\rightarrow$ |{value}|\\
%|\setkeys{fam}{key= {{{value}}}}| $\rightarrow$ |{{value}}|
%\end{quote}
%   This package \xpackage{kvsetkeys} follows a much stronger rule:
%   Exactly one level of braces are removed from an item, if the
%   item is surrounded by curly braces. An item can be a
%   the key value pair, the key or the value.
%\begin{quote}
%|\kvsetkeys{fam}{key={value}} || | $\rightarrow$ |value|\\
%|\kvsetkeys{fam}{key={{value}} | $\rightarrow$ |{value}|\\
%|\kvsetkeys{fam}{key= {{value}}| $\rightarrow$ |{value}|
%\end{quote}
% \item[Arbitrary values:] Unmatched conditionals are supported.
% \end{description}
%
% Before I describe
% \cs{kvsetkeys} in more detail, first I want to explain, how
% this package deals with key value lists. For the package also
% provides low level interfaces that can be used by package authors.
%
% \subsection{Normalizing key value lists}
%
% \begin{declcs}{kv@normalize}\,\M{key value list}
% \end{declcs}
% If the user specifies key value lists, he usually prefers
% nice formatted source code, e.g.:
% \begin{quote}
%\begin{verbatim}
%\hypersetup{
%  pdftitle    = {...},
%  pdfsubject  = {...},
%  pdfauthor   = {...},
%  pdfkeywords = {...},
%  ...
%}
%\end{verbatim}
% \end{quote}
% Thus there can be spaces around keys, around |=| or around the value.
% Also empty entries are possible by too many commas. Therefore these
% spaces and empty entries are silently removed by package \xpackage{keyval}
% and this package. Whereas the contents of the value can be protected
% by curly braces, especially if spaces or commas are used inside,
% a key name must not use spaces or other syntax characters.
%
% \cs{kv@normalize} takes a key value list and performs the cleanup:
% \begin{itemize}
% \item Spaces are removed.
% \item Syntax characters (comma and equal sign) that are active
%   are replaced by the same characters with standard catcode.
%   (Example: \xpackage{babel}'s language option \xoption{turkish}
%   uses the equal sign as active shorthand character.)
% \end{itemize}
% The result is stored in \cs{kv@list}, e.g.:
% \begin{quote}
% |\kv@list| $\rightarrow$ |,pdftitle={...},pdfsubject={...},...,|
% \end{quote}
% Curly braces around values (or keys) remain untouched.
% \begin{description}
% \item[v1.3+:]
% One comma is added in front of the list and each pair ends with
% a comma. Thus an empty list consists of one comma, otherwise
% two commas encloses the list. Empty entries other than the first
% are removed.
% \item[v1.0 -- v1.2:]
% Empty entries are removed later. In fact it adds a comma at the begin
% and end to protect the last value and an easier implementation.
% \end{description}
%
% \subsection{Parsing key value lists}
%
% \begin{declcs}{kv@parse}\,\M{key value list}\,\M{processor}
% \end{declcs}
% It is easier to parse a normalized list, thus \cs{kv@parse}
% normalizes the list and calls \cs{kv@parse@normalized}.
%
% \begin{declcs}{kv@parse@normalized}\,\M{key value list}%
%   \,\M{processor}
% \end{declcs}
% Now the key value list is split into single key value pairs.
% For further processing the key and value are given as arguments
% for the \meta{processor}:
% \begin{quote}
%   \meta{processor}\,\M{key}\,\M{value}
% \end{quote}
% Also key and value are stored in macro names:
% \begin{itemize}
% \item \cs{kv@key} stores the key.
% \item \cs{kv@value} stores the value or if the value was not
%   specified it has the meaning \cs{relax}.
% \end{itemize}
% The behaviour in pseudo code:
% \begin{quote}
%   foreach (\meta{key}, \meta{value}) in (\meta{key value list})\\
%   \hspace*{1.5em}\cs{kv@key} := \meta{key}\\
%   \hspace*{1.5em}\cs{kv@value} := \meta{value}\\
%   \hspace*{1.5em}\meta{processor}\,\M{key}\,\M{value}
% \end{quote}
%
% \begin{declcs}{kv@break}
% \end{declcs}
% Since version 2011/03/03 v1.11 \cs{kv@break} can be
% called inside the \meta{processor}
% of \cs{kv@parse} or \cs{kv@parse@normalized}, then
% the processing is stopped and the following entries discarded.
%
% \subsection{Processing key value pairs}
%
% Key value pairs can be processed in many different ways.
% For example, the processor for \cs{kvsetkeys} works similar
% to \cs{setkeys} of package \xpackage{keyval}. There unknown
% keys raise an error.
%
% Package \xpackage{xkeyval} also knows a star form of \cs{setkeys}
% that stores unknown keys in an internal macro for further processing
% with \cs{setrmkeys} and similar macros. This feature is covered
% by processor \cs{kv@processor@known}.
%
% \subsubsection{Processing similar to \xpackage{keyval}}
%
% \begin{declcs}{kv@processor@default}\,\M{family}\,\M{key}\,\M{value}
% \end{declcs}
% There are many possiblities to process key value pairs.
% \cs{kv@processor@default} is the processor used in \cs{kvsetkeys}.
% It reimplements and extends the behaviour of
% \xpackage{keyval}'s \cs{setkeys}.
% In case of unknown keys \cs{setkeys} raise an error.
% This processer, however, calls a handler instead, if it
% is provided by the family. Both \meta{family} and \meta{key}
% may contain package \xpackage{babel}'s shorthands
% (since 2011/04/07 v1.13).
%
% Since 2011/10/18 v1.15 the family handler can reject the
% successful handling of a key by calling \cs{kv@handled@false}.
%
% Since 2019/12/15 v1.18 \cs{kv@processor@default} also defines
% macro \cs{kv@fam} with meaning \meta{family} for convenience.
%
% \subsubsection{Processing similar to \cs{setkeys*} of package \xpackage{xkeyval}}
%
% \begin{declcs}{kv@processor@known}\,\M{family}\,^^A
%   \M{cmd}\,\M{key}\,\M{value}
% \end{declcs}
% The key value processor \cs{kv@processor@known} behaves similar
% to \cs{kv@processor@default}. If the \meta{key} exists in the
% \meta{family} its code is called, otherwise the family handler
% is tried. If the family handler is not set or cannot handle the
% key, the unknown key value pair is added to the macro \meta{cmd}.
% Since 2011/10/18 v1.15.
%
% The behaviour in pseudo code:
% \begin{quote}
%   if \meta{key} exists\\
%   \hspace*{1.5em}call the keyval code of \meta{key}\\
%   else\\
%   \hspace*{1.5em}if \meta{handler} for \meta{family} exists\\
%   \hspace*{3em}handled = true\\
%   \hspace*{3em}\meta{handler}\,\M{key}\,\M{value}\\
%   \hspace*{3em}if handled\\
%   \hspace*{3em}else\\
%   \hspace*{4.5em}add \texttt{"}\M{key}\texttt{=}\M{value}\texttt{"}
%     to \M{cmd}\\
%   \hspace*{3em}fi\\
%   \hspace*{1.5em}else\\
%   \hspace*{3em}add \texttt{"}\M{key}\texttt{=}\M{value}\texttt{"}
%     to \M{cmd}\\
%   \hspace*{3em}raise unknown key error\\
%   \hspace*{1.5em}fi\\
%   fi
% \end{quote}
%
% Since 2019/12/15 v1.18 \cs{kv@processor@known} also defines
% macro \cs{kv@fam} with meaning \meta{family} for convenience.
%
% \subsection{Default family handler}
%
% \cs{kv@processor@default} calls \meta{handler}, the default
% handler for the family, if the key does not exist in the family.
% The handler is called with two arguments, the key and the value.
% It can be defined with \cs{kv@set@family@hander}:
%
% \begin{declcs}{kv@set@family@handler}\,\M{family}\,\M{handler definition}
% \end{declcs}
% This sets the default family handler for the keyval family
% \meta{family}. Inside \meta{handler definition} |#1| stands for
% the key and |#2| is the value. Also \cs{kv@key} and \cs{kv@value}
% can be used for the key and the value. If the value is not given,
% \cs{kv@value} has the meaning \cs{relax}.
%
% \begin{declcs}{kv@unset@family@handler}\,\M{family}
% \end{declcs}
% It removes the family handler for \meta{family}.
% Since 2011/10/18 v1.15.
%
% \subsection{Put it all together}
%
% \begin{declcs}{kvsetkeys}\,\M{family}\,\M{key value list}
% \end{declcs}
% Macro \cs{kvsetkeys} processes the \meta{key value list} with
% the standard processor \cs{kv@processor@default}:
% \begin{quote}
%   \cs{kv@parse}\,\M{key value list}^^A
%   \texttt{\{}\cs{kv@processor@default}\,\M{family}\texttt{\}}
% \end{quote}
%
% \begin{declcs}{kvsetknownkeys}\,\M{family}\,\M{cmd}\,\M{key value list}
% \end{declcs}
% Macro \cs{kvsetknownkeys} processes the \meta{key value list}
% with processor \cs{kv@processor@known}. All key value pairs
% with keys that are not known in \meta{family} are stored
% in macro \meta{cmd}. A previous contents of macro \meta{cmd}
% will be overwritten. If all keys can be handled, \meta{cmd}
% will be empty, otherwise it contains a key value list of
% unhandled key value pairs.
% Since 2011/10/18 v1.15.
%
% Pseudo code:
% \begin{quote}
%   create macro \meta{cmdaux} with unique name (inside the current group)\\
%   \cs{def}\meta{cmdaux}\{\}\\
%   \cs{kv@parse}\,\M{key value list}^^A
%   \texttt{\{}\cs{kv@processor@known}\,\M{family}\,\M{cmdaux}\texttt{\}}\\
%   \cs{let}\meta{cmd}=\meta{cmdaux}
% \end{quote}
%
% \begin{declcs}{kvsetkeys@expandafter}\,\M{family}\,\M{list cmd}\\
%   \cs{kvsetknownkeys@expandafter}\,\M{family}\,%
%   \M{cmd}\,\M{list cmd}
% \end{declcs}
% Both macros behave like the counterparts without suffix
% |@expandafter|. The difference is that the key value list is
% given as macro that is expanded once.
% Since 2011/10/18 v1.15.
%
% Thus you can replace \cs{setkeys} of package \xpackage{keyval}
% by the key value parser of this package:
% \begin{quote}
%   |\renewcommand*{\setkeys}{\kvsetkeys}|\\
%   or\\
%   |\let\setkeys\kvsetkeys|
% \end{quote}
%
% \subsection{Comma separated lists}
%
% Since version 2007/09/29 v1.3 this package also supports the normalizing
% and parsing of general comma separated lists.
%
% \begin{declcs}{comma@normalize}\,\M{comma list}
% \end{declcs}
% Macro \cs{comma@normalize} normalizes the comma separated list,
% removes spaces around commas. The result is put in macro \cs{comma@list}.
%
% \begin{declcs}{comma@parse}\,\M{comma list}\,\M{processor}
% \end{declcs}
% Macro \cs{comma@parse} first normalizes the comma separated list
% and then parses the list by calling \cs{comma@parse@normalized}.
%
% \begin{declcs}{comma@parse@normalized}\,\M{normalized comma list}^^A
%   \,\M{processor}
% \end{declcs}
% The list is parsed. Empty entries are ignored. \meta{processor}
% is called for each non-empty entry with the entry as argument:
% \begin{quote}
%   \meta{processor}|{|\meta{entry}|}|
% \end{quote}
% Also the entry is stored in the macro \cs{comma@entry}.
%
% \begin{declcs}{comma@break}
% \end{declcs}
% Since version 2011/03/03 v1.11 \cs{comma@break} can be
% called inside the \meta{processor}
% of \cs{comma@parse} or \cs{comma@parse@normalized}, then
% the processing is stopped and the following entries discarded.
%
% \section{Example}
%
% The following example prints a short piece of HTML code using
% the tabbing environment for indenting purpose and a key value
% syntax for specifying the attributes of an HTML tag.
% The example illustrates the use of a default family handler.
%    \begin{macrocode}
%<*example>
\documentclass{article}
\usepackage[T1]{fontenc}
\usepackage{kvsetkeys}
\usepackage{keyval}

\makeatletter
\newcommand*{\tag}[2][]{%
  % #1: attributes
  % #2: tag name
  \begingroup
    \toks@={}%
    \let\@endslash\@empty
    \kvsetkeys{tag}{#1}%
    \texttt{%
      \textless #2\the\toks@\@endslash\textgreater
    }%
  \endgroup
}
\kv@set@family@handler{tag}{%
  % #1: key
  % #2: value
  \toks@\expandafter{%
    \the\toks@
    \space
    #1=\string"#2\string"%
  }%
}
\define@key{tag}{/}[]{%
  \def\@endslash{/}%
}
\makeatother

\begin{document}
\begin{tabbing}
  \mbox{}\qquad\=\qquad\=\kill
  \tag{html}\\
  \>\dots\\
  \>\tag[border=1]{table}\\
  \>\>\tag[width=200, span=3, /]{colgroup}\\
  \>\>\dots\\
  \>\tag{/table}\\
  \>\dots\\
  \tag{/html}\\
\end{tabbing}
\end{document}
%</example>
%    \end{macrocode}
%
% \StopEventually{
% }
%
% \section{Implementation}
%
% \subsection{Identification}
%
%    \begin{macrocode}
%<*package>
%    \end{macrocode}
%    Reload check, especially if the package is not used with \LaTeX.
%    \begin{macrocode}
\begingroup\catcode61\catcode48\catcode32=10\relax%
  \catcode13=5 % ^^M
  \endlinechar=13 %
  \catcode35=6 % #
  \catcode39=12 % '
  \catcode44=12 % ,
  \catcode45=12 % -
  \catcode46=12 % .
  \catcode58=12 % :
  \catcode64=11 % @
  \catcode123=1 % {
  \catcode125=2 % }
  \expandafter\let\expandafter\x\csname ver@kvsetkeys.sty\endcsname
  \ifx\x\relax % plain-TeX, first loading
  \else
    \def\empty{}%
    \ifx\x\empty % LaTeX, first loading,
      % variable is initialized, but \ProvidesPackage not yet seen
    \else
      \expandafter\ifx\csname PackageInfo\endcsname\relax
        \def\x#1#2{%
          \immediate\write-1{Package #1 Info: #2.}%
        }%
      \else
        \def\x#1#2{\PackageInfo{#1}{#2, stopped}}%
      \fi
      \x{kvsetkeys}{The package is already loaded}%
      \aftergroup\endinput
    \fi
  \fi
\endgroup%
%    \end{macrocode}
%    Package identification:
%    \begin{macrocode}
\begingroup\catcode61\catcode48\catcode32=10\relax%
  \catcode13=5 % ^^M
  \endlinechar=13 %
  \catcode35=6 % #
  \catcode39=12 % '
  \catcode40=12 % (
  \catcode41=12 % )
  \catcode44=12 % ,
  \catcode45=12 % -
  \catcode46=12 % .
  \catcode47=12 % /
  \catcode58=12 % :
  \catcode64=11 % @
  \catcode91=12 % [
  \catcode93=12 % ]
  \catcode123=1 % {
  \catcode125=2 % }
  \expandafter\ifx\csname ProvidesPackage\endcsname\relax
    \def\x#1#2#3[#4]{\endgroup
      \immediate\write-1{Package: #3 #4}%
      \xdef#1{#4}%
    }%
  \else
    \def\x#1#2[#3]{\endgroup
      #2[{#3}]%
      \ifx#1\@undefined
        \xdef#1{#3}%
      \fi
      \ifx#1\relax
        \xdef#1{#3}%
      \fi
    }%
  \fi
\expandafter\x\csname ver@kvsetkeys.sty\endcsname
\ProvidesPackage{kvsetkeys}%
  [2019/12/15 v1.18 Key value parser (HO)]%
%    \end{macrocode}
%
%    \begin{macrocode}
\begingroup\catcode61\catcode48\catcode32=10\relax%
  \catcode13=5 % ^^M
  \endlinechar=13 %
  \catcode123=1 % {
  \catcode125=2 % }
  \catcode64=11 % @
  \def\x{\endgroup
    \expandafter\edef\csname KVS@AtEnd\endcsname{%
      \endlinechar=\the\endlinechar\relax
      \catcode13=\the\catcode13\relax
      \catcode32=\the\catcode32\relax
      \catcode35=\the\catcode35\relax
      \catcode61=\the\catcode61\relax
      \catcode64=\the\catcode64\relax
      \catcode123=\the\catcode123\relax
      \catcode125=\the\catcode125\relax
    }%
  }%
\x\catcode61\catcode48\catcode32=10\relax%
\catcode13=5 % ^^M
\endlinechar=13 %
\catcode35=6 % #
\catcode64=11 % @
\catcode123=1 % {
\catcode125=2 % }
\def\TMP@EnsureCode#1#2{%
  \edef\KVS@AtEnd{%
    \KVS@AtEnd
    \catcode#1=\the\catcode#1\relax
  }%
  \catcode#1=#2\relax
}
\TMP@EnsureCode{36}{3}% $
\TMP@EnsureCode{38}{4}% &
\TMP@EnsureCode{39}{12}% '
\TMP@EnsureCode{43}{12}% +
\TMP@EnsureCode{44}{12}% ,
\TMP@EnsureCode{45}{12}% -
\TMP@EnsureCode{46}{12}% .
\TMP@EnsureCode{47}{12}% /
\TMP@EnsureCode{91}{12}% [
\TMP@EnsureCode{93}{12}% ]
\TMP@EnsureCode{94}{7}% ^ (superscript)
\TMP@EnsureCode{96}{12}% `
\TMP@EnsureCode{126}{13}% ~ (active)
\edef\KVS@AtEnd{\KVS@AtEnd\noexpand\endinput}
%    \end{macrocode}
%
% \subsection{Package loading}
%
%    \begin{macrocode}
\begingroup\expandafter\expandafter\expandafter\endgroup
\expandafter\ifx\csname RequirePackage\endcsname\relax
  \def\TMP@RequirePackage#1[#2]{%
    \begingroup\expandafter\expandafter\expandafter\endgroup
    \expandafter\ifx\csname ver@#1.sty\endcsname\relax
      \input #1.sty\relax
    \fi
  }%
  \TMP@RequirePackage{infwarerr}[2007/09/09]%
  \let\PackageError\@PackageError
\else
\fi
%    \end{macrocode}
%
%    \begin{macrocode}
\expandafter\ifx\csname toks@\endcsname\relax
  \toksdef\toks@=0 %
\fi
%    \end{macrocode}
%
% \subsection{Check for \eTeX}
%
%    \cs{unexpanded}, \cs{ifcsname}, and \cs{unless} are used if found.
%    \begin{macrocode}
\ifx\numexpr\@undefined
  \catcode`\$=9 % ignore
  \catcode`\&=14 % comment
\else % e-TeX
  \catcode`\$=14 % comment
  \catcode`\&=9  % ignore
\fi
%    \end{macrocode}
%
% \subsection{Generic help macros}
%
%    \begin{macro}{\KVS@Empty}
%    \begin{macrocode}
\def\KVS@Empty{}
%    \end{macrocode}
%    \end{macro}
%    \begin{macro}{\KVS@FirstOfTwo}
%    \begin{macrocode}
\long\def\KVS@FirstOfTwo#1#2{#1}
%    \end{macrocode}
%    \end{macro}
%    \begin{macro}{\KVS@SecondOfTwo}
%    \begin{macrocode}
\long\def\KVS@SecondOfTwo#1#2{#2}
%    \end{macrocode}
%    \end{macro}
%
%    \begin{macro}{\KVS@IfEmpty}
%    \begin{macrocode}
\long\def\KVS@IfEmpty#1{%
& \edef\KVS@Temp{\unexpanded{#1}}%
$ \begingroup
$   \toks@{#1}%
$   \edef\KVS@Temp{\the\toks@}%
$ \expandafter\endgroup
  \ifx\KVS@Temp\KVS@Empty
    \expandafter\KVS@FirstOfTwo
  \else
    \expandafter\KVS@SecondOfTwo
  \fi
}
%    \end{macrocode}
%    \end{macro}
%
% \subsection{Normalizing}
%
%    \begin{macro}{\kv@normalize}
%    \begin{macrocode}
\long\def\kv@normalize#1{%
  \begingroup
    \toks@{,#1,}%
    \KVS@Comma
    \KVS@SpaceComma
    \KVS@CommaSpace
    \KVS@CommaComma
    \KVS@Equals
    \KVS@SpaceEquals
    \KVS@EqualsSpace
    \xdef\KVS@Global{\the\toks@}%
  \endgroup
  \let\kv@list\KVS@Global
}
%    \end{macrocode}
%    \end{macro}
%    \begin{macro}{\comma@normalize}
%    \begin{macrocode}
\def\comma@normalize#1{%
  \begingroup
    \toks@{,#1,}%
    \KVS@Comma
    \KVS@SpaceComma
    \KVS@CommaSpace
    \KVS@CommaComma
    \xdef\KVS@Global{\the\toks@}%
  \endgroup
  \let\comma@list\KVS@Global
}
%    \end{macrocode}
%    \end{macro}
%    \begin{macro}{\KVS@Comma}
%    Converts active commas into comma with catcode other.
%    Also adds a comma at the end to protect the last value
%    for next cleanup steps.
%    \begin{macrocode}
\begingroup
  \lccode`\,=`\,%
  \lccode`\~=`\,%
\lowercase{\endgroup
  \def\KVS@Comma{%
    \toks@\expandafter{\expandafter}\expandafter
    \KVS@@Comma\the\toks@~\KVS@Nil
  }%
  \long\def\KVS@@Comma#1~#2\KVS@Nil{%
    \toks@\expandafter{\the\toks@#1}%
    \KVS@IfEmpty{#2}{%
    }{%
      \KVS@@Comma,#2\KVS@Nil
    }%
  }%
}
%    \end{macrocode}
%    \end{macro}
%    \begin{macro}{\KVS@SpaceComma}
%    Removes spaces before the comma, may add commas at the end.
%    \begin{macrocode}
\def\KVS@SpaceComma#1{%
  \def\KVS@SpaceComma{%
    \expandafter\KVS@@SpaceComma\the\toks@#1,\KVS@Nil
  }%
}
\KVS@SpaceComma{ }
%    \end{macrocode}
%    \end{macro}
%    \begin{macro}{\KVS@@SpaceComma}
%    \begin{macrocode}
\long\def\KVS@@SpaceComma#1 ,#2\KVS@Nil{%
  \KVS@IfEmpty{#2}{%
    \toks@{#1}%
  }{%
    \KVS@@SpaceComma#1,#2\KVS@Nil
  }%
}
%    \end{macrocode}
%    \end{macro}
%    \begin{macro}{\KVS@CommaSpace}
%    Removes spaces after the comma, may add commas at the end.
%    \begin{macrocode}
\def\KVS@CommaSpace{%
  \expandafter\KVS@@CommaSpace\the\toks@, \KVS@Nil
}
%    \end{macrocode}
%    \end{macro}
%    \begin{macro}{\KVS@@CommaSpace}
%    \begin{macrocode}
\long\def\KVS@@CommaSpace#1, #2\KVS@Nil{%
  \KVS@IfEmpty{#2}{%
    \toks@{#1}%
  }{%
    \KVS@@CommaSpace#1,#2\KVS@Nil
  }%
}
%    \end{macrocode}
%    \end{macro}
%    \begin{macro}{\KVS@CommaComma}
%    Replaces multiple commas by one comma.
%    \begin{macrocode}
\def\KVS@CommaComma{%
  \expandafter\KVS@@CommaComma\the\toks@,\KVS@Nil
}
%    \end{macrocode}
%    \end{macro}
%    \begin{macro}{\KVS@@CommaComma}
%    \begin{macrocode}
\long\def\KVS@@CommaComma#1,,#2\KVS@Nil{%
  \KVS@IfEmpty{#2}{%
    \toks@{#1,}% (!)
  }{%
    \KVS@@CommaComma#1,#2\KVS@Nil
  }%
}
%    \end{macrocode}
%    \end{macro}
%
%    \begin{macro}{\KVS@Equals}
%    Converts active equals signs into catcode other characters.
%    \begin{macrocode}
\begingroup
  \lccode`\==`\=%
  \lccode`\~=`\=%
\lowercase{\endgroup
  \def\KVS@Equals{%
    \toks@\expandafter{\expandafter}\expandafter
    \KVS@@Equals\the\toks@~\KVS@Nil
  }%
  \long\def\KVS@@Equals#1~#2\KVS@Nil{%
    \edef\KVS@Temp{\the\toks@}%
    \ifx\KVS@Temp\KVS@Empty
      \expandafter\KVS@FirstOfTwo
    \else
      \expandafter\KVS@SecondOfTwo
    \fi
    {%
      \toks@{#1}%
    }{%
      \toks@\expandafter{\the\toks@=#1}%
    }%
    \KVS@IfEmpty{#2}{%
    }{%
      \KVS@@Equals#2\KVS@Nil
    }%
  }%
}
%    \end{macrocode}
%    \end{macro}
%    \begin{macro}{\KVS@SpaceEquals}
%    Removes spaces before the equals sign.
%    \begin{macrocode}
\def\KVS@SpaceEquals#1{%
  \def\KVS@SpaceEquals{%
    \expandafter\KVS@@SpaceEquals\the\toks@#1=\KVS@Nil
  }%
}
\KVS@SpaceEquals{ }
%    \end{macrocode}
%    \end{macro}
%    \begin{macro}{\KVS@@SpaceEquals}
%    \begin{macrocode}
\long\def\KVS@@SpaceEquals#1 =#2\KVS@Nil{%
  \KVS@IfEmpty{#2}{%
    \toks@{#1}%
  }{%
    \KVS@@SpaceEquals#1=#2\KVS@Nil
  }%
}
%    \end{macrocode}
%    \end{macro}
%    \begin{macro}{\KVS@EqualsSpace}
%    Removes spaces after the equals sign.
%    \begin{macrocode}
\def\KVS@EqualsSpace{%
  \expandafter\KVS@@EqualsSpace\the\toks@= \KVS@Nil
}
%    \end{macrocode}
%    \end{macro}
%    \begin{macro}{\KVS@@EqualsSpace}
%    \begin{macrocode}
\long\def\KVS@@EqualsSpace#1= #2\KVS@Nil{%
  \KVS@IfEmpty{#2}{%
    \toks@{#1}%
  }{%
    \KVS@@EqualsSpace#1=#2\KVS@Nil
  }%
}
%    \end{macrocode}
%    \end{macro}
%
% \subsection{Parsing key value lists}
%
%    \begin{macro}{\kv@parse}
%    Normalizes and parses the key value list. Also sets \cs{kv@list}.
%    \begin{macrocode}
\long\def\kv@parse#1{%
  \kv@normalize{#1}%
  \expandafter\kv@parse@normalized\expandafter{\kv@list}%
}
%    \end{macrocode}
%    \end{macro}
%    \begin{macro}{\kv@parse@normalized}
%    |#1|: key value list\\
%    |#2|: processor
%    \begin{macrocode}
\long\def\kv@parse@normalized#1#2{%
  \KVS@Parse#1,\KVS@Nil{#2}%
}
%    \end{macrocode}
%    \end{macro}
%    \begin{macro}{\KVS@Parse}
%    |#1,#2|: key value list\\
%    |#3|: processor
%    \begin{macrocode}
\long\def\KVS@Parse#1,#2\KVS@Nil#3{%
  \KVS@IfEmpty{#1}{%
  }{%
    \KVS@Process#1=\KVS@Nil{#3}%
  }%
  \KVS@MaybeBreak
  \KVS@IfEmpty{#2}{%
  }{%
    \KVS@Parse#2\KVS@Nil{#3}%
  }%
}
%    \end{macrocode}
%    \end{macro}
%    \begin{macro}{\KVS@Process}
%    |#1|: key\\
%    |#2|: value, |=|\\
%    |#3|: processor
%    \begin{macrocode}
\long\def\KVS@Process#1=#2\KVS@Nil#3{%
  \let\KVS@MaybeBreak\relax
  \def\kv@key{#1}%
  \KVS@IfEmpty{#2}{%
    \let\kv@value\relax
    #3{#1}{}%
  }{%
    \KVS@@Process{#1}#2\KVS@Nil{#3}%
  }%
}
%    \end{macrocode}
%    \end{macro}
%    \begin{macro}{\KVS@@Process}
%    |#1|: key\\
%    |#2|: value\\
%    |#3|: processor
%    \begin{macrocode}
\long\def\KVS@@Process#1#2=\KVS@Nil#3{%
& \edef\kv@value{\unexpanded{#2}}%
$ \begingroup
$   \toks@{#2}%
$   \xdef\KVS@Global{\the\toks@}%
$ \endgroup
$ \let\kv@value\KVS@Global
  #3{#1}{#2}%
}
%    \end{macrocode}
%    \end{macro}
%    \begin{macro}{\KVS@MaybeBreak}
%    \begin{macrocode}
\let\KVS@MaybeBreak\relax
%    \end{macrocode}
%    \end{macro}
%    \begin{macro}{\KVS@break}
%    \begin{macrocode}
\def\KVS@break#1#2#3#4{%
  \let\KVS@MaybeBreak\relax
}
%    \end{macrocode}
%    \end{macro}
%    \begin{macro}{\kv@break}
%    \begin{macrocode}
\def\kv@break{%
  \let\KVS@MaybeBreak\KVS@break
}
%    \end{macrocode}
%    \end{macro}
%
% \subsection{Parsing comma lists}
%
%    \begin{macro}{\comma@parse}
%    Normalizes and parses the key value list. Also sets \cs{comma@list}.
%    \begin{macrocode}
\def\comma@parse#1{%
  \comma@normalize{#1}%
  \expandafter\comma@parse@normalized\expandafter{\comma@list}%
}
%    \end{macrocode}
%    \end{macro}
%
%    \begin{macro}{\comma@parse@normalized}
%    |#1|: comma list\\
%    |#2|: processor
%    \begin{macrocode}
\def\comma@parse@normalized#1#2{%
  \KVS@CommaParse#1,\KVS@Nil{#2}%
}
%    \end{macrocode}
%    \end{macro}
%    \begin{macro}{\KVS@CommaParse}
%    |#1,#2|: comma list\\
%    |#3|: processor
%    \begin{macrocode}
\def\KVS@CommaParse#1,#2\KVS@Nil#3{%
  \KVS@IfEmpty{#1}{%
  }{%
    \def\comma@entry{#1}%
    #3{#1}%
  }%
  \KVS@MaybeBreak
  \KVS@IfEmpty{#2}{%
  }{%
    \KVS@CommaParse#2\KVS@Nil{#3}%
  }%
}
%    \end{macrocode}
%    \end{macro}
%    \begin{macro}{\comma@break}
%    \begin{macrocode}
\def\comma@break{%
  \let\KVS@MaybeBreak\KVS@break
}
%    \end{macrocode}
%    \end{macro}
%
% \subsection{Processing key value pairs}
%
%    \begin{macro}{\kv@handled@false}
%    The handler can call \cs{kv@handled@false} or
%    \cs{kv@handled@true} so report failure or success.
%    The default is success (compatibility for versions
%    before 2011/10/18 v1.15).
%    \begin{macrocode}
\def\kv@handled@false{%
  \let\ifkv@handled@\iffalse
}
%    \end{macrocode}
%    \end{macro}
%    \begin{macro}{\kv@handled@true}
%    \begin{macrocode}
\def\kv@handled@true{%
  \let\ifkv@handled@\iftrue
}
%    \end{macrocode}
%    \end{macro}
%    \begin{macro}{\ifkv@handled@}
%    \begin{macrocode}
\kv@handled@true
%    \end{macrocode}
%    \end{macro}
%
%    \begin{macro}{\kv@processor@default}
%    \begin{macrocode}
\def\kv@processor@default#1#2{%
  \begingroup
    \csname @safe@activestrue\endcsname
    \let\ifincsname\iftrue
    \edef\KVS@temp{\endgroup
      \noexpand\KVS@ProcessorDefault{#1}{#2}%
    }%
  \KVS@temp
}
%    \end{macrocode}
%    \end{macro}
%    \begin{macro}{\KVS@ProcessorDefault}
%    \begin{macrocode}
\long\def\KVS@ProcessorDefault#1#2#3{%
  \def\kv@fam{#1}%
& \unless\ifcsname KV@#1@#2\endcsname
$ \begingroup\expandafter\expandafter\expandafter\endgroup
$ \expandafter\ifx\csname KV@#1@#2\endcsname\relax
&   \unless\ifcsname KVS@#1@handler\endcsname
$   \begingroup\expandafter\expandafter\expandafter\endgroup
$   \expandafter\ifx\csname KVS@#1@handler\endcsname\relax
      \kv@error@unknownkey{#1}{#2}%
    \else
      \kv@handled@true
      \csname KVS@#1@handler\endcsname{#2}{#3}%
      \relax
      \ifkv@handled@
      \else
        \kv@error@unknownkey{#1}{#2}%
      \fi
    \fi
  \else
    \ifx\kv@value\relax
&     \unless\ifcsname KV@#1@#2@default\endcsname
$     \begingroup\expandafter\expandafter\expandafter\endgroup
$     \expandafter\ifx\csname KV@#1@#2@default\endcsname\relax
        \kv@error@novalue{#1}{#2}%
      \else
        \csname KV@#1@#2@default\endcsname
        \relax
      \fi
    \else
      \csname KV@#1@#2\endcsname{#3}%
    \fi
  \fi
}
%    \end{macrocode}
%    \end{macro}
%
%    \begin{macro}{\kv@processor@known}
%    \begin{macrocode}
\def\kv@processor@known#1#2#3{%
  \begingroup
    \csname @safe@activestrue\endcsname
    \let\ifincsname\iftrue
    \edef\KVS@temp{\endgroup
      \noexpand\KVS@ProcessorKnown{#1}\noexpand#2{#3}%
    }%
  \KVS@temp
}
%    \end{macrocode}
%    \end{macro}
%    \begin{macro}{\KVS@ProcessorKnown}
%    \begin{macrocode}
\long\def\KVS@ProcessorKnown#1#2#3#4{%
  \def\kv@fam{#1}%
& \unless\ifcsname KV@#1@#3\endcsname
$ \begingroup\expandafter\expandafter\expandafter\endgroup
$ \expandafter\ifx\csname KV@#1@#3\endcsname\relax
&   \unless\ifcsname KVS@#1@handler\endcsname
$   \begingroup\expandafter\expandafter\expandafter\endgroup
$   \expandafter\ifx\csname KVS@#1@handler\endcsname\relax
      \KVS@AddUnhandled#2{#3}{#4}%
    \else
      \kv@handled@true
      \csname KVS@#1@handler\endcsname{#3}{#4}%
      \relax
      \ifkv@handled@
      \else
        \KVS@AddUnhandled#2{#3}{#4}%
      \fi
    \fi
  \else
    \ifx\kv@value\relax
&     \unless\ifcsname KV@#1@#2@default\endcsname
$     \begingroup\expandafter\expandafter\expandafter\endgroup
$     \expandafter\ifx\csname KV@#1@#3@default\endcsname\relax
        \kv@error@novalue{#1}{#3}%
      \else
        \csname KV@#1@#3@default\endcsname
        \relax
      \fi
    \else
      \csname KV@#1@#3\endcsname{#4}%
    \fi
  \fi
}
%    \end{macrocode}
%    \end{macro}
%    \begin{macro}{\KVS@AddUnhandled}
%    \begin{macrocode}
\long\def\KVS@AddUnhandled#1#2#3{%
& \edef#1{%
&   \ifx#1\KVS@empty
&   \else
&     \unexpanded{#1},%
&   \fi
&   \unexpanded{{#2}={#3}}%
& }%
$ \begingroup
$   \ifx#1\KVS@empty
$     \toks@{{#2}={#3}}%
$   \else
$     \toks@\expandafter{#1,{#2}={#3}}%
$   \fi
$   \xdef\KVS@Global{\the\toks@}%
$ \endgroup
$ \let#1\KVS@Global
}
%    \end{macrocode}
%    \end{macro}
%
%    \begin{macro}{\kv@set@family@handler}
%    \begin{macrocode}
\long\def\kv@set@family@handler#1#2{%
  \begingroup
    \csname @safe@activestrue\endcsname
    \let\ifincsname\iftrue
  \expandafter\endgroup
  \expandafter\def\csname KVS@#1@handler\endcsname##1##2{#2}%
}
%    \end{macrocode}
%    \end{macro}
%
%    \begin{macro}{\kv@unset@family@handler}
%    \begin{macrocode}
\long\def\kv@unset@family@handler#1#2{%
  \begingroup
    \csname @safe@activestrue\endcsname
    \let\ifincsname\iftrue
  \expandafter\endgroup
  \expandafter\let\csname KVS@#1@handler\endcsname\@UnDeFiNeD
}
%    \end{macrocode}
%    \end{macro}
%
% \subsection{Error handling}
%
%    \begin{macro}{\kv@error@novalue}
%    \begin{macrocode}
\def\kv@error@novalue{%
  \kv@error@generic{No value specified for}%
}
%    \end{macrocode}
%    \end{macro}
%    \begin{macro}{\kv@error@unknownkey}
%    \begin{macrocode}
\def\kv@error@unknownkey{%
  \kv@error@generic{Undefined}%
}
%    \end{macrocode}
%    \end{macro}
%    \begin{macro}{\kv@error@generic}
%    \begin{macrocode}
\def\kv@error@generic#1#2#3{%
  \PackageError{kvsetkeys}{%
    #1 key `#3'%
  }{%
    The keyval family of the key `#3' is `#2'.\MessageBreak
    The setting of the key is ignored because of the error.\MessageBreak
    \MessageBreak
    \@ehc
  }%
}
%    \end{macrocode}
%    \end{macro}
%
% \subsection{Do it all}
%
%    \begin{macro}{\kvsetkeys}
%    \begin{macrocode}
\long\def\kvsetkeys#1#2{%
  \kv@parse{#2}{\kv@processor@default{#1}}%
}
%    \end{macrocode}
%    \end{macro}
%
%    \begin{macro}{\kvsetkeys@expandafter}
%    \begin{macrocode}
\def\kvsetkeys@expandafter#1#2{%
  \expandafter\kv@parse\expandafter{#2}{%
    \kv@processor@default{#1}%
  }%
}
%    \end{macrocode}
%    \end{macro}
%
%    \begin{macro}{\KVS@cmd}
%    \begin{macrocode}
\def\KVS@cmd{0}%
%    \end{macrocode}
%    \end{macro}
%    \begin{macro}{\KVS@cmd@inc}
%    \begin{macrocode}
\def\KVS@cmd@inc{%
& \edef\KVS@cmd{\the\numexpr\KVS@cmd+1}%
$ \begingroup
$   \count255=\KVS@cmd\relax
$   \advance\count255 by 1\relax
$ \edef\x{\endgroup
$   \noexpand\def\noexpand\KVS@cmd{\number\count255}%
$ }%
$ \x
}
%    \end{macrocode}
%    \end{macro}
%    \begin{macro}{\KVS@cmd@dec}
%    \begin{macrocode}
\def\KVS@cmd@dec{%
& \edef\KVS@cmd{\the\numexpr\KVS@cmd-1}%
$ \begingroup
$   \count255=\KVS@cmd\relax
$   \advance\count255 by -1\relax
$ \edef\x{\endgroup
$   \noexpand\def\noexpand\KVS@cmd{\number\count255}%
$ }%
$ \x
}
%    \end{macrocode}
%    \end{macro}
%    \begin{macro}{\KVS@empty}
%    \begin{macrocode}
\def\KVS@empty{}
%    \end{macrocode}
%    \end{macro}
%
%    \begin{macro}{\kvsetknownkeys}
%    \begin{macrocode}
\def\kvsetknownkeys{%
  \expandafter
  \KVS@setknownkeys\csname KVS@cmd\KVS@cmd\endcsname{}%
}
%    \end{macrocode}
%    \end{macro}
%    \begin{macro}{\KVS@setknownkeys}
%    \begin{macrocode}
\long\def\KVS@setknownkeys#1#2#3#4#5{%
  \let#1\KVS@empty
  \KVS@cmd@inc
  #2\kv@parse#2{#5}{\kv@processor@known{#3}#1}%
  \KVS@cmd@dec
  \let#4=#1%
}
%    \end{macrocode}
%    \end{macro}
%
%    \begin{macro}{\kvsetknownkeys@expandafter}
%    \begin{macrocode}
\def\kvsetknownkeys@expandafter{%
  \expandafter
  \KVS@setknownkeys
      \csname KVS@cmd\KVS@cmd\endcsname\expandafter
}
%    \end{macrocode}
%    \end{macro}
%
%    \begin{macrocode}
\KVS@AtEnd%
%</package>
%    \end{macrocode}
%% \section{Installation}
%
% \subsection{Download}
%
% \paragraph{Package.} This package is available on
% CTAN\footnote{\CTANpkg{kvsetkeys}}:
% \begin{description}
% \item[\CTAN{macros/latex/contrib/kvsetkeys/kvsetkeys.dtx}] The source file.
% \item[\CTAN{macros/latex/contrib/kvsetkeys/kvsetkeys.pdf}] Documentation.
% \end{description}
%
%
% \paragraph{Bundle.} All the packages of the bundle `kvsetkeys'
% are also available in a TDS compliant ZIP archive. There
% the packages are already unpacked and the documentation files
% are generated. The files and directories obey the TDS standard.
% \begin{description}
% \item[\CTANinstall{install/macros/latex/contrib/kvsetkeys.tds.zip}]
% \end{description}
% \emph{TDS} refers to the standard ``A Directory Structure
% for \TeX\ Files'' (\CTANpkg{tds}). Directories
% with \xfile{texmf} in their name are usually organized this way.
%
% \subsection{Bundle installation}
%
% \paragraph{Unpacking.} Unpack the \xfile{kvsetkeys.tds.zip} in the
% TDS tree (also known as \xfile{texmf} tree) of your choice.
% Example (linux):
% \begin{quote}
%   |unzip kvsetkeys.tds.zip -d ~/texmf|
% \end{quote}
%
% \subsection{Package installation}
%
% \paragraph{Unpacking.} The \xfile{.dtx} file is a self-extracting
% \docstrip\ archive. The files are extracted by running the
% \xfile{.dtx} through \plainTeX:
% \begin{quote}
%   \verb|tex kvsetkeys.dtx|
% \end{quote}
%
% \paragraph{TDS.} Now the different files must be moved into
% the different directories in your installation TDS tree
% (also known as \xfile{texmf} tree):
% \begin{quote}
% \def\t{^^A
% \begin{tabular}{@{}>{\ttfamily}l@{ $\rightarrow$ }>{\ttfamily}l@{}}
%   kvsetkeys.sty & tex/generic/kvsetkeys/kvsetkeys.sty\\
%   kvsetkeys.pdf & doc/latex/kvsetkeys/kvsetkeys.pdf\\
%   kvsetkeys-example.tex & doc/latex/kvsetkeys/kvsetkeys-example.tex\\
%   kvsetkeys.dtx & source/latex/kvsetkeys/kvsetkeys.dtx\\
% \end{tabular}^^A
% }^^A
% \sbox0{\t}^^A
% \ifdim\wd0>\linewidth
%   \begingroup
%     \advance\linewidth by\leftmargin
%     \advance\linewidth by\rightmargin
%   \edef\x{\endgroup
%     \def\noexpand\lw{\the\linewidth}^^A
%   }\x
%   \def\lwbox{^^A
%     \leavevmode
%     \hbox to \linewidth{^^A
%       \kern-\leftmargin\relax
%       \hss
%       \usebox0
%       \hss
%       \kern-\rightmargin\relax
%     }^^A
%   }^^A
%   \ifdim\wd0>\lw
%     \sbox0{\small\t}^^A
%     \ifdim\wd0>\linewidth
%       \ifdim\wd0>\lw
%         \sbox0{\footnotesize\t}^^A
%         \ifdim\wd0>\linewidth
%           \ifdim\wd0>\lw
%             \sbox0{\scriptsize\t}^^A
%             \ifdim\wd0>\linewidth
%               \ifdim\wd0>\lw
%                 \sbox0{\tiny\t}^^A
%                 \ifdim\wd0>\linewidth
%                   \lwbox
%                 \else
%                   \usebox0
%                 \fi
%               \else
%                 \lwbox
%               \fi
%             \else
%               \usebox0
%             \fi
%           \else
%             \lwbox
%           \fi
%         \else
%           \usebox0
%         \fi
%       \else
%         \lwbox
%       \fi
%     \else
%       \usebox0
%     \fi
%   \else
%     \lwbox
%   \fi
% \else
%   \usebox0
% \fi
% \end{quote}
% If you have a \xfile{docstrip.cfg} that configures and enables \docstrip's
% TDS installing feature, then some files can already be in the right
% place, see the documentation of \docstrip.
%
% \subsection{Refresh file name databases}
%
% If your \TeX~distribution
% (\TeX\,Live, \mikTeX, \dots) relies on file name databases, you must refresh
% these. For example, \TeX\,Live\ users run \verb|texhash| or
% \verb|mktexlsr|.
%
% \subsection{Some details for the interested}
%
% \paragraph{Unpacking with \LaTeX.}
% The \xfile{.dtx} chooses its action depending on the format:
% \begin{description}
% \item[\plainTeX:] Run \docstrip\ and extract the files.
% \item[\LaTeX:] Generate the documentation.
% \end{description}
% If you insist on using \LaTeX\ for \docstrip\ (really,
% \docstrip\ does not need \LaTeX), then inform the autodetect routine
% about your intention:
% \begin{quote}
%   \verb|latex \let\install=y% \iffalse meta-comment
%
% File: kvsetkeys.dtx
% Version: 2019/12/15 v1.18
% Info: Key value parser
%
% Copyright (C)
%    2006, 2007, 2009-2012 Heiko Oberdiek
%    2016-2019 Oberdiek Package Support Group
%    https://github.com/ho-tex/kvsetkeys/issues
%
% This work may be distributed and/or modified under the
% conditions of the LaTeX Project Public License, either
% version 1.3c of this license or (at your option) any later
% version. This version of this license is in
%    https://www.latex-project.org/lppl/lppl-1-3c.txt
% and the latest version of this license is in
%    https://www.latex-project.org/lppl.txt
% and version 1.3 or later is part of all distributions of
% LaTeX version 2005/12/01 or later.
%
% This work has the LPPL maintenance status "maintained".
%
% The Current Maintainers of this work are
% Heiko Oberdiek and the Oberdiek Package Support Group
% https://github.com/ho-tex/kvsetkeys/issues
%
% The Base Interpreter refers to any `TeX-Format',
% because some files are installed in TDS:tex/generic//.
%
% This work consists of the main source file kvsetkeys.dtx
% and the derived files
%    kvsetkeys.sty, kvsetkeys.pdf, kvsetkeys.ins, kvsetkeys.drv,
%    kvsetkeys-example.tex, kvsetkeys-test1.tex,
%    kvsetkeys-test2.tex, kvsetkeys-test3.tex,
%    kvsetkeys-test4.tex.
%
% Distribution:
%    CTAN:macros/latex/contrib/kvsetkeys/kvsetkeys.dtx
%    CTAN:macros/latex/contrib/kvsetkeys/kvsetkeys.pdf
%
% Unpacking:
%    (a) If kvsetkeys.ins is present:
%           tex kvsetkeys.ins
%    (b) Without kvsetkeys.ins:
%           tex kvsetkeys.dtx
%    (c) If you insist on using LaTeX
%           latex \let\install=y\input{kvsetkeys.dtx}
%        (quote the arguments according to the demands of your shell)
%
% Documentation:
%    (a) If kvsetkeys.drv is present:
%           latex kvsetkeys.drv
%    (b) Without kvsetkeys.drv:
%           latex kvsetkeys.dtx; ...
%    The class ltxdoc loads the configuration file ltxdoc.cfg
%    if available. Here you can specify further options, e.g.
%    use A4 as paper format:
%       \PassOptionsToClass{a4paper}{article}
%
%    Programm calls to get the documentation (example):
%       pdflatex kvsetkeys.dtx
%       makeindex -s gind.ist kvsetkeys.idx
%       pdflatex kvsetkeys.dtx
%       makeindex -s gind.ist kvsetkeys.idx
%       pdflatex kvsetkeys.dtx
%
% Installation:
%    TDS:tex/generic/kvsetkeys/kvsetkeys.sty
%    TDS:doc/latex/kvsetkeys/kvsetkeys.pdf
%    TDS:doc/latex/kvsetkeys/kvsetkeys-example.tex
%    TDS:source/latex/kvsetkeys/kvsetkeys.dtx
%
%<*ignore>
\begingroup
  \catcode123=1 %
  \catcode125=2 %
  \def\x{LaTeX2e}%
\expandafter\endgroup
\ifcase 0\ifx\install y1\fi\expandafter
         \ifx\csname processbatchFile\endcsname\relax\else1\fi
         \ifx\fmtname\x\else 1\fi\relax
\else\csname fi\endcsname
%</ignore>
%<*install>
\input docstrip.tex
\Msg{************************************************************************}
\Msg{* Installation}
\Msg{* Package: kvsetkeys 2019/12/15 v1.18 Key value parser (HO)}
\Msg{************************************************************************}

\keepsilent
\askforoverwritefalse

\let\MetaPrefix\relax
\preamble

This is a generated file.

Project: kvsetkeys
Version: 2019/12/15 v1.18

Copyright (C)
   2006, 2007, 2009-2012 Heiko Oberdiek
   2016-2019 Oberdiek Package Support Group

This work may be distributed and/or modified under the
conditions of the LaTeX Project Public License, either
version 1.3c of this license or (at your option) any later
version. This version of this license is in
   https://www.latex-project.org/lppl/lppl-1-3c.txt
and the latest version of this license is in
   https://www.latex-project.org/lppl.txt
and version 1.3 or later is part of all distributions of
LaTeX version 2005/12/01 or later.

This work has the LPPL maintenance status "maintained".

The Current Maintainers of this work are
Heiko Oberdiek and the Oberdiek Package Support Group
https://github.com/ho-tex/kvsetkeys/issues


The Base Interpreter refers to any `TeX-Format',
because some files are installed in TDS:tex/generic//.

This work consists of the main source file kvsetkeys.dtx
and the derived files
   kvsetkeys.sty, kvsetkeys.pdf, kvsetkeys.ins, kvsetkeys.drv,
   kvsetkeys-example.tex, kvsetkeys-test1.tex,
   kvsetkeys-test2.tex, kvsetkeys-test3.tex,
   kvsetkeys-test4.tex.

\endpreamble
\let\MetaPrefix\DoubleperCent

\generate{%
  \file{kvsetkeys.ins}{\from{kvsetkeys.dtx}{install}}%
  \file{kvsetkeys.drv}{\from{kvsetkeys.dtx}{driver}}%
  \usedir{tex/generic/kvsetkeys}%
  \file{kvsetkeys.sty}{\from{kvsetkeys.dtx}{package}}%
  \usedir{doc/latex/kvsetkeys}%
  \file{kvsetkeys-example.tex}{\from{kvsetkeys.dtx}{example}}%
}

\catcode32=13\relax% active space
\let =\space%
\Msg{************************************************************************}
\Msg{*}
\Msg{* To finish the installation you have to move the following}
\Msg{* file into a directory searched by TeX:}
\Msg{*}
\Msg{*     kvsetkeys.sty}
\Msg{*}
\Msg{* To produce the documentation run the file `kvsetkeys.drv'}
\Msg{* through LaTeX.}
\Msg{*}
\Msg{* Happy TeXing!}
\Msg{*}
\Msg{************************************************************************}

\endbatchfile
%</install>
%<*ignore>
\fi
%</ignore>
%<*driver>
\NeedsTeXFormat{LaTeX2e}
\ProvidesFile{kvsetkeys.drv}%
  [2019/12/15 v1.18 Key value parser (HO)]%
\documentclass{ltxdoc}
\usepackage{holtxdoc}[2011/11/22]
\begin{document}
  \DocInput{kvsetkeys.dtx}%
\end{document}
%</driver>
% \fi
%
%
%
% \GetFileInfo{kvsetkeys.drv}
%
% \title{The \xpackage{kvsetkeys} package}
% \date{2019/12/15 v1.18}
% \author{Heiko Oberdiek\thanks
% {Please report any issues at \url{https://github.com/ho-tex/kvsetkeys/issues}}}
%
% \maketitle
%
% \begin{abstract}
% Package \xpackage{kvsetkeys} provides \cs{kvsetkeys}, a variant
% of package \xpackage{keyval}'s \cs{setkeys}. It allows to specify
% a handler that deals with unknown options. Active commas and equal
% signs may be used (e.g. see \xpackage{babel}'s shorthands) and
% only one level of curly braces is removed from the values.
% \end{abstract}
%
% \tableofcontents
%
% \def\M#1{\texttt{\{}\meta{#1}\texttt{\}}}
%
% \section{Documentation}
%
% First I want to recommend the very good review article
% ``A guide to key-value methods'' by Joseph Wright \cite{tb94wright}.
% It introduces the different key-value packages and compares them.
%
% \subsection{Motivation}
%
% \cs{kvsetkeys} serves as replacement for \xpackage{keyval}'s
% \cs{setkeys}. It basically uses the same syntax. But the
% implementation is more robust and predictable:
% \begin{description}
% \item[Active syntax characters:]
%   Comma `|,|' and the equals sign `|=|' are used inside
%   key value lists as syntax characters. Package \xpackage{keyval}
%   uses the catcode of the characters that is active during
%   package loading, usually this is catcode 12 (other).
%   But it can happen that the catcode setting of the syntax characters
%   changes. Especially active characters are of interest, because
%   some language adaptations uses them. For example, option \xoption{turkish}
%   of package \xpackage{babel} uses the equals sign as active shorthand
%   character. Therefore package \xpackage{kvsetkeys} deals with
%   both catcode settings 12 (other) and 13 (active).
% \item[Brace removal:]
%   Package \xpackage{keyval}'s \cs{setkeys} removes up to two
%   levels of curly braces around the value in some unpredictable way:
%\begin{quote}
%|\setkeys{fam}{key={{value}}} || || | $\rightarrow$ |value|\\
%|\setkeys{fam}{key={{{value}}}} | $\rightarrow$ |{value}|\\
%|\setkeys{fam}{key= {{{value}}}}| $\rightarrow$ |{{value}}|
%\end{quote}
%   This package \xpackage{kvsetkeys} follows a much stronger rule:
%   Exactly one level of braces are removed from an item, if the
%   item is surrounded by curly braces. An item can be a
%   the key value pair, the key or the value.
%\begin{quote}
%|\kvsetkeys{fam}{key={value}} || | $\rightarrow$ |value|\\
%|\kvsetkeys{fam}{key={{value}} | $\rightarrow$ |{value}|\\
%|\kvsetkeys{fam}{key= {{value}}| $\rightarrow$ |{value}|
%\end{quote}
% \item[Arbitrary values:] Unmatched conditionals are supported.
% \end{description}
%
% Before I describe
% \cs{kvsetkeys} in more detail, first I want to explain, how
% this package deals with key value lists. For the package also
% provides low level interfaces that can be used by package authors.
%
% \subsection{Normalizing key value lists}
%
% \begin{declcs}{kv@normalize}\,\M{key value list}
% \end{declcs}
% If the user specifies key value lists, he usually prefers
% nice formatted source code, e.g.:
% \begin{quote}
%\begin{verbatim}
%\hypersetup{
%  pdftitle    = {...},
%  pdfsubject  = {...},
%  pdfauthor   = {...},
%  pdfkeywords = {...},
%  ...
%}
%\end{verbatim}
% \end{quote}
% Thus there can be spaces around keys, around |=| or around the value.
% Also empty entries are possible by too many commas. Therefore these
% spaces and empty entries are silently removed by package \xpackage{keyval}
% and this package. Whereas the contents of the value can be protected
% by curly braces, especially if spaces or commas are used inside,
% a key name must not use spaces or other syntax characters.
%
% \cs{kv@normalize} takes a key value list and performs the cleanup:
% \begin{itemize}
% \item Spaces are removed.
% \item Syntax characters (comma and equal sign) that are active
%   are replaced by the same characters with standard catcode.
%   (Example: \xpackage{babel}'s language option \xoption{turkish}
%   uses the equal sign as active shorthand character.)
% \end{itemize}
% The result is stored in \cs{kv@list}, e.g.:
% \begin{quote}
% |\kv@list| $\rightarrow$ |,pdftitle={...},pdfsubject={...},...,|
% \end{quote}
% Curly braces around values (or keys) remain untouched.
% \begin{description}
% \item[v1.3+:]
% One comma is added in front of the list and each pair ends with
% a comma. Thus an empty list consists of one comma, otherwise
% two commas encloses the list. Empty entries other than the first
% are removed.
% \item[v1.0 -- v1.2:]
% Empty entries are removed later. In fact it adds a comma at the begin
% and end to protect the last value and an easier implementation.
% \end{description}
%
% \subsection{Parsing key value lists}
%
% \begin{declcs}{kv@parse}\,\M{key value list}\,\M{processor}
% \end{declcs}
% It is easier to parse a normalized list, thus \cs{kv@parse}
% normalizes the list and calls \cs{kv@parse@normalized}.
%
% \begin{declcs}{kv@parse@normalized}\,\M{key value list}%
%   \,\M{processor}
% \end{declcs}
% Now the key value list is split into single key value pairs.
% For further processing the key and value are given as arguments
% for the \meta{processor}:
% \begin{quote}
%   \meta{processor}\,\M{key}\,\M{value}
% \end{quote}
% Also key and value are stored in macro names:
% \begin{itemize}
% \item \cs{kv@key} stores the key.
% \item \cs{kv@value} stores the value or if the value was not
%   specified it has the meaning \cs{relax}.
% \end{itemize}
% The behaviour in pseudo code:
% \begin{quote}
%   foreach (\meta{key}, \meta{value}) in (\meta{key value list})\\
%   \hspace*{1.5em}\cs{kv@key} := \meta{key}\\
%   \hspace*{1.5em}\cs{kv@value} := \meta{value}\\
%   \hspace*{1.5em}\meta{processor}\,\M{key}\,\M{value}
% \end{quote}
%
% \begin{declcs}{kv@break}
% \end{declcs}
% Since version 2011/03/03 v1.11 \cs{kv@break} can be
% called inside the \meta{processor}
% of \cs{kv@parse} or \cs{kv@parse@normalized}, then
% the processing is stopped and the following entries discarded.
%
% \subsection{Processing key value pairs}
%
% Key value pairs can be processed in many different ways.
% For example, the processor for \cs{kvsetkeys} works similar
% to \cs{setkeys} of package \xpackage{keyval}. There unknown
% keys raise an error.
%
% Package \xpackage{xkeyval} also knows a star form of \cs{setkeys}
% that stores unknown keys in an internal macro for further processing
% with \cs{setrmkeys} and similar macros. This feature is covered
% by processor \cs{kv@processor@known}.
%
% \subsubsection{Processing similar to \xpackage{keyval}}
%
% \begin{declcs}{kv@processor@default}\,\M{family}\,\M{key}\,\M{value}
% \end{declcs}
% There are many possiblities to process key value pairs.
% \cs{kv@processor@default} is the processor used in \cs{kvsetkeys}.
% It reimplements and extends the behaviour of
% \xpackage{keyval}'s \cs{setkeys}.
% In case of unknown keys \cs{setkeys} raise an error.
% This processer, however, calls a handler instead, if it
% is provided by the family. Both \meta{family} and \meta{key}
% may contain package \xpackage{babel}'s shorthands
% (since 2011/04/07 v1.13).
%
% Since 2011/10/18 v1.15 the family handler can reject the
% successful handling of a key by calling \cs{kv@handled@false}.
%
% Since 2019/12/15 v1.18 \cs{kv@processor@default} also defines
% macro \cs{kv@fam} with meaning \meta{family} for convenience.
%
% \subsubsection{Processing similar to \cs{setkeys*} of package \xpackage{xkeyval}}
%
% \begin{declcs}{kv@processor@known}\,\M{family}\,^^A
%   \M{cmd}\,\M{key}\,\M{value}
% \end{declcs}
% The key value processor \cs{kv@processor@known} behaves similar
% to \cs{kv@processor@default}. If the \meta{key} exists in the
% \meta{family} its code is called, otherwise the family handler
% is tried. If the family handler is not set or cannot handle the
% key, the unknown key value pair is added to the macro \meta{cmd}.
% Since 2011/10/18 v1.15.
%
% The behaviour in pseudo code:
% \begin{quote}
%   if \meta{key} exists\\
%   \hspace*{1.5em}call the keyval code of \meta{key}\\
%   else\\
%   \hspace*{1.5em}if \meta{handler} for \meta{family} exists\\
%   \hspace*{3em}handled = true\\
%   \hspace*{3em}\meta{handler}\,\M{key}\,\M{value}\\
%   \hspace*{3em}if handled\\
%   \hspace*{3em}else\\
%   \hspace*{4.5em}add \texttt{"}\M{key}\texttt{=}\M{value}\texttt{"}
%     to \M{cmd}\\
%   \hspace*{3em}fi\\
%   \hspace*{1.5em}else\\
%   \hspace*{3em}add \texttt{"}\M{key}\texttt{=}\M{value}\texttt{"}
%     to \M{cmd}\\
%   \hspace*{3em}raise unknown key error\\
%   \hspace*{1.5em}fi\\
%   fi
% \end{quote}
%
% Since 2019/12/15 v1.18 \cs{kv@processor@known} also defines
% macro \cs{kv@fam} with meaning \meta{family} for convenience.
%
% \subsection{Default family handler}
%
% \cs{kv@processor@default} calls \meta{handler}, the default
% handler for the family, if the key does not exist in the family.
% The handler is called with two arguments, the key and the value.
% It can be defined with \cs{kv@set@family@hander}:
%
% \begin{declcs}{kv@set@family@handler}\,\M{family}\,\M{handler definition}
% \end{declcs}
% This sets the default family handler for the keyval family
% \meta{family}. Inside \meta{handler definition} |#1| stands for
% the key and |#2| is the value. Also \cs{kv@key} and \cs{kv@value}
% can be used for the key and the value. If the value is not given,
% \cs{kv@value} has the meaning \cs{relax}.
%
% \begin{declcs}{kv@unset@family@handler}\,\M{family}
% \end{declcs}
% It removes the family handler for \meta{family}.
% Since 2011/10/18 v1.15.
%
% \subsection{Put it all together}
%
% \begin{declcs}{kvsetkeys}\,\M{family}\,\M{key value list}
% \end{declcs}
% Macro \cs{kvsetkeys} processes the \meta{key value list} with
% the standard processor \cs{kv@processor@default}:
% \begin{quote}
%   \cs{kv@parse}\,\M{key value list}^^A
%   \texttt{\{}\cs{kv@processor@default}\,\M{family}\texttt{\}}
% \end{quote}
%
% \begin{declcs}{kvsetknownkeys}\,\M{family}\,\M{cmd}\,\M{key value list}
% \end{declcs}
% Macro \cs{kvsetknownkeys} processes the \meta{key value list}
% with processor \cs{kv@processor@known}. All key value pairs
% with keys that are not known in \meta{family} are stored
% in macro \meta{cmd}. A previous contents of macro \meta{cmd}
% will be overwritten. If all keys can be handled, \meta{cmd}
% will be empty, otherwise it contains a key value list of
% unhandled key value pairs.
% Since 2011/10/18 v1.15.
%
% Pseudo code:
% \begin{quote}
%   create macro \meta{cmdaux} with unique name (inside the current group)\\
%   \cs{def}\meta{cmdaux}\{\}\\
%   \cs{kv@parse}\,\M{key value list}^^A
%   \texttt{\{}\cs{kv@processor@known}\,\M{family}\,\M{cmdaux}\texttt{\}}\\
%   \cs{let}\meta{cmd}=\meta{cmdaux}
% \end{quote}
%
% \begin{declcs}{kvsetkeys@expandafter}\,\M{family}\,\M{list cmd}\\
%   \cs{kvsetknownkeys@expandafter}\,\M{family}\,%
%   \M{cmd}\,\M{list cmd}
% \end{declcs}
% Both macros behave like the counterparts without suffix
% |@expandafter|. The difference is that the key value list is
% given as macro that is expanded once.
% Since 2011/10/18 v1.15.
%
% Thus you can replace \cs{setkeys} of package \xpackage{keyval}
% by the key value parser of this package:
% \begin{quote}
%   |\renewcommand*{\setkeys}{\kvsetkeys}|\\
%   or\\
%   |\let\setkeys\kvsetkeys|
% \end{quote}
%
% \subsection{Comma separated lists}
%
% Since version 2007/09/29 v1.3 this package also supports the normalizing
% and parsing of general comma separated lists.
%
% \begin{declcs}{comma@normalize}\,\M{comma list}
% \end{declcs}
% Macro \cs{comma@normalize} normalizes the comma separated list,
% removes spaces around commas. The result is put in macro \cs{comma@list}.
%
% \begin{declcs}{comma@parse}\,\M{comma list}\,\M{processor}
% \end{declcs}
% Macro \cs{comma@parse} first normalizes the comma separated list
% and then parses the list by calling \cs{comma@parse@normalized}.
%
% \begin{declcs}{comma@parse@normalized}\,\M{normalized comma list}^^A
%   \,\M{processor}
% \end{declcs}
% The list is parsed. Empty entries are ignored. \meta{processor}
% is called for each non-empty entry with the entry as argument:
% \begin{quote}
%   \meta{processor}|{|\meta{entry}|}|
% \end{quote}
% Also the entry is stored in the macro \cs{comma@entry}.
%
% \begin{declcs}{comma@break}
% \end{declcs}
% Since version 2011/03/03 v1.11 \cs{comma@break} can be
% called inside the \meta{processor}
% of \cs{comma@parse} or \cs{comma@parse@normalized}, then
% the processing is stopped and the following entries discarded.
%
% \section{Example}
%
% The following example prints a short piece of HTML code using
% the tabbing environment for indenting purpose and a key value
% syntax for specifying the attributes of an HTML tag.
% The example illustrates the use of a default family handler.
%    \begin{macrocode}
%<*example>
\documentclass{article}
\usepackage[T1]{fontenc}
\usepackage{kvsetkeys}
\usepackage{keyval}

\makeatletter
\newcommand*{\tag}[2][]{%
  % #1: attributes
  % #2: tag name
  \begingroup
    \toks@={}%
    \let\@endslash\@empty
    \kvsetkeys{tag}{#1}%
    \texttt{%
      \textless #2\the\toks@\@endslash\textgreater
    }%
  \endgroup
}
\kv@set@family@handler{tag}{%
  % #1: key
  % #2: value
  \toks@\expandafter{%
    \the\toks@
    \space
    #1=\string"#2\string"%
  }%
}
\define@key{tag}{/}[]{%
  \def\@endslash{/}%
}
\makeatother

\begin{document}
\begin{tabbing}
  \mbox{}\qquad\=\qquad\=\kill
  \tag{html}\\
  \>\dots\\
  \>\tag[border=1]{table}\\
  \>\>\tag[width=200, span=3, /]{colgroup}\\
  \>\>\dots\\
  \>\tag{/table}\\
  \>\dots\\
  \tag{/html}\\
\end{tabbing}
\end{document}
%</example>
%    \end{macrocode}
%
% \StopEventually{
% }
%
% \section{Implementation}
%
% \subsection{Identification}
%
%    \begin{macrocode}
%<*package>
%    \end{macrocode}
%    Reload check, especially if the package is not used with \LaTeX.
%    \begin{macrocode}
\begingroup\catcode61\catcode48\catcode32=10\relax%
  \catcode13=5 % ^^M
  \endlinechar=13 %
  \catcode35=6 % #
  \catcode39=12 % '
  \catcode44=12 % ,
  \catcode45=12 % -
  \catcode46=12 % .
  \catcode58=12 % :
  \catcode64=11 % @
  \catcode123=1 % {
  \catcode125=2 % }
  \expandafter\let\expandafter\x\csname ver@kvsetkeys.sty\endcsname
  \ifx\x\relax % plain-TeX, first loading
  \else
    \def\empty{}%
    \ifx\x\empty % LaTeX, first loading,
      % variable is initialized, but \ProvidesPackage not yet seen
    \else
      \expandafter\ifx\csname PackageInfo\endcsname\relax
        \def\x#1#2{%
          \immediate\write-1{Package #1 Info: #2.}%
        }%
      \else
        \def\x#1#2{\PackageInfo{#1}{#2, stopped}}%
      \fi
      \x{kvsetkeys}{The package is already loaded}%
      \aftergroup\endinput
    \fi
  \fi
\endgroup%
%    \end{macrocode}
%    Package identification:
%    \begin{macrocode}
\begingroup\catcode61\catcode48\catcode32=10\relax%
  \catcode13=5 % ^^M
  \endlinechar=13 %
  \catcode35=6 % #
  \catcode39=12 % '
  \catcode40=12 % (
  \catcode41=12 % )
  \catcode44=12 % ,
  \catcode45=12 % -
  \catcode46=12 % .
  \catcode47=12 % /
  \catcode58=12 % :
  \catcode64=11 % @
  \catcode91=12 % [
  \catcode93=12 % ]
  \catcode123=1 % {
  \catcode125=2 % }
  \expandafter\ifx\csname ProvidesPackage\endcsname\relax
    \def\x#1#2#3[#4]{\endgroup
      \immediate\write-1{Package: #3 #4}%
      \xdef#1{#4}%
    }%
  \else
    \def\x#1#2[#3]{\endgroup
      #2[{#3}]%
      \ifx#1\@undefined
        \xdef#1{#3}%
      \fi
      \ifx#1\relax
        \xdef#1{#3}%
      \fi
    }%
  \fi
\expandafter\x\csname ver@kvsetkeys.sty\endcsname
\ProvidesPackage{kvsetkeys}%
  [2019/12/15 v1.18 Key value parser (HO)]%
%    \end{macrocode}
%
%    \begin{macrocode}
\begingroup\catcode61\catcode48\catcode32=10\relax%
  \catcode13=5 % ^^M
  \endlinechar=13 %
  \catcode123=1 % {
  \catcode125=2 % }
  \catcode64=11 % @
  \def\x{\endgroup
    \expandafter\edef\csname KVS@AtEnd\endcsname{%
      \endlinechar=\the\endlinechar\relax
      \catcode13=\the\catcode13\relax
      \catcode32=\the\catcode32\relax
      \catcode35=\the\catcode35\relax
      \catcode61=\the\catcode61\relax
      \catcode64=\the\catcode64\relax
      \catcode123=\the\catcode123\relax
      \catcode125=\the\catcode125\relax
    }%
  }%
\x\catcode61\catcode48\catcode32=10\relax%
\catcode13=5 % ^^M
\endlinechar=13 %
\catcode35=6 % #
\catcode64=11 % @
\catcode123=1 % {
\catcode125=2 % }
\def\TMP@EnsureCode#1#2{%
  \edef\KVS@AtEnd{%
    \KVS@AtEnd
    \catcode#1=\the\catcode#1\relax
  }%
  \catcode#1=#2\relax
}
\TMP@EnsureCode{36}{3}% $
\TMP@EnsureCode{38}{4}% &
\TMP@EnsureCode{39}{12}% '
\TMP@EnsureCode{43}{12}% +
\TMP@EnsureCode{44}{12}% ,
\TMP@EnsureCode{45}{12}% -
\TMP@EnsureCode{46}{12}% .
\TMP@EnsureCode{47}{12}% /
\TMP@EnsureCode{91}{12}% [
\TMP@EnsureCode{93}{12}% ]
\TMP@EnsureCode{94}{7}% ^ (superscript)
\TMP@EnsureCode{96}{12}% `
\TMP@EnsureCode{126}{13}% ~ (active)
\edef\KVS@AtEnd{\KVS@AtEnd\noexpand\endinput}
%    \end{macrocode}
%
% \subsection{Package loading}
%
%    \begin{macrocode}
\begingroup\expandafter\expandafter\expandafter\endgroup
\expandafter\ifx\csname RequirePackage\endcsname\relax
  \def\TMP@RequirePackage#1[#2]{%
    \begingroup\expandafter\expandafter\expandafter\endgroup
    \expandafter\ifx\csname ver@#1.sty\endcsname\relax
      \input #1.sty\relax
    \fi
  }%
  \TMP@RequirePackage{infwarerr}[2007/09/09]%
  \let\PackageError\@PackageError
\else
\fi
%    \end{macrocode}
%
%    \begin{macrocode}
\expandafter\ifx\csname toks@\endcsname\relax
  \toksdef\toks@=0 %
\fi
%    \end{macrocode}
%
% \subsection{Check for \eTeX}
%
%    \cs{unexpanded}, \cs{ifcsname}, and \cs{unless} are used if found.
%    \begin{macrocode}
\ifx\numexpr\@undefined
  \catcode`\$=9 % ignore
  \catcode`\&=14 % comment
\else % e-TeX
  \catcode`\$=14 % comment
  \catcode`\&=9  % ignore
\fi
%    \end{macrocode}
%
% \subsection{Generic help macros}
%
%    \begin{macro}{\KVS@Empty}
%    \begin{macrocode}
\def\KVS@Empty{}
%    \end{macrocode}
%    \end{macro}
%    \begin{macro}{\KVS@FirstOfTwo}
%    \begin{macrocode}
\long\def\KVS@FirstOfTwo#1#2{#1}
%    \end{macrocode}
%    \end{macro}
%    \begin{macro}{\KVS@SecondOfTwo}
%    \begin{macrocode}
\long\def\KVS@SecondOfTwo#1#2{#2}
%    \end{macrocode}
%    \end{macro}
%
%    \begin{macro}{\KVS@IfEmpty}
%    \begin{macrocode}
\long\def\KVS@IfEmpty#1{%
& \edef\KVS@Temp{\unexpanded{#1}}%
$ \begingroup
$   \toks@{#1}%
$   \edef\KVS@Temp{\the\toks@}%
$ \expandafter\endgroup
  \ifx\KVS@Temp\KVS@Empty
    \expandafter\KVS@FirstOfTwo
  \else
    \expandafter\KVS@SecondOfTwo
  \fi
}
%    \end{macrocode}
%    \end{macro}
%
% \subsection{Normalizing}
%
%    \begin{macro}{\kv@normalize}
%    \begin{macrocode}
\long\def\kv@normalize#1{%
  \begingroup
    \toks@{,#1,}%
    \KVS@Comma
    \KVS@SpaceComma
    \KVS@CommaSpace
    \KVS@CommaComma
    \KVS@Equals
    \KVS@SpaceEquals
    \KVS@EqualsSpace
    \xdef\KVS@Global{\the\toks@}%
  \endgroup
  \let\kv@list\KVS@Global
}
%    \end{macrocode}
%    \end{macro}
%    \begin{macro}{\comma@normalize}
%    \begin{macrocode}
\def\comma@normalize#1{%
  \begingroup
    \toks@{,#1,}%
    \KVS@Comma
    \KVS@SpaceComma
    \KVS@CommaSpace
    \KVS@CommaComma
    \xdef\KVS@Global{\the\toks@}%
  \endgroup
  \let\comma@list\KVS@Global
}
%    \end{macrocode}
%    \end{macro}
%    \begin{macro}{\KVS@Comma}
%    Converts active commas into comma with catcode other.
%    Also adds a comma at the end to protect the last value
%    for next cleanup steps.
%    \begin{macrocode}
\begingroup
  \lccode`\,=`\,%
  \lccode`\~=`\,%
\lowercase{\endgroup
  \def\KVS@Comma{%
    \toks@\expandafter{\expandafter}\expandafter
    \KVS@@Comma\the\toks@~\KVS@Nil
  }%
  \long\def\KVS@@Comma#1~#2\KVS@Nil{%
    \toks@\expandafter{\the\toks@#1}%
    \KVS@IfEmpty{#2}{%
    }{%
      \KVS@@Comma,#2\KVS@Nil
    }%
  }%
}
%    \end{macrocode}
%    \end{macro}
%    \begin{macro}{\KVS@SpaceComma}
%    Removes spaces before the comma, may add commas at the end.
%    \begin{macrocode}
\def\KVS@SpaceComma#1{%
  \def\KVS@SpaceComma{%
    \expandafter\KVS@@SpaceComma\the\toks@#1,\KVS@Nil
  }%
}
\KVS@SpaceComma{ }
%    \end{macrocode}
%    \end{macro}
%    \begin{macro}{\KVS@@SpaceComma}
%    \begin{macrocode}
\long\def\KVS@@SpaceComma#1 ,#2\KVS@Nil{%
  \KVS@IfEmpty{#2}{%
    \toks@{#1}%
  }{%
    \KVS@@SpaceComma#1,#2\KVS@Nil
  }%
}
%    \end{macrocode}
%    \end{macro}
%    \begin{macro}{\KVS@CommaSpace}
%    Removes spaces after the comma, may add commas at the end.
%    \begin{macrocode}
\def\KVS@CommaSpace{%
  \expandafter\KVS@@CommaSpace\the\toks@, \KVS@Nil
}
%    \end{macrocode}
%    \end{macro}
%    \begin{macro}{\KVS@@CommaSpace}
%    \begin{macrocode}
\long\def\KVS@@CommaSpace#1, #2\KVS@Nil{%
  \KVS@IfEmpty{#2}{%
    \toks@{#1}%
  }{%
    \KVS@@CommaSpace#1,#2\KVS@Nil
  }%
}
%    \end{macrocode}
%    \end{macro}
%    \begin{macro}{\KVS@CommaComma}
%    Replaces multiple commas by one comma.
%    \begin{macrocode}
\def\KVS@CommaComma{%
  \expandafter\KVS@@CommaComma\the\toks@,\KVS@Nil
}
%    \end{macrocode}
%    \end{macro}
%    \begin{macro}{\KVS@@CommaComma}
%    \begin{macrocode}
\long\def\KVS@@CommaComma#1,,#2\KVS@Nil{%
  \KVS@IfEmpty{#2}{%
    \toks@{#1,}% (!)
  }{%
    \KVS@@CommaComma#1,#2\KVS@Nil
  }%
}
%    \end{macrocode}
%    \end{macro}
%
%    \begin{macro}{\KVS@Equals}
%    Converts active equals signs into catcode other characters.
%    \begin{macrocode}
\begingroup
  \lccode`\==`\=%
  \lccode`\~=`\=%
\lowercase{\endgroup
  \def\KVS@Equals{%
    \toks@\expandafter{\expandafter}\expandafter
    \KVS@@Equals\the\toks@~\KVS@Nil
  }%
  \long\def\KVS@@Equals#1~#2\KVS@Nil{%
    \edef\KVS@Temp{\the\toks@}%
    \ifx\KVS@Temp\KVS@Empty
      \expandafter\KVS@FirstOfTwo
    \else
      \expandafter\KVS@SecondOfTwo
    \fi
    {%
      \toks@{#1}%
    }{%
      \toks@\expandafter{\the\toks@=#1}%
    }%
    \KVS@IfEmpty{#2}{%
    }{%
      \KVS@@Equals#2\KVS@Nil
    }%
  }%
}
%    \end{macrocode}
%    \end{macro}
%    \begin{macro}{\KVS@SpaceEquals}
%    Removes spaces before the equals sign.
%    \begin{macrocode}
\def\KVS@SpaceEquals#1{%
  \def\KVS@SpaceEquals{%
    \expandafter\KVS@@SpaceEquals\the\toks@#1=\KVS@Nil
  }%
}
\KVS@SpaceEquals{ }
%    \end{macrocode}
%    \end{macro}
%    \begin{macro}{\KVS@@SpaceEquals}
%    \begin{macrocode}
\long\def\KVS@@SpaceEquals#1 =#2\KVS@Nil{%
  \KVS@IfEmpty{#2}{%
    \toks@{#1}%
  }{%
    \KVS@@SpaceEquals#1=#2\KVS@Nil
  }%
}
%    \end{macrocode}
%    \end{macro}
%    \begin{macro}{\KVS@EqualsSpace}
%    Removes spaces after the equals sign.
%    \begin{macrocode}
\def\KVS@EqualsSpace{%
  \expandafter\KVS@@EqualsSpace\the\toks@= \KVS@Nil
}
%    \end{macrocode}
%    \end{macro}
%    \begin{macro}{\KVS@@EqualsSpace}
%    \begin{macrocode}
\long\def\KVS@@EqualsSpace#1= #2\KVS@Nil{%
  \KVS@IfEmpty{#2}{%
    \toks@{#1}%
  }{%
    \KVS@@EqualsSpace#1=#2\KVS@Nil
  }%
}
%    \end{macrocode}
%    \end{macro}
%
% \subsection{Parsing key value lists}
%
%    \begin{macro}{\kv@parse}
%    Normalizes and parses the key value list. Also sets \cs{kv@list}.
%    \begin{macrocode}
\long\def\kv@parse#1{%
  \kv@normalize{#1}%
  \expandafter\kv@parse@normalized\expandafter{\kv@list}%
}
%    \end{macrocode}
%    \end{macro}
%    \begin{macro}{\kv@parse@normalized}
%    |#1|: key value list\\
%    |#2|: processor
%    \begin{macrocode}
\long\def\kv@parse@normalized#1#2{%
  \KVS@Parse#1,\KVS@Nil{#2}%
}
%    \end{macrocode}
%    \end{macro}
%    \begin{macro}{\KVS@Parse}
%    |#1,#2|: key value list\\
%    |#3|: processor
%    \begin{macrocode}
\long\def\KVS@Parse#1,#2\KVS@Nil#3{%
  \KVS@IfEmpty{#1}{%
  }{%
    \KVS@Process#1=\KVS@Nil{#3}%
  }%
  \KVS@MaybeBreak
  \KVS@IfEmpty{#2}{%
  }{%
    \KVS@Parse#2\KVS@Nil{#3}%
  }%
}
%    \end{macrocode}
%    \end{macro}
%    \begin{macro}{\KVS@Process}
%    |#1|: key\\
%    |#2|: value, |=|\\
%    |#3|: processor
%    \begin{macrocode}
\long\def\KVS@Process#1=#2\KVS@Nil#3{%
  \let\KVS@MaybeBreak\relax
  \def\kv@key{#1}%
  \KVS@IfEmpty{#2}{%
    \let\kv@value\relax
    #3{#1}{}%
  }{%
    \KVS@@Process{#1}#2\KVS@Nil{#3}%
  }%
}
%    \end{macrocode}
%    \end{macro}
%    \begin{macro}{\KVS@@Process}
%    |#1|: key\\
%    |#2|: value\\
%    |#3|: processor
%    \begin{macrocode}
\long\def\KVS@@Process#1#2=\KVS@Nil#3{%
& \edef\kv@value{\unexpanded{#2}}%
$ \begingroup
$   \toks@{#2}%
$   \xdef\KVS@Global{\the\toks@}%
$ \endgroup
$ \let\kv@value\KVS@Global
  #3{#1}{#2}%
}
%    \end{macrocode}
%    \end{macro}
%    \begin{macro}{\KVS@MaybeBreak}
%    \begin{macrocode}
\let\KVS@MaybeBreak\relax
%    \end{macrocode}
%    \end{macro}
%    \begin{macro}{\KVS@break}
%    \begin{macrocode}
\def\KVS@break#1#2#3#4{%
  \let\KVS@MaybeBreak\relax
}
%    \end{macrocode}
%    \end{macro}
%    \begin{macro}{\kv@break}
%    \begin{macrocode}
\def\kv@break{%
  \let\KVS@MaybeBreak\KVS@break
}
%    \end{macrocode}
%    \end{macro}
%
% \subsection{Parsing comma lists}
%
%    \begin{macro}{\comma@parse}
%    Normalizes and parses the key value list. Also sets \cs{comma@list}.
%    \begin{macrocode}
\def\comma@parse#1{%
  \comma@normalize{#1}%
  \expandafter\comma@parse@normalized\expandafter{\comma@list}%
}
%    \end{macrocode}
%    \end{macro}
%
%    \begin{macro}{\comma@parse@normalized}
%    |#1|: comma list\\
%    |#2|: processor
%    \begin{macrocode}
\def\comma@parse@normalized#1#2{%
  \KVS@CommaParse#1,\KVS@Nil{#2}%
}
%    \end{macrocode}
%    \end{macro}
%    \begin{macro}{\KVS@CommaParse}
%    |#1,#2|: comma list\\
%    |#3|: processor
%    \begin{macrocode}
\def\KVS@CommaParse#1,#2\KVS@Nil#3{%
  \KVS@IfEmpty{#1}{%
  }{%
    \def\comma@entry{#1}%
    #3{#1}%
  }%
  \KVS@MaybeBreak
  \KVS@IfEmpty{#2}{%
  }{%
    \KVS@CommaParse#2\KVS@Nil{#3}%
  }%
}
%    \end{macrocode}
%    \end{macro}
%    \begin{macro}{\comma@break}
%    \begin{macrocode}
\def\comma@break{%
  \let\KVS@MaybeBreak\KVS@break
}
%    \end{macrocode}
%    \end{macro}
%
% \subsection{Processing key value pairs}
%
%    \begin{macro}{\kv@handled@false}
%    The handler can call \cs{kv@handled@false} or
%    \cs{kv@handled@true} so report failure or success.
%    The default is success (compatibility for versions
%    before 2011/10/18 v1.15).
%    \begin{macrocode}
\def\kv@handled@false{%
  \let\ifkv@handled@\iffalse
}
%    \end{macrocode}
%    \end{macro}
%    \begin{macro}{\kv@handled@true}
%    \begin{macrocode}
\def\kv@handled@true{%
  \let\ifkv@handled@\iftrue
}
%    \end{macrocode}
%    \end{macro}
%    \begin{macro}{\ifkv@handled@}
%    \begin{macrocode}
\kv@handled@true
%    \end{macrocode}
%    \end{macro}
%
%    \begin{macro}{\kv@processor@default}
%    \begin{macrocode}
\def\kv@processor@default#1#2{%
  \begingroup
    \csname @safe@activestrue\endcsname
    \let\ifincsname\iftrue
    \edef\KVS@temp{\endgroup
      \noexpand\KVS@ProcessorDefault{#1}{#2}%
    }%
  \KVS@temp
}
%    \end{macrocode}
%    \end{macro}
%    \begin{macro}{\KVS@ProcessorDefault}
%    \begin{macrocode}
\long\def\KVS@ProcessorDefault#1#2#3{%
  \def\kv@fam{#1}%
& \unless\ifcsname KV@#1@#2\endcsname
$ \begingroup\expandafter\expandafter\expandafter\endgroup
$ \expandafter\ifx\csname KV@#1@#2\endcsname\relax
&   \unless\ifcsname KVS@#1@handler\endcsname
$   \begingroup\expandafter\expandafter\expandafter\endgroup
$   \expandafter\ifx\csname KVS@#1@handler\endcsname\relax
      \kv@error@unknownkey{#1}{#2}%
    \else
      \kv@handled@true
      \csname KVS@#1@handler\endcsname{#2}{#3}%
      \relax
      \ifkv@handled@
      \else
        \kv@error@unknownkey{#1}{#2}%
      \fi
    \fi
  \else
    \ifx\kv@value\relax
&     \unless\ifcsname KV@#1@#2@default\endcsname
$     \begingroup\expandafter\expandafter\expandafter\endgroup
$     \expandafter\ifx\csname KV@#1@#2@default\endcsname\relax
        \kv@error@novalue{#1}{#2}%
      \else
        \csname KV@#1@#2@default\endcsname
        \relax
      \fi
    \else
      \csname KV@#1@#2\endcsname{#3}%
    \fi
  \fi
}
%    \end{macrocode}
%    \end{macro}
%
%    \begin{macro}{\kv@processor@known}
%    \begin{macrocode}
\def\kv@processor@known#1#2#3{%
  \begingroup
    \csname @safe@activestrue\endcsname
    \let\ifincsname\iftrue
    \edef\KVS@temp{\endgroup
      \noexpand\KVS@ProcessorKnown{#1}\noexpand#2{#3}%
    }%
  \KVS@temp
}
%    \end{macrocode}
%    \end{macro}
%    \begin{macro}{\KVS@ProcessorKnown}
%    \begin{macrocode}
\long\def\KVS@ProcessorKnown#1#2#3#4{%
  \def\kv@fam{#1}%
& \unless\ifcsname KV@#1@#3\endcsname
$ \begingroup\expandafter\expandafter\expandafter\endgroup
$ \expandafter\ifx\csname KV@#1@#3\endcsname\relax
&   \unless\ifcsname KVS@#1@handler\endcsname
$   \begingroup\expandafter\expandafter\expandafter\endgroup
$   \expandafter\ifx\csname KVS@#1@handler\endcsname\relax
      \KVS@AddUnhandled#2{#3}{#4}%
    \else
      \kv@handled@true
      \csname KVS@#1@handler\endcsname{#3}{#4}%
      \relax
      \ifkv@handled@
      \else
        \KVS@AddUnhandled#2{#3}{#4}%
      \fi
    \fi
  \else
    \ifx\kv@value\relax
&     \unless\ifcsname KV@#1@#2@default\endcsname
$     \begingroup\expandafter\expandafter\expandafter\endgroup
$     \expandafter\ifx\csname KV@#1@#3@default\endcsname\relax
        \kv@error@novalue{#1}{#3}%
      \else
        \csname KV@#1@#3@default\endcsname
        \relax
      \fi
    \else
      \csname KV@#1@#3\endcsname{#4}%
    \fi
  \fi
}
%    \end{macrocode}
%    \end{macro}
%    \begin{macro}{\KVS@AddUnhandled}
%    \begin{macrocode}
\long\def\KVS@AddUnhandled#1#2#3{%
& \edef#1{%
&   \ifx#1\KVS@empty
&   \else
&     \unexpanded{#1},%
&   \fi
&   \unexpanded{{#2}={#3}}%
& }%
$ \begingroup
$   \ifx#1\KVS@empty
$     \toks@{{#2}={#3}}%
$   \else
$     \toks@\expandafter{#1,{#2}={#3}}%
$   \fi
$   \xdef\KVS@Global{\the\toks@}%
$ \endgroup
$ \let#1\KVS@Global
}
%    \end{macrocode}
%    \end{macro}
%
%    \begin{macro}{\kv@set@family@handler}
%    \begin{macrocode}
\long\def\kv@set@family@handler#1#2{%
  \begingroup
    \csname @safe@activestrue\endcsname
    \let\ifincsname\iftrue
  \expandafter\endgroup
  \expandafter\def\csname KVS@#1@handler\endcsname##1##2{#2}%
}
%    \end{macrocode}
%    \end{macro}
%
%    \begin{macro}{\kv@unset@family@handler}
%    \begin{macrocode}
\long\def\kv@unset@family@handler#1#2{%
  \begingroup
    \csname @safe@activestrue\endcsname
    \let\ifincsname\iftrue
  \expandafter\endgroup
  \expandafter\let\csname KVS@#1@handler\endcsname\@UnDeFiNeD
}
%    \end{macrocode}
%    \end{macro}
%
% \subsection{Error handling}
%
%    \begin{macro}{\kv@error@novalue}
%    \begin{macrocode}
\def\kv@error@novalue{%
  \kv@error@generic{No value specified for}%
}
%    \end{macrocode}
%    \end{macro}
%    \begin{macro}{\kv@error@unknownkey}
%    \begin{macrocode}
\def\kv@error@unknownkey{%
  \kv@error@generic{Undefined}%
}
%    \end{macrocode}
%    \end{macro}
%    \begin{macro}{\kv@error@generic}
%    \begin{macrocode}
\def\kv@error@generic#1#2#3{%
  \PackageError{kvsetkeys}{%
    #1 key `#3'%
  }{%
    The keyval family of the key `#3' is `#2'.\MessageBreak
    The setting of the key is ignored because of the error.\MessageBreak
    \MessageBreak
    \@ehc
  }%
}
%    \end{macrocode}
%    \end{macro}
%
% \subsection{Do it all}
%
%    \begin{macro}{\kvsetkeys}
%    \begin{macrocode}
\long\def\kvsetkeys#1#2{%
  \kv@parse{#2}{\kv@processor@default{#1}}%
}
%    \end{macrocode}
%    \end{macro}
%
%    \begin{macro}{\kvsetkeys@expandafter}
%    \begin{macrocode}
\def\kvsetkeys@expandafter#1#2{%
  \expandafter\kv@parse\expandafter{#2}{%
    \kv@processor@default{#1}%
  }%
}
%    \end{macrocode}
%    \end{macro}
%
%    \begin{macro}{\KVS@cmd}
%    \begin{macrocode}
\def\KVS@cmd{0}%
%    \end{macrocode}
%    \end{macro}
%    \begin{macro}{\KVS@cmd@inc}
%    \begin{macrocode}
\def\KVS@cmd@inc{%
& \edef\KVS@cmd{\the\numexpr\KVS@cmd+1}%
$ \begingroup
$   \count255=\KVS@cmd\relax
$   \advance\count255 by 1\relax
$ \edef\x{\endgroup
$   \noexpand\def\noexpand\KVS@cmd{\number\count255}%
$ }%
$ \x
}
%    \end{macrocode}
%    \end{macro}
%    \begin{macro}{\KVS@cmd@dec}
%    \begin{macrocode}
\def\KVS@cmd@dec{%
& \edef\KVS@cmd{\the\numexpr\KVS@cmd-1}%
$ \begingroup
$   \count255=\KVS@cmd\relax
$   \advance\count255 by -1\relax
$ \edef\x{\endgroup
$   \noexpand\def\noexpand\KVS@cmd{\number\count255}%
$ }%
$ \x
}
%    \end{macrocode}
%    \end{macro}
%    \begin{macro}{\KVS@empty}
%    \begin{macrocode}
\def\KVS@empty{}
%    \end{macrocode}
%    \end{macro}
%
%    \begin{macro}{\kvsetknownkeys}
%    \begin{macrocode}
\def\kvsetknownkeys{%
  \expandafter
  \KVS@setknownkeys\csname KVS@cmd\KVS@cmd\endcsname{}%
}
%    \end{macrocode}
%    \end{macro}
%    \begin{macro}{\KVS@setknownkeys}
%    \begin{macrocode}
\long\def\KVS@setknownkeys#1#2#3#4#5{%
  \let#1\KVS@empty
  \KVS@cmd@inc
  #2\kv@parse#2{#5}{\kv@processor@known{#3}#1}%
  \KVS@cmd@dec
  \let#4=#1%
}
%    \end{macrocode}
%    \end{macro}
%
%    \begin{macro}{\kvsetknownkeys@expandafter}
%    \begin{macrocode}
\def\kvsetknownkeys@expandafter{%
  \expandafter
  \KVS@setknownkeys
      \csname KVS@cmd\KVS@cmd\endcsname\expandafter
}
%    \end{macrocode}
%    \end{macro}
%
%    \begin{macrocode}
\KVS@AtEnd%
%</package>
%    \end{macrocode}
%% \section{Installation}
%
% \subsection{Download}
%
% \paragraph{Package.} This package is available on
% CTAN\footnote{\CTANpkg{kvsetkeys}}:
% \begin{description}
% \item[\CTAN{macros/latex/contrib/kvsetkeys/kvsetkeys.dtx}] The source file.
% \item[\CTAN{macros/latex/contrib/kvsetkeys/kvsetkeys.pdf}] Documentation.
% \end{description}
%
%
% \paragraph{Bundle.} All the packages of the bundle `kvsetkeys'
% are also available in a TDS compliant ZIP archive. There
% the packages are already unpacked and the documentation files
% are generated. The files and directories obey the TDS standard.
% \begin{description}
% \item[\CTANinstall{install/macros/latex/contrib/kvsetkeys.tds.zip}]
% \end{description}
% \emph{TDS} refers to the standard ``A Directory Structure
% for \TeX\ Files'' (\CTANpkg{tds}). Directories
% with \xfile{texmf} in their name are usually organized this way.
%
% \subsection{Bundle installation}
%
% \paragraph{Unpacking.} Unpack the \xfile{kvsetkeys.tds.zip} in the
% TDS tree (also known as \xfile{texmf} tree) of your choice.
% Example (linux):
% \begin{quote}
%   |unzip kvsetkeys.tds.zip -d ~/texmf|
% \end{quote}
%
% \subsection{Package installation}
%
% \paragraph{Unpacking.} The \xfile{.dtx} file is a self-extracting
% \docstrip\ archive. The files are extracted by running the
% \xfile{.dtx} through \plainTeX:
% \begin{quote}
%   \verb|tex kvsetkeys.dtx|
% \end{quote}
%
% \paragraph{TDS.} Now the different files must be moved into
% the different directories in your installation TDS tree
% (also known as \xfile{texmf} tree):
% \begin{quote}
% \def\t{^^A
% \begin{tabular}{@{}>{\ttfamily}l@{ $\rightarrow$ }>{\ttfamily}l@{}}
%   kvsetkeys.sty & tex/generic/kvsetkeys/kvsetkeys.sty\\
%   kvsetkeys.pdf & doc/latex/kvsetkeys/kvsetkeys.pdf\\
%   kvsetkeys-example.tex & doc/latex/kvsetkeys/kvsetkeys-example.tex\\
%   kvsetkeys.dtx & source/latex/kvsetkeys/kvsetkeys.dtx\\
% \end{tabular}^^A
% }^^A
% \sbox0{\t}^^A
% \ifdim\wd0>\linewidth
%   \begingroup
%     \advance\linewidth by\leftmargin
%     \advance\linewidth by\rightmargin
%   \edef\x{\endgroup
%     \def\noexpand\lw{\the\linewidth}^^A
%   }\x
%   \def\lwbox{^^A
%     \leavevmode
%     \hbox to \linewidth{^^A
%       \kern-\leftmargin\relax
%       \hss
%       \usebox0
%       \hss
%       \kern-\rightmargin\relax
%     }^^A
%   }^^A
%   \ifdim\wd0>\lw
%     \sbox0{\small\t}^^A
%     \ifdim\wd0>\linewidth
%       \ifdim\wd0>\lw
%         \sbox0{\footnotesize\t}^^A
%         \ifdim\wd0>\linewidth
%           \ifdim\wd0>\lw
%             \sbox0{\scriptsize\t}^^A
%             \ifdim\wd0>\linewidth
%               \ifdim\wd0>\lw
%                 \sbox0{\tiny\t}^^A
%                 \ifdim\wd0>\linewidth
%                   \lwbox
%                 \else
%                   \usebox0
%                 \fi
%               \else
%                 \lwbox
%               \fi
%             \else
%               \usebox0
%             \fi
%           \else
%             \lwbox
%           \fi
%         \else
%           \usebox0
%         \fi
%       \else
%         \lwbox
%       \fi
%     \else
%       \usebox0
%     \fi
%   \else
%     \lwbox
%   \fi
% \else
%   \usebox0
% \fi
% \end{quote}
% If you have a \xfile{docstrip.cfg} that configures and enables \docstrip's
% TDS installing feature, then some files can already be in the right
% place, see the documentation of \docstrip.
%
% \subsection{Refresh file name databases}
%
% If your \TeX~distribution
% (\TeX\,Live, \mikTeX, \dots) relies on file name databases, you must refresh
% these. For example, \TeX\,Live\ users run \verb|texhash| or
% \verb|mktexlsr|.
%
% \subsection{Some details for the interested}
%
% \paragraph{Unpacking with \LaTeX.}
% The \xfile{.dtx} chooses its action depending on the format:
% \begin{description}
% \item[\plainTeX:] Run \docstrip\ and extract the files.
% \item[\LaTeX:] Generate the documentation.
% \end{description}
% If you insist on using \LaTeX\ for \docstrip\ (really,
% \docstrip\ does not need \LaTeX), then inform the autodetect routine
% about your intention:
% \begin{quote}
%   \verb|latex \let\install=y\input{kvsetkeys.dtx}|
% \end{quote}
% Do not forget to quote the argument according to the demands
% of your shell.
%
% \paragraph{Generating the documentation.}
% You can use both the \xfile{.dtx} or the \xfile{.drv} to generate
% the documentation. The process can be configured by the
% configuration file \xfile{ltxdoc.cfg}. For instance, put this
% line into this file, if you want to have A4 as paper format:
% \begin{quote}
%   \verb|\PassOptionsToClass{a4paper}{article}|
% \end{quote}
% An example follows how to generate the
% documentation with pdf\LaTeX:
% \begin{quote}
%\begin{verbatim}
%pdflatex kvsetkeys.dtx
%makeindex -s gind.ist kvsetkeys.idx
%pdflatex kvsetkeys.dtx
%makeindex -s gind.ist kvsetkeys.idx
%pdflatex kvsetkeys.dtx
%\end{verbatim}
% \end{quote}
%
% \begin{thebibliography}{9}
% \bibitem{tb94wright}
%   A guide to key-value methods, Joseph Wright, second draft for
%   \href{https://www.tug.org/tugboat}{TUGBoat}, 2009-03-17.
%   \url{https://www.texdev.net/uploads/2009/03/keyval.pdf}
%
% \bibitem{keyval}
%   David Carlisle:
%   \textit{The \xpackage{keyval} package};
%   1999/03/16 v1.13;
%   \CTANpkg{keyval}.
%
% \end{thebibliography}
%
% \begin{History}
%   \begin{Version}{2006/03/06 v1.0}
%   \item
%     First version.
%   \end{Version}
%   \begin{Version}{2006/10/19 v1.1}
%   \item
%     Fix of \cs{kv@set@family@handler}.
%   \item
%     Example added.
%   \end{Version}
%   \begin{Version}{2007/09/09 v1.2}
%   \item
%     Using package \xpackage{infwarerr} for error messages.
%   \item
%     Catcode section rewritten.
%   \end{Version}
%   \begin{Version}{2007/09/29 v1.3}
%   \item
%     Normalizing and parsing of comma separated lists added.
%   \item
%     \cs{kv@normalize} rewritten.
%   \item
%     Robustness increased for normalizing and parsing,
%     e.g. for values with unmatched conditionals.
%   \item
%     \eTeX\ is used if available.
%   \item
%     Tests added for normalizing and parsing.
%   \end{Version}
%   \begin{Version}{2009/07/19 v1.4}
%   \item
%     Bug fix for \cs{kv@normalize}: unwanted space removed (Florent Chervet).
%   \end{Version}
%   \begin{Version}{2009/07/30 v1.5}
%   \item
%     Documentation addition: recommendation for Joseph Wright's
%     review article.
%   \end{Version}
%   \begin{Version}{2009/12/12 v1.6}
%   \item
%     Short info shortened.
%   \end{Version}
%   \begin{Version}{2009/12/22 v1.7}
%   \item
%     Internal optimization (\cs{KVS@CommaSpace}, \dots, \cs{KVS@EqualsSpace}).
%   \end{Version}
%   \begin{Version}{2010/01/28 v1.8}
%   \item
%     Compatibility to ini\TeX\ added.
%   \end{Version}
%   \begin{Version}{2010/03/01 v1.9}
%   \item
%     Support of \cs{par} inside values.
%   \end{Version}
%   \begin{Version}{2011/01/30 v1.10}
%   \item
%     Already loaded package files are not input in \hologo{plainTeX}.
%   \end{Version}
%   \begin{Version}{2011/03/03 v1.11}
%   \item
%     \cs{kv@break} and \cs{comma@break} added.
%   \end{Version}
%   \begin{Version}{2011/04/05 v1.12}
%   \item
%     Error message with recovery action in help message
%     (request by GL).
%   \end{Version}
%   \begin{Version}{2011/04/07 v1.13}
%   \item
%     \cs{kv@processor@default} supports package \xpackage{babel}'s
%     shorthands.
%   \item
%     \cs{kv@set@family@handler} with shorthand support.
%   \end{Version}
%   \begin{Version}{2011/06/15 v1.14}
%   \item
%     Some optimizations in token register uses (GL, HO).
%   \end{Version}
%   \begin{Version}{2011/10/18 v1.15}
%   \item
%     \cs{kv@processor@known} and \cs{kvsetknownkeys} added.
%   \item
%     \cs{kvsetkeys@expandafter} and \cs{kvsetknownkeys@expandafter} added.
%   \item
%     Family handler can report success or failure by \cs{kv@handled@true}
%     or \cs{kv@handled@false}.
%   \item
%     \cs{kv@unset@family@handler} added.
%   \end{Version}
%   \begin{Version}{2012/04/25 v1.16}
%   \item
%     \cs{kv@processor@default} and \cs{kv@processor@known} define
%     macro \cs{kv@fam} for convenience.
%   \item
%     Catcode section: Catcode setting for \texttt{+} added for \hologo{eTeX}.
%   \end{Version}
%   \begin{Version}{2016/05/16 v1.17}
%   \item
%     Documentation updates.
%   \end{Version}
%   \begin{Version}{2019/12/15 v1.18}
%   \item
%     Documentation updates.
%   \item
%     Avoid \xpackage{etexcmds} and \xpackage{infwarwerr} in \LaTeX.
%   \end{Version}
% \end{History}
%
% \PrintIndex
%
% \Finale
\endinput
|
% \end{quote}
% Do not forget to quote the argument according to the demands
% of your shell.
%
% \paragraph{Generating the documentation.}
% You can use both the \xfile{.dtx} or the \xfile{.drv} to generate
% the documentation. The process can be configured by the
% configuration file \xfile{ltxdoc.cfg}. For instance, put this
% line into this file, if you want to have A4 as paper format:
% \begin{quote}
%   \verb|\PassOptionsToClass{a4paper}{article}|
% \end{quote}
% An example follows how to generate the
% documentation with pdf\LaTeX:
% \begin{quote}
%\begin{verbatim}
%pdflatex kvsetkeys.dtx
%makeindex -s gind.ist kvsetkeys.idx
%pdflatex kvsetkeys.dtx
%makeindex -s gind.ist kvsetkeys.idx
%pdflatex kvsetkeys.dtx
%\end{verbatim}
% \end{quote}
%
% \begin{thebibliography}{9}
% \bibitem{tb94wright}
%   A guide to key-value methods, Joseph Wright, second draft for
%   \href{https://www.tug.org/tugboat}{TUGBoat}, 2009-03-17.
%   \url{https://www.texdev.net/uploads/2009/03/keyval.pdf}
%
% \bibitem{keyval}
%   David Carlisle:
%   \textit{The \xpackage{keyval} package};
%   1999/03/16 v1.13;
%   \CTANpkg{keyval}.
%
% \end{thebibliography}
%
% \begin{History}
%   \begin{Version}{2006/03/06 v1.0}
%   \item
%     First version.
%   \end{Version}
%   \begin{Version}{2006/10/19 v1.1}
%   \item
%     Fix of \cs{kv@set@family@handler}.
%   \item
%     Example added.
%   \end{Version}
%   \begin{Version}{2007/09/09 v1.2}
%   \item
%     Using package \xpackage{infwarerr} for error messages.
%   \item
%     Catcode section rewritten.
%   \end{Version}
%   \begin{Version}{2007/09/29 v1.3}
%   \item
%     Normalizing and parsing of comma separated lists added.
%   \item
%     \cs{kv@normalize} rewritten.
%   \item
%     Robustness increased for normalizing and parsing,
%     e.g. for values with unmatched conditionals.
%   \item
%     \eTeX\ is used if available.
%   \item
%     Tests added for normalizing and parsing.
%   \end{Version}
%   \begin{Version}{2009/07/19 v1.4}
%   \item
%     Bug fix for \cs{kv@normalize}: unwanted space removed (Florent Chervet).
%   \end{Version}
%   \begin{Version}{2009/07/30 v1.5}
%   \item
%     Documentation addition: recommendation for Joseph Wright's
%     review article.
%   \end{Version}
%   \begin{Version}{2009/12/12 v1.6}
%   \item
%     Short info shortened.
%   \end{Version}
%   \begin{Version}{2009/12/22 v1.7}
%   \item
%     Internal optimization (\cs{KVS@CommaSpace}, \dots, \cs{KVS@EqualsSpace}).
%   \end{Version}
%   \begin{Version}{2010/01/28 v1.8}
%   \item
%     Compatibility to ini\TeX\ added.
%   \end{Version}
%   \begin{Version}{2010/03/01 v1.9}
%   \item
%     Support of \cs{par} inside values.
%   \end{Version}
%   \begin{Version}{2011/01/30 v1.10}
%   \item
%     Already loaded package files are not input in \hologo{plainTeX}.
%   \end{Version}
%   \begin{Version}{2011/03/03 v1.11}
%   \item
%     \cs{kv@break} and \cs{comma@break} added.
%   \end{Version}
%   \begin{Version}{2011/04/05 v1.12}
%   \item
%     Error message with recovery action in help message
%     (request by GL).
%   \end{Version}
%   \begin{Version}{2011/04/07 v1.13}
%   \item
%     \cs{kv@processor@default} supports package \xpackage{babel}'s
%     shorthands.
%   \item
%     \cs{kv@set@family@handler} with shorthand support.
%   \end{Version}
%   \begin{Version}{2011/06/15 v1.14}
%   \item
%     Some optimizations in token register uses (GL, HO).
%   \end{Version}
%   \begin{Version}{2011/10/18 v1.15}
%   \item
%     \cs{kv@processor@known} and \cs{kvsetknownkeys} added.
%   \item
%     \cs{kvsetkeys@expandafter} and \cs{kvsetknownkeys@expandafter} added.
%   \item
%     Family handler can report success or failure by \cs{kv@handled@true}
%     or \cs{kv@handled@false}.
%   \item
%     \cs{kv@unset@family@handler} added.
%   \end{Version}
%   \begin{Version}{2012/04/25 v1.16}
%   \item
%     \cs{kv@processor@default} and \cs{kv@processor@known} define
%     macro \cs{kv@fam} for convenience.
%   \item
%     Catcode section: Catcode setting for \texttt{+} added for \hologo{eTeX}.
%   \end{Version}
%   \begin{Version}{2016/05/16 v1.17}
%   \item
%     Documentation updates.
%   \end{Version}
%   \begin{Version}{2019/12/15 v1.18}
%   \item
%     Documentation updates.
%   \item
%     Avoid \xpackage{etexcmds} and \xpackage{infwarwerr} in \LaTeX.
%   \end{Version}
% \end{History}
%
% \PrintIndex
%
% \Finale
\endinput
|
% \end{quote}
% Do not forget to quote the argument according to the demands
% of your shell.
%
% \paragraph{Generating the documentation.}
% You can use both the \xfile{.dtx} or the \xfile{.drv} to generate
% the documentation. The process can be configured by the
% configuration file \xfile{ltxdoc.cfg}. For instance, put this
% line into this file, if you want to have A4 as paper format:
% \begin{quote}
%   \verb|\PassOptionsToClass{a4paper}{article}|
% \end{quote}
% An example follows how to generate the
% documentation with pdf\LaTeX:
% \begin{quote}
%\begin{verbatim}
%pdflatex kvsetkeys.dtx
%makeindex -s gind.ist kvsetkeys.idx
%pdflatex kvsetkeys.dtx
%makeindex -s gind.ist kvsetkeys.idx
%pdflatex kvsetkeys.dtx
%\end{verbatim}
% \end{quote}
%
% \begin{thebibliography}{9}
% \bibitem{tb94wright}
%   A guide to key-value methods, Joseph Wright, second draft for
%   \href{https://www.tug.org/tugboat}{TUGBoat}, 2009-03-17.
%   \url{https://www.texdev.net/uploads/2009/03/keyval.pdf}
%
% \bibitem{keyval}
%   David Carlisle:
%   \textit{The \xpackage{keyval} package};
%   1999/03/16 v1.13;
%   \CTANpkg{keyval}.
%
% \end{thebibliography}
%
% \begin{History}
%   \begin{Version}{2006/03/06 v1.0}
%   \item
%     First version.
%   \end{Version}
%   \begin{Version}{2006/10/19 v1.1}
%   \item
%     Fix of \cs{kv@set@family@handler}.
%   \item
%     Example added.
%   \end{Version}
%   \begin{Version}{2007/09/09 v1.2}
%   \item
%     Using package \xpackage{infwarerr} for error messages.
%   \item
%     Catcode section rewritten.
%   \end{Version}
%   \begin{Version}{2007/09/29 v1.3}
%   \item
%     Normalizing and parsing of comma separated lists added.
%   \item
%     \cs{kv@normalize} rewritten.
%   \item
%     Robustness increased for normalizing and parsing,
%     e.g. for values with unmatched conditionals.
%   \item
%     \eTeX\ is used if available.
%   \item
%     Tests added for normalizing and parsing.
%   \end{Version}
%   \begin{Version}{2009/07/19 v1.4}
%   \item
%     Bug fix for \cs{kv@normalize}: unwanted space removed (Florent Chervet).
%   \end{Version}
%   \begin{Version}{2009/07/30 v1.5}
%   \item
%     Documentation addition: recommendation for Joseph Wright's
%     review article.
%   \end{Version}
%   \begin{Version}{2009/12/12 v1.6}
%   \item
%     Short info shortened.
%   \end{Version}
%   \begin{Version}{2009/12/22 v1.7}
%   \item
%     Internal optimization (\cs{KVS@CommaSpace}, \dots, \cs{KVS@EqualsSpace}).
%   \end{Version}
%   \begin{Version}{2010/01/28 v1.8}
%   \item
%     Compatibility to ini\TeX\ added.
%   \end{Version}
%   \begin{Version}{2010/03/01 v1.9}
%   \item
%     Support of \cs{par} inside values.
%   \end{Version}
%   \begin{Version}{2011/01/30 v1.10}
%   \item
%     Already loaded package files are not input in \hologo{plainTeX}.
%   \end{Version}
%   \begin{Version}{2011/03/03 v1.11}
%   \item
%     \cs{kv@break} and \cs{comma@break} added.
%   \end{Version}
%   \begin{Version}{2011/04/05 v1.12}
%   \item
%     Error message with recovery action in help message
%     (request by GL).
%   \end{Version}
%   \begin{Version}{2011/04/07 v1.13}
%   \item
%     \cs{kv@processor@default} supports package \xpackage{babel}'s
%     shorthands.
%   \item
%     \cs{kv@set@family@handler} with shorthand support.
%   \end{Version}
%   \begin{Version}{2011/06/15 v1.14}
%   \item
%     Some optimizations in token register uses (GL, HO).
%   \end{Version}
%   \begin{Version}{2011/10/18 v1.15}
%   \item
%     \cs{kv@processor@known} and \cs{kvsetknownkeys} added.
%   \item
%     \cs{kvsetkeys@expandafter} and \cs{kvsetknownkeys@expandafter} added.
%   \item
%     Family handler can report success or failure by \cs{kv@handled@true}
%     or \cs{kv@handled@false}.
%   \item
%     \cs{kv@unset@family@handler} added.
%   \end{Version}
%   \begin{Version}{2012/04/25 v1.16}
%   \item
%     \cs{kv@processor@default} and \cs{kv@processor@known} define
%     macro \cs{kv@fam} for convenience.
%   \item
%     Catcode section: Catcode setting for \texttt{+} added for \hologo{eTeX}.
%   \end{Version}
%   \begin{Version}{2016/05/16 v1.17}
%   \item
%     Documentation updates.
%   \end{Version}
%   \begin{Version}{2019/12/15 v1.18}
%   \item
%     Documentation updates.
%   \item
%     Avoid \xpackage{etexcmds} and \xpackage{infwarwerr} in \LaTeX.
%   \end{Version}
% \end{History}
%
% \PrintIndex
%
% \Finale
\endinput
|
% \end{quote}
% Do not forget to quote the argument according to the demands
% of your shell.
%
% \paragraph{Generating the documentation.}
% You can use both the \xfile{.dtx} or the \xfile{.drv} to generate
% the documentation. The process can be configured by the
% configuration file \xfile{ltxdoc.cfg}. For instance, put this
% line into this file, if you want to have A4 as paper format:
% \begin{quote}
%   \verb|\PassOptionsToClass{a4paper}{article}|
% \end{quote}
% An example follows how to generate the
% documentation with pdf\LaTeX:
% \begin{quote}
%\begin{verbatim}
%pdflatex kvsetkeys.dtx
%makeindex -s gind.ist kvsetkeys.idx
%pdflatex kvsetkeys.dtx
%makeindex -s gind.ist kvsetkeys.idx
%pdflatex kvsetkeys.dtx
%\end{verbatim}
% \end{quote}
%
% \begin{thebibliography}{9}
% \bibitem{tb94wright}
%   A guide to key-value methods, Joseph Wright, second draft for
%   \href{https://www.tug.org/tugboat}{TUGBoat}, 2009-03-17.
%   \url{https://www.texdev.net/uploads/2009/03/keyval.pdf}
%
% \bibitem{keyval}
%   David Carlisle:
%   \textit{The \xpackage{keyval} package};
%   1999/03/16 v1.13;
%   \CTANpkg{keyval}.
%
% \end{thebibliography}
%
% \begin{History}
%   \begin{Version}{2006/03/06 v1.0}
%   \item
%     First version.
%   \end{Version}
%   \begin{Version}{2006/10/19 v1.1}
%   \item
%     Fix of \cs{kv@set@family@handler}.
%   \item
%     Example added.
%   \end{Version}
%   \begin{Version}{2007/09/09 v1.2}
%   \item
%     Using package \xpackage{infwarerr} for error messages.
%   \item
%     Catcode section rewritten.
%   \end{Version}
%   \begin{Version}{2007/09/29 v1.3}
%   \item
%     Normalizing and parsing of comma separated lists added.
%   \item
%     \cs{kv@normalize} rewritten.
%   \item
%     Robustness increased for normalizing and parsing,
%     e.g. for values with unmatched conditionals.
%   \item
%     \eTeX\ is used if available.
%   \item
%     Tests added for normalizing and parsing.
%   \end{Version}
%   \begin{Version}{2009/07/19 v1.4}
%   \item
%     Bug fix for \cs{kv@normalize}: unwanted space removed (Florent Chervet).
%   \end{Version}
%   \begin{Version}{2009/07/30 v1.5}
%   \item
%     Documentation addition: recommendation for Joseph Wright's
%     review article.
%   \end{Version}
%   \begin{Version}{2009/12/12 v1.6}
%   \item
%     Short info shortened.
%   \end{Version}
%   \begin{Version}{2009/12/22 v1.7}
%   \item
%     Internal optimization (\cs{KVS@CommaSpace}, \dots, \cs{KVS@EqualsSpace}).
%   \end{Version}
%   \begin{Version}{2010/01/28 v1.8}
%   \item
%     Compatibility to ini\TeX\ added.
%   \end{Version}
%   \begin{Version}{2010/03/01 v1.9}
%   \item
%     Support of \cs{par} inside values.
%   \end{Version}
%   \begin{Version}{2011/01/30 v1.10}
%   \item
%     Already loaded package files are not input in \hologo{plainTeX}.
%   \end{Version}
%   \begin{Version}{2011/03/03 v1.11}
%   \item
%     \cs{kv@break} and \cs{comma@break} added.
%   \end{Version}
%   \begin{Version}{2011/04/05 v1.12}
%   \item
%     Error message with recovery action in help message
%     (request by GL).
%   \end{Version}
%   \begin{Version}{2011/04/07 v1.13}
%   \item
%     \cs{kv@processor@default} supports package \xpackage{babel}'s
%     shorthands.
%   \item
%     \cs{kv@set@family@handler} with shorthand support.
%   \end{Version}
%   \begin{Version}{2011/06/15 v1.14}
%   \item
%     Some optimizations in token register uses (GL, HO).
%   \end{Version}
%   \begin{Version}{2011/10/18 v1.15}
%   \item
%     \cs{kv@processor@known} and \cs{kvsetknownkeys} added.
%   \item
%     \cs{kvsetkeys@expandafter} and \cs{kvsetknownkeys@expandafter} added.
%   \item
%     Family handler can report success or failure by \cs{kv@handled@true}
%     or \cs{kv@handled@false}.
%   \item
%     \cs{kv@unset@family@handler} added.
%   \end{Version}
%   \begin{Version}{2012/04/25 v1.16}
%   \item
%     \cs{kv@processor@default} and \cs{kv@processor@known} define
%     macro \cs{kv@fam} for convenience.
%   \item
%     Catcode section: Catcode setting for \texttt{+} added for \hologo{eTeX}.
%   \end{Version}
%   \begin{Version}{2016/05/16 v1.17}
%   \item
%     Documentation updates.
%   \end{Version}
%   \begin{Version}{2019/12/15 v1.18}
%   \item
%     Documentation updates.
%   \item
%     Avoid \xpackage{etexcmds} and \xpackage{infwarwerr} in \LaTeX.
%   \end{Version}
% \end{History}
%
% \PrintIndex
%
% \Finale
\endinput
