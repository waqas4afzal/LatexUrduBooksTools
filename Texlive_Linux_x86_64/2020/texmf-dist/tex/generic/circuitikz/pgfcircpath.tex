% Copyright 2018-2021 by Romano Giannetti
% Copyright 2015-2021 by Stefan Lindner
% Copyright 2013-2021 by Stefan Erhardt
% Copyright 2007-2021 by Massimo Redaelli
%
% This file may be distributed and/or modified
%
% 1. under the LaTeX Project Public License and/or
% 2. under the GNU Public License.
%
% See the files gpl-3.0_license.txt and lppl-1-3c_license.txt for more details.


\def\pgf@circ@direction{0.0}

% swap two coordinates
\def\pgfcirc@swap@coordinates#1#2{%
    coordinate (pgfcirc@tmp@swap) at (#1)
    coordinate (#1) at (#2)
    coordinate (#2) at (pgfcirc@tmp@swap)
}

% Names
\ctikzset{name/.style = { n={#1} } } %%%%%%%%%%%%%%%%%%%%%%%%%%%%%%%%%%%@@@
\ctikzset{n/.code = {
	\pgfkeys{/tikz/circuitikz/bipole/name={#1}}
}}

% Reflect the node along
\ctikzset{mirrored/.is choice}
\ctikzset{mirror value/.initial=1}
\ctikzset{mirrored/true/.code = {\ctikzsetvalof{mirror value}{-1}} }
\ctikzset{mirrored/false/.code = {\ctikzsetvalof{mirror value}{1}} }
\ctikzset{mirror/.style = {/tikz/circuitikz/mirrored=true}}

% Invert node along path
\ctikzset{inverted/.is choice}
\ctikzset{invert value/.initial=1}
\ctikzset{inverted/true/.code = {\ctikzsetvalof{invert value}{-1}\pgf@circuit@bipole@invertedtrue}}
\ctikzset{inverted/false/.code = {\ctikzsetvalof{invert value}{1}\pgf@circuit@bipole@invertedfalse}}
\ctikzset{invert/.style = {/tikz/circuitikz/inverted=true}}
\newif\ifpgf@circuit@bipole@inverted
\ctikzset{bipole/inverted/.is if=pgf@circuit@bipole@inverted}

\newif\ifpgf@circuit@bipole@voltage@backward
\ctikzset{bipole/voltage/direction/.is choice}
\ctikzset{bipole/voltage/direction/forward/.code={\pgf@circuit@bipole@voltage@backwardfalse}}
\ctikzset{bipole/voltage/direction/backward/.code={\pgf@circuit@bipole@voltage@backwardtrue}}

% Initialize paths
\def\pgfcircresetpath{
    \ctikzset{bipole/name=, bipole/label/name=, bipole/label/position=90, ,bipole/annotation/name=, bipole/annotation/position=-90,
        bipole/inverted=false, bipole/kind=,
        bipole/voltage/direction=backward, bipole/voltage/label/name=, bipole/voltage/position=below,
        bipole/nodes/left=none, bipole/nodes/right=none, bipole/is voltage=false,bipole/is voltageoutsideofsymbol=false,bipole/is strokedsymbol=false,
        bipole/is current=false, bipole/current/label/name=, bipole/current/x position=after,
        bipole/current/y position=above, bipole/current/direction=forward,
        mirrored=false
    }
}

%
% expandable IF for the extra nodes (thanks to Henri Menke)
% see https://chat.stackexchange.com/transcript/message/56560808#56560808
%
\def\pgfcirc@if@has@i{%
    \ifpgfcirc@has@i
        \expandafter\pgfutil@firstoftwo
    \else
        \expandafter\pgfutil@secondoftwo
    \fi}
\def\pgfcirc@if@has@v{%
    \ifpgfcirc@has@v
        \expandafter\pgfutil@firstoftwo
    \else
        \expandafter\pgfutil@secondoftwo
    \fi}
\def\pgfcirc@if@has@f{%
    \ifpgfcirc@has@f
        \expandafter\pgfutil@firstoftwo
    \else
        \expandafter\pgfutil@secondoftwo
    \fi}



%% Generic bipole path
\def\pgf@circ@bipole@path#1#2{
    % Create a bipole path from the shapes defined with \pgfcircdeclarebipole
    % or \pgfcircdeclarebipolescaled; the node shapes are named with a "shape"
    % appended to the main (path-style) name
    % #1 path-style node name
    % #2 the argument passed from the to-path structure; don't touch
    %
    % Example:
    % \def\pgf@circ@capacitor@path#1{\pgf@circ@bipole@path{capacitor}{#1}}
    %
    \pgf@circ@bipole@path@base{shape}{}{#1}{#2}
}
%%
%% ultra-generic bipole path
%% I am not sure what the last argument is needed for, but don't touch it or everything explodes
%%
\def\pgf@circ@bipole@path@base#1#2#3#4{%
    %
    % Create a path-style component based on a node-style shape
    % #1: postfix to be added to the name path to obtain the main shape name
    % #2: text to be passed as text to the node
    % #3: name of the bipole component
    % #4: this will be filled by the argument of the to-path
    %
    \pgfextra{
        \ctikzset{bipole/kind = #3}
        \edef\pgf@temp{\ctikzvalof{bipole/name}}
        \def\pgf@circ@temp{}
        \ifx\pgf@temp\pgf@circ@temp % if it has not a name
            \pgfmathrandominteger{\pgf@circ@rand}{1000}{9999}
            \ctikzset{bipole/name = pgfcirc@#3\pgf@circ@rand} % create it (re-usage should not create problem, but...)
            \edef\pgfcirc@a@prefix{pgfcirc}% do not pollute the namespace for nothing
        \else
            \edef\pgfcirc@a@prefix{\ctikzvalof{bipole/name}}% for exporting v-i-f anchors
        \fi
    }
    % save start and stop values
    % notice that we DO NOT MOVE the path position at all!
    coordinate (\ctikzvalof{bipole/name}start) at (\tikztostart)
    coordinate (\ctikzvalof{bipole/name}end) at (\tikztotarget)
    \pgfextra{
        % find the direction (angle) of the path
        \pgfmathanglebetweenpoints{\pgfpointanchor{\ctikzvalof{bipole/name}start}{center}}
            {\pgfpointanchor{\ctikzvalof{bipole/name}end}{center}}
        \edef\pgf@circ@direction{\pgfmathresult}
        \expandafter\xdef\csname pgfcirc@\pgfcirc@a@prefix-direction\endcsname{\pgf@circ@direction}
    }
    % position the component in the middle of the path. We DO NOT MOVE the current position!
    node[#3#1, rotate=\pgf@circ@direction, yscale=\ctikzvalof{mirror value},
        xscale=\ctikzvalof{invert value}] (\ctikzvalof{bipole/name})
        at ($(\tikztostart) ! .5 ! (\tikztotarget)$) {#2}
    % set start and end labels
    \ifpgf@circuit@bipole@inverted
        \ifcsname pgf@anchor@#3#1@pathstart\endcsname%if special path-anchors are defined, use them!
            coordinate	(pgfcirc@anchorstartnode) at (\ctikzvalof{bipole/name}.pathend)
            coordinate	(pgfcirc@anchorendnode) at (\ctikzvalof{bipole/name}.pathstart)
        \else
            coordinate	(pgfcirc@anchorstartnode) at (\ctikzvalof{bipole/name}.right)
            coordinate	(pgfcirc@anchorendnode) at (\ctikzvalof{bipole/name}.left)
        \fi
        \else
        \ifcsname pgf@anchor@#3#1@pathstart\endcsname%if special path-anchors are defined, use them!
            coordinate	(pgfcirc@anchorstartnode) at (\ctikzvalof{bipole/name}.pathstart)
            coordinate	(pgfcirc@anchorendnode) at (\ctikzvalof{bipole/name}.pathend)
        \else
            coordinate	(pgfcirc@anchorstartnode) at (\ctikzvalof{bipole/name}.left)
            coordinate	(pgfcirc@anchorendnode) at (\ctikzvalof{bipole/name}.right)
        \fi
    \fi
    % draw the leads unless it's an open circuit
    % stop at the component
    \pgfextra{\def\pgf@temp{open}\def\pgf@circ@temp{#3}}
    \ifx\pgf@temp\pgf@circ@temp  % if it is an open do nothing
    \else
        % it is important to start the path with -- to have correct line joins!
        -- (\tikztostart) -- (pgfcirc@anchorstartnode)
    \fi
    % Add all the "ornaments": labels, annotations, voltages, currents and flows
    \drawpoles
    \pgf@circ@ifkeyempty{bipole/label/name}\else\pgf@circ@drawlabels{label}\fi
    \pgf@circ@ifkeyempty{bipole/annotation/name}\else\pgf@circ@drawlabels{annotation}\fi
    % the following  must be made in their own path scope to avoid crash in TikZ 3.1.8/3.1.8a
    % it should be logically safe for older version too --- even if TikZ reverted the change
    % use explandable ifs too, thanks to Henri Menke
    {\pgfcirc@if@has@v{\pgf@circ@drawvoltage}{}}%
    {\pgfcirc@if@has@i{\pgf@circ@drawcurrent}{}}%
    {\pgfcirc@if@has@f{\pgf@circ@drawflow}{}}%
    % finish the path from the component to the final target
    % you never know --- re-set \pgf@temp to detect open
    \pgfextra{\def\pgf@temp{open}\def\pgf@circ@temp{#3}}
    \ifx\pgf@temp\pgf@circ@temp  % if it is an open do nothing
        (\tikztotarget)
    \else
        (pgfcirc@anchorendnode)  -- (\tikztotarget)
    \fi
    % reset internal circuit keys
    \pgfextra{\pgfcircresetpath}
    %draw pending nodes an path
    \tikztonodes
}

%% Macros for path and style activation for bipoles or path-style

\def\comnpatname{\ifpgf@circuit@compat *\else\fi}
\def\compattikzset#1{%
    % \typeout{BIPOLEDEF:\space \detokenize{#1}}%
    \tikzset{\comnpatname#1}}
%
% this is used for components that are mainly node-style but have a path-style form
%
\def\pgfcirc@node@to@path#1#2#3{%
    % add a path-style component based on a node-style one without mangling the name
    % of the shape.
    % #1: node-type shape name (existing)
    % #2: path-type name (to be created)
    % #3: additional options to add to the path style
    %
    \expandafter\def\csname pgf@circ@#1@path\endcsname##1{\pgf@circ@bipole@path@base{}{##1}{#1}{}}%
    \compattikzset{#2/.style = {\circuitikzbasekey,
        /tikz/to path=\csname pgf@circ@#1@path\endcsname{##1},
        #3}}%
    \ctikzset{bipoles/#1/height/.initial=1}%
}
%
% this one is for normal definition: path to style, directly
% the first parameter (#1) here is l,v,i (l=..., v=..., i=...)
% the last parameter are options to be inserted in the "to path" definition
%
\def\pgfcirc@path@to@style#1#2#3#4{% using #1 as label, assign \pgf@circ@#2@path to style #3
    \compattikzset{#3/.style={\circuitikzbasekey, #4, /tikz/to path=\csname pgf@circ@#2@path\endcsname, #1={##1}}}%
}
% this one create a alias style from a node definition
\def\pgfcirc@node@to@style#1#2#3#4{% using #1 as label, assign \pgf@circ@bipole@path{#2} to style #3
    \compattikzset{#3/.style={\circuitikzbasekey, #4, /tikz/to path=\pgf@circ@bipole@path{#2}, #1={##1}}}%
}
% this create an alias style
\def\pgfcirc@style@to@style#1#2{% alias style #1 to style #2
    \compattikzset{#2/.style={\comnpatname #1={##1}}}%
}
% this create an alias style, changing the labelling
\def\pgfcirc@style@to@style@label#1#2#3{% alias style #1 to style #2
    \compattikzset{#2/.style={\comnpatname #1, #3={##1}}}%
}
% create a bipole
\def\pgfcirc@activate@bipole#1#2#3#4{% path name, base node name, style name
    \expandafter\def\csname pgf@circ@#2@path\endcsname##1{\pgf@circ@bipole@path{#3}{##1}}%
    \pgfcirc@path@to@style{#1}{#2}{#4}{}% no options here, let's see
}
\def\pgfcirc@activate@bipole@simple#1#2{\pgfcirc@activate@bipole{#1}{#2}{#2}{#2}}
% create a bipole with options
\def\pgfcirc@activate@bipole@opt#1#2#3#4#5{% path name, base node name, style name
    \expandafter\def\csname pgf@circ@#2@path\endcsname##1{\pgf@circ@bipole@path{#3}{##1}}%
    \pgfcirc@path@to@style{#1}{#2}{#4}{#5}% no options here, let's see
}
\def\pgfcirc@activate@bipole@simple@opt#1#2#3{\pgfcirc@activate@bipole@opt{#1}{#2}{#2}{#2}{#3}}


%% New system, for simple object
%% \pgfcirc@activate@bipole@simple{l}{mass}
%% New system, different names
%% The old system is the following
%% 1 - define just the pgf@circ@path@whatever#1
%% (see for example the variable one)
%% 2 - set the style
%% \compattikzset{resistor/.style= {\circuitikzbasekey, /tikz/to path=\pgf@circ@resistor@path, l={#1}}}

%% Path definition with the new mechanism have been moved to where the nodes
%% are defined.

%% Handling of terminals%<<<

\ctikzset{bipole/nodes/.is family}
\ctikzset{bipole/nodes/left/.initial=none}
\ctikzset{bipole/nodes/right/.initial=none}
\tikzset{bipole nodes/.style n args={2}{%
    \circuitikzbasekey/bipole/nodes/left=#1,
    \circuitikzbasekey/bipole/nodes/right=#2
    }
}

%% Easily usable styles

\ctikzset{o-o/.style = {\circuitikzbasekey/bipole/nodes/left=ocirc, \circuitikzbasekey/bipole/nodes/right=ocirc}}
\ctikzset{-o/.style = {\circuitikzbasekey/bipole/nodes/left=none, \circuitikzbasekey/bipole/nodes/right=ocirc}}
\ctikzset{o-/.style = {\circuitikzbasekey/bipole/nodes/left=ocirc, \circuitikzbasekey/bipole/nodes/right=none}}
\ctikzset{*-o/.style = {\circuitikzbasekey/bipole/nodes/left=circ, \circuitikzbasekey/bipole/nodes/right=ocirc}}
\ctikzset{o-*/.style = {\circuitikzbasekey/bipole/nodes/left=ocirc, \circuitikzbasekey/bipole/nodes/right=circ}}
\ctikzset{d-o/.style = {\circuitikzbasekey/bipole/nodes/left=diamondpole, \circuitikzbasekey/bipole/nodes/right=ocirc}}
\ctikzset{o-d/.style = {\circuitikzbasekey/bipole/nodes/left=ocirc, \circuitikzbasekey/bipole/nodes/right=diamondpole}}
\ctikzset{*-/.style = {\circuitikzbasekey/bipole/nodes/left=circ, \circuitikzbasekey/bipole/nodes/right=none}}
\ctikzset{-*/.style = {\circuitikzbasekey/bipole/nodes/left=none, \circuitikzbasekey/bipole/nodes/right=circ}}
\ctikzset{d-/.style = {\circuitikzbasekey/bipole/nodes/left=diamondpole, \circuitikzbasekey/bipole/nodes/right=none}}
\ctikzset{-d/.style = {\circuitikzbasekey/bipole/nodes/left=none, \circuitikzbasekey/bipole/nodes/right=diamondpole}}
\ctikzset{*-*/.style = {\circuitikzbasekey/bipole/nodes/left=circ, \circuitikzbasekey/bipole/nodes/right=circ}}
\ctikzset{d-*/.style = {\circuitikzbasekey/bipole/nodes/left=diamondpole, \circuitikzbasekey/bipole/nodes/right=circ}}
\ctikzset{*-d/.style = {\circuitikzbasekey/bipole/nodes/left=circ, \circuitikzbasekey/bipole/nodes/right=diamondpole}}
\ctikzset{d-d/.style = {\circuitikzbasekey/bipole/nodes/left=diamondpole, \circuitikzbasekey/bipole/nodes/right=diamondpole}}

% rectjoinfill workarounds

\ctikzset{.-/.style = {\circuitikzbasekey/bipole/nodes/left=rectjoinfill, \circuitikzbasekey/bipole/nodes/right=none}}
\ctikzset{.-*/.style = {\circuitikzbasekey/bipole/nodes/left=rectjoinfill, \circuitikzbasekey/bipole/nodes/right=circ}}
\ctikzset{.-o/.style = {\circuitikzbasekey/bipole/nodes/left=rectjoinfill, \circuitikzbasekey/bipole/nodes/right=ocirc}}
\ctikzset{.-d/.style = {\circuitikzbasekey/bipole/nodes/left=rectjoinfill, \circuitikzbasekey/bipole/nodes/right=diamondpole}}
\ctikzset{-./.style = {\circuitikzbasekey/bipole/nodes/left=none, \circuitikzbasekey/bipole/nodes/right=rectjoinfill}}
\ctikzset{o-./.style = {\circuitikzbasekey/bipole/nodes/left=ocirc, \circuitikzbasekey/bipole/nodes/right=rectjoinfill}}
\ctikzset{*-./.style = {\circuitikzbasekey/bipole/nodes/left=circ, \circuitikzbasekey/bipole/nodes/right=rectjoinfill}}
\ctikzset{d-./.style = {\circuitikzbasekey/bipole/nodes/left=diamondpole, \circuitikzbasekey/bipole/nodes/right=rectjoinfill}}
\ctikzset{.-./.style = {\circuitikzbasekey/bipole/nodes/left=rectjoinfill, \circuitikzbasekey/bipole/nodes/right=rectjoinfill}}

\tikzset{reversed/.style = {\circuitikzbasekey/bipole/inverted=true}}

\def\drawpoles{
    \pgfextra{ \edef\pgf@circ@temp{\ctikzvalof{bipole/nodes/left}} \def\pgf@temp{none}}
    \ifx\pgf@temp\pgf@circ@temp\else(\tikztostart) node[\pgf@circ@temp] {}\fi
    \pgfextra{ \edef\pgf@circ@temp{\ctikzvalof{bipole/nodes/right}} }
    \ifx\pgf@temp\pgf@circ@temp\else(\tikztotarget) node[\pgf@circ@temp] {}\fi
}
% %>>>

%%
%% Definition of path for transistors
%%
% Transistor like bipoles

\def\pgf@circ@trans@path#1#2{
    \pgfextra{
        \edef\pgf@temp{\ctikzvalof{bipole/name}}
        \def\pgf@circ@temp{#2}
        \ifx\pgf@temp\pgf@circ@temp % if it has not a name
            \pgfmathrandominteger{\pgf@circ@rand}{1000}{9999}
            \ctikzset{bipole/name = trans\pgf@circ@rand} % create it
        \fi
    }
    \ifpgf@circuit@bipole@inverted
        (\tikztostart) node[coordinate] (\ctikzvalof{bipole/name}end) {}
        (\tikztotarget) node[coordinate] (\ctikzvalof{bipole/name}start) {}
    \else
        (\tikztostart) node[coordinate] (\ctikzvalof{bipole/name}start) {}
        (\tikztotarget) node[coordinate] (\ctikzvalof{bipole/name}end) {}
    \fi
    \pgfextra{
        \pgfmathanglebetweenpoints{\pgfpointanchor{\ctikzvalof{bipole/name}start}{center}}
        {\pgfpointanchor{\ctikzvalof{bipole/name}end}{center}}
        \pgfmathadd{\pgfmathresult}{-90}
        \pgfmathround{\pgfmathresult}
        \edef\pgf@circ@direction{\pgfmathresult}
    }
    ($(\tikztostart) ! .5 ! (\tikztotarget)$)
    node[#1, /tikz/rotate=\pgf@circ@direction, xscale=\ctikzvalof{mirror value}]
    (\ctikzvalof{bipole/name}) {}
    node {\ctikzvalof{bipole/label/name}}
    \ifcsname pgf@anchor@#1@pathstart\endcsname%if special path-anchors are defined, use them!
        (\ctikzvalof{bipole/name}start.center) --(\ctikzvalof{bipole/name}.pathstart)
        (\ctikzvalof{bipole/name}.pathend)  -- (\ctikzvalof{bipole/name}end.center)
    \else
        (\ctikzvalof{bipole/name}start.center) --(\ctikzvalof{bipole/name}.left)
        (\ctikzvalof{bipole/name}.right)  -- (\ctikzvalof{bipole/name}end.center)
    \fi
    \drawpoles
    \pgfextra{
        \pgfcircresetpath
    }
    (\tikztotarget) 	\tikztonodes  % and go on!
}

\def\pgf@circ@definetranspath#1{
	\compattikzset{T#1/.style = {\circuitikzbasekey, /tikz/to path=\pgf@circ@trans@path{#1}{}, l=##1}}
}

%
% vim: set fdm=marker fmr=%<<<,%>>>:
