%--------------------------------------------
%
% Package pgfplots
%
% Provides a user-friendly interface to create function plots (normal
% plots, semi-logplots and double-logplots).
% 
% It is based on Till Tantau's PGF package.
%
% Copyright 2007-2014 by Christian Feuersänger.
%
% This program is free software: you can redistribute it and/or modify
% it under the terms of the GNU General Public License as published by
% the Free Software Foundation, either version 3 of the License, or
% (at your option) any later version.
% 
% This program is distributed in the hope that it will be useful,
% but WITHOUT ANY WARRANTY; without even the implied warranty of
% MERCHANTABILITY or FITNESS FOR A PARTICULAR PURPOSE.  See the
% GNU General Public License for more details.
% 
% You should have received a copy of the GNU General Public License
% along with this program.  If not, see <http://www.gnu.org/licenses/>.
%
%--------------------------------------------

\def\pgfplotsset#{\pgfqkeys{/pgfplots}}

% This is *identical* to \pgfutil@IfUndefined . I copied it here
% because pgf up to and including version 2.10 does not contain it.
\def\pgfplotsutil@IfUndefined#1{%
  \begingroup\expandafter\expandafter\expandafter\endgroup
  \expandafter\ifx\csname#1\endcsname\relax
    \expandafter\pgfutil@firstoftwo
  \else
    \expandafter\pgfutil@secondoftwo
  \fi
}
\pgfplotsutil@IfUndefined{pgfutil@IfUndefined}{\let\pgfutil@IfUndefined=\pgfplotsutil@IfUndefined}{}

\def\pgfplotsloglevel{4}%

\def\pgfplots@log#1#2{%
	\ifnum#1>\pgfplotsloglevel\relax
	\else
		\ifnum\inputlineno=1
			\pgfutil@IfUndefined{@currenvline}{%
				\def\pgfplots@log@tmp{}%
			}{%
				\let\pgfplots@log@tmp=\@currenvline
			}%
		\else
			\def\pgfplots@log@tmp{ on input line \the\inputlineno}%
		\fi
		\immediate\write-1{Package pgfplots info\pgfplots@log@tmp: #2}%
	\fi
}%

\newif\ifpgfplots@LUA@supported
\pgfkeys{
	/pgfplots/lua support/.is choice,
	/pgfplots/lua support/true/.code={\pgfplots@LUA@supportedtrue},%
	/pgfplots/lua support/false/.code={\pgfplots@LUA@supportedfalse},%
	/pgfplots/lua support/auto/.code={%
		\pgfutil@IfUndefined{directlua}{%
			\pgfplots@LUA@supportedfalse
		}{%
			\pgfplots@LUA@supportedtrue
		}%
	},%
	/pgfplots/lua support/auto,% this is the default anyway
}

% this is an emergency key which allows to switch off LUA support in
% pgfplots early at boot time. 
%
% use \def\pgfplotsenablelua{0} to switch it off. '1' switches it on
% (or uncommenting the line).
\pgfutil@IfUndefined{pgfplotsenablelua}{}{%
	\ifpgfplots@LUA@supported
		\def\pgfplots@glob@TMPc{1}%
		\ifx\pgfplots@glob@TMPc\pgfplotsenablelua
			\message{Pgfplots: found emergency control flag \string\pgfplotsenablelua=1. Keeping LUA active.^^J}%
		\else
			\message{Pgfplots: found emergency control flag \string\pgfplotsenablelua!=1. Disabling LUA support.^^J}%
			\pgfplots@LUA@supportedfalse
		\fi
	\fi
}%

\def\pgfplots@alloc@catcodetable{%
	% it is task of the TeX format to load this.
	% context comes with it out-of-the box, and pgfplots.sty loads luatexbase.sty
	\pgfutil@IfUndefined{newcatcodetable}{%
		% Oh. The TeX format does not have support for
		% \newcatcodetable and the associated pgfplots module did not
		% load an addon-package!?
		%
		% We _need_ the catcodes which are active when pgfplots starts.
		% And there is no central mechanism to allocate catcodes. I will
		% take one for me now and hope that it will not be used (soon).
		\def\pgfplotsglobalretval{41}%
	}{%
		% the macro \pgfplots@luabackend@table is never used by TeX
		% code.
		% I only need the value inside of the .lua files: whenever
		% they print something, they need to use it.
		\newcatcodetable\pgfplots@luabackend@table
		\let\pgfplotsglobalretval=\pgfplots@luabackend@table
	}%
}%

\ifpgfplots@LUA@supported
	\let\pgfplotsutil@directlua=\directlua
	\let\pgfplotsutil@directlua@std=\pgfplotsutil@directlua
	\pgfplotsutil@directlua{
		tex.enableprimitives("pgfplotsutil@",{ "savecatcodetable", "luaescapestring"})}%
	\def\pgfplotsutil@directlua@trace#1{%
		\directlua{%
			texio.write_nl("CALLING\string\n\pgfplotsutil@luaescapestring{#1}");
			#1%
		}%
	}%
	\pgfplots@log3{Initializing with LUA version \pgfplotsutil@directlua{tex.print(_VERSION)}}%
	% attention: this can set 'lua support=false' as a side-effect:
	\pgfplotsutil@directlua{require("pgfplots"); pgfplots.pgfplotsversion = "\pgfplotsversion";}%
	%
	\pgfplots@alloc@catcodetable
	\c@pgf@counta=\pgfplotsglobalretval
	\pgfplotsutil@savecatcodetable\c@pgf@counta
	\pgfplotsutil@directlua{pgfplots.LOAD_TIME_CATCODETABLE = \the\c@pgf@counta}%
\fi


\def\pgfplots@error#1{\PackageError{pgfplots}{#1}{}}%
\def\pgfplotstable@error@pkg#1{\PackageError{pgfplotstable}{#1}{}}%
% DO NOT USE THIS METHOD DIRECTLY: use \pgfplotswarning{<warning name>}<args>\pgfeov
\pgfplotsutil@IfUndefined{PackageWarning}{%
	\def\pgfplots@warning#1{\pgfplots@message{! Package pgfplots Warning:
#1}}%
}{%
	\def\pgfplots@warning#1{\PackageWarning{pgfplots}{#1}{}}%
}%

\def\pgfplots@message#1{%
	\immediate\write16{#1}%
}%

\pgfutil@IfUndefined{pgfkeys}{%
	\pgfplots@error{It seems your version of PGF/Tikz is older than 2.00. Unfortunately, pgfplots requires at least version 2.00 ... you may need to update your TeX-Distribution or install PGF manually, sorry}%
}{\relax}

% Throws exception `#1' with arguments `#2'.
%
% #1 : the exception name
% #2: all what comes after the exception name is considered to be argument
%    (or arguments) for the exception '#1'.
%    the \pgfeov is IMPORTANT as it delimits the argument.
%
% Note that all standard pgfplots exceptions provide a feature to
% exchange the error message text: define \pgfplotsexceptionmsg
% set a replacement.
\def\pgfplotsthrow#1#2\pgfeov{%
	\def\pgfkeyscurrentkey{/pgfplots/exception/#1}%
	\pgfkeysvalueof{/pgfplots/exception/#1/.@cmd}#2\pgfeov
}%
\let\pgfplotsthrow@orig=\pgfplotsthrow

% Generates warning `#1' with argument(s) `#2'.
%
% It works in exactly the same way as \pgfplotsthrow.
\def\pgfplotswarning#1#2\pgfeov{%
	\def\pgfkeyscurrentkey{/pgfplots/warning/#1}%
	\pgfkeysvalueof{/pgfplots/warning/#1/.@cmd}#2\pgfeov
}%

\def\pgfplotsthrow@warning#1{%
	\pgfplots@warning{#1}%
	\immediate\write-1{ [warning \pgfkeyscurrentkey]}%
}%

\def\pgfplots@assert@not@zero#1#2{%
	\ifdim#1=0sp %
		\pgfplotsthrow{invalid argument}{\pgfplots@loc@TMPa}{#2}\pgfeov
	\fi
}%

% A primitive try #1 catch #2 end  block.
%
% It tries code #1. If any exception occurs within, it suppresses the
% exception and tries to continue. It then invokes #2 as soon as it
% can.
%
% You can use \pgfplotsrethrow in #2.
%
% ATTENTION: this is a simple attempt to simulate error control. Don't
% rely on it too heavily! In the moment, I am not even sure if it can
% be nested (perhaps you need to introduce extra scopes since
% \pgfplotstry doesn't).
\long\def\pgfplotstry#1\catch#2\endpgfplotstry{%
	\global\let\pgfplotstry@exception\pgfutil@empty
	\def\pgfplotsthrow##1##2\pgfeov{\gdef\pgfplotstry@exception{{##1}{##2}}}%
	#1\relax%
	\let\pgfplotsthrow=\pgfplotsthrow@orig
	\ifx\pgfplotstry@exception\pgfutil@empty
	\else
		#2%
	\fi
	\global\let\pgfplotstry@exception\pgfutil@empty
}%
\def\pgfplotsrethrow{\expandafter\pgfplotsthrow\pgfplotstry@exception\pgfeov}%

\pgfkeys{
	% #1: the argument which should have been assigned.
	% #2: an error message. 
	/pgfplots/exception/invalid argument/.code 2 args={%
		\ifx\pgfplotsexceptionmsg\relax
			\pgfplots@error{#2}%
		\else
			\pgfplots@error{\pgfplotsexceptionmsg}%
		\fi
		\let#1=\pgfutil@empty
	},%
	% #1 : the layer name
	% #2 : additional context information (message)
	/pgfplots/exception/inactive layer/.code 2 args={%
		\ifx\pgfplotsexceptionmsg\relax
			\pgfplots@error{Sorry, the layer '#1' has not been activated but it is referenced in #2. Perhaps you misspelled it? Each referenced layer must be activated (expect for layer 'discard').}%
		\else
			\pgfplots@error{\pgfplotsexceptionmsg}%
		\fi
	},%
	% #1: the argument which should have been assigned.
	% #2: an error message. 
	/pgfplots/exception/no such element/.code 2 args={%
		\ifx\pgfplotsexceptionmsg\relax
			\pgfplots@error{#2}%
		\else
			\pgfplots@error{\pgfplotsexceptionmsg}%
		\fi
		\let#1=\pgfutil@empty
	},%
	/pgfplots/exception/invalid color component/.code 2 args={%
		\ifx\pgfplotsexceptionmsg\relax
			\pgfplots@error{#2}%
		\else
			\pgfplots@error{\pgfplotsexceptionmsg}%
		\fi
		\let#1=\pgfutil@empty
	},%
	/pgfplots/exception/cycle list instantiation needed/.code={%
		\ifx\pgfplotsexceptionmsg\relax
			\pgfplots@error{Please invoke \string\pgfplotsset{cycle list/#1} before using cycle list '#1'}%
		\else
			\pgfplots@error{\pgfplotsexceptionmsg}%
		\fi
	},%
	% #1: an error message
	/pgfplots/exception/unsupported operation/.code={%
		\ifx\pgfplotsexceptionmsg\relax
			\pgfplots@error{#1}%
		\else
			\pgfplots@error{\pgfplotsexceptionmsg}%
		\fi
	},%
	% #1: the argument which should have been assigned.
	% #2: the file name
	% #3: the error message
	/pgfplots/exception/no such table file/.code args={#1#2#3}{%
		\ifx\pgfplotsexceptionmsg\relax
			\pgfplots@error{#3}%
		\else
			\pgfplots@error{\pgfplotsexceptionmsg}%
		\fi
		\let#1=\relax
	},
	/pgfplots/exception/too few colormap positions/.code args={#1#2}{%
		\ifx\pgfplotsexceptionmsg\relax
			\pgfplots@error{#2}%
		\else
			\pgfplots@error{\pgfplotsexceptionmsg}%
		\fi
		\let#1=\relax
	},
	/pgfplots/exception/dimension too small in ticks/.code args={#1}{%
		\ifx\pgfplotsexceptionmsg\relax
			\pgfplots@error{The argument(s) for #1tick resulted in a tick distance which is too small. Please reconfigure the xtick argument(s)}%
		\else
			\pgfplots@error{\pgfplotsexceptionmsg}%
		\fi
	},
	%
	%
	%
	%
	% Generated if you say 'ymin=0' in a log axis.
	% #1: the user argument ('ymin')
	% #2: the offending value.
	/pgfplots/warning/illegal user argument in log axis/.code args={#1#2}{%
		\pgfplotsthrow@warning{Ignoring illegal input argument #1=#2: cannot apply log.}%
	},
	/pgfplots/warning/too many colormap positions/.code={%
		\pgfplotsthrow@warning{the colormap has too many positions; ignoring position #1}%
	},
	/pgfplots/warning/plot without coordinates/.code={%
		\pgfplotsthrow@warning{the current plot has no coordinates (or all have been filtered away)}%
	},
	/pgfplots/warning/log basis required/.code 2 args={%
		\pgfplotsthrow@warning{Some input argument (#2) explicitly relies on the log basis of #1 (which is e). Please verify that this is intended and set 'log basis #1=10' otherwise. This warning will be shown only once per document.}%
	},
	/pgfplots/warning/no such legend to name/.code={%
		\pgfplotsthrow@warning{There is no `legend to name=#1', sorry. I am skipping it}%
	},
	/pgfplots/warning/no such colorbar to name/.code={%
		\pgfplotsthrow@warning{There is no `colorbar to name=#1', sorry. I am skipping it}%
	},
	/pgfplots/warning/no such addplot label/.code={%
		\pgfplotsthrow@warning{There is no \string\addplot\space label `#1', sorry. I am skipping it}%
	},
	/pgfplots/warning/missing near ticklabel at/.code={%
		\pgfplotsthrow@warning{The anchor 'near ticklabel' cannot be computed correctly because the position is missing. Please add 'near ticklabel at=coordinate' with a suitable coordinate}%
	},
	/pgfplots/warning/enlargelimits respects figure size failed/.code={%
		\pgfplotsthrow@warning{enlargelimits respects figure size=true: could not respect the prescribed width/height (\pgfplotsretval). The size and/or aspect ratio of your image will be different than expected. You can try to modify BOTH width and height.}%
	},
	% #1: the axis
	/pgfplots/warning/empty range cleared axis/.code={%
		\pgfplotsthrow@warning{You have an axis with empty range (in direction #1). Replacing it with a default range and clearing all plots.}%
	},
	/pgfplots/warning/approx empty range enlarged/.code args={#1#2#3}{%
		\pgfplotsthrow@warning{Axis range for axis #1 is approximately empty; enlarging it (it is [#2:#3])}%
	},
	/pgfplots/warning/dimension too large in ticks/.code args={#1#2#3}{%
		\pgfplotsthrow@warning{The argument for #1tick distance=#2 was capped to #3 in order to avoid 'dimension too large'}%
	},
	/pgfplots/warning/lossy colormap rgb conversion/.code 2 args={%
		\pgfplotsthrow@warning{I am about to convert colormap '#1' from #2\space to rgb (which might be lossy). Perhaps the choice 'colorbar sampled={surf,shader=interp}' yields better results since it supports #2.}%
	},
	/pgfplots/warning/cycle list size limit reached/.code={%
		\pgfplotsthrow@warning{Reached /pgfplots/@cycle/max size=\pgfkeysvalueof{/pgfplots/@cycle/max size}. Omitting all following entries.}%
	},
	/pgfplots/warning/colorbar uninitialised/.code={%
		\pgfplotsthrow@warning{Sorry, the colorbar is uninitialised - your axis has no color data. Consider using 'point meta=f(x)' or something like that. I am using a color range of [0,1] for the colorbar.}%
	},
	/pgfplots/warning/colorbar CMYK unsupported/.code={%
		\pgfplotsthrow@warning{colorbar remark: The colormap '#1' has a CMYK colorspace which is not (yet) supported by PGF. I switched colorbar styles automatically to the visually equivalent 'colorbar sampled={surf,shader=interp}' to generate a proper CMYK colorbar. This has less support in some previewers. Use \string\pgfplotscolorbarCMYKworkaroundfalse\space to disable this improvement. [This message won't be replicated for '#1']}%
	},
	/pgfplots/warning/ticklabel anchor undetermined/.code args={#1#2#3#4}{%
		\pgfplotsthrow@warning{the ticklabel anchor cannot be determined, the normal vector -(#2) and the unit #1 vector (#3) are almost parallel (abs(cos(angle)) = #4)!}%
	},
	/pgfplots/warning/point meta unbounded/.code 2 args={%
		\pgfplotsthrow@warning{The per point meta data `#1' (#2) (and probably others as well) is unbounded - using the minimum value instead.}%
	},
	/pgfplots/warning/gnuplot -V impossible/.code={%
		\pgfplotsthrow@warning{Sorry, I could not determine 'gnuplot -V' to check whether gnuplot and logscale writes results in log() or not. Please set `/pgfplots/gnuplot writes logscale=true|false' manually.}%
	},
	/pgfplots/warning/gnuplot -V format unknown/.code={%
		\pgfplotsthrow@warning{Sorry, I can't reliably check which version of gnuplot is available. I guess it is gnuplot < 4.4. Please set `/pgfplots/gnuplot writes logscale=true|false' manually if anything fails.}%
	},
	/pgfplots/warning/plot3 graphics compatibility mode/.code={%
		\pgfplotsthrow@warning{plot3 graphics is running in backwards compatibility mode. %
					Use \string\pgfplotsset{compat=1.6} or higher to benefit from upgraded scaling capabilites.}%
	},
	/pgfplots/warning/plot3 graphics too few inner anchors/.code={%
		\pgfplotsthrow@warning{plot graphics in 3D axis has just two inner anchors (those with '(x,y,z)=>(imgx,imgy)'). You should provide a third one such that I can check for correct scaling.}%
	},
	/pgfplots/warning/scatter classes no such class/.code={%
		\pgfplotsthrow@warning{scatter/classes: can't find class for '#1'!? Please make sure you have specified 'scatter src=explicit symbolic'. Ignoring class '#1' (this message will not come again).}%
	},
	/pgfplots/warning/lua colormap unavailable/.code={%
		\pgfplotsthrow@warning{there is no LUA colormap representation for #1. Using TeX arithmetics.}%
	},
	/pgfplots/warning/interior colormap inefficient/.code={%
		\pgfplotsthrow@warning{the combination 'shader=interp,interior colormap name' is inefficient (it uses 'shader=faceted interp,faceted color=none') and may slow down the *display* of your document}%
	},
	/pgfplots/warning/matrix delinearize uses sqrt/.code args={#1#2#3#4}{%
		\pgfplotsthrow@warning{%
									mesh processing did not find '[mesh/rows=<N>]' or '[mesh/cols=<N>]'. 
									Assuming [mesh/rows=#1,mesh/cols=#2] 
									as sqrt(num points) = sqrt(#3) [ = #4].}%
	},
	/pgfplots/warning/matrix check/.code args={#1#2#3#4#5}{%
		#1{the arguments of [mesh/rows=#2,mesh/cols=#3] assume  #5\space points, but I actually got N = #4\space points! The data matrix appears to be incomplete or overcomplete!? [Use mesh/check=false to disable this message]}%
	},
	/pgfplots/warning/imagesc limits incomplete/.code args={#1}{%
		\pgfkeysgetvalue{/pgfplots/empty line}\pgfplots@loc@TMPa%
		\def\pgfplots@loc@TMPb{scanline}
		\ifx\pgfplots@loc@TMPb\pgfplots@loc@TMPa
			\pgfplotsthrow@warning{Automatic computation of axis limits for 'matrix plot' is INACCURATE. Please add the key '#1'. You can also ignore this warning and deal with axis limits manually}%
		\else
			\pgfplotsthrow@warning{Automatic computation of axis limits for 'matrix plot' is INACCURATE. Please add the keys '#1,empty line=scanline' (current value is empty line=\pgfplots@loc@TMPa). You can also ignore this warning and deal with axis limits manually}%
		\fi
	},
	/pgfplots/warning/tick computation failed/.code 2 args={%
		\pgfplotsthrow@warning{Tick computation for direction #1 failed; there are always too few tick labels (try min ticks has already been advanced to #2)}%
	},
	/pgfplots/warning/scale uniformly unsupported/.code={%
		\pgfplotsthrow@warning{Sorry, 'scale uniformly' failed because its actual implementation works only if y_x = 0 and x_y = 0 (for 2d axes) or if z_x = 0 (for 3d axes). The result will not fill the prescribed dimensions. Falling back to 'scale uniformly strategy=units only. (use scale uniformly warning=false to disable this warning)}%
	},
	/pgfplots/warning/axis equal incompatible change/.code={%
		\pgfplotsthrow@warning{the content of your 3d axis has changed compared to previous versions of pgfplots. please review it. ^^J   %
		[continued]   explanation: you have a 3d axis with 'axis equal' and/or 'unit vector ratio' which has (probably) been optimized for an older version of pgfplots. any version older than 1.6 produced wrong output.^^J   %
	[continued]	  to remove this warning, write \string\pgfplotsset{compat=1.6} (or newer) in your preamble (may change all figures in your document) or by adding that to the affected axis.}%
	},
	/pgfplots/warning/unit vector ratio axis undetermined/.code={%
		\pgfplotsthrow@warning{The algorithm to implement 'unit vector ratio' failed! It could not determine the axis which shall be scaled and decided to use 'unit vector ratio axis=#1'.}%
	},
	/pgfplots/warning/linear system singular/.code={%
		\pgfplotsthrow@warning{Linear system is singular up to machine precision! Continuing with small threshold.}%
	},
	/pgfplots/warning/center lines upgrade required/.code={%
		\pgfplotsthrow@warning{Please use '\string\pgfplotsset{compat=1.8}' to enable correct label placement for 3d axes (at least compat/labels=1.8).}%
	},
	/pgfplots/warning/empty column name/.code 2 args={%
		\pgfplotsthrow@warning{Table '#1' has an empty column name; replacing it by '#2'}%
	},
}
\let\pgfplotsexceptionmsg=\relax

\def\pgfplots@EOI{\pgfplots@EOI}%

\newif\ifpgfplots@loc@tmp
\newtoks\t@pgfplots@toka
\newtoks\t@pgfplots@tokb
\newtoks\t@pgfplots@tokc
\newdimen\pgfplots@tmpa
\newcount\c@pgfplots@coordindex
\newcount\c@pgfplots@scanlineindex
\pgfutil@IfUndefined{r@pgf@reada}{%
	\csname newread\endcsname\r@pgfplots@reada
}{%
	\let\r@pgfplots@reada=\r@pgf@reada
}

% use these macros for GLOBAL temporary assignments.
% you can NEVER rely on their values unless you know exactly what you are doing.
\gdef\pgfplots@glob@TMPa{}%
\gdef\pgfplots@glob@TMPb{}%
\gdef\pgfplots@glob@TMPc{}%

% use these macros for LOCAL temporary assigments.
% you can NEVER rely on their values unless you know exactly what you are doing.
\def\pgfplots@loc@TMPa{}%
\def\pgfplots@loc@TMPb{}%
\def\pgfplots@loc@TMPc{}%


% Invokes code #2 if file '#1' exists and #3 if not.
\long\def\pgfplots@iffileexists#1#2#3{%
	\openin\r@pgfplots@reada=#1
	\ifeof\r@pgfplots@reada
		#3\relax
	\else
		\closein\r@pgfplots@reada
		#2\relax
	\fi
}
\let\pgfplotsiffileexists=\pgfplots@iffileexists

\pgfutil@ifundefined{pgfplots@texdist@protect}{%
	\def\pgfplots@texdist@protect{}%
}{}

\input pgfplotssysgeneric.code.tex
\endinput
