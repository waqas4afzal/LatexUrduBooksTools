% \iffalse meta-comment
%
% Copyright (C) 1993-2020
% The LaTeX3 Project and any individual authors listed elsewhere
% in this file.
%
% This file is part of the LaTeX base system.
% -------------------------------------------
%
% It may be distributed and/or modified under the
% conditions of the LaTeX Project Public License, either version 1.3c
% of this license or (at your option) any later version.
% The latest version of this license is in
%    http://www.latex-project.org/lppl.txt
% and version 1.3c or later is part of all distributions of LaTeX
% version 2008 or later.
%
% This file has the LPPL maintenance status "maintained".
%
% The list of all files belonging to the LaTeX base distribution is
% given in the file `manifest.txt'. See also `legal.txt' for additional
% information.
%
% The list of derived (unpacked) files belonging to the distribution
% and covered by LPPL is defined by the unpacking scripts (with
% extension .ins) which are part of the distribution.
%
% \fi
% Filename: ltnews19.tex
%
% This is issue 19 of LaTeX News.

\documentclass{ltnews}

\usepackage[T1]{fontenc}

\usepackage{lmodern,url}

\publicationmonth{September}
\publicationyear{2009}

\publicationissue{19}

\begin{document}

\maketitle

\section{New \LaTeX\ release}

This issue of \LaTeX~News marks the first release of a new version of
\LaTeXe\ since the publication of The \LaTeX\ Companion in 2005--2006.

Just in time for \TeX\ Live 2009, this version is a maintenance
release and introduces no new features. A number of small changes have
been made to correct minor bugs in the kernel, slightly extend the Unicode
support, and improve various
aspects of some of the \texttt{tools} packages.

\section{New code repository}

Since the last \LaTeX\ release, the entire code base has been moved
to a public \textsc{svn} repository%
\footnote{\url{http://www.latex-project.org/svnroot/latex2e-public/}}
and the entire build architecture re-written. In fact, it has only
been possible for us to consider a new \LaTeX\ release since earlier
this year when the test suite was finally set up with the new system.
In the process, a bug in the \LaTeX\ picture fonts distributed with
\TeX\ Live was discovered, proving that the
tests are working and are still very valuable.

Now that we can easily generate new packaged versions of the \LaTeXe\
distribution, we expect to be able to roll out bug fixes in a much
more timely manner than over the last few years. New versions should
be distributed yearly with \TeX\ Live. Having said this, the
maintenance of the \LaTeXe\ kernel is slowing down as the bugs become
fewer and more subtle. Remember that we cannot change any of the
underlying architecture of the kernel or any design decisions of the
standard classes because we must preserve backwards compatibility with
legacy documents at all costs.

Even new features cannot be added, because any new documents using
them will not compile in systems (such as journal production engines)
that are generally not updated once they've been proven to work as
necessary.

None of this is to say that we consider \LaTeXe\ to be any less
relevant for document production than in years past: a stable system
is a useful one.
Moreover, the package system continues to provide a flourishing and stable
means
for the development of a wide range of extensions.

\newpage

\section{Babel}

One area of the \LaTeXe\ code base that is still receiving feedback to
be incorporated into the main distribution is the Babel system for
multilingual typesetting.
While the Babel sources have already been added to the \textsc{svn} repository the
integration of the test system for Babel is still outstanding.


\section{The future}

While work on \LaTeXe\ tends to maintenance over active development,
the \LaTeX3 project is
seeing new life. Our goals here are to provide a transition from the
\LaTeXe\ document processing model to one with a more flexible
foundation. Work is continuing in the \textsf{expl3} programming
language and the \textsf{xpackages} for document design. Future
announcements
about \LaTeX3 will be available via the \LaTeX\ Project
website and in TUGboat.

\end{document}
