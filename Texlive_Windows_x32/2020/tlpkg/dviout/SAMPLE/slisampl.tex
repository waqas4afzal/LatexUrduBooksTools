\documentclass{slides}
\usepackage{graphicx, color, myhyper}
\def\pause{\special{pause}}

\paperwidth=10.24in
\paperheight=7.68in
\textwidth=9in
\textheight=6.6in
\voffset=-.9in
\hoffset=-.48in

\title{Presentation with dviout%
% 		1024 x 768, black screen, cover from bottom, 
%		cover sheet On for Pause, Presentation ; Fit
\special{dviout -e=0 -y=XGAP !AN2N5NP!p;!bdf}
\special{dviout `timer 5000 je}%	wait 5 sec go to the next page
}
\author{Toshio OSHIMA}
\date{January 1, 2000}
\pagestyle{plain}

%%%%%%    TEXT START    %%%%%%
\begin{document}
\maketitle

\begin{slide}
\special{dviout `timer 5000 je}%	wait 5 sec at the first pause
\special{dviout `timer 5000 je}%	wait 5 sec at the second pause
\special{dviout `timer 5000 je}%	wait 5 sec and display whole page
\special{dviout `timer 5000 je}%	wait 5 sec and go to the next page
{\large \S0 Presentation by \TeX}

Many technical documents including mathematical formulas are written using
{\TeX} and they can be used for presentations by dviout for Windows.

{\color{red} Wait without any key input!}

\pause
Study the solution $u(x)$ of the Shr\"odinger equation
\[
 \left(-\frac12\sum_{1\le j\le n}\frac{\partial^2}{\partial x_j^2} + U(x)\right)u(x) = \lambda u(x)
\]

with the potential function

\pause
\[
 U(x) = \sum_{1\le i < j\le n}\frac{C}{\sinh^2(x_i - x_j)}.
\]
\pause

It can be written by Gauss hypergeometric function when $n=2$.
\end{slide}
%%%%%%%%%%%%%%%%%%%%%%%%%%%%%%%
\begin{slide}
% \special{dviout `timer 30000}
% \special{dviout `href file:sample.dvi}
One may try to present this document by a key or mouse operation.
If it is displayed under a cover sheet,
\begin{itemize}
\item
push [Space] key to proceed step by step,
\item
move the mouse under pushing its left button.
\end{itemize}
For this purpose, open \href{file:slisamp2.dvi}{slisamp2.dvi} (a sample 
with the slightly yellow background).

Or open \href{file:slisamp3.dvi}{slisamp3.dvi} or \href{file:slisamp4.dvi}{slisamp4.dvi}  (without the cover sheet).

Otherwise jump to \href{file:sample.dvi}{sample.dvi}. % after 30 seconds. 

Note that the [ESC] key is the toggle switch between the presentation mode and
the normal mode.
\end{slide}
\end{document}
