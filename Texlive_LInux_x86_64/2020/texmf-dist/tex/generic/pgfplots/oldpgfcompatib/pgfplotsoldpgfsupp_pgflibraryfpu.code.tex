%%%%%%%%%%%%%%%%%%%%%%%%%%%%%%%%%%%%%%%%%%%%%%%%%
%%% This file is a copy of some part of PGF/Tikz.
%%% It has been copied here to provide :
%%%  - compatibility with older PGF versions
%%%  - availability of PGF contributions by Christian Feuersaenger
%%%    which are necessary or helpful for pgfplots.
%%%
%%% For reasons of simplicity, I have copied the whole file, including own contributions AND
%%% PGF parts. The copyrights are as they appear in PGF.
%%%
%%% Note that pgfplots has compatible licenses.
%%%
%%% This copy has been modified in the following ways:
%%%  - nested \input commands have been updated
%%%
%
% Support for the contents of this file will NOT be done by the PGF/TikZ team.
% Please contact the author and/or maintainer of pgfplots (Christian Feuersaenger) if you need assistance in conjunction
% with the deployment of this patch or partial content of PGF. Note that the author and/or maintainer of pgfplots has no obligation to fix anything:
% This file comes without any warranty as the rest of pgfplots; there is no obligation for help.
%%%%%%%%%%%%%%%%%%%%%%%%%%%%%%%%%%%%%%%%%%%%%%%%%%
%%% Date of this copy: Mo 30. Apr 17:24:35 CEST 2018 %%%



% Copyright 2008/2009 by Christian Feuersaenger
%
% This file may be distributed and/or modified
%
% 1. under the LaTeX Project Public License and/or
% 2. under the GNU General Public License.
%
% See the file doc/generic/pgf/licenses/LICENSE for more details.

\newif\ifpgfmathfloatparseactive
\newif\ifpgfmathfloat@scaleactive

% public macro which invokes '#1' if the fpu is installed and ready and '#2'
% otherwise.
\def\pgflibraryfpuifactive#1#2{%
	\ifpgfmathfloatparseactive
		#1%
	\else
		#2%
	\fi
}%

\pgfqkeys{/pgf}{
	% enable the FPU parser if it is not yet active
	%
	% It will be deactivated after the current TeX group
	fpu/.is choice,
	fpu/true/.code={%
		\ifpgfmathfloatparseactive
		\else
			\pgfutil@ifundefined{pgfmathdeclarefunction}{%
				% Ohoh - we are running on a TeX distribution with
				% PGF 2.00 which doesn't have the new math engine.
				% I can provide special treatment here, provided that
				% all float commands are still able to run (that means
				% more information needs to be copied from the pgf cvs
				% to pgf 2.00 - for example pgfmathfloat.code.tex).
				%
				% I employ this to work with pgfplots and pgf 2.00
				% using all new features.
				\pgfmathfloat@parser@install@pgf@two@null@null%
			}{%
				\pgfmathfloat@parser@install%
			}%
			\pgfmathfloatparseactivetrue
			% improve compatibility with Marks FP library:
			\pgfkeysalso{/pgf/fixed point arithmetic/.prefix style={/pgf/fpu=false}}%
		\fi
	},%
	fpu/false/.code={%
		\ifpgfmathfloatparseactive
			\pgfmathfloat@uninstall%
			\pgfmathfloatparseactivefalse
		\fi
	},%
	fpu/.default=true,
	% Use this to introduce a result scaling.
	% Every expression in which the prefix '*' occurs
	% will be multiplied with the result and converted to fixed point
	% representation.
	fpu/scale results/.code={%
		\ifpgfmathfloatparseactive
			\pgfmathparse{#1}%
		\else
			\pgfmathfloatparsenumber{#1}%
		\fi
		\let\pgfmathfloatscale=\pgfmathresult%
	},%		
	% determines the output format of each complete expression parsing
	% process. If 'scale results' is active, 'fixed' is assumed
	% automatically.
	fpu/output format/.is choice,
	fpu/output format/float/.code=	{%
		\def\pgfmathfloatparse@output@choice{Y}%
		\let\pgfmathfloatparse@output=\relax
	},
	fpu/output format/sci/.code=	{%
		\def\pgfmathfloatparse@output@choice{S}%
		\def\pgfmathfloatparse@output{\pgfmathfloattosci@{\pgfmathresult}}%
	},
	fpu/output format/fixed/.code=	{%
		\def\pgfmathfloatparse@output@choice{F}%
		\def\pgfmathfloatparse@output{\pgfmathfloattofixed@{\pgfmathresult}}%
	},
	fpu/output format/float,
	fpu/rel thresh/.code={%
		\pgfmathfloatparsenumber{#1}%
		\let\pgfmathfloat@relthresh=\pgfmathresult
	},
	fpu/rel thresh=1e-4,
}

\pgfmathfloatcreate{1}{1.0}{0}\let\pgfmathfloatscale=\pgfmathresult

\let\pgfmathfloatone=\pgfmathresult

% This is the replacement parser invocation.
% It does two things which are different to \pgfmathparse:
% 1. it disables any dimension dependand scalings,
% 2. it implements the 'scale results' feature.
\def\pgfmathfloatparse{%
	\begingroup%
		% disable any dimension-dependant scalings:
		\let\pgfmathpostparse=\relax%
		\pgfmath@catcodes%
		\pgfmath@quickparsefalse%
		\pgfmathfloatparse@}

% for pgf 2.00 :
\def\pgfmathfloatparse@pgf@two@null@null{%
	\pgfmath@quickparsefalse%
	\pgfmathfloatparse@}

\def\pgfmathfloatparse@#1{%
	\edef\pgfmathfloat@expression{#1}%
	\expandafter\pgfmathfloatparse@@\pgfmathfloat@expression\pgfmathfloat@
	\ifpgfmathfloat@scaleactive
		\expandafter\pgfmathfloatmultiply@\expandafter{\pgfmathresult}{\pgfmathfloatscale}
		\pgfmathfloattofixed{\pgfmathresult}%
	\else
		\pgfmathfloatparse@output
	\fi
}

\def\pgfmathfloat@char@asterisk{*}
\def\pgfmathfloatparse@@#1#2\pgfmathfloat@{%
	\def\pgfmathfloat@test{#1}%
	\ifx\pgfmathfloat@test\pgfmathfloat@char@asterisk%
		\def\pgfmathfloat@expression{#2}%
		\pgfmathfloat@scaleactivetrue
	\fi%
	\expandafter\pgfmathparse@\expandafter{\pgfmathfloat@expression}%
	% \endgroup provided by \pgfpathmarse@end
}

% Crude handling of file plots
%
\pgfkeys{/pgf/fpu/.cd,
  scale file plot x/.code=\pgfmathfloatparse{#1}\edef\pgfmathfloatplotscalex{\pgfmathresult*},
  scale file plot y/.code=\pgfmathfloatparse{#1}\edef\pgfmathfloatplotscaley{\pgfmathresult*},
  scale file plot z/.code=\pgfmathfloatparse{#1}\edef\pgfmathfloatplotscalez{\pgfmathresult*}
}

\def\pgfmathfloat@uninstall@appendcmd#1{%
	\expandafter\gdef\expandafter\pgfmathfloat@uninstall\expandafter{\pgfmathfloat@uninstall #1}%
}%

% If the uninstall command is already assembled, it will skip the
% uninstall assemblation.
\def\pgfmathfloat@plots@checkuninstallcmd{%
	\pgfutil@ifundefined{pgfmathfloat@uninstall}{%
		\global\let\pgfmathfloat@uninstall=\pgfutil@empty
	}{%
		% We already HAVE an uninstall command (prepared globally).
		% So: don't waste time assembling one!
		\def\pgfmathfloat@uninstall@appendcmd##1{}%
		\def\pgfmathfloat@prepareuninstallcmd##1{}%
	}%
}%

% This assembles an uninstall command globally ON FIRST USAGE.
% See \pgfmathfloat@plots@checkuninstallcmd
\def\pgfmathfloat@prepareuninstallcmd#1{%
	% and store backup information (globally - I don't want to do that
	% all the time when the FPU is used!):
	\expandafter\global\expandafter\let\csname pgfmathfloat@backup@\string#1\endcsname=#1%
	\expandafter\gdef\expandafter\pgfmathfloat@uninstall\expandafter{\pgfmathfloat@uninstall
		\expandafter\let\expandafter#1\csname pgfmathfloat@backup@\string#1\endcsname%
	}%
}

\def\pgfmathfloat@install#1=#2{%
	\pgfmathfloat@prepareuninstallcmd{#1}%
	\let#1=#2%
}
\def\pgfmathfloat@install@csname#1#2{%
	\expandafter\pgfmathfloat@prepareuninstallcmd\csname #1\endcsname%
	\pgfutil@namelet{#1}{#2}%
}
\def\pgfmathfloat@install@unimplemented#1{%
	\expandafter\pgfmathfloat@prepareuninstallcmd\csname pgfmath#1@\endcsname%
	\expandafter\def\csname pgfmath#1@\endcsname##1{\pgfmathfloat@notimplemented{#1}}%
}
\def\pgfmathfloat@plots@install{%
	\let\pgfmathfloatplotscalex=\pgfutil@empty
	\let\pgfmathfloatplotscaley=\pgfutil@empty
	\let\pgfmathfloatplotscalez=\pgfutil@empty
	\pgfmathfloat@install\pgf@parsexyline=\pgfmathfloat@parsexyline%
	\pgfmathfloat@install\pgf@parsexyzline=\pgfmathfloat@parsexyzline%
}
\def\pgfmathfloat@parsexyline#1 #2 #3\pgf@stop{%
	\edef\pgfmathfloat@marshal{%
		\noexpand\pgfplotstreampoint{\noexpand\pgfpointxy{\pgfmathfloatplotscalex#1}{\pgfmathfloatplotscaley#2}}%
	}%
	\pgfmathfloat@marshal%
}
\def\pgfmathfloat@parsexyzline#1 #2 #3 #4\pgf@stop{%
  \edef\pgfmathfloat@marshal{%
		\noexpand\pgfplotstreampoint{%
			\noexpand\pgfpointxyz{\pgfmathfloatplotscalex#1}{\pgfmathfloatplotscaley#2}{\pgfmathfloatplotscalez#3}%
		}%
	}%
	\pgfmathfloat@marshal%
}

%
\def\pgfmathfloat@parser@install@functions{%
	% Install float commands...
	%
	\pgfmathfloat@install\pgfmathadd@=\pgfmathfloatadd@%
	\pgfmathfloat@install\pgfmathsubtract@=\pgfmathfloatsubtract@%
	\pgfmathfloat@install\pgfmathneg@=\pgfmathfloatneg@%
	\pgfmathfloat@install\pgfmathmultiply@=\pgfmathfloatmultiply@%
	\pgfmathfloat@install\pgfmathdivide@=\pgfmathfloatdivide@%
	\pgfmathfloat@install\pgfmathabs@=\pgfmathfloatabs@%
	\pgfmathfloat@install\pgfmathsign@=\pgfmathfloatsign@%
	\pgfmathfloat@install\pgfmathround@=\pgfmathfloatround@%
	\pgfmathfloat@install\pgfmathfloor@=\pgfmathfloatfloor@%
	\pgfmathfloat@install\pgfmathceil@=\pgfmathfloatceil@
	\pgfmathfloat@install\pgfmathint@=\pgfmathfloatint@
	\pgfmathfloat@install\pgfmathmod@=\pgfmathfloatmod@%
	\pgfmathfloat@install\pgfmathmax@=\pgfmathfloatmax@%
	\pgfmathfloat@install\pgfmathmin@=\pgfmathfloatmin@%
	\pgfmathfloat@install\pgfmathsin@=\pgfmathfloatsin@%
	\pgfmathfloat@install\pgfmathcos@=\pgfmathfloatcos@%
	\pgfmathfloat@install\pgfmathtan@=\pgfmathfloattan@%
	\pgfmathfloat@install\pgfmathdeg@=\pgfmathfloatdeg@%
	\pgfmathfloat@install\pgfmathrad@=\pgfmathfloatrad@%
	\pgfmathfloat@install\pgfmathatan@=\pgfmathfloatatan@%
	\pgfmathfloat@install\pgfmathasin@=\pgfmathfloatasin@%
	\pgfmathfloat@install\pgfmathacos@=\pgfmathfloatacos@%
	\pgfmathfloat@install\pgfmathcot@=\pgfmathfloatcot@%
	\pgfmathfloat@install\pgfmathsec@=\pgfmathfloatsec@%
	\pgfmathfloat@install\pgfmathcosec@=\pgfmathfloatcosec@%
	\pgfmathfloat@install\pgfmathexp@=\pgfmathfloatexp@%
	\pgfmathfloat@install\pgfmathln@=\pgfmathfloatln@%
	\pgfmathfloat@install@csname{pgfmathlog10@}{pgfmathfloatlog10@}%
	\pgfmathfloat@install@csname{pgfmathlog2@}{pgfmathfloatlog2@}%
	\pgfmathfloat@install\pgfmathsqrt@=\pgfmathfloatsqrt@%
	\pgfmathfloat@install\pgfmath@pi=\pgfmathfloatpi@%
	\pgfmathfloat@install\pgfmathpi=\pgfmathfloatpi@%
	\pgfmathfloat@install\pgfmathe@=\pgfmathfloate@%
	\pgfmathfloat@install\pgfmathe=\pgfmathfloate@%
	\pgfmathfloat@install\pgfmathlessthan@=\pgfmathfloatlessthan@%
	\pgfmathfloat@install\pgfmathnotless@=\pgfmathfloatnotless@%
	\pgfmathfloat@install\pgfmathnotgreater@=\pgfmathfloatnotgreater@%
	\pgfmathfloat@install\pgfmathless@=\pgfmathfloatlessthan@%
	\pgfmathfloat@install\pgfmathgreaterthan@=\pgfmathfloatgreaterthan@%
	\pgfmathfloat@install\pgfmathgreater@=\pgfmathfloatgreaterthan@%
	\pgfmathfloat@install\pgfmathifthenelse@=\pgfmathfloatifthenelse@%
	\pgfmathfloat@install\pgfmathequal@=\pgfmathfloatequal@%
	\pgfmathfloat@install\pgfmathequalto@=\pgfmathfloatequal@%
	\pgfmathfloat@install\pgfmathnotequal@=\pgfmathfloatnotequal@%
	\pgfmathfloat@install\pgfmathnotequalto@=\pgfmathfloatnotequal@%
	\pgfmathfloat@install\pgfmathpow@=\pgfmathfloatpow@
	\pgfmathfloat@install\pgfmathrand@=\pgfmathfloatrand@
	\pgfmathfloat@install\pgfmathrand=\pgfmathfloatrand@
	\pgfmathfloat@install\pgfmathrnd@=\pgfmathfloatrnd@
	\pgfmathfloat@install\pgfmathrnd=\pgfmathfloatrnd@
	\pgfmathfloat@install\pgfmathtrue@=\pgfmathfloattrue@
	\pgfmathfloat@install\pgfmathfalse@=\pgfmathfloatfalse@
	\pgfmathfloat@install\pgfmathnot@=\pgfmathfloatnot@
	\pgfmathfloat@install\pgfmathhex@=\pgfmathfloathex@
	\pgfmathfloat@install\pgfmathHex@=\pgfmathfloatHex@
	\pgfmathfloat@install\pgfmathoct@=\pgfmathfloatoct@
	\pgfmathfloat@install\pgfmathbin@=\pgfmathfloatbin@
	\pgfmathfloat@install\pgfmathand@=\pgfmathfloatand@
	\pgfmathfloat@install\pgfmathor@=\pgfmathfloator@
	\pgfmathfloat@install\pgfmathfactorial@=\pgfmathfloatfactorial@
	\pgfmathfloat@install\pgfmathveclen@=\pgfmathfloatveclen@
	\pgfmathfloat@install\pgfmathcosh@=\pgfmathfloatcosh@
	\pgfmathfloat@install\pgfmathsinh@=\pgfmathfloatsinh@
	\pgfmathfloat@install\pgfmathtanh@=\pgfmathfloattanh@
	\pgfmathfloat@install\pgfmath@iftrue=\pgfmathfloat@iftrue
	\expandafter\pgfmathfloat@install\csname pgfmathatan2@\endcsname=\pgfmathfloatatantwo@
	\pgfmathfloat@install@unimplemented{isprime}%
	\pgfmathfloat@install@unimplemented{iseven}%
	\pgfmathfloat@install@unimplemented{isodd}%
	\pgfmathfloat@install@unimplemented{gcd}%
	\pgfmathfloat@install@unimplemented{frac}%
	\pgfmathfloat@install@unimplemented{random}%
	\pgfmathfloat@install@unimplemented{setseed}%
	\pgfmathfloat@install@unimplemented{Mod}%
	\pgfmathfloat@install@unimplemented{div}%
	\pgfmathfloat@install@unimplemented{real}%
%	\pgfmathfloat@install@unimplemented{height}%
	%
	%
	\pgfmathfloat@install\pgfmathscientific=\pgfmathfloatscientific%
}

\def\pgfmathfloat@iftrue{%
	\if Y\pgfmathfloatparse@output@choice
		\let\pgfmathfloat@@iftrue@v=\pgfmathfloatone
		\let\pgfmathfloat@@iftrue@next=\pgfmathfloat@iftrue@
	\else
		\if S\pgfmathfloatparse@output@choice
			\def\pgfmathfloat@@iftrue@v{1.0e0}%
			\let\pgfmathfloat@@iftrue@next=\pgfmathfloat@iftrue@
		\else
			\def\pgfmath@next{\pgfutilifstartswith{1.0}}%
			\expandafter\pgfmath@next\expandafter{\pgfmathresult}{%
				\ifdim\pgfmathresult pt=1.0pt %
					\let\pgfmathfloat@@iftrue@next=\pgfutil@firstoftwo
				\else
					\let\pgfmathfloat@@iftrue@next=\pgfutil@secondoftwo
				\fi
			}{%
				\let\pgfmathfloat@@iftrue@next=\pgfutil@secondoftwo
			}%
		\fi
	\fi
	\pgfmathfloat@@iftrue@next
}%
\def\pgfmathfloat@iftrue@{%
	\ifx\pgfmathresult\pgfmathfloat@@iftrue@v
		\let\pgfmath@next=\pgfutil@firstoftwo
	\else
		\let\pgfmath@next=\pgfutil@secondoftwo
	\fi
	\pgfmath@next
}%

\def\pgfmathfloat@parser@install{%
	\pgfmathfloat@plots@checkuninstallcmd
	\pgfmathfloat@plots@install%
	\pgfmathfloat@parser@install@functions
	%
	%
	%
	% The following methods actually enable the parser to work with
	% the internal floating point number representation.
	%
	% The idea is as follows:
	% 1. Every operand must be given in internal float representation.
	% 2. The internal float repr can be distinguished by a normal
	% number. This is accomplished by introducing a new "exponent"
	% token.
	% 3. The stack-push-operation checks whether the argument is a
	% float. If not, it is parsed properly before pushing it.
	\pgfmath@tokens@make{exponent}{\pgfmathfloat@POSTFLAGSCHAR}%
	\pgfmathfloat@uninstall@appendcmd{%
		\expandafter\let\csname pgfmath@token@exponent@\pgfmathfloat@POSTFLAGSCHAR\endcsname=\relax
	}%
	\let\pgfmath@basic@parse@exponent=\pgfmath@parse@exponent%
	\let\pgfmath@basic@stack@push@operand=\pgfmath@stack@push@operand
	\pgfmathfloat@install\pgfmath@stack@push@operand=\pgfmathfloat@stack@push@operand
	\pgfmathfloat@install\pgfmath@parse@operand@quote=\pgfmathfloat@parse@operand@quote
	\pgfmathfloat@install\pgfmath@parse@exponent=\pgfmathfloat@parse@float@or@exponent
	%
	\pgfmathfloat@install\pgfmathparse=\pgfmathfloatparse%
	%\pgfmathfloat@install\pgfmathparse@trynumber@token=\pgfmathfloat@parse@trynumber@token
	\pgfmathfloat@install\pgfmathparse@expression@is@number=\pgfmathfloat@parse@expression@is@number
}%

% This here might bring speed improvements... if implemented
% correctly.
% However, this heuristics might fail in cases like "1+1" vs "1e+1" ...
%\def\pgfmathfloat@parse@trynumber@token{numericfpu}
%\pgfmath@tokens@make{numericfpu}{eE+-Y.0123456789}

\def\pgfmathfloat@parse@expression@is@number{%
	\pgfmathfloatparsenumber{\pgfmath@expression}%
    \pgfmath@smuggleone\pgfmathresult%
  \endgroup
  \ignorespaces
}%

\def\pgfmathfloat@defineadapter@for@pgf@two@null@null@ONEARG#1{%
	\edef\pgfmathfloat@loc@TMPa{%
		\noexpand\def\expandafter\noexpand\csname pgfmath@parsefunction@#1\endcsname{%
			\noexpand\let\noexpand\pgfmath@parsepostgroup\expandafter\noexpand\csname pgfmath@parsefunction@#1@\endcsname%
			\noexpand\expandafter\noexpand\pgfmath@parse@}%
		\noexpand\def\expandafter\noexpand\csname pgfmath@parsefunction@#1@\endcsname{%	
			\noexpand\expandafter\expandafter\noexpand\csname pgfmath#1@\endcsname\noexpand\expandafter{\noexpand\pgfmathresult}%
			\noexpand\pgfmath@postfunction%
		}%
	}%
	\pgfmathfloat@loc@TMPa
}%
\def\pgfmathfloat@parser@install@pgf@two@null@null{%
	\pgfmathfloat@plots@checkuninstallcmd
	\pgfmathfloat@plots@install%
	\pgfmathfloat@parser@install@functions
	\let\pgfmathrand@=\pgfmath@basic@rand@
	\let\pgfmathrnd@=\pgfmath@basic@rnd@
	\pgfmathfloat@install\pgfmathmax@=\pgfmathfloatmaxtwo%
	\pgfmathfloat@install\pgfmathmin@=\pgfmathfloatmintwo%
	\pgfmathfloat@defineadapter@for@pgf@two@null@null@ONEARG{factorial}%
	\pgfmathfloat@defineadapter@for@pgf@two@null@null@ONEARG{hex}%
	\pgfmathfloat@defineadapter@for@pgf@two@null@null@ONEARG{bin}%
	\pgfmathfloat@defineadapter@for@pgf@two@null@null@ONEARG{oct}%
	\pgfmathfloat@defineadapter@for@pgf@two@null@null@ONEARG{tanh}%
	\pgfmathfloat@defineadapter@for@pgf@two@null@null@ONEARG{sinh}%
	\pgfmathfloat@defineadapter@for@pgf@two@null@null@ONEARG{cosh}%
	%
	% The following methods actually enable the parser to work with
	% the internal floating point number representation.
	%
	% The idea is as follows:
	% 1. Every operand must be given in internal float representation.
	% 2. The internal float repr can be distinguished by a normal
	% number. This is accomplished by introducing a new "exponent"
	% token.
	% 3. The stack-push-operation checks whether the argument is a
	% float. If not, it is parsed properly before pushing it.
	\let\pgfmath@basic@parsedecimalpoint=\pgfmath@parsedecimalpoint%
	\let\pgfmath@basic@stack@push@operand=\pgfmath@stackpushoperand
	\pgfmathfloat@install\pgfmath@stackpushoperand=\pgfmathfloat@stack@push@operand
	\pgfmathfloat@install\pgfmath@parsedecimalpoint=\pgfmathfloat@parsedecimalpoint@pgf@two@null@null
	\pgfmathfloat@install\pgfmath@endparse=\pgfmathfloat@endparse@pgf@two@null@null
	\pgfmathfloat@install\pgfmath@endparsegroup=\pgfmathfloat@endparsegroup@pgf@two@null@null
	\pgfmathfloat@install\pgfmath@postfunction=\pgfmathfloat@postfunction@pgf@two@null@null
	\pgfmathfloat@install\pgfmath@@parseoperandgroup=\pgfmathfloat@@parseoperandgroup
	%
	\pgfmathfloat@install\pgfmathparse=\pgfmathfloatparse@pgf@two@null@null%
}%

\pgfutil@ifundefined{pgfmathdeclarefunction}{%
	% BACKWARDS COMPATIBILITY: We have PGF 2.00 :
	\def\pgfmathdeclarepseudoconstant#1#2{%
		\begingroup
		\toks0=\expandafter{\csname pgfmath#1@\endcsname}%
		\toks1={\pgfmath@postfunction}%
		\xdef\pgfmathfloat@glob@TMP{\the\toks0 \the\toks1 }%
		\xdef\pgfmathfloat@glob@TMPb{\the\toks0 }%
		\endgroup
		\expandafter\let\csname pgfmath@parsefunction@#1\endcsname=\pgfmathfloat@glob@TMP
		\expandafter\let\csname pgfmath#1\endcsname=\pgfmathfloat@glob@TMPb
		\expandafter\def\csname pgfmath#1@\endcsname{#2}%
	}%
	\let\pgfmathredeclarepseudoconstant=\pgfmathdeclarepseudoconstant
}{%
	\pgfutil@ifundefined{pgfmathdeclarepseudoconstant}{%
		\def\pgfmathdeclarepseudoconstant#1#2{\pgfmathdeclarefunction*{#1}{0}{#2}}
	}{}%
}%

\pgfmathdeclarepseudoconstant{inf}{\def\pgfmathresult{inf}}
\pgfmathdeclarepseudoconstant{INF}{\def\pgfmathresult{inf}}
\pgfmathdeclarepseudoconstant{Inf}{\def\pgfmathresult{inf}}
\pgfmathdeclarepseudoconstant{infty}{\def\pgfmathresult{inf}}
\pgfmathdeclarepseudoconstant{nan}{\def\pgfmathresult{nan}}
\pgfmathdeclarepseudoconstant{NaN}{\def\pgfmathresult{nan}}

%%%%%%%%%%%%%%%%%%%%%%%%%%%%%%%%%%%%%%%%%%%
%
% Hacks to the basic level pgf math engine:
%
% WARNING: These methods rely heavily on the internal float representation!
%%%%%%%%%%%%%%%%%%%%%%%%%%%%%%%%%%%%%%%%%%%

% for pgf2.00 :
\def\pgfmathfloat@parsedecimalpoint@pgf@two@null@null#1{%
	\expandafter\ifx\pgfmathfloat@POSTFLAGSCHAR#1% check whether it is a float
		\let\pgfmath@next=\pgfmathfloat@return@float@pgf@two@null@null%
	\else
		\def\pgfmath@next{\pgfmath@basic@parsedecimalpoint#1}%
	\fi
	\pgfmath@next
}
% for pgf2.00:
\def\pgfmathfloat@return@float@pgf@two@null@null#1]{%
	\edef\pgfmathresult{\the\c@pgfmath@parsecounta\pgfmathfloat@POSTFLAGSCHAR#1]}%
	\let\pgfmath@resulttemp=\pgfmathresult
	\pgfmath@parseoperator%
}%
% for pgf2.00:
\def\pgfmathfloat@endparse@pgf@two@null@null#1\pgfmath@empty{%
    \pgfmath@processalloperations%
    \pgfmath@stackpop{\pgfmathresult}%
	% delete the final unit scalings
    \pgfmath@smuggleone{\pgfmathresult}%
  \endgroup%
  \ignorespaces%
}
% for pgf2.00:
\def\pgfmathfloat@endparsegroup@pgf@two@null@null{%
		\pgfmath@processalloperations%
		\pgfmath@stackpop{\pgfmathresult}%
		% eliminated register usage here...
		\pgfmath@smuggleone{\pgfmathresult}%
	\endgroup%
	\pgfmath@parsepostgroup%
}
% for pgf2.00:
\def\pgfmathfloat@postfunction@pgf@two@null@null{%
	\let\pgfmath@parsepostgroup\pgfmath@parseoperator%
	\ifnum\pgfmath@sign1<0
		\pgfmathfloatneg@{\pgfmathresult}%
		\let\pgfmath@sign\pgfutil@empty
	\fi
	\pgfmath@parseoperator}
% for pgf2.00:
\def\pgfmathfloat@@parseoperandgroup{%
	\let\pgfmath@postparsegroup\pgfmath@parseoperator%
	\ifnum\pgfmath@sign1<0
		\pgfmathfloatneg@{\pgfmathresult}%
		\let\pgfmath@sign\pgfutil@empty
	\fi
	\pgfmath@parseoperator%
}






% PRECONDITION:
% either
% 	<number>e
% 	        ^
% 	-> read the exponent.
% or
%   <sign>\pgfmathfloat@POSTFLAGSCHAR
%         ^
%   -> we have a parsed floating point number -> read it.
\def\pgfmathfloat@parse@float@or@exponent{%
	\if\pgfmath@token \pgfmathfloat@POSTFLAGSCHAR%
		% Ok, we actually HAVE a pre-parsed floating point number!
		% Return it.
		\expandafter\pgfmathfloat@return@float\expandafter\pgfmath@token@next
	\else
		% We have a standard number in scientific format. Parse it.
		\expandafter\pgfmath@basic@parse@exponent
	\fi
}%
\def\pgfmathfloat@return@float#1]{%
	\edef\pgfmathresult{\pgfmath@number \pgfmathfloat@POSTFLAGSCHAR#1]}%
	\expandafter\pgfmath@basic@stack@push@operand\expandafter{\pgfmathresult}%
	\pgfmath@parse@@operator%
}%

\def\pgfmathfloat@parse@operand@quote#1"{%
  \edef\pgfmathresult{\pgfmath@fpu@stringmarker #1}%
  \expandafter\pgfmath@basic@stack@push@operand\expandafter{\pgfmathresult}%
  \pgfmath@parse@@operator%
}

\def\pgfmath@fpu@stringmarker{@@str@@:}%

% This extends the functionality of the basic level operand stack: it
% assures every element on the stack is a float.
\def\pgfmathfloat@stack@push@operand#1{%
	\pgfutil@ifnextchar\bgroup{%
		\let\pgfmathfloat@stack@push@operand@list@=\pgfutil@empty
		\pgfmathfloat@stack@push@operand@list
	}{%
		\pgfmathfloat@stack@push@operand@single
	}%
	#1\relax
}%
\def\pgfmathfloat@stack@push@operand@single#1\relax{%
	\expandafter\pgfutil@in@\pgfmathfloat@POSTFLAGSCHAR{#1}%
	\ifpgfutil@in@
		\pgfmath@basic@stack@push@operand{#1}%
	\else
		\expandafter\pgfutil@in@\expandafter{\pgfmath@fpu@stringmarker}{#1}%
		\ifpgfutil@in@
			\pgfmathfloat@stack@push@operand@single@str#1\relax
		\else
			\pgfmathfloatparsenumber{#1}%
			\expandafter\pgfmath@basic@stack@push@operand\expandafter{\pgfmathresult}%
		\fi
	\fi
}%

\expandafter\def\expandafter\pgfmathfloat@stack@push@operand@single@str\pgfmath@fpu@stringmarker #1\relax{%
	\pgfmath@basic@stack@push@operand{#1}%
}%
\def\pgfmathfloat@stack@push@operand@GOBBLE#1\relax{}%
\def\pgfmathfloat@stack@push@operand@list#1{%
	\expandafter\pgfutil@in@ \pgfmathfloat@POSTFLAGSCHAR{#1}%
	\ifpgfutil@in@
		\expandafter\def\expandafter\pgfmathfloat@stack@push@operand@list@\expandafter{%
			\pgfmathfloat@stack@push@operand@list@{#1}%
		}%
	\else
		\pgfmathfloatparsenumber{#1}%
		\begingroup
		\toks0=\expandafter{\pgfmathfloat@stack@push@operand@list@}%
		\toks1=\expandafter{\pgfmathresult}%
		\xdef\pgfmathfloat@glob@TMP{\the\toks0 {\the\toks1}}%
		\endgroup
		\let\pgfmathfloat@stack@push@operand@list@=\pgfmathfloat@glob@TMP
	\fi
	\pgfutil@ifnextchar\relax{%
		\expandafter\pgfmath@basic@stack@push@operand\expandafter{\pgfmathfloat@stack@push@operand@list@}%
		\pgfmathfloat@stack@push@operand@GOBBLE
	}{%
		\pgfmathfloat@stack@push@operand@list
	}%
}%
%%%%%%%%%%%%%%%%%%%%%%%%%%%%%%%%%%%%%%%%%%%
%
% --- END --- Hacks to the basic level pgf math engine
%
%%%%%%%%%%%%%%%%%%%%%%%%%%%%%%%%%%%%%%%%%%%



%%%%%%%%%%%%%%%%%%%%%%%%%%%%%%%%%%%%%%%%%%%%%%%%%%%%%%%%
%
% Here starts the implementation of the floating point
% routines.
%
% They can be used even if the FPU parser is not active.
%
%%%%%%%%%%%%%%%%%%%%%%%%%%%%%%%%%%%%%%%%%%%%%%%%%%%%%%%%

% Remember the basic math commands. They will be invoked as subroutines in floating point routines.
\let\pgfmath@basic@add@=\pgfmathadd@
\let\pgfmath@basic@subtract@=\pgfmathsubtract@
\let\pgfmath@basic@multiply@=\pgfmathmultiply@
\let\pgfmath@basic@divide@=\pgfmathdivide@
\let\pgfmath@basic@reciprocal@=\pgfmathreciprocal@
\let\pgfmath@basic@abs@=\pgfmathabs@
\let\pgfmath@basic@round@=\pgfmathround@
\let\pgfmath@basic@rand@=\pgfmathrand@
\let\pgfmath@basic@rnd@=\pgfmathrnd@
\let\pgfmath@basic@setseed@=\pgfmathsetseed@
\let\pgfmath@basic@random@=\pgfmathrandom@
\let\pgfmath@basic@floor@=\pgfmathfloor@
\let\pgfmath@basic@ceil@=\pgfmathceil@
\let\pgfmath@basic@mod@=\pgfmathmod@
\let\pgfmath@basic@max@=\pgfmathmax@
\let\pgfmath@basic@min@=\pgfmathmin@
\let\pgfmath@basic@sin@=\pgfmathsin@
\let\pgfmath@basic@cos@=\pgfmathcos@
\let\pgfmath@basic@tan@=\pgfmathtan@
\let\pgfmath@basic@deg@=\pgfmathdeg@
\let\pgfmath@basic@rad@=\pgfmathrad@
\let\pgfmath@basic@atan@=\pgfmathatan@
\let\pgfmath@basic@asin@=\pgfmathasin@
\let\pgfmath@basic@acos@=\pgfmathacos@
\let\pgfmath@basic@cot@=\pgfmathcot@
\let\pgfmath@basic@sec@=\pgfmathsec@
\let\pgfmath@basic@cosec@=\pgfmathcosec@
\let\pgfmath@basic@pow@=\pgfmathpow@
\let\pgfmath@basic@exp@=\pgfmathexp@
\let\pgfmath@basic@ln@=\pgfmathln@
\let\pgfmath@basic@sqrt@=\pgfmathsqrt@
\let\pgfmath@basic@@pi=\pgfmath@pi
\let\pgfmath@basic@veclen@=\pgfmathveclen@
\let\pgfmath@basic@e@=\pgfmathe@
\let\pgfmath@basic@lessthan@=\pgfmathlessthan@
\let\pgfmath@basic@greaterthan@=\pgfmathgreaterthan@
\let\pgfmath@basic@equalto@=\pgfmathequalto@
\let\pgfmath@basic@equal@=\pgfmathequal@
\let\pgfmath@basic@true@=\pgfmathtrue@
\let\pgfmath@basic@false@=\pgfmathfalse@

\def\pgfmathfloatscientific#1#2{%
	\edef\pgfmathresult{#1e#2}%
	\expandafter\pgfmathfloatparsenumber\expandafter{\pgfmathresult}%
}
% Compares #1 with #2 and sets \pgfmathresult either to 1.0 or 0.0.
%
% It also sets the boolean \ifpgfmathfloatcomparison (globally).
\def\pgfmathfloatlessthan@#1#2{%
%\def\pgfmathfloatlessthan#1#2#3\and#4#5#6{%
	\global\pgfmathfloatcomparisonfalse
	\begingroup
	\edef\pgfmathfloat@loc@TMPa{#1}%
	\edef\pgfmathfloat@loc@TMPb{#2}%
	\expandafter\pgfmathfloat@decompose\pgfmathfloat@loc@TMPa\relax\pgfmathfloat@a@S\pgfmathfloat@a@M\pgfmathfloat@a@E
	\expandafter\pgfmathfloat@decompose\pgfmathfloat@loc@TMPb\relax\pgfmathfloat@b@S\pgfmathfloat@b@M\pgfmathfloat@b@E
	\ifcase\pgfmathfloat@a@S
		% x = 0 -> (x<y <=> y >0)
		\ifcase\pgfmathfloat@b@S
			% y = 0
		\or% y > 0
			\global\pgfmathfloatcomparisontrue
		\or% y < 0
		\or% y = nan
		\or% y = + infty
			\global\pgfmathfloatcomparisontrue
		\or% y = -infty
		\fi
	\or
		% x > 0 -> (x<y <=> ( y > 0 && |x| < |y|) )
		\ifcase\pgfmathfloat@b@S
			% y = 0
		\or% y>0:
			\pgfmathfloatlessthan@positive
		\or% y < 0
		\or% y = nan
		\or% y = + infty
			\global\pgfmathfloatcomparisontrue
		\or% y = -infty
		\fi
	\or
		% x < 0 -> (x<y <=> (y >= 0 ||   |x| > |y|) )
		\ifcase\pgfmathfloat@b@S
			% y = 0
			\global\pgfmathfloatcomparisontrue
		\or%y > 0
			\global\pgfmathfloatcomparisontrue
		\or% 'y<0':
			\pgfmathfloatgreaterthan@positive
		\or% y = nan
		\or% y = + infty
			\global\pgfmathfloatcomparisontrue
		\or% y = -infty
		\fi
	\or
		% x = nan.
	\or
		% x = +infty
	\or
		% x = -infty
		\ifnum\pgfmathfloat@b@S=3
		\else
			\global\pgfmathfloatcomparisontrue
		\fi
	\fi
	\endgroup
	\ifpgfmathfloatcomparison
		\def\pgfmathresult{1.0}%
	\else
		\def\pgfmathresult{0.0}%
	\fi
}
\let\pgfmathfloatlessthan=\pgfmathfloatlessthan@
\let\pgfmathfloatless@=\pgfmathfloatlessthan@

% ! (#1 < #2)  <=>  (#1 >= #2)
\def\pgfmathfloatnotless@#1#2{%
	\pgfmathfloatless@{#1}{#2}%
	\ifpgfmathfloatcomparison
		\def\pgfmathresult{0.0}%
	\else
		\def\pgfmathresult{1.0}%
	\fi
}%
% ! (#1 > #2)  <=>  (#1 <= #2)
\def\pgfmathfloatnotgreater@#1#2{%
	\pgfmathfloatless@{#2}{#1}%
	\ifpgfmathfloatcomparison
		\def\pgfmathresult{0.0}%
	\else
		\def\pgfmathresult{1.0}%
	\fi
}%

% compares \pgfmathfloat@a@[SME] < \pgfmathfloat@b@[SME]
\def\pgfmathfloatlessthan@positive{%
	\ifnum\pgfmathfloat@a@E<\pgfmathfloat@b@E
		\global\pgfmathfloatcomparisontrue
	\else
		\ifnum\pgfmathfloat@a@E=\pgfmathfloat@b@E
			\ifdim\pgfmathfloat@a@M<\pgfmathfloat@b@M
				\global\pgfmathfloatcomparisontrue
			\fi
		\fi
	\fi
}

% compares \pgfmathfloat@a@[SME] > \pgfmathfloat@b@[SME]
\def\pgfmathfloatgreaterthan@positive{%
	\ifnum\pgfmathfloat@a@E>\pgfmathfloat@b@E
		\global\pgfmathfloatcomparisontrue
	\else
		\ifnum\pgfmathfloat@a@E=\pgfmathfloat@b@E
			\ifdim\pgfmathfloat@a@M>\pgfmathfloat@b@M
				\global\pgfmathfloatcomparisontrue
			\fi
		\fi
	\fi
}


\def\pgfmathfloatgreaterthan@#1#2{\pgfmathfloatlessthan@{#2}{#1}}
\let\pgfmathfloatgreaterthan=\pgfmathfloatgreaterthan@
\let\pgfmathfloatgreater@=\pgfmathfloatgreaterthan@

\def\pgfmathfloatmax@#1{%
	\begingroup
		\pgfmathfloatcreate{2}{1.0}{2147483644}%
		\let\pgfmathmaxsofar=\pgfmathresult
		\pgfmathfloatmax@@#1{}%
}%
\def\pgfmathfloatmax@@#1{%
	\def\pgfmath@temp{#1}%
	\ifx\pgfmath@temp\pgfmath@empty%
		\expandafter\pgfmathfloatmax@@@%
	\else%
		\pgfmathfloatlessthan{\pgfmathmaxsofar}{#1}%
		\ifpgfmathfloatcomparison
			\edef\pgfmathmaxsofar{#1}%
		\fi
		\expandafter\pgfmathfloatmax@@%
	\fi%
}%
\def\pgfmathfloatmax@@@{%
	\let\pgfmathresult=\pgfmathmaxsofar
	\pgfmath@smuggleone{\pgfmathresult}%
	\endgroup
}%
\def\pgfmathfloatmin@#1{%
	\begingroup
		\pgfmathfloatcreate{1}{1.0}{2147483644}%
		\let\pgfmathminsofar=\pgfmathresult
		\pgfmathfloatmin@@#1{}%
}%
\def\pgfmathfloatmin@@#1{%
	\def\pgfmath@temp{#1}%
	\ifx\pgfmath@temp\pgfmath@empty%
		\expandafter\pgfmathfloatmin@@@%
	\else%
		\pgfmathfloatlessthan{#1}{\pgfmathminsofar}%
		\ifpgfmathfloatcomparison
			\edef\pgfmathminsofar{#1}%
		\fi
		\expandafter\pgfmathfloatmin@@%
	\fi%
}%
\def\pgfmathfloatmin@@@{%
	\let\pgfmathresult=\pgfmathminsofar
	\pgfmath@smuggleone{\pgfmathresult}%
	\endgroup
}%

\def\pgfmathfloatmaxtwo#1#2{%
	\pgfmathfloatlessthan{#1}{#2}%
	\ifpgfmathfloatcomparison
		\edef\pgfmathresult{#2}%
	\else
		\edef\pgfmathresult{#1}%
	\fi
}
\let\pgfmathfloatmax=\pgfmathfloatmaxtwo

\def\pgfmathfloatmintwo#1#2{%
	\pgfmathfloatlessthan{#1}{#2}%
	\ifpgfmathfloatcomparison
		\edef\pgfmathresult{#1}%
	\else
		\edef\pgfmathresult{#2}%
	\fi
}
\let\pgfmathfloatmin=\pgfmathfloatmintwo

% Renormalizes #1 to extended precision mantisse, meaning
% 100 <= m < 1000
% instead of 1 <= m < 10.
%
% The 'extended precision' means we have higher accuracy when we apply pgfmath operations to mantissas.
%
% The input argument is expected to be a normalized floating point number; the output argument is a non-normalized floating point number (well, normalized to extended precision).
%
% The operation is supposed to be very fast.
%
% @see \pgfmathfloatsetextprecision
%
% There is a routine for internal usage,
% \pgfmathfloattoextentedprecision@a. It also provides exponent and
% sign of #1 in output arguments and may be used to increase speed.
\def\pgfmathfloattoextentedprecision#1{%
	\begingroup
	\pgfmathfloattoextentedprecision@a{#1}%
	\pgfmathfloatcreate{\pgfmathfloat@a@S}{\pgfmathresult}{\pgfmathfloat@a@E}%
	\pgfmath@smuggleone\pgfmathresult
	\endgroup
}%

\def\pgfmathfloattoextentedprecision@@zero#1\pgfmathfloat@EOI{%
	\edef\pgfmathresult{#1}%
}%
\def\pgfmathfloattoextentedprecision@@one#1.#2#3\pgfmathfloat@EOI{%
	\edef\pgfmathresult{#1#2.#3}%
}%
\def\pgfmathfloattoextentedprecision@@two#1.#2#3#4\pgfmathfloat@EOI{%
	\edef\pgfmathresult{#1#2#3.#4}%
}%
\def\pgfmathfloattoextentedprecision@@three#1.#2#3#4#5\pgfmathfloat@EOI{%
	\edef\pgfmathresult{#1#2#3#4.#5}%
}%

% Sets extended precision to 10^#1.
%
% The different choices are
%
% - 0:  normalization      0 <= m < 1 (disable extended precision)
% - 1:  normalization     10 <= m < 100
% - 2:  normalization    100 <= m < 1000  (default)
% - 3:  normalization   1000 <= m < 10000
%
% #1 is the exponent, #1 = 0,1,2 or 3.
%
% This setting applies to \pgfmathfloattoextentedprecision and friends.
\def\pgfmathfloatsetextprecision#1{%
	\ifcase#1\relax
		\let\pgfmathfloattoextentedprecision@@=\pgfmathfloattoextentedprecision@@zero
		\def\pgfmathfloatextprec@shift{0}%
	\or
		\let\pgfmathfloattoextentedprecision@@=\pgfmathfloattoextentedprecision@@one
		\def\pgfmathfloatextprec@shift{1}%
	\or
		\let\pgfmathfloattoextentedprecision@@=\pgfmathfloattoextentedprecision@@two
		\def\pgfmathfloatextprec@shift{2}%
	\else
		\let\pgfmathfloattoextentedprecision@@=\pgfmathfloattoextentedprecision@@three
		\def\pgfmathfloatextprec@shift{3}%
	\fi
}%
\pgfmathfloatsetextprecision{2}%

% Does the "hard" work for \pgfmathfloattoextentedprecision. It
% provides additional outputs.
%
% INPUT:
% #1  normalized floating point number. Maybe a macro (it will be expanded ONCE)
%
% OUTPUT:
% - \pgfmathresult : the mantisse in extended precision
% - \pgfmathfloat@a@S : the sign of #1
% - \pgfmathfloat@a@E : the exponent of #1, adjusted for extended precision
% - \pgfmathfloat@a@Mtok : undefined (its contents will be destroyed.
%
\def\pgfmathfloattoextentedprecision@a#1{%
	\edef\pgfmathresult{#1}%
	\expandafter\pgfmathfloat@decompose@tok\pgfmathresult\relax\pgfmathfloat@a@S\pgfmathfloat@a@Mtok\pgfmathfloat@a@E
	\ifnum\pgfmathfloat@a@S<3
		\advance\pgfmathfloat@a@E by-\pgfmathfloatextprec@shift\relax% compensate for shift
		\expandafter\pgfmathfloattoextentedprecision@@\the\pgfmathfloat@a@Mtok 000\pgfmathfloat@EOI
	\fi
}%


% Similar to \pgfmathfloattoextentedprecision@a, this one here fills the '@b' registers.
\def\pgfmathfloattoextentedprecision@b#1{%
	\edef\pgfmathresult{#1}%
	\expandafter\pgfmathfloat@decompose@tok\pgfmathresult\relax\pgfmathfloat@b@S\pgfmathfloat@a@Mtok\pgfmathfloat@b@E
	\ifnum\pgfmathfloat@b@S<3
		\advance\pgfmathfloat@b@E by-\pgfmathfloatextprec@shift\relax
		\expandafter\pgfmathfloattoextentedprecision@@\the\pgfmathfloat@a@Mtok 00\pgfmathfloat@EOI
	\fi
}%

% Addition of two floating point numbers using 8 significant digits.
\def\pgfmathfloatadd@#1#2{%
	\begingroup
	%
	% renormalize argument to 100 <= m < 1000 for extended accuracy:
	\pgfmathfloattoextentedprecision@a{#1}%
	\let\pgfmathfloat@arga=\pgfmathresult
	%
	\pgfmathfloattoextentedprecision@b{#2}%
	\let\pgfmathfloat@argb=\pgfmathresult
	%
	\pgfmathfloatcomparisontrue% re-use this boolean here to handle special cases.
	\ifcase\pgfmathfloat@a@S
		\edef\pgfmathresult{#2}%
		\pgfmathfloatcomparisonfalse
	\or
	\or
		\edef\pgfmathfloat@arga{-\pgfmathfloat@arga}%
	\else
		\pgfmathfloatcomparisonfalse
		\pgfmathfloatcreate{\the\pgfmathfloat@a@S}{0.0}{0}%
	\fi
	\ifcase\pgfmathfloat@b@S
		\edef\pgfmathresult{#1}%
		\pgfmathfloatcomparisonfalse
	\or
	\or
		\edef\pgfmathfloat@argb{-\pgfmathfloat@argb}%
	\else
		\pgfmathfloatcomparisonfalse
		\pgfmathfloatcreate{\the\pgfmathfloat@b@S}{0.0}{0}%
	\fi
	\ifpgfmathfloatcomparison
		% Shift lesser mantisse to fit the larger one:
		\ifnum\pgfmathfloat@a@E<\pgfmathfloat@b@E
			\pgfmathfloatadd@shift{\pgfmathfloat@arga}{\pgfmathfloat@a@E}{\pgfmathfloat@b@E}%
		\else
			\pgfmathfloatadd@shift{\pgfmathfloat@argb}{\pgfmathfloat@b@E}{\pgfmathfloat@a@E}%
		\fi
		% add them!
		\pgfmath@basic@add@{\pgfmathfloat@arga}{\pgfmathfloat@argb}%
		% renormalize sum. This is the only part were an expensive routine comes into play:
		\edef\pgfmathresult{\pgfmathresult e\the\pgfmathfloat@a@E}%
		\expandafter\pgfmathfloatqparsenumber\expandafter{\pgfmathresult}%
	\fi
	\pgfmath@smuggleone\pgfmathresult
	\endgroup
}%

% #1= floating point number
% #2= TeX code to execute if #1 == 0
% #3= TeX code to execute if #1 != 0
\def\pgfmathfloatifzero#1#2#3{%
	\pgfmathfloatgetflagstomacro{#1}\pgfmathfloat@loc@TMPa
	\if\pgfmathfloat@loc@TMPa0 #2\else#3\fi
}%

\def\pgfmathfloatiffinite#1#2#3{%
	\pgfmathfloatgetflagstomacro{#1}\pgfmathfloatiffinite@
	\ifnum\pgfmathfloatiffinite@>2 #3\else #2\fi
}%

\def\pgfmathfloatifthenelse@#1#2#3{%
	\pgfmathfloatifflags{#1}{0}{%
		\edef\pgfmathresult{#3}%%
	}{%
		\edef\pgfmathresult{#2}%
	}%
}
\def\pgfmathfloatequal@#1#2{%
	\pgfmathfloatifapproxequalrel{#1}{#2}{%
		\def\pgfmathresult{1}%
		\pgfmathfloatcomparisontrue
	}{%
		\def\pgfmathresult{0}%
		\pgfmathfloatcomparisonfalse
	}%
}
\let\pgfmathfloatequalto@=\pgfmathfloatequal@

\def\pgfmathfloatnotequal@#1#2{%
	\pgfmathfloatifapproxequalrel{#1}{#2}{%
		\def\pgfmathresult{0}%
		\pgfmathfloatcomparisonfalse
	}{%
		\def\pgfmathresult{1}%
		\pgfmathfloatcomparisontrue
	}%
}
\let\pgfmathfloatnotequalto@=\pgfmathfloatnotequal@

% Computes the relative error between #1 and #2 (assuming #2 != 0) and
% invokes #3 if the relative error is below `/pgf/fpu/rel thresh' and
% #4 if that is not the case.
\long\def\pgfmathfloatifapproxequalrel#1#2#3#4{%
	\begingroup
	\pgfmathfloatparsenumber{#1}%
	\let\pgfmathfloatarga=\pgfmathresult
	\pgfmathfloatparsenumber{#2}%
	\let\pgfmathfloatargb=\pgfmathresult
	\pgfmathfloatrelerror@\pgfmathfloatarga\pgfmathfloatargb
	\let\pgfmathfloatarga=\pgfmathresult
	\pgfmathfloatlessthan@\pgfmathfloatarga\pgfmathfloat@relthresh
	\ifpgfmathfloatcomparison
		\def\pgfmathfloat@loc@TMPa{#3}%
	\else
		\def\pgfmathfloat@loc@TMPa{#4}%
	\fi
	\expandafter\endgroup
	\pgfmathfloat@loc@TMPa
}%

% Invokes code '#3' if the flags of the floating point number '#1'
% match the flag provided in '#2'.
%
% \pgfmathfloatcreate{1}{1.0}{2}
% \pgfmathfloatifflags{\pgfmathresult}{0}{It's zero!}{It's not zero!}%
% \pgfmathfloatifflags{\pgfmathresult}{1}{It's positive!}{It's not positive!}%
% \pgfmathfloatifflags{\pgfmathresult}{2}{It's negative!}{It's not negative!}%
% or, equivalently
% \pgfmathfloatifflags{\pgfmathresult}{+}{It's positive!}{It's not positive!}%
% \pgfmathfloatifflags{\pgfmathresult}{-}{It's negative!}{It's not negative!}%
% it also supports #2=u which means 'unbounded'
\def\pgfmathfloatifflags#1#2#3#4{%
	\if#2-%
		\pgfmathfloatifflags{#1}{2}{#3}{#4}%
	\else
		\if#2+%
			\pgfmathfloatifflags{#1}{1}{#3}{#4}%
		\else
			\pgfmathfloatgetflagstomacro{#1}\pgfmathfloat@loc@TMPa
			\if#2u%
				\ifnum\pgfmathfloat@loc@TMPa>2
					#3\relax
				\else
					#4\relax
				\fi
			\else
				\if\pgfmathfloat@loc@TMPa#2%
					#3\relax
				\else
					#4\relax
				\fi
			\fi
		\fi
	\fi
}%

% #1=mantisse which needs to be shifted (with smaller exponent)
% #2=smaller exponent
% #3=larger exponent
%
% ATTENTION: this helper method DESTROYS contents of \pgfmathfloat@a@S.
\def\pgfmathfloatadd@shift#1#2#3{%
	\pgf@xa=#1 pt%
	\pgfmathfloat@a@S=#3\relax
	\advance\pgfmathfloat@a@S by-#2\relax
	\ifcase\pgfmathfloat@a@S
	\or
		\divide\pgf@xa by10\relax
	\or
		\divide\pgf@xa by100\relax
	\or
		\divide\pgf@xa by1000\relax
	\or
		\divide\pgf@xa by10000\relax
	\or
		\divide\pgf@xa by10000\relax
		\divide\pgf@xa by10\relax
	\or
		\divide\pgf@xa by10000\relax
		\divide\pgf@xa by100\relax
	\or
		\divide\pgf@xa by10000\relax
		\divide\pgf@xa by1000\relax
	\or
		\divide\pgf@xa by10000\relax
		\divide\pgf@xa by10000\relax
	\else
		\pgf@xa=0pt%
	\fi
	#2=#3\relax
	\edef#1{\pgf@sys@tonumber\pgf@xa}%
}

\let\pgfmathfloatadd=\pgfmathfloatadd@


% Subtracts two floating point numbers.
\def\pgfmathfloatsubtract@#1#2{%
	\begingroup
	\edef\pgfmathresult{#2}%
	\expandafter\pgfmathfloat@decompose@tok\pgfmathresult\relax\pgfmathfloat@b@S\pgfmathfloat@a@Mtok\pgfmathfloat@b@E
	\ifcase\pgfmathfloat@b@S
		\edef\pgfmathresult{#1}%
	\or
		\pgfmathfloatcreate{2}{\the\pgfmathfloat@a@Mtok}{\the\pgfmathfloat@b@E}%
		\let\pgfmathfloatsub@arg=\pgfmathresult
		\pgfmathfloatadd@{#1}{\pgfmathfloatsub@arg}%
	\or
		\pgfmathfloatcreate{1}{\the\pgfmathfloat@a@Mtok}{\the\pgfmathfloat@b@E}%
		\let\pgfmathfloatsub@arg=\pgfmathresult
		\pgfmathfloatadd@{#1}{\pgfmathfloatsub@arg}%
	\else
		\pgfmathfloatcreate{\the\pgfmathfloat@b@S}{0.0}{0}%
	\fi
	\pgfmath@smuggleone\pgfmathresult
	\endgroup
}%

\let\pgfmathfloatsubtract=\pgfmathfloatsubtract@

% Scales a floating point number #1 with a fixed point number #2 using pgfmathmultiply.
%
% Use this method if #2 is small number.
\def\pgfmathfloatmultiplyfixed@#1#2{%
	\begingroup
	%
	% renormalize argument to 100 <= m < 1000 for extended accuracy:
	\pgfmathfloattoextentedprecision@a{#1}%
	\let\pgfmathfloat@arga=\pgfmathresult
	%
	\pgfmathfloatcomparisontrue% re-use this boolean here to handle special cases.
	\ifcase\pgfmathfloat@a@S
		\edef\pgfmathresult{#1}%
		\pgfmathfloatcomparisonfalse
	\or
	\or
		\edef\pgfmathfloat@arga{-\pgfmathfloat@arga}%
	\else
		\pgfmathfloatcomparisonfalse
		\pgfmathfloatcreate{\the\pgfmathfloat@a@S}{0.0}{0}%
	\fi
	\ifpgfmathfloatcomparison
		\pgfmath@basic@multiply@{\pgfmathfloat@arga}{#2}%
		% renormalize product. This is the only part were an expensive routine comes into play:
		\edef\pgfmathresult{\pgfmathresult e\the\pgfmathfloat@a@E}%
		\expandafter\pgfmathfloatqparsenumber\expandafter{\pgfmathresult}%
	\fi
	\pgfmath@smuggleone\pgfmathresult
	\endgroup
}%

\let\pgfmathfloatmultiplyfixed=\pgfmathfloatmultiplyfixed@


\def\pgfmathfloatmultiply@#1#2{%
	\begingroup
	\pgfmathfloatsetextprecision{1}%
	\pgfmathfloattoextentedprecision@a{#1}%
	\let\pgfmathfloat@arga=\pgfmathresult
	%
	\pgfmathfloattoextentedprecision@b{#2}%
	\let\pgfmathfloat@argb=\pgfmathresult
	%
	\pgfmathfloatcomparisontrue% re-use this boolean here to handle special cases.
	\ifcase\pgfmathfloat@a@S
	% 0
		\pgfmathfloatcreate{0}{0.0}{0}%
		\pgfmathfloatcomparisonfalse
	\or% +
		\ifcase\pgfmathfloat@b@S
			\pgfmathfloatcreate{0}{0.0}{0}%
			\pgfmathfloatcomparisonfalse
		\or
			\def\pgfmathresult@S{1}%
		\or
			\def\pgfmathresult@S{2}%
		\else
			\expandafter\pgfmathfloatcreate\the\pgfmathfloat@b@S{0.0}{0}%
			\pgfmathfloatcomparisonfalse
		\fi
	\or% -
		\ifcase\pgfmathfloat@b@S
			\pgfmathfloatcreate{0}{0.0}{0}%
			\pgfmathfloatcomparisonfalse
		\or
			\def\pgfmathresult@S{2}%
		\or
			\def\pgfmathresult@S{1}%
		\or
			\pgfmathfloatcreate{3}{0.0}{0}%
			\pgfmathfloatcomparisonfalse
		\or
			\pgfmathfloatcreate{5}{0.0}{0}%
			\pgfmathfloatcomparisonfalse
		\or
			\pgfmathfloatcreate{4}{0.0}{0}%
			\pgfmathfloatcomparisonfalse
		\fi
	\or% nan
		\pgfmathfloatcreate{3}{0.0}{0}%
		\pgfmathfloatcomparisonfalse
	\or% +infty
		\ifcase\pgfmathfloat@b@S
			\pgfmathfloatcreate{0}{0.0}{0}%
			\pgfmathfloatcomparisonfalse
		\or
			\pgfmathfloatcreate{4}{0.0}{0}%
			\pgfmathfloatcomparisonfalse
		\or
			\pgfmathfloatcreate{5}{0.0}{0}%
			\pgfmathfloatcomparisonfalse
		\or
			\pgfmathfloatcreate{3}{0.0}{0}%
			\pgfmathfloatcomparisonfalse
		\or
			\pgfmathfloatcreate{4}{0.0}{0}%
			\pgfmathfloatcomparisonfalse
		\or
			\pgfmathfloatcreate{5}{0.0}{0}%
			\pgfmathfloatcomparisonfalse
		\fi
	\or% -infty
		\ifcase\pgfmathfloat@b@S
			\pgfmathfloatcreate{0}{0.0}{0}%
			\pgfmathfloatcomparisonfalse
		\or
			\pgfmathfloatcreate{5}{0.0}{0}%
			\pgfmathfloatcomparisonfalse
		\or
			\pgfmathfloatcreate{4}{0.0}{0}%
			\pgfmathfloatcomparisonfalse
		\or
			\pgfmathfloatcreate{3}{0.0}{0}%
			\pgfmathfloatcomparisonfalse
		\or
			\pgfmathfloatcreate{5}{0.0}{0}%
			\pgfmathfloatcomparisonfalse
		\or
			\pgfmathfloatcreate{4}{0.0}{0}%
			\pgfmathfloatcomparisonfalse
		\fi
	\fi
	\ifpgfmathfloatcomparison
		\pgfmath@basic@multiply@{\pgfmathfloat@arga}{\pgfmathfloat@argb}%
		\advance\pgfmathfloat@a@E by\pgfmathfloat@b@E
		% renormalize sum. This is the only part were an expensive routine comes into play:
		\edef\pgfmathresult{\pgfmathresult e\the\pgfmathfloat@a@E}%
		\expandafter\pgfmathfloatqparsenumber\expandafter{\pgfmathresult}%
		\expandafter\pgfmathfloat@decompose@tok\pgfmathresult\relax\pgfmathfloat@a@S\pgfmathfloat@a@Mtok\pgfmathfloat@a@E
		\pgfmathfloatcreate{\pgfmathresult@S}{\the\pgfmathfloat@a@Mtok}{\the\pgfmathfloat@a@E}%
	\fi
	\pgfmath@smuggleone\pgfmathresult
	\endgroup
}%
\let\pgfmathfloatmultiply=\pgfmathfloatmultiply@

% Defines \pgfmathresult to be #1 / #2 for two floating point numbers.
%
% It employs the basic math engine internally to divide mantissas.
\def\pgfmathfloatdivide@#1#2{%
	\begingroup
	\pgfmathfloatsetextprecision{1}% is not too important, I think. After all, 0.1 <= #1/#2 < 10 or so due to normalization (no matter, which)
	\edef\pgfmathfloat@arga{#1}%
	\pgfmathfloattoextentedprecision@a{\pgfmathfloat@arga}%
	\let\pgfmathfloat@arga=\pgfmathresult
	%
	\edef\pgfmathfloat@argb{#2}%
	\pgfmathfloattoextentedprecision@b{\pgfmathfloat@argb}%
	\let\pgfmathfloat@argb=\pgfmathresult
	%
	\pgfmathfloatcomparisontrue% re-use this boolean here to handle special cases.
	\ifcase\pgfmathfloat@a@S
	% 0
		\pgfmathfloatcreate{0}{0.0}{0}%
		\pgfmathfloatcomparisonfalse
	\or% +
		\ifcase\pgfmathfloat@b@S
			\pgfmathfloatcreate{4}{0.0}{0}%
			\pgfmathfloatcomparisonfalse
		\or
			\def\pgfmathresult@S{1}%
		\or
			\def\pgfmathresult@S{2}%
		\or
			\pgfmathfloatcreate{3}{0.0}{0}%
			\pgfmathfloatcomparisonfalse
		\else
			\pgfmathfloatcreate{0}{0.0}{0}%
			\pgfmathfloatcomparisonfalse
		\fi
	\or% -
		\ifcase\pgfmathfloat@b@S
			\pgfmathfloatcreate{5}{0.0}{0}%
			\pgfmathfloatcomparisonfalse
		\or
			\def\pgfmathresult@S{2}%
		\or
			\def\pgfmathresult@S{1}%
		\or
			\pgfmathfloatcreate{3}{0.0}{0}%
			\pgfmathfloatcomparisonfalse
		\else
			\pgfmathfloatcreate{0}{0.0}{0}%
			\pgfmathfloatcomparisonfalse
		\fi
	\or% nan
		\pgfmathfloatcreate{3}{0.0}{0}%
		\pgfmathfloatcomparisonfalse
	\or% +infty
		\ifcase\pgfmathfloat@b@S
			\pgfmathfloatcreate{4}{0.0}{0}%
			\pgfmathfloatcomparisonfalse
		\or
			\pgfmathfloatcreate{4}{0.0}{0}%
			\pgfmathfloatcomparisonfalse
		\or
			\pgfmathfloatcreate{5}{0.0}{0}%
			\pgfmathfloatcomparisonfalse
		\or
			\pgfmathfloatcreate{3}{0.0}{0}%
			\pgfmathfloatcomparisonfalse
		\or
			\pgfmathfloatcreate{4}{0.0}{0}% what is inf/inf ?
			\pgfmathfloatcomparisonfalse
		\or
			\pgfmathfloatcreate{5}{0.0}{0}% or inf/-inf ?
			\pgfmathfloatcomparisonfalse
		\fi
	\or% -infty
		\ifcase\pgfmathfloat@b@S
			\pgfmathfloatcreate{5}{0.0}{0}%
			\pgfmathfloatcomparisonfalse
		\or
			\pgfmathfloatcreate{5}{0.0}{0}%
			\pgfmathfloatcomparisonfalse
		\or
			\pgfmathfloatcreate{4}{0.0}{0}%
			\pgfmathfloatcomparisonfalse
		\or
			\pgfmathfloatcreate{3}{0.0}{0}%
			\pgfmathfloatcomparisonfalse
		\or
			\pgfmathfloatcreate{5}{0.0}{0}%
			\pgfmathfloatcomparisonfalse
		\or
			\pgfmathfloatcreate{4}{0.0}{0}%
			\pgfmathfloatcomparisonfalse
		\fi
	\fi
	\ifpgfmathfloatcomparison
		\pgfmath@basic@divide@{\pgfmathfloat@arga}{\pgfmathfloat@argb}%
		\advance\pgfmathfloat@a@E by-\pgfmathfloat@b@E
		% renormalize. This is the only part were an expensive float routine comes into play:
		\edef\pgfmathresult{\pgfmathresult e\the\pgfmathfloat@a@E}%
		\expandafter\pgfmathfloatqparsenumber\expandafter{\pgfmathresult}%
		% And re-insert the proper sign:
		\expandafter\pgfmathfloat@decompose@tok\pgfmathresult\relax\pgfmathfloat@a@S\pgfmathfloat@a@Mtok\pgfmathfloat@a@E
		\pgfmathfloatcreate{\pgfmathresult@S}{\the\pgfmathfloat@a@Mtok}{\the\pgfmathfloat@a@E}%
	\fi
	\pgfmath@smuggleone\pgfmathresult
	\endgroup
}%
\let\pgfmathfloatdivide=\pgfmathfloatdivide@

\def\pgfmathfloatreciprocal@#1{%
	\begingroup
	% FIXME optimize
	\edef\pgfmathfloat@loc@TMPa{#1}%
	\pgfmathfloatcreate{1}{1.0}{0}%
	\pgfmathfloatdivide@{\pgfmathresult}{\pgfmathfloat@loc@TMPa}%
	\pgfmath@smuggleone\pgfmathresult
	\endgroup
}%

% Computes sqrt(#1) in floating point arithmetics.
%
% It employs sqrt( m * 10^e ) = sqrt(m) * sqrt(10^e).
\def\pgfmathfloatsqrt@#1{%
	\begingroup
	\pgfmathfloatsetextprecision{3}%
	\edef\pgfmathfloat@arga{#1}%
	\pgfmathfloattoextentedprecision@a{\pgfmathfloat@arga}%
	\let\pgfmathfloat@arga=\pgfmathresult
	%
	\ifcase\pgfmathfloat@a@S
	% 0
		\pgfmathfloatcreate{0}{0.0}{0}%
	\or% +
		\pgfmath@basic@sqrt@{\pgfmathfloat@arga}%
		\ifodd\pgfmathfloat@a@E
			\ifnum\pgfmathfloat@a@E>0
				\expandafter\pgfmath@basic@multiply@\expandafter{\pgfmathresult}{3.16227766}% * sqrt(10)
			\else
				\expandafter\pgfmath@basic@multiply@\expandafter{\pgfmathresult}{0.316227766}% * sqrt(0.1)
			\fi
		\fi
		\divide\pgfmathfloat@a@E by2 % sqrt(10^e) = 10^{e/2} (see above for odd e)
		% renormalize sum. This is the only part were an expensive routine comes into play:
		\edef\pgfmathfloat@arga{\pgfmathresult e\the\pgfmathfloat@a@E}%
		\pgfmathfloatqparsenumber{\pgfmathfloat@arga}%
	\or% -
		\pgfmathfloatcreate{3}{0.0}{0}%
	\or% nan
		\pgfmathfloatcreate{3}{0.0}{0}%
	\or% +infty
		\pgfmathfloatcreate{4}{0.0}{0}%
	\or% -infty
		\pgfmathfloatcreate{3}{0.0}{0}%
	\fi
	\pgfmath@smuggleone\pgfmathresult
	\endgroup
}%
\let\pgfmathfloatsqrt=\pgfmathfloatsqrt@

% Returns the integer part of the floating point number #1.
%
% The result is returned as floating point as well.
%
% This operation is not limited to TeX's range of count registers (it
% works symbolly)
%
% @see \pgfmathfloattoint
% POSTCONDITION: \pgfmathresult contains the result and
%	\pgfmathfloatintwasnoop=1 if there was nothing to do
%   \pgfmathfloatintwasnoop=0 if there where non-zero digits after the period
%   \pgfmathfloatintwasnoop=2 if there where digits after the period. The digits will be stored in \pgfmathfloatintremainder in this case.
\def\pgfmathfloatint@#1{%
	\begingroup
	\edef\pgfmathfloatint@input{#1}%
	\expandafter\pgfmathfloat@decompose@tok\pgfmathfloatint@input\relax\pgfmathfloat@a@S\pgfmathfloat@a@Mtok\pgfmathfloat@a@E
	\gdef\pgfmathfloatintwasnoop{1}%
	\gdef\pgfmathfloatintremainder{}%
	\ifcase\pgfmathfloat@a@S
		% 0: nothing to do.
	\or% +
		\expandafter\pgfmathfloatint@@\the\pgfmathfloat@a@Mtok\pgfmathfloat@EOI
		\pgfmathfloatcreate{\the\pgfmathfloat@a@S}{\the\pgfmathfloat@a@Mtok}{\the\pgfmathfloat@a@E}%
	\or% -
		\expandafter\pgfmathfloatint@@\the\pgfmathfloat@a@Mtok\pgfmathfloat@EOI
		\pgfmathfloatcreate{\the\pgfmathfloat@a@S}{\the\pgfmathfloat@a@Mtok}{\the\pgfmathfloat@a@E}%
	\else
		% nothing to do
	\fi
	%\message{ XXXXX int(\pgfmathfloatint@input) = \pgfmathresult -> was no op = \pgfmathfloatintwasnoop\space (remainder \pgfmathfloatintremainder)^^J}%
	\pgfmath@smuggleone\pgfmathresult
	\endgroup
}%
\def\pgfmathfloatint@@#1.{%
	\ifnum\pgfmathfloat@a@E<0
		\pgfmathfloat@a@S=0
		\pgfmathfloat@a@Mtok={0.0}%
		\pgfmathfloat@a@E=0
		\gdef\pgfmathfloatintwasnoop{0}%
		\expandafter\pgfmathfloatint@@loop@gobble
	\else
		\pgfmathfloat@a@Mtok={#1.}%
		\pgfmathfloat@b@E=\pgfmathfloat@a@E
		\expandafter\pgfmathfloatint@@loop
	\fi
}%
\def\pgfmathfloatint@@loop#1{%
	\def\pgfmathfloatint@@loop@{#1}%
	\ifx\pgfmathfloatint@@loop@\pgfmathfloat@EOI
		\gdef\pgfmathfloatintwasnoop{1}%
		\let\pgfmathfloatint@@loop@next=\relax
	\else
		\ifnum\pgfmathfloat@b@E=0
			\def\pgfmathfloatint@@loop@next{\pgfmathfloatint@@loop@gobble#1}%
		\else
			\pgfmathfloat@a@Mtok=\expandafter{\the\pgfmathfloat@a@Mtok#1}%
			\advance\pgfmathfloat@b@E by-1
			\let\pgfmathfloatint@@loop@next=\pgfmathfloatint@@loop
		\fi
	\fi
	\pgfmathfloatint@@loop@next
}%
\def\pgfmathfloatint@@loop@gobble#1\pgfmathfloat@EOI{%
	\if0\pgfmathfloatintwasnoop
	\else
		\gdef\pgfmathfloatintwasnoop{2}%
		\gdef\pgfmathfloatintremainder{#1}%
	\fi
}%
\let\pgfmathfloatint=\pgfmathfloatint@

\def\pgfmathfloatfloor#1{%
	\edef\pgfmath@orig{#1}%
	\pgfmathfloatint@{#1}%
	\pgfmathfloatifflags{\pgfmath@orig}{2}{%
		\let\pgfmath@trunc=\pgfmathresult
		\ifcase\pgfmathfloatintwasnoop\relax
			% ah - we stripped something! Round DOWN
			\pgfmathfloatcreate{2}{1.0}{0}% -1
			\expandafter\pgfmathfloatadd@\expandafter{\pgfmathresult}{\pgfmath@trunc}%
		\or
			% was no-op
			\let\pgfmathresult=\pgfmath@trunc
		\else
			% ok, we have to inspect the remainder:
			\pgfmathfloatparsenumber{0.\pgfmathfloatintremainder}%
			\pgfmathfloatifflags{\pgfmathresult}{1}{%
				% ah - we stripped a non-zero remainder! Round DOWN
				\pgfmathfloatcreate{2}{1.0}{0}% -1
				\expandafter\pgfmathfloatadd@\expandafter{\pgfmathresult}{\pgfmath@trunc}%
			}{%
				% was no-op
				\let\pgfmathresult=\pgfmath@trunc
			}%
		\fi
	}{}%
}
\let\pgfmathfloatfloor@=\pgfmathfloatfloor

\def\pgfmathfloatceil#1{%
	\edef\pgfmath@orig{#1}%
	\pgfmathfloatint@{#1}%
	\pgfmathfloatifflags{\pgfmath@orig}{1}{%
		\let\pgfmath@trunc=\pgfmathresult
		\ifcase\pgfmathfloatintwasnoop\relax
			% ah - we stripped something! Round UP
			\pgfmathfloatcreate{1}{1.0}{0}% +1
			\expandafter\pgfmathfloatadd@\expandafter{\pgfmathresult}{\pgfmath@trunc}%
		\or
			% was no-op
			\let\pgfmathresult=\pgfmath@trunc
		\else
			% ok, we have to inspect the remainder:
			\pgfmathfloatparsenumber{0.\pgfmathfloatintremainder}%
			\pgfmathfloatifflags{\pgfmathresult}{1}{%
				% ah - we stripped a non-zero remainder! Round UP
				\pgfmathfloatcreate{1}{1.0}{0}% +1
				\expandafter\pgfmathfloatadd@\expandafter{\pgfmathresult}{\pgfmath@trunc}%
			}{%
				% was no-op
				\let\pgfmathresult=\pgfmath@trunc
			}%
		\fi
	}{}%
}
\let\pgfmathfloatceil@=\pgfmathfloatceil

\def\pgfmathfloat@notimplemented#1{%
	\pgfmath@error{Sorry, the operation '#1' has not yet been implemented in the floating point unit}{}%
	\pgfmathfloatcreate{0}{0.0}{0}%
}%

% Divides or multiplies the input number by 10^#4 using an arithmetic
% left/right shift.
%
% Input:
% #1 a normalised floating point number.
% #2 a positive or negative integer number denoting the shift.
%
% Example:
% \pgfmathfloatshift{11e3}{4}%
% -> pgfmathresult = 11e7
\def\pgfmathfloatshift@#1#2{%
	\begingroup
	\expandafter\pgfmathfloat@decompose@tok#1\relax\pgfmathfloat@a@S\pgfmathfloat@a@Mtok\pgfmathfloat@a@E
	\advance\pgfmathfloat@a@E by#2\relax
	\pgfmathfloatcreate{\the\pgfmathfloat@a@S}{\the\pgfmathfloat@a@Mtok}{\the\pgfmathfloat@a@E}%
	\pgfmath@smuggleone\pgfmathresult
	\endgroup
}
\let\pgfmathfloatshift=\pgfmathfloatshift@

% Defines \pgfmathresult to be |#1|, the absolute value of the
% normalized floating point number #1.
\def\pgfmathfloatabs@#1{%
	\begingroup
	\expandafter\pgfmathfloat@decompose@tok#1\relax\pgfmathfloat@a@S\pgfmathfloat@a@Mtok\pgfmathfloat@a@E
	\ifcase\pgfmathfloat@a@S
		% 0: do nothing.
	\or
		% +: ok, is positive.
	\or
		% -: multiply with -1:
		\pgfmathfloat@a@S=1
	\or
		% nan: do nothing.
	\or
		% +infty: ok.
	\or
		% -infty: multiply with -1:
		\pgfmathfloat@a@S=4
	\fi
	\pgfmathfloatcreate{\the\pgfmathfloat@a@S}{\the\pgfmathfloat@a@Mtok}{\the\pgfmathfloat@a@E}%
	\pgfmath@smuggleone\pgfmathresult
	\endgroup
}%
%
% Defines \pgfmathresult to be sign(#1)
\def\pgfmathfloatsign@#1{%
	\begingroup
	\expandafter\pgfmathfloat@decompose@tok#1\relax\pgfmathfloat@a@S\pgfmathfloat@a@Mtok\pgfmathfloat@a@E
	\ifcase\pgfmathfloat@a@S
		% 0:
		\pgfmathfloatcreate{0}{0.0}{0}%
	\or
		% +: ok, is positive.
		\pgfmathfloatcreate{1}{1.0}{0}%
	\or
		% -:
		\pgfmathfloatcreate{2}{1.0}{0}%
	\or
		% nan: do nothing.
		\pgfmathfloatcreate{\the\pgfmathfloat@a@S}{\the\pgfmathfloat@a@Mtok}{\the\pgfmathfloat@a@E}%
	\or
		% +infty:.
		\pgfmathfloatcreate{1}{1.0}{0}%
	\or
		% -infty:
		\pgfmathfloatcreate{2}{1.0}{0}%
	\fi
	\pgfmath@smuggleone\pgfmathresult
	\endgroup
}%
\let\pgfmathfloatsign=\pgfmathfloatsign@

% Computes the absolute error |#1 - #2| into \pgfmathresult.
\def\pgfmathfloatabserror@#1#2{%
	\pgfmathfloatsubtract@{#1}{#2}%
	\pgfmathfloatabs@{\pgfmathresult}%
}%
\let\pgfmathfloatabserror=\pgfmathfloatabserror@

% Computes the relative error |#1 - #2|/|#2| into \pgfmathresult,
% assuming #2 != 0.
\def\pgfmathfloatrelerror@#1#2{%
	\pgfmathfloatsubtract@{#1}{#2}%
	\let\pgfmathfloat@subtract=\pgfmathresult
	\pgfmathfloatifflags{#2}{0}{%
		\let\pgfmathresult=\pgfmathfloat@subtract
	}{%
		\pgfmathfloatdivide@{\pgfmathfloat@subtract}{#2}%
	}%
	\pgfmathfloatabs@{\pgfmathresult}%
}%
\let\pgfmathfloatrelerror=\pgfmathfloatrelerror@

% Computes \pgfmathresult = #1 mod #2 using truncated division.
%
\def\pgfmathfloatmod@#1#2{%
	\begingroup
	\pgfmathfloattoint{#1}%
	\let\pgfmathfloat@loc@TMPa=\pgfmathresult
	\pgfmathfloattoint{#2}%
	\let\pgfmathfloat@loc@TMPb=\pgfmathresult
	\c@pgfmath@counta=\pgfmathfloat@loc@TMPa\relax
	\divide\c@pgfmath@counta by\pgfmathfloat@loc@TMPb\relax
	\expandafter\pgfmathfloatparsenumber\expandafter{\the\c@pgfmath@counta}%
	%
	\let\pgfmathfloat@loc@TMPa=\pgfmathresult
	\pgfmathfloatmultiply@{\pgfmathfloat@loc@TMPa}{#2}%
	\let\pgfmathfloat@loc@TMPb=\pgfmathresult
	\pgfmathfloatsubtract@{#1}{\pgfmathfloat@loc@TMPb}%
	\pgfmath@smuggleone\pgfmathresult
	\endgroup
}
\let\pgfmathfloatmod=\pgfmathfloatmod@


% A modification of \pgfmathfloatmod@ where #3 = 1/#2 is already
% known. This may be faster.
\def\pgfmathfloatmodknowsinverse@#1#2#3{%
	\pgfmathfloatmod@{#1}{#2}%
	%--------------------------------------------------
	% \begingroup
	% % FIXME : is this function correct? \pgfmathfloatmod had a
	% % rounding flaw...
	% \pgfmathfloatmultiply@{#1}{#3}%
	% \pgfmathfloatint@{\pgfmathresult}%
	% \let\pgfmathfloat@loc@TMPa=\pgfmathresult
	% \pgfmathfloatmultiply@{\pgfmathfloat@loc@TMPa}{#2}%
	% \let\pgfmathfloat@loc@TMPb=\pgfmathresult
	% \pgfmathfloatsubtract@{#1}{\pgfmathfloat@loc@TMPb}%
	% \pgfmath@smuggleone\pgfmathresult
	% \endgroup
	%--------------------------------------------------
}
\let\pgfmathfloatmodknowsinverse=\pgfmathfloatmodknowsinverse@

\def\pgfmathfloatpi@{%
	\pgfmathfloatcreate{1}{3.14159265358979}{0}%
}%
\let\pgfmathfloatpi=\pgfmathfloatpi@

\def\pgfmathfloate@{%
	\pgfmathfloatcreate{1}{2.71828182845905}{0}%
}
\let\pgfmathfloate=\pgfmathfloate@

% Converts #1 from radians to degrees.
\def\pgfmathfloatdeg@#1{%
	\expandafter\ifx\csname pgfmfltdeg@factor\endcsname\relax
		% Lazy evaluation:
		\pgfmathfloatcreate{1}{5.72957795130823}{1}%
		\global\let\pgfmfltdeg@factor=\pgfmathresult
	\fi
	\pgfmathfloatmultiply@{#1}\pgfmfltdeg@factor%
}
\let\pgfmathfloatdeg=\pgfmathfloatdeg@

% Converts #1 from degree to radians.
\def\pgfmathfloatrad@#1{%
	\expandafter\ifx\csname pgfmfltrad@factor\endcsname\relax
		% Lazy evaluation:
		\pgfmathfloatcreate{1}{1.74532925199433}{-2}%
		\global\let\pgfmfltrad@factor=\pgfmathresult
	\fi
	\pgfmathfloatmultiply@{#1}\pgfmfltrad@factor%
}
\let\pgfmathfloatrad=\pgfmathfloatrad@

% Computes #1(#2) where #1 is a trigonometric function, i.e.
% #1(#2) = #1( #2 + r*360 )
%
% #1 is a one-argument macro which assigns \pgfmathresult.
\def\pgfmathfloatTRIG@#1#2{%
	\if0\pgfmath@trig@format@choice
		% trig format=deg
		\expandafter\ifx\csname pgfmathfloatTRIG@NUM\endcsname\relax%
			% Lazy evaluation:
			\pgfmathfloatcreate{1}{3.6}{2}%
			\global\let\pgfmathfloatTRIG@NUM=\pgfmathresult
			\pgfmathfloatcreate{1}{2.77777777777778}{-3}%
			\global\let\pgfmathfloatTRIG@NUM@INV=\pgfmathresult
		\fi
		\pgfmathfloatmodknowsinverse@{#2}{\pgfmathfloatTRIG@NUM}{\pgfmathfloatTRIG@NUM@INV}%
	\else
		% trig format=rad
		\expandafter\ifx\csname pgfmathfloatTRIG@rad@NUM\endcsname\relax%
			% Lazy evaluation:
			\pgfmathfloatcreate{1}{6.28318530717959}{0}%
			\global\let\pgfmathfloatTRIG@rad@NUM=\pgfmathresult
			\pgfmathfloatcreate{1}{1.59154943091895}{-1}%
			\global\let\pgfmathfloatTRIG@rad@NUM@INV=\pgfmathresult
		\fi
		\pgfmathfloatmodknowsinverse@{#2}{\pgfmathfloatTRIG@rad@NUM}{\pgfmathfloatTRIG@rad@NUM@INV}%
	\fi
	\pgfmathfloattofixed@{\pgfmathresult}%
	\expandafter#1\expandafter{\pgfmathresult}%
	\pgfmathfloatparsenumber{\pgfmathresult}%
}%

\def\pgfmathfloatsin@#1{\pgfmathfloatTRIG@\pgfmath@basic@sin@{#1}}
\let\pgfmathfloatsin=\pgfmathfloatsin@
\def\pgfmathfloatcos@#1{\pgfmathfloatTRIG@\pgfmath@basic@cos@{#1}}
\let\pgfmathfloatcos=\pgfmathfloatcos@
\def\pgfmathfloattan@#1{%
	% compute sin(#1) / cos(#1)
	\begingroup
	\pgfmathfloatcos@{#1}%
	\let\pgfmathfloat@loc@TMPa=\pgfmathresult
	\pgfmathfloatsin@{#1}%
	\expandafter\pgfmathfloatdivide@\expandafter{\pgfmathresult}{\pgfmathfloat@loc@TMPa}%
	\pgfmath@smuggleone\pgfmathresult
	\endgroup
}
\let\pgfmathfloattan=\pgfmathfloattan@

\def\pgfmathfloatcot@#1{%
	% compute cos(#1) / sin(#1)
	\begingroup
	\pgfmathfloatsin@{#1}%
	\let\pgfmathfloat@loc@TMPa=\pgfmathresult
	\pgfmathfloatcos@{#1}%
	\expandafter\pgfmathfloatdivide@\expandafter{\pgfmathresult}{\pgfmathfloat@loc@TMPa}%
	\pgfmath@smuggleone\pgfmathresult
	\endgroup
}%
\let\pgfmathfloatcot=\pgfmathfloatcot@

\def\pgfmathfloatatan@#1{%
	\begingroup
	\expandafter\ifx\csname pgfmathfloatatan@TMP\endcsname\relax%
		\pgfmathfloatcreate{1}{1.6}{4}%
		\global\let\pgfmathfloatatan@TMP=\pgfmathresult
		\pgfmathfloatcreate{2}{1.6}{4}%
		\global\let\pgfmathfloatatan@TMPB=\pgfmathresult
	\fi
	\pgfmathfloatgreaterthan@{#1}{\pgfmathfloatatan@TMP}%
	\ifpgfmathfloatcomparison
		\pgfmathiftrigonometricusesdeg{%
			\pgfmathfloatcreate{1}{9.0}{1}%
		}{%
			\pgfmathfloatcreate{1}{1.570796326794}{0}%
		}%
	\else
		\pgfmathfloatlessthan{#1}{\pgfmathfloatatan@TMPB}%
		\ifpgfmathfloatcomparison
			\pgfmathiftrigonometricusesdeg{%
				\pgfmathfloatcreate{2}{9.0}{1}%
			}{%
				\pgfmathfloatcreate{2}{1.570796326794}{0}%
			}%
		\else
			\pgfmathfloattofixed@{#1}%
			\expandafter\pgfmath@basic@atan@\expandafter{\pgfmathresult}%
			\pgfmathfloatparsenumber{\pgfmathresult}%
		\fi
	\fi
	\pgfmath@smuggleone\pgfmathresult
	\endgroup
}%
\let\pgfmathfloatatan=\pgfmathfloatatan@

\def\pgfmathfloatatantwo#1#2{%
  % Note: first parameter is y (!), second is x (!)
  \begingroup%
	\let\pgfmath@trig@format@choice@@=\pgfmath@trig@format@choice
	\def\pgfmath@trig@format@choice{0}%
	%
	\expandafter\pgfmathfloat@decompose@tok#1\relax\pgfmathfloat@a@S\pgfmathfloat@a@Mtok\pgfmathfloat@a@E
	\expandafter\pgfmathfloat@decompose#2\relax\pgfmathfloat@b@S\pgfmathfloat@b@M\pgfmathfloat@b@E
	\ifnum\pgfmathfloat@a@S=0
		% ok, #1 = 0.  Substitute by 1e-16 such that the next \ifnum catches it:
		\pgfmathfloat@a@E=-16 %
	\fi
	%
	\ifnum\pgfmathfloat@a@E<-3 %
		\ifnum\pgfmathfloat@b@S=2 %
			% #2 < 0
			\pgfmathfloatcreate{1}{1.8}{2}% +180
		\else
			\ifnum\pgfmathfloat@b@S=1 %
				% #2 >0
				\pgfmathfloatcreate{0}{0.0}{0}%
			\else
				% + or - 90, just use the sign of #1:
				\pgfmathfloatcreate{\the\pgfmathfloat@a@S}{9.0}{1}%
			\fi
		\fi
    \else%
      \pgfmathfloatabs@{#1}\let\pgfmath@tempa\pgfmathresult%
      \pgfmathfloatabs@{#2}\let\pgfmath@tempb\pgfmathresult%
	  \pgfmathfloatgreaterthan@{\pgfmath@tempa}{\pgfmath@tempb}%
	  \ifpgfmathfloatcomparison
        \pgfmathfloatdivide@{#2}{\pgfmath@tempa}%
        \expandafter\pgfmathfloatatan@\expandafter{\pgfmathresult}%
		\let\pgfmath@tempa=\pgfmathresult
		\pgfmathfloatcreate{1}{9.0}{1}%
		\let\pgfmath@tempb=\pgfmathresult
		\pgfmathfloatsubtract@{\pgfmath@tempb}{\pgfmath@tempa}%
      \else%
        \pgfmathfloatdivide@{\pgfmath@tempa}{#2}%
        \expandafter\pgfmathfloatatan@\expandafter{\pgfmathresult}%
		\expandafter\pgfmathfloatifflags\expandafter{\pgfmathresult}{2}{%
			\let\pgfmath@tempa=\pgfmathresult
			\pgfmathfloatcreate{1}{1.8}{2}%
			\let\pgfmath@tempb=\pgfmathresult
			\pgfmathfloatadd@{\pgfmath@tempa}{\pgfmath@tempb}%
		}{}%
      \fi%
	  %
	  \pgfmathfloatifflags{#1}{-}{%
	  	%  #1 < 0:
		\pgfmathfloatmultiplyfixed@{\pgfmathresult}{-1}%
	  }{}%
    \fi%
	\if1\pgfmath@trig@format@choice@@
		% trig format=rad
		\pgfmathfloat@scale@deg@to@rad\pgfmathresult
	\fi
    \pgfmath@smuggleone\pgfmathresult%
  \endgroup%
}%
\let\pgfmathfloatatantwo@=\pgfmathfloatatantwo
\expandafter\let\csname pgfmathfloatatan2\endcsname=\pgfmathfloatatantwo
\expandafter\let\csname pgfmathfloatatan2@\endcsname=\pgfmathfloatatantwo@

\def\pgfmathfloat@scale@deg@to@rad#1{%
	\edef\pgfmathfloat@loc@TMPb{#1}%
	\pgfmathfloatcreate{1}{1.74532925199433}{-2}% = pi / 180
	\pgfmathfloatmultiply@{\pgfmathresult}{\pgfmathfloat@loc@TMPb}%
}%

\def\pgfmathfloatsec@#1{\pgfmathfloatTRIG@\pgfmath@basic@cos@{#1}\pgfmathfloatreciprocal@{\pgfmathresult}}
\let\pgfmathfloatsec=\pgfmathfloatsec@
\def\pgfmathfloatcosec@#1{\pgfmathfloatTRIG@\pgfmath@basic@sin@{#1}\pgfmathfloatreciprocal@{\pgfmathresult}}
\let\pgfmathfloatcosec=\pgfmathfloatcosec@

% Expands #2 using \edef and invokes #1 with the resulting string.
%
% DEPRECATED
% Example:
%   \pgfmath@y=7.9pt
%   \pgfmathlog@invoke@expanded\pgfmathexp@{{\pgf@sys@tonumber{\pgfmath@y}}}%
% will invoke
%   \pgfmathexp@{7.9}
\def\pgfmathlog@invoke@expanded#1#2{%
	\edef\pgfmath@resulttemp{#2}%
	\expandafter#1\pgfmath@resulttemp
}

\def\pgfmathfloatln@#1{%
	\pgfmathlog@float{#1}%
	\ifx\pgfmathresult\pgfutil@empty
		\pgfmathfloatcreate{3}{0.0}{0}%
	\else
		\pgfmathfloatparsenumber{\pgfmathresult}%
	\fi
}
\let\pgfmathfloatln=\pgfmathfloatln@

\expandafter\def\csname pgfmathfloatlog10@\endcsname#1{%
	\pgfmathfloatln@{#1}%
	\let\pgfmathfloat@log@e=\pgfmathresult
	\pgfmathfloatcreate{1}{4.34294481903252}{-1}% 1/ln(10)
	\pgfmathfloatmultiply@{\pgfmathresult}{\pgfmathfloat@log@e}%
}%
\pgfutil@namelet{pgfmathfloatlog10}{pgfmathfloatlog10@}%

\expandafter\def\csname pgfmathfloatlog2@\endcsname#1{%
	\pgfmathfloatln@{#1}%
	\let\pgfmathfloat@log@e=\pgfmathresult
	\pgfmathfloatcreate{1}{1.44269504088896}{0}% 1/ln(2)
	\pgfmathfloatmultiply@{\pgfmathresult}{\pgfmathfloat@log@e}%
}%
\pgfutil@namelet{pgfmathfloatlog2}{pgfmathfloatlog2@}%

\expandafter\let\expandafter\pgfmathfloatlogtwo\csname pgfmathfloatlog2\endcsname
\expandafter\let\expandafter\pgfmathfloatlogtwo@\csname pgfmathfloatlog2@\endcsname
\expandafter\let\expandafter\pgfmathfloatlogten\csname pgfmathfloatlog10\endcsname
\expandafter\let\expandafter\pgfmathfloatlogten@\csname pgfmathfloatlog10@\endcsname

% Computes log(x) into \pgfmathresult.
%
% This allows numbers such at 10000000 or 5.23e-10 to be represented
% properly, although TeX-registers would produce overflow/underflow
% errors in these cases.
%
% The natural logarithm is computed using log(X*10^Y) = log(X) + log(10)*Y
%
% FIXME This routine is only kept for backwards compatibility!
% It does not work as expected because
% 1. it calls \pgfmathfloatparsenumber
% 2. it returns the result as fixed point number
% Use \pgfmathln@ instead!
\def\pgfmathlog@#1{%
	\pgfmathfloatparsenumber{#1}%
	\pgfmathlog@float{\pgfmathresult}%
}
\let\pgfmathlog=\pgfmathlog@
\def\pgfmathlog@float#1{%
	\begingroup%
		% compute #1 = M*10^E with normalised mantisse M = [+-]*[1-9].XXXXX
		\expandafter\pgfmathfloat@decompose@tok#1\relax\pgfmathfloat@a@S\pgfmathfloat@a@Mtok\pgfmathfloat@a@E
		\ifnum\pgfmathfloat@a@S=1
			% Now, compute log(#1) = log(M) + E*log(10)
			\expandafter\pgfmath@basic@ln@\expandafter{\the\pgfmathfloat@a@Mtok}%
			\pgfmathfloat@b@M=\pgfmathresult pt%
			\pgfmathfloat@a@M=2.302585pt% = log(10)
			\multiply\pgfmathfloat@a@M by\pgfmathfloat@a@E\relax
			\advance\pgfmathfloat@b@M by\pgfmathfloat@a@M
			\edef\pgfmathresult{\pgf@sys@tonumber{\pgfmathfloat@b@M}}%
		\else
			\let\pgfmathresult=\pgfutil@empty%
		\fi
	\pgfmath@smuggleone\pgfmathresult
	\endgroup%
}


% Computes exp(#1) in floating point.
%
% The algorithm employs the identity
%  exp(x) = exp(x - log(10^k) + log(10^k)
%         = 10^k exp( x - k*log 10 )
% with k choosen such that exp( x - k*log10) can be computed with the
% basic level math engine.
%
% The precision (relative error) is between 10^{-4} and 10^{-6}. For
% #1 = 700, it is even 10^{-3}. I will need to improve that someday.
\def\pgfmathfloatexp@#1{%
	\begingroup
		\expandafter\pgfmathfloat@decompose@tok#1\relax\pgfmathfloat@a@S\pgfmathfloat@a@Mtok\pgfmathfloat@a@E
		\ifcase\pgfmathfloat@a@S
		% #1 = 0:
			\pgfmathfloatcreate{1}{1.0}{0}%
		\or% #1 > 0
			\pgfmathfloatexp@@{#1}%
		\or% #1 < 0
			\pgfmathfloatexp@@{#1}%
		\else
			\edef\pgfmathresult{#1}%
		\fi
		\pgfmath@smuggleone\pgfmathresult
	\endgroup
}%
\def\pgfmathfloatexp@@#1{%
	% Employ the identity
	% exp(x) = exp(x - log(10^k) + log(10^k)) = 10^k exp( x - k *log(10))
	%
	% I'd like to have x - k*log(10) <= 1
	% => compute k := int( (x - 1) * 1/log(10) )
	% that should suffice since \pgfmathexp@ should be
	% accurate enough for those numbers.
	%
	% please note that we can do all this in TeX registers.
	% exp(700) is almost the maximum of double precision
	% anyway, and exp(16000) is certainly the largest we will
	% ever need.
	\pgfmathfloattofixed@{#1}%
	\pgf@xa=\pgfmathresult pt
	\pgf@xa=0.434294481\pgf@xa\relax
	\edef\pgfmathfloat@loc@TMPa{\pgf@sys@tonumber{\pgf@xa}}%
	\expandafter\pgfmathfloatexp@@toint\pgfmathfloat@loc@TMPa\relax
	\pgf@xa=2.302585092pt
	\multiply\pgf@xa by-\pgfmathfloat@k\relax
	\advance\pgf@xa by\pgfmathresult pt
	\edef\pgfmathfloat@loc@TMPa{\pgf@sys@tonumber{\pgf@xa}}%
%\message{computing exp(\pgfmathresult) = 10^\pgfmathfloat@k * exp(\pgfmathfloat@loc@TMPa)...}%
	\pgfmath@basic@exp@{\pgfmathfloat@loc@TMPa}%
	\let\pgfmathfloat@loc@TMPa=\pgfmathresult
	\pgfmathfloatparsenumber{\pgfmathfloat@loc@TMPa}%
	\let\pgfmathfloat@loc@TMPa=\pgfmathresult
	\pgfmathfloatshift@{\pgfmathfloat@loc@TMPa}{\pgfmathfloat@k}%
}
% determine 'k'. This is a heuristics. The exponential series
% converges best for |x| <= 1. However, the fixed point arithmetics
% for tex results in best results for large |x|. Well, I'll need to
% tune this here.
\def\pgfmathfloatexp@@toint#1.#2\relax{%
	\c@pgf@counta=#1\relax
	\ifnum\c@pgf@counta<0
\advance\c@pgf@counta by-1 % FIXME . this is a test for optimizations.
		\c@pgf@countb=#2\relax
		\ifnum\c@pgf@countb>0
			\advance\c@pgf@counta by-1
		\fi
	\fi
	\edef\pgfmathfloat@k{\the\c@pgf@counta}%
}%
\let\pgfmathfloatexp=\pgfmathfloatexp@

\def\pgfmathfloatround@#1{%
	\begingroup
	\pgfkeysvalueof{/pgf/number format/precision/.@cmd}0\pgfeov
	\pgfmathfloattofixed{#1}%
	\pgfmathroundto{\pgfmathresult}%
	\pgfmathfloatparsenumber{\pgfmathresult}%
	\pgfmath@smuggleone\pgfmathresult
	\endgroup
}%


\def\pgfmathfloatneg@#1{%
	\begingroup
	\expandafter\pgfmathfloat@decompose@tok#1\relax\pgfmathfloat@a@S\pgfmathfloat@a@Mtok\pgfmathfloat@a@E
	\ifcase\pgfmathfloat@a@S\relax
		% 0:
		\edef\pgfmathresult{#1}%
	\or
		% +:
		\pgfmathfloatcreate{2}{\the\pgfmathfloat@a@Mtok}{\the\pgfmathfloat@a@E}%
	\or
		% -:
		\pgfmathfloatcreate{1}{\the\pgfmathfloat@a@Mtok}{\the\pgfmathfloat@a@E}%
	\or
		% nan:
		\edef\pgfmathresult{#1}%
	\or
		% +infty:
		\pgfmathfloatcreate{5}{\the\pgfmathfloat@a@Mtok}{\the\pgfmathfloat@a@E}%
	\or
		% -infty:
		\pgfmathfloatcreate{4}{\the\pgfmathfloat@a@Mtok}{\the\pgfmathfloat@a@E}%
	\fi
	\pgfmath@smuggleone\pgfmathresult
	\endgroup
}%


\def\pgfmathfloatpow@#1#2{%
	\begingroup%
	\expandafter\pgfmathfloat@decompose@tok#2\relax\pgfmathfloat@a@S\pgfmathfloat@a@Mtok\pgfmathfloat@a@E
	\ifcase\pgfmathfloat@a@S\relax
		% #1 ^ 0 = 1
		\pgfmathfloatcreate{1}{1.0}{0}%
	\or
		% #2 > 0
		\pgfmathfloatpow@@{#1}{#2}%
	\or
		% #2 < 0
		\pgfmathfloatpow@@{#1}{#2}%
	\or
		% #2 = nan
		\edef\pgfmathresult{#2}%
	\or
		% #2 = inf
		\edef\pgfmathresult{#2}%
	\or
		% #2 = -inf
		\pgfmathfloatcreate{0}{0.0}{0}%
	\fi
	\pgfmath@smuggleone\pgfmathresult
	\endgroup
}%

% computes #1^#2
% PRECONDITIONS
% - #2 is positive.
\def\pgfmathfloatpow@@#1#2{%
	\pgfmathfloattofixed@{#2}%
	\afterassignment\pgfmath@x%
	\expandafter\c@pgfmath@counta\pgfmathresult pt\relax%
	\ifdim\pgfmath@x=0pt %
		% loop "manually"; we have an integer exponent!
		\ifnum\c@pgfmath@counta<0
			\pgfmathfloatreciprocal@{#1}%
			\let\pgfmathfloat@loc@TMPa=\pgfmathresult
			\c@pgfmath@counta=-\c@pgfmath@counta
		\else
			\edef\pgfmathfloat@loc@TMPa{#1}%
		\fi
		\pgfmathfloatcreate{1}{1.0}{0}%
		\let\pgfmathfloat@loc@TMPb=\pgfmathresult
		\pgfmathloop
			\ifnum\c@pgfmath@counta>0\relax%
				\ifodd\c@pgfmath@counta%
					\pgfmathfloatmultiply@{\pgfmathfloat@loc@TMPb}{\pgfmathfloat@loc@TMPa}%
					\let\pgfmathfloat@loc@TMPb=\pgfmathresult
				\fi
				\ifnum\c@pgfmath@counta>1\relax%
					\pgfmathfloatmultiply@{\pgfmathfloat@loc@TMPa}{\pgfmathfloat@loc@TMPa}%
					\let\pgfmathfloat@loc@TMPa=\pgfmathresult
				\fi%
				\divide\c@pgfmath@counta by2\relax%
		\repeatpgfmathloop%
	\else
		\pgfmathfloatgetflags{#1}\c@pgfmath@counta
		\ifnum0=\c@pgfmath@counta
			% ah: 0^x
			\pgfmathfloatgetflags{#2}\c@pgfmath@counta
			\ifnum0=\c@pgfmath@counta
				% ah: 0^0
				\pgfmathfloatcreate{1}{1.0}{0}%
			\else	
				% ah: 0^x with x!=0:
				\pgfmathfloatcreate{0}{0.0}{0}%
			\fi
		\else
			% employ #1^#2 = exp( #2 * ln(#1) )
			\pgfmathfloatln@{#1}%
			\let\pgfmathfloat@loc@TMPa=\pgfmathresult
			\edef\pgfmathfloat@loc@TMPb{#2}%
			\pgfmathfloatmultiply@{\pgfmathfloat@loc@TMPa}{\pgfmathfloat@loc@TMPb}%
			\pgfmathfloatexp@{\pgfmathresult}%
		\fi
	\fi
}

\def\pgfmathfloat@definemethodfrombasic@NOARG#1{%
	\pgfutil@ifundefined{pgfmath@basic@#1@}{%
		\pgfutil@namelet{pgfmath@basic@#1@}{pgfmath#1@}%
	}{}%
	\edef\pgfmathfloat@glob@TMP{%
		\expandafter\noexpand\csname pgfmath@basic@#1@\endcsname
		\noexpand\pgfmathfloatparsenumber{\noexpand\pgfmathresult}%
	}%
	\expandafter\let\csname pgfmathfloat#1@\endcsname=\pgfmathfloat@glob@TMP%
	\expandafter\let\csname pgfmathfloat#1\endcsname=\pgfmathfloat@glob@TMP%
}%
\def\pgfmathfloat@definemethodfrombasic@ONEARG#1{%
	\pgfutil@ifundefined{pgfmath@basic@#1@}{%
		\pgfutil@namelet{pgfmath@basic@#1@}{pgfmath#1@}%
	}{}%
	\edef\pgfmathfloat@glob@TMP##1{%
		\noexpand\pgfmathfloattofixed{##1}%
		\noexpand\expandafter
		\expandafter\noexpand\csname pgfmath@basic@#1@\endcsname\noexpand\expandafter%
			{\noexpand\pgfmathresult}%
		\noexpand\pgfmathfloatparsenumber{\noexpand\pgfmathresult}%
	}%
	\expandafter\let\csname pgfmathfloat#1@\endcsname=\pgfmathfloat@glob@TMP%
	\expandafter\let\csname pgfmathfloat#1\endcsname=\pgfmathfloat@glob@TMP%
}%
\def\pgfmathfloat@definemethodfrombasic@TWOARGS#1{%
	\pgfutil@ifundefined{pgfmath@basic@#1@}{%
		\pgfutil@namelet{pgfmath@basic@#1@}{pgfmath#1@}%
	}{}%
	\edef\pgfmathfloat@glob@TMP##1##2{%
		\noexpand\pgfmathfloattofixed{##2}%
		\noexpand\let\noexpand\pgfmathfloat@loc@TMPa=\noexpand\pgfmathresult
		\noexpand\pgfmathfloattofixed{##1}%
		\noexpand\expandafter
		\expandafter\noexpand\csname pgfmath@basic@#1@\endcsname\noexpand\expandafter%
			{\noexpand\pgfmathresult}{\noexpand\pgfmathfloat@loc@TMPa}%
		\noexpand\pgfmathfloatparsenumber{\noexpand\pgfmathresult}%
	}%
	\expandafter\let\csname pgfmathfloat#1@\endcsname=\pgfmathfloat@glob@TMP%
	\expandafter\let\csname pgfmathfloat#1\endcsname=\pgfmathfloat@glob@TMP%
}%
\pgfmathfloat@definemethodfrombasic@NOARG{rand}
\pgfmathfloat@definemethodfrombasic@NOARG{rnd}
\pgfmathfloat@definemethodfrombasic@NOARG{false}
\pgfmathfloat@definemethodfrombasic@NOARG{true}
% arcsin, arccos
\pgfmathfloat@definemethodfrombasic@ONEARG{asin}
\pgfmathfloat@definemethodfrombasic@ONEARG{acos}
\pgfmathfloat@definemethodfrombasic@ONEARG{not}
\pgfmathfloat@definemethodfrombasic@ONEARG{hex}
\pgfmathfloat@definemethodfrombasic@ONEARG{Hex}
\pgfmathfloat@definemethodfrombasic@ONEARG{oct}
\pgfmathfloat@definemethodfrombasic@ONEARG{bin}
\pgfmathfloat@definemethodfrombasic@TWOARGS{and}
\pgfmathfloat@definemethodfrombasic@TWOARGS{or}

\pgfutil@ifundefined{pgfmathdeclarefunction}{%
	% special treatment: \pgfmathrand@ was not properly defined for pgf 2.00:
	\let\pgfmath@basic@rand=\pgfmathrand
	\let\pgfmath@basic@rand@=\pgfmathrand@
	\def\pgfmathfloatrand@{%
		\pgfmath@basic@rand
		\pgfmathfloatparsenumber{\pgfmathresult}%
	}%
	\let\pgfmathfloatrand=\pgfmathfloatrand@%
	%
	% special treatment: \pgfmathrnd@ was not properly defined for pgf 2.00:
	\let\pgfmath@basic@rnd=\pgfmathrnd
	\let\pgfmath@basic@rnd@=\pgfmathrnd@
	\def\pgfmathfloatrnd@{%
		\pgfmath@basic@rnd
		\pgfmathfloatparsenumber{\pgfmathresult}%
	}%
	\let\pgfmathfloatrnd=\pgfmathfloatrnd@%
}{}

% Implements the factorial of '#1'.
% This does only work if '#1 < 2^32'.
\def\pgfmathfloatfactorial@#1{%
	\begingroup
	\pgfmathfloattofixed{#1}%
	% collect integer part into a 32 bit register:
	\afterassignment\pgfmath@gobbletilpgfmath@%
	\expandafter\c@pgfmath@counta\pgfmathresult\relax\pgfmath@%
	\pgfmathfloatcreate{1}{1.0}{0}%
	\let\pgfmathfloat@loc@TMPa=\pgfmathresult
	\pgfmathloop
	\ifnum\c@pgfmath@counta<2 %
	\else
		\expandafter\pgfmathfloatparsenumber\expandafter{\the\c@pgfmath@counta}%
		\expandafter\pgfmathfloatmultiply@\expandafter{\pgfmathresult}{\pgfmathfloat@loc@TMPa}%
		\let\pgfmathfloat@loc@TMPa=\pgfmathresult
		\advance\c@pgfmath@counta by-1\relax%
	\repeatpgfmathloop
	\pgfmath@smuggleone\pgfmathresult
	\endgroup
}%

% Implements the vector length of a 2D vector.
%
% ATTENTION: this does NOT use the improved code of the basic layer!
% It simply computed sqrt( #1^2 + #2^2 )!
\def\pgfmathfloatveclen@#1#2{%
	\begingroup
	\edef\pgfmathfloat@@a{#1}%
	\pgfmathfloatmultiply@{\pgfmathfloat@@a}{\pgfmathfloat@@a}%
	\let\pgfmathfloat@@a=\pgfmathresult
	%
	\edef\pgfmathfloat@@b{#2}%
	\pgfmathfloatmultiply@{\pgfmathfloat@@b}{\pgfmathfloat@@b}%
	\let\pgfmathfloat@@b=\pgfmathresult
	%
	\pgfmathfloatadd@{\pgfmathfloat@@a}{\pgfmathfloat@@b}%
	\pgfmathfloatsqrt@{\pgfmathresult}%
	\pgfmath@smuggleone\pgfmathresult
	\endgroup
}%

\def\pgfmathfloatcosh@#1{%
	\begingroup
	\pgfmathfloatexp@{#1}%
	\let\pgfmathfloat@@a=\pgfmathresult
	%
	\pgfmathfloatneg@{#1}%
	\pgfmathfloatexp@{\pgfmathresult}%
	%
	\pgfmathfloatadd@{\pgfmathresult}{\pgfmathfloat@@a}%
	\expandafter\pgfmathfloatmultiplyfixed@\expandafter{\pgfmathresult}{0.5}%
	%
	\pgfmath@smuggleone\pgfmathresult
	\endgroup
}%
\def\pgfmathfloatsinh@#1{%
	\begingroup
	\pgfmathfloatexp@{#1}%
	\let\pgfmathfloat@@a=\pgfmathresult
	%
	\pgfmathfloatneg@{#1}%
	\pgfmathfloatexp@{\pgfmathresult}%
	%
	\pgfmathfloatsubtract@{\pgfmathfloat@@a}{\pgfmathresult}%
	\expandafter\pgfmathfloatmultiplyfixed@\expandafter{\pgfmathresult}{0.5}%
	%
	\pgfmath@smuggleone\pgfmathresult
	\endgroup
}%
\def\pgfmathfloattanh@#1{%
	\begingroup
	\pgfmathfloatsinh@{#1}%
	\let\pgfmathfloat@@a=\pgfmathresult
	%
	\pgfmathfloatcosh@{#1}%
	\let\pgfmathfloat@@b=\pgfmathresult
	%
	\pgfmathfloatdivide@{\pgfmathfloat@@a}{\pgfmathfloat@@b}%
	%
	\pgfmath@smuggleone\pgfmathresult
	\endgroup
}%
\endinput
