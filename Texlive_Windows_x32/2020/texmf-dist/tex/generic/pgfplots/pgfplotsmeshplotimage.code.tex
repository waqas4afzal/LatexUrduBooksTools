%--------------------------------------------
%
% Package pgfplots
%
% Provides a user-friendly interface to create function plots (normal
% plots, semi-logplots and double-logplots).
% 
% It is based on Till Tantau's PGF package.
%
% Copyright 2007-2015 by Christian Feuersänger.
%
% This program is free software: you can redistribute it and/or modify
% it under the terms of the GNU General Public License as published by
% the Free Software Foundation, either version 3 of the License, or
% (at your option) any later version.
% 
% This program is distributed in the hope that it will be useful,
% but WITHOUT ANY WARRANTY; without even the implied warranty of
% MERCHANTABILITY or FITNESS FOR A PARTICULAR PURPOSE.  See the
% GNU General Public License for more details.
% 
% You should have received a copy of the GNU General Public License
% along with this program.  If not, see <http://www.gnu.org/licenses/>.
%
%--------------------------------------------

% This file essentially belongs to the 'mesh' plot handler.
% It implements the (rather involved) feature 'mesh input=image'.
%
% 'mesh input=image' is implemented during the visualization phase
% and, partially, during the survey phase. The survey phase merely
% updates the data limits. The visualization phase has the complete
% knowledge about the input matrix size and can compute the cells.

\def\pgfplotsplothandlersurveystart@mesh@image{%
	\let\pgfplotsplothandlermesh@image@lastA=\pgfutil@empty
	\let\pgfplotsplothandlermesh@image@lastlastA=\pgfutil@empty
	\let\pgfplotsplothandlermesh@image@lastB=\pgfutil@empty
	\def\b@pgfplotsplothandlermesh@image@updatedlimits@A@next{1}%
	\def\b@pgfplotsplothandlermesh@image@updatedlimits@B@next{0}%
	\def\c@pgfplotsplothandlermesh@image@numscanlines{0}%
	%
	\if\pgfplots@plot@mesh@ordering0%
		% ordering = x varies= rowwise -> scanline is cols!
		\def\pgfplotsplothandlermesh@image@A@dir{x}%
		\def\pgfplotsplothandlermesh@image@B@dir{y}%
		\def\pgfplotsplothandlermesh@image@updatelimits@AB##1##2##3{%
			\pgfplotsaxisupdatelimitsforcoordinate{##1}{##2}{##3}%
		}%
	\else
		% ordering = y varies = colwise: scanline is rows!
		\def\pgfplotsplothandlermesh@image@A@dir{y}%
		\def\pgfplotsplothandlermesh@image@B@dir{x}%
		\def\pgfplotsplothandlermesh@image@updatelimits@AB##1##2##3{%
			\pgfplotsaxisupdatelimitsforcoordinate{##2}{##1}{##3}%
		}%
	\fi
}%
\def\pgfplotsplothandlersurveyend@mesh@image{%
	\if1\pgfplotsaxisplothasjumps
		% the approach to interpolate cell corners using the adjacent cell midpoints becomes extremely complicated once 
		% we allow all kinds of holes in the input matrix.
		% If you want to make invisible cells, you can add "nan" as color data
		\pgfplots@error{Sorry, 'matrix plot' supports no holes (jumps) in the input data. Please implement jumps using an unbounded point meta value (i.e. inf or nan)}%
	\fi
	%
}%
\def\pgfplotsplothandlersurveypoint@mesh@image{%
	% already called:
	%\pgfplotsplothandlersurveypoint@default
	%
	% The 'update limits' routine for 'mesh input=image' is a
	% HEURISTICS.
	%
	% It works well for "real" image plots, i.e. for those which are
	% not skewed or changed parametrically.
	%
	% The main challenge is that we have to update the matrix limits
	% while we pass through the scanlines.
	%
	% The idea is:
	% - update cell midpoint limits using the standard PGFPlots
	%   methods. We would like to have the cell limits instead,
	%   though... to this end:
	% - update the lower x cell limit for the first encountered X in every
	%   scanline
	% - update the upper x cell limit for the last encountered X in every
	%   scanline
	% - update the lower AND upper y cell limit for the first encountered Y
	%   in every scanline.
	%
	% That means: 
	% - we do NOT update X limits for mid points
	% - we do NOT update Y limits for the second, third, ..., nth
	%   point on scanlines. Only for the first.
	%
	% For parametric image plots, these limits will be inaccurate. I
	% suppose these are rather exotic anyway, so I accept that to
	% simplify the implementation.
	%
	\expandafter\let\expandafter\pgfplotsplothandlermesh@cur\csname pgfplots@current@point@\pgfplotsplothandlermesh@image@A@dir\endcsname
	\expandafter\let\expandafter\pgfplotsplothandlermesh@cur@B\csname pgfplots@current@point@\pgfplotsplothandlermesh@image@B@dir\endcsname
	%
	\if1\b@pgfplotsplothandlermesh@image@updatedlimits@A@next
		\ifx\pgfplotsplothandlermesh@cur\pgfutil@empty
		\else
			\ifx\pgfplotsplothandlermesh@image@lastA\pgfutil@empty
			\else
				\pgfplotscoordmath{x}{parse}{\pgfplotsplothandlermesh@image@lastA - 0.5*(\pgfplotsplothandlermesh@cur - \pgfplotsplothandlermesh@image@lastA)}%
				\let\pgfplots@current@point@xh=\pgfmathresult
				\pgfplotsplothandlermesh@image@updatelimits@AB\pgfplots@current@point@xh\pgfplotsplothandlermesh@cur@B\pgfplots@current@point@z
				%
				\def\b@pgfplotsplothandlermesh@image@updatedlimits@A@next{0}%
			\fi
		\fi
	\fi
	%
	\if1\b@pgfplotsplothandlermesh@image@updatedlimits@B@next
		\ifx\pgfplotsplothandlermesh@cur@B\pgfutil@empty
		\else
			\ifx\pgfplotsplothandlermesh@image@lastB\pgfutil@empty
			\else
				\pgfplotscoordmath{y}{parse}{\pgfplotsplothandlermesh@image@lastB - 0.5*(\pgfplotsplothandlermesh@cur@B - \pgfplotsplothandlermesh@image@lastB)}%
				\let\pgfplots@current@point@yh=\pgfmathresult
				\pgfplotsplothandlermesh@image@updatelimits@AB\pgfplotsplothandlermesh@cur\pgfplots@current@point@yh\pgfplots@current@point@z
				%
				%
				\pgfplotscoordmath{y}{parse}{\pgfplotsplothandlermesh@cur@B + 0.5*(\pgfplotsplothandlermesh@cur@B - \pgfplotsplothandlermesh@image@lastB)}%
				\let\pgfplots@current@point@yh=\pgfmathresult
				\pgfplotsplothandlermesh@image@updatelimits@AB\pgfplotsplothandlermesh@cur\pgfplots@current@point@yh\pgfplots@current@point@z
				%
				\def\b@pgfplotsplothandlermesh@image@updatedlimits@B@next{0}%
			\fi
		\fi
	\fi
	\let\pgfplotsplothandlermesh@image@lastlastA=\pgfplotsplothandlermesh@image@lastA
	\let\pgfplotsplothandlermesh@image@lastA=\pgfplotsplothandlermesh@cur
	%
	\let\pgfplotsplothandlermesh@image@lastB=\pgfplotsplothandlermesh@cur@B
}%

\def\pgfplotsplothandlersurveyend@mesh@image@scanlinecomplete{%
	\ifx\pgfplotsplothandlermesh@image@lastlastA\pgfutil@empty
	\else
		\ifx\pgfplotsplothandlermesh@image@lastA\pgfutil@empty
		\else
			\expandafter\let\expandafter\pgfplotsplothandlermesh@cur@B\csname pgfplots@current@point@\pgfplotsplothandlermesh@image@B@dir\endcsname
			\pgfplotscoordmath{x}{parse}{\pgfplotsplothandlermesh@image@lastA + 0.5*(\pgfplotsplothandlermesh@image@lastA - \pgfplotsplothandlermesh@image@lastlastA)}%
			\let\pgfplots@current@point@xh=\pgfmathresult
			\pgfplotsplothandlermesh@image@updatelimits@AB\pgfplots@current@point@xh\pgfplotsplothandlermesh@cur@B\pgfplots@current@point@z
		\fi
	\fi
	%
	\def\b@pgfplotsplothandlermesh@image@updatedlimits@B@next{1}%
	%
	\let\pgfplotsplothandlermesh@image@lastlastA=\pgfutil@empty
	\let\pgfplotsplothandlermesh@image@lastA=\pgfutil@empty
	\def\b@pgfplotsplothandlermesh@image@updatedlimits@A@next{1}%
	%
	\pgfplotsutil@advancestringcounter\c@pgfplotsplothandlermesh@image@numscanlines
}%

%
% The 'mesh' plot handler has a "visualization pipeline" as part of
% its visualization:
%
% if it is active, the coordinate stream used for the visualization
% (\pgfplotsplothandlermesh@stream) invokes
% \pgfplotsplothandlermesh@PIPE@DECODE as first item. This, in turn,
% fires other elements of the mesh plot pipeline, but its most
% important task is to assemble patches while it goes. 
%
% That's where 'mesh input=image' hooks in: it simply generates special patches.

% this is part of the implementation of mesh input=image
% the pgf@plotstreampoint routine for the first scan line:
\def\pgfplotsplothandlermesh@PIPE@DECODE@image@fillscanline#1{%
	\pgfplotsplothandlermesh@PIPE@DECODE@matrix@fillscanline@common{#1}%
	\ifnum\c@pgfplots@scanlineindex=\pgfplotsplothandlermesh@scanlinelength\relax
		% second line!
		\let\pgfplotsplothandlermesh@PIPE@DECODE=\pgfplotsplothandlermesh@PIPE@DECODE@image@secondscanline%%
		\c@pgfplots@scanlineindex=0
		\def\c@pgfplotsplothandlermesh@image@nextrownum{2}%
	\fi
}%

% IDEA: 
%
% 1. The first scanline is merely buffered and collected into the deque 'lastscanline' (just as any mesh plot).
% 2. The second scanline 
%  - generates cells for the first scanline _and_
%  - appends the computed intermediate vertices to the deque 'lastinterpscanline'.
%  - during the sweep, a single cell of the first scanline is
%  generated
%  - only the _last_ element of the scanline generates two cells
% 3. All scanlines after the second 
%  - pop the intermediate vertices from the deque 
%  - use them to construct cells for the previous scanline
%  - append their computed intermediate vertices to the deque 'lastinterpscanline'.
%  - during the sweep, a single cell of the first scanline is
%  generated
%  - only the _last_ element of the scanline generates two cells (just as in step 2)
% 4. the last scanline is special in that it generates cells for the
% previous scanline _and_ the last scanline.
%
% The "computed intermediate vertices" are mean values and
% extrapolation points. My idea is to make this special cell-based
% mesh more flexible than a simple imagesc, hoping that this pays off.
\def\pgfplotsplothandlermesh@PIPE@DECODE@image@secondscanline#1{%
	\pgfplotsplothandlermesh@PIPE@DECODE@matrix@collect@and@callback{#1}{%
		%
		% create cell:
		\pgfplotsplothandlermesh@PIPE@DECODE@image@sweep@create@cell{1}%
		%
		\advance\c@pgfplots@scanlineindex by1
		\ifnum\c@pgfplots@scanlineindex=\pgfplotsplothandlermesh@scanlinelength\relax
			% ok, we have to extrapolate the last column.
			\pgfplotsplothandlermesh@PIPE@DECODE@image@sweep@create@cell@lastcolumn{1}%
		\fi
		\advance\c@pgfplots@scanlineindex by-1
	}{%
		% if scanline is completed, switch states:
		\let\pgfplotsplothandlermesh@PIPE@DECODE=\pgfplotsplothandlermesh@PIPE@DECODE@image@after@second%%
		\c@pgfplots@scanlineindex=0
		\pgfplotsutil@advancestringcounter\c@pgfplotsplothandlermesh@image@nextrownum
	}%
}%
\def\pgfplotsplothandlermesh@PIPE@DECODE@image@after@second#1{%
	\pgfplotsplothandlermesh@PIPE@DECODE@matrix@collect@and@callback{#1}{%
		\pgfplotsplothandlermesh@PIPE@DECODE@image@sweep@create@cell{0}%
		%
		\advance\c@pgfplots@scanlineindex by1
		\ifnum\c@pgfplots@scanlineindex=\pgfplotsplothandlermesh@scanlinelength\relax
			% ok, we have to extrapolate the last column.
			\pgfplotsplothandlermesh@PIPE@DECODE@image@sweep@create@cell@lastcolumn{0}%
		\fi
		\advance\c@pgfplots@scanlineindex by-1
	}{%
		% if scanline is completed: advance to next
		\c@pgfplots@scanlineindex=0
		\pgfplotsutil@advancestringcounter\c@pgfplotsplothandlermesh@image@nextrownum
	}%
}%

% To be used within the DECODE "loop" only! It assembles temporary
% variables for the next iteration (and consumes it).
%
% #1: '1' if topmost row is to be extrapolated, '0' if it should be
% acquired from the datastructure 'lastinterpscanline'
\def\pgfplotsplothandlermesh@PIPE@DECODE@image@sweep@create@cell#1{%
		\let\pgfplotsplothandlermesh@patchclass\pgfplotsplothandlermesh@patchclass@input
		%
		% we have to compute the four corners by means of
		% averaging/extrapolation.
		%
		% We place the rectangle around (i-1,j-1), i.e. imjm
		% Notation: the four corners are aa, ab, ba, and bb.
		%  aa -----  ab
		%  |         |
		%  |         |
		%  |   imjm  |          
		%  |         |     
		%  |         |
		%  ba ------ bb
		%
		% bb is the average of the four cell mid points of our
		% scanline, the others are acquired from previous iterations
		% or from extrapolation schemes.
		%
		% compute bb:
		\pgfplotspatchvertexaccumstart
		\pgfplotspatchvertexaccum@configure@ignore@unboundedmetadata
		\expandafter\pgfplotspatchvertexadd\pgfplotsplothandlermesh@im@jm\times{0.25}%
		\expandafter\pgfplotspatchvertexadd\pgfplotsplothandlermesh@im@j\times{0.25}%
		\expandafter\pgfplotspatchvertexadd\pgfplotsplothandlermesh@i@j\times{0.25}%
		\expandafter\pgfplotspatchvertexadd\pgfplotsplothandlermesh@i@jm\times{0.25}%
		\pgfplotspatchvertexfinish\pgfplotsplothandlermesh@bb
		\pgfplotspatchvertexcopymetaifbounded\pgfplotsplothandlermesh@i@j\to\pgfplotsplothandlermesh@bb
		%
		%
		\if1#1%
			% Ah -- this here is the very first row! That means we
			% have to extrapolate the topmost coordinates.
			%
			% Apply extrapolation for ab:
			%
			% compute ab = bb + 2* ( 1/2 (imj + imjm) - bb)
			% which resolves to the following:
			\pgfplotspatchvertexaccumstart
			\pgfplotspatchvertexaccum@configure@ignore@unboundedmetadata
			\expandafter\pgfplotspatchvertexadd\pgfplotsplothandlermesh@bb\times{-1.0}%
			\expandafter\pgfplotspatchvertexadd\pgfplotsplothandlermesh@im@j\times{1.0}%
			\expandafter\pgfplotspatchvertexadd\pgfplotsplothandlermesh@im@jm\times{1.0}%
			\pgfplotspatchvertexfinish\pgfplotsplothandlermesh@ab
			\pgfplotspatchvertexcopymetaifbounded\pgfplotsplothandlermesh@im@j\to\pgfplotsplothandlermesh@ab
		\else
			\ifnum\c@pgfplots@scanlineindex=1
				\pgfplotsdequepopfront{lastinterpscanline}\to\pgfplotsplothandlermesh@aa
			\fi
			\pgfplotsdequepopfront{lastinterpscanline}\to\pgfplotsplothandlermesh@ab
		\fi
		%
		\ifnum\c@pgfplots@scanlineindex=1
			% ah, we have to apply special treatment for the leftmost
			% cell.
			%
			%
			% compute ba = bb + 2* ( 1/2 (imjm + ijm) - bb)
			% which resolves to the following:
			\pgfplotspatchvertexaccumstart
			\pgfplotspatchvertexaccum@configure@ignore@unboundedmetadata
			\expandafter\pgfplotspatchvertexadd\pgfplotsplothandlermesh@bb\times{-1.0}%
			\expandafter\pgfplotspatchvertexadd\pgfplotsplothandlermesh@i@jm\times{1.0}%
			\expandafter\pgfplotspatchvertexadd\pgfplotsplothandlermesh@im@jm\times{1.0}%
			\pgfplotspatchvertexfinish\pgfplotsplothandlermesh@ba
			\pgfplotspatchvertexcopymetaifbounded\pgfplotsplothandlermesh@i@jm\to\pgfplotsplothandlermesh@ba
			%
			\if1#1%
				% ok, extrapolate 'aa' as well:
				%
				% compute aa = ba + (ab - bb) 
				% which resolves to the following:
				\pgfplotspatchvertexaccumstart
				\pgfplotspatchvertexaccum@configure@ignore@unboundedmetadata
				\expandafter\pgfplotspatchvertexadd\pgfplotsplothandlermesh@bb\times{-1.0}%
				\expandafter\pgfplotspatchvertexadd\pgfplotsplothandlermesh@ba\times{1.0}%
				\expandafter\pgfplotspatchvertexadd\pgfplotsplothandlermesh@ab\times{1.0}%
				\pgfplotspatchvertexfinish\pgfplotsplothandlermesh@aa
				\pgfplotspatchvertexcopymetaifbounded\pgfplotsplothandlermesh@im@jm\to\pgfplotsplothandlermesh@aa
			\else
				% in this case, 'aa' has been fetched from the deque
				% (from the previous scanline)
			\fi
			%
			% as soon as we process the next scanline, we need to know the
			% interpolated values -- they will become the top row of the
			% next scanline's cells.
			\expandafter\pgfplotsdequepushback\pgfplotsplothandlermesh@ba\to{lastinterpscanline}%
		\else
			% ok, we take the left edge (ba and aa) from the previous
			% iteration (see the end of this code where the previous
			% loop iteration prepared them).
		\fi
		%
		\pgfplotsplothandlermesh@image@set@first@color@to\pgfplotsplothandlermesh@im@jm\to\pgfplotsplothandlermesh@aa
		\expandafter\pgfplotsplothandlermesh@setnextvertex\expandafter{\pgfplotsretval}%
		\expandafter\pgfplotsplothandlermesh@setnextvertex\expandafter{\pgfplotsplothandlermesh@ab}%
		\expandafter\pgfplotsplothandlermesh@setnextvertex\expandafter{\pgfplotsplothandlermesh@bb}%
		\expandafter\pgfplotsplothandlermesh@setnextvertex\expandafter{\pgfplotsplothandlermesh@ba}%
		%
		% ... and communicate this as well to the next scanline (see
		% \pgfplotsplothandlermesh@PIPE@DECODE@image@after@second)
		\expandafter\pgfplotsdequepushback\pgfplotsplothandlermesh@bb\to{lastinterpscanline}%
		%
		\ifnum\c@pgfplotsplothandlermesh@image@nextrownum=\pgfplotsplothandlermesh@numscanlines\relax
			\pgfplotsplothandlermesh@PIPE@DECODE@image@sweep@create@cell@lastrow
		\fi
		%
		% merely shift the current points such that the next loop
		% iteration can use them without modifications:
		\let\pgfplotsplothandlermesh@ba=\pgfplotsplothandlermesh@bb
		\let\pgfplotsplothandlermesh@aa=\pgfplotsplothandlermesh@ab
		%
}%

% The color handling of 'matrix input=image' is quite involved.
% There are the following requirements:
% 1. the point meta of the cell's first vertex _must_ be equal to the point meta of the cell mid point, i.e. equal to the user input. 
% 2. All other vertices should be interpolated "as smart as possible". This interpolation scheme should work even if adjacent cell midpoints have _no_ bounded meta data!
% 3. The routines which glue adjacent cells together have to ensure that they do not accidentally intermix (1) and (2). This is complicated since the sweep loop intermingles both responsibilities (sigh).
%
% This helper function ensures (1): it copies the point meta of the \pgfplotspatchvertex #1 to the values of of #2 and assigns the result to \pgfplotsretval.
% #1 a valid \pgfplotspatchvertex
% #2 a valid \pgfplotspatchvertex
%
% POSTCONDITION: \pgfplotsretval contains the result, i.e. a copy of '#2' which contains the point meta of '#1'.
\def\pgfplotsplothandlermesh@image@set@first@color@to#1\to#2{%
	\let\pgfplotsretval=#2%
	\pgfplotspatchvertexcopymeta#1\to\pgfplotsretval
}%

\def\pgfplotsplothandlermesh@PIPE@DECODE@image@sweep@create@cell@lastrow{%
	%
	\let\pgfplotsplothandlermesh@last@aa=\pgfplotsplothandlermesh@ba
	\let\pgfplotsplothandlermesh@last@ab=\pgfplotsplothandlermesh@bb
	%
	% ASSUMPTION: we want to compute Laa, Lab, Lba and Lbb in the following setup:
	%
	%  aa -----  ab
	%  |         |
	%  |         |
	%  |   imjm  |          
	%  |         |     
	%  |         |
	%  ba ------ bb
	% =Laa       =Lab
	%  |         |
	%  |         |
	%  |   ijm   |          
	%  |         |     
	%  |         |
	%  Lba------ Lbb
	%
	% compute Lbb = bb + 2 * (1/2 ( ijm + ij) - bb ). We assume that
	% we can still access 'bb', the coordinate of the current point.
	% This resolves to the following:
	\pgfplotspatchvertexaccumstart
	\pgfplotspatchvertexaccum@configure@ignore@unboundedmetadata
	\expandafter\pgfplotspatchvertexadd\pgfplotsplothandlermesh@bb\times{-1.0}%
	\expandafter\pgfplotspatchvertexadd\pgfplotsplothandlermesh@i@jm\times{1.0}%
	\expandafter\pgfplotspatchvertexadd\pgfplotsplothandlermesh@i@j\times{1.0}%
	\pgfplotspatchvertexfinish\pgfplotsplothandlermesh@last@bb
	% FIXME : which is the correctly extrapolated cdata!?
	%\pgfplotspatchvertexcopymetaifbounded\pgfplotsplothandlermesh@i@j\to\pgfplotsplothandlermesh@last@bb
	%
	\ifnum\c@pgfplots@scanlineindex=1
		% compute Lba = Lbb + Laa - Lab
		\pgfplotspatchvertexaccumstart
		\pgfplotspatchvertexaccum@configure@ignore@unboundedmetadata
		\expandafter\pgfplotspatchvertexadd\pgfplotsplothandlermesh@last@bb\times{1.0}%
		\expandafter\pgfplotspatchvertexadd\pgfplotsplothandlermesh@last@aa\times{1.0}%
		\expandafter\pgfplotspatchvertexadd\pgfplotsplothandlermesh@last@ab\times{-1.0}%
		\pgfplotspatchvertexfinish\pgfplotsplothandlermesh@last@ba
		% FIXME : which is the correctly extrapolated cdata!?
		%\pgfplotspatchvertexcopymetaifbounded\pgfplotsplothandlermesh@i@jm\to\pgfplotsplothandlermesh@last@ba
	\else
		% transported from previous iteration.
	\fi
	%
	\pgfplotsplothandlermesh@image@set@first@color@to\pgfplotsplothandlermesh@i@jm\to\pgfplotsplothandlermesh@last@aa
	\expandafter\pgfplotsplothandlermesh@setnextvertex\expandafter{\pgfplotsretval}%
	\expandafter\pgfplotsplothandlermesh@setnextvertex\expandafter{\pgfplotsplothandlermesh@last@ab}%
	\expandafter\pgfplotsplothandlermesh@setnextvertex\expandafter{\pgfplotsplothandlermesh@last@bb}%
	\expandafter\pgfplotsplothandlermesh@setnextvertex\expandafter{\pgfplotsplothandlermesh@last@ba}%
	%
	\let\pgfplotsplothandlermesh@last@ba=\pgfplotsplothandlermesh@last@bb
}%

\def\pgfplotsplothandlermesh@PIPE@DECODE@image@sweep@create@cell@lastrow@lastcol{%
	%
	% ASSUMPTION: we want to compute Laa, Lab, Lba and Lbb in the following setup:
	%
	%  aa -----  ab
	%  |         |
	%  |         |
	%  |   imj   |          
	%  |         |     
	%  |         |
	%  ba ------ bb
	% =Laa       =Lab
	%  |         |
	%  |         |
	%  |   ij    |          
	%  |         |     
	%  |         |
	%  Lba------ Lbb
	\let\pgfplotsplothandlermesh@last@aa=\pgfplotsplothandlermesh@ba
	\let\pgfplotsplothandlermesh@last@ab=\pgfplotsplothandlermesh@bb
	%
	% I mirrored the formula for lastrow.
	\pgfplotspatchvertexaccumstart
	\pgfplotspatchvertexaccum@configure@ignore@unboundedmetadata
	\expandafter\pgfplotspatchvertexadd\pgfplotsplothandlermesh@last@bb\times{1.0}%
	\expandafter\pgfplotspatchvertexadd\pgfplotsplothandlermesh@last@aa\times{-1.0}%
	\expandafter\pgfplotspatchvertexadd\pgfplotsplothandlermesh@last@ab\times{1.0}%
	\pgfplotspatchvertexfinish\pgfplotsplothandlermesh@last@bb
	\pgfplotspatchvertexcopymetaifbounded\pgfplotsplothandlermesh@i@j\to\pgfplotsplothandlermesh@last@bb
	%
	\pgfplotsplothandlermesh@image@set@first@color@to\pgfplotsplothandlermesh@i@j\to\pgfplotsplothandlermesh@last@aa
	\expandafter\pgfplotsplothandlermesh@setnextvertex\expandafter{\pgfplotsretval}%
	\expandafter\pgfplotsplothandlermesh@setnextvertex\expandafter{\pgfplotsplothandlermesh@last@ab}%
	\expandafter\pgfplotsplothandlermesh@setnextvertex\expandafter{\pgfplotsplothandlermesh@last@bb}%
	% this value is transported from previous iteration:
	\expandafter\pgfplotsplothandlermesh@setnextvertex\expandafter{\pgfplotsplothandlermesh@last@ba}%
}%

\def\pgfplotsplothandlermesh@PIPE@DECODE@image@sweep@create@cell@lastcolumn#1{%
	\let\pgfplotsplothandlermesh@patchclass\pgfplotsplothandlermesh@patchclass@input
	%
	% we have to compute the four corners by means of
	% averaging/extrapolation.
	%
	% We place the rectangle around (i-1,j), i.e. imj.
	% This is in contrast to
	% \pgfplotsplothandlermesh@PIPE@DECODE@image@sweep@create@cell
	% where we placed the rectange around (i-1,j-1).
	%
	% Notation: the four corners are aa, ab, ba, and bb.
	%  aa -----  ab
	%  |         |
	%  |         |
	%  |   imj   |          
	%  |         |     
	%  |         |
	%  ba ------ bb
	%
	% We can simply copy aa and ba from the previous cell (as in
	% the standard case). Note that this has already been prepared
	% for us, i.e. the variables have already been set by the
	% previous iteration.
	%
	% compute ab and bb by means of extrapolation:
	%
	% compute bb:
	\pgfplotspatchvertexaccumstart
	\pgfplotspatchvertexaccum@configure@ignore@unboundedmetadata
	\expandafter\pgfplotspatchvertexadd\pgfplotsplothandlermesh@ba\times{-1.0}%
	\expandafter\pgfplotspatchvertexadd\pgfplotsplothandlermesh@im@j\times{1.0}%
	\expandafter\pgfplotspatchvertexadd\pgfplotsplothandlermesh@i@j\times{1.0}%
	\pgfplotspatchvertexfinish\pgfplotsplothandlermesh@bb
	% FIXME : which is the correctly extrapolated cdata!?
	%
	%
	\if1#1%
		% Ah -- this here is the very first row! That means we
		% have to extrapolate the topmost coordinates.
		%
		% Apply extrapolation for ab:
		%
		% compute ab = bb + (aa - ba)
		\pgfplotspatchvertexaccumstart
		\pgfplotspatchvertexaccum@configure@ignore@unboundedmetadata
		\expandafter\pgfplotspatchvertexadd\pgfplotsplothandlermesh@bb\times{1.0}%
		\expandafter\pgfplotspatchvertexadd\pgfplotsplothandlermesh@aa\times{1.0}%
		\expandafter\pgfplotspatchvertexadd\pgfplotsplothandlermesh@ba\times{-1.0}%
		\pgfplotspatchvertexfinish\pgfplotsplothandlermesh@ab
		% FIXME : which is the correctly extrapolated cdata!?
	\else
		\pgfplotsdequepopfront{lastinterpscanline}\to\pgfplotsplothandlermesh@ab
	\fi
	%
	\pgfplotsplothandlermesh@image@set@first@color@to\pgfplotsplothandlermesh@im@j\to\pgfplotsplothandlermesh@aa
	\expandafter\pgfplotsplothandlermesh@setnextvertex\expandafter{\pgfplotsretval}%
	\expandafter\pgfplotsplothandlermesh@setnextvertex\expandafter{\pgfplotsplothandlermesh@ab}%
	\expandafter\pgfplotsplothandlermesh@setnextvertex\expandafter{\pgfplotsplothandlermesh@bb}%
	\expandafter\pgfplotsplothandlermesh@setnextvertex\expandafter{\pgfplotsplothandlermesh@ba}%
	%
	% ... and communicate this as well to the next scanline (see
	% \pgfplotsplothandlermesh@PIPE@DECODE@image@after@second)
	\expandafter\pgfplotsdequepushback\pgfplotsplothandlermesh@bb\to{lastinterpscanline}%
	%
	\ifnum\c@pgfplotsplothandlermesh@image@nextrownum=\pgfplotsplothandlermesh@numscanlines\relax
		\pgfplotsplothandlermesh@PIPE@DECODE@image@sweep@create@cell@lastrow@lastcol
	\fi
}%
