% tipaman3.tex
% Copyright 2002 FUKUI Rei
%
% This program may be distributed and/or modified under the
% conditions of the LaTeX Project Public License, either version 1.2
% of this license or (at your option) any later version.
% The latest version of this license is in
%   http://www.latex-project.org/lppl.txt
% and version 1.2 or later is part of all distributions of LaTeX 
% version 1999/12/01 or later.
%
% This program consists of all files listed in Manifest.txt.
%

\begingroup
\raggedbottom

\chapter{Recent Changes}

\section{Changes from Version 1.2 to 1.3}

Some symbols included in the xipa and related font families have been
modified. 

\section{Changes from Version 1.1 to 1.2}

The following 

\begin{itemize}
\item The following symbols are added to the \texttt{tipx} fonts:

  Varieties of glottal stop symbols and a new symbol:\\
  \textglotstopvari\ (\texttt{\tbs textglotstopvari})\\
  \textglotstopvarii\ (\texttt{\tbs textglotstopvarii})\\
  \textglotstopvariii\ (\texttt{\tbs textglotstopvariii})\\
  \textlfishhookrlig\ (\texttt{\tbs textlfishhookrlig})

\item Symbol shapes of the \texttt{xipa} and \texttt{xipx} font
  families slightly modified.
\end{itemize}

\section{Changes from Version 1.0 to 1.1}

The following changes have been made since the first release of
\tipa\footnote{The first release of \tipa{} has been known as
  `beta0624'. I originally intended to change it to
  something like `tipa-1.0' soon after the release but unfortunately
  I didn't have the opportunity to do so.}.

\begin{itemize}
\item The following typefaces are newly added in Version
  1.1. Font description files (\texttt{*.fd}) modified accordingly.

  Bold Extended Slanted Roman: \textbf{\textipa{[\textsl{""Ekspl@"neIS@n}]}}\\
  Sans Serif Bold Extended: \textbf{\textsf{\textipa{[""Ekspl@"neIS@n]}}}\\
  Sans Serif Slanted:  \textsf{\textipa{[\textsl{""Ekspl@"neIS@n}]}}\\
  Typewriter Text: \texttt{\textipa{[\textsl{""Ekspl@"neIS@n}]}}\\
  Typewriter Text Slanted: \textsl{\texttt{\textipa{[\textsl{""Ekspl@"neIS@n}]}}}

\item Many bugs fixed in the \MF\ source codes; modifications made for
  almost every symbol. The \texttt{xipa} family of fonts now more closely
  simulates Times Roman style.\footnote{%
    I'm not fully satisfied with the result of this simulation and
    further changes will be made in the next release. However, I have
    no intention of simulating too closely in order to avoid any
    possible copyright problems.}

\item \texttt{t3enc.def} and \texttt{tipa.sty} modified.

\item New series of fonts, \textbf{tipx} and \textbf{xipx} have been
  created. These fonts are collections of symbols missing in the
  previous version of \tipa{} and cover almost all the symbols that
  appear in the second edition of \emph{PSG} (1996). (Remember that
  \tipa{} was released in 1996 and at the time the second edition of
  \emph{PSG} was not available.) Some of the symbols included in the
  previous version of \tipa{} are now moved into \textbf{tipx} and
  \textbf{xipx}. Thus the T3 encoding is slightly modified.

  In order to use newly created fonts, add the following after the
  declaration of \tipa{}.

  \verb|\usepackage{tipx}|

  For a list of newly created symbols, see next section.

  The encoding of \texttt{tipx} and \texttt{xipx} still has no
  definite name. The style file (\texttt{tipx.sty}) uses the
  U encoding and new family names (tipx and xipx which are arbitrary).
  In the future, it may be possible to use a new encoding name TS3
  (I experimentally put \texttt{ts3enc.def} and \texttt{ts3*.fd} in
  the \texttt{sty} directory of the package. Use
  these files at your own risk, if the system doesn't complain.)

\item Some new tone letter commands, \verb|\stone| and \verb|\rtone|.
\item Manual updated.
\item Manual for the \texttt{vowel.sty} completed.
\item Some diacritic commands added.
\end{itemize}

\subsection{Newly created symbols}\label{sec:newsymbols}

The following two symbols are newly adopted in the \texttt{tipa}
encoding (i.e., T3).

\begin{quote}
Hooktop right-tail D --- \texthtrtaild\\
Left-hooktop long Y --- \textlhtlongy
\end{quote}

The following command was realized by a macro in the previous version
but now is assigned a code of its own in the \texttt{tipa} encoding
(i.e., T3).

\begin{quote}
Crossed lambda --- \textcrlambda
\end{quote}

The following symbols are (mostly) newly created symbols in the
\texttt{tipx} fonts. (Note that some are moved from the \texttt{tipa}
because of the encoding change.)

\begin{quote}
Right-hook A --- \textrhooka\\
Left-hook four --- \textlhookfour\\
Inverted script A --- \textinvscripta\\
A-O ligature --- \textaolig\\
Inverted small capital A --- \textinvsca\\
Small capital A-O ligature --- \textscaolig\\
Stretched C (original form) --- \textstretchcvar\\
Curly-tail stretched C --- \textctstretchc\\
Curly-tail stretched C (original form) --- \textctstretchcvar\\
Front-hook D --- \textfrhookd\\
Front-hook D (Original) --- \textfrhookdvar\\
D-B ligature --- \textdblig\\
Small capital delta --- \textscdelta\\
Right-hook E --- \textrhooke\\
Right-hook epsilon --- \textrhookepsilon\\
Small capital F --- \textscf\\
Greek gamma --- \textgrgamma\\
Front-tail gamma --- \textfrtailgamma\\
Back-tail gamma --- \textbktailgamma\\
Right-tail hooktop H --- \textrtailhth\\
Heng --- \textheng\\
Curly-tail J (a variety found in 1996 IPA) --- \textctjvar\\
Hooktop barred dotless J (a variety) --- \texthtbardotlessjvar\\
Small capital K --- \textsck\\
Turned small capital K --- \textturnsck\\
Reversed small capital L --- \textrevscl\\
H-M ligature --- \texthmlig\\
Small capital M --- \textscm\\
Front-bar N --- \textfrbarn\\
Right leg N --- \textnrleg\\
Bull's eye (an old version) --- \textObullseye\\
Female sign --- \textfemale\\
Uncrossed female sign --- \textuncrfemale\\
Right-hook open O --- \textrhookopeno\\
Inverted omega --- \textinvomega\\
Left-hook P --- \textlhookp\\
Small capital P --- \textscp\\
A variety of thorn (1) --- \textthornvari\\
A variety of thorn (2) --- \textthornvarii\\
A variety of thorn (3) --- \textthornvariii\\
A variety of thorn (4) --- \textthornvariv\\
Q-P ligature --- \textqplig\\
Reversed small capital R --- \textrevscr\\
Reversed esh with top loop --- \textlooptoprevesh\\
Front-hook T --- \textfrhookt\\
Curly-tail turned T --- \textctturnt\\
Turned small capital U --- \textturnscu\\
Turned two --- \textturntwo\\
Bent-tail yogh --- \textbenttailyogh\\
Turned three --- \textturnthree\\
Curly-tail inverted glottal stop --- \textctinvglotstop\\
Turned glottal stop (PSG 1996:211) --- \textturnglotstop\\
Pipe (a variety with no descender) --- \textpipevar\\
Double pipe (a variety with no descender) --- \textdoublepipevar\\
Double-barred pipe (a variety with no descender) --- \textdoublebarpipevar\\
Superscript left arrow --- \textspleftarrow\\
Down full arrow --- \textdownfullarrow\\
Up full arrow --- \textupfullarrow\\
Subscript right arrow --- \textsubrightarrow\\
Subscript double arrow --- \textsubdoublearrow\\
Reversed Polish hook --- an accent command e.g., \textrevpolhook{o}\\
Retracting sign (a variety) --- \textretractingvar\\
Right hook (long) --- \textrthooklong\\
Palatalization hook (long) --- \textpalhooklong\\
Palatalization hook (a variety) --- \textpalhookvar
\end{quote}

\subsection{Symbol shape changes}

Shapes of the following symbols have been modified from the first
version to the present.

\begin{center}\tabcolsep.2em
\begin{tabular}{llccl}
Name  & Macro name  & New  & Old  & Old symbol name\\
\hline
Pipe                     & \Tt{textpipe}         & \textpipe
 & \textpipevar          & \Tt{textpipevar}\\
Double pipe              & \Tt{textdoublepipe}   & \textdoublepipe
 & \textdoublepipevar    & \Tt{textdoublepipevar}\\
Double-barred pipe       & \Tt{textdoublebarpipe}& \textdoublebarpipe
 & \textdoublebarpipevar & \Tt{textdoublebarpipevar}\\
Down arrow               & \Tt{textdownstep}     & \textdownstep
 & \textdownfullarrow    & \Tt{textdownfullarrow}\\
Up arrow                 & \Tt{textupstep}       & \textupstep
 & \textupfullarrow      & \Tt{textupfullarrow}\\
Bull's eye               & \Tt{textbullseye}     & \textbullseye
 & \textObullseye        & \Tt{textObullseye}\\
Hooktop barred           & \Tt{texthtbardotlessj}& \texthtbardotlessj
 & \texthtbardotlessjvar & \Tt{texthtbardotlessjvar}\\
dotless J \\
\end{tabular}
\end{center}

For each symbol, the old shape is preserved in the \texttt{tipx} fonts
and can be accessed by a new name (in most cases \texttt{var} or
\texttt{O} is attached) indicated at the rightmost column of the above
table.


\clearemptydoublepage
\chapter{Symbols not included in TIPA}

Although the present version of \tipa{} includes almost all the
symbols found in \PSG\ and \Handbook, there are still some symbols not
included or defined in \tipa{}.

Some such symbols can be realized by writing appropriate
macros, while some others cannot be realized without resorting to
the Metafont.

This section discusses these problems by classifying such symbols into
three categories, as shown below.

\begin{enumerate}
\itemsep0pt
\item Symbols that can be realized by \TeX{}'s macro level and/or by using
  symbols from other fonts.
\item Symbols that can be imitated by \TeX{}'s macro level and/or by using
  symbols from other fonts (but may not look quite satisfactory).
\item Symbols that cannot be realized at all, without creating a new
  font.
\end{enumerate}

With the addition of the \tipx{} fonts, symbols that belong to the
third category are virtually non-existent now.

As for the symbols that belong to the first and second categories,
\tipa{} provides a variety of macros and parts of symbols that can be
used to compose a desired symbol if you can write an appropriate
macro for it.

The following table shows symbols that belong to the first category.
For each symbol, an example of input method and its output is also
given. Note that barred or crossed symbols can be easily made by
\tipa{}'s \verb|\ipabar| macro.

\def\SecLine{\>}

\medskip
\begin{tabbing}
\iftimes
x \=xxxxxxxxxxxxxxxxxxxxxxxx \=xxxxxxxxxxxxxxxxxxxxxxxxxxxxxxxxxxxxxx \= \kill
\else
x \=xxxxxxxxxxxxxxxxxxxxxxxx \=xxxxxxxxxxxxxxxxxxxxxxxxxxxxxxxxxxxx \= \kill
\fi
\> Barred small capital I 
  \SecLine \verb|\ipabar{\textsci}{.5ex}{1.1}{}{}| \>
  \ipabar{\textsci}{.5ex}{1.1}{}{} \\
\> Barred J 
  \SecLine \verb|\ipabar{j}{.5ex}{1.1}{}{}| \>
  \ipabar{j}{.5ex}{1.1}{}{} \\
\> Crossed K 
  \SecLine \verb|\ipabar{k}{1.2ex}{.6}{}{.4}| \>
  \ipabar{k}{1.2ex}{.6}{}{.4} \\
\> Barred open O 
  \SecLine \verb|\ipabar{\textopeno}{.5ex}{.6}{.4}{}| \>
  \ipabar{\textopeno}{.5ex}{.5}{.5}{} \\
\> Barred small capital omega 
  \SecLine \verb|\ipabar{\textscomega}{.5ex}{1.1}{}{}| \>
  \ipabar{\textscomega}{.5ex}{1.1}{}{} \\
\> Barred P 
  \SecLine \verb|\ipabar{p}{.5ex}{1.1}{}{}| \>
  \ipabar{p}{.5ex}{1.1}{}{} \\
\> Half-barred U 
  \SecLine \verb|\ipabar{u}{.5ex}{.5}{}{.5}| \>
  \ipabar{u}{.5ex}{.5}{}{.5} \\
\> Barred small capital U 
  \SecLine \verb|\ipabar{\textscu}{.5ex}{1.1}{}{}| \>
  \ipabar{\textscu}{.5ex}{1.1}{}{} \\
\iftimes
\> Double slash 
  \SecLine \verb|/\kern-.1em/| \>
  /\kern-.1em/ \\
\> Triple slash 
  \SecLine \verb|/\kern-.1em/\kern-.1em/| \>
  /\kern-.1em/\kern-.1em/ 
\else
\> Double slash 
  \SecLine \verb|/\kern-.25em/| \>
  /\kern-.25em/ \\
\> Triple slash 
  \SecLine \verb|/\kern-.25em/\kern-.25em/| \>
  /\kern-.25em/\kern-.25em/ 
\fi
\end{tabbing}

The next definitions attach a tiny `left hook' (which shows
palatalization) to a symbol. For example:

\iftimes
\newcommand\textlhookb{{\tipaencoding
  b\hspace{-.1em}\raisebox{.0ex}{\textpalhookvar}}}
\newcommand\textlhookm{{\tipaencoding
  m\hspace{-.1em}\raisebox{.0ex}{\textpalhook}}}
\begin{verbatim}
  % Left-hook B
  \newcommand\textlhookb{{\tipaencoding
    b\hspace{-.1em}\raisebox{.0ex}{\textpalhookvar}}}
  % Left-hook M
  \newcommand\textlhookm{{\tipaencoding
    m\hspace{-.1em}\raisebox{.0ex}{\textpalhook}}}
\end{verbatim}
\else
\newcommand\textlhookb{{\tipaencoding
  b\hspace{-.15em}\raisebox{.0ex}{\textpalhookvar}}}
\newcommand\textlhookm{{\tipaencoding
  m\hspace{-.15em}\raisebox{.0ex}{\textpalhook}}}
\begin{verbatim}
  % Left-hook B
  \newcommand\textlhookb{{\tipaencoding
    b\hspace{-.15em}\raisebox{.0ex}{\textpalhookvar}}}
  % Left-hook M
  \newcommand\textlhookm{{\tipaencoding
    m\hspace{-.15em}\raisebox{.0ex}{\textpalhook}}}
\end{verbatim}
\fi

The former example uses a left-hook called \Tt{textpalhookvar},
(\KK\textpalhookvar\KK) and the latter uses a hook called \Tt{textpalhook},
(\KK\textpalhook\KK).

\begin{quote}
  Left-hook B --- \textlhookb \\
  Left-hook M --- \textlhookm
\end{quote}

Symbols that belong to the second category are shown below. Note that
slashed symbols can be in fact easily made by a macro. For example, a
slashed b i.e., \ipaclap{\textipa{b}}{\textipa{/}} can be made by
\verb|\ipaclap{\textipa{b}}{\textipa{/}}|. The reason why slashed
symbols are not included in \tipa{} is as follows: first, a simple
overlapping of a symbol and a slash does not always result in a good
shape, and secondly, it doesn't seem significant to devise fine-tuned
macros for symbols which were created essentially for typewriters.

\medskip
\begin{tabbing}
xxxx \=xxxxxxxxxxxxxxxxxxxxxxxx \= \kill
\> Slashed B \>
  \ipaclap{\textipa{b}}{\textipa{/}} \\
\> Slashed C \>
  \ipaclap{\textipa{c}}{\textipa{/}} \\
\> Slashed D \>
  \ipaclap{\textipa{d}}{\textipa{/}} \\
\> Slashed U \>
  \ipaclap{\textipa{u}}{\textipa{/}} \\
\> Slashed W \>
  \ipaclap{\textipa{w}}{\textipa{/}}
\end{tabbing}

\endgroup

\clearemptydoublepage
\chapter{FAQ}

\newcount\FAQcnt \FAQcnt=0

\newcommand\QandA[2]{{\par\bigskip\parindent0pt
  \global\advance\FAQcnt 1
  \hangindent2em\hangafter1 \textbf{Q\the\FAQcnt:} #1\par\medskip
  \hangindent2em\hangafter1 \textbf{A\the\FAQcnt:} #2\par\medskip}}

\newcommand\NextPar{\par\hangindent2em\hangafter0\relax}

\QandA {I have installed all the \tipa{} fonts. But the system can't
  find them. What's wrong?}{Please don't forget to run the command
  \texttt{mktexlsr} after the installation. Also, try to run the
  command:\par\begin{quote}\texttt{kpsewhich tipa10.mf}\end{quote}
  \NextPar If the system shows nothing in return, you must have
  installed them in a wrong place.}

\QandA {I'm using shortcut letters but there are still many symbols
  which have no shortcut letters. What can I do? Do I have to use all
  these long names?}{You are free to define shorter names. \LaTeX's
  \texttt{\tbs newcommand} is a safe way to do this. For example:

  \begin{quote}
  \texttt{\tbs newcommand\tbi\tbs vef\tbii\tbi\tbs textbarrevglotstop\tbii}
  \end{quote}

  \vspace{-5mm}
  \begin{tipaexample}
    \yitem \texttt{[\tbs vef] is a voiced epiglottal fricative.}
    \yitem \textipa{\let\vef\textbarrevglotstop [\vef]} is a
      voiced epiglottal fricative.
  \end{tipaexample}}

\QandA {I want to use the \LaTeX\ command \texttt{\tbs |} in the IPA
  environment. But I don't want to specify the \texttt{safe}
  option. Is it possible?}{Use a command called \texttt{\tbs Vert}
  instead of \texttt{\tbs |}. It has the same meaning. Other possibly
  dangerous commands such as \texttt{\tbs:}, \texttt{\tbs:} and
  \texttt{\tbs!} have a similar substitute command. For more
  details, see page~\pageref{unsafemode}.}
  
\QandA {I can't input Eng (\texttt{\tbs ng})
  properly. Why?}{Use \texttt{\tbs textipa\tbi N\tbii}. Technically
  speaking, this is a matter of priority among the \texttt{OT1},
  \texttt{T1} and \texttt{T3} encodings. But may be called a bug. I'll
  work out this problem in the next release.}

\QandA {How can I input \emph{capital letters}, I mean real capital
  letters, not small capitals, within the IPA environment?}{Use the
  command \texttt{\tbs*}. For example:

  \begin{tipaexample}
    \yitem \texttt{\tbs textipa\tbi["pI\tbs *Di]\tbii}
    \yitem \textipa{["pI\*Di]}
  \end{tipaexample}\NextPar This command is explained in
  section~\ref{sec:specialmacros}.}

\QandA {How can I output an accent or diacritic symbol alone? For
  example, I want to print the umlaut symbol alone, in order to
  explain the usage of this symbol.}{Try to add an empty argument
 to the umlaut command.

  \begin{tipaexample}
    \yitem \texttt{\tbs texipa\tbi[\tbs"\tbi\tbii]\tbii}
    \yitem \textipa{[\"{}]}
  \end{tipaexample}}

\QandA {Are there only a limited number of tone letters?}
  {Absolutely not! Please read section \ref{sec:tone} carefully.}

\QandA {How to create a PDF file?} {You can find a few examples in
  section \ref{sec:pdf}.}

\QandA {I have succeeded in creating a PDF document. But \tipa{} fonts
  don't look good (jaggy).  What's wrong?}{Type1 fonts are not
  embedded in your document and pk fonts are used instead. Install
  Type1 font files and/or map file correctly.}

\QandA {I have succeeded in creating a PDF document with Type1 fonts
  embedded. But some symbols are missing. Why?}{In some versions of
  \texttt{dvips}, the character shifting switch is turned on by
  default. In order to prevent this, try to invoke \texttt{dvips} in
  the following way.\par
  \begin{quote}\texttt{dvips -Ppdf -G0} \textsl{filename}\end{quote}}

\QandA {I find no description on hyphenation of phonetic texts in this
  manual.}{I haven't seen any description on hyphenation in \Handbook\ 
  nor in \emph{Principles}. }

\QandA {Why is italic font not included in \tipa? Slanted fonts can be
  used as substitutes. But I want real italic fonts.}{It isn't
  difficult to create italic shapes for a limited number of symbols
  such as Schwa, Turned script A, and so on. However, creating a whole
  set of IPA symbols in italic is quite a different story. It is
  difficult to distinguish, for example, Lower-case A and Script A in
  italic. In the IPA's \emph{Principles}, it is recommended
  that the IPA symbols should be roman, excluding italic shapes in
  some of the examples.  Another point that should be made is that
  there exist several systems of phonetic symbols in which all the
  symbols appear in italic. These are the ones mainly used in
  Scandinavian countries, and the problem is, there is no one-to-one
  correspondences between such systems and the IPA.  Aside from the
  strictly phonetic use of symbols, however, there is a practical need
  for italic versions of symbols such as italic Schwa. Therefore, it
  may be helpful to create a new auxiliary font containing limited
  number of italic symbols.}

\QandA {Which is the first name of the author of \tipa? I'm confused.}
  {Rei is his first name.}

\QandA {I can't send e-mail to the author.}
  {I recently changed my e-mail address.

  \begin{quote}\texttt{fkr@l.u-tokyo.ac.jp}\end{quote}}
  

\vspace{\stretch{2}}

\betacomment
\thetacomment

\endinput

%%% Local Variables: 
%%% mode: latex
%%% TeX-master: "tipaman"
%%% End: 
