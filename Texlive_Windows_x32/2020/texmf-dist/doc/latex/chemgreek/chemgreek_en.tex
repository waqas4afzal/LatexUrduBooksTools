% arara: pdflatex: { shell: on , interaction: nonstopmode }
% !arara: biber
% !arara: pdflatex
% !arara: pdflatex
% --------------------------------------------------------------------------
% the CHEMGREEK package
%
%   interface for upright greek letters for use in chemistry
%
% --------------------------------------------------------------------------
% Clemens Niederberger
% --------------------------------------------------------------------------
% https://github.com/cgnieder/chemgreek/
% contact@mychemistry.eu
% --------------------------------------------------------------------------
% If you have any ideas, questions, suggestions or bugs to report, please
% feel free to contact me.
% --------------------------------------------------------------------------
% Copyright 2015--2020 Clemens Niederberger
%
% This work may be distributed and/or modified under the
% conditions of the LaTeX Project Public License, either version 1.3
% of this license or (at your option) any later version.
% The latest version of this license is in
%   http://www.latex-project.org/lppl.txt
% and version 1.3 or later is part of all distributions of LaTeX
% version 2005/12/01 or later.
%
% This work has the LPPL maintenance status `maintained'.
%
% The Current Maintainer of this work is Clemens Niederberger.
% --------------------------------------------------------------------------
\documentclass[load-preamble+]{cnltx-doc}
\KOMAoption{listof}{totoc}
\usepackage[utf8]{inputenc}
\usepackage{chemgreek}
\setcnltx{
  package  = {chemgreek},
  info     = {interface for upright greek letters for use in chemistry},
  url      = http://www.mychemistry.eu/forums/forum/chemgreek/ ,
  authors  = Clemens Niederberger ,
  email    = contact@mychemistry.eu ,
  abstract = {%
    \centering
    \includegraphics{chemmacros-logo.pdf}
    \par
  } ,
  add-cmds = {
    activatechemgreekmapping,
    changechemgreeksymbol, chemalpha, chembeta, chemgamma, chemdelta,
      chemDelta, chemomega, chemphi, chemPhi,
    chemgreekmappingsymbol,
    declarechemgreekmapping,
    newchemgreekmapping,
    printchemgreekalphabet,
    renewchemgreekmapping,
    selectchemgreekmapping
  } ,
  index-setup = { level = \section , headers={\indexname}{\indexname} , noclearpage }
}

\usepackage{array,booktabs}

\expandafter\def\csname libertine@figurestyle\endcsname{LF}
\usepackage[libertine]{newtxmath}
\expandafter\def\csname libertine@figurestyle\endcsname{OsF}

\usepackage{chemmacros}
\selectchemgreekmapping{newtx}

\usepackage[biblatex]{embrac}
\ChangeEmph{[}[,.02em]{]}[.055em,-.08em]
\ChangeEmph{(}[-.01em,.04em]{)}[.04em,-.05em]

\usepackage[accsupp]{acro}
\acsetup{
  long-format  = \scshape ,
  short-format = \scshape
}
\DeclareAcronym{iupac}{
  short     = iupac ,
  long      = International Union of Pure and Applied Chemistry ,
  pdfstring = IUPAC ,
  accsupp   = IUPAC
}
\DeclareAcronym{pdf}{
  short     = pdf ,
  long      = portable document file ,
  pdfstring = PDF ,
  accsupp   = PDF
}

\defbibheading{bibliography}{\section{References}}

\newcommand*\tablehead[1]{\textrm{\bfseries#1}}

% \BeforeBeginEnvironment{example}{\vspace{\baselineskip}}
% \AfterEndEnvironment{example}{\vspace{\baselineskip}}
% \BeforeBeginEnvironment{sourcecode}{\vspace{\baselineskip}}
% \AfterEndEnvironment{sourcecode}{\vspace{\baselineskip}}

\begin{document}
\selectchemgreekmapping{newtx}

\listoftables

\section{Introduction}
The \chemgreek{} package is an auxiliary package for other chemistry packages
such as \pkg{chemmacros}.  In chemistry there is often the need for upright
greek letters.  The \chemgreek{} package provides an interface to various
other packages that provide upright greek letters.  One could mention
\pkg{textgreek}, \pkg{upgreek}, \pkg{newtx} or \pkg{kpfonts}.  All of these
packages provide upright greek letters, some a whole alphabet some only the
upright variants of the standard italic symbols for which macros are defined
in base \LaTeX.

\chemgreek{} offers a possibility to map those different interfaces to a
unified set of macros for usage in a chemistry package.  This is useful as
then for example names like \iupac{\b-\D-gluco|pyranose} can be typeset
with a semantic interface and still have matching greek letters while the user
is not limited to a certain package or font.  Consequently this package is
used by the \pkg{chemmacros} package~\cite{pkg:chemmacros} and its \acs{iupac}
naming commands, for example, and by the \pkg{chemnum}
package~\cite{pkg:chemnum}.

\section{Licence and Requirements}
\license

\chemgreek{} loads the following packages:
\pkg{expl3}\footnote{\CTANurl{l3kernel}}~\cite{bnd:l3kernel},
\pkg{xparse}\footnote{\CTANurl{l3packages}}~\cite{bnd:l3packages} and
\pkg{amstext}\footnote{\url{http://ctan.org/tex-archive/macros/latex/required/amslatex/math}}%
~\cite{pkg:amstext}.

\section{News}
\subsection{Version~0.2}
\begin{itemize}
  \item The mapping ``mathdesign'' has been added.  In order to use it you
    need the \pkg{mathdesign} package~\cite{pkg:mathdesign} loaded and use one
    of its options.
  \item The mapping ``fourier'' has been added.  In order to use it you
    need the \pkg{fourier} package~\cite{pkg:fourier} loaded.
\end{itemize}

\subsection{Version~0.3}
\begin{itemize}
  \item The provided macros have been renamed from \cs*{Chem\meta{\ldots}}
    into \cs*{chem\meta{\ldots}}.  The uppercase version still are provided
    for backwards compatibility but issue a warning message and will be
    removed some time in the future.
  \item The commands for defining mappings have gotten an optional argument
    which allows to specify the name of the package a mapping needs.  The
    command \cs{selectchemgreekmapping} now checks for this package and gives
    a warning if it doesn't find it loaded.
  \item The mapping ``textalpha'' has been added.  In order to use it you
    need the \pkg{textalpha} package (part of
    \bnd{greek-fontenc}~\cite{bnd:greek-fontenc}) loaded.
  \item If the package \pkg{hyperref}~\cite{pkg:hyperref} is loaded with the
    \code{unicode} option \emph{and} the \pkg{textalpha} package has been
    loaded at begin document all the \cs*{chem\meta{\ldots}} commands are let
    to \pkg{textalpha}'s \cs*{text\meta{\ldots}} commands for the \acs{pdf}
    bookmarks.  This allows Greek letters in the bookmarks without worrying
    about \cs*{texorpdfstring}.
\end{itemize}

\subsection{Version~0.4}
\begin{itemize}
  \item The mapping ``fontspec'' has been added.  In order to use it you
    need the \pkg{fontspec} package~\cite{pkg:fontspec} loaded.  This means it
    can only be used with \LuaLaTeX\ or \XeLaTeX.
  \item New command \cs{printchemgreekalphabet}.
\end{itemize}

\subsection{Version~0.5}
\begin{itemize}
  \item The package is now distributed independently from the \pkg{chemmacros}
    package.
\end{itemize}

\subsection{Version~0.5}
\begin{itemize}
  \item The deprecated macros \cs*{Chem\meta{\ldots}} have been dropped and
    will now cause an error if used.
\end{itemize}

\subsection{Version~1.0}
\begin{itemize}
  \item If a mapping is activated that needs a package and the package is
    missing \chemgreek\ falls back to the `default' mapping now.
  \item If exactly one package for one of the mappings has been loaded the
    corresponding mapping is activated at begin document, see also
    section~\ref{sec:pred-mapp-select}.
  \item expl3 versions of the letter macros (\verbcode+\chemgreek_alpha:+,
    \verbcode+\chemgreek_Alpha:+, \ldots)
  \item New macros \cs{chemgreekmappingsymbol} (see
    section~\ref{sec:additional-macros}), \cs{newchemgreekmappingalias},
    \cs{renewchemgreekmappingalias} and \cs{declarechemgreekmappingalias} (see
    section~\ref{sec:define-mappings}).
\end{itemize}

\section{Define Mappings}\label{sec:define-mappings}
\selectchemgreekmapping{default}

\chemgreek's main commands are:
\begin{commands}
  \command{newchemgreekmapping}[\oarg{package}\marg{name}\marg{mapping list}]
    \changedversion{0.3}Add a new mapping to \chemgreek.  Issues an error if a
    mapping with the same name already exists.  With the optional argument the
    package that is needed for this mapping can (and should) be specified.
  \command{renewchemgreekmapping}[\oarg{package}\marg{name}\marg{mapping list}]
    \changedversion{0.3}Renew a \chemgreek{} mapping.  Issues an error if the
    mapping doesn't exist yet.  With the optional argument the package that is
    needed for this mapping can (and should) be specified.
  \command{declarechemgreekmapping}[\oarg{package}\marg{name}\marg{mapping list}]
    \changedversion{0.3}Declare a new mapping to \chemgreek.  If the mapping
    already exists it will be overwritten.  With the optional argument the
    package that is needed for this mapping can (and should) be specified.
  \command{newchemgreekmappingalias}[\marg{new mapping name}\marg{existing mapping name}]
    \sinceversion{1.0}Define an alias mapping.  Issues an error if \meta{new
    mapping name} already exists.
  \command{renewchemgreekmappingalias}[\marg{new mapping name}\marg{existing mapping name}]
    \sinceversion{1.0}Redefine a mapping to an alias of an existing mapping.
    Issues an error if \meta{new mapping name} doesn't exist, yet.
  \command{declarechemgreekmappingalias}[\marg{new mapping name}\marg{existing mapping name}]
    \sinceversion{1.0}Define an alias mapping.  Doesn't check, if \marg{new
      mapping name} exists or not.
\end{commands}

The command \cs{newchemgreekmapping} needs to get a comma separated list of
24 pairs divided by a slash.  The first entry is the lowercase version und the
second the uppercase version for the corresponding greek letter at the current
position.  This will become clearer if you look at how the \code{default}
mapping is defined:

\begin{sourcecode}
  \newchemgreekmapping{default}
    {
      \ensuremath{\alpha}   / \ensuremath{\mathrm{A}} , %  1: alpha
      \ensuremath{\beta}    / \ensuremath{\mathrm{B}} , %  2: beta
      \ensuremath{\gamma}   / \ensuremath{\Gamma} ,     %  3: gamma
      \ensuremath{\delta}   / \ensuremath{\Delta} ,     %  4: delta
      \ensuremath{\epsilon} / \ensuremath{\mathrm{E}} , %  5: epsilon
      \ensuremath{\zeta}    / \ensuremath{\mathrm{Z}} , %  6: zeta
      \ensuremath{\eta}     / \ensuremath{\mathrm{H}} , %  7: eta
      \ensuremath{\theta}   / \ensuremath{\Theta} ,     %  8: theta
      \ensuremath{\iota}    / \ensuremath{\mathrm{I}} , %  9: iota
      \ensuremath{\kappa}   / \ensuremath{\mathrm{K}} , % 10: kappa
      \ensuremath{\lambda}  / \ensuremath{\Lambda} ,    % 11: lambda
      \ensuremath{\mu}      / \ensuremath{\mathrm{M}} , % 12: mu
      \ensuremath{\nu}      / \ensuremath{\mathrm{N}} , % 13: nu
      \ensuremath{\xi}      / \ensuremath{\Xi} ,        % 14: xi
      \ensuremath{o}        / \ensuremath{\mathrm{O}} , % 15: omikron
      \ensuremath{\pi}      / \ensuremath{\Pi} ,        % 16: pi
      \ensuremath{\rho}     / \ensuremath{\mathrm{P}} , % 17: rho
      \ensuremath{\sigma}   / \ensuremath{\Sigma} ,     % 18: sigma
      \ensuremath{\tau}     / \ensuremath{\mathrm{T}} , % 19: tau
      \ensuremath{\upsilon} / \ensuremath{\Upsilon} ,   % 20: upsilon
      \ensuremath{\phi}     / \ensuremath{\Phi} ,       % 21: phi
      \ensuremath{\psi}     / \ensuremath{\Psi} ,       % 22: psi
      \ensuremath{\chi}     / \ensuremath{\mathrm{X}} , % 23: chi
      \ensuremath{\omega}   / \ensuremath{\Omega}       % 24: omega
  }
\end{sourcecode}

There \emph{must} be 24 pairs of entries, \ie, a complete mapping!  Those
entries are the ones that will be used by the interface macros.  For each
letter a pair \cs{chemalpha}/\cs{chemAlpha} is defined that uses the entries
of the currently active mapping.  That means there are 48 (robust) macros
defined each beginning with \cs*{chem...} followed by the lowercase or
uppercase name of the Greek letter.

The default mapping is -- as you can probably see -- \emph{not an upright
  one}.  This is because \chemgreek{} will not make any choice for a specific
package but lets the user (or another package) choose.  \chemgreek\ however
recognizes if \emph{an unambiguous choice} for one of the upright mappings is
possible and if it is will select the appropriate mapping at begin document.

\begin{example}
  Default mapping: \chemphi\ and \chemPhi, $\phi$ and $\Phi$
\end{example}

\begin{commands}
  \expandable\command{l\_chemgreek\_active\_mapping\_tl}
    The currently active mapping is always available in the token list variable.
\end{commands}

\section{Predefined Mappings and Selection of a Mapping}\label{sec:pred-mapp-select}
\chemgreek\ predefines some mappings.  Some of the mappings require additional
packages to be loaded.  The mapping names and the required packages are listed
in table~\ref{tab:mappings}.  The mapping \code{fontspec} is a bit different
here:  if you use this mapping then the fact is used that \pkg{fontspec} also
defines commands like \cs*{textalpha}.  However, they only work if you also
use a font that has the Greek glyphs.

\begin{table}
  \centering
  \begin{tabular}{>{\ttfamily}ll}
    \toprule
      \tablehead{mapping} & \tablehead{package} \\
    \midrule
      default     & --- \\
      var-default & --- \\
      textgreek   & \pkg{textgreek} \cite{pkg:textgreek} \\
      upgreek     & \pkg{upgreek} \cite{pkg:upgreek} \\
      newtx       & \pkg*{newtxmath} \cite{pkg:newtx} \\
      kpfonts     & \pkg{kpfonts} \cite{pkg:kpfonts} \\
      pxgreeks    & \pkg{pxgreeks} \cite{pkg:pxgreeks} \\
      mathdesign  & \pkg{mathdesign} \cite{pkg:mathdesign} \\
      fourier     & \pkg{fourier} \cite{pkg:fourier} \\
      textalpha   & \pkg{textalpha} \cite{bnd:greek-fontenc} \\
      fontspec    & \pkg{fontspec} \cite{pkg:fontspec} \\
    \bottomrule
  \end{tabular}
  \caption{Predefined mappings.}
  \label{tab:mappings}
\end{table}

If exactly \emph{one} of the packages required by one of the mappings has been
loaded such that an unambiguous choice is possible then \chemgreek\ will
choose and activate the corresponding mapping at begin document.  If an
umbiguous choice isn't possible then \chemgreek\ will select the `default'
mapping at begin document.  If a user has selected a mapping manually in the
preamble (with one of the commands explained in a bit) then \chemgreek\ will
do nothing on its own.

A mapping is selected and activated manually with one of the following
commands:
\begin{commands}
  \command{activatechemgreekmapping}[\sarg\marg{name}]
    \changedversion{1.0}This commands selects and activates the mapping
    \meta{name}.  If the star variant is used also the package of mapping
    \meta{name} (as defined with \cs{newchemgreekmapping} is loaded.
    Otherwise a required package has to be loaded additionally the usual
    way via \cs*{usepackage} or \cs*{RequirePackage}.  If the package hasn't
    been loaded a warning will be written to the log and the `default' mapping
    will be activated instead.  \emph{The command can only be used in the
      document preamble}.
  \command{selectchemgreekmapping}[\marg{name}]
    \changedversion{1.0}This commands selects and activates the mapping
    \meta{name}.  A required package has to be loaded additionally the usual
    way via \cs*{usepackage} or \cs*{RequirePackage}.  If the package hasn't
    been loaded a warning will be written to the log and the `default' mapping
    will be activated instead.  \emph{The command can be used throughout the
    document}.
\end{commands}

\begin{example}
  % requires the `newtxmath' package to be loaded:
  \chemphi\ and \chemPhi, $\phi$ and $\Phi$\par
  \selectchemgreekmapping{newtx}
  \chemphi\ and \chemPhi, $\upphi$ and $\upPhi$
\end{example}

Since the \code{fontspec} mapping is a little bit different than the others
I'd like to show a little example for it.  The difference is subtle: you need
to choose a font containing the needed glyphs.

\begin{example}[compile,program=xelatex,runs=1,add-frame=false]
  \documentclass[margin=3pt]{standalone}
  \usepackage{fontspec}
  \setmainfont{LinLibertine_R.otf}% need a font that has the glyphs!
  \usepackage{chemgreek}
  \selectchemgreekmapping{fontspec}
  \begin{document}
  \printchemgreekalphabet
  \end{document}
\end{example}

\section{Changing a Specific Symbol in an Existing Mapping}\label{sec:chang-spec-symb}
If you should want to change a specific entry of a specific mapping it would
be rather tedious to redefine the whole mapping.  That is why \chemgreek\
provides a command for that purpose:
\begin{commands}
  \command{changechemgreeksymbol}[\marg{mapping
    name}\Marg{upper|lower}\marg{entry name}\marg{entry}]
    Changes the \code{upper}- or \code{lower}case entry \meta{entry name} in
    the mapping \meta{mapping name}.
\end{commands}

In order to activate the change you need the (re-) activate the affected
mapping afterwards:
\begin{example}
  \chemalpha
  \changechemgreeksymbol{default}{lower}{alpha}{xxx}%
  \selectchemgreekmapping{default}
  \chemalpha
\end{example}

\section{Inspecting a Mapping}\label{sec:inspecting-mapping}
\selectchemgreekmapping{newtx}
If you want to check if a mapping has been correctly set you can use the
following commands:
\begin{commands}
  \command{printchemgreekmapping}[\marg{mapping}]
    \sinceversion{0.3}This will typeset a table (using a simple \code{tabular}
    environment) with all~48 characters like the one shown in
    table~\ref{tab:showmapping}.
  \command{printchemgreekalphabet}
    \sinceversion{0.4}This will print the twentyfour pairs of lower- and
    uppercase letters of the currently active mapping: \printchemgreekalphabet.
  \command{showchemgreekmapping}[\marg{mapping}]
    \changedversion{0.3}This command will write information about the
    definition of all 48~macros for a mapping to the log file.
\end{commands}

\begin{table}
  \centering
  \printchemgreekmapping{newtx}
  \caption{A demonstration of the \cs*{printchemgreekmapping} command.}
  \label{tab:showmapping}
\end{table}

\section{Additional Macros}\label{sec:additional-macros}

\begin{commands}
  \command{chemgreekmappingsymbol}[\marg{mapping name}\marg{symbol name}]
    A command which prints a specific symbol of a specific mapping.  The mapping
    \meta{mapping name} doesn't need to be active but package dependencies
    must be taken care of, \emph{i.e.}, if \meta{mapping name} needs a certain
    package to be loaded you should make sure that it is.
\end{commands}

\appendix

\section{Overviews Over the Mappings}\label{sec:overv-over-mapp}

\newcommand*\demonstratemapping[2][]{%
  \ifblank{#1}
    {
      \subsection{Mapping `\texttt{#2}'}
      \includegraphics{chemgreek-#2.pdf}
    }
    {
      \subsection{Mapping `\texttt{#2}' with package option \option*{#1}}
      \includegraphics{chemgreek-#2-#1.pdf}
    }
}

\demonstratemapping{default}
\demonstratemapping{var-default}
\demonstratemapping{textgreek}
\demonstratemapping{upgreek}
\demonstratemapping{newtx}
\demonstratemapping{kpfonts}
\demonstratemapping{pxgreeks}
\demonstratemapping[utopia]{mathdesign}
\demonstratemapping[charter]{mathdesign}
\demonstratemapping{fourier}
\demonstratemapping{textalpha}

\subsection{Mapping `\texttt{fontspec}' with Font `Linux Libertine'}
\includegraphics{chemgreek-fontspec.pdf}

\end{document}
