% Copyright 1995--1999 Stephan I. B"ottcher <stephan@nevis1.columbia.edu>
%
% This program can be redistributed and/or modified under the terms
% of the LaTeX Project Public License Distributed from CTAN
% archives in directory macros/latex/base/lppl.txt; either
% version 1 of the License, or any later version.
%
% $Id: ulineno.tex,v 3.1 2001/08/03 03:29:19 stephan Exp $
%
% 1999-06-11 SiB: updated for lineno.sty v3.00
\ProvidesFile{ulineno.tex}[2001/08/03 lineno.sty users manual] 
    %% <- UL 2011/02/13 for readprov.sty

\documentclass[12pt,a4paper,titlepage]{article}

                                              \date{\def\$##1: ##2 ##3${##2}%$
                                                                      Version 
   \$Revision: 3.1 $, 
   \$Date: 2001/08/03 03:29:19 $
                                                                             }
                                 \def\lineno.sty{\texttt{\itshape lineno.sty}}


\title{           \lineno.sty
\\                Users Manual
}
\author{      Stephan I. B\"ottcher
}

\usepackage{lineno}
\usepackage{url}

\def~{\verb~}
\catcode`\<\catcode`\~
\def<#1>{$\langle${\itshape#1}\/$\rangle$}
\catcode`\|\catcode`\~
\def|#1{{\ttfamily\string#1}}
\def\NL{<number>]}

\begin{document}
\begin{runninglinenumbers}
\renewcommand\linenumberfont{\normalfont\small\bfseries}
\setlength\linenumbersep{1cm}
\maketitle
\thispagestyle{empty}
\resetlinenumber
\tableofcontents
\clearpage
\setcounter{page}{1}
\pagewiselinenumbers

\section{When to use \lineno.sty}

The primary use of \lineno.sty is to get line numbers in your LaTeX
document.  \lineno.sty directly supports line numbers in various
modes, and if you have any wishes that are not directly supported,
chances are good that \lineno.sty can be adapted to do it.  And if it
turns out to be difficult with \lineno.sty, then it will be even more
difficult without.

There is one class of cases however, where \lineno.sty is not the most
appropriate tool.  When your text already comes in lines, like source
listings, tables, poetry,  then \lineno.sty will propably do the job,
but other tools could do it much more efficiently.

\lineno.sty can only number lines in the main text.  Lines in any kind
of boxes cannot be numbered.  But see section \ref{sec:ilineno} below.

\lineno.sty is especially useful when the text is broken into lines by
\TeX\ itself.  In this case, the task to attach line numbers is not
easy, and a tool like \lineno.sty has good chances to break when used
together with other packages.

On the other hand, \lineno.sty proved to be quite stable, and I did
not get any reports of complete failure yet.

With \lineno.sty, you can refer to the line numbers via \LaTeX s
~\ref~ and ~\pageref~ macros.

There are other applications, not directly related to line numbering.
Whenever you want to attach something to your text, after a paragraph
was broken into lines, across pagebreaks, lineno may do the job.

\begin{bframe}
  Say, you need a frame around some text, but the text should be able
  to break across pages, \lineno.sty can be used to draw rules on both
  margins, which can be completed at the top and bottom by some
  special environment, see section\,\,\ref{sec:bframe}.
\end{bframe}

There may be cases, where other packages do a better job, e.g.,
\url{CTAN:macros/latex/contrib/supported/numline}.


\section{Where to get \lineno.sty}

\lineno.sty is available on the \emph{Comprehensive TeX Archive
  Network} (CTAN), e.g, \url{ftp:// ftp.tex.ac.uk/ tex-archive/
  macros/ latex/ contrib/ supported/ lineno/ lineno.sty}.

The home of \lineno.sty is \url{http:// www.nevis.columbia.edu/
  ~stephan/ tex/ #lineno}.  Here you can get the latest version.

\end{runninglinenumbers}

\section{How to use \lineno.sty}

As any other \LaTeXe\ package, you need to load it with
\begin{quote}
   ~\usepackage[~<options>~]{lineno}~
\end{quote}

\subsection{Package options}

The available <options> are:
\begin{description}

\item[\texttt{[left]}:]  Put the line numbers into the left margin
  (default). 

\item[\texttt{[right]}:]  Put the line numbers into the right margin.

\item[\texttt{[switch]}:]  Put the line numbers into the outer margin
  (left on even numbered pager, right on odd numbered pages).

\item[\texttt{[switch*]}:]  Put the line numbers into the inner
  margin.
  
\item[\texttt{[pagewise]}:] Number the lines from 1 on each page.

\item[\texttt{[running]}:]  Continuously number the lines (default).
  
\item[\texttt{[modulo]}:] Print line numbers only if they are
  multiples of five.

\item[\texttt{[mathlines]}:]  When using the ~{linenomath}~
  environment for display math, put line numbers also on the lines of
  the display.

\item[\texttt{[displaymath]}:]  Automatically wrap the standard LaTeX
  display math environments in ~{linenomath}~.

  \begin{itshape}
    Change from v2 to v3:  ~[mathlines]~ was called ~[displaymath]~
    before, and ~[displaymath]~ was ~\usepacke{mlineno}~.
  \end{itshape}
  
\end{description}
All effects of these option can also be obtained by calling certain
macros later in the document, except for ~[displaymath]~.

\subsection{How to turn on line numbering}

The line numbering is not activated by default.  If you want line
numbers, you have to turn it on.  The simplest way is to say
\begin{quote}
    ~\linenumbers~
\end{quote}
This will turn on line numbers within the current group, according to
the currently selected mode.  If you do it right at the beginning, the
whole document will be numbered.

~\linenumbers~ takes effect only when a paragraph is finished.  This
is important if you use ~\linenumbers~ for a single paragraph only.
This will \emph{not} work:
\begin{quote}\obeylines
   ~\begingroup~\\%
   ~\linenumbers~
     Text of your paragraph.
   ~\endgroup~
\end{quote}
because the ~\endgroup~ terminates the effect of ~\linenumbers~ before
the paragraph is finished.  The solution is to insert an empty line,
or an explicit ~\par~ before the ~\endgroup~.

The prefered solution is to use an enviroment:
\begin{quote}\obeylines
   ~\begin{linenumbers}~ 
     Text of your paragraph. 
   ~\end{linenumbers}~
\end{quote}
~\end{linenumbers}~ terminates the paragraph for you.  It will also
make sure that the next paragraph is not indented, if there is no
empty line in between.

This form will not finish any preceeding paragraph.  If you do not
leave an empty line before ~\begin{linenumbers}~, the previous
  paragraph may also get line numbers (this is not true after a list
  environment).

A quite esoteric form to number a single paragraph is:
\begin{quote}\obeylines
     Text of your paragraph.
     ~{\linenumbers\par}~
\end{quote}
Here, the ~\linenumbers~ is turned on in a group that just executes
the ~\par~ to terminate the previous paragraph.  Only this paragraph
will have line numbers. (~\par~ is exactly equivalent to an empty
line).


\subsection{How to turn off line numbering}

You do not usualy turn of line numbering explicitly, but rather
restrict line numbering to a group or an environment.  However, you
can turn it off with the command
\begin{quote}
   ~\nolinenumbers~
\end{quote}
This macro can also be used as an environment.  The same restrictions
apply as for ~\linenumbers~.


\section{Numbering modes}

Attaching line numbers after \TeX\ has broken the paragraph into lines
is already quite a difficult job for \TeX.  Why?  Because \TeX\ does
not normally give you any control after line breaking.  \lineno.sty
does it by pretending that the current page ends after each and every
line.  This keeps \TeX\ quite busy, but works.

Often, you want the line numbers to depend on the page they finally end
up.  E.g., you want the numbers to start with~ ~1 one each new page,
or you want them to appear on the outer margins in twoside mode.  This
is an almost impossible demand, but \lineno.sty can do it, at some
cost, in terms of computing time, memory, and disk space.  \lineno.sty
needs to write a note into the auxiliary file for each line in the
numbered part of the text, and \TeX\ keeps some memory for each page,
and all that takes some computation time to figure out on which page
the line was in the last \LaTeX\ run.

Thus, there are two basic numbering modes
\begin{enumerate}
  
\item \emph{Running} line numbers.  The numbers run continuously along
  the document, without considering page breaks.  
  
  The line numbers can be reset, or preset at any point in the
  document (except within a paragraph).  Margin switching (or any
  other dependence on the page breaks) is not possible.
  
\item \emph{Pagewise} line numbers.  The lines on each page are
  numbered from one.  Automatic margin switching is possible.

  \emph{Running} mode can be simulated in pagewise mode, to get
  automatic margin switching with continuous line numbers, but without
  the ability to reset/preset the line numbers anywhere in the
  document.  
  
  Any feature that depends on the page breaks must use this mode.

\end{enumerate}

\begin{runninglinenumbers*}

\subsection{\emph{Running} line numbers}

This is the default mode of operation.  The lines of your text are
numbered continuously across page breaks.  

You cannot get automatic margin switching in this mode, because at the
time when the line numbers are attached, it is not yet decided on
which page the line will end up.  However, you can simulate continuous
line numbers in \emph{pagewise} mode (see below), where margin
switching becomes possible.

You get this mode by
\begin{itemize}\raggedright
\item \emph{not} specifying any of the package options
  ~[switch,pagewise]~,
\item specifying the package option ~[running]~,
\item give the command ~\setrunninglinenumbers~ in the document, or
\item turn on line numbering with ~\runninglinenumbers~ or
  ~\begin{runninglinenumbers}~. 
\end{itemize}

\subsubsection{Reseting or setting the line number}

You can reset the line number with
\begin{quote}
  ~\resetlinenumber[~<number>~]~
\end{quote}
where the default is to reset the line number to one.

Alternatively, you can give the line number in an optional argument to
any command that turns on the line numbering in this mode:
\begin{quote}\obeylines
  ~\linenumbers[~<number>~]~ 
  ~\runninglinenumbers[~<number>~]~ 
  ~\begin{linenumbers}[~<number>~]~
  ~\begin{runninglinenumbers}[~<number>~]~
\end{quote}
or reset the line number to 1 with any of the following:
\begin{quote}\obeylines
  ~\linenumbers*~ 
  ~\runninglinenumbers*~ 
  ~\begin{linenumbers*}~
  ~\begin{linenumbers}*~
  ~\begin{runninglinenumbers*}~
  ~\begin{runninglinenumbers}*~
\end{quote}
This section of the manual is enclosed in
\begin{quote}\obeylines
  ~\begin{runninglinenumbers*}~
  ~\end{runninglinenumbers*}~
\end{quote}
\end{runninglinenumbers*}


\begin{pagewiselinenumbers}

\subsection{\emph{Pagewise} line numbers}

Basically, this mode prints line numbers starting with 1 on top of
each page, counting only lines where this mode is active.  If you
turn on this mode in the middle of a page, the numbers start with 1
at that point.

The \emph{pagewise} mode of operation is quite a bit more work for
\TeX.  When the line numbers are printed, \TeX\ does not know on which
page the current line will go.  To get \emph{pagewise} mode working,
\LaTeX\ has to remember on which page the line went in a previous run.
This information is written into the ~.aux~ file, and like a table of
contents, you will not see the numbers until you run \LaTeX\ at least
twice on the document.  Sometimes the information in the ~.aux~ file
becomes invalid in a way that causes \lineno.sty to hang.  In such a
case you'll have to delete all ~.aux~ files of the document to get it
going again.

You get this mode by
\begin{itemize}\raggedright
\item specifying the package option
  ~[pagewise]~, but not ~[running]~,
\item give the command ~\setpagewiselinenumbers~ in the document, or
\item turn on line numbering with ~\pagewiselinenumbers~ or
  ~\begin{pagewiselinenumbers}~. 
\end{itemize}

You cannot reset or preset the line numbers in \emph{pagewise} mode.
This mode uses a different counter than \emph{running} mode.  You can
switch between the mode, and the \emph{running} numbers will continue
where they left of, unless you reset the counter.

\subsubsection{Margin switching}

As a side effect of all the business to find out on which page the lines
end up, margin switching becomes possible.  It is now possible to
print the line number differently on odd and even pages.  The command
\begin{quote}
  ~\switchlinenumbers~
\end{quote}
will cause the line numbers to be printed on the outer margins, on the
right side for odd pages, and on the left side for even numbered
pages.  If you add a star, the numbers go on the inner margins
\begin{quote}
  ~\switchlinenumbers*~
\end{quote}
You can achieve the same effect by specifying the package
option ~[switch]~ or ~[switch*]~ together with ~[pagewise]~. 

Margin switching is independent of any ~[twoside]~ option of the
document class.

\subsubsection{\emph{Running} mode with margin switching}

Margin switching is not possible in \emph{running} mode, but you can
abuse the pagewise mode to produce continuous line numbers.  This mode
is activated with the command
\begin{quote}
  ~\runningpagewiselinenumbers~
\end{quote}
It is not possible to reset the line number counter for this mode, the
lines will be numbered starting from one at the beginning of the
document, and continue.  You can switch back to real pagewise mode with
\begin{quote}
  ~\realpagewiselinenumbers~
\end{quote}
but when you resume \emph{running-pagewise}, the numbers will not
continue where they left off, but jump by the number of lines that
were numbered \emph{real-pagewise}.

The package options ~[switch]~ and ~[switch*]~ will turn on
\emph{running pagewise} mode with margin switching, unless you also
specify ~[pagewise]~.

Finally, if you give these options together with ~[running]~, like
\begin{quote}
  ~\usepackage[switch,pagewise,running]{lineno}~
\end{quote}
the mode is set to plain \emph{running} line numbers, but later in
your document, if you say ~\pagewiselinenumbers~, the setings of the
~[switch]~ and ~[pagewise]~ options will take effect.

\subsubsection{Page numbers}

Any pages that contain \emph{pagewise} numbered lines must have a
different page number.  This may cause trouble with titlepages or
simmilar fudging with the page number counter in the document.  (I am
talking about ~\c@page~ or ~\count 0~ here, not ~\thepage~).

If you get strange errors, like \emph{``the use of} ~\nextLN~
\emph{does not match its definition''}, you should first delete all
~.aux~ files, and rerun \LaTeX\ several times.  If the error does not
reappear, fine, else check your page numbers, including those that are
not printed.  You may use \emph{running} line numbers on your
titlepages, since the page boundaries are predefined, usually.

\end{pagewiselinenumbers}

\begin{linenumbers*}

\subsection{Margin selection}

In any mode you can select a fixed margin on which the numbers shall
be printed.  You can either give a package option ~[left]~ or
~[right]~, or change the margin within the document with
\begin{quote}\obeylines
  ~\leftlinenumbers~
\rightlinenumbers
  ~\rightlinenumbers~
\end{quote}
\rightlinenumbers
Neither of these commands actually turns on line numbering.  The
margin setting is changed for the currently selected mode
\emph{running} or \emph{pagewise}.  If you want to set the margin for
both modes, append a star
\begin{quote}\obeylines
  ~\leftlinenumbers*~
  ~\rightlinenumbers*~
\end{quote}
The default is ~\leftlinenumbers*~.

\end{linenumbers*}


\subsection{Summary of mode selection commands}

The following commands change the major line numbering mode.
\begin{description}\itemsep=0pt\parskip=0pt
\item [|\setrunninglinenumbers]
\item [|\setpagewiselinenumbers] \leavevmode\\
  set the current mode, without activating or deactivating line
  numbering. 
\item [|\runninglinenumbers\texttt{*[\NL}]
\item [|\pagewiselinenumbers] \leavevmode\\
  set the current mode and activate line numbering, optionally reset
  (~*~), or set the line number for \emph{running} mode. 
\end{description}
The following commands do not change the major mode, but modify the
the behaviour of \emph{pagewise} mode
\begin{description}\itemsep=0pt\parskip=0pt
\item [|\realpagewiselinenumbers]
\item [|\runningpagewiselinenumbers] \leavevmode\\
  select if \emph{pagewise} mode should number continuously, or really
  pagewise.
\end{description}


\subsection{Summary of margin selection commands}

The following command select into which margin the numbers should be
printed 
\begin{description}\itemsep=0pt\parskip=0pt
\item [|\leftlinenumbers\texttt*]
\item [|\rightlinenumbers\texttt*] \leavevmode\\
  put the line numbers in the left/right margin in both modes.
\item [|\leftlinenumbers]
\item [|\rightlinenumbers] \leavevmode\\
  put the line number into the left/right margin in the currently
  selected mode.
\item [|\switchlinenumbers] \leavevmode\\
  put the line numbers into the outer margin, when using
  \emph{pagewise} mode.
\item [|\switchlinenumbers\texttt*] \leavevmode\\
  put the line numbers into the inner margin, when using
  \emph{pagewise} mode.
\end{description}

\begin{linenumbers*}
\modulolinenumbers[2]

\subsection{Numbering only one in five lines}

You can ask lineno to print the line numbers only if they devide by
<n>, by saying
\begin{quote}
  ~\modulolinenumbers[~<n>~]~
\end{quote}
If you omit the optional argument, it will be every fifth line, or
whatever you used last (the current value of the counter
~linenumbermodulo~).  You get the same effect by giving the package
option ~[modulo]~.

If you set <n> to 1, the modulo value is not changed, but modulo
numbering is turned off.  

Changing the counter ~linenumbermodulo~ does not turn on modulo
numbering, it just changes the modulo, if it is turned on without the
optional argument.

This section of the manual is enclosed in
\begin{quote}
  ~\begin{linenumbers*}~ \\ 
  ~\modulolinenumbers[2]~

  ~\end{linenumbers*}~
\end{quote}

\end{linenumbers*}

\begin{pagewiselinenumbers}
\switchlinenumbers

\subsection{How the line numbers look like}

This section of the manual is enclosed in
\begin{quote}
 ~\begin{pagewiselinenumbers}~ \\
 ~\switchlinenumbers~

 ~\end{pagewiselinenumbers}~
\end{quote}
By default, the line numbers are set in
\begin{quote}
  ~\normalfont\tiny\sffamily~
\end{quote}
right justified to 10\,pt left of the nominal left edge of the text.
If the number is to appear on the right margin, it is right justified
to 20\,pt right of the nominal right edge of the text.

You can easily change that, of course.  The font is defined by the
macro
\begin{quote}
  ~\linenumberfont~
\end{quote}
You can redefine it to make the numbers appear in a differnt font
(with ~\renewcommand~).  I'll say right here in this manual
\renewcommand\linenumberfont{\normalfont\bfseries\small}
\begin{quote}
  ~\renewcommand\linenumberfont{\normalfont\bfseries\small}~
\end{quote}
The distance on the left margin can be changed with
\begin{quote}
  ~\setlength\linenumbersep{~<distance>~}~
\end{quote}
The distance on the right margin is ~\linenumbersep~ plus
~\linenumberwidth~, both are set to 10\,pt by default.
Let's do right here
\setlength\linenumberwidth{1cm}
\begin{quote}
  ~\setlength\linenumbersep{1cm}~
\end{quote}
I put the command at the end of the paragraph, but it effects the
whole paragraph, because that is where everything happens: at the end
of the paragraphs.

The line numbers are printed as arabic numerals, as defined in the
macro
\begin{quote}
  ~\thelinenumber~
\end{quote}
for the \LaTeX\ counter ~{linenumber}~.  You can redefine that, as
for any other \LaTeX\ counter.  E.g.,
\renewcommand\thelinenumber{\roman{linenumber}}
\begin{quote}
  ~\renewcommand\thelinenumber{\roman{linenumber}}~
\end{quote}
(For heavy duty line numbering, you should use \LaTeX s internal
format, e.g., ~\@roman\c@linenumber~.)

\subsubsection{Deeper customization}

The line number is attached to the line by the macro
\begin{quote}
  ~\makeLineNumber~
\end{quote}
which is expanded inside a zero width ~\hbox~ alligned to the
origin of the current line (left edge, baselines alligned).  You can
do anything in this macro.  Let's assume you do not want line numbers
at all, but just attach something else to each line of text: this is
the macro to modify.  You can refer to the line number by the macro
~\LineNumber~, which prints the current value of the counter
~{linenumber}~, or nothing if \emph{modulo} mode is active, and the
number does not divide by ~{linenumbermodulo}~.  You should not modify
this macro when using any form of \emph{pagewise} mode.  At the end of
this paragraph I put the following
\begin{runninglinenumbers}
\renewcommand\makeLineNumber
   {\hss$\rightarrow$\ \rlap{\hskip\textwidth\ $\leftarrow$}}
\end{runninglinenumbers}
\begin{quote}
  ~\begin{runninglinenumbers}~\\
  ~\renewcommand\makeLineNumber~\\
  ~ {\hss$\rightarrow$\ \rlap{\hskip\textwidth\ $\leftarrow$}}~\\
  ~\end{runninglinenumbers}~
\end{quote}

Redefining ~\makeLineNumber~ cuts deep into the linenumber mode
setting.  If you still want to use the basic modes \emph{running} and
\emph{pagewise} you should go one level up and modify one of the
following macros, which are called by (or assigned to)
~\makeLineNumber~ depending on the current mode, and with the counter
~{linenumber}~ allready adjusted for the current page when using
\emph{pagewise} mode.
\begin{quote}\obeylines
  ~\makeLineNumberRunning~
  ~\makeLineNumberOdd~
  ~\makeLineNumberEven~
\end{quote}
The first is used in \emph{running} mode, the other two in
\emph{pagewise} mode.  If you modify those, you cut into the
margin selection mechanism as advertised above.  You will have to call
a mode selection command after changing these macros, to make them take
effect (at least for ~\makeLineNumberRunning~).  

You can go up one more step.  The margin selection commands redefine
the above macros by making them equivalent (~\let~) to one of the
following macros
\begin{quote}\obeylines
  ~\makeLineNumberLeft~
  ~\makeLineNumberRight~
\end{quote}
You may redefine these, and afterwards issue a margin selection
command followed by a mode selection command.  The default definitions
are
\begin{quote}
  ~\def\makeLineNumberLeft~\\
  ~  {\hss\linenumberfont\LineNumber\hskip\linenumbersep}~
\end{quote}
and
\begin{quote}
  ~\def\makeLineNumberRight~\\
  ~  {\linenumberfont\hskip\linenumbersep\hskip\textwidth~\\
  ~    \hbox to\linenumberwidth{\hss\LineNumber}\hss}~
\end{quote}
Whatever you do, use ~\LineNumber~ to print the digits.  This ensures
that the modulo mode setting is acknowledged.  If you want to
customize that as well, this is the macro to change.  ~\LineNumber~
eventually refers to ~\thelinenumber~, which is the place to customize
the numerals themselves.  At the end of this paragraph I said
\renewcommand\LineNumber{\ifodd\value{linenumber} \thelinenumber)\fi}
\begin{quote}
   ~\renewcommand\LineNumber~\\
   ~   {\ifodd\value{linenumber} \thelinenumber)\fi}~
\end{quote}
Line number references are not affected by ~\LineNumber~, but you
should keep track of ~\thelinenumber~.  Look at the following
reference
\begin{quote}
  The setion title of the next section is on page \pageref{ll1}, line
  \ref{ll1}.
\end{quote}
There is a mismatch in the definition of ~\thelinenumber~, from the
point of the refered line to the reference here.
\end{pagewiselinenumbers}

\begin{pagewiselinenumbers}

\section[Line number references]
        {Line number references\linelabel{ll1}}

\lineno.sty allows to refer to a line number with ~\ref{~<label>~}~.
The label is set by ~\linelabel{~<label>~}~, anywhere in a paragraph
that gets line numbers.  If you put a ~\linelabel~ somewhere else, the
line number it refers to is pretty meaningless, but no error message
is issued.  (This section begins on line \ref{ll1}, page
\pageref{ll1} and ends on line \ref{ll2}, page \pageref{ll2}.)

References work both in \emph{running} and \emph{pagewise} modes.
However, you have to be very careful when using both \emph{real
  pagewise} mode and \emph{running pagewise} mode: The mode setting
must be the same during reference as during labeling.

The restrictions on placing a ~\linelabel~ are the same as for a
~\marginpar~.  (In fact, they are implemented as fake ~\marginpar~s.)
E.g., the linelabel on the section header of this chapter could not be
placed in the straight forward way, because it would cause an error in
the table of contents.  (It should be disabled during \emph{toc}
processing.)  Instead, I typed
\begin{quote}
  ~\section[Line number references]~\\
  ~        {Line number references\linelabel{ll1}}~
\end{quote}
This works fine, with the standard \LaTeX\ ~{article}~ document
style.  If more generic markup is required, you should avoid things
like this (you should avoid \lineno.sty altogether, in that case).

~\linelabel~ does not work in display math.  (Why,  you do not use
equation numbers?)

The ~\linelabel~ command starts a new paragraph if it is issued after
an empty line.  It will properly ignore further whitespace before the
first word of text. Thus you can easily get a label to the first line
of a paragraph.  

Furthermore, ~\linelabel~ uses \LaTeX s standart tricks to avoid
unnecessary spaces if you place is between two words with spaces
around, but you should not do that anyway, if you want to make sure
that the label applies to a certain word in the text.  You should
attach the line label to the word in question without intervening
spaces, either before of after the word.  If you leave spaces both
before and after the ~\linelabel~ command, the current implemenation
attaches the line label to the following word (by ignoring the
trailing space).

Special care should be taken at the end of a paragraph.  You better
put the line label without intervening spaces behind the period, like
this 
\begin{quote}
  ~This is the end of this section.%~\\
  ~\linelabel{ll2}~
\end{quote}
This is the end of this section.%
\linelabel{ll2}

\end{pagewiselinenumbers}

\begin{linenumbers*}


\section{Known incompatibilities}

This section will expand whenever somebody discovers problems when
using \lineno.sty together with other \LaTeX\ packages.

\subsection{\texttt{wrapfig.sty}}

\lineno.sty used to have problems with \texttt{wrapfig.sty}, but this
is solved but \lineno.sty, version 2.05.


\section{Extension packages}

The extension packages were abolished with version v3.00 of
\lineno.sty.  One package, ~itemrule.sty~, was dropped, the others
were put directly into the the main \lineno.sty package file.

\subsection{Display math, \texttt{\itshape mlineno}}

\lineno.sty does not work well with display math.  The parts of a
paragraph preceeding a display math will not get line numbers.  The
reason is that the paragraph is not finished, but the part above the
display is broken into lines anyway.

Let me demonstrate the effect.  This paragraph contains a
~{displaymath}~ environment.  The part before the Formula does not get
line numbers.  Here comes the Formula
\begin{displaymath}
    \mathrm{6\,H_2O + 6\,CO_2 \ \rightarrow\  C_6H_{12}O_6 + 6\,O_2}
\end{displaymath}
which makes the line numbers disappear.  The trailing part of the
paragraph does get line numbers.  

There are several means to solve the problem, but none of them is
fully satisfactory.  \lineno.sty defines a ~{linenomath}~ environment,
which can be wrapped around a display math to make the line numbering
work.  This will work with any kind of display math, even if you use
explicit ~$$~ pairs.  There is a ~*~-form ~{linenomath*}~ which also
numbers the lines of the display itself, as good as it can.  Here
comes an example use of ~{linenomath*}~
\begin{linenomath*}
\begin{eqnarray*}
  b_1 &=& a_{11}x_1 + a_{12}x_2 \\
  b_2 &=& a_{21}x_1 + a_{22}x_2   
\end{eqnarray*}
\end{linenomath*}
wrapping an ~{egnarray*}~ like this
\begin{quote}
  ~\begin{linenomath*}~\\
  ~  \begin{eqnarray*}~\\
  ~    b_1 &=& a_{11}x_1 + a_{12}x_2 \\~\\
  ~    b_2 &=& a_{21}x_1 + a_{22}x_2~\\   
  ~  \end{eqnarray*}~\\
  ~\end{linenomath*}~
\end{quote}
Specifying the package option ~[mathlines]~, causes the
~{linenomath}~ environment to switch its function with its *-form.
This allows you to later decide if you want the numbers on the formulas
or not, without putting in stars everywhere (as long as you have been
consistent).

The ~{linenomath}~ wrapper does nothing if line numbering is not
activated.  You do not need to remove them if you decide to omit the
line numbering later (although you still have to load \lineno.sty).

The \texttt{\itshape mlineno} extension redefines the standard \LaTeX\ 
display math environments to wrap themselves automatically into
~{linenomath}~:
\begin{quote}\obeylines
   ~\[~ ~\]~
   ~{displaymath}~
   ~{equation}~
   ~{eqnarray}~
   ~{eqnarray*}~
\end{quote}
As of version 3.00 you can activate this by the package option
~[displaymath]~.  You can keep existing explicit ~{linenomath}~
wrappers without harm.  

This will probably not work with AMSmath.

Other solutions to make \lineno.sty work with display math are either
ugly or less stable.  The ugly possibility is to precede every display
math with an empty line, or better with
\begin{quote}
  ~{\par\nobreak\noindent}~
\end{quote}
but this gains you nothing compared to using the ~{linenomath}~
environment, since you still have to modify your document.

\end{linenumbers*}


\subsection{\texttt{\itshape rlineno}}

\begin{runninglinenumbers}
It is currently not possible to put a ~\linelabel~ into math mode.
This should not impose problems with inline math, since you can attach
the label just outside the math shift, like\linelabel{rll1} 
\begin{quote}
  ~... \linelabel{~<label>~}$x=f(y)$ ...~
\end{quote}
but to refer to a line in display math, you cannot label it directly. 
\end{runninglinenumbers}

\begin{pagewiselinenumbers}
The extension \texttt{\itshape rlineno} offers a solution,
allowing to refer to a line number with an offset.  

This extension is now available with plain \lineno.sty v3.00.

You put the ~\linelabel~ at the end of the preceeding line, and refer
to that number, plus 1 or, in case of multiline displays, by any other
offset.\linelabel{rll2}
\begin{quote}
  ~...\linelabel{~<label>~}~ \\
  ~\begin{equation}~ \\
  ~     y=f'(x)~ \\
  ~\end{equation}~ \\
  ~... the formula in line \lineref[+1]{~<label>~}~ 
\end{quote}
The offset calculation is different for pagewise or running line
numbers, and you will have to make sure the right method is used. 
\end{pagewiselinenumbers}

The macro ~\lineref[~<offset>~]{~<label>~}~ uses the method of the
currently selected mode (line numbering need not be activated).  To
specify the mode explicitly, use ~\linerefr~ or ~\linerefp~ for
running or pagewise mode respectively.

For example, the manual sets a ~\linelabel{rll1}~ at the end of the
line preceeding the single line quote environment at
line\,\linerefr[+1]{rll1} in the first paragraph of this chapter, and
refered to it via
\begin{quote}
  ~\linerefr[+1]{rll1}~
\end{quote}
and another one preceeding the quote environment in the second
paragraph of this chapter, with the central line
numbered\,\linerefp[+3]{rll2}, which I just refered to via
\begin{quote}
  ~\linerefp[+3]{rll2}~
\end{quote}
\begin{runninglinenumbers}
If the correct mode is selected (e.g., ~\setrunninglinenumbers~),
you can use ~\lineref[+1]{rll1}~ to refer to
line\,\lineref[+1]{rll1}.
\end{runninglinenumbers}


\subsection{\texttt{\itshape numquote}}

This extension defines two environments, each of them with a *-form or
alternatively an optional argument.
\begin{quote}\obeylines
  ~\begin{numquote}~
  ~\end{numquote}~
\end{quote}
is like ~{quote}~,
\begin{quote}\obeylines
  ~\begin{numquotation}~
  ~\end{numquotation}~
\end{quote}
is like ~{quotation}~, but with the lines numbered.  Numbering
restarts with 1 each time it is used, except you use the *-form, or
specify the line number to start with in an optional argument.
\begin{numquote}
The line numbers set to the left of the text, indented by the same
amount as the quote or quotation.  This works as expected even if the
quote appears in a nested list. 
\end{numquote}
To customize the distance from the text or the shape of the
numbers, you can modify the macros ~\quotelinenumbersep~ and
~\quotelinenumberfont~ respectively.

As of version 3.00, this is available directly with \lineno.sty.

\clearpage

\subsection{\texttt{\itshape ilineno}}
\label{sec:ilineno}

\begin{center}
\noindent\fbox{\parbox{0.8\textwidth}{
  \internallinenumbers \resetlinenumber[13]
  Sometimes you need line numbers within a |\parbox\ or figure.  This
  extension can do that, if the text is simple enough.  It works by
  drawing a ladder of numbers next to the paragraph, each number
  vertically separated by |\baselineskip\ from the other.  This
  obviously dose not work nicely with anything but straight text.
  Extra high lines (like this: $\displaystyle x=\frac AB$) will
  disturb the alignment.  The power of \lineno.sty is not necessary to
  achieve this, but you may need both in one document with a somewhat
  coherent interface, and some of the necessary infrastructure from
  \lineno.sty is actually useful.
}}\end{center}
% Why is the v-spacing wrong here?

The preceding paragraph was set like this:
\begin{quote}
  ~\begin{center}\fbox{\parbox{0.8\textwidth}{~ \\
    ~\internallinenumbers~ \\
    ~\resetlinenumber[13]~ \\
    ~ ~ Sometimes you need \dots \\
  ~}}\end{center}~
\end{quote}

\subsection{\texttt{\itshape itemrule.sty}}

Somebody asked me if he can get a rule next to an itemized item, from
the bullet downwards.  This style did that.  It was unstable and of
little interest, so it is gone from the \lineno.sty distribution.

\subsection{\texttt{\itshape bframe}}
\label{sec:bframe}

\setlength\bframesep{10pt}
\begin{bframe}
\lineno.sty can be used for unexpected purposes.  The environment 
\begin{quote}\obeylines
  ~\begin{bframe}~
  ~\end{bframe}~
\end{quote}
draws boxes around some text, even across page breaks.  At a page
break, the box is open (it is difficult to fix that).  There are two
parameters:
\begin{description}\itemsep=0pt\parskip=0pt
\item [|\bframesep] \leavevmode\\
  defines the separation of the box from the text.
\item [|\bframerule] \leavevmode\\
  defines the thickness of the lines around the text.
\end{description}

It works by asking \lineno.sty to draw small rules left and right to
the lines.  The total height of the rules is ~\baselineskip~ plus
~\bframesep~. This leaves some extra space for extra high lines.

The current version is very preliminary.  In a future version there
may be a possibility to move the vertical lines inwards for indented
text.  The box may optionally be closed at page breaks.  It will not
likely be possible to have a colored background in the box, at least
not with a graphics driver that cannot put a background behind
existing foreground (Postscript cannot, I've been told).  Tell me how
to make a gray background, like block dots on transparent, and I can
do shaded boxes.

\end{bframe}

\section{How to print the documented source}

\begin{bframe}
\lineno.sty is written in three programming languages: \TeX, sh,
and awk.

The file can be loaded as a \LaTeXe\ package as it is.  However, if
you feed it into a Unix shell, like
\begin{quote}
   ~csh>  source ./lineno.sty~
\end{quote}
it will produce the files ~lineno.tex~ and ~lineno.dvi~, which is the
documented source of \lineno.sty.  If you want to learn how it works,
or you need to adapt \lineno.sty for some special requirements, you
should read that document.
\end{bframe}

\end{document}
